{
 \renewcommand{\baselinestretch}{1.25}
 
 \bfseries\Large Предисловие\par\bigskip
 \mdseries\normalsize
 
 %Честно сказать, на лекциях профессора Голода мне не раз хотелось
 %есть. Сказывалось, что алгебра была второй / третьей парой, а
 %завтрак -- в 6 утра. Поэтому концентрация на преподаваемом материале
 %была нулевая, особенно учитывая тот факт, что Евгений Соломонович не
 %отличается гроким и понятным голосом, а изложение материала порой
 %бывало каким-то странным; я выбрал в данном конспекте несколько
 %необычную последовательность изложения материала, которая не
 %совпадает ни с одним из них уже написанных учебников, но она мне
 %кажется наиболее логичной в плане применения уже полученных знаний к
 %вновь приобретаемым.
 
 %Cтоило огромного труда впоследствии понять, что же такого профессор
 %Голод пытался донести до сытых до знаний студентов-раздолбаев. В
 %составлении этого документа мне помогли конспекты {\bfseries Алёны
 %Соболевой} [104] и, разумеется, мои собственные. Также выражаю
 %благодарность тем, кто принял участие в исправлении ошибок и
 %опечаток, выполнил часть работы за меня, а именно {\bfseries Ивану
 %Пузыревскому} [108] и многим другим.\medskip

 %\begin{center}Дерзайте!\end{center}

\vskip10mm
\begin{center}
{\color{red}\bf Внимание! Готовиться к экзамену по этому конспекту опасно
  для жизни! Имеются противопоказания!}
\end{center} 

\vskip20mm
\begin{flushright}
Если вам помог этот документ в сдаче экзамена, не поленитесь, подойдите и скажите "спасибо"... 
Знаете, как приятно? Любое другое вознаграждение приветствуется.
\end{flushright}
 
 \begin{flushright}
 \emph{{\bfseries Борис Агафонцев}, 102 группа[3mm]}
 \end{flushright}
}
 
 \hrule
 
 \begin{thebibliography}{books}
   \bibitem{lectures} Конспекты лекций по алгебре. \copyright\, МехМат, I курс, 1-й поток, 2006-2007 уч.г.
   \bibitem{golod} Е.С. Голод. \emph{Курс лекций по алгебре}. М., Издательство Центра прикладных исследований при механико-математическом факультете МГУ, 2004
   \bibitem{kurosh} А.Г. Курош. \emph{Курс высшей алгебры}. М.: Наука, 1971
   \bibitem{kostrikin_lectures} А.И. Кострикин. \emph{Введение в алгебру. Основы алгебры.}. М.: Физматлит, 1994
   \bibitem{fairytail} Пузыревский И.В. \emph{Сказка про $\ep$-окрестности}. М.: Hewlett-Packard, 2006
   \bibitem{l1557} Конспекты лекций И.Б. Кожухова в физико-математическом лицее №1557, \copyright\, Первая группа, 2004-2006.
   \bibitem{other}\dots

 \end{thebibliography}
 
 \vskip10mm
\dmvntrail

\markboth{}{}

\newpage
