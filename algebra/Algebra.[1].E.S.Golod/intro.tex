\subsection*{Предисловие}
 
 Честно сказать, на лекциях профессора Голода мне не раз хотелось
 есть. Сказывалось, что алгебра была второй / третьей парой, а
 завтрак -- в 6 утра. Поэтому концентрация на преподаваемом материале
 была нулевая, особенно учитывая тот факт, что Евгений Соломонович не
 отличается гроким и понятным голосом, а изложение материала порой
 бывало каким-то странным; я выбрал в данном конспекте несколько
 необычную последовательность изложения материала, которая не
 совпадает ни с одним из них уже написанных учебников, но она мне
 кажется наиболее логичной в плане применения уже полученных знаний к
 вновь приобретаемым.
 
 Cтоило огромного труда впоследствии понять, что же такого профессор
 Голод пытался донести до сытых до знаний студентов-раздолбаев. В
 составлении этого документа мне помогли конспекты {\bfseries Алёны
 Соболевой} [104] и, разумеется, мои собственные. Также выражаю
 благодарность тем, кто принял участие в исправлении ошибок и
 опечаток, выполнил часть работы за меня, а именно {\bfseries Ивану
 Пузыревскому} [108] и многим другим.



\begin{petit}
{\color{red}\bf Внимание! Готовиться к экзамену по этому конспекту опасно
  для жизни! Имеются противопоказания!}
\end{petit} 

\begin{petit}
Если вам помог этот документ в сдаче экзамена, не поленитесь, подойдите и скажите "спасибо"\ldots
Знаете, как приятно? Любое другое вознаграждение приветствуется.
({\bfseries Борис Агафонцев}, 102 группа)
\end{petit}
 
 
