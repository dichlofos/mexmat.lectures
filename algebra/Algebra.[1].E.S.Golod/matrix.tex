\section{Матрицы}
\label{matrix}

\epigraph{Errors using inadequate data are much less than those using
  no data at all.}{Charles Babbage}

\subsection{Основные понятия}

\begin{df}
  \emph{Матрица}\index{матрица} -- прямоугольная таблица из чисел или любых других объектов.
  $$
  A =
  \begin{pmatrix}
   a_{11}&a_{12}&\dots&a_{1n}\\
   a_{21}&a_{22}&\dots&a_{2n}\\
   \vdots&\vdots&\ddots&\vdots\\
   a_{n1}&a_{n2}&\dots&a_{mn}
  \end{pmatrix}
  $$
\end{df}

Над матрицами определены следующие операции:
\begin{enumerate}
\item Умножение на скаляр.
\item Транспонирование.\index{матрица!транспонированная} Строки
  записываются в столбцы. Обозначается $A^T$
\item Сложение матриц. Определено только для матриц одинакового
  размера, производится почленное сложение всех элементов матрицы.
\item Умножение матриц. Определяется следующим способом: пусть даны
  матрицы $A$ размером $m\times n$ и $B$ размером $n\times k$. Тогда
  матрица $C=AB$ будет состоять из элементов
  $c_{i_j}=\sum\limits_{k=1}^n a_{i_k}b_{k_j}$. 
\end{enumerate}
  
Так как матрицы являются обобщением векторов, то операциям сложения и
умножения на скаляр соответствуют все заявленные для векторов 8
свойств, некоторые другие свойства теории линейной зависимости также
сохраняются.
  
Можно также доказать следующие свойства, относящиеся к вновь определённым операциям:
\begin{enumerate}
\item $(A+B)^T = A^T+B^T$
\item $(A+B)C=AC+BC$
\item $A(BC)=(AB)C$
  \begin{proof}
    Определим сначала базис пространства матриц $M_{m\times
      n}(\R)$. Он будет состоять из \emph{матричных
      единиц}\index{матричная единица} $E_{ij}$ \emph{(на
      пересечении i-й строки и j-го столбца стоит 1, а все
      остальные элементы равны 0}. Всякая матрица будет
    представима в виде линейной комбинации матричных единиц.
    
    Теперь осталось проверить только ассоциативность умножения
    матричных единиц. В общем
    случае 
    \begin{displaymath}
      E_{ij}E_{kl}=\left\{\begin{array}{ll}E_{il}, &   k=j\\0,&k\ne j\end{array}\right..
    \end{displaymath}
    Легко проверяется, что
    умножение матричных единиц ассоциативно. Из этого следует, что
    и умножение любых матриц ассоциативно, так как ранее мы
    доказали свойство дистрибутивности.
  \end{proof}
\item $(AB)^T=B^TA^T$
\end{enumerate}

\begin{df}
  \emph{Следом}\index{след матрицы} матрицы $\tr A$ называется сумма диагональных элементов матрицы. 
  %Можно заметить, что для этой функции существует ряд очень интересных свойств, 
  % о которых, может быть, будет рассказано позднее.
\end{df}
  
Докажем следующее важное свойство: $\tr(AB)=\tr(BA)$:
\begin{proof}
  Проведём доказательство одной строкой:
  $$\tr(AB)=\sum_i\sum_k a_{ik}b_{ki}=\sum_k\sum_i a_{ik}b_{ki}=\sum_k\sum_i b_{ki}a_{ik}=\tr(BA)$$%
\end{proof}

\begin{df}
  Единичной матрицей\index{матрица!единичная} называется такая матрица $E$, что $AE\bw=EA\bw=A$. Такая матрица имеет следующий вид (все элементы, кроме элементов главной диагонали, равны нулю):
  \begin{center}
    \scalebox{0.7}{
      $
        A =
        \begin{pmatrix}
          1&0&\dots&0\\
          0&1&\dots&0\\
          0&0&\ddots&\vdots\\
          0&0&\dots&1
        \end{pmatrix}
      $
    }
  \end{center}
\end{df}
  
\begin{df}
  Если в строке матрицы все элементы -- нули, то такая строка
  называется \emph{нулевой}. Первый ненулевой элемент в строке
  называется \emph{главным} или \emph{лидером}.
\end{df}
  
\begin{df}
  \emph{Элементарными преобразованиями (далее: ЭП) строк
    матрицы}\index{элементарные преобразования} называются
    преобразования вида:
  \begin{enumerate}
  \item К i-й строке прибавить j-ю, умноженную на $\lambda\in\mathbb R$
  \item Поменять местами i-ю и j-ю строки
  \item Умножить i-ю строку на $\lambda\ne0$
  \end{enumerate}
  Элементарные преобразования строк или столбцов матрицы можно
  записать как произведение исходной матрицы на одну из
  \emph{элементарных матриц}\index{элементарные матрицы} (прямых или
  транспонированных). Если матрицу умножить на одну из элементарных
  матриц слева ($A'=E_iA$), то соответствующее преобразование затронет
  строки, если справа -- столбцы.
  
  
  \begin{center}
    \scalebox{0.75}{
  $
    E_1 =
  \begin{pmatrix}
   1&      & &       &       &      & \\
    &\ddots& &       &       &      & \\
    &      &1&\dots  &\lambda   &\dots & \\
    &      & &\ddots &       &      & \\
    &      & &       &\ddots &      & \\
    &      & &       &       &1     &
  \end{pmatrix}\spc
  $
  }
  \scalebox{0.5}{
  $
  E_2 =
  \begin{pmatrix}
   1&      & &      & &      & &      & \\
    &\ddots& &      & &      & &      & \\
    &      &0&      & &      &1&      & \\
    &      & &\ddots& &      & &      & \\
    &      & &      &1&      & &      & \\
    &      & &      & &\ddots& &      & \\
    &      &1&      & &      &0&      & \\
    &      & &      & &      & &\ddots& \\
    &      & &      & &      & &     &1
  \end{pmatrix}
  \spc
  $
  }
  \scalebox{0.65}{
  $
    E_3 =
  \begin{pmatrix}
   1&      & &       & &      & \\
    &\ddots& &       & &      & \\
    &      &1&       & &      & \\
    &      & &\lambda& &  &      & \\
    &      & &       &1&      & \\
    &      & &       & &\ddots& \\
    &      & &       & &      &1\\
  \end{pmatrix}
  $
  }
  \end{center}
\end{df}

\newpage
\begin{df}
  Матрица называется ступенчатой\index{матрица!ступенчатая}, если она удовлетворяет следующим условиям:
  \begin{enumerate}
     \item Ниже нулевой строки (если она есть) находятся только нулевые строки
     \item В каждой ненулевой строке её лидер стоит строго правее, чем в предыдущей.
  \end{enumerate}
  %% FIXME: Bug somewhere here, but I am too asleep to fix it.
  %% \begin{center}
  %%   \scalebox{0.7}{
  %%     $
  %%     \tilde A' =
  %%     \left(
  %%     \begin{array}{c}
  %%       \begin{array}{m{3mm}|m{3mm}m{3mm}m{3mm}m{3mm}m{3mm}m{3mm}|m{3mm}}
  %%         &*& . & . & . & . & . & .\\
  %%         \cline{2-2}
  %%       \end{array}\\
  %%       \begin{array}{m{3mm}m{3mm}|m{3mm}m{3mm}m{3mm}m{3mm}m{3mm}|m{3mm}}
  %%         & & * & . & . & . & . & .\\
  %%         \cline{3-4}
  %%       \end{array}\\
  %%       \begin{array}{m{3mm}m{3mm}m{3mm}m{3mm}|m{3mm}m{3mm}m{3mm}|m{3mm}}
  %%      & &   &   & * & . & . & .\\
  %%         \cline{5-8}
  %%       \end{array}\\
  %%       \begin{array}{m{3mm}m{3mm}m{3mm}m{3mm}m{3mm}m{3mm}m{3mm}|m{3mm}}
  %%         &0&   &   &   &   &   & .\\
  %%       \end{array}\\
  %%   \begin{array}{m{3mm}m{3mm}m{3mm}m{3mm}m{3mm}m{3mm}m{3mm}|m{3mm}}
  %%     & &   &   &   &   &   &  \\
  %%   \end{array}\\
  %%     \end{array}  
  %%     \right)
  %%     $
  %% }
  %% \end{center} 
\end{df}
  
\begin{theorem}
  Любую матрицу можно привести к ступенчатому виду используя только ЭП 1 типа.
  \label{mat:echelon}
\end{theorem}
  
\begin{proof}
  Докажем теорему при помощи метода математической индукции. Индукцию
  проведём по количеству строк матрицы.
  
  Для $m = 1$ утверждение верно, так как матрица, состоящая из одной строки ступенчатая.
    
  Для удобства будем рассматривать матрицу без нулевых
  столбцов. Рассмотрим матрицу из m строк и её первую строку. Возможно
  две ситуации:
  \begin{enumerate}
  \item $a_{11}\ne0$. Тогда с помощью ЭП 1 типа мы можем сделать все остальные элементы этого столбца нулевыми, добавив к каждой (i-й) строке этой матрицы первую, умноженную на $-\frac{a_{i1}}{a_{11}}$. 
  \item $a_{11}=0$. Так как мы исключили из рассмотрения все нулевые столбцы, то существует ненулевой элемент в первом столбце. Добавим к первой строке строку, содержащую этот элемент и реализуем алгоритм, описанный в пункте 1.
  \end{enumerate}
    
    По предположению индукции подматрицу этой матрицы из $m-1$ строки (со 2-й по последнюю) можно привести к ступенчатому виду, одновременно с этим и вся матрица приведётся к ступечатому виду. Теорема доказана.
  \end{proof}
  
  \begin{df}
    Матрица называется \emph{сильноступенчатой}\index{матрица!сильноступенчатая} (приведённой к улучшенному ступенчатому виду), если над лидерами её строк находятся только нулевые элементы, а сами лидеры равны 1. Сильноступенчатый вид матрицы единственен.
  \end{df}

Приведение матрицы к сильноступенчатому виду включено в алгоритм нахождения базиса любой системы векторов, а также нахождения ФСР ОСЛУ.
  
  %\begin{df}
  %  Обратной матрицей к $A$ называется такая матрица $A^{-1}$, что $AA^{-1}=A^{-1}A=E$. Обратная матрица определена только у квадратных матриц. Если обратная матрица существует, то она единственна. К обратным матрицам мы вернёмся в процессе рассмотрения теории определителей (см. \mref{det}).
  %\end{df}
  
\newpage
\subsection{Ранг матрицы}

\begin{df}
  \emph{Рангом матрицы} называется ранг системы её столбцов.\index{ранг!матрицы}
\end{df}

\begin{theorem}[о ранге матрицы]\index{теорема!о ранге матрицы}
  Ранг системы столбцов матрицы равен рангу системы её строк и равен числу ненулевых строк в её ступенчатом виде.
\end{theorem}

\begin{proof}
  Для доказательства этой теоремы докажем несколько вспомогательных утверждений.
  
  \begin{lemma}
    При ЭП строк матрицы ранг системы столбцов не меняется.
  \end{lemma}
  \begin{proof}
    Если какая-то система столбцов была линейно зависима, то и преобразованная система столбцов остаётся линейно зависимой. Тогда сохраняется и линейная независимость столбцов. В таком случае базис исходной системы эквивалентен базису новой системы, таким образом они имеют одинаковое число векторов. Ранг системы векторов не изменяется.
  \end{proof}

  \begin{lemma}
    При ЭП строк матрицы ранг системы строк не меняется.
  \end{lemma}

  \begin{proof}
    Если система строк имела базис, то и любая система, состоящая из линейных комбинаций этих строк будет иметь такой же базис (как один из возможных вариантов).
  \end{proof}

  
  Приведём матрицу к сильноступенчатому виду. Теперь очевидно, что ранг системы строк равен числу ненулевых строк в ступенчатом виде (т.е. числу <<ступенек>> или главных элементов) и равен рангу системы столбцов.
\end{proof}

Очевидно, что если $C=AB$, то столбцы (строки) $C$ являются линейной комбинацией столбцов $A$ (строк $B$). В таком случае понятно, что если система столбцов $A$ была линейно независима, то и система столбцов $C$ будет также линейно независимой и если система строк $B$ была линейно независима, то и система строк $C$ будет линейно независимой.\index{ранг!произведения матриц}

Таким образом получаем $\rk (AB)\le\min(\rk A, \rk B)$. Если $B$ -- невырожденная матрица, то $\rk A=\rk (AB)$.\index{матрица!невырожденная}
