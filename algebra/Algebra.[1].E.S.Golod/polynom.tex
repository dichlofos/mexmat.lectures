\section{Кольцо многочленов}
\label{polynom}

\epigraph{I was more interested in skating and the girls and traveling
  than I was in calculus.}{Scott Hamilton}

\subsection{Построение кольца многочленов}

Пусть $\Kb$ -- коммутативное кольцо с единицей, тогда обозначим через $\Sb=\Kb[x]$ кольцо многочленов от одной переменной, если оно удовлетворяет следующим требованиям:\index{кольцо!многчленов}\index{многочлен}

\begin{enumerate}
  \item $\Kb\subset\Sb$
  \item $\exists\xb\in\Sb\spc\forall f\in\Sb\spc f$ однозначно представляется в виде многочлена от $\xb$ с коэффициентами из $\Kb$.
\end{enumerate}

Определим некоторые тривиальные понятия, а также проведём некоторые тривиальные проверки:

\begin{enumerate}
  \item $a_ix^i\cdot b_jx^j = a_ib_jx^{i+j}$
  \item $\sum\limits_i a_ix^i\cdot\sum\limits_jb_jx^j=\sum\limits_{i,j}a_ib_jx^{i+j}=\sum\limits_kc_kx^k$, где $c_k=\sum\limits_{i+j=k}a_ib_j$.
  \item $\dots$
\end{enumerate}

\begin{df}
  \emph{Степенью} многочлена $f$ называется наибольшее из чисел $k$, что одночлен $a_kx^k$ входит в представление $f$. Обозначается $\deg f$. Принято считать, что $\deg0=-\infty$, $\deg a = 0$.\index{степень!многочлена}
\end{df}

Отметим, что свойства кольца $\Kb[x]$ напрямую зависят от свойств кольца $\Kb$, то есть если $\Kb$ является областью целостности, то и $\Kb[x]$ является областью целостности; например: если $a$ неделитель нуля, то и $g=ax^n+\dots$ не является делителем нуля. В случае, если $g,h$ -- неделители нуля, то $\deg(gh)=\deg g+\deg h$.

Если в $\Kb$ нет делителей нуля, то обратимых неконстант в $\Kb[x]$ тоже нет.

\subsection{Функциональный взгляд}

Несмотря на то, что под многочленом мы прежде всего понимаем формальную запись, можно также составить функцию $f\colon \Kb\to\Kb$, такую, что для $$\forall f=a_0+a_1x+\dots\in\Kb[x]\spc\forall c\in\Kb\spc f(c)=a_0+a_1c+\dots\in\Kb.$$
Очевидно, что $f(c)+g(c)=(f+g)(c)$ и $f(c)g(c)=(fg)(c)$. Стоит отметить, что если $\Kb$ -- бесконечная область целостности, то равенство многочленов в алгебраическом и функциональном смысле равносильны. Составить контрпример для случая конечного поля не составит труда: в $\spc0\ne\prod\limits_i^n(x-c_i)$, но $\forall i\spc f(c_i)=0$. В случае $\Zb/p$ нетрудно вывести теорему Вильсона с помощью похожей конструкции: $(p-1)!\equiv-1\mod p$.

\begin{theorem}[Безу]
Если $f(x)=(x-c)h(x)+r$, то $r=f(c)$.\index{теорема!Безу}
\end{theorem}

\begin{df}
  Элемент $c\in\Kb$ называется \emph{корнем} $f$, если $f(c)=0$. Как следствие из теоремы Безу также получаем, что $c$ -- корень, если $x-c\mid f$. Элемент $c$ называется корнем кратности $k$, если $(x-c)^k\mid f$, но $(x-c)^{k+1}\nmid f$.\index{корень!многочлена}
\end{df}

\begin{theorem}
  Пусть $\Kb$ -- область целостности. Тогда любой многочлен единственным образом представим в виде $f(x)=(x-c_1)^{k_1}\dots(x-c_s)^{k_s}\cdot h(x)\in\Kb[x]$, где $h(x)$ не имеет корней в $\Kb$.
\end{theorem}
\begin{proof}
  Доказательство существования такого представления проводится по индукции и тривиально. Теперь к доказательству единственности. Тут тоже всё достаточно просто: пусть не так, тогда существует два представления:
  $$(x-c_1)^{k_1}\dots(x-c_s)^{k_l}\cdot h_1(x)= (x-d_1)^{l_1}\dots(x-d_t)^{l_l}\cdot h_2(x)$$
  
  За несколько операций нетрудно убедиться, что число различных корней и их кратности совпадают, а, следовательно, и $h_1=h_2$. Необходимо использовать тот факт, что в области целостности нет делителей нуля.
\end{proof}

Из этой теоремы имеется важное {\bfseries следствие}: число корней с учётом кратности не превосходит степени многочлена.

\subsection[Теория делимости многочленов]{Теория делимости в кольце многочленов от одной переменной над полем}

\begin{theorem}
  Пусть $\Pb$ -- поле. Тогда для любых двух многочленов $f$ и $g\ne0$
  существует единственное представление $f=gh+r$, где $\deg r<\deg
  g$.
\end{theorem}
\begin{proof}
  Пусть $\deg g=n$, $\deg f=m\ge n$ (иначе тривиально: $f\bw=g\cdot0+f$). Доказательство проведём по индукции по степени многочлена $f$. Пусть утвереждение теоремы верно для всех многочленов степени меньшей $m$. 
  
  Пусть $g = bx^n+\dots$, $b$ -- обратим в $\Pb$ и $f=ax^m+\dots$. Тогда рассмотрим разность $f-b^{-1}ax^{m-n}g = f_1,\spc\deg f_1<\deg f$, то есть по индуктивному предположению $f_1=gh_1+r$ и $f=f_1+b^{-1}ax^{m-n}g+r=g(h_1+b^{-1}ax^{m-n})+r=gh+r$.
  
  Осталось выяснить, почему такое представление единственно. Пусть $f=gu_1+v_1=gu_2+v_2$. Тогда $g(u_1-u_2)=r_1-r_2$. Если $u_1\ne u_2$, то $\deg (v_1-v_2)\ge\deg g$, чего не может быть, откуда следует, что $u_1=u_2$ и $v_1=v_2$.
\end{proof}

Пусть $\Ab$ -- область целостности. Говорят, что $b\mid a$, если $\exists u\in\Ab\spc a=bu$. Если деление возможно, то оно единственно в силу отсутствия делителей нуля. Два элемента называются \emph{ассоциированными} друг другу, если $b\mid a$ и $a\mid b$. Например, в поле $\Zb$ ассоциирванными друг другу являются только числа $\pm1$. В кольце $\Kb[x]$ многочлены $f$ и $g$ являются ассоциированными, если $\exists\alpha\in\Kb^*=\Kb\setminus\{0\}\colon f=\alpha g$.

Продолжая исследовать многочлены в кольце $\Kb[x]$, введём понятие \emph{неприводимого} многочлена:

\begin{df}
  Многочлен $f,\spc\deg f>0$, называется \emph{неприводимым}, если он не представим в виде произведения двух многочленов строго меньшей степени. Очевидно, что любой многочлен первой степени неприводим над любым полем. В дальнейшем же, приводимость одного и того же многочлена, но над разными полями может меняться.\index{многочлен!неприводимый}
  
  Понятие неприводимого многочлена в кольце многочленов от одной переменной в некоторой степени является аналогом простого числа в поле целых чисел. 
  
  Для области целостности $\Ab$ элемент $a\in\Ab$ называется неприводимым, если его нельзя разложить в произведение необратимых. Область целостности называется \emph{факториальной}, если для неё справедливо утверждение об однозначном разложении на неприводимые элементы. Примерами нефакториальных областей целостности являются %$\Zb[i]=\{a+bi\mid a,b\in\Zb\}$ (целые гауссовы числа)
$\Zb[i\sqrt{5}]$ и другие.\index{кольцо!факториальное}
\end{df}

Неприводимых многочленов над любым полем существует бесконечное количество. В случае бесконечного поля утверждение очевидно, а в случае конечного поля мы можем гарантировать существование неприводимых многочленов сколь угодно большой степени.

\begin{theorem}
  Всякий ненулевой многочлен в кольце многочленов над полем однозначно (с точностью до ассоциированности) представим в виде произведения неприводимых.
\end{theorem}
\begin{proof}
  Существование такого представления докажем индукцией по степени многочлена. В качестве основания индукции можно взять $\deg f=0$ или $\deg f\bw=1$. Индуктивный переход тривиален.
  
  Единственность такого представления докажем тоже по индукции. Пусть для всех многочленов, степени меньшей $n$ верно; проверим для $f,\spc\deg f\bw=n$. Пусть не так, и имеется два представления, занумеруем все неприводимые многочлены в порядке возрастания степени:
  $$f=ap_1p_2\dots p_s\qquad f=bq_1q_2\dots q_t$$
  
  Возможно 2 взаимоисключающих случая:
  \begin{enumerate}
    \item Среди $q_j\spc\exists j\colon p_1\mid q_j\ra q_j=p_1\cdot u$. Так как $q_j$ неприводим, то $u$ -- константа из поля, а значит мы можем сократить на $p_1$ и получим два разложения для многочлена строго меньшей степени, что невозможно по индуктивному предположению.
    \item Все $q_i$ не делятся на $p_1$. Тогда разделим $q_j,\spc\deg q_j\ge\deg p_1$, на $p_1$ с остатком. Далее считаем, что $q_j = q_1$ (перенумеруем для удобства). Тогда имеем следующее:
    $$f=ap_1p_1\dots p_s=b(p_1u+r)q_2\dots q_t=bp_1uq_2\dots q_t+brq_2\dots q_t=m+h$$
    
    Так как $p_1\mid f$ и $p_1\mid m$, то $p_1\mid h$. Отметим, что $\deg h<\deg f$. Тогда по индуктивному предположению представление $h$ в виде произведния неприводимых однозначно, а значит в его разложении обязательно присутствует $p_1\colon h=p_1g$. Но с другой стороны, так как $\deg r<\deg p_1$ и ни один из $q_i$ не делится на $p_1$, $p_1$ не может присутствовать в этом разложении.
    
    Получили противоречие, теорема доказана.
  \end{enumerate}
\end{proof}

Аналогично определению кратных корней для многочлена определятся понятие кратного вхождения неприводимых многочленов в разложение.

Стоит отметить, что кольцо многочленов факториально не только над полем, но и над любым факториальным кольцом. Об этом будет сказано позже.

\subsection{Наибольший общий делитель}

\begin{df}
Наибольшим общим делителем элементов $f$ и $g$ одновременно не равных нулю называется такой элемент $d$, что
 \begin{enumerate}
   \item $d\mid f$ и $d\mid g$
   \item $\forall d'\colon d'\mid f,d'\mid g\ra d'\mid d$
 \end{enumerate}\index{НОД}
\end{df}

%Пусть имеется два наибольших общих делителя -- $d_1$ и $d_2$. По определению $d_1\mid d_2$ и $d_2\mid d_1$, а значит они отличаются на обратимые множители, то есть равны с точностью до ассоциированности.

\begin{df}
  \emph{Наименьшим общим кратным} двух элементов $f$ и $g$ называется такой элемент $m$, что
 \begin{enumerate}
   \item $f\mid m$ и $g\mid m$
   \item $\forall m'\colon f\mid m',g\mid m'\ra m'\mid m$
 \end{enumerate}\index{НОК}
\end{df}

\begin{theorem}[Алгоритм Евклида]
  Пусть $\Pb$ -- поле. Тогда для $\forall (f,g)\ne(0,0)$, $f,g\in\Pb[x]$ наибольший общий делитель $f$ и $g$ определятся из системы равенств $(f,g)=\dots=(r_{s-1},r_s)=(r_s,0)=r_s$, где
  $$
   \left\{
     \begin{array}{lll}
       f  &=&   gq_0 + r_1\\
       g  &=& r_1q_1 + r_2\\
       r_1&=& r_2q_2 + r_3\\
       &&\dots\\
       r_{s-2}&=&q_{s-1}r_{s-1}+r_s\\
       r_{s-1}&=&q_sr_s + 0
     \end{array}
   \right.
  $$
\end{theorem}\index{алгоритм!Евклида}
\begin{proof}
  Доказательство этой теоремы проводится в два этапа: <<снизу вверх>>, когда проверяется, что $r_s$ действительно является делителем $f$ и $g$ и <<сверху вниз>>, когда проверяется, что $r_s$ -- наибольший из всех делителей.
\end{proof}

Стоит отметить, что НОД существует для любых элементов в факториальном кольце. Можно доказать факториальность евклидовых колец, то есть колец, в которых возможно задать функцию \emph{норма}, отвечающую свойствам $N(ab)\ge N(a)$ и $\forall a,b\spc\exists q,r\colon a=bq+r$ и $N(r)<N(b)$ или $r=0$.\index{кольцо!факториальное}\index{кольцо!евклидово}

Следствием из алгоритма Евклида является тот факт, что наибольший общий делитель $d$ элементов $f$ и $g$ представим в виде $d=fu+gv$. Более сильное утверждение (для многочленов) сообщает нам о следующем:

\begin{theorem}
  Для любых многочленов $f,g$ существует единственное представление $(f,g)=d=fu+gv$, где $\deg u<\deg g/d$ и $\deg v<\deg f/d$.
\end{theorem}

\begin{proof}
То, что существуют произвольные $\tilde u,\tilde v$, для которых выполенено $d=f\tilde u+g\tilde v$, нам уже известно. Пусть $\tilde f=f/d$, $\tilde g=g/d$. Тогда $1=\tilde f\tilde u+\tilde g\tilde v$. Разделим $\tilde u$ на $\tilde g$ с остатком: $\tilde u=\tilde gq+u,\spc\deg u<\deg\tilde g$ и подставим это выражение в вышенаписанное равенство: $1=\tilde f(\tilde gq+u)+\tilde g\tilde v=\tilde fu+\tilde g(q\tilde f+\tilde v)$.

Осталось проверить, что $\deg v$, где $v=q\tilde f+\tilde v$, меньше $\deg \tilde f$. Действительно, так как $\tilde gv=1-\tilde fu$, то
$$\deg\tilde g+\deg v=\deg(\tilde gv)=\deg(1-\tilde fu)<\deg (\tilde f\tilde g)=\deg\tilde g+\deg\tilde f\ra \deg v<\deg\tilde f$$
\end{proof}

\begin{theorem}[об остатках]
  Пусть $f_1,f_2,\dots,f_s$ и $g_1,g_2,\dots,g_s$ -- произвольные наборы многочленов с тем лишь условием, что $\forall i\ne j\spc (f_i,f_j)=1$. Тогда существует такой многочлен $h$, что остатки от деления $h$ на $f_i$ равны $g_i$.
\end{theorem}
\begin{proof}
  Доказательство проведём, естественно, по индукции по числу многочленов. Для $n=2$ верно: $$h=q_1f_1+g_1=q_2f_2+g_2\la g_2-g_1=q_1f_1-q_2f_2$$
  
  Последнее представление возможно, так как $(f_1,f_2)=1$.
  
  Пусть для $n=s-1$ такой многочлен $h_{s-1}$ существует, тогда, используя технологию доказательства для $n=2$ можем обобщить от $n=s-1$ к $n=s$, введя многочлен $\tilde f_2=f_2\cdot\dots\cdot f_s$ и получив многочлен $h_s$: для многочленов $f_2,\dots, f_s$ и $f_2,\dots , g_s$ требуемый условием теоремы многочлен $h_{s-1}$ существует. Тогда для многочленов $f_1,f_2$ и $g_1, h_{s-1}$ существует такой многочлен $h_s=f_1q_1+g_1=\tilde f_2\tilde q_2+h_{s-1}$. Этот многочлен удовлетворяет нашим требованиям. 
\end{proof}

\subsection{Многочлены над факториальными кольцами}\index{кольцо!факториальное}

\begin{df}
  \emph{Содержанием}\index{содержание} многочлена $c(f)$ будем называть наибольший общий делитель его коэффицинтов. Многочлены над факториальным кольцом называются \emph{примитивными}\index{многочлен!примитивный}, если его содержание равно 1, что равносильно тому, что содержание принадлжеит классу обратимых элементов кольца $\Kb$. 
\end{df}

В кольце $\Kb[x]$ неразложимыми многочленами являются неприводимые примитивные многочлены.

\begin{theorem}[лемма Гаусса]\index{лемма!Гаусса}
\label{polynom:gauss}
  Произведение примитивных -- примитивно.
\end{theorem}
\begin{proof}
Пусть
$$P(x)=a_kx^k+a_{k-1}x^{k-1}+\dots+a_1x+a_0,\spc (a_k,\dots,a_0)=1$$
$$Q(x)=b_lx^l+b_{l-1}x^{l-1}+\dots+b_1x+b_0,\spc (b_l,\dots,b_0)=1$$
Рассмотрим их произведение
\begin{multline*}
P(x)\cdot Q(x)=a_kb_lx^{k+l}+(a_kb_{l-1}+a_{k-1}b^l)x^{k+l-1}+\\+(a_kb_{l-2}+a_{k-1}b_{l-1}+a_{k-2}b_l)x^{k+l-2}+\dots+a_0b_0=\sum_i^{k+l} c_ix^i,\spc c_i=\sum_{s+t=i}a_sb_t
\end{multline*}

Предположим, что $P(x)Q(x)$ ~--- не примитивный. Тогда все $c_i\dv p$. Пусть $a_i$ ~--- первый, не делящийся на $p$, а $b_j$ ~--- последний, не делящийся на $p$. Тогда рассмотрим коэффициент при $x^{i+j}\colon$ $$c_{i+j}=a_0b_{i+j}+a_1b_{i+j-1} \dots+a_ib_j + \dots a_{i+j}b_0\spc\text{(c поправками, если $i>l$ или $j>k$)} $$
Тогда все слагаемые кроме $a_ib_j$ делятся на $p$, а само это слагаемое не делится на $p$. Значит $c_{i+j}\not div p$. Противоречие.
\end{proof}

\begin{theorem}
Если многочлен $f(x)$ неприводим над $\Kb[x]$, то он неприводим и над $\Kb(x)$.
\end{theorem}
\begin{proof}
Пусть $f(x)$ ~--- примитивный (иначе вынесем наибольший общий делитель его коэффициентов и рассмотрим полученный многочлен; приводимость многочлена над $\Kb(x)$ не изменится, если его умножить или разделить на константу), и он приводим над $\Kb(x)$, но неприводим над $\Kb[x]$ $$f(x)\bw=\underbrace{g(x)\cdot h(x)}_{\text{коэфф. из $\Kb$}}$$
Приведём все коэффициенты в $g(x)$ и $h(x)$ к общему знаменателю, вынесем его и вынесем наибольший общий множитель коэффициентов в числителе.
$$f(x)=\frac{a}{b}\varphi(x)\cdot\frac{c}{d}\psi(x)=\frac{ac}{bd}\varphi(x)\psi(x),\spc\varphi(x),\psi(x)\text{~--- примитивные}$$
$$bd\cdot f(x)=ac\cdot\varphi(x)\psi(x)$$
По лемме Гаусса (\ref{polynom:gauss}) имеем, что $bd=ac,\spc f(x)=\varphi(x)\cdot\psi(x)$. То есть мы получили, что $f(x)$ приводим над $\Kb[x]$. Противоречие.
\end{proof}

\begin{theorem}
  Кольцо многочленов над любым факториальным кольцом факториально.
\end{theorem}
\begin{proof}
  Индукцией по степени многочлена получаем, что требуемое разложение всегда существует. Докажем единственность такого представления. Пусть сущестует два представления многочлена $f$:
  $$f=a_1a_2\dots a_k p_1(x)p_2(x)\dots p_s(x)=b_1b_2\dots b_lq_1(x)q_2(x)\dots q_t(x)$$
  
  В силу факториальности кольца $\Kb$ и того, что произведение примитивных примитивно, имеем тот факт, что константы из $\Kb$ определены в разложении однозначно (с точностью до ассоциированности). Покажем теперь однозначность разложения многочленов. В силу предыдущей теоремы приводимость каждого из многочленов в записи не изменится, если рассматривать многочлены не над кольцом $\Kb$, а над его полем частных $\Fb$. Нам известно, что $\Fb[x]$ -- факториально как кольцо многочленов, над полем, но это означает, что $\exists r_i,v_i\in\Kb\colon p_i(x)=\frac{r_i}{v_i}q_i(x)$. Домножив последнее равенство на $v_ip_i(x)=r_iq_i(x)$. Так как $p_i(x)$ и $q_i(x)$ примитивны, то получаем, что они равны с точностью до ассоциированности уже в данном случае над $\Kb[x]$. Теорема доказана.
\end{proof}

\subsection{Многочлены на полем комплексных чисел}

Кольцо многочленов $\Cbb[x]$ является наиболее важным случаем полей / колец многочленов от одной переменной. Введём следующее понятие:

\begin{df}
  Поле $\Pb$ называется алгебраически замкнутым, если $\forall f\in\Pb[x]$,\linebreak $\deg f>0$, имеет корень в $\Pb$.\index{поле!алгебраически замкнутое}
\end{df}
\begin{theorem}[основная теорема алгебры]
  Поле $\Cbb$ алгебраически замкнуто.\index{теорема!основная теорема алгебры}
\end{theorem}

Доказательство этой теоремы мы проведём несколько позже, а пока будем пользоваться важным следствием: $\forall f\in\Cbb[x]$ представим в в виде $a_n\cdot\prod\limits_{i=1}^s(x-z_i)^k_i$, где $z_i$ -- корни этого многочлена, $k_i$ -- их кратности, а $a_n$ -- старший коэффициент многочлена.

Одной из самых интересных задач для нас пока является исследоваение корней произвольного многочлена $f\in\Cbb[x]$. Первым шагом к этому будет изучение границы для модуля корня многочлена в $\Cbb[x]$. Утверждение состоит в следующем:

\begin{theorem}
  Пусть $f=a_nx^n+a_{n-1}x^{n-1}+\dots+a_1x+a_0$ и $z\colon f(z)=0$. Тогда $|z|<1+A$, где $A=\max\limits_{0\le i\le n-1}\left|\frac{a_i}{a_n}\right|>0$
\end{theorem}
\begin{proof}
  Покажем, что если $|z|\ge1+A$, то $z$ не может быть корнем $f$. Рассмотрим выражение:
  \begin{multline*}  
    \left|\frac{f(z)}{a_n}\right|=\left|z^n+\frac{a_{n-1}}{a_n}z^{n-1}+\dots+\frac{a_0}{a_n}\right|\ge|z^n|-\left|\frac{a_{n-1}}{a_n}z^{n-1}+\dots+\frac{a_0}{a_n}\right|\ge\\ \ge
    |z^n|-A\left(|z|^{n-1}+\dots+1\right)=|z^n|-A\frac{|z|^n-1}{|z|-1}=\frac{|z|^n(|z|-(1+A))}{|z|-1}>0
  \end{multline*}
\end{proof}

Определим операцию комлексного сопряжения для многочленов: если $f\bw=a_nx^n\bw+\dots\bw+a_0$, то $\overline{f}=\overline{a_n}x_n+\dots+\overline{a_0}$. Очевидно, что $\overline{f+g}=\overline{\mathstrut f}+\overline{\mathstrut g}$ и т.д. Верны также увтереждения, что если $z$ -- корень кратности $k$ для $f$, то $\overline{z}$ -- корень кратности $k$ для $\overline{f}$.

Теперь, если у нас есть $f\in\Cbb[x]$, то тот факт, что $f\in\R[x]\subset\Cbb[x]$ будет означать нам не что иное, что $f=\overline{f}$. Обобщая всё вышесказанное имеем следующее утверждение:

\begin{theorem}
  Всякий многочлен $f\in\R[x]$ однозначно представим в виде произведения линейных множителей и квадратных трёхчленов с отрицательным дискриминантом.
\end{theorem}
\begin{proof}
  $f\in\R[x]\lra f=\overline{f}$. Выделив сначала действительные корни многочлена, его комплексные корни мы можем сгруппировать по парам $z$ и $\overline{z}$. Теперь, если раскрыть скобки $(x-z)(x-\overline{z})$, то получим квадратный трёхчлен с действительными коэффициентами. Легко проверить, что его дискриминант отрицателен, если только не $z=\overline{z}$.
\end{proof}

\subsection{Основная теорема алгебры}

\begin{theorem}
  Пусть дано поле $\Pb$ и неприводимый многочлен $f\in\Pb[x]$. Тогда существует расширение поля $\Lb\supset\Pb$, в котором $f$ имеет корень.
\end{theorem}
\begin{proof}
  Имеет смысл рассматривать подстановку в многочлен $f\!=\!a_nx^n\bw+\dots\bw+a_1x+a_0$ над полем $\Pb$ не элемента из поля $\Pb$, а матрицы элементов из поля $\Pb$. В качестве матрицы, которую мы будем подставлять возьмём так называемую сопровождающую матрицу многочлена, изначально приняв $a_n=1$:
$$
A_f=\left(
\begin{matrix}
  & &      & & -a_0\\
 1& &      & & -a_1\\
  &1&      & & -a_2\\
  & &\dots & &     \\
  & &      &1& -a_{n-1}
\end{matrix}
\right)
$$

Рассмотрим строение выражения $B=f(A_f)$. В частности заметим, что
$$A_fe_1=e_2,$$
$$A^2_fe_1 = A_fe_2=e_3,$$
$$\dots$$
$$A^n_fe_1 = A_fe_n=-a_0e_1-a_1e_2-\dots-a_{n-1}e_n.$$

Тогда 
\begin{multline*}f(A_f)e_1 = A_f^ne_1+a_{n-1}A^{n-1}_fe_1+\dots+a_1A_fe_1+a_0e_1=\\=-a_0e_1-a_1e_2-\dots-a_{n-1}e_n+a_{n-1}e_n+\dots+a_1e_2+a_0e_1=0\end{multline*}

Аналогично $f(A_f)e_i=f(A_f)A_f^{i-1}e_1=A_f^{i-1}\cdot f(A_f)e_1=0$. Коммутативность была применена в силу того факта, что в выражении присутствуют степени одной и той же матрицы.\\

Рассмотрим теперь все множество всех матриц $\Ab_f = \{g(A_f)\mid \forall g\in\Pb[x]\}$. Это множество уже образует кольцо; корень многочлена $f$ в этом кольце есть. Осталось показать, что оно образует поле. Заметим, что в качестве многочленов $g$ из $\Pb[x]$ мы можем рассматривать только те, чья степень строго меньше степени $f$. Действительно, если разделить $g$ на $f$ с остатком, имеем $g(A_f)\bw=q(A_f)f(A_f)+r(A_f)=r(A_f),\spc\deg r<\deg f$.

Любой элемент из $\Ab_f$ представим в виде $h=\al_0E+\al_1A_f+\dots+\al_{n-1}A_f^{n-1}$, что соответствует некоторому многочлену из $\Pb[x]$. Покажем, что если $h\ne0$, то $h$ обратим в $\Ab_f$. Так как $f$ неприводим в $\Pb[x]$, то $\exists u,v\colon fu+hv=1$. Подставляя в это равенство $A_f$ получаем $f(A_f)\cdot u(A_f)+h(A_f)\cdot v(A_f)\bw=E$. Так как $f(A_f)=0$ имеем $v(A_f)=h^{-1}(A_f)$.

Доказано, что $\Ab_f$ -- поле, в котором $f$ имеет корень.
\end{proof}

В качестве примера мы можем взять $f=x^2+1\in\R[x]$. Для него $A_f=\left(\begin{smallmatrix}0&-1\\1&0\end{smallmatrix}\right)$. Такая матрица отвечает одному из способов построения поля комплексных чисел $\Cbb$.\\

Очевидно, что мы можем построить поле, в котором любой многочлен будет представим в виде представления линейных множителей, может быть придётся расширять поле не один раз. Доказательство по индукции.

\begin{theorem}[Основная теорема алгебры]
  Любой многочлен $f\in\Cbb[x]$ представим в виде произведения линейных множителей.
\end{theorem}
\begin{proof}
  Рассмотрим сначала вопрос существования корней у произвольного многочлена $f\in\R[x]$. Пусть его степень $\deg f=2^km$, где $m$ -- нечётное. Докажем существование у него хотя бы одного комплексного корня. По индукции: для $k=0$ утверждение верно, так как всякий многочлен нечётной степени с действительными коэфициентами имеет даже хотя бы один действительный корень.
    
    По доказанной выше теореме существует расширение поля $\Lb\supset\Cbb$, в котором $f$ раскладывается на линейные множители. Пусть $x_1,x_2,\dots x_n\in\Lb$ -- его корни. Для любого $t\in\R$ построим многочлен $g(x)$, корнями которого будут являться элементы вида $u_{ij}=x_i+x_j+tx_ix_j$, их число равно $\frac{n(n-1)}2=2^{k-1}m'$. Очевидно, что коэффициенты многочлена $g$ являются элементарными симметрическими выражениями от корней $f$, то есть лежат в поле действительных чисел. По предположению индукции многочлен $g$ имеет комплексный корень.
    
    Это означает, что при любом выборе числа $t$ можно указать такую пару индексов $i$ и $j$, что элемент $x_i+x_j+tx_ix_j$ будет являться комплексным числом. Верно даже более сильное утверждение: $\exists t_1\ne t_2\in\R$, что для одних и тех же индексов $i$ и $j$
    $$
    \left\{
      \begin{array}{lll}
        x_i+x_j+t_1x_ix_j&=&a\\
        x_i+x_j+t_2x_ix_j&=&b
      \end{array}\right.,\spc a,b\in\Cbb
    $$
    
    Легко показать, что из этого следует, что $x_i+x_j$ и $x_ix_j$ по отдельности также принадлежат полю $\Cbb$, а значит элементы $x_i$ и $x_j$ являются корнями квадратного уравнения $\xb^2\bw-(x_i\bw+x_j)\xb\bw+x_ix_j\bw=0$ с комплексными коэффициентами, а значит являются комплексными числами. Утверждение для $f\in\R[x]$ доказано.
    
    Для произвольного многочлена $f\in\Cbb[x]$ рассмотрим следующую конструкцию: $F(x)=f(x)\cdot\overline{f}(x)$, коэффициенты многочлена $F(x)$ $b_k=\sum\limits_{i+j=k}a_i\overline{a_j}$. Заметим, что $\overline{b}_k=b_k$, то есть коэффициенты многочлена $F$ -- действительные числа, а значит $\exists\beta\colon f(\beta)\overline{f}(\beta)=0$. Это означает, что $f$ имеет своим корнем или $\beta$, или $\overline\beta$. Теорема доказана.
\end{proof}

\subsection{Формальная алгебраическая производная}

Пусть $f=a_nx^n+a_{n-1}x^{n-1}\bw+\dots\bw+a_1x\bw+a_0$. Тогда\index{производная}
$$f'\eqdef na_nx^{n-1}+(n-1)a_{n-1}x^{n-2}+\dots+2a_2x+a_1$$

Легко (а может быть и не очень) доказать следующие свойства (доказательства проводятся с помощью обобщения верности этих свойств для одночленов на произволные многочлены):

\begin{enumerate}
  \item $(f+g)'=f'+g'$
  \item $(fg)'=f'g+g'f$
  \item $(f(g(x)))'_x=f'_g(g(x))\cdot g'_x$
\end{enumerate}

Рассмотрим следующее применение производной: пусть $f\in\Pb[x]$, где $\Pb$ -- поле; $f=\prod\limits p_i^{k-i}$, $p_i$ -- неприводимые, и $k_i>0$ -- их кратности. Наше утверждение гласит следующее:
\begin{theorem}\par\strut\\
\begin{enumerate}
  \item Кратность $p$ в $f'$ больше либо равна $k-1$
  \item Если $\Char\Pb=0$, то кратность $p$ в $f'$ в точности равна $k-1$
\end{enumerate}
\end{theorem}
\begin{proof}
  Пусть $f=p^kg,\spc p\nmid g$. Вычислим $f'$ по правилам дифференцирования, описанным выше:
  $$f'=kp^{k-1}p'g+p_kg'=p^{k-1}(kp'g+pg')$$
  
  В случае $\Char\Pb=0\ra p'\ne0$
\end{proof}

Следтвием этого утверждения является тот факт, что в случае поля характеристики 0 $f$ не имеет кратных множителей тогда и только тогда, когда $(f,f')=1$. В случае поля характеристики $p$ это верно только в обратную сторону. Также очевидно, что при дифференцировании кратность корня убывает не более чем на единицу (строго на единицу в случае $\Char\Pb=0$).

И ещё одним следствием нашего утвереждения является тот факт, что в случае поля характеристики 0 можно решить задачу построения многочлена, имеющего такие же корни, но кратности 1, не находя самих корней: если $f\bw=\prod\limits_ip_i^{k_i}\cdot h$, то $f'=\prod\limits_ip_i^{k_i-1}\cdot h$, а значит $g=f/(f,f')$ имеет те же корни, что и $f$, но первой кратности.

\subsection{Теорема Штурма}

Пусть дан многочлен $f\in\R[x]$. Нас будет интересовать задача нахождения таких интервалов на числовой прямой, в которых содержится ровно один корень этого многочлена.

\begin{df}
  \emph{Системой Штурма} для многочлена $f$ на отрезке $[a;b]$ называется последовательность многочленов $f_0, f_1,\dots, f_s$, обладающая следующими свойствами:
  \begin{enumerate}
    \item $f_0$ и $f$ имеют на $[a,b]$ одни и те же корни без учёта кратности. То есть имеем право взять в качестве $f_0$ многочлен $f/(f,f')$. Здесь и далее будем считать, что $f$ не имеет кратных корней, то есть $f=f_0$.
    \item $f_s$ не имеет корней на $[a,b]$
    \item $f_i$ и $f_{i+1}$ не имеют общих корней на $[a,b]$ для $0\le i\le s-1$
    \item Если $f_i(c)=0\spc c\in[a;b]$, то $f_{i-1}(c)f_{i+1}(c)<0$ для $1\le i\le s-1$
    \item Если $f(c)=0\spc c\in[a;b]$, то $f\cdot f_1$ меняет знак с <<-->> на <<+>> при прохождении точки $c$ слева направо.
  \end{enumerate}
\end{df}

Функция $\omega(x)$ будет возвращать нам целое неотрицательное число -- число перемен знаков в последовательности значений многочленов из последовательности Штурма в точке $x$.

\begin{theorem}[Штурма]\par\strut\\
  \begin{enumerate}
    \item Число корней многочлена $f$ в полуинтервале $(a;b]$ (без учёта кратности) равно $\omega(a)-\omega(b)$.
    \item для любого многочлена система Штурма существует, и будет указан алгоритм её нахождения.
  \end{enumerate}\index{теорема!Штурма}
\end{theorem}
\begin{proof}\par\strut\\
  \begin{enumerate}
    \item Будем идти по всем точкам из полуинтервала $(a;b]$ слева направо. Заметим, что если в точке $x$ ни один из $f_i,\spc 0\bw\le i\bw\le s-1$ не имеет корня, то значение $\omega(x)$ не меняется. Теперь, если в точке $x$ какой-то из многочленов $f_i,\spc 1\bw\le i\bw\le s-1$ имеет корень, то, так как $f_{i-1}(x)f_{i+1}<0$, число перемен знаков в последовательности Штурма тоже не меняется.
    
    Если же теперь $f(x)=0$, то так как $f$ и $f_1$ не имеют общих корней и $f\cdot f_1$ меняет знак с <<-->> на <<+>>, то число перемен знаков при прхождении через точку $x$, являющуюся корнем $f$ убывает ровно на 1.
    \item В качестве системы Штурма можно взять следующую последовательность многочленов $f_0,\dots,f_s$, определяемую системой равенств:
    \begin{gather*}
      f_0 = f\\
      f_1 = f_0'\\
      f_0 = g_1f_1 - f_2\\
      \dots\\
      f_{i-1} = g_if_i - f_{i+1}\\
      \dots\\
      f_s = - (f,f')
    \end{gather*}
    
    Легко проверить, что данная последовательность многочленов удовлетворяет свойствам системы Штурма.
  \end{enumerate}
\end{proof}

\subsection{Кольцо многочленов от нескольких переменных}

Пусть $\Kb$ -- коммутативное кольцо с единицей. Кольцом многочленов от $n$ переменных над $\Kb$ называется такое кольцо $\Sb$, что
\begin{enumerate}
  \item $\Kb\subset\Sb$
  \item $\exists x_1,x_2,\dots,x_n\in\Sb$, что любой многочлен $f\in\Sb$ однозначно представим в виде $\sum\limits_i a_{i_1}\dots a_{i_n}x_1^{i_1}\dots x_n^{i_n}$, $a_{ij}\in\Kb$.
\end{enumerate}

Существование такого кольца можно провести индукцией по числу переменных, определяя $\Kb[x_1,\dots,x_n]$ как $\Kb[x_1,\dots,x_{n-1}][x_n]$.

Выражение $x_1^{i_1}\cdot x_2^{i_2}\cdot\dots\cdot x_n^{i_n}$ называется \emph{мономом}. Степенью монома называется число $i_1+i_2+\dots+i_n$. Многочлен $f$ называется однородным степени $m$, если все мономы в его представлении имеют степень $m$. Сумма и произведение однородных многочленов однородны.\index{моном}

На множестве мономов нам необходимо ввести отношение порядка. Логично упорядочить все монономы лексикографически, то есть при сравнения мономов смотреть на показатели степени при переменных $x_1, x_2,\dots, x_n$; моном с более высоким показателем степени при $x_1$ будет считаться старше. Вполне очевидно, что высший член произведения двух многочленов равен произведению их высших членов.

\subsection{Симметрические многочлены. Формулы Виета}

\begin{df}
  Многочлен $f(x_1,x_2,\dots,x_n)$ называется \emph{симметрическим}\index{многочлен!симметрический}, если для любой перестановки $\pi$ выполнено $(\pi\circ f)(x_1,x_2,\dots,x_n)=f(x_{\pi(1)},x_{\pi(2)},\dots,x_{\pi(n)})$. Любая комбинация симметрических многочленов является симметрическим многочленом.
  
  Вводятся \emph{элементарные} симметрические многочлены $$\sigma_i(x_1,x_2,\dots,x_n)=\sum\limits_{1\le k_i\le n}\left(x_{k_1}x_{k_2}\dots x_{k_i}\right).$$
\end{df}

Легко показать, что, если $f(x)=x^n+a_{n-1}x^{n-1}+\dots+a_1x+a_0$ и $x_1,\dots,x_n$ -- его корни, то связь между коэффициентами многочлена и корнями выражается как $a_{n-i}=(-1)^i\cdot \sigma_i(x_1,x_2,\dots x_n)$. Такое выражение называется \emph{формулами Виета}\index{формулы!Виета}. Стоит отметить, что хоть и $x_i$ не обязательно принадлежат тому кольцу, над которым построено кольцо многочленов, в котором лежит многочлен $f$, но значение элементарных симметрических многочленов от этих корней лежит в этом кольце.\\

Оказывается, что наиболее общим способом получения симметрических многочленов является подстановка в многочлен $g\in\Kb[y_1,y_2,\dots,y_n]$ в качестве переменных $y_i$ элементарных симметрических многочленов $\sigma_i\in\Kb[x_1,x_2,\dots,x_n]$. Но ещё более удивителен тот факт, что \dots

\begin{theorem}
  Для любого симметрического многочлена $f\in\Kb[x_1,x_2,\dots,x_n]$ существует и притом единственный многочлен $g(\sigma_1,\sigma_2,\dots,\sigma_n)=f(x_1,x_2,\dots,x_n)$. Коэффициенты многочлена $g$ являются целочисленными линейными комбинациями коэффициентов многочлена $f$.
\end{theorem}
\begin{proof}\par\strut\\
  \begin{enumerate}
    \item Можно считать многочлен $f$ однородным, иначе его можно единственным образом представить в виде суммы однородных многочленов различных степеней.
    \item Высший член любого симметрического многочлена монотонен, то есть в представлении $u=a\cdot x_1^{i_1}x_2^{i_2}\dots x_n^{i_n}$ имеем $i_1\ge i_2\ge\dots\ge i_n$.
    \item Пусть старший член $f$ равен $u=a\cdot x_1^{i_1}x_2^{i_2}\dots x_n^{i_n}$. Рассматривая многочлен $f_1=f - a\cdot\sigma_1^{i_1-i_2}\sigma_2^{i_2-i_3}\dots\sigma_n^{i_n}$, имеем $\deg f_1 = \deg f$, в $f_1$ не входит старший член $f$ и его коэффициенты линейным образом целочисленно выражаются через коэффициенты $f$. Далее проделав точно такую же схему для $f_1$, получаем многочлен $f_2$, для которого получаем многочлен $f_3$ и так далее конечное число раз. Таким образом было получено требуемое разложение $f$ через элементарные симметрические.
    \item Докажем единственность такого представления. В случае существования двух различных многочленов $g_1(\sigma_1, \sigma_2, \dots,\sigma_n)=g_2(\sigma_1, \sigma_2, \dots,\sigma_n)=f$ существовал бы отличный от нуля многочлен $g=g_1-g_2$, для которого $g(\sigma_1,\sigma_2,\dots,\sigma_n)=0$. Для каждого из одночленов в представлении $g$ при подстановке в него элементарных симметрических получаются разные старшие члены от $x_1,x_2,\dots,x_n$, то есть среди них обязательно имеется самый высший и $g(\sigma_1,\sigma_2,\dots,\sigma_n)\ne0$.
  \end{enumerate}
\end{proof}

\subsection{Результант пары многочленов}

Пусть дано два многочлена $f,g\in\Kb[x]$
\begin{gather*}
  f(x) = a_nx^n+a_{n-1}x^{n-1}+\dots+a_1x+a_0,\spc a_n = 0\\
  g(x) = b_mx^m+b_{m-1}x^{m-1}+\dots+b_1x+b_0,\spc b_m = 0
\end{gather*}

За определение результанта возьмём выражение
$$
\res(f,g) = \left|\begin{matrix}
a_n & a_{n-1} &         & \dots & a_0   &     &        &     \\
    & a_n     & a_{n-1} & \dots &       & a_0 &        &     \\
    & \dots   & \ddots  & \dots & \dots &     & \ddots &     \\
    &         &         & a_n   &       &     & &      & a_0 \\
b_m & b_{m-1} &         & \dots & b_0   &     &        &     \\
    & b_m     & b_{m-1} & \dots &       & b_0 &        &     \\
    & \dots   & \ddots  & \dots & \dots &     & \ddots &     \\
    &         &         & b_m   &       &     & &      & b_0
\end{matrix}\right|
$$

Свойства:

\begin{enumerate}
  \item $\res(f,g)=0 \lra \deg(f,g)>0$ (имеются общие корни)
    \begin{proof}
      Пусть $\res(f,g)=0$. Тогда существует нетривиальная линейная комбинация строк в определителе матрицы из коэффициентов многочленов. За $\al_1,\dots,\al_{m+n}$ обозначим строки этого определителя, имеем $c_1\al_1\bw+\dots\bw+c_{m+n}\al_{m+n}=0$. Умножая такую строку на столбец $(x^{m+n-1} // \dots // 1)$ получаем выражение $f(x)(c_1x^{m-1}+\dots+c_m)+g(x)(c_{m+1}x^{n-1}+\dots+c_{m+n})=0$ или $fu+gv=0,\spc\deg u<m,\,\deg v<n$. Очевидно, что $u,v\ne0$. Если бы $(f,g)=1$, то т.к. $f\mid gv$, то $f\mid v$. Противоречие с тем, что $\deg v<\deg f$.\\
      
      Пусть $(f,g)=h,\spc \deg h>0$. Тогда $f=hf_1$, $g=-hg_1$; отсюда $fg_1+gf_1=0$, значит существует набор коэффициентов, при котором линейная комбинация строк данного нам определителя равна нулю, а значит и $\res(f,g)=0$.
    \end{proof}
  \item $\res(f,g)=a_0^m\prod\limits_{i=1}^n g(\alpha_i)=(-1)^{mn}b_0^n\prod\limits_{j=1}^m f(\beta_j)=a_0^mb_0^n\prod\limits_{i,j}(\alpha_i-\beta_j)$.
    \begin{proof}
      Рассмотрим $\res(f,g-y)=(-1)^na_0^my^n+\dots+\res(f,g)$, где $y$ -- некоторая новая переменная. Рассмотретим это выражение как многочлен степени $n$ от $y$. Придадим $y$ значение $g(\alpha_i)$. Многочлены $f(x)$ и $g(x)-g(\alpha_i)$ имеют общий корень, а значит делятся на $x-\alpha_i$, или $\res(f,g-g(\alpha_i))=0$. В таком случае многочлен $\res(f,g-y)$ должен делиться на все $g(\alpha_i)-y$, или $\res(f,g-y)=a_0^m\prod\limits_{i=1}(g(\alpha_i)-y)$. При $y=0$ получаем требуемое равенство.
    \end{proof}
  \item Результант применим для решения систем полиномиальных уравнений вида
    $$
      \left\{\begin{array}{lcl}
        F(x,y) &=& 0\\
        G(x,y) &=& 0
      \end{array}\right.
    $$
\end{enumerate}

\subsection{Дискриминант многочлена}

Рассмотрим многочлен $f(t)=a_nt^n+a_{n-1}t^{n-1}+\dots+a_1t+a_0$. Он имеет ровно $n$ корней в некотором расширении кольца, из которого взяты его коэффициенты. Дискриминантом\index{дискриминант} этого многочлена по определению называется элемент $D(f)\eqdef\prod\limits_{1\le j<i\le n}(x_i-x_j)^2$, лежащий в том же кольце, что и коэффициенты $f$, так как является симметрической функцией от корней многочлена. Нетрудно заметить, что значение дискриминанта равно квадрату значения определителя Вандермонда, составленного из корней многочлена, в свою очередь равного определителю, составленному из степенных сумм $s_0, s_1,\dots s_{2n-2}$:

$$
 D = 
 \left(
   \begin{matrix}
     s_0 & s_1 & \dots & s_{n-1}\\
     s_1 & s_2 & \dots & s_{n}  \\
     \vdots & \vdots & \ddots & \vdots\\
     s_{n-1} & s_{n} & \dots & s_{2n-2}
   \end{matrix}
 \right)
$$

Как известно, степенные формулы выражаются через элементарные симметрические по формулам Ньютона\index{формулы!Ньютона}: для $s^k = x_1^k+x_2^k+\dots+x_n^k$ определено рекуррентное соотношение 
$$
 s_k - \sigma_1s_{k-1}+\sigma_2s_{k-2}-\dots+(-1)^i\sigma_is_{k-i}+\dots+(-1)^{k-1}\sigma_{k-1}s_1+(-1)^kk\sigma_k=0
$$

Получим ещё одно выражение дискриминанта. Рассмотрим $\res(f,f')$. Вспомним, что если $f=a_n(t-x_1)\dots(t-x_n)$, то $f'=a_n\sum\limits_{i=1}^n(t-x_1)\dots\widehat{(t-x_i)}\dots(t-x_n)$; очевидно, что $f'(x_i)=a_n(t-x_1)\dots\widehat{(t-x_i)}\dots(t-x_n)$. Тогда\index{результант}
\begin{multline*}
  \res(f,f')=a_n^{n-1}\prod_{i=1}^nf'(x_i)=a_n^{2n-1}\prod_{i=1}^n(t-x_1)\dots\widehat{(t-x_i)}\dots(t-x_n)=\\=a_n^{2n-1}\cdot(-1)^{\frac{n(n-1)}2}\cdot\prod_{1\le j<i\le n}(x_i-x_j)^2 = a^{2n-1}_n\cdot(-1)^{\frac{n(n-1)}2}\cdot D(f)
\end{multline*}

\subsection{Поле рациональных дробей}

Руководствуясь теоремой \ref{group:pole}, вложим кольцо многочленов от одной переменной в поле. В этом поле определим некоторые понятия:\index{поле!рациональных дробей}

\begin{df}
Дробь $\frac{f}{g}$ называется правильной, если $\deg f<\deg g$\index{дробь!правильная}
\end{df}

\begin{theorem}
Всякую неправильную дробь можно представить в виде <<многочлен + правильная дробь>>\index{многочлен}
\end{theorem}

\begin{df}
 Простейшая дробь ~-- дробь вида $\frac{f}{p^m}$, где $p$ ~--- неприводимый и $\deg f\bw<\deg p$.\index{дробь!простейшая}
\end{df}

\begin{theorem}
Всякую правильную дробь можно представить в виде суммы простейших.
\end{theorem}
\begin{proof}
  Доказательство, разумеется, проведём по индукции. Индукцию проведём по степени $g$. Для $\deg g=1$ верно. Пусть верно для всех правильных робей, степень знаменателя которых меньше $n$. Возможно два случая:
  \begin{enumerate}
    \item $g$ раскладывается в произведение двух взаимнопростых многочленов. Тогда по следствию из алгоритма Евклида и исходя из индуктивного предположения получаем, что требуемое представление существует.
    \item $g=p^k$. Всё равно разделим $f$ на $p$ с остатком, получим, что $f = qp+r$ или $\frac{f}g=\frac{q}{p^{k-1}}+\frac{r}{p^k}, \deg r<\deg p$. По индуктивному предположению опять получаем, что требуемое представление существует.
  \end{enumerate}
\end{proof}
