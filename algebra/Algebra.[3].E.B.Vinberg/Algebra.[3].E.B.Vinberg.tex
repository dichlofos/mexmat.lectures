\documentclass[a4paper]{article}
\usepackage[utf8]{inputenc}
\usepackage[russian]{babel}
\usepackage{dmvn}
\usepackage{amsmath}
\usepackage{xypic}

\newcounter{lec}
\renewcommand{\thelec}{\Roman{lec}}
\newcommand*{\lecture}[1]{\refstepcounter{lec}\vspace{20pt}
\begin{center}{\rmfamily\textsc{Лекция \thelec. \\ \textbf{#1}}}\vspace{5pt}
\end{center}}

\newcommand*{\tema}[1]{\vspace{20pt}
\begin{center}{\textbf{\textsc{#1.}}}\vspace{5pt}
\end{center}}

\newcommand{\svoy}{\vspace{5pt}\noindent\textbf{Свойства.}\vspace{-6pt}}

%% A lot of legacy code
\renewcommand{\Isom}{\mathop{\mathrm{Isom}}\nolimits}
\newcommand{\Sym}{\mathop{\mathrm{Sym}}\nolimits}
\renewcommand{\Aut}{\mathop{\mathrm{Aut}}\nolimits}

\renewcommand{\Tor}{\mathop{\mathrm{Tor}}\nolimits}
\newcommand{\Mor}{\mathop{\mathrm{Mor}}\nolimits}
\renewcommand{\rk}{\mathop{\mathrm{rk}}\nolimits}
\renewcommand{\diag}{\mathop{\mathrm{diag}}\nolimits}
\renewcommand{\Im}{\mathop{\mathrm{Im}}\nolimits}

\newcommand*{\p}[1]{#1\nobreak\discretionary{}{\hbox{$\mathsurround=0pt #1$}}{}}


\begin{document}
%----------------------Lecture 1---------------------------%
\lecture{Определение и примеры групп.}

\textit{Группой} называется множество $G$ с операцией умножения,
удовлетворяющей условиям

1) $(ab)c=a(bc)$\quad (\textit{ассоциативность})

2) $\exists \, e \;\mathrm{(\textit{единица})}: ae=ea=a \quad
\forall \, a\in G$

3) $\forall a\in G \quad\exists \, a^{-1}\in G\;
\mathrm{(\textit{обратный элемент})}: aa^{-1}=a^{-1}a=e$.

Группа $G$ называется \textit{коммутативной} (или
\textit{абелевой}), если $ab\p=ba \quad \forall a,b\in G$.

\emph{Аддитивной группой} называется множество $G$ с операцией
сложения, удовлетворяющей условиям

1) $(a+b)+c=a+(b+c)$\quad (\textit{ассоциативность})

2) $\exists \, 0 \;\mathrm{(\textit{нуль})}: a+0=0+a=a \quad \forall
a\in G$

3) $\forall  a\in G \;\exists \, (-a)\in G\;
\mathrm{(\textit{противоположный элемент})}: a+(-a)\p=(-a)+a=0$.

Обычно аддитивная группа предполагается абелевой: $a+b=b\p+a\quad
\forall a,b\in G$.

Подмножество $H$ группы $G$ называется \emph{подгруппой}, если

1) $ab\in H \quad \forall a,b\in H$

2) $a^{-1}\in H\quad \forall a\in H$

3) $e\in H$.

Подгруппа сама является группой относительно той же операции.

Отображение $f\colon G\to H$ называется \emph{изоморфизмом} группы
$G$ на группу $H$, если

1) $f$ биективно

2) $f(ab)=f(a)f(b)\quad \forall a,b\in G$.

Свойства изоморфизма: $f(e)=e$, $f(a^{-1})=f(a)^{-1}$.

\begin{ex}
\begin{enumerate}
  \item $\mathbb{Z}$ (по сложению)~--- абелева группа. По определению,
  всякое кольцо является абелевой группой по сложению.
  \item $\mathbb{R}^+=\mathbb{R}\setminus \{0\}$~--- абелева группа по
  умножению. По определению, совокупность ненулевых элементов любого
  поля $K$ является абелевой группой по умножению и обозначается
  через $K^*$.
  \item $\mathbb{T}=\{z\in\mathbb{C}: |z|=1\}$~--- подгруппа в
  $\mathbb{C}^*$.
  \item $C_n=\{z\in\mathbb{C}: z^n=1\}$~--- подгруппа в
  $\mathbb{T}$.
  \item Векторы плоскости (или пространства) образуют абелеву группу
  относительно сложения. П определению, всякое векторное
  пространство является абелевой группой по сложению.
  \item $S(X)$~--- группа преобразований (биективных отображений в
  себя) множества $X$ (единица~--- $\mathrm{id}_X$). В частности, $S(\{1,2,\ldots,
  n\})=S_n$~--- симметрическая группа подстановок степени $n$.
  Всякая подгруппа группы $S(X)$ называется \emph{группой
  преобразований} множества $X$.
  \item $\Isom \mathbb{E}^2$ ($\Isom\mathbb{E}^3$)~--- группа движений
  евклидовой плоскости (пространства). $\Isom_+ \mathbb{E}^2$
  ($\Isom_+\mathbb{E}^3$)~--- подгруппа собственных (сохраняющих ориентацию)
  движений.
  \item Группа симметрий правильных многоугольников
  (или многогранников): $P\subset \mathbb{E}^2$ ($\mathbb{E}^3$). $\Sym P=
  \{g\in\Isom \mathbb{E}^2 \: (\mathbb{E}^3): gP=P\}$. $D_n$~---
  группа симметрий правильного $n$-угольника (\emph{группа диэдра}).
  $|D_n|=2n$. $C_n$~--- это группа вращений правильного
  $n$-угольника. $|C_n|=n$.
  \item Кристаллографические группы (группы симметрий кристаллических
  структур).
  \item $\mathrm{\mathrm{GL}}(V)$~--- группа невырожденных линейных преобразований
  $n$-мерного векторного пространства над полем $K$. $\mathrm{\mathrm{GL}}(V)\simeq
  \mathrm{\mathrm{GL}}_n(K)$~--- группа невырожденных матриц $n\times n$ над полем
  $K$.
  \item $\mathrm{O}(V)$~--- группа преобразований евклидова векторного
  пространства $V$. $\mathrm{O}(V)\simeq \mathrm{O}_n$~--- группа ортогональных
  матриц. Отказываясь от требования положительной определенности
  скалярного умножения (но предполагая невырожденность), получаем
  группу псевдоортогональных преобразований, изоморфную $\mathrm{O}_{p,q}$ (группа
  псевдоортогональных матриц), где $(p,q)$~--- \emph{сигнатура}
  скалярного умножения (т.е. число плюсов и минусов). В частности,
  $\mathrm{O}_{3,1}$~--- группа Лоренца.
  \item $\mathrm{\mathrm{GL}}_n(\mathbb{Z})$~--- группа обратимых целочисленных
  матриц.
  \item $K$~--- поле, $f\in K[x]$~--- неприводимый многочлен степени
  $n$; $K\subset L$~--- поле разложения многочлена $f$. $\mathrm{Gal}\: L/K=
  \{\varphi\in \Aut L: \varphi|_K\p=\mathrm{id}\}$~--- группа Галуа поля $L$
  над $K$. $\mathrm{Gal}\: L/K\subset S_n$. Например, $\mathrm{Gal}\: \mathbb{C}/\mathbb{R}
  \simeq C_2$.
\end{enumerate}
\end{ex}
\tema{Циклические группы}

\emph{Степень} элемента: $g^n=\underbrace{g\cdot\ldots\cdot g}_{n}$
при $n>0$, $e$ при $n=0$ и $\underbrace{g^{-1}\cdot\ldots\cdot
g^{-1}}_{n}$ при $n<0$. $g^m\cdot g^n=g^{m+n}$, $(g^n)^{-1}=g^{-n}$.

$\langle g\rangle=\{g^n: n\in \mathbb{Z}\}$~--- \emph{циклическая
группа, порожденная элементом $g$}. Если $\exists \, g\in G:
G=\langle g\rangle$, то $G$ называется \emph{циклической группой}.

Либо все $g^n$ различны, либо нет. Во втором случае они циклически
повторяются с некоторым периодом $m=\ord g$ (\emph{порядок элемента
$g$}). $\ord g=\min\limits_{g^n=e}\{n\p>0\}$.

\begin{theorem}
1. Все бесконечные циклические группы изоморфны $\mathbb{Z}$.

2. Всякие конечные циклические группы порядка $n$ изоморфны $C_n$.
$\square$
\end{theorem}
%----------------------------------------------------------%
%%%%%%%%%%%%%%%%%%%%%%%%%%%%%%%%%%%%%%%%%%%%%%%%%%%%%%%%%%%%
%----------------------Lecture 2---------------------------%
\lecture{Факторгруппа.}

\emph{Отношение} на множестве $X$~--- это подмножество
$\mathcal{R}\in X\times X$. Если $(x,y)\in \mathcal{R}$, то говорят,
что $x$ и $y$ \emph{находятся в отношении $\mathcal{R}$} и пишут
$x\mathcal{R}y$. Отношение $\mathcal{R}$ называется \emph{отношением
эквивалентности}, если

1) $x\mathcal{R}x\quad \forall x\in X$ \quad (\emph{рефлексивность})

2) $x\mathcal{R}y\Rightarrow y\mathcal{R}x$\quad
(\emph{симметричность})

3) $x\mathcal{R}y\; \& \;y\mathcal{R}z\Rightarrow
x\mathcal{R}z$\quad (\emph{транзитивность}).

Обычно пишут $x\mathop{\sim}\limits_{\mathcal{R}}y$, или просто
$x\sim y$.

\emph{Класс эквивалентности, содержащий $x$}:
$\mathcal{R}(x)=[x]=\{y\in X: x\p\sim y\}$. Классы эквивалентности
задают разбиение множества $X$. Множество классов эквивалентности
называется \emph{фактормножеством} множества $X$ по отношению
эквивалентности $\mathcal{R}$ и обозначается $X/\mathcal{R}$.

Есть естественное отображение $\pi\colon X\to X/\mathcal{R}$,
$x\mapsto \mathcal{R}(x)$. Оно называется \emph{отображением
факторизации}.

Для любого отображения $f\colon X\to Y$ определяется отношение
эквивалентности $\mathcal{R}_f$ на $X$: $x_1\sim x_2$, если
$f(x_1)=f(x_2)$. Получается следующая диаграмма:
$$\xymatrix{
 X \ar[rr]^f \ar[dr]_\pi && Y\\
&X/\mathcal{R}_f \ar[ur]_{\bar{f}} }$$ По определению,
$\bar{f}([x])=f(x)$. $f=\bar{f}\circ \pi$, $\pi$ сюръективен,
$\bar{f}$ инъективен. Это разложение называется \emph{факторизацией
$f$}.

Пусть $(X, \circ)$~--- множество с операцией. Отношение
эквивалентности $\mathcal{R}$ на множестве $X$ называется
\emph{согласованным с операцией $\circ$}, если $x\sim x'$, $y\sim
y'$ $\Rightarrow$ $x\circ y\sim x'\circ y'$. Тогда на
фактормножестве $X/\mathcal{R}$ можно ввести операцию $\circ$:
$[x]\circ [y]=[x\circ y]$. Это определение корректно. Из определения
следует, что отображение факторизации является гомоморфизмом, т.е.
$\pi(x\circ y)=\pi(x)\circ\pi(y)$.

Пусть $(X,\circ)$ и $(Y, \ast)$~--- два множества с операциями,
$f\colon X\to Y$~--- \emph{гомоморфизм}, т.е. $f(x_1\circ
x_2)=f(x_1)\ast f(x_2)\quad \forall x_1,x_2\in X$. Имеем:
$f=\bar{f}\pi$, где $\pi\colon X\to X/\mathcal{R}_f$,
$\bar{f}([x])=f(x)$.

\begin{theorem}
$\bar{f}$~--- гомоморфизм, а если $f$ сюръективен, то $\bar{f}$~---
изоморфизм $X/\mathcal{R}_f$ на $Y$.
\end{theorem}

\begin{proof}
$\bar{f}([x_1]\circ[x_2])=\bar{f}([x_1\circ x_2])=f(x_1\circ
x_2)=f(x_1)\ast f(x_2)\p=\bar{f}([x_1])\ast \bar{f}([x_2])$.
\end{proof}

Пусть $G$~--- группа, $H\subset G$~--- подгруппа. \emph{Отношение
сравнимости по модулю $H$}: $g_1\equiv g_2\pmod{H}$, если
$g_1^{-1}g_2\in H$. Это отношение является отношением
эквивалентности. Классы эквивалентности имеют вид $[g]\p=gH=\{gh:
h\in H\}$ и называются \emph{левыми смежными классами группы $G$ по
$H$}.

Аналогично, можно определить $g_1\equiv g_2\pmod{H}$, если
$g_2g_1^{-1}\in H$. Тогда классы эквивалентности имеют вид $[g]=Hg$
и называются \emph{правыми смежными классами группы $G$ по $H$}.

\emph{Инверсия} (взятие обратного элемента) в группе $G$
осуществляет биекцию между множествами левых и правых смежных
классов: $(gH)^{-1}\p=Hg^{-1}$. Количество левых и правых смежных
классов одинаково и обозначается $|G:H|$.

\begin{theorem}[Лагранж]
Если $|G|<\infty$, то $|G|=|G:H|\cdot |H|$. $\square$
\end{theorem}

Подгруппа $H\subset G$ называется \emph{нормальной}, если $\forall
g\in G\quad gH=Hg$ ($\Leftrightarrow gHg^{-1}=H$). Обозначение:
$H\triangleleft G$.

\begin{ex}
\begin{enumerate}
  \item $S_{n-1}\ntriangleleft S_n$ при $n\geqslant 3$.
  \item Если $G$ абелева, то всякая ее подгруппа нормальна.
\end{enumerate}
\end{ex}

\begin{theorem}
Отношение сравнимости по модулю подгруппы $H$ согласовано с
операцией в $G$ тогда и только тогда. когда $H$ нормальна.
\end{theorem}

\begin{proof}
Пусть $H\triangleleft G$, $g_1\equiv g_1'\pmod{H}$, $g_2\equiv
g_2'\pmod{H}$. Тогда $g_1'=g_1h_1$, $g_2'=g_2h_2$ (где $h_1,h_2\in
H$) $\Rightarrow$
$g_1'g_2'=g_1(h_1g_2)h_2=g_1(g_2h_1')h_2\p=(g_1g_2)(h_1'h_2)\equiv
g_1g_2\pmod{H}$.

Обратно, пусть отношение сравнимости согласовано с операцией. Тогда
$\forall g\in G,\: h\in H\quad ghg^{-1}\equiv geg^{-1}\equiv
e\pmod{H}$ $\Rightarrow$ $ghg^{-1}\in H$, т.е. $gHg^{-1}\subset H$.
Аналогично, $g^{-1}Hg\subset H$. Но тогда $H\subset gHg^{-1}$
$\Rightarrow$ $gHg^{-1}\p=H$.
\end{proof}

\begin{theorem}
Всякое отношение эквивалентности в $G$, согласованное с операцией,
есть отношение сравнимости по модулю некоторой подгруппы.
\end{theorem}

\begin{proof}
Рассмотрим $H=[e]=\{h\in G: e\sim h\}$. Докажем, что $H$~---
подгруппа: $h_1,h_2\sim e$ $\Rightarrow$ $h_1h_2\sim e$; $h\sim e$
$\Rightarrow$ $e=hh^{-1}\sim e h^{-1}=h^{-1}$; $e\sim e$. Тогда
$g_1\sim g_2$ $\Leftrightarrow$ $e\sim g_1^{-1}g_2$
$\Leftrightarrow$ $g_1^{-1}g_2\in H$ $\Leftrightarrow$ $g_1\equiv
g_2\pmod{H}$.
\end{proof}

Т.о., если $N\triangleleft G$, то на множестве классов сравнимости
по модулю $N$ определяется операция: $(g_1N)(g_2N)=(g_1g_2)N$.
Множество классов сопряженности обозначается $G/N$  и относительно
такой операции оно является группой. Она называется
\emph{факторгруппой группы $G$ по $N$}.

Отображение факторизации $\pi\colon G\to G/N$, $g\mapsto gN$
является гомоморфизмом. Обратно, пусть $f\colon G\to H$~---
гомоморфизм групп. Тогда соответствующее отношение эквивалентности
$\mathcal{R}_f$ согласовано с операцией. Значит, это отношение
сравнимости по модулю нормальной подгруппы $N=\{g\in G: f(g)=e\}$.
Эта подгруппа называется \emph{ядром} и обозначается как $N=\ker f$.

Имеет место следующая диаграмма:
$$\xymatrix{
 G \ar[rr]^f \ar[dr]_\pi && H\\
&G/N \ar[ur]_{\bar{f}} }$$

\begin{theorem}
Если $f\colon G\to H$~--- гомоморфизм групп, то $\bar{f}\colon
G/N\to H$~--- гомоморфизм. Если $f$ сюръективен, то $\bar{f}$~---
изоморфизм. $\square$
\end{theorem}

\note{В общем случае $G/\ker f\simeq \Im f$}.

\begin{ex}
\begin{enumerate}
  \item $\sgn\colon S_n\to \{\pm1\}$. $\ker \sgn=A_n$. Т.о., $S_n/A_n\simeq
  \{\pm1\}$.
  \item $\det\colon \mathrm{GL}_n(K)\to K^*$. $\ker \det = \mathrm{SL}_n(K)$.
  Т.о., $\mathrm{GL}_n(K)/SK_n(K)\simeq K^*$.
  \item $f\colon \mathbb{C}^*\to\mathbb{C}^*$, $z\mapsto z^n$. $\ker
  f=C_n$ $\Rightarrow$ $\mathbb{C}^*/C_n\simeq \mathbb{C}^*$.
  \item $f\colon \mathbb{C}^*\to\mathbb{R}^*_+$, $z\mapsto |z|$. $\ker
  f=\mathbb{T}$ $\Rightarrow$ $\mathbb{C}^*/\mathbb{T}\simeq
  \mathbb{R}^*_+$.
  \item $\exp\colon \mathbb{R}\to\mathbb{R}^*$, $x\mapsto e^x$. $\ker
  \exp=\{0\}$, $\Im \exp=\mathbb{R}^*_+$ $\Rightarrow$ $\mathbb{R}\simeq
  \mathbb{R}^*_+$.

  $\exp\colon \mathbb{C}\to\mathbb{C}^*$, $z\mapsto e^z$. $\ker \exp=2\pi
  i\mathbb{Z}$, $\Im \exp=\mathbb{C}^*$ $\Rightarrow$ $\mathbb{C}/2\pi i\mathbb{Z}\simeq
  \mathbb{C}^*$ (также $\mathbb{C}/\mathbb{Z}\simeq \mathbb{C^*}$).
  \item Линейное отображение $f\colon K^n\to K^m$,
  $(x_1,\ldots,x_n)\mapsto (y_1,\ldots,y_n)$,
  $y_i=\sum\limits_{j=1}^na_{ij}x_j$, $i=1,\ldots,m$. $\ker f$~---
  множество решений системы однородных линейных уравнений
  $\sum\limits_{j=1}^na_{ij}x_j=0$. $b=(b_1,\ldots,b_m)\in K$
  $\Rightarrow$ $f^{-1}(b)$ либо пуст, либо класс сопряженности $\pmod{\ker
  f}$, т.е. $f^{-1}(b)=x_0+\ker f$. С другой стороны, $f^{-1}(b)$
  есть множество решений системы $\sum\limits_{j=1}^na_{ij}x_j=b_i$
  ($i=1,\ldots,m$).
  \item $f\colon S_4\to S_3$. $y_1=x_1x_2+x_3x_4$,
  $y_2=x_1x_3+x_2x_4$, $y_3=x_1x_4\p+x_2x_3$. $\ker f=\{e, (12)(34), (13)(24),
  (14)(23)\}=V_4$ (\emph{четверная группа Клейна}) $\Rightarrow$ $S_4/V_4\simeq
  S_3$.
\end{enumerate}
\end{ex}
%----------------------------------------------------------%
%%%%%%%%%%%%%%%%%%%%%%%%%%%%%%%%%%%%%%%%%%%%%%%%%%%%%%%%%%%%
%----------------------Lecture 3---------------------------%
\lecture{Прямые произведения групп.}

Говорят, что группа $G$ \emph{разлагается в прямое произведение
своих подгрупп $G_1,\ldots,G_k$}, если

1) каждый элемент $g\in G$ единственным образом представляется в
виде $g=g_1\ldots g_k$, где $g_i\in G_i$

2) при $i\neq j$ $g_i g_j=g_jg_i\quad \forall g_i\in G_i, g_j\in
G_j$.

Правило умножения: $(g_1\ldots g_k)(g_1'\ldots
g_k')=(g_1g_1')\ldots(g_kg_k')$.

Обозначение: $G=G_1\times\ldots\times G_k$.

\svoy
\begin{enumerate}
  \item $G_i\cap G_j=\{e\}$ при $i\neq j$: $g\in G_i\cap G_j$
  $\Rightarrow$ $g=e\ldots \stackrel{i}{g}\ldots e=e\ldots \stackrel{j}{g}\ldots
  e$ $\Rightarrow$ $g=e$.
  \item $G_i\triangleleft G$: $h\in G_i$, $g_1,\ldots,g_k\in G$, $g=g_1\ldots
  g_k$ $\Rightarrow$ $$ghg^{-1}=(g_1eg_1^{-1})\ldots(g_ihg_i^{-1})\ldots(g_keg_k^{-1})=g_ihg_i^{-1}\in
  G_i.$$
\end{enumerate}

\begin{lemma}
\label{1.III}Если $G', G''\in G$, $G',G''\triangleleft G$ и $G'\cap
G''=\{e\}$, то $g'g''\p=g''g'\quad \forall g'\in G', g''\in G''$.
\end{lemma}

\begin{proof}
$g'g''g'^{-1}g''^{-1}\in G'\cap G''=\{e\}$ $\Rightarrow$
$g'g''g'^{-1}g''^{-1}=e$.
\end{proof}

\begin{theorem}
Пусть $G_1,G_2\triangleleft G$, $G_1\cap G_2=\{e\}$, $G_1G_2=G$.
Тогда $G\p=G_1\times G_2$.
\end{theorem}

\begin{proof}
1) Любой элемент $g\in G$ представляется в виде $g\p=g_1g_2$ (где
$g_1\in G_1, g_2\in G_2$) по условию теоремы. Докажем, что такое
представление единственно. $g\p=g_1g_2=g_1'g_2'$ $\Rightarrow$
$G_1\ni g_1'^{-1}g_1=g_2'g_2^{-1}\in G_2$ $\Rightarrow$
$g_1'^{-1}g_1=g_2'g_2^{-1}=e$ $\Rightarrow$ $g_1=g_1'$, $g_2=g_2'$.

2) $\forall g_1\in G_1, g_2\in G_2\quad g_1g_2=g_2g_1$ по
лемме~\ref{1.III}.
\end{proof}

\begin{ex}
\begin{enumerate}
  \item Разложение векторного пространства в прямую сумму
  подпространств: $V=V_1\oplus \ldots\oplus V_k$ есть разложение
  аддитивной группы в прямую сумму подгрупп.
  \item $\mathbb{C}^*=\mathbb{R}^*_+\times \mathbb{T}$, т.е.
  $z=r(\cos\varphi+i\sin\varphi)$.
  \item $\mathrm{GL}_n^+(\mathbb{R})$~--- группа вещественных матриц с
  положительным определителем; $\mathrm{GL}_n^+(\mathbb{R})=\mathrm{SL}_n(\mathbb{R})\times\{\lambda
  E\}$, где $\lambda\in \mathbb{R}_+$.
\end{enumerate}
\end{ex}

Пусть $G_1,\ldots,G_k$~--- произвольные группы. \emph{Внешним
произведением групп $G_1,\ldots,G_k$} называется прямое произведение
множеств $G_1,\ldots,G_k$ с операцией умножения
$(g_1,\ldots,g_k)(g_1',\ldots,g_k')=(g_1g_1',\ldots,g_kg_k')$. Это
также группа, обозначаемая $G_1\times \ldots\times G_k$. Если группа
$G$ разлагается в прямое произведение подгрупп $G_1,\ldots,G_k$, то
$G\simeq G_1\times\ldots \times G_k$ (внешнее прямое произведение):
$g=g_1\cdot\ldots\cdot g_k\leftrightarrow (g_1,\ldots,g_k)$.

\begin{ex}
\begin{enumerate}
  \item $K^*\times\ldots\times K^*=(K^*)^n$ изоморфна группе невырожденных
диагональных матриц порядка $n$.
\end{enumerate}
\end{ex}

\tema{Абелевы группы\footnote{В этой теме все рассматриваемые
подгруппы считаются аддитивными, если не оговорено противное.}}

Пусть $A$~--- абелева группа. $\forall a\in A$, $\forall k\in
\mathbb{Z}$ определен элемент $ka\in A$.

\svoy
\begin{enumerate}
  \item $k(a+b)=ka+kb$
  \item $(k+l)a=ka+la$
  \item $(kl)a=k(la)$
  \item $1\cdot a=a$.
\end{enumerate}

\emph{Линейная комбинация} элементов $a_1,\ldots,a_n\in A$ есть
элемент $k_1a_1\p+\p\ldots+k_na_n$ ($k_1,\ldots,k_n\in \mathbb{Z}$).
Совокупность всех линейных комбинаций элементов $a_1,\ldots,a_n$
есть наименьшая подгруппа, содержащая $a_1,\ldots,a_n$. Она
называется подгруппой, \emph{порожденной} $a_1,\ldots,a_n$ и
обозначается как $\langle a_1,\p\ldots,a_n\rangle$. Если $\langle
a_1,\ldots,a_n\rangle=A$, то говорят, что группа $A$
\emph{порождается} элементами $a_1,\ldots,a_n$. Группа, порожденная
конечным числом элементов, называется \emph{конечно-порожденной}. В
частности, группа, порожденная одним элементом,~--- это циклическая
группа.

Элементы $a_1,\ldots,a_n$ называются \emph{линейно зависимыми}, если
существуют числа $k_1,\ldots,k_n\in \mathbb{Z}$, не все равные 0,
такие, что $k_1a_1+\ldots+k_na_n\p=0$. В противном случае
$a_1,\ldots,a_n$ называются \emph{линейно независимыми}. Линейно
независимая система элементов, порождающих группу $A$, называется
\emph{базисом группы $A$}. Не всякая конечно порожденная абелева
группа обладает базисом, например, группа $\mathbb{Z}_m$ не обладает
базисом.

Конечно порожденная абелева группа, обладающая базисом, называется
\emph{свободной}. Если $\{e_1,\ldots,e_n\}$~--- базис группы $A$, то
$$A=\langle e_1\rangle\oplus\ldots\oplus\langle e_n\rangle\simeq
\mathbb{Z}\oplus\ldots\oplus\mathbb{Z}=\mathbb{Z}^n.$$

\begin{theorem}
Все базисы свободной абелевой группы равномощны.
\end{theorem}

\begin{proof}
Пусть $\hc{\enumun e}$ и $\hc{e'_1,\ldots,e'_m}$~--- два базиса.
Предположим, что $m>n$. Имеем
$$(e'_1,\ldots,e'_m)=(e_1,\ldots,e_n)\mathop{C}\limits_{n\times m}$$
Можно рассматривать $C$ как матрицу над $\mathbb{Q}$. Т.к. $m>n$, то
столбцы линейно зависимы над $\mathbb{Q}$, а значит, и над
$\mathbb{Z}$. Но тогда $e_1',\ldots,e_m'$ линейно зависимы в $A$~---
противоречие.
\end{proof}

Число элементов базиса свободной абелевой группы называется ее
\emph{рангом} и обозначается $\rk A$.

Опишем все базисы свободной абелевой группы. Пусть $\{e_1,\ldots,
e_n\}$~--- базис и $(e'_1,\ldots, e'_n)=(e_1,\ldots,e_n)C$.

\begin{theorem}
$\{e'_1,\ldots, e'_n\}$~--- базис $\Leftrightarrow$ $\det C=\pm1$.
\end{theorem}

\begin{proof}
  \begin{enumerate}
  \item Пусть $\det C=\pm1$. Тогда $C^{-1}$ целочисленна и
    $(e_1,\p\ldots, e_n)=(e'_1,\ldots,e'_n)C^{-1}$ $\Rightarrow$
    $e_1',\ldots,e_n'$ порождают $A$. Т.к. столбцы матрицы $C$ линейно
    независимы, то $\{e'_1,\ldots, e'_n\}$ линейно независимы.

  \item Пусть $\{e'_1,\ldots, e'_n\}$~--- базис. Тогда $(e_1,\ldots,
    e_n)=(e'_1,\ldots,e'_n)D$, где $D$~--- целочисленная матрица, откуда
    \begin{equation*}
      (e_1,\ldots, e_n)=(e_1,\ldots,e_n)CD\Rightarrow
      CD=E \Rightarrow \det C\cdot \det D=1 \Rightarrow \det C=\pm1
    \end{equation*}
  \end{enumerate}
\end{proof}
%----------------------------------------------------------%
%%%%%%%%%%%%%%%%%%%%%%%%%%%%%%%%%%%%%%%%%%%%%%%%%%%%%%%%%%%%
%----------------------Lecture 4---------------------------%
\lecture{}

\begin{theorem}
\label{1.IV}Всякая подгруппа $N$ свободной абелевой группы $L$ ранга
$n$ есть свободная абелева группа ранга не больше $n$.
\end{theorem}

\begin{proof}
Индукция по $n$.

$n=1$ $\Rightarrow$ $L\simeq\mathbb{Z}$. Будем считать, что
$L=\mathbb{Z}$. Если $N=\{0\}$, то $N$~--- свободная абелева группа
ранга 0. Если $N\neq\{0\}$, то $N$ содержит положительные числа, и
$k$~--- наименьшее из них. Докажем, что $N=k\mathbb{Z}$. Пусть $m\in
N$, тогда $m=qk+r$, $0\leq\mathrm{SL}ant r<k$. Тогда $r=m-qk\in N$
$\Rightarrow$ $r=0$.

Пусть теперь $\{e_1,\ldots,e_n\}$~--- базис $L$ и $L_1=\langle
e_1,\ldots,e_{n-1}\rangle$. Тогда $L_1$~--- это свободная абелева
группа ранга $n-1$. Рассмотрим группу $N_1\p=N\p\cap L_1\subset
L_1$. По предположению индукции $N_1$~--- свободная абелева группа
ранга $m\leqslant n-1$. Пусть $\{f_1,\ldots,f_m\}$~--- базис $N_1$.
Если $N=N_1$, то все доказано. Если $N\neq N_1$, то рассмотрим
последние координаты всех элементов из $N$ в базисе
$\{e_1,\ldots,e_n\}$ группы $L$. Они образуют ненулевую подгруппу в
группе $\mathbb{Z}$. Значит, она имеет вид $k\mathbb{Z}$. Пусть
$f_{m+1}\in N$ имеет последнюю координату $k$. Тогда
$\{f_1,\ldots,f_m,f_{m+1}\}$~--- базис $N$.
\end{proof}

\begin{note}
$\rk N=n\nRightarrow N=L$. Например, $k\mathbb{Z}\subsetneqq
\mathbb{Z}$ при $k>1$ и $\rk k\mathbb{Z}=\rk\mathbb{Z}=1$.
\end{note}

Пусть $\mathbb{E}^n$~--- $n$-мерное евклидово векторное
пространство, $\{e_1,\ldots,e_n\}$ --- его базис. Тогда
$L=\Big\{\sum\limits_{i=1}^n k_ie_i: k_i\in\mathbb{Z}\Big\}$~---
свободная абелева группа ранга $n$ с базисом $\{e_1,\ldots,e_n\}$.
Такие подгруппы называются \emph{решетками в $\mathbb{E}^n$}.

Подмножество $A\subset\mathbb{E}^n$ \emph{дискретно}, если в любом
ограниченном подмножестве $K\subset\mathbb{E}^n$ имеется лишь
конечное число точек из $A$ (по-другому: у $A$ нет предельных
точек). Очевидно, что всякая решетка является дискретным
подмножеством.

\begin{theorem}
\label{2.IV}Всякая дискретная подгруппа $L$ в $\mathbb{E}^n$,
порождающая $\mathbb{E}^n$ как векторное пространство, является
решеткой.
\end{theorem}

\begin{proof}
Существует базис $\{e_1,\ldots,e_n\}$ пространства $\mathbb{E}^n$,
содержащийся в $L$. Пусть $L_0$~--- решетка, порожденная этим
базисом. Ясно, что $L_0\subset L$.

Докажем, что $L_0$~--- подгруппа конечного индекса в $L$. Рассмотрим
параллелепипед $P=\Big\{\sum\limits_{i=1}^n x_ie_i: 0\leqslant
x_i\leqslant 1\Big\}$. Тогда $\forall x\in L\quad\exists \,
k_1,\ldots,k_n\p\in\mathbb{Z}: x-(k_1e_1+\ldots+k_ne_n)\in P$. Это
означает, что каждый смежный класс $L$ по $L_0$ содержит элемент из
$P$. По условию дискретности $L\cap P$ конечно, а значит,
$|L:L_0|=d<\infty$.

$|L/L_0|=d$ $\Rightarrow$ $dL\subset L_0$. Т.о., $L_0\subset
L\subset d^{-1}L_0$. $d^{-1}L_0$ есть свободная абелева группа ранга
$n$ с базисом $\{d^{-1}e_1,\ldots,d^{-1}e_n\}$. По
теореме~\ref{1.IV} $L$~--- свободная абелева группа ранга $n$.
Всякий базис группы $L$ содержит $n$ элементов и порождает
$\mathbb{E}^n$, а значит, является базисом $\mathbb{E}^n$.
\end{proof}

\emph{Кристаллической структурой в $\mathbb{E}^3$} называется
конечный набор дискретных подмножеств $A_1,\ldots,A_k\subset
\mathbb{E}^3$ со следующим свойством: существует такой базис
$\{e_1,e_2,e_3\}$ пространства $\mathbb{E}^3$, то $A_i+e_j=A_i$,
$i\p=1,\ldots,k$, $j=1,2,3$. Рассмотрим группу
$L=\{a\in\mathbb{E}^3: t_aA_i=A_i, i\p=1,\ldots,k\}$. По
теореме~\ref{2.IV} это решетка.

Группа симметрий кристаллической структуры
$A=\{A_1,\ldots,A_k\}$~--- это группа $\Gamma=\Sym A=\{g\in
\Isom\mathbb{E}^3: gA_i=A_i, i=1,\ldots,k\}$. Такие группы
называются \emph{кристаллографическими}.

\emph{Группа симметрий направлений в кристаллической структуре}~---
это группа $G=d\Gamma=\{dg:g\in\Gamma\}\subset \mathrm{O}_3$.

\begin{theorem}
Группа $G$ конечна и может содержать повороты или зеркальные
повороты только на углы
$0,\frac{\pi}{3},\frac{\pi}{2},\frac{2\pi}{3},\pi$.
\end{theorem}

\begin{proof}
Пусть $L=\{a\in\mathbb{E}^3:t_a\in\Gamma\}$~--- решетка и
$\{e_1,e_2,e_3\}$~--- базис этой решетки. Тогда $\forall
a\in\mathbb{E}^n$, $\forall \gamma\in\Isom\mathbb{E}^n\quad \gamma
t_a\gamma^{-1}=t_{d\gamma(a)}$ $\Rightarrow$ $d\gamma(a)\in\Gamma$.
Т.о., $\forall g\in G\quad \mathrm{GL}=L$, т.е. в базисе
$\{e_1,e_2,e_3\}$ $g$ записывается целочисленной матрицей. Значит,
$G$~--- дискретное подмножество в пространстве всех матриц. Но
$G\subset\mathrm{O}_3$~--- ограниченное подмножество (в
ортонормированном базисе все матричные элементы по модулю не больше
1). Значит, $|G|<\infty$. Далее, $\forall g\in G\quad
\mathrm{tr}g\in\mathbb{Z}$. Но в некотором ортонормированном базисе
$g$ записывается матрицей $\left(\begin{smallmatrix}\cos\varphi&
-\sin\varphi& 0\\ \sin\varphi& \cos\varphi& 0\\
0& 0& \pm1\end{smallmatrix}\right)$ $\Rightarrow$
$\mathrm{tr}g=2\cos\varphi\pm 1$ $\Rightarrow$
$2\cos\varphi\in\mathbb{Z}$ $\Rightarrow$ $|\cos\varphi|=\{\frac 12,
1\}$.
\end{proof}
%----------------------------------------------------------%
%%%%%%%%%%%%%%%%%%%%%%%%%%%%%%%%%%%%%%%%%%%%%%%%%%%%%%%%%%%%
%----------------------Lecture 5---------------------------%
\lecture{}

\emph{Элементарные преобразования базисов}:

1) $e_i'=e_i+ce_j$ ($c\in\mathbb{Z}$), $e_k'=e_k$ при $k\neq i$

2) $e_i'=e_j$, $e_j'=e_i$, $e_k'=e_k$ при $k\neq i,j$

3) $e_i'=-e_i$, $e_k'=e_k$ при $k\neq i$.

Прямоугольная матрица $C=(c_{ij})$ размера $m\times n$ называется
\emph{диагональной}, если $c_{ij}=0$ при $i\neq j$. Обозначение:
$C=\diag(c_{11},\ldots,c_{pp})$, $p=\min\{m,n\}$.

\begin{lemma}
\label{1.V}Всякую целочисленную матрицу $C$ размера $m\times n$ с
помощью целочисленных элементарных преобразований строк и столбцов
можно привести к виду $\diag(u_1,\ldots,u_p)$ ($p=\min\{m,n\}$), где
$u_i\in\mathbb{Z}$, $u_i\geqslant 0$ и $u_i\mid u_{i+1}$ при
$i=1,\ldots,p-1$.
\end{lemma}

\begin{proof}
Если $C=0$, то доказывать нечего. Если $C\neq 0$, то путем
элементарных преобразований строк и столбцов можно добиться, чтобы
$c_{11}>0$. Далее будем минимизировать $c_{11}$.

Если $c_{i1}$ не делится на $c_{11}$, то разделим с остатком:
$c_{i1}=qc_{11}+r$, $0<r<c_{11}$ и, вычитая из $i$-й строки 1-ю,
умноженную на $q$, получим $r$ на месте $i,1$. Переставив 1-ю и
$i$-ю строки, получим $r$ на месте $(1,1)$.

Аналогично, если $c_{1j}$ не делится на $c_{11}$, то с помощью
целочисленных элементарных преобразований столбцов можно также
уменьшить $c_{11}$.

Пусть все элементы 1-й строки и 1-го столбца делятся на $c_{11}$.
Тогда с помощью целочисленных элементарных преобразований строк и
столбцов их можно сделать нулями, т.е. привести $C$ к виду
$$\begin{pmatrix}
c_{11}& 0& \cdots& 0\\
0 & *& \cdots& *\\
\vdots & \vdots&\ddots &\vdots \\
0 & *&\cdots &*
\end{pmatrix}$$

Предположим теперь, что $c_{ij}$ ($i,j\geq\mathrm{SL}ant 2$) не
делится на $c_{11}$. Прибавив к 1-й строке $i$-ю строку, мы не
изменим $c_{11}$, но получим, что $c_{1j}$ не делится на $c_{11}$ и
придем к рассмотренной ранее ситуации.

В конце концов
$$C=\begin{pmatrix}
c_{11}& 0& \cdots& 0\\
0 & c_{22}& \cdots& c_{2p}\\
\vdots & \vdots&\ddots &\vdots \\
0 & c_{p2}&\cdots &c_{pp}
\end{pmatrix}$$
где всякий элемент матрицы
$$C_1=\begin{pmatrix}
c_{22}& \cdots& c_{2p}\\
\vdots & \ddots &\vdots \\
c_{p2} & \cdots &c_{pp}
\end{pmatrix}$$
делится на $c_{11}=u_1$. Далее, делая то же самое с матрицей $C_1$,
свойство делимости на $u_1$ сохранится, и мы приведем матрицу $C$ к
требуемому виду.
\end{proof}

\begin{theorem}
\label{2.V}Для всякой подгруппы $N$ свободной абелевой группы $L$
существует такой базис $\{e_1,\ldots,e_n\}$ группы $L$ и такие
натуральные числа $u_1,\ldots,u_m$ ($m\leqslant n$), что
$\{u_1e_1,\ldots,u_me_m\}$~--- базис $N$ и $u_i\mid u_{i+1}$ при
$i=1,\ldots,m-1$.
\end{theorem}

\begin{proof}
Пусть $\{e_1,\ldots,e_n\}$~--- произвольный базис группы $L$ и
$\{f_1,\ldots,f_m\}$~--- базис $N$. Тогда
$$(f_1,\ldots,f_m)\p=(e_1,\ldots,e_n)\mathop{C}\limits_{n\times m}$$
($C$~--- целочисленная матрица). При элементарных преобразованиях
базиса группы $L$ $$(e_1',\ldots,e_n')=(e_1,\ldots,e_n)U$$ ($U$~---
элементарная матрица) $\Rightarrow$
$(f_1,\ldots,f_m)\p=(e_1',\ldots,e_n')U^{-1}C$, т.е. в $C$
происходят целочисленные элементарные преобразования строк.

При элементарных преобразованиях базиса подгруппы $N$ получаем:
$$(f_1',\ldots,f_m')=(f_1,\ldots,f_m)V \Rightarrow
(f_1',\ldots,f_m')=(e_1,\ldots,e_n)CV$$
т.е. в $C$ происходят целочисленные элементарные преобразования
столбцов.

По лемме~\ref{1.V} матрицу $C$ можно таким образом привести к виду
$C\p=\diag(u_1,\ldots,u_m)$. Т.к. $\rk C=m$, то $u_i\neq 0$ и
$f_i=u_1e_i$, $i=1,\ldots,m$.
\end{proof}

\begin{theorem}
Всякая конечно порожденная абелева группа $A$ разлагается в прямую
сумму циклических групп.
\end{theorem}

\begin{proof}
Пусть $A=\langle a_1,\ldots,a_n\rangle$. Рассмотрим гомоморфизм
$$\varphi\colon \mathbb{Z}^n\stackrel{\text{на}}{\to}A,\quad
(k_1,\ldots,k_n)\mapsto k_1a_1+\ldots+a_ne_n.$$ Пусть $N=\ker
\varphi$, тогда $A\simeq \mathbb{Z}^n/N$. По теореме~\ref{2.V}
существуют базис $\{e_1,\ldots,e_n\}$ группы $\mathbb{Z}^n$ и
натуральные числа $u_1,\ldots,u_m$ ($m\leqslant n$), такие, что
$\{u_1e_1,\ldots,u_me_m\}$~--- базис $N$ и $u_i\mid u_{i+1}$ при
$i=1,\ldots,m-1$.

Рассмотрим гомоморфизм
$$\psi\colon\mathbb{Z}^n\stackrel{\text{на}}{\to}\mathbb{Z}_{u_1}
\oplus\ldots\oplus\mathbb{Z}_{u_m}\oplus\underbrace{\mathbb{Z}
\oplus\ldots\oplus\mathbb{Z}}_{n-m},$$ $$l_1e_1+\ldots+l_ne_n\mapsto
([l_1]_{u_1},\ldots,[l_m]_{u_m},l_{m+1},\ldots,l_n).$$
$\ker\psi=\langle u_1e_1,\ldots,u_me_m\rangle=N$. Следовательно,
$$A\simeq\mathbb{Z}^n/N\simeq\mathbb{Z}_{u_1}
\oplus\ldots\oplus\mathbb{Z}_{u_m}\oplus\underbrace{\mathbb{Z}
\oplus\ldots\oplus\mathbb{Z}}_{n-m}.$$
\end{proof}

\begin{note}
1) На самом деле мы доказали, что $A$ разлагается в прямую сумму
циклических групп порядков $u_1,\ldots,u_m,\infty$, где $u_i\mid
u_{i+1}$.

2) Если $A$ конечна, то слагаемых $\mathbb{Z}$ нет.
\end{note}

\begin{lemma}
\label{3.V}Если $n=kl$, $(k,l)=1$, то
$\mathbb{Z}_n\simeq\mathbb{Z}_k\oplus\mathbb{Z}_l$.
\end{lemma}

\begin{proof}
Нужно доказать, что группа $\mathbb{Z}_k\oplus\mathbb{Z}_l$
циклическая, т.е. что в ней есть элемент порядка $n$. Таким
элементом является $([1]_k,[1]_l)$. В самом деле,
$m([1]_k,[1]_l)=([m]_k,[m]_l)=0$ $\Leftrightarrow$ $k,l\mid m$
$\Leftrightarrow$ $n\mid m$. Следовательно, $\ord([1]_k,[1]_l)=n$.
\end{proof}

\begin{theorem}
Если $n=p_1^{k_1}\ldots p_s^{k_s}$ ($p_1,\ldots,p_s$~--- различные
простые числа), то
$\mathbb{Z}_n\simeq\mathbb{Z}_{p_1^{k_1}}\oplus\ldots\oplus\mathbb{Z}_{p_s^{k_s}}$.
\end{theorem}

\begin{proof}
По лемме~\ref{3.V}$$\mathbb{Z}_n\simeq\mathbb{Z}_{p_1^{k_1}}\oplus
\mathbb{Z}_{p_2^{k_2}\ldots
p_s^{k_s}}\simeq\ldots\simeq\mathbb{Z}_{p_1^{k_1}}\oplus \ldots
\oplus\mathbb{Z}_{p_s^{k_s}}.$$
\end{proof}

\begin{ex}
\begin{enumerate}
  \item $\mathbb{Z}_{60}\simeq \mathbb{Z}_4\oplus\mathbb{Z}_3\oplus\mathbb{Z}_5=
  \langle[15]\rangle\oplus\langle[20]\rangle\oplus\langle[12]\rangle$.
  Например, $[1]=-[15]\p-[20]+3\cdot[12]$.
\end{enumerate}
\end{ex}
Группа называется \emph{примарной}, если ее порядок есть степень
простого числа.

\begin{theorem}
Всякая конечно порожденная абелева группа $A$ разлагается в прямую
сумму примарных и бесконечных циклических групп, причем число
слагаемых и набор порядков определены однозначно.
\end{theorem}

\begin{proof}
1) Существование такого разложения следует из теорем 2 и 3.

2) Покажем единственность числа слагаемых и наборов их порядков.
Пусть $A=\langle a_1\rangle_{p_1^{k_1}}\oplus\ldots\oplus\langle
a_s\rangle_{p_s^{k_s}}\oplus\langle
a_{s+1}\rangle_\infty\oplus\ldots\oplus\langle
a_{s+t}\rangle_\infty$ (среди чисел $p_1,\ldots,p_s$ могут быть
одинаковые).

Рассмотрим \emph{подгруппу кручения} $\Tor A=\{a\in A:\ord
a<\infty\}$. Ясно, что $\Tor A=\langle
a_1\rangle_{p_1^{k_1}}\oplus\ldots\oplus\langle
a_s\rangle_{p_s^{k_s}}$ и $A/\Tor A\simeq\mathbb{Z}^t$. Т.к.
определение $\Tor A$ не зависит от разложения, то и число $t$ не
зависит от разложения.

Рассмотрим \emph{подгруппу $p$-кручения} $\Tor_p A=\{a\in
A:p^ka=0\}$. Ясно, что $\Tor A$~--- сумма тех $\langle
a_i\rangle_{p_i^{k_i}}$, для которых $p_i=p$. Т.к. определение
$\Tor_pA$ не зависит от разложения, то и
$\bigoplus\limits_{p_i=p}\langle a_i\rangle_{p_i^{k_i}}$ не зависит
от разложения.

Т.о., доказательство теоремы сводится к случаю, когда $A$~---
примарная группа.

3) Случай примарной группы: $A=\langle
a_1\rangle_{p^{k_1}}\oplus\ldots\oplus\langle a_r\rangle_{p^{k_r}}$
($k_1\leqslant\ldots\leqslant k_r$), $|A|=p^k$, $k=k_1+\ldots+k_r$.

Докажем индукцией по $k$, что набор $(k_1,\ldots,k_r)$ определен
однозначно. При $k=1$ это очевидно. Предположим, что утверждение
верно для групп порядка $p^l$, $l\leqslant k$. Пусть
$k_1=\ldots=k_s=1$, $k_{s+1}>1$.

Рассмотрим подгруппу $pA=\hc{pa: a\in A}$. Ясно, что
$$pA=\langle
pa_s\rangle_{p^{k_{s+1}-1}}\oplus\ldots\oplus\langle
pa_r\rangle_{p^{k_r-1}}.$$ По предположению индукции (для $pA$)
набор $(k_{s+1}-1,\ldots,k_r-1)$ определен однозначно. Значит, и набор
$(k_{s+1},\ldots,k_r)$ определен однозначно.

Число $s$ определяется из равенства $s+k_{s+1}+\ldots+k_r=k$.
\end{proof}
%----------------------------------------------------------%
%%%%%%%%%%%%%%%%%%%%%%%%%%%%%%%%%%%%%%%%%%%%%%%%%%%%%%%%%%%%
%----------------------Lecture 6---------------------------%
\lecture{}

\begin{note}
Сами слагаемые разложения, о которых идет речь в теореме, вообще
говоря. не определены однозначно. Например, $\langle
a_1\rangle_2\oplus\langle a_2\rangle_2\p=\langle
a_1+a_2\rangle_2\oplus\langle a_2\rangle_2$. Вообще. если
$G=\mathbb{Z}_p^r$, то $G$ можно рассматривать как $r$-мерное
векторное пространство над $\mathbb{Z}_p$, и разложение $G$ в прямую
сумму циклических подгрупп~--- это разложение векторного
пространства в сумму одномерных подпространств.
\end{note}

\emph{Экспонентой} конечной группы $G$ называется н.о.к. порядков
всех своих элементов. Обозначение: $e(G)$. Ясно, что $e(G)\mid|G|$ и
что $g^{e(G)}\p=e\quad\forall g\in G$.

Вообще говоря, элемента порядка $e(G)$ не существут: $e(S_3)=6$, но
элементов порядка 6 в $S_3$ нет.

\begin{theorem}
\label{1.VI}В любой конечной абелевой группе $A$ существует элемент
порядка $e(A)$.
\end{theorem}

\begin{proof}
$A=\langle a_1\rangle_{u_1}\oplus\ldots\langle
a_m\rangle_{u_m}\oplus$, где $u_i\mid u_{i+1}$ ($i=1,\ldots,m-1$).
$e(A)=u_m=\ord a_m$.
\end{proof}

\begin{theorem}
Мультипликативная группа $F^*$ любого конечного поля $F$
циклическая.
\end{theorem}

\begin{proof}
$|F|=q$ $\Rightarrow$ $|F^*|=q-1$. Докажем, что $e(F^*)=q-1$.
$\forall x\in F^*\quad x^{e(F^*)}-1=0$ $\Rightarrow$ $q-1\leqslant
e(F^*)$ $\Rightarrow$ $e(F^*)=q-1$. По теореме~\ref{1.VI} $\exists
\, a\in F^*: \ord a=q-1$ $\Rightarrow$ $F^*=\langle a\rangle$.
\end{proof}

Пусть $p$~--- нечетное простое число. $\mathbb{Z}^*_p=\langle
a\rangle_{p-1}$.

Элемент $c\in\mathbb{Z}^*_{p}$ называется \emph{квадратичным
вычетом}, если он является квадратом в $\mathbb{Z}^*_p$. $c=a^k$~---
квадратичный вычет $\Leftrightarrow$ $k$ четно.

\begin{ex}
\begin{enumerate}
  \item $\mathbb{Z}^*_7=\{1,2,3,4,5,6\}$, $1,2,4$~--- квадратичные
  вычеты, $3, 5, 6$~--- квадратичные невычеты.
\end{enumerate}
\end{ex}

\begin{theorem}
Уравнение $x^2+1$ имеет корень в $\mathbb{Z}^*_p$ $\Leftrightarrow$
$p\equiv 1\pmod{4}$.
\end{theorem}

\begin{proof}
-1~--- единственный элемент порядка 2 в $\mathbb{Z}^*_p$. Если
$\mathbb{Z}^*_p\p=\langle a\rangle$, то $-1=a^{\frac{p-1}{2}}$
$\Rightarrow$ -1~--- квадратичный вычет $\Leftrightarrow$
$\frac{p-1}{2}$ четно.
\end{proof}

\tema{Действия групп}

Пусть $S(X)$~--- группа всех преобразований множества $X$.
\emph{Действием группы $G$ на множестве $X$} называется всякий
гомоморфизм $\alpha\colon G\p\to S(X)$:
$\alpha(gh)=\alpha(g)\alpha(h)$ $\Rightarrow$ $\alpha(e)=id$,
$\alpha(g^{-1})=\alpha(g)^{-1}$. Обозначения:
$G\mathop{:}\limits_\alpha X$, $\alpha(g)x=gx$; условие
гомоморфизма: $(gh)x=g(hx)$. $\ker\alpha\triangleleft G$~---
\emph{ядро неэффективности действия $\alpha$}. Если
$\ker\alpha=\{e\}$, то действие \emph{эффективно}.

$\Im\alpha$~--- группа преобразований множества $X$. По теореме о
гомоморфизме $\Im\alpha\simeq G/\ker\alpha$.

Если $G:X$, то $G$ действуют

1) на любом инвариантном подмножестве $Y\subset X$

2) на множестве всех подмножеств множества $X$.

\begin{ex}
\begin{enumerate}
  \item $\Isom\mathbb{E}^2:\mathbb{E}^2$ $\Rightarrow$ $\Isom
  \mathbb{E}^2$ действует на множестве треугольников.
\end{enumerate}
\end{ex}

Если $G:X$ и $H\subset G$~--- подгруппа, то $H:X$.

Действие $G:X$ определяет отношение эквивалентности:
$x\mathop{\sim}\limits_Gy$, если $\exists \, g\in G: y=gx$. Классы
эквивалентности называются \emph{орбитами} данного действия. Класс
эквивалентности $x$ обозначается как $Gx=\{gx:g\in G\}$. Действия с
одной орбитой называются \emph{транзитивными}. Число точек в орбите
называется ее \emph{длиной} и обозначается $|Gx|$.

\emph{Стабилизатор элемента $x$}~--- это $G_x=\{g\in G: gx=x\}$.

\begin{theorem}
Пусть $G:X$. Тогда $G_{gx}=gG_xg^{-1}$.
\end{theorem}

\begin{proof}
$h\in G_x$ $\Rightarrow$ $(ghg^{-1})(gx)=g(hx)=gx$ $\Rightarrow$
$ghg^{-1}\in G_{gx}$.

$h\in G_{gx}$ $\Rightarrow$ $(g^{-1}hg)x=g^{-1}(gx)=x$ $\Rightarrow$
$g^{-1}hg\in G_x$.
\end{proof}

\begin{ex}
\begin{enumerate}
  \item $SO_2:\mathbb{E}^2$, $G_o=SO_2$, $G_p=\{e\}$, $p\neq o$.
  \item $\mathrm{GL}_n(\mathbb{C}):\mathrm{GL}_n(\mathbb{C})$, $A\circ X=AXA^{-1}$.
  Ядро неэффективности есть $\{\lambda E: \lambda\in\mathbb{C}^*\}$.
  $A\sim B$ $\Leftrightarrow$ $A$ и $B$ имеют одну и ту же жорданову
  форму.
  \item $\mathrm{GL}_n(\mathbb{C}):L_n(\mathbb{C})$, $A\circ X=AXA^t$.
  \item $S_4:\{1,2,3,4\}\rightsquigarrow V_4:\{1,2,3,4\}$. Действие
  $V_4:\{1,2,3,4\}$ транзитивно, стабилизаторы тривиальны.
\end{enumerate}
\end{ex}
%----------------------------------------------------------%
%%%%%%%%%%%%%%%%%%%%%%%%%%%%%%%%%%%%%%%%%%%%%%%%%%%%%%%%%%%%
%----------------------Lecture 7---------------------------%
\lecture{}

\begin{theorem}
\label{1.VII}Если группа $G$ конечна, то $|Gx|=|G:G_x|$.
\end{theorem}

\begin{proof}
Рассмотрим отображение $G/G_x\to Gx$, $gG_x\mapsto gx$. Это
определение корректно: $\forall h\in G_x\quad (gh)x=g(hx)=gx$.
Построенное отображение сюръективно по определению орбиты. Оно также
инъективно: $g_1x=g_2x$ $\Rightarrow$ $(g_2g_1^{-1})x=x$, т.е.
$g_2g_1^{-1}\in G_x$ $\Rightarrow$ $g_1G_x=g_2G_x$.
\end{proof}

Пусть $P$~--- выпуклый многогранник. \emph{Флагом} многогранника $P$
назовем тройку $\{v,e,f\}$, где $v$~--- вершина, $e$~--- ребро,
содержащее $v$, $f$~--- грань, содержащая $e$. $P$~---
\emph{правильный многогранник}, если $\Sym P$ действует транзитивно
на множестве флагов.

Пусть $V$~--- множество вершин многогранника $P$. Рассмотрим
действие $\Sym P\p: V$. Это транзитивное действие. По
теореме~\ref{1.VII} $|\Sym P|\p=|V|\cdot|(\Sym P)_v|$.

Пусть $E_v$~--- множество ребер, выходящих из $v$. Действие $(\Sym
P)_v:E_v$ транзитивно $\Rightarrow$ по теореме~\ref{1.VII} $|(\Sym
P)_v|=|E_v|\cdot 2$.

Окончательно получаем, что $$|\Sym P|=2(\text{число
вершин})(\text{степень вершины}).$$ Для куба $|\Sym P|=48$, для
икосаэдра $|\Sym P|=120$.

$G\mathop{:}\limits_l G$, $l(g)x=gx$~--- действие группы на себе:
$$l(g_1g_2)x=g_1(g_2x)\p=l(g_1)l(g_2)x.$$ Это действие транзитивно.
Стабилизатор тривиален.

Если $H\subset G$~--- подгруппа, то орбитами $H$ будут правые
смежные классы $Hx$.

Аналогично, $G\mathop{:}\limits_r G$, $r(g)x=xg^{-1}$:
$$r(g_1g_2)x=x(g_1g_2)^{-1}=xg_2^{-1}g_1^{-1}=r(g_1)r(g_2)x.$$ Орбиты
подгруппы~--- левые смежные классы $xH$.

$G\mathop{:}\limits_a G$, $a(g)x=gxg^{-1}$:
$$a(g_1g_2)x=g_1g_2xg_2^{-1}g_1^{-1}=a(g_1)a(g_2)x.$$ $\forall x\in G
\quad a(g)$~--- автоморфизм группы $G$:
$$a(g)(xy)=g(xy)g^{-1}=(gxg^{-1})(gyg^{-1})=(a(g)x)(a(g)y).$$
Эквивалентные элементы называются \emph{сопряженными}, т.е. $x$ и
$y$ сопряжены, если $\exists \, g\in G: gxg^{-1}=y$. Орбиты~---
\emph{классы сопряженности}. Обозначение: $C(x)$.

Стабилизатор элемента $x$ называется \emph{централизатором элемента
$x$} и обозначается $Z(x)$. По определению, $Z(x)=\{g\in G:
gx=xg\}$. Ядро неэффективности~--- \emph{центр $Z(G)$ группы $G$}.

\begin{imp}
Если $G$ конечна. то $|C(x)|=\frac{|G|}{|Z(x)|}$. $\square$
\end{imp}

\begin{ex}
\begin{enumerate}
  \item $G=S_n$. Пусть $\sigma=(i_1\ldots
  i_k)(j_1\ldots j_l)\ldots$~--- разложение на независимые циклы, $\tau\in
  S_n$. Если $\sigma(p)=q$, то $\tau\sigma\tau^{-1}(\tau
  (p))=\tau(q)$. Следовательно, $\tau\sigma\tau^{-1}=(\tau(i_1)\ldots\tau(i_k))
  (\tau(j_1)\ldots\tau(j_l))\ldots$ Т.о., сопряженные подстановки
  характеризуются тем, что наборы длин независимых циклов в их
  разложениях совпадают.

  Рассмотрим $S_4$: $e$~---1, $(ij)$~--- 6, $(ij)(kl)$~--- 3,
  $(ijk)$~--- 8, $(ijkl)$~--- 6.

  Докажем, что $Z(S_n)=\{e\}$. Пусть $\tau\in Z(S_n)$ $\Rightarrow$
  $$\tau(ij)\tau^{-1}=(\tau(i)\tau(j))=(ij)\quad \forall i,j,$$ т.е.
  $\tau$ сохраняет любую пару $\{i,j\}$ $\Rightarrow$ $\tau=e$, т.к.
  любой элемент из $\{1,2,\ldots,n\}$ есть пересечение двух пар.
  \item $G=\mathrm{GL}_n(\mathbb{C})$. $A$ и $B$ сопряжены тогда и только
  тогда. когда они имеют одну и ту же жорданову форму. $Z(\mathrm{GL}_n(\mathbb{C}))=\{\lambda E:\lambda\in
  \mathbb{C}^*\}$.
\end{enumerate}
\end{ex}
Рассмотрим действие группы $G$ на множестве своих подгрупп
сопряжениями: $a(g)H=gHg^{-1}$. Эквивалентные подгруппы называются
\emph{сопряженными}, т.е. $H_1$ и $H_2$ сопряжены, если $\exists \,
g\in G: gH_1g^{-1}=H_2$. Орбита называется \emph{классом сопряженной
подгруппы}. Стабилизатор подгруппы $H$ называется ее
\emph{нормализатором} и обозначается $N(H)$. Т.о., $N(H)=\{g\in G:
gHg^{-1}=H\}$. Очевидно, что $H\triangleleft N(H)$.

\begin{theorem}
Если $G$ конечна, то число подгрупп, сопряженных $H$, делит $|G:H|$.
\end{theorem}

\begin{proof}
По теореме~\ref{1.VII} это число равно
$$|G:N(H)|=\frac{|G|}{|N(H)|}\p=\frac{|G|}{|H|}:\frac{|N(H)|}{|H|}$$ и
делит $\frac{|G|}{|H|}=|G:H|$.
\end{proof}

\begin{theorem}
Центр примарной конечной группы нетривиален.
\end{theorem}

\begin{proof}
Пусть $|G|=p^k$, $k\in \mathbb{N}$. Разложим $G$ на классы
сопряженности, тогда $G=Z\sqcup C(x_1)\sqcup\ldots\sqcup C(x_s)$.
$\forall i=1,\ldots,s\quad |C(x_i)|=p^l$, $l\in \mathbb{N}$
$\Rightarrow$ $p\mid |C(x_i)|$ $\Rightarrow$ $p\mid |Z|$
$\Rightarrow$ $Z\neq\{e\}$.
\end{proof}

\begin{imp}
Всякая группа порядка $p^2$ абелева.
\end{imp}

\begin{proof}
Пусть $|G|=p^2$, $Z=Z(G)$. Предположим, что $|Z|=p$. Тогда
$|G:Z|=p$, и значит, $G/Z$~--- циклическая группа. Пусть $aZ$~--- ее
порождающий элемент $\Rightarrow$ $\forall g\in G\quad
gZ=(aZ)^k=a^kZ$ $\Rightarrow$ $g=a^kz$, $z\in Z$ $\Rightarrow$ $G$
абелева~--- противоречие.
\end{proof}

\tema{Теоремы Силова}

Пусть $|G|=p^km$, где $p$ простое, $p\nmid m$.

\emph{Силовской $p$-подгруппой группы $G$} называется всякая
подгруппа порядка $p^k$. Если $G$ абелева, то ее единственная
силовская $p$-подгруппа есть подгруппа $p$-кручения $\Tor_pG$.

\begin{theorem}
\label{4.VII} Силовские $p$-подгруппы существуют.
\end{theorem}

\begin{theorem}
\label{5.VII} Все силовские $p$-подгруппы сопряжены. Более того,
всякая $p$-подгруппа содержится в некоторой силовской $p$-подгруппе.
\end{theorem}

\begin{theorem}
\label{6.VII} Число силовских $p$-подгрупп сравнимо с 1 $\pmod{p}$.
\end{theorem}

\begin{ex}
\begin{enumerate}
  \item $|A_5|=60=2^2\cdot3\cdot5$. Силовские 2-подгруппы: $V_4\subset A_4\subset
  A_5$~--- 5; 3-подгруппы: $\langle(ijk)\rangle$~--- 10;
  5-подгруппы: $\langle(i_1\ldots i_5)\rangle$~--- 6.
\end{enumerate}
\end{ex}
%----------------------------------------------------------%
%%%%%%%%%%%%%%%%%%%%%%%%%%%%%%%%%%%%%%%%%%%%%%%%%%%%%%%%%%%%
%----------------------Lecture 8---------------------------%
\lecture{}

\begin{proof}[Доказательство теоремы~\ref{4.VII}]
Доказывать будем индукцией по $|G|$. Если $|G|=1$, то утверждение
тривиально.

Пусть $|G|=n>1$ и для всех групп порядка меньше $n$ утверждение
верно. $G=Z\sqcup C(x_1)\sqcup\ldots\sqcup C(x_s)$, $|C(x_i)|>1$.
Рассмотрим два случая.

1) $\exists \, i: p\nmid |C(x_i)|$. $|C(x_i)|=\frac{|G|}{|Z(x_i)|}$
$\Rightarrow$ $p^k\mid |Z(x_i)|$. Но $|Z(x_i)|<n$, поэтому по
предположению индукции существует силовская $p$-подгруппа в
$Z(x_i)$. Она будет силовской $p$-подгруппой в $G$.

2) $\forall i\quad p\mid |C(x_i)|$. Тогда $p\mid |Z|$. Пусть
$|Z|=p^{k_0}m_0$, где $0<k_0\p\leqslant k$ и $p\nmid m_0$, и пусть
$Z_0=\Tor_pZ$ (силовская $p$-подгруппа в $Z$). Имеем
$|Z_0|=p^{k_0}$. Рассмотрим $G/Z_0$ и канонический гомоморфизм
$\pi\colon G\to G/Z_0$. Имеем $|G/Z_0|=p^{k-k_0}m$. По предположению
индукции в $G/Z_0$ существует силовская $p$-подгруппа $S_1$,
$|S_1|=p^{k-k_0}$. Тогда $\pi^{-1}(S_1)=S$ имеет порядок $|S_1|\cdot
|Z_0|=p^k$ и является силовской $p$-подгруппой в $G$.
\end{proof}

\begin{proof}[Доказательство теоремы~\ref{5.VII}]
Пусть $S$~--- какая-то силовская $p$- подгруппа и $H$~--- какая-то
$p$-подгруппа. Рассмотрим $H:G/S$, $h\circ gS=hgS$. Длина каждой
нетривиальной орбиты делится на $p$ (т.к. она делит $|H|\p=p^l$). Но
$|G/S|=|G:S|$ не делится на $p$. Значит, существуют неподвижные
точки, т.е. $\exists \, g\in G: H\subset gSg^{-1}$. Т.о., $H$
содержится в силовской $p$-подгруппе $gSg^{-1}$. Если же $|H|=p^k$,
то $H=gSg^{-1}$.
\end{proof}

\begin{proof}[Доказательство теоремы~\ref{6.VII}]
Пусть $S$~--- какая-то силовская $p$-подгруппа и $C(S)$~---
множество всех подгрупп, сопряженных с $S$, т.е. по теореме
множество всех силовских $p$-подгрупп. Рассмотрим действие $S:C(S)$
сопряжениями. Длина каждой нетривиальной орбиты делится на $p$.
Найдем все тривиальные орбиты, т.е. неподвижные точки данного
действия. Если $S_1\in C(S)$~--- неподвижная точка, то $S\subset
N(S_1)=\{g\in G:gS_1g^{-1}\p=S_1\}$. Но тогда $S$ и $S_1$~---
силовские $p$-подгруппы в $N(S_1)$ и по теореме они сопряжены в
$N(S_1)$. Т.к. $S_1\triangleleft N(S_1)$, то $S_1=S$.

Итак, для действия $S:C(S)$ имеется единственная неподвижная точка,
а именно сама подгруппа $S$. Следовательно, $|C(S)|\equiv1\pmod{p}$.
\end{proof}

\begin{ex}
\begin{enumerate}
  \item $|G|=pq$, где $p>q$~--- различные простые числа. Тогда число
  силовских $p$-подгрупп $N_p\equiv1\pmod{p}$ и $N_p\mid q$
  $\Rightarrow$ $N_p=1$, т.е. силовская $p$-подгруппа нормальна и
  единственна. Обозначим ее $G_p$. Тогда $|G_p|=p$ $\Rightarrow$ $G_p\simeq
  \mathbb{Z}_p$ и $G_p$~--- циклическая. Далее, $N_q\equiv1\pmod{q}$
  и $N_q\mid p$. Если $p\not\equiv1\pmod{q}$, то $N_q=1$, т.е.
  силовская $q$-подгруппа $G_q$ также единственна и нормальна. Т.к.
  $G_p\cap G_q=\{e\}$, то $G_p\cdot G_q=G$ и, значит, $G=G_p\times
  G_q$, т.е. $G$~--- циклическая.
  \item $|G|=45=3^2\cdot5$. $N_3\equiv1\pmod{3}$ и $N_3\mid 5$
  $\Rightarrow$ $N_3=1$. $N_5\equiv1\pmod{q}$ и $N_5\mid 9$
  $\Rightarrow$ $N_5=1$. $G=G_3\times G_5$. $G_5$ циклическая, $G_3$
  абелева $\Rightarrow$ $G$ абелева.
\end{enumerate}
\end{ex}
\tema{Полупрямые произведения групп}

Группа $G$ разлагается в \emph{полупрямое произведение своих
подгрупп $N$ и $H$}, если

1) $N\triangleleft G$

2) $N\cap H=\{e\}$

3) $NH=G$, т.е. $\forall g\in G\quad g=nh$, где $n\in N$, $h\in H$.

Из этих условий следует, что представление $g=nh$ единственно:
$g\p=n_1h_1=n_2h_2$ $\Rightarrow$ $N\ni n_2^{-1}n_1=h_2h_1^{-1}\in
H$ $\Rightarrow$ $n_1=n_2$, $h_1=h_2$. Обозначение:
$G=N\leftthreetimes H=H\rightthreetimes N$.

\begin{ex}
\begin{enumerate}
  \item $S_n=A_n\leftthreetimes \langle(12)\rangle$
  \item $S_4=V_4\leftthreetimes S_3$
  \item $D_n=C_n\leftthreetimes \langle r\rangle$, $r\in D_n$~---
  отражение
  \item $\mathrm{GL}_n(K)=\mathrm{SL}_n(K)\leftthreetimes \{\diag(1,\ldots,\lambda)\}$
  \item $GA(S)=N\leftthreetimes GA(S)_o$.
\end{enumerate}
\end{ex}
Правило умножения: $(n_1h_1)(n_2h_2)=(n_1(h_1n_2h_1^{-1}))(h_1h_2)$.
В частности, отображение $G\to H$, $nh\mapsto h$, является
гомоморфизмом, и по теореме о гомоморфизме $G/N\simeq H$.

Отображение $N\to N$, $n\mapsto hnh^{-1}$ является автоморфизмом
группы $N$. Обозначим его через $\alpha(h)$. Отображение
$\alpha\colon H\to \Aut N$ является гомоморфизмом. Оно определяет
структуру полупрямого произведения. В частности, это произведение
является прямым $\Leftrightarrow$ $\alpha$ тривиален:
$\alpha\p=\mathrm{id}\quad \forall h\in H$.

\emph{Внешнее полупрямое произведение групп $N$ и $H$} определяется
гомоморфизмом $\alpha\colon H\to \Aut N$. Тогда $G=N\times H$,
$(n_1,h_1)(n_2,h_2)\p=(n_1(\alpha(h_1)n_2),h_1h_2)$. Выполнены все
аксиомы группы: $e=(e_N,e_H)$ и $(n,h)^{-1}=(\alpha(h^{-1})n^{-1},
h^{-1})$.

Опишем полупрямые произведения циклической группы. Для этого пишем
группу автоморфизмов циклической группы.

\begin{theorem}
Всякий автоморфизм циклической группы $\langle a\rangle_n$ имеет вид
$\varphi_k(x)=x^k$, где $(k,n)=1$.
\end{theorem}

\begin{proof}
Пусть $\varphi\in \Aut\langle a\rangle_n$, $\varphi(a)=a^k$. Тогда
$\forall x=a^m\quad \varphi(x)\p=\varphi(a)^m=a^{km}=x^k$.

$\ker\varphi=\{a^m: n\mid km\}$. Если $(k,n)=1$, то
$\ker\varphi=\{e\}$. Если $(k,n)\p=d>1$, то $\ker\varphi=\langle
a^\frac nd\rangle\neq\{e\}$. Т.о., если $\varphi\in\Aut \langle
a\rangle_n$, то $(k,n)=1$.

Обратно, пусть $(k,n)=1$ . Рассмотрим $\varphi_k\colon\langle
a\rangle_n\to \langle a\rangle_n$, $\varphi_k(x)=x^k$. Это
гомоморфизм: $(xy)^k=x^ky^k$ и $\ker\varphi_k=\{e\}$ $\Rightarrow$
$\Im \varphi_k=\langle a\rangle_n$ $\Rightarrow$ $\varphi_k\p\in
\Aut \langle a\rangle_n$.
\end{proof}

\begin{imp}
$\Aut \langle a\rangle_n\simeq \mathbb{Z}_n^*$.
\end{imp}

\begin{proof}
Автоморфизмы нумеруются элементами этого кольца: $\mathbb{Z}_n^*\to
\Aut\langle a\rangle_n$, $[k]_n\mapsto \varphi_k$. Это гомоморфизм:
$\varphi_{kl}=\varphi_k\varphi_l$, и он биективен $\Rightarrow$ он
изоморфизм.
\end{proof}

Т.о., полупрямое произведение $\langle
a\rangle_n\leftthreetimes\langle b\rangle_m$ задается гомоморфизмом
$\alpha\colon \langle b\rangle_m\to \Aut \langle a\rangle_n\simeq
\mathbb{Z}_n^*$, определяющийся образом $b$: $\alpha(b)=[k]_n$,
$k^m\equiv 1\pmod{n}$.
%----------------------------------------------------------%
%%%%%%%%%%%%%%%%%%%%%%%%%%%%%%%%%%%%%%%%%%%%%%%%%%%%%%%%%%%%
%----------------------Lecture 9---------------------------%
\lecture{}

Таким образом, полупрямое произведение $\langle
a\rangle_n\mathop{\leftthreetimes}\limits_k\langle b\rangle_m$
определяется образом $\varphi_k\in\Aut\langle a\rangle_n$ элемента
$b$. При этом должны выполняться следующие условия: $(k,n)=1$ и
$k^m\equiv 1\pmod{n}$. Произведение будет прямым $\Leftrightarrow$
$k\equiv 1\pmod{n}$.

Отсюда получается следующая формула умножения:
$(a^pb^s)(a^qb^t)\p=a^p(b^sa^qb^{-s})(b^sb^t)=a^{p+k^sq}b^{s+t}$.

\begin{ex}
\begin{enumerate}
  \item Группа диэдра $D_n=\langle a\rangle_n\leftthreetimes \langle
  b\rangle_2$. Поскольку $bab^{-1}=a^{-1}$, то $k=-1$ и $D_n=\langle a\rangle_n
  \mathop{\leftthreetimes}\limits_{-1} \langle b\rangle_2$.
\end{enumerate}
\end{ex}

\begin{note}
Может быть так, что $\langle
a\rangle_n\mathop{\leftthreetimes}\limits_{k}\langle
b\rangle_m\simeq \langle
a\rangle_n\mathop{\leftthreetimes}\limits_{k'}\langle b\rangle_m$
при $k\not\equiv k'\pmod{n}$ при другом выборе порождающего элемента
группы $\langle b\rangle_m$. А именно, при замене $b$ на $b'=b^s$,
где $(s,m)=1$, $k$ заменяется на $k^s$.
\end{note}

Рассмотрим группу порядка $pq$, где $p>q$~--- простые.

\begin{theorem}
1) Если $p\not\equiv 1\pmod{q}$, то всякая группа порядка $pq$
циклическая.

2) Если $p\equiv 1\pmod{q}$, то существуют ровно две неизоморфные
группы порядка $pq$: одна циклическая, другая неабелева.
\end{theorem}

\begin{proof}
Пусть $G_p=\langle a\rangle_p$~--- силовская $p$-подгруппа,
$G_q=\langle b\rangle_q$~--- силовская $q$-подгруппа. Тогда
$G_p\triangleleft G$, $G_p\cap G_q=\{e\}$, $G_p\cdot G_q=G$
$\Rightarrow$ $G=\langle
a\rangle_p\mathop{\leftthreetimes}\limits_k\langle b\rangle_q$, где
$(k,p)=1$ и $k^q\equiv1\pmod{p}$, т.е. $[k]^q=1$ в $\mathbb{Z}^*_p$.

1) $p\not\equiv 1\pmod{q}$. Тогда в $\Aut\langle
a\rangle_p\simeq\mathbb{Z}^*_p$ (циклическая группа порядка $p-1$)
нет элементов порядка $q$ $\Rightarrow$ $[k]_p=1$, т.е. $G=\langle
a\rangle_p\times\langle b\rangle_q$ $\Rightarrow$ $G$ циклическая.

2) $p\equiv 1\pmod{q}$. Тогда в $\Aut\langle
a\rangle_p\simeq\mathbb{Z}^*_p$ есть единственная циклическая
подгруппа порядка $q$, скажем, $\langle\varphi_k\rangle_q$. Либо
$[k]_p=1$, и тогда $G$ циклическая, либо $[k]_p\neq 1$, и тогда для
любого $[l]_p=[k]_p^s$ (где $(s,q)=1$) заменяя $b$ на $b'=b^s$,
перейдем от $k$ к $l$. В этом случае  $G\simeq\langle
a\rangle_p\mathop{\leftthreetimes}\limits_k\langle b\rangle_q
\p\simeq\langle a\rangle_p\mathop{\leftthreetimes}\limits_l\langle
b'\rangle_q$.
\end{proof}

\tema{Разрешимые группы}

Пусть $G$~--- группа. \emph{Коммутатор элементов $x,y\in G$}~--- это
элемент $(x;y)=xyx^{-1}y^{-1}$.

\svoy
\begin{enumerate}
  \item $(x;y)=e$ $\Leftrightarrow$ $xy=yx$
  \item $(y;x)=(x;y)^{-1}$.
\end{enumerate}

\emph{Коммутант группы $G$}~--- это подгруппа $G'=(G;G)$,
порожденная всеми коммутаторами, т.е. совокупность всех произведений
вида $(x_1;y_1)\p\cdot\ldots\cdot(x_n;y_n)$. $G$ абелева
$\Leftrightarrow$ $G'=\{e\}$.

Если $\varphi\colon G\to H$~--- гомоморфизм группы $G$ на группу
$H$, то $\varphi(G')\p=H'$.

\begin{theorem}
\label{1.IX}Коммутант $G'$ группы $G$~--- это наименьшая нормальная
подгруппа, фактор по которой абелев.
\end{theorem}

\begin{proof}
1) Докажем, что $G'\triangleleft G$. Коммутант $G'$ инвариантен
относительно всех автоморфизмов группы $G$, и, в частности,
относительно внутренних автоморфизмов $a(g)$, $g\in G$ $\Rightarrow$
$G'\triangleleft G$.

2) Докажем минимальность. Пусть $N\triangleleft G$ и $\pi\colon G\to
G/N$~--- канонический гомоморфизм. Тогда $G/N=A$~--- абелева
$\Leftrightarrow$ $A'=\{e\}$ $\Leftrightarrow$ $\pi(G')=\{e\}$
$\Leftrightarrow$ $G'\subseteq N$.
\end{proof}

\begin{ex}
\begin{enumerate}
  \item $S_3'\subset A_3$, но $S_3'\neq\{e\}$, т.к. $S_3$ неабелева
  $\Rightarrow$ $S_3'=A_3$.
  \item $S_4'\subset A_4$, $S_4'\neq\{e\}$ и $S_4'\supset S_3'=A_3$
  $\Rightarrow$ $S_4'$ содержит все тройные циклы $\Rightarrow$ $|S_4'|\geqslant
  9$ $\Rightarrow$ $S_4'=A_4$.
  \item $V_4\triangleleft A_4$, $A_4/V_4$ циклическая порядка 3
  $\Rightarrow$ $A_4'\subset V_4$, $A_4'\neq\{e\}$. Пусть $A_4'\ni
  (12)(34)$. Но все произведения двух нетривиальных транспозиций
  сопряжены в $A_4$ $\Rightarrow$ $A_4'=V_4$.
\end{enumerate}
\end{ex}

\begin{lemma}
При любом $n$ $A_n$ порождается тройными циклами, а при $n\geqslant
5$~--- также произведениями пар независимых транспозиций.
\end{lemma}

\begin{proof}
Т.к. группа $S_n$ порождается транспозициями, то группа $A_n$
порождается произведениями пар транспозиций. Но $(ij)(jk)\p=(ijk)$,
$(ij)(kl)=(ijk)(jkl)$. Значит, $A_n$ порождается тройными циклами.
Аналогично, при $n\geqslant 5$ $(ij)(jk)=[(ij)(lm)][(jk)(lm)]$, и
$A_n$ порождается произведениями пар независимых транспозиций.
\end{proof}

\begin{theorem}
$S_n'=A_n$, при $n\geqslant 5$ $A_n'=A_n$.
\end{theorem}

\begin{proof}
$S_n'\subset A_n$, $S_n'\supset S_3'=A_3$ $\Rightarrow$ $S_n'$
содержит все тройные циклы $\Rightarrow$ $S_n'=A_n$.

При $n\geqslant 5$ $A_n'\supset A_4=V_4$ $\Rightarrow$ $A_n'$
содержит все произведения пар независимых транспозиций $\Rightarrow$
$A_n'=A_n$.
\end{proof}

\begin{note}
Все произведения пар независимых транспозиций сопряжены не только в
$S_n$ но и в $A_n$:
\begin{equation*}
  \forall \, i,j,k,l\quad
  (ij)(kl)=\tau((12)(34))\tau^{-1}
\end{equation*}
Если $\tau$ четна, то все
доказано. Если $\tau$ нечетна, то заменим $\tau$ на
$\tau'=\tau(12)$. Тогда
\begin{equation*}
  \tau'((12)(34))\tau'^{-1}=\tau((12)(34))\tau^{-1}
\end{equation*}
\end{note}

\begin{lemma}
\label{2.IX}Группа $\mathrm{SL}_n(K)$ порождается элементарными
матрицами первого типа.
\end{lemma}

\begin{proof}
Пусть $\det A=1$. Докажем, что матрицу $A$ можно привести к $E$ с
помощью элементарных преобразований строк первого типа. Вначале
сделаем $a_{11}=1$. Если $a_{i1}\neq 0$, то добавим к первой строке
$i$-ю строку с подходящим коэффициентом.

Если все $a_{i1}=0$ при $i>1$, то $a_{11}\neq 0$ и, прибавив ко
второй строке первую, придем к предыдущему случаю.

Пусть теперь $a_{11}=1$. Вычитаем из всех строк первую с подходящими
коэффициентами, получаем, что $a_{i1}=0$ при $i>1$.

Аналогично $A$ приводится к унитреугольному виду. Дальше~---
обратный ход метода Гаусса.
\end{proof}
%----------------------------------------------------------%
%%%%%%%%%%%%%%%%%%%%%%%%%%%%%%%%%%%%%%%%%%%%%%%%%%%%%%%%%%%%
%----------------------Lecture 10---------------------------%
\lecture{}

\begin{theorem}
При $|K|>3$ $\mathrm{GL}_n(K)'=\mathrm{SL}_n(K)=\mathrm{SL}_n(K)'$.
\end{theorem}

\begin{proof}
Во-первых, $\mathrm{GL}_n(K)/\mathrm{SL}_n(K)\simeq K^*$~---
абелева, поэтому $\mathrm{GL}_n(K)'\subset \mathrm{SL}_n(K)$.

Во-вторых, $\left(\bigl(\begin{smallmatrix}\lambda & 0\\
0 & \lambda^{-1}\end{smallmatrix}\bigr); \bigl(\begin{smallmatrix}1
& c\\ 0 & 1\end{smallmatrix}\bigr)\right)=\bigl(\begin{smallmatrix}
1& (\lambda^2-1)c\\ 0& 1\end{smallmatrix}\bigr)$. Если $|K|>3$, то
беря $\lambda\p\neq 0, \pm1$ и подходящее $c$, можно получить любую
матрицу вида $\bigl(\begin{smallmatrix}1& a\\
0& 1\end{smallmatrix}\bigr)$.

$\forall \, n\geqslant 2$, $\forall \, i,j\in\{1,\ldots,n\}$, $i\neq
j$ имеется вложение $\mathrm{SL}_2(K)\hookrightarrow
\mathrm{SL}_n(K)$:
$$\begin{pmatrix}a& b\\ c& d\end{pmatrix}\mapsto
\left(
\begin{smallmatrix}
1 & & & & & & & & \\
  &\ddots & & & & & & & \\
 & & 1& & & & & & \\
 & & &a &\cdots & c & \\
 & & &\vdots & \ddots &\vdots & & & \\
 & & &d &\cdots & b& & & \\
 & & & & & & 1& & \\
 & & & & & & & \ddots &\\
 & & & & & & & & 1
\end{smallmatrix}\right)
$$

Из предыдущего вычисления следует, что $E+aE_{ij}\in SK_n(K)'$. По
лемме~\ref{2.IX} получаем, что $\mathrm{SL}_n(K)=\mathrm{SL}_n(K)$.
Т.к. $\mathrm{GL}_n(K)'\supset \mathrm{SL}_n(K)'\p=\mathrm{SL}_n(K)$
и $\mathrm{GL}_n(K)'\subset \mathrm{SL}_n(K)$, то
$\mathrm{GL}_n(K)'=\mathrm{SL}_n(K)$.
\end{proof}

\emph{Кратные коммутанты} $G^{(n)}$ определяются по индуктивному
правилу: $G^{(0)}=G$, $G^{(1)}=G'$, $G^{(n+1)}=(G^{(n)})'$. Если
$\varphi\colon G\stackrel{\text{на}}{\to} H$, то
$\varphi(G^{(n)})\p=H^{(n)}$. Отсюда следует, что $\forall \, n\quad
G^{(n)}\triangleleft G$.

Группа $G$ называется \emph{разрешимой}, если $\exists \,
n\in\mathbb{N}: G^{(n)}=\{e\}$.

\begin{ex}
\begin{enumerate}
  \item $S_n$ разрешима $\Leftrightarrow$ $n\leqslant 4$ ($S_4^{(3)}
  =\{e\}$, $S_3^{(2)}=\{e\}$, $S_2'=\{e\}$).
  \item $\mathrm{GL}_n(K)$ не разрешима при $n\geqslant 2$ и $|K|>3$.
\end{enumerate}
\end{ex}

\svoy
\begin{enumerate}
  \item $G$ разрешима $\Rightarrow$ всякая подгруппа $H\subset G$ и
  всякая факторгруппа $G/N$ разрешима: $G^{(n)}=\{e\}$ $\Rightarrow$
  $H^{(n)}=\{e\}$; пусть $\pi\colon G\to G/N$~--- канонический
  гомоморфизм, тогда $(G/N)^{(n)}=\pi(G^{(n)})=\pi(e)=e$.
  \item Если нормальная подгруппа $N\triangleleft G$ и факторгруппа
  $G/N$ разрешимы, то и группа $G$ разрешима: пусть $N^{(k)}=\{e\}$
  и $(G/N)^{(l)}=\{e\}$, тогда $\pi(G)=(G/N)^{(l)}=\{e\}$
  $\Rightarrow$ $G^{(l)}\subset N$ $\Rightarrow$ $G^{l+k}\subset
  N^{(k)}=\{e\}$.
\end{enumerate}

\begin{theorem}
Всякая $p$-примарная конечная группа разрешима.
\end{theorem}

\begin{proof}
Индукция по $n$. При $n=1$~--- очевидно. Пусть $n>1$, тогда
$Z(G)\neq \{e\}$~--- абелева (а значит, и разрешимая), $G/Z(G)$
разрешима по предположению индукции.
\end{proof}

\begin{theorem}
Группа $\mathrm{B}_n(K)$ треугольных матриц порядка $n$ над полем
$K$ разрешима.
\end{theorem}

\begin{proof}
Рассмотрим гомоморфизм $\varphi\colon \mathrm{B}_n(K)\to (K^*)^n$:
$$
\begin{pmatrix}
\lambda_1 &* &* \\
\vdots& \ddots & *\\
0&\cdots & \lambda_n
\end{pmatrix}\mapsto (\lambda_1,\ldots,\lambda_n),
$$
причем группа $(K^*)^n$ абелева. $\ker\varphi=\mathrm{U}_n(K)$. Если
$\mathrm{U}_n(K)$ разрешима, то и $\mathrm{B}_n(K)$ разрешима.

Докажем разрешимость группы $\mathrm{U}_n(K)$ индукцией по $n$. При
$n=1$~--- очевидно. При $n>1$ рассмотрим гомоморфизм $\psi\colon
\mathrm{U}_n(K)\to \mathrm{U}_{n-1}(K)$,
$$
\begin{pmatrix}
1 &* &* &\vdots\\
\vdots& \ddots & *&\vdots\\
0&\cdots & 1&\vdots\\
\cdots& 0 &\cdots & 1
\end{pmatrix}\mapsto
\begin{pmatrix}
1 &* &* \\
\vdots& \ddots & *\\
0&\cdots & 1
\end{pmatrix}.$$

Очевидно, что
$$\ker\psi=
\begin{pmatrix}
1 &\cdots &0 &a_1\\
\vdots& \ddots & \vdots&\vdots\\
0&\cdots & 1&a_{n-1}\\
\cdots& 0 &\cdots & 1
\end{pmatrix}\simeq (K^*)^{n-1}
$$
---абелева группа. $\mathrm{U}_n(K)/\ker\psi\simeq \mathrm{U}_{n-1}(K)$~--- разрешима
по предположению индукции. Значит, $\mathrm{U}_n(K)$ разрешима.
\end{proof}

\tema{Простые группы}

Группа $G$ называется \emph{простой}, если она не содержит
нетривиальных нормальных подгрупп.

Простая группа $G$ разрешима $\Leftrightarrow$ $G$~--- циклическая
группа простого порядка.

Существуют некоммутативные простые группы.

\begin{lemma}
\label{1.X}Если $G$~--- конечная группа и $p\mid|G|$, то существует
элемент $g\in G$ порядка $p$.
\end{lemma}

\begin{proof}
Возьмем нетривиальную силовскую $p$-подгруппу $S\p\subset G$. Тогда
$\forall \, g\in S, g\neq e\quad \ord g=p^k$ и $\ord g^{p^{k-1}}=p$.
\end{proof}

\begin{theorem}
Группа $A_5$ проста.
\end{theorem}

\begin{proof}
Поскольку $|A_5|=2^2\cdot 3\cdot 5$, то все элементы, не равные $e$,
имеют порядок 2, 3 или 5. Пусть $N$~--- нетривиальная нормальная
подгруппа.

1) Если $2\mid|N|$, то по лемме~\ref{1.X} $N$ содержит элемент
порядка 2 $\Rightarrow$ $N$ содержит все транспозиции вида
$(ij)(kl)$ $\Rightarrow$ $N=A_5$~--- противоречие.

2) Если $3\mid|N|$, то по лемме~\ref{1.X} $N$ содержит тройной цикл
$\Rightarrow$ $N$ содержит все тройные циклы $\Rightarrow$
$N=A_5$~--- противоречие.

3) Если $|N|=5$, то $N=\langle(ijklm)\rangle$~--- силовская
5-подгруппа. Но в $A_5$ силовская 5-подгруппа не единственна, а
значит, не нормальна~--- противоречие.
\end{proof}

\begin{note}
Можно доказать, что не существует некоммутативных простых групп
порядка меньше 60. Более того. всякая группа порядка меньше 60
разрешима. Группа
$\mathrm{P\mathrm{SL}}_n(K)=\mathrm{SL}_n(K)/\{\lambda E:
\lambda^n=1\}$ проста, кроме случая $n=2$, $|K|=2,3$.
\end{note}
%----------------------------------------------------------%
%%%%%%%%%%%%%%%%%%%%%%%%%%%%%%%%%%%%%%%%%%%%%%%%%%%%%%%%%%%%
%----------------------Lecture 11---------------------------%
\lecture{Линейные представления групп.}

\emph{Линейным представлением группы $G$ в векторном пространстве
$V$} называется всякий гомоморфизм $R\colon G\to \mathrm{GL}(V)$.
Пространство $V$ называется \emph{пространством представления}, а
его размерность~--- \emph{размерностью представления}.

\emph{Матричным представлением группы $G$} называется всякий
гомоморфизм $R\colon G\to \mathrm{GL}_n(K)$ ($K$~--- поле).

Всякую матрицу $A\in \mathrm{GL}_n(K)$ можно рассматривать как
линейный оператор $X\mapsto AX$ в пространстве $K^n$.
Соответственно, всякое матричное представление можно рассматривать
как линейное представление в пространстве $K^n$.

Обратно, если $R\colon G\to \mathrm{GL}(V)$~--- линейное
представление и $e=(e_1,\p\ldots,e_n)$~--- базис пространства $V$,
то, записывая линейные операторы $R(g)$ матрицами в базисе $e$,
получим следующее матричное представление: $R_e\colon G\p\to
\mathrm{GL}_n(K)$.

При переходе от старого базиса к новому $e'=eC$, получаем другое
матричное представление $R_{e'}$, связанное с $R_e$ формулой
$R_{e'}(g)\p=C^{-1}R_e(g)C$.

Линейные представления одной и той же группы $R\colon G\to
\mathrm{GL}(V)$ и $S\colon G\to \mathrm{GL}(U)$ \emph{изоморфны},
если есть такой изоморфизм $\varphi\colon V\to U$ векторных
пространств, что $\forall \, g\in G$ \quad $\varphi
R(g)=S(g)\varphi$, т.е. следующая диаграмма коммутативна:
$$\xymatrix{
 V \ar[r]^{R(g)} \ar[d]_\varphi & V \ar[d]^\varphi\\
 U \ar[r]_{S(g)} & U}$$

Пусть $e=(e_1,\ldots,e_n)$~--- базис $V$. Тогда
$\varphi(e)=(\varphi(e_1),\ldots,\varphi(e_n))$~--- базис $U$.
Условие коммутативности диаграммы означает, что
$R_e(g)\p=S_{\varphi(e)}(g)\quad \forall \, g\in G$.

\begin{ex}
\begin{enumerate}
  \item $G=\mathbb{R}$, $R_1(t)=\left(\begin{smallmatrix}\cos t& -\sin t\\ \sin t& \cos
  t\end{smallmatrix}\right)$, $R_2(t)=\left(\begin{smallmatrix}\ch t& \sh t\\ \sh t& \ch
  t\end{smallmatrix}\right)$, $R_3(t)=\left(\begin{smallmatrix} e^t& 0\\ 0&
  e^{-t}\end{smallmatrix}\right)$, $R_4(t)=\left(\begin{smallmatrix} 1& t\\ 0 &
  1\end{smallmatrix}\right)$. $R_2\simeq R_3$.
  \item $G=S_4$, $R_1\colon S_4\widetilde{\to}\Sym T\subset
  \mathrm{GL}(\mathbb{E}^3)$, $R_2\colon S_4\widetilde{\to} \Sym_+K\subset
  \mathrm{GL}(\mathbb{E}^3)$. $\det R_2(g)\equiv 1\quad \forall \, g$, $\det
  R_1(g)\neq1$ при $g\not\in A_4$ $\Rightarrow$ $R_1\not\simeq R_2$.
  \item $G=S_3$, $R\colon S_3\widetilde{\to}\Sym\triangle\subset
  \mathrm{GL}(\mathbb{E}^2)$.
  \item $G=S_4$, $S\colon S_4\to S_3\widetilde{\to}\Sym\triangle\subset
  \mathrm{GL}(\mathbb{E}^2)$.
  \item Одномерные представления~--- гомоморфизмы $G\to K^*$. В
  частности, $\det\colon \mathrm{GL}_n(K)\to K^*$, $\sgn\colon S_n\to
  \{\pm1\}$.
  \item Тривиальные представления: $I\colon G\to \mathrm{GL}(V)$, $I(g)=\mathcal{E}\quad \forall \, g\in
  G$.
\end{enumerate}
\end{ex}

\emph{Расширение поля} $K\subset L$ (например,
$\mathbb{R}\subset\mathbb{C}$), $R\colon G\to
\mathrm{GL}_n(K)\p\subset \mathrm{GL}_n(L)$.

\emph{Сумма представлений} $R\colon G\to \mathrm{GL}(V)$ и $S\colon
G\to \mathrm{GL}(U)$~--- это представление $R+S\colon G\to
\mathrm{GL}(V\oplus U)$, определяемое по следующим формулам:
$(R\p+S)(g)(v,u)=(R(g)v, S(g)u)$ или
$(R+S)(g)=\left(\begin{smallmatrix} R(g) & 0\\
0 & S(g)\end{smallmatrix}\right)$.

\begin{ex}
\begin{enumerate}
  \item Пусть $R_3\colon\mathbb{R}\to \mathrm{GL}(\mathbb{R})$, тогда оно
  является суммой двух представлений $t\mapsto e^t$, $t\mapsto e^{-t}$.
\end{enumerate}
\end{ex}
Пусть $R\colon G\to \mathrm{GL}(V)$~--- некоторое представление
Подпространство $U\subset V$ называется \emph{инвариантным
относительно представления $R$}, если оно инвариантно относительно
всех операторов $R(g)$, $g\in G$, т.е. $R(g)u\in U\quad \forall \,
u\in U, g\in G$. В матричной форме (в базисе пространства $V$,
согласованным с $U$) это означает, что
$R(g)\p=\left(\begin{smallmatrix}* & *\\ 0 &
*\end{smallmatrix}\right)\quad \forall \, g\in G$.

Если $U$ инвариантно, то можно рассматривать ограничение
представления $R$ на $U$: $R_U(g)u\p=R(g)u\quad\forall \, u\in U$. В
матричной форме $R(g)\p=\left(\begin{smallmatrix} R_U(g) & *\\ 0 &
*\end{smallmatrix}\right)$.

Если $V=U\oplus W$, $U$, $W$~--- инвариантные подпространства, то
$R(g)\p=\left(\begin{smallmatrix} R_U(g) & 0\\ 0 &
 R_W(g)\end{smallmatrix}\right)$, т.е. $R\simeq R_U+R_W$.

Линейное представление $R\colon G\to \mathrm{GL}(V)$ называется
\emph{неприводимым}, если в $V$ нет нетривиальных инвариантных
подпространств.

Линейное представление $R\colon G\to \mathrm{GL}(V)$ называется
\emph{вполне приводимым}, если для всякого инвариантного
подпространства $U\subset V$ существует инвариантное дополнительное
подпространство $W\subset V$.

Всякое неприводимое представление вполне приводимо.

Всякое одномерное представление неприводимо.

Неприводимое представление может стать приводимым после расширения
поля.

\begin{ex}
\begin{enumerate}
  \item $G=\mathbb{R}$. Представление $R_1$ неприводимо над
  $\mathbb{R}$, но разлагается в сумму двух одномерных представлений
  над $\mathbb{C}$: $R_1=\left(\begin{smallmatrix} e^{it} & 0\\ 0 &
  e^{-it}\end{smallmatrix}\right)$ в базисе $(e_1-ie_2, e_1+ie_2)$.

  Представление $R_3$ разлагается в сумму двух одномерных над
  $\mathbb{R}$.

  Представление $R_4$ приводимо, но не вполне приводимо.
  \item $G=S_4$. Представления $R_1$ и $R_2$ неприводимы, т.к. у них
  повороты на $120^\circ$ вокруг осей, проходящих через вершины, не
  имеют 1-мерных инвариантных подпространств.
\end{enumerate}
\end{ex}
%----------------------------------------------------------%
%%%%%%%%%%%%%%%%%%%%%%%%%%%%%%%%%%%%%%%%%%%%%%%%%%%%%%%%%%%%
%----------------------Lecture 12---------------------------%
\lecture{}

\begin{theorem}
Ограничение вполне приводимого представления на инвариантное
подпространство $U$ также вполне приводимо.
\end{theorem}

\begin{proof}
Пусть $U_1\subset U$~--- инвариантное подпространство. Тогда
существует инвариантное подпространство $V_2\subset V$: $V=U_1\oplus
V_2$. Рассмотрим инвариантное подпространство $U_2=V_2\cap U$.
Докажем, что $U=U_1\oplus U_2$.

1) $U_1\cap U_2\subset U_1\cap V_2=0$.

2) $\forall \, u\in U\quad u=u_1+u_2$, где $u_1\in U_1$, $u_2\in
V_2$. Но $u_2=u-u_1\in U$ $\Rightarrow$ $u_2\in U_2$.
\end{proof}

\begin{theorem}
Линейное представление $R\colon G\to \mathrm{GL}(V)$ является вполне
приводимым $\Leftrightarrow$ оно раскладывается в сумму неприводимых
представлений.
\end{theorem}

\begin{proof}
1) Пусть $R$ вполне приводимо. Пусть $0\neq V_1\subset V$~---
минимальное инвариантное подпространство. Тогда $R|_{V_1}=R_1$
неприводимо. Существует инвариантное дополнение~--- подпространство
$V_1'$: $V\p=V_1\oplus V_1'$. Пусть $0\neq V_2\subset V_1'$~---
минимальное инвариантное подпространство и $V_2'$~--- инвариантное
дополнительное подпространство: $V_1'=V_2\oplus V_2'$. Тогда
$R|_{V_2}=R_2$ неприводимо, и т.д. В конце концов мы получим сумму
минимальных инвариантных подпространств: $V=V_1\oplus\ldots\oplus
V_s$. Это означает, что $R=R_1+\ldots+R_s$, где $R_i=R|_{V_i}$~---
неприводимые представления.

2) Обратно, пусть $R$ разлагается в сумму неприводимых
представлений. Это означает, что пространство $V$ разлагается в
прямую сумму минимальных инвариантных подпространств:
$V=V_1\oplus\ldots\oplus V_s$.

Пусть $U\subset V$~--- инвариантное подпространство. Будем искать
дополнительное инвариантное подпространство в виде суммы некоторых
из $V_1,\ldots, V_s$. Для всякого $I\subset\{1,\ldots,s\}$ положим
$V_I=\bigoplus\limits_{i\in I}V_i$. Пусть $I$~--- максимальное
подмножество, для которого $U\cap V_I=0$. Докажем, что $V=U\oplus
V_I$. По построению $U\cap V_I=0$. $\forall \, j\not\in I\quad U\cap
V_{I\cup\{j\}}\neq 0$, т.е. $\exists \, u\in U: u=\sum\limits_{i\in
I}v_i+v_j$, где $v_i\in V_I$, $v_j\in V_j$. Тогда
$v_j=u-\sum\limits_{i\in I}v_i\in U\oplus V_i$. Значит, $V_j\cap
(U\oplus V_i)\neq 0$. Т.к. $V_j$~--- минимальное инвариантное
подпространство, то $V_j\subset U\oplus V_I$. Значит, $V=U\oplus
V_I$.
\end{proof}

\begin{ex}
\begin{enumerate}
  \item $G=\mathbb{Z}$, $R\colon \mathbb{Z}\to \mathrm{GL}(V)$, $R(1)=\mathcal{A}\in
  \mathrm{GL}(V)$ $\Rightarrow$ $R(k)=\mathcal{A}^k$. Т.о., представление $R$
  определяется однозначно линейным оператором $\mathcal{A}$.
  Обратно, $\forall \, \mathcal{A}\in \mathrm{GL}(V)$ формула
  $R(k)=\mathcal{A}^k$ определяет линейное представление группы
  $\mathbb{Z}$.

  Если $R$~--- комплексное представление группы $\mathbb{Z}$, то его
  неприводимость означает, что $\dim V=1$, а полная приводимость~---
  что $R$ есть сумма одномерных представлений, т.е. матрица
  оператора $\mathcal{A}$ приводится к диагональному виду.
\end{enumerate}
\end{ex}

\begin{theorem}
Всякое линейное представление $R\colon G\to \mathrm{GL}(V)$ конечной
группы $G$ над полем характеристики 0 вполне приводимо.
\end{theorem}

\begin{proof}
Пусть $U\subset V$~--- инвариантное подпространство и $W\p\subset
V$~--- дополнительное подпространство к $U$: $V=U\oplus W$. Пусть
$\mathcal{P}$~--- проектор на $U$ параллельно $W$, т.е. $\forall \,
v=u+w$, где $u\in U$, $w\in W$ \quad $\mathcal{P}v=u$. Рассмотрим
\emph{усреднение} проектора $\mathcal{P}$ по группе $G$:
$$\mathcal{P}_0=\frac{1}{|G|}\sum\limits_{g\in
G}R(g)\mathcal{P}R(g)^{-1}.$$

Докажем некоторые свойства оператора $\mathcal{P}_0$.

1) $\forall \, u\in U\quad \mathcal{P}_0u=u$: т.к. $R(g)^{-1}u\in
U$, то $\mathcal{P}R(g)^{-1}u=R(g)^{-1}u$ и
$\mathcal{P}_0u=\frac{1}{|G|}\sum\limits_{g\in G}R(g)R(g)^{-1}u=u$.

2) $\forall \, v\in V\quad \mathcal{P}_0v\in U$: т.к.
$\mathcal{P}R(g)^{-1}v\in U$, то
$$\mathcal{P}_0v=\frac{1}{|G|}\sum\limits_{g\in
G}R(g)\mathcal{P}R(g)^{-1}v\in U.$$

Положим $W_0=\ker \mathcal{P}_0$. Тогда

1) $U\cap W_0=0$

2) $U+W_0=V$: $\forall \, v\in V\quad
v=\mathcal{P}_0v+(v-\mathcal{P}_0v)$.

Таким образом, $V=U\oplus W_0$. Покажем, что $W_0$ инвариантно.
Пусть $w\in W_0$, $h\in G$. Тогда
$$
\begin{aligned}
\mathcal{P}_0R(h)w &=\frac{1}{|G|}\sum\limits_{g\in
G}R(g)\mathcal{P}R(g)^{-1}R(h)w=\\
&=\frac{1}{|G|}R(h)\sum\limits_{g\in
G}(R(h)^{-1}R(g))\mathcal{P}(R(g)^{-1}R(h))w=\\
&=\frac{1}{|G|}R(h)\sum\limits_{g\in
G}R(h^{-1}g)\mathcal{P}R(h^{-1}g)^{-1}w=\\
&=\frac{1}{|G|}R(h)\sum\limits_{g\in
G}R(g)\mathcal{P}R(g)^{-1}w=R(h)\mathcal{P}_0w=0.
\end{aligned}$$
\end{proof}

\begin{ex}
\begin{enumerate}
  \item Докажем, что два трехмерных представлений группы $S_4$
  неприводимы над $\mathbb{C}$.

  $R_{1,2}\colon S_4\to \mathrm{GL}(\mathbb{E}^3)$. Если существует двумерное
  комплексное инвариантное подпространство, то в силу полной
  приводимости есть и одномерное инвариантное подпространство,
  например. $\langle z=x+iy\rangle$. Тогда $\forall \, g\in G\quad
  R(g)z=(\lambda+i\mu)z$, где $\lambda,\mu\in \mathbb{R}$, т.е. $R(g)x=\lambda x-\mu
  y$, $R(g)y=\mu x+\lambda y$ $\Rightarrow$ вещественное
  подпространство $\langle x,y\rangle$ инвариантно~--- противоречие.
  \item \emph{Мономиальное представление группы $S_n$}. Пусть
  $V$~--- векторное пространство с базисом $\{e_1,\ldots,e_n\}$ (т.е. $\dim V=n$).
  Определим представление $M\colon S_n\to \mathrm{GL}(V)$ по правилу
  $R(\sigma)e_i=e_{\sigma(i)}$. Подпространства $\langle
  e_1+\ldots+e_n\rangle$ и $V_0=\Big\{\sum\limits_{i=1}^n x_ie_i: \sum\limits_{i=1}^n
  x_i=0\Big\}$ являются инвариантными и взаимно дополнительными.
  Тогда $M$ раскладывается в сумму одномерного представления и
  $(n-1)$-мерного представления $M_0=M|_{V_0}$. Докажем, что оно
  неприводимо.

  Пусть $U\subset V_0$~--- инвариантное подпространство. Возьмем $0\neq u\p=\sum\limits_i x_ie_i\in
  U$. Т.к. мы можем переставлять координаты, то можно считать. что
  $X_1\neq x_2$. Тогда $R((12))u-u=(x_2-x_1)(e_1-e_2)\in U$
  $\Rightarrow$ $e_1-e_2\in U$ $\Rightarrow$ $e_i=e_j\in U\quad \forall \,
  i,j$ $\Rightarrow$ $V_0=U$. В частности при $n=4$ $M_0=R_1$.
\end{enumerate}
\end{ex}
%----------------------------------------------------------%
%%%%%%%%%%%%%%%%%%%%%%%%%%%%%%%%%%%%%%%%%%%%%%%%%%%%%%%%%%%%
%----------------------Lecture 13---------------------------%
\lecture{}

\begin{theorem}[Лемма Шура]
Пусть $R\colon G\to \mathrm{GL}(V)$~--- неприводимое комплексное
линейное представление группы $G$. Тогда всякий линейный оператор
$\mathcal{A}$ в пространстве $V$, перестановочный со всеми
операторами $R(g)$ (где $g\in G$), скалярен.
\end{theorem}

\begin{proof}
Пусть $\lambda$~--- собственное значение оператора $\mathcal{A}$ и
$V_\lambda\p=\{v\in V: \mathcal{A}v=\lambda v\}$. Тогда $V_\lambda$
инвариантно относительно всех операторов представления: $\forall \,
v\in V_\lambda$ \quad $\mathcal{A}R(g)v=R(g)\mathcal{A}v=\lambda
R(g)v$ $\Rightarrow$ $V_\lambda=V$, т.е.
$\mathcal{A}=\lambda\mathcal{E}$.
\end{proof}

\begin{imp}
Всякое неприводимое комплексное представление абелевой группы
одномерно.
\end{imp}

\begin{proof}
Пусть $G$~--- абелева группа и $R\colon G\to \mathrm{GL}(V)$~---
неприводимое комплексное представление. Тогда $$\forall \, g,h\in
G\quad R(g)R(h)=R(gh)=R(hg)=R(h)R(g),$$ т.е. $R(h)$ перестановочен
со всеми операторами представления, и по лемме Шура $R(h)$~---
скалярный оператор.
\end{proof}

Опишем все комплексные линейные представления конечных абелевых
групп.

Т.к. всякое представление есть сумма неприводимых, а всякое
неприводимое представление одномерно, то достаточно описать
одномерные представления.

Пусть $G=\langle a_1\rangle_{n_1}\times\ldots\times \langle
a_s\rangle_{n_s}$. Одномерное представление есть гомоморфизм
$R\colon G\to \mathbb{C}^*$. Оно определяется числами
$R(a_1)=\varepsilon_1$, \ldots, $R(a_s)=\varepsilon_s$, т.к.
$R(a_1^{k_1}\ldots a_s^{k_s})$. Далее, т.к. $a_i^{n_i}=e$, то должно
быть $\varepsilon_i^{n_i}=1$. Обратно, если
$\varepsilon_1,\ldots,\varepsilon_s$ удовлетворяют этим условиям, то
предыдущая формула определяет одномерное представление группы $G$.
Т.о., получается $n_1\ldots n_s=|G|$ представлений.

\begin{theorem}
Пусть $R$~--- одномерное представление группы $G$ и $\pi\colon
G\p\to G/(G,G)$~--- канонический гомоморфизм. Тогда существует такое
одномерное представление $\bar{R}$ группы $G/(G,G)$, что
$R=\bar{R}\circ\pi$.
\end{theorem}

\begin{proof}
Очевидно, что если $\bar{R}$~--- одномерное представление группы
$G/(G,G)$, то $R=\bar{R}\circ \pi$~--- одномерное представление
группы $G$.

Докажем, что $(G,G)\subset\ker R$: $R((g,h))=(R(g),R(h))=1$.
Следовательно, все элементы каждого смежного класса $g(G,G)$ при
представлении $R$ переходят в одно и то же число. Значит, $\exists
\, \bar{R}: G/(G,G)\to K^*$: $R=\bar{R}\circ\pi$. А именно,
$\bar{R}(g(G,G))=R(g)$.

Отображение $\bar{R}$~--- гомоморфизм: $\bar{R}(g(G,G)\cdot
h(G,G))=\bar{R}(gh(G,G))\p=R(gh)=R(g)R(h)=\bar{R}(g(G,G))\cdot
\bar{R}(h(G,G))$.
\end{proof}

\begin{ex}
\begin{enumerate}
  \item $G=S_n$, $(G,G)=A_n$, $S_n/A_n\simeq C_2$. Значит, группа
  $S_n$ имеет два одномерных комплексных представления: тривиальное
  и $\sgn$.
  \item $G=D_n=\langle a,b\rangle$, где $a$~--- поворот на угол
  $\frac{2\pi}{n}$, $b$~--- отражение. Элементы $a$ и $b$
  удовлетворяют соотношениям $a^n=e$, $b^2=e$, $(ab)^2=e$. Значит,
  $(a,b)=a^2$ и $(G,G)\supset \langle a^2\rangle$. Если $n$ четно,
  то $\ord a^2=n/2$ $\Rightarrow$ $|D_n.\langle a^2\rangle|=4$
  $\Rightarrow$ $D_n/\langle a^2\rangle$ абелева $\Rightarrow$ $(G,G)=\langle
  a^2\rangle$. Если $n$ нечетно, то $\ord a^2=n$ $\Rightarrow$ $\langle
  a^2\rangle=C_n$ $\Rightarrow$ $(G,G)=C_n$. Т.о., группа $D_n$
  имеет 4 одномерных представления, если $n$ четно, и 2, если $n$
  нечетно.
\end{enumerate}
\end{ex}
Опишем все неприводимые комплексные представления группы $D_n$.
Заметим, что всякий элемент группы $D_n$ представляется в виде $a^k$
или $a^kb$, причем этот вид определен однозначно с точностью до
прибавления к $k$ целого кратного $n$.

Пусть $R\colon D_n\to \mathrm{GL}(V)$~--- неприводимое комплексное
представление, $\dim V>1$. Положим $R(a)=\mathcal{A}$,
$R(b)=\mathcal{B}$. Операторы $\mathcal{A}$ и $\mathcal{B}$
удовлетворяют соотношениям $\mathcal{A}^n=\mathcal{E}$,
$\mathcal{B}^2=\mathcal{E}$, $\mathcal{BAB}=\mathcal{A}^{-1}$.
Обратно, формулы $R(a^k)=\mathcal{A}^k$,
$R(a^kb)=\mathcal{A}^k\mathcal{B}$ определяют представление группы
$D_n$ в пространстве $D_n$: $R(a^k\cdot
a^l)=R(a^{k+l})=\mathcal{A}^{k+l}=\mathcal{A}^k\cdot\mathcal{A}^l=R(a^k)\cdot
R(a^l)$, $R(a^k\cdot
a^lb)=\mathcal{A}^{k+l}\mathcal{B}=R(a^{k+l}b)=R(a^k)\cdot R(a^lb)$,
$R(a^kb\cdot
a^l)=R(a^k(ba^lb^{-1})b)=R(a^{k-l}b)=\mathcal{A}^{k-l}\mathcal{B}=\mathcal{A}^k\mathcal{B}\cdot
\mathcal{A}^l=R(a^kb)\cdot R(a^l)$, $R(a^kb\cdot a^lb)=R(a^kb)\cdot
R(a^lb)$.

Пусть $e\in V$~--- собственный вектор оператора $\mathcal{A}$:
$\mathcal{A}e=\lambda e$. Положим $f=\mathcal{B}e$. Заметим, что $e$
и $f$ не коллинеарны, т.к. иначе $\langle e\rangle$ инвариантно и
$V=\langle e\rangle$ одномерно. Далее,
$\mathcal{A}f=\mathcal{AB}e=\mathcal{BA}^{-1}e=\lambda^{-1}\mathcal{B}e=\lambda^{-1}f$,
т.е. $f$~--- собственный вектор оператора $\mathcal{A}$.
$\mathcal{B}f=\mathcal{B}^2e=e$ $\Rightarrow$ подпространство
$\langle e,f\rangle$ инвариантно $\Rightarrow$ $V=\langle
e,f\rangle$. В базисе $\{e,f\}$
$\mathcal{A}=\left(\begin{smallmatrix} \lambda & 0\\
0 & \lambda^{-1} \end{smallmatrix}\right)$, $\mathcal{B}=\left(\begin{smallmatrix} 0 & 1\\
1 & 0 \end{smallmatrix}\right)$. При этом $\lambda^n=1$ и
$\lambda\neq \pm$, т.к. иначе $\lambda=\lambda^{-1}$ и
подпространство $\langle e+f\rangle$ инвариантно.

Построенное таким образом неприводимое двумерное представление
группы $D_n$ обозначим $R_\lambda$.

Очевидно, что $R_\lambda\simeq R_\mu$ $\Leftrightarrow$
$\mu=\lambda^{\pm1}$. Т.о., получается $\frac{n-2}{2}$ двумерных
неприводимых представлений при четном $n$ и $\frac{n-1}{2}$ при
нечетном $n$.

\tema{Морфизмы представлений}

\emph{Морфизмом} представления $R\colon G\to \mathrm{GL}(V)$ в
представление $S\colon G\p\to \mathrm{GL}(U)$ называется всякое
линейное отображение $f\colon V\to U$, для которого коммутативна
следующая диаграмма:
$$\xymatrix{
 V \ar[r]^{R(g)} \ar[d]_f & V \ar[d]^f\\
 U \ar[r]_{S(g)} & U}$$

Все линейные отображения $f\colon V\to U$ образуют векторное
пространство, а морфизмы представлений образуют в нем
подпространство, обозначаемое $\Mor(R,S)$.

\begin{prop}
$\Mor(R_1+R_2,S)\simeq \Mor(R_1,S)\oplus\Mor(R_2,S)$.
\end{prop}

\begin{proof}
Пусть $R_1\colon G\to \mathrm{GL}(V_1)$, $R_2\colon G\to
\mathrm{GL}(V_2)$, $S\colon G\p\to \mathrm{GL}(U)$. Тогда
$R_1+R_2\colon G\to \mathrm{GL}(V_1\oplus V_2)$ и любой морфизм
$f\p\in\Mor(R_1+R_2,S)$ имеет вид
$$f((v_1,v_2))=f_1(v_1)+f_2(v_2), \quad \text{где $f_1\in \Mor(R_1,S)$,
$f_2\in\Mor(R_2,S)$.}$$ Отображение $f\mapsto (f_1,f_2)$ и есть
искомый изоморфизм.
\end{proof}

В частности, $\Mor(R,R)$ есть пространство линейных операторов,
перестановочных со всеми операторами представления.

Если $R$~--- неприводимое комплексное представление, то по лемме
Шура $\dim\Mor(R,R)=1$.

\begin{theorem}
\label{3.XIII}Если $R,S$~--- неприводимые комплексные представления,
то
$$\dim\Mor(R,S)=
\begin{cases}
1,&\text{если $R\simeq S$}\\
0,&\text{если $R\not\simeq S$}.
\end{cases}$$
\end{theorem}

\begin{proof}
Если $R\simeq S$, то можно считать, что $R=S$ и тогда
$\dim\Mor(R,R)=1$.

Пусть $R\not\simeq S$ и $0\neq f\in\Mor(R,S)$. Тогда $\ker f$
инвариантно $\Rightarrow$ $\ker f=0$. Далее, $\Im f$ инвариантно
$\Rightarrow$ $\Im f=U$ $\Rightarrow$ $f$~--- изоморфизм~---
противоречие.
\end{proof}

\begin{imp}
Пусть $R=\sum\limits_i k_iR_i$~--- разложение представления $R$ в
сумму неприводимых. Тогда $k_j=\dim\Mor(R,R_j)$.
\end{imp}

\begin{proof}
$\Mor(R,R_j)\simeq \bigoplus\limits_ik_i\Mor(R_i,
R_j)=k_j\Mor(R_j,R_j)=k_j$.
\end{proof}

\begin{ex}
\begin{enumerate}
  \item Найдем число 4-мерных комплексных представлений группы
  $D_6$. Имеется 4 одномерных и 2 двумерных неприводимых\footnote{Здесь
  в лекции было сказано <<нетривиальных>>. Полагаю, это оговорка.} комплексных
  представления. $4=\underbrace{2+2}_{3}=\underbrace{2+1+1}_{2(4+6)=20}=
  \underbrace{1+1+1+1}_{CC^4_4=25}$, значит, всего 58 4-мерных
  представлений.
\end{enumerate}
\end{ex}
%----------------------------------------------------------%
%%%%%%%%%%%%%%%%%%%%%%%%%%%%%%%%%%%%%%%%%%%%%%%%%%%%%%%%%%%%
%----------------------Lecture 14---------------------------%
\lecture{Регулярные представления.}

Пусть $G$~--- конечная группа, $A$~--- векторное пространство с
базисом $\{a_g: g\in G\}$ ($\dim A=|G|$). Рассмотрим представление
$L\colon G\to \mathrm{GL}(A)$, $L(g)a_h=a_{gh}$. Это представление
называется \emph{регулярным}.

Для простоты будем писать просто $g$ вместо $a_g$. Тогда $L(g)h=gh$.

\begin{theorem}
Кратность вхождения каждого неприводимого\footnote{В лекциях
опять-таки было сказано <<нетривиального>>.} комплексного линейного
представления $R$ группы $G$ в регулярное представление равна $\dim
R$.
\end{theorem}

\begin{proof}
Опишем пространство $\Mor(L, R)$, где $R\colon G\to \mathrm{GL}(V)$
--- любое представление. Пусть $f\in \Mor(L,R)$, $f\colon A\to V$
перестановочно с действием $G$. $f(e)=v$ однозначно определяет $f$:
$f(g)=f(L(g)e)\p=R(g)v$.

Обратно, $\forall \, v\in V$ определим линейное отображение
$f_v\colon A\to V$ по формуле $f_v(g)=R(g)v$. Оно будет
перестановочно с действием $G$:
$$f_v(L(g)h)=f_v(gh)=R(gh)v=R(g)R(h)v=R(g)f_v(h).$$ Кроме того,
$f_v$ линейно зависит от $v$: $f_{v_1+v_2}=f_{v_1}+f_{v_2}$,
$f_{\lambda v}=\lambda f_v$.

Т.о., $\Mor(L,R)\simeq V$ и $\dim \Mor(L,R)=\dim V$.

Если $R$ неприводимо, то по теореме~\ref{3.XIII} $\dim \Mor(L,R)$
равно кратности вхождения $R$ в $L$.
\end{proof}

\begin{imp}
Конечная группа имеет лишь конечное число неприводимых\footnote{И
здесь в лекции было <<нетривиальных>>.} комплексных представлений и
сумма квадратов их размерностей равна порядку группы. $\square$
\end{imp}

\begin{ex}
\begin{enumerate}
  \item $G$ абелева: $\underbrace{1^2+\ldots+1^2}_{n=|G|}=n$.
  \item $G=D_n$. Пусть $n$ четно $\Rightarrow$ $\frac{n-2}{2}\cdot 2^2+4\cdot
  1^2=2n$. Пусть $n$ нечетно $\Rightarrow$ $\frac{n-1}{2}\cdot 2^2+2\cdot
  1^2=2n$.
  \item $G=S_4$: $1^2+1^2+2^2+3^2+3^2=24=|G|$. Значит, других
  нетривиальных комплексных линейных представлений нет.
\end{enumerate}
\end{ex}

\begin{theorem}[Без доказательства]
Число неприводимых комплексных линейных представлений конечной
группы $G$ равно числу классов сопряженности в $G$.
\end{theorem}

\begin{ex}
\begin{enumerate}
  \item $G$ абелева: число неприводимых представлений равно $|G|$.
  \item $G=S_4$: 5 классов сопряженности $\Rightarrow$ есть 5
  неприводимых представлений.
  \item $G=A_5$. $|G|=60=1^2+3^2+4^2+34$. Число классов
  сопряженности равно 5 $\Rightarrow$
  $|G|=1^2+3^2+4^2+(\underbrace{3^2+5^2}_{34})$.
\end{enumerate}
\end{ex}

\tema{Линейные представления группы $\mathbb{R}$}

Рассмотрим представление $F\colon \mathbb{R}\to \mathrm{GL}(V)$
аддитивной группы $\mathbb{R}$, где $V$~--- векторное пространство
над полем $K=\mathbb{R}$ или $K=\mathbb{C}$. $F(t+u)\p=F(t)\cdot
F(u)\quad \forall \, t,u\in\mathbb{R}$ (\emph{однопараметрическая
группа линейных операторов}).В матричной записи $F(t)=(F_{ij}(t))$.
Потребуем, чтобы функции $F_{ij}(t)$ были непрерывно
дифференцируемы.

\begin{theorem}
Дифференцируемое отображение $F\colon \mathbb{R}\to \mathrm{GL}(V)$
является линейным представлением $\Leftrightarrow$
$F'(t)=\mathcal{A}F(t)$ для некоторого оператора $\mathcal{A}\in
V^*$, и $F(0)=\mathcal{E}$.
\end{theorem}

\begin{note}
$\mathcal{A}=F'(0)$.
\end{note}

\begin{proof}
1) Пусть $F$~--- гомоморфизм. Тогда $\forall \,
t,u\in\mathbb{R}\quad F(u\p+t)=F(u)\cdot F(t)$. Дифференцируя по $u$
при $u=0$, получаем $F'(t)\p=F'(0)\cdot F(t)=\mathcal{A} F(t)$.

2) Обратно, пусть $F'(t)=\mathcal{A}F(t)$, $F(0)=\mathcal{E}$.
Данное дифференциальное уравнение можно рассматривать как систему из
$n^2$ обычных дифференциальных уравнений с $n^2$ неизвестными
функциями. По общей теореме о решениях системы дифференциальных
уравнений решение однозначно определяется начальным условием.

$\forall \, \mathcal{C}\in V^*$ рассмотрим функцию
$F_\mathcal{C}(t)=F(t)\mathcal{C}$. Она удовлетворяет тому же
уравнению:
$F'_\mathcal{C}(t)=F'(t)\mathcal{C}=\mathcal{A}F(t)\mathcal{C}=\mathcal{A}F_\mathcal{C}(t)$
с начальным условием $F_\mathcal{C}(0)=\mathcal{C}$.

$\forall \, u\in \mathbb{R}$ рассмотрим матричную функцию
$F_u(t)=F(t+u)$. Имеем:
$F'_u(t)=F'(t+u)=\mathcal{A}F(t+u)=\mathcal{A}F_u(t)$.
Следовательно, $F_u(t)=F(t)\mathcal{C}$, где
$\mathcal{C}=F_u(0)=F(u)$.

Т.о., $F(t+u)=F(t)\cdot F(u)\quad \forall \, t,u\in \mathbb{R}$.
\end{proof}

В случае $\dim V=1$ мы получаем обычное дифференциальное уравнение
$f'(t)=af(t)$, $f(0)=1$. Его решение~--- это экспонента:
$f(t)=e^{at}$.

Экспонента --- это сумма бесконечного ряда:
$e^a=\sum\limits_{k=0}^\infty \frac{a^k}{k!}$. Можно попробовать
написать такой же ряд для линейного оператора $\mathcal{A}$:
$e^\mathcal{A}\p=\sum\limits_{k=0}^\infty \frac{\mathcal{A}^k}{k!}$.
Чтобы придать смысл этому ряду, надо определить сходимость
последовательности матриц.

Рассмотрим матрицы над полем $K=\mathbb{R}$ или $K=\mathbb{C}$.
\emph{Норма матрицы} $\mathop{A}\limits_{n\times n}=(a_{ij})$: $\|
A\|= \max\limits_i\sum\limits_j |a_{ij}|$.

\svoy
\begin{enumerate}
  \item $\|A\|\geqslant 0$, причем $\|A\|=0$ $\Leftrightarrow$ $A=0$
  \item $\sum\limits_j|a_{ij}|\leqslant \|A\|$
  \item $\|A+B\|\leqslant \|A\|+\|B\|$: $\forall \, i\quad \sum\limits_i
  |a_{ik}+b_{ij}|\leqslant \sum\limits_i|a_{ik}|+\sum\limits_i|b_{ij}|\leqslant
  \|A\|\p+\|B\|$.
  \item $\|AB\|\leqslant \|A\|\cdot \|B\|$: пусть $C=AB=(c_{ij})$
  $\Rightarrow$ $\forall \, i$ $$\sum\limits_k|c_{ik}|=\sum\limits_k\sum
  \limits_j|a_{ij}|\cdot|b_{jk}|\leqslant \sum\limits_j|a_{ij}|\cdot \|B\|\leqslant
  \|A\|\cdot\|B\|.$$
\end{enumerate}

Последовательность матриц $A_k$ \emph{сходится к матрице} $A$, если
$\|A\p-A_k\|\to 0$. Это равносильно поэлементной сходимости. Пишут:
$A_k\to A$.

\begin{theorem}[Критерий Коши]
Ряд $\sum\limits_{k=1}^\infty A_k$ сходится $\Leftrightarrow$
$A_{p+1}+\ldots\p+A_q\to 0$ при $p,g\to\infty$. $\square$
\end{theorem}

\begin{prop}
Если числовой ряд $\sum\limits_{k=1}^\infty \|A_k\|$ сходится, то и
матричный ряд $\sum\limits_{k=1}^\infty A_k$ сходится, причем его
сумма не зависит от порядка слагаемых.
\end{prop}

\begin{proof}
Если ряд $\sum\limits_{k=1}^\infty \|A_k\|$ сходится, то
$\|A_{p+1}+\ldots+A_q\|\p\leqslant\|A_{p+1}\|+\ldots+\|A_q\|\to 0$
при $p,q\to \infty$ $\Rightarrow$ ряд $\sum\limits_{k=1}^\infty A_k$
сходится по критерию Коши, причем каждый ряд из матричных элементов
сходится абсолютно $\Rightarrow$ сумма ряда не зависит от порядка
слагаемых.
\end{proof}

\begin{theorem}
$\forall \, A$ ряд $\sum\limits_{k=1}^\infty \frac{A^k}{k!}$
сходится абсолютно.
\end{theorem}

\begin{proof}
$\sum\limits_{k=0}^\infty \|\frac{A^k}{k!}\|\leqslant
\sum\limits_{k=0}^\infty \frac{\|A\|^k}{k!}$~-- сходится абсолютно.
\end{proof}

\emph{Экспонента матрицы $A$}: $e^A=\sum\limits_{k=1}^\infty
\frac{A^k}{k!}$.
%----------------------------------------------------------%
%%%%%%%%%%%%%%%%%%%%%%%%%%%%%%%%%%%%%%%%%%%%%%%%%%%%%%%%%%%%
%----------------------Lecture 15---------------------------%
\lecture{}

$e^{C^{-1}AC}=\sum\limits_{k=0}^\infty
\frac{C^{-1}A^kC}{k!}=C^{-1}e^AC$. Это позволяет определить
экспоненту линейного оператора: $e^\mathcal{A}$~--- это
\emph{линейный оператор с матрицей} $e^A$, где $A$~--- матрица
оператора $\mathcal{A}$.

\begin{lemma}
Если $AB=BA$, то $e^{A+B}=e^A\cdot e^B$.
\end{lemma}

\begin{proof}
Т.к. ряд $\sum\limits_{k=0}^\infty\sum\limits_{\substack{p,q=0\\
p+q=k}}^k\frac{A^pB^q}{p!q!}$ сходится абсолютно, то суммировать
можно в любом порядке $\Rightarrow$
$$\begin{aligned}
e^{A+B}&= \sum\limits_{k=0}^\infty\frac{(A+B)^k}{k!}=
\sum\limits_{k=0}^\infty\sum\limits_{\substack{p,q=0\\p+q=k}}^k\frac{C_k^pA^pB^q}{k!}=\\
&=\sum\limits_{k=0}^\infty\sum\limits_{\substack{p,q=0\\
p+q=k}}^k\frac{A^pB^q}{p!q!}=\sum\limits_{p=0}^\infty\frac{A^p}{p!}\cdot
\sum\limits_{q=0}^\infty\frac{B^q}{q!}=e^A\cdot e^B.
\end{aligned}$$
\end{proof}

\begin{theorem}
$\forall \, A$ отображение $F_A\colon t\mapsto e^{tA}$ есть линейное
представление группы $\mathbb{R}$, причем $F'(0)=A$.
\end{theorem}

\begin{proof}
Нужно проверить, что $F'_A(t)=AF_A(t)$, $F_A(0)=E$. Вычислим
производную $F'_A(0)$:
$$\frac{e^{tA}-E}{t}=\sum\limits_{k=1}^\infty \frac{t^{k-1}A^k}{k!}=
A+t\sum\limits_{k=2}^\infty\frac{A^kt^{k-2}}{k!}.$$ При $|t|<1$ ряд
мажорируется числовым рядом
$$\frac{\|A\|^2}{2!}+\frac{\|A\|^3}{3!}+\ldots=C(=e^{\|A\|}-1-\|A\|),$$
и, следовательно, сходится, причем его сумма по норме не больше $C$.
Значит, $F'_A(0)=\lim\limits_{t\to 0}\frac{e^{tA}-E}{t}=A$.

Т.к. матрицы $tA$ и $uA$ коммутируют $\forall \, t,u\in\mathbb{R}$,
то $F_A(t+u)=F_A(t)\p\times F_A(u)$.
$F'_A(t)=\left.\frac{d}{du}F_A(u+t)\right|_{u=0}=\left.\frac{d}{du}
F_A(u)\right|_{u=0}\cdot F_A(t)=AF_A(t)$.
\end{proof}

\begin{ex}
\begin{enumerate}
  \item Рассмотрим 4 двумерных представления группы $\mathbb{R}$:\\
  $R_1(t)=\left(\begin{smallmatrix} \cos t& -\sin t\\ \sin t& \cos t
  \end{smallmatrix}\right)$, $R_1'(0)=\left(\begin{smallmatrix} 0& -1\\ 1&
  0\end{smallmatrix}\right)=A_1$ $\Rightarrow$ $R_1(t)=e^{t\left(\begin{smallmatrix}
  0& -1\\ 1& 0\end{smallmatrix}\right)}$.\\
  $R_2(t)=\left(\begin{smallmatrix} \ch t& \sh t\\ \sh t& \ch t
  \end{smallmatrix}\right)$, $R_2'(0)=\left(\begin{smallmatrix} 0& 1\\ 1&
  0\end{smallmatrix}\right)=A_2$ $\Rightarrow$ $R_2(t)=e^{t\left(\begin{smallmatrix}
  0& 1\\ 1& 0\end{smallmatrix}\right)}$.\\
  $R_3(t)=\left(\begin{smallmatrix} e^t& 0\\ 0& e^{-t}
  \end{smallmatrix}\right)$, $R_3'(0)=\left(\begin{smallmatrix} 1& 0\\ 0&
  -1\end{smallmatrix}\right)=A_3$ $\Rightarrow$ $R_3(t)=e^{t\left(\begin{smallmatrix}
  1& 0\\ 0& -1\end{smallmatrix}\right)}$.\\
  $R_4(t)=\left(\begin{smallmatrix} 1& t\\ 0& 1
  \end{smallmatrix}\right)$, $R_4'(0)=\left(\begin{smallmatrix} 0& 1\\ 0&
  0\end{smallmatrix}\right)=A_4$ $\Rightarrow$ $R_1(t)=e^{t\left(\begin{smallmatrix}
  0& 1\\ 0& 0\end{smallmatrix}\right)}$.
\end{enumerate}
\end{ex}

\tema{Идеалы и факторкольца}

Если в кольце $A$ имеется отношение эквивалентности, согласованное
со сложением и умножением, то на множестве классов эквивалентности
можно ввести операции сложения и умножения по формулам
$[a]+[b]\p=[a+b]$, $[a]\cdot[b]=[ab]$.

Найдем отношения эквивалентности, согласованные с операциями. Т.к.
кольцо~--- это абелева группа, то всякое такое отношение есть
отношение сравнимости по модулю подгруппы $I$: $a\sim b$
$\Leftrightarrow$ $a\equiv b\pmod{I}$ $\Leftrightarrow$ $a-b\in I$.

Найдем, какой должна быть подгруппа $I$, чтобы это отношение было
согласовано с операцией умножения.

\begin{theorem}
Отношение сравнимости по модулю подгруппы $I\subset A$ согласовано с
умножением $\Leftrightarrow$ $I$~--- \emph{идеал кольца $A$} (т.е.
$AI=IA=I$).
\end{theorem}

\begin{proof}
1) Пусть отношение эквивалентности по модулю $I$ согласовано с
умножением. Тогда $\forall \, u\in I\quad u\p\equiv 0\pmod{I}$, и,
значит, $au\p\equiv a\cdot 0\p\equiv0\pmod{I}$, $ua\equiv0\cdot
a\equiv0\pmod{I}$.

2) Обратно, пусть $I$~--- идеал и $a\equiv a'\pmod{I}$, $b\equiv
b'\pmod{I}$. Тогда $a'=a+u$, $b'=b+v$, где $u,v\in I$. Значит,
$a'b'=ab+(ub+av+uv)\equiv ab\pmod{I}$.
\end{proof}

Если $I$~--- идеал кольца $A$, то в факторгруппе $A/I$ можно
определить операцию умножения по формуле $(a+I)(b+I)=ab+I$.
Определенное таким образом умножение в $A/I$ дистрибутивно
относительно сложения.

Построенное таким образом кольцо $A/I$ называется
\emph{факторкольцом кольца $A$ по идеалу $I$}.

\begin{ex}
\begin{enumerate}
  \item $\mathbb{Z}/n\mathbb{Z}=\mathbb{Z}_n$ (кольцо вычетов по модулю
  $n$).
\end{enumerate}
\end{ex}

Имеется канонический гомоморфизм $\pi\colon A\to A/I$, $a\mapsto
a+I$.

\begin{theorem}[О гомоморфизме колец]
Пусть $f\colon A\to B$~--- гомоморфизм колец. Тогда $\ker f=I$~---
идеал кольца $A$ и $f=\bar{f}\circ \pi$, где $\pi\colon A\to
A/I$~--- канонический гомоморфизм, а $\bar{f}\colon A/I\to B$~---
некоторый гомоморфизм. Если $f$ сюръективен. то $\bar{f}$~---
изоморфизм. $\square$
\end{theorem}

Пусть $A$~--- коммутативное ассоциативное кольцо с единицей. Тогда
всякое факторкольцо также является коммутативным ассоциативным
кольцом с единицей.

$\forall \, a\in A$ определим \emph{главный идеал} $(a)=\{ua: u\in
A\}$. Легко проверить, что это идеал.

\begin{ex}
\begin{enumerate}
  \item В кольце $\mathbb{Z}$ $(n)=n\mathbb{Z}$.
\end{enumerate}
\end{ex}

\emph{Отношение сравнимости по модулю главного идеала} $(u)$~--- это
то, что в 1-м семестре называлось отношением сравнимости по модулю
$u$: $a\equiv b\pmod{(u)}$ $\Leftrightarrow$ $a-b\in(u)$
$\Leftrightarrow$ $a-b=cu$ $\Leftrightarrow$ $a\equiv b\pmod{u}$.

Не все идеалы являются главными.

\begin{ex}
\begin{enumerate}
  \item В кольце $K[x,y]$ ($K$~--- поле) идеал $I=\{f\in K[x,y]:
  f(0,0)=0\}$ не является главным.
\end{enumerate}
\end{ex}

Целостное кольцо $A$, не являющееся полем, называется
\emph{евклидовым кольцом}, если задано отображение $N\colon
A\setminus \{0\}\to \mathbb{Z}_+$ (которое называется
\emph{нормой}), удовлетворяющее условиям

1) $N(ab)\geqslant N(a)$, причем равенство достигается
$\Leftrightarrow$ $b$ обратим

2) $\forall \, a\in b,\; \forall \, b\in A\setminus \{0\}\quad
\exists \, q,r\in A: a=bq+r$ и либо $r=0$, либо $N(r)<N(b)$.

\begin{theorem}
В евклидовом кольце всякий идеал главный.
\end{theorem}

\begin{proof}
Пусть $I$~--- идеал евклидова кольца $A$. Если $I=0$, то $I=(0)$.
Если $I\neq 0$, то пусть $u\in I$~--- элемент наименьшей нормы.
Тогда $I\supset (u)$. Докажем, что на самом деле $I=(u)$. Пусть
$a\in I$, тогда $a=qu+r$, где $r=0$ или $N(r)<N(u)$. Но $r=a-qu\in
I$ $\Rightarrow$ $r=0$.
\end{proof}
%----------------------------------------------------------%
%%%%%%%%%%%%%%%%%%%%%%%%%%%%%%%%%%%%%%%%%%%%%%%%%%%%%%%%%%%%
%----------------------Lecture 16---------------------------%
\lecture{}

$a\equiv b\pmod{(u)}$ $\Leftrightarrow$ $a\equiv b\pmod{u}$. Т.о.,
факторкольцо $A/(u)$~--- это то же самое, что кольцо вычетов по
модулю $u$. Например, $\mathbb{Z}/(n)=\mathbb{Z}_n$.

\begin{theorem}
Пусть $A$~--- евклидово кольцо и $u\in A$. Тогда $A/(u)$ является
полем $\Leftrightarrow$ $u$~--- простой элемент.
\end{theorem}

\begin{proof}
1) Если $u=0$, то $A/(u)=A$~--- не поле по определению евклидова
кольца.

2) Если $u$ обратим, то $(u)=A$ и $A/(u)=\{0\}$~--- не поле по
определению поля.

3) Если $u=v\cdot w$, где $v$ и $w$ необратимы, то $[v]\cdot[w]=0$ в
$A/(u)$. Но $[v],[w]\neq0$, т.к. $v$ и $w$ не делятся на $u$.
Значит, $A/(u)$~--- не поле.

4) Пусть $u$~--- простой элемент, $a\notin(u)$. Тогда $(a,u)=1$
$\Rightarrow$ $\exists \, x,y\in A: ax+uy=1$ $\Rightarrow$
$[a]\cdot[x]=1$, т.е. $[x]=[a]^{-1}$. Значит, $A/(u)$~--- поле.
\end{proof}

Применим эту теорему к кольцу многочленов $K[x]$ над полем $K$.
Получаем, что $K[x]/(f(x))$~--- поле $\Leftrightarrow$ $f(x)$~---
неприводимый многочлен.

Пусть $f(x)$~--- любой многочлен степени $n>0$. Тогда для $a,b\in K$
$a\p\equiv b\pmod{f(x)}$ $\Leftrightarrow$ $a=b$. Следовательно, $K$
вкладывается в кольцо $K[x]/(f(x))\p=L$. Будем отождествлять элемент
$a\in K$ с $[a]\in L$. Введем обозначение: $\alpha=[x]\in L$. Тогда
$f(\alpha)=[f(x)]=0$, т.е. $\alpha$~--- корень многочлена $f(x)$.
Если $f(x)$ неприводим, то $L$~--- поле. Говорят, что $L$ получается
из $K$ \emph{присоединением корня многочлена $f(x)$}.

В силу единственности деления с остатком в кольце $K[x]$ в каждом
классе $g(x)+{f(x)}$ имеется единственный многочлен степени меньше
$\deg f(x)$. Это означает, что каждый элемент кольца $L$
единственным образом представляется в виде
$a_0+a_1\alpha+\ldots+a_{n-1}\alpha^{n-1}$ ($a_0,\ldots,a_{n-1}\in
K$).

Кольцо $L$ можно рассматривать как векторное пространство над $K$.
Из предыдущего следует, что
$$\hc{1,\alpha,\alpha^2,\ldots,\alpha^{n-1}}$$ является базисом этого
пространства, и, значит, $\dim_K L=n$.

\begin{ex}
\begin{enumerate}
  \item $\mathbb{R}[x]/(x^2+1)\simeq \mathbb{C}$.
  \item $\mathbb{Q}[x]/(x^3-2)=\{a_0+a_1\alpha+a_2\alpha^2: a_0,a_1,a_2\in\mathbb{Q},
  \alpha^3=2\}$. В поле $L=\mathbb{Q}[x]/(x^3-2)$ многочлен $x^3-2$
  имеет ровно один корень.
\end{enumerate}
\end{ex}
Расширение $L$ поля $K$ называется \emph{конечным}, если
$\dim_KL<\infty$.

\begin{ex}
\begin{enumerate}
  \item $\mathbb{R}\supset \mathbb{Q}$.
  \item $K(x)\supset K$.
\end{enumerate}
\end{ex}
Пусть $F$~--- конечное поле характеристики $p$. Тогда
$\langle1\rangle$ (аддитивная подгруппа, порожденная 1) есть
подкольцо:
$(\underbrace{1+\ldots+1}_{k})\cdot(\underbrace{1+\ldots+1}_{l})\p=(\underbrace{1+\ldots+1}_{kl})$.
Более того, это подполе, изоморфное $\mathbb{Z}_p$:
$$[k]_p\leftrightarrow \underbrace{1+\ldots+1}_{k}.$$ Т.о., $F\supset
\mathbb{Z}_p$, и, следовательно, $|F|=p^n$.

Построим поле из $p^2$ элементов. $F=\mathbb{Z}_p[x]/(f(x))$, где
$f(x)\in\mathbb{Z}_p[x]$~--- неприводимый многочлен степени 2.

Если $p\neq2$, то $\mathbb{Z}_p^*$~--- циклическая группа четного
порядка $p-1$. Ровно половина ее элементов являются квадратами.
Пусть $a\in \mathbb{Z}_p^*$~--- квадратичный невычет. Тогда $x^2-a$
не имеет корней в $\mathbb{Z}_p$ и, следовательно, неприводим.

Если $p=2$, то можно взять $f(x)=x^2+x+1$.

На самом деле, верна

\begin{theorem}[Без доказательства]
Для любого простого $p$ и $\forall \, n\in \mathbb{N}$ существует
поле из $p^n$ элементов. Более того, все такие поля изоморфны.
\end{theorem}
%----------------------------------------------------------%
\end{document}
