\documentclass[a4paper]{article}
\usepackage[utf8]{inputenc}
\usepackage[russian]{babel}
\usepackage{dmvn}

\newcommand{\sumg}{\sums{g\in G}}

\begin{document}
\dmvntitle{Курс лекций по}{высшей алгебре}{Лектор Евгений Соломонович Голод}
{II курс, 3 семестр, поток математиков}{Москва, 2003 г.}

\pagebreak
\tableofcontents
\pagebreak

\section*{Введение}
\subsection*{Предисловие}

Спасибо всем, кто принимал прямое/косвенное участие в создании этого документа! Прежде всего, хочется
поблагодарить нескольких студентов МехМата: \textbf{Домбровскую~А.} за предоставление великолепного
конспекта лекций; \textbf{Трушина~Д.} и \textbf{Малыхина~Ю.} за ценные советы и доказательства некоторых
теорем; \textbf{Краснобаева~И.}, \textbf{Степанову~М.}, \textbf{Короткова~В.} за обнаружение
нескольких ошибок и опечаток.

Отдельная благодарность выносится \textbf{Юрию Кудряшову}, которого смело можно назвать главным редактором
данного издания.

К сожалению, часть утверждений в этом курсе пришлось доказывать самостоятельно (лектору они казались совсем
очевидными), так что вероятность наличия ошибок не равна нулю, но (по идее) должна стремиться к нему справа.
Часть доказательств было заменено их аналогами, взятыми из других источников (в основном это учебник
Э.\,Б.\,Винберга), поскольку они казались более простыми для понимания. Данная версия сего опуса уже далеко
не первая, ибо благодаря стараниям вышеупомянутых лиц в тексте было исправлено огромное количество ошибок и
опечаток. Остаётся надеяться, что он будет полезен будущим поколениям слушателей лекций автора курса.

В данной версии была сделано ещё несколько улучшений, на этот раз в основном с целью улучшить читаемость текста.

\subsection*{Список литературы}

\begin{itemize}
\setlength{\itemsep}{-3pt}

\item Конспекты лекций по высшей алгебре. \copyright \, МехМат, II курс, первый поток, 2003--2004 уч.г.
\item Винберг\,Э.\,Б. \emph{Курс алгебры.} 3-е изд. М.: Факториал-Пресс, 2002. 544 с.
\item Кострикин\,А.\,И. \emph{Введение в алгебру: Основные структуры.} 2-е изд. М.: ФизМатЛит, 2001. 272 с.
\item B.\,L.\,van\,der\,Waerden. \emph{Алгебра.} М.: Наука, 1979. 624 с.
\end{itemize}

\medskip
\dmvntrail

\pagebreak

\section{Теория групп}

\subsection{Основные понятия и теоремы}

\subsubsection{Группы и подгруппы}

\begin{df}
Множество с ассоциативной бинарной операцией называется \emph{полугруппой}. Полугруппа с нейтральным
элементом называется \emph{моноидом}. Моноид, в котором каждый элемент обратим, называется
\emph{группой}.
\end{df}

\begin{df}
\emph{Подгруппой} в группе $G$ называется подмножество $H \subseteq G$, само являющееся группой.
\end{df}

Пусть $M_1 \sco M_s \subseq G$. Тогда $M_1 \sd M_s := \hc{x_1 x_2 \ldots x_s \vl x_i \in M_i}$.
Если $M_i$ конечны, то $|M_1 \sd M_s| \bw\le |M_1|\ldots|M_s|$.

\begin{df}
Пусть заданы группы $(G, \cdot)$ и $(G', *)$. \emph{Гомоморфизмом} групп $G$ и $G'$ называется отображение
$f\cln G \ra G'$, такое, что $f(a \cdot b)=f(a)*f(b)$ для $\fa a,b \in G$. \emph{Эпиморфизмом} называется
сюръективный гомоморфизм.
\end{df}

При гомоморфизме единица переходит в единицу. В самом деле, $f(e) = f(e^2) = f(e)f(e)$, следовательно,
$f(e)$ единица группы $G'$.

\begin{df}
\emph{Циклической подгруппой} элемента $a \in G$ называется множество $\ha{a} := \hc{a^n}, n \in \Z$.
\end{df}

\begin{df}
\emph{Порядком} элемента $a \in G$ называется число $O(a) := \min\hc{k>0\cln a^k = e}$.
\end{df}

\subsubsection{Смежные классы}

\begin{df}
\emph{Левым смежным классом} элемента $g$ по подгруппе $H$ группы $G$ называется множество  $gH \bw= \hc{gh
\vl h \in H}$. Аналогично определяется правый смежный класс.
\end{df}

Рассмотрим отображение $f\cln H \ra aH$, где элемент $a \in G$ фиксирован, $f\cln h \mapsto ah$.
Сюръективность $f$ очевидна, докажем инъективность. Действительно, $f(h_1) = f(h_2) \Lra a h_1 = a h_2 \Lra
h_1 = h_2$. Отсюда следует, что $|H| = |aH|$.

\begin{stm}
Смежные классы либо не пересекаются, либо совпадают. $aH=bH \Lra a^{-1}b\in H$.
\end{stm}
\begin{proof}
Действительно, пусть $aH \cap bH \neq \es$, тогда найдутся $h_1, h_2 \in H$ такие, что $ah_1 = bh_2$,
откуда $b=ah_1h_2^{-1}$. Значит, $bH = ah_1 h_2^{-1}H \bw= aH$, к $h_1h_2^{-1} \in H$.
Теперь докажем второе утверждение. Пусть $aH=bH$. Тогда $ah_1=bh_2$, откуда $a^{-1}b=h_1h_2^{-1} \in H$.
Обратно: $a^{-1}b \in H \Ra a^{-1}b=h \Ra bH = ahH = aH$.
\end{proof}

Значит, имеется разбиение группы $G$ на левые смежные классы: $G = \bigcup g_iH$. Есть биекция между левыми и
правыми смежными классами: левому классу $aH$ сопоставим правый класс  $Ha^{-1}$. Это делает корректным

\begin{df}
\emph{Индексом} подгруппы $H$ называется число $(G:H)$ смежных классов по этой подгруппе.
\end{df}

Из всего вышесказанного вытекает
\begin{theorem}[Лагранжа]
Порядок группы является произведением порядка подгруппы на её индекс.
\end{theorem}

\begin{df}
Подгруппа $N \subseq G$ называется \emph{нормальной} в $G$, если $\fa g \in G$ имеем $gN = Ng$.
Обозначение: $N \nl G$.
\end{df}

\begin{stm}
Следующие утверждения эквивалентны:

\pt{1} Подгруппа $N$ нормальна в $G$;

\pt{2} $\fa g \in G$ имеем $gNg^{-1} = N$;

\pt{3} $\fa g \in G$ имеем $gNg^{-1} \subseq N$;

\pt{4} Произведение множеств $(g_1 N)(g_2 N)$ левый смежный класс $g_1 g_2 N$;

\pt{5} Cуществует гомоморфизм $f\cln G \ra H$, такой что $N = \Ker f$.
\end{stm}

\begin{proof}
Эквивалентность пунктов \pt{1} и \pt{2} очевидна. Докажем, что из \pt{3} следует \pt{2}. Рассмотрим
отображение $\ph_g(x) \bw= gxg^{-1}, \; x \in N$. Нам дано, что $\Img \ph \subseq N$. Проверим инъективность.
Пусть $\ph_g(x_1) = \ph_g(x_2)$. Тогда $gx_1g^{-1}=gx_2g^{-1} \Ra x_1=x_2$. Проверим пункт \pt{4}. Если для
$\fa g_1, g_2 \in G$ имеем равенство множеств $(g_1N)(g_2N)\bw=g_1g_2N$, то для $\fa n_1, n_2 \in N$ найдется
элемент $n_3 \in N\cln g_1n_1g_2n_2=g_1g_2n_3$. Сократим в равенстве на $g_1$, домножим справа на $g_2^{-1}$ и
слева на $n_1^{-1}$. Получим $g_2n_2g_2^{-1} = n_1^{-1}n_3 \in N$, то есть при сопряжении с любым элементом
мы снова попадаем в $N$, а значит, верно свойство \pt{3}. Наоборот: если $N$ нормальна, то $g_1Ng_2N \bw= g_1(N
g_2)N \bw= g_1(g_2N)N \bw= g_1g_2N$. Теперь докажем эквивалентность \pt{1} и \pt{5}. Если $N = \Ker f$, то
\eqn{f(gNg^{-1}) \bw= f(g)f(N)f(g^{-1})\bw=f(g) e f(g^{-1})=f(gg^{-1}) = f(e) = e \in \Ker f=N,}
то есть выполняется \pt{3}. Наоборот: возьмём отображение $f$, ставящее в соответствие
элементу из $G$ его смежный класс по подгруппе $N$. Очевидно, что это гомоморфизм, и его ядро есть $N$.
\end{proof}

\begin{df}
Отношение эквивалентности, сохраняющее операцию, называется \emph{конгруэнцией}.
\end{df}

\subsubsection{Факторгруппы}

Введём на множестве смежных классов по подгруппе $H \nl G$ естественную операцию $fH \cdot gH = fgH$.  Таким
образом, мы получили \emph{\emph{факторгруппу}} $G/H$. (свойства группы легко проверить).

\begin{theorem}[О гомоморфизме]
Пусть $\ph\cln G \ra H$ эпиморфизм, $\Ker \ph =: K$, а $\pi\cln G \ra G/K$ канонический гомоморфизм.
Тогда $G/K \cong H$. Более точно, $\exists !$ изоморфизм $\ol{\pi}\cln G/K\ra H$, такой, что $\ol{\pi}\pi = \ph$.
\end{theorem}
\begin{proof}
Определим $\ol{\pi}$ следующим образом: $\ol{\pi}(gK) := \ph(g)$. Это корректно, к оно не зависит от
выбора представителя смежного класса: если $g'=gk$, то $\ph(g')=\ph(g)\ph(k)=\ph(g)$. Операция сохраняется:
\eqn{\ol{\pi}(g_1K \cdot g_2K) \bw= \ol{\pi}(g_1g_2K) \bw= \ph(g_1g_2) \bw= \ph(g_1)\ph(g_2) =
\ol{\pi}(g_1K)\ol{\pi}(g_2K).}
Значит, $\ol{\pi}$ гомоморфизм. Докажем его инъективность:
\eqn{\ol{\pi}(g_1K) = \ol{\pi}(g_2K) \Lra \ph(g_1) \bw= \ph(g_2) \Lra e=\ph(g_1^{-1})\ph(g_2)
=\ph(g_1^{-1}g_2) \Lra g_1^{-1}g_2 \in K,}
а тогда по замечанию $g_1K=g_2K$. Итак, $\ol{\pi}$ изоморфизм. Из условия $\ol{\pi}\pi = \ph$
следует единственность $\ol{\pi}$, так как если $\ph(g) = g'$, а $\pi(g) = [g]$, то
$\ol{\pi}([g])=\ol{\pi}\bigl(\pi(g)\bigr) = \ph(g) = g'$, е $\ol{\pi}$ определён однозначно.
\end{proof}


\subsubsection{Произведения подгрупп}

\begin{stm}
Пусть $K, L \subs G$. Множество $KL$ является подгруппой $\Lra K L=L K$.
\end{stm}
\begin{proof}
Докажем, что если $L K=K L$, то $L K$ подгруппа. Проверим, что $l_1k_1 \cdot l_2k_2 \in LK$. Поскольку  наши
множества совпадают, то $k_1 l_2=l'k' \in LK$. Тогда $l_1k_1 l_2k_2 = l_1 l'k'k_2 \in LK$. Обратный элемент:
$(lk)^{-1}=k^{-1}l^{-1} \in K L = L K$. Таким образом, $L K$ подгруппа. В обратную сторону: пусть $H=KL$
подгруппа. Мы имеем тождество $H^{-1} = H$. Тогда $(KL)^{-1}=\hc{(ab)^{-1}=b^{-1}a^{-1}}=LK$, что и
требовалось.
\end{proof}

\begin{note}
Рассмотрим частный случай: $L \nl G$. Тогда $K L = \cups{g \in K}g L \eqdef \cups{g \in K} L g = L K$.
\end{note}

\begin{theorem}[О соответствии]
Пусть есть эпиморфизм $\ph\cln G \ra F, \; \Ker \ph =: H$. Рассмотрим все подгруппы в $A \subseq G$, содержащие
$H$ (назовём их выделенными). Сопоставим каждой выделенной подгруппе $A$ её образ $U := \ph(A)$.  Тогда такое
сопоставление есть биекция между выделенными подгруппами и всеми подгруппами в $F$. При этом соответствующие
друг другу группы одновременно нормальны и факторгруппы по ним изоморфны, е $A \nl G, \; H \subseq A \Lra
U \nl F$, и $ G/A \cong F/U$.
\end{theorem}
\begin{proof}
Докажем биективность соответствия. Сюръективность: рассмотрим полный прообраз $A$ любой подгруппы  $U \subseq
F$. Единица группы $F$ лежит в $U$, а полный прообраз единицы и есть ядро. Значит, $\Ker \ph \subseq A$,
е $A$ выделенная подгруппа. Теперь докажем инъективность: пусть есть такие выделенные подгруппы $A$ и
$B$, что $\ph(A)=\ph(B)$. Тогда $\fa a \in A$ найдется $b \in B\cln \ph(a)=\ph(b)$. Следовательно,
$\ph(b)^{-1}\ph(a)=e \Lra \ph(b^{-1}a) = e$, то есть $b^{-1}a \in \Ker\ph$. Но $\Ker\ph \subseq B \Ra a = bh
\in B$, а значит $A \subseq B$. Аналогично $B \subseq A$. Значит, $A=B$ и первое утверждение доказано.

Докажем одновременную нормальность соответствующих подгрупп. Пусть $A$ выделенная подгруппа и $A \nl G$.
Тогда $\fa g \in G$ имеем $gAg^{-1}=A$. Применим $\ph$ к этому равенству: $\ph(g)\ph(A)\ph(g^{-1})=\ph(A)$.
Так как ${\Img \ph=F}$, то если $g$ пробегает по всей $G$, то $\ph(g)$ пробегает по всей $F$. Тогда подгруппа
$\ph(A)$ нормальна в $F$. Остается показать изоморфность соответствующих факторгрупп. Обозначим $U := \ph(A)$
и построим отображение $\pi\cln G \bw\ra F/U$ по правилу $\pi(g) = \ph(g)U$. Оно является гомоморфизмом, так как
\eqn{\pi(g_1g_2)=\ph(g_1g_2)U=\bigl(\ph(g_1)\ph(g_2)\bigr)U=\bigl(\ph(g_1)U\bigr)\bigl(\ph(g_2)U\bigr) =
\pi(g_1)\pi(g_2).}
Оно сюръективно, к если $g$ пробегает всю $G$, то $\ph(g)U$ пробегает всю факторгруппу
$F/U$. Теперь найдём его ядро: $g \in \Ker\pi \Lra \pi(g)=\ph(g)U=U \Lra \ph(g) \in U=\ph(A) \Lra g \in A$.
Значит, $A=\Ker\pi$, а тогда по теореме о гомоморфизме $G/A \cong F/U$, что и требовалось.
\end{proof}

\begin{imp}
Пусть $L \nl K \nl G$ и $ L, K \nl G$. Тогда $K/L \nl G/L$ и $\frac{G/L}{K/L} \cong G/K$.
\end{imp}
\begin{proof}
Рассмотрим эпиморфизм $\ph\cln G \ra G/L$ и применим к нему утверждение теоремы. Очевидно, что $\Ker\ph \bw= L$.
Имеем $\ph(K) = K/L$, а значит, $\frac{G/L}{K/L} \cong G/K$.
\end{proof}

\begin{theorem}[Об изоморфизме]
Пусть $K \nl G, \; H \subseq G, \; HK$ подгруппа в $G$. Тогда $H \cap K \nl H$ и $HK/K \bw\cong H/(H\cap K)$.
\end{theorem}
\begin{proof}
Если $K \nl G$, то и подавно $K \nl HK$, значит, можно рассмотреть факторгруппу $HK/K$. Рассмотрим канонический
эпиморфизм $\pi\cln G\ra G/K$. Через $\pi_0$ обозначим ограничение $\pi$ на $H$. Тогда
$h \corr{\pi_0} hK$. Теперь рассмотрим факторгруппу $HK/K$. Она состоит из смежных классов вида $hkK$, е
из множеств $hK$. Значит, $\Img\pi_0 \bw= HK/K$. Тогда $\Ker\pi_0$ состоит из тех $h \in H$, для которых
$\pi_0(h) = eK=K$. Но $hK = K \Lra h \in K$, следовательно, имеем $h \in H \cap K$, е $\Ker \pi_0 = H
\cap K$. По теореме о гомоморфизме $HK/K \cong H/(H \cap K)$.
\end{proof}

%==================================================================
\subsection{Автоморфизмы и классы сопряжённости}
%==================================================================

\subsubsection{Преобразования. Автоморфизмы. Примеры.}

$\Sb_M$ группа биективных отображений множества $M$ в себя. Рассмотрим группу движений плоскости,
сохраняющих правильный $n$-угольник. В ней $n$ вращений на угол $\frac{2\pi k}{n}$ и $n$ осевых симметрий.
Эта группа называется \emph{\emph{группой диэдра}} и обозначается $\Db_n$. Очевидно, $|\Db_n| = 2n$.
Обозначим поворот на угол $\frac{2\pi}{n}$ через $a$ и отражение через $b$, тогда любое преобразование имеет
вид $a^k b$. Композиция отражений есть поворот, значит, $b'b=a^k \Ra b'=a^k b$.

\begin{df}
Изоморфизм группы на себя называется \emph{автоморфизмом}.
\end{df}
Все автоморфизмы группы $G$ образуют группу, обозначаемую $\Aut G$.

\begin{stm}
$\Aut\ha{a}_\infty = \hc{\id, a \mapsto a^{-1}}$.
\end{stm}
\begin{proof}
Пусть $a$ порождающий элемент, $\ph(a) \neq a, a^{-1}$. Тогда $\ph(a) = a^k, \; k \neq \pm 1$. Если $a$
порождает всю группу, то и $\ph(a)$ также порождает всю группу. Но $\ha{\ph(a)} = \hc{(a^k)^m \vl m
\in Z} \neq G$, значит, $\ph$ не автоморфизм.
\end{proof}

Аналогично можно показать, что $\Aut\ha{a}_n = \hc{\ph(a) = a^k\cln (n,k)=1}$. Таким образом, автоморфизмы
циклической группы соответствуют обратимым элементам  в $\Z/n\Z$, е группа $\Aut\ha{a}_n$ изоморфна
группе обратимых элементов в $\Z/n\Z$.

\begin{df}
\emph{Внутренним автоморфизмом} группы $G$ называется отображение вида $\ph_g(x) = gxg^{-1}$.
\end{df}

Определение корректно, так как
\eqn{\ph_g(x_1 x_2)= g x_1 x_2 g^{-1} = (gxg^{-1})(gxg^{-1}) = \ph_g(x_1)\ph_g(x_2),}
а биективность очевидна. Группа внутренних автоморфизмов обозначается $\Int G$.
Покажем, что $\Int G \nl \Aut G$. Рассмотрим внутренний автоморфизм $f(x)=gxg^{-1}$. При сопряжении
произвольным автоморфизмом $\ph$ имеем
\eqn{(\ph^{-1} \circ f \circ \ph)(x)=\ph\bigl(f\bigl(\ph^{-1}(x)\bigl)\bigr) \bw= \ph\bigl(g\ph^{-1}(x)g^{-1}\bigr)=\ph(g)x\ph(g)^{-1},}
е снова получился некоторый внутренний автоморфизм.

\begin{df}
Группа \emph{внешних автоморфизмов} есть факторгруппа $\Out G := \Aut G/\Int G$.
\end{df}

\subsubsection{Центр группы}

\begin{df}
\emph{Центром} группы $G$ называется множество $Z(G) = \hc{g\cln gx=xg \fa x}$.
\end{df}

Рассмотрим отображение $\pi\cln G\ra \Int G$, определенное по правилу $g \mapsto \ph_g$. Очевидно, что это
гомоморфизм. Посмотрим на его ядро: $\pi(g) = \id \Lra \fa x \; gxg^{-1}=x \Lra \fa x \; gx = xg \Lra g \in
Z(G)$. Значит, $\Ker \pi = Z(G)$.

\begin{stm}
Если группа $G$ не абелева, то $G/Z(G)$ не может быть циклической.
\end{stm}
\begin{proof}
Допустим, $G/Z = \ha{aZ}$. Тогда $gZ = (aZ)^k = a^k Z$. Возьмём два элемента  $g_1 = a^kz_1$ и
$g_2=a^lz_2$. Тогда имеем $g_1g_2 = a^k z_1 a^l z_2 = a^l z_2 a^k z_1 = g_2 g_1$, так как $z_1$ и $z_2$
коммутируют со всеми. Противоречие.
\end{proof}

\subsubsection{Классы сопряженных элементов}

\begin{df}
Классом элементов, \emph{сопряженных} c $x \in G$, называется множество $x^G := \hc{gxg^{-1} \vl g \in G}$.
\end{df}

Подгруппа $H \subseq G$ является нормальной, если она является объединением классов сопряженности (очевидно).

\begin{ex}
Наличие одинаковой жордановой формы у двух матриц является критерием их сопряженности.
\end{ex}

Рассмотрим группу перестановок $\Sb_n$.

\begin{stm}
Две перестановки сопряжены $\Lra$ они имеют одинаковую цикловую структуру.
\end{stm}
\begin{proof}
Разложим перестановку $\pi \in \Sb_n$ на независимые циклы:
$\pi = (i_1 \sco i_{k_1})(i_{k_1+1} \sco i_{k_2})\cdots(i_{k_s+1} \sco i_n)$. Имеем
$$g \pi g^{-1} = \rbmat{i_1 i_2 \ldots i_{k_1} i_{k_1+1} \ldots i_n \\ j_1 j_2 \ldots j_{k_1} j_{k_1+1} \ldots j_n}
\pi\rbmat{j_1 j_2 \ldots j_{k_1} j_{k_1+1} \ldots j_n \\ i_1 i_2 \ldots i_{k_1} i_{k_1+1} \ldots i_n}=
(j_1,j_2 \sco j_{k_1})(j_{k_1+1} \sco j_{k_2})\cdots(j_{k_s+1} \sco j_n).$$ Отсюда следует, что если
перестановки сопряжены, то длины циклов одинаковые. Очевидно также то, что если  у двух перестановок
одинаковая цикловая структура, то они сопряжены сопрягающую перестановку легко предъявить.
\end{proof}

Теперь рассмотрим $\Ab_n \subset \Sb_n$ знакопеременную группу чётных перестановок. Наличие  одинаковой
цикловой структуры необходимо и для $\Ab_n$. Чтобы найти достаточное условие, рассмотрим более общий случай.
Вначале докажем вспомогательное

\begin{stm}
Подгруппы индекса 2 всегда нормальны.
\end{stm}
\begin{proof}
$(G:H)=2 \Ra $ имеется только 2 смежных класса: сама подгруппа $H=eH=He$ и некоторый левый класс  $gH, \; g
\notin H$. Тогда правый класс $Hg$ совпадает либо с $H$, либо с $gH$. Но первая возможность отпадает, так как
$g \notin H$, значит, $gH=Hg$, что и требовалось.
\end{proof}

\begin{stm}
Пусть $H \subset G$ и $(G:H)=2$. Тогда для $\fa x \in H$ возможны 2 случая:

\pt{1} $x^H=x^G \Lra \exi t \notin H\cln tx=xt$.

\pt{2} $x^G=x^H \cup x_1^H, \; x_1=txt^{-1}, t \notin H, |x^H|=|x_1^H| \Lra \fa t \notin H \; tx \neq xt$.
\end{stm}

\begin{proof}
Докажем \pt{1}. Пусть $x^G=x^H$. Из этого следует, что $\fa z \notin H \; \exi h \in H\cln zxz^{-1}=hxh^{-1}$,
то есть $x=(z^{-1}h)x(h^{-1}z)$. Положим $t=z^{-1}h \notin H$, тогда $h^{-1}z = t^{-1}$, то есть $x=txt^{-1} \Lra
tx=xt$. Наоборот: пусть $\exi t \notin H\cln tx=xt$. Тогда $\fa z \notin H$ имеем $z=ht$. Значит,
\eqn{zxz^{-1} = h \ub{txt^{-1}}_x h^{-1} = hxh^{-1} \in x^H,}
то есть $x^G=x^H$.

Теперь докажем \pt{2}. Пусть $z,t \notin H, z=ht$. Тогда $zxz^{-1}= h\ub{txt^{-1}}_{x_1}h^{-1} =hx_1h^{-1}
\in x_1^H$. Докажем, что $|x^H|=|x_1^H|$.  Заметим, что $x \corr{\ph} txt^{-1}$ автоморфизм, а при нём
классы сопряженных элементов переходят также в классы сопряженных. $\ph(x) = x_1 \Ra \ph(x^H)= x_1^H$.
\end{proof}

Вернёмся к группе $\Ab_n$. Выясним, когда для чётной перестановки существует коммутирующая с ней нечётная.
В следующем утверждении под циклами подразумеваются в том числе и циклы длины 1.

\begin{stm}Пусть $\pi \in \Ab_n$. Тогда:

\pt{1} $\exi \tau \notin \Ab_n\cln \tau\pi = \pi\tau \Lra \pi$ содержит либо цикл чётной длины, либо 2  цикла
равной нечётной длины.

\pt{2} $\tau\pi \neq \pi\tau \fa \tau \notin \Ab_n \Lra$ все циклы в $\pi$ разной нечётной длины.
\end{stm}

\begin{proof}
Разложим $\pi$ на независимые циклы: $\pi=\si_1\si_2\si_3\cdots\si_s$. Пункт \pt{1}. Пусть
(первый случай) $\si_1$ имеет чётную длину. Тогда просто положим $\tau = \si_1$. Второй случай: $\pi =
(i_1 \sco i_k)(j_1 \sco j_k)\si_3\cdots\si_s$ и $k=2n+1$. В этом случае положим $\tau = (i_1 j_1)(i_2
j_2)\cdots (i_k j_k)$. Тогда $\tau \pi \tau^{-1} = (i_1 \sco i_k)(j_1 \sco j_k)\si_3\cdots\si_s = \pi$.
Пункт \pt{2}. Пусть в $\pi$ все циклы разной нечётной длины. Заметим, что сопряжение действует на независимые
циклы независимо, т.е $\tau\pi\tau^{-1} = \pi \Lra \tau \si_i \tau^{-1} = \si_i \fa i$. Значит, надо
выяснить, какие перестановки коммутируют с одним циклом. Достаточно посмотреть, что происходит с циклом
$\si=(1,2 \sco n)$. Докажем, что не существует перестановки $\tau \notin \Ab_n\cln \tau\si=\si\tau$.
Предположим противное. Тогда цикловая структура $\tau$ и $\si$ должна быть одинаковая, а значит, $\tau =
\si^k$. Но это значит, что $\tau$ чётная перестановка. Противоречие.
\end{proof}

%==================================================================
\subsection{Свободные группы}
%==================================================================

\subsubsection{Системы порождающих элементов}

\begin{df}
Пусть $G$ группа. Рассмотрим подмножество $S \subset G$ и всевозможные произведения элементов из $S$  и
обратных к ним: $H=\hc{s_{i_1}^{\ep_1}s_{i_2}^{\ep_2} \cdots s_{i_k}^{\ep_k}}, \ep_i = \pm 1$. Оно называется
подгруппой, \emph{порождённой} множеством $S$. Обозначение: $H = \ha{S}$.
\end{df}

Множество $H$ действительно будет подгруппой, так как для $\fa a \; \exi a^{-1} = s_{i_k}^{-\ep_k} \cdots
s_{i_1}^{-\ep_1}, \; e = s_{i_1}s_{i_1}^{-1}$. Поскольку элементы не обязательно коммутируют, один и тот же
элемент может встречаться несколько раз. Договоримся считать пустое произведение единицей. Очевидно, $H$
наименьшая подгруппа, содержащая $S$.

\begin{ex}
Группа с 1 порождающим элементом циклическая. Пустая система порождает $\hc{e}$.
\end{ex}

\begin{df}
$S$ \emph{система порождающих} для $G$, если для
$\fa g \in G \quad g=s_{i_1}^{\ep_1}s_{i_2}^{\ep_2} \cdots s_{i_k}^{\ep_k}, \; s_{i_j} \in S, \; \ep_j = \pm 1$.
\end{df}

\begin{note}
Однозначности разложения в определении не требуется!
\end{note}

\begin{ex}
В группе диэдра $\Db_n$ есть система из двух порождающих поворот $a$ на угол $\frac{2\pi}{n}$ и
симметрия $b$ относительно некоторой оси.
\end{ex}

\begin{df}
Будем называть комбинации порождающих элементов \emph{словами}. \emph{Правильными} назовём те слова,  в
которых не встречаются комбинации вида "$\ldots s_i s_i^{-1}\ldots$".
\end{df}

\begin{df}
Если в $G$ имеет место равенство двух правильных слов $a$ и $b$, будем говорить о соотношении между этими
словами: $a=b \Lra ab^{-1}=e$.
\end{df}

\begin{ex}
В группе $\Db_n$ есть соотношения $a^n=e, b^2=e, (ab)^2=e$.
\end{ex}

\subsubsection{Свободные группы}

\begin{df}
Если из некоторого набора соотношений следуют все остальные соотношения, то этот набор называется набором
\emph{определяющих} соотношений.
\end{df}

Пусть $S$ абстрактное множество. Берём все правильные слова (\emph{формальные выражения}), составленные
из элементов $S$ и обратных к ним, е выражений вида "$s$"\; и "$s^{-1}$". Определим умножение слов $u$ и
$v$: приписываем одно слово к другому и производим сокращения на стыке слов.

\begin{stm}
Построенное таким образом множество произведений является группой.
\end{stm}
\begin{proof}
Единица есть (пустое произведение). Обратный элемент также имеется. Проверим ассоциативность.  Пусть
$u,v,w$ правильные слова. Докажем, что $(uv)w=u(vw)$. Если на стыках слов нет сокращений, то всё ясно,
иначе рассмотрим 3 случая.

\pt{1} $u=ab, \; v=b^{-1}cd, \; w=d^{-1}f$, и подслово $c \neq \es$. Тогда $(uv)w \bw= (abb^{-1}cd)d^{-1}f \bw= (acd)d^{-1}f = acf$.
С другой стороны, $u(vw) = acf$.

\pt{2} Если $c=\es$, то $u=ab, \; v=b^{-1}d, \; w \bw= d^{-1}f$, и также получаем, что $(uv)w=u(vw)$.

\pt{3} $u=acb, \; v=b^{-1}c^{-1}d, \; w=d^{-1}cf$ аналогично.
\end{proof}

\begin{df}
Построенная таким образом группа называется \emph{свободной} группой с множеством свободных  порождающих
$S$. (Название объясняется тем, что в такой группе нет нетривиальных соотношений.)
\end{df}

Теперь уточним понятие определяющих соотношений. Пусть есть свободная группа $F = \ba{\wt{S}}, \; \wt{S} =
\hc{x_i}_{i \in I}$ и группа $G$, порожденная  семейством $S = \hc{s_i}_{i \in I}$. Рассмотрим эпиморфизм
$f\cln F\ra G$, определённый по правилу $f(x_i) = s_i$. В силу однозначности записи элемента свободной группы
заданное отображение корректно. При этом $\Ker f =: N$ состоит в точности из тех слов, которые при
подстановке $x_i \mapsto s_i$ переходят в единицу группы $G$, е $x_{i_1}^{\ep_1}\cdots x_{i_k}^{\ep_k}
\in \Ker f \Lra s_{i_1}^{\ep_1} \cdots s_{i_k}^{\ep_k}=e$. По теореме о гомоморфизме $G \cong F/N$.

\begin{df}
\emph{Определяющая система соотношений} это такая совокупность правильных слов, равных в $G$ единице,
что соответствующие слова в свободной группе порождают $N$ как нормальную подгруппу.
\end{df}

$N$ является минимальной подгруппой, содержащей все эти правильные слова и порождается выбранными
соотношениями и сопряженными к ним. Заметим, что не существует алгоритма, определяющего, равны ли два
некоторых правильных слова.

\begin{ex}
Свободная группа с одним порождающим элементом бесконечная циклическая группа.
\end{ex}

\subsubsection{Определяющие соотношения в группах $\Db_n$ и $\Sb_n$}

\begin{stm}
Соотношения в группе диэдра $a^n=e, \; b^2=e, \; (ab)^2=e$ являются определяющими.
\end{stm}
\begin{proof}
$\Db_n=\ha{a,b}$. Рассмотрим свободную группу $F=\ha{x,y}$ и группу $N = \ha{x^n,y^2, (xy)^2}$.  Рассмотрим
гомоморфизм $f\cln F \ra \Db_n$ такой, что $f(x)=a, f(y)=b$. Очевидно, что $N \subseq \Ker f$. Докажем, что $F/N \cong
\Db_n$. Рассмотрим гомоморфизм $\ol{f}\cln F/N\ra \Db_n$. Найдём число смежных классов в факторгруппе:
$(x^k)N$ их $n$ штук, и $(x^k y)N$ ещё $n$ штук. Но в группе $\Db_n$ ровно $2n$ элементов, значит,
$\ol{f}$ изоморфизм и $N=\Ker f$, а это и требовалось.
\end{proof}

Рассмотрим $\Sb_n$. Очевидно, что системами порождающих являются множества всех транспозиций и всех циклов.
$\Sb_n = \ha{\tau_1=(12), \tau_2=(23) \sco \tau_{n-1}=(n-1,n)} = \ha{(12), (123) \sco (12 \sco n)}$.

\begin{problem}
Доказать, что для системы $\hc{\tau_i}$ верны соотношения
\eqn{\tau^2=e; \quad \tau_i\tau_j=\tau_j\tau_i, |i-j| \ge 2; \quad \tau_i\tau_{i+1}\tau_i=\tau_{i+1}\tau_i\tau_{i+1}.}
\end{problem}

%==================================================================
\subsection{Прямое произведение групп}
%==================================================================

\subsubsection{Понятие прямого произведения и его свойства}

\begin{df}
Дана группа $G$ и $H_1 \sco H_s \subset G$. Группа $G$ \emph{прямое произведение} $H_1 \sco H_s$, если:

\pt{1} $H_i \nl G$;

\pt{2} $G = H_1 \sd H_s$, е $\fa g \in G$ имеем $g=g_1 \ldots g_s$, где $g_i \in H_i$;

\pt{3} Разложение в пункте \pt{2} единственно.

Обозначение: $G= H_1 \st H_s$.
\end{df}

\begin{ex}
Прямая сумма подпространств по операции сложения.
\end{ex}

\begin{imp}
Если $G = H_1 \st H_s$, то $H_i \cap H_j = \hc{e}, \; i \neq j$.
\end{imp}
\begin{proof}
Допустим противное: $x \in H_i \cap H_j$. Тогда разложение неоднозначно:
$e \ldots g_i \ldots e = x = e\ldots g_j\ldots e$.
\end{proof}

\begin{df}
\emph{Коммутатором} элементов группы $x$ и $y$ называется элемент $[x,y] := xyx^{-1}y^{-1}$.
\end{df}

\begin{stm}
Если $X \nl G, \; Y \nl G$, и $X \cap Y = \hc{e}$, то их элементы коммутируют.
\end{stm}
\begin{proof}
Пусть $x \in X, \; y \in Y$. Условие $xy=yx \Lra [x,y]=e$. Но так как $xyx^{-1} \in Y$, а $yx^{-1}y^{-1} \in
X$,  то имеем $[x,y] \in X, \; [x,y] \in Y$. В силу тривиальности пересечения $[x,y]=e$.
\end{proof}

\begin{imp}
$(g_1 \ldots g_s)(g'_1\ldots g'_s) = (g_1g'_1)\ldots(g_sg'_s)$.
\end{imp}

Очевидно, что $g_1^{k_1}\ldots g_s^{k_s}=e \Lra g_i^{k_i}=e$. Следовательно,  $O(g_1\ldots g_s) =
\LCM\hc{O(g_1) \sco O(g_s)}$. Кроме того, если $|H_i| < \infty$, то $|G| = \prod |H_i|$.

\begin{stm}
Следующие условия эквивалентны:

\emph{I.} $G = H_1 \st H_s$ Группа $G$ есть прямое произведение;

\emph{II.} \pt{1}, \pt{2} - те же, что и в определении; \pt{3} $H_j \cap \bigl(H_1 \ldots H_{j-1} H_{j+1} \ldots H_s \bigr) = \hc{e}$;

\emph{III.} \pt{1}, \pt{2} - те же самые; \pt{3} $H_j \cap \bigl(H_1 \ldots H_{j-1}\bigr) = \hc{e}$;

\emph{IV.} \pt{1} $H_i$ подгруппы, $\fa g_i \in H_i, \fa g_j \in H_j \;\; g_i g_j = g_j g_i$; \pt{2} - то же самое; \pt{3}
любое из \emph{I}, \emph{II}, \emph{III}.
\end{stm}

\begin{proof}
\emph{II.} То, что из определения следует тривиальность пересечения, было доказано выше.  Теперь выведем
единственность разложения из свойств пункта \emph{II}. Допустим противное, е существует $g=h_1 \ldots
h_s = h_1' \ldots h_s'$. Преобразуем равенство к виду $h_1'^{-1}h_1=h_2' \ldots h_s'h_s^{-1} \ldots
h_2^{-1}$. Слева стоит произведение элементов из $H_1$, справа из $H_2 \ldots H_s$. Пересечение
тривиально $ \Ra h_1'^{-1}h_1=e \Lra h_1=h_1'$. Аналогично $h_i=h_i' \fa i$, что и означает единственность
разложения. \emph{III}: Доказывается аналогично \emph{II}, пользуясь индукцией по числу подгрупп.
\emph{IV}: Докажем нормальность подгрупп $H_i$. Пусть $h_j \in H_j$. Тогда $g h_j g^{-1} = g_1 \ldots g_s
h_j g_s^{-1} \ldots g_1^{-1}$. По условию элементы коммутируют, значит, их можно переставить с $h_j$, и они
сократятся, а значит, $gh_jg^{-1}=h_j$, что и даёт нормальность.
\end{proof}

Теперь рассмотрим случай конечной группы $G$. Тогда можно привести ещё несколько эквивалентных  определений
прямого произведения.

\begin{stm}
Следующие условия эквивалентны определению прямого произведения для конечных групп:

\emph{V.} \pt{1}, \pt{2} те же, что и в первых 4 группах условий; \pt{3} $|G|=|H_1| \ldots |H_s|$;

\emph{VI.} \pt{1} то же самое; \pt{2} $|G|=|H_1|\ldots|H_s|$; \pt{3} любое из предыдущих.
\end{stm}

\begin{proof}
\emph{VI}: Рассмотрим $H= H_1 \st H_s$ прямое произведение. Тогда $|H|=|H_1|\ldots|H_s|$, так как
группы конечны. Следовательно, $H=G$. \emph{V}: Выведем единственность. Из \pt{1} и \pt{2} следует, что
$|G| \le |H_1|\ldots|H_s|$. Рассмотрим всевозможные произведения $g_1 \ldots g_s$. Если бы какие-то из них
совпадали, то неравенство было бы строгим, а такого по условию не бывает. Утверждение доказано.
\end{proof}

В аддитивной терминологии прямое произведение называется прямой суммой: $G = H_1 \sop H_s$.

\subsubsection{Разложение циклических групп}

\begin{ex}
$(\Z,+)$ не разлагается в прямое произведение, так как для $\fa m,n$ имеем $mn\Z \subset m\Z \cap n\Z$.
\end{ex}

\begin{ex}
$G = \ha{a}_n$, где $n$ конечно. Если $H$ подгруппа в $G$, то $H=\ha{a^d}, \; |H| = \frac{n}{d}$. Если
$n = p^k$, где число $p$ простое, то $H_i = \ha{a^{p^i}}, i \le k$. Тогда прямого произведения нет:
\eqn{\ha{a} \sups \ha{a^p} \sups \ldots \sups \ha{a^{p^{k-1}}} \sups \hc{e}.}
\end{ex}

\begin{df}
Группа порядка $p^k$ называется $p$-\emph{примарной} группой.
\end{df}

Пусть $n=p_1^{k_1} \ldots p_s^{k_s}$, и $p_i \neq p_j, \; i \neq j$. Пусть $m_i:= p_i^{k_i}$. Рассмотрим
группы $H_i=\ha{a^\frac{n}{m_i}}_{m_i}$ и их произведение $G= \prod H_i$. Если $x \in H_j$, то $O(x)$
некоторая степень числа $p_j$. Пусть $x \in \bigcap H_i$. Тогда  очевидно, что $O(x)=1$ и $x=e$. Кроме того,
имеем $\prod |H_i| = \prod p_i^{k_i}=|G|$. Тем самым любая циклическая группа разлагается в прямое
произведение примарных циклических групп.

\subsubsection{Внешнее прямое произведение}

\begin{df}
Пусть $G_1 \sco G_s$ группы. Составим из них группу
\eqn{G := \hc{(g_1 \sco g_s) \vl g_i \in G_i} = G_1 \dot{\times} \ldots \dot{\times} G_s.}
Она является множеством векторов строк из $g_i$ и называется \emph{внешним прямым произведением} групп $G_i$.
\end{df}

Можно отождествить внешнее и внутреннее произведение следующим образом. Пусть $G = H_1 \st H_s$
(произведение подгрупп), а $\wt{G} = G_1 \dot{\times} \ldots \dot{\times} G_s$ (внешнее произведение). Построим
изоморфизм $\ph\cln G \ra \wt{G}$ по правилу $g_1 \ldots g_s \corr{\ph} (g_1 \sco g_s)$. Отображение задано
корректно, так как элементы из разных подгрупп коммутируют, и $(g_1 \ldots g_s)(g_1' \ldots g_s')=(g_1 g_1')
\ldots (g_s g_s') \mapsto (g_1g_1' \sco g_s g_s')$. Подгруппы $H_i$ можно отождествить с множествами векторов
$\hc{(e \sco e, g_i, e \sco e)}$, е c подгруппами во внешнем произведении, и таким образом, внешнее и
внутреннее произведения можно не различать.

\subsubsection{Гомоморфизмы произведений групп}

Пусть есть гомоморфизм $G \ra H= H_1 \st H_s$ и набор гомоморфизмов $\ph_i\cln G \ra H_i$. Определим  гомоморфизм
$\ph\cln G \ra H$ следующим образом: $\ph(g):=\bigl(\ph_1(g) \sco \ph_s(g)\bigr)$. Наоборот, по отображению
$\ph$ можно определить $\ph_i$. Имеется биекция между набором гомоморфизмов из $G$ в $H$ и множеством
$\hc{\ph_i \vl \ph_i\cln G\ra H_i}$.

Зададим гомоморфизм $\ph\cln G= G_1 \st G_s \ra H$. Достаточно задать ограничения
$\ph_i=\ph\evn{G_i}, \; \ph_i\cln G_i \ra H$. Отображение $\ph$ задается ограничениями  однозначно: $\ph(g) = \ph(g_1 \ldots
g_s) = \bigl(\ph_1(g_1) \ldots \ph_s(g_s)\bigr)$. Отображение $\ph$ будет гомоморфизмом, когда элементы
образов различных $\ph_i$ коммутируют: $y_i \in \Img \ph_i, \; y_j \in \Img \ph_j \Ra y_i y_j = y_j y_i$. В
том случае, когда группа $H$ абелева, то всегда есть биективное соответствие между множеством
гомоморфизмов из $G$ в $H$ и наборами $\hc{\ph_1 \ldots \ph_s}$.

\begin{theorem}
Пусть $G = G_1 \st G_s$, и $H_i \nl G_i$. Тогда $H \nl G$ и $G/H \cong G_1/H_1 \st G_s/H_s$.
\end{theorem}
\begin{proof}
Построим эпиморфизм $\ph\cln G \ra G_1/H_1 \st G_s/H_s$. Положим $\ph(g_1 \ldots g_s) = (g_1H_1 \sco g_s H_s)$.
Очевидно, что отображение задано корректно. Найдём его ядро. Имеем
\eqn{\Ker \ph = \hc{g_1 \ldots g_s\cln g_i H_i = H_i \Ra g_i \in H_i \fa i}.}
Следовательно, $\Ker \ph = H_1 \st H_s$, а по теореме о гомоморфизме $G/\Ker \ph
\cong \Img \ph$.
\end{proof}

\begin{imp}
Частный случай теоремы: $G=H_1 \times H_2 \Ra G/H_1 \cong H_2$.
\end{imp}

%==================================================================
\subsection{Абелевы группы}
%==================================================================

\subsubsection{Основные свойства}

В абелевой группе всякая подгруппа нормальна.

Рассмотрим гомоморфизмы вида $\ph\cln G \ra K$, где $G$ произвольная группа, $K$ абелева. Попробуем
ввести на множестве гомоморфизмов структуру группы. Определим произведение гомоморфизмов так:  $(\ph_1 \cdot
\ph_2)(x) \bw{:=} \ph_1(x) \ph_2(x)$. Если $K$ абелева, то произведение гомоморфизмов есть гомоморфизм
(\emph{для неабелевых групп это неверно!!!}):

$$(\ph \cdot \psi)(xy) = \ph(xy)\psi(xy)=\ph(x)\ub{\ph(y)\psi(x)}_{\text{комм.}}\psi(y) = \bigl(\ph(x)\psi(x)\bigr)\bigl(\ph(y)\psi(y)\bigr)
= (\ph \cdot \psi)(x) (\ph \cdot \psi)(y).$$

Определим обратное отображение $\ph^{-1}(x) := \ph(x)^{-1}$. Оно будет гомоморфизмом только тогда,  когда
$K$ абелева группа. В качестве нейтрального элемента возьмём гомоморфизм, переводящий всё в единицу.
Таким образом мы построили группу $\Hom(G,K)$. Заметим, что она будет абелевой.

\subsubsection{Системы порождающих в абелевой группе}

Поскольку в абелевых группах все элементы коммутируют, правильными словами будут те, в которых все  элементы
попарно различны. Пусть дана группа $(G,+)$. Тогда любой элемент представляется в виде целочисленной линейной
комбинации порождающих, в которой только конечное число коэффициентов отличны от нуля:
$g = \sums{i \in I} n_i a_i, \; n_i \in \Z$. Однако запись элемента по-прежнему неоднозначна.

\begin{problem}
Доказать, что какую бы мы не взяли систему порождающих в группах $(\Q, +)$ и $(\Q, \cdot)$, из неё  можно
выкинуть какой-то элемент.
\end{problem}

Будем теперь рассматривать абелевы группы с конечной системой порождающих.

\begin{df}
Конечнопорождённая абелева группа $G$ называется \emph{свободной абелевой} группой со свободной  системой
порождающих $\hc{a_1 \sco a_n}$, если запись элемента в виде целочисленной линейной комбинации этих
порождающих однозначна. Система порождающих называется в этом случае \emph{базисом}, а число $n$
\emph{рангом} группы.
\end{df}

Абелева группа свободна, если она обладает базисом. $G = \ha{a_1}_\infty \sop \ha{a_n}_\infty$.

\begin{theorem}
Конечнопорождённая абелева группа изоморфна некоторой факторгруппе свободной группы того же ранга, е если
$F$ свободная абелева группа с базисом $\hc{x_1 \sco x_n}$, а $G = \ha{a_1 \sco a_n}$, то
найдётся подгруппа $H \subs F$, такая что $G \cong F/H$.
\end{theorem}
\begin{proof}
Определим эпиморфизм $\ph\cln F \ra G$ так: $\ph(x_i) := a_i$. Тогда $\ph(\sum n_i x_i) = \sum n_i a_i$.
Очевидно, его образ есть вся группа $G$. По теореме о гомоморфизме $F/\Ker \ph \cong G$. Ядро $\ph$  и будет
искомой подгруппой в $F$.
\end{proof}

\subsubsection{Разложение конечнопорождённых абелевых групп}

Пусть $F$ свободная абелева группа с базисом $\hc{x_1 \sco x_n}$, и $H \subseq F$, $\Ga$ некоторое множество
индексов, и $\bc{y_\ga = \sum a_{i\ga} x_i \bvl  a_{i\ga} \in \Z, \ga \in \Ga}$ система порождающих
подгруппы $H$. Числа $a_{i\ga}$ образуют матрицу $A$, в которой $n$ строк и,  возможно, бесконечное число
столбцов. Есть 3 типа целочисленных элементарных преобразований (ЭП) такой матрицы:

\pt{1} К одной строке можно прибавить другую, умноженную на целое число;

\pt{2} Можно менять строки местами;

\pt{3} Можно умножать строку на обратимые элементы кольца $\Z$, то есть на $\pm 1$.

Аналогичные преобразования можно осуществлять со столбцами.

\begin{theorem}
Посредством элементарных преобразований можно привести матрицу коэффициентов к «диагональному» виду
$\diag(d_1 \sco d_n)$. При этом $d_1 | d_2 | \ldots | d_n$ и итоговый вид матрицы определён однозначно.
\end{theorem}
\begin{proof}
Докажем индукцией по числу строк (их конечное число). Если матрица нулевая, доказывать нечего. База
индукции: $n=1$. Выберем наименьший по модулю ненулевой элемент $a_{i\ga}$. Если все остальные элементы
делятся на него, то можно путём ЭП первого типа обнулить все эти элементы. Если существует элемент
$a_{i\de}$, не делящийся на $a_{i\ga}$, то с помощью столбца $\ga$ поделим $a_{i\de}$ с остатком,
и получим меньший по модулю элемент (остаток). После конечного числа шагов все элементы строки будут делиться
на какой-то из её элементов. База индукции есть.

Теперь предположим, что всё доказано для $n-1$ строки, тогда докажем для $n$ строк. Выберем наименьший  по
модулю ненулевой элемент, и проведём аналогичные действия для той строки и того столбца, в котором стоит этот
элемент. (Если всё на него делится, тогда можно всё кроме него обнулить, а если не делится, то поделим с
остатком, и т.д.) Таким образом можно обнулить некоторый «крест» в матрице, а на пересечении строки и столбца
этого креста стоит ненулевой элемент $d_k$, и все остальные элементы на него делятся. Путём перестановки
строк и столбцов можно сдвинуть крест в левый верхний угол матрицы. Тогда останется матрица с $n-1$ строкой,
и шаг индукции доказан. Очевидно, что на каждом шаге каждый ненулевой элемент $d_k$ делит все элементы минора
матрицы порядка $k$ и является наибольшим общим делителем чисел этого минора. Из алгоритма Евклида следует,
что при элементарных преобразованиях указанного типа НОДы элементов не меняются (сам алгоритм Евклида есть
последовательность таких преобразований). Значит, набор $d_1 \sco d_n$ определён однозначно.
\end{proof}

Изучим влияние ЭП на базис. Очевидно, что ЭП 2 и 3 типа несущественны. Остается разобрать  случай ЭП первого
типа. Для строк $(i), (j)$: при ЭП $(i) \mapsto (i) + \la (j), \; \la \in \Z$ имеем \eqn{y_\ga = (a_{ij}
+ \la a_{j\ga})x_i + a_{j\ga}(x_j - \la x_i) + \sums{k\neq i,j} a_{k\ga}x_k.} Очевидно, что такое
преобразование соответствует элементарной замене базиса $x_j \mapsto x_j-\la x_i$. Рассуждения для
столбцов аналогичны.

\begin{theorem}
Пусть $F$ свободная абелева группа, $H \subseq F$. Тогда можно выбрать новый базис $F=\ha{x_1' \sco
x_n'}$ и  новую систему порождающих $H=\ha{y_1' \sco y_n'}$ так, что $y_j'=d_j x_j', \; d_1 | d_2 | \ldots |
d_n$ и $H$ является свободной группой с рангом $\rk H \le n$.
\end{theorem}
\begin{proof}
Следует из предыдущей теоремы. Приведением матрицы к диагональному виду элементарными преобразованиями и
соответствующими им заменами базисов можно найти требуемый базис и систему порождающих. Пусть $d_1 \sco d_k
\neq 0, d_{k+1} \sco d_n = 0$. Тогда $H = \ha{y_1' \sco y_k'}$. Эти порождающие свободны, так как если бы
$\la_1 y_1' \spl \la_k y_k'=0$, то и $\la_1 d_1 x_1' \spl \la_k d_k x_k'=0$, а это
противоречит тому, что $\hc{x_1 \sco x_n}$ базис $F$.
\end{proof}

\begin{theorem}[Существование разложения]
Любая конечно порождённая абелева группа разлагается в прямую сумму конечного числа бесконечных циклических
групп и примарных циклических групп.
\end{theorem}
\begin{proof}
Пусть $G = \ha{a_1 \sco a_n}$ абелева группа. Поскольку $G \cong F/H$, по предыдущей теореме существует базис:
$F=\ha{x_1 \sco x_n}$, $H= \ha{d_1 x_1 \sco d_k x_k, d_{k+1} x_{k+1} \sco d_n x_n}$, где $d_{k+1}= \ldots = d_n =0$. Имеем
$$F=\ha{x_1}_\infty  \sop \ha{x_n}_\infty,
H=\ha{d_1 x_1}_\infty \sop \ha{d_k x_k}_\infty \oplus \ub{\ha{d_{k+1}x_{k+1}}_\infty \sop \ha{d_n x_n}_\infty}_0.$$
По теореме о факторизации по прямым слагаемым
$$G \cong \frac{\ha{x_1}}{\ha{d_1 x_1}} \sop \frac{\ha{x_k}}{\ha{d_k x_k}} \oplus \frac{\ha{x_{k+1}}}{\hc{0}} \sop \frac{\ha{x_n}}{\hc{0}}.$$
Если $d_i > 1$, то $i$-е слагаемое есть циклическая группа $\Z/d_i\Z$. Если $d_i = 0$, остается бесконечное циклическое слагаемое $\Z$. Если же $d_i=1$,
то $\ha{x}/\ha{x}=\hc{0}$, и такое слагаемое можно отбросить. Итак, получаем разложение
$$G \cong \oplusl{i=1}{k} \br{\Z/d_i\Z} \oplus \Z^{n-k}.$$
В свою очередь, каждую конечную циклическую группу разложим на примарные, что и требовалось.
\end{proof}

\begin{df}
Приведённое выше разложение называется \emph{каноническим}, если $d_1 | \ldots | d_k$.
\end{df}

\begin{theorem}[Единственность разложения]
В разложении абелевой группы в прямую сумму циклических групп число слагаемых и их порядки определены однозначно.
\end{theorem}
\begin{proof}
Пусть $G= \ha{c_1}_{p_1^{k_1}} \sop \ha{c_s}_{p_s^{k_s}} \oplus \ha{c_{s+1}}_\infty  \sop \ha{c_{s+t}}_\infty$.
Рассмотрим \emph{\emph{подгруппу кручения}} (е подгруппу элементов конечного порядка)
$$\Tor G := \hc{a \in G\cln ma=0 \text{ для некоторого } m \in \Z, \; m \neq 0}.$$
Очевидно, что $\Tor G$ есть сумма конечных слагаемых ($s$ штук). Тогда имеем
$G=\Tor G \oplus \Z^t$, или $G/\Tor G \cong \Z^t$.
Определение $\Tor G$
не зависит от разложения, а значит и число $t$ не зависит от разложения.

Разберёмся теперь с конечными слагаемыми. Для каждого $p$ рассмотрим подгруппы $p$-кручения
$$\Tor_p G := \hc{a \in G\cln p^k a = 0 \text{ для некоторого } k \in \Z},$$
е суммы $p$-примарных слагаемых при фиксированном $p$. Аналогично первому случаю число таких подгрупп
определяется однозначно. Остаётся рассмотреть случай, когда $G$ примарная группа  порядка $p^k$. Пусть
есть разложение $G= \ha{c_1}_{p^{k_1}} \sop \ha{c_r}_{p^{k_r}}$, и $k_1 \spl k_s = k$. Докажем индукцией по
$k$, что набор чисел $\hc{k_1 \sco k_r}$ от разложения не зависит. При $k=1$ всё очевидно. Пусть $k>1$. Тогда
рассмотрим подгруппу $pG := \hc{pa \vl a \in G}$. Очевидно, что
\eqn{p G = \ha{pc_1}_{p^{k_1-1}} \sop \ha{pc_r}_{p^{k_r-1}}.}
Если $k_i=1$, то это слагаемое при умножении на $p$ исчезнет. Определение $pG$ от
разложения не зависит, а по предположению индукции для порядка меньше $p^k$ набор чисел $\hc{k_i}$ не зависит
от разложения. Тем самым теорема доказана.
\end{proof}

\begin{note}
Подгруппу кручения иногда называют \emph{\emph{периодической частью}} абелевой группы.
\end{note}

Как по разложению определить, изоморфны ли группы? Нужно разложить их в примарные циклические. В~силу
единственности разложения можно по нему судить об изоморфности групп.

\subsubsection{Конечные абелевы группы}

\begin{df}
\emph{Показателем} группы называется число $d := \min\hc{k > 0\cln x^k = e \fa x \in G}$.
\end{df}

Выясним, когда прямая сумма циклических групп циклическая. Пусть $G= \ha{a_1}_{n_1} \sop \ha{a_s}_{n_s}$, и
$|G| \bw = n \bw = n_1 \ldots n_s$. Группа $G$ циклическая, когда в ней есть элемент порядка $n$. Для $\fa x=(x_1
\sco x_s)$ имеем  $O(x_i) | n_i$. Возьмём элемент $a=(a_1 \sco a_s)$ он, очевидно, имеет наибольший
порядок. $O(a)=\LCM\hc{n_1 \sco n_s}$. Значит, $G$ циклическая $\Lra$ числа $n_1 \sco n_s$ попарно
взаимно просты.

\begin{stm}
Конечная абелева группа $G$ циклическая $\Lra$ её показатель $d$ равен порядку группы.
\end{stm}
\begin{proof}
Очевидно, что $d=\LCM\hc{O(x) | x \in G}$. Разложим группу в прямую сумму циклических групп. Если их  порядки
не взаимно просты, то $d < |G|$ и группа не циклическая. Наоборот, если порядки слагаемых взаимно просты, то
$d=|G|$ и группа циклическая.
\end{proof}

Есть биекция между конечными абелевыми группами порядка $n$ и разложениями числа $n$ в произведение  степеней
простых чисел.

\begin{ex}
$|G|=72=8\cdot 9 = 4\cdot 2\cdot 9 = 2\cdot 2\cdot 2\cdot 9 = 8\cdot 3\cdot 3 = 4\cdot 2\cdot 3\cdot 3 = 2\cdot 2\cdot 2\cdot 3\cdot 3$.
$$G_1 = \Z_8 \oplus \Z_9$$
$$G_2 = \Z_4 \oplus \Z_2 \oplus \Z_9$$
$$\ldots$$
$$G_6 = \Z_2 \oplus \Z_2 \oplus \Z_2 \oplus \Z_3 \oplus \Z_3$$
\end{ex}

\subsubsection{Свойства подгрупп в мультипликативной группе поля}

Пусть $K$ поле, $K^*=K \wo \hc{0}$ его мультипликативная группа.

\begin{stm}
Любая конечная подгруппа в мультипликативной группе поля циклическая.
\end{stm}
\begin{proof}
Пусть $G \subseq K^*, \; |G|=n, \; d$ показатель $G$. Имеем $x^d = 1 \fa x \in G$. Поскольку уравнение
$x^d-1=0$ имеет не более чем $d$ корней, то $|G| \le d \Ra |G|=d$, что и означает цикличность группы.
\end{proof}

\subsubsection{Геометрические приложения абелевых групп. Дискретные подгруппы в $\R^n$}

\begin{df}
Пусть $V=\R^n$ евклидово пространство. Подгруппа $G \subseq (V, +)$ называется \emph{решёткой}, если:

\pt{1} $G$ дискретна, е $\exi \ep > 0\cln \hc{x \in V\cln |x| < \ep} \cap G = \hc{0}$;

\pt{2} $\ha{G}=V$, е линейная оболочка $G$ с вещественными коэффициентами совпадает с $V$.
\end{df}

\begin{theorem}
Любая решётка $G$ в $\R^n$ является свободной абелевой группой, порождённой некоторым базисом  пространства
$V$, е $G = \hc{\sum k_i e_i \vl k_i \in \Z}$.
\end{theorem}
\begin{proof}
Из второго пункта определения $G$ следует, что существует базис $u_1 \sco u_n$ пространства $V$, где $u_i \in G$. Пусть
$x \in G$, тогда $x = \sum \xi_i u_i$. Возможны 3 случая для коэффициентов решётки:

\pt{1} $\fa x \in G \fa \xi_i \in \Q$, и все знаменатели у $\xi_i$ ограничены сверху;

\pt{2} $\xi_i \in Q$, но имеют сколь угодно большие знаменатели;

\pt{2} $\exi x$, у которого хотя бы одна из координат иррациональна.

Покажем, что из всех случаев возможен только случай \pt{1}. Пусть $F = \hc{\sum k_i u_i  \bvl k_i \in \Z}$
свободная абелева группа ранга $n$ (здесь и далее суммирование идёт по $i=\ol{1,\,n}$). Взяв НОК всех знаменателей
коэффициентов, можно найти $N$ такое, что $Nx \in F$ для $\fa x \in G$. Очевидно, что  отображение $x \corr{\ph} Nx$ есть инъективный гомоморфизм $G \ra
F$. Тогда $G \cong \Img \ph \subseq F \Ra G$ свободная группа (как подгруппа свободной группы), и $\rk G
\le n$. Но так как $\ha{G}=V$, то $\rk G = n$. Значит, базис $V$ есть базис $G$.

Докажем, что случаи \pt{2} и \pt{3} для дискретной подгруппы невозможны. Во втором случае, так как
знаменатели коэффициентов неограниченны, можно отбросить целые части координат и рассмотреть в $G$
подмножество $M := \hc{x \in G \vl x = \sum \xi_i u_i, \xi_i \in [0,1]}$. Оно, очевидно,
ограничено и бесконечно, а значит, обладает предельной точкой, в любой окрестности которой есть точки из $G$.
Это противоречит дискретности.

В третьем случае пусть $x=\xi_1 u_1 \spl \xi_n u_n$ и для определённости координата $\xi_1 \notin \Q$.
Рассмотрим последовательность  векторов $x_m = \hc{mx \vl m \in \Z}$ (аналогично отбросим целые части
координат и оставим только дробные). Докажем, что все элементы этой последовательности различны. Пусть $mx \bw=lx$,
тогда $mx\bw-lx \bw= \suml{i=1}{n}k_i u_i, k_i \in \Z$, а значит, $m\xi_1 - l\xi_1 = k_1$, е $\xi_1 \in
\Q$, а по условию это не так. Значит, все $x_i$ различны. Но они все находятся в ограниченной области, а
значит есть предельная точка и противоречие с дискретностью $G$.
\end{proof}

%==================================================================
\subsection{Нормальные ряды группы. Теорема Жордана Гёльдера}
%==================================================================

\begin{df}
\emph{Нормальным рядом} называется последовательность подгрупп
\eqn{G=H_0 \nr H_1 \nr H_2 \nr \ldots \nr H_k = \hc{e}.}
Число $k$ называется \emph{длиной} нормального ряда,  а факторгруппы $H_{k-1}/H_k$ \emph{факторами} нормального ряда.
\end{df}

\begin{df}
Пусть дан нормальный ряд $G=H_0 \nr H_1 \nr H_2 \nr \ldots \nr H_k = \hc{e}$. Другой нормальный ряд
$G \bw= F_0 \nr F_1 \nr F_2 \nr \ldots \nr F_m = \hc{e}$ называется
\emph{уплотнением} первого, если все подгруппы первого ряда встречаются во втором.
\end{df}
В ряду любой член может повторяться несколько раз. Если этого нет, то говорят о ряде без повторений.

\begin{df}
\emph{Композиционный} ряд нормальный ряд, который нельзя уплотнить (без повторений).
\end{df}

\begin{df}
Группа называется \emph{простой}, если в ней нет нетривиальных нормальных подгрупп.
\end{df}

Очевидно, что ряд композиционный $\Lra$ все его факторы простые.

\begin{theorem}[Жордана Гёльдера]
Пусть $G$ обладает композиционным рядом длины $k$. Тогда все нормальные ряды в $G$ имеют длину не  больше
$k$, а все композиционные ряды имеют одинаковую длину и их факторы изоморфны после некоторой перестановки.
\end{theorem}
\begin{proof}
Пусть в $G$ есть соответственно композиционный и нормальный ряды
$$G = H_0 \nr H_1 \nr H_2 \nr \ldots \nr H_k = \hc{e}, \quad G = K_0 \nr K_1 \nr K_2 \nr \ldots \nr K_m = \hc{e}.$$
Докажем, что $m \le k$. Проведём индукцию по числу $k$. Если $k=1$, то группа простая и всё очевидно.  Пусть
утверждение теоремы верно для рядов длины меньше $k$, докажем для рядов длины $k$.

\pt{1} Первый случай: $K_1 \subseq H_1$. Тогда $H_1 \nr K_1 \nr K_2 \nr \ldots K_m$ нормальный ряд  в
$H_1$. Но по условию в $H_1$ есть композиционный ряд длины $k-1$, а тогда по предположению индукции все ряды
в $H_1$ имеют длину $\le k-1$. Значит, $m-1 \le k-1 \Ra m \le k$.

\pt{2} Второй случай: $K_1 \nsubseteq H_1$. Но тогда и $H_1 \nsubseteq K_1$, поскольку в противном случае  по
теореме о соответствии $K_1/H_1 \nl G/H_1$, а этого не бывает, так как все факторы простые. Обозначим $L :=
(H_1 \cap K_1)$. Итак, имеем $L \nl H_1, L \neq H_1$. Рассмотрим $H_1 K_1$. Имеем $H_1 \varsubsetneqq H_1 K_1
\nl G \Ra H_1K_1 = G$. По второй теореме об изоморфизме $G/H_1 = H_1 K_1/H_1 \cong K_1/L$ и $G/K_1 =
H_1K_1/K_1 \cong H_1/L$. В $H_1$ есть композиционный ряд длины $\le k-1$, а в $L$ есть композиционный ряд
длины $\le k-2$, так как $L \nl H_1$. С другой стороны, $L \nl K_1, L \neq K_1$, и $K_1/L$ простая
группа. Значит, и в $K_1$ есть композиционный ряд длины $k-1$. Значит, можно применить индуктивное
предположение, и во втором случае также получаем $m \le k$. Отсюда следует, что все композиционные ряды в $G$
имеют одинаковую длину.

Теперь докажем второе утверждение об изоморфности факторов. Пусть в $G$ есть два композиционных ряда

$$G = H_0 \nr H_1 \nr H_2 \nr \ldots \nr H_k = \hc{e}, \quad G = K_0 \nr K_1 \nr K_2 \nr \ldots \nr K_k = \hc{e}.$$
Проведём индукцию по длине ряда. Рассмотрим два случая. Первый случай: $H_1=K_1$. Тогда применим
предположение индукции к $H_1$ и сведём утверждение к меньшему числу факторов. Второй случай: $H_1 \neq K_1$.
Тогда рассмотрим подгруппу $L_2 := (H_1 \cap K_1)$. Имеем $L_2 \nl H_1 \nl G$ и $L_2 \nl K_1 \nl G$. Тогда
имеем ряды

(1) ~~ $G \nr H_1 \nr H_2 \nr H_3 \nr \ldots$

(2) ~~ $G \nr H_1 \nr L_2 \nr L_3 \nr L_4 \nr \ldots$

(3) ~~ $G \nr K_1 \nr L_2 \nr L_3 \nr L_4 \nr \ldots$

(4) ~~ $G \nr K_1 \nr K_2 \nr K_3 \nr \ldots$

Здесь $L_3 \nr L_4 \nr \ldots$ некоторый композиционный ряд в $L_2$. Посмотрим на первые два ряда.  В
$H_1$ имеем два композиционных ряда длины $k-1$, а значит по индуктивному предположению их факторы
$\frac{H_1}{H_2}, \frac{H_2}{H_3}, \ldots$ и $\frac{H_1}{L_2}, \frac{L_2}{L_3}, \ldots$ изоморфны после
некоторой перестановки. То же самое можно сказать про два последних ряда: $\frac{K_1}{K_2}, \frac{K_2}{K_3},
\ldots$ и $\frac{K_1}{L_2}, \frac{L_2}{L_3}, \ldots$ Остаётся заметить, что $G=H_1K_1$, тогда $\frac{G}{H_1}
= \frac{H_1K_1}{H_1} \cong \frac{K_1}{L_2}$, и $\frac{G}{K_1} = \frac{H_1K_1}{K_1} \cong \frac{H_1}{L_2}$, то
есть первые и вторые факторы во втором и третьем рядах изоморфны «крест-накрест». Теорема доказана.
\end{proof}

\begin{ex}
Композиционные ряды векторных пространств есть цепочки вложенных подпространств. Все ряды одинаковой  длины
$\Ra$ число элементов в базисе одинаково.
\end{ex}

%==================================================================
\subsection{Коммутант. Разрешимые группы. Простые группы}
%==================================================================

\subsubsection{Коммутант}

\begin{df}
Группа называется \emph{разрешимой}, если она обладает нормальным рядом с абелевыми факторами.
\end{df}

Попытаемся построить такой ряд. Нужно построить подгруппу $N \nl G$, такую, что $G/N$ абелева.  Для этого
возьмём коммутаторы $[a,b]$ всех элементов группы.

\begin{lemma}
Дана группа $G$ и $N \nl G$. Факторгруппа $G/N$ будет абелевой $\Lra [a,b] \in N \fa a, b \in G$.
\end{lemma}
\begin{proof}
$G/N$ абелева $\Lra [aN, bN] = eN = N \Lra (aN)(bN)(a^{-1}N)(b^{-1}N) = aba^{-1}b^{-1}N = N \Lra [a,b] \in N$.
\end{proof}

Свойства коммутатора: $[a,b]^{-1}=[b,a]$. Кроме того, сопряженный к коммутатору есть коммутатор  сопряженных:
\eqn{g[a,b]g^{-1} = gaba^{-1}b^{-1}g^{-1}= gag^{-1}gbg^{-1}ga^{-1}g^{-1}gb^{-1}g^{-1} = [gag^{-1}, gbg^{-1}].}

\begin{df}
Подгруппа, порожденная всеми коммутаторами в группе $G$, называется \emph{коммутантом} $G$ и  обозначается
$G'$. Коммутант иногда называют \emph{производной} подгруппой.
\end{df}

Очевидно, что коммутант есть наименьшая нормальная подгруппа в $G$, факторгруппа по которой абелева.
Приведём некоторые другие свойства коммутанта.

\begin{lemma}
При гомоморфизме $f\cln G \ra K$ имеем $f(G') \subset K'$. Если $f$ эпиморфизм, то $f(G')=K'$.
\end{lemma}
\begin{proof}
Имеем
\eqn{f\br{[a,b]} = f(a)f(b)f(a)^{-1}f(b)^{-1} = \bs{f(a),f(b)}.}
Пусть $f(a)=x, f(b)=y$. По доказанному $[x,y] \bw\in K'$. Если $f$ сюръективен, то любой элемент
$K'$ есть образ некоторого коммутатора, значит, $f(G')=K'$.
\end{proof}

По индукции определяются коммутанты высших порядков: $G^{(i+1)} := \bigl(G^{(i)}\bigr)'$.
По индукции очевидным образом доказывается предыдущее утверждение для коммутантов произвольного порядка.

\subsubsection{Разрешимость групп}

Построим ряд из коммутантов группы: $G \nr G^{(1)} \nr G^{(2)} \nr \ldots$

\begin{theorem}
Группа $G$ разрешима $\Lra \exi l\cln G^{(l)}= \hc{e}$, е существует нормальный ряд из коммутантов группы.
\end{theorem}
\begin{proof}
Пусть есть ряд с абелевыми факторами $G \nr H_1 \nr H_2 \nr \ldots$ Докажем по индукции,  что
$G^{(i)}\subseq H_i \fa i$. База: $G' \subseq H_1$, так как $G/H_1$ абелева. Шаг индукции: пусть
$G^{(i-1)} \subseq H_{i-1}$. Так как $H_{i-1}/H_i$ абелева, то $H_{i-1}' \subseq H_i$. Тогда имеем
$G^{(i)} \subseq H_{i-1}' \subseq H_i$, что и требовалось доказать.
\end{proof}

\begin{imp}
Всякая подгруппа разрешимой группы разрешима.
\end{imp}
\begin{proof}
В самом деле, если $G^{(l)}=\hc{e}$ и $H \subseq G$, то $H^{(l)}\subseq G^{(l)} \Ra H^{(l)} = \hc{e}$.
\end{proof}

\begin{stm}
Пусть $H \nl G$, группы $H$ и $G/H$ разрешимы. Тогда $G$ разрешима.
\end{stm}
\begin{proof}
В силу разрешимости $\exi n, m\cln (G/H)^{(n)}=\hc{\ol{e}}$, и $H^{(m)}=\hc{e}$.
Рассмотрим гомоморфизм $\ph\cln G \ra G/H$. По доказанному $\ph(G^{(n)}) \subseq (G/H)^{(n)}=\hc{\ol{e}}$, е
$G^{(n)} \subseq \Ker \ph = H$. Значит, $G^{(n+m)}=\hc{e}$.
\end{proof}

\subsubsection{Примеры разрешимых групп}

Любая абелева группа, очевидно, разрешима. Группа $\Db_n$ разрешима, так как в ней есть нормальная подгруппа
вращений $R$, а $\Db_n/R \cong \Z_2$ абелева.
Группы $\Sb_3$ и $\Ab_3$ разрешимы, так как $\Ab_3$ абелева, а $\Sb_3 \cong \Db_3$.

Рассмотрим более сложный пример: невырожденные верхнетреугольные матрицы $\Tb_n(\K)$ над полем $\K$.

\begin{stm}
Группа верхнетреугольных матриц $\Tb_n(\K)$ разрешима.
\end{stm}
\begin{proof}
Зададим гомоморфизм $f\cln \Tb_n(\K) \ra (\K^*)^n$, где $\K^*$ мультипликативная группа поля $\K$:
$$\rbmat{a_1 & & * \\ & \ddots & \\ 0 & & a_n} \corr{f} (a_1 \sco a_n).$$
Очевидно, что $\Ker f = \UT_n(\K)$ группа верхних унитреугольных матриц. $\Img f$ есть абелева  группа
(множество векторов-строк). Остаётся доказать разрешимость $\Ker f$. Проведём индукцию по размерности $n$.
База индукции очевидна. Пусть утверждение верно для $n-1$. Тогда построим гомоморфизм $g\cln \UT_n(\K) \ra
\UT_{n-1}(\K)$, при котором угловой минор размерности $n-1$ отображается тождественно (грубо говоря, отрезаем
от матрицы последнюю строку и последний столбец). Сюръективность $g$ очевидна, а $\Ker g$ состоит из матриц
вида
$$\rbmat{1&&0& b_1 \\ &\ddots&& \vdots \\ 0&&1& b_{n-1}\\ 0&\cdots&0&1}$$
Тогда $\Ker g \cong \K^{n-1}$, к $\Ker g \cong \hc{(b_1 \sco b_{n-1})}$. Произведение матриц при  этом
переходит в сумму строк, а образ есть абелева группа по сложению, и шаг индукции доказан. По предыдущему
утверждению $\Tb_n(\K)$ разрешима.
\end{proof}

Очевидно, что если в нормальном ряду есть хоть один неразрешимый фактор, то и вся группа неразрешима.

\subsubsection{Простые группы}

Абелева группа проста $\Lra$ она циклическая простого порядка.

\begin{lemma}
Группа $\Ab_n$ порождается тройными циклами.
\end{lemma}
\begin{proof}
Любая чётная подстановка есть произведение чётного числа транспозиций. Любую пару транспозиций можно
получить из тройных циклов: $(ac)(bd)=(abc)(abd)$, а все тройные циклы у нас есть по условию.
\end{proof}

\begin{lemma}
Если нормальная подгруппа $N \nl \Ab_n$ содержит хотя бы один тройной цикл, то $N = \Ab_n$.
\end{lemma}
\begin{proof}
При $n \ge 5$ все тройные циклы сопряжены, а нормальная подгруппа есть объединение классов сопряженности,
значит, она содержит все тройные циклы и тем самым по предыдущей лемме порождает $\Ab_n$.
\end{proof}

\begin{theorem}
Группа $\Ab_n$ при $n \ge 5$ простая.
\end{theorem}
\begin{proof}
Пусть $N \nl \Ab_n$ и $N \neq \hc{e}$. Докажем, что в $N$ есть тройной цикл. Тогда утверждение теоремы  будет
следовать из лемм. Пусть $\si = \si_1\si_2\si_3 \ldots \si_s \in N$. Рассмотрим циклическую
группу $\ha{\si}$. Она содержит циклическую группу простого порядка, значит, можно считать, что $\si$
имеет простой порядок $p$, и число $p$ минимальное. Тогда без ограничения общности можно считать, что
первый цикл в $\si$ имеет длину $p$, е $\si_1 = (1 \sco p)$. Поскольку все сопряженные с $\si$
элементы лежат в $N$, то у нас есть перестановка $\tau=\si_1\si_2^{-1}\si_3^{-1} \ldots
\si_s^{-1}$. Тогда $\si\tau = \si_1^2$, е тоже цикл длины $p$. Для $p$ есть три возможности:
$p=2, p=3, p \ge 5$. Если $p=3$, то $\si\tau$ тройной цикл и всё доказано.

Рассмотрим случай $p \ge 5$. Мы уже знаем, что в $N$ есть все циклы длины $p$, тогда возьмём перестановки
$\pi_1:=(1,2,3,4,5 \sco p)$ и $\pi_2:=(1,3,4,2,5 \sco p)$. Легко  видеть, что перестановка $\pi_1^{-1}\pi_2$
оставляет на месте число $p$, а значит, в ней есть цикл длины меньше $p$. Противоречие.

Теперь рассмотрим случай $p=2$. Тогда $\si$ имеет вид $(12)(34)\rho$, где $\rho$ произведение
некоторых других транспозиций. Рассмотрим сопряжённую перестановку $\tau:=(13)(24)\rho$. Поскольку
$\rho^2=e$, то имеем $\si\tau = (14)(23) \in N$. Значит, в $N$ имеются все пары транспозиций. Тогда
рассмотрим перестановку $\pi_1 := (12)(34)$ и $\pi_2:=(12)(45)$. Имеем $\pi_1\pi_2 = (345)$, е тройной
цикл. Значит, $N=\Ab_n$.
\end{proof}

\begin{theorem}
Группа $\SO_3(\R)$ простая.
\end{theorem}
\begin{proof}
Из курса аналитической геометрии известно, что любая матрица $A \in \SO_3(\R)$ подобна матрице
\eqn{\label{RotationMatrix}A=\rbmat{\cos\al & -\sin\al & 0 \\ \sin\al & \cos\al & 0 \\ 0 & 0 & 1}.}
Отсюда следует, что все матрицы в $\SO_3(\R)$ сопряжены. Пусть в $\SO_3(\R)$ есть нетривиальная нормальная
подгруппа $N$. Тогда в ней есть матрица $(\ref{RotationMatrix})$ с некоторым фиксированным углом поворота $\al$.
Рассмотрим матрицу
$$B(\ph)=\rbmat{1 & 0 & 0 \\ 0 & \cos\ph & -\sin\ph \\ 0 & \sin\ph & \cos\ph}$$
Положим $C(\ph):=A B(\ph)A^{-1}B^{-1}(\ph) \in N$ коммутатор матриц $A$ и $B$. Докажем, что в $N$  есть
матрицы $X$ со следом $\tr X \in [-1;3]$. Рассмотрим функцию $f(\ph) = \tr C(\ph)$. Имеем $f(0) = 3$, и
$f(\ph) < 3$ при $\ph \neq 0$. Очевидно, что $f$ неперерывна. Тогда в $N$ есть матрицы со сколь угодно малым
углом поворота $\psi$, так как $\cos\psi=\frac{\tr C -1}{2}$. Но тогда есть и матрица с углом поворота
$-\psi$. Значит, для любой матрицы $D$ с углом поворота $\ph$ можно найти такую матрицу $G \in N$ и
подходящую степень $k$, что $D=G^k$. Тем самым $N=\SO_3(\R)$.
\end{proof}

%==================================================================
\subsection{Действия групп на множествах}
%==================================================================

\subsubsection{Понятие действия}

\begin{df}
\emph{Действием} группы $G$ на множестве $\Mc$ называется гомоморфизм $\rho\cln G\ra \Sb_\Mc$, где $\Sb_\Mc$
группа биективных отображений $\Mc$ на себя. Обозначение: $G\cln \Mc$. Ядро гомоморфизма $\rho$ называется
\emph{ядром действия}. Действие называется \emph{точным}, если $\Ker \rho = \hc{e}$.
\end{df}
Иначе говоря, каждому элементу $g \in G$ ставится в соответствие некоторое преобразование $\rho(g)$
множества $\Mc$. При этом произведению элементов соответствует композиция преобразований, е
$\rho(gh)=\rho(g)\circ\rho(h)$. Из определения следует, что:

\pt{1} $\fa g, h \in G, \fa x \in \Mc$ имеем $(gh)x=g(hx)$, так как
$(gh)x=\rho(gh)(x)=\bigl(\rho(g)\circ\rho(h)\bigr)(x)=\rho(g)(hx)=g(hx)$;

\pt{2} $ex \mapsto x$, поскольку $ex=\rho(e)(x)=\id_\Mc(x)=x$.

\subsubsection{Орбиты и стабилизаторы}

\begin{df}
\emph{Стабилизатором}\footnote{Иногда стабилизатор называют \emph{стационарной подгруппой}.} элемента $m \in \Mc$ называется подгруппа
$\St(m):=\hc{g \in G\cln gm=m}$.
\end{df}
Покажем, что это действительно подгруппа. Проверим свойства. Очевидно, $e \in \St(m) \fa m$. Произведение:
если $g_1, g_2 \in \St(m)$, то $(g_1g_2)m=g_1(g_2m)=g_1m=m \Ra g_1g_2\in \St(m)$. Обратный элемент: $g \in
\St(m) \Ra gm=m$. Умножим равенство слева на $g^{-1}$. Тогда $m=g^{-1}m$, е $g^{-1} \in \St(m)$.

Очевидно, что ядро действия есть пересечение всех стабилизаторов.

\begin{df}
\emph{Орбитой} элемента $m \in \Mc$ называется множество $\Orb(m)= G m := \hc{gm \vl g \in G}$. Точка $m$
называется \emph{\emph{неподвижной точкой}} действия, если $\Orb(m)=\hc{m}$. Мощность  орбиты называется
её \emph{длиной}.
\end{df}

\begin{stm}
\label{OrbEqLCStabStm}
Элементы орбиты находятся в биективном соответствии с левыми смежными классами по стабилизатору
фиксированного элемента $m$.
\end{stm}
\begin{proof}
Имеем $gm=hm \Lra (h^{-1}g)m=m \Lra h^{-1}g \in \St(m) \Lra g \in h\St(m)$.
\end{proof}

\begin{stm}
Любые две орбиты либо не пересекаются, либо совпадают.
\end{stm}
\begin{proof}
Орбита определяется одним элементом: $gm \in \Orb(m) \Ra \Orb(gm) = G(gm) = (Gg)m = Gm = \Orb(m)$.
\end{proof}

\begin{imp}
Имеется разбиение множества $\Mc$ на орбиты.
\end{imp}

Введём отношение эквивалентности: $m_1 \sim m_2 \Lra m_1, m_2$ лежат на одной орбите $\Lra \exi g\cln gm_1=m_2$.

\begin{df}
Пусть $H \subset G, g \in G$. Подгруппа $gHg^{-1}$ называется \emph{сопряжённой} с $H$.
\end{df}

Фиксируем некоторый элемент $m \in \Mc$, и рассмотрим стабилизатор элемента $gm \in \Mc$ для некоторого $g \in G$:

$$\St(gm)=\hc{h \in G\cln h(gm)= gm} = \hc{h \in G\cln (g^{-1}hg)m=m} \Lra g^{-1}hg \in \St(m) \Lra h \in g\St(m)g^{-1}.$$

Таким образом, $\St(gm)=g\St(m)g^{-1}$.
\begin{imp}
Если $m$ и $m'$ лежат на одной орбите, е $m'=gm$, то их стабилизаторы сопряжены:
$$\St(m')= g\St(m)g^{-1}.$$
\end{imp}

\begin{df}
Действие \emph{транзитивно}, если его орбита единственна, е $\fa m_1, m_2 \in \Mc \; \exi g\cln gm_1=m_2$.
\end{df}

Итак, мы имеем разбиение множества $\Mc=\bigcup\Orb(m_i)$. Из утверждения \ref{OrbEqLCStabStm}
и теоремы Лагранжа следует, что
\eqn{\label{OrbitFormula}|\Mc| = \sum|\Orb(m_i)| = \sum \br{G : \St(m_i)},}
так как длина орбиты элемента $m_i$ равна числу левых смежных классов по $\St(m_i)$, то есть индексу $\St(m_i)$.

\begin{ex}
Группа $\Sb_n$ действует на множестве $\hc{1 \sco n}$.
\end{ex}

\begin{ex}
Группа $\GL_n(\K)$ действует на $\K^n$.
\end{ex}

\subsubsection{Действия группы на себе. Централизаторы и нормализаторы}
Можно также определить действие группы на самой себе (например, \emph{\emph{левыми сдвигами}}): $g\cdot x = gx$.
Все действия, рассматриваемые выше, были левыми действиями. Правое действие: $g\cdot x \bw= xg^{-1}$.

\begin{df}
\emph{Централизатором} элемента $x$ называется множество $Z(x):=\hc{g \in G\cln xg=gx}$.
\end{df}

Рассмотрим действие группы на себе сопряжениями: $g\cdot x = gxg^{-1}$. Тогда стабилизатором каждого элемента
будет его централизатор, а орбиты превращаются в классы сопряжённости.
Пусть $x \in G$, а $C(x)$ класс сопряжённых ему элементов. Тогда $|C(x)|\cdot|Z(x)|=|G|$.

Теперь пусть $M$ множество всех подгрупп в $G$. Рассмотрим действие сопряжениями $g\cdot H = gHg^{-1}$. Тогда имеем
$\St(H) = \hc{g\cln gHg^{-1}=H}$. Подгруппа $\St(H)$ будет наибольшей подгруппой, в которой $H$ нормальна. Она называется
\emph{\emph{нормализатором}} $H$ и обозначается $N(H)$.

\begin{stm}
Число подгрупп, сопряжённых с данной, равно индексу её нормализатора.
\end{stm}
\begin{proof}
Длина орбиты равна числу подгрупп, сопряжённых с $H$. Остается применить формулу (\ref{OrbitFormula}).
\end{proof}

Пусть $S$ некоторое подмножество в $G$. Рассмотрим действие $g\cdot S = gS$. Тогда $H:=\St(S) = \hc{h\cln hS=S}$.
Если $HS=S$, то $S$ объединение смежных классов по $H$.

%==================================================================
\subsection{Конечные $p$-группы. Теоремы Силова}
%==================================================================

\subsubsection{Формула классов. Конечные $p$-группы}

\begin{df}
Группа $G$ называется $p$-группой, если $p$ простое и $|G| = p^k$.
\end{df}

Пусть группа $G$ разбита на классы сопряжённых элементов $x_1^G \sco x_r^G$. Очевидно,  что если $x \in
Z(G)$, то $x^G = \hc{x}$. Пусть $|Z(G)| = q$. Тогда получаем так называемую \emph{\emph{формулу
классов}}:
$$|G| = |Z(G)| + \suml{i=q+1}{r}\frac{|G|}{|Z(x_i)|} = |Z(G)| + \suml{i=q+1}{r}|x_i^G|.$$

\begin{theorem}
Всякая $p$-группа имеет нетривиальный центр.
\end{theorem}
\begin{proof}
Если группа абелева, то тогда её центр есть вся группа. Если она не абелева, то в формуле классов размер
каждого нецентрального класса делится на $p$. Тогда имеем $|G| = p^k = |Z(G)| + pm$, а значит, и $|Z(G)|$
делится на $p$, е в центре кроме единицы ещё что-то есть.
\end{proof}

\begin{imp}
Всякая $p$-группа разрешима.
\end{imp}
\begin{proof}
Докажем по индукции по порядку группы. Пусть $|G|=p^k$. Имеем $Z(G) \nl G$. Тогда $|G/Z(G)| < |G|$, так  как
центр нетривиален. Факторгруппа имеет меньший порядок, и можно применить индукцию.
\end{proof}

\begin{theorem}
Всякая группа $G$ порядка $p^2$ абелева.
\end{theorem}
\begin{proof}
Центр $G$ имеет порядок либо $p$, либо $p^2$. Во втором случае доказывать нечего, а иначе $|G/Z(G)|=p$, но факторгруппа неабелевой группы по центру не может быть циклической. Противоречие.
\end{proof}

Рассмотрим несколько примеров $p$-групп.

\begin{ex}
Группа \emph{\emph{кватернионов}} $\Qb_8$.
\end{ex}

\begin{ex}
Группа унитреугольных матриц $\UT_3(\F_p)$.
\end{ex}

\subsubsection{Полупрямое произведение групп}

Пусть $N \nl G$, а $H$ подгруппа в $G$.
Тогда произведение подгрупп $NH$ является подгруппой, так как

$$(n_1h_1)(n_2h_2)=(n_1 \ub{h_1n_2h_1^{-1}}_{\in N})(h_1h_2) \in NH, \text{ и } (nh)^{-1}=\ub{h^{-1}n^{-1}h}_{\in N}h^{-1} \in NH.$$

Это обстоятельство позволяет дать следующее

\begin{df}
Группа $G$ есть \emph{полупрямое произведение} подгрупп $N$ и $H$ (обозначение: $G = N \lhdp H$), если:

\pt{1} $N \nl G, H \subset G$;

\pt{2} $NH=G$;

\pt{3} $N \cap H = \hc{e}$.
\end{df}

\begin{note}
Полупрямое произведение несимметрично!
\end{note}

Пусть $G = N \lhdp H$. Для каждого $h \in H$ рассмотрим ограничение внутреннего автоморфизма $\Phi_h(x):=
hxh^{-1}$  на подгруппу $N$. В силу нормальности подгруппы $N$ получаем, что $\Phi_h \in \Aut N$ и
отображение $h \mapsto \Phi_h$ является гомоморфизмом $H \ra \Aut N$. Тогда умножение элементов из $N \lhdp
H$ происходит так:

$$(n_1h_1)(n_2h_2)=\bigl(n_1\Phi_{h_1}(n_2)\bigr)(h_1h_2).$$

Также можно определить \emph{\emph{внешнее полупрямое произведение}}. Пусть есть какие-то группы $N$ и
$H$,  задан гомоморфизм $\ph\cln H \ra \Aut N$, и элемент $h \corr{\ph} \Phi_h$. Определим в декартовом
произведении $N \times H$ умножение по формуле $(n_1, h_1)\cdot(n_2, h_2):=\bigl(n_1
\Phi_{h_1}(n_2),h_1h_2\bigr)$.

Очевидно, что аксиомы группы выполняются. Полученная группа и будет внешним полупрямым произведением групп
$N \lhdp H$. Аналогично прямому произведению, можно отождествить группы $N$ и $H$ с множествами пар
$\hc{(n,e)}$ и $\hc{(e,h)}$ соответственно и не различать внешнее и внутреннее произведения.

Вернёмся к $p$-группам. Пусть $N = \ha{a}_{p^2}, H = \ha{b}_p$. Группа автоморфизмов циклической группы
изоморфна группе обратимых элементов кольца вычетов $\Z/p^2\Z$, поэтому имеем $|\Aut N|=p(p-1)$. В $\Aut N$
есть элемент порядка $p$, следовательно, есть циклическая подгруппа порядка $p$. Значит, существует
нетривиальный гомоморфизм $H \ra \Aut N$ и можно построить полупрямое произведение $G = N \lhdp H$ порядка
$p^3$.

\subsubsection{Теоремы Силова}

\begin{df}
Пусть $|G| = p^n m$, где $p$ простое и $(p,m)=1$. Рассмотрим подгруппу $H \subset G$ порядка $p^n$.  Она
называется \emph{силовской} $p$-подгруппой.
\end{df}

\begin{theorem}[Первая теорема Силова]
Силовская $p$-подгруппа существует.
\end{theorem}
\begin{proof}
Если группа $G$ абелева, то разложим её на примарные циклические. Очевидно, что силовской $p$ подгруппой
будет произведение всех тех слагаемых, порядки которых являются степенями числа $p$. В общем случае
применим индукцию по $|G|$. Если $|G|=1$, то доказывать нечего. Пусть $|G| > 1$. Рассмотрим разбиение $G$ на
классы сопряжённых элементов. Возможны 2 случая:

\pt{1} Есть нетривиальный класс $C(x)$, число элементов которого не делится на $p$. Тогда,  так как
$|Z(x)|\cdot|C(x)|\bw=|G| \bw= p^nm$, то $|Z(x)|$ делится на $p^n$. Порядок централизатора меньше $|G|$, значит, по
индуктивному предположению в $Z(x)$ есть силовская $p$-подгруппа порядка $p^n$. Тогда она же будет искомой
подгруппой в $G$.

\pt{2} Такого класса нет, е количество элементов во всех нетривиальных классах делится на  $p$. Тогда по
формуле классов $|Z(G)|$ делится на $p$. Пусть $|Z(G)|=p^kl$, и $(p,l)=1$. Тогда в центре $Z(G)$ есть
подгруппа $Z' \subset Z(G)$ порядка $p^k$. Факторгруппа $G/Z'$ имеет порядок $p^{n-k}m$, и снова по
индуктивному предположению в ней есть подгруппа порядка $p^{n-k}$. Её полный прообраз при каноническом
гомоморфизме $G \ra G/Z'$ и будет силовской $p$-подгруппой в $G$.
\end{proof}

\begin{theorem}[Вторая теорема Силова]
\label{SecondSilovTheorem}
Всякая $p$-подгруппа содержится в некоторой силовской $p$-подгруппе. Все силовские $p$-подгруппы сопряжены.
\end{theorem}
\begin{proof}
Пусть $S \subset G$ силовская $p$-подгруппа в $G$, и $T$ какая-то $p$-подгруппа. Рассмотрим  действие
$T$ на фактормножестве\footnote[1]{$G/S$ вовсе не обязано быть группой! (Прим. наб.)} $G/S$ левыми сдвигами.
При таком действии длина любой нетривиальной орбиты будет делиться на $p$, так как порядок стабилизатора
делит порядок группы, и, стало быть, является некоторой степенью числа $p$. Заметим, что $|G/S|$ не делится
на $p$. Значит, у данного действия есть неподвижные точки. Пусть $gS$ такая точка. Тогда $\fa t \in T$
имеем $t\cdot gS \subseq gS$, е $\fa s \in S \;\; tgs = gs'$. После преобразования этого равенства имеем

$$t=g\ub{s's^{-1}}_{\in S}g^{-1}, \text{ е } t \in gSg^{-1} \Ra T \subseq gSg^{-1}.$$

Таким образом, первое утверждение доказано, так как $gSg^{-1}$ будет некоторой силовской $p$-подгруппой.  А~если порядки у $T$ и $S$ совпадают, то $T=gSg^{-1}$, что и даёт сопряжённость всех силовских $p$-подгрупп.
\end{proof}

\begin{theorem}[Третья теорема Силова]
Число силовских $p$-подгрупп сравнимо с 1 по модулю $p$.
\end{theorem}
\begin{proof}
Пусть $S$ силовская $p$-подгруппа в $G$ и $C(S)$ класс подгрупп, сопряженных с $S$, е класс всех
силовских $p$-подгрупп. Рассмотрим действие группы $G$ сопряжениями на $C(S)$. При таком действии
стабилизатор любой подгруппы $S'$ равен её нормализатору $N(S')$. Ограничим это действие на $S$. Тогда всё
множество $C(S)$ разобьётся на нетривиальные орбиты (длина каждой из них делится на $p$, как и в теореме
\ref{SecondSilovTheorem}), и на неподвижные точки. Докажем, что единственной неподвижной точкой
будет сама подгруппа $S$, откуда и будет следовать, что $|C(S)| \equiv 1 \pmod p$.

Пусть $S' \in C(S)$ какая-то неподвижная точка. Это значит, что любой элемент из $S$ действует  на $S'$
тождественно, то есть лежит в стабилизаторе $\St(S')$. Таким образом, $S \subset \St(S')=N(S')$. Тогда $S$ и
$S'$ будут силовскими $p$-подгруппами в группе $N(S')$ и, значит, сопряжены в ней. Но $S'$ нормальная
подгруппа в своём нормализаторе, то есть сопряжена только сама себе. Следовательно, $S'=S$.
\end{proof}

\begin{imp}\label{SylowFirstCor}
Силовская подгруппа единственна $\Lra$ она нормальна.
\end{imp}

\begin{imp}\label{SylowSecondCor}
Из сопряженности всех силовских $p$-подгрупп вытекает, что их количество $N_p$ равно индексу  нормализатора
одной из этих подгрупп, е если $|G|=p^nm$, то $N_p | m$.
\end{imp}

\begin{imp}
Если $\fa p_i \bigl||G|$ силовская подгруппа $G_{p_i}\nl G$, то $G = G_{p_1} \st G_{p_q}$ и $|G| \bw= |G_{p_1}| \ldots |G_{p_q}|$.
\end{imp}
\begin{proof}
Докажем, что пересечение каждой подгруппы с произведением остальных тривиально. Допустим противное.  Пусть
существует $x \in G\cln x=x_1=x_2 \ldots x_q$, где $x_i \in G_{p_i}$. Тогда порядок элемента слева есть
некоторая степень $p_1$, а у любого $x_i$ справа порядок не может делиться на $p_1$. Противоречие.
\end{proof}

\subsubsection{Группы порядка $pq$}

Рассмотрим группу $G$ порядка $pq$, где $p$ и $q$ простые, и $p < q$. Рассмотрим силовские $q$-подгруппы.
По следствию \ref{SylowSecondCor} имеем $N_q | p$ и $N_q \equiv 1 \pmod p \Ra N_q=1$. Тогда по следствию
\ref{SylowFirstCor} теорем Силова получаем, что $G_q \nl G, |G_q|=q,
|G/G_q|=p$, а значит, группа разрешима. Теперь рассмотрим силовские $p$-подгруппы. Имеем $N_p | q$ и $N_p \equiv
1 \pmod p$. Возможны 2 случая:

\pt{1} $q$ не сравнимо с $1 \pmod p \Ra N_p=1 \Ra G_p \nl G \Ra G = G_p \times G_q$ циклическая группа.

\pt{2} $q \equiv 1 \pmod p$. Тогда покажем, что существует неабелева группа порядка $pq$. Возьмём группы $N\cln |N|=q$ и $H\cln |H|=p$. Имеем $\Aut N \cong \F_q^*$. Тогда $\Aut N$ содержит подгруппу порядка $p$ и можно
построить вложение $H \hookrightarrow \Aut N$ и получить тем самым внешнее полупрямое произведение групп
$G=N\lhdp H$.

%##################################################################
\section{Кольца. Поля. Алгебры}
%##################################################################

%==================================================================
\subsection{Основные понятия и теоремы}
%==================================================================

\subsubsection{Кольца. Гомоморфизмы колец. Идеалы и факторкольца}

Напомним, что \emph{\emph{кольцом}} называется множество $(R, +, \cdot)$ с двумя операциями. По  сложению
$R$ есть абелева группа, а умножение дистрибутивно по отношению к сложению.

\begin{df}
\emph{Гомоморфизмом} колец $R$ и $S$ называется отображение $f\cln R \ra S$ со следующими свойствами:

\pt{1} $f(a+b)=f(a)+f(b)$ для $\fa a,b \in R$;

\pt{2} $f(ab)=f(a)f(b)$ для $\fa a,b \in R$;

\pt{3} $f(1)=1$ это требование только для колец с 1. Если $f$ сюръективно, то \pt{3} есть  следствие
\pt{1} и \pt{2}.
\end{df}

Будем пока рассматривать кольца с 1.

\begin{ex}
Рассмотрим кольцо квадратных матриц $\Mb_n$. В нём есть единица единичная матрица. Рассмотрим теперь
подкольцо $\Mb_{n-1} \subset \Mb_n$. Оно не является подкольцом с единицей, так как единичная матрица
размерности $n-1$ \emph{не совпадает} с единицей в всём кольце.
\end{ex}

Рассмотрим гомоморфизм колец $f$, и обозначим $I := \Ker f$. Оно обладает двумя свойствами:

\pt{1} $I$ - подгруппа по сложению;

\pt{2} $\fa x \in I, \; r \in R$ имеем $rx \in I, \; xr \in I$, так как $f(rx)=f(r)f(x)=0 \Ra rx \in I$,  и
аналогично $xr \in I$.

\begin{df}
Подгруппа $I$ аддитивной группы кольца $R$, удовлетворяющая этим двум свойствам, называется (\emph{двусторонним}) \emph{идеалом} кольца. Если идеал
выдерживает только левое умножение, он называется левым идеалом. Аналогично определяется правый идеал. Обозначение: $I \nl R$.
\end{df}

\begin{note}
Идеал является аналогом нормальной подгруппы в теории групп.
\end{note}

\begin{df}
\emph{Факторкольцом} кольца $R$ по идеалу $I$ называется множество $R/I:=\hc{r+I \vl r \in R}$  смежных
классов по $I$.
\end{df}

Введём операцию умножения смежных классов по правилу $(a+I)(b+I):=ab+I$. Проверим корректность, е что
произведение не зависит от выбора представителя смежного класса. Пусть $a+I=a'+I, \; b+I=b'+I$. Докажем, что
$ab+I=a'b'+I$. Пусть $a'=a+x, \; x \in I; \; b' = b+y, \; y \in I$. Тогда $a'b'=ab + \ub{ay+xb+xy}_{\in I}$,
то есть $R/I$ действительно является кольцом. Часто говорят, что $R/I$ \emph{\emph{кольцо вычетов по
модулю}} $I$.

\subsubsection{Основные теоремы о кольцах}

\begin{theorem}[О гомоморфизме колец]
Пусть $\ph\cln A \ra B$ эпиморфизм колец, $\pi\cln A \ra A/\Ker \ph$ канонический гомоморфизм. Тогда
существует изоморфизм $\ol{\pi}\cln A/\Ker \ph \ra B$, такой, что $\ph = \pi \ol{\pi}$.
\end{theorem}
\begin{proof}
Для аддитивных групп существование искомого изоморфизма уже установлено. Проверим сохранение операции
умножения. Пусть $\ph(x)=u$ и $\ph(y)=v$. Тогда $\ph(xy)=uv$ и
$\ol{\pi}(uv)=\pi(xy)=\pi(x)\pi(y)=\ol{\pi}(u)\ol{\pi}(v)$.
\end{proof}

Для колец, как и для групп, верна
\begin{theorem}[О соответствии]
Существует биекция между подкольцами в кольце, содержащими идеал, и всеми подкольцами в факторкольце  по
этому идеалу. Верно также, что если $I \subseq J \nl R$, то $R/J \cong \frac{R/I}{J/I}$.
\end{theorem}

\begin{stm}
Существует биекция между идеалами в $R/I$ и идеалами в $R$, содержащими $I$.
\end{stm}
\begin{proof}
Пусть $\pi\cln R \ra R/I$ канонический гомоморфизм. Пусть $I \subseq J \nl R$, и $\ol{J} \nl R/I$. Имеем
$\ol{J} = J/I = \pi(J), \; J=\pi^{-1}(\ol{J})$ полный прообраз $\ol{J}$.
\end{proof}

\begin{theorem}[Об изоморфизме]
Пусть $R$ кольцо, $K$ его подкольцо и $I \nl R$. Тогда
\eqn{(K+I)/I \cong K/(K \cap I).}
\end{theorem}
\begin{proof}
Рассмотрим естественный эпиморфизм $\pi\cln R \ra R/I$ и его ограничение $\pi_0:=\pi$\raisebox{-2pt}{$\rvert_K$}. Его образ
состоит из смежных классов $x+I, \; x \in K$, е $\Img \pi_0 = (K+I)/I$. Ядро $\Ker \pi_0$ состоит из всех
тех элементов из $K$, для которых $x+I=I$. Значит, $\Ker \pi_0 = K \cap I$. По теореме о гомоморфизме
получаем, что $(K+I)/I \cong K/(K \cap I)$.
\end{proof}

\begin{note}
Идеал в кольцах без единицы является подкольцом. В кольцах с 1 очевидно, что если единица лежит в идеале, то он совпадает со всем кольцом.
Очевидно также, что если некоторый обратимый элемент кольца лежит в идеале, то он также совпадает со всем кольцом: в самом деле, умножим такой
элемент на обратный к нему, получим единицу. При этом по определению идеала результат умножения лежит в нём, е $1 \in I$.
\end{note}

\begin{df}
Кольцо называется \emph{простым}, если в нём нет нетривиальных идеалов.
\end{df}

\begin{ex}
Примеры простых колец поля и тела. Заметим, что вследствие простоты гомоморфизмы полей и тел будут
вложениями.
\end{ex}

\subsubsection{Идеалы в кольце квадратных матриц}

\begin{theorem}
Пусть $R$ кольцо с 1. Для всякого идеала $I \nl R$ имеем $\Mb_n(I) \nl \Mb_n(R)$. Верно и обратное:
любой идеал в кольце квадратных матриц $\Mb_n(R)$ есть кольцо матриц над некоторым идеалом кольца $R$.
\end{theorem}
\begin{proof}
Первое утверждение теоремы очевидно и следует из правила умножения матриц. Докажем второе утверждение.  Пусть
$M$ идеал в $\Mb_n(R)$. Покажем, что $M=\Mb_n(I)$, где $I$ некоторый идеал в $R$. В самом деле, идеал
$M$ выдерживает левое и правое умножение, в частности, на матричные единицы $E_{ij}$. Рассмотрим матрицу
$A=(a_{ij}) \in M$. Умножим её слева и справа на $E_{ki}$ и $E_{jl}$. Получим $E_{ki}AE_{jl}=a_{ij}E_{kl}$.
По определению идеала результат лежит в $M$, значит, вместе с матрицей $A$ идеал $M$ содержит все матрицы,
составленные из её элементов, поставленных в произвольную строку и столбец. Рассмотрим множество всех чисел
из $R$, которые могут составлять матрицы из $M$, и обозначим его через $I$. Покажем, что это идеал в $R$,
е покажем, что для $\fa a \in I$ верно $xa \in I, \fa x \in R$. Рассмотрим матрицу $aE_{ij} \in M$,
причём $i,j$ можно брать любыми. Так как $M$ идеал, то для $\fa x \in R$ имеем $(xa)E_{ij} \in M$.
Значит, по определению множества $I$, число $xa$ также лежит в $I$. Аналогично $ax \in I$, значит, $I$
идеал.
\end{proof}

\begin{imp}
В частности, кольцо квадратных матриц над простым кольцом простое.
\end{imp}

Будем теперь рассматривать коммутативные кольца.

\begin{stm}
Простое коммутативное кольцо с 1 является полем.
\end{stm}
\begin{proof}
Докажем, что каждый ненулевой элемент в таком кольце обратим. Пусть $x \in R, \; x \neq 0$. Тогда  $xR=R$,
так как кольцо простое. Следовательно, $\exi y\cln xy=1$. Элемент $y$ и будет обратным к $x$.
\end{proof}

\begin{df}
Идеал, порождённый одним элементом, называется \emph{главным}. Кольцо, в котором все идеалы главные,
называется \emph{кольцом главных идеалов}.
\end{df}
Будем обозначать для краткости идеалы, порождённые элементом $x$, через $(x)$, если из контекста ясно,  о
каком кольце идёт речь.

\begin{stm}
Кольцо многочленов $\K[x]$ является кольцом главных идеалов.
\end{stm}
\begin{proof}
Пусть $I \nl \K[x]$ и $f \in I$ многочлен наименьшей степени. Докажем, что $I=(f)$. Очевидно,  $(f)
\subseq I$. Докажем, что любой элемент $g \in I$ делится на $f$. Поделим $g$ с остатком: $g=fq+r$. Многочлен
$fq \in I \Ra r \in I$. Но так как $\deg r < \deg f$, то $r=0$.
\end{proof}

Абсолютно также доказывается, что \emph{любая евклидова область есть кольцо главных идеалов}.

%==================================================================
\subsection{Алгебры}
%==================================================================

\subsubsection{Основные определения и примеры}

\begin{df}
\emph{Алгеброй} над полем $\K$ называется множество $(R, +, *_{R}, *_{\K})$ с операциями сложения,
умножения и умножения на элементы поля $\K$, обладающими следующими свойствами:

\pt{1} По сложению и умножению на элементы поля $\K$ множество $R$ есть векторное пространство;

\pt{2} По сложению и умножению $R$ есть кольцо;

\pt{3} $\fa \la \in \K, a,b \in R \; \; \la(ab)=(\la a)b = a(\la b)$.
\end{df}

\begin{ex}
$\Mb_n(\K), \K[x]$.
\end{ex}

\begin{df}
\emph{Центром} кольца (алгебры) $R$ называется подмножество $Z(R) := \hc{r \in R\cln rx=xr \fa x \in R}$.
\end{df}

Заметим, что центр тела является полем.

\begin{df}
Ненулевая алгебра называется \emph{простой}, если ней нет нетривиальных идеалов.
\end{df}

Пусть $R$ алгебра над $\K$, и $R$ обладает единицей. Тогда поле $\K$ вкладывается в $\R$  очевидным
образом: $\la \mapsto \la \cdot 1$. Очевидно, что поле при этом оказывается в центре алгебры.
Наоборот, если в $Z(R)$ содержится поле $\K$, то $R$ алгебра над $\K$.

\begin{df}
\emph{Идеал} алгебры то же самое, что и идеал кольца, но он должен выдерживать умножение на  элементы
поля, е быть подпространством.
\end{df}

\begin{df}
\emph{Гомоморфизмом} алгебр $R$ и $S$ над полем $\K$ называется отображение $f\cln R \ra S$ со свойствами:

\pt{1} $f(x+y)=f(x)+f(y)$;

\pt{2} $f(xy)=f(x)f(y)$;

\pt{3} $f(\la x)=\la f(x) \; \fa x \in R, \; \la \in \K$.
\end{df}

Множество $R$ является алгеброй с 1 над $\K \Lra R$ кольцо с 1 и $\K \subset Z(R)$. В одну  сторону
очевидно, в другую проверить аксиомы алгебры. Если алгебра с 1, то $f(\la \cdot 1) = \la \cdot
f(1) = \la \in S$. Получаем следующее

\begin{stm}
Если алгебра обладает единицей, то при гомоморфизме алгебр элементы поля отображаются тождественно.  Верно и
обратное: если $f$ гомоморфизм колец с 1 и он тождественно действует на элементы поля, то $f$ является
гомоморфизмом алгебр. В этом случае говорят, что $f$ является гомоморфизмом над $\K$.
\end{stm}

Поскольку алгебра $R$ есть векторное пространство над полем $\K$, можно выбрать базис: $R=\ha{e_1 \sco e_n}$.
Зададим умножение на базисных векторах:

$$e_i e_j = \suml{k=1}{n}c_{ij}^ke_k.$$
Числа $c_{ij}$ называются \emph{структурными константами}.

\begin{note}
Задание структурных констант не гарантирует ассоциативности!
\end{note}

\begin{ex}
В матричной алгебре базис составляют матричные единицы.
\end{ex}

\subsubsection{Групповая алгебра конечной группы}

Пусть дана конечная группа $G$. Занумеруем (формально) базисные векторы элементами группы: $\hc{e_g \vl g \in G}$.
\begin{df}
\emph{Групповой алгеброй} группы $G$ над полем $\K$ называется множество
$$\K G := \BC{a=\sumg a_g e_g \bvl a_g \in \K} \text{ с правилом умножения базисных векторов } e_g e_h=e_{gh}.$$
\end{df}

Обычно базисные элементы отождествляют с элементами группы, поэтому любой элемент алгебры записывается в
виде $a=\sumg a_g g$. Из ассоциативности умножения элементов группы следует ассоциативность умножения в
алгебре.

Структурные константы алгебры зависят от базиса и являются тензорами типа $(2,1)$. Умножение элементов
алгебры $R$ билинейное отображение $m\cln R \times R \ra R$, $m(x,y)=xy$. Сопоставим  этому отображению
трилинейное отображение $T\cln R\times R \times R^* \ra \K$, $T(x,y,f)=f\bigl(m(x,y)\bigr)$.

\subsubsection{Факторалгебра алгебры многочленов}

Рассмотрим $\K[x]$. Возьмём идеал, порождённый ненулевым многочленом $f$ степени $n$. Построим  факторалгебру
$\K[x]\bigl/(f)=\hc{g+(f)}$. Каждый смежный класс, очевидно, содержит единственный многочлен степени меньше
$n$. Поэтому факторалгебру можно отождествить (как множество) с множеством многочленов степени меньше $n$.
Обозначим $\ol{g}:=g+(f)$ остаток от деления на $f$. Сумма и произведение остатков также есть остаток, а
значит, данное множество действительно является факторалгеброй. Базис её составляют многочлены $\hc{\ol{1},
\ol{x}, \ol{x}^2 \sco \ol{x}^{n-1}}$.

\begin{stm}
Если $R$ конечномерная алгебра с 1 над $\K$, то всякий неделитель нуля обратим.
\end{stm}
\begin{proof}
Пусть $a \in R$ и $a$ неделитель нуля. Зададим отображение $\ph\cln R \ra R$ по правилу $\ph(x) = ax$.
Оно невырожденно, а значит, сюръективно и $\exi y\cln y \mapsto 1 \Ra ay=1$. Аналогично найдём левый обратный элемент.
\end{proof}

\begin{imp}
Конечномерная алгебра без делителей нуля является телом.
\end{imp}
\begin{proof}
Докажем, что в такой алгебре есть единица. Пусть $\al$ неделитель нуля. Рассмотрим отображение  $x
\mapsto \al x$. Оно невырожденно, а потому сюръективно, и найдётся элемент $e\cln f(e)=\al$, е
$\al e = \al$. Заметим, что тогда выполняется и равенство $ex=x$, так как равенство $\al e=\al$
можно домножить на $\al$ справа, а затем заменить $\al$ в правой части на $\al e$. Получим $\al
e\al=\al \al e$, откуда $e\al =\al e$. Докажем, что для $\fa x \neq 0$ верно $xe=x$. Умножим
равенство $e \al =\al$ слева на $x$ и вынесем $\al$ за скобки. Получим $(xe-x)\al=0 \Lra xe-x=0
\Lra xe=x$.
\end{proof}

\begin{df}
Алгебра, являющаяся телом, называется \emph{алгеброй с делением}.
\end{df}
Коммутативная конечномерная алгебра без делителей нуля является полем.

Вернёмся к алгебре многочленов. Если порождающий элемент $f$ идеала $(f)$ приводим, е $f=gh$, то в
факторалгебре есть делители нуля: $\bigl(g+(f)\bigr)\bigl(h+(f)\bigr)=f+(f) = (f)=\ol{0}.$ Верно и обратное:
если $f$ неприводим, то делителей нуля нет, так как если бы $\bigl(g+(f)\bigr)\bigl(h+(f)\bigr)=\ol{0}$, то
$gh \in (f) \Ra f | gh$, е $f|g$ или $f|h$.

%==================================================================
\subsection{Поля и их расширения}
%==================================================================

\subsubsection{Расширения. Алгебраические и трансцендентные элементы}

\begin{df}
Говорят, что поле $L$ есть \emph{расширение} поля $K$, если $K \subset L$.
\end{df}

Если $L$ расширение $K$, то $L$ является алгеброй над $K$.

\begin{df}
Размерность расширения $L$ как алгебры над $K$ называется \emph{степенью} расширения и обозначается
$\dim_KL$ или $(L:K)$.
\end{df}

В частности, $K \subset K[x]\bigl/(f)$.

\begin{theorem}
\label{FieldExtRootExistsTheorem}
Пусть $K$ поле, и $f(x) \in K[x], \; f(x) \neq \const$. Тогда $\exi L \supset K$, в котором  $f$
имеет корень.
\end{theorem}
\begin{proof}
Пусть $p(x)$ неприводимый множитель $f(x)$. Рассмотрим факторкольцо $L:=K[x]\bigl/(p)$. В силу
неприводимости $p$ оно будет полем. Пусть $\ol{x} = x + (p) \in L$. Тогда $p(\ol{x})=p(x) + (p) = \ol{0}$,
е $\ol{x}$ корень $p(x)$.
\end{proof}

\begin{df}
Дано расширение $L$ поля $K$. Число $\al \in L$ называется \emph{алгебраическим} элементом над $K$,
если существует ненулевой многочлен $f \in K[x]$, такой, что $f(\al)=0$, и \emph{трансцендентным} в
противном случае.
\end{df}

Можно рассмотреть наименьшее подполе в расширении $L$, содержащее $K$ и корень $\al$. Это будет поле
рациональных функций.

\begin{df}
Многочлен $f$ называется \emph{аннулирующим} для числа $\al$, если $f(\al)=0$.
\end{df}

\begin{theorem}
\label{FiniteExtensionTheorem}Если расширение конечное, то любой элемент $\al$ в нём является алгебраическим.
\end{theorem}
\begin{proof}
Пусть $R$ конечномерная алгебра с 1 и $K \subset R$. Покажем, что для $\al$ есть аннулирующий
многочлен. Пусть степень расширения равна $n$. Тогда рассмотрим элементы $\hc{1, \al, \al^2 \sco
\al^n}$. Они линейно зависимы, е $\exi \la_i \in K\cln \la_0 + \la_1\al + \la_2
\al^2 \spl \la_n \al^n=0$. Значит, многочлен с коэффициентами $\la_i$ и будет аннулирующим.
\end{proof}

Очевидно, что множество всех аннулирующих многочленов для корня $\al$ является идеалом.

\begin{df}
\emph{Минимальный} многочлен корня $\al$ его аннулирующий многочлен наименьшей степени.
\end{df}

\begin{stm}
Пусть $\al$ алгебраический элемент. Тогда его минимальный многочлен $p$ неприводим.
\end{stm}
\begin{proof}
Пусть $p=gh$. Тогда $p(\al)=g(\al)h(\al) \Ra g(\al)=0$ или $h(\al)=0 \; \Ra \; \deg g, \deg h < \deg p$. Противоречие.
\end{proof}

\subsubsection{Простое расширение}

Пусть $L$ расширение $K$, число $\al \in L$ алгебраический элемент над $K$, и $p$ минимальный
многочлен для $\al$. Построим гомоморфизм $\ph\cln K[x] \ra L$. Положим $\ph(x) := \al, \,
\ph(g(x))=g(\al)$. По теореме о гомоморфизме имеем $\Img \ph \cong K[x]\bigl/\Ker \ph$. Ядро $\ph$ будет
множеством всех аннулирующих многочленов, а тогда $\Img \ph \cong K[x]\bigl/(p)$. Но мы знаем, что
$K[x]\bigl/(p) = \hc{g(\ol{x}) \vl \deg g <  \deg p}$. Имеем $\ol{x} = x + (p) \mapsto \al \Ra \Img
\ph$ множество многочленов от $\al$ степени меньше $\deg p$. Это множество является полем и, кроме
того, это наименьшее поле, содержащее $K$ и $\al$. Оно обозначается $K(\al)$ и называется
\emph{\emph{простым расширением}} поля $K$. Итак, если $p$ минимальный многочлен для $\al$, то
$K(\al) \cong K[x]\bigl/(p)$.

\begin{stm}
Пусть есть расширения $L_1, L_2$ поля $K$, $\al_1 \in L_1, \; \al_2 \in L_2$ и минимальные  многочлены
этих двух корней совпадают. Тогда $K(\al_1)$ и $K(\al_2)$ изоморфны как алгебры над $K$, е
существует изоморфизм $\ph\cln \ph(\al_1)=\al_2$, при котором элементы основного поля отображаются
тождественно.
\end{stm}
\begin{proof}
$K(\al_1)$ и $K(\al_2)$ изоморфны одной и той же факторалгебре. Изоморфизм: $\ph(g(\al_1)):=g(\al_2)$.
\end{proof}

\begin{ex}
Пусть $z \in \Cbb, \; z \notin \R, \; p(x) \in \R[x], \; p(z)=0$. Тогда $\R[x]\bigl/(p) \cong \R(z) = \Cbb$.
\end{ex}

\subsubsection{Башни полей}

Будем рассматривать цепочки вложенных друг в друга полей так называемые «башни полей».

\begin{theorem}
Пусть $K \subset L \subset F$. Тогда $(F:K)=(L:K)\cdot (F:L)$.
\end{theorem}
\begin{proof}
Выберем базис $L$ над $K$: $\hc{\omega_1 \sco \omega_m}$, и базис $F$ над $L$: $\hc{\xi_1 \sco \xi_n}$. Докажем,
что всевозможные попарные произведения базисных векторов $\hc{\omega_i \xi_j}$ образуют базис $F$ над $K$.
Любой элемент выражается:
$$u \in F, u=\suml{j=1}{n}x_j\xi_j, \; x_j\in L, \quad x_j = \suml{i=1}{m}a_{ij}\omega_i, \; a_{ij} \in K \; \Ra \; u= \suml{j=1}{n}\suml{i=1}{m}a_{ij}\omega_i\xi_j.$$
Линейная независимость: пусть $\suml{j=1}{n}\ub{\suml{i=1}{m}a_{ij}\omega_i\xi_j}_{\in L} =0 \; \Ra \; \fa j \; \suml{i=1}{m}a_{ij}\omega_i=0 \; \Ra \; a_{ij}=0$.
Значит, это базис.
\end{proof}

\begin{imp}
Пусть $\al_1 \sco \al_s \in L$ алгебраические элементы над $K$. Рассмотрим наименьшее поле,
содержащее $K$ и все эти элементы: $K(\al_1 \sco \al_s) = K(\al_1)(\al_2)\ldots(\al_s)$. По
предыдущей теореме полученное простое расширение имеет конечную степень над $K$.
\end{imp}

\begin{imp}
Множество всех алгебраических над $K$ элементов в данном расширении является полем.
\end{imp}
\begin{proof}
Пусть $\al_1$ и $\al_2$ алгебраичны над $K$. Тогда $K(\al_1, \al_2)$ конечное
расширение, а по теореме~\ref{FiniteExtensionTheorem} любой элемент в нём (в частности, сумма
и произведение любых двух элементов) будет алгебраическим.
\end{proof}

\subsubsection{Поле разложения многочлена}

\begin{df}
\emph{Полем разложения} многочлена $f$ над $K$ называется такое расширение $L \supset K$, что в $L[x]$
многочлен $f$ разлагается на линейные множители, и $L$ наименьшее поле, содержащее все корни $f$.
\end{df}

\begin{theorem}
Для любого многочлена $f$ над полем $K$ существует поле разложения.
\end{theorem}
\begin{proof}
Докажем индукцией по степени $f$. Если $\deg f=1$, то полем разложения будет поле $K$. Пусть $\deg
f=n$ и для многочленов меньшей степени всё доказано. По теореме \ref{FieldExtRootExistsTheorem}
существует поле $K_1 \supset K$, в котором
$f$ имеет корень (обозначим его $\al_1$). Тогда над $K_1$ многочлен $f$ имеет разложение
$f(x)=(x-\al_1)g$, где $g \in K_1[x]$. Имеем $\deg g = n-1$, и можно применить предположение
индукции к $g$ и полю $K_1$. Пусть $K_2$ поле разложения для $g$. Тогда в нём он разлагается на
линейные множители. Значит, и $f$ разлагается в $K_2$ на линейные множители: $f(x)=
\prodl{i=1}{n}(x-\al_i)$. Тогда $L:=K(\al_1 \sco \al_n)$ будет искомым полем разложения.
\end{proof}

\subsubsection{Конечные поля}

Пусть $F$ конечное поле характеристики $p$. Очевидно, что $F$ содержит $\F_p$ поле вычетов по  модулю
$p$ и $F$ является конечным расширением для $\F_p$. Пусть $(F:\F_p)=n$. Выберем базис $\hc{w_1 \sco w_n}$,
тогда любой элемент $x \in F$ записывается в виде $x=\sum a_i w_i, \; a_i \in \F_p$. Следовательно,
$|F|=p^n$.

Поскольку $|F^*|= p^n-1$, то для $\fa a \in F^*$ по теореме Лагранжа имеем $a^{p^n-1}=1$, то есть
$a^{p^n}=a$.  Таким образом, любой элемент поля $F$ является корнем многочлена $f(x):=x^{p^n}-x \in \F_p[x]$,
а значит, он разлагается над $F$ на линейные множители. Следовательно, $F$ поле разложения $f(x)$.

\begin{theorem}
Для любого простого числа $p$ и любого числа $n \in \N$ существует поле из $p^n$ элементов.
\end{theorem}
\begin{proof}
Рассмотрим $\F_p \supset L$ -- поле разложения многочлена $f(x):=x^{p^n}-x$ над $\F_p$. Заметим,  что
$f'=-1$, а значит, многочлен $f$ взаимно прост со своей производной и потому не имеет кратных корней.
Рассмотрим множество корней этого многочлена $\hc{\al_1 \sco \al_{p_n}}$. Докажем, что они образуют
искомое поле. В самом деле, пусть $x, y$ корни. Тогда $(x+y)^{p^n}=x^{p^n}+y^{p^n}=x+y$, е число
$x+y$ также является корнем. Аналогично $(xy)^{p^n}=x^{p^n}y^{p^n}=xy$. Очевидно, что если $x$ корень, то
и $x^{-1}$ тоже корень.
\end{proof}

\begin{lemma}
Над полем $\F_p$ существуют неприводимые многочлены любой степени.
\end{lemma}
\begin{proof}
Рассмотрим поле $F$ из $p^n$ элементов (мы уже знаем, что оно есть), и его мультипликативную группу $F^*$.
Пусть $F^* = \ha{\al}$. Рассмотрим отображение $\ph\cln \F_p[x] \ra F$ по правилу $\ph\cln f \mapsto f(\al)$.
Поскольку $\ph(0) = 0$, а $\ph(x^k) = \al^k$, получаем, что $\ph$ -- эпиморфизм. По теореме о
гомоморфизмах колец имеем $\F_p[x]\bigr/\Ker \ph \cong F$. Заметим, что $\Ker \ph$ -- главный идеал,
порождённый некоторым неприводимым многочленом $d$. Действительно, если бы он был приводим, то факторкольцо
$\F_p[x]\bigr/\Ker \ph$ не было бы полем. Его степень будет в точности $n$.
\end{proof}

\begin{theorem}
Конечные поля, содержащие одинаковое число элементов, изоморфны между собой.
\end{theorem}
\begin{proof}
Пусть $|F_1|=|F_2| = p^n$. Докажем, что $F_1 \cong F_2$. Пусть $F_1=\F_p(\al), \; p(x)$ минимальный
многочлен для $\al$ степени $n$. Все элементы $F_1$ корни многочлена $f(x):=x^{p^n}-x$, то есть он
является аннулирующим для $\al$. Пусть $f(x)$ разлагается над $\F_p$ так: $f(x)=p(x)g(x)$. С другой
стороны, в поле $F_2$ многочлен $f(x)$ также разлагается на линейные множители. Пусть $\be$ некоторый
корень $f(x)$ в поле $F_2$. Тогда $F_2=\F_p(\be)$. Таким образом, поля $F_1$ и $F_2$ содержат корни одного
и того же многочлена, а значит, изоморфны.
\end{proof}

\begin{problem}
Доказать, что $n \divs \ph(p^n-1)$, где $\ph$ функция Эйлера, а $p$ простое число.
\end{problem}

%=============================================
\subsection{Алгебры с делением}
%=============================================

\subsubsection{Определения, примеры. Алгебры с делением над $\Cbb$ и $\R$}

\begin{df}
\emph{Алгеброй с делением} над полем $\K$ называется ассоциативная алгебра с единицей, в которой каждый
ненулевой элемент обратим по умножению. Другими словами, алгебра с делением это тело, являющееся
алгеброй.
\end{df}

\begin{problem}
Доказать, что центр простого кольца с единицей является полем.
\end{problem}

\begin{stm}
Пусть $R$ алгебра с делением над полем $\K$. Тогда для любого элемента $\al \in R$ его минимальный
многочлен неприводим над $\K$.
\end{stm}
\begin{proof}
От противного: пусть $p(x)=g(x)h(x)$. Тогда $p(\al)=g(\al)h(\al)=0$, но так как делителей нуля нет, то либо $g(\al)=0$, либо
$h(\al)=0$, е есть аннулирующий многочлен меньшей степени. Противоречие.
\end{proof}

Рассмотрим минимальную алгебру, содержащую поле $\K$ и элемент $\al$, $p(x)$ минимальный многочлен
для $\al$. Эта алгебра содержит все многочлены от $\al$ (линейные комбинации всех степеней $\al$).
Совокупность таких выражений будет подалгеброй (сумма и произведение элементов данного множества принадлежит
этому же множеству это следует из пункта \pt{3} определения алгебры). $\hc{f(\al)}=\K(\al), \;
\K(\al) \cong \K[x]\bigl/(p)$. Но так как $p(x)$ неприводим, то $\K(\al)$ поле.

Далее слово «алгебра» означает «алгебра с делением».

\begin{theorem}
Всякая конечномерная алгебра над $\Cbb$ совпадает с $\Cbb$.
\end{theorem}
\begin{proof}
Поле $\Cbb$ алгебраически замкнуто, значит, неприводимыми являются только многочлены первой степени, а
значит, минимальный многочлен любого элемента имеет вид $p(x)=x-z$. Если $p(\al)=0$, то $\al = z \Ra
\al \in \Cbb$.
\end{proof}

\begin{theorem}
Коммутативная конечномерная алгебра $D$ над полем $\R$ совпадает либо с $\R$, либо с~$\Cbb$.
\end{theorem}
\begin{proof}
Имеем $\R \subset D$. Рассмотрим произвольный элемент $\al \in D$ и его минимальный многочлен $p(x)$.  Он
либо линейный, либо квадратный (других неприводимых над $\R$ не бывает). Если для всех $\al$ минимальные
многочлены линейны, то аналогично предыдущей теореме получаем, что $\al \in \R$ и $D=\R$. Если же среди минимальных
многочленов есть квадратные, то $\R(\al) \cong \R[x]\bigl/(p) \cong \Cbb$.
\end{proof}

\subsubsection{Тело кватернионов. Теорема Фробениуса}

Поставим вопрос о том, можно ли построить алгебру над $\R$ размерности больше 2. Ответ: можно, но только она
не будет коммутативной. Четырёхмерная алгебра над $\R$ называется \emph{\emph{алгеброй кватернионов}} и
обозначается $\Hbb$. Строится она следующим образом: берём векторное пространство $\R^4 = \ha{1,i,j,k}$ и
задаем умножение базисных векторов (структурных констант) так: $i^2=j^2=k^2=-1; \; ij=k, \; jk=i, \; ki=j; \;
ji=-k, \; kj=-i, \; ik=-j$, е элементы антикоммутируют. Проверка ассоциативности неинтересна (хотя и нужна),
и мы её здесь опустим. \emph{\emph{Сопряжённым}} к кватерниону $u=a+bi+cj+dk$
называют кватернион $\ol{u}=a-bi-cj-dk$. \emph{\emph{Нормой}} кватерниона $u$ называется число
$N(u):=u\ol{u}=a^2=b^2+c^2+d^2$. Любой ненулевой элемент обратим, так как
$$u \cdot \frac{\ol{u}}{|u|^2}=1.$$

Очевидно, что $\ol{u+v}=\ol{u}+\ol{v}, \; \ol{\ol{u}}=u, \; |u|\cdot|v|=|uv|$. А вот произведением
сопряженных будет сопряженный к произведению в обратном порядке: $\ol{uv}=\ol{v} \cdot \ol{u}$. Таким
образом, кватернионы образуют тело.

\begin{note}
Отображение $u \mapsto \ol{u}$ называют \emph{\emph{антиавтоморфизмом}}.
\end{note}

\begin{theorem}[Фробениуса]
Любая конечномерная алгебра $D$ с делением над $\R$ изоморфна либо $\R$, либо $\Cbb$, либо $\Hbb$.
\end{theorem}
\begin{proof}
Рассмотрим центр алгебры $Z(D)$. Он является конечномерной коммутативной алгеброй над $\R$, и по теореме 2
изоморфен либо $\R$, либо $\Cbb$. Во втором случае можно рассмотреть $D$ как алгебру над $Z(D)$, и по теореме
1 получаем, что $D \cong \Cbb$. В первом случае рассмотрим элемент $\al \notin Z(D)=\R$. Тогда имеем
$\R(\al) = \Cbb, \; \Cbb \subset D, \; i \in \Cbb \Ra i \in D$. Рассмотрим $D$ как векторное пространство
над $\Cbb$. Зададим умножение на скаляры: для $\fa u \in D, \; z \in \Cbb$ положим $z \cdot u := zu$.
Рассмотрим линейный оператор $\ph\cln D \ra D$, определённый по правилу $\ph(u):=ui$. Проверим корректность: в
силу ассоциативности имеем $\ph(zu)=(zu)i=z(ui)=z\ph(u)$, е это действительно линейный оператор. Заметим,
что $\ph^2(u)=(ui)i=u(i^2)=-u$, поэтому $\ph^2(u)+u=0$, е многочлен $t^2+1$ является аннулирующим для
$\ph$. Собственными значениями $\ph$ являются числа $\pm i$. Значит, $D$ как векторное пространство есть
прямая сумма собственных подпространств: $D = D_+ \oplus D_-$, и

$$D_+ = \hc{u \in D\cln \ph(u)=iu}=\hc{u \in D\cln ui=iu},$$
$$D_- = \hc{u \in D\cln \ph(u)=-iu}=\hc{u \in D\cln ui=-iu}.$$

Таким образом, подпространство $D_+$ есть всё, что коммутирует с $\Cbb$. Значит, $D_+$ есть подалгебра в  $D$
и она содержится в центре $D$, откуда вытекает, что $D_+ = \Cbb$. Если подпространство $D_-$ нулевое (е $D$
коммутативна), то $D=\Cbb$. Если же $D$ некоммутативна, то $D_- \neq \hc{0}$. В этом случае докажем, что $D
\cong \Hbb$. Для этого рассмотрим элемент $h \in D_-, \; h \neq 0$. Пусть $u \in D_-$, тогда
$i(uh)=-uih=uhi$, е для $\fa u \in D_-$ имеем $uh \in D_+$. Теперь рассмотрим линейное отображение
$\psi\cln D_- \ra D_+$ над $\Cbb$, определённое таким образом: $\psi(u)=uh$. Делителей нуля в алгебре нет, значит,
$\Ker \psi = 0$ и это инъективное отображение. Таким образом, $\psi(D_-) \subset D_+$. Но $D_+$ одномерно, а
значит, и $\psi(D_-)$ также одномерно, а потому $\psi(D_-)=D_+$. Теперь возьмём любой ненулевой элемент $h
\in D_-$ в качестве базисного, тогда $D_- = \hc{zh}_{z \in \Cbb}$. Рассмотрим элемент $\psi(h) = h^2 \in D_+
= \Cbb$. Этот элемент коммутирует с $\Cbb$, а так как $D_-=\hc{zh}$, то $h^2$ коммутирует и с $D_-$. Значит,
$h^2 \in Z(D)$, то есть $h^2=:a \in \R$. Но поскольку $h \notin \R$, то многочлен $x^2-a$ будет минимальным
для $a$ и неприводимым над $\R$. Значит, $a < 0$. Теперь рассмотрим элемент $j:=\frac{h}{\sqrt{|a|}}$. Имеем
$j^2=-1, \; j \in D_- \Ra ij=-ji$. Любой элемент $u \in D$ однозначно записывается в виде $u=z_1 + z_2 j$,
где $z_1, z_2 \in \Cbb$. Пусть $z_1=a+bi, \; z_2=c+di$. Тогда $u=a+bi+cj+dij$. Обозначим $ij=:k$, получим,
что $D \cong \Hbb$ (все свойства легко проверить).
\end{proof}

\begin{note}
Процесс расширения алгебр над $\R$ можно продолжать и дальше, однако 8-мерная алгебра над $\R$
неассоциативна,  а 16-мерная имеет делители нуля.
\end{note}

Одним из важных утверждений, связанных с алгебрами над $\R$, является теорема о <<причёсывании
ежа>>:\footnote{Доказательства этой теоремы в нашем курсе не будет.}
\begin{theorem}
Если существует $n$-мерная алгебра с делением, то сфера $S^{n-1}$ параллелизуема, е на ней существует касательное
векторное поле без особых точек.
\end{theorem}

\subsubsection{Геометрические приложения кватернионов}

Используем свойство $|uv|=|u||v|$. Рассмотрим сферу $S^3 :=\hc{u \in \Hbb\cln |u|=1}$. Она является группой по
умножению.
\begin{stm}
$S^3 \cong \SU(2)$ группа двумерных унитарных матриц с определителем $1$.
\end{stm}
\begin{proof}
Пусть $u \in S^3$. Имеем $u=a+bj, \; |a|^2+|b|^2=1$. Построим изоморфизм по  правилу $u \mapsto
\rbmat{\ol{a},-\ol{b}\\ b,~~a}$.
\end{proof}

Найдём связь $S^3$ и $\SO_3$. Пусть $w \in \Hbb, u \in S^3$. Учитывая то, что при этом $u^{-1}=\ol{u}$,
рассмотрим отображение $\ph_u(w):=uwu^{-1}=uv\ol{u}$. Оно сохраняет единицу и сохраняет длины векторов,
значит, это ортогональный оператор. Отсюда получаем, что $\ha{1}^\bot=\ha{i,j,k}=:V$, и $\dim_\R V =3$, е,
построено отображение $\ph\cln S^3 \ra \Ob(V)$. Множество $S^3$ линейно связно, а так как
определитель линейного оператора на $V$ есть непрерывная функция, то в $\Img \ph$ все операторы имеют один и
тот же определитель $\Ra \Img \ph \subseq \SO_3$.

\begin{problem}
Доказать, что на самом деле $\Img \ph = \SO_3$.
\end{problem}

%#########################################################
\section{Модули над кольцами и алгебрами}
%#########################################################

%================================================
\subsection{Основные понятия}
%================================================

\subsubsection{Модули, подмодули, гомоморфизмы модулей. Фактормодули}

\begin{df}
\emph{Левым модулем} над кольцом $R$ называется множество $M$ с операциями сложения и умножения  (слева) на
элементы кольца. При этом должны выполняться аксиомы:

\pt{1}--~\pt{4} $(M,+)$ абелева группа;

\pt{5} $r(x+y)=rx+ry$ для $\fa x,y \in M, \; r \in R$;

\pt{6} $(r+s)x=rx+sx$ для $\fa x \in M, \; r,s \in R$;

\pt{7} $(rs)x=r(sx)$ для $\fa x \in M, \; r,s \in R$ (а для правых модулей $x(rs)=(xr)s$);

\pt{8} Если $R$ кольцо с 1, то должно быть $1 \cdot x = x$ для $\fa x \in R$.
\end{df}

Определим модуль над алгеброй $R$ над полем $\K$. К набору аксиом добавится ещё 2 условия:
$(M,+,*_R,*_\K)$ векторное пространство над $\K$, и для $\fa x \in M, \; r \in R, \; \la \in \K$
выполняется  равенство $r(\la x)=\la(rx)=(\la r)x$.

\begin{note}
Если алгебра обладает единицей, то последнее свойство выводится из остальных, так как поле~$\K$  содержится в
алгебре.
\end{note}

\begin{df}
\emph{Гомоморфизмом} левых $R$-модулей $M$ и $N$ называется такое отображение $\ph\cln M \ra N$, что:

\pt{1} $\ph(x+y)=\ph(x)+\ph(y)$ для $\fa x, y \in M$;

\pt{2} $\ph(rx)=r\ph(x)$ для $\fa x \in M, \; r \in R$.
\end{df}

\begin{note}
Алгебру можно рассматривать как левый (или правый) модуль над собой: $r \cdot x = rx \Ra R$ левый
модуль, а если $r \cdot x = xr$, то правый.
\end{note}

Любой элемент $r$ из кольца $R$ задает линейный оператор на модуле $M$.

\begin{df}
\emph{Изоморфизмом} левых $R$-модулей $M$ и $N$ называется биективное отображение $\ph\cln M\ra N$, являющееся
изоморфизмом векторных пространств. При этом $r \ph(x)= \ph(r x)$, е диаграмма коммутативна:
$$\dsquare[M`N`M`N;\ph`r`r`\ph]$$
\end{df}

\begin{df}
\emph{Подмодулем} $N$ модуля $M$ называется подмножество модуля $M$, замкнутое относительно сложения  и
умножения на элементы кольца (алгебры), е выполнены свойства:

\pt{1} $N$ подгруппа по сложению;

\pt{2} $rx \in N \fa x \in N, \; r \in R$;

\pt{3} Если $R$ алгебра без 1, то требуем, чтобы $N$ было подпространством: $\la x \in N$ для  $\fa x
\in N, \; \la \in \K$.
\end{df}

\begin{df}
\emph{Ядром} гомоморфизма $\ph$ называется множество $\Ker \ph := \hc{x \in M\cln \ph(x)=0}$. Оно,  очевидно,
является подмодулем.
\end{df}

Рассмотрим \emph{смежные классы} в модуле $M$. Для $\fa y_0 \in \Img \ph$ рассмотрим полный прообраз
$\ph^{-1}(y_0) = x_0 + N$, где $N=\Ker \ph$. Легко видеть, множество таких классов является модулем.

\begin{df}
Множество смежных классов $M/N:=\hc{x + N \vl x \in M}$ называется \emph{фактормодулем}.
\end{df}

Умножение на элементы кольца в фактормодуле задаётся так: $r(x+N)=rx+N$. Корректность определения  очевидна.
Если $R$ без 1, то определим умножение на скаляры: $\la(x + N)=\la x + N$. Как и в случае групп,
можно рассматривать каноническую проекцию $\pi\cln M \ra M/N\cln \pi(x)=x+N$.

\subsubsection{Основные теоремы о модулях}

\begin{theorem}[О гомоморфизме]
Пусть $f\cln M \ra N$ гомоморфизм $R$-модулей, $\pi\cln M\ra M/\Ker f$ канонический гомоморфизм.  Тогда
существует изоморфизм $\ph\cln \Img f \ra M/\Ker f$, такой, что $f=\ph \circ \pi$.
\end{theorem}
\begin{proof}
Мы уже знаем, что между фактормодулем и $\Img f$ имеется изоморфизм групп. Проверим коммутирование $\ph$
c умножением на элементы кольца $R$. Пусть $f(x)=y$. Тогда $f(rx)=ry$, и $\ph(ry)=\pi(rx)=r\pi(x)=r\ph(y)$.
\end{proof}

\begin{theorem}[О соответствии]
Имеется биективное соответствие между подмодулями в фактормодуле по $\Ker f$ и подмодулями в исходном
модуле, содержащими $\Ker f$.
\end{theorem}

\begin{theorem}[Об изоморфизме]
Пусть $P,Q$ подмодули в $M$. Тогда $(P+Q)/Q\cong P/(P\cap Q)$.
\end{theorem}

%============================================================
\subsection{Прямые суммы и ряды модулей. Системы порождающих модуля}
%============================================================

\subsubsection{Прямые суммы модулей}

\begin{df}
Пусть $Q_1 \sco Q_s$ подмодули в $M$. Говорят, что $M$ \emph{прямая сумма} $Q_1 \sco Q_s$,  если
$M$ как абелева группа есть прямая сумма подгрупп $Q_i$. Это эквивалентно тому, что любой элемент $x \in M$
записывается однозначно в виде суммы $x=x_1  \spl  x_s$, где $x_i \in Q_i$. Обозначение: $M= Q_1 \sop Q_s$.
\end{df}
Умножение на скаляр в прямой сумме почленное, так как слагаемые являются подмодулями:  $rx=rx_1  \spl  rx_s$.
Аналогично группам определяется внешняя прямая сумма. Если $R$ алгебра, то прямая сумма модулей будет
прямой суммой подпространств.

\subsubsection{Ряды подмодулей. Простые модули}

Рассмотрим \emph{\emph{ряд}} вложенных модулей $M=M_0 \supset M_1 \supset \ldots \supset M_s = \hc{0}$.
Рассмотрим фактормодули $M_i/M_{i+1}$.

\begin{df}
Ненулевой модуль называется \emph{простым}, если в нём нет нетривиальных подмодулей (отличных от  нуля и
его самого). Простые модули иногда называют \emph{неприводимыми}.
\end{df}

\begin{df}
Ряд из модулей называется \emph{композиционным}, если все его факторы простые модули.
\end{df}

Для модулей имеет место
\begin{theorem}[Жордана Гёльдера]
Если модуль обладает композиционным рядом, то любой его ряд уплотняется до композиционного, все
композиционные ряды имеют одинаковую длину и факторы этих рядов изоморфны после некоторой перестановки.
\end{theorem}

\begin{imp}
Пусть есть 2 разложения модуля на простые: $M= Q_1 \sop Q_s=P_1 \sop P_t$. Тогда $s=t$ и слагаемые  изоморфны
после некоторой перестановки.
\end{imp}
\begin{proof}
Рассмотрим ряд подмодулей в $M$:
$$M=M_0 \supset \ub{(Q_2 \sop Q_s)}_{M_1} \supset \ub{(Q_3 \sop Q_s)}_{M_2} \supset \ldots \ub{(Q_s)}_{M_{s-1}} \supset \hc{0}.$$

Факторы этого ряда будут простыми по условию: $M_{i-1}/M_i\cong Q_i$. Аналогичным образом построим  ряд из
$P_i$. Остается лишь применить теорему Жордана Гёльдера.
\end{proof}

\begin{df}
\emph{Длиной} модуля называется длина его композиционного ряда.
\end{df}

\begin{note}
Векторное пространство частный случай модуля, его размерность совпадает с длиной.
\end{note}

\subsubsection{Системы порождающих модуля. Циклические модули}

Начиная с этого момента все рассматриваемые кольца и алгебры с единицей.

\begin{df}
Пусть $Q \subset M$. Система $S \subset Q$ называется \emph{системой порождающих} для $Q$, если  любой
элемент $x \in Q$ записывается в виде $x=r_1 x_1  \spl  r_k x_k$, где $x_i \in S, \; r_i \in R$. Обозначение:
$Q=\ha{S}\bw=\hc{\sum r_i x_i \vl x_i \in S, \; r_i \in R}$. Если порождающее семейство конечно, модуль
называется \emph{конечнопорождённым}.
\end{df}

Кольцо частный случай модуля, идеалы кольца подмодули, поэтому можно говорить о системе  порождающих
для левых идеалов. Пусть $N$ левый идеал в $R$. Система $S \subset N$ будет системой порождающих для $N$,
если $N=\ha{S}$. Очевидно, что любая система $S$ порождает некоторый левый идеал.

\begin{df}
Подмодуль, порождённый одним элементом $a$, называется \emph{циклическим}: $M=\hc{ra \vl r \in R}$.
\end{df}

\begin{ex}
В кольце циклическими подмодулями будут главные левые идеалы.
\end{ex}

Если $M$ циклический модуль, то и $M/Q$ также циклический: $M=\ha{a} \Ra M/Q=\ha{a+Q}$.  Очевидно
также, что любой простой модуль является циклическим.

\begin{theorem}
Всякий циклический $R$-модуль $M$ изоморфен модулю вида $R/I$, где $I$ левый идеал в $R$.
\end{theorem}
\begin{proof}
Пусть $M=\ha{a}$. Рассмотрим гомоморфизм $\ph\cln R\ra M$, при котором $\ph(r)=ra$. Очевидно, что $\ph$
сюръективен. По теореме о гомоморфизме $M \cong R/I$, где $I=\Ker \ph$.
\end{proof}

\begin{ex}
Любая абелева группа является $\Z$-модулем. Циклические подмодули в ней циклические подгруппы.
\end{ex}

%========================================================================
\subsection{Свободные модули. Конечнопорождённые модули над кольцом многочленов}
%========================================================================

\subsubsection{Свободные модули}

Пусть $V=\ha{e_1,\ldots,e_n}_R$ конечномерное векторное пространство над $R$. В нём любой  элемент
однозначно выражается через базис. Однако в случае модулей базис есть не всегда.

\begin{df}
$R$-модуль $M$ называется \emph{свободным}, если в нём существует такая система  порождающих $e_1 \sco
e_n$, что любой элемент $x \in M$ однозначно представляется в виде $x=r_1e_1  \spl  r_n e_n$, где $r_i \in
R$, е модуль обладает базисом.
\end{df}

\begin{ex}
Кольцо $R$, как левый модуль над собой, обладает базисом: $R=\ha{1}$, а значит, является свободным.
\end{ex}

Пусть есть свободный $R$-модуль $M=\ha{e_1 \sco e_n}$. Имеем $R \cong \ha{e_i}$ (изоморфизм  очевиден: $r \ra
r e_i$). Тогда получаем, что $M= \ha{e_1} \sop \ha{e_n} \Ra$ \emph{прямая сумма нескольких экземпляров
кольца есть свободный модуль}.

\begin{theorem}
$\forall$ конечнопорождённый модуль изоморфен фактормодулю свободного модуля по некоторому подмодулю.
\end{theorem}
\begin{proof}
Пусть $M=\ha{a_1 \sco a_n}$. Рассмотрим свободный модуль $F=\ha{e_1,\ldots,e_n}$. Рассмотрим  гомоморфизм
$\ph\cln F\ra M$, ставящий в соответствие элементу $x=r_1e_1+\ldots+r_ne_n \in F$ элемент $\ph(x)=r_1a_1  \spl
r_na_n$. Поскольку $F$ свободен, то отображение задано корректно. Элементы $a_i$ порождающие $\Ra \ph$
сюръективен. Обозначая $Q:=\Ker \ph$, получаем, что $M \cong F/Q$.
\end{proof}

\subsubsection{Конечнопорождённые модули над кольцом многочленов}

Рассмотрим алгебру многочленов $R:=\K[\la]$. Рассмотрим модуль $V$ над $R$. Для этого $V$ должно  быть
векторным пространством над $\K$, и нужно для $\fa x \in V$ задать умножение на $\la$, е определить
линейное отображение (оператор) $x\ra \la \cdot x$. Наоборот, если задано векторное пространство $V$ над
$\K$ и оператор $\ph$, тогда $V$ естественным образом становится модулем над $R$: зададим умножение на
элементы $R$ по правилу $f(\la)\cdot x := f(\ph)x$, где $x \in V, \; f \in \K[\la]$, е
подействуем на $x$ многочленом от оператора.

Теперь рассмотрим некоторый набор многочленов $f_1 \sco f_k \in \K[\la]$ и идеал, порождённый этими
многочленами: $\ha{f_1 \sco f_k}=\hc{g_1(\la)f_1  \spl  g_k(\la)f_k}$. Поскольку этот идеал главный,
то он порождается одним элементом и равен $d(x)\K[\la]$, где $d(x)=\GCD(f_1 \sco f_k)$.

Рассмотрим $R=\K[\la]$ свободный циклический бесконечномерный модуль и циклический модуль $M$,
который изоморфен фактормодулю свободного модуля: $M \cong \K[\la]\bigl/(f)$, где $f \in \K[\la], \;
f \neq 0$. Пусть $\deg f = n$, тогда $\dim_\K M=n$. Это число называется \emph{\emph{порядком}} модуля.

\begin{df}
Конечномерный циклический $R$-модуль называется \emph{примарным}, если $f(\la)=p(\la)^k$
степень неприводимого многочлена.
\end{df}

\begin{lemma}
Если $u, v$ взаимно простые элементы кольца $R$ главных идеалов, то $R/(uv)\cong R/(u) \oplus R/(v)$.
\end{lemma}
\begin{proof}
Рассмотрим отображение $f\cln R \ra R/(u) \oplus R/(v)$, определённый так: $x \corr{f} \bigl(x + (u),
x+(v)\bigr)$.  Оно является гомоморфизмом колец. По условию существуют элементы кольца $a, b$ такие, что
$au+bv=1$. Тогда
$$f(bv)=\bigl(bv+(u),bv+(v)\bigr)=\bigl(1-au+(u),0+(v)\bigr)=\bigl(1+(u),0+(v)\bigr), \; \text{и аналогично} \; f(au)=\bigl(0+(u),1+(v)\bigr).$$
Значит, $f$ сюръективен. Очевидно, что $\Ker f = (uv)$. Остается применить теорему о гомоморфизме.
\end{proof}
\begin{theorem}
Пусть $M$ циклический модуль над $\K[\la]$, и $M = \K[\la]\bigr/(f)$. Пусть $f=gh$, и $g, h$ взаимно
просты. Тогда $M \cong \K[\la]\bigl/(g) \oplus \K[\la]\bigl/(h)$.
\end{theorem}
\begin{proof}
Очевидно, что выполняются условия леммы ($\K[x]$ кольцо главных идеалов). Изоморфизм,  построенный при
доказательстве леммы, является и изоморфизмом модулей. Теорема доказана.
\end{proof}

\begin{imp}Любой конечномерный циклический модуль изоморфен прямой сумме примарных циклических  модулей. Прямая сумма конечномерных циклических модулей
является циклическим модулем $\Lra$ их порядки взаимно просты.
\end{imp}

\begin{note}
Всё это верно только для модулей над кольцом многочленов.\footnote{На самом деле не только над $\K[x]$.
См. Э.\,Б.\,Винберг. <<Курс алгебры>>. Стр. 368-369 (Прим. наб.)}
\end{note}

\begin{theorem}
Всякий конечнопорождённый модуль $M$ над $R:=\K[\la]$ есть прямая сумма конечного числа  бесконечномерных
циклических модулей и конечного числа примарных циклических модулей.
\end{theorem}
\begin{proof}
Докажем по аналогии с абелевыми группами. Пусть $M=\ha{a_1 \sco a_n}$, и $F=\ha{x_1 \sco x_n}$  свободный
модуль. Мы знаем, что $\exi Q\cln M \cong F/Q$. Пусть $Q=\ha{b_i}_{i \in I}$, где $b_j=b_{1j}e_1  \spl
b_{nj}e_n$ соотношения между $a_i$. Составим матрицу $B=(b_{ij})$ размера $n \times I$. Приведём её с
помощью элементарных преобразований и алгоритма Евклида к диагональному виду, осуществляя соответствующие
замены базиса: $B'=\diag(b'_1 \sco b'_n)$. Новые базисы будут иметь вид $F=\ha{e'_1 \sco e'_n}, \;
Q=\ha{b'_1e'_1 \sco b'_n e'_n}$. Тогда $F=Re'_1 \sop Re'_n, \; Q= R b'_1 e'_1 \sop R b'_n e'_n$. Отсюда
следует, что $F/Q \cong R\bigl/(b'_1) \sop R\bigl/(b'_n)$. Заметим, что если в каком-то слагаемом $b_i=0$, то
оно будет бесконечномерным.
\end{proof}

Имеет место и теорема о единственности такого разложения.

\subsubsection{Альтернативное доказательство теоремы о жордановом базисе}

Применим нашу теорию для конечномерного векторного пространства $V$, на котором задан оператор $\ph$.
Рассмотрим $V$ как модуль над $\K[\la]=:R$, задав умножение на скаляры так: $\la \cdot x :=
\ph(x)$, е $f(\la)\cdot x=f(\ph)x$. Пусть $V=\ha{e_1 \sco e_n}_\K$, оператор $\ph$ имеет матрицу
$A=(a_{ij})$. Тогда

$$\la \cdot e_j = \ph(e_j) = a_{1j}e_1  \spl  a_{nj}e_n, \; j=\ol{1, n} \; \Ra \; a_{1j}e_1  \spl  (a_{jj}-\la)e_j  \spl  a_{nj}e_n=0.$$

Пусть $F=\ha{u_1 \sco u_n}_{\K[\la]}$ свободный $R$-модуль. Представим $V$ как фактормодуль
свободного модуля: $V=F/N$ и рассмотрим канонический гомоморфизм $\pi\cln F \ra V\cln \pi(u_i)=e_i$. Имеем $\Ker
\pi = N$. Рассмотрим элементы $y_j:=a_{1j}u_1  \spl  (a_{jj}-\la)u_j  \spl  a_{nj}u_n \in N$. Покажем,
что $\hc{y_j}$ есть набор определяющих соотношений, е что они порождают $N$. Возьмём их линейную оболочку
$N':=\ha{y_1 \sco y_n}_{\K[\la]}$ и докажем, что она совпадает с $N$. Очевидно, что $N' \subseq N$.
Рассмотрим гомоморфизм $F/N' \ra F/N \cong V$, при котором $h + N' \mapsto h + N$. Покажем, что $F/N'$ как векторное
пространство имеет размерность $n$ (достаточно показать, что она не превосходит $n$). Имеем $\la u_j =
a_{1j}u_1  \spl  a_{nj}u_n - y_j$. Значит, если вместо $\la$ подставить произвольный многочлен
$f(\la)$, то получается, что $f(\la)u_j \in \ha{u_1 \sco u_k}_\K + N'$. Тогда $\fa x \in F$ лежит в
$\ha{u_1 \sco u_k}_\K + N'$, так как $x = \sum f_i(\la) u_i$. Значит, если $\ol{x} \in F/N'$, то $\ol{x}
\in \ha{\ol{u_1} \sco \ol{u_n}}_\K$, е факторпространство $F/N'$ есть линейная оболочка $n$ векторов $\Ra
\dim_\K F/N' \le n$. Таким образом, $N'=N$.

Теперь представим модуль $V$ в виде суммы циклических модулей. Для этого приведём матрицу определяющих
соотношений (это в точности $A-\la E$) к диагональному виду:

$$\rbmat{a_{11}-\la & \ldots & a_{1n} \\ \ldots & \ldots & \ldots \\ a_{n1} & \ldots & a_{nn}-\la} \rightsquigarrow \rbmat{d_1(\la) & & 0 \\ & \ldots &  \\ 0 & & d_n(\la)}.$$

При этом $d_1(\la) | \ldots | d_n(\la)$. Таким образом,  $V \cong \K[x]\bigl/(d_1) \sop
\K[x]\bigl/(d_n)$. В данном случае $d_i \neq 0$, так как пространство конечномерное. Заметим, что
характеристический многочлен оператора $\chi_\ph(\la)$ с точностью до константы равен $d_1(\la)
\ldots d_n(\la)$. Минимальный многочлен для $\ph$ делится на все $d_i \Ra$ он равен $d_n(\la)$.
Теперь доразложим каждое слагаемое в сумму примарных циклических модулей: $d_j(\la) =
\prodl{i=1}{s_j}p_{j i}(\la)^{k_{j i}}$, где $p_{i j}$ неприводимы.

Итак, мы перешли к новым базисам в $F$ и в $N$ следующего вида:  $F=\ha{u'_1 \sco u'_n}, \; N=\ha{y_1' \sco
y_n'}, \; y'_j=d_j(\la)u'_j$. Порождающие циклических модулей это образы $\ol{u'}_j$ элементов $u'_j$
в пространстве $V$ (это не обязательно векторный базис!). Выразим $\ol{u'}_j$ через $u'_j$:

$$\rbmat{
\begin{array}{ccc|ccc}
a_{11}-\la & \ldots & a_{1n}          & 1 &  & 0 \\
               &        &                 & & \ddots &  \\
a_{n1}         & \ldots & a_{nn}-\la  & 0 & & 1 \\
\end{array}}
\rightsquigarrow
\rbmat{
\begin{array}{ccc|ccc}
d_1(\la) &        & 0             & & \\
             & \ldots &               & & C^{-1} & \\
0            &        & d_n(\la)  & &
\end{array}
}.$$

Матрица в правой части после приведения матрицы $A-\la E$ к диагональному виду будет  обратной к матрице
перехода к новому базису в $F$.

Отсюда получается теорема Жордана: Пусть $\K=\Cbb$. Тогда неприводимыми будут только  линейные многочлены
вида $\la - \la_i$. Следовательно,

$$V \cong \K[x]\Bigl/\bigl((\la - \la_1)^{k_1}\bigl) \sop \K[x]\Bigl/\bigl((\la - \la_s)^{k_s}\bigl) = \ha{e_1 \sco e_s}.$$

Векторный базис каждой жордановой клетки $\hc{e_i, (\la - \la_i) e_i \sco (\la - \la_i)^{k_i-1} e_i}$. Вид матрицы в силу
единственности разложения определён однозначно с точностью до перестановки клеток.

%==========================================================
\subsection{Прямые произведения колец (алгебр) и модулей над ними}
%==========================================================

\subsubsection{Прямые произведения колец}

Пусть $R$ кольцо с 1, $I_1 \sco I_s$ его идеалы.

\begin{df}
Кольцо $R$ есть \emph{прямое произведение} $I_1 \sco I_s$, если $R$ как абелева группа есть  прямая сумма:
$R= I_1 \sop I_s$. Обозначение: $R= I_1 \st I_s$.
\end{df}

Из определения следует, что $I_i \cap I_j = \hc{0}$. Если $x_i \in I_i, \; x_j \in I_j \Ra x_i x_j \in I_i
\cap I_j \Ra x_i x_j = 0$. Значит, произведение элементов прямого произведения покомпонентное:
\eqn{x=x_1 \spl  x_s, \quad  y=y_1  \spl  y_s \; \Ra \; xy\bw=x_1 y_1  \spl  x_s y_s.}
Представим единицу в виде суммы: $1=e_1 \spl  e_s$, где $e_i \in I_i$. Тогда $x_i=x_i e_i = e_i x_i$, е элементы $e_i$ являются единицами для
соответствующих подгрупп (однако $I_i$ будут всего лишь подгруппами по сложению, а не подкольцами в $R$, так
как $e_i \neq 1$). Заметим также, что $e_i ^2 = e_i$, а $e_ix_j = x_j e_i = 0$, если $i\neq j$. Таким
образом, получаем свойства системы элементов $e_i$:

\pt{1} $e_i^2=e_i$ (идемпотентность);

\pt{2} $e_ie_j=0, \; i \neq j$ (ортогональность);

\pt{3} $e_i  \spl  e_s =1$ (полная система);

\pt{4} $e_i  \in Z(R)$.

Наоборот, если задана полная система ортогональных идемпотентов, то $I_i := e_iR = Re_i$ идеал в $R$,  и
тогда $R=I_1 \sop I_s$.

Теперь определим внешнее произведение. Рассмотрим декартово произведение $R = R_1 \st R_s$, введём  операции
(покомпонентно). Рассмотрим $I_i = \hc{(0,\ldots,0,x_i,0,\ldots,0)}$, где $x_i \in R$. Очевидно, что $I_i$
будет идеалом. Значит, можно не различать внешнее и внутреннее произведения.

\begin{ex}
$R=\K[x]\bigl/(f), \; f=g_1 \ldots g_s$, и $(g_i,g_j)=1$ для $i \neq j$. Рассмотрим кольца  $R_i :=
\K[x]\bigl/(g_i)$. Тогда $(f) \subseq (g_i)$. Рассмотрим отображения $\ph_i\cln \K[x] \ra R_i$. Построим
отображение $\ph\cln \K[x]\ra R_1 \st R_s$, заданное по правилу $\ph(h):=\bigl(\ph_1(h) \sco \ph_s(h)\bigr)$.
Каждому многочлену сопоставим набор его классов вычетов по модулю $g_i$. По китайской теореме об остатках оно
будет сюръективным, так как для любого набора остатков найдётся элемент, которой при делении на заданный
набор элементов даёт эти остатки. А тогда по теореме о гомоморфизме $\K[x]\bigl/\Ker \ph \cong R_1 \st R_s$.
Но ядро состоит в точности из тех многочленов, которые делятся на каждый из $g_i$, е $\Ker \ph = (f)$.
Значит, $R \cong \prodl{i=1}{s} R_i$.
\end{ex}

\begin{ex}
Множество блочно-диагональных матриц над полем $\K$ образует алгебру. Идеалами, очевидно, будут  матрицы, в
которых в одном из блоков стоит произвольная подматрица, а все остальные блоки нулевые. Вся алгебра будет
произведением таких идеалов.
\end{ex}

\begin{stm}
Центр прямого произведения колец равен произведению их центров.
\end{stm}
\begin{proof}
Пусть $R=R_1 \st R_s, \; x=(x_1 \sco x_s), \; y = (y_1 \sco y_s)$. Умножение покомпонентное,
следовательно, $xy=yx \Lra x_i y_i = y_i x_i \fa i$. Это и означает, что $Z(R)=Z(R_1) \st Z(R_s)$.
\end{proof}

\begin{stm}
Дан модуль $M$ над прямым произведением колец $R = R_1 \st R_s$. Тогда $M$ однозначно разлагается в прямую сумму
$M=M_1 \sop M_s$, где $M_i$ модуль над $R_i$, и $y_jM_i = 0$ для $\fa y_j \in R_j$ при $i \neq j$.
\end{stm}
\begin{proof}
В силу наличия прямого произведения колец имеем разложение для единицы: $1=e_1  \spl  e_s$, где $e_i$ единицы в $R_i$. Положим
$M_i = e_i M$. Покажем, что $M_i$ будут подмодулями. Поскольку $e_i \in Z(R)$, то для $\fa r \in R, \fa x_i = e_i x$, где $x \in M$, имеем
$rx_i=re_i x=e_irx \in M_i$. Если $x_i \in M_i$, то $x_i = e_i x$, и $e_ix_i=e_i^2x=e_ix$, е $M_i$ подмодуль. Пусть
$y_j \in R_j, \; x_i=e_ix \in M_i$. Тогда при $j \neq i$ получаем, что $y_ix_i = \ub{y_ie_i}_0 x=0$, и следовательно,
$M_i \cap \sums{i\neq j}M_j = \hc{0}$. Теперь представим любой элемент $x \in M$ в виде суммы.
Имеем $x=1\cdot x = (e_1  \spl  e_s)x =e_1x \spl  e_s x$. Значит, $M$ есть прямая сумма $M_i$.
\end{proof}

Обратно, пусть заданы модули $M_i$ над каждым $R_i$. Построим из них прямую сумму $M$. Для $\fa r_j \in R_j, \; x_i \in M_i$ при $i \neq j$ положим
$r_j x_i=0$. Определим действие $r=(r_1 \sco r_s)$ на $x \in M$ так: $r \cdot x_i := r_i x_i$. Тогда $M=M_1 \sop M_s$.

\begin{imp}
Пусть задан простой модуль над прямым произведением. Тогда он совпадает с модулем над одним из множителей.
\end{imp}

\subsubsection{Модули над конечномерными алгебрами}

\begin{stm}
Пусть $R$ конечномерная алгебра над $\K$, и $V$ конечнопорождённый $R$-модуль. Тогда $V$
конечномерное векторное пространство.
\end{stm}
\begin{proof}
Пусть $V=\ha{e_1 \sco e_n}_R$. Рассмотрим свободный модуль $F=\ha{u_1 \sco u_s}$. Имеем  $F \cong R \sop R$
($s$ штук). Представим $V$ как фактормодуль $F/Q$. Модуль $F$ конечномерный $\Ra V$ также
конечномерный.
\end{proof}

Пусть $R=\ha{e_1 \sco e_k}_\K$ векторное пространство. Введём умножение на базисных векторах  (превратим
$R$ в алгебру) по формулам $e_i e_j = \suml{k=1}{n}c_{ij}^k e_k$. Построим модуль $V$ над $R$. Для этого
зададим действие элементов $r \in R$ на элементы $x \in V$ (линейные операторы) $\rho_r(x) = rx$. Достаточно
задать такие линейные операторы для базиса: $\rho_i(x)=e_i x$, где $\rho_i = \rho_{e_i}$. При этом
произведение элементов алгебры должно соответствовать произведению операторов: $\rho_i\rho_j =
\suml{k=1}{n}c_{ij}^k\rho_k$, и единице соответствует тождественный оператор $\rho_e = \Ec$. Таким образом,
модуль $V$ над алгеброй $R$ это векторное пространство и семейство линейных операторов. Фактически при
задании модуля рассматривается гомоморфизм из $R$ в алгебру линейных операторов $\Lin(V)$, е
\emph{линейное представление} $R$.

Рассмотрим гомоморфизм $R$-модулей $\ph\cln V \ra W$, е линейное отображение модулей как векторных
пространств, перестановочное с умножением на элементы из $R$, е $\ph(rx)=r\ph(x)$. Пусть $\rho_r$ и
$\widetilde{\rho}_r$ операторы умножения на $r$ в модулях $V$ и $W$ соответственно. Перестановочность
$\rho$ и $\ph$ достаточно проверять только для базисных элементов. Выберем базис в модулях: $V=\ha{v_1 \sco
v_n}, \; W=\ha{w_1 \sco w_n}$. Пусть $T$ матрица $\ph$ в этих базисах, $A_i$ матрица $\rho_i$
относительно базиса $V$, а $B_i$ матрица $\widetilde{\rho}_i$ относительно базиса $W$. Получаем, что
матрицы $B_i$ и $A_i$ должны быть сопряжены одной и той же матрицей $T$: Для $\fa x \; B_i T x = T A_i x \;
\Ra \; B_i T = TA_i \; \Ra \; B_i = TA_iT^{-1}$. Рассмотрим частный случай: $V=W$, и выбрано 2 разных базиса.
Тогда матрицы операторов на $V$ при переходе к другому базису изменяются при помощи сопряжения (хорошо
известное утверждение из линейной алгебры).

%===========================================================
\subsection{Простота и полупростота модулей. Полупростые алгебры}
%===========================================================

\subsubsection{Простые модули}

Пусть дан $R$-модуль $V$. Рассмотрим подмодуль $L \subset V$. Он является подпространством. Наоборот,
подпространство $L$ будет подмодулем, если оно инвариантно относительно всех операторов из $R$, е
$\rho_r(L) \subseq L \Lra \fa x \in L, \; r \in R$ выполнено $rx \in L$. Это свойство достаточно проверять на
базисных операторах $\rho_i$.

\begin{df}
Если существует подпространство, инвариантное относительно всех операторов $\rho_i$, то пространство
называется \emph{приводимым}.
\end{df}
На матричном языке свойство приводимости выглядит так: существует матрица перехода к новому базису, в
котором матрицы всех операторов имеют общий угол нулей:
$$B_i = \rbmat{\begin{array}{c|c} * & * \\ \hline 0 & * \end{array}}.$$

\begin{df}
Модуль называется \emph{полупростым} (\emph{вполне приводимым}), если он является прямой суммой простых.
\end{df}

\begin{lemma}[Эквивалентное условие полупростоты модуля]
Конечномерный модуль $V$ над конечномерной алгеброй $R$ является полупростым $\Lra$ для любого
подмодуля $L \subset V$ существует подмодуль $L' \subset V$, такой, что $V=L \oplus L'$.
\end{lemma}
\begin{proof}
Справа налево докажем по индукции по размерности подмодулей. Имеем $V=L \oplus L'$. Пусть один из подмодулей
(для определённости $L$) не простой. Докажем, что $L$ также расщепляется в прямую сумму. Пусть $M \subset L$.
Тогда по условию $V=M \oplus M'$. Покажем, что $L=M \oplus (L \cap M')$. Очевидно, что $M \oplus (L \cap M')
\subseq L$, остаётся показать обратное включение. Пусть $x \in L \; \Ra \; x=x_1 + x_2$, где $x_1 \in M, \;
x_2 \in M'$. Тогда $x_2 = x-x_1 \in L \; \Ra \; x_2 \in (L \cap M')$. Поскольку модули $L$ и $L'$ имеют
меньшую размерность, то они предположению индукции обладают свойством отщепляемости.

Обратно: пусть $V= V_1 \sop V_s$, где $V_i$ простые подмодули, $L$ некоторый подмодуль в $V$.  Если
$V_i \subset L$ для всех $i$, то тогда $L=V$ и доказывать нечего. Пусть нашлось $i\cln V_i \nsubseteq L$.
Тогда $L \cap V_i = \hc{0}$, так как пересечение $L \cap V_i$ есть подмодуль в $V_i$, а $V_i$ простой.
Следовательно, можно рассмотреть прямую сумму $L \oplus V_i$. Если $L \oplus V_i \neq V$, то повторяем
процедуру отщепления. Рано или поздно всё закончится, так как модуль конечномерный.
\end{proof}

\begin{imp}
Подмодуль или фактормодуль полупростого модуля является полупростым.
\end{imp}
\begin{proof}
Для подмодулей справедливость утверждения обеспечивается леммой. Докажем для фактормодулей.  Пусть $L \subset
V, \; L''=V/L$. Вследствие полупростоты $V$ найдётся подмодуль $L' \subset V\cln \; V=L\oplus L'$. Но тогда
$V/L \cong L'$. Так как $L'$ подмодуль полупростого модуля, то он сам полупростой, а значит, $V/L$
полупростой.
\end{proof}

\subsubsection{Полупростые алгебры}

\begin{df}
Конечномерная алгебра называется \emph{полупростой (слева)}, если как левый модуль над собой она  является
полупростым модулем.
\end{df}

\begin{df}
\emph{Минимальный} левый идеал ненулевой идеал, не содержащий ненулевых подидеалов.
\end{df}

Рассмотрим алгебру $R$ как левый модуль над собой. Пусть она полупроста, е $R= I_1 \sop I_s$. Тогда
простые подмодули $I_i$, очевидно, будут левыми идеалами. Наоборот, алгебра $R$ полупроста слева, если
она является прямой суммой своих минимальных левых идеалов.

\begin{stm}
Любой конечномерный модуль $V$ над полупростой алгеброй $R$ является полупростым.
\end{stm}
\begin{proof}
Модуль $V$ конечномерен $\Ra$ имеет конечную систему порождающих, а значит, может быть представлен как
$V=F/Q$, где $F=\ha{e_1 \sco e_k}_R = R \sop R$ свободный модуль. Если $R$ полупроста, то и $F$
разлагается в прямую сумму простых, е является полупростым. А значит, и $F/Q$ по следствию
также полупростой.
\end{proof}

\begin{stm}
Всякая простая конечномерная алгебра $R$ является полупростой.
\end{stm}
\begin{proof}
Алгебра $R$ конечномерна $\Ra$ в ней есть минимальные левые идеалы (см. определение идеала в алгебре).  Пусть
$I$ некоторый минимальный левый идеал в $R$. Пусть $x \in I, \; r \in R$. Рассмотрим отображение $f\cln x
\mapsto xr$, е $f\cln I \ra Ir$ эпиморфизм. Поскольку $Ir$ левый идеал, то $f$ будет гомоморфизмом
левых модулей (простых). Тогда его ядро либо нулевое, либо совпадает с $I$, так как в $I$ нет нетривиальных
подмодулей. Значит, либо $Ir \cong I$, либо $Ir=0$. Теперь рассмотрим все такие идеалы вида $Ir_i$ для всех
$r_i \in R$ и их сумму
$$J:=\sums{r_i \in R} I r_i = \hc{r \in R\cln \exi x_i \in I, \; r_i \in R \; (i=1,\ldots,s)\cln \; r = x_1 r_1  \spl  x_s r_s}.$$

Это множество будет двусторонним ненулевым идеалом, содержащим $I$, так как $I$ левый идеал, и  умножение
слева ничего не меняет, а при умножении справа снова получается элемент такого же вида. Но алгебра простая, и
нетривиальных идеалов там нет, а значит, $J=R$. В частности, есть разложение для единицы: $1=x_1 r_1  \spl
x_s r_s$. Рассмотрим сумму $Ir_1  \spl  Ir_s$. Она содержит единицу, а следовательно совпадает с $R$.

Теперь рассмотрим внешнюю прямую сумму $V := I r_1 \sop I r_s$. Она будет полупростым модулем (суммой простых).
Построим гомоморфизм $g\cln V \ra R$ по формуле $g(y_1 \sco y_s)=y_1  \spl y_s$, где $y_i = x_i r_i, \; x_i \in I$.
Он будет эпиморфизмом левых модулей, а тогда по теореме о гомоморфизме $R \cong V/\Ker g$. Тогда по следствию
леммы $R$ будет полупростой.
\end{proof}

\begin{note}
Пусть все слагаемые в разложении модуля $V$ между собой изоморфны $I$. Тогда, так как по теореме
Жордана Гёльдера разложение в прямую сумму простых модулей однозначно с точностью до изоморфизма, то любой
подмодуль и фактормодуль в $V$ также однозначно разлагается в прямую сумму модулей, изоморфных $I$. Значит,
$R$ будет прямой суммой минимальных левых идеалов, изоморфных $I$.
\end{note}

\begin{stm}
Пусть $R$ полупростая алгебра, $R= I_1 \sop I_s$, где $I_i$ минимальные левые идеалы. Тогда  любой
неприводимый $R$-модуль $V$ изоморфен одному из $I_i$.
\end{stm}
\begin{proof}
Модуль $V$ порождается одним своим элементом. В самом деле, если бы порождающих было больше, то  тогда в $V$
существовал бы нетривиальный подмодуль, порождённый одним из них. Пусть $V=Rx_0$, где $0\neq x_0 \in V$.
Рассмотрим эпиморфизм $f\cln R \ra V\cln f(r)=rx_0$. Тогда $V \cong R/\Ker f$. Поскольку $R$ полупроста, то $R =
\Ker f \oplus I$, где $I$ некоторый её идеал. Следовательно, $I \cong R/\Ker f \cong V$. Но так как $\Ker
f$ подмодуль полупростого модуля, то он сам полупростой, е $\Ker f = I'_1 \sop I'_{k-1}$, где
$I'_i$ минимальные левые идеалы. Значит, $R=I'_1 \sop I'_{k-1} \oplus I$. В силу однозначности разложения
$k=s$ и слагаемые изоморфны (с точностью до перестановки). Следовательно, $\exi j\cln V \cong I \cong I_j$.
\end{proof}

\begin{imp}
Над простой конечномерной алгеброй все простые модули между собой изоморфны.
\end{imp}

\begin{ex}
Пусть $R=\Mb_n(\K)$ (полная матричная алгебра), и $V=\K^n$. Модуль $V$ будет неприводимым, так как  не
существует подпространства, инвариантного относительно всех операторов. Рассмотрим множество матриц $I_j$, у
которых $j$-тый столбец произвольный, а все остальные столбцы нулевые. Очевидно, что это левый идеал
алгебры, кроме того, он будет минимальным. Имеем $I_j \cong \K^n$, $\Mb_n(\K) = I_1 \sop I_n$. Значит,
алгебра $\Mb_n(\K)$ полупростая.
\end{ex}

\begin{problem}
Доказать, что если над полупростой алгеброй все неприводимые модули между собой изоморфны, то она простая.
\end{problem}

%===========================================
\subsection{Кольцо (алгебра) эндоморфизмов модулей}
%===========================================

\subsubsection{Основные понятия}

\begin{df}
\emph{Эндоморфизмом} модуля называется его гомоморфизм на себя.
\end{df}

Пусть $R$ кольцо или алгебра, $V, W$ $R$-модули. Рассмотрим множество гомоморфизмов $\Hom(V,W)$.
Введём на нём операцию сложения: $(f+g)(x):=f(x)+g(x)$. Очевидно, что это корректно. Значит, $\Hom(V,W)$ есть
абелева группа по сложению. Если $R$ алгебра над $\K$, то зададим умножение на элементы поля: $(\la
f)(x):=\la f(x)$, и группа гомоморфизмов превращается в векторное пространство над $\K$.

Теперь рассмотрим $\End(V):=\Hom(V,V)$. В таком множестве можно ввести ещё одну операцию  композицию
(умножение). Значит, $\End(V)$ кольцо. Оно и называется \emph{\emph{кольцом эндоморфизмов}} модулей.

\begin{note}
Эндоморфизм аналог линейного оператора в векторном пространстве.
\end{note}

Если $R$ алгебра, то $\End(V)$ становится алгеброй. Можно считать, что $\K \subset \End(V)$, так как
можно отождествить скалярные операторы с умножением на числа. Очевидно, что $\K \subset Z(\End(V))$.

Пусть $V$ конечномерный $R$-модуль, где $R$ алгебра над $\K$. Рассмотрим алгебру всех $\K$-линейных
операторов $\Lin(V)$. Имеем $\End(V) \subset \Lin(V)$. Как уже говорилось, можно рассматривать элементы из
$R$ как линейные операторы. По определению гомоморфизма должно выполняться свойство $f(rx)=rf(x)$, а значит,
эндоморфизмы это те операторы, которые коммутируют с операторами из $R$.

\subsubsection{Лемма Шура}

Будем рассматривать гомоморфизмы простых модулей.

\begin{lemma}[Шура]
Пусть $R$ конечномерная алгебра над $\K$, и $V, W$ простые $R$-модули. Тогда:

\pt{1} $f \in \Hom(V,W) \; \Ra \; f$ либо нулевой, либо изоморфизм;

\pt{2} Любой эндоморфизм $V$ является автоморфизмом, и $\End(V)$  алгебра с делением;

\pt{3} Если $\K=\Cbb$, то любой эндоморфизм простого модуля скалярен, е $\End(V)=\Cbb$.
\end{lemma}

\begin{proof}
\pt{1} Ядро эндоморфизма подмодуль, а $V$ простой модуль, значит, либо $\Ker f =\hc{0}$, либо  $\Ker f =
V$. В~первом случае получаем, что $f$ инъекция. $\Img f \subseq W, \; \Img f \neq \hc{0} \; \Ra \; \Img f
= W$, так как $W$ простой модуль.

\pt{2} Очевидно: любой изоморфизм на себя (е автоморфизм) обратим, и значит,
$\End(V)$ алгебра с делением.

\pt{3} Можно сослаться на теорему о том, что все конечномерные алгебры с делением над $\Cbb$ совпадают  с
$\Cbb$. Но не будем «стрелять из пушки по воробьям» и докажем это по-другому. Пусть $f \in \End(V)$. У него
есть собственное значение $\la$, так как $\K=\Cbb$. Если $f=\la \Ec$, то всё ясно, а если нет, то
тогда оператор $f-\la \Ec\neq 0$ также будет эндоморфизмом. Тогда $\Ker(f -\la \Ec)$ подмодуль в
$V$, но вследствие простоты это либо $\hc{0}$, либо весь модуль $V$.
\end{proof}

\subsubsection{Кольцо эндоморфизмов прямой суммы модулей}

Пусть $V=V_1 \sop V_n$. Рассмотрим эндоморфизм $f\cln V \ra V$. Пусть мы уже знаем, как устроены $\Hom(V_i,
V_j)$.  Тогда достаточно задать $f$ на прямых слагаемых. Пусть $x =(x_1 \sco x_n) \in V, \; f(x)=(y_1,
\ldots,y_n)$, где $y_i \in V_i$. $f\bigl((0, \ldots,x_j,\ldots,0)\bigr)=(y_{1j} \sco y_{nj})$. Рассмотрим
$f_{ij}\cln V_j \ra V_i$. Эти гомоморфизмы можно рассматривать как гомоморфизмы всего модуля, если считать, что
$f_{ij}\bigl((x_1 \sco x_n)\bigr)=(0 \sco y_{ij}, \ldots,0)$. Если мы зададим $f_{ij}$ для всех $i$ и $j$, то
тогда мы зададим и весь эндоморфизм $f$. Пусть $x=x_1  \spl  x_n$, тогда $f(x)=f(x_1)  \spl  f(x_n)$. Каждый
«элементарный» гомоморфизм $f_{ij}$ отображает $V_j$ в $V_i$, а всё остальное переводит в 0. Значит,

$$f=\suml{i,j=1}{n}f_{ij}=\rbmat{f_{11} & \ldots & f_{1n} \\ & & \\ f_{n1} & \ldots & f_{nn}}.$$

При таком задании $f_{ij}$ сумме эндоморфизмов соответствует сумма, а композиции произведение матриц:

$$f_{ij} \cdot g_{kl} = \case{0, \; j \neq k; \\ (f \circ g)_{il}, \; j=k.}$$
Значит, можно отождествить элементы кольца эндоморфизмов прямой суммы модулей с матрицами из $f_{ij}$.

%========================================================
\subsection{Основная теорема о полупростой алгебре над $\Cbb$}
%========================================================
\subsubsection{Гомоморфизмы полупростых модулей}

Рассмотрим частный случай: $V$ полупростой $R$-модуль, а $R$ алгебра над $\Cbb$. Разложим $V$ на
простые и сгруппируем изоморфные слагаемые в блоки:

$$V=(\ub{V_1 \sop V_{n_1}}_{n_1})\oplus(\ub{V_{n_1+1} \sop V_{n_1+n_2}}_{n_2}) \sop  V_s.$$

 Пусть $\ph_{ij} \in \Hom(V_j, V_i)$.
Тогда, по лемме Шура, если $V_j$ и $V_i$ в одном блоке, то $\Hom(V_i, V_j) \cong \Cbb$ (только умножение   на
скаляры), а если в разных, то $\Hom(V_i, V_j) = \hc{0}$. Значит, матрицы из алгебры эндоморфизмов будут иметь
следующий блочный вид:

$$\rbmat{\mat{\Mb_{n_1}(\Cbb) \\ n_1 \times n_1}& & & \text{{\LARGE $0$}} \\ &\mat{\Mb_{n_2}(\Cbb)\\ n_2 \times n_2}&  & \\ & & \ddots & \\ \text{{\LARGE $0$}} & & &\mat{\Mb_{n_s}(\Cbb)\\ n_s \times n_s}}$$

Получается следующее

\begin{stm}
Алгебру эндоморфизмов полупростого модуля можно отождествить с прямым произведением полных матричных  алгебр
$\Mb_{n_1}(\Cbb) \st \Mb_{n_s}(\Cbb)$.
\end{stm}

\subsubsection{Основная теорема и её следствия}

\begin{theorem}
Полупростая алгебра $R$ над $\Cbb$ изоморфна прямому произведению полных матричных алгебр.
\end{theorem}
\begin{proof}
Рассмотрим алгебру как левый модуль над собой. Представим её как прямую сумму минимальных левых идеалов,
сгруппировав их в блоки:
$$R=(\ub{V_1 \sop V_{n_1}}_{n_1})\oplus(\ub{V_{n_1+1} \sop V_{n_1+n_2}}_{n_2}) \sop V_{n_k}, \; \dim R = n.$$

Из предыдущего утверждения следует, что $\End_R(R) = \Mb_{n_1}(\Cbb) \st \Mb_{n_k}(\Cbb)$. Теперь построим
изоморфизм алгебр между $R$ и $\End_R(R)$. Сопоставим каждому элементу алгебры $r \in R$ эндоморфизм
$\ph_r(x):=xr$. Проверим, что это действительно эндоморфизм:
$\ph_r(x_1+x_2)=(x_1+x_2)r=x_1r+x_2r=\ph_r(x_1)+\ph_r(x_2)$, и $\ph_r(\la x)=(\la x)r=\la(x
r)=\la \ph_r(x)$. Таким образом, получается отображение $\mu\cln R \ra \End_R(R)$. Покажем, что оно
биективно. Инъективность: если $r \neq 0$, то $\ph_r(1)=r \Ra \ph_r \neq 0$. Сюръективность: возьмём любой
эндоморфизм $\ph \in \End_R(R)$, положим $r:=\ph(1)$. Тогда, пользуясь тем, что $\ph$ эндоморфизм левых
модулей и он перестановочен с умножением на элементы из $R$, получаем, что $\ph(x)=\ph(x \cdot 1)=x \ph(1) =
x r = \ph_r(x)$, е любому эндоморфизму соответствует некоторый $\ph_r$. Биективность доказана. Остается
построить изоморфизм алгебр $R$ и $\End_R(R)$. Первым кандидатом на роль изоморфизма является само
отображение $\mu$, но беда в том, что если $r_1, r_2 \in R$, то
$$\ph_{r_1 r_2}(x)=x(r_1 r_2)=(x r_1)r_2 = \ph_{r_1}(x) r_2 = \ph_{r_2}(\ph_{r_1}(x))=(\ph_{r_1} \circ \ph_{r_2})(x).$$

Перемножение происходит в обратном порядке: $\mu(r_1 r_2)=\mu(r_2)\mu(r_1)$. Значит, $\mu$ не является
изоморфизмом алгебр. Теперь «подправим» наше отображение, вспомнив, что произведение транспонированных матриц
есть транспонированное произведение в обратном порядке. Пусть $T$ отображение транспонирования. Тогда $(T
\circ \mu)$ уже будет изоморфизмом алгебр, откуда и следует, что $R \cong \Mb_{n_1}(\Cbb) \sop
\Mb_{n_k}(\Cbb)$.
\end{proof}

Сформулируем следствия теоремы. $R$ полупростая алгебра над $\Cbb$. Имеем $R \cong \Mb_{n_1}(\Cbb) \st \Mb_{n_k}(\Cbb)$.
Тогда:

\begin{imp}
Если $V$ простой модуль над $R$, то он будет модулем над одним из сомножителей $\Mb_{n_i}(\Cbb)$ для
некоторого $i$. Все простые модули над $\Mb_n(\Cbb)$ изоморфны минимальному левому идеалу, состоящему из
матриц, у которых все столбцы, кроме одного, нулевые. Таким образом, если есть $k$ блоков, то есть $k$
попарно неизоморфных модулей $V$ над блоками, и $\dim_\Cbb V=n_i$, где $n_i$ размер блока.
\end{imp}

\begin{imp}
Размерность полупростой алгебры над $\Cbb$ равна сумме квадратов размерностей простых модулей над ней, е
$\dim_\Cbb R = n_1^2  \spl  n_k^2 = \dim \Mb_{n_1}(\Cbb)  \spl  \dim \Mb_{n_k}(\Cbb)$.
\end{imp}

\begin{imp}
\label{ThirdCorMainTheorem}
Имеем $Z(R)= Z(M_{n_1}(\Cbb)) \st Z(M_{n_k}(\Cbb))$. Центр образован скалярными матрицами  $\Ra Z(R)=\Cbb^k$,
и $\dim_\Cbb Z(R) = k$, где $k$~-- число блоков. Следовательно, число неизоморфных простых модулей над
полупростой комплексной алгеброй равно размерности её центра.
\end{imp}

\begin{imp}
Если полупростая алгебра над $\Cbb$ коммутативна, то все простые модули над ней одномерны. Верно также и
обратное утверждение.
\end{imp}
\begin{proof}
Пусть задана совокупность коммутирующих линейных операторов $\hc{\Ac_i}$ на комплексном конечномерном
векторном пространстве. Докажем, что у них есть общий собственный вектор, е общее одномерное инвариантное
подпространство. Проведём индукцию по $n:=\dim V$. Если $n=1$, доказывать нечего. Пусть всё доказано для
размерности меньше $n$. Если все операторы скалярные, то всё ясно. Пусть есть какой-то нескалярный оператор,
для определённости $\Ac_1$, и $\la$ его собственное значение. Рассмотрим его собственное
подпространство $V_\la = \Ker(\Ac - \la \Ec)$. Очевидно, оно ненулевое и не совпадает со всем
пространством. Покажем, что оно инвариантно относительно всех $\Ac_i$. Пусть $x \in V_\la$. Тогда
$\Ac_1(\Ac_i(x)) \eqvl{комм.}{20} \Ac_i(\Ac_1(x))=\Ac_i(\la x)=\la \Ac_i(x)$, е
$\la$ будет собственным значением для всех $\Ac_i$. Имеем $\dim V_\la < n \Ra$ можно применить
предположение индукции. Значит, все простые модули одномерны.
\end{proof}

%############################################################
\section{Линейные представления групп}
%############################################################

%===============================================
\subsection{Основные понятия}
%===============================================

\subsubsection{Понятие линейного представления}

\begin{df}
Пусть $G$ группа, $V$ конечномерное векторное пространство над полем $\K$. \emph{Линейным
представлением} группы $G$ называется гомоморфизм $\rho\cln G \ra \GL(V)$.
\end{df}

Линейное представление является частным случаем действия группы на векторном пространстве $V$. Однако
каждому элементу группы в данном случае сопоставляется не произвольное биективное отображение $V$ в себя, а
линейный оператор. Произведению элементов соответствует композиция операторов: $g_1 g_2 \mapsto \rho(g_1
g_2)=\rho(g_1)\rho(g_2)$. Единица переходит в тождественный оператор $\Ec$.

Пусть в пространстве $V$ выбран какой-то базис: $V=\ha{e_1 \sco e_n}$. Тогда можно перейти к матричному
представлению $\rho\cln G \ra \GL_n(\K)$.

\begin{df}
\emph{Размерностью представления} называется размерность пространства $V$.
\end{df}

\begin{df}
Пусть есть 2 представления $\rho_1\cln G \ra \GL(V)$ и $\rho_2\cln G \ra \GL(W)$. \emph{Гомоморфизмом}  линейных
представлений называется линейное отображение векторных пространств $\ph\cln V \ra W$, при котором
$\ph\bigl(\rho_1(g)(x)\bigr)\bw=\rho_2(g)\bigl(\ph(x)\bigr)$ для $\fa g \in G$, е следующая диаграмма
коммутативна:
$$\dsquare[V`W`V`W;\ph`\rho_1`\rho_2`\ph]$$
\end{df}

\begin{df}
Гомоморфизм $\ph$ линейных представлений $\rho_1$ и $\rho_2$ называется \emph{изоморфизмом} в том случае,
когда он является изоморфизмом векторных пространств. В этом случае говорят, что представления
\emph{эквивалентны}.
\end{df}

Выясним, что означает эквивалентность представлений $\rho_1$ и $\rho_2$ в терминах матриц. Пусть
$V=\ha{e_1 \sco e_n}$, и $W=\ha{f_1 \sco f_n}$. Пусть $C$ матрица изоморфизма $\ph$, а $\rho_1(g)$ и $\rho_2(g)$
матрицы операторов в соответствующих базисах. Тогда, по определению изоморфизма,
$$\rho_2(g)C=C\rho_1(g) \Lra \rho_2(g)=C\rho_1(g)C^{-1} \; \fa g \in G.$$
Таким образом, матрицы всех операторов, соответствующих элементам группы, должны быть подобны, и (что  самое
важное) сопрягающая матрица одна для всех элементов группы.

\subsubsection{Приводимость представлений}

\begin{df}
Линейное представление называется \emph{приводимым}, если существует нетривиальное подпространство  $W
\subset V$, являющееся инвариантным относительно $\rho(g)$ для всех $g \in G$. В этом случае индуцируется
представление на пространстве $W$. Если такого подпространства нет, представление \emph{неприводимо}.
\end{df}

На матричном языке приводимость означает, что если выбрать базис $V=\ha{e_1 \sco e_k, e_{k+1} \sco e_n}$, такой, что
$W=\ha{e_1 \sco e_k}$, то у всех операторов $\rho(g)$ будет общий угол нулей.

\begin{note}
Иногда для краткости пишут $\rho(g)(x)=gx$.
\end{note}

Если $G=\hc{g_1 \sco g_m}$, то очевидно, что представление неприводимо $\Lra \; \ha{g_1x \sco g_m x}_\K = V$.

\begin{df}
Линейное представление $\rho$ называется \emph{вполне приводимым (полупростым)}, если оно разлагается в
прямую сумму неприводимых, то есть $V = V_1 \sop V_s$, где $V_i$ инвариантные подпространства
относительно $\rho(g)$ для всех $g \in G$. Обозначение: $\rho = \rho_1 \sop \rho_s$.
\end{df}

Вполне приводимость означает, что матрицы $\rho(g)$ в подходящем базисе будут блочно-диагональными, и в
каждом блоке стоит матрица ограничения представления $\rho_i := \rho\evn {V_i}$ на подпространство $V_i$.

\subsubsection{Примеры линейных представлений}

\begin{ex}
Рассмотрим линейное представление бесконечной циклической группы $\rho\cln \ha{a}_\infty \ra \GL(V)$.
Достаточно задать одну невырожденную матрицу $\rho(a)$. Очевидно, два представления будут эквивалентными
$\Lra$ матрицы для порождающих элементов подобны, е обладают одинаковой жордановой формой.
\end{ex}

\begin{ex}
$G=\ha{a}_n$. В этом случае опять достаточно задать матрицу для порождающего элемента, такую, что
$\rho(a)^n=\Ec$. Аннулирующим для оператора $\rho(a)$ будет многочлен $t^n-1$, а значит, над полем $\Cbb$
матрица диагонализируема, и все жордановы клетки будут одномерными, так как у многочлена нет кратных корней.
\end{ex}

\begin{note}
Над полем $\Cbb$ все конечномерные представления будут вполне приводимыми.
\end{note}

\begin{ex}
Пусть $G$ свободная группа с базисом $x_1 \sco x_n$. На порождающих элементах представление можно
задавать произвольным образом. Представления эквивалентны, если $\rho_2(x_i)=C\rho_1(x_i)C^{-1}, \, i=\ol{1,\,n}$.

Рассмотрим более общий случай. Пусть группа $G$ порождается элементами $a_1 \sco a_n$. Чтобы задать  линейное
представление, нужно определить $\rho(a_1) \sco \rho(a_n)$, и при этом должны выполняться определяющие
соотношения, е если $a_{i_1}^{\ep_1} \ldots a_{i_s}^{\ep_s}=e$, то тогда $\rho(a_{i_1})^{\ep_1} \ldots
\rho(a_{i_s})^{\ep_s} = \Ec$.
\end{ex}

\subsubsection{Связь модулей с линейными представлениями групп}

Убедимся в том, что линейное представление частный случай модуля над алгеброй.  Пусть $G=\hc{g_1 \sco
g_n}$, $\K$ поле. Рассмотрим групповую алгебру $\K G=\ha{g_1 \sco g_n}=\BC{\sumg a_g g \vl a_g \in \K}$,
е $n$-мерное векторное пространство над $\K$, у которого базисные векторы формально занумерованы
элементами группы. Рассмотрим модуль $V$ над $\K G$. Зададим умножение в модуле на элементы алгебры, е
действие $r \in \K G$ на $x \in V$. Для этого зададим умножение на базисных векторах: для $r = g \in G$ и  $x
\in V$ определим линейный оператор $\rho(g)(x)=gx$. Наоборот, пусть задано представление $\rho\cln G \ra \GL(V)$.
Чтобы задать умножение $r \cdot x$, достаточно задать его на базисных элементах: $gx\bw=\rho(g)(x)$.
При этом произведение базисных элементов переходит в произведение операторов, поэтому аксиомы модуля
выполняются автоматически: $(\sum a_g g)x \eqvl{дистр.}{25} \sum a_g \rho(g) x$.

Таким образом, мы видим, что рассмотрение линейных представлений равносильно рассмотрению модулей.  В
частности, практически одинаковым оказывается понятие гомоморфизма, приводимости, тд

%================================================
\subsection{Основные теоремы о линейных представлениях}
%================================================

\subsubsection{Лемма Шура для линейных представлений. Теорема Машке}

Переформулируем лемму Шура для линейных представлений.
\begin{lemma}
Пусть $V, W$ неприводимые линейные представления группы $G$ над полем $\Cbb$. Тогда любой гомоморфизм
$\ph\cln V \ra W$ либо изоморфизм, либо нулевой. Гомоморфизм неприводимого комплексного представления в себя
является скалярным (Матрица, коммутирующая со всеми операторами, может быть только скалярной).
\end{lemma}

\begin{theorem}[Машке]
Пусть $G$ конечная группа порядка $n$, $\K$ поле и $\Char \K$ не является делителем порядка группы
(в частности, $\Char \K = 0$). Тогда групповая алгебра $\K G$ полупроста.
\end{theorem}
\begin{proof}
Покажем, что любой конечномерный модуль $V$ над $\K G$ обладает свойством отщепляемости, е если
существует инвариантное подпространство $L \subset V$, то существует и дополнительное инвариантное
подпространство $L'$ такое, что $V=L \oplus L'$. Это равносильно тому, что существует гомоморфизм модулей $\pi\cln V \ra L$,
являющийся проекцией на $L$, е $\pi(x)=x$, если $x \in L$. Если мы найдём такой гомоморфизм, то положим
$L':=\Ker \pi$ и утверждение будет доказано. В самом деле, покажем, что $V=\Img \pi \oplus \Ker \pi$.
Проектор это такой оператор $\pi$, что $\pi^2(x)=\pi(x)$. Запишем тождество $x=\pi(x)+x-\pi(x)$.
Поскольку $\pi(x-\pi(x))=\pi(x)-\pi^2(x)=\pi(x)-\pi(x)=0$, то $x-\pi(x) \in \Ker \pi$. Таким образом, вектор
разлагается в сумму $\pi(x) \in \Img \pi$ и $x-\pi(x) \in \Ker\pi$. Пересечение ядра с образом нулевое,
поэтому сумма прямая.

Построим проекцию на $L$, являющуюся гомоморфизмом модулей. Пусть $\wt{\pi}\cln V \ra L$ произвольная
проекция на $L$, при которой элементы из $L$ отображаются тождественно. Построим отображение
$\pi(x):=\frac{1}{n} \sumg g \wt{\pi}(g^{-1}x)$. Коэффициент $\frac{1}{n}$ имеет смысл, так как $\Char \K
\nmid n$ и в поле $\K$ число $n\cdot 1$ не равно 0 и, стало быть, обратимо. Покажем, что $\pi$ является
гомоморфизмом модулей, а именно, $\pi(hx)=h\pi(x)$. В самом деле,

\eqn{\label{ProjectionFormula}\pi(h x)=\frac{1}{n}\sumg g \wt{\pi}(g^{-1}h x)=\frac{1}{n}\sumg h (h^{-1} g) \wt{\pi}\bigl((g^{-1}h)x\bigr)
\eqvl{дистр.}{25} h \cdot \frac{1}{n}\sumg (h^{-1} g) \wt{\pi}\bigl((h^{-1}g)^{-1}x\bigr).}

Если $g$ при суммировании пробегает всю группу, то и $h^{-1}g$ также пробегает всю группу. Значит,
$(\ref{ProjectionFormula})$ равно $h \pi(x)$. Теперь проверим, что это проекция на $L$. Если $x \in L$, то и
$g^{-1}x \in L$, так как $L$ инвариантное подпространство. Но тогда и $\wt{\pi}(g^{-1}x) \in L$, а значит, и
$g \wt{\pi}(g^{-1}x) \in L$. Кроме того, \equ{\pi(x)=\frac{1}{n}\sumg
g\ub{\wt{\pi}(g^{-1}x)}_{g^{-1}x}=\frac{1}{n}\sumg g g^{-1}x=\frac{1}{n}\ub{\sum x}_{nx}=x.}
\hfill\end{proof}

\subsubsection{Ортогональные и унитарные представления}

Пусть $V$ векторное пространство над полем $\R$ или $\Cbb$. Введём скалярное произведение (евклидово или эрмитово).

\begin{df}
Представление называется \emph{ортогональным} (соответственно, для $\Cbb$~-- \emph{унитарным}), если  все
операторы $\rho(g)$ ортогональны (унитарны).
\end{df}

Пользуясь этим понятием, можно легко доказать теорему Машке для полей $\R$ и $\Cbb$. Рассмотрим случай $\K=\Cbb$.
Покажем, что можно ввести такое скалярное произведение, относительно которого линейное представление будет унитарным. Сначала введём обычное эрмитово
произведение $(x,y)=\suml{i=1}{n}\ol{x_i}y_i$. Построим новое скалярное произведение $\ha{x,y}:=\frac{1}{n}\sumg(gx,gy)$. Ясно, что это также
невырожденная эрмитова форма. Тогда любое представление будет унитарным, поскольку если
$h \in G$, то
\eqn{\ha{h x,h y}=\frac{1}{n}\sum (g h x,g h y)=\frac{1}{n}n\ha{x,y}=\ha{x,y}}
(здесь $gh$ также пробегает всю $G$).

Теперь доказательство теоремы Машке тривиально: если есть инвариантное подпространство относительно
ортогонального (унитарного) оператора, то ортогональное дополнение также инвариантно, а этот факт был доказан
в курсе линейной алгебры.

\begin{problem}
Доказать обратную теорему Машке: если групповая алгебра полупроста, то $\Char\K \nmid |G|$. Идея решения: от противного, пусть $\Char \K$
делит порядок группы. Нужно рассмотреть алгебру $\K G$ как модуль над собой и доказать, что подпространство $L:= \bba{\sumg g}$ не отщепляется.
\end{problem}

\subsubsection{Свойства линейных представлений. Регулярное представление}

Дано представление $\rho\cln G \ra \GL(V)$, где $V$ векторное пространство над $\Cbb$. Перечислим  его свойства.

\begin{stm}
Любое комплексное представление вполне приводимо (уже доказывалось).
\end{stm}

\begin{df}
\emph{Регулярным представлением} группы $G$ называется представление $\Lambda$ на групповой алгебре  $\K
G$, заданное по правилу $\Lambda(h)(\sumg a_g g)=\sumg a_g h g$.
\end{df}

Разложим групповую алгебру (а вместе с ней и регулярное представление) на неприводимые и сгруппируем
изоморфные слагаемые в блоки:

$$\K G=(\ub{V_1 \sop V_{n_1}}_{n_1}) \oplus (\ub{V_{n_1+1} \sop V_{n_1+n_2}}_{n_2}) \sop (\ub{V_{n_1 \spl n_{s-1}} \sop V_{n_1 \spl n_s}}_{n_s}).$$

\begin{stm}
Любое неприводимое представление входит в регулярное представление.\footnote[1]{На лекциях это утверждение
не доказывалось. (Прим. наб.)}
\end{stm}
\begin{proof}
Пусть $\La: G \ra \GL(\K G)$ регулярное представление группы $G$, и $\rho\cln G \ra \GL(V)$
произвольное неприводимое представление. Фиксируем вектор $x \in V$. Рассмотрим линейное отображение $\ph_x\cln \K G \ra V$, заданное по правилу $\sumg a_g g \corr{\ph_x} \sumg a_g \rho(g) x$. Покажем, что $\ph_x$
гомоморфизм линейных представлений. Имеем
$$\ph_x\bigl(\La(h)(\sumg a_g g)\bigr) \eqdef \; \ph_x\bigl(\sumg a_g h g\bigr)=\sumg a_g \rho(hg)x=\rho(h)\bigl(\sumg a_g \rho(g)x\bigr)=\rho(h)\cdot \ph_x\bigl(\sumg a_g g\bigr).$$

Заметим также, что $\ph_x(e)=\rho(e)(x)=\Ec x =x$. Рассмотрим произвольный гомоморфизм представлений  $\ph\cln \Lambda \ra \rho$. Тогда найдётся $x \in V$, для которого $\ph = \ph_x$. В самом деле, положим $x:=\ph(e)$.
Тогда для базисных элементов имеем $\ph(g)=\ph(g \cdot e)=\ph(\Lambda(g)e)=\rho(g)\ph(e)=\rho(g)x=\ph_x(g)$,
е никаких других гомоморфизмов, кроме $\ph_x$, тут быть не может. Далее, применяя первое утверждение
леммы Шура, получаем, что всякое неприводимое представление изоморфно подпредставлению регулярного
представления.
\end{proof}

\begin{imp}
Кратность вхождения неприводимого представления в регулярное представление равна его размерности:
$\suml{i=1}{s}n_i^2=\dim \K G = |G|$.
\end{imp}

\begin{stm}
Число неприводимых комплексных представлений равно размерности центра групповой алгебры и равно числу
классов сопряжённости группы.
\end{stm}
\begin{proof}
Очевидно, что элемент лежит в центре алгебры тогда и только тогда, когда он коммутирует с базисом, е если
$x=\sumg a_g g$, то $x \in Z(\Cbb G) \; \Lra \; hx=xh \; \fa h \in G$. А это значит, что $hxh^{-1}=x \; \Lra
\; \sumg a_g h g h^{-1} = \sumg a_g g$. Таким образом, у вектора должны быть одинаковые координаты при
сопряжённых базисных элементах. Поэтому можно сгруппировать сопряжённые элементы из каждого класса и вынести
их за скобки. Следовательно, каждый базисный вектор центра это сумма элементов в некотором классе
сопряжённости. По следствию \ref{ThirdCorMainTheorem} основной теоремы число
неизоморфных неприводимых модулей (а вместе с тем и
число неприводимых представлений) равно размерности центра алгебры.
\end{proof}

%===============================================================
\subsection{Линейные комплексные представления различных классов групп}
%===============================================================

\subsubsection{Представления абелевых групп}

Здесь и далее предполагаем, что $\K=\Cbb$. Всякое неприводимое комплексное представление абелевой группы
будет одномерным, е $\chi\cln G \ra \GL_1(\Cbb)=\Cbb^*$. Эквивалентность представлений в силу
коммутативности $\Cbb^*$ есть обычное равенство. Разложим группу на циклические: $G = \ha{a_1}_{n_1} \st
\ha{a_k}_{n_k}$. Зададим представление на порождающих: $\chi(a_i) \in \Cbb^*, \; a_i^{n_i}=1 \Ra
\chi(a_i)^{n_i} = 1$, е $\chi(a_i)$ корни $n_i$-той степени из 1. Для каждого $a_i$ есть $n_i$
возможностей, поэтому всего $n_1 \ldots n_k=n=|G|$ различных гомоморфизмов $\chi$. Разложим регулярное
представление абелевой группы на неприводимые. Построим одномерные подпространства, являющиеся собственными
для всех операторов. Для каждого $\chi$ рассмотрим вектор $v_\chi:=\sumg \chi(g^{-1})g \in \Cbb G$. Он будет
собственным, к
$$hv_\chi = \sumg \chi(g^{-1})hg=\sumg \chi(h)\chi(g^{-1}h^{-1})hg=\chi(h)\sumg  \chi\bigl((hg)^{-1}\bigr)(hg)=\chi(h) v_\chi,$$
и $\chi(h)$ будет собственным значением. Проверка линейной независимости предоставляется читателю в
качестве элементарного упражнения. Таким образом, $\Cbb G$ является прямой суммой $n$ собственных
подпространств.

\subsubsection{Одномерные представления произвольной конечной группы}

Пусть $G$ конечная группа, $\chi\cln G \ra \Cbb^*$ её одномерное представление.  Тогда $G/\Ker \chi =
\Img \chi \subset \Cbb^*$ абелева группа. Значит, она содержит коммутант $G'$ группы $G$. Рассмотрим
образ смежного класса $gG'$ при гомоморфизме $\chi$. Имеем $gG' \subseq g\Ker \chi \Ra \chi(gG') \subset
\chi(g \Ker \chi)=\chi(g)$. Вывод: гомоморфизмы $\chi$ находятся в биективном соответствии с гомоморфизмами
$G/G' \ra \Cbb^*$, и число одномерных комплексных представлений равно $|G/G'|$.

\begin{problem}
Доказать, что у неабелевых групп существуют неприводимые многомерные представления.
\end{problem}

\subsubsection{Линейные представления групп $\Db_n, \; \Qb_8, \; \Sb_n, \; \Ab_n$}

Рассмотрим группу диэдра $\Db_n$. Имеем $\Db_n=\ha{a,b}$, где $a$ поворот, а $b$ симметрия.
Определяющие соотношения: $a^n=e, b^2=e, bab=a^{-1}$. Пусть $n$ нечётно. Тогда $|\Db_n/\Db_n'|=2$, е
существует 2 одномерных представления. Группа диэдра естественным образом действует на плоскости, поэтому
положим $\rho(b)=b, \; \rho_k(a)=a^k$ при $k=1 \sco \frac{n-1}{2}$. Соотношения, очевидно, выполняются. Такие
представления неприводимы, поскольку собственный вектор симметрии имеет вещественные координаты, а
собственный вектор поворота нет, и значит, не может быть общих собственных векторов (е одномерных
инвариантных подпространств). Построенные представления будут неэквивалентными, так как $\rho_k(a)$ и
$\rho_m(a)$ при $k \neq m$ имеют разные собственные значения, а потому матрицы не могут быть подобными. Итого
получилось $2+2^2 \frac{n-1}{2}=2n=|\Db_n|$. Значит, это все неприводимые представления группы $\Db_n$ при
нечётном $n$. В случае $n=2k$ имеем $|\Db_n/\Db_n'|=4$. Поступим аналогично, только число в данном случае $k
= 1 \sco \frac{n-2}{2}$. Всего $4+2^2 \frac{n-2}{2}=2n$. Значит, это все представления.

Теперь рассмотрим группу $\Sb_3$. Можно было бы свести задачу к предыдущей, заметив,  что $\Sb_3 \cong
\Db_3$. Но поступим по-другому. Имеем $|\Sb_3/\Sb_3'|=2$. Значит, есть ещё одно двумерное представление.
Пусть $V=\ha{e_1, e_2, e_3}$. Зададим представление (так называемое \emph{\emph{мономиальное
представление}}), переставляющее базисные векторы, т.е если $\pi \in \Sb_3$, то $\pi e_i = e_{\pi(i)}$.
Рассмотрим подпространство $L:=\ha{e_1+e_2+e_3}$. Оно, очевидно, будет инвариантным. Ортогональное дополнение
к нему $\ha{e_1+e_2+e_3}^\bot$ будет также инвариантным и неприводимым.

Перечислим неприводимые представления группы $\Qb_8$. Вспомним, что $\Qb_8 \subset \Hbb, \; \Cbb \subset
\Hbb$.  Рассмотрим $\Hbb$ как векторное пространство над $\Cbb$ и зададим умножение на скаляры: $\la
\cdot x := x\la$ для $\fa \la \in \Cbb, \; x \in \Hbb$. Представление зададим так: пусть $g \in
\Qb_8$, тогда положим $\rho(g)(x)=gx$ (обычное умножение слева). Это будет линейный оператор, так как
$\rho(g)(\la \cdot x) = \rho(g)(x\la)=g(x\la)=(gx)\la = \la \cdot
\bigl(\rho(g)(x)\bigr)$. Это представление двумерно, теперь покажем, что оно неприводимо. Допустим противное,
пусть существует нетривиальное инвариантное подпространство. Тогда, так как $\ha{\Qb_8}=\Hbb$, то оно было бы
инвариантным и относительно всех элементов из $\Hbb$, е это был бы левый идеал, а в алгебре с делением он
совпадает со всей алгеброй. Поскольку $|\Qb_8/\Qb_8'|=4$, то будет 4 одномерных представления. Получаем
$4+2^2=8=|\Qb_8|$, е перечислены все представления.

Рассмотрим представления группы $\Ab_n$. Пусть $V=\ha{e_1 \sco e_n}_\Cbb$. Рассмотрим мономиальное
представление, переставляющее базисные векторы. Как и в случае $\Sb_3$, оно приводимо и инвариантным
подпространством будет $\ha{e_1  \spl  e_n}$. Покажем, что ортогональное дополнение $L:=\ha{e_1  \spl
e_n}^\bot$ будет инвариантным и для $\Ab_n$. Имеем $L=\hc{x=(x_1 \sco x_n)\cln \sum x_i=0}$. Пусть $n \ge 4$.
Докажем, что из любого ненулевого вектора путём перестановок координат можно получить базис $L$. Ясно, что
если у вектора $x$ больше трёх ненулевых координат, то можно тройным циклом переставить какие-то 3 из них и
затем вычесть результат из исходного вектора, после чего останется только 3 ненулевых координаты. Поэтому без
ограничения общности можно считать, что $x=(x_1,x_2,x_3,0,\ldots,0)$. Если одна из первых трёх координат (для
определённости $x_3$) равна 0, то $x_1=-x_2$ и можно построить набор векторов

\begin{align*}
u_1=(1,    -1, \phm0, \phm0, \phm0, \phm\ldots \phm0, \phm0)\\
u_2=(0, \phm1,    -1, \phm0, \phm0, \phm\ldots \phm0, \phm0)\\
u_3=(0, \phm0, \phm1,    -1, \phm0, \phm\ldots \phm0, \phm0)\\
.\dots\dots\dots\dots\dots\dots\dots\dots\dots\dots\dots\dots\\
u_n=(0, \phm0, \phm0, \phm0, \phm0, \phm\ldots \phm1,    -1)
\end{align*}
Они, как легко видеть, образуют базис $L$. А если все три координаты ненулевые, то рассмотрим  вектор
$y=(123)\cdot x=(x_3, x_1, x_2,0,\ldots,0)$. Если $x$ и $y$ линейно независимы, то вектора $u_1$ и $u_2$
содержатся в их линейной оболочке и можно, как и в первом случае, построить базис $L$. Если же $x$ и $y$
линейно зависимы, то тогда имеем $\begin{array}{|cc|}x_2 & x_3 \\ x_1 & x_2\end{array} = x_2^2-x_1x_3=0$.
Рассмотрим ещё 2 вектора: $(12)(34)\cdot x=(x_2, x_1, 0, x_3 \sco 0)$, и
$(234)\cdot x \bw= (x_1,0,x_2,x_3,0,\ldots,0)$. Они линейно независимы, так как $\begin{array}{|ccc|}x_2 & x_3 & 0 \\ x_1 & 0 &
x_3 \\ 0 & x_2 & x_3\end{array}=-x_1x_3^2-x_2^2x_3=-2x_1x_3^2 \neq 0$. Далее рассуждения аналогичны. Итак,
получаем, что ограничение мономиального представления на подпространство $L$ неприводимо.

Отдельно разберём случай $n=4$. Для группы $\Ab_4$ имеем $|\Ab_4/\Ab_4'|=3$. Выше было построено
неприводимое представление размерности $n-1$, значит, в нашем случае $3+3^2=12=|\Ab_4|$, е это все
неприводимые представления. Для группы $\Sb_4$ имеем 2 одномерных представления, так как $|\Sb_4/\Sb_4'|=2$.
Пусть $\rho$ мономиальное представление. Построим по нему ещё одно неприводимое представление. Положим
$\rho'(\pi)(x)=(\sgn \pi) \rho(\pi)(x)$. Неприводимость следует из того, что при ограничении на подгруппу
$\Ab_4$ получается обычное мономиальное представление. Покажем, что $\rho' \nsim \rho$. Допустим противное,
е что $\rho'(\pi)=C^{-1}\rho(\pi)C$ для $\fa \pi \in \Sb_4$. В частности, что должно быть верно для $\pi
\in \Ab_4$, но в $\Ab_4$ $\rho'$ совпадает с $\rho$, а потому $C$ скалярная матрица. Значит,
$\rho'(\pi)=\rho(\pi)$ для $\fa \pi \in \Sb_4$, а это не так, если $\pi$ нечётная перестановка. Чтобы
найти ещё одно двумерное представление, воспользуемся следующим обстоятельством. Если есть эпиморфизм групп
$f\cln G \ra H$, и $\rho$ неприводимое представление группы $H$, то есть и неприводимое представление
$\wt{\rho}$ группы $G$ той же размерности, которое можно определить по правилу $\wt{\rho}(g):=\rho(f(g))$. В
случае группы $S_4$ имеем $\Vb_4 \nl \Sb_4$, и $\Sb_4/\Vb_4 \cong \Sb_3$, поэтому можно рассмотреть
эпиморфизм $f\cln \Sb_4 \ra \Sb_3$ и получить искомое представление. Итого получается $2+2 \cdot 3^2 + 2^2=24$,
е ровно столько, сколько нужно.

\subsubsection{Неприводимые комплексные представления\\ и нормальные подгруппы простого индекса}

Пусть $H \nl G$, и $(G:H)=p$ простое число. Тогда имеем $G/H=\ha{aH}_p$, $a^p=b \in H$, и любой  элемент
группы $G$ записывается в виде $g=a^k \cdot h$, где $h \in H$. Пусть дано множество всех попарно неизоморфных
неприводимых представлений группы $H$: $\Mc := \hc{\rho_1 \sco \rho_s}$, и $\Mc \ni \rho\cln H \ra \GL(V)$.
Определим действие $G$ на этом множестве: $\rho_g(h):=\rho(g^{-1}hg)$ для $\fa g \in G$. Если элемент $g$
лежит в $H$, то $\rho_g(h)=\rho(g)^{-1}\rho(h)\rho(g)$ для $\fa h \in H$, е $\rho_g \sim \rho$. Значит,
$H \subseq \St(\rho)$. Поскольку $(G:H)$ простое число, то промежуточных подгрупп между $G$ и $H$ нет.
Поэтому возможно только 2 случая: $\St(\rho)=G$ и $\St(\rho)=H$.

В первом случае покажем, что можно продолжить представление $\rho$ до представления всей группы $G$. Для этого достаточно задать $\rho(a)$, и при этом
должны выполняться условия:

\pt{1} \; $\rho(a)^p=\rho(b)$;

\pt{2} \; $\rho(\ub{a^{-1}ha}_{\in H})=\rho(a)^{-1}\rho(h)\rho(a)$ для любого $h \in H$.

Они же будут и достаточными условиями, поскольку если $g=a^k h$, то $\rho(g)=\rho(a)^k\rho(h)$ и свойства
гомоморфизма легко проверяются. Поскольку $\St(\rho)=G$, то при действии любого элемента (в частности,
элемента $a$) $\rho$ остаётся на месте, е $\rho_a \sim \rho$. Поэтому $\rho_a(h)=\rho(a^{-1}h
a)=C^{-1}\rho(h)C$ для $\fa h \in H$. По индукции очевидным образом получаем, что
$\rho(a^{-k}ha^k)=C^{-k}\rho(h)C^{k}$. Вспоминая, что $a^p=b$, получаем, что при $k=p$ имеем $\rho(b^{-1}h
b)=C^{-p}\rho(h)C^{p}$. Так как $b \in H$, то $\rho(b^{-1}hb)=\rho(b)^{-1}\rho(h)\rho(b)$. Отсюда
$$C^p\rho(b)^{-1}\rho(h)\rho(b)C^{-p}=\rho(h) \; \Lra \; \bigl[C^p \rho(b)^{-1}\bigr]\rho(h)\bigl[C^p\rho(b)^{-1}\bigr]^{-1}=\rho(h).$$
Но представление $\rho$ неприводимо, и по лемме Шура $C^p\rho(b)^{-1}$ скалярная матрица.  Пусть
$C^p\rho(b)^{-1}=\la^{-1}E$, где $\la$ некоторое комплексное число. Тогда
$\rho(b)=(\sqrt[p]{\la}C)^p$. Вспомним теперь про равенство $\rho(a^{-1}ha)=C^{-1}\rho(h)C$. От умножения
матрицы на ненулевой скаляр ничего не изменится, поэтому можно считать, что $C^p=\rho(b)$. Положим
$\rho(a)=C$ и тем самым получим продолжение представления, но не единственное: пусть $\xi_0 \sco
\xi_{p-1}$ корни из 1 $p$-й степени. Тогда получаем $p$ представлений: $\rho_i(a)=\xi_i C$. Остаётся
показать, что они неэквивалентны. От противного: пусть $\rho_i \sim \rho_j$ при $i \neq j$. Тогда
$\rho_i(g)=C^{-1}\rho_j(g)C$ для всех $g \in G$. В частности, это должно быть верно для $g \in H$, но в этом
случае, очевидно, $\rho_i(g)=\rho(g)$. Значит, $\rho(g)=C^{-1}\rho(g)C$ для $\fa g \in H$, следовательно,
$C$ скалярная матрица. Значит, $\rho_i(g)=\rho_j(g)$ для любого $g \in G$, но это неверно, если
подставить $g=a$: справа и слева от знака равенства будут стоять различные корни из единицы. Случай 1
разобран.

Пусть теперь $\St(\rho)=H$. Тогда орбита $\rho$ состоит из элементов $\rho, \rho_a, \rho_{a^2} \sco
\rho_{a^{p-1}}$.  Возьмём $p$ экземпляров пространства $V$ и построим внешнюю прямую сумму $W:=V_0 \sop
V_{p-1}$. На каждом из $V_i$ рассмотрим представление $\rho_{a^i}$  подгруппы $H$. Определим представление
$\wt{\rho}\cln H \ra \GL(W)$:

$$\wt{\rho}(h):=\rbmat{\rho(h) & & & \text{{\LARGE $0$}} \\ & \rho_a(h)&  & \\ & & \ddots & \\ \text{{\LARGE $0$}} & & & \rho_{a^{p-1}}(h)}$$

Теперь зададим продолжение $\wt{\rho}$ на группу $G$, е $\rho(a)$:

$$\wt{\rho}(a):=\rbmat{
0 &   &  & \text{{\LARGE $0$}} &  \rho(b)\\
E & 0 &  & & 0\\
  & E & 0  \\
& & \ddots & 0 & \vdots \\
\text{{\LARGE $0$}} & & & E & 0}$$ Оно осуществляет циклический сдвиг подпространств: $V_0 \mapsto V_1, \;
V_1 \mapsto V_2, \ldots$  Проверка условий \pt{1} и \pt{2} выполняется «в лоб» умножением матриц. Докажем,
что получилось неприводимое представление $\wt{\rho}\cln G \ra \GL(W)$. Пусть $L$ инвариантное
подпространство в $W$. Тогда $L$  будет инвариантным и относительно $H$. Разложим его на неприводимые: $L=L_1
\sop L_k$ (относительно $H$). Докажем, что любое из  $L_i$ совпадает с одним из $V_j$. Рассмотрим проекции
$L_i$ на $V_j$ для всех $j$. Очевидно, они все не могут одновременно быть нулевыми. Тогда эти проекции будут
ненулевыми гомоморфизмами неприводимых представлений, а по лемме Шура они должны быть изоморфизмами. Таким
образом, $L=V_0 \sop V_k$. Если $k \neq p-1$, то $L$, очевидно, не будет инвариантным подпространством
(поскольку под действием $\wt{\rho}(a)$ переставляются \emph{все} $V_i$). Значит, $L=W$.

Покажем, что так получаются все неприводимые представления $G$. Пусть в множестве $\Mc$ представления
$\rho_1 \sco \rho_k$ имеют одноэлементные орбиты (случай 1) и их размерности соответственно $n_1 \sco n_k$, а
остальные $\rho_{k+1} \sco \rho_s$ имеют орбиты из $p$ элементов и размерности $n_{k+1} \sco n_s$
соответственно (случай 2). Имеем $\sum n_i^2\bw=|H|$. Для первых $k$ представлений получаем $p$ неэквивалентных
представлений $G$. Для всех остальных получаем по одному представлению группы $G$, каждое размерности $p \cdot n_i$. Всего
\eqn{pn_1^2  \spl  pn_k^2 + p n_{k+1}^2  \spl  p n_s^2 =p\sum n_i^2=p|H|=|G|.}

%==========================================
 \subsection{Характеры линейных представлений}
%==========================================

 \subsubsection{Понятие характера}

\begin{df}
Пусть $G$ конечная группа, $\rho\cln G \ra \GL_n(\Cbb)$ матричное представление.  \emph{Характером
представления} называется функция $\chi_\rho\cln G \ra \Cbb$, равная следу оператора $\rho(g)$, е
$\chi_\rho(g):=\tr \rho(g)$.
\end{df}

Перечислим основные свойства характеров:

\pt{1} Если $\rho_1 \sim \rho_2$, то $\chi_{\rho_1}(g)=\chi_{\rho_2}(g)$, так как след инвариант
линейного оператора.

\pt{2} Характеры постоянны на классах сопряжённости:
\eqn{\chi_\rho(h^{-1}gh)=\tr\rho(h^{-1}gh)=\tr\rho(h)^{-1}\rho(g)\rho(h)=\tr\rho(g)=\chi_\rho(g).}

\pt{3} $\chi_\rho(g^{-1})=\ol{\chi_\rho(g)}$. В самом деле, если $|G|=n$, то $\rho(g)^n=\Ec$. Тогда все собственные значения $\la_1 \sco \la_n$ оператора
$\rho(g)$ комплексные корни из 1 степени $n$. Поскольку $\rho(g^{-1})=\rho(g)^{-1}$, и $\la_i^{-1}=\ol{\la_i}$ для всех $i$, а след оператора есть сумма
собственных значений, то $\chi(g^{-1})=\ol{\chi(g)}$.

\pt{4} $\chi_{\rho_1 \oplus \rho_2}(g) = \chi_{\rho_1}(g) + \chi_{\rho_2}(g)$. Это очевидно,  поскольку
матрица прямой суммы представлений блочно-диагональная, и след такой матрицы равен сумме следов блоков.

Рассмотрим групповую алгебру $\Cbb G$. Пусть $r=\sumg a_g g \in \Cbb G$. Тогда $\rho(r)=\sumg a_g \rho(g)$ и
можно рассмотреть характер представления на всей групповой  алгебре: $\chi_\rho(r)=\sumg a_g \chi_\rho(g)$.
Он будет линейной функцией на $\Cbb G$.

\subsubsection{Основная теорема о характерах}

Рассмотрим пространство $\Xc$ всех комплекснозначных функций на группе $G$, постоянных на классах
сопряжённых элементов. Пусть этих классов $s$ штук и $|G|=n$. Имеем $\dim_\Cbb \Xc = s$. Введём на $\Xc$
эрмитово скалярное произведение: для $f_1, f_2\cln G \ra \Cbb$ положим
\eqn{(f_1, f_2):=\frac{1}{n}\sumg \ol{f_1(g)}f_2(g).}

\begin{theorem}
Пусть $\rho_1 \sco \rho_s$ неприводимые представления группы $G$. Тогда характеры $\chi_i:=\chi_{\rho_i}$ образуют ортонормированный базис в $\Xc$.
\end{theorem}
\begin{proof}
Разложим $\Cbb G$ в прямое произведение: $\Cbb G = \Mb_{n_1}(\Cbb) \st \Mb_{n_s}(\Cbb)$.  Выберем
нумерацию блоков так, что $\rho_i$ представление на минимальном левом идеале в $\Mb_{n_i}(\Cbb)$. Тогда
$\rho_i=0$ на $\Mb_{n_j}$ при $i \neq j$. Рассмотрим регулярное представление $\La\cln G \ra \GL(\Cbb G)$.
Напомним, что $\La(h)(\sumg a_g g)=\sumg a_g h g$. Найдём характер регулярного представления $\chi_r$. На
базисных элементах имеем $\La(h)(g)=hg$. Если $h=e$, то так как матрица элемента $e$ есть единичная
матрица размера $n \times n$, получаем, что $\chi_r(h)=n$. Если же $h \neq e$, то $hg \neq g$ (происходит
перестановка базисных векторов), и на диагонали будут нули, е $\chi_r(h)=0$.

Пусть $e= e_1 \sop e_s$, где $e_i$ единица $\Mb_{n_i}(\Cbb)$, и $e_i = \sumg a_g g$. Найдём  коэффициенты
$a_g$. Умножив слева на $g^{-1}$, преобразуем равенство к виду
\eqn{g^{-1}e_i=a_g + \sums{h \neq g} a_h g^{-1} h.}
Отсюда получаем $\chi_r(g^{-1}e_i) = n a_g + 0 \; \Ra \; a_g = \frac{1}{n} \chi_r(g^{-1}e_i)$. Но так как $\chi_r =
\suml{i=1}{s}n_i \chi_i$, то
\eqn{a_g = \frac{1}{n} \suml{j=1}{s} n_j \chi_j(g^{-1}e_i).}
Поскольку $g^{-1}e_i \in \Mb_{n_i}(\Cbb)$, то останется только одно слагаемое, е $a_g = \frac{n_i}{n}\chi_i(g^{-1}e_i)$. Имеем
\eqn{\chi_i(g^{-1})=\chi_i(g^{-1}e)=\chi_i\bigl(\suml{j=1}{s}g^{-1}e_j\bigr)=\chi_i(g^{-1}e_i),}
так как все слагаемые, кроме $i$-того, равны 0. Окончательно получаем $a_g = \frac{n_i}{n}\chi_i(g^{-1})$, а потому
\eqn{e_i = \frac{n_i}{n}\sumg \chi_i(g^{-1})g.}

Применим к последнему равенству характер $\chi_j$.  Если $i=j$, то
\eqn{\chi_i(e_i) = n_i = \frac{n_i}{n}\sumg \chi_i(g^{-1})\chi_i(g).}
Поскольку $\rho_i(e_i)$ единичная матрица со следом $n_i$, то $\frac{1}{n}\sumg \ol{\chi_i(g)}\chi_i(g)=1$, то есть
$(\chi_i, \chi_i) = 1$. Если же $i \neq j$, то очевидно, что $(\chi_i,
\chi_j)=0$, что и требовалось доказать.
\end{proof}

\begin{imp}
Любое представление определяется своим характером.
\end{imp}
\begin{proof}
Пусть $\rho = \suml{i=1}{s}k_i \rho_i$. Тогда $\chi_\rho = \suml{i=1}{s}k_i \chi_i$. Имеем $k_i = (\chi_i, \chi_\rho)$. Значит, $k_i$
однозначно определяются.
\end{proof}

\begin{imp}
Представление неприводимо $\Lra$ скалярный квадрат его характера равен 1.
\end{imp}
\begin{proof}
Очевидно: $(\chi_\rho, \chi_\rho) = \suml{i=1}{s}k_i^2=1$. Значит, все $k_i$, кроме одного, равны  0, что и
даёт неприводимость.
\end{proof}

\end{document}
