\documentclass[a4paper]{article}
\usepackage{dmvn}

\newcommand{\kph}{\Ker \ph}
\newcommand{\sumlg}{\suml{g \in G}{}}
\newcommand{\bignull}{\text{{\Large 0}}}
\newcommand{\bigast}{\text{{\Large $*$}}}
\newcommand{\ulmatrix}[2]{\rbmat{\tab{c|c}{$#1$&$#2$\\\hline0&1}}}

\begin{document}
\dmvntitle{Курс лекций по}{высшей алгебре}{Лектор\т Виктор Николаевич Латышев}
{II курс, 3 семестр, поток математиков}{Москва, 2004 г.}

\pagebreak
\tableofcontents
\pagebreak

\section*{Введение}
\subsection*{Предисловие}

Автор данного документа будет признателен, если ему сообщат о замеченных в документе опечатках и неточностях.
Документ не раз подвергался исправлениям, но мелкие ошибки всё ещё могут оставаться.
В данной версии проведена ещё одна правка, в основном с целью улучшения читаемости.

\medskip\dmvntrail

\subsection*{Список сокращений}

\begin{points}{-3}
\item Аббревиатуры: <<КПАГ>>=<<конечно порождённая абелева группа>>,
      <<САГ>>=<<свободная абелева группа>>, <<КГИ>>=<<кольцо главных идеалов>>.
\item Квадратными скобками мы будем обозначать коммутаторы элементов группы и наименьшее общее кратное
      целых чисел. Например, $[a,b] = aba^{-1}b^{-1}$ и $[6,4,5] = 60$.
      К сожалению, обозначения одинаковые, но из контекста, как правило, всегда ясно,
      о чём идёт речь. Для полноты картины заметим, что через $[a,b]$ может обозначаться векторное
      произведение и коммутатор векторных полей, но в нашем тексте этих вещей не встретится.
\item Символом $(a,b)$ мы будем обозначать наибольший общий делитель двух чисел или многочленов.
      Скалярное произведение, также часто обозначаемое круглыми скобками, в тексте не встретится.
\item Буквой $\Pg$ мы будем обозначать множество простых чисел.
\end{points}

\begin{thebibliography}{4}
    \setlength\itemsep{-3pt}
    \bibitem{lectures:latyshev} М.\,Н.\,Вельтищев. \emph{Конспекты лекций В.\,Н.\,Латышева по алгебре.} 2004.
    \bibitem{lectures:golod}    А.\,В.\,Домбровская. \emph{Конспекты лекций Е.\,С.\,Голода по алгебре.} 2004.
    \bibitem{kostrikin}         А.\,И.\,Кострикин. \emph{Введение в алгебру. Часть III: Основные структуры.} М.: ФизМатЛит, 2001.
    \bibitem{vinberg}           Э.\,Б.\,Винберг. \emph{Курс алгебры.} М.: Факториал\ч Пресс, 2002.
\end{thebibliography}

\pagebreak

\section{Теория групп}

\subsection{Группы}

\begin{df}
\emph{Группой} называется непустое множество $G$ c операцией $*\cln G \times G \to G$, которая удовлетворяет
аксиомам:

\pt{1} $(a*b)*c=a*(b*c)$\т ассоциативность операции.

\pt{2} $\exi e \in G\cln \fa a \in G$ имеем $e*a=a*e=a$\т существование нейтрального элемента.

\pt{3} $\fa a \in G \exi a^{-1}\cln a*a^{-1}=a^{-1}*a=e$\т существование обратного элемента.

Если $\fa a,b \in G$ имеет место свойство $a*b=b*a$, то группа называется \emph{коммутативной}
или \emph{абелевой}. Знак операции,
как и умножение, часто опускают, если используется мультипликативная терминология.

Нейтральный элемент группы называется \emph{единицей}.
\end{df}

Легко видеть, что единица группы единственна: пусть $e_1$ и $e_2$\т две единицы
группы. Тогда из свойств единицы следует, что $e_1 = e_1e_2 =
e_2$, значит, $e_1=e_2$. Обратный элемент также единствен: пусть
$a^{-1}a=e$ и $b^{-1}a=e$. Тогда домножим второе равенство справа
на $a^{-1}$, получим $b^{-1}aa^{-1}=a^{-1}$. Отсюда
$b^{-1}=a^{-1}$.

\begin{df}
\emph{Подгруппой} $H$ группы $G$ называется подмножество $H\subs G$,
которое относительно операции в $G$ само является группой.
\end{df}

Достаточными условиями для подгруппы является её замкнутость
относительно операции:

\pt{1} $a, b \in H \ra ab \in H$.

\pt{2} $e \in H$.

\pt{3} $\fa a \in H$ имеем $a^{-1} \in H$.

\begin{stm}[Эквивалентное определение подгруппы]
Непустое подмножество $H \subs G$ будет подгруппой в $G$ тогда и только тогда, когда
$a, b \in H \Ra a^{-1}b \in H$.
\end{stm}
\begin{proof}
В одну сторону это очевидно, докажем в обратную сторону. Пусть $\fa a, b \in H$ имеем $a^{-1}b \in H$. Рассмотрим
$a \in H$. Тогда $a^{-1}a = e \in H$. Тогда $\fa a \in H$ имеем $a^{-1}e = a^{-1} \in H$.
Пусть теперь $a, b \in H$, тогда $a^{-1} \in H$, значит, $\hr{a^{-1}}^{-1}b=ab \in H$, и мы видим,
что все свойства подгруппы выполнены.
\end{proof}

\begin{ex}

\pt{1} $\Ac_n \subs \Sc_n$\т знакопеременная группа\т подгруппа в группе подстановок.

\pt{2} $\SO_n \subs \GL_n(\R)$\т подгруппа ортогональных матриц с определителем $1$.
\end{ex}

\begin{df}
\emph{Порядок} элемента $g \in G$\т число $O(g) := \min \hc{n\cln g^n=e}$. По определению, элемент бесконечного
порядка\т элемент, не имеющий порядка.
\end{df}

В конечных группах все элементы имеют конечный порядок, ибо все степени одного элемента
$e, g, g^2, g^3, \dots$ не могут быть различными. Положим $n = O(g)$. Ясно, если $g^m=e$, то
$n \divs m$. В самом деле, пусть $m=nq+r, r < n$. Тогда $g^m=\hr{g^n}^q\cdot g^r=g^r=e$, что невозможно,
поскольку $r < n$. Значит, $r=0$.

\subsection{Гомоморфизмы групп}

\begin{df}
\emph{Гомоморфизм групп} $(G,*)$ и $(L,\circ)$\т отображение $\ph\cln G \ra L$, сохраняющее операцию: $\fa
a,b\bw\in G$ имеем $\ph(a*b)=\ph(a)\circ\ph(b)$.
\end{df}

Пусть $e$\т единица $G$, а $e'$\т единица в $L$. Покажем, что $\ph(e)=e'$. В самом деле,
$\ph(e)=\ph(ee)=\ph(e)\ph(e)$, значит,
$\ph(e)$\т нейтральный элемент в $L$. Покажем, что $\ph(a^{-1})= \ph(a)^{-1}$. В самом деле, рассмотрим
$e'\bw=\ph(e)\bw=\ph(aa^{-1})=\ph(a)\ph(a^{-1})$, \те $\ph(a^{-1})$\т обратный элемент по отношению к $\ph(a)$.

\begin{df}
Биективный гомоморфизм $\ph\cln G \ra L$ называется \emph{изоморфизмом} групп. При этом говорят, что группы
$G$ и $L$ \emph{изоморфны}, и пишут $G \cong L$. Изоморфизм группы на себя называется \emph{автоморфизмом}.
\end{df}

\begin{df}
\emph{Ядро} гомоморфизма $\ph$\т множество $\kph := \hc{g \in G\cln \ph(g)=e'}$. \emph{Образ} гомоморфизма\т
множество $\Img \ph := \hc{x \in L\cln \exi g \in G\cln \ph(g)=x}$.
\end{df}

\begin{stm}
$\kph$ является подгруппой в $G$, $\Img \ph$ является подгруппой в $L$.
\end{stm}
\begin{proof}
Пусть $g, h\in \kph$. Тогда $\ph(gh)=\ph(g)\ph(h)=e'e'=e' \Ra gh \in \kph$. Очевидно, $e \in \kph$. Кроме
того, если $g \in \kph$, то $\ph(g^{-1})=\ph(g)^{-1}=e'^{-1}=e' \Ra g^{-1} \in \kph$.

Пусть $x,y\in \Img \ph$, тогда $\exi g,h\in G\cln x=\ph(g), y=\ph(h)$. Тогда
$x^{-1}y=\ph(g)^{-1}\ph(h)=\ph(g^{-1}h)\in \Img\ph$. Используя эквивалентное определение подгруппы, получаем
требуемое утверждение.
\end{proof}

\begin{df}
Инъективный гомоморфизм называется \emph{вложением} группы $G$ в группу $L$ и обозначается $\ph\cln G \inj
L$. В этом случае $G \cong \Img \ph$.
\end{df}

Заметим, что гомоморфизм инъективен $\Lra \kph = \hc{e}$. В самом деле, пусть $\kph =\hc{e}$. Тогда
$\ph(a)=\ph(b) \bw\Lra \ph(a^{-1}b)=e' \Lra a^{-1}b=e \Lra a=b$. Обратно, пусть гомоморфизм инъективен. Тогда
$a \in \kph \Lra \ph(a)=e'$. Но $\ph(e)=e'$, и в силу инъективности $a=e$. Значит, $\kph$ содержит только $e$.

Пусть $\ph\cln G \ra L$\т гомоморфизм, и $H$\т подгруппа в $G$. Покажем, что $K = \ph(H)$\т подгруппа в~$L$.
Рассмотрим $x, y \in K$. Тогда $\exi a, b \in H\cln x = \ph(a), y = \ph(b)$. Тогда $xy
=\ph(a)\ph(b)=\ph(ab)$, значит, $xy \in K$. Поскольку $e \in H$, а $\ph(e) =e'$, получаем, что $e' \in K$.
Покажем, что $x^{-1} \in K$. Действительно, $x^{-1}=\ph(a^{-1}) \in K$, поскольку $a^{-1} \in H$.

\begin{df}
Сюръективный гомоморфизм называется \emph{эпиморфизмом}.
\end{df}

Пусть $\ph\cln G \ra L$\т эпиморфизм, и $K$\т подгруппа в $L$. Тогда $H = \ph^{-1}(K)$\т подгруппа в $G$. В
самом деле, пусть $x,y \in H$, тогда $\ph(x), \ph(y) \in K$. Рассмотрим $\ph(x^{-1}y)=\ph(x)^{-1}\ph(y) \in
K$. Следовательно, $x^{-1}y \in H$, значит, $H$\т подгруппа.

\begin{ex}

\pt{1} Пусть $n \in \N$. Рассмотрим $\ph\cln (\Z, +) \ra \Cbb^*$, определённый по правилу $\ph\cln k \mapsto
\ep_k = \cos\frac{2\pi k}{n} + i\sin \frac{2\pi k}{n}$. Тогда $\ph$\т гомоморфизм. Очевидно, $\kph = n\Z$.

\pt{2} \emph{Левый сдвиг}. Пусть $G$\т группа. Фиксируем $g \in G$, и рассмотрим отображение
$L_g\cln G \ra G$, определённое по правилу $L_g\cln x \mapsto gx$. Покажем, что $L_g \in \Sc_G$, где
$\Sc_G$\т группа подстановок множества $G$. Действительно, оно сюръективно:
$L_g(g^{-1}x)=g(g^{-1}x)=x$. Кроме того, $L_g(x)=L_g(y) \Lra gx=gy \Lra x=y$. Значит, это инъекция.
Теперь рассмотрим множество всех левых сдвигов $L_G := \hc{L_g\cln g \in G}$. Это подгруппа в $\Sc_G$,
поскольку произведение сдвигов на элементы $g_1$ и $g_2$ есть левый сдвиг на элемент $g_1g_2$,
нейтральным элементом в $L_G$ будет левый сдвиг на $e \in G$, обратным к сдвигу на $g$\т
сдвиг на $g^{-1}$. Таким образом, $L_G \subs \Sc_G$.
\end{ex}

\begin{theorem}[Кэли]
Пусть $G$\т группа. Тогда $\exi$ инъективный гомоморфизм $G \inj \Sc_G$.
\end{theorem}
\begin{proof}
Рассмотрим $\ph\cln G \ra \Sc_G$ по правилу $\ph\cln g \mapsto L_g$. Тогда $\ph$\т искомый гомоморфизм, ибо
$\ph(g_1g_2) \bw= L_{g_1g_2} \bw= L_{g_1}L_{g_2}=\ph(g_1)\ph(g_2)$, а его инъективность очевидна.
\end{proof}

\subsection{Системы порождающих}

Очевидно, что если $A \subs G$ и $B \subs G$\т подгруппы, то $A \cap B$ тоже будет подгруппой $G$.

\begin{df}
Фиксируем в группе $G$ несколько элементов $A=\hc{a_1\sco a_m,\dots}$. Подгруппой $H$,
\emph{порождённой} системой $A$, называется пересечение всех подгрупп $G$, содержащих $A$.
\end{df}

Изучим строение $H$. Ясно, что $H$ содержит все элементы вида $a_{i_1}^{\ep_1}\sd a_{i_r}^{\ep_r}$,
где $\ep_j = \pm 1$, а $r \in \N$, причём среди $i_1\sco i_r$ могут быть одинаковые числа. Такие
произведения образуют подгруппу в $G$, и она является одной из подгрупп $G$, содержащих $A$.
Значит, элементов, не представимых в таком виде, в $H$ быть не может, откуда
$H= \hc{a_{i_1}^{\ep_1}\sd a_{i_r}^{\ep_r}}$. Если $H=G$, то говорят, что $A$ порождает $G$.

\subsection{Циклические группы}

\begin{df}
Пусть $a \in G$ порождает группу $G$. Тогда $G$\т \emph{циклическая группа}, обозначаемая $G = \ha{a}$.
\end{df}

\begin{theorem}
Всякая подгруппа циклической группы является циклической группой.
\end{theorem}
\begin{proof}
В самом деле, пусть $G= \ha{a}$, и $H \subs G$. Если $H=\hc{e}$, то доказывать нечего, ибо
$H=\ha{e}$. Пусть $a^m \in H$, тогда $a^{-m} \in H$. Положим $k=\min \hc{m \in \N\cln a^m \in H}$.
Покажем, что $H = \ha{a^k}$. В самом деле, пусть $a^m \in H$, тогда, поделив с остатком, имеем
$m = kq+r, r < k$. Тогда $a^m= \hr{a^k}^q a^r$, следовательно, $a^r=a^m \hr{a^k}^{-q} \in H$,
поскольку каждый множитель принадлежит $H$. Но поскольку мы выбрали $k$ минимальным, $r=0$, и
любой элемент из $H$ есть некоторая степень элемента $a^k$.
\end{proof}

\begin{theorem}
Все бесконечные циклические группы изоморфны $(\Z, +)$. Все конечные циклические группы изоморфны $U_n$.
\end{theorem}
\begin{proof}
Пусть $G = \ha{a}$. Возможны 2 случая.

\pt{1} $O(a)=\infty$. Тогда $G = \hc{\dots, a^{-2}, a^{-1}, e, a^1, a^2, \dots}$. Все степени
элемента $a$ здесь различны, поэтому $|G| \bw= \infty$. Рассмотрим $\ph\cln G \ra \Z$ по правилу
$a^m \mapsto m$. Биективность отображения очевидна. Докажем сохранение операции:
$\ph(a^ka^l)=\ph(a^{k+l})=k+l = \ph(a^k) + \ph(a^l)$. Значит, $\ph$\т изоморфизм.

\pt{2} $O(a)=n$. Тогда $G = \hc{e, a^1, a^2\sco a^{n-1}}$. В самом деле, рассмотрим
$a^m=a^{nq+r}=e^q \cdot a^r = a^r$. Теперь можно построить изоморфизм $\ph\cln G \ra U_n$ по
правилу $a^m \mapsto \ep^m$, где $\ep = \cos\frac{2\pi}{n} + i \sin\frac{2\pi}{n}$. Проверим
корректность: $a^m=a^l \Ra \ep^m=\ep^l$. Действительно, имеем $a^{m-l}=e$, значит, $n \divs (m-l)$,
отсюда $\ep^{m-l}=1$, отсюда $\ep^m=\ep^l$.
Сюръективность очевидна. Покажем инъективность. Рассмотрим
$\ph(a^m)=\ph(a^l) \Lra \ep^m=\ep^l \Lra n \divs (m-l) \bw\Lra a^{m-l}=e \Lra a^m=a^l$.
Сохранение операции очевидно. Следовательно, $\ph$\т изоморфизм.
\end{proof}

\subsection{Разложение группы по подгруппе, теорема Лагранжа и её следствия}

\begin{df}
Пусть $H$\т подгруппа в $G$. Фиксируем $a \in G$, тогда $aH=\hc{ah\cln h \in H}$\т \emph{левый
смежный класс элемента~$a$ по подгруппе $H$}.
\end{df}

Аналогично определяется \emph{правый смежный класс} $Ha$. Рассмотрим отображение $\ph\cln H \ra aH$ по
правилу $\ph\cln h \mapsto ah$. Сюръективность очевидна, а инъективность следует из того, что
если $ah_1=ah_2$, то, домножая слева на $a^{-1}$, получаем $h_1=h_2$. Следовательно, $|H| =|aH|$.

\begin{theorem}[Лагранжа]
Пусть $j$\т количество левых смежных классов по $H$. Тогда $|G|=|H| \cdot j$.
\end{theorem}
\begin{proof}
Различные смежные классы не пересекаются. В самом деле, рассмотрим $aH$ и $bH$.  Пусть $g \in aH \cap bH$,
значит, $g=ah_1=bh_2$, отсюда $a = bh_2h_1^{-1}$, значит, $a \in bH$, поскольку $h_2h_1^{-1} \in H$. Но тогда
$aH \subs bH$, поскольку $\fa h \in H$ имеем $bh_2h_1^{-1}h \in bH$. Из соображений симметрии, $bH \subs aH$.
Значит, $aH=bH$. Значит, смежные классы либо не пересекаются, либо совпадают. Значит, они образуют разбиение
группы $G$ на $j$ частей. Поскольку все смежные классы равномощны $H$, получаем, что $|G|=|H|\cdot j$.
\end{proof}

\begin{note}
Точно также можно доказать, что $|G| = |H| \cdot j$, где $j$\т количество правых  смежных классов по $H$.
Отсюда следует, что число правых и число левых смежных классов совпадает. Оно называется индексом подгруппы
$H$ и обозначается $(G:H)$.
\end{note}

\begin{imp}
$|H| \divs |G|$, $(G:H) \divs |G|$, $O(a) \divs |G|$, $\fa a \in G$.
\end{imp}
\begin{proof}
Первые два утверждения следуют из теоремы Лагранжа, а третье следует из того, что  можно рассмотреть
циклическую группу $\ha{a}$, для которой имеем $|\ha{a}|=O(a)$.
\end{proof}

\begin{imp}
Если $|G|=p$, где $p \in \Pg$, то $G$\т циклическая группа.
\end{imp}
\begin{proof}
$\exi a \in G\cln O(a)=p$, поскольку порядки всех элементов не могут быть равны единице. Тогда $\ha{a}=G$.
\end{proof}

\subsection{Разложение по подгруппе. Конгруэнции. Нормальные подгруппы. Факторгруппы}

Пусть $H$\т подгруппа в $G$. Рассмотрим отношение $a \sim b \Lra a^{-1}b \in H, a, b \in G$.  Покажем, что
$\sim$ задаёт отношение эквивалентности на $G$. Действительно, $a \sim a$, ибо $a^{-1}a = e \in H$.
Симметричность: $b \sim a \Lra b^{-1}a \bw\in H \Lra \hr{b^{-1}a}^{-1} \in H \Lra a^{-1}b \in H \Lra a \sim b$.
Транзитивность: $a \sim b, b \sim c \Lra a^{-1}b \in H, b^{-1}c \in H \Ra (a^{-1}b)(b^{-1}c) \bw\in H \Ra
a^{-1}c \in H \Ra a \sim c$.

Заметим, что $aH=bH \Lra a^{-1}b \in H$. В самом деле, $b=ah \Lra a^{-1}b=h\in H$.  Следовательно, классы
эквивалентности, задаваемые $\sim$, есть в точности смежные классы: $[a]=aH$.

\begin{df}
\emph{Конгруэнция на множестве с операцией}. Рассмотрим $(G, *)$, и пусть $\sim$  задаёт некоторое отношение
эквивалентности на $G$. Тогда назовём $\sim$ конгруэнцией, если она согласована с операцией в $G$:
$a \bw\sim a', b \bw\sim b' \Ra a*b \bw\sim a'* b'$. Это означает, что можно выбрать других представителей из классов
эквивалентности, и это не повлияет на результат умножения.
\end{df}

\begin{df}
Если $\sim$\т конгруэнция, то имеет смысл рассмотреть \emph{фактормножество} $\br{\fact{G}{\sim}, *}$ классов
эквивалентности с операцией $*$. Поскольку $\sim$\т конгруэнция, имеем  $[a]*[b]=[a*b]$, и тогда $*$
называется \emph{естественной операцией}. Если $G$\т группа, то для фактормножества выполняются все аксиомы
группы: ассоциативность: $\br{[a]*[b]}*[c]=[a*b]*[c]=[a*b*c]=[a]*[b*c]=[a]*\br{[b]*[c]}$, существование
нейтрального элемента: $[a]*[e]=[e]*[a]=[a]$ и существование обратного элемента: $[a]*[a^{-1}]=[e]$,
следовательно, $[a^{-1}]=[a]^{-1}$. Итак, $\fact{G}{\sim}$\т \emph{факторгруппа}.
\end{df}

Заметим, что разбиение на смежные классы по подгруппе не обязательно задаёт конгруэнцию. Определим класс
подгрупп, факторизация по которым её задаёт.

\begin{df}
\emph{Нормальная подгруппа} (инвариантная подгруппа, нормальный
делитель)\т подгруппа $H$ группы $G$, такая, что $\fa a \in G$
имеем $aH=Ha$. Обозначение: $H \lhd G$.
\end{df}

Заметим, что в абелевой группе любая подгруппа нормальна.
Нормальные подгруппы существуют в любой группе:
$G \lhd G, \hc{e} \lhd G$.

\begin{df}
Группа называется \emph{простой}, если в ней нет нормальных подгрупп, кроме тривиальных.
\end{df}

\begin{df}
Элементы $x,y \in G$ называются \emph{сопряжёнными}, если $\exi g \in G\cln y=gxg^{-1}$. Говорят, что
$x$ сопряжён с $y$ через $g$.
\end{df}

Покажем, что сопряжение задаёт отношение эквивалентности: $x\sim y \Lra y=gxg^{-1}$ для некоторого $g \in G$.
В самом деле, имеет место рефлексивность: $x \sim x$, ибо $x=exe^{-1}$. Докажем симметричность: $x \sim y \Ra
y \sim x$.  Действительно, если $y=gxg^{-1}$, то $x=g^{-1}yg=\hr{g^{-1}} y \hr{g^{-1}}^{-1}$. Транзитивность:
пусть $x \sim y, y \sim z$. Тогда $y=g_1xg_1^{-1}, z=g_2yg_2^{-1}$. Отсюда $z=(g_2g_1)x(g_2g_1)^{-1}$,
значит, $x \sim z$. Таким образом, группу $G$ можно разбить на \emph{классы сопряжённых элементов}:
$G = \bigcup K_x$, $K_x := \hc{y \in G\cln y \sim x}$.

Рассмотрим \emph{группу автоморфизмов $\Aut G \subs \Sc_G$ группы $G$}. Фиксируем  $g \in G$. Рассмотрим
отображение $I_g\cln G\ra G$ по правилу $I_g\cln x \mapsto gxg^{-1}$. Покажем, что $I_g \in \Aut G$. В самом деле,
$I_g$ сюръективно: на $x$ отображается элемент $g^{-1}xg$: $I_g(g^{-1}xg)=gg^{-1}xgg^{-1}=x$. Инъективность
очевидна: $gxg^{-1}=gyg^{-1} \Lra x=y$. Значит, это как минимум биекция. Покажем сохранение операции:
$I_g(xy)=g(xy)g^{-1}=(gxg^{-1})(gyg^{-1})=I_g(x)I_g(y)$. Значит, это автоморфизм.

\begin{df}
Автоморфизмы вида $I_g$ образуют \emph{подгруппу внутренних автоморфизмов}  $\Int G \subs \Aut G$. Покажем,
что это действительно подгруппа. Имеем $\id = I_e \in \Int G$, кроме того, $(I_{g_1}\circ
I_{g_2})(x) \bw= g_1g_2xg_2^{-1}g_1^{-1} \bw= (g_1g_2)x(g_1g_2)^{-1}=I_{g_1g_2}(x)$. Отсюда  видно, что обратным к
$I_g$ является автоморфизм $I_{g^{-1}} \in \Int G$. Итак, $\Int G$\т подгруппа.
\end{df}

Заметим, что $H \lhd G \Lra gHg^{-1}=H$. Получаем эквивалентное определение:  $H\lhd G$ тогда и только тогда,
когда она инвариантна относительно $\Int G$. Следовательно, $H$ состоит из нескольких классов сопряжённых
элементов, ибо вместе с любым элементом класса сопряжённости она содержит и любой элемент этого же класса.

\begin{theorem}
Все конгруэнции на $G$ являются эквивалентностями, связанными с разложениями по нормальным подгруппам.
\end{theorem}
\begin{proof}
\pt{1} Пусть $H \lhd G$, и по определению $a \sim b \Lra a^{-1}b \in H$. Покажем,  что это конгруэнция. Пусть
$a \sim a', b \sim b'$. Покажем, что $ab \sim a'b'$. Имеем $a^{-1}a' \in H, b^{-1}b' \in H$. Тогда
$(ab)^{-1}(a'b')=b^{-1}a^{-1}a'b'= \br{b^{-1}(a^{-1}a')b}b^{-1}b'$. Но $a^{-1}a' \in H$ по условию, тогда в
силу нормальности $H$ имеем $b^{-1}(a^{-1}a')b \in H$. Но и $b^{-1}b' \in H$ по условию, значит, произведение
этих множителей также лежит в $H$. Но это и означает, что $ab \sim a'b'$. Тогда $\fact{G}{\sim}=\hc{aH, a \in
G}$ с единицей $eH=H$, $(aH)^{-1} = a^{-1}H$, $(aH)(bH)=(ab)H$.

\pt{2} Обратно, пусть $\sim$ задаёт конгруэнцию. Покажем, что $\exi H \lhd G$,  для которой $a \sim b \Lra
a^{-1}b \in H$. Рассмотрим $H := \hc{a \in G\cln a \sim e}$. Покажем, что $H$\т искомая нормальная подгруппа в
$G$. Поскольку $e \sim e$, имеем $e \in H$. Пусть $a,b\in H, a \sim e, b \sim e$. По свойству конгруэнции,
$ab \sim ee=e$, \те $ab \in H$. Далее, поскольку $a \sim e, a^{-1} \sim a^{-1}$, получаем $aa^{-1}\sim
ea^{-1}$, откуда $e \sim a^{-1} \Lra a^{-1} \in H$. Значит, $H$\т подгруппа в $G$. Покажем нормальность:
пусть $h \in H \Lra h \sim e$, кроме того, $g \sim g, g^{-1} \sim g^{-1}$. Отсюда $ghg^{-1} \sim geg^{-1}=e
\Ra ghg^{-1} \in H$.
\end{proof}

\begin{note}
Когда речь идёт о конгруэнции, всегда можно иметь ввиду соответствующую  нормальную подгруппу. Поэтому
говорят обычно не о фактормножестве по конгруэнции, а о \emph{факторгруппе} $\fact{G}{H}$, где $H \lhd G$.
\end{note}

\begin{note}
Заметим, что отношение нормальности не является \emph{транзитивным}, \те если $A \lhd B$, $B \lhd C$, то $A$
не обязательно нормальна в $C$. Пример: $C = \Ac_4$, $B = \Vc_4$, $A= \hc{\id, (12)(34)}$.
\end{note}

\begin{theorem}
Пусть $H\subs G$\т подгруппа индекса 2. Тогда $H \lhd G$.
\end{theorem}
\begin{proof}
Рассмотрим левые смежные классы по $H$. Их будет два: $eH$ и $g_lH$, где $g_l \notin H$.
Теперь рассмотрим правые смежные классы: $He$ и $Hg_r$, где $g_r \notin H$. Ясно,
что $eH=H=He$, но тогда $g_lH=Hg_r$, поскольку смежные классы образуют разбиение $G$. Но
если левые и правые смежные классы совпадают, то $H \lhd G$.
\end{proof}

\subsection{Теорема о гомоморфизме}

\begin{theorem}
Ядра гомоморфизмов, и только они, являются нормальными подгруппами.
\end{theorem}
\begin{proof}
\pt{1} Пусть $\ph\cln G\ra L$\т гомоморфизм. Покажем, что $H=\kph \lhd G$.  Пусть $h \in H, g \in G$. Рассмотрим
$\ph(ghg^{-1})=\ph(g)\ph(h)\ph(g^{-1})= \ph(g)e'\ph(g)^{-1}=e'$. Значит, $ghg^{-1} \in H$, но это и означает,
что $H \lhd G$.

\pt{2} Пусть $H \lhd G$. Рассмотрим естественный эпиморфизм  $\pi\cln G \ra \fact{G}{H}$ по правилу
$\pi\cln a \mapsto aH$. Тогда $a \bw\in \Ker \pi \Lra \pi(a)=aH=H \Lra a \in H \Ra \Ker \pi = H$.
\end{proof}

\begin{theorem}[О гомоморфизмах групп]
Пусть $\ph\cln G \ra L$\т эпиморфизм, $H := \kph$, $\pi\cln G \ra \fact{G}{H}$\т  естественный эпиморфизм.
Тогда существует изоморфизм $\psi\cln L \ra \fact{G}{H}$, для которого диаграмма $(\ref{group.hom.th})$
коммутативна, то есть $\pi = \psi \circ \ph$. \eqn{\label{group.hom.th}\Vtriangle[G`L`\fact{G}{H};\ph`\pi`\psi]}
Иными словами, $\fact{G}{\kph} \cong \Img \ph =L$.
\end{theorem}
\begin{proof}
Пусть $x \in L$. Тогда $\exi a \in G\cln x=\ph(a)$ в силу того, что $\Img \ph = L$.  Рассмотрим
$\psi\cln L \ra \fact{G}{H}$ по правилу $\psi\cln x \mapsto aH$, где $\ph(a)=x$. Покажем, что $\psi$\т искомый изоморфизм. В
самом деле, проверим корректность. Пусть $x=\ph(a)=\ph(b), a,b \in G$. Тогда $\ph(a)=\ph(b) \Lra
\ph(a^{-1}b)=e_L$, где $e_L$\т единица в $L$. Отсюда $a^{-1}b \in \kph = H$, значит, $aH=bH$. Значит, $\psi$
переведёт $\ph(a)$ и $\ph(b)$ в один и тот же смежный класс. Тем самым корректность доказана.

Покажем, что $\psi$\т биекция. Сюръективность сразу следует из определения  $\psi$: достаточно взять такой
$x$, что $x=\ph(a)$. Проверим инъективность: $\psi\br{\ph(a)} =\psi\br{\ph(b)} \Lra aH=bH \Lra a^{-1}b \in H
\Lra \ph(a^{-1}b) = e_L \Lra \ph(a)=\ph(b)$.

Покажем, что $\psi$ сохраняет операцию: $\psi\br{\ph(a)\ph(b)}=\psi\br{\ph(ab)}=(ab)H=aH \cdot
bH=\psi\br{\ph(a)}\cdot\psi\br{\ph(b)}$. Здесь свойство $(ab)H =aH \cdot bH$ следует из того, что $H =\kph$,
и, следовательно, задаёт конгруэнцию.

Проверим коммутативность. Пусть $a \in G$. Тогда $\pi(a)=aH$. Но $(\psi \circ \ph)(a)=aH$ по определению
$\psi$.
\end{proof}

\subsection{Теорема о соответствии групп при эпиморфизме}

Говоря о гомоморфизмах, мы уже показали, что прообраз подгруппы при эпиморфизме есть подгруппа.

\begin{theorem}[О соответствии]
Пусть $\ph\cln G \ra L$\т эпиморфизм групп. Положим $H := \kph \lhd G$.  Назовём $A \subs G$ выделенной
подгруппой, если $H \subs A$. Тогда сопоставление $\theta$ выделенной подгруппе $A \subs G$ её образа
$\ph(A)$ определяет биекцию между выделенными подгруппами в $G$ и всеми подгруппами $L$. При этом
соответствующие подгруппы одновременно нормальны и факторгруппы по ним изоморфны.
\end{theorem}
\begin{proof}
\pt{1} Покажем, что $\theta$ биективно. Сюръективность очевидна, ибо для  любой подгруппы $U \subs L$ имеем
$e_L \in U$, следовательно, $H \subs \ph^{-1}(U) = A$, откуда заключаем, что $A$\т выделенная подгруппа.
Докажем инъективность. Пусть $A, B \subs G$\т выделенные подгруппы, и $\ph(A)=\ph(B)$. Покажем, что $A=B$.
Рассмотрим произвольное $a \in A$. Тогда в силу совпадения образов, $\exi b \in B\cln \ph(a)=\ph(b)$. Отсюда
$\ph(b^{-1}a)=e_L \Lra b^{-1}a \in \kph =H$, значит, $a=bh$, но поскольку $H \subs B$, получаем $bh \in B$,
\те $a \in B$. Следовательно, $A \subs B$. По симметричным соображениям $B \subs A$. Значит, $A=B$.

\pt{2} Пусть $A$\т выделенная подгруппа в $G$. Покажем, что  $A \lhd G \Lra \ph(A) \lhd L$. В самом деле,
пусть $A \lhd G$. Тогда $\fa g \in G$ имеем $gAg^{-1} \in A$. Применим $\ph$ к этому тождеству, получим
$\ph(g)\ph(A)\ph(g)^{-1}=\ph(A)$. Но поскольку $\ph$\т эпиморфизм, то когда $g$ пробегает $G$, $\ph(g)$
пробегает всю $L$. Отсюда следует нормальность $\ph(A)$ в $L$. Обратно, пусть $M \lhd L$, предположим, что
$A := \ph^{-1}(M)$ не является нормальной в $G$. Значит, $\exi g \in G,\, x \in A\cln gxg^{-1} \notin A$. Тогда
$\ph(gxg^{-1}) \notin M$, следовательно, $\ph(g)\ph(x)\ph(g)^{-1} \notin M$. Но это уже противоречит тому,
что $M \lhd L$, поскольку $\ph(x) \in M$. Тем самым одновременная нормальность доказана.

\pt{3} Докажем изоморфность факторгрупп, соответствующих нормальным  подгруппам. Пусть $A \lhd G$\т
выделенная подгруппа, и $U := \ph(A)$. Рассмотрим отображение $\psi\cln G \ra \fact{L}{U}$, определённое по
правилу $g \mapsto \ph(g)U$. Докажем, что это эпиморфизм. Сюръективность очевидна: когда $g$ бегает по $G$,
$\ph(g)$ бегает по всей $L$, значит, любой смежный класс накроется. Проверим сохранение операции:
$\psi(g_1g_2)=\ph(g_1g_2)U=\br{\ph(g_1)\ph(g_2)} U \bw= \ph(g_1)U \cdot \ph(g_2)U \bw= \psi(g_1) \cdot \psi(g_2)$ по
свойствам естественной операции. Теперь найдём ядро: $g \in \Ker \psi \Lra \psi(g) \bw= \ph(g)U=U \Lra \ph(g)
\in U \Lra g \in A$. Отсюда следует, что $A = \Ker \psi$. По теореме о гомоморфизме получаем $\fact{G}{A}
\cong \fact{L}{U}$.
\end{proof}

\begin{theorem}
Пусть $L \lhd K \lhd G$ и $L \lhd G$. Тогда $\fact{G/L}{K/L} \cong \fact{G}{K}$.
\end{theorem}
\begin{proof}
Рассмотрим естественный эпиморфизм $\ph\cln G \ra \fact{G}{L}$. Имеем  $\kph = L$. Тогда $\ph(K) = \fact{K}{L}$.
Применим теорему о соответствии к $\ph$, тогда получим $K \overset{\theta}{\longleftrightarrow} \fact{K}{L}$.
Отсюда $\fact{K}{L} \lhd \fact{G}{L}$. По теореме о соответствии, факторгруппы по соответствующим нормальным
подгруппам изоморфны, поэтому $\fact{G}{K} \cong \fact{G/L}{K/L}$.
\end{proof}

\section{Поля и кольца}

\subsection{Кольца. Гомоморфизмы колец. Факторкольца}

\begin{df}
\emph{Кольцом} называется непустое множество $R$ с двумя операциями,  сложением и умножением:
$(R, +, \cdot)$. При этом по сложению кольцо\т абелева группа, по умножению выполнена левая и правая дистрибутивность:
$a(b+c)=ab+ac, (a+b)c=ac+bc$. \emph{Ассоциативное} кольцо\т кольцо с ассоциативным умножением,
\emph{коммутативное} кольцо\т кольцо с коммутативным умножением, \emph{кольцо с единицей}\т кольцо, в котором
есть нейтральный по умножению элемент.
\end{df}

\begin{df}
\emph{Гомоморфизм} колец $R$ и $Q$\т отображение $\ph\cln R \ra Q$, сохраняющее  обе операции. Поскольку это, в
частности, гомоморфизмы их аддитивных групп, имеем $\ph(0)=0$, $\ph(-a)=-\ph(a)$. Гомоморфизм не обязан
сохранять единицу кольца, даже если она есть.
\end{df}

\begin{df}
\emph{Конгруэнция} кольца\т отношение
эквивалентности $\sim$, согласованное с операциями: $a \bw\sim b, a' \bw\sim b' \Ra a+a' \bw\sim b+b', aa' \bw\sim bb'$.
\end{df}

\emph{Естественные операции} в кольцах обладают свойствами:

\pt{1} $[a]+[b]=[a+b]$,

\pt{2} $[a][b]=[ab]$,

\pt{3} $[c]\br{[a]+[b]}=[c][a+b]=[c(a+b)]=[ca+cb]=[ca]+[cb]=[c][a]+[c][b]$,

\pt{4} $\br{[a]+[b]}[c]=[a][c]+[b][c]$ (доказывается аналогично).

\begin{df}
\emph{Идеалом} кольца $R$ называется подгруппа $(I,+) \subs (R,+)$,  выдерживающая левое и правое умножение
на элементы кольца: $\fa a \in R$ имеем $aI \subs I, Ia \subs I$. Если $I$ выдерживает только левое
умножение, он называется \emph{левым} идеалом. Аналогично определяется \emph{правый} идеал.
\end{df}

В кольцах идеалы играют роль нормальных подгрупп, и обозначение такое же:  $I \lhd R$. Поскольку кольцо по
сложению является абелевой группой, можно определить эквивалентность $a\sim b \Lra a - b \in I$. Тогда это
будет конгруэнция относительно сложения. Покажем, что это будет конгруэнцией и по умножению. Пусть $a \sim
a', b \sim b'$. Рассмотрим $ab-a'b'=(ab-ab')+(ab'-a'b')=a(b-b')+(a-a')b' \in I$, поскольку $a-a' \in I, b-b'
\in I$. Но $ab-a'b' \in I \Lra ab \sim a'b'$.

\begin{df}
Определим теперь \emph{факторкольцо} по идеалу $I$: это факторгруппа  $\fact{A}{I}$ c операцией умножения
$(a+I)(b+I)= ab + I$.
\end{df}

\begin{df}
\emph{Ядром} гомоморфизма колец $\ph\cln R \ra Q$ называется множество $\kph = \hc{x \in R\cln \ph(x)=0}$.
\end{df}

\begin{theorem}
Ядра гомоморфизмов, и только они, являются идеалами кольца.
\end{theorem}
\begin{proof}
\pt{1} Пусть $\ph\cln R \ra Q$\т гомоморфизм колец, обозначим $I := \kph$.  Пусть $x \in I, a \in R$. Проверим,
что $I$ выдерживает умножение: $\ph(ax)=\ph(a)\ph(x)=\ph(a)\cdot 0=0$, значит, $ax \in I$. Аналогично, $xa
\in I$. Кроме того, из теоремы о ядре гомоморфизма групп следует, что $(I,+) \subs (R, +)$. Значит, $I \lhd
R$.

\pt{2} Обратно, пусть $I \lhd R$. Рассмотрим естественный эпиморфизм  $\pi\cln R \ra \fact{R}{I}$, тогда
$\Ker \pi = I$.
\end{proof}

\begin{theorem}[О гомоморфизмах колец]
Пусть $\ph\cln A \ra B$\т эпиморфизм, $I := \kph$, $\pi\cln A \ra \fact{A}{I}$\т  естественный эпиморфизм.
Тогда существует изоморфизм $\psi\cln B \ra \fact{A}{I}$, для которого диаграмма $(\ref{ring.hom.th})$
коммутативна, то есть $\pi = \psi \circ \ph$. \eqn{\label{ring.hom.th}\Vtriangle[A`B`\fact{A}{I};\ph`\pi`\psi]}
\end{theorem}
\begin{proof}
Поскольку все участвующие в теореме кольца являются абелевыми группами, для  аддитивных групп существование
$\psi$ доказано. Покажем, что $\psi$ будет изоморфизмом колец. Нам надо проверить только сохранение
умножения. Рассмотрим $x,y \in A$, тогда $\ph(x), \ph(y) \in B$. Рассмотрим
$\psi\br{\ph(x)\ph(y)}\bw=\psi\br{\ph(xy)}\bw=(xy)+I\bw=(x+I)(y+I)\bw=\psi\br{\ph(x)}\psi\br{\ph(y)}$, что и требуется.
\end{proof}

Заметим, что тривиальные идеалы в кольцах есть всегда. Таково всё кольцо и  нулевое подкольцо. Покажем, что в
полях все идеалы тривиальны. Действительно, если $x \in I$, где $x \neq 0$, то $1 \in I$, поскольку $x^{-1}x
\in I$. А если в идеале есть единица, то там содержится и всё поле, поскольку идеал выдерживает умножение на
любой элемент поля.

\begin{df}
Рассмотрим коммутативное кольцо $A$. Зафиксируем систему элементов  $\hc{x_1\sco x_r}, x_i \in A$. Рассмотрим
множество $I := \hc{a_1x_1 \spl a_rx_r\vl a_i \in A}$. Очевидно, что $I \lhd A$, поскольку оно замкнуто
относительно сложения и операции умножения на элементы кольца. Такой идеал называется \emph{порождённым}
элементами $x_1\sco x_r$ и обозначается $(x_1\sco x_r) \lhd A$. Идеал, порождённый одним элементом,
называется \emph{главным}. Кольцо, в котором каждый идеал главный, называется \emph{кольцом главных идеалов}.
\end{df}

\begin{theorem}
Пусть $K$\т поле. Тогда $K[x]$\т КГИ.
\end{theorem}
\begin{proof}
Рассмотрим ненулевой идеал $I \lhd K[x]$. Выберем в $I$ ненулевой многочлен  минимальной степени и обозначим
его через $d$. Покажем, что $I = (d)$. В самом деле, пусть $f \in I$. Допустим, что $d \ndivs f$. Поделим $f$
на $d$ с остатком: $f=dq+r$. Тогда $r=f-dq$, откуда $r \in I$. Действительно, $f \in I$, $dq \in I$, значит, их
разность тоже лежит в идеале. Но $\deg r < \deg d$, а это противоречит тому, что мы брали $d$ минимальной
степени. Значит, $d \divs \!f$, откуда $I=(d)$.
\end{proof}

Из этой теоремы следует, что идеал в $K[x]$ содержит единственный ненулевой  многочлен минимальной степени со
старшим коэффициентом 1. Кроме того, если $K[x]$ заменить на $\Z$, а «степень»\т на «модуль», то мы получим
доказательство теоремы о том, что $\Z$\т кольцо главных идеалов.

\begin{problem}
Докажите, что $K[x_1\sco x_n], n \ge 2$ и $\Z[x]$ не являются КГИ.
\end{problem}

\begin{theorem}[Об инъективности]
Пусть $\ph\cln F \ra L$\т гомоморфизм полей. Тогда либо $\ph \equiv 0$, либо $\ph$ инъективен.
\end{theorem}
\begin{proof}
Поскольку $\kph \lhd F$, а в полях нетривиальных идеалов нет, получаем, что либо $\kph = 0$,
и тогда $\ph$ инъективен; либо $\kph = F$, и $\ph \equiv 0$.
\end{theorem}

\subsection{Факторкольца многочленов. Алгебраические расширения. Поля разложения многочленов}

Рассмотрим факторкольцо $A=\fact{K[x]}{(d)}$, где $d = a_nx^n \spl a_0, n > 0$. Рассмотрим эпиморфизм
$\ph\cln K[x] \ra A$, определённый по правилу  $\ph\cln f \mapsto f + I$. Кроме того, поскольку $K \inj K[x]$, можно
рассмотреть и ограничение этого эпиморфизма $\ph\evn{K}\mskip-10mu \cln K \ra A$. Введём обозначение
для смежных классов по идеалу: $\ol{f}=f+I$. Для элементов поля $K$ имеем
$\ph(\al + \be)= (\al + \be) + I = \ph(\al) + \ph(\be)$ и
$\ph(\al\be)=(\al\be)+I=\ph(\al)\ph(\be)$. Тогда $\ol{K}=\hc{\ol{\al}=\al + I\cln \al \in K} \bw\cong K$. Значит,
по теореме об инъективности, $K \inj A$, и тогда $A$ будет линейным пространством над полем $K$. Заметим, что
каждый смежный класс порождается многочленом степени меньше $n$: поделив на $d$ с остатком, получаем
$\ol{f}=f+I=r+(qd + I)= r+I$. Отсюда $\dim_K A=n$. Базисом будут степени $\ol{x}$ до $n-1$ включительно:
$\ol{1}\sco \ol{x}^{n-1}$. Через них всё выражается, докажем линейную независимость: если бы
$\ol{\al_{n-1}x^{n-1} \spl \al_0} \bw=\ol{0} \bw=0+I \in I$, то это противоречило бы тому, что в $I$ нет многочленов
степени ниже $n$, кроме нулевого. Заметим, что здесь мы воспользовались свойствами естественных операций.
Отсюда следует, что $\al_i = 0$.

\begin{theorem}
Факторкольцо $A=\fact{K[x]}{(d)}$ является полем тогда и только тогда, когда~$d$ неприводим.
\end{theorem}
\begin{proof}
В самом деле, если $d=d_1d_2$, и $\deg d_1, \deg d_2 < n$, то $\ol{d}_1, \ol{d}_2 \neq \ol{0}$. Но
$\ol{d}_1\ol{d}_2=\ol{d}=\ol{0}$. Значит, тут есть делители нуля, и $A$ не может быть полем. Пусть $d$
неприводим, тогда пусть $\ol{f} \neq \ol{0}$, значит, $d \ndivs f$.  Поскольку $d$ неприводим, получаем, что
$(d,f)=1$, а отсюда по формуле «$fu+gv$» имеем $\ol{f}\ol{u}+\ol{d}\ol{v} \bw=\ol{1}$. Но $\ol{d}=\ol{0}$, отсюда
$\ol{f}\ol{u}=\ol{1}$, и мы нашли обратный элемент к ненулевому многочлену $\ol{f}$. Значит, это поле.
\end{proof}

\begin{df}
Пусть $K$\т подполе $F$, тогда $F$ называют \emph{расширением} поля~$K$.
Пусть $p$\т неприводимый над $K$ многочлен. Тогда $\fact{K[x]}{(p)}$ называют \emph{простым
алгебраическим расширением поля} $K$. Также можно рассмотреть \emph{простое трансцендентное
расширение} $K \inj K(x)$ с помощью \emph{поля рациональных функций} над $K$, но мы не будем этого делать.
\end{df}

Неприводимый над $K$ многочлен $p$ имеет корни в $\fact{K[x]}{(p)}$. Действительно, $p(\ol{x})=
\al_n\ol{x}^n \spl \al_0=\ol{p(x)}=\ol{0}$.

\begin{theorem}[О вложении]
Пусть $F$\т расширение поля $K$, содержащее корень $\theta$ неприводимого над  $K$ многочлена $p \in K[x]$.
Тогда существует вложение $\fact{K[x]}{(p)} \inj F$.
\end{theorem}
\begin{proof}
Рассмотрим отображение $\ph\cln K[x] \ra F$ по правилу $\ph\cln f(x) \mapsto f(\theta)$.  Это, очевидно,
гомоморфизм. Ясно, что $\kph$ есть множество всех многочленов из $K[x]$, имеющих корень $\theta$, в
частности, $p \in \kph$. Заметим, что $\kph$\т главный идеал, порождённый некоторым многочленом $d$. Значит,
$p = d \cdot g$, но $p$\т неприводим, значит, $p \sim d$. Тогда можно считать, что $\kph = (p)$, и применив
теорему о гомоморфизме колец, получаем требуемое. Заметим, что мы таким образом построили минимальное
подполе, содержащее корень неприводимого многочлена $p$. Это вытекает из соображений размерности: не
существует собственного подполя, содержащего корень $\theta$ с размерностью, равной размерности данного поля.
\end{proof}

\begin{note}
Расширение поля~$K$, полученное присоединением корня $\ta$ неприводимого над $K$ многочлена, обозначается
$K(\theta)$.
\end{note}

\begin{ex}
Построим поле $\Cbb$. Возьмём многочлен $p$ второй степени с отрицательным  дискриминантом, и в качестве
$\theta$ рассмотрим какой-нибудь его корень. Мы знаем, что $\dim_\R \Cbb=2$. Имеем $\ph(1)=1, \ph(x)=\theta$.
Поскольку 1 и $\theta$ линейно независимы, $\Cbb = \ha{1,\theta}$. Тогда $\fact{\R[x]}{(p)} \cong \Cbb$.
\end{ex}

Пусть $K \subs F, K \subs L$. Пусть $\ph\cln F \ra L$\т изоморфизм полей.  $\ph$ называется \emph{изоморфизмом
над} $K$, если $\fa \al \in K$ имеем $\ph(\al)=\al$, \те поле $K$ при таком изоморфизме неподвижно.

Пусть $f \in K[x]$. Многочлен $f$ может не разлагаться на линейные множители  над $K$, если $n=\deg f > 1$.
Наша цель\т построить поле, в котором он разложится на линейные множители. Рассмотрим $L_1$\т простое
алгебраическое расширение $K$, содержащее корень $\theta_1$ многочлена $f$. $L_1$ строится очевидным образом:
надо профакторизовать $K[x]$ по любому неприводимому множителю $f$, тогда это будет поле, содержащее корень
данного неприводимого множителя. Итак, $f=(x-\theta_1)g \in L_1[x]$, причём $\deg g = n - 1$. Значит, можно
рассмотреть простое алгебраическое расширение $L_2$ для $g$, там будет ещё один корень $\theta_2$, и так
далее. Поскольку степень многочлена уменьшается на 1 при каждом расширении, не более, чем за $n$ расширений
мы придём к полю $L_r$, над которым $f$ разлагается на линейные множители.

\begin{df}
Теперь рассмотрим пересечение всех подполей, содержащих корни $\theta_i$.  Оно тоже разлагает многочлен $f$
на линейные множители, и кроме того обладает свойством минимальности, \те не существует промежуточного
подполя, над которым $f$ разлагается на линейные множители. Такое поле называется \emph{полем разложения} многочлена $f$.
\end{df}

\begin{theorem}
Все поля разложения многочлена $f \in K[x]$ изоморфны между собой над $K$.
\end{theorem}
\begin{proof}
Будем вести доказательство индукцией по $n=\deg f \ge 1$. Для $n=1$ доказывать  нечего, ибо поле разложения
просто совпадает с $K$. Пусть утверждение теоремы верно для всех многочленов степени меньше $n$ над любым
полем. Покажем, что и для $n$ это справедливо. Пусть $E$ и $E'$\т поля разложения $f \in K[x]$. Выберем
неприводимый множитель $p$ многочлена $f$, и пусть $\theta \in E\cln p(\theta)=0$ и $\theta' \in E'\cln
p(\theta')=0$. Тогда по теореме о вложении получаем $L := K(\theta)=\fact{K[x]}{(p)} \inj E$ и
$L':=K(\theta')=\fact{K[x]}{(p)} \inj E'$. Тогда ясно, что $L\cong L'$ над $K$. Отождествим в силу этого
изоморфизма поля $L$ и $L'$, а также корни $\theta$ и $\theta'$. Поэтому далее можно считать, что $L \subs E,
L \subs E'$. Заметим, что многочлен $f$ представим над $L$ как $f=(x-\theta)g$, где $\deg g = n-1, g \in
L[x]$. Покажем, что $E$ и $E'$ есть поля разложения $g$. Действительно, если $f$ разлагается над $E$ на
линейные множители, то и $g$ разлагается. Однако, если бы существовало промежуточное подполе $F$ над которым
$g$ разлагается, и $L \subs F \subsetneq E$, то $F$ было бы полем разложения $f$, что невозможно. Аналогично
устанавливаем, что $E'$ есть поле разложения~$g$. По предположению индукции имеем $E \cong E'$ над $L$.
Поскольку $K \subs L$, тем более имеет место изоморфизм $E \cong E'$ над $K$.
\end{proof}

\subsection{Алгебраические элементы. Алгебраическая замкнутость}

\begin{theorem}[О простых подполях]
Пусть $K$\т поле. Если $\Char K = 0$, то $\Q \inj K$. Если $\Char K \neq 0$, то $F_p \inj K$.
\end{theorem}
\begin{proof}
Пусть $\Char K = 0$. Рассмотрим $\ph\cln \Q \ra K$, определённое по правилу  $\ph\cln \frac{m}{n} \mapsto (m\cdot
1)(n\cdot 1)^{-1}$. Очевидно, $\ph$ сохраняет операцию. Но $\ph$ инъективно по теореме об инъективности.
Пусть $\Char K = p$, тогда $p \in \Pg$. Пусть $\ph\cln \Z \ra K$\т гомоморфизм, где $\ph\cln m \mapsto m \cdot 1$.
Имеем: $m \in \kph \Lra \ph(m)=me=0 \Lra p \divs m$. Значит, $\kph = (p) \lhd \Z$. Отсюда по теореме о
гомоморфизмах колец $F_p = \fact{\Z}{(p)} \inj K$.
\end{proof}

\begin{df}
Пусть $F$\т простое алгебраическое расширение поля $K$. Обозначим через $[F:K]$ \emph{размерность} $\dim_K F$,
которая, по доказанному выше, есть степень неприводимого многочлена, по которому мы проводили факторизацию.
\end{df}

\begin{df}
Элемент $\al \in F$ называется \emph{алгебраическим над} $K$, если $\exi f \in K[x]\cln f(\al)=0$.
\end{df}

Если $[F:K] < \infty$, то все элементы $F$ алгебраические. В самом деле, пусть $m := [F:K]$. Рассмотрим $\al
\in F$ и пусть $n>m$. Рассмотрим элементы $e, \al, \al^2\sco \al^n$. Они будут линейно зависимы, значит,
найдётся нетривиальная линейная комбинация $c_0e+c_1\al \spl c_n\al^n=0$. Тогда $c_0\sco c_n$ и будут
коэффициентами искомого многочлена, имеющего корень $\al$.

\begin{theorem}[О размерности короткой башни полей]
Пусть $K \subs L \subs F$\т короткая башня полей, и $[F:L]<\infty$ и $[L:K]<\infty$.  Тогда
$[F:K]=[F:L]\cdot[L:K]$.
\end{theorem}
\begin{proof}
В самом деле, пусть $x_1\sco x_m$\т базис $L$ над $K$, а $y_1\sco y_n$\т  базис $F$ над $L$. Докажем, что
$\Bc=\hc{x_iy_j}$ будет базисом $F$ над $K$, откуда и будет следовать утверждение теоремы. Покажем, что через
$\Bc$ выражается любой элемент $F$. В самом деле, пусть $z \in F$, тогда $z=\be_1y_1 \spl \be_ny_n$, где
$\be_j \in L$. Отсюда $\be_j = \al_{1j}x_1 \spl \al_{mj}x_m$. Значит, если мы подставим $\be_j$ в выражение
для $z$, получим $z = \suml{j=1}{n}\suml{i=1}{m}\al_{ij}x_iy_j$, и выразимость доказана. Теперь докажем
линейную независимость. Пусть $0=\sum c_{ij} x_iy_j=\sum\br{\sum c_{ij}x_i}y_j=0$. Но поскольку $c_{ij} \in
K$, получаем, что $r_j := \sum c_{ij}x_i \in L$, значит, поскольку $\hc{y_j}$\т базис, получаем, что $r_j =
0, \fa j$. Но так как $\hc{x_i}$\т базис, получаем, что при каждом $j$ имеем $c_{ij}=0, \fa i$. Итак, все
$c_{ij}=0$, значит, только тривиальные линейные комбинации векторов $\Bc$ могут быть равны 0. Значит, $\Bc$\т
базис.
\end{proof}

\begin{imp}
Пусть $E$\т поле разложения многочлена $f \in K[x]$. Тогда $[E:K]<\infty$.
\end{imp}
\begin{proof}
Как было показано ранее, если $p$\т неприводим, то $\fact{K[x]}{(p)}$\т поле  размерности $\deg p$ над $K$,
поэтому, как следует из алгоритма построения поля разложения, его размерность над $K$ по предыдущей теореме
будет конечна, ибо мы расширяем поле не более $\deg f$ раз.
\end{proof}

\begin{df}
Пусть $F$\т алгебраически замкнутое расширение поля $K$. Напомним, что  \emph{алгебраическая замкнутость}
поля означает, что любой многочлен положительной степени над этим полем имеет в нём корни. Пусть $\ol{K}$\т
множество корней в $F$ многочленов из $K[x]$. $\ol{K}$ называется \emph{алгебраическим замыканием} поля $K$.
\end{df}

\begin{theorem}
Множество $\ol{K}$ является алгебраически замкнутым подполем $F$.
\end{theorem}
\begin{proof}
Покажем, что $\ol{K}$\т подполе. Пусть $f_1, f_2\in K[x]$ и  $f_1(\al_1)=0, f_2(\al_2)=0$, где $\al_i \in
\ol{K}$. Возьмём $f:=f_1f_2$, тогда $\al_i$\т его корни. Пусть $E \inj F$\т поле разложения $f$. $E$\т поле,
значит, $\al_1\al_2, \al_1^{-1}, \al_1 + \al_2 \in E$. Так как $[E:K]<\infty$, все элементы $E$ алгебраичны
над $K$, значит, все они лежат в $\ol{K}$.

Покажем, что $\ol{K}$ алгебраически замкнуто. Рассмотрим  $f = c_nx^n \spl c_0 \in \ol{K}[x]$.  Расширим поле
$K$, присоединив к нему все $c_i$, получим поле $E$. Тогда $E \inj F$ и $[E:K] < \infty$. Теперь можно
утверждать, что $f \in E[x]$. В силу алгебраической замкнутости $F$, $\exi \theta \in F\cln f(\theta)=0$. Пусть
$E(\theta) \inj F$\т расширение, полученное присоединением корня $\theta$. Тогда $[E(\theta):E]< \infty$. По
теореме о короткой башне полей имеем $[E(\theta):K]< \infty$, откуда все элементы $E(\theta)$ алгебраичны над
$K$. Значит, $\theta$ есть корень некоторого многочлена из $K[x]$ и потому $\theta \in \ol{K}$.
\end{proof}

\begin{ex}
Рассмотрим $\Cbb \supset \Q$. Рассмотрим подполе \emph{алгебраических чисел}  $\ol{\Q}$. Остальные числа
$\Cbb \wo \ol{\Q}$ называются \emph{трансцендентными}. Заметим, что множество трансцендентных чисел имеет
мощность континуум, в то время как $\hm{\ol{\Q}}=\aleph_0$.
\end{ex}

\subsection{Конечные поля}

Пусть $F_q$\т конечное поле порядка $q$. Очевидно,  $\Char F_q = p \in \Pg$. Мы знаем, что $F_p \inj F_q$ и
$[F_q:F_p] < \infty$ в силу конечности полей. Пусть $y \in F_q$, тогда $\exi!$ разложение вида $y=c_1x_1 \spl
c_nx_n$, где $c_i \in F_p$. Тогда мощность $F_q$ будет равна количеству строк вида $(c_1\sco c_n)$, а их,
очевидно, будет $p^n$ штук, отсюда $|F_q|=p^n$.

Пусть $x,y \in F_q$. Покажем, что $(x+y)^p=x^p+y^p$. В самом  деле, имеем
$(x+y)^p=\suml{0}{p}\binom{p}{i}x^iy^{p-i}$. Но так как $p \divs \binom{p}{i}$ при $i \neq 0, i \neq p$, то
для таких $i$ в поле характеристики $p$ будем иметь $\binom{p}{i}=0$. Тогда $(x+y)^p=x^p+y^p$.

Имеем $|F_q^*|=p^n-1$. По теореме Лагранжа имеем для $a \in F_q^*$  соотношение $a^{p^n-1}=1$. Тогда
$a^{p^n}=a$. Рассмотрим $f(x)=x^{p^n}-x \in F_p[x]$. Тогда любой элемент $F_q$ является его корнем. Но у него
не может быть больше $p^n$ корней по теореме Безу, поэтому $F_q$\т поле разложения этого многочлена. Отсюда
следует единственность с точностью до изоморфизма поля из $p^n$ элементов, поскольку все они будут изоморфны
как поля разложения некоторого многочлена.

Теперь докажем, что такие поля существуют. Рассмотрим  $p \in \Pg$ и $n \in \N$. Пусть $f=x^{p^n}-x \in
F_p[x]$. Рассмотрим формальную производную этого многочлена: получим $f'=p^nx^{p^n-1}-1=-1$ в $F_p[x]$.
Значит, у него нет кратных корней. Пусть $E$\т поле разложения $f$. Покажем, что все корни $f$ образуют
подполе $\mu \subs E$. Пусть $f(\al)=0, f(\be)=0$. Имеем $\al^{p^n}=\al$ и $\be^{p^n}=\be$. Рассмотрим
$(\al+\be)^{p^n}=\al^{p^n}+\be^{p^n}=\al+\be$. Отсюда $f(\al+\be)=0$. Совершенно аналогично показывается, что
произведение корней $\al\be$ и $\al^{-1}$ также лежат в подполе $\mu$. Тогда, поскольку $E$\т минимальное
поле, над которым $f$ разложим на линейные множители, получаем, что $E = \mu$. Отсюда $|E|=p^n$.

\begin{theorem}
Инъекция $F_{p^n} \inj F_{p^m}$ существует $\Lra$ $n\divs m$.
\end{theorem}
\begin{proof}
Пусть инъекция существует. Тогда положим $r:=[F_{p^m}:F_{p^n}]$.  По теореме о размерности короткой башни
полей $F_p \subs F_{p^n} \subs F_{p^m}$ получаем $nr=m$, что и требуется.

Обратно, пусть $m=nr$. Имеем $p^m-1=p^{nr}-1=\hr{p^n}^r-1=(p^n-1)t$. Имеем также $x^{p^m}-x \bw=
\hr{x^{p^m-1}-1} x \bw= x\hr{x^{(p^n-1)t}-1} \bw= x\hr{x^{p^n-1}-1}g(x)$,  где вид $g$ нас не интересует. Отсюда
$x^{p^m}-x=(x^{p^n}-x)g(x)$. Значит, все корни многочлена $x^{p^n}-x$ являются корнями многочлена $x^{p^m}-x$.
Из этого следует вложение.
\end{proof}

\begin{theorem}
Пусть $a,b \in G, ab=ba, \ha{a}_m\cap\ha{b}_n=\hc{e}$. Тогда $O(ab)=[m,n]$.
\end{theorem}
\begin{proof}
Имеем $(ab)^s=e \Lra a^sb^s=e \Lra a^s=b^{-s}$. Но
$a^s \in \ha{a}$, а $b^{-s} \in \ha{b}$, значит, поскольку
пересечение только по единице, получаем $a^s=e, b^{-s}=e$. Значит,
$m \divs s, n \divs s$, откуда $[m,n] \divs s$. Поэтому
$O(ab)=[m,n]$. В случае, если $(m,n)=1$, получаем, что $O(ab)=mn$.
\end{proof}

\begin{df}
Пусть $n=p^s \cdot k$, причём $(k,p)=1$ и $p \in \Pg$. Тогда $p^s$  называется \emph{примарным делителем}
числа $n$ по $p$.
\end{df}

\begin{theorem}
Мультипликативная группа конечного поля является циклической.
\end{theorem}
\begin{proof}
Пусть $F_q$\т поле порядка $q$, значит, $|F_q^*|=q-1$. Рассмотрим  элемент $a$ максимального порядка в
$F_q^*$: $O(a)=n$. Покажем, что $F_q^*=\ha{a}_n$. Допустим противное, именно, $\exi b\cln b \notin \ha{a}$.
Рассмотрим произвольное $p \in \Pg$ среди делителей $n$. Возьмём примарный по $p$ делитель: $n=p^sk$.
Рассмотрим $m := O(b)$. Представим $m$ в виде $m=p^tr$, и покажем, что $s \ge t$. В самом деле, допустим, $s
< t$. Тогда рассмотрим $\ha{a^{p^s}}_k\cap\ha{b^r}_{p^t}=\hc{e}$, поскольку по теореме Лагранжа порядок
элемента из пересечения делит и $k$, и $p^t$, а они взаимно просты. Тогда по предыдущей теореме получаем, что
$O\hr{a^{p^s}\cdot b^r}=kp^t>kp^s=n$. Этого не может быть, ведь мы выбрали элемент максимального порядка.
Значит, получаем, что $m \divs n$, ибо число $p$ можно было брать любым. Тогда получаем, что уравнение
$x^n=1$ имеет $n+1$ корень: $n$ корней из $\ha{a}_n$ и ещё корень $b$. Такого не бывает, значит,
$\ha{a}_n=F_q^*$.
\end{proof}

\begin{theorem}
Над конечным полем $F_q$, где $q = p^m$ и $\Char F_q = p$, существуют неприводимые многочлены любой степени.
\end{theorem}
\begin{proof}
Найдём многочлен степени $n$ над полем $F_q$. Рассмотрим поле $F_r$,  где $r = q^n$. Такое поле существует,
ибо $r = p^{mn}$. Так как $m \divs mn$, существует вложение $F_q \inj F_r$, причём $\dim_{F_q} F_r=n$. По
предыдущей теореме $F^*_r=\ha{\theta}_{r-1}$. Рассмотрим отображение $\ph\cln F_q[x] \ra F_r$ по правилу $\ph\cln f
\mapsto f(\theta)$. Поскольку $\ph(0) = 0$, а $\ph(x^k) = \theta^k$, получаем, что $\ph$\т эпиморфизм. По
теореме о гомоморфизмах колец имеем $\fact{F_q[x]}{\kph} \cong F_r$. Заметим, что $\kph$\т главный идеал,
порождённый некоторым неприводимым многочленом $d$. Действительно, если бы он был приводим, фактор по нему не
был бы полем. Поскольку $\dim_{F_q} F_r = n$, получаем, что $\deg d = n$, и многочлен найден.
\end{proof}

\section{Конечнопорождённые абелевы группы}

\subsection{Прямые произведения групп}

\subsubsection{Внутренние произведения}

\begin{df}
\emph{Произведением} подмножеств $A_1\sco A_m$ группы $G$ назовём множество
$$A_1\sd A_m := \hc{a_1\sd a_m \vl a_i \in A_i}.$$
Пусть $A,B \subs G$. Скажем, что $A$ и $B$ \emph{коммутируют как подмножества}, если $AB=BA$.
Подмножества $A$ и $B$ \emph{коммутируют поэлементно},
если $\fa a \in A, \fa b \in B$ имеем $ab=ba$.
\end{df}

Очевидно, если подмножества коммутируют поэлементно, то они коммутируют и как подмножества.

\begin{problem}
Приведите пример двух коммутирующих подмножеств, не коммутирующих поэлементно.
\end{problem}

Пусть $A, B$\т подгруппы $G$.

\begin{problem}
Приведите пример, когда $AB$ не будет подгруппой $G$.
\end{problem}

Заметим, что если $B \lhd G$, то $AB=BA$. В самом деле, рассмотрим  $a \in A, b \in B$. Тогда $ab=(aba^{-1})a
\in BA$, поскольку в силу нормальности $B$ имеем $aba^{-1} \in B$. Значит, $AB \subs BA$. Аналогично, $BA
\subs AB$. Значит, $AB=BA$.

\begin{theorem}
Если подгруппы $H_i \subs G$ попарно коммутируют, то $H:=H_1\sd H_m$\т подгруппа в $G$, не зависящая от порядка
сомножителей. Если подгруппа $H_i$ коммутирует поэлементно со всеми остальными множителями $H_j$, то $H_i \lhd H$.
\end{theorem}
\begin{proof}
Докажем, что $H$\т подгруппа в $G$. Доказывать будем индукцией по $m$.  При $m=1$ доказывать нечего. Пусть
$m=2$, тогда $A, B \subs G$. Рассмотрим $a_1b_1, a_2b_2 \in AB$. Имеем
$(a_1b_1)^{-1}(a_2b_2)=\ub{b_1^{-1}a_1^{-1}a_2}_{a_3b_3}b_2 \in AB$. Значит, $AB$\т подгруппа. Докажем шаг
индукции: пусть $m \ge 3$ и $H=\ub{H_1\sd H_{m-1}}_L H_m$. По предположению индукции $L$\т подгруппа,
поскольку там $m-1$ множитель. Кроме того, поскольку $H_i$ попарно коммутируют, получаем $LH_m=H_mL$, и тогда
$LH_m$\т подгруппа, поскольку здесь тоже меньше, чем $m$ сомножителей. Итак, $H$\т подгруппа в $G$.

Пусть имеет место поэлементное коммутирование. Покажем, что $H_i \lhd H$. В самом деле, пусть $h \in H$.
Тогда $h=h_1\sd h_i\sd h_m$. Далее $\wh{h_i}$ обозначает пропуск соответствующего множителя. Имеем
$hH_i=(h_1\sd h_i\sd h_m) H_i = \lcomm$ в силу поэлементного коммутирования $\rcomm =(h_1\sd \wh{h_i}\sd
h_m)(h_iH_i)=(h_1\sd \wh{h_i}\sd h_m)H_i=\lcomm$ в силу поэлементного коммутирования, $H_i$ можно протащить
через произведение  $\rcomm=H_i(h_1\sd \wh{h_i}\sd h_m)= (H_ih_i)(h_1\sd \wh{h_i}\sd h_m)=H_i(h_1\sd h_i\sd
h_m)=H_ih$, а, значит, левый смежный класс совпадает с правым.
\end{proof}

\begin{imp}
Если все $H_i \lhd G$, то $H = H_1\sd H_m \lhd G$.
\end{imp}
\begin{proof}
Поскольку нормальные подгруппы заведомо коммутируют как
подмножества, получаем, что $H$\т подгруппа в $G$. Рассмотрим $g \in G$, тогда
$gHg^{-1}=g(H_1\sd H_m)g^{-1}=(gH_1g^{-1})\sd (gH_mg^{-1})=H_1\sd H_m=H
\Ra H \lhd G$, ибо нормальные подгруппы инвариантны
относительно внутренних автоморфизмов.
\end{proof}

\begin{df}
Произведение подгрупп $H_i \subs G$ называется \emph{прямым}, если $H_i$ коммутируют между собой поэлементно и
$\fa h \in H=H_1\sd H_m$ существует единственный набор $(h_1\sco h_m)\cln h=h_1\sd h_m, h_i \in H_i$.
Заметим, что в силу поэлементного коммутирования имеем $H_i \lhd H$.
\end{df}

Обозначения: $H=H_1\st H_m$ для мультипликативной терминологии, $H=H_1\sop H_m$\т для аддитивной.

\begin{theorem}[О прямом произведении]
Пусть $H_i \lhd G$. Тогда $H=H_1\st H_m \Lra \fa i$ имеем $H_i \cap (H_1\sd \wh{H_i}\sd H_m) = \hc{e}$.
\end{theorem}
\begin{proof}
Пусть $H=H_1\st H_m$. Тогда $M := H_i \cap (H_1\sd \wh{H_i}\sd H_m)$.  Рассмотрим $x \in M$. С одной стороны,
поскольку $x \in H_i$, получаем $x = e\sd\underset{i}{x}\sd e$. С другой стороны, $x=h_1\sd
\underset{i}{e}\sd h_m$. Но представление $x$ в виде произведения множителей из $H_j$ единственно, значит,
$x=e$. Следовательно, $M=\hc{e}$.

Обратно, пусть пересечение тривиально. Рассмотрим коммутатор  $[h_i, h_j]=h_ih_jh_i^{-1}h_j^{-1}$. Имеем
$h_ih_jh_i^{-1} \bw\in H_j$ в силу нормальности $H_j$. Но $h_jh_i^{-1}h_j^{-1} \in H_i$ в силу нормальности
$H_i$. Значит, $[h_i,h_j] \in H_i \cap H_j = \hc{e}$. Следовательно, $h_ih_j=h_jh_i$, и мы доказали
поэлементное коммутирование. Теперь докажем единственность разложения. Пусть $x \in H = h_1\sd h_i\sd
h_m=h_1'\sd h_i'\sd h_m'$. Тогда, пользуясь коммутированием, получаем $h_i(h_1\sd \wh{h_i}\sd
h_m)=h_i'(h_1'\sd \wh{h_i'}\sd h_m')$. Далее, $h_i(h_i')^{-1}=(h_1'\sd \wh{h_i'}\sd h_m')(h_m^{-1}\sd
\wh{h_i^{-1}}\sd h_1^{-1})$. Теперь, поскольку $h_m'h_m^{-1} \in H_m$, он коммутирует с элементами других
подгрупп, поэтому этот множитель можно переставить в «хвост» произведения. Теперь, поскольку
$h_{m-1}'h_{m-1}^{-1} \in H_{m-1}$, его тоже можно переставить на предпоследнее место. Так будем переставлять
их, пока не получим $h_i(h_i')^{-1}=(h_1'h_1^{-1})\sd \wh{(h_i'h_i^{-1})}\sd (h_m'h_m^{-1})$. Но поскольку
пересечение подгрупп единичное, получаем $h_i(h_i')^{-1}=\hc{e}$. Значит, $h_i=h_i'$, и мы доказали
единственность разложения.
\end{proof}

\subsubsection{Внешние произведения}

\begin{df}
Пусть $G_1\sco G_m$\т группы. Рассмотрим обычное декартово произведение $G := G_1\st G_m$ и определим на нём
операцию покомпонентного умножения:  $(g_1\sco g_m)\cdot(g_1'\sco g_m'):=(g_1g_1'\sco g_mg_m')$. Очевидно,
это будет группой с единицей в виде строки единиц групп $G_i$. В этом случае $G$ называется \emph{прямым
произведением групп} $G_1\sco G_m$ и обозначается так же, как и внутреннее произведение.
\end{df}

Рассмотрим $\wt{G_i} := \hc{\wt{g_i}=(e_1\sco g_i\sco e_m), g_i \in G_i}$.  Очевидно что $\wt{G_i}$ образуют
подгруппы в $G$. Осуществим канонический изоморфизм $G_i \ra \wt{G_i}$ по правилу $g_i \mapsto \wt{g_i}$.
Тогда можно мысленно отождествить $G_i$ и $\wt{G_i}$. Тогда $G_i \inj G$. Заметим, что $G=\wt{G_1}\st
\wt{G_m}$. В самом деле, элементы между собой коммутируют, разложение единственно, а пересечение, очевидно,
единичное.

\begin{theorem}[О факторизации по прямым множителям]
Пусть $G=G_1\st G_m$. Пусть $H_i \lhd G_i$. Рассмотрим $H=H_1\st H_m$.  Тогда $H \lhd G$ и $\fact{G}{H} \cong
\fact{G_1}{H_1}\st \fact{G_m}{H_m}$.
\end{theorem}
\begin{proof}
Рассмотрим отображение $\ph\cln G \ra \fact{G_1}{H_1}\st \fact{G_m}{H_m}$,
определённое по правилу
$$\ph\cln g = (g_1\sco g_m) \mapsto (g_1H_1\sco g_mH_m).$$
Оно сохраняет операцию: $\ph(gg')=\br{(g_1g_1')H_1\sco  (g_mg_m')H_m}=(g_1H_1\sco g_mH_m)\cdot(g_1'H_1\sco
g_m'H_m)=\ph(g)\ph(g')$. Сюръективность $\ph$ очевидна, ибо на любую строку смежных классов отображается
строка их представителей. Поэтому $\ph$\т эпиморфизм. Далее, $g \in \kph \Lra \ph(g)= (g_1H_1\sco
g_mH_m)=(H_1\sco H_m) \Lra g_i \in H_i$. Отсюда $\kph = H$, и, факторизуя по $H$, по теореме о гомоморфизме
групп получаем требуемое.
\end{proof}

\subsection{Конечнопорождённые абелевы группы}

\begin{df}
Пусть $G$\т абелева группа. Мы будем придерживаться аддитивной  терминологии для абелевых групп. Говорят, что
$G$ \emph{порождается} системой $g_1\sco g_m$, если $\fa g \in G$ имеем $g=n_1g_1 \spl n_mg_m$, где $n_i \in
\Z$. Тогда $G$ называют \emph{конечнопорождённой абелевой группой}.
\end{df}

\begin{df}
Система порождающих $x_1\sco x_m$ группы $G$ называется \emph{базой}  (\emph{базисом}), если она «линейно
независима» над $\Z$, \те только тривиальные целочисленные линейные комбинации базисных векторов могут быть
равны 0. Очевидно, если $x_1\sco x_m$\т база, то представление любого элемента группы через базисные вектора
единственно. В самом деле, если бы их было два, то можно вычесть одно из другого и получить нетривиальную, но
равную нулю линейную комбинацию.
\end{df}

\begin{df}
КПАГ $G$ называется \emph{свободной}, если она обладает базой. Заметим,  что далеко не каждая абелева группа
свободна. Например, конечная ненулевая абелева группа имеет конечное число порождающих, но никогда не бывает
свободной, поскольку там все элементы имеют конечный порядок, значит, для любого ненулевого базисного вектора
$x$ найдётся $n \in \N\cln nx=0$, значит, система даже из одного базисного вектора будет линейно зависимой.
\end{df}

Рассмотрим группу $G=\ha{x_1}_\infty\sop \ha{x_m}_\infty \cong \ub{\Z\sop\Z}_{m} \inj \Q^m$. Заметим, что
если существует нетривиальная линейная зависимость над $\Q$,  то она есть и над $\Z$. Достаточно все дроби
домножить на их общий знаменатель.

\begin{theorem}
Все базы САГ $G$ имеют одинаковое количество элементов.
\end{theorem}
\begin{proof}
Пусть $\Xc=\hc{x_1\sco x_m}$ и $\Yc=\hc{y_1\sco y_n}$\т две различные базы.  Допустим, что $m \neq n$, и для
определённости $m > n$. Тогда, поскольку каждый вектор $x_i$ выражается через вектора базы $\Yc$, получаем
$(x_1\sco x_m)=(y_1\sco y_n)C$, где $C$\т целочисленная матрица, в которой $n$ строк и $m$ столбцов. Но по
основной лемме о линейной зависимости столбцы $C$ линейно зависимы над $\Q$ (ведь мы доказывали эту лемму для
произвольного поля). Значит, как уже было сказано, имеет место и нетривиальная линейная зависимость между
векторами $\Xc$ над $\Z$, а это невозможно.
\end{proof}

\begin{df}
Число векторов базиса САГ называется её \emph{рангом}. Из теоремы следует,  что определение корректно. По
определению, нулевая группа считается \emph{свободной абелевой группой ранга} 0.
\end{df}

Пусть $G$\т САГ с базисом $x_1\sco x_m$. Произвольное отображение векторов  базиса в элементы некоторой
абелевой группы $L$ продолжается естественным образом до гомоморфизма. Действительно, пусть $\ph\cln G \ra L$,
причём $\ph(x_i)=a_i$. Имеем $\ph(n_1x_1 \spl n_mx_m)= n_1a_1 \spl n_ma_m$, причём отображение задано
корректно благодаря единственности разложения элемента $G$ по базису.

\begin{theorem}
Всякая КПАГ $L$ с $m$ порождающими изоморфна некоторому фактору САГ $G$ ранга $m$.
\end{theorem}
\begin{proof}
Пусть $G$\т САГ с базисом $x_1\sco x_m$, а $L=\ha{a_1\sco a_m}$. Как было  сказано выше, корректно задан
гомоморфизм $\ph\cln G \ra L$, при котором $x_i \mapsto a_i$. Поскольку $a_1\sco a_m$ порождают $G$, этот
гомоморфизм будет сюръективен, ведь мы перебираем все строки вида $(n_1\sco n_m)$. Тогда по теореме о
гомоморфизме групп получаем $\fact{G}{\kph} \cong L$.
\end{proof}

\begin{theorem}[О согласованных базах]
Пусть $G$\т САГ ранга $m$ и $H$\т ненулевая подгруппа в $G$. Тогда $H$\т САГ  ранга $l \le m$, причём
существуют согласованные базы $\Xc = \hc{x_1\sco x_m}$ группы $G$ и $\Yc=\hc{y_1\sco y_l}$ группы $H$, такие,
что $y_i=n_ix_i$.
\end{theorem}
\begin{proof}
Будем вести индукцию по рангу $m$ САГ. При $m=1$ доказывать нечего:  $G \cong \Z = \ha{1}_\infty$, а
поскольку любая подгруппа $H \subs \Z$ имеет строение $n\Z$, то для неё базисом будет вектор $y_1=n\cdot 1$.

\pt{1} Рассмотрим $h=a_1x_1 \spl a_mx_m$, для которого $a_1 \in \N$ реализует  минимально возможный
коэффициент по всем $a_i$ и по всем возможным базам группы $G$.

\pt{2} Рассмотрим другой элемент $h'=a_1'x_1 \spl a_m'x_m \in H$. Докажем,  что $a_1 \divs a_1'$. В самом
деле, допустим, что это не так, поделим с остатком: $a_1'=a_1q+r$. Рассмотрим элемент $h'-qh=rx_1+\dots$. Но
этого не может быть, поскольку мы договорились, что $a_1$ реализует минимум среди коэффициентов. Значит,
$r=0$, и тогда $a_1 \divs a_1'$.

\pt{3} Покажем, что $a_1 \divs a_i, i=1\sco m$. Допустим, что $a_i=a_1q+r$.  Перейдём к новой базе
$\hc{x_1+qx_i,x_2\sco x_m}$ и положим $x_1' := x_1+qx_i$. Ясно, что целочисленные элементарные преобразования
не нарушают линейной независимости. Но тогда в этой базе $h=a_1x_1' \spl rx_i+\dots$. Этого тоже не может
быть, поскольку мы договорились, что $a_1$ реализует минимум по всем возможным базам $G$.

\pt{4} Теперь мы точно знаем, что $a_i=a_1q_i, i=2\sco m$. Выберем новый первый  вектор для базы $G$, а
остальные оставим без изменения: $x_1'=x_1+q_2x_2 \spl q_mx_m$. В новой базе имеем $h=a_1x_1'$. Пусть
$H_1:=\ha{h}_\infty$, $G_0:=\ha{x_2}_\infty\sop \ha{x_m}_\infty$. Рассмотрим $H_0:=H \cap G_0$. Очевидно,
$H_1 \cap H_0 = \hc{0}$. Заметим, что $H=H_1 \oplus H_0$. В самом деле, покажем, что $\fa h' \in H$
представим в виде суммы элементов из $H_1$ и $H_0$. Пусть $h'=a_1'x_1'+\dots$. Тогда по доказанному имеем
$a_1\divs a_1'$, \те $a_1'=a_1q_1$. Рассмотрим $h'-q_1h=0\cdot x_1'+\dots=h'' \in H_0$, поскольку оставшаяся
часть будет некоторой линейной комбинацией $x_2\sco x_m$. Следовательно, $h'=q_1h_1+h''$, и представимость
доказана.

\pt{5} Вспомним, что мы ведём индукцию по рангу $G$. Имеем $\rk G_0 = m-1$, и  поскольку $H_0 \subs G_0$\т
САГ, и по предположению индукции её ранг не превосходит $m-1$, и существуют согласованные базы $x_2'\sco
x_m'$ группы $G_0$ и $y_2\sco y_l$ группы $H_0$, и эти базы согласованы числами $n_2\sco n_m$. В силу того,
что $H=\ha{h}_\infty\oplus H_0$, откуда $\rk H = l$. Мы знаем, что $h=a_1x_1'$. Значит, чтобы получить
согласованные базы для $G$ и $H$, достаточно взять $x_1'$ и уже выбранную базу $G_0$, а для $H$ взять $y_1$ и
уже выбранную базу $H_0$. При этом $n_1 := a_1$.
\end{proof}

\begin{imp}
Всякая КПАГ является прямой суммой циклических подгрупп (конечных и бесконечных).
\end{imp}
\begin{proof}
Пусть $L$\т КПАГ. Тогда $L \cong \fact{G}{H}$, где $G$\т САГ некоторого  ранга, и $H \subs G$\т подгруппа.
Выберем согласованные базы в $G$ и $H$, и пусть $x_1\sco x_m$\т база $G$, а $y_1\sco y_l$\т база $H$, причём
$y_i=n_ix_i$. Тогда $G=\ha{x_1}_\infty\sop \ha{x_l}_\infty\bw\oplus\ha{x_{l+1}}_\infty\sop \ha{x_m}_\infty$, а
$H$ можно представить в виде $H=\ha{n_1x_1}_\infty\sop \ha{n_lx_l}_\infty\bw\oplus\ha{0}_1\sop \ha{0}_1$. Тогда
можно профакторизовать по прямым слагаемым, и получим $\fact{G}{H} \cong \fact{\ha{x_1}}{\ha{n_1x_1}}\sop
\fact{\ha{x_l}}{\ha{n_lx_l}}\oplus \fact{\ha{x_{l+1}}}{\ha{0}}\sop\fact{\ha{x_m}}{\ha{0}}$. Но поскольку
$\fact{\ha{x_i}}{\ha{n_ix_i}} \cong \fact{\Z}{n\Z}$, а $\fact{\ha{x_i}}{\ha{0}} \cong \Z$, получаем, что $L
\cong \fact{G}{H} \cong \Z_{n_1}\sop\Z_{n_l}\oplus\Z\sop \Z$, что и требуется.
\end{proof}

\subsection{Разложение на примарные циклические группы}

Опишем способ вычисления порядка элемента прямого произведения групп.  Пусть $G=G_1\st G_m$, причём
$G$ не обязательно абелева. Имеем $g=g_1\sd g_m$. Тогда $O(g)=\bs{O(g_1)\sco O(g_m)}$. В самом деле, пусть
$g^s=\lcomm$ поскольку элементы из различных сомножителей коммутируют между собой $\rcomm = g_1^s\sd
g_m^s=e\sd e$. Значит, имеем $O(g_i) \divs s$, отсюда следует наше утверждение. В частности, если порядки
элементов взаимно просты, их наименьшее общее кратное совпадает с их произведением.

\begin{df}
Группа называется \emph{прямо не разложимой}, если она не представляется  в виде прямого произведения.
\end{df}

Покажем, что $\Z$ прямо не разложима. В самом деле, любая подгруппа $\Z$ имеет вид $n\Z$. Следовательно,
любые две подгруппы $\Z$, скажем, $n\Z$ и $m\Z$, содержат в своём пересечении подгруппу $mn\Z$, поэтому сумма
никаких двух подгрупп не может быть прямой.

\begin{df}
Если группа $G$ имеет порядок $p^n$, где $p \in \Pg$, то она называется $p$-\emph{группой}.
\end{df}

\pt{3} Циклическая $p$-группа прямо не разложима. В самом  деле, допустим, что имеет место нетривиальное
представление $G=\ha{a}_{p^n}=\ha{a_1}_{p^s}\oplus\ha{a_2}_{p^t}$. Здесь из соображений порядков подгрупп
имеем $p^n=p^s\cdot p^t$. Рассмотрим $b = b_1+b_2\in G$, где $b_1 \in \ha{a_1}_{p^s}$, а $b_2 \in
\ha{a_2}_{p^t}$. Тогда имеем $O(b_1)\divs p^s, O(b_2)\divs p^t$, откуда $O(b)=[p^s,p^t]=p^{\max\hc{s,t}}$. Но
мы же договорились, что $s, t>0$. Значит, в $G$ нет элементов порядка $p^n$, ибо $O(b) < p^n$. Противоречие.

\pt{4} Пусть $G$\т конечная циклическая группа, и $|G| \neq p^n, p \in \Pg$.  Тогда $G$ разлагается в прямую
сумму. Действительно, пусть $G=\ha{a}_n$, $n=st, (s,t)=1$. Тогда $G=\ha{ta}_s\oplus\ha{sa}_t$. Сумма
действительно прямая, ибо пересечение явно нулевое, в силу взаимной простоты $s$ и $t$, а
$O(ta+sa)=[s,t]=st=n$. Значит, количество элементов $\ha{ta}_s\oplus\ha{sa}_t$ совпадает с $n=|G|$. Значит,
$G$ действительно разлагается в прямую сумму.

\begin{imp}
Любую конечную циклическую группу можно разложить в прямую сумму примарных циклических.
\end{imp}
\begin{proof}
Будем разлагать, пока не останутся только примарные слагаемые. Рано или поздно мы остановимся.
\end{proof}

\subsection{Инварианты групп}

\begin{df}
\emph{Инвариантом} группы $G$ назовём любой объект, который никак не зависит от примарного разложения этой
группы. Мы извлечём единственность разложения на примарные и бесконечные циклические группы из инвариантов.
\end{df}

\begin{df}
Во всякой абелевой группе $G$ множество элементов конечного порядка образует  подгруппу, называемую
\emph{подгруппой кручения} и обозначаемую $\Tor G$. То, что это подгруппа, легко
проверяется.
\end{df}

По определению, $\Tor G$\т инвариант $G$. Не менее очевидно, что в примарном разложении подгруппе
$\Tor G$ соответствуют все прямые слагаемые конечного порядка. Значит, $G=L\oplus \Tor G$, где $L$\т САГ. На
основании этого мы сейчас докажем, что число бесконечных слагаемых разложения инвариантно. Заметим для
начала, что если $C=A\st B$, то $\fact{C}{B} \cong A$, что тривиально выводится из теоремы о факторизации по
прямым множителям. В нашем случае имеем $\fact{G}{\Tor G} \cong L$. Но тогда $\rk L$\т инвариант, ибо
$\fact{G}{\Tor G}$\т инвариант. А ранг $L$ в точности равен количеству бесконечных слагаемых в разложении.

Значит, нам остаётся доказать, что разложение $\Tor G$ единственно. Выделим  из разложения $\Tor G$
\emph{примарные компоненты}\т слагаемые порядка $p_i^s$, где $p_i \in \Pg$, и обозначим их $G_{p_i}$. Ясно,
что $p$-примарная компонента есть прямая сумма $p$-примарных циклических подгрупп, порядки которых суть
степени $p$, и потому инвариантна. Таким образом, нам осталось доказать, что каждая $p$-примарная компонента
разложима единственным образом.

\begin{theorem}[О единственности]
Разложение $p$-примарной компоненты единственно.
\end{theorem}
\begin{proof}
Пусть $|G|=p^k$ и $G=\ha{c_1}_{p^{k_1}}\sop \ha{c_r}_{p^{k_r}}$, причём  $\sum k_i = k$. Будем вести индукцию
по $k$, и докажем, что набор $k_1\sco k_r$ не зависит от разложения. При $k=1$ доказывать нечего, ибо в
группе порядка $p$ нетривиальных подгрупп нет. Пусть $k>1$. Рассмотрим гомоморфизм $\ph\cln G \ra G$,
определённый по правилу $\ph\cln g \mapsto pg$. Положим $pG := \ph(G)$. Легко видеть, что
$pG=\ha{pc_1}_{p^{k_1-1}}\sop \ha{pc_r}_{p^{k_r-1}}$. При этом обратятся в нули ровно те слагаемые, для
которых $k_i=1$. Ясно, что $pG$\т инвариант, и, кроме того, к $pG$ применимо предположение индукции, ибо
$|pG|<|G|$. Значит, набор не обратившихся в нули $k_i$ инвариантен, поскольку инвариантен набор
$\hc{k_i-1\cln k_i > 1}$. Но и количество ушедших в нули слагаемых инвариантно, поскольку оно определяется из соотношения
$k_1 \spl k_r=k$ однозначно.
\end{proof}


\section{Действия. Разрешимые группы. Теоремы Силова}

\subsection{Действие группы на множестве}

\begin{df}
Говорят, что группа $G$ \emph{действует на множестве} $\Mc$, если задан  гомоморфизм $\rho\cln G \ra
\Sc_\Mc$, где $\Sc_\Mc$\т группа подстановок множества $\Mc$. Обозначение: $gx = \rho(g)(x) \in \Mc$. Само
действие принято обозначать либо $G\cln\Mc$, либо тройкой $(G, \rho, \Mc)$.
\end{df}

Рассмотрим очевидные свойства действия:

\pt{1} $\fa g_1, g_2 \in G, \fa x \in \Mc$ имеем $(g_1g_2)x=g_1(g_2x)$, что  следует из определения операции
в $\Sc_\Mc$.

\pt{2} $\fa x \in \Mc$ имеем $ex=x$, поскольку $\rho(e) = \id_\Mc$.


\begin{df}
\emph{Стабилизатором (стационарной подгруппой)} элемента $x \in \Mc$ называется множество $\St x \bw=
\hc{g\cln gx\bw=x}$. Покажем, что это действительно подгруппа. Пусть $g_1, g_2 \bw\in \St x$. Тогда
$(g_1g_2)x\bw=g_1(g_2x) \bw= g_1x\bw=x$, поэтому $g_1g_2 \bw\in \St x$. Поскольку $ex\bw=x$, получаем $e \in
\St x$. Далее, пусть $g \in \St x$, тогда $gx=x \Lra g^{-1}(gx)\bw= g^{-1}x \Lra (g^{-1}g)x=g^{-1}x \Lra x \bw=
g^{-1}x$, а это и означает, что $g^{-1} \in \St x$. Итак, $\St x$\т подгруппа~в~$G$.
\end{df}

\begin{df}
\emph{Орбитой} элемента $x \in \Mc$ называется множество $\Orb x = Gx = \hc{gx\cln g \in G}$.
\end{df}

Покажем, что орбита порождается любым своим элементом, иными словами, если  $x, y \in \Orb x$, то $\Orb
x\bw=\Orb y$. В самом деле, поскольку $y \in \Orb x$, имеем $gx=y$. Но ясно, что $\Orb gx = \Orb x$, ибо $\Orb
gx \bw= \hc{hgx, h \in G}$, а когда $h$ бегает по $G$, то и $hg$ бегает по $G$, поэтому $\Orb y$ ничем не
отличается от $\Orb x$.

Следовательно, орбиты образуют разбиение множества $\Mc$. В самом деле, если  орбиты пересекаются, то они
совпадают. Действительно, если $z\in\Orb x \cap \Orb y$, то $\Orb x=\Orb z=\Orb y$. Кроме того, сам элемент
всегда лежит на своей орбите, поскольку $ex=x$.

Покажем, что стабилизаторы элементов одной и той же орбиты сопряжены. Рассмотрим $\St x, \St y$, причём $y=gx$. Рассмотрим $a \in G$ и посмотрим,
когда $a \in \St gx$. Имеем $a(gx)=gx \Lra (g^{-1}ag)x=x \Lra g^{-1}ag \in \St x$. Положим $h = g^{-1}ag$, тогда $a=ghg^{-1} \in g(\St x)g^{-1}$,
что и требовалось доказать.

Покажем, что $|\Orb x| = (G: \St x)$, если $|G| < \infty, |\Mc| < \infty$. В  самом деле, если элементы
орбиты совпадают, \те $g_1x=g_2x \Lra (g_1^{-1}g_2)x=x \Lra g_1^{-1}g_2 \in \St x$, откуда $g_1\St x =
g_2\St x$, \те смежные классы по стабилизатору совпадают. Следовательно, если мы будем брать разные элементы
из орбиты, то им будут соответствовать разные смежные классы. А их количество как раз и есть $(G: \St x)$.

\begin{ex}
Пусть $H \subs G$\т подгруппа, $\Mc := G$, а действовать на $\Mc$ будем левыми  сдвигами: $\rho(g) = L_g$.
Тогда $\Orb g = Hg$.
\end{ex}

\begin{ex}
Группа $G$ действует на себе сопряжениями: $I_g x=gxg^{-1}$. Рассмотрим орбиты  при сопряжении: $\Orb x =
\hc{gxg^{-1}\cln g \in G} = K_x$\т класс сопряжённости элемента $x$. Отсюда следует, что классы сопряжённости не
пересекаются. При таком действии стабилизатор называется \emph{централизатором} и обозначается $\Nc_x$.
Заметим, что $\Nc_x$ содержит в себе все элементы группы, коммутирующие с $x$.
\end{ex}

\begin{df}
Множество элементов, коммутирующих со всеми элементами группы, называется  \emph{центром} и
обозначается $\Zc(G)$.
\end{df}

Ясно, что центр\т абелева группа. Центр нормален в $G$. Действительно, пусть $z \in \Zc(G)$. Тогда
$gzg^{-1}\bw=gg^{-1}z\bw=z$, \те элементы центра неподвижны при сопряжении. Следовательно $\Zc(G)$ состоит из
неподвижных элементов и потому инвариантна относительно внутренних автоморфизмов. Значит, $\Zc(G) \lhd G$.

\begin{ex}
Группа действует на множестве своих подгрупп сопряжениями: пусть $H \subs G$,  тогда $I_g H \bw= gHg^{-1} \bw=
H^g$. Рассмотрим стабилизатор $\St H = \hc{g \in G\cln gHg^{-1}=H \Lra Hg = gH}$. Такие стабилизаторы называются
\emph{нормализаторами} и обозначаются $\Nc_H$. Очевидно, $H \lhd \Nc_H$. Таким образом, нормализатор
подгруппы есть максимальная подгруппа, в которой нормальна подгруппа $H$ группы $G$. Обозначим через $K_H$
орбиты этого действия. Они называются классами сопряжённых подгрупп. По доказанному выше имеем $|K_H| =
(G: \Nc_H)$.
\end{ex}

\subsection{Разрешимые группы}

\begin{theorem}
Пусть $A$\т подгруппа в $G$, а $B \lhd G$. Тогда $\fact{AB}{B} \cong \fact{A}{A \cap B}$.
\end{theorem}
\begin{proof}
Мы уже знаем, что $AB$\т подгруппа в $G$. Поскольку $B \lhd G$, тем более  $B \lhd AB$. Далее, рассмотрим
отображение $\ph\cln A \ra \fact{AB}{B}$, определённое по правилу $\ph\cln a \ra aB$. Покажем, что $\ph$\т
эпиморфизм. $(ab)B \bw= a(bB) \bw= aB$, поэтому на каждый смежный класс что-то отобразится. Сохранение операции:
$\ph(a_1a_2) \bw= (a_1a_2)B \bw= (a_1B)(a_2B) \bw= \ph(a_1)\ph(a_2)$. Рассмотрим его ядро: $a \in \kph \Lra \ph(a)=aB=B \Lra
a \in B$, откуда $a \in A \cap B$. Итак, $\ph$\т эпиморфизм. Осталось профакторизовать по его ядру и получить
требуемое утверждение.
\end{proof}

\begin{theorem}
Пусть $B \lhd G$, а $B$ и $\fact{G}{B}$ являются $p$-группами, $p \in \Pg$.  Тогда $G$\т тоже $p$-группа.
\end{theorem}
\begin{proof}
Пусть $|B| = p^n$, а $\bm{\fact{G}{H}} = p^k$. Тогда имеем  $|G|=|B|\cdot\bm{\fact{G}{B}}=p^n \cdot p^k =
p^{n+k}$.
\end{proof}

\begin{df}
Назовём \emph{коммутатором} элементов $a, b \in G$ элемент  $[a,b]=aba^{-1}b^{-1}$.
\end{df}

Очевидны свойства: $[a,b] = e \Lra ab=ba$ и $[a,b]^{-1}=[b,a]$.

\begin{df}
Рассмотрим подгруппу $G' \subs G$, порождённую всеми коммутаторами
элементов $G$. Ввиду второго свойства коммутатора любой элемент $G'$ есть произведение нескольких
коммутаторов. Такая подгруппа называется \emph{коммутантом} группы $G$. Индуктивно определяются
\emph{коммутанты высших порядков}: пусть $G^{(k)}$\т коммутант $k$-ого порядка, тогда
$G^{(k+1)}=\hr{G^{(k)}}'$. Группа $G$ называется \emph{разрешимой}, если $\exi m \in \N\cln G^{(m)}=\hc{e}$.
\end{df}

\begin{theorem}
Пусть $\ph\cln G \ra L$\т гомоморфизм групп. Тогда $\ph\br{G^{(i)}} \subs L^{(i)}$.
\end{theorem}
\begin{proof}
Доказывать будем по индукции. При $i=1$ это верно, поскольку произведение  коммутаторов переходит в
произведение коммутаторов:
$\ph\br{[a,b][c,d]}=\ph\br{[a,b]}\ph\br{[c,d]}=\bs{\ph(a),\ph(b)}\bs{\ph(c),\ph(d)}$, и потому всё, что
порождено коммутаторами элементов из $G$, при гомоморфизме окажется в $L'$, и база индукции доказана. Пусть
теперь $\ph\br{G^{(i)}} \subs L^{(i)}$. Мы уже знаем, что $\ph\hr{\br{G^{(i)}}'} \in \br{L^{(i)}}'$. Но это и
означает, что $\ph\br{G^{(i+1)}} \subs L^{(i+1)}$.
\end{proof}

\begin{theorem}
Пусть $p \in \Pg$. Центр $p$-группы нетривиален.
\end{theorem}
\begin{proof}
Имеем $|G|=p^n$. Рассмотрим разбиение $G$ на классы сопряжённых элементов:  $G = \bigcup K_a$. Заметим, что
если $a \in \Zc(G)$, то его класс сопряжённости состоит только из самого элемента $a$. Имеем $|K_a| =
(G: \Nc_a)$. Если $a \notin \Zc(G)$, то $\Nc_a \neq G$, но тогда по теореме Лагранжа $p$ делит $|K_a|$. Имеем
тогда: $|G|=p^n=|\Zc(G)|+ \sum |K_a|$. Отсюда следует, что $p$ делит $|\Zc(G)|$. Но это означает, что центр
нетривиален.
\end{proof}

\begin{theorem}
Пусть $p \in \Pg$. Тогда $p$-группа разрешима.
\end{theorem}
\begin{proof}
В самом деле, будем вести индукцию по $n$, где $n$\т степень множителя  $p$ в $|G|$. При $n=1$ утверждение
очевидно, ибо тогда $|G|=p$ и она циклическая, а потому абелева. Пусть $n \ge 2$, тогда $\Zc(G) \neq \hc{e}$.
Рассмотрим $\ol{G} := \fact{G}{\Zc(G)}$, тогда $|\ol{G}|=p^{n-1}$ и она разрешима по индуктивному
предположению, \те $\ol{G}^{(m)}=\hc{\ol{e}}$. Рассмотрим естественный эпиморфизм $\ph\cln G \ra \ol{G}$.
Поскольку гомоморфный образ коммутанта содержится в коммутанте фактора, получаем, что $\ph\br{G^{(m)}} =
\hc{\ol{e}}$, откуда $G^{(m)} \subs \kph = \Zc(G)$. Отсюда $G^{(m+1)} = \hc{e}$, поскольку $\Zc(G)$\т
абелева.
\end{proof}

\begin{theorem}
Пусть $G$\т неабелева группа. Тогда $\fact{G}{\Zc(G)}$ не может быть циклическим.
\end{theorem}
\begin{proof}
Допустим, что $\ol{G} =\fact{G}{\Zc(G)}$ является циклической. Тогда пусть  $\ol{a}=a\Zc(G)$\т порождающий
элемент в $\ol{G}$. Рассмотрим разбиение $G$ на смежные классы. Пусть $x, y \in G$. Каждый из них лежит в
каком-то из смежных классов, которые в силу цикличности фактора можно записать как $\Zc(G)\sco
a^{k-1}\Zc(G)$, где $k = O(\ol{a})=|\ol{G}|$. Следовательно, имеют место равенства $x=a^tz_1$ и $y=a^sz_2$,
где $z_i \in \Zc(G)$. Тогда $xy=a^tz_1 a^sz_2=\lcomm$ поскольку $z_i$ коммутируют со всеми элементами группы
$\rcomm = a^{t+s} z_2 z_1 = a^s z_2 a^t z_1 = yx$, и мы получили, что любые 2 элемента группы коммутируют
между собой, \те $G$\т абелева. Это невозможно по условию. Значит, фактор не может быть циклическим.
\end{proof}

\begin{imp}
Группа $|G| = p^2$, где $p \in \Pg$, является абелевой.
\end{imp}
\begin{proof}
Имеем: $\Zc(G)$ нетривиален. Допустим, что $G$\т не абелева,  тогда $\Zc(G) \neq G$. По теореме Лагранжа
$|\Zc(G)|=p$, и потому порядок фактора по центру тоже равен $p$. Но все группы порядка $p$\т циклические, а,
как мы знаем, фактор по центру неабелевой группы не может быть циклическим. Противоречие, значит, $G$\т
абелева.
\end{proof}

\subsection{Теоремы Силова}

\begin{df}
Рассмотрим группу $|G| = p^nm$, причём $p \in \Pg$ и взят примарный  по $p$ делитель. Если $H \subs G$ и
$|H|=p^n$, то такая подгруппа называется \emph{силовской} $p$-подгруппой. Мы будем обозначать силовскую
подгруппу символом $\Pc$ и в рассуждениях всегда будем подразумевать, что это $p$-подгруппа.
\end{df}

\begin{theorem}[Первая теорема Силова]
Силовская $p$-подгруппа существует.
\end{theorem}
\begin{proof}
Итак, пусть $|G|=p^nm$, где $p \in \Pg$, причём взят примарный по  $p$ делитель. Докажем теорему индукцией по
порядку $G$. При $|G| = p$ доказывать нечего. Пусть $|G| > p$, и для групп меньшего порядка всё доказано.
Возможны два случая.

\pt{1} Пусть $p \divs |\Zc(G)|$. Отсюда $\Zc(G)$ нетривиален.  Рассмотрим $\ha{a}_p \subs \Zc(G)$. Такая
есть, поскольку $\Zc(G)$ абелева, и к ней применима теорема о разложении на примарные циклические. Тогда,
поскольку элементы центра коммутируют со всеми элементами группы, получаем $\ha{a}_p \lhd G$. Рассмотрим
$\ol{G} := \fact{G}{\ha{a}_p}$, причём тогда $|\ol{G}|=p^{n-1}m$.

Рассмотрим канонический эпиморфизм $\ph\cln G \ra \ol{G}$, тогда по  предположению индукции $\exi L \subs
\ol{G}$\т силовская $p$-подгруппа, поскольку порядок фактора меньше порядка группы. Кроме того, имеем $|L| =
p^{n-1}$. Рассмотрим $\Pc := \ph^{-1}(L) \subs G$, тогда, поскольку $L = \ol{H} = \fact{H}{\ha{a}_p}$,
получаем $|\Pc| = p \cdot p^{n-1} = p^n$. Значит, $\Pc$\т искомая силовская подгруппа.

\pt{2} Пусть $p \ndivs |\Zc(G)|$. Тогда рассмотрим разбиение $G$ на  классы сопряжённости: $p^nm=|\Zc(G)| +
\sum |K_a|$. Из соображений делимости, найдётся нетривиальный смежный класс $K_a$, порядок которого не
делится на $p$. Отсюда, поскольку $|K_a|=(G:\Nc_a)$ и $|G| = p^nm = (G:\Nc_a) \cdot |\Nc_a|$, получаем, что
$p \ndivs (G:\Nc_a)$, откуда $p^n \divs |\Nc_a|$. Поскольку $|\Nc_a| < |G|$, по предположению индукции $\exi
\Pc \subs \Nc_a\cln |\Pc|=p^n$. Она и будет искомой силовской подгруппой в $G$.
\end{proof}

\begin{theorem}[Вторая и третья теоремы Силова]
Всякая $p$-подгруппа группы $G$ содержится в некоторой силовской $p$-подгруппе.  Все силовские $p$-подгруппы
сопряжены.
\end{theorem}
\begin{proof}
Пусть $H \subs G$\т какая либо $p$-подгруппа. Рассмотрим действие $H$ на  фактормножестве $\fact{G}{\Pc}$
левыми сдвигами. Так как число элементов любой нетривиальной орбиты обязано делиться на $p$, а $p \ndivs
\hm{\fact{G}{\Pc}}$, то такое действие должно иметь неподвижные точки. Пусть $g\Pc$\т такая неподвижная
точка, то есть для $\fa h \in H$ имеем $h \cdot ga \bw= hga \bw= ga'$, где $a, a' \in \Pc$. Тогда $h = g\ub{a'a^{-1}}_{\in
\Pc}g^{-1}$, а отсюда ясно, что $\fa h \in H$ имеем $h \subs g\Pc g^{-1}$. Тем самым доказано первое
утверждение. Теперь, пусть $H$\т силовская $p$-подгруппа. По доказанному имеем $H \subs g\Pc g^{-1}$, но
$|H|=|g\Pc g^{-1}|=|\Pc|$, откуда $H = g\Pc g^{-1}$.
\end{proof}

\begin{theorem}[Последняя теорема Силова]
Число силовских $p$-подгрупп сравнимо с 1 по модулю $p$.
\end{theorem}
\begin{proof}
По предыдущей теореме, множество всех силовских подгрупп есть  $X = \hc{g\Pc g^{-1}\cln g \in G}$. Рассмотрим
действие $(\Pc, I, X)$. Покажем, что единственной неподвижной точкой при таком действии будет сама $\Pc$. В
самом деле, пусть $H \in X$\т неподвижная точка. Это означает, что $\Pc \subs \Nc_H$. Действительно, если при
сопряжении элементами из $\Pc$ подгруппа $H$ остаётся на месте, это означает, что $\St H$ содержит $\Pc$. Но
стабилизатор при сопряжении\т это и есть нормализатор. Но поскольку $H \subs \Nc_H$, мы нашли две силовские
подгруппы в группе $\Nc_H$, а они сопряжены по предыдущей теореме. Но $H \lhd \Nc_H$, откуда $H = \Pc$.
Далее, поскольку порядки всех нетривиальных орбит кратны $p$, получаем, что $|X| \equiv 1 \mod p$.
\end{proof}

\section{Элементы теории представлений}

\subsection{Линейные представления групп}

\begin{df}
Пусть $G$\т группа, $V$\т векторное пространство над полем $K$.  \emph{Линейным представлением} $\rho$ группы
$G$ называется гомоморфизм вида $\rho\cln G \ra \Aut V$, где $\Aut V$\т группа невырожденных линейных операторов
на $V$. Заметим, что $\Aut V \subs \Sc_V$, и таким образом, представление\т частный случай действия $(G,
\rho, V)$. Пространство $V$ называется \emph{несущим пространством} линейного представления. Мы будем изучать
конечномерные векторные пространства. Пусть $\dim V = n$, тогда $n$ называется \emph{размерностью}
представления.
\end{df}

\begin{df}
Пусть $U \subs V$\т подпространство. $U$ называется $G$-\emph{инвариантным},  если $U$ инвариантно
относительно всех операторов $\rho(g), g \in G$.
\end{df}

Такое свойство $U$ позволяет рассмотреть ограничение нашего
представления $\rho$ на $U$, поскольку $\fa g \in G$ получаем $\rho(g)\evn{U}\cln U \ra U$\т корректно
определённый линейный оператор. Тогда $\rho\evn{U}$ называется подпредставлением. Очевидно, что любое
представление имеет тривиальные $G$-инвариантные подпространства: нулевое и само пространство $V$.

\begin{df}
Если у представления есть нетривиальные подпредставления, то оно называется \emph{приводимым}. Если же
все подпредставления тривиальны, то оно называется \emph{неприводимым}.
\end{df}

\begin{df}
Пусть даны два представления $(G, \rho, V)$ и $(G, \tau, U)$. Естественно, что  $V$ и $U$ рассматриваются над
одним и тем же полем $K$. Линейное отображение $\ph\cln V \ra U$ называется \emph{гомоморфизмом представлений}
$\rho$ и $\tau$, если оно согласовано с действием группы: $\fa x \in V, \fa g \in G$ имеем
$\ph\hr{\rho(g)x}=\tau(g)\ph(x)$. На языке операторов: $\fa g \in G$ имеем $\ph \circ \rho(g) = \tau(g) \circ
\ph$. Если $\ph$\т изоморфизм линейных пространств $V$ и $U$, то $\ph$ называют \emph{изоморфизмом}
представлений. Гомоморфизм представлений обычно обозначается так: $\ph\cln \rho \ra \tau$.
\end{df}

Со всяким линейным представлением группы связано её \emph{матричное представление},  если зафиксировать в $V$
базис. Тогда получаем сквозной гомоморфизм $G \map{\rho} \Aut V \ra \GL_n(K)$. Недостатком матричных
представлений является их зависимость от базиса. Из курса линейной алгебры следует, что если зафиксировать
два базиса, то матрицы одного и того же линейного оператора $\rho(g)$ будут связаны равенством
$B_{\rho(g)}=CA_{\rho(g)}C^{-1}$, где $C$\т матрица перехода от одного базиса к другому. Такие матричные
представления называются эквивалентными. Заметим, что очень похожее по виду соотношение получается, если
записать по-другому операторное равенство: $\fa g \in G$ имеем $\tau(g)=\ph\circ\rho(g)\ph^{-1}$.

\begin{df}
Пусть представление $(G, \rho, V)$ имеет два $G$-инвариантных подпространства $U$  и $W$, причём $V = U
\oplus W$. Тогда имеет смысл разложить наше представление в \emph{прямую сумму}: $\rho = \rho_U \oplus
\rho_W$. Представление, разлагающееся в прямую сумму неприводимых, называется \emph{вполне приводимым}.
\end{df}

На языке матриц существование $G$-инвариантного подпространства $U$ означает, что  матрицы $\rho(g)$ имеют
вид $\rbmat{\tab{c|c}{$*$ & $*$ \\ \hline 0 & $*$}}$, причём первый блок матрицы соответствует
подпространству $U$. Аналогично, если $\rho = \rho_U \oplus \rho_W$, то матрицы $\rho(g)$ имеют вид
$\rbmat{\tab{c|c}{$*$ & 0 \\ \hline 0 & $*$}}$.

\begin{df}
Представление называется \emph{точным}, если $\Ker \rho = \hc{e}$, \те $\rho$  инъективно. Тогда $G \inj
\Aut V$.
\end{df}

\begin{theorem}
Пусть $\ph\cln \rho \ra \tau$\т гомоморфизм представлений c несущими пространствами  $V$ и $U$ соответственно.
Тогда $\Img \ph$ и $\kph$ являются $G$-инвариантными подпространствами.
\end{theorem}
\begin{proof}
Пусть $x \in V$, тогда $\ph(x) \in \Img \ph$. Из согласованности действий следует,  что
$\tau(g)\ph(x)=\ph\hr{\rho(g)x} \in \Img \ph$, а это означает, что $\Img \ph$ является $G$-инвариантным.
Пусть теперь $x\in \kph$, то есть $\ph(x) = 0$. Тогда $\ph\hr{\rho(g)x}\bw=\tau(g)\ph(x)\bw=\tau(g)0=0$, значит,
$\rho(g)x \in \kph$, а это означает, что $\kph$ является $G$-инвариантным.
\end{proof}

\begin{theorem}[Лемма Шура]
Пусть $\ph$\т гомоморфизм неприводимых представлений с несущими пространствами~$V$~и~$U$.
Тогда либо $\ph = 0$, либо $\ph$ есть изоморфизм линейных пространств~$V$~и~$U$.
\end{theorem}
\begin{proof}
Пусть $\ph \neq 0$. Тогда $\Img \ph$ есть ненулевое $G$-инвариантное подпространство.  Но в $U$ нет
нетривиальных $G$-инвариантных подпространств. Значит, $\Img \ph = U$. Иначе говоря, $\ph$ сюръективно.
Далее, поскольку $\kph \neq V$, получаем, что $\kph = 0$, ибо в $V$ тоже нет нетривиальных $G$-инвариантных
подпространств. Следовательно, $\ph$ инъективно. Но тогда $\ph$ биективно, а это и означает, что $\ph$\т
изоморфизм линейных пространств $V$ и $U$.
\end{proof}

\subsection{Теорема Машке}

\begin{df}
Пусть $\Phi\cln V \ra V$\т линейный оператор. Если $\Phi^2=\Phi$, то такой оператор  называется
\emph{проектором}.
\end{df}

\begin{theorem}
Пусть $\Phi$\т проектор. Тогда $V = \Img \Phi \oplus \Ker \Phi$.
\end{theorem}
\begin{proof}
Пусть $x \in V$. Рассмотрим тождество $x = \ub{\Phi x}_{\in \Img} + \ub{x - \Phi x}_{\in \Ker}$. В  самом
деле, поскольку $\Phi$\т проектор, имеем $\Phi(x-\Phi x) = \Phi x - \Phi^2 x = \Phi x - \Phi x = 0$. Значит,
действительно $x - \Phi x \in \Ker \Phi$. Однако $\Ker \Phi \cap \Img \Phi = 0$. Но это и означает, что $V =
\Img \Phi \oplus \Ker \Phi$.
\end{proof}

Говорят, что $\Phi$ \emph{проектирует} $\Img \Phi$ \emph{параллельно} $\Ker \Phi$.

\begin{theorem}
Пусть $G$\т конечная группа, а $K$\т поле нулевой характеристики или $\Char K \ndivs |G|$.  Тогда всякое
подпредставление выделяется прямым слагаемым.
\end{theorem}
\begin{proof}
Рассмотрим произвольное подпредставление $(G, \rho_U, U) \subs (G, \rho, V)$. Из общей  теории линейной
алгебры следует, что всегда можно выбрать $W \subs V$ так, чтобы $V = U \oplus W$. В нашем случае $U$ было и
остаётся $G$\д инвариантным, а вот $W$\т не обязательно. Покажем, что его можно «подправить» таким образом,
чтобы $\rho_U$ выделилось в качестве прямого слагаемого.

Пусть $\Theta\cln V \ra U$ задаёт проекцию на $U$, то есть $\Theta x = u + w \mapsto u$.  Рассмотрим линейный
оператор
$$
  \Phi = \frac{1}{|G|}\sumlg \rho(g) \Theta \rho(g)^{-1}.
$$

Покажем, что $\Phi$ согласован с действием $G$. В самом деле, пусть $h \in G$, а $x \in V$. Тогда
\begin{multline*}
\rho(h)\Phi x = \frac{1}{|G|}\sumlg \rho(h)\rho(g)\Theta \rho(g)^{-1} x=
\frac{1}{|G|}\sumlg \rho(h)\rho(g)\Theta \rho(g)^{-1}\rho(h)^{-1}\rho(h) x=\\
=\ub{\frac{1}{|G|}\sumlg \rho(hg)\Theta \rho(hg)^{-1}}_{\Phi}\rho(h)x = \Phi\rho(h)x,
\end{multline*}
поскольку когда $g$ бегает по $g$, то $hg$ тоже бегает по всей $G$. Иначе говоря, $\Phi\cln \rho \ra \rho$\т
гомоморфизм представлений.

Покажем, что $\Img \Phi \subs U$. Действительно, возьмём $x \in V$, тогда
$$
  \Phi x =  \frac{1}{|G|}\sumlg \rho(g)\ub{\Theta \rho(g)^{-1}x}_{\in U} \in U
$$
в силу того, что $\Theta$\т проекция на $U$, а $\rho(g)$
вектора из $U$ переводит в вектора из $U$, поскольку $U$ есть $G$\д инвариантное подпространство.

Наконец, заметим, что если $u \in U$, то $\Phi u = u$. Действительно, $\Phi u =  \frac{1}{|G|}\sumlg
\rho(g)\Theta \rho(g)^{-1} u = \lcomm$ в силу $G$\д инвариантности $U$ имеем $\rho(g)^{-1}u \in U$, а потому
$\Theta \rho(g)^{-1}u = \rho(g)^{-1}u$, следовательно $\rcomm = \frac{1}{|G|}\sumlg \rho(g) \rho(g)^{-1}u \bw=
\frac{|G|}{|G|}u=u$.

Заметим, что наша конструкция работает благодаря тому, что по условию теоремы в  поле $K$ число $|G| \neq 0$.
Мы доказали, что $\Phi$ является проектором, а потому $V = \Img \Phi \oplus \Ker \Phi$. Но, как мы знаем,
$\Img \Phi$ и $\Ker \Phi$ являются $G$-инвариантными подпространствами.
\end{proof}

\begin{theorem}[Машке]
Пусть $|G|<\infty$. Если $\Char K = 0$ или $\Char K \ndivs |G|$, то любое её  представление вполне приводимо.
\end{theorem}
\begin{proof}
Рассмотрим произвольное представление. Если оно неприводимо, то доказывать  нечего. Если оно приводимо, то
оно разлагается в прямую сумму подпредставлений: $(G, \rho, V) = (G, \rho_U, U) \oplus (G, \rho_W, W)$. Если
они приводимы, то их можно разлагать дальше. Поскольку $\dim U, \dim W < \dim V$, процесс разложения когда-то
остановится.
\end{proof}

\subsection{Линейные представления абелевых групп}
Пусть $K$\т алгебраически замкнутое поле.

\begin{theorem}
Пусть $\Phi = \hc{\ph_i}$\т множество попарно коммутирующих линейных операторов  на $V(K)$, причём $\dim V <
\infty$. Тогда в $V$ существует общий собственный вектор для всех этих операторов.
\end{theorem}
\begin{proof}
Будем вести индукцию по $n = \dim V$. При $n=1$ доказывать нечего. Пусть теперь  $n > 1$, и наше утверждение
верно для пространств меньшей размерности. Если все $\ph_i$ являются гомотетиями, то доказывать нечего, ибо
любой вектор $x \neq 0$ будет искомым. Если же это не так, то $\exi \ph \in \Phi$, не являющийся гомотетией.
Поскольку $K$ алгебраически замкнуто, у $\ph$ есть собственные векторы. Пусть $\la$\т одно из собственных
значений $\ph$, а $V_\la$\т подпространство, отвечающее этому собственному значению. Тогда $V_\la \neq V$ в
силу негомотетичности оператора $\ph$ и $\dim V_\la < n$.

Мы знаем, что $V_\la = \Ker(\ph - \la \Ec)$. Докажем, что $V_\la$ инвариантно  относительно всех операторов
из $\Phi$. В самом деле, пусть $x \in V_\la \Lra (\ph - \la \Ec)x = 0$. Пусть $\psi \in \Phi$. Заметим, что
$\la \Ec$ коммутирует с любым оператором. Тогда $(\ph - \la \Ec)\psi x = \psi (\ph - \la \Ec) x = \psi 0 =
0$, откуда $\psi x \in V_\la$. Тогда по предположению индукции, $\exi x \in V_\la$\т собственный для всех
операторов из $\Phi$, и вектор найден.
\end{proof}

\begin{theorem}
Всякое неприводимое представление абелевой группы над алгебраически замкнутым полем одномерно.
\end{theorem}
\begin{proof}
Поскольку $G$\т абелева, получаем, что операторы из $\Img \rho$ попарно  коммутируют. По предыдущей теореме,
$\exi x \in V$, собственный для операторов из $\Img \rho$. Тогда $U = \ha{x}$, очевидно, является
$G$-инвариантным. Но $\rho$ неприводимо, а потому $U = V$, то есть $\dim V = 1$.
\end{proof}

\begin{imp}
В условиях теоремы Машке любое представление абелевой группы над алгебраически  замкнутым полем разлагается в
прямую сумму одномерных представлений.
\end{imp}

Зафиксируем базис $V$. Одномерные представления представляют собой  гомоморфизмы вида
$\rho\cln G \ra K^*$.

\begin{lemma}
Пусть $G=G_1\st G_m$ и $L$\т абелевы группы. Пусть $\ph_i\cln G_i \ra L$\т  гомоморфизмы прямых множителей.
Тогда отображение $\ph\cln G \ra L$, определённое по правилу $\ph\cln g = g_1\sd g_m \mapsto \ph_1(g_1)\sd
\ph_m(g_m)$\т гомоморфизм, и наоборот, если задан гомоморфизм $\ph\cln G \ra L$, то он индуцирует $\ph_i$.
\end{lemma}
\begin{proof}
Имеем $\ph(gg')=\ph_1(g_1g_1')\sd \ph_m(g_mg_m')= \lcomm$ $L$ абелева  $\rcomm = \br{\ph_1(g_1)\sd
\ph_m(g_m)}\cdot \br{\ph_1(g_1')\sd \ph_m(g_m')}\bw=\ph(g)\ph(g')$.
Обратно, пусть $\ph\cln G \ra L$\т гомоморфизм, а $\ph_i = \ph\evn{G_i}$.  Тогда из свойств гомоморфизма и
свойств ограничений следует, что $\ph(g)=\ph(g_1)\sd \ph(g_m)=\ph_1(g_1)\sd \ph_m(g_m)$.
\end{proof}

Рассмотрим  все неприводимые представления $G$ в $\Cbb^*$, где $G$\т КПАГ. Как мы знаем, наша группа
разлагается в прямое произведение (нам будет удобна  мультипликативная терминология): $G = \ha{x_1}_\infty\st
\ha{x_r}_\infty\st\ha{a_1}_{n_1}\st \ha{a_s}_{n_s}$. В данном случае неважно, являются ли конечные прямые
множители примарными или нет. Нам необходимо изучить гомоморфизмы вида $\ph\cln G \ra \Cbb^*$. По лемме,
необходимо изучить ограничения $\ph$ на каждое из слагаемых. Поэтому можно считать, что теперь группа $G$
состоит всего из одного прямого множителя. Возможны два случая.

\pt{1} Пусть $G$\т САГ ранга 1, \те $G = \ha{x}_\infty$. Поскольку $\ph$\т  гомоморфизм, достаточно задать
образ порождающей: пусть $\ph(x) = c \in \Cbb^*$. Тогда $\ph(x^k)=\ph(x)^k = c^k$, \те $\ph\cln x^m \mapsto
c^m$, и других возможностей нет.

\pt{2} Пусть $G = \ha{a}_n \cong U_n$. Мы знаем, что $\ph$ определяется  образом одного порождающего
элемента. Пусть $\ph(a) =\ep \in \Cbb^*$, тогда $\ph(a^m)=\ep^m$. Имеем $\ph(a^n)=\ph(e)=1$, откуда
$\ep^n=1$. Таким образом, $\ep$ должен быть некоторым корнем $n$-ой степени из 1, \те $\ep \in U_n$.
Обратно, если $\ep \in U_n$, то отображение, заданное по правилу $\ph\cln a \mapsto \ep$, продолжается до
гомоморфизма по правилу $a^m \mapsto \ep^m$. Здесь, конечно, надо проверить корректность данного отображения:
если $a^k=a^l$, то $\ph(a^k)\overset{?}{=}\ph(a^l)$. Но мы не первый раз сталкиваемся с такого рода
гомоморфизмами: подобную проверку мы уже проводили, когда доказывали изоморфизм $\ha{a}_n \cong U_n$ в первой
главе.

Итак, если $G = \ha{x_1}_\infty\st \ha{x_r}_\infty\st\ha{a_1}_{n_1}\st \ha{a_s}_{n_s}$.  Тогда гомоморфизмы
$\ph\cln G \ra \Cbb^*$ определяются по правилу $\ph\cln g = x_1^{k_1}\sd x_r^{k_r}\cdot a_1^{l_1}\sd a_s^{l_s}
\mapsto c_1^{k_1}\sd c_r^{k_r}\cdot\ep_1^{l_1}\sd \ep_s^{l_s}$, где $c_i \in \Cbb^*$, а $\ep_i \in U_{n_i}$,
и других нет.

\subsection{Регулярные представления групп}

\begin{df}
Пусть $K$\т поле, а $|G| = n$. Приступим к изготовлению групповой алгебры $KG$. Пусть $G = \hc{g_1\sco g_n}$, тогда $KG
:= \BC{\suml{i=1}{n}\al_ig_i}$, где $\al_i \in K$. Определим сложение и умножение на скаляры: пусть $x = \sum
\al_i g_i, y = \sum \be_i g_i, \la \in K$, тогда $x + y := \sum(\al_i + \be_i)g_i$ и $\la x := \sum (\la
\al_i) g_i$. Можно отождествить $1 \cdot g_i \leftrightarrow g_i$, и тогда $G \inj KG$. Из определения
следует, что базисом в $KG$ будут элементы группы, отсюда $\dim KG = n$. Можно ввести умножение элементов в
$KG$: его достаточно задать на базисных векторах, но их мы умеем перемножать, поскольку это просто элементы
группы $G$. Тогда это будет ассоциативная алгебра с единицей над $K$. Построенное пространство $KG$
называется \emph{групповой алгеброй} над $K$.
\end{df}

\begin{df}
Рассмотрим представление $(G, \La, KG)$, называемое \emph{регулярным}.  Пусть $g \in G$, Определим линейные
операторы $\La\cln KG \ra KG$ по правилу $\La(g) = L_g$, \те это левый сдвиг на элемент $g$. Таким образом,
$\La(g)$ будет как-то переставлять базисные вектора. Следовательно, он невырожден, поскольку базис переходит
в базис.
\end{df}

Рассмотрим регулярное представление $(G, \La, KG)$ и другое представление  $\br{G, \rho, V(K)}$. Фиксируем
$x \bw\in V$. Рассмотрим $\ph_x\cln KG \ra V$, действующее по правилу $\ph_x \cln \sum \al_ig_i \mapsto \sum \al_i
\rho(g_i)x$. Несложно показать, что $\ph_x$ является линейным отображением $KG \ra V$. Покажем, что $\ph_x$
задаёт гомоморфизм представлений $\La \ra \rho$. В самом деле, проверим согласованность с действием:
$\ph_x\br{\La(g)\sum \al_i g_i} = \ph_x \br{\sum \al_i gg_i}= \sum \al_i \rho(gg_i)x \bw= \rho(g)\br{\sum \al_i
\rho(g_i)x} \bw= \rho(g) \ph_x\br{\sum \al_ig_i}$, и таким образом, операция действительно согласована. Далее,
имеем $\ph_x(e)\bw=\rho(e)x\bw= \id x \bw= x$. Таким образом,

Покажем, что никаких других гомоморфизмов $\La \ra \rho$ не существует.  Именно, если $\ph\cln \La \ra \rho$\т
гомоморфизм представлений, то $\ph$ совпадает с некоторым $\ph_x$. В самом деле, пусть $x := \ph(e)$. Тогда
для базисных векторов имеем $\ph(g)=\ph(ge)=\ph\br{\La(g)e}=\rho(g)\ph(e)=\rho(g)x=\ph_x(g)$.

\begin{theorem}[Универсальное свойство регулярного представления]
В условиях теоремы Машке всякое неприводимое представление $G$ изоморфно
некоторому подпредставлению регулярного представления.
\end{theorem}
\begin{proof}
Рассмотрим регулярное представление $(G, \La, KG)$. По теореме Машке оно  разложимо в прямую сумму
неприводимых представлений: $\La = \oplusl{i=1}{m}\rho_i$. Тогда $KG = \oplusl{i=1}{m}V_i$, где $V_i$
являются $G$-инвариантными подпространствами. Рассмотрим произвольное неприводимое представление $(G, \rho,
V(K))$. Пусть $x \in V$\т ненулевой вектор. Рассмотрим $\ph_x\cln \La \ra \rho$, а это отображение, как мы
знаем, есть гомоморфизм представлений. Оно ненулевое, поскольку $\ph_x(e)=x$. Имеем $\Img \ph_x \neq 0$. Но
$\Img \ph_x$ есть $G$-инвариантное подпространство. По лемме Шура получаем $\Img \ph_x = V$. Рассмотрим
ограничение $\ph_x\evn{V_i}\cln \rho_i \ra \rho$. По лемме Шура получаем, что это либо нулевое отображение,
либо изоморфизм. Но поскольку $\ph_x$ ненулевое, значит, $\exi i\cln \rho_i \cong \rho$. В самом деле, в силу
неприводимости $\rho$, оно не может быть изоморфно прямой сумме нескольких $\rho_i$, но кому-то из них\т
обязательно.
\end{proof}

\begin{imp}
В условиях теоремы Машке группа обладает конечным числом различных неприводимых представлений.
\end{imp}
\begin{proof}
Любое неприводимое представление оказалось изоморфно некоторому  подпредставлению регулярного представления,
а их конечное число.
\end{proof}

\section{Замечания и приложения большой теории}

\subsection{Лирическое отступление о цикличности конечной подгруппы $K^*$}

\begin{df}
\emph{Экспонента} группы $G$\т число $d := \min \hc{k \in \N\cln \fa x \in G\, x^k = e}$. Обозначение: $\exp G$.
\end{df}

Рассмотрим группу $G$ и её примарное разложение. Пусть $|G| = n$. Группа будет циклической тогда и только
тогда, когда в ней есть элемент порядка $n$. Пусть $G = \ha{a_1}_{n_1}\st \ha{a_s}_{n_s}$. Тогда $n = n_1\sd
n_s$. Далее, для $\fa x \bw= (x_1\sco x_s)$ имеем $O(x_i) \divs n_i$. Ясно, что элемент $a = (a_1\sco a_s)$ имеет
наибольший порядок в группе, поскольку $O(a_i) = n_i$. Очевидно, что $O(a) = \bs{n_1\sco n_s}$. Значит, $G$
будет циклической группой $\Lra$ числа $n_1\sco n_s$ взаимно просты, поскольку тогда их наименьшее общее
кратное совпадёт с их произведением, то есть с $n$. Отсюда следует, что группа циклическая $\Lra \exp G = |G|$.

\begin{theorem}
Конечная подгруппа $G \subs K^*$ является циклической.
\end{theorem}
\begin{proof}
Пусть $d = \exp G$. Имеем $x^d = 1 \fa x \in G$ по определению экспоненты.  Поскольку уравнение $x^d-1=0$
имеет не более $d$ корней, получаем, что $|G| \le d$. Но это означает, что $|G| = d$, поскольку случай $|G| <
d$ невозможен. А это и означает цикличность $G$.
\end{proof}

\subsection{Разрешимость и неразрешимость групп}

\pt{1} Подгруппа разрешимой группы разрешима, поскольку если $H \subs G$, то $H^{(i)} \subs G^{(i)}$.

\pt{2} Пусть $H \lhd G$ разрешима и $\ol{G} = \fact{G}{H}$ разрешима.  Тогда $G$ разрешима. В самом деле,
рассмотрим естественный эпиморфизм $\pi\cln G \ra \ol{G}$. Имеем $\ol{G}^{(m)}=\hc{\ol{e}}$, а поскольку
коммутант при гомоморфизме лежит в коммутанте образа, получаем $G^{(m)} \subs H$. Имеем $H^{(n)}=\hc{e}$.
Поэтому $\hr{G^{(m)}}^{(n)} \subs H^{(n)} = \hc{e}$, откуда $G^{(m+n)} = \hc{e}$.

\pt{3} Пусть $H \lhd G$. Тогда $\ol{G}=\fact{G}{H}$ абелева $\Lra G' \subs H$.  Действительно, $\ol{G}$
абелева $\Lra [\ol{a},\ol{b}]=\ol{e} \bw\Lra [a,b] \in H \bw\Lra G' \subs H$.

\begin{theorem}
Группа верхнетреугольных матриц $\UT_n(K) \subs \GL_n(K)$ разрешима.
\end{theorem}
\begin{proof}
Обозначим нашу группу через $G_n$. Рассмотрим $\ph\cln G_n \ra \GL_n$, определённый  по правилу
$$
  \ph\cln \rbmat{\la_1 & & \bigast\\ & \ddots & \\ \bignull & & \la_n}
  \mapsto \rbmat{\la_1 & & \bignull \\ & \ddots & \\ \bignull & & \la_n}.
$$
То, что это гомоморфизм, очевидно, поскольку диагональный элемент  произведения верхнетреугольных матриц есть
произведение соответствующих диагональных элементов. Не менее очевидно, что $\kph$ есть подгруппа
унитреугольных матриц (это верхнетреугольные матрицы с единицами на главной диагонали), обозначим её $H_n$.
Тогда $H_n \lhd G_n$. Более того, поскольку $\Img \ph$\т это просто диагональные матрицы, по теореме о
гомоморфизме получаем, что $\fact{G_n}{H_n}$ абелева и потому разрешима. Осталось доказать, что $H_n$
разрешима, тогда в силу \pt{2} группа $G_n$ разрешима.

Проведём доказательство разрешимости $H_n$ по индукции: для $n=1$  доказывать нечего. Пусть $n>1$, и
предположение индукции верно для $H_k$ при $k < n$. Рассмотрим $A = \ulmatrix{A'}{u}$, $B =
\ulmatrix{B'}{v}$, где $A', B' \in H_{n-1}$, а $u,v$\т столбцы высоты $n-1$. Заметим, что
$AB=\ulmatrix{A'B'}{A'v+u}$. Рассмотрим отображение $\ph\cln H_n \ra H_{n-1}$, определённое по правилу
$\ph\cln A \mapsto A'$. Очевидно, что $\ph$\т гомоморфизм с ядром $K_n := \hc{A_u := \ulmatrix{E_{n-1}}{u}}$, где
$E_{n-1}$\т единичная матрица, а $u$\т столбец. Легко видеть, что $K_n$ абелева, поскольку $A_u \cdot A_v =
A_{u+v}=A_{v+u}=A_v \cdot A_u$, ибо сложение столбцов коммутативно. По теореме о гомоморфизме имеем $H_{n-1}
= \fact{H_n}{K_n}$. Тогда, со ссылкой на \pt{2} получается, что $H_n$ разрешима, ибо $H_{n-1}$ разрешима по
предположению индукции.
\end{proof}

\begin{lemma}\label{all.same.cyc}
Пусть $N \lhd \Sc_n$. Пусть $\si \in N$. Тогда $N$ содержит все подстановки  того же циклического строения,
что и $\si$.
\end{lemma}
\begin{proof}
Разложим $\si$ в независимые циклы. Без ограничения общности можно считать,  что $\si = (i_1\sco i_k)$,
поскольку для нескольких циклов рассуждения аналогичны. В силу того, что $N \lhd \Sc_n$, эта подгруппа должна
выдерживать сопряжения. Покажем, что $\fa \tau \in \Sc_n$ имеем $\tau \si \tau^{-1} = \br{\tau(i_1)\sco
\tau(i_k)}$. Пусть
$$\tau = \rbmat{1 & 2 & \dots & s & \dots & n \\j_1 & j_2 & \dots & j_s & \dots & j_n}.$$
Посмотрим, как подстановка $\tau\si\tau^{-1}$ действует на элемент $j_s$. Если $s \in (i_1\sco i_k)$,  то
$\tau\si\tau^{-1}(j_s)=\tau\si(s) = j_{\si(s)}$. Если же $s \notin (i_1\sco i_k)$, то $\tau\si\tau^{-1}(j_s)
= j_s$. Но это и требовалось доказать, поскольку $\tau$ пробегает всю $\Sc_n$.
\end{proof}

\begin{lemma}\label{tri.cyc.span}
Все тройные циклы порождают группу $\Ac_n$.
\end{lemma}
\begin{proof}
Поскольку любая подстановка есть произведение транспозиций, а чётная подстановка  содержит чётное число
транспозиций, нам достаточно доказать, что имея все тройные циклы, можно получить любую пару транспозиций.
Пусть нам надо получить пару $(ac)(bd)$. Имеем $(abc)(abd)=(ac)(bd)$.
\end{proof}

\begin{lemma}\label{one.normal}
Если $N \lhd \Ac_n$ содержит хотя бы один тройной цикл, то $N = \Ac_n$.
\end{lemma}
\begin{proof}
По лемме \ref{all.same.cyc}, $N$ содержит все тройные  циклы\footnote{Устраните логический пробел в этом
утверждении! Первая лемма применима к $\Sc_n$, а не к $\Ac_n$! Доказательство можно посмотреть в учебнике
Э.\,Б.\,Винберга\т \emph{Прим. ред.}}. Применяем лемму \ref{tri.cyc.span} и получаем требуемое.
\end{proof}

\begin{theorem}
Группа $\Ac_n$ при $n \ge 5$ простая.
\end{theorem}
\begin{proof}
Пусть $N \lhd \Ac_n$ и $N \neq \hc{e}$. Докажем, что в $N$ есть тройной цикл.  Тогда утверждение теоремы
будет следовать из лемм. Пусть $\si = \si_1\sd \si_s \in N$\т разложение подстановки в независимые циклы.
Рассмотрим циклическую группу $\ha{\si}$. Она содержит циклическую группу простого порядка, значит, можно
считать, что $\si$ имеет простой порядок $p$, и число $p$\т минимальное. Тогда без ограничения общности можно
считать, что первый цикл в $\si$ имеет длину $p$, \те $\si_1 = (1,\ldots,p)$. Поскольку все сопряженные с
$\si$ элементы лежат в $N$, то у нас есть перестановка $\tau=\si_1\si_2^{-1}\si_3^{-1}\ldots\si_s^{-1}$.
Тогда $\si\tau = \si_1^2$, \те тоже цикл длины $p$. Для $p$ есть три возможности: $p=2, p=3, p \ge 5$. Если
$p=3$, то $\si\tau$\т тройной цикл и всё доказано.

Рассмотрим случай $p \ge 5$. Мы уже знаем, что в $N$ есть все циклы длины $p$, тогда возьмём перестановки
$\pi_1:=(1,2,3,4,5,\ldots,p)$ и  $\pi_2:=(1,3,4,2,5,\ldots,p)$. Легко видеть, что перестановка
$\pi_1^{-1}\pi_2$ оставляет на месте число $p$, а значит, в ней есть цикл длины меньше $p$. Противоречие.

Теперь рассмотрим случай $p=2$. Тогда $\si$ имеет вид $(12)(34)\rho$,  где $\rho$\т произведение некоторых
других транспозиций. Рассмотрим сопряжённую перестановку $\tau:=(13)(24)\rho$. Поскольку $\rho^2=e$, то имеем
$\si\tau = (14)(23) \in N$. Значит, в $N$ имеются все пары транспозиций. Тогда рассмотрим перестановку $\pi_1
:= (12)(34)$ и $\pi_2:=(12)(45)$. Имеем $\pi_1\pi_2 = (345)$, \те тройной цикл. Значит, $N=\Ac_n$.
\end{proof}

\subsection{Явный вид функции Эйлера и малая теорема Ферма}
\begin{df}
Пусть $n \in \N$, тогда $\ph(n) := \hm{\hc{m \in \N\cln (m,n)=1, m < n}}$\т \emph{функция Эйлера}.
\end{df}

\begin{df}
Функция $f$ называется \emph{мультипликативной}, если для любых взаимно простых $m, n \in \N$ имеем
$f(mn)=f(m)f(n)$.
\end{df}

\begin{theorem}
Функция Эйлера $\ph$ мультипликативна.
\end{theorem}
\begin{proof}
В самом деле, пусть $(m,n)=1$. Рассмотрим $G = \ha{a}_m\st\ha{b}_n$. Мы знаем,  что тогда $G = \ha{c}_{mn}$.
Заметим, что число порождающих в $G$ равно $\ph(mn)$. Рассмотрим $d \in G$. Тогда $\exi!$ разложение вида
$d=a^k\cdot b^l$. Заметим, что $O(d)=\hs{O(a^k), O(b^l)} = mn \Lra O(a^k) =m, O(b^l)=n \Lra$ элемент $d$\т
порождающий. Поэтому количество порождающих есть $\ph(m)\ph(n)$: на каждую порождающую из $\ha{a}$ можно
взять одну из порождающих $\ha{b}$. Но это и означает, что $\ph(mn)=\ph(m)\ph(n)$.
\end{proof}

Выведем формулу для $\ph$: пусть $n=p_1^{k_1}\sd p_s^{k_s}$\т каноническое  разложение числа $n$ на простые
множители. Используя мультипликативность $\ph$, получаем $\ph(n)=\ph\br{p_1^{k_1}}\sd \ph\br{p_s^{k_s}}$.
Легко видеть, что все числа, не взаимно простые с $p^k$ и меньшие его, суть числа $p,2p\sco \hr{p^{k-1}-1}p$,
и всего их $p^{k-1}-1$ штук. Существует $p^k-1$ чисел, меньших $p^k$. Поэтому $\ph\br{p^k}=p^k-1 -
\br{p^{k-1}-1}=p^k-p^{k-1}=p^k\hr{1-\frac{1}{p}}$. Отсюда выводим формулу:
$$\ph(n)=p_1^{k_1}\hr{1-\frac{1}{p_1}}\sd p_s^{k_s}\hr{1-\frac{1}{p_s}}= p_1^{k_1}\sd
p_s^{k_s}\cdot\hr{1-\frac{1}{p_1}}\sd\hr{1-\frac{1}{p_s}}=n\hr{1-\frac{1}{p_1}}\sd \hr{1-\frac{1}{p_s}}.$$

\begin{theorem}[Малая теорема Ферма]
Пусть $(m,n)=1$. Тогда $m^{\ph(n)}\equiv 1 \mod n$.
\end{theorem}
\begin{proof}
Рассмотрим кольцо $\Z_n$. Зададимся вопросом: когда элемент $[m] \in \Z_n$ обратим по умножению? Если
$(m,n)\neq 1$, то $[m]$ является делителем нуля и потому не может быть обратим. Действительно, допустим, что
$[m][k] = [0]$, причём $[m] \neq [0]$, $[k] \neq [0]$ и $[m]$ обратим, \те $\exi [m]^{-1}\cln [m]^{-1}[m]=[1]$.
Тогда имеем $[0]\bw=[m]^{-1}[0]\bw=[m]^{-1}[m][k]=[m^{-1}m][k]=[1][k]=[k]\neq [0]$, противоречие. Если же
$(m,n)=1$, тогда по формуле «$fu+gv$» получаем $mu+nv=1$, и производя редукцию по модулю $n$, получаем
$[m][u]=[1]$, откуда $[u]=[m]^{-1}$. Обозначим через $\Z_n^*$ группу обратимых элементов в $\Z_n$ (проверка
того, что это группа, предоставляется читателю). Тогда, поскольку $\hm{\Z_n^*}=\ph(n)$, по теореме Лагранжа
$\fa [m] \in \Z_n^*$ получаем  $[m]^{\ph(n)}=[1]$, но это и означает, что $m^{\ph(n)} \equiv 1 \mod n$.
\end{proof}

\subsection{Китайская теорема об остатках}

\begin{theorem}[Китайская теорема об остатках]
Пусть $n_1\sco n_s$\т попарно взаимно простые числа,
$n_i \in \N$. Пусть $0 \le r_i < n_i \in \N$. Тогда $\exi m \in \N\cln m \equiv r_i \mod n_i$.
\end{theorem}
\begin{proof}
Рассмотрим $G = \ha{a}_n$, где
$n=n_1\sd n_s$. Пусть $a_i := a^{n_1\sd \wh{n_i}\sd n_s}$.
Очевидно, $G = \ha{a_1}_{n_1}\st \ha{a_s}_{n_s}$. Рассмотрим $b =
a_1\sd a_s$. Имеем $O(b)=\bs{O(a_1)\sco O(a_s)}=n_1\sd n_s=n$,
откуда $b$\т порождающий элемент $G$. Рассмотрим элемент
$a_1^{r_1}\sd a_s^{r_s}$. Он есть некоторая степень порождающей
$b$. Поэтому $\exi m < n$, для которого $a_1^{r_1}\sd
a_s^{r_s}=b^m=a_1^m\sd a_s^m$. Отсюда, по свойствам прямого
произведения групп получаем $a_1^{m-r_1}\sd a_s^{m-r_s}=e=e\sd e$,
и в силу единственности представления элемента в виде произведения
элементов из прямых множителей, получаем $a_i^{m-r_i}=e$, а это
означает, что $n_i \divs (m-r_i)$, \те $m \equiv r_i \mod n_i$.
\end{proof}

\subsection{Теорема Вильсона}

\begin{theorem}[Вильсона]
Пусть $p \in \Pg$, тогда $(p-2)! \equiv 1 \mod p$.
\end{theorem}
\begin{proof}
Рассмотрим $\Sc_p$. Поскольку $|\Sc_p|=p!$, все силовские $p$-подгруппы имеют порядок $p$ и потому
циклические и пересекаются только по $\id$. Циклов длины $p$ в $\Sc_p$ существует всего $(p-1)!$ штук, и
каждая силовская $p$-подгруппа содержит по $p-1$ циклу. Тогда $\frac{(p-1)!}{p-1}=(p-2)!  \equiv 1 \mod p$ по
последней теореме Силова.
\end{proof}

\section{Return to Linear Representations}

\subsection{Пространство линейных отображений}

Рассмотрим векторные пространства $V$ и $W$ над полем $K$. Рассмотрим
$\Lin(V, W)$\т \emph{пространство линейных отображений} из $V$
в $W$. Оно наделено операциями $+$ и $\cdot$ на скаляры, \те
естественной структурой линейного пространства. Пусть $\ph, \psi
\in \Lin(V,W)$. Тогда $(\ph + \psi)(x):=\ph x+\psi x$ и
$(\la\ph)(x)=\la\ph(x)$. Очевидно, что если $\dim V = m$, а $\dim
W=n$, то $\dim \Lin(V,W) = mn$, поскольку $\Lin(V,W) \cong
\Mb(m\times n, K)$ при зафиксированных базисах $V$ и $W$.

Пусть $(G, \rho, V)$ и $(G, \tau, W)$\т представления. Пусть
$\Hom(\rho,\tau)$\т множество гомоморфизмов представлений. Тогда
$\Hom(\rho, \tau) \subs \Lin(V,W)$\т пока это всего лишь
подмножество. Покажем, что это подпространство. Для этого
достаточно доказать, что если $\ph, \psi \in \Hom(\rho, \tau)$, то
$(\ph + \psi)$ и $\la \ph$ тоже лежат в $\Hom(\rho,\tau)$. В самом
деле, рассмотрим $(\ph + \psi)\br{\rho(g)x}=\lcomm$ по определению
суммы операторов $\rcomm =\ph\br{\rho(g)x} +
\psi\br{\rho(g)x}=\lcomm$ в силу того, что $\ph$ и $\psi$
согласованы с действием $\rcomm = \tau(g)\ph(x) + \tau(g)\psi(x) =
\tau(g)\br{\ph(x)+\psi(x)}=\tau(g)\br{(\ph+\psi)x}$, что и
требовалось доказать. Аналогично доказывается свойство умножения
на скаляр $\la$. Итак, $\Hom(\rho,\tau)$\т подпространство в
$\Lin(V, W)$.

Пусть $\ph\cln V \ra W$. Пусть $V = V_1 \sop V_m$. Тогда $\fa x \in V$
имеет место единственное представление $x = x_1 \spl x_m$. Ввиду
того, что $\ph(x) = \ph(x_1) \spl \ph(x_m)$, можно ограничить
действия $\ph$ на каждое из слагаемых и ввести операторы $\ph_i\cln
V_i \ra W$, действующие по правилу $\ph_i(x) =
\ph\evn{V_i}(x_i)$. Важно, что при этом можно расширить область
действия каждого из $\ph_i$ на всё пространство $V$, получив $m$
операторов $\wh{\ph}_i\cln V\ra W$, и мысленно отождествить их с
$\ph_i$. Тогда получается, что $\ph = \ph_1 \spl \ph_m$. Тогда
ясно, что $\Lin(V_1 \sop V_m, W) \cong \Lin(V_1,W) \sop
\Lin(V_m,W)$. Однако не будем рассматривать всё $\Lin$ целиком, а
ограничимся рассмотрением
$\Hom(\rho,\tau)\cong\oplusl{i=1}{m}\Hom(\rho_i,\tau)$.

\subsection{Кратность неприводимого представления}

\begin{df}
Определим \emph{кратность неприводимого представления}: пусть $\rho
=\oplusl{i=1}{m}\rho_i^{r_i}$, где
$\rho_i$\т неприводимые представления. Тогда число $r_i$ называется кратностью неприводимого представления
$\rho_i$ в $\rho$. Ясно, что необходима проверка корректности данного определения.
\end{df}

\begin{theorem}
Пусть $K$ алгебраически замкнуто, а $\rho$ и $\tau$\т неприводимые представления $G$.
Тогда
$$\dim \Hom(\rho,\tau)=\case{0, \rho \ncong \tau,\\1, \rho \cong \tau.}$$
\end{theorem}
\begin{proof}
В самом деле, пусть $\rho \ncong \tau$. Тогда по лемме
Шура получаем $\Hom(\rho,\tau)=\hc{0}$, и потому $\dim
\Hom(\rho,\tau) = 0$. Теперь рассмотрим случай $\rho \cong \tau$.
Тогда их можно отождествить: теперь мы будем рассматривать
$\Hom(\rho,\rho)$. Пусть $\ph \in \Hom(\rho,\rho)$. Тогда,
поскольку $\ph\cln V \ra V$ сохраняет действие группы, получаем
$\ph\br{\rho(g)x}=\rho(g)\ph(x)$. В~силу алгебраической
замкнутости $K$, у $\ph$ есть собственный вектор с собственным
значением $\la$. Покажем, что $V_\la$ есть $G$-инвариантное
подпространство. Действительно, $x \in V_\la \Lra \ph(x)=\la x$,
тогда $\ph\br{\rho(g)x} \bw= \rho(g)\ph(x) \bw= \rho(g)(\la x) \bw= \la
\rho(g)x$. Таким образом, вектор $\rho(g)x$ является собственным
вектором с собственным значением $\la$. Значит, $V_\la$
действительно $G$-инвариантно. Но $\rho$ неприводимо, а потому
$V_\la = V$, ведь оно ненулевое. Значит, наш оператор есть просто
гомотетия с коэффициентом $\la$, то есть $\ph = \la \Ec$. Но пространство
гомотетий, очевидно, одномерно. Поэтому $\dim \Hom(\rho,\rho) =1$.
\end{proof}

\begin{imp}
Кратность неприводимого представления для алгебраически замкнутых полей является инвариантом.
\end{imp}
\begin{proof}
Рассмотрим $\rho = \oplusl{i=1}{m}\rho_i^{r_i}$, причём $\rho_i$ попарно неизоморфны.
Рассмотрим
$$
  \Hom(\rho,\rho_k)\cong\Hom(\rho_k^{r_k},\rho_k)\oplus
  \Hom\bbr{\oplusl{i\neq k}{}\rho_i^{r_i}, \rho_k}\cong
  \Hom(\rho_k,\rho_k)^{r_k}\oplus\bbr{\oplusl{i\neq k}{}\Hom(\rho_i,\rho_k)^{r_i}}\cong
  \Hom(\rho_k,\rho_k)^{r_k},
$$
поскольку по доказанной выше теореме получаем, что последние слагаемые будут
нулевыми. Отсюда получаем, вновь используя доказанную теорему: $\dim \Hom(\rho,\rho_k) = \dim
\Hom(\rho_k,\rho_k)^{r_k}= r_k \cdot \dim \Hom(\rho_k,\rho_k)= r_k$. Следовательно, $r_k$ есть размерность
некоторого пространства, а она инвариантна.
\end{proof}

\begin{note}
Несущие подпространства $V_i$ определены однозначно, если $r_i = 1$.
\end{note}

\subsection{Кратность неприводимых представлений в регулярном представлении}

Рассмотрим регулярное представление $(G,\La,KG)$ и $(G,\rho,V)$. Пусть  $\ph_x \in \Hom(\La, \rho)$, причём
$x\in V$ и $\ph_x(e)\bw=x$, а $\ph_x(g)=\rho(g)x$. Из свойств линейных отображений выводим правило сложения
гомоморфизмов $(\ph_x \bw+\ph_y)(e)=\ph_x(e) \bw+ \ph_y(e)=x\bw+y = \ph_{x\bw+y}(e)$, откуда $\ph_{x+y}=\ph_x + \ph_y$;
и умножения на скаляры: $\ph_{\la x}(e)\bw=\la x \bw= (\la \ph_x)(e)$, откуда $\ph_{\la x} \bw= \la \ph_x$.

Рассмотрим отображение $\Phi\cln \Hom(\La, \rho) \ra V$, определённое по правилу  $\Phi\cln \ph_x \mapsto x$.
Сюръективность очевидна. Проверим линейность: $\Phi(\ph_x + \ph_y) = \Phi(\ph_{x+y})=x+y=\Phi(\ph_x) +
\Phi(\ph_y)$. Аналогично, $\Phi(\la \ph_x) = \la \Phi(\ph_x)$. Отсюда следует, что $\Ker \Phi = 0$, поскольку
имеет место равенство $\ph_x(e)=x$. Значит, $\Phi$ осуществляет изоморфизм линейных пространств.
Следовательно, $\dim \Hom(\La,\rho) = \dim \rho$.

\begin{theorem}
Кратность неприводимого представления группы в её регулярном представлении  совпадает с его размерностью.
\end{theorem}
\begin{proof}
Как мы знаем, $\La = \rho_i^{r_i}\sop\rho_m^{r_m}$. Здесь $\rho_i$\т неприводимые  представления. Заметим,
что $r_i = \dim \rho_i$. Действительно, возьмём некоторое представление $(G, \rho_k, V_k)$. Используя
рассуждения, аналогичные тем, что были в теореме об инвариантности кратности, получаем $\Hom(\La, \rho_k)
\cong \Hom(\rho_k,\rho_k)^{r_k}$, и потому $\dim \Hom(\La, \rho_k) \bw= \dim \rho_k \bw= \dim V_k$.
\end{proof}

\begin{imp}
В условиях теоремы Машке $|G| = \suml{i=1}{m}\br{\dim \rho_i}^2$.
\end{imp}
\begin{proof}
Действительно, мы знаем, что $\dim \La = |G|$, поскольку элементы группы образуют  базис групповой алгебры.
Далее, $|G| = \suml{i=1}{m}r_i \dim \rho_i = \suml{i=1}{m} \br{\dim \rho_i}^2$, поскольку $\dim \rho_i =
r_i$.
\end{proof}

\end{document}
