\title{Лекции по Высшей Алгебре}
\author{(лектор --- В. Н. Латышев)}
\maketitle
\hrule
\rule{0pt}{20pt}
\marginpar{01.09.03}
\par{\bf Содержание курса Высшей Алгебры в третьем семестре:}
\begin{enumerate}
    \item Факторгруппы и факторкольца;
    \item Конечно порожденные абелевы группы;
    \item Теоремы Силова;
    \item Элементы теории представлений групп.
\end{enumerate}
\par
%\rule{280pt}{.1pt}\par\rule{0pt}{25pt}
\centerline{\bf Воспоминания}
\par\rule{0pt}{20pt}
В прошлом семестре основные алгебраические структуры изучались по схеме объект --- подобъект --- изоморфизм.
Четвертым пунктом можно добавить рассмотрение приложений, это
будет рассмотрено в разделе о представлениях групп. Итак, основным
алгебраическим объектом, изучаемом в этом курсе, будет группа.
\par\de Группа $\rr{G}$ --- это непустое множество с операцией, которая удовлетворяет следующим свойствам:
\begin{enumerate}
    \item $(ab)c=a(bc)\quad\forall a,b,c\in \rr{G};$
    \item $\ex e\;:\; ae=ea=a\quad\forall a\in \rr{G};$
    \item $\forall a\in\rr{G}\ \ex a^{-1}\;:\; aa^{-1}=a^{-1}a=e.$
\end{enumerate}
\par Здесь групповая операция называется умножением и никак не обозначается, это соответствует мультипликативной терминологии
теории групп, наряду с которой существует аддитивная терминология, и в ней операция, называемая сложением, обозначается знаком ''$+$''.
\par В группе нейтральный элемент единствен. Действительно, пусть $e_1$ и $e_2$ --- нейтральные элементы, тогда $e_1e_2=e_1=e_2$, и они равны.
Аналогичным образом понятно, что для каждого обратимого элемента (в полугруппе --- непустом множестве с ассоциативной операцией) обратный
элемент тоже единствен: пусть $b$ и $c$ --- два обратных к $a$ элемента, тогда
$(ba)c=ec=c=b(ac)=be=b$, значит, $b=c$.
\par Теперь рассмотрим подобъект, который называется подгруппой: \par\de Непустое подмножество $\rr{H}$ группы $\rr{G}$ называется
подгруппой, если оно само относительно операции, определенной в $\rr{G}$, образует группу.
\par Единица подгруппы совпадает с единицей группы (что, например, нехарактерно для других алгебраических объектов):
пусть $e'$ --- единица в \mr{H}, тогда $e'^2=e',$ и, умножая слева на $e'^{-1},$ имеем $e'^2e'^{-1}=e'e'^{-1}=e=e'.$
\par Существуют две совокупности признаков подгруппы:
\begin{enumerate}
    \item
     $a,b \in \rr{H}\ra ab\in \rr{H},$\\
     $e\in \rr{H},$\\
     $a\in H\ra a^{-1}\in\rr{H};$
    \item $a,b\in \rr{H}\ra a^{-1}b\in \rr{H}$\qquad(в этом пункте предполагается, что \mr{H} --- непустое подмножество \mr{G}).
\end{enumerate}
Очевидно, что эти две совокупности признаков эквивалентны между собой и действительно задают подгруппу.
\par Продолжим воспоминания. Рассмотрим подстановки (то есть биекции на самого себя) произвольного множества ${M}$,
они образуют группу относительно операции суперпозиции. Эта группа
называется группой подстановок множества ${M}$ и обозначается
$\rr{S}_{{M}}$. Если же множество $M$ имеет $n$ элементов, то
можно мыслить, что $M=\{1,2,\dots,n\}$, и тогда группа его
подстановок обозначается $\rr{S}_n$ и называется симметрической
группой $n-$й степени. В этой группе выделяется одна важная
подгруппа --- подгруппа четных подстановок (то есть тех, которые
разлагаются в четное число транспозиций). Она обозначается
$\rr{A}_n\subset \rr{S}_n$ и называется знакопеременной подгруппой
$n-$й степени.
\par Еще один пример группы --- полная линейная группа $\rr{GL}(n,k)$ невырожденных матриц порядка $n$ над полем $k$. В ней выделяется
специальная линейная подгруппа $\rr{SL}(n,k)\subset\rr{GL}(n,k)$ матриц с единичным определителем.
\par\de Порядком элемента группы называется натуральное число $O(a)=\min\{m\in\mathbb{N}\;:\;a^m=e\}.$
Если порядка элемента не существует, то говорят, что он имеет бесконечный порядок (Например, в группе $\mathbb{C}^*=(\mathbb{C}\backslash\{0\},\cdot)$
конечный порядок имеют только корни из единицы). Если же $|\rr{G}|<\infty,$ то для $\forall a\in \rr{G}$ найдутся $k,l$ такие, что $a^k=a^l$ и $k>l$,
поскольку все степени элемента $a$ не могут быть различными. Но тогда $a^{k-l}=e$, то есть, у каждого элемента в конечной группе есть порядок.
\par Отметим некоторые свойства порядка:
\begin{enumerate}
    \item Если $O(a)=n,$ то $a^m=e\lra {}^n\,\vrule\,{}_m.$ Действительно, поделим $m=nq+r,$ тогда $a^m=(a^n)^q\cdot a^r=\\=a^r=e\lra r=0,$ так как $0\le r<n$.
    \item $O(a)=n\ra O(a^k)=\frac{\displaystyle n}{\displaystyle \rr{HOD}(k,n)}.$
\end{enumerate}
\par Рассмотрим гомоморфизмы и изоморфизмы групп.
\par\de Отображение $\phi : \rr{G}\rightarrow \rr{K}$ (где $\rr{G}$ и $\rr{K}$ --- группы) называется гомоморфизмом, если оно сохраняет
операции: $\phi(ab)=\phi(a)\phi(b),$ где $a,b\in\rr{G};\;\phi(a),\phi(b)\in\rr{K}$. Рассмотрим некоторые свойства гомоморфизмов групп.
Пусть $e\in\rr{G},\; e'\in\rr{K}$ --- единицы групп.
\begin{enumerate}
    \item $\phi(e)=e'$ для любого гомоморфизма $\phi$. Действительно, $\phi(e)=\phi(e^2)=\phi(e)^2,$ таким образом $\phi(e)^2\phi(e)^{-1}=\phi(e)\phi(e^{-1})=\phi(e)=e'$
    \item $\phi(a)^{-1}=\phi(a^{-1}).$ Действительно, $\phi(a\cdot a^{-1})=\phi(e)=e',$ но $\phi(a)\phi(a^{-1})=e'\ra\\\ra\phi(a)^{-1}=\phi(a^{-1}),$ так как
    обратный элемент единствен.
\end{enumerate}
\par\de Биективный гомоморфизм называется изоморфизмом. Если между группами существует изоморфизм, то они называются изоморфными, обозначается $\rr{G}\cong\rr{K}$.
В этом случае группы неразличимы с точки зрения свойств операций.
\par У гомоморфизмов есть образ и ядро:
\par\de $\rr{Im}\phi=\{x\in\rr{K}\ :\ \exists a\in \rr{G}\ $ такое что $x=\phi(a)\}\subseteq \rr{K}$ --- образ;
\par\de $\rr{Ker}\phi=\{a\in\rr{G}\ :\ \phi(a)=e'\}\subseteq\rr{G}$ --- ядро.
\te{Предложение}{Ядро и образ гомоморфизма являются подгруппами в соответствующих группах}.
\par\dok Пусть $x,y\in\rr{Im}\phi\lra\ex a,b\in\rr{G}\ :\ \phi(a)=x,\ \phi(b)=y.$ Тогда $x^{-1}y=\phi(a^{-1}b)\in\rr{Im}\phi.$\\
Теперь пусть $a,b \in\rr{Ker}\phi,$ то есть $\phi(a)=\phi(b)=e'.$ Но тогда $\phi(a^{-1}b)=\phi(a)^{-1}\phi(b)=e'^{-1}e'=e'\ra\ra a^{-1}b\in\rr{Ker}\phi.$\quad\qed
\par Инъективный гомоморфизм (то есть такой, при котором из каждый элемент второй группы имеет ровно один прообраз) короче называется инъекцией.
Инъекция осуществляет изоморфизм между $\rr{G}$ и \mr{Im\phi}, или изоморфное вложение \mr{G} в \mr{K}. Обозначается это так: $\phi : \rr{G}\hra\rr{K}$
\te{Замечание}{Гомоморфизм $\phi$ является инъекцией тогда и только тогда, когда $\rr{Ker\phi}=\{e\}.$}
\par\dok $\rr{Ker}\phi=\{e\},\ a,b\in\rr{G},\ \phi(a)=\phi(b)\lra\phi(a^{-1}b)=e'\ra a^{-1}b=e\ra a=b.$\\ Обратно,
пусть $a\in\rr{Ker}\phi\lra \phi(a)=e',$ но $\phi(e)=e',$ и поскольку $\phi$ --- инъекция, то $a=e$ и $\rr{Ker}\phi=\{e\}.$\quad\qed
\par Рассмотрим $\phi : (\mathbb{Z},+)\rightarrow\mathbb{C}^*,$ такое, что $\phi : k\mapsto\eps_k=\cos\frac{2\pi k}{n}+i\sin\frac{2\pi k}n,$ где
$n\in\mathbb{N}$ --- фиксировано, $\eps_1$ --- примитивный (первообразный) корень, $\eps_k=\eps_1^k.$ Тогда
$$\phi(k+l)=\eps_{k+l}=\eps_1^{k+l}=\eps_1^k\cdot\eps_1^l=\eps_k\cdot\eps_l=\phi(k)\cdot\phi(l),$$ то есть поскольку образ суммы есть
произведение образов, $\phi$ является гомоморфизмом. Его образ $\rr{Im}\phi=\rr{U}_n$ --- группа комплексных корней $n-$й степени из 1,
ядро --- это $n\mathbb{Z}$, то есть $\phi$ не является инъекцией.
\par Рассмотрим еще один пример: функция $\ln : x\mapsto\ln x$ осуществляет изоморфизм $(\mathbb{R}_+,\,\cdot)$ и $(\mathbb{R},\,+).$
\par Теперь рассмотрим группу \mr{G} и фиксируем произвольный элемент $g\in\rr{G}$ и рассмотрим отображение $L_g : \rr{G}\rightarrow\rr{G},$
при котором $x\mapsto gx.$ Это отображение называется левым сдвигом группы на элемент $g$. Докажем, что $L_g\in \rr{S}_\rr{G}.$ Действительно,
это сюръекция: для каждого $x\in\rr{G}$ рассмотрим $L_g(g^{-1}x)=gg^{-1}x=x,$ поэтому на каждый элемент что-то отображается. Это также и инъекция:
если $L_g(x)=L_g(y),$ то $gx=gy$ и следовательно, домножая на $g^{-1}$ слева, получаем, что $x=y$.
Таким образом левый сдвиг есть подстановка группы $\rr{G}$, в этом заключается его теоретико-множественное свойство. Отметим другие свойства левого сдвига:
\begin{enumerate}
    \item $L_{g_1}L_{g_2}=L_{g_1g_2}.$ Действительно, $(L_{g_1}L_{g_2})(x)=g_1(g_2x)=(g_1g_2)x=L_{g_1g_2}(x).$
    \item $L_e=\mathcal{E},$ так как $L_e(x)=ex=x.$
    \item $(L_g)^{-1}=L_{g^{-1}}$, потому что $L_gL_{g^{-1}}=L_{gg^{-1}}=L_e=\mathcal{E}.$
\end{enumerate}
Таким образом, все левые сдвиги образуют подгруппу $L_\rr{G}\subseteq \rr{S_G}.$
\te{Теорема (Кэли)}{Существует инъекция $G\hookrightarrow \rr{S_G}$}.
\par\dok Рассмотрим отображение $\phi : G\rightarrow \rr{S_G},$ при котором $g\mapsto L_g.$ Заметим, что $\phi$ --- гомоморфизм.
Действительно, $\phi(g_1g_2)=L_{g_1g_2}=L_{g_1}L_{g_2}=\phi(g_1)\phi(g_2).$ Докажем, что его ядро состоит
из одного элемента. Пусть $g\in\rr{Ker}\phi\lra L_g=\mathcal{E},$ то есть $L_g(x)=gx=x,$ и домножая справа на $x^{-1},$ имеем $g=e.$
Таким образом $\rr{Ker}\phi=\{e\}\ra \phi$ --- инъекция.\quad\qed
\par Образ предыдущего гомоморфизма --- это $\rr{Im}\phi=L_{\rr{G}},$ то есть $\rr{G}\cong L_\rr{G}\subset\rr{S_G}$. Таким образом,
всякая группа вложима в группу подстановок, то есть реализуется подстановками.

Если $|\rr{G}|=n,$ то $\rr{G}\hookrightarrow\rr{S}_n,$ и поэтому {все группы порядка $n$ суть подгруппы в $\rr{S}_n$}. Но если вложить
произвольную группу в группу подстановок достаточно просто, то вынуть ее обратно оказывается неизмеримо сложнее. И напоследок дадим еще одно
определение:

\de Пусть $\{a_1,\dots,a_n,\dots\}\in \rr{G}.$ Подгруппа $\rr{H}\subset\rr G$ называется порожденной элементами\\ $\{a_1,\dots,a_n,\dots\},$ если она
является пересечением всех подгрупп, содержащих эти элементы.
\smallskip
\hrule
\rule{0pt}{10pt}
\marginpar{08.09.03}
\centerline{Порождающие (образующие) группы}

Рассмотрим группу $\rr{G}$ и рассмотрим в ней совокупность элементов $A=\{a_1,\dots,a_n,\dots\}.$ Подгруппа, порожденная
элементами $A$, это $\rr H=\bigcap\limits_i\rr H_i,$ где $\rr H_i$ --- подгруппы и $A\subseteq \rr H_i.$ $\rr H$, очевидно, подгруппа как
пересечение подгрупп.

Однако данное определение неконструктивно, поскольку непонятно, какие элементы содержатся в $\rr H.$
В \mr H содержатся все произведения вида $a_{i_1}^{\eps_1}a_{i_2}^{\eps_2}\dots a_{i_r}^{\eps_r},\ \eps_j=\pm 1$ (здесь не предполагается, что
соседние элементы различны). Однако такие произведения сами образуют подгруппу, и она
содержится в любой из подгрупп $\rr H_i,$ поэтому подгруппа \mr H с ней совпадает.

Если $\rr{H=G},$ то \mr G --- группа, порожденная элементами $\{a_1,\dots,a_n,\dots\},$ и это обозначается как $G=\lob{a_1,\dots,a_n,\dots}.$
Если порождающая у группы одна, то такая группа называется циклической, и при этом используются обозначения $\lob{a}_m$ в случае конечной группы или
$\lob{a}_\fn$ в случае бесконечной (наряду с обычным обозначением $\lob{a}$, разумеется).

\centerline{Циклические группы}

Примеры циклических групп дают $(\mathbb{Z},+)=\lob{1}_\fn$ --- группа целых чисел по сложению и $\rr U_n=\lob{\eps_1}_n$ --- группа всех комплексных корней $n-$й степени из 1
(в последнем примере $\eps_k=\cos\frac{2\pi k}n+i\sin\frac{2\pi k}n,\ k=0,1,\dots,n-1;\ \ \eps_k=\eps_1^k$, а в качестве порождающей можно взять также
любой другой примитивный корень).

Рассмотрим теперь подгруппы циклических групп. Докажем теорему:

\te{Теорема}{Всякая подгруппа циклической группы сама является циклической группой.}

\dok Пусть $\rr G=\lob{a},\rr H\subseteq\rr G$ --- подгруппа.\\ Если $\rr H=\{e\}$, то доказывать нечего: \mr H --- циклическая подгруппа.
\\Пусть теперь $\rr H\ne\{e\}.$ Возьмем $k=\min\{m\in\mathbb{N} : a^m\in\rr H\}\in\mathbb{N}$ и
докажем, что $\rr H=\lob{a^k}.$ Действительно, пусть $a^m\in\rr H.$ Разделим с остатком $m=kq+r,\ o\le r<k$ и рассмотрим $a^m=(a^k)^q\cdot a^r,$
откуда $a^r=a^m\cdot(a^k)^{-q}\in \rr H$ как произведение элементов, лежащих в $\rr H$. Это означает, что $r=0,$ то есть что $a^m=(a^k)^q.$\quad\qed

Из этой теоремы следует, что других подгрупп в $(\mathbb{Z}, +)$ кроме $(n\mathbb Z,+)$ нет.

\te{Теорема}{Все бесконечные циклические группы изоморфны
$(\mathbb Z, +)$; все циклические группы порядка $n$ изоморфны
$\rr U_n$.}

\dok Рассмотрим $\rr G=\lob{a}.$ Доказательство распадается {на} два случая:
\begin{enumerate}
    \item $O(a)=\fn$ . Тогда $\rr G=\{\dots,a^{-m},\dots,a^{-2},a^{-1},a^0=e,a^1,a^2,\dots,a^m,\dots\},$ причем все
    степени элемента $a$ различны (если $a^k=a^l,\ k>l\ra a^{k-l}=e,$ и при этом $k-l>0,$ что противоречит с отсутствием
    порядка у элемента $a$). Поэтому $\rr G=\lob{a}_\fn.$ Рассмотрим отображение $\rr G \rightarrow (\mathbb Z,+)$ при котором
    $a^m\mapsto m.$ Очевидно, что $\phi$ --- биекция, так как все степени $a$ различны, и при этом $\phi$ сохраняет операцию:
    $\phi(a^ka^l)=\phi(a^{k+l})=k+l=\phi(a^k)+\phi(a^l)\ra\phi$ --- изоморфизм.
    \item Теперь пусть $O(a)=n.$ Тогда $\rr G=\{a^0=e,a^1,\dots,a^{n-1}\},$ поскольку других степеней $a$ нет в силу того, что
    $a^n=e$, и все степени $a$ от $0$ до $n-1$, очевидно, различны. Таким образом $\rr G=\lob{a}_n.$ Рассмотрим $\phi : \rr G\rightarrow\rr U_n=\lob{\eps_1}_n,$
    при котором $a^m\mapsto \eps_1^m.$ Определяя отображение таким образом, мы должны убедиться в его корректности. Пусть
    $a^m\mapsto \eps_1^m,\ a^m=a^l.$ Будет ли тогда $\eps_1^m=\eps^l_1?$ Рассмотрим $a^m=a^l\lra a^{m-l}=e\lra{}^n\,\vrule\,_{m-l}\ra
    \eps_1^{m-l}=1\ra{\eps_1^m=\eps_1^l.}$ Таким образом, отображение корректно и, очевидно, сюръективно. Докажем его
    инъективность. Рассмотрим $\phi(a^m)=\phi(a^l)\lra\eps_1^m=\eps_1^l\lra\eps_1^{m-l}=1\lra{}^n\,\vrule\,_{m-l}\ra a^{m-l}=e\ra a^m=a^l.$
    Таким образом, отображение $\phi$ биективно. Оно также, очевидно, сохраняет операции: $\phi(a^ma^l)=\phi(a^{m+l})=\eps_1^{m+l}=
    \eps_1^m\eps_1^l=\phi(a^m)\phi(a^l).$ Поэтому $\phi$ --- изоморфизм и группы оказываются изоморфны.\quad\qed
\end{enumerate}
\centerline{Разложение группы по подгруппе}\medskip

Пусть $\rr H\subseteq\rr G$ --- подгруппа. Фиксируем $a\in \rr G$; рассмотрим два множества: $a\rr H=\{ah : h\in\rr H\},$ которое называется
левым смежным классом элемента $a$ по подгруппе \mr H, и $\rr Ha=\{ha : h\in\rr H\}$ --- правый смежный класс. Вообще говоря,
эти два смежных класса не обязательно должны совпадать, но если \mr G --- абелева группа, то это так.

Рассмотрим теперь $\phi : \rr H\rightarrow a\rr H,$ при котором $h\mapsto ah,$ где $a\in\rr G$ --- фиксированный элемент. Сюръективность
этого отображения очевидна, инъективность несложно устанавливается: $\phi(h_1)=\phi(h_2)\lra ah_1=ah_2\lra h_1=h_2.$ Значит,
$|\rr H|=|\rr Ha|=|a\rr H|.$

Теперь рассмотрим арифметическое строение группы. Число $n=|\rr G|$ называется порядком группы, число $k=|\rr H|$ --- порядком
подгруппы, а количество левых смежных классов по данной подгруппе $j=(\rr G;\ \rr H)$ называется индексом этой подгруппы во всей группе.
\ste{Теорема}{Лагранжа о группах}{В любой группе $n=kj.$}

\dok Докажем сначала, что каждый смежный класс порождается любым своим элементом. Пусть $a\in b\rr H$, покажем включение $\forall h\in\rr H\ \  ah\in b\rr H.$
Действительно, $a\in b\rr H\ra a=bh_1$ для некоторого $h_1\in \rr H.$ Рассмотрим $ah=bh_1h\in b\rr H,$ поскольку $h_1h\in\rr H.$
Таким образом, $a\rr H\subseteq b\rr H.$ Аналогично $b\rr H\subseteq a\rr H,$ поэтому $a\rr H=b\rr H.$

Значит, если два класса имеют непустое пересечение, то они совпадают. Поэтому левые смежные классы образуют разбиение группы $\rr G.$
Отсюда простым подсчетом типа ''вес груза равен количеству мешков, умноженному на вес мешка'' получаем, что $n=kj$,\quad\qed
\te{Следствия из теоремы Лагранжа о группах}{\begin{enumerate}
    \item ${}^k\,\vrule\,_n;$
    \item ${}^j\,\vrule\,_n;$
    \item ${}^{O(a)}\,\vrule\,_n.$
\end{enumerate}    }

\dok 1,2 --- очевидно вытекает из теоремы. Чтобы доказать следствие 3, надо заметить, что $O(a)=|\lob{a}|,$ но $\lob a$ --- подгруппа
в \mr G, поэтому ${}^{O(a)}\,\vrule\,_n.$\quad\qed

Также можно заметить, что все группы простого порядка суть циклические: ведь если $|\rr G|=p$ --- простое число, то
$\forall a\ne e\ \ \lob a=\rr G$ по теореме Лагранжа, значит, $\rr G=\lob a.$ Также из теоремы можно заключить, что
количество левых и правых смежных классов по одной и той же подгруппе одинаково и равно $j=\frac nk.$

Понятно, что разбиения и отношения эквивалентности на множестве --- это одно и то же. Рассмотрим левое разложение группы по подгруппе.
Очевидно, что оно задает отношение эквивалентности $a\rr H=b\rr H\lra a^{-1}b\in \rr H.$ Действительно, $b=ah\lra a^{-1}b=h\in \rr H.$
Определим это отношение: $a\sim b\lra a^{-1}b\in \rr H.$ Оно является отношением эквивалентности:
\begin{enumerate}
    \item $a\sim a:\ \ a^{-1}a=e\in \rr H;$
    \item $a\sim b\ra b\sim a:\ \ a^{-1}b\in\rr H\ra b^{-1}a=(a^{-1}b)^{-1}\in\rr H;$
    \item $a\sim b,\ b\sim c\ra a\sim c:\ \ a^{-1}b\in\rr H,\ b^{-1}c\in\rr H\ra (a^{-1}b)(b^{-1}c)=a^{-1}c\in\rr H.$
\end{enumerate}

Можно сказать, что $a\sim b\lra a^{-1}b\in\rr H\lra a\rr H=b\rr H\lra a$ и $b$ лежат в одном смежном классе по подгруппе $\rr H.$
Таким образом, класс эквивалентности элемента $a$ --- это его левый смежный класс:
$[a]=a\rr H.$

\centerline{Конгруэнция на множествах с операцией}
\medskip

Рассмотрим множество с операцией $(\rr G, *)$ и рассмотрим отношение эквивалентности ''$\sim$'' на нем.
Это отношение называется конгруэнцией (то есть согласованным с операцией $*$), если из того, что $a\sim a'$ и $b\sim b'$
вытекает, что $a*b\sim a'*b'.$

\de Множество классов эквивалентности по отношению ''$\sim$'' называется фактормножеством по этому отношению и
обозначается $\rr G/_\sim$\,. И если при этом отношение ''$\sim$'' является конгруэнцией, то на фактормножестве $\rr G/_\sim$
можно ввести естественные операции, определяемые поэлементно: $[a]*[b]=[a*b].$
Если же взять произвольное отношение эквивалентности, естественная операция может быть некорректна: если $a\sim a'$ и $b\sim b',$ то
$[a]=[a'],\ [b]=[b']$, но не всегда $a*b\sim a'*b'$ (а при конгруэнции всегда), поэтому необязательно $[a]*[b]=[a*b]$
 будет совпадать с $[a']*[b']=[a'*b'].$

Свойства естественной операции: если исходная операция была ассоциативна, то ассоциативность переносится на операцию на фактормножестве.
Если исходное множество $\rr G$ имело нейтральный элемент и если всякий элемент исходного множества был обратим,
то и по отношению к естественной операции данные свойства будут сохраняться (это все очевидно следует из определения
естественной операции, докажем, например, ассоциативность: $([a]*[b])*[c]=[a*b]*c=[(a*b)*c]=[a*(b*c)]=[a]*[b*c]=[a]*([b]*[c])$, все остальное доказывается аналогично).

Значит, если $(\rr G,*)$ --- группа, то $(\rr G/_\sim,*)$ --- тоже группа. Конгруэнции на группе дают факторгруппы, и существует
другой способ описания конгруэнций --- на языке так называемых нормальных подгрупп.

\centerline{Примеры}
\medskip

Источники групп:
\begin{enumerate}
    \item Аддитивные группы классических колец: $(\mathbb Z,+),\ (\mathbb Q,+),\ (\mathbb R,+),\ (\mathbb C,+),\ (k_{n\times n},+)$ --- аддитивная группа кольца квадратных матриц над полем $k$;
    \item Мультипликативные группы классических полей: $\mathbb Q^*,\ \mathbb R^*,\ \mathbb C^*,\ ,$ также $\rr{GL}(n,k)$ и ее подгруппы;
    \item Группы симметрий и группы вращений тел: рассмотрим $\rr E^n$ --- точечное $n-$мерное евклидово пространство, и рассмотрим произвольное подмножество $M\subseteq\rr E^n.$ Рассмотрим движения $\rr E^n,$ отображающие
    $M$ на себя, они образуют подгруппу в группе движений $\rr E^n,$ которая называется группой симметрий множества $M$ и обозначается $\rr{Sym} M.$
    В этой группе можно выделить подгруппу собственных движений с детерминантом, равным единице, это будет группа вращений $M$, которая обозначается $\rr{Rot}M\subseteq\rr{Sym}M.$
\end{enumerate}

Если у множества есть гиперплоскость симметрии, то детерминант отражения относительно нее равен $-1$. Обозначим это отражение за $\sigma.$
Пусть $\alpha\in\rr{Sym}M,\ \det\alpha=-1\ra\sigma\alpha=\beta\in\rr{Rot}M\ra\sigma^2\alpha=\alpha=\sigma\beta\in\sigma\rr{Rot}M\ra$
$\rr{Sym}M=\rr{Rot}M\cup\sigma\rr{Rot}M.$

Группа симметрий правильного $n-$угольника называется группой диэдра и обозначается $\rr D_n.$
В $\rr D_n$ содержится $2n$ элементов, и если $b$ --- фиксированное отражение, $a$ --- поворот по часовой стрелке на угол $\frac{2\pi}n$ около центра многоугольника, то
$\rr D_n=\{a^0=e,a^1,\dots,a^{n-1},b,ab,\dots,a^{n-1}b\}$, других элементов нет, все присутствующие различны.\\ Таким образом,
$\rr D_n=\lob{a,b},\ a^n=b^2=e,\ \rr{Rot}M=\lob a_n,\ ab=ba^{n-1}=ba^{-1}.$
