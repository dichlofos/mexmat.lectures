\documentclass[a4paper]{article}
\usepackage[utf8]{inputenc}
\usepackage[russian]{babel}
\usepackage[simple]{dmvn}

\title{Программа спецкурса <<Алгебраическая геометрия\ теория инвариантов>> (1 год)}
\author{Лектоpы Д.\,А.\,Тимашёв, И.\,В.\,Аржанцев}
\date{Approx 2004~г.}
\begin{document}
\maketitle

\section*{Основы алгебраической геометрии [1], [2], [3], [4]}
\begin{items}{-2}
\item Аффинные алгебраические многообразия, определяющие идеалы, регулярные функции, морфизмы.
\item Теорема Гильберта о~нулях. Двойственность между категориями аффинных многообразий и~конечнопорождённых алгебр без нильпотентов.
\item Топология Зарисского, её нётеровость, разложение многообразия на~неприводимые компоненты.
\item Теорема об образе доминантного морфизма.
\item Рациональные функции и отображения.
\item Произведение многообразий.
\item Размерность. Теорема Крулля, теорема о~размерности слоев морфизма.
\item Касательные пространства и~отображения. Гладкие и~особые точки.
\item Конечные морфизмы. Нормальные многообразия.
\item Общее понятие алгебраического многообразия. Проективные многообразия, их полнота.
\end{items}

\section*{Алгебраические группы [2], [3], [5]}
\begin{items}{-2}
\item Понятие алгебраической группы. Связная компонента единицы. Гомоморфизмы
алгебраических групп (ядро и~образ замкнутые подгруппы).
\item Касательная алгебра Ли.
\item Действия алгебраических групп, локальная замкнутость орбит, представление в~алгебре регулярных функций.
\item Линеаризуемость аффинных алгебраических групп.
\item Однородные пространства, теорема Шевалле. Факторгруппы.
\item Разложение Жордана. Алгебраические торы. Унипотентные группы, теорема Энгеля.
\item Коммутант алгебраической группы. Разрешимые группы, теорема Бореля о~неподвижной точке, теорема Ли Колчина.
\item Редуктивные группы, полная приводимость представлений.
\end{items}

\section*{Теория инвариантов [3], [6], [7]}
\begin{items}{-2}
\item Теорема Гильберта об инвариантах. Категорный фактор аффинного многообразия
по~действию редуктивной группы, свойства морфизма факторизации.
\item Инварианты конечных групп, теорема Шевалле.
\item Рациональные инварианты, теорема Розенлихта о~разделении орбит инвариантами.
\item Замкнутые орбиты. Критерий Мацусимы. Стабильность действия, критерий Попова.
\item Нильпотентные орбиты, критерий Гильберта Мамфорда.
\item Присоединенное представление редуктивной группы, его орбиты и инварианты.
\item Классическая теория инвариантов систем тензоров.
\item Полустабильные и~стабильные точки проективного действия. Фактор Мамфорда.
\end{items}

\section*{Литература}
\begin{nums}{-2}
\item Шафаревич\,И.\,Р. Основы алгебраической геометрии. Т. 1. М., Наука, 1988.
\item Винберг\,Э.\,Б., Онищик А.\,Л. Семинар по группам Ли и~алгебраическим группам. М., Наука, 1988.
\item Крафт\,X. Геометрические методы в теории инвариантов. М., Мир, 1987.
\item Данилов\,В.\,Л. Алгебраические многообразия и~схемы. // Итоги науки и~техники.
Современные проблемы математики. Фундаментальные направления. Т. 23. М., ВИНИТИ, 1988.
\item Хамфри\,Дж. Линейные алгебраические группы. М., Наука, 1980.
\item Дьедонне\,Ж., Кэролл\,Дж., Мамфорд\,Д. Геометрическая теория инвариантов. М., Мир, 1974.
\item Спрингер\,Т.\,А. Теория инвариантов. М., Мир, 1981.
\end{nums}

\medskip\dmvntrail
\end{document}
