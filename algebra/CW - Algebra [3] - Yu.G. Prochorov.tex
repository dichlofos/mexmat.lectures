\documentclass[a4paper]{article}
\usepackage[utf8]{inputenc}
\usepackage[russian]{babel}
\usepackage[simple]{dmvn}

\begin{document}

\section*{Задачи с контрольной по высшей алгебре}
\subsection*{2003 год. IV семестр. Преподаватель Ю.\,Г.\,Прохоров}

\begin{problem}
Найти все силовские $p$-подгруппы в $\Sb_{p^2}$.
\end{problem}
\begin{solution}
Пусть $H$ силовская $p$ подгруппа. Очевидно, $|H|=p^{p+1}$. Отсюда следует, что любой
элемент $H$ либо $e$, либо имеет порядок вида~$p^k$.
А что дальше?..
\end{solution}

\begin{problem}
Найти коммутант группы движений правильного тетраэдра.
\end{problem}
\begin{solution}
Обозначим нашу группу движений через $G$. Очевидно, $G \cong \Sb_4$, поскольку при этих движениях
переставляются 4 вершины этого тетраэдра. Действительно, рассмотрим любую перестановку $\ph$ вершин 1, 2, 3, 4.
Покажем, что наш тетраэдр можно движением преобразовать так, что вершины совпадут со своими образами при
перестановке. Возьмём  первую вершину и поместим её в $\ph(1)$. Далее нам надо совместить движением $\ph(2)$,
$\ph(3)$ и $\ph(4)$ с 2, 3, 4 соответственно. Это несложно: все они уже находятся в одной плоскости (если
правильно разместить передвигаемый тетраэдр), и осталось только проверить ориентацию: если ориентация
треугольников (2,3,4) и $\ph(2),\ph(3),\ph(4)$ разная, то надо отразить тетраэдр, а если она одинаковая, то
они без труда совместятся поворотом вокруг оси $(1,C)$, где $C$ центр грани, противолежащей вершине 1.

Теперь перед нами более противная задача: надо найти $\Sb_4'$. Но мы знаем, что коммутант нормальная
подгруппа. Поскольку $[a,b]=aba^{-1}b^{-1}$, получаем, что любой коммутатор подстановок чётная
подстановка. Отсюда $\Sb_4' \subs \Ab_4$. Поскольку нормальная подгруппа обязана содержать вместе с
некоторой подстановкой все подстановки данного циклического строения, несложно подобрать несколько
подстановок так, чтобы их коммутаторы порождали всю $\Ab_4$. Например, если $a=(12)$ и $b=(234)$, то
$[a,b]=(124)$, т.е. один тройной цикл найден. Значит, коммутант содержит все тройные циклы. А они порождают
всю знакопеременную группу. Отсюда получаем ответ: $G' \cong \Ab_4$.
\end{solution}

\begin{problem}
Найти центр группы диэдра $\Db_n$.
\end{problem}
\begin{solution}
Мы знаем, что группа диэдра порождается симметрией $s$ и элементарным поворотом $r$. Имеют место соотношения
$r^n=\id$, $s=s^{-1}$ и $srs = r^{-1}$. Центр -- множество элементов, коммутирующих со всеми элементами
группы. Вся группа исчерпывается $2n$ элементами $\hc{r^k, sr^k | k = 0 \sco n-1}$. Очевидно, повороты
коммутируют с поворотами. Посмотрим, когда элементы вида $r^i$ коммутируют с $sr^k$. По определению, это
будет тогда и только тогда, когда $s r^k r^i r^{-k} s = r^i$, и, поскольку повороты коммутируют между собой,
получаем $sr^is=r^i$. Мы знаем, что в $D_n$ выполнено $srs=r^{-1}$, поэтому, домножая это равенство само на
себя $i$ раз, получаем $\ub{srs \cdot srs \sd srs}_{i~раз} = r^{-i}$, откуда,
поскольку $s^2=\id$, имеем $sr^is=r^{-i}$. Это означает, что для коммутирования необходимо равенство
$r^i=r^{-i}$. Спрашивается, часто ли такое бывает? Да не очень-то, поскольку в $D_{2m+1}$ это вообще верно
только при $i=0$, и тогда $Z(D_{2m+1})=\hc{e}$. Что касается чётного диэдра, то здесь есть ещё поворот на
угол $\pi$, т.е. попросту $r^{\frac{n}{2}}$. Очевидно, такой поворот коммутирует со всеми элементами. Итак,
ответ: $Z(D_{2m+1})=\hc{e}$, а $Z(D_{2m})=\hc{e, r^m}$.
\end{solution}

\begin{problem}
Дана простая группа $G$ порядка 168. Найти количество элементов порядка 7 в ней.
\end{problem}
\begin{solution}
Имеем $168=2^3\cdot3\cdot7$. Рассмотрим силовские 7-подгруппы: они имеют порядок 7 и пересекаются по $e$,
поскольку они циклические простого порядка. Заметим, что каждая из $G_7^{(i)}$ содержит ровно 6
элементов порядка 7 (все, кроме $e$). Найдём их количество:
$\verb"#" G_7 = \hc{1, 2, 4, 8, 3, 6, 12, 24}$.

Поскольку $\verb"#" G_7 \equiv 1 \pmod 7$, сразу не подходят случаи 2, 4, 3, 6, 12, 24. Остаются
только 1 и 8. Но и 1 не подходит, поскольку по условию группа $G$ -- простая, значит, в ней нет
нетривиальных нормальных подгрупп. Осталась единственная возможность -- 8, и она-то и реализуется
в действительности. Ежу ясно, что все элементы порядка 7 лежат в $G_7^{(i)}$. Для не-ежей поясняем:
если $O(a)=7$, то $\ha{a}_7$ -- силовская 7-подгруппа по определению. Ну, теперь можно написать
ответ: $6 \cdot8=48$ элементов порядка 7.
\end{solution}

\begin{problem}
Доказать, что группа $G$ порядка 80 не является простой.
\end{problem}
\begin{solution}
Имеем $80=2^4\cdot 5$. Следовательно, тут надо исследовать силовские $G_2$ и $G_5$. Начнём с
2 подгрупп. Поскольку количество силовских подгрупп есть индекс нормализатора, а подгруппа обязана
лежать в своём нормализаторе, получаем $\verb"#" G_2 = \hc{1, 5}$.

Первый случай нас устраивает, поскольку тогда силовская 2-подгруппа уже единственна и потому нормальна,
откуда $G$ -- не простая. Во втором случае всё не так просто,
и надо посмотреть на $G_5$. Имеем $\verb"#" G_5 \in \hc{1, 2, 4, 8, 16}$.
Здесь сразу не подходят случаи 2, 4, 8, поскольку $\verb"#" G_5 \equiv 1 \pmod 5$. Случай 1 нас устраивает, и
осталось выяснить, почему одновременно невозможны случаи $\verb"#" G_2 = 5$ и $\verb"#" G_5 = 16$. В самом
деле, если бы это было так, то, поскольку силовские 5-подгруппы имеют порядок 5, значит, они циклические и
пересекаются только по $e$, откуда количество элементов порядка 5 есть $4 \cdot 16=64$ шт. Но тогда на
элементы других порядков остаётся всего 16 мест, а силовская 2-подгруппа имеет порядок 16. Значит, она
обязана содержать их все, и потому она только одна. Полученное противоречие говорит о том, что силовская
подгруппа хотя бы одного типа единственна, а, значит, нормальна.
\end{solution}

\begin{problem}
Найти коммутант группы $G=\ha{A, B} \subs \GL_2(\Cbb)$, где
$A=\rbmat{\ep & 0 \\ 0 & \ep^{-1}}$, $B=\rbmat{0 & i \\ i & 0}$, а $\ep$ -- примитивный корень группы
$U_n$ нечётного порядка.
\end{problem}
\begin{solution}
<<Мы дорогих лекарств не употребляем>>, поэтому в лоб вычислим группу, а потом найдём коммутант.  По
определению, $G = \hc{a_1^{\ep_1}\sd a_s^{\ep_s}}$, где $a_i \in \hc{A, B}$, а $\ep_i = \pm 1$. Вынесем
из $B$ множитель $i$, тогда пусть $C := \frac{1}{i}B=(-i)B=\rbmat{0 & 1 \\ 1 & 0}$. Ясно, что множители $i$
можно вынести вперёд, тогда наше произведение перепишется в виде $G = \hc{(-i)^t \cdot
a_1^{\ep_1}\cdot\dots\cdot a_s^{\ep_s}}$, где $a_i \in \hc{A, C}$, а $t$ -- количество матриц $C$ в этом
произведении. Далее, поскольку $C^2=E$, можно считать, что в произведениях, образующих группу $G$, нет двух
подряд идущих матриц $C$ и все степени при матрицах $C$ равны 1. Далее, поскольку $A^m \cdot A^k=
\rbmat{\ep^{m+k} & 0 \\ 0 & \ep^{-(m+k)}}$, то можно считать, что все произведения выглядят так:
$(-i)^tA^{m_1}CA^{m_2}C\sd CA^{m_p}$, т.е. степени матриц $A$ перемежаются матрицами $C$, которых ровно
$t$ штук. Заметим, что матрица $C$ действует как перестановка столбцов умножаемой на неё матрицы. Фактически,
получив такое представление любого элемента $G$, несложно найти общий вид коммутанта. Заметим, что любой
коммутатор матриц из $G$ будет содержать чётное число матриц $C$ в своём выражении, а потому произойдёт
четное количество перестановок столбцов. Очевидно, все числовые множители при взятии коммутатора двух
элементов уничтожатся, поскольку элементы $\Cbb$ коммутируют по умножению. Это означает, что во всяком
случае коммутатор (если перемножить все его множители) будет представлен матрицей вида $A^{2m}$, что очевидно
следует из того, что каждая степень $\ep$ встретится в элементах коммутатора чётное число раз, и после
чётного числа перестановок столбцов матрица сохранит диагональный вид. Но заметим, что поскольку 2 взаимно
просто с порядком корня $\ep$ в $U_n$, матрицы $A$ с чётными степенями породят циклическую группу,
изоморфную, как несложно видеть, самой $U_n$.
\end{solution}

\begin{problem}
По мотивам предыдущей задачи: найти ещё и все одномерные комплексные представления этой группы $G$.
\end{problem}
\begin{solution}
Во-первых, найдём количество таких представлений. Оно в точности равно $\frac{|G|}{|G'|}$ по теореме  о
коммутанте. Заметим, что в качестве представлений имеет смысл рассматривать только такие гомоморфизмы $\rho:
G \to \Cbb^*$, которые убивают коммутант. Действительно, гомоморфный образ коммутант лежит в коммутанте образа,
а $\hr{\Cbb^*}'=\hc{e}$, поскольку $\Cbb^*$ -- абелева. (...)
\end{solution}

\begin{problem}
Описать все силовские подгруппы в группе $\Ab_5$.
\end{problem}
\begin{problem}
Описать все одномерные комплексные представления неабелевой группы $|G|=pq$.
\end{problem}
\begin{problem}
Описать все одномерные комплексные представления $\Ab_4$.
\end{problem}

\end{document}
