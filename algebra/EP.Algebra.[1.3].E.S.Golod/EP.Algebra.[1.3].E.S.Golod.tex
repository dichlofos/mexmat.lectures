\documentclass[a4paper]{article}
\usepackage[utf8]{inputenc}
\usepackage[russian]{babel}
\usepackage[simple]{dmvn}

\title{Программа экзаменов по высшей алгебре}
\author{Лектор Евгений Соломонович Голод}
\date{1, 3 семестры}

\begin{document}
\maketitle

\section*{1 семестр (2002 г.)}

\begin{nums}{0}
\item Элементарные преобразования. Приведение к ступенчатому виду. Метод Гаусса.
\item Множества и отображения. Композиция, обратное отображение, бином Ньютона. Бинарная операция, группа,
обобщенный закон ассоциативности.
\item Группа $\Sb_n$. Разложение перестановки на независимые циклы и транспозиции. Чётность перестановки.
\item Определитель. Свойство полилинейности. Неизменность при транспонировании. Свойство кососимметричности.
Поведение определителя при ЭП.
Вычисление посредством приведения к треугольной матрице. Критерий $\det=0$ в терминах ступенчатого вида.
\item Теорема и формулы Крамера. Случай однородной системы.
\item Определитель с углом нулей. Разложение определителя по строке/столбцу. Фальшивое разложение.
\item Определитель Вандермонда. Приложение к задаче интерполяции.
\item Линейная зависимость векторов и её свойства. Критерий $\det=0$ в терминах линейной зависимости.
\item Основная лемма о линейной зависимости. Базис системы векторов. Существование и свойства. Стандартный
базис в $\R^n$. Алгоритм поиска базиса
конечной системы векторов. Ранг системы векторов и его свойства.
\item Подпространства. Линейная оболочка. Свойства множеств решений СЛУ. Задание подпространства однородной
СЛУ. Размерность и базис
подпространства решений ОСЛУ. Фундаментальная система решений.
\item Операции над матрицами и их свойства. Поведение произведения при транспонировании.
\item Элементарные матрицы и их связь с ЭП. Представление невырожденной матрицы в виде произведения элементарных
матриц. Определитель произведения матриц. Аксиоматическое задание определителя.
\item Матричное уравнение $AX=B$. Обратная матрица, существование и единственность, способы вычисления.
\item Ранг матрицы. Совпадение его с рангом системы строк и столбцов. Ранг произведения матриц. Поведение ранга
при ЭП. Ранг ступенчатой матрицы.
Выражение ранга через миноры. Теорема о ранге матрицы.
\item Теорема Кронекера Капелли (Критерий совместности и определенности в терминах ранга матрицы). Выбор
главных и свободных неизвестных.
\item Степени элементов в полугруппе и группе. Порядок элемента и его вычисление в $\Sb_n$.
\item Кольцо и его свойства. Группа обратимых элементов. Делители нуля. Поле и его характеристика.
\item Построение поля $\Cbb$. Тригонометрическая форма комплексного числа. Формула Муавра.
\item Изоморфизм групп и колец. Существование и единственность $\Cbb$. Комплексное сопряжение изоморфизм.
\item Корни из единицы. Порядок элемента в циклической группе. Порождающие элементы. Изоморфизм циклических групп.
\item Кольца и поля классов вычетов. Малая теорема Ферма.
\item Кольцо многочленов от одной переменной над коммутативным кольцом. Старший член, степень, делители нуля,
обратимые элементы. Деление с остатком в $\K[x]$.
\item НОД в $\K[x]$ и в $\Z$. Существование и единственность для произвольных евклидовых областей. Взаимно
простые элементы в $\Z$ и в $\K[x]$. НОК.
\item Неприводимые множители в $\Z$ и в $\K[x]$. Однозначность разложения на на неприводимые множители.
Представление НОД в виде линейной комбинации $f$ и $g$. Китайская теорема об остатках.
\item Дифференцирование в $\K[x]$. Поведение кратности неприводимого множителя при дифференцировании.
Отделение кратных множителей.
\item Многочлен как функция. Теорема Безу. Кратность корня. Поведение кратности корня при дифференцировании.
Отделение кратных корней.
\item Число корней с учетом кратностей. Функциональное и алгебраическое равенство многочленов.
Интерполяционный многочлен Лагранжа.
\item Многочлены над $\Cbb$ как функции $f\cln \Cbb \ra \Cbb$. Лемма Даламбера. Алгебраическая замкнутость $\Cbb$.
\item Разложение многочленов над $\Cbb$. Число корней многочлена из $\Cbb[x]$ с учетом кратностей.
Граница для корней. Теорема Штурма.
\item Поле частных области целостности. Разложение дробей в сумму простейших дробей и многочлена.
\item Формулы Виета. Многочлены от $n$ переменных. Лексико-графическо-степенной порядок на мономах.
Старший член и его свойства. Симметрические многочлены. Результант и дискриминант.
\item Подгруппы циклических групп. Таблица Кэли и ее свойства. Теорема Кэли.
\item Разложение группы на смежные классы по подгруппе $H$. Теорема Лагранжа. Нормальные подгруппы.
Факторгруппы. Теорема о гомоморфизме.
\item Факториальность $\K[x]$. Лемма Гаусса и ее следствия. Признак неприводимости Эйзенштейна.
\item Алгебраические и целые алгебраические числа.
\end{nums}
\pagebreak

\section*{3 семестр (2005 г.)}

\begin{nums}{-1}
\item
Понятие факторгруппы. Структура гомоморфизма групп. Теорема о
гомоморфизме.
\item
Связь между подгруппами в группе и факторгруппе. Изоморфизм
$\frac{G/N}{H/N}\cong G/H$.
\item
Произведения подгрупп. Нормализатор подгруппы. Изоморфизм
$\frac{HK}{H}\cong \frac{K}{K\cap H}$.
\item
Центр группы. Факторгруппы по центру.
\item
Группа автоморфизмов. Внутренние автоморфизмы. Группа
автоморфизмов циклической группы.
\item
Прямое произведение групп. Изоморфизм $\frac{G_1\st G_n}{H_1\st
H_n}\cong (G_1/H_1)\st (G_n/H_n)$.
\item
Порядок элемента в прямом произведении. Прямое произведение
циклических групп.
\item
Системы порождающих в группах. Примеры $(\mathbf{S}_n,
\mathbf{A}_n)$.
\item
Теорема о приведении целочисленной матрицы к диагональному виду.
\item
Свободные абелевы группы. Конечно порожденные абелевы группы как
факторгруппы свободных групп. Теорема о согласованных базисах для
конечно порожденной абелевой группы и ее подгруппы. Разложения
конечно порожденной абелевой группы и ее подгруппы в прямую сумму
бесконечных и примарных циклических (существование).
\item
Теорема единственности разложения конечно порожденной абелевой
группы в прямую сумму бесконечных и примарных циклических.
\item
Конечные подгруппы мультипликативной группы поля. Классификация
конечных абелевых групп с точностью до изоморфизма.
\item
Дискретные подгруппы в конечномерном вещественном векторном
пространстве.
\item
Коммутаторы и коммутанты, их свойства.
\item
Понятие разрешимой группы. Связь разрешимости группы с
разрешимостью ее подгрупп и факторгруппы.
\item
Разрешимость группы невырожденных верхних треугольных матриц.
\item
Понятие классов сопряженных элементов. Классы сопряженных
элементов в $\mathbf{GL}_n(\mathbb{K}), \mathbb{K}\subset\Cbb$ и
$\mathbf{S}_n$.
\item
Классы сопряженных элементов в $\mathbf{A}_n$.
\item
Понятие простой группы. Простота группы $\mathbf{A}_n$, $n\ge 5$.
\item
Простота группы $\mathbf{SO}_3$.
\item
Действие группы н множестве (представление группы перестановками).
Теорема Кэли. Орбиты, стабилизаторы, неподвижные точки. Число
элементов в орбите. Классы сопряженных элементов и классы
сопряженных подгрупп как орбиты.
\item
Транзитивное действие. Эквивариантные отображения и изоморфизмы действий.
\item
Конечные $p$-группы. Центр и разрешимость конечных $p$-групп.
Группы порядка $p^2$.
\item
Полупрямое произведение групп. Примеры групп порядка $p^3$.
\item
Силовские $p$-подгруппы. Существование $p$-подгрупп (первая
теорема Силова).
\item
Теорема о сопряженности и число силовских $p$-подгрупп (вторая и
третья теоремы Силова).
\item
Группы порядка $pq$ ($p$, $q$ простые числа).
\item
Идеалы (левые, правые, двусторонние) в кольце. Структура
гомоморфизма кольца. Факторкольцо. Теорема о гомоморфизме для
колец. Соответствие между подкольцами, идеалами в факторкольце и
исходном идеале. Изоморфизм $\frac{K+I}{I}\cong \frac{K}{I\cap
K}$.
\item
Главные левые (правые) идеалы. Тело как кольцо с единицей без
нетривиальных левых идеалов. Коммутативные кольца главных идеалов
(примеры).
\item
Понятие простого кольца. Идеалы в кольце квадратных матриц над
кольцом с единицей.
\item
Понятие алгебры над полем. Случай алгебры с единицей. Гомоморфизм
алгебр, идеалы в алгебре. Конечномерные алгебры, структурные
константы. Конечномерные алгебры без делителей нуля.
\item
Факторалгебра алгебры многочленов от одной переменной над полем.
Когда она является полем? Поле комплексных чисел как такая
факторалгебра.
\item
Существование расширение поля, в котором данный многочлен
разлагается на линейные множители.
\item
Подалгебра, порожденная одним элементом алгебры. Алгебраические и
и трансцендентные элементы. Минимальный многочлен алгебраического
элемента. Случай алгебры с делением (тела).
\item
Расширение полей. Простое алгебраическое расширение. Размерность
башни полей. Алгебраические и трансцендентные числа. Поле
алгебраических чисел.
\item
Конечные поля: число элементов, существование конечных полей,
существование неприводимых многочленов заданной степени.
\item
Конечные поля: единственность конечного поля данного порядка.
\item
Построение алгебры кватернионов.
\item
Конечномерные алгебры с делением над полями комплексных и
действительных чисел (теорема Фробениуса).
\item
Понятие модуля. Подмодули, гомоморфизм модулей, фактормодуль,
теорема о гомоморфизме, связь между подмодулями в модуле и
фактормодуле. Прямая сумма модулей.
\item
Произведение левого идеала и модуля. Произведение и степень левых
идеалов. Свойство левого идеала, являющегося прямым слагаемым в
кольце. Прямая сумма колец.
\item
Модули над алгеброй (линейные представления алгебр). Интерпретация
на языке линейных операторов и матриц. Изоморфизм
(эквивалентность) линейных представлений. Линейные представления
конечных групп как модули над групповой алгеброй. Регулярное
представление алгебры (группы).
\item
Простые модули (неприводимые представления). Интерпретация
приводимости на языке инвариантных подпространств. Аннуляторы
модуля $n$ элементов модуля. Циклические модули. Простые модули
как фактормодули кольца по максимальному левому идеалу.
\item
Алгебра эндоморфизмов модуля. Лемма Шура. Неприводимые
представления коммутативной алгебры (группы) над полем комплексных
чисел.
\item
Полупростые модули (вполне приводимые представления), их
характеризация. Полупростота подмодулей и фактормодулей
полупростого модуля.
\item
Полупростые алгебры, полупростота модулей над ними. Вхождение
любого неприводимого представления в регулярное. Однозначная
определенность кратностей вхождения простых модулей в разложение
полупростого модуля.
\item
Характеризация полупростых алгебр в терминах нильпотентных левых
идеалов.
\item
Характеры линейных представлений. Сепарабельные алгебры и их
полупростота. Теорема Машке.
\item
Ортогональные и унитарные представления групп.
\item
Теорема о кратности вхождения неприводимого комплексного
представления в регулярное.
\item
Разложение полупростой алгебры над полем комплексных чисел в
прямую сумму матричных алгебр. Число неприводимых комплексных
представлений.
\item
Одномерные комплексные представления групп.
\item
Неприводимые комплексные представления групп $\mathbf{S}_3$,
$\mathbf{S}_4$, $\mathbf{A}_4$, $\mathbf{D}_n$, $\mathbf{Q}_8$.
\item
Характеры неприводимых комплексных представлений.
\end{nums}

\pagebreak

\section*{3 семестр (2003 г.)}

\begin{nums}{-3.5}
\item Нормальные подгруппы, факторгруппы, теорема о гомоморфизме. Соответствие между подгруппами
в факторгруппе и исходной группе. Теорема об изоморфизме для факторгрупп.
\item Произведение подгрупп. Изоморфизм $HK/K \cong H/(H \cap K)$. Группа автоморфизмов. Внутренние автоморфизмы, центр группы.
Классы сопряженных элементов. Сопряженные элементы в $\Sb_n$ и $\Ab_n$.
\item Свободная группа. Задание группы порождающими и соотношениями. Прямое произведение групп.
Прямое произведение циклических групп. Приведение целочисленной матрицы к каноническому виду.
\item Свободные абелевы группы и их подгруппы. Строение конечно порожденных абелевых
групп. Теорема единственности для разложения конечно порожденной  абелевой группы в
прямую сумму бесконечных и примарных циклических. Классификация конечных
абелевых групп.
\item Конечные подгруппы мультипликативной группы поля. Дискретные подгруппы в $\R^n$. Нормальные ряды. Теорема Жордана Гёльдера. Разрешимые группы. Коммутант. Примеры.
Простота группы $\Ab_n$ при $n \ge 5$. Простота группы $\SO_3$.
\item Действие групп. Орбиты, стабилизаторы. Приложение к классам сопряженных элементов
и подгрупп.
\item $p$-группы, их центр, разрешимость. Группы порядка $р$ и $р^3$. Три теоремы Силова. Группы порядка $pq$.
\item Гомоморфизмы колец. Идеалы, факторкольца. Теорема о гомоморфизме для колец.
Соответствие между подкольцами, идеалами в кольце н факторкольце. Изоморфизм
$(K + I)/I \cong K/(K \cap I)$.
\item Простые кольца. Идеалы в кольце квадратных матриц. Кольца главных идеалов. Примеры.
\item Понятие алгебры над полем. Конечномерные алгебры. Структурные константы. Неделители
нуля в конечномерной алгебре с единицей.
\item Факторкольцо кольца многочленов от одной переменной над полем. Существование
расширения поля, в котором данный многочлен имеет корень. Простые расширения поля.
Алгебраические и трансцендентные элементы. Размерность башни полей.
Поле разложения многочлена.
\item Конечные поля (число элементов, существование, неприводимые многочлены, единственность).
\item Алгебры с делением (над полем комплексных чисел). Алгебра кватернионов. Теорема Фробениуса.
\item Понятие модуле над кольцом (алгеброй): подмодули, гомоморфизмы, фактормодули.
\item Теорема о гомоморфизме для модулей, прямая сумма модулей. Соответствие между подмодулями в модуле и его фактормодуле.
Простые (неприводимые) модули. Теорема Жордана Гёльдера для модулей.
\item Система порождающих модуля. Всякий простой модуль циклический. Свободные модули.
Представление конечно порожденного модуля как фактормодуля свободного.
\item Конечнопорожденные модули над кольцом многочленов от одной переменной над полем.
Приложение к теореме о жордановой форме.
\item Прямое произведение колец (алгебр), его центр. Модули над прямым произведением колец (алгебр).
\item Полупростые модули (эквивалентные условия). Полупростота подмодуля и фактормодуля полупростого модуля.
Единственность разложения полупростого модуля в прямую сумму простых.
\item Полупростые алгебры. Всякий конечномерный модуль над полупростой алгеброй полупрост. Всякий простой модуль над полупростой алгеброй изоморфен минимальному левому идеалу.
\item Конечномерная простая алгебра является полупростой. Разложение полной матричной алгебры в прямую сумму минимальных левых идеалов.
\item Кольцо эндоморфизмов модуля. Лемма Шура. Матричное описание кольца эндоморфизмов прямой суммы модулей.
Полупростая алгебра над полем $\Cbb$ является прямым произведением полных матричных алгебр.
\item Следствия из теоремы о строении полупростых алгебр над полем $\Cbb$ (сумма квадратов размерностей неприводимых модулей, центр алгебры и число неприводимых модулей).
\item Неприводимые модули над коммутативной алгеброй над полем $\Cbb$.
\item Понятие линейного представления группы. Гомоморфизм и изоморфизм (эквивалентность)
линейных\break представлений. Приводимые, неприводимые и вполне приводимые представления.
Описание этих понятий на языке матриц.
\item Связь между линейными представлениями группы и модулями над ее групповой
алгеброй. Теорема Машке.
\item Унитарные и ортогональные представления. Число неприводимых представлений группы над полем $\Cbb$, кратность их
вхождения в регулярное представление, сумма квадратов размерностей.
\item Неприводимые представления конечной абелевой группы над полем $\Cbb$
и разложение ее регулярного представления в прямую сумму неприводимых. Одномерные представления группы и их число над полем $\Cbb$.
\item Описание всех неприводимых комплексных представлений группы $\Sb_3$, $\Db_n$ и $\Qb_8$.
\item Описание всех неприводимых комплексных представлений групп $\Ab_4$ и $\Sb_4$.
\item Связь между неприводимыми комплексными представлениями конечной группы и её
нормальной подгруппы простого индекса. Характеры линейных представлений.
\end{nums}

\medskip\dmvntrail


\end{document}
