\documentclass[a4paper]{article}
\usepackage[simple]{dmvn}

\title{Программа экзамена по высшей алгебре}
\author{Лектор\т Эрнест Борисович Винберг}
\date{3 семестр, 2004 г.}
\begin{document}
\maketitle
\begin{nums}{-1}
\item Отношения эквивалентности в группе, согласованные с операцией. Нормальные подгруппы. Факторгруппа.
\item Ядро и образ гомоморфизма групп. Теорема о гомоморфизме.
\item Разложение группы в прямое произведение подгрупп. Случай двух множителей.
\item Внешнее прямое произведение групп, его связь с внутренним. Факторгруппа прямого
      произведения групп по произведению нормальных подгрупп множителей.
\item Свободные (конечнопорожденные) абелевы группы. Связь между различными базисами.
\item Свобода и ранг подгруппы свободной абелевой группы.
\item Дискретные подгруппы группы $\Eb^n$.
\item Кристаллографические группы. Возможные порядки осей симметрии кристалла.
\item Существование базиса свободной абелевой группы, согласованного с подгруппой.
\item Фундаментальный параллелепипед решетки, его объем. Индекс подрешетки как отношение
      объёмов фундаментальных параллелепипедов.
\item Разложение конечнопорождённой абелевой группы в прямую сумму циклических групп.
\item Разложение циклической группы в прямую сумму примарных циклических групп.
\item Разложение конечнопорождённой абелевой группы в прямую сумму примарных циклических групп, его единственность.
\item Экспонента конечной абелевой группы. Критерий цикличности. Цикличность мультипликативной группы конечного поля.
\item Действия групп. Ядро неэффективности. Орбиты и стабилизаторы. Стабилизаторы эквивалентных точек.
\item Действие группы на себе сопряжениями. Центр группы и централизатор элемента. Классы сопряженности.
\item Классы сопряженности в группах $\Sb_n$ и $\Ab_n$.
\item Нетривиальность центра примарной конечной группы. Группы порядка $p^2$.
\item Действие группы на множестве смежных классов по подгруппе. Описание всех транзитивных действий
      с точностью до изоморфизма.
\item Силовские подгруппы конечной группы. Силовские подгруппы группы $\Ab_5$.
\item Первая теорема Силова.
\item Вторая теорема Силова.
\item Третья теорема Силова.
\item Группы порядка~$45$ и группы порядка~$pq$.
\item Разложение группы в полупрямое произведение подгрупп. Разложение в полупрямое произведение
      группы~$\Sb_4$ и группы аффинных преобразований.
\item Порождающие групп~$\Sb_n$ и $\Ab_n$.
\item Порождающие групп $\GL_n(K)$ и $\SL_n(K)$.
\item Коммутант группы, его свойства.
\item Коммутанты групп $\Sb_n$ и $\Ab_n$.
\item Коммутанты групп $\GL_n(K)$ и $\SL_n(K)$.
\item Разрешимые группы. Разрешимость всякой примарной конечной группы.
\item Разрешимость группы треугольных матриц.
\item Простота группы $\Ab_n$ при $n\ge5$.
\item Линейные и матричные представления групп, связь между ними. Инвариантные подпространства. Неприводимые представления. Примеры.
Примеры двумерных представлений группы $(\R,+)$.
\item Вполне приводимые линейные представления групп. Сумма представлений. Разложение вполне приводимого
представления в сумму неприводимых представлений. Полная приводимость линейных представлений конечных групп.
\item Изоморфизм (эквивалентность) представлений. Построение двух неизоморфных неприводимых трехмерных
комплексных линейных представлений группы $\Sb_4$.
\item Мономиальное представление группы $\Sb_n$, его разложение в сумму неприводимых представлений.
\item Лемма Шура. Одномерность неприводимых комплексных линейных представлений абелевых групп.
\item Линейные представления конечных абелевых групп.
\item Одномерные линейные представления групп.
\item Линейные представления группы $\Db_n$.
\item Линейные представления группы $(\R,+)$ (однопараметрические группы линейных
      операторов) как решения операторного дифференциального уравнения.
\item Экспонента линейного оператора (матрицы). Описание однопараметрических групп линейных операторов
      в терминах экспонент.
\item Отношения эквивалентности в кольце, согласованные с операциями. Идеалы. Факторкольцо.
\item Ядро и образ гомоморфизма колец. Теорема о гомоморфизме.
\item Идеалы и факторкольца евклидова кольца.
\item Присоединение к полю корня неприводимого многочлена.
\item Число элементов конечного поля. Построение поля из $p^2$ элементов.
\end{nums}

\medskip\dmvntrail
\end{document}
