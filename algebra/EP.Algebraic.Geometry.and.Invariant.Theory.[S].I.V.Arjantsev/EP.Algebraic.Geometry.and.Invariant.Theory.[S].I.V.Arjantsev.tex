\documentclass[a4paper]{article}
\usepackage[utf8]{inputenc}
\usepackage[russian]{babel}
\usepackage[simple]{dmvn}



\title{Программа спецкурса <<Введение в алгебраическую геометрию и теорию инвариантов>>}
\author{Лектоp И.\,В.\,Аржанцев}
\date{2005--2006 г.}

\begin{document}
\maketitle

\subsection*{Часть 1. Алгебраическая геометрия}

\begin{nums}{0}
\item Аффинные алгебраические подмногообразия. Примеры. Идеал подмногообразия.
      Теорема Гильберта о базисе. Алгебра регулярных функций. Теорема Гильберта о нулях.
\item Морфизмы аффинных подмногообразий. Аффинные многообразия.
\item Топология Зарисского. Главное открытое подмножество. Спектр аффинной алгебры.
\item Нётерово топологическое пространство. Однозначность разложения на неприводимые компоненты.
\item Произведение аффинных многообразий. Алгебра регулярных функций на произведении.
\item Рациональные функции. Область определения и идеал знаменателей. Локализация алгебры.
\item Морфизмы: доминантные, конечные, замкнутые вложения, рациональные и простые накрытия.
      Лемма Накаямы и детерминантный трюк. Лемма Нётер о нормализации.
      Разложение доминантного морфизма в композицию рационального накрытия и проекции.
\item Размерность многообразия. Теорема Крулля о гиперповерхности. Высота простого идеала.
      Теорема о размерности слоёв морфизма. Теорема Шевалле о полунепрерывности.
\item Рациональные морфизмы. Теорема о факторизации морфизма. Теорема об образе морфизма.
\item Касательное пространство. Дифференциал морфизма. Гладкие и особые точки.
\item Пучки функций и абстрактные алгебраические предмногообразия.
\item Проективные и квазипроективные многообразия. Примеры: многообразия Грассмана и многообразие флагов.
\item Произведение предмногообразий. Аксиома отделимости и алгебраические многообразия.
      Отделимость квазипроективных многообразий.
\item Полные многообразия. Полнота проективных многообразий.
\item Нормальные многообразия. Теорема об устранении особенности. Дивизоры на многообразиях.
      Дивизор рациональной функции.
\end{nums}

\subsection*{Часть 2. Алгебраические группы}

\begin{nums}{0}
\item Алгебраические группы. Определение и примеры. Замкнутые подгруппы. Теорема о компонентах связности.
Замкнутость ядра и образа при гомоморфизме групп.
\item Действие алгебраической группы. Теорема о размерности орбит. Стабилизаторы.
\item Теорема о замкнутом эквивариантном вложении. Существование точного линейного представления.
\item Однородные пространства. Теорема Шевалле о квазипроективности. Факторгруппа алгебраической группы.
\item Касательная алгебра. Свойства функтора Ли.
\item Разложение Жордана в алгебраической группе и её касательной алгебре.
\item Алгебраические торы: полная приводимость представлений и решётки характеров. Квазиторы.
\item Разрешимые группы.  Теорема Бореля о неподвижной точке. Теорема Ли Колчина. Унипотентные группы.
\item Максимальные торы и подгруппы Бореля: теорема о сопряжённости. Проективность обобщённого многообразия полных флагов.
\item Редуктивные группы: пять эквивалентных определений. Строение связных редуктивных групп.
\end{nums}

\subsection*{Часть 3. Теория инвариантов}

\begin{nums}{0}
\item Теорема Гильберта об инвариантах. Оператор Рейнольдса.
\item Морфизм факторизации для действия редуктивной группы на аффинном многообразии. Его свойства. Критерий Игусы.
Категорный фактор.
\item  Рациональные инварианты. Теорема Розенлихта. График действия.
\item Инварианты конечных групп. Геометрический фактор. Теорема Шевалле Шепарда Тодда.
\item Классическая теория инвариантов: постановка задачи, поляризация, полный набор типовых инвариантов,
первая основная теорема для классических групп.
\item Присоединённое представление: нильпотентные и полупростые элементы, лемма Ричардсона, группа Вейля, теорема Шевалле об инвариантах.
\end{nums}

\medskip\dmvntrail
\end{document}
