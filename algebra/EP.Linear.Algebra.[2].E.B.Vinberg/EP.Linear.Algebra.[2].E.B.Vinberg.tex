\documentclass[a4paper]{article}
\usepackage[utf8]{inputenc}
\usepackage[russian]{babel}
\usepackage[simple]{dmvn}

\title{Программа экзамена по линейной алгебре}
\author{Лектор Э.\,Б.\,Винберг}
\date{II семестр, 2003 г.}

\begin{document}

\maketitle
\begin{nums}{-1}
\item Базис и размерность векторного пространства.
\item Преобразования координат в векторном пространстве.
\item Подпространства как множества решений систем однородных линейных уравнений.
\item Связь между размерностями суммы и пересечения двух подпространств.
\item Линейная независимость подпространств. Базис и размерность прямой суммы.
\item Линейные отображения, их запись в координатах. Образ и ядро линейного отображения, связь между их размерностями.
\item Линейные функции, их запись в координатах. Сопряжённое пространство и сопряжённые базисы.
\item Канонический изоморфизм $V \cong V^{**}$ при $\dim V < \infty$.
\item Билинейные функции, их запись в координатах. Изменение её матрицы при переходе к другому базису.
\item Ортогональное дополнение к подпространству относительно симметрической или кососимметрической билинейной функции.
\item Связь между симметрическими билинейными и квадратичными функциями. Существование ортогонального базиса
для симметрической билинейной функции.
\item Нормальный вид вещественной квадратичной функции. Закон инерции.
\item Процесс ортогонализации. Нахождение индексов инерции квадратичной функции методом Якоби.
\item Критерий Сильвестра.
\item Существование симплектического базиса для кососимметрической билинейной функции.
\item Евклидовы пространства. Длина вектора и угол между векторами.
\item Матрица и определитель Грама системы векторов евклидова пространства.
\item Ортонормированные базисы евклидова пространства и ортогональные матрицы.
\item Расстояние от вектора до подпространства в евклидовом пространстве, его выражение через определители Грама.
\item Объем параллелепипеда в евклидовом пространстве (две формулы).
\item Полуторалинейные функции, их запись в координатах. Изменение матрицы полуторалинейной функции при переходе к другому базису.
Эрмитовы и косоэрмитовы полуторалинейные функции, связь между ними.
\item Нормальный вид эрмитовой функции. Закон инерции.
\item Эрмитовы пространства. Ортонормированные базисы эрмитова пространства и унитарные матрицы.
\item Линейные операторы, их запись в координатах. Изменение матрицы линейного оператора при переходе к другому базису. Ранг и
определитель линейного оператора. Невырожденные линейные операторы.
\item Собственные векторы и собственные значения линейного оператора.
\item Собственные подпространства линейного оператора, их свойства. Достаточное условие существования собственного базиса.
\item Инвариантные подпространства линейного оператора. Существование одномерного или двумерного инвариантного
подпространства для линейного оператора в вещественном векторном пространстве.
\item Связь между линейными операторами и билинейными (полуторалинейными) функциями в евклидовом (эрмитовом)
пространстве. Сопряженные операторы.
\item Существование ортонормированного собственного базиса для симметрического оператора. Приведение квадратичной функции в
евклидовом пространстве к каноническому виду.
\item Приведение к каноническому виду матрицы ортогонального оператора.
\item Существование ортонормированного собственного базиса для эрмитова (унитарного) оператора.
\item Полярное разложение невырожденного линейного оператора в евклидовом (эрмитовом) пространстве.
\item Корневые подпространства линейного оператора и разложение пространства в их прямую сумму.
\item Нильпотентные операторы. Разложение пространства в прямую сумму циклических подпространств нильпотентного оператора.
\item Приведение матрицы линейного оператора к жордановой форме. Минимальный многочлен. Теорема Гамильтона Кэли.
Критерий существования собственного базиса.
\item Аффинные пространства. Векторизация. Аффинные системы координат. Барицентрические линейные комбинации точек.
\item Плоскости аффинного пространства, их задание системами линейных уравнений. Аффинная оболочка системы точек.
\item Взаимное расположение плоскостей в аффинном пространстве.
\item Выпуклые множества. Выпуклая оболочка системы точек. Симплексы.
\item Полупространства. Выпуклые многогранники, их грани. Грани симплекса и параллелепипеда.
\item Теорема о том, что всякий ограниченный выпуклый многогранник есть выпуклая оболочка своих вершин. Задача линейного программирования.
\item Аффинные отображения, их свойства. Аффинные преобразования. Существование и единственность аффинного преобразования,
переводящего один заданный репер в другой. Координатный признак равенства фигур в аффинной геометрии.
\item Дифференциал как гомоморфизм аффинной группы в линейную. Параллельные переносы и гомотетии.
\item Квадрики в аффинном пространстве. Центральные, конические и цилиндрические квадрики.
\item Аффинная классификация невырожденных вещественных квадрик.
\item Евклидовы аффинные пространства. Расстояние между точками и между плоскостями.
\item Движения. Дифференциал как гомоморфизм группы движений в ортогональную группу. Собственные и несобственные движения.
\item Ось движения. Геометрическое описание движений плоскости и трехмерного пространства.
\item Прямоугольные системы координат в евклидовом аффинном пространстве. Свойство максимальной подвижности и координатный
признак равенства фигур в евклидовой геометрии.
\item Приведение уравнения невырожденной квадрики в евклидовом пространстве к каноническому
виду (без доказательства единственности в случае параболоида).
\item Проективные пространства, их аффинные карты. Однородные и неоднородные координаты.
\item Плоскости в проективном пространстве, их взаимное расположение.
\item Проективные преобразования. Существование и единственность проективного преобразования $n$-мерного проективного пространства,
переводящего одну заданную систему $n+2$ точек общего положения в другую.
\item Двойное отношение четверки точек, лежащих на одной прямой. Его инвариантность при проективных преобразованиях.
\item Квадрики в проективном пространстве, их аффинные изображения. Проективная классификация
невырожденных вещественных квадрик, ее сопоставление с аффинной классификацией.
\item Векторные модели сферической и гиперболической геометрий. Плоскости, расстояние между точками и движения в этих моделях.
\item Свойство максимальной подвижности в сферической и гиперболической геометриях.
\item Сумма углов сферического и гиперболического треугольника.
\item Тензорная алгебра векторного пространства.
\item Симметрическая алгебра векторного пространства ($\Char K = 0$), ее связь с алгеброй многочленов.
\item Внешняя алгебра векторного пространства (над полем нулевой характеристики).
\item Разложимые поливекторы и подпространства векторного пространства.
\item Канонический вид и критерий разложимости бивектора.
\item Определитель как единственная кососимметрическая $n$-линейная функция в $n$-мерном пространстве.
\end{nums}

\medskip\dmvntrail
\end{document}
