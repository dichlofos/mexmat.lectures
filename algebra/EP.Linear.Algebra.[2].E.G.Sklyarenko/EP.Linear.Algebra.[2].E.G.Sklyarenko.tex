\documentclass[a4paper]{article}
\usepackage[simple]{dmvn}

\title{Программа экзамена по линейной алгебре}
\author{Лектор\т \framebox{Е.\,Г.\,Скляренко}}
\date{II семестр, 2003 г.}

\begin{document}
\maketitle

\begin{nums}{-2}
\item Векторные пространства, примеры. Подпространства
 и линейные подмногообразия. Факторпространства.
\item Линейная зависимость и независимость векторов. Свойства
разложения векторов по линейно независимым системам, признаки зависимости векторов.
\item Ранг множества векторов, его свойства. Размерность, базисы, координаты.
Размерность факторпространства.
\item Теоремы об изоморфизме векторных пространств. Изоморфизм
факторпространства дополнительному пространству.
\item Суммы и пересечения подпространств. Размерность суммы.
\item Прямые суммы двух и более подпространств. Внешняя прямая сумма.
\item Аффинные пространства, их однородность. Координатный репер.
 Координаты точек и векторов.
\item Переход к новым координатам: преобразование координат точек и векторов.
Формулы и матрицы перехода.
\item Подпространства аффинного пространства, способы задания. Параллелепипеды.
Подпространства как множества решений систем линейных уравнений.
\item Взаимное расположение двух подпространств в аффинном пространстве (пересечение,
параллельность, скрещиваемость).
\item Деление отрезка в заданном отношении, координаты делящей точки.
\item Теоремы об изоморфизме аффинных пространств. Эквивалентность понятий аффинного
и векторного пространства.
\item Скалярное произведение векторов. Евклидовы векторные и точечные пространства.
Линейная независимость ортогональных векторов. Процесс ортогонализации. Ортонормированные базисы и реперы.
\item Теоремы об изоморфизме для евклидовых векторных и точечных пространств.
Следствия (длина вектора и расстояние между точками, неравенство треугольника, угол между
векторами, перпендикулярность ортогональных векторов, теорема Пифагора, неравенство Коши\ч
Буняковского). Примеры. Изоморфизмы как изометрии.
\item Ортогональное дополнение к векторному подпространству, его свойства. Угол между вектором
и подпространством, расстояние от вектора до подпространства.
\item Проектирование вектора на подпространство. Коэффициенты Фурье. Метод наименьших квадратов.
\item Нормальный вектор к гиперплоскости, расстояние от точки до гиперплоскости и
между параллельными гиперплоскостями.
\item Объём $k$\д мерного параллелепипеда и определитель Грама, свойства определителя Грама.
\item Линейные отображения векторных пространств, координатная запись. Векторное пространство линейных отображений.
Композиция линейных отображений, матрица композиции.
\item Ядро и образ линейного отображения, их размерности. Условия изоморфизма.
\item Линейные операторы, координатная запись. Зависимость матрицы от выбора базиса
(в том числе в тензорной записи).
\item Кольцо линейных операторов, изоморфизм кольцу матриц. Условия невырожденности.
\item Инвариантные подпространства, влияние на вид матрицы. Инвариантные подпространства над полями действительных и
комплексных чисел. Ступенчатый вид матрицы оператора.
\item Собственные векторы и значения оператора. Собственные подпространства, в том числе для различных
собственных чисел.
\item Характеристический многочлен. Кратность корня и размерность собственного подпространства.
Условия диагонализируемости матрицы оператора. Свойства жордановых клеток.
\item Существование и единственность жордановой (нормальной) формы матрицы
оператора над алгебраически замкнутым полем.
\item Корневые подпространства, свойства и связь с жордановой формой.
\item Аннулирующие многочлены, минимальный многочлен. Теорема Гамильтона\ч Кэли.
\item Комплексификация пространства и линейного отображения. Нормальная форма матрицы
вещественного оператора.
\item Матричная интерпретация рекуррентных формул и жорданова форма.
\item Линейные отображения и преобразования аффинных пространств, аналитическая запись.
Частные случаи. Аффинные преобразования и понятие об аффинной классификации.
\item Пространство линейных функций. Дуальный (сопряжённый) базис.
 Переход к другому базису и преобразование координат линейных функций.
\item Взаимная сопряжённость пространства и пространства линейных функций на нём.
\item Дуальные линейные отображения и операторы в пространствах линейных функций.
Взаимная дуальность с основными отображениями.
\item Билинейные функции, их координатная запись. Полуторалинейные функции.
Базисные функции, отвечающие базису подпространства.
\item Зависимость матрицы билинейной (полуторалинейной) функции от выбора базиса
(в том числе в тензорной записи).
\item Билинейные функции и линейные отображения векторного пространства в пространство линейных функций.
\item Симметричные, кососимметричные и эрмитовы функции. Совпадение ядер.
\item Ортогональность векторов и подпространств относительно симметричных, кососимметричных
и эрмитовых функций. Свойства ортогональных дополнений.
\item Нормальный вид симметричных, кососимметричных и эрмитовых функций.
\item Единственность нормального вида, симметрической кососимметрической и эрмитовой функций.
Закон инерции для симметричных и эрмитовых функций.
\item Квадратичные функции и их связь с билинейными, нормальный вид.
Приведение к нормальному виду выделением полных квадратов.
\item Метод Грама приведения к нормальному виду и теорема Якоби.
\item Положительно определённые симметричные и эрмитовы функции.
Критерий Сильвестра.
\item Симметричные, антисимметричные и эрмитовы скалярные произведения. Псевдоевклидовы,
эрмитовы и симплектические векторные пространства. Теоремы об изоморфизме.
Ортонормированные и симплектические базисы, связь с координатами векторов.
\item Естественный изоморфизм пространства со скалярным произведением своему пространству функций
(общий вид линейной функции на пространстве со скалярным произведением).
\item Изотропные векторы и подпространства. Линейная независимость ортогональных неизотропных векторов.
 Ортогональное дополнение. Определитель матрицы Грама. Процесс ортогонализации.
\item Симплектические векторные пространства. Гамильтоновы базисы. Изотропные подпространства.
\item Унитарное пространство. Неравенство Коши\ч Буняковского, неравенство треугольника.
\item Переход к новым ортонормированным или симплектическим координатам. Специальные группы матриц
(ортогональных, псевдоортогональных, унитарных, псевдоунитарных, симплектических).
\item Операторы, сохраняющие скалярные произведения (ортогональные, псевдоортогональные,
симплектические, унитарные, псевдоунитарные), их матрицы. Изометрии. Группы операторов.
\item Канонический вид ортогонального и унитарного операторов, его единственность.
Собственные подпространства. Ортогональные преобразования точечных пространств.
\item Группы $O(1)$, $O(2)$, $U(1)$, $Sp(2)$, $O(1,1)$. Псевдовращения плоскости.
Трёхмерное псевдоевклидово пространство.
\item Овеществление комплексного пространства и оператора. Псевдоевклидова и симплектические структуры,
определяемые эрмитовым пространством.
\item Овеществление псевдоунитарного оператора. Группа $U(p,q)$ как пересечение $O(2p,2q) \cap Sp(2p+2q)$.
\item Существование и единственность сопряжённого оператора в пространстве со скалярным произведением,
его свойства. Сопряжённость для оператора, сохраняющего скалярное произведение. Связь с дуальными операторами
в пространстве линейных функций.
\item Самосопряжённые операторы в евклидовом и унитарном пространстве. Канонический вид.
Свойства собственных векторов и подпространств.
\item Естественный изоморфизм пространству операторов пространств билинейных и полуторалинейных
функций на пространствах со скалярным произведением. Канонический вид симметрической (эрмитовой)
функции на евклидовом (унитарном) пространстве. Приведение к каноническому виду уравнения гиперповерхности
второго порядка.
\item Полярные разложения оператора в композиции самосопряжённых и ортогональных (унитарных).
\item Инварианты пары квадратичных форм, одна из которых положительно определена. Их канонический базис.
\item Тензоры. Примеры. Тензоры и полилинейные функции. Пространства тензоров.
\item Умножение тензоров. Базисы и координаты в пространствах тензоров.
\item Свёртка тензоров. Примеры.
\item Опускание и поднятие индексов тензора для пространства со скалярным произведением.
Примеры.
\item Симметрия и косая симметрия тензоров и их координат. Операции симметрирования и альтернирования.
\item Кососимметричные тензоры. Операция внешнего умножения, свойства.
\item Простые поливекторы (и внешние формы), их координаты. Примеры.
Плюккеровы координаты подпространства.
\item Базис и размерность пространства поливекторов (внешних форм).
\end{nums}

\medskip\dmvntrail
\end{document}
