\documentclass[a4paper]{article}
\usepackage[simple]{dmvn}

\title{Программа экзамена по линейной алгебре}
\author{Лектор\т М.\,В.\,Зайцев}
\date{II семестр, 2004~г.}

\begin{document}
\maketitle

\begin{nums}{-2}
\item Векторные пространства. Линейная зависимость и независимость векторов.
Базис, размерность.
\item Матрица перехода от одного базиса к другому. Координаты, их изменение при
замене базиса. Изоморфизм пространств одинаковой размерности.
\item Подпространства, их суммы и пересечения. Прямая сумма подпространств.
Размерность суммы и пересечения подпространств.
\item Сопряженное пространство и его размерность. Канонический изоморфизм.
Критерий линейной независимости векторов.
\item Задание подпространств линейной однородной системой уравнений.
\item Линейные отображения, их задание матрицами. Размерность ядра и образа.
Критерий инъективности.
\item Алгебра линейных операторов. Матрица линейного оператора и ее изменение
при замене базиса.
\item Определитель и след линейного оператора. Критерий невырожденности
оператора.
\item Инвариантные подпространства. Собственные векторы и собственные значения.
\item Характеристический многочлен. Алгебраическая и геометрическая кратности
корня.
\item Спектр оператора. Критерии диагонализируемости линейного оператора.
\item Минимальный многочлен, его существование и единственность.
\item Теорема Гамильтона\ч Кэли и ее следствия.
\item Разложение пространства в сумму корневых подпространств.
\item Нормальный базис для нильпотентного оператора.
\item Жордановы матрицы. Существование жордановой нормальной формы у
комплексной матрицы.
\item Единственность жордановой нормальной формы.
\item Билинейные формы и их матрицы. Изменение матрицы при замене базиса.
Канонический базис для симметрической билинейной формы.
\item Квадратичные формы и их матрицы. Канонический и нормальный вид
квадратичной формы. Алгоритм Лагранжа.
\item Закон инерции для вещественных квадратичных форм.
\item Теорема Якоби. Критерии Сильвестра.
\item Канонический вид кососимметрической билинейной формы.
\item Евклидово пространство. Неравенство Коши\ч Буняковского и его следствия.
\item Ортогональность векторов. Существование ортонормированного базиса в
евклидовом пространстве. Изоморфизм евклидовых пространств одинаковой
размерности.
\item Процесс ортогонализации Грамма\ч Шмидта. Ортогональное дополнение.
\item Сопряженный оператор и его матрица. Существование ортонормированного
базиса из собственных векторов для самосопряженного оператора.
\item Ортогональные матрицы. Приведение квадратичной формы к главным осям.
\item Ортогональный оператор и его канонический базис.
\item Полярное разложение линейного оператора.
\item Унитарное пространство, существование ортонормированного базиса, матрица
перехода от одного ортонормированного базиса к другому.
\item Эрмитовы и унитарные операторы, их канонический вид.
\item Аффинные пространства, их изоморфизм, координаты точки в разных системах
координат.
\item Подпространства в аффинном пространстве и их пересечение. Задание
подпространств системами линейных уравнений.
\item Евклидовы пространства, расстояние от точки до плоскости.
\item Расстояние между плоскостями в евклидовом пространстве. Определитель
Грама и объем параллелепипеда.
\item Аффинная группа, подгруппа сдвигов и подгруппа, оставляющая неподвижной
фиксированную точку.
\item Движения евклидова пространства, их представление в виде произведения
сдвига и движения с неподвижной точкой.
\item Классификация движений в двумерном и трехмерном пространствах.
\item Квадратичные функции в аффинном пространстве, их центральные точки,
канонический вид.
\item Квадрики в аффинном пространстве. Единственность уравнения, задающего квадрику.
\item Центр квадрики и центральные точки квадратичной функции.
\item Понятие тензора, тензоры малых рангов, произведение тензоров. Базис и
размерность пространства тензоров типа $(p,q)$.
\item Изменение координат тензора при замене базиса.
\item Свертка тензора, ее координаты.
\item Симметризация и альтернирование тензоров.
\item Тензорная алгебра. Внешняя алгебра векторного пространства.
\item Базис и размерность внешней алгебры векторного пространства.
\item Связь внешнего произведения с определителем.
\end{nums}

\medskip\dmvntrail
\end{document}
