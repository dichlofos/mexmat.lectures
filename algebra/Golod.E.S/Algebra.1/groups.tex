\section{Алгебраические структуры}
\label{groups}

\newcommand{\mnod}[2]{\left( #1 ; #2 \right)}
\newcommand{\mnok}[2]{\left[ #1 ; #2 \right]}

\epigraph{It is not enough to have a good mind; the main thing is to use it well.}{Rene Descartes}

\subsection{Полугруппы}

\begin{df}
  Пусть $\Gb$ -- некоторое множество. Тогда \emph{бинарной операцией}\index{операция!бинарная} на этом множестве будет называться правило, согласно которому каждой упорядоченной паре $(a,b)$, $a,b\in\Gb$, ставится в соответствие некоторый элемент $c\in\Gb$. Операция называется \emph{частичной}\index{операция!частичная}, если она определена не для всех пар.
\end{df}

\begin{df}
  Операция называется \emph{ассоциативной}\index{операция!ассоциативная}, если $\forall a,b,c\in\Gb$ $(ab)c\bw=a(bc)$. Множество, на котором задана ассоциативная операция, называется \emph{полугруппой}\index{полугруппа}. В случае ассоциативной операции значение выражения вообще не зависит от расстановки скобок:
\end{df}
\begin{note}
 В случае частичной бинарной операции следует сказать, что закон ассоциативности состоит в следующем: если определено хотя бы одно из произведений $a(bc)$ или $(ab)c$, то второе также определено и они равны.
\end{note}

\begin{theorem}[обобщённый закон ассоциативности]
  Произведение $n$ элементов в полугруппе не зависит от способа расстановки скобок.
\end{theorem}
\begin{proof}
  Проведём доказательство индукцией по $n$. База индукции -- $n=3$ -- известно.

  Теперь пусть $n\ge4$ фиксировано и известно, что для произведения меньшего чем $n$ числа элементов утверждение теоремы верно. Выделим скобки, над которыми производится последняя операция: $(a_1\cdot\dots\cdot a_k)\cdot(a_{k+1}\cdot\dots\cdot a_n)$. Назовём такую расстановку скобок расстановкой типа $k$. Внутри каждой из них возможна любая расстановка скобок по предположению индукции.

  Нам надо доказать, что для любой расстановки типа $m$, где $1\le m\le n-1$ произведение определено и результат одинаков. Для этого нам дано, что для некоторой расстановки типа $k$ это произведение определено.

  Достаточно показать, что определены произведения для расстановок типа $k\pm1$ тоже определены и результат одинаков во всех трёх случаях. Покажем, например, переход от $k$ к $k-1$:
  $$((a_1\cdot\dots\cdot a_{k-1})\cdot a_k)\cdot(a_{k+1}\cdot\dots\cdot a_n)=(a_1\cdot\dots\cdot a_{k-1})\cdot(a_k\cdot(a_{k+1}\cdot\dots\cdot a_n))$$

  Нетрудно заметить, что доказательство на этом завершается.
\end{proof}

\begin{df}
  Пусть $\Gb$ -- множество с бинарной операцией. Элемент $e\in\Gb$ называется \emph{левой единицей}, если $ea=a\cln\forall a\in\Gb$. Определение \emph{правой единицы}\index{единица} аналогично. Если в $\Gb$ имеется левая единица $e_1$ и правая единица $e_2$, то они совпадают и элемент $e=e_1=e_2$ называется (двусторонней) единицей.
\end{df}
\begin{df}
  Пусть $(\Gb,\cdot)$ обладает единицей. Тогда для элемента $a$ элемент $a'$ называется \emph{левым} (соответственно \emph{правым}) \emph{обратным}\index{элемент!обратный}, если $a'a=e$ ($aa'=e$). Также определяется двусторонний обратный элемент, который обозначается как $a^{-1}$. Элемент, имеющий обратный, называется \emph{обратимым}.\index{элемент!обратимый}
\end{df}
\begin{theorem}
  Если элемент обладает левым обратным и правым обратным, то они равны между собой.
\end{theorem}
\begin{note}
  На экзамене необходимо проиллюстрировать примерами все эти свойства. %\nota{Чего только не необходимо сделать на экзамене...}
\end{note}

\begin{df}
  По определению $a^n=a\cdot\dots\cdot a$ ($n$ раз), $a^0=e$, $a^{-n}=(a^{-1})^n\bw=(a^n)^{-1}$.
\end{df}
\begin{df}
  Элементы $a,b\in\Gb$ называются \emph{коммутирующими}, если $ab\bw=ba$. Если это выполнено для любых элементов, то операция (множество с этой операцией) называется коммутативной.\index{операция!коммутативная}
\end{df}

\subsection{Группы}

\begin{df}
 \emph{Группа}~--- множество с операцией, удовлетворяющее условиям:
\index{группа}%
 \begin{enumerate}
  \item Ассоциативность
  \item $\exists e\colon\cln ae=ea=a$
  \item $\forall a\cln\exists b\colon\cln ab = ba = e$. Обозначается $b = a^{-1}\cln(a = b^{-1})$
 \end{enumerate}
\end{df}

\begin{df}
Порядком элемента $a$ в группе называется наименьшее $n \in \mathbb N $, такое что $a^n = e$. Обозначается o(a) = n. Если $\nexists o(a)$, то принято считать, что $o(a) = \infty$.\index{порядок!элемента группы}
\end{df}

Порядок элемента группы отвечает следующему свойству:

\begin{theorem}
Пусть $o(a)=n$. Тогда $o(a^k)=\frac{n}{(k,n)}$
\end{theorem}
\begin{proof}
  Пусть $m$ -- положительное число такое, что $(a^k)^m\bw=e$. Значит $n\mid km$. Поэтому $m$ -- наименьший такой показатель, когда $km$~--- наименьшее общее кратное чисел $k$ и $n$, т.е. $km=\frac{kn}{(k,n)}\ra m=\frac{n}{(k,n)}$.
\end{proof}

\begin{theorem}
  \label{group:chaos}
  Порядок любого элемента конечной группы является делителем порядка группы (под порядком множества понимается количество его элементов).\index{порядок!группы}
\end{theorem}
\begin{proof}
  Пусть $|\Gb|=n$ и $a_1,\dots a_n$ -- её элементы. Пусть порядок произвольного элемента $a\in\Gb\cln o(a)=k$. Рассмотрим такое отображение $\pi_a\colon \pi_a(a_i)\bw=aa_i$. Это отображение биективно, следовательно может быть рассмотрено как перестановка на множестве $\Gb$. Разложим её в произведение независимых циклов. Каждый из циклов имеет вид $(a_i,aa_i,\dots a^{k-1}a_i)$ так как $a^la_i\ne a_i$ при $1\le l<k$; таким образом, если их число равно $m$, то $km=n$.
\end{proof}

\subsection{Кольца}

\begin{df}
  Если на множестве $K$ задано две операции (условно <<$+$>> и <<$\cdot$>>) и выполнены следующие условия:\index{кольцо}
  \begin{enumerate}
    \item $(K,+)$ -- абелева (коммутативная) группа\index{группа!абелева}
    \item $(K,\cdot)$ -- полугруппа
    \item $a(b+c)=ab+ac$ и $(a+b)c=ac+bc$,
  \end{enumerate}
  то $(K,+,\cdot)$ называется \emph{кольцом}.

  Если в $(K,\cdot)$ существует единица ($1\ne0$), то $K$ называется кольцом с единицей. Кольцо называется коммутативным, если $\forall x,y\cln xy=yx$. Обратимость элементов в кольце зависит от свойств алгебраической структуры $(K,\cdot)$.
\end{df}

\begin{df}
  Элемент $a$ называется левым (правым, двусторонним) \emph{делителем нуля}, если $\exists b\ne0$ такой, что $ab=0$ ($ba=0$, $ab=ba=0$ соответственно).\index{делитель нуля}
\end{df}

\begin{df}
  \emph{Поле} -- коммутативное кольцо с $1\ne0$, в котором всякий ненулевой элемент обратим.\index{поле}
\end{df}


\subsection{Кольца вычетов}

Фиксируем $n\in \N$ и будем рассматривать кольца вычетов по модулю $n$.

На множестве целых чисел введём отношение эквивалентности таким
образом, что два числа $a$ и $b$ будут считаться эквивалентными, если
$a\equiv b \mod n$. Все целые числа сгруппируются по классам
эквивалентности, множество этих классов будем обозначать как $\Zc/n$,
а сами классы обозначим как $[a]=\{x\mid x\equiv a \mod n\}$. На
множестве классов логично определить операции сложения и умножения как
$[a]+[b]=[a+b]$ и $[a][b]=[ab]$, но требует проверки корректность
такого определения, то есть чтобы результат не зависел от выбранных
представителей в каждом из классов. В данном случае это не составит
большого труда. Также тривиальны и остальные необходимые проверки,
которые должны нам показать, что множество классов вычетов с такими
операциями является коммутативным кольцом с
единицей.\index{кольцо!вычетов}

Также легко заметить, что элемент $[k]$ в $\Zc/n$ обратим в том и только том случае, если $(k,n)=1$. Так как кольцо вычетов конечно и коммутативно, то всякий неделитель нуля в нём обратим. Действительно, если $a$ -- неделитель нуля, то все элементы $ax$ различны, значит один из них равен $1$.

Далее, обозначим за $U(\Zc/n)$ группу всех обратимых элементов кольца $\Zc/n$. Порядок этой группы равен $\varphi(n)$. Кольцо $\Zc/n$ является полем тогда и только тогда, когда $n$ -- простое.

\begin{theorem}[обобщённая малая теорема Ферма]
  $(k,n)=1\ra k^{\varphi(n)}\equiv 1 \mod n$.\index{теорема!Ферма}
\end{theorem}
\begin{proof}
  Если $(k,n)=1$, то $[k]\subset U(\Zc/n)$. По теореме \ref{group:chaos} имеем, что $[k]^{\varphi(n)}=[1]$, что и требовалось доказать.
\end{proof}
\begin{note}
  В частном случае, когда $n$ -- простое, имеем, что почти все $k$ таковы, что $(k,n)=1$ и $\varphi(n)=n-1$, то есть $k^n\equiv k \mod n$.
\end{note}

\begin{df}
  \emph{Характеристикой поля} называется такое наименьшее неотрицательное число $n$, что $1+\dots+1 \text{\:($n$ раз)} = 0$ . В случае, если такое не выполняется ни при каком натуральном $n$, принято считать, что характеристика поля $\Char\Pb = 0$.\index{характеристика поля} Если $\Char\Pb=n$, то $n$ -- простое.
\end{df}

\subsection{Построение поля частных области целостности}

\begin{theorem}
\label{group:pole}\index{поле!частных}
Любое коммутативное кольцо с единицей и без делителей нуля вкладывается в поле.
\end{theorem}
\begin{proof}
Пусть $\Kb$ ~-- коммутативное кольцо без делителей нуля. Это означает, что
$$ab=0\ra (a=0)\vee(b=0)$$

Для доказательства построим некоторое множество, состоящее из элементов кольца $\Kb$, если данное множество с определёнными на нём операциями будет являться полем, и будет существовать вложение $\varphi\colon\cln\Kb\rightarrow \Fb$, то кольцо $\Kb$ будет вложено в это поле.

Рассмотрим множество пар $(a,b),\cln a,b\in\Kb,\cln b\ne0$. Пусть пары $(a,b)$ и $(c,d)$ эквивалентны, если $ad=bc$. Эквивалентные пары будем обозначать таким образом: $(a,b)\sim(c,d)$. Пусть $\Ab$ ~--- множество всех таких пар. Отождествим эквивалентные пары. Это значит, что $$(a,b)\sim(c,d)\cln\&\cln(c,d)\sim(u,v)\ra(a,b)\sim(u,v)$$

$\blacktriangleright$
$$
\left\{
\begin{array}{lll}
ad&=&bc\\
cv&=&du
\end{array}
\right.
\lra
\left\{
\begin{array}{lll}
adv&=&bcv\\
bcv&=&bdu
\end{array}
\right.
\ra adv=bdu \ra av=bu \blacktriangleleft
$$

Обозначим за $[(a,b)]$ класс всех пар $(a_i,b_i)$, эквивалентных между собой. Тогда пусть множество всех классов $\Fb=\{ [(a,b)] \mid (a,b)\in \Ab\}$. Докажем, что это множество с указанными ниже операциями является полем, и в него будет вложено исходное кольцо $\Kb$.

На множестве $\Fb$ определим сложение двух классов:
$$[(a,b)]+[(c,d)]=[(ad+bc,bd)]$$

Далее будем обозначать класс или как $[(a,b)]$, или как
$\frac{a}{b}$. Тогда определение сложения примет привычный вид
$\frac{a}{b}+\frac{c}{d}=\frac{ad+bc}{bd}$. Покажем, что класс суммы
остаётся неизменным при различном выборе складываемых пар в классах
слагаемых.


\begin{proof}
  Пусть $(a,b)+(c,d)=(ad+bc,bd),\cln(a',b')+(c',d') = (a'd'+b'c',b'd')$. Иначе,
  $$
  \left\{
  \begin{array}{lll}
    ab'&=&ba'\\
    cd'&=&dc'
  \end{array}
  \right.
  \lra
  \left\{
  \begin{array}{lll}
    ab'dd'&=&dd'ba'\\
    cd'bb'&=&dc'bb'
  \end{array}
  \right.
  $$

  Сложим оба равенства:
  $$ ab'dd'+cd'bb'= dd'ba'+dc'bb'$$
  $$(ad+cb)b'd'   =(d'a'+c'b')bd $$

  Последнее означает не что иное, как $(ad+bc,bd)\bw\sim(a'd'+b'c',b'd')$.
\end{proof}

Теперь определим операцию умножения двух классов как $\frac{a}{b}\times\frac{c}{d}=\frac{ac}{bd}$. Необходимо теперь доказать корректность умножения, то есть что при различном выборе пар в классах $\frac{a}{b}$ и $\frac{c}{d}$ класс их произведения будет оставаться неизменным. Это доказательство аналогично вышеприведённому для сложения.

Для определённых нами операций сложения и умножения проверим аксиомы поля.

\begin{enumerate}
 \item $\frac{a}{b}+\frac{c}{d}=\frac{c}{d}+\frac{a}{b}$ (по определению)
 \item $\left(\frac{a}{b}+\frac{c}{d}\right)+\frac{s}{t} = \frac{a}{b}+\left(\frac{c}{d}+\frac{s}{t}\right)$

   $\blacktriangleright$ Рассмотрим правую и левую части. Если мы их равносильными преобразованиям приведём к равным выражениям, то доказательство будет завершено.
   $$\left(\frac{a}{b}+\frac{c}{d}\right)+\frac{s}{t}=\frac{ad+bc}{bd}+\frac{s}{t}=\frac{adt+bct+sbd}{bdt}$$
   $$\frac{a}{b}+\left(\frac{c}{d}+\frac{s}{t}\right)=\frac{a}{b}+\frac{ct+ds}{dt}=\frac{adt+bct+sbd}{bdt}\blacktriangleleft$$

 \item Класс $[(0;a)]$ отождествим с нулём.
 \item $\forall\frac{a}{b}\cln\exists\frac{-a}{b}\colon\cln\frac{a}{b}+\frac{-a}{b}=0$
 \item $\left( \frac{a}{b}+\frac{c}{d} \right)\frac{s}{t}= \frac{a}{b}\frac{s}{t}+\frac{c}{d}\frac{s}{t}$
 \item $\frac{a}{b}\cdot\frac{c}{d}=\frac{c}{d}\cdot\frac{a}{b}$
 \item $\left(\frac{a}{b}\cdot\frac{c}{d}\right)\frac{s}{t}=\frac{a}{b}\left(\frac{c}{d}\cdot\frac{s}{t}\right)$
 \item Класс $[(a,a)]$ отождествим с единицей.
 \item Обратным к $\frac{a}{b}$ будет являться $\frac{b}{a}$.
\end{enumerate}

Таким образом мы показали, что $\Fb$ с определёнными нами операциями
является полем. $\Fb$ называется полем частных кольца $\Kb$.

Определим вложение $\varphi\colon\cln a\rightarrow\frac{ax}{x},\cln a\in\Kb,\:\frac{ax}{x}\in\Fb$.

Это вложение сохраняет операцию: $a+b\stackrel{\varphi}{\longrightarrow} [((a+b)x,x)] =[(ax+bx, x)]=[(ax,x)] + [(bx,x)]$ (аналогично проверяется для умножения) и разные элементы переходят в разные:

\begin{proof}
  Пусть $a \to \frac{ab}{b},\cln a'\to \frac{a'b}{b}$. Необходимо
  доказать, что при различных $a$ и $a'$, $\frac{ab}{b}$ и
  $\frac{a'b}{b}$ также не совпадают. Докажем от противного. Пусть
  $a\ne a',\cln \frac{ab}{b}=\frac{a'b}{b}$. Последнее выражение
  означает, что $(ab,b)\sim(a'b,b)$ или $abb=a'bb$. Так как исходное
  кольцо $\Kb$ без делителей нуля, то или $a=a'$, или $b=0$. Но мы
  определили классы в множестве $\Ab$ так, что $b\ne 0$. Следовательно
  $a=a'$. Противоречие.
\end{proof}

Таким образом, мы построили такое множество $\Fb$ из элементов $\Kb$ и определили на нём две операции таким образом, что множество с этими операциями является полем и показали такое отображение $\Kb \stackrel{\varphi}{\longrightarrow} \Fb$, что оно является вложением.
\end{proof}

\begin{ex}
Кольцо целых чисел $\mathbb Z$ вложено в поле рациональных чисел $\mathbb Q\colon\mathbb Z \subset \mathbb Q$
\end{ex}

\subsection{Понятие гомоморфизма и изоморфизма}

\begin{df}
Отображение $\varphi\colon (G,\cdot)\to(K,*)$ называется \emph{гомоморфизмом}, если оно <<сохраняет операцию>>, т.е. $f(a\cdot b)=f(a)*f(b)\cln\forall a,b\in G$.\index{гомоморфизм}
\end{df}

\begin{stm}
  Образ группы при гомоморфизме является группой. При этом $\varphi(e)\!\!\bw=e'$. Однако при гомоморфизме полугрупп образ единицы не обязательно является единицей.
\end{stm}

Гомоморфизм колец уже обязан сохранять обе операции. В общем случае, опять же, единица не обязательно переходит в единицу.

\begin{theorem}
  Пусть $\Pb$ -- поле и $\Kb$ -- кольцо. В таком случае если существует гомоморфизм $\varphi\colon\Pb\to\Kb$, то $\varphi$ либо тривиален, либо инъективен.
\end{theorem}
\begin{proof}
  Пусть $x\neq y$ и $f(x)=f(y)\ra f(x-y)=0$. В силу существования $(x-y)^{-1}$ имеем
  \begin{equation*}
    f(1)=f\hr{(x-y)(x-y)^{-1}}=0
  \end{equation*}
  то есть $\forall a\in\Pb\colon f(a)=f(a\cdot 1)=0$.
\end{proof}

\begin{df}
  Отображение $\varphi$ называется \emph{изоморфизмом}, если $\varphi$ является гомоморфизмом и $\varphi$ -- биективно. Два множества с алгебраическими операциями называются изоморфными, если существует изоморфизм, переводящий одно в другое.\index{изоморфизм}
\end{df}

\begin{theorem}[Кейли]\index{теорема!Кейли}
  Всякая группа порядка $n$ изоморфна некоторой подгруппе симметрической группы $S_n$.
\end{theorem}
\begin{proof}
  Пусть $\Gb=\{a_1,a_2,\dots,a_n\}$ -- группа порядка $n$. Составим для группы $\Gb$ таблицу Кейли и сопоставим каждому элементу $a_i\in\Gb$ перестановку на множестве $\Gb$, задаваемую $i$-й строкой таблицы Кейли, то есть рассмотрим отображение $f\colon \Gb\to S_n$, при котором $f(a)=\pi_a,\cln \pi_a(a_j)=aa_j$. Разным элементам группы $\Gb$ соответствуют разные перестановки, таким образом отображение $\Gb\to \Im f$ -- биективно.

  Докажем, что оно является гомоморфизмом. $\forall a_j\in\Gb$ имеем $\pi_{ab}(a_j)\bw=(ab)a_j\bw=a(ba_j)\bw=\pi_a(\pi_b(a_j))\bw=(\pi_a\circ\pi_b)(a_j)$.
\end{proof}

\begin{theorem}
  Пусть $\Pb$ -- поле. Имеется единственный инъективный гомоморфизм
    \begin{enumerate}
      \item $f\colon\Q\to\Pb$, если $\Char\Pb=0$;
      \item $f\colon \Zc/p\to\Pb$, если $\Char\Pb=p$.
    \end{enumerate}
  В обоих случаях $\Im f$ является единственным минимальным полем поля $\Pb$.
\end{theorem}
\begin{proof}
  \begin{enumerate}
    \item $\Char\Pb=0$. Тогда $\forall n\colon n\cdot1\ne0$. Зададим
      отображение $f$ по правилу
      $f\left(\frac{m}n\right)\bw=(m\cdot1)(n\cdot 1)^{-1}$. Так как
      запись рационального числа в виде дроби неоднозначна, то
      необходимо проверить, что легко сделать, что подобное задание
      $f$ корректно. Далее проверяем, что $f$ -- гомоморфизм, который
      является инъективным, так как $\Qb$ -- поле. Так как
      $f(1_\Q)=1_{\Pb}$, то $f$ определён однозначно.
    \item $\Char\Pb=p$. Определим $f([k])=k\cdot1$. Так как $o(1)=p$, то $k\cdot1=l\cdot1\lra k\equiv l\mod p$ и отображение определено корректно. Очевидно, что оно гомоморфизм, инъективно и однозначно.
  \end{enumerate}

  В обоих случаях $\Im f$ содержится в любом подполе $\Lb$ поля $\Pb$ так как $1\in\Lb$, а значит все элементы $n\cdot1\in\Lb$ и $(m\cdot1)(n\cdot)^{-1}\in\Lb\cln(1\cdot1\ne0)$. $\Im f$ называется \emph{простым подполем}\index{подполе простое} поля $\Pb$.
\end{proof}
\subsection{Циклические группы}

\begin{df}
Пусть $(\Gb,\cdot)$ -- группа и $a\in\Gb$. Тогда множество $\left\langle a\right\rangle=\{a^n\mid n\in\Zc\}$ называется \emph{циклической подгруппой} с \emph{порождающим элементом} $a$. Очевидно, что $|\left\langle a\right\rangle|=o(a)$. Может так оказаться, что $\left\langle a\right\rangle = \Gb$. В таком случае сама группа $\Gb$ называется циклической.\index{группа!циклическая}
\end{df}

Если $\left\langle a\right\rangle$ -- циклическая группа порядка $n$, то порядок элемента $o(a^k)=\frac{n}{(n,k)}$. Если $(n,k)=1$, то элемент $a^k$ является порождающим в этой группе. Легко понять, что число образующих в группе порядка $n$ равно $\varphi(n)$. Очевидно, что если группа простого порядка, то каждый элемент этой группы будет являться порождающим. Если $|\left\langle a\right\rangle|=\infty$, то образующими являются только элементы $a$ и $a^{-1}$.

\begin{theorem}
  Пусть $G=\left\langle a \right\rangle$ циклическая группа и $(K, \cdot)$ -- произвольная группа. В таком случае существует единственный гомоморфизм $f\colon G\to K,$ при котором $f(a)=b$ для любого $b$, если $|\left\langle a\right\rangle|=\infty$ и для $b\colon b^n=e$, если $|\left\langle a\right\rangle|=n$.
\end{theorem}
\begin{proof}
  Так как при гомоморфизме $f(a^k)=f^k(a)$, то $f$ однозначно задаётся своим значением $f(a)$. В случае бесконечной группы $G$ утверждение теоремы становится очевидным. В случае конечной, в общем-то, тоже.
\end{proof}

\begin{theorem}
  Две любые группы одного и того же простого порядка изоморфны между собой.
\end{theorem}

\subsection{Разложение группы на смежные классы. Теорема Лагранжа}

\begin{df}
  Пусть $H$ -- подгруппа в $\Gb$. Множества вида $gH = \{gh\mid h\in H\}$ при фиксированном $g\in\Gb$ называются \emph{левыми смежными классами} группы $\Gb$ по подгруппе $H$. Аналогично определяются правые смежные классы.\index{классы!смежные}
\end{df}

\begin{theorem}
  Каждый левый смежный класс определяется любым своим элементом, т.е. если $g'\in gH$, то $g'H=gH$. Два левых смежных класса либо не пересекаются, либо совпадают. Объединение всех левых смежных классов равно $\Gb$. Отображение $H\to gH$ биективно.
\end{theorem}
\begin{proof}
  Пусть $g'\in gH$, т.е. $g'=gh,\cln h\in H$. Тогда $g'H=ghH\subset gH$; так как $g=g'h^{-1}$, то $g\in g'H$ и $gH\subset g'H$, из чего следует $gH=g'H$. Остальные утверждения теоремы выводятся не менее просто.
\end{proof}

Разбиение множества $\Gb$ на попарно непересекающиеся классы равносильно введению отношения эквивалентности на этом множестве.

\begin{theorem}
  Два элемента $g_1,g_2\in\Gb$ принадлежат одному левому смежному классу по подгруппе $H$ тогда и только тогда, когда $g_1^{-1}g_2\in H$.
\end{theorem}
\begin{proof}
  $g_2\in g_1H \lra g_2=g_1h$ для некоторого $h\in H$. Следовательно, $g^{-1}g_2\in H$.
\end{proof}

\begin{theorem}
  Отображение $x\to x^{-1}$ группы $\Gb$ на себя задаёт биективное соответствие между множествами левых и правых смежных классов по подгруппе $H$.
\end{theorem}
\begin{proof}
  $$(gH)^{-1}=\{(gh)^{-1}\mid g\in H\} = \{h^{-1}g^{-1}\mid h\in H\}=\{hg^{-1}\mid h\in H\}=Hg^{-1}$$
\end{proof}

\begin{df}
  Число (левых или правых) смежных классов группы $\Gb$ по подгруппе $H$ называется \emph{индексом}\index{индекс подгруппы} подгруппы $H$ в $\Gb$ и обозначается $(\Gb:H)$.
\end{df}

\begin{theorem}[Лагранжа]\index{теорема!Лагранжа}
  Пусть $\Gb$ -- конечная группа. Тогда $|H|\cdot (\Gb:H)=|G|$.
\end{theorem}
\begin{proof}
  Если $|H|=d$, то для каждого элемента $h\in H$ имеем $(\Gb:H)$, например, левых смежных классов, которые не пересекаются между собой и при объединении дают $\Gb$. Таким образом утверждение теоремы очевидно.
\end{proof}

Примеры разбиения множества на смежные классы:
\begin{enumerate}
  \item $G=(\Zc,+),\cln H=n\Zc$. Получаемые таким образом классы называются \emph{классами вычетов}.
  \item $G=(\R,+),\cln H=2\pi\Zc$. Смежные классы $\al+2\pi\Zc$ находятся в биективном соответствии с с углами, которые образуют векторы на плоскости с положительным направление оси абсцисс.
  \item \dots
\end{enumerate}
