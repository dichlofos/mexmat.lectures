\section{Перестановки}
\label{permutation}

\epigraph{There's no sense in being precise when you don't even know what you're talking about.}{John von Neumann}

\subsection{Группа перестановок}
\index{группа!перестановок}

\begin{df}
  \emph{Отображение} -- правило, которое каждому 
  элементу из множества $\Xb$ ставит в соответствие некоторый элемент множества $\Yb$. 
\end{df}

%Основные типы отображений:
%\begin{enumerate}
% \item Сюръективное отображение.
%   $$\mathbf F\colon A\rightarrow  B,\spc\forall y\in \mathbf B\spc\exists x\in\mathbf A: y=F(x)$$
% \item Инъективное отображение.
%   $$F(x_1) = F(x_2) \ra x_1 = x_2$$
% \item Биективное отображение, или взаимооднозначное соответствие. Определяется как отображение, обладающее предыдущими двумя признаками.
%\end{enumerate}

Под перестановками на множестве понимаются биективные отображения множества в себя. Рассмотрим множество, состоящее из конечного числа элементов. Занумеруем их от 1 до n. Множество всех биективных отображений из $(1,2,\dots n)$ в себя обозначим через $S_n$, $|S_n|=n!$. Всякую из таких перестановок можно записать в виде таблицы

$$\sigma =\left(\begin{array}{cccc}1&2&\dots&n\\i_1&i_2&\dots& i_n\end{array}\right),$$
которая означает, что при выполнении такого отображения элемент с номером 1 переходит в элемент с номером $i_1$ и т.п.

Перестановки не обладают свойством коммутативности, но обладают свойством ассоциативности. Существование единичной и обратной перестановки гарантируют нам то, что множество всех перестановок $S_n$ является \emph{группой}.

Условимся перемножать перестановки справа налево.

\begin{df}%
\index{цикл}
\emph{Цикл}~--- перестановка, такая что $\alpha_1 \rightarrow \alpha_2 \rightarrow \dots \rightarrow \alpha_k \rightarrow \alpha_1$, $\alpha_1,\alpha_2\dots\alpha_k \in \{1,2,\dots n\}$. Обозначается $(\alpha_1,\alpha_2,\dots \alpha_k)$. 

Если при выполнении какой-то циклической перестановки элементы множества переходят сами в себя, то они называются неподвижными, иначе -- перемещаемыми. Два цикла называются независимыми, если у них нет общих перемещаемых элементов. Два независимых цикла коммутативны.
\end{df}

\begin{theorem}
Всякая перестановка представима в виде произведения независимых циклов.
\end{theorem}

\begin{df}%
\index{транспозиция}
\emph{Транспозиция} -- цикл длины 2, или изменение мест двух элементов между собой.
\end{df}

\begin{note}
  Любой цикл длины $k$ представим в виде произведения $k-1$ транспозиций:
  $$ (k_1,k_2\dots,k_n) = (k_1,k_n)\dots(k_1,k_3)(k_1,k_2)=(k_1,k_2)(k_2,k_3)\dots(k_{n-1},k_n) $$
\end{note}
\begin{note}
  Число циклов длины $k$ в $S_n$ равно $\Cb_n^k \cdot(k-1)!$
\end{note}

\begin{df}\index{перестановка!декремент}
  \emph{Декрементом} подстановки называется сумма длин независимых
  циклов в её разложении, уменьшенных на 1:
  $d=d(\sigma)=\sum\limits_{i=1}^s(k_i-1)$. Легко понять, что
  $\sgn\sigma=(-1)^d$.
\end{df}

\newpage
\subsection{Чётность, знак перестановки}

\begin{df}
  Назовём пару элементов $(i,j)$ \emph{правильной}, если $i\bw<j\ra\sigma(i)\bw<\sigma(j)$ и \emph{неправильной} в обратном случае. Чётностью\index{перестановка!чётность} перестановки будет называть чётность количества неправильных пар $k$ в этой перестановке. Знак\index{перестановка!знак} перестановки $\sgn\sigma\eqdef(-1)^k$.
\end{df}

\begin{note}
  Транспозиция меняет знак перестановки. При умножении перестановок из знаки также перемножаются: $\sgn(\pi\sigma)=\sgn\pi\cdot\sgn\sigma$.
\end{note}

\begin{note}
  Число чётных перестановок равно числу нечётных и равно $n!/2$.
\end{note}

\begin{df}
	$\Ab_n\subset\Sb_n=\{\sigma\vert\sgn\sigma>0\}$~--- подгруппа, называемая \emph{знакопеременной
	группой перестановок}\index{перестановка!знакопеременная}.
\end{df}
