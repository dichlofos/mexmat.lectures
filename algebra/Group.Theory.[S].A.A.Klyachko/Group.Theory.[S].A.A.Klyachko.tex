\documentclass[10pt,a4paper]{article}%
\usepackage[russian]{babel}%
\usepackage[rus,two]{stylefile}%
%\usepackage{russlh}%

\oddsidemargin=0pt%
\topmargin=-60pt%
\textwidth=16cm%
\textheight=686pt%
\makeindex

\begin{document}
\begin{center}{\sf Записи лекций СПЕЦКУРСА по ТЕОРИИ ГРУПП (А.А.
Клячко)}
\end{center}

\noindent {\small Везде нет различия между обозначениями $\subset$
и $\subseteq$ --- оба означают нестрогое включение. Как правило,
мы пишем $H<G$, если $H$ --- подгруппа $G$, и $S\subset G$, если
$S$
--- подмножество (группы) $G$. Как и в случае множеств, не различаются
строгое $<$ и нестрогое $\leq$ включения.}
%%%%%%%%%%%%%%%%%%%%%%%%%%%%%%%%%%%%%%%%%%%%%%%%%%%%%%%%%%%%%%%%%%%%%
\section{Вводная часть}
\subsection{О теории множеств}

\begin{exercise}$\Mq$ не раскладывается в прямую
сумму.
\end{exercise}

\emph{Свободная абелева группа}\index{группа!свободная абелева }
--- прямая сумма некоторого семейства группы $\Mz$. Т.е. $F\cong
\bigoplus\limits_{i\in I}G_i=\{\sum\limits_{finite}n_i x_i\colon
n_i\in\Mz\}=F(\{x_i\}_{i\in I})$. Любая абелева группа
--- факторгруппа некоторой свободной абелевой группы.
Действительно, рассмотрим свободную группу $F(\{x_a\}_{a\in A})$ и
возьмём гомоморфизм $\phi\colon F\rightarrow A$, т.ч. $\sum n_a
x_a\mapsto\sum n_a a$ (здесь мы пользуемся конечностью формальной
суммы).

$A$ ---  \emph{полная}\index{группа!полная} (или
\emph{делимая})\index{группа!делимая}, если для всякого $a\in A$ и
любого $n\in\Mn$ найдётся $b\in A$, такой, что $nb=a$. Например,
$\Mq$.

\begin{fact} Если для любых $a\in A$ и
$p\in \Mp$ существует $b\in A$, такой, что $pb=a$, то $A$ ---
полная.
\end{fact}

\begin{fact} Свободная группа вкладывается в полную
$(F=\bigoplus\limits_{i\in I}\Mz\subset\bigoplus\limits_{i\in
I}\Mq)$.
\end{fact}

\begin{fact} Факторгруппы, прямые слагаемые, прямые суммы,
декартовы произведения полных групп полны.
\end{fact}

\begin{fact}
Объединение возрастающей цепочки полных подгрупп полно.
\end{fact}

Принимается нами

\vskip 2pt\noindent{\bf Аксиома выбора.}\index{аксиома выбора}
{\sl Пусть $I$
--- непустое множество, и $\{X_i\}_{i\in I}$ --- индексированное
семейство непустых множеств. Тогда и произведение
$\prod\limits_{i\in I}X_i$ непусто.}
\\ Её ещё можно переформулировать так:
\par {\sl Пусть $I$ --- непустое множество, и $\{X_i\}_{i\in I}$ ---
индексированное семейство непустых множеств. Тогда существует
такая функция $f$ из $I$ в $\bigcup\limits_{i\in I}X_i$, что при
всех $i\in I$ выполнено $f(i)\in X_i$.}

Чаще нам будем удобнее пользоваться эквивалентным аксиоме выбора
условием --- леммой Цорна. Для её формулировки понадобятся
несколько понятий. \emph{Бинарное
отношение}\index{отношение!бинарное} на множестве $A$ это
подмножество $\mathcal{R}$ декартова произведения $A\times A$. Оно
называется {\em рефлексивным}\index{отношение!рефлексивное}, если

\hfil$\forall a\in A~a\mathcal{R}a$,\hfil\\ {\em
транзитивным}\index{отношение!транзитивное}, если

\hfil$\forall a,b\in A~
(a\mathcal{R}b~\&~b\mathcal{R}c\rightarrow a\mathcal{R}c)$,\hfil\\
{\em антисимметричным}\index{отношение!антисимметричное}, если

\hfil$\forall a,b\in A~ (a\mathcal{R}b~\&~
b\mathcal{R}a\rightarrow a=b)$.\hfil\\ {\em Частично
упорядоченным}\index{множество!частично упорядоченное (ЧУМ)}
множеством (ЧУМ) называется множество вместе с заданным на нём
рефлексивным, транзитивным и антисимметричным бинарным отношением.
{\em Линейно упорядоченным}\index{множество!линейно упорядоченное
(ЛУМ)} множеством (ЛУМ) называется ЧУМ, в котором сравнимы любые
два элемента. {\em Цепь}\index{цепь} в ЧУМ'е это подмножество,
линейно упорядоченное относительно индуцированного порядка. {\em
Максимальным}\index{максимальный элемент} элементом в ЧУМ'е $A$
называется такой элемент $a$, для которого выполнено $\forall b\in
A (b\leqslant a \rightarrow b=a)$.

\vskip 2pt\noindent{\bf Лемма Цорна.}\index{лемма!Цорна} {\sl Если
каждая цепь в непустом ЧУМ'е ограничена сверху, то в нём есть
максимальный элемент.}

\begin{theorem}Во всяком векторном пространстве существует
базис.
\end{theorem}

\begin{proof} Действительно, если у нас есть возрастающая цепь
подмножеств, состоящих из линейно независимых векторов, то она
ограничена сверху --- достаточно взять её объединение и заметить,
что если там есть некоторое соотношение, то оно выполнено и в
некотором из элементов цепи.
\end{proof}

\subsection{Полные абелевы группы}

\begin{theorem}Для всякой
абелевой группы $A$ существует полная абелева группа $B$,
содержащая некоторую подгруппу $\widetilde{A}$, изоморфную $A$.
\end{theorem}

\begin{proof} Для некоторой свободной абелевой $F$ и её подгруппы $H$
$A\cong F/H$. Возьмём $B=\widetilde{F}/H$, где $\widetilde{F}$
--- полная группа, в которую вкладывается $F$. Она полная и $A$,
очевидно, туда вкладывается.\end{proof}

Абелева группа $B$ называется {\em пополнением}\index{пополнение
абелевой группы} абелевой группы $A$, если
\par 1. $A<B$
\par 2. $B$ --- полная
\par 3. Всякая полная группа $C$, для которой $A<C<B$, совпадает с $B$.

\begin{theorem}
Пополнение абелевой группы единственно с точностью до изоморфизма,
тождественного на $A$:
\begin{figure}[h]
\hfil\epsffile{groups1.1}\hfil
\end{figure}\label{ab gr comp}
\end{theorem}

\begin{proof} Пусть $\widetilde{A}$ и $\widehat{A}$
--- пополнения абелевой группы $A$. Рассмотрим гомоморфизмы вида

$$
\Phi=\{\phi\colon H\hookrightarrow\widehat{A}\colon~
A<H<\widetilde{A}\textmd{ и }\phi(A)=A\}.
$$

Положим $\phi_1\leqslant\phi_2$, если
$\Dom\phi_1\subset\Dom\phi_2$ и
$\phi_2\left|_{\Dom\phi_1}\right.=\phi_1$. Рассмотрим какую-нибудь
цепь в $\Phi$:
${\cdots<H_i\stackrel{\phi_i}\rightarrow\widehat{A}<\cdots.}$
Ограничивающий её элемент --- $\bigcup
H_i\stackrel{\phi}\rightarrow\widehat{A}$, такой, что
$\phi(x)=\phi_i(x)$. По лемме Цорна можем найти максимальный
элемент $\phi_0\colon H\rightarrow\widehat{A}$. Пусть
$H=\Dom\phi_0$ и $H\ne \widetilde{A}$. Тогда $H$ --- неполная
группа, т.е. найдутся такие $h_0\in H$ и $p\in\Mp$, что $h_0\notin
pH$. С другой стороны $h_0\in p\widetilde{A}$, т.е. для некоторого
$x\in\widetilde{A}\setminus H$ будет $px=h_0$.

Заметим, что $\langle H,x\rangle\cong H\oplus\langle
x\rangle/\langle (h_0,-px)\rangle$. Для этого рассмотрим
$\alpha:H\oplus\langle x\rangle\rightarrow\widetilde{A}$,
действующий следующим образом: $(h,kx)\mapsto h+kx$. Ясно, что
$\Image\alpha=\langle H,x\rangle$ и
$\ker\alpha\supset\langle(h_0,-px)\rangle$.

Предположим, что $(h,-kx)\in\ker\alpha$, т.е. $h-kx=0$ в
$\widetilde{A}$, т.е. $kx=h\in H$. Если $p|k$, то $k=sp$ и
$h-kx=h-spx=h-sh_0$, значит $h=sh_0$, поэтому $(h,kx)\in\langle
(h_0,-px)\rangle$. Если $(k,p)=1$, то существуют $a,b:~ap+bk=1$,
поэтому $apx+bkx=x$, но и $kx$ и $px$ лежат в $H$, значит и $x\in
H$, что противоречит выбору $x$. Получаем, что
$\ker\alpha=\langle(h_0,-px)\rangle$.

Получается, что в $\phi_0(H)$ не извлекается корень $p$-ой
степени, в то время как в $\widehat{A}$ он извлекается, поэтому
можем взять такой $y$, что $y\notin\phi_0(H)$ и
$\langle\phi_0(H),y\rangle\cong\langle H,x\rangle$, т.е. $\phi_0$
не максимально. Из этого получаем, что $H=\widetilde{A}$. Но тогда
имеем $\Image\phi_0=\widehat{A}$, т.к.\, в $\widehat{A}$ нет
меньших полных подгрупп, содержащих $A$. \end{proof}

%%%%%%%%%%%%%%%%%%%%%%%%%%%%%%%%%%%%%%%%%%%%%%%%%%%%%%%%%%%%%%%%%%%%%%%%%%%%%%%%

\begin{theorem}
Пусть $A$ --- полная абелева группа, и $B$ --- абелева группа,
причём $A<B,~H_0<B$ и $H_0\cap A=\{0\}$. Тогда существует такая
$H<B$, что $H_0<H$ и $B=A\oplus H$.
\end{theorem}

\begin{proof} Рассмотрим семейство $\{H<B:~H_0<H,~H\cap A=\{0\}\}$ Оно
непусто --- $H_0$ там лежит
--- и упорядочено по включению. Очевидно, что всякая цепь
ограничена сверху объединением своих членов (они все лежат в $B$,
содержат $H_0$ и пересекают $A$ по единице). Вообще, если
$\cdots<H_\alpha<\cdots_{\alpha\in\mathcal{A}}$ --- цепь подгрупп,
то $\bigcup\limits_{\alpha\in\mathcal{A}}H_\alpha$ --- группа.

По лемме Цорна находим в этом семействе максимальный элемент $H$.
Т.к.\, $A\cap H=\{0\}$, то $A+H=A\oplus H$.

Покажем, что $A+H=B$. Предположим обратное и возьмём $b\in
B\setminus H\oplus A$. $H<\langle H,b\rangle$, поэтому ввиду
максимальности $H$ $\langle H,b\rangle\cap A\ne\{0\}$. Выберем
$a=h+kb\in\langle H,b\rangle\cap A$, так, чтобы $k$ было
минимальным положительным числом, среди получающихся таким образом
из таких $a$. Понятно, что можно добиться того, чтобы $k$ было
простым, за счёт замены $b$: если $k=mn$, то либо $nb\in A\oplus
H$, либо $m(nb)\in A\oplus H$. $A$ --- полная, значит $a=ka'$.
Положим $b'=a'-b$, для него выполнено $kb'=k(a'-b)=h$. $b'\notin
H\oplus A$ (иначе и $b$ там лежит). Проверим, что $\langle
H,b'\rangle\cap A\ne\{0\}$. Если $0\ne x=h'+k'b'$, то $k'b'\in
A\oplus H$. В случае, если $k|k'$, то $k'b'=skb'=sh\in H$, а тогда
$x\in H$. Если же $(k,k')=1$, то $b'$ выражается через $k'b'$ и
$kb'$, а значит лежит в $A\oplus H$ --- противоречие.\end{proof}

\begin{theorem}Всякая абелева группа $A$ полна \ifif из того, что
$A<B$ следует существование $H$, такой, что $B=A\oplus H$.
\end{theorem}

\begin{proof} $(\Rightarrow)$ --- только что доказали.

$(\Leftarrow)$ $A$ вкладывается в некоторую полную абелеву группу
$B$. По условию $B=A\oplus H$. Но прямое слагаемое в полной
абелевой группе само полно, что и требовалось.\end{proof}

Один должок с прошлой лекции --- существование пополнения в
теореме \ref{ab gr comp}.

\begin{proof} $A$ вкладывается в некоторую полную группу $B$.

Пусть $H=\max\{H<B:~A\cap H=\{0\},~H\mbox{ --- полная}\}$ (лемма
Цорна!). $B=H\oplus C$. $C$ --- полна как прямое слагаемое.
Предположим, что $C$ не минимальна, т.е. есть такая $C'$, что
$A<C'<C$. Тогда $C=C'\oplus C''$ и $B=H\oplus C''\oplus C'$, что
противоречит максимальности $H$, т.к.\,$H\oplus C''\cap A=\{0\}$.
Получаем, что $C''=\{0\}$ и $C'=C$.\end{proof}

\begin{remark}
Пересечение полных подгрупп не всегда полно.
\end{remark}

\begin{example}$\widetilde{\Mz}=\Mq$.
\end{example}

\begin{example}$\widetilde{\Mz_p}=\Mz_{p^\infty}=\cup(\Mz_p\subset\Mz_{p^2}\subset\Mz_{p^3}\subset
\dots)$.
\end{example}

\begin{exercise} Проверить, что это
пополнение.
\end{exercise}
\inst Пусть $x\in\Mz_{p^n}$ и $(q,p)=1$.
Тогда $1=p^nd_1+qd_2$ и $x=xp^nd_1+xqd_2=qd_2x$ $d_2x$ --- искомый
корень. В $\Mz_{p^\infty}$ нет собственных полных подгрупп,
содержащих $\Mz_p$
--- любой простой корень из элемента из $\Mz_{p^k}$ лежит в
$\Mz_{p^{k+1}}$

\begin{theorem}
{$A$ --- полная абелева группа \ifif $A\cong
\bigoplus\limits_{i\in I}C_i$, где $C_i\cong\Mq$ или
$C_i\cong\Mz_{p^\infty}$.}\label{completeness_criterion}
\end{theorem}

\begin{proof} Рассмотрим в $A$ семейства независимых подгрупп, изоморфных
$\Mq$ и $\Mz_{p^\infty}$. Семейства упорядочены по включению.
Любая цепь ограничена сверху объединением. По лемме Цорна возьмём
максимальное семейство $\{C_i\}_{i\in I}$ Т.к.\,$C_i$ ---
независимы, то $\mathop+\limits_{i\in I}C_i=\bigoplus\limits_{i\in
I}C_i<A$

$\bigoplus\limits_{i\in I}C_i$ --- полная как прямая сумма полных.
Значит она в $A$ выделяется прямым слагаемым:
$A=\bigoplus\limits_{i\in I}C_i\oplus H$. Возьмём $h\in H$.
Возводя $h$ в степень, можно считать, что $h$ либо простого, либо
бесконечного порядка. $\widetilde{\langle h\rangle}<H$ --- либо
$\Mq$, либо $\Mz_{p^\infty}$. Но тогда имеем независимое семейство
$\{C_i\}_{i\in I}\cup\{\widetilde{\langle h\rangle}\}$ ---
противоречие с выбором $\{C_i\}_{i\in I}$. Поэтому $h=0$. Значит
$H=\{0\}$ и $A=\bigoplus\limits_{i\in I}C_i$.\end{proof}

\begin{exercise} Подгруппы $\Mz_{p^\infty}$ --- это только
$\Mz_p$.
\end{exercise}

\begin{theorem}Число неизоморфных подгрупп в $\Mq$ имеет мощность
$\mathfrak{c}$.
\end{theorem}

\begin{proof} Понятно, что их количество не больше, чем $\mathfrak{c}$,
т.к.\,оно не превосходит количества подмножеств в $\Mq$.

Рассмотрим подгруппы вида
$$
A_\pi=\{\frac{m}{n}:~q|n,~q\in\pi,~\pi\subset \Mp\}
$$
Пусть $A_\pi\cong A_{\pi'}$ и $p\in\pi$. Всегда $1\in A_\pi$. В
$A_\pi$ извлекается корень $p$-ой степени, значит $\frac{1}{p}\in
A_{\pi'}$, т.е. $p\in\pi'$, поэтому $\pi=\pi'$. Т.к.\,$|2^\Mp|=\,$
$\mathfrak{c}$, то подгрупп не меньше, чем
$\mathfrak{c}$.\end{proof}

\begin{remark}
Это не все подгруппы в $\Mq$.
\end{remark}

\begin{theorem}Существует
универсальная счётная абелева группа\index{группа!универсальная
счётная абелева} (т.е. такая счётная абелева группа, что все
счётные абелевы группы в неё вкладываются).
\end{theorem}

\begin{proof}
$\bigoplus\limits_{i\in\Mn}\Mq\oplus\bigoplus\limits_{i\in\Mn}C_i$
--- искомая.\end{proof}

\begin{remark}
Если опустить слово 'абелева', то утверждение становится неверным.
\end{remark}

\begin{remark}
Если слово 'счётная' заменить на какой-нибудь другой кардинал, то
утверждение останется верным.
\end{remark}

Абелева группа называется {\em
редуцированной}\index{группа!редуцированная абелева}, если она не
содержит полных нетривиальных подгрупп.

Если $G$ --- абелева, то можно найти максимальную полную подгруппу
$B$ и $G=B\oplus H$. Тогда $H$ --- редуцирована. Таким образом,
доказана

\begin{theorem}Любая абелева группа раскладывается в
прямое произведение полной и редуцированной.$\square$
\end{theorem}

\subsection{Категорный язык}

Говорят, что задана \emph{категория}\index{категория} $\Ck$, если:
\par $1^\circ$. Задан класс (т.е. не обязательно множество)
{\em объектов}\index{объект} $\Ob\Ck$.
\par $2^\circ$. Для каждой пары объектов $X,Y\in\Ob\Ck$ задан
класс {\em морфизмов}\index{морфизм} (ещё называемых {\em
стрелками}\index{стрелка}) $\Mor_{\Ck}(X,Y)$, (ещё обозначаемые
$\Hom\Ck$ и $\Arr\Ck$) и для всякого морфизма $f\in\Mor\Ck$
определены объекты $\dom f$ ({\em домен}\index{домен} $f$) и $\cod
f$ ({\em кодомен}\index{кодомен} $f$), такие, что
$f\in\Mor_{\Ck}(A,B)$.
\par $3^\circ$. Для всех $X,Y,Z\in\Ob\Ck$ задано отображение $\Mor_{\Ck}(Y,Z)
\times\Mor_{\Ck}(X,Y)\to\Mor_{\Ck}(X,Z)$ (композиция).
\par $4^\circ$. Если $X\stackrel{\phi}\to Y$,
$Y\stackrel{\psi}\to Z$ и $X\stackrel{\chi}\to Z$, то
$(\chi\circ\psi)\circ\phi=\chi\circ(\psi\circ\phi)$.
\par $5^\circ$. У каждого объекта $X\in\Ob\Ck$ существует такой
{\em тождественный}\index{морфизм!тождественный} морфизм
$1_X\in\Mor_{\Ck}(X,X)$, что при всех $\phi:X\to Y$ и $\psi:Y\to
X$ выполнено $\phi\circ 1_X=\phi$ и $1_X\circ\psi=\psi$. Можно
обозначать тождественные морфизмы так $\id X$.

\begin{example}
Категория $\Set$:\index{категория!$\Set$} объектами являются
множества, морфизмами
--- отображения. Категория $\Group$\index{категория!$\Group$}: объекты --- группы, стрелки
--- гомоморфизмы. Категория $\Top$\index{категория!$\Top$}: объекты --- топологические
пространства, стрелки --- непрерывные отображения. Моноид можно
рассматривать как одноэлементную категорию, там стрелки ---
элементы моноида.
\end{example}

Объект $A\in\Ob\Ck$ называется {\em
инициальным}\index{объект!инициальный}, если для каждого объекта
$B\in\Ob\Ck$ найдётся единственный морфизм $f:A\to B$. В $\Set$
таковым объектом является пустое множество, а в $\Group$
--- тривиальная группа.

{\em Мономорфизмом}\index{мономорфизм} называется такой морфизм
$f$, что для всех стрелок $h_1$ и $h_2$, для которых $f\circ
h_1=f\circ h_2$ выполнено $h_1=h_2$. На диаграммах мономорфизмы
часто обозначают $\rightarrowtail$. Легко проверить, что в $\Set$
это просто инъективные отображения, а в $\Group$ --- инъективные
гомоморфизмы.

{\em Изоморфизмом}\index{изоморфизм} называется такая стрелка
$f:A\to B$, что существует стрелка $f':B\to A$ такая, что $f\circ
f'=\id B$ и $f'\circ f=\id A$. Соответственно, два объекта
называются изоморфными, если между ними устанавливается
изоморфизм. Легко увидеть, что все инициальные объекты изоморфны.

Категория $\op{\Ck}$ называется {\em
двойственной}\index{категория!двойственная} к категории $\Ck$,
если $\Ob\op{\Ck}=\Ob\Ck$, $\Mor\op{\Ck}=\Mor\Ck$, причём для всех
объектов $A,B\in\Ob\Ck$ $\Mor_{\op{\Ck}}(A,B)=\Mor_{\Ck}(B,A)$ и
закон композиции выглядит так: $g\circ_{\Ck}f=f\circ_{\op{\Ck}}g$.
Естественно имеем $\op{(\op{\Ck})}=\Ck$.

{\em Эпиморфизмы}\index{эпиморфизм} в $\Ck$ это монострелки в
$\op{\Ck}$. На диаграммах эпиморфизмы часто обозначают
$\twoheadrightarrow$. {\em
Терминальные}\index{объект!терминальный} объекты в $\Ck$ это
инициальные объекты в $\op{\Ck}$.

Объект $P$ категории $\Ck$ называется {\em
проективным}\index{объект!проективный}, если для каждых
эпиморфизма $\alpha:A\to B$ и стрелки $f:P\to B$ существует
морфизм $g:P\to A$, такой, что $f=\alpha g$. Двойственным образом
определяются инъективные объекты.

\begin{exercise} Показать, что проективность и инъективность в классе
абелевых групп равносильны свободе и полноте соответственно.
\end{exercise}

%%%%%%%%%%%%%%%%%%%%%%%%%%%%%%%%%%%%%%%%%%%%%%%%%%%%%%%%%%%%%%%%%%%%%%%%%%%%%%%%%%%
Прямая сумма полных групп полна. Двойственное утверждение не
верно.

\vskip-12pt \inspicright{groups5} \hangindent=-30mm \hangafter=-5
{\em Произведением}\index{произведение!объектов} семейства
объектов $\{X_s\}_{s\in S}$ называется такой объект $X$ вместе с
семейством морфизмов $p_s:X\to X_s$, что для любых объекта $M$ и
семейства стрелок $f_s:M\to X_s$ существует единственный морфизм
$f$, такой, что при всех $s\in S$ выполнено $f_s=p_sf$.
Двойственным образом определяется {\em
копроизведение}\index{копроизведение!объектов}.

\begin{exercise}
Построить копроизведение в $\mathfrak{Group}$.
\end{exercise}

\begin{exercise} $\prod\limits_\omega \Mz$ --- не свободна в
$\mathfrak{AbGroup}$.
\end{exercise}

\subsection{Первая теорема Прюфера}

$T(A)=\{x\in A:~\exists n\in N:~nx=0\}$ --- {\em периодическая
часть}\index{периодическая!часть} $A$. $T(A)<A:~(a,b\in
T(A)\rightarrow a+b\in T(A))$.

\begin{remark}
Для неабелевых групп последнее утверждение неверно.
\end{remark}

$A/T(A)$ --- без кручения (если есть смежный класс конечного
порядка $(aT)^n\subset T$, то $a^nT\subset T$ и $a^n\in T$, а
значит и $a$ лежит в $T$, следовательно $aT=T$).

\begin{example}\index{пример!когда $T(A)$ не выделяется прямым слагаемым}
$T(A)$ может не выделяться прямым слагаемым.

Если $A=\prod\limits_{p\in\Mp}\Mz_p$, то
$T(A)=\bigoplus\limits_{p\in\Mp}\Mz_p$. Предположим, что
$A=T(A)\oplus C$. $C\cong
A/T(A)=\prod\limits_{p\in\Mp}\Mz_p/\bigoplus\limits_{p\in\Mp}\Mz_p$
--- это элементы $\prod\limits_{p\in\Mp}\Mz_p$ с точностью до
конечного числа координат. $A/T(A)$ --- полная, поскольку если
$\{x_\alpha\}_{\alpha\in\Mp}\in A/T(A)$, то по всем координатам,
отличным от $p$, извлекается корень $p$-ой степени. Покажем, что в
$\prod\limits_{p\in\Mp}\Mz_p$ нет полных нетривиальных подгрупп.
Если $\{t_\alpha\}_{\alpha\in\Mp}\ne 0$, то найдётся $p$, такой
что $t_p\ne 0$, значит из $\{t_\alpha\}$ не извлекается корень
$p$-ой степени.$\square$
\end{example}

Группа $A$ --- {\em периодическая}\index{периодическая!группа},
если существует $n$, такой что $nA=1$.

\begin{theorem}Если $A$ --- периодическая абелева группа
с периодом $n$, то $A=\bigoplus\limits_{i\in I}C_i$, где $C_i$
--- примарные циклические.
\end{theorem}

\begin{proof} Обозначим $T_p(A)=\{x\in A:~p^kx=0\mbox{ для
некоторого}~k\}$ --- {\em $p$-компонента}\index{$p$-компонента}
группы $A$. $T_p(A)<A$ и, кроме того,
$T(A)=\bigoplus\limits_{p\in\Mp}T_p(A)$. Поэтому можно считать,
что $A$ --- $p$-группа и $n=p^k$. Проведём индукцию по $n$.

Рассмотрим независимые семейства подгрупп
$\{C_i\cong\Mz_{p^k}\}_{i\in I}$. Они упорядочены по включению.
Также выполнено условие леммы Цорна
--- всякая упорядоченная цепь ограниченна сверху её
объединением.

Пусть $\{C_i\}_{i\in I}$
--- максимальное семейство. Рассмотрим $\bigoplus\limits_{i\in I}C_i<A$ (сумма прямая ввиду
независимости). Возьмём $H=\max\{H<A:~H\cap\bigoplus\limits_{i\in
I}C_i=\{0\}\}$, и обозначим $Y=\bigoplus\limits_{i\in I}C_i$.
Покажем, что $A=Y\oplus H$. Поскольку очевидно, что сумма прямая,
достаточно показать, что просуммировав получим всё $A$.

Возьмём тот $x\in A\setminus(Y\oplus H)$, для которого такой
$l>0$, что $lx\in H\oplus Y$ минимален. $l$ --- простое; это
делается также, как в теореме \ref{completeness_criterion}. Тогда
$l=p$. Можем взять ненулевой $y\in(H+\langle x\rangle) \cap Y$;
для него $y=h+mx$. Понятно, что $p|m$, поэтому путём замены $x$ на
некоторый $dx$ можем считать, что $y=h+px$.

$y=py'$ (иначе по построению $Y$ $y$ имеет порядок $p^k$ и
$O(h-y)=p^k$, а значит и $O(px)=p^k$, т.е. $O(x)=p^{k+1}$ --- а
это противоречие с тем, что период $n=p^k$). Поэтому из $y$
извлекается корень $p$-ой степени. Имеем $p(x-y')=h$, $x-y'=x'$.
$x'\notin Y\oplus H$, иначе $x=x'+y'\in H\oplus Y$. Получаем, что
$H+\langle x'\rangle\cap Y=\{0\}$, т.е. $H$ не максимально.

Если бы в $H$ был элемент порядка $p^k$, то семейство $\{C_i\}$ не
было бы максимальным, значит к $H$ применима индукция.\end{proof}

\section{Нильпотентные группы}
\subsection{Основные свойства}

Обозначим $a^x=x^{-1}ax$.
\\Тогда
\begin{center}
$(a^x)^y=a^{xy}$\\
$a^xb^x=(ab)^x$\\
$(a^x)^n=(a^n)^x=a^{nx}$\\
$a^{-x}=(a^{-1})^x$
\end{center}
{\em Коммутатор}\index{коммутатор} элементов $a$ и $b$ ---
$[a,b]=a^{-1}b^{-1}ab=b^{-a}b=a^{-1}a^b$. \\Обозначим
$[a_1,\dots,a_n]=\big[\dots\big[[a_1,a_2]a_3\big]\dots\big]$.
Имеют место следующие равенства:
\begin{center}
$ab=ba[a,b]$
\\ $[a,b]^{-1}=[b,a]$
\\ $[a,bc]=[a,c][a,b][a,b,c],$
\\ $[ab,c]=[a,c][a,c,b][b,c]$
\\ $[a,b,c^a][c,a,b^c][b,c,a^b]=1$ (тождество Якоби)\index{тождество!Якоби}
\end{center}
Если $A,B\lhd G$, то $A\cap B\lhd G$ и $AB\lhd G$
($a_1b_1a_2b_2=a_1a_2b_1^{a_2}b_2$). $[A,B]=\langle[a,b]:~a\in
A,~b\in B\rangle$. $[A,B,C]=\langle\{[a,b,c]:~a\in A,~b\in B,~c\in
C\}\rangle$

\begin{exercise} Привести пример, когда $[A,B]\ne\{[a,b]:~a\in
A,~b\in B\}$.
\end{exercise}

$[A,B]\lhd G,~[A,B]=[B,A]<A\cap B$. Заметим также, что если ещё
$C\lhd G$, то $[A,B,C]\subset[C,A,B]\cdot[B,C,A]$ (по тождеству
Якоби).
%%%%%%%%%%%%%%%%%%%%%%%%%%%%%%%%%%%%%%%%%%%%%%%%%%%%%%%%%%%%%%%%%%%%%%%%%%%%%%%%%%%%%%%%%%%

$\langle
X\rangle=\{\prod\limits_{finite}x_i^{\varepsilon_i}:~x_i\in X,
~\varepsilon_i=\pm 1\}$
\\ $\ncl{X}_G=\{\prod\limits_{finite}x_i^{g_i\varepsilon_i}:~x_i\in
 X,~g_i\in G,~\varepsilon_i=\pm 1\}$
--- {\em нормальная подгруппа, порождённая $X$ {\em (\/}нормальное замыкание\index{нормальное замыкание}{\em )\/}}.

\begin{example} Если $A=\langle X\rangle$ и $B=\langle Y\rangle$, то
$[A,B]=\ncl{\{[x,y]:~x\in X,~y\in Y\}}=C$.

Включение $C\subset[A,B]$ очевидно. Чтобы убедиться в
справедливости обратного включения рассмотрим $G/C$. В этой группе
$a^{-1}b^{-1}ab=(x_1\dots x_k)^{-1}(y_1\dots y_m)^{-1}x_1\dots
x_ky_1\dots y_m=$ \par\hfill$=(x_1\dots x_k)^{-1}{y_m}^{-1}\dots
y_2^{-1}x_1\dots x_ky_2\dots y_m=\dots=e$.$\square$
\end{example}

\begin{example} Если $A=\ncl{X}$ и $B=\langle Y\rangle$, то $[A,B]=\ncl{\{[x,y]:~x\in X,~y\in
Y\}}=C$.

Опять же, нас интересует включение $[A,B]\subset C$. Рассмотрим
$G/C$. Для любых $x\in X$ и $y\in Y$ $[x,y]=e\pmod C$, Значит
$[x,B]=e\pmod C$ и $[x^g,B^g]=e\pmod C$. Т.к.\,$B^g=B\pmod{C}$, то
всякий множитель в $\prod {x_i}^{g_i\varepsilon_i}$ коммутирует с
$B$. Окончательно имеем $[A,B]=e\pmod C$, т.е. $[A,B]\subset
C$.$\square$
\end{example}

$G'=[G,G]$ --- {\em коммутант}\index{коммутант};
$G^{(m)}=[G^{(m-1)},G^{(m-1)}]$
--- {\em $m$-й коммутант\index{$m$-й коммутант}}.
\\ $\gamma_1(G)=G$ --- {\em 1-й централ}; $\gamma_{(m)}(G)=[\gamma_{(m-1)}(G),G]$
--- {\em $m$-й централ\index{$m$-й централ}}.

$G=\gamma_1(G)\rhd\gamma_2(G)\rhd\gamma_3(G)\rhd\dots$
--- нижний центральный ряд\index{ряд!нижний центральный}.

\begin{lemma}
{$[\gamma_i(G),\gamma_j(G)]\subset\gamma_{i+j}(G)$.}
\end{lemma}

\begin{proof} Индукция по $\min(i,j)$; считаем, что $j\leq i$. Пусть
$j<i$.
$[\gamma_i,\gamma_j]=\big[\gamma_i,[\gamma_{j-1},G]\big]=[\gamma_{j-1},G,\gamma_i]\subset
{\subset\big[[G,\gamma_i],\gamma_{j-1}\big]}\cdot\big[[\gamma_i,\gamma_{j-1}],G\big]\subset[\gamma_{i+1},
\gamma_{j-1}]
\cdot[\gamma_{i+j-1},G]\subset\gamma_{i+j}\cdot\gamma_{i+j}\subset\gamma_{i+j}$.\end{proof}

Применяя индукцию, легко получить, что
$[x_1,\dots,x_n]\in\gamma_n$.

\vskip 2pt Пусть $a\in\gamma_i,~b\in\gamma_j,~c\in\gamma_k$, тогда
$[a,b]=[b,a]^{-1}$, $[a,bc]=[a,c][a,b][a,b,c]=
[a,b][a,c]\big[[a,c],[a,b]\big][a,b,c]=[a,b][a,c]\pmod{\gamma_{i+j+k}}$,
поскольку $[a,b,c]\in\gamma_{i+j+k}$ и
$\big[[a,c],[a,b]\big]\in\gamma_{2i+j+k}$, а также
$[a,b,c^a]=\big[[a,b],c[c,a]\big]=\big[[a,b],c\big]
\big[[a,b],[c,a]\big]\pmod{\gamma_{j+i+k+i+k}}$. Отсюда имеем:
\begin{center}
$ab=ba\pmod{\gamma_{i+j}}$
\\ $[a,bc]=[a,b][a,c]\pmod{\gamma_{i+j+k}}$
\\ $[ab,c]=[a,c][b,c]\pmod{\gamma_{i+j+k}}$
\\ $[a,b,c^a]=[a,b,c]\pmod{\gamma_{i+j+k+1}}$
\\ $[a,b,c][c,a,b][b,c,a]=1\pmod{\gamma_{i+j+k+1}}$
\end{center}

\begin{remark} Получаются аксиомы кольца Ли!\index{кольцо!Ли}
\end{remark}

$G$ --- {\em нильпотентная группа\index{группа!нильпотентная}
ступени $s$}\index{ступень!нильпотентности}, если
$\gamma_{s+1}(G)=1$ и $\gamma_s\ne 1$

\begin{fact} Подгруппа нильпотентной группы
нильпотентна, причем не большей ступени.
\end{fact}

\begin{fact}
Факторгруппы нильпотентной группы нильпотентны.
\end{fact}
\inst Построить естественный эпиморфизм.

\begin{fact} Декартово и
прямое произведения семейства нильпотентных групп нильпотентны,
если их ступени нильпотентности ограниченны в совокупности.
\end{fact}

\begin{fact} Центр
нильпотентной группы нетривиален.
\end{fact}
\inst Это так, поскольку $e=[\gamma_s,G]$, то $\gamma_s\subset
Z(G)$.

\begin{fact} Конечные
$p$-группы нильпотентны.
\end{fact}
\inst Центр нетривиален, применима индукция.

\begin{fact} Если фактор по центру нильпотентен, то и
вся группа нильпотентна на единицу большей ступени.
\end{fact}
\inst Имеем: $G/Z\rhd\gamma_2\rhd\dots\rhd\gamma_s=1$. Пусть
$\pi:G\rightarrow G/Z$ --- естественный эпиморфизм, тогда
$G\rhd\pi^{-1}\gamma_2\rhd\dots\rhd Z\rhd 1$.

\begin{example}
$\mathrm{UT}_n(F)$.
$$
\left(%
\begin{array}{ccc}
  1&\overbrace{0\dots 0}^{k}&*\\
  &&0\\
  &\ddots&\vdots\\
  &&0\\
  0&&1\\
\end{array}%
\right)\cdot
\left(%
\begin{array}{ccc}
  1&&*\\
  &\ddots&\\
  0&&1\\
\end{array}%
\right)=
\left(%
\begin{array}{ccc}
  1&\overbrace{0\dots 0}^{k+1}&*\\
  &&0\\
  &\ddots&\vdots\\
  &&0\\
  0&&1\\
\end{array}%
\right)
$$
\end{example}

\begin{fact} Нильпотентные группы разрешимы.
\end{fact}

\begin{fact} Обратное не верно. Пример --- $S_3$ --- её центр тривиален.
\end{fact}

\noindent{\bf Группа Гейзенберга.}\index{группа!Гейзенберга} Это
группа $\langle
a,b,c|~cac^{-1}a^{-1},~cbc^{-1}b^{-1},~b^{-1}a^{-1}bac\rangle\cong\mathrm{UT}_3(\Mz)$
$$
a^ib^jc^k\mapsto
\left(%
\begin{array}{ccc}
  1&j&k\\
  0&1&i\\
  0&0&1\\
\end{array}%
\right)
$$
$(a^ib^jc^k)a=a^{i+1}b^jc^{k+j}\\(a^ib^jc^k)b=a^ib^{j+1}c^k\\(a^ib^jc^k)c=a^ib^jc^{k+1}$

Кольцо $N$ {\em нильпотентно\index{кольцо!нильпотентное}
{\em(\/}ступени не больше $s${\em)\/}}, если для любых
$x_1,\dots,x_{s+1}\in N$ выполнено $x_1\cdot\dots\cdot x_{s+1}=0$.

\begin{exercise} Пусть $R$ --- ассоциативное кольцо с единицей, $N$ ---
его нильпотентное подкольцо. Тогда $(1+N)^\times$ - нильпотентная
группа.
\end{exercise}

Если $G=\langle X\rangle$, то $\gamma_i=\ncl{\{[x_1,\dots,x_i]:
~x_k\in X\}}=\langle\{[x_1,\dots,x_i]:~x_k\in
X\}\rangle\gamma_{i+1}$. Действительно,
$\langle\{[x_1,\dots,x_i]:~x_k\in
X\}\rangle\gamma_{i+1}\subset\gamma_i,~\langle\{[x_1,\dots,x_i]:~x_k\in
X\}\rangle\gamma_{i+1}\lhd G$. Если $G$ --- конечно порождена, то
$\gamma_i/\gamma_{i+1}$ --- конечно порождённая абелева группа.
Тем самым, имеет место следующая

\begin{theorem}Если $G$ конечно
порождённая нильпотентная группа, то любая её подгруппа $H<G$ тоже
конечно порождена.
\end{theorem}

\begin{proof} $H=\bigcup\limits_i H\cap\gamma_i$. Заметим, что
$\frac{H\cap\gamma_i}{H\cap\gamma_{i+1}}\subset\frac{\gamma_i}{\gamma_{i+1}}$.
Поэтому можем написать $H\cap\gamma_1=\\=\langle
y_1,\dots,y_{n_1}\rangle H\cap\gamma_2= \langle
y_1,\dots,y_{n_1}\rangle\langle y'_1,\dots,y'_{n_2}\rangle
H\cap\gamma_3=...=\langle y_1,\dots,y_{n_k}^{(k)}\rangle$, что и
доказывает теорему.\end{proof}
%%%%%%%%%%%%%%%%%%%%%%%%%%%%%%%%%%%%%%%%%%%%%%%%%%%%%%%%%%%%%%%%%%%%%%%%%%%%%%%

\begin{example}\index{пример!не конечно порождённой подгруппы в конечно порождённой группе}
Подгруппа конечно
порождённой группы не обязательно конечно порождена.

Рассмотрим группу перестановок $\Mz$. $S(\Mz)>\langle
c,(1,2)\rangle$, где $c(n)=n+1$.
\\ $\langle c^{2k}(1,2)c^{-2k}\rangle_{k\in\Mz}=\langle(2k+1,2k+2)\rangle_{k\in\Mz}=
\langle\dots,(1,2),(3,4),\dots\rangle<\langle c,(1,2)\rangle$
\\ $\langle\dots,(1,2),(3,4),\dots\rangle$ --- абелева, т.к.\,циклы независимы. Поскольку все
элементы имеют порядок 2, то из её конечной порожденности
следовала бы её конечность.$\square$
\end{example}

В абелевой группе $x^ny^n=(xy)^n$.
%%%%%%%%%%%%%%%%%%%%%%%%%%%%%%%%%%%%%%%%%%%%%%%%%%%%%%%%%%%
\begin{lemma}
Если $G$ нильпотентная группа ступени $s>2$ и $a\in G$, то
$\langle a,G'\rangle$ --- нильпотентная группа ступени меньше $s$.
\end{lemma}

\begin{proof} Обозначим $H=\langle a,G'\rangle$. $G'<z_{s-1}(G)\cap
H<z_{s-1}(H)$, поэтому $H/z_{s-1}(H)$ циклическая. Поскольку
группа неабелева, то фактор тривиален, т.е. $z_{s-1}(H)=H$.
\end{proof}

\begin{theorem}
Если $G$
--- нильпотентная группа без кручения, и $a^n=b^n$, то
$a=b$.
\end{theorem}

\begin{proof} Доказывать будем индукцией по ступени нильпотентности. База
тривиальна. Группа $\langle a,G'\rangle$ нормальна (любая
подгруппа, содержащая коммутант нормальна: $g^{-1}xg=x\pmod{G'}$)
и имеет меньшую ступень нильпотентности. $a,a^b\in\langle
a,G'\rangle$ и $(a^b)^n=a^n$, ($b^{-n}a^nb^n=a^n\Rightarrow
(a^n)^b=b^{n-1}a^nb^{1-n}$, откуда, сворачивая, получим требуемое)
откуда по индукции получаем $a^b=a$, значит ввиду их
перестановочности $(ab^{-1})^n=1$, и т.к.\,в $G$ нет кручения, то
$a=b$.\end{proof}

\begin{corollary}\label{from pow to comm}
Если $G$
--- нильпотентная группа без кручения, то из $x^my^n=y^nx^m$
следует $[x,y]=1$.
\end{corollary}

\begin{proof} Напишем цепочку следствий: $x^my^n=y^nx^m\Rightarrow
(y^{-n}xy^n)^m=x^m\Rightarrow y^{-n}xy^n=x\Rightarrow
(x^{-1}yx)^n=y^n\Rightarrow x^{-1}yx=y$. \end{proof}
%%%%%%%%%%%%%%%%%%%%%%%%%%%%%%%%%%%%%%%%%%%%%%%%%%%%%%%%

\begin{theorem}В нильпотентной группе $G$ ступени $s$ к любым $x$ и
$y$ найдётся такой $z$, что $x^{n^s}y^{n^s}=z^n$.
\end{theorem}

\begin{proof} Покажем, что $\langle G^{n^s}\rangle\subset
G^n=\{g^n\}_{g\in G}$, отсюда будет следовать утверждение леммы.
Доказывать будем индукцией по $s$.

$\langle G^n\rangle/\gamma_s(\langle G^n\rangle)$ ---
нильпотентная группа ступени не больше $s-1$ (как фактор по
$s$-ому члену нижнего центрального ряда). Для неё свойство
выполнено, т.е. для каждого $x\in\langle\!\langle
G^n\rangle^{n^{s-1}}\rangle$ найдётся $y$, для которого
$x=y^n\pmod{\gamma_s}$ или $x=y^nz,~z\in\gamma_s(\langle
G^n\rangle)$. Для $z$ можем написать
$z=\prod\limits_i[t_i,g_i^n]=\prod\limits_i[t_i,g_i]^n$, поскольку
для $n=0,1$ всё и так тривиально, а при $n>1$ имеем
$[t,g^n]=[t,g]^n\pmod{\gamma_{s+1}}=[t,g]^n$. Далее. $s$-ый член
нижнего центрального ряда лежит в центре, значит последнее
произведение равно $(\prod\limits_i[t_i,g_i])^n$ и можно записать
$x=y^nz=y^nw^n=(yw)^n$.\end{proof}

Группа $G$ называется {\em
$\pi$-примарной}\index{группа!$\pi$-примарная}, где $\pi\in\Mp$,
если порядок всякого элемента из $G$ делится только на простые
числа из $\pi$.

\begin{theorem}
{$G$ --- нильпотентная. Следующие свойства переносятся с $G/G'$ на
$G$:
\par $\left.0\right)$ нетривиальность;
\par $\left.1\right)$ порождённость множеством $X$, т.е. если $X\subset
G$ такое, что $\langle X\rangle=G\pmod{G/G'}$ (т.е. ${\langle
X\rangle G'=G}$), то $\langle X\rangle$=G;
\par $\left.2\right)$ $\pi$-примарность;
\par $\left.3\right)$ конечность;
\par $\left.4\right)$ полнота.}\label{from quotient to group}
\end{theorem}

\begin{proof} $\left.1\right)$ Индукция по членам нижнего центрального
ряда: дано $G=\gamma_1=\langle X\rangle\gamma_2$. Пусть
$\gamma_n=\langle X,\gamma_{n+1}\rangle$ при всех $n$ меньших $k$.
Тогда
$\gamma_k=\langle{[\gamma_{k-1},X],[\gamma_{k-1},\gamma_k]}\rangle\gamma_{k+1}$,
но $[\gamma_{k-1},\gamma_k]\subset\gamma_{k+1}$, и
$\gamma_{k-1}=\langle X,\gamma_k\rangle$. Поэтому
$\gamma_k=\langle X\rangle\gamma_k$ для всех $k$; спускаясь до
$\gamma_{s+1}$ будем иметь утверждение теоремы.

$\left.2\right)$ Индукция по членам нижнего центрального ряда:
нужно доказать, что $G/\gamma_s(G)$ --- $\pi$-примарна, т.е. что
для каждого $g\in G$ существует $n$ --- $\pi$-число, такое, что
$g^n\in\gamma_s$. Для этого достаточно показать, что
$\gamma_i/\gamma_{i+1}$ --- $\pi$-примарна при всех $i$.
$[\gamma_i,\gamma_i]\subset\gamma_{i+1}$ стало быть, $\gamma_i$
абелева по модулю $\gamma_{i+1}$. Для абелевых групп достаточно
доказывать $\pi$-примарность образующих. Пусть $x\in\gamma_{i-1}$
и $y\in G$, a $n$ --- такой, что $y^n\in\gamma_2$. Рассмотрим
$[x,y]$ ($\gamma_i=[\gamma_{i-1},G]$). Тогда
$[x,y]^n=[x,y^n]\pmod{\gamma_{i+1}}=$

\hfill$=1\pmod{\gamma_{i+1}}$.

$\left.3\right)$ Из $\left.1\right)$ имеем конечную порождённость.
В то же время $G/G'$
--- конечная абелева группа (значит и периодична), т.е. по доказательству $\left.2\right)$
и все факторы нижнего центрального ряда периодичны абелевы и
конечно порождены, т.е. конечны. Тем самым, поднимаясь по членам
нижнего центрального ряда, можем получить конечность $G$.

$\left.4\right)$  Группа $G$ полна, когда при всех $n$ $G\subset
G^n$. Мы знаем, что $\langle G^{n^s}\rangle\subset G^n$. Теперь
заметим, что $\langle G^{n^s}\rangle=G$, что следует из 1) и
полноты $G/G'$: $G/G'\subset(G/G')^{n^{s}}$.\end{proof}

Пусть $G$ --- нильпотентная группа. Рассмотрим $H=\{x:~\exists
n:~x^n=1\}\subset G$, тогда $\langle H\rangle$
--- нильпотентная подгруппа, порождённая элементами конечного порядка. Тогда
$\langle H\rangle/\langle H\rangle'$ --- абелева группа,
порождённая элементами конечного порядка, поэтому применяя теорему
\ref{from quotient to group} получим равенство $H=\angle{H}$, что
означает, что периодическая часть выделяется подгруппой.
Аналогично можем получить $\pi$-примарную часть.

Говорят, что группа $G$ {\em почти обладает
свойством}\index{группа!почти обладает свойством} $\mathcal{P}$,
если в ней существует подгруппа конечного индекса, обладающая
свойством $\mathcal{P}$.

\begin{theorem}Конечно порождённая нильпотентная группа почти без
кручения.
\end{theorem}

\begin{proof} Итак, в нильпотентной группе $T(G)\lhd G$ (порядки
сопряжённых элементов равны). Рассмотрим $H=\langle
G^{(\mbox{\scriptsize период }T(G))^s}\rangle\subset
G^{\mbox{\scriptsize период }T(G)}$.\marginpar{непонятно, почему
существует период T(G)} $G^{\mbox{\scriptsize период }T(G)}$ без
кручения, значит и $H$ без кручения, кроме того, $H\lhd G$. $G/H$
--- периодическая конечно порождённая, а её фактор по коммутанту ещё и абелев, значит
конечен, поэтому $H$ --- искомая подгруппа конечного
индекса.\end{proof}

\begin{theorem}Если $G$
--- конечная нильпотентная группа, то $p$-примарные части
выделяются в качестве подгрупп.
\end{theorem}

\begin{proof} $T_p(G)\lhd G$, $T_{p_1}(G)\cdot T_{p_2}(G)\cdot\dots=G$,
$T_{p_i}\cap\bigcup\limits_{j\ne i} T_{p_j}=\{1\}$, поэтому
$G=\bigoplus\limits_{i}T_{p_i}$ --- прямое произведение конечных
$p$-групп.\end{proof}

%%%%%%%%%%%%%%%%%%%%%%%%%%%%%%%%%%%%%%%%%%%%%%%%%%%%%%%%%%%%%%%%%%%%%%%%%%%%%
Пусть $z_1(G)=Z(G)$ и $\pi :G\rightarrow G/Z(G)$
--- естественная проекция. \marginpar{\sf\footnotesize 24.10.03}
\\Положим $z_k(G)=\pi^{-1}(Z(G/z_{k-1}(G)))$ --- {\em гиперцентралы}\index{гиперцентрал}.

\begin{theorem}Имеют место следующие включения:
$$
\begin{array}{cccccccc}
z_1&\leq&z_2&\leq&\dots&\leq&z_s&=G\\
\vee&&\vee&&&&\vee&\\
\gamma_s&\leq&\gamma_{s-1}&\leq&\dots&\leq&\gamma_1&=G.
\end{array}
$$
Верхний ряд называется соответственно верхним
центральным\index{ряд!верхний центральный} рядом.
\end{theorem}

\begin{proof} Как мы уже видели выше, $z_1\supset\gamma_s$. Пусть
$z_{k-1}\supset\gamma_{s-k+2}$. Тогда
$[G,\gamma_{s-k+1}]=\gamma_{s-k+2}\subset z_{k-1}$ и поэтому
$z_k\supset\gamma_{s-k+1}$. Действительно, если $x\in G\setminus
z_k$, то найдётся $y\in G$ такой, что $xz_{k-1}yz_{k-1}\ne
yz_{k-1}xz_{k-1}$, т.е. $x^{-1}y^{-1}xy\notin z_{k-1}$, а значит
$\gamma_{s-k+1}\cap G\setminus z_{k}=\varnothing$.\end{proof}

Ряд $G=G_1\rhd G_2\rhd\dots$ называется {\em
центральным}\index{ряд!центральный}, если $[G_i,G]\subset
G_{i+1}$.

\begin{exercise}
Любой центральный ряд заключён между нижним и верхним:
$z_k\supset\sigma_{s-k}\supset\gamma_{s-k+1}$.
\end{exercise}

\begin{exercise}
Нильпотентность группы равносильна обрываемости её верхнего
центрального ряда.
\end{exercise}

\begin{theorem}
{$G$
--- нильпотентная. Следующие свойства переносятся с $Z(G)$ на $G$:
\par $\left.0\right)$ нетривиальность;
\par $\left.1\right)$ тривиальность нормальной подгруппы (если $N\cap
Z=\{e\}$, то $N={e}$);
\par $\left.2\right)$ $\pi$-примарность;
\par $\left.3\right)$ конечность (для конечно порождённых);
\par $\left.4\right)$ свобода абелевых подгрупп.}\label{from center to group}
\end{theorem}

\begin{proof} $\left.1\right)$ Доказываем индукцией по верхнему
центральному ряду. Пусть $N\cap z_{k-1}=\{1\}$. Если $x\in N\cap
z_k$, и для любого $g\in G$ $[x,g]=1\pmod{z_{k-2}}$, то $x\in
z_{k-1}$, (рассматриваем $\pi:G\rightarrow G/z_{k-2}$, прообраз
при $\pi$ $Z(G/z_{k-2})$ есть $z_{k-1}$, а в единицу, естественно
переходит $z_{k-2}$). Если же существует такой $g$, что $[g,x]\ne
1\pmod{z_{k-2}}$, что означает, что $[x,g]\in z_{k-1}\cap N$, и
$[x,g]=1$. В обоих случаях имеем $N\cap z_k=1$.

\begin{remark}
Это свойство двойственно свойству 1) теоремы \ref{from quotient to
group}, которое утверждает, что если есть $\langle
X\rangle=H\stackrel{\phi}\rightarrow G\stackrel{\pi}\rightarrow
G/G'$, то из того что $\pi\phi$ эпиморфно следует, что $\phi$
эпиморфно. Новое же свойство говорит, что если есть
$G/N=H\stackrel{\phi}\leftarrow G\stackrel{i}\leftarrow Z$, то из
того что $\phi i$ мономорфизм следует, что $\phi$ мономорфизм.
\end{remark}

$\left.2\right)$ Докажем только для конечно порождённых и только
периодичность для членов верхнего центрального ряда. Пусть
$z_{k-1}$ $\pi$-примарен. Возьмём $x\in z_k$ и покажем, что
найдётся $n$, такой, что $x^n\in z_{k-1}$. Пусть $G=\langle
g_1,\dots,g_t\rangle$. При всех $i$ рассмотрим $[x^n,g_i]$
...\marginpar{\sf\LARGE ?}

$\left.3\right)$ Центр конечен, следовательно и периодичен, откуда
и вся группа периодическая. Поскольку она ещё и конечно
порождённая, то фактор по коммутанту конечно порождён, периодичен
и абелев, а значит конечен. Ввиду нильпотентности $G$ по теореме
\ref{from quotient to group} имеем её конечность.\end{proof}

\begin{exercise} Доказать $\left.4\right)$.
\end{exercise}

\begin{remark}
В 4) просто свободность потребовать нельзя.
\end{remark}

Из теорем \ref{from quotient to group} и \ref{from center to
group} вытекает

\begin{corollary}Если $G$ конечно
порождённая нильпотентная группа, то следующие условия
эквивалентны:
\par $\left.1\right)$ $G$ --- конечна;
\par $\left.2\right)$ $G'$ конечного индекса;
\par $\left.3\right)$ центр конечен.$\square$
\end{corollary}

\begin{theorem}Пусть
группа $G$ конечно порождена. Тогда следующие условия
эквивалентны:
\par $\left.1\right)$ коммутант конечен
\par $\left.2\right)$ центр имеет конечный индекс
\par $\left.3\right)$ количество коммутаторов конечно.
\end{theorem}

\begin{proof} $(1\Rightarrow 3)$ Доказывать нечего.

$(3\Rightarrow 2)$ Пусть $G=\langle g_1,\dots,g_t\rangle$.
$[g_1,x]=[g_1,y]$ \ifif $g_1^{-1}x^{-1}g_1x=g_1^{-1}y^{-1}g_1y$,
т.е. $g_1^x=g_1^y$, или $g_1^{xy^{-1}}=g_1$, что означает, что
$xy^{-1}$ коммутирует c $g_1$. Имеем $xy^{-1}\in C(g_1)$, т.е.
централизатор $g_1$ конечного индекса ($x\in yC\Leftrightarrow
xy^{-1}\in C$). Поскольку $Z(G)=\bigcap\limits_{i} C(g_i)$, а
пересечение конечного числа подгрупп конечного индекса само
конечного индекса, то имеем условие 1).

\begin{exercise} Пусть $G$
--- конечно порождённая группа, и $H$
--- её подгруппа конечного индекса. Тогда $H$ --- конечно
порождена.
\end{exercise}

$(2\Rightarrow 1)$ $Z(G)$ --- конечно порождённая абелева группа,
значит имеем конечную сумму $Z(G)=\Mz^r\oplus\bigoplus\Mz_{k_i}$.
Пусть $G'$ бесконечен. В то же время, т.к.\,$Z$ конечного индекса,
по модулю $Z$ он конечен. Значит $Z\cap G'$ и $Z$ бесконечны,
поэтому одна из координат в сумме принимает бесконечное количество
значений, лежащих в $G'$. Профакторизуем по сумме по другим
координатам. С первой координатой ничего не случится, индекс
центра останется конечным. В новой группе центр содержит $\langle
a\rangle_\infty$ и $\langle a\rangle_\infty$ имеет конечный
индекс. Пусть $g_1,\dots,g_n$ --- элементы фактора новой группы по
$\langle a\rangle_\infty$. На смежных классах $g_i\langle
a\rangle_\infty$ группа $G$ действует перестановками:
$g(g_ia^k)=g_ja_l$ (переставляет и сдвигает).

Рассмотрим $\phi:G\rightarrow\Mz$, который элементу $g$
сопоставляет суммарный сдвиг, отвечающий $g$. Заметим, что элемент
вида $g_ia^k$ сдвигает на $k$: $g_ia^kg_j=g_ig_ja^k=g_la^k$, тем
самым имеет ненулевой суммарный сдвиг. Значит $\ker\phi$ ---
конечная группа. Фактор по ядру абелев, значит ядро содержит
коммутант, это значит, что коммутант конечен.\end{proof}

\subsection{Аппроксимируемость}
Говорят, что $G$ {\em аппроксимируется классом
групп}\index{группа!аппроксимируемая классом групп} $\mathcal{K}$,
если для всякого нетривиального элемента $1\ne g\in G$ существует
$H\in\mathcal{K}$ и такой эпиморфизм $\phi:G\rightarrow H$, что
$\phi(g)\ne 1$. Если $G$ аппроксимируется классом всех конечных
групп, она называется {\em финитно
аппроксимируемой}\index{группа!финитно аппроксимируемая}.

Равносильные определения финитной аппроксимируемости:

$\left.1\right)$ Для любого набора $g_1,\dots,g_k$ нетривиальных
элементов найдётся конечная группа и гомоморфизм $\phi$ группы $G$
в неё, такой, что образы данных элементов
$\phi(g_1),\dots,\phi(g_k)$ нетривиальны.

Действительно, финитная аппроксимируемость из условия, очевидно,
следует, а наоборот, в качестве искомой группы следует взять
произведение групп, удовлетворяющих определению для элементов
$g_1,\dots,g_k$, и в качестве морфизма --- диагональный.

$\left.2\right)$ $G$ вкладывается в произведение
$\prod\limits_{g\in G}H_g$ конечных групп.

В одну сторону: берём конечные группы для каждого элемента из
определения, перемножаем, берём диагональ. Она является
мономорфизмом, т.к.\,иначе какой-то нетривиальный элемент
переходил бы в единицу, что означало бы, что он переходит в
единицу при всех выбранных гомоморфизмах. Наоборот, для заданного
элемента выбираем группу в произведении, и устраиваем композицию
вложения и проекции.

$\left.3\right)$ Для каждого $g\in G$ найдется подгруппа конечного
индекса $H$, такая, что $g\notin H$.

Чтобы из определения получить условие достаточно взять ядро
соответствующего гомоморфизма. Наоборот, если $H$ нормальна, то
нужно взять фактор по ней и соответствующее факторотображение. В
любой подгруппе конечного индекса $k$ можно найти нормальную
подгруппу конечного индекса (не выше $k$) (T.22).

$\left.4\right)$ $\bigcap\limits_{\mbox{\scriptsize конечного
индекса}}H_i=\{e\}$.

Оно, очевидно, эквивалентно предыдущему условию.
%%%%%%%%%%%%%%%%%%%%%%%%%%%%%%%%%%%%%%%%%%%%%%%%%%%%%%%%%%%%%%%%%%%%%%%%%%%%
\subsection{Хопфовость}

\marginpar{\sf\footnotesize 31.10.03}Группа $G$ называется {\em
хопфовой}\index{группа!хопфова}, если она не изоморфна никакой
своей собственной факторгруппе, т.е. если $G/N\cong G$, то
$N=\{e\}$. Двойственное свойство
--- {\em кохопфовость}\index{группа!кохопфова} --- когда группа неизоморфна никакой своей собственной подгруппе.

\begin{theorem}Если $G$ --- конечно порождённая финитно аппроксимируемая
группа, то $G$ --- хопфова.
\end{theorem}

\begin{proof} Зафиксируем произвольную конечную группу $K$. Рассмотрим
семейство отображений $G\stackrel{\phi}\rightarrow K$. Ввиду
конечной порождённости группы $G$ это семейство конечно.
Предположим, что

\begin{figure}[h]%
\hfil\epsffile{groups1.eps}\hfil
\end{figure}

Рассмотрим $\Phi:\Hom(G,K)\rightarrow\Hom(G,K)$, который
$\phi\stackrel{\Phi}\mapsto\phi i\pi$. Далее $i$ будем опускать.
$\Phi$ инъективно: если $\phi_1\circ\pi=\phi_2\circ\pi$, то ввиду
эпиморфности $\pi$ $\phi_1=\phi_2$. $\Phi$ сюръективно: по $\phi$
берём композицию $\phi'\circ\pi$, которая отображается в $\phi$.

Итак, для любого $\phi\in\Hom(G,K)$ существует $\psi\in\Hom(G,K)$,
такой что $\phi=\psi\pi$. Но $\psi\pi$ переводит $N\lhd G$ в
единицу. Значит для любых $\phi$ и конечной $K$ имеем $\phi(N)=e$.
Из финитной аппроксимируемости окончательно получаем, что
$N=\{e\}$.\end{proof}

\begin{theorem}Факторы верхнего центрального ряда нильпотентной
группы без кручения сами без кручения.
\end{theorem}

\begin{proof} Пусть $\{0\}=z_0<z_1<z_2<\dots<z_s=G$. Предположим, что
нашёлся такой $x\in z_i\setminus z_{i-1}$, для которого $x^n\in
z_{i-1}$ (т.е. в $z_i/z_{i-1}$ кручение). Тогда для любого $y$
имеем $[x,y]\in z_{i-1}$, и существует такой $y$, что $[x,y]\notin
z_{i-2}$ (по определению $z_i$). Поскольку $x^n\in z_{i-1}$, то
$z_{i-2}\ni [x^n,y]=[x,y]^n\pmod{z_{i-2}}$. Итак, по индукции
можно прийти к тому, что имеется кручение в центре.
\end{proof}

\begin{theorem}
{$G$
--- конечно порождённая нильпотентная группа. Тогда $G$ финитно аппроксимируема,
а если ещё $G$ без кручения, а $p\in\Mp$, то $G$ аппроксимируется
конечными $p$-группами.}
\end{theorem}

\begin{proof} Любая конечно порождённая нильпотентная группа почти без
кручения.

\begin{fact} Если $G$ почти финитно
аппроксимируема, то $G$ финитно аппроксимируема.
\end{fact}

Отсюда с учётом второго утверждения теоремы получается первое.

Фиксируем произвольный элемент $g\in G$. Найдём для него конечную
$p$-группу $K$ и морфизм $\phi$, такие, что $\phi(g)\ne 1$.

Если $g\notin Z(G)$, то профакторизовав по $Z$, получим группу без
кручения меньшей ступени нильпотентности, применим индукцию.

Если $g\in Z(G)$, то выберем максимальную нормальную подгруппу
$N\lhd G$, не содержащую $g$, но содержащую $g^p$(такая есть ввиду
леммы Цорна и существенно, что $g\in Z(G)$). Рассмотрим $G/N$.
Понятно, что $g\in Z(G/N)$. $Z(G/N)=\langle h\rangle_{p^n}$, т.к.
$Z(G/N)$
--- конечно порождённая абелева и $N$ --- максимальна. Т.к.\,группа
$G/N$ конечно порождена и нильпотентна, то она конечна. $Z(G/N)$
$p$-примарен, значит $G/N$ тоже. Поэтому она и является искомой.
\end{proof}

\begin{remark}
Условие, что $G$ без кручения существенно
--- если есть элемент конечного порядка $p\in\Mp$, то группа не
может аппроксимироваться $q$-группами при $q\ne p$.
\end{remark}

\begin{corollary}Конечно порождённая нильпотентная группа хопфова.
$\square$
\end{corollary}

\begin{exercise} Доказать непосредственно, что конечно
порождённая нильпотентная группа хопфова.
\end{exercise}

Не любая нильпотентная группа пополняется.

\begin{theorem} Если $G$ --- полная нильпотентная группа и $g$ ---
элемент конечного порядка, то $g\in Z(G)$.
\end{theorem}

\begin{proof} Пусть $g$ не лежит в $Z(G)$. Покажем, что если $g\in z_2$,
то $g\in z_1$. Пусть $g^n=1$ и $g\notin Z(G)$. Профакторизуем по
центру. $[g,x]=[g,y^n]=[g,y]^n=[g^n,y]=1$,\marginpar{обосновать
равенства} значит $g\in z_1$, противоречие.\end{proof}

\begin{example}
$\mathbb{H}$ не вкладывается в полную нильпотентную группу.
\end{example}

\begin{theorem}[Мальцев]\index{теорема!Мальцева} Если $G$ нильпотентная группа
без кручения, то существует полная группа $\widetilde{G}>G$ той же
ступени нильпотентности.
\end{theorem}

\begin{proof} Для каждого $p\in\Mp$ $G$ аппроксимируется конечными
$p$-группами. Т.е. имеем вложение
$G\hookrightarrow\Times\limits_{i\in I,~g\in G}K_{i,p}=D_p$.
Рассмотрим
$G\hookrightarrow\Times\limits_{p\in\Mp}D_p\rhd\prod\limits_{p\in\Mp}D_p$.
Теперь профакторизуем: $G\hookrightarrow\Times D_p/\prod D_p$.
Последняя факторгруппа полна, т.к.\,по всем координатам кроме
$p$-ой извлекается корень $p$-ой степени.
\end{proof}
\marginpar{\footnotesize{Остаётся показать, что ступень
нильпотентности та же.}}

Говорят, что группа $G$ {\em локально обладает
свойством}\index{группа!локально обладающая свойством}
$\mathcal{P}$, если любая её конечно порождённая подгруппа,
обладает свойством $\mathcal{P}$.

\begin{fact} Если $\mathcal{L}$ --- класс групп, замкнутый
относительно декартовых произведений и факторгрупп и $G$ локально
вкладывается в группу из $\mathcal{L}$, то $G$ вкладывается в
группу из $\mathcal{L}$.
\end{fact}

Пусть $\{H\}$ --- множество всех конечно порождённых подгрупп в
$G$. Пусть $H<\widetilde{H}\in\mathcal{L}$. Рассмотрим
$\prod\limits_{H}\widetilde{H}/\sim$, где $\sim$ такой: $X\sim Y$
\ifif существует такая конечно порождённая $H<G$, что для любой
конечно порождённой $M$, содержащей $H$ выполнено $X_M=Y_M$. Легко
проверить, что $\sim$ --- отношение эквивалентности.

\begin{exercise} Проверить согласованность с умножением: $(X\sim
Y)\rightarrow(ZX\sim ZY)$.
\end{exercise}

%%%%%%%%%%%%%%%%%%%%%%%%%%%%%%%%%%%%%%%%%%%%%%%%%%%%%%%%%%%
Вложим в рассматриваемое произведение $G$. А именно на координате
$H$
$$
\phi: g\mapsto\left\{
\begin{array}{rcl}
g, \\ 1,
\end{array}
\begin{array}{rcl}
g\in H \\ \mbox{иначе}
\end{array}
\right.
$$
Очевидно, что для всех $g,h\in H$ выполнено
$(\phi(g),\phi(h))_H=(\phi(gh))_H$, а также, что если
$(\phi(g))_H=(\phi(h))_H$, то $g=h$ (начиная с некоторого места
координаты совпадают).
%%%%%%%%%%%%%%%%%%%%%%%%%%%%%%%%%%%%%%%%%%%%%%%%%%%%%%%%%%%%%%%%%%%%%%%%%%%%%%
\section{Разрешимость}
\marginpar{\sf\footnotesize 14.11.03}{\em Разрешимая
группа\index{группа!разрешимая} ступени
$m$}\index{ступень!разрешимости}--- такая группа $G$, что
$G^{(m)}=\{1\}$ и $G^{(m-1)}\neq\{1\}$.

Пусть задан гомоморфизм $\phi:B\rightarrow\mathrm{Aut}A$. Для
каждого $\sigma\in\mathrm{Aut}A$ и $a\in A$ положим
$a^\sigma=\sigma^{-1}(a)$.

\begin{remark} Это сделано так, чтобы
$a^{\sigma_1\sigma_2}=(a^{\sigma_1})^{\sigma_2}$. Действительно,
$a^{\sigma_1\sigma_2}=(\sigma_1\sigma_2)^{-1}(a)=
\sigma_2^{-1}(\sigma_1^{-1}(a))=(a^{\sigma_1})^{\sigma_2}.$
\end{remark}

{\em Полупрямое произведение групп\index{произведение!групп
полупрямое}} $A$ и $B$:
$$
B\leftthreetimes_\phi A=\{ba:~b\in B,~a\in A\},
$$
где умножение задано по формуле
$(b_1,a_1)(b_2,a_2)=(b_1b_2,a_1^{\phi(b_2)}a_2)$. Группы $A$ и $B$
называют соответственно {\em пассивной}\index{подгруппа!пассивная}
и {\em активной}\index{подгруппа!активная}.

\begin{exercise} $G\cong B\leftthreetimes A$ \ifif $G=B_0A_0$, $B_0<G$,
$B\cong B_0$, $A_0\lhd G$, $A\cong A_0$ и $A_0\cap B_0=\{1\}$.
\end{exercise}

\begin{example} $D_n=\langle b\rangle_2\leftthreetimes_\phi\langle
a\rangle_n$; $\phi(b)(a^k)=a^{-k}$.

$D_\infty=\langle b\rangle_2\leftthreetimes_\phi\langle
a\rangle_\infty$ --- разрешимая группа, в которой элементы
конечного порядка не образуют группу. Действительно,
$(ba)(ba)=b^2a^{-1}a=1$ и $b\cdot ba=a$.
\end{example}

\begin{fact} $B\leftthreetimes A/A\cong B$.
\end{fact}
Пусть $B$ действует на $\prod\limits_{b\in B}A$ автоморфизмами
(переставляя индексы). {\em Прямым
{\em(\/}декартовым{\em)\/}\index{сплетение!прямое}\index{сплетение!декартово}
сплетением} $B$ и $A$ называется $B\imath
A=B\leftthreetimes(\coprod\limits_{b\in B}A_b)$ (соответственно,
$B\bar{\imath}A=B\leftthreetimes(\prod\limits_{b\in B}A_b)$).
$\coprod\limits_{b\in B}A_b$ и $\prod\limits_{b\in B}A_b$
называются {\em базой}\index{база!сплетения} сплетения.

\begin{example} Конечно
порождённая группа (разрешимая ступени 2), с базой сплетения, не
являющейся конечно порождённой: $\langle
b\rangle_\infty\imath\langle a\rangle_2$.

Порождающий --- $(b,a_1)$; база сплетения
--- $\Mz_2^\infty$.$\square$
\end{example}

\begin{fact} Если факторгруппа по разрешимой группе разрешима, то и
исходная группа разрешима.
\end{fact}

Отсюда и из (Ф.15) вытекает

\begin{fact} Сплетение (оба) разрешимых групп разрешимо.
\end{fact}

\begin{example}\index{пример!конечно порождённой не финитно аппроксимируемой группы}
Существует конечно порождённая (разрешимая ступени 3) группа, не
являющаяся финитно аппроксимируемой.

Рассмотрим $\langle b\rangle_\infty\imath S_3=\langle
b,(12)_1,(123)_1\rangle$. Пусть имеется гомоморфизм $\psi:\langle
b\rangle_\infty\imath S_3\rightarrow K$, где $K$ --- конечная и
$\psi((123)_1)\ne 1$. $S_3\hookrightarrow K$, т.к.\,$A_3<S_3$ не
отображается в единицу. $(S_3)_{b^l}\hookrightarrow K$, поскольку
она сопряжена с $(S_3)_1$, а значит и их образы при $\psi$
сопряжены. Таким образом в $K$ имеется бесконечное количество
групп, изоморфных $S_3$, обозначим $H_i=\psi((S_3)_{b^i})$.
Поэтому найдутся $i$ и $j$, для которых $H_i=H_j$. Поскольку
$(S_3)_{b^i}$ и $(S_3)_{b^j}$ коммутируют, то $H_i$ --- абелева.
Но $H_i\cong S_3$. Противоречие показывает, что $\psi:A_3\mapsto
1$, т.е. финитная аппроксимируемость места не имеет.$\square$
\end{example}

Рассмотрим над $G$ уравнение
$$
(*)~~~~~~~x^{n_1}g_1x^{n_2}g_2\dots x^{n_k}g_k=1.
$$
Говорят, что уравнение $(*)$ {\em разрешимо над}\index{уравнение
разрешимое над группой} $G$, если $G$ вкладывается в $H$, а в $H$
оно разрешимо.

\begin{fact} Если
уравнение $xg_1x^{-1}=g_2$ разрешимо, то $g_1$ и $g_2$ имеют
одинаковые порядки.
\end{fact}

\begin{hypothesis} [Левин]\index{гипотеза!Левина} Если группа $G$ без кручения, то любое
нетривиальное уравнение разрешимо над $G$.
\end{hypothesis}

\begin{hypothesis}[Кервери, Лауденбах]\index{гипотеза!Кервери---Лауденбаха} Если уравнение невырожденно, т.е. сумма
показателей при $x$ не равна нулю, то уравнение разрешимо над $G$.
\end{hypothesis}

\begin{theorem}[Левин]\index{теорема!Левина} Если показатели положительны,
то уравнение разрешимо над $G$.
\end{theorem}

\begin{proof} Будем считать, что все показатели равны единице, т.к.\,в
$(*)$ между $g_i$ если надо можно вписать единицы. Рассмотрим
$\langle b\rangle_k\imath G=\{b^l(h_1,\dots,h_k)\}$ ($b$ действует
на строчку циклическим сдвигом). $G$ вкладывается в сплетение по
формуле $g\mapsto(g,\dots,g)$.

Будем искать решение в виде $b(x_1,\dots,x_k)=\widetilde{x}$.
Обозначим $f(x)=xg_1\dots xg_k$.
$$
f(\widetilde{x})=b(x_1,\dots,x_k)g_1\dots b(x_1,\dots,x_k)g_k
$$
$$
f(\widetilde{x})=(x_kg_1x_{k-1}g_2\dots x_1g_k,x_1g_1x_kg_2\dots
x_2g_k,\dots,x_{k-1}g_1x_{k-2}g_2\dots x_kg_k)
$$
Положим $x_k=g_1^{-1},~x_{k-1}=g_k^{-1},\dots,~x_1=g_2^{-1}$.
Легко видеть, что в этом случае $f(\widetilde{x})=1$.\end{proof}
\begin{theorem}[Фробениус]\index{теорема!Фробениуса} Пусть $A\lhd G$ и
$G/A=B$. Тогда $G$ вкладывается в $B\bar{\imath}A$.
\end{theorem}

\begin{proof} Будем решать такую задачу: пусть $A\lhd G$ и $A$ действует
на $M$ справа, нас будет интересовать вопрос о продолжении
действия на $G$.

В каждом смежном классе $Ag$ зафиксируем элемент $\bar{g}$.
$G=\coprod\limits_{\bar{g}}A\bar{g}$. Расширим также $M$:
$m\cdot\bar{g}=m_{\bar{g}}$,
$M'=\coprod\limits_{\bar{g}}M\bar{g}$. Рассмотрим элемент
$g=a\bar{g}$. Тогда
$mg=mg\bar{g}^{-1}\bar{g}=(ma)\bar{g}=(ma)_{\bar{g}}$, а также
$m_{\bar{h}}g=m\bar{h}g=m(\bar{h}g)=
(m\bar{h}g(\overline{\bar{h}g})^{-1})_{\overline{\bar{h}g}}$.
Проверим, что получилось действие:
$$
m_{\bar{h}}(xy)=(m\bar{h}xy(\overline{hxy})^{-1})_{\overline{hxy}}=
(m\bar{h}x(\overline{hx})^{-1}\overline{hx}y(\overline{\overline{hx}y})^{-1})_{\overline{hxy}}
=
((m\bar{h}x(\overline{hx})^{-1})_{\overline{hx}})y=(m_{\bar{h}}x)y
$$
Вложение $G$ в сплетение ---

{\em вложение Фробениуса:\index{вложение Фробениуса} $g\mapsto
b(g)\prod\limits_{\overline{h}}
(\overline{h}g(\overline{\overline{h}g})^{-1})_{b({\overline{h}g})}$.}
\end{proof}

\begin{remark}
Если $M$ --- векторное пространство, а $A$
--- линейное действие, то по формуле получится линейное действие,
сохраняющее точность.
\end{remark}

%%%%%%%%%%%%%%%%%%%%%%%%%%%%%%%%%%%%%%%%%%%%%%%%%%%%%%%%%%%%%%%
Группа\marginpar{\sf\footnotesize 21.11.03} называется {\em
\index{группа!линейная} линейной}, если она вкладывается в группу
невырожденных матриц над некоторым полем.

\begin{theorem}Почти линейная группа линейна.
\end{theorem}

\begin{proof}\end{proof} \marginpar{\sf\LARGE ?}

Существует разрешимая группа ступени $s$, содержащая все
разрешимые группы ступени $s$ и имеющие порядок $\mathfrak{m}$.
Если $G$ --- разрешимая группа ступени $s$, то $A=G'$
--- разрешимая группа ступени $s-1$, а $G/A=B$ --- абелева.
Получаем вложение $G\hookrightarrow B\bar{\imath}A$. В частности,
разрешимая данной ступени $s$ группа мощности $\mathfrak{c}$,
содержащая все счётные разрешимые группы ступени $s$, а именно в
качестве $B$ нужно взять универсальную абелеву группу, а в
качестве $A$ универсальную группу на 1 меньшей ступени
разрешимости.

Если имеется субнормальный ряд $G\rhd G_1\dots\rhd G_s=\{1\}$, и
факторы $G_i/G_{i+1}$ абелевы конечно порождённые,\footnote{можно
уплотнить ряд, чтобы факторы стали циклическими} то G называется
{\em полициклической}\index{группа!полициклическая}.

\begin{theorem}Подгруппа $H$ конечно порождённой полициклической группы
$G$ конечно порождена.
\end{theorem}

\begin{proof} Рассмотрим соответствующий $G$ ряд с циклическими
факторами. Будем доказывать индукцией по длине ряда. $H\cap G_1<G$
--- конечно порождена, $G/G_1$ --- циклическая. $H/H\cap
G_1<G/G_1$, откуда и следует утверждение.\end{proof}

\begin{fact}
Конечно порождённая нильпотентная группа является полициклической.
\end{fact}

Подгруппа называется {\em вполне
инвариантной}\index{подгруппа!вполне инвариантная}, если она
переходит в себя при действии всех эндоморфизмов группы.

\begin{theorem}Если $H$ --- подгруппа конечно порождённой
группы $G$ конечного индекса, то существует $N\vartriangleleft G$
конечного индекса, лежащая в $H$; более того, существует такая
вполне инвариантная $N$.
\end{theorem}

\begin{proof} Нам далее утверждение будет нужно только для случая, когда
$H$ нормальна, его и будем доказывать.

Рассмотрим семейство $\{G\rightarrow G/H\}=\Phi$. Оно конечно, и
ядра гомоморфизмов его элементов имеют в $G$ конечный индекс.
Возьмём $N=\bigcap\limits_{\phi\in\Phi}\ker\phi$. $N$ имеет в $G$
конечный индекс, покажем, что она вполне инвариантна. Рассмотрим
какой-нибудь $\psi:G\rightarrow G$ и предположим, что
$x\in\psi(N)$. Тогда $x\in\ker\phi\psi$, значит $\psi
x\in\ker\phi$, откуда
$\psi(N)\subset\bigcap\limits_{\phi\in\Phi}\ker\phi$, что и
требовалось.\end{proof}

\begin{theorem}Пусть $G=B\leftthreetimes A$,
где $A$ и $B$ финитно аппроксимируемы и $A$ конечно порождена.
Тогда $G$ финитно аппроксимируема.\label{FA of semiprod}
\end{theorem}

\begin{proof} Если $x\notin A$, то ввиду финитной аппроксимируемости $B$
имеем нужный гомоморфизм $G\stackrel{\pi}\rightarrow
G/A=B\stackrel{\phi}\rightarrow K$.

Если $x\in A$, ввиду её финитной аппроксимируемости можем выбрать
$N\vartriangleleft A$ конечного индекса, такую, что $x\notin N$.
Внутри $N$ выбираем вполне инвариантную подгруппу $N'$ конечного
индекса. $N'$ нормальна в $G$, поскольку при сопряжении элементами
из $B$ $A$ остаётся на месте и внутри неё имеется некий
эндоморфизм, а относительно него $N'$ инвариантна. Рассмотрим
$G_1=G/N'=B\leftthreetimes(A/N')$. $A/N'$ --- конечна и,
соответственно $\Aut(A/N')$ --- конечна. Имеется
$\phi:~B\rightarrow\Aut(A/N')$. $\ker\phi$ нормальная подгруппа
конечного индекса в $B$, кроме того, $\ker\phi$ является таковым в
$G$: конечность индекса тривиальна, а нормальность вытекает из
равенства
$(e,a)(b,e)(e,a^{-1})=(e,a)(b,a^{-1})=(b,a^{\phi(b)}a^{-1})=(b,e)$
при $b\in\ker\phi$.\end{proof}

\begin{theorem}Полициклические группы финитно аппроксимируемы.
\end{theorem}

\begin{proof} Будем доказывать индукцией по длине соответствующего ряда,
поэтому считаем, что $G_1$ конечно порождена и финитно
аппроксимируема. $G/G_1$ --- циклическая, если она конечна, то
группа почти финитно аппроксимируема. Пусть она бесконечна:
$G/G_1=\langle gG_1\rangle_\infty$. $\langle g\rangle\cap
G_1=\{1\}$, $G=\langle g\rangle G_1$ и $G_1\lhd G$, поэтому
$G=\langle g\rangle_\infty\leftthreetimes G_1$, откуда с помощью
\ref{FA of semiprod} следует утверждение.\end{proof}

\section{Связь с модулями}

Разрешимая группа ступени 2 (коммутант абелев) называется {\em
метаабелевой}\index{группа!метаабелева}.

Метаабелевы группы близки к коммутативным кольцам. Группа $G$
действует на своём коммутанте сопряжением: если $y\in G'$, а $x\in
G$, то $y\cdot x=y^x$. На самом деле, можно считать, что действует
фактор по коммутанту. Рассмотрим $(G',+,\cdot)$. Для всех $a,b\in
G',~\lambda,\mu\in G/G'$ выполнены
\begin{center}
$(a+b)\lambda=a\lambda+b\lambda$\\
$a(\lambda\mu)=(a\lambda)\mu$\\
$a1=a$\\
$a(\lambda+\mu)=a\lambda+a\mu$
\end{center}
Тем самым, $G'$ --- модуль над групповым кольцом $\Mz[G/G']=\{\sum
r_ig_i:~g_i\in G/G',~r_i\in\Mz\}$.

Ассоциативное кольцо с единицей называется {\em
нётеровым}\index{кольцо!нётерово}, если всякий идеал конечно
порождается как идеал. Аналогично, модуль называется {\em
нётеровым}\index{модуль!нётеров}, если любой подмодуль конечно
порождается как модуль.

%%%%%%%%%%%%%%%%%%%%%%%%%%%%%%%%%%%%%%%%%%%%%%%%%%%%%%%%%%%%%%%%%%%%%%
\begin{theorem}[теорема Гильберта о базисе]\index{теорема!Гильберта о базисе} Кольцо многочленов над
нётеровым кольцом нётерово.
\end{theorem}

Без доказательства.

\begin{fact} Факторкольцо нётерова кольца
нётерово.
\end{fact}

\begin{fact} Конечно порождённое ассоциативное
коммутативное кольцо нётерово.
\end{fact}

\begin{fact} Конечно
порождённый модуль над нётеровым кольцом нётеров.
\end{fact}

%%%%%%%%%%%%%%%%%%%%%%%%%%%%%%%%%%%%%%%%%%%%%%%%%%%%%%%%%%%%%%%%%

\begin{theorem}Всякий конечно порождённый модуль над нётеровым кольцом
нётеров. \marginpar{\sf\footnotesize 28.11.03}
\end{theorem}

\begin{proof} Пусть $M$ --- данный модуль над данным кольцом $R$. По
условию $M=\sum\limits_{i=1}^{n}Rx_i$. Будем доказывать индукцией
по $n$. База тривиальна. Пусть $N$ подмодуль в $M$. Рассмотрим
$N\cap\langle x_1,\dots,x_n\rangle=\langle y_1,\dots,y_k\rangle$.
Рассмотрим коэффициенты при $x_n$ элементов $N$. Они образуют
подкольцо $S<R$, которой по условию нётерово, т.е.
$S=a_1R+\dots+a_sR$. Осталось заметить, что элементы
$a_1x_n,\dots,a_sx_n,y_1,\dots,y_k$ порождают $n$.
\end{proof}

\begin{corollary}Конечно порождённое кольцо нётерово.$\square$
\end{corollary}

\begin{corollary}Пусть $G$
--- конечно порождённая метаабелева группа. Тогда всякая её
нормальная подгруппа $N$ есть нормальное замыкание конечного числа
элементов.
\end{corollary}

\begin{proof} $G'$ --- модуль над $\Mz[G/G']$, $N\cap G'$ --- подмодуль в
нём. Тогда $N\cap G'=\ncl{y_1,\dots,y_k}$ (T.27). $G/G'$ ---
конечно порождённая абелева, значит $N/(G'\cap N)<G/G'$
--- конечно порождена. Т.е. имеем $N/(G'\cap N)=\langle z_1(G'\cap
N),\dots,z_k(G'\cap N)\rangle$ и $N=\langle
z_1,\dots,z_k,y_1,\dots,y_s\rangle$.\end{proof}

\begin{theorem}Пусть
$R$
--- конечно порождённое ассоциативное коммутативное кольцо с единицей.
Если $R$ --- поле, то оно конечно.
\end{theorem}

\begin{proof} Пусть $R$ конечно порождается как кольцо, т.е. $R=\langle
x_1,\dots,x_n\rangle$. Рассмотрим в $R$ минимальное подполе $S$
(это либо $\Mz_p$, либо $\Mq$). Пусть $\{y_1,\dots,y_k\}=\{
x_{i_1},\dots,x_{i_k}\}$
--- максимальное алгебраически независимое подмножество над $S$. Рассмотрим
$P$, порождённое $y_1,\dots,y_k$ как поле. $P=\langle
y_1,\dots,y_k\rangle=S(y_1,\dots,y_k)$ --- поле рациональных
функций. Тогда все $x_i$ алгебраичны над $P$ (иначе имелось бы
противоречие с максимальностью множества $\{y_1,\dots,y_k\}$).
Значит $R$ --- алгебраически замкнутое конечномерное поле над $P$.

Если $k\ne 0$, выберем базис $\langle b_1,\dots,b_s\rangle=R$ над
$P$. Рассмотрим $c_{ijk}$ и $d_{ij}$ удовлетворяющие равенствам

\hfil$b_ib_j=\sum\limits_{k}c_{ijk}b_k$,\hfil

\hfil$x_i= \sum\limits_{j}d_{ij}b_j$.\hfil
\\ Рассмотрим множество состоящее из знаменателей $c_{ijk}$ и $d_{ij}$. Множество
неприводимых многочленов ....\end{proof}\marginpar{\sf\LARGE ?}

Пусть $R$ --- кольцо. Его идеал $I$ называется {\em
минимальным}\index{идеал!минимальный}, если любой нетривиальный
идеал, лежащий в $I$ с ним совпадает. Легко видеть, что любой
минимальный идеал является главным.

\begin{theorem}Пусть $R$
--- ассоциативное коммутативное кольцо с единицей и $I=xR,~x\ne 0$ --- его минимальный идеал.
Тогда
\par $\left.1\right)$ $\Ann x=\{y\in R:~yx=0\}$ --- минимальный идеал и
\par $\left.2\right)$ если $xR$ --- наименьший идеал и $R$
--- нётерово, то $\Ann x$ --- нильпотентный, т.е. существует
такой $k$, что $(\Ann x)^k=0$.
\end{theorem}

\begin{proof}$\left.1\right)$ Пусть $\Ann x\subsetneq J$. $Jx\ne \{0\}$,
поскольку в нём есть элементы, не обнуляющие $x$. $Jx\subset Rx$ и
ввиду минимальности $Rx$ имеем $Jx=Rx$, поэтому для всех $r\in R$
есть такой $j\in J$, что $rx=jx$, а это значит, что $r-j\in\Ann
x$. Но $j\in J$ и $\Ann x\subset J$, поэтому $r\in J$, т.е. мы
получили, что $J=R$, что и требовалось.

$\left.2\right)$ %\marginpar{\sf\LARGE ?}
\end{proof}

\begin{theorem}Пусть $R$
--- конечно порождённое ассоциативное коммутативное кольцо с единицей. Тогда $R$ финитно
аппроксимируемо.
\end{theorem}

\begin{theorem}[Мальцев]\index{теорема!Мальцева} Если $G\subset GL_n(R)$
--- конечно порождённой коммутативное ассоциативное кольцо с
единицей, то $G$ финитно аппроксимируемо.
\end{theorem}

\begin{theorem}Пусть $M$ --- конечно порождённый модуль над конечно
порождённым ассоциативным коммутативным кольцом с единицей $R$.
Тогда $M$ финитно аппроксимируемо.
\end{theorem}

%\begin{proof}\end{proof}

%%%%%%%%%%%%%%%%%%%%%%%%%%%%%%%%%%%%%%%%%%%%%%%%%%%%%%%%%%%%%%%%

\begin{theorem}Конечно\marginpar{\sf\footnotesize 05.12.03} порождённая
метаабелева группа финитно аппроксимируема.
\end{theorem}

\begin{theorem}[Холл]\index{теорема!Холла} Пусть $G$ --- конечно порождённая
группа, $A$
--- её абелева нормальная подгруппа, и фактор $G/A$
--- полициклический. Тогда $G$ --- финитно аппроксимируемая.
\end{theorem}

%\begin{proof}\end{proof}

\noindent{\bf Проблема Бернсайда}.\index{проблема Бернсайда} {\sl
Всякая ли периодическая группа локально конечна}.
\footnote{отрицательный ответ был получен Е.С.~Голодом; известно,
что утверждение $(\exists n\in\Mn ~\forall g\in
G~g^n=1)\to(G\mbox{
--- локально конечна})$ истинно при $n=1,2,3,4,6$, ложно при нечётных $n$, больших $665$,
и достаточно больших чётных}
\begin{theorem}[Шмидт]\index{теорема!Шмидта} Расширение локально конечной группы при
помощи локально конечной тоже локально конечно.
\end{theorem}

\begin{proof} Можно считать, что $G$ --- конечно порождена и доказывать
её конечность. Тогда $G/N$
--- конечно порождённая локально конечная группа, т.е. конечная,
поэтому $N$
--- подгруппа конечного индекса. Следовательно $N$ --- конечно
порождена (У.12), а с учётом локальной конечности и конечна. Но
тогда и $G$ конечна.\end{proof}

\begin{theorem}Разрешимая периодическая группа локально конечна.
\end{theorem}

\begin{proof} Индукция по ступени разрешимости. Тогда можно считать, что
$G'$
--- локально конечна. Но тогда $G/G'$ --- абелева и периодическая,
а таковая всегда локально конечна. Осталось применить теорему
Шмидта.\end{proof}

\begin{theorem}Подгруппа $H$ конечного индекса в
конечно порождённой группе $G$ конечно порождена.
\end{theorem}

\begin{proof} Представим $\langle
y_1,\dots,y_m\rangle=G=x_1H\sqcup\dots\sqcup x_nH$, где $x_1=1$ и
$y_i^{-1}\in\{y_1,\dots,y_m\}$. Для всех $i$ можем написать
$y_i=x_{j(i)}h_i$ и $y_ix_k=x_{k,i}h_{k,i}$. Покажем, что $h_i$ и
$h_{j,k}$ порождают $H$:

\hfil$H\ni h=y_{i_1}\dots y_{i_s}=y_{i_1}\dots
y_{i_{s-1}}x_{j(i)}h_{i_s}=\dots=x_t\prod h_jh_{i_s}$.\hfil

Но поскольку правая часть равенства лежит в $H$, то и $x_t\in H$,
т.е. $x_t=1$. Таким образом, $H$ --- конечно порождена.\end{proof}

\section{Тождества}

Пусть
$$
(2)~~~~~~~x_{i_1}^{\varepsilon_{1}}\dots
x_{i_n}^{\varepsilon_{n}}=1,~~\varepsilon_i=\pm 1.
$$
{\em Тождество\index{тождество} $(2)$ выполнено} в $G$, если для
любых $g_1,\dots,g_n\in G$ $g_{i_1}^{\varepsilon_{1}}\dots
g_{i_n}^{\varepsilon_{n}}=1$ в $G$.

\begin{example}
$x^{-1}y^{-1}xy=1$ выполнено в $G$ \ifif $G$ --- абелева.
$[x_1,\dots,x_s]=1$ выполнено в $G$ \ifif $G$ нильпотентна ступени
не больше $s$.
\end{example}

\begin{exercise} Пусть $D_0(x_1)=x_1$, а
$D_k(x_1,\dots,x_{2^k})=[D_{k-1}(x_1,\dots,x_{2^{k-1}}),D_{k-1}(x_{2^{k-1}+1},
\dots,x_{2^k})]$.
Показать, что $D_k(x_1,\dots,x_k)=1$ \ifif $G$ --- разрешима
ступени $\leqslant k$.
\end{exercise}

Если задан набор тождеств
$$
(3)~~~~~~~\{v_i(x_1,\dots)=1:~i\in I\},
$$
то класс всех групп, в котором выполнены эти тождества называется
{\em многообразием}\index{многообразие групп} групп.

\begin{exercise} Из $x^2=1$ следует,
что для
любого $y$ $xy=yx$.
\end{exercise}

Если $G_i\in\mathcal{M}$, то $\prod G_i,\bigoplus
G_i\in\mathcal{M}$, $~\mathcal{M}$ замкнуто относительно подгрупп
и факторгрупп. И наоборот, справедлива

\begin{theorem}[Биркгоф]\index{теорема!Биркгофа} $\mathcal{K}$ является многообразием
\ifif $\mathcal{K}$ замкнут относительно декартовых произведений,
подгрупп и факторгрупп.
\end{theorem}

\begin{proof} Любая система тождеств эквивалентна некоторой системе
тождеств в счётном алфавите. Действительно, любое тождество вида
$v(x_{\xi_1},\dots,x_{\xi_s})$ эквивалентно тождеству
$v(x_1,\dots,x_s)$, т.е. для записи всех тождеств достаточно иметь
буквы $x_1,x_2,\dots$. Но тогда можно считать, что всего тождеств
счётное число, как конечных подмножеств счётного множества.

Выпишем все тождества, не выполненные в классе $\mathcal{K}$. Т.е.
такие $v_i$, что существуют $G_i\in\mathcal{K}$ и $a_{ij}\in G_i$,
что $v_i(a_{i1}\dots a_{ik_i})\ne 1$ в $G_i$.

Рассмотрим $\prod\limits_{i\in\Mn}G_i\supset
F=\langle(a_{11},a_{21},\dots)=b_1,(a_{12},a_{22},\dots)=b_2,\dots\rangle$.
Заметим, что всякое соотношение между $b_i$
($u(b_{i_1},b_{i_2},\dots)$) выполнено \ifif $u$ --- тождество в
$\mathcal{K}$. Действительно, если это не тождество, то это есть
некоторый $v_i$, тогда на $i$-ой координате имеем с одной стороны
$v_i(a_{i1},a_{i2},\dots)$ (т.к.\,для $b_i$ верно), а с другой оно
по выбору не тождество, значит не верно. Пусть теперь, есть группа
$G$, такая, что в ней выполнены все равенства, отличные от
$v_i,~i\in\Mn$. Тогда она является фактором группы $G$, а значит
лежит в классе $\mathcal{K}$, что и требовалось.
\end{proof}

На самом деле, доказано большее

\begin{corollary}Пусть $\mathcal{K}$ --- класс
групп. Породим им многообразие (обозначается $\var\mathcal{K}$).
Тогда $\var\mathcal{K}$ состоит из факторгрупп подгрупп декартовых
произведений групп из $\mathcal{K}$.$\square$
\end{corollary}

Группа $F\in\mathcal{K}$ называется {\em
свободной}\index{группа!свободная} в классе $\mathcal{K}$, если
можно найти такие $b_1,\dots$ ({\em свободный
базис\index{свободный базис}}; его мощность называется {\em
рангом\index{ранг}}), что $F=\langle b_1,\dots\rangle$ и $u$
--- тождество в $\mathcal{K}$, \ifif $u(b_{i_1},\dots)=1$.

\begin{fact} Пусть
$\mathcal{M}$
--- многообразие, тогда в $\mathcal{M}$ существуют свободные
группы любого ранга, и любая группа $G\in\mathcal{M}$ факторгруппа
свободной. Свободные группы одинакового ранга изоморфны.
\end{fact}

В {\em абсолютно свободных группах\index{группа!абсолютно
свободная}} есть единственное тождество: $1=1$. Ещё пример ---
если тождества $[x,y]=1$ и $x^n=1$, то соответствующий класс ---
$\var\{\Mz_n\}$. $\Mq$ свободна с базисом $\{1\}$ в классе
$\{\Mq\}$.

Будем обозначать $F(x_1,\dots)$ всевозможные слова, составленные
из букв $x_1,\dots$

\begin{exercise} Доказать,
что если $u,v\in F(x_1,x_2)$, то $uv=vu$ \ifif существует $w$,
такой, что $u,v\in \langle w\rangle$.
\end{exercise}

Группа $F$ называется {\em категорно
свободной}\index{группа!категорно свободная} в классе
$\mathcal{M}$ с базисом $X$, если для любого отображения множества
$X$ в группу $G\in\mathcal{M}$ существует единственный
гомоморфизм, такой, что следующая диаграмма коммутативна:
\begin{figure}[h]%
\hfil\epsffile{groups2.eps}\hfil
\end{figure}

\begin{fact}$F$ --- свободная группа в классе $\mathcal{K}$ с базисом
$X\subset F$ \ifif $F$ категорно свободна и $F=\langle
X\rangle$.\footnote{последние условие для единственности}
\end{fact}

\begin{exercise}$F$ свободна в $\mathcal{K}$ \ifif $F$ свободна в
$\var\Ck$.
\end{exercise}

Если $\Cm$ --- многообразие, то $\Cm$ задаётся так: $\Cm=\var
F_{\Cm}(x_i)_{i\in\Mn}$ ($\Cm$ порождается свободными в $\Cm$
группами счётного ранга).

{\em Коразмерностью}\index{коразмерность} называется минимальное
количество тождеств для задания многообразия.

\begin{theorem}[Красильников, Шмелькин]\index{теорема!Красильникова---Шмелькина} $G'$ нильпотентна \ifif
$\var G$ конечно базируемо.
\end{theorem}

\begin{theorem}[Оутс, Пауэл]\index{теорема!Оутс---Пауэла} Если $G$ конечна, то $\var G$ ---
конечно базируемо.
\end{theorem}

Без доказательства.

\begin{remark} Последняя теорема справедлива также для ассоциативных колец, а
в полугруппах это не так.
\end{remark}

\begin{theorem}Если $G$
--- конечная группа, а $H$ --- конечно порождённая группа,
удовлетворяющая всем тождествам $G$, то $H$ --- конечна.
\end{theorem}

%%%%%%%%%%%%%%%%%%%%%%%%%%%%%%%%%%%%%%%%%%%%%%%%%%%%%%%%%%%%%%%%
%%%%%%%%%%%%%%%%%%%%%%%%%%%%%%%%%%%%%%%%%%%%%%%%%%%%%%%%%%%%%%%%

\section{Геометрия в группах}
\subsection{Граф Кэли}

Для\marginpar{\sf\footnotesize 20.02.04} ориентированного графа
далее будем обозначать $V$ --- множество его вершин, $V_n$ ---
множество вершин кратности $n$, $E$
--- рёбер, $\alpha,\omega:E\rightarrow V$ --- отображения
сопоставляющие ребру соответственно его начало и конец. {\em
Графом Кэли\index{граф Кэли}} $\Gamma(G)$ группы $G=\langle
X\rangle$ называется ориентированный граф, вершинами которого
служат элементы группы $G$, каждое ребро которого имеет метку (или
ещё их называют раскраской) $x\in X$ и из вершины $u$ выходит в
вершину $v$ ребро с меткой $x$ \ifif существует $x\in X$ такой,
что $ux=v$.

\begin{remark}
Граф Кэли зависит от системы образующих.
\end{remark}

\begin{example} Графы Кэли группы диэдра $D_{12}=\langle
a\rangle_{2}\leftthreetimes\langle b\rangle_{12}$ и свободной
группы с двумя образующими:
\begin{figure}[h]%
\hfil\epsffile{groups3.eps}\hfil\epsffile{groups4.eps}\hfil
\end{figure}
\end{example}

Как по графу Кэли восстановить операцию в группе? По правилу
треугольника: если нужно получить произведение $g_1g_2$, то
отшагиваем от единицы по метке $g_1$, затем от $g_1$ по $g_2$ ---
попадём как раз в $g_1g_2$.

{\em Автоморфизм}\index{автоморфизм графа} ориентированного графа
это пара взаимно однозначных отображений $f:V(G)\rightarrow V(G)$
и $f:E(G)\rightarrow E(G)$, сохраняющих начало и конец рёбер и
раскраску, т.е. для любого $e\in E$
$\alpha(f(e))=f(\alpha(e)),~\omega(f(e))=f(\omega(e)),~\lambda(f(e))=\lambda(e)$,
где $\lambda$ --- раскраска. Легко видеть, что $\Aut G$
--- группа.


\begin{theorem}
{$\Aut G\cong G$.}
\end{theorem}

\begin{proof}  Построим изоморфизм $\phi:G\rightarrow\Aut\Gamma(G)$, а
именно, положим $\phi(g)(v)=gv$. \break\vskip-18pt
\insepsright{groups5}\vskip-3pt\hangindent=-32mm \hangafter=-5
\noindent Понятно, что левый сдвиг на $g$
--- автоморфизм графа, а также, что
$\phi(gh)=\phi(g)\phi(h)$. Поскольку $\phi(g)$
--- это умножение слева, то $\ker\phi$ тривиально. Пусть $f\in\Aut\Gamma(G)$ и $f(1)=g$.
Покажем, что $f=\phi(g)$. Рассуждаем индукцией по длине пути,
соединяющего 1 и $v$. Рассмотрим такой $v_1$, что существует $x\in
X$, для которого $v_1x=v$. Тогда $f(v_1)=\phi(g)(v_1)=gv_1$ по
предположению. Но поскольку $gv=gv_1x$, то ребро с меткой $x$ из
$gv_1$ направлено в $gv$, т.е. $f(v)=\phi(g)(v)=gv$.\end{proof}
%\begin{figure}[h]%
%\hfil\epsffile{groups5.eps}\hfil
%\end{figure}
Следующие условия необходимы для того, чтобы граф был графом Кэли
некоторой группы:

$\left.1\right)$ связность;

$\left.2\right)$ из любой вершины исходит (и входит) ровно одно
ребро каждого цвета.

\begin{theorem}Чтобы они были достаточными
необходимо ещё такое условие:
\par $\left.3\right)$ если задано выражение от образующих $\prod
x_{i_j}^{\varepsilon_j}$, то из любой вершины можно отложить путь
с такой меткой, и если такой путь замкнут для одной из вершин, то
всякий путь с такой меткой тоже замкнут.
\end{theorem}

\begin{proof} Покажем, что если для графа выполнены $\left.1\right)$,
$\left.2\right)$ и $\left.3\right)$, то он --- граф Кэли некоторой
группы. Группу определим по правилу треугольника. Докажем
корректность, т.е. что если есть два пути $p_1$ и $p_2$ из $1$ в
$g_1$, то для любого $g_2$ пути $p_1$ и $p_2$, отложенные от него
идут в точку $g_2g_1$. Но это действительно так, поскольку путь
$p_1p_2^{-1}$ из $1$ замкнут, а значит такой путь замкнут и от
$g_2$. Аксиомы группы проверяются совсем легко, также легко
видеть, что первоначальный граф --- граф Кэли построенной группы.
\end{proof}

\begin{exercise} В предположении $\left.1\right)$ и
$\left.2\right)$ свойство $\left.3\right)$ равносильно следующему:
граф однороден, т.е. любая вершина переводится в другую
автоморфизмом, или, что тоже самое, $\Aut\Gamma$ действует на
$V(\Gamma)$ транзитивно.
\end{exercise}

\begin{theorem}Если $H$ --- подгруппа конечного индекса в конечно
порождённой группе $G$ и $\Gamma(G)$ построен по конечной системе
образующих, то существует такой $r$, что $O_r(H)=\Gamma(G)$.
\end{theorem}

\begin{proof} Представим $G$ как $Hg_1\sqcup\dots\sqcup Hg_k$. Фиксируем
некоторый $g\in G$, для него при некоторых $h\in H$ и
$i\in\{1,\dots,k\}$ выполнено $g=hg_i$. Рассмотрим расстояние
$\rho(g,h)=\rho(hg_i,h)=\rho(g_i,1)\leqslant\max\limits_{i\in
\{1,\dots,k\}}\{\rho(g_i,1)\}=m$. Значит в качестве $r$ достаточно
брать $m+1$.\end{proof}

\begin{exercise} Сформулировать и доказать
обратное утверждение.
\end{exercise}

\begin{theorem}Подгруппа $H$ конечного индекса в
конечно порождённой группе $G=\langle x_1,\dots,x_n\rangle$
конечно порождена.
\end{theorem}

\begin{proof} Рассмотрим $Y=O_{3r}(1)\cap H$, где $r$ из предыдущей
теоремы. $Y$ конечно, поскольку граф локально конечен. Покажем,
что $H=\langle Y\rangle$. Предположим обратное. $\langle
Y\rangle\subset H$, выберем $h\in H\setminus\langle Y\rangle$ ---
ближайший к $Y$, а $h_1\in H$ --- ближайший к нему, выражающийся
через $Y$. Рассмотрим $O_r(h)$ и $O_r(h_1)$ --- они не
пересекаются, поскольку $\rho(h_1,h)>3r$, т.к.\,$H\cap
O_{3r}(h_1)=H\cap h_1O_{3r}(1)$. Но тогда середина пути $hh_1$ не
покрывается $r$-окрестностью подгруппы $H$ --- противоречие.
\par\end{proof}

%%%%%%%%%%%%%%%%%%%%%%%%%%%%%%%%%%%%%%%%%%%%%%%%%%%%%%%%%%%

\begin{example}\marginpar{\sf\footnotesize 27.02.04} Если $G=\langle
G\rangle$ --- заданная система образующих, то граф Кэли группы
относительно неё --- полный граф.
\end{example}

\subsection{Квазиизометричность}
\begin{example} Построим графы
Кэли для группы $G=\langle a\rangle_{\infty}=\langle
a^{-2},a^3\rangle=\langle a^{-2},a^3\rangle$:

\begin{figure}[h]%
\hfil\epsffile{groups6.eps}\hfil\epsffile{groups7.eps}\hfil\epsffile{groups8.eps}\hfil
\end{figure}

А теперь граф группы $G=\langle a\rangle_{\infty}\times\langle
b\rangle_{\infty}$:

\begin{figure}[h]%
\hfil\epsffile{groups9.eps}\hfil
\end{figure}
\end{example}

Мы можем заметить, что графы на первых трёх рисунках похожи и
отличны от четвёртого, что приводит к следующему определению:

Пусть $X,Y$ --- метрические пространства. $X$ называется {\em
квазиизометричным}\index{пространства!квазиизометричные} $Y$, если
существует отображение $f:X\rightarrow Y$ со следующими
свойствами:

$\left.1\right)$ $f$ --- {\em
квазиизометрия}\index{квазиизометрия}, т.е. существуют константы
$\lambda>0,c\geqslant 0$, для которых при всех $x,y\in X$

\hfil$\frac{1}{\lambda}\rho_X(x,y)-c\leqslant\rho_Y(f(x),f(y))
\leqslant\lambda\rho_X(x,y)+c$,\hfil

$\left.2\right)$ $f$ --- {\em почти
сюръективно}\index{отображение!почти сюрьективное}, т.е.
существует такая константа $\varepsilon$, что для любого $y\in Y$
найдётся такой $x\in X$, что

\hfil$\rho_Y(f(x),y)<\varepsilon~(Y\subset
O_\varepsilon(f(X)))$.\hfil

\begin{example}
$\Mz\hookrightarrow\Mr$ --- почти сюръективно. Обратная
квазиизометрия $\Mr\rightarrow\Mz$ --- целая часть.
\end{example}

\begin{exercise} Проверить, что $X$ квазиизометрично
$Y$ если существуют квазиизометрические отображения
$f:X\rightarrow Y$ и $g:Y\rightarrow X$ такие, что найдётся
$\varepsilon>0$ для которого $\rho(f(g(y)),y)\leqslant\varepsilon$
и $\rho(g(f(x)),x)\leqslant\varepsilon$. Заметить, что
квазиизометричность
--- отношение эквивалентности.
\end{exercise}

\begin{exercise} Показать, что если $Y\subset X$ и
$O_\varepsilon(Y)=X$, то $X$ и $Y$ квазиизометричны.
\end{exercise}

\begin{corollary}Любая группа квазиизометрична своей подгруппе конечного
индекса.
\end{corollary}

\begin{proof} Т.к.\,тогда существует такой $\varepsilon$, что
$O_\varepsilon(H)\supset G$.\end{proof}

\begin{theorem}При выборе разных
конечных систем образующих конечно порождённой группы её графы
Кэли относительно них квазиизометричны.
\end{theorem}

\begin{proof} Пусть $G=\langle x_1,\dots,x_n\rangle=\langle
y_1,\dots,y_m\rangle$ --- данные системы образующих и $\Gamma_1$ и
$\Gamma_2$ --- соответствующие графы Кэли. Обозначим $G_{\rho_1}$
группу с метрикой
$\rho_1(g,h)=\inf\{n:~x_{i_1}^{\varepsilon_1}\dots
x_{i_n}^{\varepsilon_n}=h^{-1}g\}$, аналогично определим
$G_{\rho_2}$. Понятно, что $\Gamma_1$ квазиизометрично
$G_{\rho_1}$ и $\Gamma_2$ квазиизометрично $G_{\rho_2}$, покажем,
что $G_{\rho_1}$ квазиизометрично $G_{\rho_2}$. А именно,
квазиизометрией является тождественное отображение $G$ на себя,
где константа $\lambda$ есть максимум из
$\lambda_1=\max\limits_{i\in\{1,\dots,n\}}|x_i|_{\rho_2}$ и
$\lambda_2=\max\limits_{i\in\{1,\dots,m\}}|y_i|_{\rho_1}$.\end{proof}

Свойство группы называется {\em
геометрическим}\index{свойство!геометрическое}, если оно
сохраняется при квазиизометриях.

\begin{fact} Почти
нильпотентность --- геометрическое свойство, а почти разрешимость
--- нет.
\end{fact}

\begin{theorem}Пусть $G=\langle x_1,\dots,x_n\rangle$, $|G|=\aleph_0$,
$\Gamma(G)$
--- её граф Кэли. Тогда в $\Gamma$ существует бесконечная
геодезическая (т.е. подмножество изометричное $\Mr$).
\end{theorem}

\begin{proof} Сначала заметим, что для любого $k$ существует отрезок
длины $k$, иначе $\Gamma\subset O_k(1)$ и поскольку граф локально
конечен, то и группа была бы конечной. Затем заметим, что
существует отрезок длины $2k$, середина которого $1$ --- некоторый
отрезок такой длины есть, осталось его середину перегнать
автоморфизмом в $1$. Поскольку через каждую точку и каждое ребро
проходит бесконечное число путей, то получаем и существование
прямой.\footnote{известно, что без использования аксиомы выбора
этого доказать нельзя, однако, достаточен её счётный
случай}\end{proof}

\begin{theorem}Пусть $f:\Mr\rightarrow\Mr$
--- квазиизометрично, тогда $f$ почти сюръективно.
\end{theorem}

\begin{proof} Положим $a_i=f(i)$. Ввиду квазиизометричности
$|a_i-a_{i+1}|\leqslant\lambda+c=\varepsilon$. Слушателям
предоставляется проверить, что в этом случае либо в $\{a_i\}$ есть
сходящаяся к конечному пределу подпоследовательность, либо она
сходится к $+\infty$, или к $-\infty$ при $i\to\infty$. Разбирая
случаи получаем требуемое.\end{proof}

\begin{theorem}Граф конечно
порождённой группы $G$ содержит подграф, квазиизометричный $\Mr$
\ifif в $G$ существует подгруппа конечного индекса вида $\langle
a\rangle_\infty$.
\end{theorem}

\begin{proof} В нетривиальную сторону. Рассмотрим функцию
$f(x):\Gamma\to\Gamma$, точке $x$ сопоставляющее точку $f(x)$
ближайшую к $x$ на прямой $l$. $f$ --- квазиизометрия и
$\rho(f(x),f(y))\leqslant\rho(x,y)+2\varepsilon$. Выберем элемент
$g$, такой, что $|g|>10\varepsilon$. Рассмотрим элементы $1$, $g$
и $g^2$. $\rho(f(1),f(g))\geqslant 8\varepsilon$ и
$\rho(f(g),f(g^2))\geqslant 8\varepsilon$. Понятно, что для
завершения доказательства теоремы достаточно показать, что среди
таких $g$ найдутся такие, что
$\rho(f(1),f(g^2))\geqslant\varepsilon$. Но для всех $g$ противное
выполняться не может ввиду ограниченности графа.\end{proof}

%%%%%%%%%%%%%%%%%%%%%%%%%%%%%%%%%%%%%%%%%%%%%%%%%%%%%%%%%%%%

\begin{theorem}Если\marginpar{\sf\footnotesize 05.03.04} $G$ --- такая
конечно порождённая группа, для которой $|\{g^2\}_{g\in
G}|<\aleph_0$, то $|G|<\aleph_0$.
\end{theorem}

\begin{proof} Пусть $G=\langle x_1,\dots,x_n\rangle$, причём система
образующих $\{x_1,\dots,x_n\}$ содержит все квадраты элементов
группы. Для каждого элемента $g$ группы выберем его кратчайшую
запись через $x_i$ ---
$\prod\limits_{j=1}^{m}x_j^{\varepsilon_j}$. В такой записи буквы
не повторяются. Действительно, пусть $g=ux_ivx_iw$, где $u,v,w$
--- некоторые слова. Тогда
$ux_ivx_iw=u(x_iv)^2v^{-1}w=ux_jv^{-1}w$ --- слово, которое короче
выбранного. Но конечных слов без повторений из конечного числа
букв конечное число, поэтому $|G|<\aleph_0$.\end{proof}

\subsection{Свободные группы}
Напомним определение свободного объекта в категории
$\mathfrak{K}$. Пусть $X\stackrel{id_X}\hookrightarrow G$. $G$ ---
{\em свободный} объект\index{объект!свободный} с {\em базисом}
$X$, если для любого объекта $H$ и отображения
$X\stackrel{f}\rightarrow H$ существует единственный морфизм
$G\stackrel{\phi}\rightarrow H$, такой, что $f=\phi i$. Будем в
этом случае писать $G=F(X)$. Мощность множества $X$ называется его
{\em рангом}\index{ранг}.\footnote{вообще говоря определён
некорректно, но для групп годится}

\begin{example}
$F[x_1,\dots,x_n]$ --- многочлены без свободного члена
--- свободная ассоциативная коммутативная алгебра над $F$ с
базисом $\{x_i\}_{i\in\{1,\dots,n\}}$.
\par Свободные группы в
любом многообразии.
\end{example}

\begin{exercise} Свободная
группа единственна: свободные группы одинакового ранга изоморфны.
\end{exercise}

Любая биекция $X\to Y$ единственным образом продолжается до
изоморфизма $F(X)\to F(Y)$. Рассмотрим диаграммы:
\begin{figure}[h]
\hfil\epsffile{groups2.1}\hfil\epsffile{groups3.1}\hfil
\end{figure}
\\Отображение $\phi\psi:F(Y)\to F(Y)$ ограниченное на $Y$
тождественно по определению свободной группы. По тому же
определению имеем и что $\phi\psi=id_{F(X)}$. Аналогичное
равенство имеется и для $\psi\phi$. Значит, $\phi$ (очевидно,
единственный) изоморфизм.

\begin{theorem}
{$F(X)=\langle X\rangle$.}
\end{theorem}

\begin{proof} Действительно, поскольку $F(X)$ --- свободная, то можно
взять встречный к вложению $i:\langle X\rangle\to F(X)$
гомоморфизм $\phi$, для него $\phi|_X=id_X$. Следовательно, по
замеченному выше, имеем равенство $\langle X\rangle=F(X)$.
\end{proof}

\begin{fact} Если $Y\subset X$, то $\langle Y\rangle\subset
F(X)$
--- свободна с базисом $Y$.
\end{fact}

\begin{theorem}Если $Y\subset X$, то
$F(X)/\ncl{Y}=F(\mbox{смежные классы }\ncl{Y}\doteqdot Z\ncl{Y})$.
\end{theorem}

\begin{proof} Нужно проверить, что
%диаграмма

\begin{figure}[h]
\hfil\epsffile{groups4.1}\hfil
\end{figure}

Рассмотрим  отображение $F(X)\stackrel{\psi}\to H$, задающееся на
базисе следующим образом: $Y\mapsto 1$ и $zy\mapsto
h(z(\ncl{Y}))$. Поскольку $Y\subset\ker\phi$, то $\phi$ определён
корректно.\end{proof}

\begin{fact} $F(X)/F(X)'$
--- свободная абелева группа.
\end{fact}

Положим $\rk F(X)=\min\{\mbox{количество порождающих}\}$. Проверим
корректность. Будем считать, что $n=|X|<\infty$, поскольку иначе
$|X|=|F(X)|$ и утверждение тривиально. Пусть $F(X)=\langle
y_1,\dots,y_m\rangle_{m<n}$. Но тогда и любая $n$-порождённая
группа порождается $m$ элементами. Но из курса алгебры мы знаем,
что это вообще говоря не так, например для абелевых групп.

Покажем, что свободные группы существуют. Рассмотрим произвольное
множество $X$ и его копию $X^{-1}$ --- элементы $X$, к которым
формально приписана степень $-1$. $F(X)$ будет группа, элементами
которой будут конечные слова из букв $X\sqcup
X^{-1}\sqcup\varnothing$ с отношением эквивалентности, полагающим
два слова эквивалентными, когда одно получается из другого путём
вписывания или вычёркивания подслов вида $xx^{-1}$ и $x^{-1}x$, а
групповая операция это просто операция приписывания слов друг к
другу.

\begin{exercise} Проверить, что указанная
конструкция обладает аксиомами группы. Проверить её свободность.
\end{exercise}

\noindent{\bf Лемма о пинг-понге.}\index{лемма!о пинг-понге} {\sl
Пусть $G=\langle a,b\rangle$ и $G$ действует на $M$, в котором
выбраны непустые непересекающиеся подмножества $X$ и $Y$.
Известно, что для любых $k\in\Mz\setminus\{0\}$ и $x\in X,y\in Y$
$a^k(x)\in Y$ и $b^k(y)\in X$. Тогда $G=F(a,b)$.}

\begin{proof} Нужно показать, что между $a$ и $b$ нет соотношений.
Рассмотрим гомоморфизм $F(a,b)\stackrel{\phi}\to G$ и покажем, что
его ядро тривиально. Пусть $w$
--- слово в $\ker\phi$, и $w=a^{k_1}b^{l_1}\dots a^{k_n}b^{l_n}a^{k_{n+1}}$,
где все $k_i$ и все $l_i$ не нули. Тогда $\phi(w)(x)$ лежит в $Y$
при всех $x\in X$, т.е. $\phi(w)\ne 1$ и $w\notin\ker\phi$. Если
же $w=a^{k_1}b^{l_1}\dots a^{k_n}b^{l_n}$, то можно рассмотреть
$a^{k_1}wa^{-k_1}$, он лежит в $\ker\phi$, поскольку $\ker\phi\lhd
G$, но это предыдущий случай. Поэтому $w=1$.\end{proof}

\begin{example} Пусть $\langle
a=(_{2~1}^{1~0}),b=(_{0~1}^{1~2})\rangle\subset\mathbf{SL}_2(\Mz)$.
Эта группа свободна, т.к.\,действует в $\Mr^2$ умножением слева
($a^k(_y^x)=(_{y+2kx}^{~~~x})$, $b^k(_y^x)=(_{~~~y}^{x+2ky})$),
переставляя области $X$ и $Y$:

\begin{figure}[h]%
\hfil\epsffile{groups10.eps}\hfil
\end{figure}
\end{example}

\begin{fact} Свободная группа финитно аппроксимируема.
\end{fact}

Действительно, она вкладывается в $\mathbf{SL}_2(\Mz)$ и если
$(_{z~t}^{x~y})$ её элемент, то при $n>x,y,z,t$ можем устроить
нетривиальное на данном элементе отображение в
$\mathbf{SL}_2(\Mz_n)$.

\begin{corollary}Не существует
нетривиального тождества, выполненного во всех конечных группах.
\end{corollary}

\begin{proof} Предположим таковое имеется, тогда запишем его в виде
$x_1^{\varepsilon_1}\dots x_n^{\varepsilon_n}=1$. В свободной
группе правая часть равенства не тривиальна. Ввиду финитной
аппроксимируемости может устроить гомоморфизм в конечную группу,
при котором слово $x_1^{\varepsilon_1}\dots x_n^{\varepsilon_n}$
будет нетривиальным
--- противоречие.\end{proof}
%%%%%%%%%%%%%%%%%%%%%%%%%%%%%%%%%%%%%%%%%%%%%%%%%%%%%%%%%%%%

Свободная\marginpar{\sf\footnotesize 12.03.04} группа ранга 2 (а
значит и любого $\leqslant\omega_0$) финитно аппроксимируема
откуда следует, что свободная группа конечного ранга хопфова.

Если $F$ --- свободная группа, то $\{x_1,\dots,x_n\}$
--- {\em свободный базис}\index{свободный базис}, если

$\left.1\right)$ $\langle x_1,\dots,x_n\rangle=F$

$\left.2\right)$ $x_1,\dots,x_n$ --- независимы, т.е.\,$\prod
x_{i_j}^{\varepsilon_j}=1$ \ifif $\prod x_{i_j}^{\varepsilon_j}$
формально равен $1$.

\begin{theorem}Если $F_n=F(x_1,\dots,x_n)$
--- свободная группа ранга $n$ и $\langle
y_1,\dots,y_n\rangle=F_n$, то $\{y_1,\dots,y_n\}$ --- свободный
базис.
\end{theorem}

\begin{proof} Отображение $\phi:F_n\to F_n$ заданное по правилу
$\phi(x_i)=y_i$ является сюръективным гомоморфизмом. Значит по
теореме о гомоморфизмах $f_n\cong F_n/\ker\phi$. Далее из
хопфовости будем иметь $\ker\phi=1$; а при изоморфизме базис
переходит в базис.\end{proof}

\begin{corollary}Если $F(X)$ --- свободная группа и $u,v\in F(X)$, то
$uv=vu$ \ifif существует такой $w\in F(X)$, что $u,v\in\langle
w\rangle$.$\square$
\end{corollary}

\begin{theorem}Граф Кэли свободной группы $F(X)$
относительно свободных образующих является деревом.
\end{theorem}

\begin{proof} Предположим в графе нашёлся нетривиальный цикл. Это значит,
что выполнено тождество $x_{i_1}^{\eps_1}\dots
x_{i_k}^{\eps_k}=1$, что невозможно, поскольку $x_1,\dots,x_n$ ---
свободный базис.\end{proof}

Напомним, что группа $G$ действует на графе $\Gamma$, если задан
гомоморфизм $\phi:G\to\Aut\Gamma$. Действие называется {\em
свободным}\index{действие!свободное}, если стабилизатор всех точек
тривиален, т.е. $\phi_g(v)\ne v$ при всех $v\in V(\Gamma)$ и $g\in
G\setminus\{1\}$.

\begin{theorem}Пусть $G$ свободно действует на дереве. Тогда для всех
нетривиальных $g\in G$ существует единственная инвариантная прямая
$l$, т.е.\,такая, что $gl=l$.
\end{theorem}

\begin{proof} Зафиксируем $g$ и выберем вершину $t$ графа, сдвигающуюся
при $g$ на минимальное расстояние: $\rho(t,gt)=\min$. Существует
единственный путь, соединяющий $t$ и $gt$. Дальше строим путь
$[gt,g^2t]$ и т.д. Могут ли $[gt,g^2t]$ и $[t,gt]$ пересекаться не
в концевой точке? Допустим они пересеклись по точке $p\in[t,gt]$.
Тогда $gp\in[gt,g^2t]$, а значит $\rho(p,gp)<\rho(t,gt)$, что
противоречит выбору $t$. Это замечание и то, что граф у нас
древесный, приводит к существованию прямой.

Предположим такая прямая не единственна. Рассмотрим две такие
прямые $l_1$ и $l_2$. Их взаимное расположение имеет один из трёх
видов: могут непересекаться или пересекаться, совпадать по
некоторому кусочку. Разберём случай $l_1\cap l_2=\empty$.
Существует единственный отрезок $[p_1,p_2]$, который их соединяет.
Подействуем на всё элементом $g$. $gp_i\ne p_i$ и $gp_i\in l_i$.
Легко видеть, что $\rho(gp_1,gp_2)\ne\rho(p_1,p_2)$,
противоречие.\end{proof}

\begin{theorem}
Пусть $G$ свободно действует на дереве. Если $u,v\ne 1$, $uv=vu$ и
$l$ --- инвариантная прямая для $u$, тогда $vl$ --- инвариантная
прямая для $u$ (и $vl=l$).
\end{theorem}

\begin{proof} $uv(l)=vu(l)=v(l)$, но ввиду единственности $vl=l$.\end{proof}

\begin{theorem}
Пусть $G$ свободно действует на дереве. Пусть $w\in G$ такой, что
$wl=l$ и при некотором $x\in l$ $\min\limits_{w\in
G,~wl=l}=\rho(x,wx)$. Тогда для всех таких $g\in G$, что $gl=l$
выполнено $g\in\langle w\rangle$.
\end{theorem}

\begin{proof} Ввиду минимальности $w$, имеем
$\mbox{сдвиг}(g)(x)=n\cdot\mbox{сдвиг}(w)(x)$ при некотором
$n\in\Mn$. Отсюда $w^ng^{-1}$ точку $x$ никуда не сдвигает, это
значит, что $w^n=g$.\end{proof}

\subsection{Подгруппы свободных групп}

\begin{example}
Покажем, что $F_{\infty}<F_2$.

Пусть $F_2=F(a,b)$. Рассмотрим множество
$\{a,b^{-1}ab,b^{-2}ab^{2},\dots\}=\{a^{b^k}:~k\in\Mn\cup\{0\}\}$.
Пусть есть нетривиальное соотношение $w(a,a^{b},\dots,a^{b^n})\ne
1$.\marginpar{?}
\end{example}

Пусть $G$ эффективно\index{действие!эффективное} действует на
связном графе $\Gamma$ (т.е. $\ker(G\to\Aut\Gamma)=\{1\}$).
Подграф $\Phi\subset\Gamma$ называется {\em фундаментальной
областью}\index{фундаментальная!область}, если
\par 1) $\Phi$ --- связный;
\par 2) для каждой вершины $v\in\Gamma$ существует единственный
$u\in\Phi$ и существует $g\in G$, такие, что $gv=u$.

\begin{theorem}
Для любого действия любой группы на любом связном графе существует
фундаментальная область.
\end{theorem}

\begin{proof} Пусть $M=\{\mbox{связный подграф }S:~\forall v\in
V(\Gamma)~|Gv\cap S|\leqslant 1\}$. По лемме Цорна можем выделить
максимальный элемент в $M$ (относительно включения). Он и будет
фундаментальной областью.

Действительно, иначе возьмём вершину $v$, ближайшую к $\Phi$ из
тех, что $Gv\cap\Phi=\empty$. Если $\rho(v,\Phi)=1$, то добавив
$v$ и отрезок, соединяющий её с $\Phi$ получим граф больше. Иначе
можно на отрезке, соединяющим $v$ и $\Phi$ взять точку $u$ с
$\rho(u,v)=1$. Существует $g$, такой, что $gu\in\Phi$, а тогда
пара $gu,gv$ по свойствам ничем не отличается от $u,v$, но при
этом $\rho(gv,gu)=1$, что противоречит выбору $v$.\end{proof}

Перед тем как доказать главную теорему этого параграфа,
сформулируем ещё одну лемму о пинг-понге.

\begin{theorem}\index{лемма!о пинг-понге}
$G=\langle X\rangle$ действует на $M$, для всех $x,y\in X$ при
$x\ne y$ выполнено $A_x\cap A_y=\empty$. Пусть ещё для всех
$k\in\Mz\setminus\{0\}$ верно $x^kA_y\subset A_x$. Тогда $G=F(X)$.
\end{theorem}

\begin{proof} Упражнение.

\begin{theorem}
Пусть $G$ свободно действует на дереве $T$. Тогда $G$ ---
свободна.
\end{theorem}

\begin{proof} Пусть $\Phi\subset T$ --- фундаментальная область
относительно заданного действия. Для всякой вершины $v$ существует
единственный сдвиг $g\in G$, при котором $v\in g\Phi$.
Действительно, если $v\in g_1\Phi\cap g_2\Phi$ при $g_1\ne g_2$,
то $u=g_1^{-1}v\in\Phi\cap g_1^{-1}g_2\Phi=\Phi\cap h\Phi$, а это
значит что $u,h^{-1}u\in\Phi$ и ввиду свободности действия и
определения $\Phi$ будет $h=1$.

Рассмотрим множество $Y=\{g\in G:~\rho(\Phi,g\Phi)=1\}$. Покажем,
что $\angle{Y}=G$. Предположим обратное и возьмём $g\in
G\setminus\angle{Y}$ для которого достигается $\min\limits_{g\in
G\setminus\angle{Y}}\rho(\Phi,g\Phi)$. тот минимум не равен
единице по выбору $Y$. Соединим путём $\Phi$ и $g\Phi$ и возьмём
точку $\xi$ на нём на расстоянии 1 от $g\Phi$. $\xi\in h\Phi$ при
некотором $h$. $\rho(h\Phi,g\Phi)=1$ и $h$ выражается через $Y$.
Поскольку $\rho(\Phi,h^{-1}g\Phi)=1$, то $h^{-1}g$ выражается
через $Y$, а стало быть $g$ тоже.

Понятно, что сам $Y$ не базис, поскольку $Y=Y^{-1}$. Возьмём $X$,
удовлетворяющий равенству $Y=X\cup X^{-1}$.

\vskip-6pt \inspicright{groups6} \hangindent=-80mm \hangafter=-12
Всякое конечный путь делит дерево на два поддерева, растущих из
его концов. За $A_x$ обозначим объединение двух таких поддеревьев,
полученных из пути от $x^{-1}\Phi$ через $\Phi$ к $x\Phi$. Теперь
заметим, что для предложенной конструкции выполнено условие леммы
о пинг-понге. Рассмотрим инвариантную прямую $l_x$ для $x$. Она
проходит через $\Phi$, иначе $\rho(\Phi,x\Phi)=\rho(\Phi,p\in
l_x)+\rho(p,xp)+\rho(xp,x\Phi)>1$. Значит, инвариантная прямая
выглядит так, как на картинке. Покажем, что $x^kA_y\subset A_x$.
Это видно из того, что $x^k\subset A_x$. Наконец, применяя лемму,
получаем требуемое $F(X)=G$.\end{proof}

Так как подгруппа свободной группы действует свободно на графе
Кэли группы, то доказана

\begin{theorem}[Нильсен, Шрайер]\index{теорема!Нильсена---Шрайера}
Подгруппа свободной группы свободна.$\square$
\end{theorem}

%%%%%%%%%%%%%%%%%%%%%%%%%%%%%%%%%%%%%%%%%%%%%%%%%%%%%%%%%%%%
\subsection{Немного топологии}
Нас будут полезны некоторые понятия топологии, однако,
рассматривать их мы будем применительно к графам.

Пусть $X$ линейно связное хаусдорфово топологическое пространство.
Пусть $p,q:\Mi\to X$ --- пути в $X$. Они называются {\em
гомотопными}\index{гомотопные пути} (обозначение $p\approx q$),
если существует непрерывное отображение $h:\Mi\times\Mi\to X$,
называемое {\em гомотопией}\index{гомотопия}, для которого
$h(0,x)=p(x)$ и $h(1,x)=q(x)$ для всех $x\in\Mi$. Пусть $x_0\in
X$, тогда классы гомотопных петель образуют группу: единица ---
тривиальная петля $p(x)=x_0$; композиция петель $p$ и $q$ ---
петля, которая на первой половине отрезка пробегает $p$ с вдвое
большей скоростью, на второй --- $q$; обратная петля --- та же
самая петля, пробегаемая в другую сторону. Эта группа обозначается
$\pi_1(X,x_0)$ и называется {\em
фундаментальной}\index{фундаментальная!группа}\index{группа!фундаментальная}
группой $X$. Ввиду линейной связности $X$ фундаментальная группа
не зависит от выбора точки $x_0$.

Граф $\Gamma$ можно рассматривать как метрическое пространство.
Пусть $x_0$ --- вершина $\Gamma$. Фундаментальная группа графа
может быть определена так: $\pi_1(\Gamma,x_0)=\{\mbox{несократимые
петли с началом в }x_0\}$. Поскольку
$\pi_1(\Gamma,x_0)=\pi_1(\Gamma,y_0)$, то для сокращения записи
будем иногда писать $\pi_1(\Gamma)$.
\vskip-11pt\insepsright{groups11}\hangindent=-26mm
\hangafter=-5{\em Букетом}\index{букет} семейства топологических
пространств $\{X_s\}_{s\in S}$ называется пространство
$\bigvee\limits_{s\in S}X_s:=\bigsqcup\limits_{s\in S}X_s/\sim$,
где $x_s\sim x_{s'}$ при всех $s,s'\in S$ для некоторого семейства
точек $x_s\in X_s$, и $y\nsim z$ если $y\ne x_s$ или $z\ne x_s$ ни
для какого $s\in S$. Понятно, что с точностью до гомотопии, в
случае линейно связных пространств можно выбирать семейство точек
$x_s\in X_s$ произвольно.

\begin{theorem}
Группа $\pi_1(\Gamma)$ свободна.
\end{theorem}

\begin{proof} Достаточно показать, что $\Gamma\approx\bigvee S^1$. Стянем
в точку максимальное поддерево в $\Gamma$. Понятно, что при этом
фундаментальная группа не изменится, и что получится букет
окружностей. При этом ранг свободной группы будет равен количеству
рёбер.\end{proof}

{\em Эйлеровой характеристикой}\index{эйлерова характеристика}
графа $\Gamma$ называется число
$e(\Gamma)=|V(\Gamma)|-|E(\Gamma)|$. В терминах эйлеровой
характеристики можно написать $\rk\pi_1(\Gamma)=1-e(\Gamma)$.

Если задано отображение $f:X\to Y$ линейно связных пространств c
отмеченными точками $x_0\in X,~y_0\in Y$, то оно естественным
образом индуцирует гомоморфизм групп
$f_*:\pi_1(X,x_0)\to\pi_1(Y,y_0)$.

Пусть заданы $X$ и $\Mb$ --- линейно связные топологические
пространства. Отображение $p:X\to\Mb$ называется {\em
накрытием\index{накрытие}}, если прообраз $p^{-1}(b)$ любой точки
$b\in\Mb$ дискретен, и у всех точек $b\in\Mb$ найдутся такие
окрестности $U_b$, что $p^{-1}(U)=\bigsqcup\limits_{i\in
p^{-1}(b)}U_i$ и ограничение $p|_{U_i}:U_i\to U$ --- гомеоморфизм.
$\Mb$ называется {\em базой накрытия\index{база!накрытия}}, $X$
--- {\em накрывающим\index{пространство!накрывающее} пространством}, а $|p^{-1}(b)|$ --- {\em
степенью накрытия}\index{степень накрытия}, или ещё если
$n=|p^{-1}(b)|$, то накрытие $p$ называют
{\em\index{накрытие!$n$-листное} $n$-листным}.

\vskip-12pt \inspicright{groups7} \hangindent=-18mm \hangafter=-4
Покажем, что степень накрытия определена корректно, т.е.\,не
зависит от точки $b$. Пусть $x,y\in\Mb$ --- какие-то две различные
точки, в которых степени не равны. Поскольку $\Mb$ линейно связно,
то существует путь $\gamma:\Mi\to\Mb$ их соединяющий. В $\Mi$
можем выбрать точную нижнюю грань того множества точек, для
которых степень их образов при $\gamma$ не совпадает со степенью
начала. Рассмотрев окрестность образа при $\gamma$ этой точки,
участвующую в определении накрытия получим противоречие.

Дадим также определение накрытия для графов. А именно, морфизмом
графов будем называть пару отображений из вершин и рёбер одного в,
соответственно вершины и рёбра другого, уважающее начало, конец и
ориентацию рёбер и отмеченную точку. Если $X$ и $B$ связные графы
с отмеченными точками, то накрытием называется такой морфизм
$p:X\to B$, что при $p(x)=b$ отображение $p|_{\{e\in
E(X):~\alpha(e)=x\}}:X\to\{e\in E(B):~\alpha(e)=b\}$ ---
биективно, и соответственно $p|_{\{e\in
E(X):~\omega(e)=x\}}:X\to\{e\in E(B):~\omega(e)=b\}$
тоже.

\begin{example}  Окружность можно накрыть прямой, а
именно берётся отображение $p:\Mr\to S^1$, заданное формулой
$p(x)=e^{2\pi ix}$.
\end{example}

%здесь что-то про графы
Пусть $p:X\to B$ --- накрытие. {\em Поднятием}\index{поднятие
пути} пути $\gamma(t)\subset B$ называется такой путь
$\widetilde{\gamma}(t)\subset X$, что
$p(\widetilde{\gamma}(t))=\gamma(t)$ для всех $t\in\Mi$

\noindent{\bf Лемма о поднятии пути}\index{лемма!о поднятии пути}.
{\sl Если $p$
--- накрытие $\alpha(\gamma(t))=b_0$ и $x_0\in p^{-1}(b_0)$, то
существует единственное поднятие пути $\gamma(t)$ с началом в
точке $x_0$.}

\begin{proof} Выберем для всех точек пути $\gamma$ выбираем окрестность,
удовлетворяющую определению накрытия. Поскольку $\gamma$
бикомпактен как непрерывный образ компакта, то из покрытия
выбранными окрестностями можно выделить конечное подпокрытие. По
ним легко построить поднятие пути $\gamma$.

Предположим, что существуют два поднятия $\gamma_1$ и $\gamma_2$
пути $\gamma(t)$ с заданной начальной точкой. Возьмём инфимум $s$
по тем точкам $t$ отрезка $\Mi$, в которых
$\gamma_1(t)\ne\gamma_2(t)$. Теперь легко получить противоречие c
определением накрытия.\end{proof}

\noindent{\bf Лемма о поднятии гомотопии}\index{лемма!о поднятии
гомотопии}. {\sl Пусть $p:X\to B$
--- накрытие, a $u_1$ и $u_2$ --- петли с совпадающими началами в $X$.
Тогда из гомотопности путей $p(u_1)$ и $p(u_2)$ следует
гомотопность путей $u_1$ и $u_2$.}

\begin{proof} Обозначим $v_1(t)=p(u_1(t))$ и $v_2(t)=p(u_2(t))$. Возьмём
гомотопию $\gamma(s,t)$, соединяющую $v_1(t)$ и $v_2(t)$. При
фиксированном $t_0\in\Mi$ имеем путь $\gamma(s,t_0)$, соединяющий
$v_1(t_0)$ и $v_2(t_0)$. Рассмотрим конечную точку
$\widetilde{\gamma}(s,t_0)$ поднятия пути $\gamma(s,t_0)$ с
начальной точкой $u_1(t_0)$. При $t_0$ пробегающим весь $\Mi$
будем иметь путь, составленный из конечных точек
$\widetilde{\gamma}$, который проецируется в $v_1$, с начальной
точкой $u_1(0)$. Поскольку такой путь единствен, то это есть не
что иное, как $u_1$. А значит, $\widetilde{\gamma}(s,t)$ и
образует гомотопию между $u_1$ и $u_2$.\end{proof}

\begin{corollary}{Если $p$
--- накрытие, то $p_*$ --- мономорфизм.}$\square$
\end{corollary}

\begin{theorem}Пусть $B$ --- линейно связное локально линейно связное локально односвязное
хаусдорфово пространство. Тогда для любой подгруппы
$H\subset\pi_1(B)$ существует единственное накрывающее
пространство $X$ вместе с накрытием $p_H$, такое, что
$p_{H_*}(\pi_1(X))=H$. \label{sub_fund_gr}
\end{theorem}

\begin{proof} Приведём доказательство этой теоремы для случая графов.
Пусть $H\subset\pi_1(B,v)$ --- данные группы. Можем взять в $B$
максимальное поддерево $T$. Тогда для любого $x\in V(B)$
существует единственный путь $\overline{vx}$ из $v$ в $x$ по $T$.
Пусть $e$
--- ребро в $B$, с началом в $x$ и концом в $y$. Определим элемент
$\widetilde{e}\in\pi_1(B,v)$ как класс петли
$\overline{vx}e\overline{vy}^{-1}$. Пусть
$W=\{Hg\}_{g\in\pi_1(B,v)}$ и положим $V(X)=V\times W$ и
$E(X)=E\times W$. Если $e'=(e,Hg)$, скажем, что
$\alpha(e')=(x,Hg)$ и $\omega(e')=(y,Hg\widetilde{e})$, где
$x=\alpha(e)$ и $y=\omega(e)$. Очевидно, что определённый таким
образом $X$ --- связный граф.

Теперь определим накрытие. Положим $p_H|_{V(X)}:V\times W\to V$ и
$p_H|_{E(X)}:E\times W\to E$ --- естественные проекции. Покажем,
что $p_{H_*}(\pi_1(X,v'))\subset H$, где $v'=(v,H)\in V(X)$.
Возьмём замкнутый путь $p'=e_1'\dots e_n'$ в $X$, начинающийся в
$v'$. $e_i'=(e_i,Hg_i)$, поэтому $p_H(p')=e_1\dots e_n$, а
поскольку $p'$ --- путь в $X$, то $Hg_{i+1}=Hg_i\widetilde{e}_i$,
а значит концом $p'$ является
$y'=(y,H\widetilde{e}_1\dots\widetilde{e}_n)$, где $y=\omega(p)$
поэтому $v=y$ и $H=H\widetilde{e}_1\dots\widetilde{e}_n$, т.е.
$\widetilde{e}_1\dots\widetilde{e}_n\in H$. Следовательно $p$
представляет элемент $\widetilde{e}_1\dots\widetilde{e}_n\in H$,
значит $p_{H_*}(\pi_1(X,v))\subset H$. Чтобы доказать равенство,
возьмём петлю $p=e_1\dots e_n$ с началом в $v\in B$,
соответствующую элементу $g\in H$. Заметим, что
$p\sim\tilde{e}_1\dots\tilde{e}_n$, а
$\tilde{e}_1\dots\tilde{e}_n\in H$, поэтому $p=e_1'\dots e_n'$ с
началом в $v'$, где $e_i'=(e_i,H\tilde{e}_1\dots\tilde{e}_{i-1})$.
Это и есть петля в $X$, проектирующаяся в $g$.

Пусть $q:Y\to B$ другое накрытие из условия теоремы. Покажем, что
существует морфизм графов $f:Y\to X$, такой, что $pf=q$ --- из
этого будет следовать единственность, поскольку применяя это же
наблюдение в другом порядке ($f:X\to Y$), получим изоморфизм $X$ и
$Y$.

Если $\gamma$ --- путь из отмеченной точки $Y$ в некоторую другую
точку $y$, то за $f(y)$ возьмём конец поднятия пути $f\gamma$ с
началом в отмеченной точке $X$. Корректность отображения $f$
следует из того, что ${p_H}_*\pi_1(X)=q_*\pi_1(Y)$.\end{proof}

\vskip 2pt\noindent{\bf Формула Шрайера}\index{формула!Шрайера}.
{\sl Пусть $H$
--- подгруппа конечного индекса $j$ в свободной группе $F$
конечного ранга. Тогда имеет место следующая формула: $\rk
H-1=j(\rk F-1)$}

\begin{proof} Возьмём граф, фундаментальная группа которого есть $\rk F$,
например букет окружностей $\bigvee\limits_{i\in\rk F}S^1$. Его
эйлерова характеристика есть $\rk F-1$. Возьмём накрытие $p_H$.
Понятно, что эйлерова характеристика накрывающего графа равна $\rk
H-1$, а по конструкции из доказательства теоремы \ref{sub_fund_gr}
степень накрытия равна $j$.\end{proof}
%%%%%%%%%%%%%%%%%%%%%%%%%%%%%%%%%%%%%%%%%%%%%%%%%%%%%%%%%%%%

{\em Головой}\index{голова графа} графа $\Gamma$ мы будем называть
те вершины и рёбра, которые лежат на несократимых замкнутых путях.
Остальное естественно назвать {\em хвостами}\index{хвост графа}.

Легко заметить, что $\pi_1(\Gamma)$ конечно порождена \ifif голова
$\Gamma$ конечна (хвосты на фундаментальную группу не влияют).
Поскольку при накрытии несократимые пути переходят в несократимые,
то образ головы лежит в голове.

\vskip-12pt \inspicright{groups8} \hangindent=-24mm \hangafter=-4
Пусть $p_1:X_1\to B$ и $p_2:X_2\to B$ --- накрытия.
Соответствующие отмеченные точки обозначим $x_1,x_2,b_0$. Построим
обратный образ, а именно, такой $M$, что $V(M)=\{(v_1,v_2)\in
V(X_1)\times V(X_2):~p_1(v_1)=p_2(v_2)\}$ и $E(M)=\{(e_1,e_2)\in
E(X_1)\times E(X_2):~p_1(e_1)=p_2(e_2)\}$, где ребро $(e_1,e_2)$
соединяет точки
$\big(\alpha(v_1),\alpha(v_2)\big)$\break\vskip-18pt
\inspicleft{groups9}\vskip-5pt\noindent \hangindent=24mm
\hangafter=-5 и $\big(\omega(v_1),\omega(v_2)\big)$. Пусть
$\Gamma_0$
--- связная компонента $M$ с отмеченной точкой $(x_1,x_2)$.
Очевидно,  $s_1,s_2,p_1s_1,p_2s_2$ --- накрытия. Заметим, что
$\Image(p_1s_1)_*=\Image{p_1}_*\cap\Image{p_2}_*$. Действительно,
включение $\Image(p_1s_1)_*\subset\Image{p_1}_*\cap\Image{p_2}_*$
тривиально. Обратное включение следует из того, что если
$f\in\Image{p_1}_*\cap\Image{p_2}_*$, то существуют
$f_1\in\pi_1(X_1)$ и $f_2\in\pi_1(X_2)$, переходящие в $f$ при
применении ${p_1}_*$ и ${p_2}_*$ соответственно.

\begin{theorem}[Хаусон]\index{теорема!Хаусона} Пусть
$F$ --- свободная группа, а $H$ и $K$ --- её конечно порождённые
подгруппы. Тогда $H\cap K$ тоже конечно порождена.
\end{theorem}

\vskip-12pt \inspicright{groups10} \hangindent=-24mm
\hangafter=-5\begin{proof}  Можно рассматривать подгруппы в
$F_2=F(a,b)$. $\pi_1(\Theta)=F_2$, где $\Theta$ граф с двумя
вершинами и тремя рёбрами в форме буквы $\Theta$. Пусть $X_1$ и
$X_2$ --- пространства, накрывающие $H$ и $K$ соответственно.
Пусть $\Gamma'_0=\mbox{голова }\Gamma_0$, $X'_1=\mbox{голова }X_1$
и $X'_2=\mbox{голова }X_2$. Заметим, что образ головы графа снова
голова. Поскольку хвосты ни на что не влияют, можно считать, что
все вершины имеют кратность 2 или 3, а вершины кратности 2
заменить на длинное ребро. В графе, в котором все вершины имеют
кратность 3 имеет место формула
$rk\pi_1(\Gamma)-1=|E(\Gamma)|-|V(\Gamma)|=\frac{1}{2}|V(\Gamma)|$.
Также справедлива
$|V_3(\Gamma'_0)|\leqslant|V_3(X'_1)||V_3(X'_2)|$, откуда
окончательно получаем $\rk(H\cap K)-1\leqslant 2(\rk H-1)(\rk
K-1)$.\end{proof}

В связи с этим есть

\begin{hypothesis}[Х. Нейман]\index{гипотеза!Х. Нейман} $\rk H\cap
K-1\leqslant(\rk H-1)(\rk K-1)$.
\end{hypothesis}

%\subsection{Факторгруппы свободных групп}
%\subsection{Алгоритмические проблемы теории групп}
%\subsection{HNN-расширения}
%\subsection{Диаграммы ван Кампена и Хауи}
%Начнём с примера.

%Пусть $S$ --- ориентированная замкнутая двумерная компактная
%поверхность без края. {\em Картой}\index{карта} на $S$ называется
%граф $\Gamma\subset S$, делящий её на односвязные области. Другими
%словами, $S\setminus\Gamma=\sqcup(D_i\cong\Mr^2)$; $D_i$
%называются {\em клетками\index{клетка}} или {\em
%гранями}\index{грань}.

\newpage
\addcontentsline{toc}{section}{Литература}
\renewcommand{\refname}{Литература}
\begin{thebibliography}{MKS}
\bibitem[KM]{KM}\litr{Каргаполов М.И., Мерзляков Ю.И.}{Основы теории групп}{М.:
Наука, 1982}
\bibitem[Kur]{Kur}\litr{Курош А.Г.}{Теория групп}{М.: Наука,
1967}
\bibitem[BOl]{BOl}\litr{Бахтурин Ю.А., Ольшанский
А.Ю.}{Тождества}{Итоги науки и техн. ВИНИТИ. Соврем. пробл. матем.
Фундам. направл., 1988, 18, С. 117--240}
\bibitem[OlSh]{OlSh}\litr{Ольшанский А.Ю., Шмелькин
А.Л.}{Бесконечные группы}{Итоги науки и техн. ВИНИТИ. Соврем.
пробл. матем. Фундам. направл., 1989, 37, С. 5--113}
\bibitem[Ne]{Ne}\litr{Нейман
Х.}{Многообразия групп}{М.: Мир, 1969}
\bibitem[Ol]{Ol}\litr{Ольшанский А.Ю.}{Геометрия определяющих соотношений в
группах}{М.: Наука, 1989}
\bibitem[LSh]{LSh}\litr{Линдон Р., Шупп П.}{Комбинаторная теория
групп}{М.: Мир, 1980}
\bibitem[MKS]{MKS}\litr{Магнус В., Каррас А., Солитер
Д.}{Комбинаторная теория групп}{М.: Наука, 1974}
\end{thebibliography}

\newpage
\section{Дополнение\protect\footnote{нарисовано тут просто так...}}
\subsection{Теория множеств}
Дадим ещё несколько определений из теории множеств, которые
понадобятся для формулировки некоторых других утверждений,
эквивалентных аксиоме выбора.

Множество $X$ с линейным порядком $<$ называется {\em
вполне\index{множество!вполне упорядоченное} упорядоченным}, если
у любого непустого подмножества $Y\subset X$ есть наименьший
элемент, или, что тоже самое, любая бесконечная убывающая
последовательность начиная с некоторого места стабилизируется.

Функция $f:X\to Y$ называется {\em вложением}\index{вложение},
если для всех $x,y\in X$ $x<y$ \ifif $f(x)<f(y)$. Вложение $f:X\to
Y$ вполне упорядоченных множеств --- {\em начальное
вложение}\index{вложение!начальное}, если для каждых $x\in X$,
$y\in Y$ из того, что $y<f(x)$ следует существование такого
$x_1\in X$, что $f(x_1)=y$.

\begin{theorem}Если $f_1:X\to Y$ и $f_2:X\to Y$ --- начальные
вложения, то $f_1=f_2$.
\end{theorem}

\begin{proof} Если $f_1\ne f_2$, то существует наименьшее $x\in X$, для
которого $f_1(x)\ne f_2(x)$; пусть, например, $f_1(x)\ne f_2(x)$.
Тогда ни для какого $y$ не выполнено $f_2(y)=f_1(x)$, поскольку
при $y<x$ имеем $f_2(y)=f_1(y)<f_1(x)$, а при $y>x$ ---
$f_2(y)>f_2(x)>f_1(x)$. Потому $f_2$ вопреки предположению не
начально.\end{proof}

\begin{theorem}Для всяких вполне упорядоченных множеств $X$ и $Y$ существует
либо начальное вложение $X\to Y$, либо наоборот.
\end{theorem}

\begin{proof} Рассмотрим множество $A$ всех начальных вложений начальных
кусков $X$ (т.е.\,с каждой своей точкой, содержащих и все меньшие)
в $Y$.

\hfil$A=\{x\in X:~\exists f:[0,x]\to Y\}$, где $0=\min X$\hfil\\
Если $A\subsetneq X$, то рассмотрим элемент $b=\min X\setminus A$.
Для всех $x<b$ есть начальное вложение $[0,x]$ в $Y$, причём, как
показано выше, при расширении домена, одна продолжает другую.
Поэтому имеется начальное вложение $f:\left[0,b\right)\to Y$.
Доопределим $f$ в точке $b$ значением $\min Y\setminus
f\left[0,b\right)$ (если эта разность пуста, то очевидно $Y$
вкладывается в $X$) и получим противоречие. Если же $A=X$, то
сразу получаем вложение $X$ в $Y$.\end{proof}

\begin{theorem}[Цермело]\index{теорема!Цермело} Всякое
множество можно вполне упорядочить.
\end{theorem}

\begin{theorem}[Тихонов]\index{теорема!Тихонова} Произведение
$\prod\limits_{i\in I}X_i$ компактных (хаусдорфовых) пространств
$X_i$ является компактным (хаусдорфовым) пространством.
\end{theorem}

\vskip 2pt\emph{Вывод аксиомы выбора из леммы
Цорна.}\index{лемма!Цорна}\index{аксиома выбора} Пусть
$\{X_s\}_{s\in S}$
--- данное семейство непустых множеств. Рассмотрим множество пар
$(T,f)$, где $T\subset S$ и $f$ --- функция выбора для
$\{X_s\}_{s\in T}$. Эти пары естественно упорядочены отношением
$(U,f_U)\leqslant(V,f_V)$, при $U\subset V$ и $f_V|_U=f_U$.
Очевидно, выполнено условие леммы Цорна. Пусть $(T,f)$ ---
максимальный элемент. Если бы нашёлся $t\in S\setminus T$, то
дополнив $f$ в точке $t$ некоторым значением из $X_t$, получили бы
пару больше, чем максимальную.\end{proof}

\vskip 2pt\emph{Вывод аксиомы выбора из теоремы
Тихонова.}\index{аксиома выбора}\index{теорема!Тихонова} Пусть
$\{X_s\}_{s\in S}$
--- данное семейство. Возьмём семейство $Y_s=X_s\sqcup \{x_s\}$ с
топологией, в которой точка $x_s$ открыта, а также открыты все
коконечные множества. Заметим, что все $Y_s$ компактны, а значит и
произведение $\prod\limits_{s\in S}Y_s$ тоже компактно.

Множество точек, у которых на фиксированной координате $a$ всегда
$x_a$, а другие произвольные, открыто в $\prod\limits_{s\in S}
Y_s$. Соответственно, дополнение к нему замкнуто.

Пересечение конечного числа таких замкнутых дополнений непусто
(метод математической индукции мы можем применять по аксиоме
регулярности из ZF). Значит, из компактности, пересечение и всех
таких замкнутых множеств непусто. Но это пересечение как раз и
есть $\prod\limits_{s\in S} X_s$. $\square$

Множество $x$ называется {\em
транзитивным}\index{множество!транзитивное}, если выполнено

\hfil$\forall y,z~(z\in y\in x~\to~z\in x)$\hfil\\
{\em Ординалом}\index{ординал} называется транзитивное множество,
вполне упорядоченное отношением $\in$. Класс всех ординалов
обозначается через $\On$.

\vskip 2pt\noindent{\bf Принцип трансфинитной
индукции.}\index{принцип трансфинитной!индукции} {\sl Пусть
множество $X$ вполне упорядочено отношением $<$ и $\phi(x)$
--- некоторое свойство его элементов (предикат). Пусть также
известно, что $\phi(0)$ --- истинно. Тогда выполнено}

\hfil$\forall x,y~\big(y<x\to(\phi(y)\to\phi(x))\big)\to\forall
x\phi(x)$.\hfil\par Доказательство тривиально.

\vskip 2pt\noindent{\bf Принцип трансфинитной
рекурсии.}\index{принцип трансфинитной!рекурсии} {\sl Пусть
имеется множество $Y$, множество $X$ вполне упорядочено отношением
$<$ и задано отображение $h:X\times Y^{\left[0,x\right)}\to Y$.
Тогда существует единственное отображение $f$, удовлетворяющее
формуле}

\hfil$f(x)=h(x,f|_{\left[0,x\right)})$.\hfil

\begin{proof} Единственность следует из того, что в противном случае,
взяв за $x$ минимальный элемент, на котором отличаются функции
$f_1$ и $f_2$, можем записать
$f_1(x)=h(x,f_1|_{[0,x)})=h(x,f_2|_{[0,x)})=f_2(x)$.

Для доказательства существования, рассмотрим множество всех
начальных отрезков, на которых такая функция есть. Если они
покрывают $X$, то всё доказано. Иначе рассмотрим минимальный
элемент среди тех $x$, для которых на отрезке $[0,x]$ нет такой
функции. Но тогда можно доопределить $f(x)$ как
$h(x,f|_{\left[0,x\right)})$.\end{proof}

\begin{theorem}Всякое вполне упорядоченное множество изоморфно
единственному ординалу.
\end{theorem}

\begin{proof} Пусть $X$ данное множество. Построим функцию
$f(x)=\{f(y):~y<x\}=\Image f|_{[0,x)}$. В образе получился искомый
ординал. Единственность следует из транзитивности отношения
изоморфности.\end{proof}

\vskip 2pt{\em Вывод теоремы Цермело из аксиомы
выбора.}\index{аксиома выбора}\index{теорема!Цермело} Пусть $X$
--- данное множество. Построим функцию выбора для
$2^{X}\setminus\{\empty\}$ и положим $f(\empty)=x\in X$. Пусть
$\alpha$ --- наименьший ординал, больший порядковых типов тех
подмножеств $X$, которые допускают полный порядок. По
трансфинитной рекурсии определяем последовательность
$x_0,\dots,x_{\xi},\dots_{\xi<\alpha}$ по правилу

\hfil$x_{\xi}=f(X\setminus\bigcup\limits_{\zeta<\xi}\{x_\zeta\})$.\hfil\par
Если при всех $\xi<\alpha$ выполнено $x_{\xi}\ne x$, то
$x_{\xi}\in X\setminus\bigcup\limits_{\zeta<\xi}\{x_\zeta\}$, а
это значит, что нашлась последовательность типа $\alpha$ с
различными элементами из $X$ --- противоречие с выбором $\alpha$.
Значит имеем право взять наименьший ординал $\xi$, такой, что
$x_{\xi}=x$. Но тогда $X=\bigcup\limits_{\zeta<\xi}\{x_{\zeta}\}$,
что и требовалось. $\square$

\vskip 2pt{\em Вывод леммы Цорна из теоремы
Цермело.}\index{теорема!Цермело}\index{лемма!Цорна} Пусть $X$
--- данное множество с порядком $\prec$. По теореме Цермело
существует полный порядок $\leqslant$ на $X$:
$x_0,\dots,x_{\xi},\dots_{\xi<\alpha}$. По трансфинитной индукции
строим множества $A_0,\dots,A_{\xi},\dots_{\xi<\alpha+1}$, следуя
правилу
\begin{center}$A_0=\empty$,\\
$A_{\xi}=\left\{
\begin{array}{ll}
\bigcup\limits_{\zeta<\xi}A_{\zeta}\cup\{x_{\xi}\}, & \mbox{если }
\bigcup\limits_{\zeta<\xi}A_{\zeta}\cup\{x_{\xi}\} \mbox{ линейно упорядоченно,}\\
\bigcup\limits_{\zeta<\xi}A_{\zeta}, & \mbox{иначе,}
\end{array}
\right.$ \\ $A_{\alpha}=\bigcup\limits_{\xi<\alpha}A_{\xi}$.
\end{center}
В $A_{\alpha}$ находится, по условию максимальный элемент. Он, как
легко заметить, максимален во всём множестве. $\square$

Ординал $\alpha$ называется {\em кардиналом}\index{кардинал}, если
для любого ординала $\beta$ условие $\beta<\alpha$ равносильно
$|\beta|<|\alpha|$.

\begin{lemma}[Кёниг]\index{лемма!Кёнига}
У любого бесконечного дерева с конечным ветвлением существует
бесконечная ветвь.
\end{lemma}

\noindent{\bf Континуум-гипотеза.}\index{гипотеза!континуума}
$|\exp\aleph_{\alpha}|=\aleph_{\alpha+1}$.

\subsection{Абстрактная чепуха}

\begin{theorem}
Доказать, что в категории групп эпиморфизмы это в точности
сюръективные гомоморфизмы.
\end{theorem}

\begin{proof} Понятно, что все сюръективные гомоморфизмы эпиморфны. Для
доказательства обратного, предположим, что задан эпиморфизм
$\phi:G\rightarrow H$ и $\Image G\ne H$. Достаточно построить два
различных отображения $\alpha$ и $\beta$ из $H$ в какую-нибудь
группу, так, что $\phi\alpha=\phi\beta$, но $\alpha\ne\beta$. Если
$\Image G$ имеет в $H$ индекс 2, то в качестве таковых подойдут
тождественное и соответствующее факторотображение из $H$ в
$\Mz_2$. Если же $M=\Image G$ имеет в $H$ индекс больше двух, то
рассмотрим группу $S_H$ перестановок множества $H$. Пусть $\psi$
это гомоморфизм из $H$ в $S_H$, сопоставляющий элементу $h$ левый
сдвиг $\psi_h$ на $h$. Пусть $M,Mu,Mv$ --- разные смежные классы.
Определим $\sigma$ как перестановку, сопоставляющую элементу $xu$,
где $x\in M$, элемент $xv$ и наоборот, а всё остальное оставляющее
на месте. Теперь легко видеть, что $\alpha=\psi$ и
$\beta=\psi'=\sigma^{-1}\psi\sigma$ удовлетворяют нужному
условию.\end{proof}

\subsection{Фильтрованные произведения}

Если $\{A_i\}_{i\in I}$ --- семейство $T$-алгебр и $D$ --- фильтр
на $I$, то следующее отношение эквивалентности является
конгруэнцией: $a\sim b$, где $a,b\in\prod\limits_{i\in I}$ \ifif
$\{i\in I:~a_i=b_i\}\subset D$. Тогда факторалгебра
$\prod\limits_{i\in I}/\sim$ называется {\em
фильтрованным\index{произведение!фильтрованное} произведением} (а
когда $D$
--- ультрафильтр, то соответственно {\em ультрапроизведением}\index{ультрапроизведение}),
(см. например, \cite{Sk,Mal}).

\subsection{Модули}

\begin{lemma}Во всяком ЧУМ'е $A$ каждое непустое подмножество $X\subset
A$ содержит максимальный в $X$ элемент ($=A$ удовлетворяет условию
максимальности) \ifif для любой последовательности элементов $A$
вида $a_1\leq a_2\leq\dots\leq a_n\leq\dots$ существует число $k$,
начиная с которого цепочка стабилизируется ($=A$ удовлетворяет
условию обрыва возрастающих цепей).
\end{lemma}

Не будем останавливаться на её доказательстве. См. например,
\cite{Fa}.

Модуль $M_R$ называется {\em нётеровым}\index{модуль!нётеров},
если выполнено одно из следующих эквивалентных условий:

$(a)$ множество подмодулей модуля $M$, упорядоченное по включению,
удовлетворяет условию максимальности;

$(b)$ множество подмодулей модуля $M$, упорядоченное по включению,
удовлетворяет условию обрыва возрастающих цепей;

$(c)$ каждый подмодуль модуля $M$ конечно порождён.

$(a)\Rightarrow(c)$ Возьмём максимальный элемент $A$ в
совокупности всех конечно порождённых подмодулей в $S<M$. Если
$x\in S$, то $\langle x,A\rangle$ конечно порождён. Значит
$\langle A,x\rangle=A$, поэтому $A=S$, т.е.\,$S$ конечно порождён.

$(c)\Rightarrow(b)$ Пусть $S_1\subset S_2\subset\dots\subset
S_n\subset\dots$, где $S_i<M$, и
$S=\bigcup\limits_{i\in\mathbb{N}}$.
$S=\sum\limits_{i=1}^{n}x_iR$, где $x_i\in M$. Найдётся такое
число $k$, что $x_1,\dots,x_n\in S_k$. Но тогда $S=S_{k+j}$ для
всех $j\in\mathbb{N}$.

Пусть $R$ кольцо, $M$ правый $R$-модуль, а $\mathbb{N}$ будет до
конца параграфа обозначать множество неотрицательных целых чисел.
См. \cite{Fa}.

$M\langle x\rangle=\prod\limits_{\mathbb{N}}M$
--- формальные
 степенные ряды над $M$.
\\ $M[x]=\bigoplus\limits_{\mathbb{N}}M$ --- многочлены над $M$.
\\ $M[x]$ --- $R$-модуль: если $m(x)=\sum\limits_{i=0}^{n}a_ix^i\in M[x]$ и
$p(x)=\sum\limits_{j=0}^{q}r_jx^j\in R[x]$, то положим
$m(x)p(x)=\sum\limits_{k=0}^{n+q}\sum\limits_{i+j=k}a_ir^jx^k$.

Пусть $S$ --- $R$-модуль, и $S<M[x]$. Положим $L_n(S)$
--- $0$ и множество элементов $M$, встречающихся в качестве
старших коэффициентов в многочленах степени $n$ из $S$.
$L_n(S)_R<M_R$, $L_n(S)\subseteq L_{n+1}(S)$.

\begin{lemma}
Пусть $S_{R[x]},T_{R[x]}<M[x]$ и $S\subset T$. Если
$L_i(S)=L_i(T)$ при $i\in\mathbb{N}$, то $S=T$.
\end{lemma}

\begin{proof} Предположим обратное и возьмём многочлен вида
$f(x)=\sum\limits_{i=0}^{n}a_ix^i\in T\setminus S$ наименьшей
возможной степени. $a_n\in L_n(T)=L_n(S)$, поэтому существует
многочлен $g(x)\in S$ степени $n$ со старшим коэффициентом $a_n$.
$\deg h\leq n-1$, где $h(x)=f(x)-g(x)$, поэтому $h\in S$, но тогда
и $f=g+h\in S$.\end{proof}

\begin{theorem}[Гильберт]\index{теорема!Гильберта о базисе} $R$ --- кольцо, $M$ ---
нётеров $R$-модуль. Тогда $M[x]$ --- нётеров $R[x]$-модуль.
\end{theorem}

\begin{proof} Пусть $M_1\subset M_2\subset\dots\subset M_n\subset\dots$
--- последовательность $R[x]$-модулей. Рассмотрим семейство
$\{L_i(M_j):~i,j\in\mathbb{N}\}$. Понятно, что $L_0(M_j)\subset
L_0(M_j)\subset\dots\subset L_0(M_j)\subset\dots$ и т.к.
$M_j\subset M_m$ при $m\geq j$, то $L_i(M_j)\subset L_i(M_m)$.
Поэтому имеем $L_0(M_0)\subset L_1(M_1)\subset\dots\subset
L_n(M_n)\subset\dots$. $M$ --- нётерово, значит существует $q$,
для которого $L_q(M_q)=L_{q+1}(M_{q+1})=\dots$. Опять же ввиду
нётеровости для любого числа $i$ существует наименьшее число
$n(i)$, такое, что $L_i(M_{n(i)+k})=L_i(M_{n(i)})$ при всех
$k\in\mathbb{N}$. Понятно, что числа $n(i)$ ограниченны сверху
некоторым $n_0$. Тогда $L_i(M_j)=L_i(M_{n_0})$ для
$i\in\mathbb{N}$. Из (Л.4) вытекает, что при $j\geq n_0$
$M_j=M_{n_0}$. Но это и означает, что $M[x]$ нётеров
$R[x]$-модуль.\end{proof}

\subsection{Топология}
\begin{example} Пример линейно связного, но не локально линейно
связного топологического пространства. Рассмотрим на плоскости
семейство отрезков $\big[(0,0),(1,\frac{1}{n})\big]_{n\in\Mn}$. Их
объединение, очевидно линейно связно. Однако, рассмотрев
$O_{\frac{1}{2}}(0,0)$, убеждаемся, что пространство не локально
линейно связно.
\end{example}

Докажем теорему \ref{sub_fund_gr} в общем случае.

{\em Существование.} Будем строить $X$. Пусть $x_0\in B$, тогда в
качестве $X$ возьмём множество путей с началом в $x_0$ со
следующим отношением эквивалентности: $\upsilon$ и $\nu$ c
одинаковыми концами эквивалентны, если класс пути
$\upsilon\nu^{-1}$ принадлежит $H$. На множестве $X$ введём такую
топологию: рассмотрим точку $\gamma$ из $X$ --- пусть $x_1$
--- конец пути $\gamma$. Поскольку $B$ --- локально односвязно и
локально линейно связно, то находится локально связная односвязная
окрестность $U$ точки $x_1$. Окрестностью точки $\gamma$ назовём
все пути в $B$, являющиеся продолжениями пути $\gamma$ посредством
путей, лежащих в $U$. Понятно, что такое семейство путей не
зависит от выбора точки внутри $U$. Легко видеть, что построенные
окрестности образуют базу некоторой топологии (чтобы получить
пересечение построенных окрестностей в $X$, нужно взять
пересечение окрестностей в $B$, участвующих в построении). Чтобы
показать, что $X$ --- линейно связно достаточно заметить, что
любая точка $\gamma\in X$ соединяется с точкой $x_0\in X$,
являющейся в $B$ тривиальным путём. В пространстве $B$ выберем
путь $\vartheta(s)$, $s\in\Mi$, соответствующий классу $\gamma$.
Для всех $t\in\Mi$ возьмём путь $\vartheta(st)$. Это семейство
путей параметризованное по $s$ и будет путём, соединяющим $x$ с
$x_0$.

В качестве $p_H$ возьмём отображение, сопоставляющее точке
$\gamma\in X$ точку $b\in B$, являющуюся концом пути $\gamma$ в
$B$. $p_H$ сюръективно ввиду линейной связности $B$. Рассмотрим
прообраз при $p_H$ линейно связной односвязной окрестности $U$ в
$B$. Он, как легко видеть, есть объединение окрестностей из базы
$X$. Значит $p_H$ непрерывно. $p_H$ также локальный гомеоморфизм.
Теперь проверим, что $p_{H_*}(\pi_1(X))=H$.

{\em Единственность.}

\newpage
\addcontentsline{toc}{section}{Литература к дополнению}
\renewcommand{\refname}{Литература к дополнению}
\begin{thebibliography}{MacL}
\bibitem[Fa]{Fa}\litr{Фейс К.}{Алгебра: кольца, модули и категории. В 2-х тт}{М.:
Мир, 1977}
\bibitem[Sk]{Sk}\litr{Скорняков Л.А. и др.}{Общая алгебра. В
2-х тт}{М.: Наука, 1991}
\bibitem[Mal]{Mal}\litr{Мальцев А.И.}{Алгебраические системы}{М.: Наука,
1970}
\bibitem[MacL]{MacL}\litr{MacLane S.}{Categories for working
mathematician}{New York: Springer, 1971}\tran{Маклейн
С.}{Категории для работающего математика}{М.: Физматлит, 2004}
\bibitem[Eng]{Eng}\litr{Энгелькинг Р.}{Общая топология}{М.: Мир, 1986}
\bibitem[Kel]{Kel}\litr{Kelley J.L.}{The Tychonoff product theorem
implies the axiom of choice}{Fund. Math., 1950, v.37, p.75--76}
\end{thebibliography}

\newpage
\input Group.Theory.[S].A.A.Klyachko.ind

\newpage
\tableofcontents

\end{document}
