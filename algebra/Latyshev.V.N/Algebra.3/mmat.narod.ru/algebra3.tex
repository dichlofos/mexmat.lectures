\smallskip
\hrule \rule{0pt}{10pt}
\marginpar{13.10.03}
\te{Лемма}{Если $a, b\in\rr G,\ ab=ba$ и $\lob{a}_m\cap\lob{b}_n=\{e\},$ то $O(ab)=HOK(O(a),O(b))=HOK(m,n).$}

\dok Если $(ab)^s=e,$ то $a^sb^s=e\lra a^s=b^{-s}=e,$ так как $a$ и $b$ коммутируют. $a^s=b^{-s}=e\in \lob{a}_m\cap \lob{b}_n=\{e\}\ra m\,\vrule\,s$ и $n\,\vrule\,s\ra HOK(m,n)\,\vrule\,s\ra HOK(m,n)=O(ab).$\qquad\qed

\te{Следствие}{Если в условиях леммы $(m,n)=1,$ то $O(ab)=mn$.}\quad\qed
\te{Предложение}{Мультипликативная группа конечного поля является циклической.}

\dok Пусть $F_q$ --- конечное поле, $|F_q^*|=q-1.$ Все элементы $F_q^*$ имеют конечный порядок в этой группе. Пусть $a\in F_q^*$ имеет максимально возможный порядок $n=O(a).$
Покажем, что $F_q^*=\lob{a}_n.$ Предположим противное, что $\exists b\in F_q^*,\ b\notin\lob{a}_n.$
Пусть $n=p^sn',\ (p,n')=1$ (где $n$ --- порядок элемента $a$), и возьмем $m=p^tm',\ (p,m')=1$ --- порядок элемента $b$ (для любого простого $p$), и докажем, что $s\ge t.$
Предположим, что $s<t.$ Рассмотрим $\lob{a^{p^s}}_{n'}\cap\lob{b^{m'}}_{p^t}=\{e\},$ поскольку $(n', p^t)=1$ (по теореме Лагранжа порядок элемента из пересечения делит $n'$ и $p^t,$ то есть этот элемент --- единица).
По лемме, $O(a^{p^s}b^{m'})=p^tn'>p^sn'=n.$ Противоречие (так как $n$ --- максимально возможный порядок), значит $m\,\vrule\,n,$ так как $p$ было любым простым.

Рассмотрим теперь в $F_q$ уравнение $x^n-e=0.$ Получается, что оно имеет $\ge n+1$ корней в $F_q$: вся группа $\lob{a}_n$ и еще элемент $b$ удовлетворяют ему,
получаем противоречие (корней у многочлена больше, чем его степень?). Значит, $b\in\lob{a}_n.$\qquad\qed

\te{Теорема}{Над любым конечным полем существуют неприводимые многочлены любой наперед заданной степени.}

\dok Рассмотрим $F_q$, $q=p^m,$ где $p=\rr{char\,}F_q.$ Рассмотрим поле $F_{q^n},$ то есть $q^n=p^{mn},\ p=\rr{char\,}F_{q^m},$ мы знаем о существовании такого поля.
Так как $m\,\vrule\,mn$, то существует инъекция $F_q\hookrightarrow F_{q^n},$ и $\dim_{F_q}{F_{q^n}}=n.$ Только что было доказано, что $F^*_{q^n}=\lob{\theta}_{q^n-1},$ и
можно мыслить, что элементы $F_{q^n}$ --- линейные комбинации степеней $\theta$ с коэффициентами из $F_q$.
Рассмотрим эпиморфизм $\phi : F_q[x]\rightarrow F_{q^n},$ при котором $f(x)\mapsto f(\theta)$ (так как линейные комбинации степеней $\theta$ исчерпывают все $F_{q^n}$, то это сюръективный гомоморфизм).
Пусть $I=\rr{Ker}\phi=(d(x)) \lhd F_{q}[x]$. Значит, по теореме о гомоморфизмах колец, $F_{q^n}\cong F_q[x]/(d(x)).$ Так как $F_{q^n}$ --- поле, что $d(x)$ неприводим.
Так как $\dim_{F_q}{F_{q^n}}=n,$ то $\deg f(x)=n.$ Значит, для всякого $n$ над $F_q$ существует неприводимый многочлен этой степени.\qquad\qed
\smallskip

\centerline{\bf Прямые произведения групп.}

Начнем с внутреннего подхода.

\de Пусть $A_1,\dots,A_m\subset\rr G$ --- подмножества. Подмножество $A_1\dots A_m=\{a_1\dots a_m\,\vrule\, a_i\in A_i\}$ называется их произведением.

Пусть $A,B\subseteq\rr G$ --- подмножества.
\begin{enumerate}
    \item Они коммутируют как подмножества, если $AB=BA$ (это равенство понимается в смысле равенства множеств).
    \item Они коммутируют поэлементно, если $\forall a\in A\ \forall b\in B\ \ ab=ba.$
\end{enumerate}
Понятно, что второе условие сильнее первого.

Если взять произвольные подгруппы $A, B\subset\rr G$, то вообще говоря, $AB$ --- не подгруппа, и $AB\ne BA$. Но если $B\lhd\rr G$, то $AB=BA$.
Действительно, рассмотрим $a\in A,\ b\in B.$ Тогда $AB\ni ab=(aba^{-1})a\in BA,$ так как $aba^{-1}\in B\lhd\rr G,\ a\in A.$
Значит, $AB\subseteq BA,$ аналогично $BA\subseteq AB,$ значит, $AB=BA$. Но будет ли $AB$ подгруппой?

\te{Предложение}{Если подгруппы $H_i\in\rr G,\ i=q,\dots, m$ попарно коммутируют, то есть $H_iH_j=H_jH_i\ \forall i,j,$ то их произведение является подгруппой, не
зависящей от порядка сомножителей. Если, кроме того, ''$H_i$'' коммутируют попарно поэлементно, то $H_i\lhd H\ \forall i=1,\dots,m$.}

\dok Докажем сначала, что $H$ --- подгруппа, индукцией по $m$.
\\ Основание $m=2$. Пусть $A,B\in\rr G$ --- подгруппы, пусть $a_1b_1,a_2b_2\in AB,$ где $a_1,a_2\in A,\ b_1,b_2\in B.$
Посмотрим, где лежит $(a_1b_1)^{-1}(a_2b_2)=b_1^{-1}(a_1^{-1}a_2)b_2.$ Но $b_1^{-1}\in B,\ (a_1^{-1}a_2)\in A,$ значит, так как подгруппы коммутируют, то
$(a_1b_1)^{-1}(a_2b_2)=a_3b_3b_2=a_3(b_3b_2)\in AB,$ для некоторых $a_3\in A,\ b_3\in B.$ Значит, $AB$ --- подгруппа.
Если $B\lhd\rr G,$ то $AB=BA$ --- подгруппа в $\rr G.$
\\Шаг индукции. Пусть $m\ge 3,$ для меньших количеств сомножителей, чем $m$, теорема верна.
Рассмотрим $H=(H_1\dots H_{m-1})H_m=LH_m.$ Но так как подгруппы ''$H_i$'' коммутируют, то $H_mL=LH_m,$ значит, $H$ --- подгруппа по базе индукции.

Пусть теперь подгруппы коммутируют поэлементно. Докажем, что каждая из них нормальна в произведении.
Рассмотрим $H\ni h=h_1\dots h_i\dots h_m,\ h_j\in H_j\ \forall j.$ Далее ''$\widehat{h_i}$'' будет означать пропуск элемента $h_i$. Рассмотрим
$$
hH_i=(h_1\dots \widehat{h_i}\dots h_m)H_i=(h_1\dots \widehat{h_i}\dots h_m)(h_iH_i)=H_ih_i(h_1\dots \widehat{h_i}\dots h_m)=H_i(h_1\dots h_i\dots h_m)=H_ih,
$$
так как подгруппы коммутируют поэлементно и $H_ih_i=h_iH_i$.
Значит, $H_i\lhd H,\ \ \forall i=1,\dots, m.$\quad\qed

\te{Следствие}{Если $H_i\lhd\rr G,\ i=1,\dots,m,$ то $H=H_1\dots H_m\lhd\rr G,$ причем эта подгруппа не зависит от порядка сомножителей.}

\dok По предложению подгруппы коммутируют, значит, их произведение --- подгруппа в $\rr G.$ Возьмем $g\in\rr G$ и рассмотрим
$$
gHg^{-1}=g(H_1\dots H_m)g^{-1}=(gH_1g^{-1})\dots(gH_mg^{-1})=H_1\dots H_m\lhd G.\mbox{\qquad\qed}
$$

\de Произведение подгрупп $H_i\subseteq \rr G, i=1,\dots,m$ называется прямым, если:
\begin{enumerate}
    \item $h_ih_j=h_jh_i\ i\ne j$ (подгруппы коммутируют поэлементно);
    \item $\forall h\in H=H_1\dots H_m$ $\exists !$ представление вида $h=h_1\dots h_m,$ где $h_i\in H_i,\ i=1,\dots,m.$
\end{enumerate}
По предложению и следствию из него, $H\subseteq\rr G$ --- подгруппа, и каждое из сомножителей нормально в произведении.
Обозначается прямое произведение в мультипликативной терминологии так: $H=H_1\dot{\times}\dots\pr H_m,$ а в аддитивной терминологии это называется прямой суммой и
обозначается $H=H_1\oplus\dots\oplus H_m.$ Если $H=(\underbrace{A_1\pr\dots\pr A_r}_A)(\underbrace{B_1\pr\dots\pr B_s}_B),$ то $H=A\pr B,$ что вытекает из определения.
Это дает пример ''укрупнения'' или ''измельчения'' прямых множителей (в зависимости от того, в какую сторону читать).

\te{Теорема}{Пусть $H_i\lhd\rr G,\ i=1,\dots,m.$ Тогда $$H=H_1\pr\dots\pr H_m\lra\forall i=1,\dots,m\ \  H_i\cap(H_1\dots H_{i-1}H_{i+1}\dots H_m)=\{e\}.$$}

\dok Пусть $H=H_1\pr\dots\pr H_m.$ Тогда Рассмотрим $x\in H_i\cap(H_1\dots\widehat{H_i}\dots H_m).$ Тогда $x=ee\dots exe\dots e=h_1\dots h_{i-1}eh_{i+1}\dots h_m,$
где в первой записи $x$ стоит на $i-$м месте так как $x\in H_i$, а во второй записи на том же месте стоит единица. В силу единственности представления $x=e.$

Обратно, пусть $H_i\cap(H_1\dots\widehat{H_i}\dots H_m)=\{e\}\ \forall i.$ Тогда так как коммутатор $[h_i,h_j]=(h_ih_jh_i^{-1})h_j\in H_j$ в силу нормальности $H_j$, и аналогично $[h_i,h_j]\in H_i$,
то $[h_i,h_j]\in H_i\cap H_j=\{e\},$ потому что $\{e\}=H_i\cap(H_1\dots\widehat{H_i}\dots H_m)\supset H_i\cap H_j.$ Поэтому $h_ih_j=h_jh_i,$ то есть
наши подгруппы коммутируют поэлементно. Докажем теперь однозначность. Пусть $x\in H,\ x=h_1\dots h_i\dots h_m=h_1'\dots h_i'\dots h_m'.$
Тогда, пользуясь тем, что подгруппы попарно коммутируют, запишем: $$H_i\ni h_ih_i'=(h_1h_1')\dots(h_{i-1}h_{i-1}')(h_{i+1}h_{i+1}')\dots(h_mh_m')\in H_1\dots{H_{i-1}}{H_{i+1}}\dots H_m.$$
Поэтому $h_ih_i'=e\ \ \forall i\ra $ представление каждого элемента группы $H$ в виде произведения элементов из подгрупп единственно.\qquad\qed

\smallskip
\hrule \rule{0pt}{10pt}
\marginpar{20.10.03}

Переходим к ''внешнему'' определению прямого произведения групп.

Пусть $G_1,\dots, G_m$ --- группы. Рассмотрим $G=G_1\times\dots\times G_m=\{(g_1,\dots,g_m)\,\vrule\,g_i\in G_i\}$ --- декартово произведение групп ''$ G_i$'' как множеств.
Введем на декартовом произведении покомпонентную операцию: пусть $g=(g_1,\dots,g_m),\ g'=(g_1',\dots,g_m'),$ тогда $gg':=(g_1g_1',\dots,g_mg_m').$
Относительно этой операции $G$ --- группа. Действительно, покомпонентная операция ассоциативна, так как она ассоциативна по каждой компоненте,
есть единица --- это строка из единиц каждой группы $e=(e_1,\dots,e_m),$ также ко всякому элементу существует обратный --- это строка обратных по каждой компоненте $(g_1,\dots,g_m)^{-1}=(g_1^{-1},\dots,g_m^{-1}).$

Группа $G$ называется прямым произведением групп $G_1,\dots,G_m$, и обозначается также, как и во внутреннем определении: в мультипликативной терминологии $G=G_1\pr\dots\pr G_m,$  в
аддитивной терминологии (это называется прямой суммой) $G=G_1\oplus\dots\oplus G_m.$

Обозначим
$\wt{G_i}=\{\wt{g_i}=(e_1,\dots,e_{i-1},g_i,e_{i+1},\dots,e_m)\,\vrule\,
g_i\in G_i\}.$ Ясно, что $\wt{G_i}\subset G$ --- подгруппа, и что
$G_i\cong \wt{G_i}$, существует канонический изоморфизм
$G_i\rightarrow \wt{G_i},\ g_i\mapsto \wt{g_i}$. Если мысленно
отождествить $g_i\equiv \wt{g_i}$ согласно этому изоморфизму, то
получим, что $G_i\subset G.$ Понятно, что
$G=\wt{G_1}\pr\dots\pr\wt{G_m}$ во внутреннем смысле, то есть как
прямое произведение подгрупп (действительно, все элементы из
разных подгрупп коммутируют, любая строка представляется
единственным образом в виде
$g=(g_1,\dots,g_n)=(g_1,e_2,\dots,e_m)\dots(e_1,\dots,e_{m-1},g_m)=\wt{g_1}\dots\wt{g_m}$).

\ste{Теорема}{''о факторизации по прямым множителям (слагаемым)''}{Пусть есть группа $G=G_1\pr\dots\pr G_m,$\ $H_i\lhd G_i,\ H=H_1\pr\dots\pr H_m$ (это произведение автоматически прямое, так как
множители ''$H_i$'' взяты из прямых сомножителей ''$G_i$''). Тогда $G/H=G_1/H_1\pr\dots\pr G_m/H_m$}.

\dok Рассмотрим $\phi : G\rightarrow G_1/H_1\pr\dots\pr G_m/H_m$, при котором $\phi : g=(g_1,\dots,g_m)\mapsto (g_1H_1,\dots,g_mH_m).$
Отображение $\phi,$ очевидно, является гомоморфизмом: $$\phi(gg')=(g_1g_1'H_1,\dots,g_mg_m'H_m)=(g_1H_1,\dots,g_mH_m)\cdot(g_1'H_1,\dots,g_m'H_m)=\phi(g)\phi(g').$$
Из определения $\phi$ видно, что он сюръективен: на всякую строку смежных классов отразилась строка их представителей.

Посмотрим теперь ядро этого отображения. $g\in\rr{Ker}\phi\lra \phi(g)=(g_1H_1,\dots,g_mH_m)=(H_1,\dots,H_m)$ --- единица произведения факторгрупп, а это равносильно тому, что
$g_iH_i=H_i\ \ \forall i=1,\dots,m\lra g_i\in H_i\ \ \forall i\lra g\in H.$ Значит, $\rr{Ker}\phi=H,$ и по теореме о гомоморфизмах групп, все доказано, то есть
$H\lhd G$ и $G/H=G_1/H_1\pr\dots\pr G_m/H_m$.\qquad\qed

\bigskip
\noindent\verb"--------------------------конец материалов к коллоквиуму-------------------------------"
