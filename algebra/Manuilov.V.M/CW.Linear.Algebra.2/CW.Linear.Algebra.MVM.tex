\documentclass[a4paper]{article}
\usepackage[simple,utf]{dmvn}

\newcommand{\PM}{\phantom{-}}

\begin{document}
\section*{Контрольная работа по линейной алгебре}
\subsection*{2003 год. II семестр. Преподаватель В.\,М.\,Мануйлов}

\begin{problem}
Канонизировать матрицу $A$ и найти ортогональную матрицу $Q$, такую, что $B=Q^TAQ$.

\eqn{A=\rbmat{
\PM 0 &    -2 & \PM 4 & -2 \\
\PM 2 & \PM 0 &    -2 & -4 \\
   -4 & \PM 2 & \PM 0 & \PM 1 \\
\PM 2 & \PM 4 &    -1 & \PM 0
}}

\end{problem}

\begin{solution}
Характеристический многочлен $\det(A-\la E) = \la^4+45\la^2+324$ имеет корни $\hc{\pm 6i, \pm 3i}$.
Они попарно сопряженные, поэтому рассмотрим только корни $\la_1=6i, \la_2=3i$.

\eqn{\Ker(A_\Cbb-\la_1E) = \ha{\hr{-1+\frac{1}{2}i, \frac{1}{2}+i, -i, 1}}, \quad
\Ker(A_\Cbb-\la_2E) = \ha{\hr{-i, 1, 1-\frac{1}{2}i, -\frac{1}{2}-i}}.}

Базис образуют вектора из мнимых и действительных частей базисных векторов ядер:
\eqn{\hr{-2, 1, 0, 2},\quad \hr{1, 2, -2, 0}, \quad \hr{0, 2, 2, -1},\quad \hr{-2, 0, -1, -2}.}
Нормируя вектора, получаем ответ:

\eqn{Q=\rbmat{-2/3 & 1/3 & 0 & -2/3 \\
1/3 & 2/3 & 2/3 & 0 \\
0 & -2/3 & 2/3 & -1/3 \\
2/3 & 0 & -1/3 & -2/3
}, \quad
B=\rbmat{0 & 6 & 0 & 0 \\
-6 & 0 & 0 & 0 \\
0 & 0 & 0 & 3 \\
0 & 0 & -3 & 0
}.}

\end{solution}

\begin{problem}
Привести две билинейные формы к общему каноническому базису:

$$
F=\begin{pmatrix}
4 & -8 & 2 & 2 \\
-8 & 16 & -4 & -4 \\
2 & -4 & 1 & 1 \\
2 & -4 & 1 & 16
\end{pmatrix},~~
G=\begin{pmatrix}
1 & -1 & 0 & 0 \\
-1 & 2 & -1 & 0 \\
0 & -1 & 2 & -1 \\
0 & 0 & -1 & 2
\end{pmatrix}.
$$
\end{problem}

\begin{solution}
Положительно определенной является матрица $G$. Обобщенный характеристический многочлен $\det(F-\lambda G)=\lambda^4+18\lambda^3+81\lambda^2$ имеет корни: $\{0, 0, 9, 9\}$. Найдем базис из собственных векторов:

\eqn{\Ker(F-0G) = \ba{\hr{1,0,-2,0}, \hr{0,1,4,0}}, \quad \Ker(F-9G) = \ba{\hr{0,-2,0,1}, \hr{1,3,1,0}}.}

Ортогонализуем полученный базис относительно матрицы $G$:

$$f_1=e_1, ~~~ f_2=e_2-\frac{\left(e_2,e_1\right)_G}{\left(e_1,e_1\right)_G}e_1, ~~~ f_3=e_3, ~~~ f_4=e_4-\frac{\left(e_4,e_3\right)_G}{\left(e_3,e_3\right)_G}e_3$$

Получим следующую систему векторов:

$$f_1=(1,0,-2,0), ~~~ f_2=\left(\frac{5}{3},1,\frac{2}{3},0\right), ~~~ f_3=\left(0,-2,0,1\right), ~~~ f_4=\left(1,\frac{6}{5},1,\frac{9}{10}\right)$$

Осталось пронормировать этот базис по матрице $G$:

$$\left|f_1\right|_G=3, ~~~ \left|f_2\right|_G=1, ~~~ \left|f_3\right|_G=\sqrt{10}, ~~~ \left|f_4\right|_G=\frac{3}{\sqrt{10}}$$

Выпишем матрицу перехода:
$$C=\begin{pmatrix}
\frac{1}{3}         & \frac{5}{3} & 0                       & \frac{\sqrt{10}}{3} \\
& & & \\
0                   & 1           & -\frac{\sqrt{2}}{\sqrt{5}}   & \frac{2\sqrt{2}}{\sqrt{5}} \\
& & & \\
-\frac{2}{3}         & \frac{2}{3} & 0                       & \frac{\sqrt{10}}{3} \\
& & & \\
0                   & 0           & \frac{1}{\sqrt{10}}  & \frac{3}{\sqrt{10}}
\end{pmatrix}
$$

Канонический вид матриц: $$F^\prime=\diag(0,0,9,9), ~ G^\prime=E.$$
\end{solution}

\end{document}
