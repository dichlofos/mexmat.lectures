\documentclass[a4paper]{article}
\usepackage[simple]{dmvn}

\title{Материалы с контрольных и зачетов по линейной алгебре}
\author{Преподаватель\т Иван Алексеевич Дынников}
\date{2 семестр. 2005 г.}

\begin{document}
\maketitle
\centerline{\small Набрано П. Рахмоновым, отредактировано и свёрстано DMVN Corporation.}

\medskip
\dmvntrail

\section{Контрольная 15 марта 2005 г.}

1. Найти базис суммы $U+V$ и пересечения $U\cap V$ пространств $U$
и $V$, если $U=\langle(1, 1, -1, 2), (1, -1, 7, 0)\rangle$,
$V=\langle(2, -1, 1, 4), (1, 1, 8, -1)\rangle$.

2. В пространстве, двойственном пространству симметричных матриц
размера $2\times 2$ дан базис:
$$
e^1=\text{tr}(AX), e^2=\text{tr}(BX), e^3=\text{tr}(CX)
$$
Найти координаты $\text{tr} X$, если

$$
A= \left(
\begin{array}{cc}
1 & 1\\
0 & 0
\end{array} \right) ,\ B= \left(
\begin{array}{cc}
0 & 1\\
0 & 0
\end{array} \right),\ C=\left(
\begin{array}{cc}
0 & 0\\
1 & -1
\end{array} \right),
$$


3. Найти матрицу оператора проектирования пространства
$U=\langle(1, -1, 1, 1)$, $(1, 0, 0, 1)\rangle$ параллельно
пространству $V=\langle(1, 1, 0, -1), (0, 0, 0, 1)\rangle$ в
стандартном базисе $\mathbb{R}^4$

4. Найти $\cos(\frac{\pi}{2}A)$, если $$ A= \left(
\begin{array}{ccc}
4 & 3 & 1\\
0 & 1 & -1 \\
1 & 2 & 3
\end{array} \right)  $$

\section{Контрольная 28 апреля 2005 г.}

\subsection{Вариант 1}

1. Найти расстояние между прямой $l:\ (9, -2, -1, 1)+\langle(2,
-2, -1, -1)\rangle$ и подпространством $R$, где
$$
R:\ \left\{
\begin{array}{lcr}
2x+4y+z+t  & = & 8 \\
2x+7y+4z-2t & = & 29

\end{array} \right.
$$

2. Найти канонический вид и соответствующий ортонормированный
базис ортогонального оператора, заданного в стандартном базисе
$\mathbb{R}^4$ матрицей:

$$ A= \frac{1}{9}\left(
\begin{array}{rrr}
4 & 7 & -4 \\
1 & 4 & 8 \\
8 & -4 & 1
\end{array} \right)  $$

3. Представить матрицу $\mathcal{A}$ линейного оператора в виде
суммы $\mathcal{B+C}$ симметрического (самосопряженного) оператора
$\mathcal{B}$ и кососимметрического оператора $\mathcal{C}$, если
данный оператор и скалярное произведении имеют матрицы:

$$ A= \left(
\begin{array}{rrr}
0 & 0 & 0 \\
0 & 0 & 0  \\
0 & 2 & 0
\end{array} \right)  ,\qquad G=
\left(
\begin{array}{rrr}
3&  -1 & 1\\
-1 & 1 & 0 \\
1 & 0 & 1
\end{array} \right)  $$


4. Представить оператор $\mathcal{A}$, заданный своей матрицей в
некотором ортонормированном базисе в виде произведения
$\mathcal{A=BU}$ положительного самосопряженного $\mathcal{B}$ и
ортогонального $\mathcal{U}$ операторов, если

$$ A= \left(
\begin{array}{rrr}
\frac{10}{3} & \frac{2}{3} & -\frac{2}{3}\\
\frac{17}{3} & \frac{10}{3} & -\frac{4}{3} \\
6 & 3 & 0
\end{array} \right)  $$

\subsection{Вариант 2}

1. Найти расстояние между прямой $l:\ (2, 4, 0, 14)+\langle(0, 1,
-2, 5)\rangle$ и подпространством $R$, где
$$
R:\ \left\{
\begin{array}{lcr}
2x-2y+z+t & = & 9 \\
4x+2y+3z+t & = & 17

\end{array} \right.
$$

2. Найти канонический вид и соответствующий ортонормированный
базис кососимметрического оператора, заданного в стандартном
базисе $\mathbb{R}^4$ матрицей:

$$ A= \left(
\begin{array}{cccc}
0 & -3 & 4 & 4\\
3 & 0 & 2 & 2 \\
-4 & -2 & 0 & 0 \\
-4 & -2 & 0 & 0
\end{array} \right)  $$

3. Представить матрицу $\mathcal{A}$ линейного оператора в виде
суммы $\mathcal{B+C}$ симметрического (самосопряженного) оператора
$\mathcal{B}$ и кососимметрического оператора $\mathcal{C}$, если
данный оператор и скалярное произведении имеют матрицы:

$$ A= \left(
\begin{array}{rrr}
0 & 0 & 0 \\
0 & 0 & 0  \\
0 & 0 & 2
\end{array} \right)  ,\qquad G=
\left(
\begin{array}{rrr}
2& 1 & -1\\
1 & 2 & -1 \\
-1 & -1 & 1
\end{array} \right)  $$


4. Представить оператор $\mathcal{A}$, заданный своей матрицей в
некотором ортонормированном базисе в виде произведения
$\mathcal{A=BU}$ положительного самосопряженного $\mathcal{B}$ и
ортогонального $\mathcal{U}$ операторов, если

$$ A= \left(
\begin{array}{rrr}
-\frac{14}{9} & -\frac{8}{9} & -\frac{19}{9}\\
-\frac{11}{9} & \frac{4}{9} & -\frac{22}{9} \\
-\frac{2}{9} & \frac{4}{9} & \frac{14}{9}
\end{array} \right)  $$

\subsection{Вариант 5 (задания 1, 4 отсутствуют)}

2. Найти канонический вид и соответствующий ортонормированный
базис ортогонального оператора, заданного в стандартном базисе
$\mathbb{R}^4$ матрицей:

$$ A= \frac{1}{9}\left(
\begin{array}{rrr}
-4 & 7 & -4 \\
1 & -4 & -8 \\
8 & 4 & -1
\end{array} \right)  $$

3. Представить матрицу $\mathcal{A}$ линейного оператора в виде
суммы $\mathcal{B+C}$ симметрического (самосопряженного) оператора
$\mathcal{B}$ и кососимметрического оператора $\mathcal{C}$, если
данный оператор и скалярное произведении имеют матрицы:

$$ A= \left(
\begin{array}{rrr}
2 & 0 & 0 \\
0 & 0 & 0  \\
0 & 0 & 0
\end{array} \right)  ,\qquad G=
\left(
\begin{array}{rrr}
2& -1 & 0\\
-1 & 5 & 2 \\
0 & 2 & 1
\end{array} \right)  $$

\section{Самостоятельная работа \No 1}

\subsection{Вариант 7.2}

Найти жорданову нормальную форму матрицы $A$ и жорданово
разложение матрицы $A$ (т.е. представить матрицу $A$ виде
$A=\mathcal{N} + \mathcal{P}$, где $\mathcal{N}$
--- нильпотентная, $\mathcal{P}$ --- полупростой операторы),
вычислить $e^{-A}$, если
$$
A= \left(
\begin{array}{rrrrr}
-9 & 12 & 4 & -9 & 12 \\
-1 & 0 & 1 & -1 & 2 \\
-2 & 3 & 0 & -3 & 3 \\
3  & -5&-2 &  2 & -5 \\
0 & 0&  -1 & 0  &-2
\end{array} \right)
$$


\section{Самостоятельная работа \No 2}

Доказать, что для любого $n\in \mathbb{N}$ существует такой
многочлен $P_n(x_1, \ldots, x_n)$, что для всех матриц порядка $n$
имеет место равенство $\det A=P_n(\tr A, \tr A^2,\ldots, \tr
A^n)$. Найти $P_2$, $P_3$.

\section{Контрольная 13 мая 2005г.}

\subsection{Вариант 1}

1. Привести кососимметрическую билинейную функцию $f(x,
y)=(x_1y_2-x_2y_1)+(x_1y_3-x_3y_1)-2(x_1y_4-x_4y_1)-(x_2y_3-x_3y_2)+3(x_2y_4-x_4y_2)+(x_3y_4-x_4y_3)$
к каноническому виду методом Лагранжа.

2. Для данной пары квадратичных функций $f_1(x, y)=8x^2-12xy+5y^2$
и $f_2(x, y)=4x^2-4xy+5y^2$ выяснить, какая из них является
положительно определенной, и найти базис, в котором эта функция
приводится к нормальному, а другая
--- к каноническому виду, найти матрицы квадратичных функций в
этом базисе.

\subsection{Вариант 3}

1. Привести кососимметрическую билинейную функцию $f(x,
y)=(x_1y_2-x_2y_1)-(x_1y_3-x_3y_1)-(x_1y_4-x_4y_1)+3(x_2y_3-x_3y_2)+2(x_2y_4-x_4y_2)-4(x_3y_4-x_4y_3)$
к каноническому виду методом Лагранжа.

2. Для данной пары квадратичных функций $f_1(x, y)=4x^2+4xy-y^2$ и
$f_2(x, y)=8x^2-20xy+13y^2$ выяснить, какая из них является
положительно определенной, и найти базис, в котором эта функция
приводится к нормальному, а другая
--- к каноническому виду, найти матрицы квадратичных функций в
этом базисе.

\section{Зачёт \No 1}

\subsection{Вариант 1.4}

1. Найти расстояние от матрицы $A$ до подпространства
симметрических матриц в мет\-ри\-ке $(X, Y)=\tr (XGY^\top)$, если
$$
G=\left(\begin{array}{rr} 3 & 0\\
0 & 1
\end{array}\right) ,\qquad
A=\left(
\begin{array}{rr} 1 & -2\\
2 & -1
\end{array}\right)
$$

2. Найти все гиперплоскости, инвариантные относительно оператора,
заданного матрицей
$$
\left(
\begin{array}{rrrr}
0 & -1& -1 & 2\\
0 & 1 & 0 & 1\\
-1 & -1 & 0 &3 \\
-1 & -1 & -1 &4
\end{array}\right )
$$

3. Для ортогональной матрицы
$$
A=\frac{1}{33} \left(
\begin{array}{rrr}
32 & -7 & 4\\
4 & 28 & 17 \\
-7 & -16 & 28
\end{array}\right)
$$
решить уравнение $\exp X=A$. ({\it Указание:} использовать
канонический вид ортогонального оператора, найти линейные
соотношения между $A$ и $A^{-1}$  и искомой матрицей).

\subsubsection{Вариант 1.5}

1. Найти расстояние от матрицы $A$ до подпространства
верхнетреугольных матриц в метрике $(X, Y)=\tr (XGY^\top)$, если
$$
G=\left(\begin{array}{rr} 3 & 1\\
1 & 1
\end{array}\right) ,\qquad
A=\left(
\begin{array}{rr} 1 & 2\\
2 & 1
\end{array}\right)
$$

2. Найти все гиперплоскости, инвариантные относительно оператора,
заданного матрицей
$$
\left(
\begin{array}{rrrr}
0 & 1& 1 & -2\\
-2 & 1 & 2 & -4\\
-1 & -1 & 2 &-2 \\
-1 & -1 & 1 &-1
\end{array}\right )
$$

3. Для ортогональной матрицы
$$
A=\frac{1}{11} \left(
\begin{array}{rrr}
7 & 6 & 6\\
-6 & 9 & -2 \\
-6 & -2 & 9
\end{array}\right)
$$
решить уравнение $\exp X=A$. ({\it Указание:} использовать
канонический вид ортогонального оператора, найти линейные
соотношения между $A$ и $A^{-1}$  и искомой матрицей).

\subsection{Вариант 1.6}

1. Найти расстояние от матрицы $A$ до подпространства
нижнетреугольных матриц в метрике $(X, Y)=\tr (XGY^\top)$, если
$$
G=\left(\begin{array}{rr} 2 & 1\\
1 & 2
\end{array}\right) ,\qquad
A=\left(
\begin{array}{rr} 1 & 2\\
1 & 1
\end{array}\right)
$$

2. Найти все гиперплоскости, инвариантные относительно оператора,
заданного матрицей
$$
\left(
\begin{array}{rrrr}
0 & -2& 2 & -4\\
0 & 2 & 3 & 0\\
0 & 0 & -1 &0 \\
1 & 1 & -4 &4
\end{array}\right )
$$

3. Для ортогональной матрицы
$$
A=\frac{1}{25} \left(
\begin{array}{rrr}
16 & 15 & 12\\
-12 & 20 & -9 \\
-15 & 0 & 20
\end{array}\right)
$$
решить уравнение $\exp X=A$. ({\it Указание:} использовать
канонический вид ортогонального оператора, найти линейные
соотношения между $A$ и $A^{-1}$  и искомой матрицей).

\section{Зачёт \No 2}

\subsection{Вариант 6.2}

1. Задать системой уравнений аффинную оболочку многочленов $1-x$,
$x^2$, $x^3$ в пространстве $\mathbb{R}_4[x]$, взяв за координаты
многочлена $P$ набор $(P(0)$, $P(1)$, $P(2)$, $P'(0)$, $P'(-1))$.

2. Найти жорданову форму и жорданов базис для оператора
$\mathcal{A}$ в пространстве матриц размера $2\times 2$
действующей по формуле $\mathcal{A}: X \longmapsto LXM$, где
$L=\left(\begin{array}{rr} 7 & 2 \\ 2 & 10 \end{array}\right)$,
$M=\left(\begin{array}{rr} 1 & 4 \\ -1 & 5 \end{array}\right)$.
({\it Указание:}\ использовать жорданову форму матриц $L$ и $M$).

3. Используя метод Лагранжа, представить матрицу $\left(
\begin{array}{rrr}
1 & -1 & 3 \\
-1 & 10 & -12 \\
3 & -12 & 19
\end{array}
\right)$, в виде произведения $R^\top R$, где $R$ ---
верхнетреугольная матрица с положительными числами на диагонали.

\section{Зачёт \No 3}

\subsection{Вариант 7.1}

1. Используя процесс ортогонализации Грама\ч Шмидта, представить
мат\-ри\-цу $\left(
\begin{array}{rrr}
-3 & 0 & -1 \\
-2 & -1 & 1 \\
6 & 2 & 1
\end{array}
\right) $ в виде произведения $QR$ ортогональной матрицы $Q$ на
верхнетреугольную матрицу $R$ с положительными числами на
диагонали.

2. Представить матрицу $\displaystyle\frac{1}{3}\left(
\begin{array}{rrr}
1 & 0 & 2 \\
-1 & 0 & 2 \\
2 & 0 & 1
\end{array}\right)
$ в виде произведения симметрической неотрицательной матрицы на
ортогональную.

$$
\textit{Ответ:\ } S=\displaystyle\frac{1}{15\sqrt{10}} \left(
\begin{array}{ccc}
25 & 15 & 20 \\
15 & 9+16\sqrt 2 & 12-12\sqrt 2 \\
20 & 12-12\sqrt 2 & 16+9\sqrt 2
\end{array}
\right), Q=\displaystyle\frac{1}{5\sqrt{10}}\left(
\begin{array}{ccc}
5 & -5\sqrt 5 & 10 \\
3-8\sqrt 2 & 3\sqrt 5 & 6+4\sqrt 2 \\
4+6\sqrt 2 & 4\sqrt 5 & 8-3\sqrt 2
\end{array}
\right)
$$

3. Оператор $\mathcal{A}$ действует в пространстве
$\mathbb{C}_3[x]$ следующим образом: для многочлена
$P\in\mathbb{C}_3[x]$ результат $\mathcal{A}P$ применения
оператора $\mathcal{A}$ есть остаток от деления многочлена $xP(x)$
на $x^4-x^3-3x^2+x+2$. Найти жорданову нормальную форму оператора
$\mathcal{A}$ и какой-нибудь жорданов базис.

$$
\textit{Ответ:\ } C=\left(
\begin{array}{rrrr}
2 & -1 & 2 & 2 \\
-3 & -1 & -1 & -3 \\
0 & 1 & -2 & 1 \\
1 & 1 & 1 & 0
\end{array}
\right), J=\left(
\begin{array}{rrrr}
1 & 0 & 0 & 0 \\
0 & 2 & 0 & 0 \\
0 & 0 & -1 & 1 \\
0 & 0 & 0 & -1
\end{array}
\right).
$$

\section{Пересдача}

\subsection{Вариант 5.2}

1. Найти базис суммы $U+V$ и пересечения $U\cap V$ подпространств
$U$ и $V$, если
$$U=\ha{(1, 1, 1, 1, 1),(1, 1, 0, 0, 2), (2, 0, 1, -1, 3)}, \quad V=\ha{(1, 2, 0, 0, 0), (0, -1, 1, 1, 1), (1, -1, 2, 1, 2)}.$$

2. Пусть $f$ --- оператор в пространстве многочленов степени $\leq
2$, который переводит многочлен $P(x)$ в многочлен
$Q(x)=P(2)+P'(1)\cdot x+P''(2)\cdot x^2$. Найти матрицу оператора
$f$ в базисе $e_1=1$, $e_2=1+x, e_3=1+x+x^2$.

3. Найти жорданову форму и жорданов базис матрицы
$$
\left(
\begin{array}{rrrr}
7 & -4 & 2 & 5 \\
8 & -5 & 2 & 7 \\
-5 & 2 & -2 & -2 \\
1 & -1 & 0 & 2
\end{array}
\right)
$$

Вычислить $\exp A$.

4. Для пары $Q_1=x^2+2xy$ и $Q_2=5x^2+2xy+2y^2$ квадратичных
функций найти базис, в котором одна из них приводится к
диагональному виду, а другая к единичному. Какой вид имеет
квадратичная функция $Q_3=x^2-4xy-y^2$ в этом базисе?

5. Найти угол между вектором $(2, -2, 0, 1, 3)$ и подпространством
$$
\left\{
\begin{array}{lll}
x_1-x_2+x_3-x_4+x_5&=&0\\
x_1+x_2+2x_3-2x_4&=&0
\end{array}\right.
$$

6. Представить матрицу $
\left(
\begin{array}{rrr}
-2 & 16 & 8 \\
8 & -1 & 4 \\
-4 & -4 & 7
\end{array}\right)$ в виде произведения $A=QS$ ортогональной
матрицы $Q$ на симметричную $S$.

\medskip
\dmvntrail

\end{document}
