\documentclass[a4paper]{article}
\usepackage[simple]{dmvn}

\begin{document}

\subsection*{Базис суммы и пересечения подпространств}

Пусть $a_1,\dots,a_k$ и $b_1,\dots,b_l$ задают два подпространства $U$ и $W$. Составим из
них матрицу $A$, выписывая вектора по столбцам. Приведём матрицу $A$ к ступенчатому виду,
используя элементарные преобразования строк; получим матрицу $B$. Тогда вектора $a_i$ и
$b_j$, соответствующие столбцам матрицы $B$, несущим лидеры, будут базисом для $U+W$.
Найдем все вектора $\{v: v \in U \cap W \}$. Запишем это в виде
$$v=\sum_{i=1}^k x_ia_i = - \sum_{j=1}^l y_jb_j \; \Lra \; \rbmat{A} \rbmat{x_1 \\ \dots \\ x_{k+l}}=0.$$
Найдём её ФСР, пусть это будут вектора $X^1,\dots,X^p$. Тогда
искомый базис будет содержать $p$ векторов, определяемых соотношением
$$Y^s=\sum_{i=1}^k X_i^sa_i.$$

\subsubsection*{Приведение кососимметрической билинейной функции к симплектическому базису}
Пусть $a_{12} \neq 0$. Заменим
$$x_1=u_1,\; x_2=u_2-\frac{a_{13}}{a_{12}}u_3-\dots-\frac{a_{1n}}{a_{12}}u_n, \; x_3=u_3, \; \dots, \; x_n=u_n.$$
Для $y_i$ -- то же самое, но используем новую переменную $w_i$. Тогда функция приведётся
к виду $$f=u_1w_2-u_2w_1-u_2(b_3w_3+\dots+b_nw_n)+w_2(b_3u_3+\dots+b_nu_n)+\dots.$$
Тогда используем вторую подстановку
$$u_1=z_1-b_3z_3-\dots-b_nz_n,\; u_2=z_2, \; \dots, \; u_n=z_n.$$
Теперь функция приведена к виду $f=t_1z_2-z_1t_2+\dots$, причём ненаписанные члены имеют индексы
строго больше 2.

\subsubsection*{Метод Лагранжа для БЛФ и ПЛФ}
$b_{11} \neq 0: \; x_1=y_1-\frac{b_{12}}{b_{11}}y_2-\dots-\frac{b_{1n}}{b_{11}}y_n$.
Если все $b_{ii}=0$, сделать замену $x_1=y_1-y_2, \; x_2=y_1+y_2$.
В случае ПЛФ заменить $b_{1i}$ на $\overline{b}_{1i}$. Товарищ, помни, что эрмитово скалярное
произведение антилинейно по первому аргументу, и  $(x,y)=\overline{x}_1y_1+\dots$.

\subsubsection*{Приведение пары функций}
Пусть $g>0$. Приведём её методом Лагранжа к нормальному виду. Пусть $C_1$ -- матрица перехода.
Пусть $A^\prime=C_1^\top AC_1$. Пусть $C_2$ -- ортогональная матрица приведения $A^\prime$ к
главным осям. Тогда $C_1C_2$ -- искомое преобразование. Формула преобразования координат:
$X=CY$, т.е. каждая строка $C$ есть строка коэффициентов при переменных $y_i$.

\subsubsection*{Приведение симметрических, ортогональных и кососимметрических операторов}
\textit{Симметрический оператор:} находим $\{\lambda_i \}$, и ищем ФСР $\Ac-\lambda \Ec$.
Теперь выписываем каноническую матрицу $\diag(\lambda_i)$, а матрица перехода состоит из
нормированных векторов ФСР по столбцам. \textit{Ортогональный оператор:} пусть $\lambda=\alpha+i\beta$ --
СЗ $\Ac_\mathbb{C}$, а $z=x+iy$ -- СВ. Тогда в базисе $\frac{x}{|x|}, \; \frac{y}{|y|}$
матрица будет иметь клетку $\rbmat{\alpha & \beta \\ -\beta & \alpha}$. \textit{Кососимметрический
оператор:} аналогично ортогональному, только у него $\alpha=0$.

\subsubsection*{Жорданова форма и базис}
Пусть $\Ac$ -- исходный оператор. Пусть $S_k$ -- число ЖК размера $k \times k$
для СЗ $\lambda$. Пусть $\Bc = \Ac - \lambda \Ec$. Пусть $r_k = \rk \Bc^k$. Тогда число
ЖК для $\lambda$ есть $\dim V - \rk \Bc$, а $S_1=\dim V - 2r_1+r_2, \; S_2=r_1-2r_2+r_3, \dots$.
Отсюда находим ЖНФ. Поиск базиса: выбираем клетку максимального размера, и ищем вектор
$e_k: \Bc^{k-1}e_k \neq 0, \; \Bc^ke_k=0$. Надо найти все степени матрицы $\Bc$ до
порядка $k$ включительно и посчитать ФСР. Тогда вектора
$\Bc^{k-1}e_k, \; \Bc^{k-2}e_k, \; \dots, \; \Bc e_k, \; e_k$
составляют часть искомого базиса, соответствующего $\lambda$. Для клеток меньшего размера
надо добавить требование линейной независимости со всеми остальными уже найденными векторами.


\end{document}
