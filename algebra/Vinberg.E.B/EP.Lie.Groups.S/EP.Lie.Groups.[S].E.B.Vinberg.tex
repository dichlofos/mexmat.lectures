\documentclass[a4paper]{article}
\usepackage[simple,utf]{dmvn}

\title{Программа спецкурса <<Гpуппы Ли>>}
\author{Лектоp Э.\,Б.\,Винберг}
\date{2004--2005 г.}

\begin{document}
\maketitle

\begin{nums}{0}
\item Группы Ли. Подгруппы Ли. Линейные группы Ли. Классические линейные группы
Ли $\GL_n(K)$, $\SL_n(K)$, $\SO_n(K)$, $\Sp_n(K)$, $\SU_n$. Прямые произведения
групп Ли. Векторная группа Ли и $n$-мерный тор.
\item Действия групп Ли, их орбиты и стабилизаторы. Орбиты компактных групп Ли.
\item Многообразие смежных классов и факторгруппа Ли. Теоремы о транзитивном
действии группы Ли и о гомоморфизме групп Ли.
\item Связные компоненты группы Ли. Связность групп Ли $\SL_n(K)$, $\SO_n$, $\SU_n$.
\item Фундаментальная группа связной группы Ли. Отрезок точной гомотопической
последовательности, связанной с транзитивным действием группы Ли.
Односвязность групп Ли $\SL_n({\Cbb})$ и $\SU_n$.
\item Накрывающие гомоморфизмы связных групп Ли. Односвязная накрывающая группы Ли.
Накрывающие гомоморфизмы $\SU_2\to \SO_3$, $\SU_2\times \SU_2\to \SO_4$,
$\SL_2(\Cbb)\to \SO_3(\Cbb)$, $\SL_2(\Cbb)\times \SL_2(\Cbb)\to \SO_4(\Cbb)$.
Фундаментальная группа группы $\SO_n(K)$.
\item Присоединенное представление и касательная алгебра Ли группы Ли. Касательная
алгебра линейной группы Ли.
\item Дифференциал гомоморфизма групп Ли. Дифференциальное уравнение пути в группе Ли.
Восстановление гомоморфизма связной группы Ли по его дифференциалу.
\item Касательные алгебры стабилизатора точки при действии группы Ли и ядра
гомоморфизма групп Ли. Полный прообраз подгруппы Ли при гомоморфизме.
\item Соответствие между связными подгруппами Ли и подалгебрами касательной алгебры.
Связь между инвариантными подпространствами для линейного представления связной
группы Ли и его дифференциала.
\item Экспоненциальное отображение в группе Ли. Связные коммутативные вещественные
группы Ли.
\item Существование гомоморфизма односвязной группы Ли с заданным дифференциалом.
\item Описание связных групп Ли с данной касательной алгеброй. Связные двумерные
вещественные группы Ли.
\item Группа автоморфизмов алгебры и ее касательная алгебра. Автоморфизмы групп
и алгебр Ли. Внутренние автоморфизмы.
\item Полупрямое произведение групп Ли, его касательная алгебра.
\item Коммутант группы и алгебры Ли. Разрешимые группы и алгебры Ли.
\item Существование группы Ли с заданной разрешимой касательной алгеброй.
\item Теорема Ли.
\item Радикал алгебры Ли, его образ при неприводимом комплексном представлении.
\item Полупростые алгебры и группы Ли. Полупростота связной неприводимой унимодулярной
линейной группы Ли в комплексном векторном пространстве. Связь между
полупростотой вещественной алгебры Ли и ее комплексификации. Полупростота
групп Ли $\SL_n(K)$, $\SO_n(K)$, $\Sp_n(K)$, $\SO_{р,q}$, $\SU_{р,q}$.
\item Радикал группы Ли.
\item Разложение Жордана линейного оператора в комплексном векторном пространстве.
Разложение Жордана ограничения оператора на инвариантное подпространство
и присоединенного оператора.
\item Теорема Энгеля.
\item Инвариантное скалярное умножение в полупростой алгебре Ли.
Дифференцирования полупростой алгебры Ли. Существование группы Ли с заданной
полупростой касательной алгеброй.
\item Разложение полупростой алгебры Ли в прямую сумму простых.
\item Лемма о неподвижной точке компактной аффинной группы. Полная приводимость
линейных представлений компактной группы. Унитарный трюк Вейля.
\item Разложение касательной алгебры компактной группы Ли в прямую сумму полупростой
и коммутативной алгебр Ли.
\item Оператор Казимира. Теорема Вейля о полной приводимости линейных
  представлений полупростой алгебры Ли.
\item Разложение Жордана в полупростой алгебре Ли.
\item Линейные представления группы Ли $\SL_2(\Cbb)$.
\item Картановская подалгебра и корневое разложение полупростой комплексной
алгебры Ли. Свойства корневого разложения.
\item Абстрактные системы корней. Группа Вейля, камера Вейля и система простых
корней. Восстановление системы корней по системе простых корней. Матрица
Картана и схема Дынкина.
\item Корневые разложения, группы Вейля и схемы Дынкина классических комплексных алгебр Ли.
\end{nums}

\bigskip

\textbf{Примечание.} Все используемые общие топологические теоремы
принимаются без доказательства.

\medskip\dmvntrail
\end{document}
