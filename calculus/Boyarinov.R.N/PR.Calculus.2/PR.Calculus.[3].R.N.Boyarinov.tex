\documentclass[a4paper,12pt]{article}
\usepackage[russian]{babel}
\usepackage[utf8]{inputenc}
\usepackage{dmvn}
\title{Контрольные работы по математическому анализу}
\date{III семестр, 2006~г.}
\author{Преподаватель Р.\,Н.\, Бояринов}
\begin{document}
\maketitle

\centerline{\textbf{Зачет №1}}
\smallskip

Исследовать на абсолютную и условную сходимость ряды:
\eqa{1}{\sums{n=2}^{+\bes} \tg\hr{\frac1{n^{\al}}}
\ln\hr{\frac{n^4+1}{n^4-1}}, \; \al>0.}
\eqa{2}{\sums{n=2}^{+\bes} \frac{\cos\hr{\frac{\pi n}3}}{n^p +
\sin\hr{\frac{\pi n}4}}, \; \forall p.}

Исследовать функциональный ряд на равномерную сходимость на
множестве $X=(0,1)$:
\eqa{3}{\sumnui \frac{\sin\hr{x\sqrt{\ln n}}
e^{-x\sqrt{n}}}{\sqrt{n+x}}.}

Найти радиус сходимости и исследовать на абсолютную
и условную сходимость в концевых точках:
\eqa{4}{\sums{n=3}^{+\bes} \frac{\ln\ln n}{n^{\al} + \cos n}x^n, \;
\forall \al.}

Исследовать бесконечное произведение на абсолютную и условную
сходимость:
\eqa{5}{\prods{n=2}^{+\bes} \hr{1+\frac{\sin^2 n}{n^{\al}}}, \;
\forall \al.}

\bigskip

\centerline{\textbf{Зачет №2}}
\smallskip

Исследовать на абсолютную и условную сходимость ряды:
\eqa{1}{\sums{n=2}^{+\bes} \tg\hr{\frac{n^4+1}{n^4-1}}
\ln\hr{\frac{n^{\al}+1}{n^{\al}-1}}, \; \forall \al.}
\eqa{2}{\sums{n=10}^{+\bes} \frac{\cos\hr{\frac{\pi n}4}}{(\ln n)^p
+ 2 + \sin\hr{\frac{\pi n}3}}, \; \forall p.}

Исследовать функциональный ряд на равномерную сходимость на
множестве $X=(0,1)$:
\eqa{3}{\sums{n=2}^{+\bes} \frac{\sin(x\ln n)}{\sqrt[3]{n+\cos x}}
e^{-xn^{1/3}}.}

Найти радиус сходимости и исследовать на абсолютную
и условную сходимость в концевых точках:
\eqa{4}{\sums{n=10}^{+\bes} \frac{\ln n}{(\ln n)^{\al} + \cos n +
1}x^n, \; \forall \al.}

Исследовать бесконечное произведение на абсолютную и условную
сходимость:
\eqa{5}{\prods{n=3}^{+\bes} \hr{1+\frac{\ln n\cos^2 n}{n^{\al}+1}},
\; \forall \al.}

Для зачёта надо было решить задачу №3 (в обоих вариантах ответ~--- сходится неравномерно)
и любые две другие задачи. Ну и сдать
коллоквиум, конечно же!
\medskip
\dmvntrail

\end{document}
