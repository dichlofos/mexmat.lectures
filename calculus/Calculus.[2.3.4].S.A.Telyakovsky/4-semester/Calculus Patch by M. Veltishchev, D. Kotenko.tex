\documentclass[a4paper]{article}
\usepackage[utf8]{inputenc}
\usepackage[russian]{babel}
\usepackage[simple]{dmvn}

\title{Математический анализ. Service Release v2.0}
\author{Д.\,Котенко, М.\,Вельтищев}
\date{егодня~г.}

\begin{document}
\maketitle

\section{Введение}
Данный документ является своего рода <<пакетом исправлений>> к лекциям по математическому анализу,
читаемым C.\,А.\,Теляковским на Механико-математическом факультете МГУ. Настоящее издание является
частичным замыканием~\cite{lectures} в кольце истины.

\section{Ошибки, допущенные в \cite{lectures} по вине лектора}
Речь идёт о следующем утверждении:

\begin{theorem}
Пусть $A, B \in \Jc$, причём $B \subs A$. Пусть $f \in \Rb(A)$. Тогда $f \in \Rb(B)$ и
$\ints{A} f = \ints{B} f + \ints{A \wo B} f$.
\end{theorem}
Это утверждение не было доказано лектором для случая, когда $f$ не является ограниченной на
множестве $A$. Мы сейчас полностью докажем это утверждение без предположения ограниченности
функции, а затем укажем, где в лекциях оно неявно используется. При доказательстве мы, естественно,
будем пользоваться тем, что мы знаем для ограниченных функций.

\begin{proof}
Поскольку $f \in \Rb(A)$, найдётся такое $h\bw>0$, что $f \in \Bb\br{A(h)}$. Кроме того, очевидно,
что $B(h) \subs A(h)$, а из построения подстриженных множества следует их измеримость по Жордану.
Обозначим $C := A(h)$, а $D := B(h)$. По известной теореме для ограниченных функций получаем
$\ints{C} f = \ints{D} f + \ints{C\wo D}f$. По той же теореме выводим, что
$f \cdot \chi_C \in \Rb(A)$ и $f \cdot \chi_D \in \Rb(B)$, поскольку эти две функции
ограничены. Кроме того, $\ints{A} f \cdot \chi_C = \ints{C} f$ и $\ints{B} f \cdot \chi_D = \ints{D} f$.
Применим теорему 1.9 из \cite{lectures} к функциям $f$ и $f \cdot \chi_C$, получим
равенство $\ints{A} f = \ints{A} f \cdot \chi_C$. Аналогично, $\ints{B} f = \ints{B} f \cdot \chi_D$.
Теперь остаётся только провести аналогичные рассуждения для $C\wo D$, после чего круг замкнётся.
\end{proof}

\begin{note}
В предыдущей версии данного издания \cite{patch} данная теорема была доказана несколько более сложно,
однако это доказательство не использовало теорию. Для сравнения мы приведём текст этого доказательства
в конце документа.
\end{note}

Указанное утверждение используется в теореме~1.31 (отмечено знаком $\bigstar$) и~1.28 (неявно используется).
Естественно, сказанное по поводу этой теоремы в предисловии к~\cite{lectures} с этого момента
следует считать устаревшим.

\section{Ошибки, допущенные в \cite{lectures} по вине наборщика и редактора}

\emph{Нет оправдания любым ошибкам, будь то опечатки или прочие глупости. Ответственность за
них целиком лежит на том, кто набирает и редактирует.} Обоснование этого утверждения
затрагивает философию, поэтому мы не будем его трогать.

Исправлять оригинальный документ за несколько дней до экзамена показалось нецелесообразным,
да и глупо перепечатывать документ ради небольшого количества исправлений. Все исправления
будут внесены в \cite{lectures} после 25.06.2004.

\subsection{Непрерывная продолжаемость на замыкание}

Отображение $\Phi$, рассматриваемое после леммы~1.25 на стр.~10, должно быть таким,\
чтобы $\Phi$ и $\pf{\ph_i}{t_j}$ можно было непрерывно продолжить на $\Cl G$. Того, что
было написано раньше, недостаточно, поскольку только из непрерывной продолжаемости
частных производных на замыкание будет следовать их ограниченность.

Указанное место в \cite{lectures} может быть и не единственным, но проверку этого мы оставляем читателю,
предупредив его о такой возможности.

\subsection{Неправильная оценка}

В теореме о замене переменных~1.27 имеет место ошибка: вместо <<$O\br{\om(h)}$>>
следует читать <<$O(h)$>> в формуле с оценкой разности частных производных после слов
<<По ФКПЛ имеем>>. Это обычная ошибка, связанная с тем, что мысли наборщика в этот момент
были где-то не там.

\section{Старое доказательство теоремы}

В процессе доказательства мы частенько будем ругаться словами типа <<предел по базе>>, но при
желании их можно не произносить.

\begin{theorem}
Пусть $f \in \Rb(A)$ и $B \subs A$ измеримое по Жордану множество.
Тогда $f \in \Rb(B)$.
\end{theorem}
\begin{proof}
Пусть $I = \ints{A} f$. Тогда $\fa \ep > 0 \exi \de > 0\cln \fa T_A\cln \la_T < \de, \fa P$
имеем $|S_{T_A} - I| < \ep$. Отсюда для двух достаточно мелких разбиений $T_A^1$ и $T_A^2$
имеем $|S_{T_A^1} - S_{T_A^2}| < 2\ep$. Такая оценка проведена для того, чтобы убедить читателя
в том, что для предела по базе также работает критерий Коши.

Рассмотрим два произвольных разбиения $T_B^1$ и $T_B^2$, для которых $\la_{T_B} <\de$. Дополним их
до разбиения множества $A$ следующим образом: возьмём достаточно мелкую сетку $\hc{N_i}$ в $\Ebb^m$
диаметра менее $\de$ и добавим к $T_B^j$ все непустые множества $(A \wo B) \cap N_i$. Получится два
мелких разбиения для $A$, которые содержит в себе в качестве подразбиения $T_B^j$. Обозначим их $\wt{T}^1_A$
и $\wt{T}_A^2$ соответственно. Поскольку на $A \wo B$ мы будем выбирать одинаковые наборы точек $P$,
имеем $|S_{T_B^1} - S_{T_B^2}| = |S_{\wt{T}_A^1} - S_{\wt{T}_A^2}| < 2\ep$. Тем самым показано,
что совокупность $\hc{S_{T_B}}$ фундаментальна по базе $\la_{T_B} \ra 0$, значит, она имеет предел.
Последнее следует из справедливости критерия Коши для предела по базе.
Но это и означает, что $f \in \Rb(B)$.
\end{proof}

Таким образом, существование интегралов обосновано. Отсюда уже следует аддитивность,
поскольку в силу того, что $f \in \Rb(A)$, $f \in \Rb(A \wo B)$ и $f \in \Rb(B)$, можно
нехитрыми рассуждениями сблизить соответствующие интегральные суммы. Мы здесь не будем
этого проделывать.

Поясним внутреннюю причину того, почему неверно обращение этой теоремы: если $f \in \Rb(B)$ и $f \in \Rb(A \wo B)$,
то $f \in \Rb(A)$. Дело в том, что не всякую интегральную сумму для разбиения $A$ можно <<разделить>>
на куски и отделить точки $B$ от точек $A \wo B$. Трудности, понятное дело,
возникнут на стыке, и никакие фокусы, подобные тем, что были в теореме, к успеху не приведут.


\begin{thebibliography}{[1]}
\bibitem{lectures} М.\,Н.\,Вельтищев. \emph{Лекции С.\,А.\,Теляковского по математическому
анализу. PS-версия.} 08.06.2004.
\bibitem{patch} Д.\,Котенко, М.\,Вельтищев. \emph{Математический анализ. Service Release v1.0.} 2004.
\bibitem{zorich} В.\,А.\,Зорич. \emph{Математический анализ.} М.: МЦНМО. 2002.
\end{thebibliography}

\end{document}
