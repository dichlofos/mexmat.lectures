\documentclass[a4paper]{article}
\usepackage{dmvn}

\tocsubsectionparam{2.7em}
\tocsubsubsectionparam{3.7em}
\newenvironment{exx}{\par\vskip\theoremskip\textbf{Пример.}}{\par\vskip\theoremskip}

\renewenvironment{dfn}[1]{\par\vskip\theoremskip\textbf{Определение~{#1}.}}{\par\vskip\theoremskip}
\newenvironment{thh}{\par\vskip\theoremskip\textbf{Теорема.}\normalfont \itshape}{\par\vskip\theoremskip}
\newenvironment{imp*}{\par\vskip\theoremskip\textbf{Следствие.}\normalfont \itshape}{\par\vskip\theoremskip}
\newenvironment{stm*}{\par\vskip\theoremskip\textbf{Утверждение.}\normalfont \itshape}{\par\vskip\theoremskip}
\newenvironment{lem}{\par\vskip\theoremskip\textbf{Лемма.}\normalfont \itshape}{\par\vskip\theoremskip}
\newenvironment{thn}[1]{\par\vskip\theoremskip\textbf{Теорема~{#1}.}\normalfont \itshape}{\par\vskip\theoremskip}

\newcommand{\emd}[1]{\emph{#1}}

\begin{document}
\dmvntitle{Курс лекций по}{математическому анализу}{Лектор\т Валериан Иванович Гаврилов}
{I курс, 2 семестр, поток механиков}{Москва, 2006 г.}

\pagebreak

\section*{Предисловие}

Этот конспект был набран Евгением Кудашевым (в 2005--2006 г. студентом 126 группы).
Материал соответствует изложению 2006 года.

Не стоит удивляться тому, что текст начинается со слов <<Часть 3>>. Это глобальная нумерация, используемая
лектором на протяжении всего курса. Данный семестр содержит третью и четвёртую части.


После завершения первичного набора текст был перевёрстан и немного отредактирован DMVN Corporation.

Опечатки вполне возможны, кое-где могут быть не совсем верно расставлены ссылки на параграфы.
Чаще всего встречаются ссылки вида 2.3, в таких случаях их нужно понимать как два последних числа
тройного номера параграфа, данной главы, например, 3.2.3.

\medskip
\dmvntrail


\tableofcontents
\pagebreak


\hrule\begin{center}\LARGE \bf Часть 3.\hfill И н т е г р а л ь н о е\quad и с ч и с л е н и е\end{center}\hrule

\section{Неопределённый интеграл}

\subsection{Неопределённый интеграл}

\subsubsection{Точная первообразная функция}

$f, \exists f', D_{f'} \subset D_f$

$f, F, F'=f$ (обратная задача), $D_f \subset D_F$

$f(x)=\frac{1}{\sqrt{1-x^2}}, D_f \in (-1,1)$

$F(x)=\arcsin x$, $D_F=[-1,1]$

$F'(x)=f(x)$

Если $\exists F: F'=f$, то для произвольного $c\in \R, (F +
c)'=F'=f$

Рассмотрим $f(x)=\frac{1}{x}$, $D_f=(-\infty;0) \cup (0;+\infty)$
\par $F(x)=\ln |x|$ \par Справедливо, что $D_f \subset D_F$. \par
Если $x>0$, то $F(x)=\ln x, F'(x)=\frac{1}{x}=f(x)$
\par Если $x<0$, то $F(x)=\ln(-x),
F'(x)=(\ln(-x))'=\frac{1}{-x}(-x)'=\frac{1}{x}=f(x)$ \par $x \in
D_f, x \in (-\infty, 0) \cup (0,+\infty)$ \par
$\Phi(x)=\case{\ln(x) + c_1, \mbox{ если } x>0; \\ \ln(-x) + c_2,
\mbox{ если } x<0}$

\subsubsection{Точная первообразная функция на промежутке}

\begin{df*} Функцию $F$, определенную на промежутке $\ha{a,b}$, $-\infty\le a
\le b\le+\infty$, называют точной первообразной функцией для функции
$f$, если:
\begin{points}{-2}
\item $F$ --- непрерывна на $\ha{a,b}$,
\item $F$ дифференцируема на $(a,b)$, и
\item $F'(x)=f(x), x \in (a,b)$.
\end{points}
\end{df*}

\begin{theorem} Всякие две точные первообразные функции для одной и той же
функции, заданные на промежутке, отличаются друг от друга на
постоянную. \end{theorem}

\begin{proof} Пусть $F_1$ и $F_2$ --- точные первообразные функции для функции
$f$ на $\ha{a,b}$. Согласно определению~1, $F_1, F_2$ непрерывны на
$\ha{a,b}$ и $F_1'(x)=F_2'(x)=f(x), x \in (a,b)$ Тогда $\ph=F_2-F_1$
непрерывна на $\ha{a,b}$ и
$\ph'(x)=(F_2(x)-F_1(x))'=F_2'(x)-F_1'(x)=f(x)-f(x)=0, \ph(x)=c, c
\in \R, x \in \ha{a,b}$ \par $F_2=F_1 + c, x \in \ha{a,b}$
\end{proof}

\begin{imp}
Если $F$ --- какая-либо точная первообразная функция для функции $f$
на $\ha{a,b}$, то совокупность всех точных первообразных для $f$
совпадает с совокупностью функций $F + c$, где $c=\const$ (любая),
т.е. совпадает с множеством \eqa{1}{\{F + c \; | \; c \in \R\}}
\end{imp}

$\Phi'(x)=f. \;\exists x_0, \;y_0=\Phi(x_0)$

$\Phi(x)=F(x) + c$

$y_0=\Phi(x_0)=F(x_0) + c$

$c=y_0-F(x_0)$

\subsubsection{Дифференциальная форма на промежутке}
\begin{df*}
Дифференциальной формой на промежутке $I$ называют семейство
однородных линейных функций на $\R$, зависящих от параметра,
пробегающего $I$.
\end{df*}
Таким образом, с каждой точкой $x\in I$ ассоциирована однородная
линейная функция на $\R$, скажем, $l(x)$. Всякая однородная
линейная функция имеет вид $k \cdot h$, где $h$ --- аргумент, $k \in
\R$, $k$ --- число, не зависящее от $h$. В частности,
$l(x)(h)=k(x)\cdot h, h \in \R$. $k(x)$ --- функция,
определённая на $I$.

$\,dx(h)=h, h \in \R$.

$l(x)(h)=k(x)\,dx(h), h \in \R$

$l(x)=k(x)\,dx, x \in I$

\subsubsection{Неопределённый интеграл}

$F'(x)=f(x) \Lra dF(x)=f(x)\,dx$

\begin{df*}
  Пусть функция $f$ имеет на $\ha{a,b}$ точную первообразную функцию $F$.
Произвольную функцию $\hc{F + c \vl c \in \R}$ из множества~(1) назовём неопределенным интегралом функции $f(x)$
дифференциальной формы $f(x)\,dx$ на $\ha{a,b}$ и обозначим символом
$\int f(x)\,dx$. $f$ --- подинтегральная форма, $f(x)\,dx$ -
подынтегральное выражение.
\end{df*}

Таким образом, согласно определению 3 и следствию к теореме 1,

\eqa{2}{\int f(x)\,dx=F(x) + c, x \in \ha{a,b},} где $c$ ---
произвольная постоянная.

\textbf{Примеры}

1. $\int \frac{\,dx}{\sqrt{1-x^2}}=\arcsin x + c, x \in [-1,1]$

2. Если $k$ --- число, то $\int k \,dx=kx + c, x \in \R$

3. $k=1$, то $\int 1\,dx=x + c, x \in \R$

4. $k=0$, то $\int 0\,dx=c.$

\begin{theorem}
Если функция $f$ имеет $\int f(x)\,dx$ на $\ha{a,b}$, то

\eqa{3}{\hr{\int f(x)\,dx}'=f(x), \;d\int f(x)\,dx=f(x)\,dx, \quad x \in (a,b)}

$$\int dF(x)=F(x) + c, \quad x \in \ha{a,b}.$$
\end{theorem}

\begin{proof}
Согласно~(2),
$$\bbr{\int f(x)\,dx}'=(F(x) + c)'=F'(x)=f(x), \quad \text{ и } \quad d\int f(x)\,dx=\bbr{\int f(x)\,dx}'\,dx=f(x)\,dx.$$
Так как $F$ --- точная первообразная функция $F'(x)\,dx$, то, согласно~(2),
$\int dF(x)=F(x) + c$, где $c \in \ha{a,b}$.
\end{proof}

\subsubsection{Линейные операции над неопределёнными интегралами}

\begin{prop}
Пусть $k$ --- число, $k\ne 0$. Функция $f$ имеет точную
первообразную на $\ha{a,b}$, если и только если её имеет функция
$kf$, причём тогда \eqa{4}{\int kf(x)\,dx=k\int f(x)\,dx}
\end{prop}

\begin{proof}
Функция $F$ --- точная первообразная функция $f$ на $\ha{a,b}$, если
и только если $kF$ --- точная первообразная функция для $kf$ на
$\ha{a,b}$ (в силу свойства монотонности непрерывных функций и
операции дифференцирования). Согласно~(2), $\int kf(x)\,dx=kF(x) +
c=k(F(x) + \frac{c}{k})=k \int f(x)\,dx$, т.к. $\frac{c}{k}$ ---
такая же произвольная постоянная, как и $c$.
\end{proof}

\begin{note}
Условие $k\ne 0$ --- важное, т.к. $\int 0f(x)\,dx=c$, но $0 \cdot
\int f(x)\,dx=0$ (если $\exists\int f(x)\,dx$).
\end{note}

\begin{prop}
Если $f$ и $g$ имеют точные первообразные на $\ha{a,b}$, то это же
верно и для $f + g$, причём \eqa{5}{\int (f(x) + g(x))\,dx=\int
f(x)\,dx + \int g(x)\,dx, x \in \ha{a,b}}
\end{prop}
\begin{proof}
Пусть $F$ и $G$ --- точные первообразные функции для $f$ и $g$ на
$\ha{a,b}$ соответственно. Тогда $F$ и $G$ непрерывны на $\ha{a,b}$.
$F'(x)=f(x), G'(x)=g(x), x \in (a,b)$. Тогда $F(x)+G(x)$ непрерывна
на $\ha{a,b}$ и $(F(x) + G(x))'=f(x) + g(x), x \in (a,b)$, т.е. $F +
G$ --- точная первообразная функция для $f + g$ на $\ha{a,b}$.
Согласно~(2), \eqa{6}{\int (f(x) + g(x))\,dx=F(x) + G(x) + c, x \in
\ha{a,b}}

С другой стороны, согласно~(2), $\int f(x)\,dx=F(x) + c_1$ и $\int
g(x)\,dx=G(x) + c_2, x \in \ha{a,b}, c_1, c_2$ --- произвольные
постоянные. Поэтому \eqa{7}{\int f(x)\,dx +  \int g(x)\,dx=F(x)  +
G(x)  + c_1 + c_2, x \in \ha{a,b},} так как $c_1  +  c_2$ --- такая же произвольная постоянная, как и
$c$ ($\R+\R=\R$). Таким образом, из~(6) и~(7) следует формула~(5).
\end{proof}

\eqa{2'}{\int dF(x)=F(x) + c}

\subsubsection{Таблица интегралов}

\begin{nums}{-2}
\item $$\int k\,dx= kx  +  c, \int 0\,dx=c, x\in\R$$
\item $$\int x^\al \,dx=\frac{x^{\al+1}}{\al+1}+c$$

\begin{proof}
Т.к. $d\hr{\frac{x^{\al+1}}{\al+1}}=x^\al \,dx$, то
согласно (2$'$) справедливо 2.
\end{proof}
\item $$\int\frac{\,dx}{x}=\bcase{&\ln x + c_1, &x>0; \\ &\ln(-x) + c_2, &x<0.}$$
\item $$\int a^x \,dx=\frac{a^x}{\ln a} + c, 0<a\ne 1, x\in\R$$
$$\int e^x \,dx=e^x + c, x\in\R$$
\item $$\int\cos x \,dx=\sin x + c, \quad x \in\R,$$ в силу равенства $d(\sin x)=\cos \,dx$ и формулы~(2$'$).
\item $$\int\sin x \,dx=-\cos x + c, \quad x \in\R,$$ так как $d(-\cos x)=\sin \,dx$.
\item $$\int\frac{\,dx}{\cos^2 x}=\tg x + c_k, \quad x \in (-\frac{\pi}{2}+k\pi, \frac{\pi}{2}+k\pi), \quad k \in \Z$$
\item $$\int\frac{\,dx}{\sin^2 x}=-\ctg x + c_k, \quad  x\in (k\pi, (k+1)\pi),  \quad k \in \Z$$
\item $$\int\frac{\,dx}{\sqrt{1-x^2}}=\arcsin x + c = -\arccos x + c,  \quad x \in [-1,1].$$
\item $\int\frac{\,dx}{1+x^2}=\arctg x + c = -\arcctg x + c, \quad x\in\R$
\end{nums}

\subsubsection{Интегрирование по частям в неопределенном интеграле}

\begin{theorem}
Если у функций $u(x)$ и $v(x)$ на промежутке $\ha{a,b}$ существуют
производные $u'(x)$ и $v'(x)$ (в концевых точках предполагается
существование соответствующих односторонних производных), то из
существования на $\ha{a,b}$ одного из неопределённых интегралов
$\int v(x)u'(x)\,dx, \int u(x)v'(x)\,dx$ следует существование
другого и равенство \eqa{8}{\int u(x)v'(x)\,dx=u(x)v(x)-\int
v(x)u'(x)\,dx}
\end{theorem}
\begin{proof}
По условию, функции $u, v$ дифференцируемы в $(a,b)$ (и,
следовательно, непрерывны в $(a,b)$) и имеют односторонние
производные в концевых точках $\ha{a,b}$, и, следовательно, $u,v$
--- непрерывны на $\ha{a,b}$. Таким образом, $uv$ --- непрерывна на
$\ha{a,b}$ и $(u(x)v(x))'=u'(x)v(x) + u(x)v'(x), x\in (a,b).$ Пусть,
для определённости, существует $\int v(x)u'(x)\,dx$ на $\ha{a,b}$,
т.е. функция $\int v(x)u'(x)\,dx$ непрерывна на $\ha{a,b}$ и $(\int
v(x)u'(x)\,dx)'=v(x)u'(x), x\in (a,b)$. Тогда функция $u(x)v(x)$ ---
$\int v(x)u'(x)\,dx$ непрерывна на $\ha{a,b}$ и дифференцируема в
$(a,b)$ и
\mla{9}{\hs{u(x)v(x) - \int v(x)u'(x)\,dx}'= v'(x)u(x) + v(x)u'(x) - \hr{\int v(x)u'(x)\,dx)}'=\\=
v'(x)u(x) + v(x)u'(x) - v(x)u'(x)= u(x)v'(x),}

$x\in (a,b)$, и

\eqa{10}{\int u(x)v'(x)\,dx=u(x)v(x) - \int v(x)u'(x)\,dx + c, x\in
\ha{a,b}}

Формула (10) $\Lra$ (8), так как произвольная постоянная
$c$ убирается в любом интеграле:

\eqa{8'}{\int u\,dv=uv-\int v\, du.}
\end{proof}

\begin{exx}
\ml{\int x \arctg x\,dx =
\mbmat{
u=\arctg x & du=\frac{\,dx}{1+x^2} \\
v=\frac{x^2}{2} & dv=x\,dx}
= \frac{x^2}{2}\arctg x - \int\frac{x^2}{2}\frac{\,dx}{1+x^2}=\\=
\frac{x^2}{2}\arctg x - \frac12 \int\frac{x^2+1-1}{1+x^2}\,dx=\frac{x^2}{2}\arctg x - \frac12 \hs{\int \,dx - \int\frac{\,dx}{1+x^2}}
= \frac{x^2}2\arctg x - \frac{1}{2}x+\frac{1}{2}\arctg x + c.}
\end{exx}


\subsubsection{Замена переменной интегрирования в неопределенном интеграле}
\begin{theorem}
Пусть функция $F(t)$ есть точная первообразная функция для функции
$f(t)$ на промежутке $\ha{\al, \beta}$, а функция $\om(x)=t$
непрерывна на промежутке $\ha{a,b}$, дифференцируема в интервале
$(a,b)$ и $f(\ha{a,b}) \bw\subset \ha{\al, \beta}$. Тогда на
промежутке $\ha{a,b} \exists \int f(\om(x)) \om'(x) \,dx$ и
$\int f(\om(x)) \om'(x) \,dx=F(\om(x))+c, x\in \ha{a,b}$
\end{theorem}
\begin{proof}
  Согласно определению 1, $F$ непрерывна на $\ha{\al, \beta}$,
  дифференцируема в интервале $(\al,\beta)$ и $F'(t)=f(t)$ для
  всех $t \in (\al,\beta)$. Из условий теоремы следует, что
  сложная функция $F(\om(x))$ непрерывна на $\ha{a,b}$ (как
  композиция непрерывных функций) и дифференцируема в интервале
  $(a,b)$ (как композиция дифференцируемых функций), причём
  $(F(\om(x)))'=F'(t)\cdot\om'(x)=f(t)\om'(x)=f(\om(x))\om'(x),
  x\in (a,b)$. Таким образом, функция $F(\om(x))$, согласно
  определению 1, есть точная первообразная для функции
  $f(\om(x))\cdot\om'(x)$ на промежутке $\ha{a,b}$. По теореме
  1, $\exists \int f(\om(x))\om'(x)\,dx$ на $\ha{a,b}$,
  следовательно, в силу~(2), справедливо~(1).
\end{proof}
\begin{theorem}
  Пусть функция $\om(x)=t$ непрерывна на промежутке $\ha{a,b}$,
  дифференцируема в интервале $(a,b)$ и $\om'(x)>0$ для всех
  $x\in(a,b)$. Обозначим $\al=\infl{x\in\ha{a,b}} \om(x)$, $\beta=\supl{x\in\ha{a,b}}\om(x)$ (возможно,
  $\al=-\infty$ или $\beta=+\infty$), так что
  $\om(\ha{a,b})=\ha{\al,\beta}$ и на $\ha{\al,\beta}$
  определена функция $\Omega(t)=x$, обратная к $\om(x)=t$. Если
  для функции $f(t)$, определенной на $\ha{\al,\beta}$, функция
  $f(\om(x))\om'(x)$ имеет на $\ha{a,b}$ точную первообразную
  функцию $\Phi(x)$, то функция $F(t)=\Phi(\Omega(t))$ будет точной
  первообразной функцией для функции $f(t)$ и справедливо
  \eqa{2}{\int f(t) \, dt=F(t)+c=\Phi(\Omega(t))+c, t \in
  \ha{\al,\beta}}
\end{theorem}
\begin{proof}
  Из условий теоремы следует, что обратная функция $\Omega(t)$
  непрерывна на $\ha{\al,\beta}$, дифференцируема в $(\al,\beta)$ и
  $\Omega'(t)=\frac{1}{\om'(x)}$ для всех $t\in(\al,\beta)$
  (по теореме о существовании непрерывной и дифференцируемой
  обратной функции).

  Кроме того, согласно определению 1, функция $\Phi(x)$ непрерывна на
  $\ha{a,b}$ дифференцируема в $(a,b)$ и
  $\Phi'(x)=f(\om(x))\cdot\om'(x), x\in(a,b)$.

  Поэтому, функция $F(t)=\Phi(\Omega(t))$ непрерывна на
  $\ha{\al,\beta}$ (как композиция непрерывных функций),
  дифференцируема в $(\al,\beta)$ (как композиция
  дифференцируемых функций), и
$$F'(t)=\Phi'(x)\cdot\Omega'(t)=f(\om(x))\om'(x)\cdot\frac{1}{\om'(x)}=  f(\om(x))=f(t)$$
для всех $t\in(\al,\beta)$. Таким образом,
  функция $F(t)=\Phi(\Omega(t))$ есть точная первообразная функция
  для функции $f(t)$ на $\ha{\al,\beta}$.

  Согласно формуле~(2), $\exists \int f(t) \,dt$ на
  $\ha{\al,\beta}$ и справедливо~(2) из условия теоремы.
\end{proof}

Одним из важных следствий результата следующей главы служит следующая
\begin{theorem}
Всякая функция, непрерывная на отрезке, имеет на нём точную первообразную.
\end{theorem}

\subsection{Первообразная функция на промежутке}
$f(x)=\sgn x, x\in\ha{a,b}, a<0<b$ не имеет точной первообразной
$F(x)$ на $\ha{a,b}$, т.е. не существует на $\ha{a,b}$ функции
$F(x)$, $F'(x)=\sgn(x)$, $x\in(a,b)$, т.к. $\sgn x$ имеет разрыв
первого рода в $x_0 \in (a,b)$, а производная функция может иметь
точки разрыва только второго рода.

\subsubsection{Первообразная функция на промежутке}
\begin{df*}
  Функцию $F(x)$, определенную на промежутке $\ha{a,b}$ назовём
  первообразной функцией для функции $f(x)$, определённой на $\ha{a,b}$,
  если: 1)$F(x)$ непрерывна на $\ha{a,b}$, 2)$F(x)$ дифференцируема
  всюду в интервале $(a,b)$, за возможным исключением некоторого
  конечного множества $K$ точек на $(a,b)$ и 3)$F'(x)=f(x)$ для всех
  $x\in(a,b)\wo K$.
\end{df*}
\begin{note}
  В соответствии с этим определением, точная первообразная функция
  есть первообразная функция с пустым исключительным множеством $K$
  (то есть, если $K=\es$, то определение 1 $\rightarrow$
  определение 1 из параграфа 1).
\end{note}
 Согласно определению 1, функция $F(x)=|x|$ будет
первообразной функцией для $f(x)=\sgn x$ на любом $\ha{a,b}, 0<a<b$,
с исключительным множеством $K=\{0\}$ (одноэлементное множество).
Т.к. $|x|'=\sgn x, x \ne 0$ и $|x|$ --- непрерывная функция.
\begin{note}
  $F(x)=\sgn x$ не является первообразной (в смысле определения 1)
  для $f(x)=0$ на $\ha{a,b}, a<0<b$, хотя $F'(x)=0, x\ne0$, но
  $F(x)=\sgn x$ не является непрерывной на $\ha{a,b}$.
\end{note}

\subsubsection{Множество первообразных функций на промежутке}
\begin{theorem}
  Если функция $f(x)$ определена на $\ha{a,b}$ и имеет на $\ha{a,b}$
  первообразную $F(x)$, то она имеет на $\ha{a,b}$ бесконечно много
  первообразных и их множество состоит из функций $\Phi(x)=F(x)+c, x\in
  \ha{a,b}$, $c$ --- произвольная постоянная.
\end{theorem}
\begin{proof}
  По условию и определению 1, функция $F(x)$ непрерывна на $\ha{a,b}$,
  существует $F'(x)=f(x)$ для всех $x\in(a,b)\wo K$, где $K$
  --- некоторое конечное множество точек из $(a,b)$. Тогда для
  произвольного $c\in\R$ функция $\Phi(x)=F(x)+c$ непрерывна на
  $\ha{a,b}$ и $\Phi'(x)=(F(x)+c)'=F'(x)=f(x)$ для всех $x\in(a,b) \wo K$. Таким образом, функция $\Phi(x)=F(x)+c$ есть первообразная
  для $f(x)$ на $\ha{a,b}$ с исключительным множеством $K$. Обратно,
  пусть функция $\Phi(x)$ есть первообразная для $f(x)$ на $\ha{a,b}$
  (в смысле определения 1) с некоторым исключительным множеством
  $K_1\subset(a,b)$, то есть $\Phi(x)$ непрерывна на $\ha{a,b}$ и
  $\Phi'(x)=f(x)$ для всех $x\in(a,b)\wo K_1$. Тогда функция
  $g(x)=\Phi(x)-F(x)$ непрерывна на $\ha{a,b}$ (как разность
  непрерывных функций) и
  $g'(x)=(\Phi(x)-F(x))'=\Phi'(x)-F'(x)=f(x)-f(x)=0$ для всех
  $x\in(a,b)\wo(K\cup K_1)$. Множество $K\cup K_1$ ---
  конечное, и следовательно по теореме о стирании особенностей
  непрерывной функции, $g(x)=c, x\in\ha{a,b}$. Итак,
  $\Phi(x)=F(x)+c$, $x\in(a,b)$ и $K_1=K$. %имхо, полный бред и x\in(a,b) \wo K_1\cup K.
\end{proof}

\subsubsection{Неопределённый интеграл}
\begin{df*}
  Неопределенным интегралом функции $f$, определённой на $\ha{a,b}$ и
  имеющей на $\ha{a,b}$ первообразную $F(x)$ с исключительным
  множеством $K$, назовём произвольную функцию $\Phi(x)$ из
  множества \eqa{2}{\{F(x)+c \; | \; c \in \R \}} и обозначим $\int
  f(x) \,dx = F(x)+c, x \in \ha{a,b}, c\in\R$.
\end{df*}
$\int \sgn x \,dx=|x|+c, x\in \ha{a,b}$

На $\left<a,0\right]$, $\left[0,b\right>$ функция $F(x)=|x|$ есть
точная первообразная для $f(x)=\sgn x$.

\begin{theorem}
  Если функция $f$ имеет первообразную $F$ на $\ha{a,b}$ с
  исключительным множеством $K$, то в любой $x\in(a,b)\wo К$
  справедливо $(\int f(x) \,dx)'=f(x)$ и $d \int f(x) \,dx=f(x)\,dx$.
\end{theorem}

\begin{proof}
  Следствие формулы~(2).
\end{proof}

\subsubsection{Линейное свойство}
\begin{theorem}
  Если функции $f,g$ имеют на $\ha{a,b}$ неопределенные интегралы
  $\int f(x) \,dx$, $\int g(x)\,dx$, то для любых
  $\la_1,\la_2 \in \R$ функция $\la_{1}f + \la_{2}g$
  имеет $\int(\la_{1}f(x)+\la_{2}g(x))\,dx$ и \eqa{3}
  {\int(\la_{1}f(x)+\la_{2}g(x)) \,dx=\la_1\int f(x)\,dx + \la_2\int g(x)\,dx.}
\end{theorem}
\begin{proof}
  Согласно определению 1 и теоремам 1 и 2, в $(a,b)$ существуют
  конечные множества $K_1$ и $K_2$, что $(\int f(x)\,dx)'=f(x)$ и
  $(\int g(x) \,dx)'=g(x)$ для всех $x\in(a,b)\wo (K_1 \cup
  K_2)$ и также $\int f(x)\,dx$, $\int g(x)\,dx$ непрерывны на
  $\ha{a,b}$. Тогда функция $F(x)=\la_{1}\int f(x) \,dx +
  \la_2 \int g(x)\,dx$ непрерывна на $\ha{a,b}$ и

$$F'(x)=\bbr{\la_{1}\int f(x) \,dx + \la_{2}\int g(x)\,dx}'=\la_{1}\bbr{\int f(x) \,dx}' + \la_{2}\bbr{\int g(x) \,dx}'=
\la_{1}f(x)+\la_{2}g(x), x\in(a,b)\wo(K_1  \cup K_2).$$
Так как множество $K_1 \cup K_2$ --- конечное,
  то, согласно определению 1, функция $F(x)$ есть первообразная для
  $\la_{1}f+\la_{2}g$ на $\ha{a,b}$ с исключительным
  множеством $K=K_1\cup K_2$. По определению 2, существует интеграл
$$\int(\la_{1}f(x)+\la_{2}g(x))\,dx=F(x)+c=\la_1\int f(x)\,dx+\la_{2}\int g(x)\,dx+c, \quad
x\in\ha{a,b}\Lra(3),$$ так как постоянную $c$ можно убрать
  в любом из неопределённых интегралов правой части.
\end{proof}
\subsubsection{Интегрирование по частям}
\begin{theorem}
  Пусть функции $u(x), v(x)$ определены на $\ha{a,b}$,
  дифференцируемы в $(a,b)$ и имеют односторонние производные в его
  концевых точках (входящих в $\ha{a,b}$). Если на $\ha{a,b}$
  существует $\int v(x)u'(x) \,dx$ с некоторым исключительным
  множеством $K \subset (a,b)$, то на $\ha{a,b}$ существует $\int
  u(x)v'(x) \,dx$ с тем же исключительным множеством $K$ и
  справедлива формула \eqa{4}{\int u(x)v'(x) \,dx = u(x)v(x)-\int
  v(x) u'(x)\,dx, x\in\ha{a,b}.}
\end{theorem}
\begin{proof}
Из условий следует, что функция $u(x)v(x)$ непрерывна на $\ha{a,b}$, дифференцируема в $(a,b)$ и $(u(x)v(x))'=u'(x)v(x)+u(x)v'(x),
x\in(a,b)$, а также, что функция $\Phi(x)=\int v(x)u'(x) \,dx$
непрерывна на $\ha{a,b}$ и $\Phi'(x)=v(x)u'(x)$ для всех
$x\in(a,b)\wo K$. Поэтому, функция $u(x)v(x)-\Phi(x)$
непрерывна на $\ha{a,b}$, дифференцируема в $(a,b)\wo K$.
Значит,
$$(u(x)v(x)-\Phi(x))'=u'(x)v(x)+u(x)v'(x)-\Phi'(x)=u(x)v'(x)+v(x)u'(x)-v(x)u'
(x)=u(x)v'(x), \quad x\in(a,b)\wo K.$$

Таким образом, функция $u(x)v(x)-\Phi(x)$ есть первообразная для
$u(x)v'(x)$ на $\ha{a,b}$ с исключительным множеством $K$. Согласно
определению 2, существует интеграл
$$\int u(x)v'(x) \,dx = u(x)v(x)-\Phi(x)+c=u(x)v(x) - \int v(x)u'(x) \,dx + c, \quad x\in \ha{a,b},$$
что равносильно~(4), т.к. постоянную $c$ можно убрать в любом
интеграле в правой части.
\end{proof}
\begin{note}
Всякая функция, имеющая конечное множество точек разрыва на отрезке,
обладает на нём первообразной.
\end{note}

\section{Определённый интеграл Римана}

\subsection{Определённый интеграл Римана}

\subsubsection{Размеченные разбиения отрезка}

Рассмотрим на $\R$ произвольный $[a,b]$. Разбиение $T$ отрезка
$[a,b]$ --- всякое множество ${x_0, x_1,\dots, x_n}$ точек
$x_k\in[a,b], k=\ol{0,n}$, удовлетворяющих условиям
$a=x_0<x_1<\dots<x_{n-1}<x_n=b$. Отрезок $\De_k=[x_{k-1},x_k],
k=\ol{1,n}$ --- отрезок разбиения $T$. Его длина
$\hm{\De_k}=\De x_k=x_k-x_{k-1}, k=\ol{1,n}$. Число
$d(T)=\max\limits_{1\le k \le n} \De x_k$ --- диаметр $T$,
$0<d(T)\le b-a$.

На каждом $\De_k$ рассмотрим произвольную точку
$\zeta_k\in\De_k, k=\ol{1,n}$ и рассмотрим
$\hc{\zeta_1,\dots,\zeta_n}=\zeta$. Присоединяя множество $\zeta$ к
разбиению $T$, получим размеченное разбиение $T_\zeta$, где
$T:\;a=x_0<x_1<\dots<x_{n-1}<x_n=b$ --- точки размеченного разбиения
$T_\zeta$. По определению, $d(T_\zeta)=d(T)$.

Множество всех размеченных разбиений $[a,b]$ обозначается $\Ps$.

\subsubsection{База размеченных разбиений отрезка}

На $\Ps$ рассмотрим систему $\B=\hc{B_\de}$, где
$B_\de=\hc{T_\zeta\in\Ps\; | \; d(T_\zeta)<\de}, \de>0$.
Покажем, что $\B$ образует базу.

Чтобы проверить свойство 1) базы, рассмотрим произвольное $\de>0$
и выберем $n\in\N$, чтобы $n>\frac{b-a}{\de}$. Разделим $[a,b]$
на $n$ отрезков $\De_k=[x_{k-1},x_k], k=\ol{1,n}$
одинаковой длины $\frac{b-a}{n}$. Получим некоторое разбиение $T,
d(T)=\frac{b-a}{n}<\de$. Выбирая, например,
$\zeta_k=x_{k-1}\in\De_k, k=\ol{1,n}$, получим
$\zeta=\hc{\zeta_1,\dots,\zeta_n}$ и различные $T_\zeta,
d(T_\zeta)=d(T)<\de$, то есть $T_\zeta\in B_\de$.

Чтобы проверить свойство 2) базы, заметим, что $B_{\de_1} \subset
B_{\de_2}$, если $0<\de_1\le\de_2$. Так как для
произвольного $T_\zeta\in B_{\de_1},
d(T_\zeta)<\de_1\le\de_2$, и, следовательно, $T_\zeta\in
B_{\de_2}$. Рассмотрим произвольное $B_{\de_1}$ и
$B_{\de_2}$, и, следовательно, $B_\de\subset
B_{\de_1}\cap B_{\de_2}$.

Система $\B=\hc{B_\de}$ образует базу на $\Ps$, которую обозначим
$\boxed{d(T)\ra 0}$.

\subsubsection{Интегральные суммы}

Рассмотрим на $[a,b]$ произвольную $f$ и произвольное размеченное
разбиение
$$T_\zeta\colon a=x_0<x_1<\dots<x_n=b, \quad \De_k=[x_{k-1},x_k], \quad k=\ol{1,n},\quad \De x_k=\hm{\De_k}, k=\ol{1,n}$$
и $\zeta=\hc{\zeta_1,\dots,\zeta_n}, \zeta_k\in\De_k$, $k=\ol{1,n}$.

Число
$\si(f,T_\zeta):$\eqa{1}{\si(f,T_\zeta)=\sum\limits_{k=1}^{n}
f(\zeta_k)\De x_k} --- интегральная сумма функции $f$, отвечающей
размеченному разбиению $T_\zeta$ отрезка $[a,b]$. Суммы~(1)
определяют на $\Ps$ отображение (функцию) $\Phi_f$ по правилу
\eqa{2}{\Phi_f\colon\Ps\ra\R{;}\;\Phi_f(T_\zeta)=\si(f,T_\zeta),\;T_\zeta\in\Ps
.}

\subsubsection{Определённый интеграл Римана}

\begin{dfn}{1}
Число $I\in\R$ --- определённый интеграл функции $f$, определённой
на $[a,b]$, если для произвольного
$\varepsilon>0\;\exi \de>0$ такое, что для произвольного
размеченного разбиения $T_\zeta$ отрезка $[a,b]$, имеющего
$d(T_\zeta)<\de$, справедливо неравенство

\eqa{3}{\hm{\si(f,T_\zeta)-I}=\hm{I-\sum\limits_{k=1}^{n}
f(\zeta_k)\De x_k}<\ep.}

\eqa{3'}{\hm{\Phi_f(T_\zeta)-I}=\hm{\si(f,T_\zeta)-I}<\ep,}
для $\fa T_\zeta\colon d(T_\zeta)<\de$, то есть $T_\zeta \in
B_\de$.

Неравенство~(3)$\Lra$(3').

\end{dfn}

\begin{dfn}{1$'$}
Число $I\in\R$ --- определённый интеграл Римана функции $f$,
определённой на $[a,b]$, если $I=\liml{d(T)\ra 0} \Phi_f$,
где $\Phi_f$ определена~(2).

Обозначение: $I=\intl{a}{b} f(x)\,dx$, где $a,b$ --- начальная и
концевая точки интеграла.

Определение 1 $\Lra$ определению 1$'$ (1$'$ --- лучше,
так как из него следует единственность определённого интеграла, если
он существует, поскольку интеграл --- предел некоторой функции по
некоторой базе).

\eqa{4}{\intl{a}{b} f(x)\,dx=\liml{d(T)\ra 0} \Phi_f =
\liml{d(T)\ra 0} \si(f,T_\zeta)=\liml{d(T)\ra 0}
\sum\limits_{k=1}^{n} f(\zeta_k)\De x_k.}
\end{dfn}

\begin{dfn}{2}
Множество всех функций $f$, имеющих определённый интеграл на
$[a,b]$, обозначается $\Rc[a,b]$, а сами функции называются
интегрируемыми (по Риману) на $[a,b]$.
\end{dfn}

Обозначение: $f\in \Rc[a,b]$.

\begin{stm*}
$\Rc[a,b]\ne\es$ для $\fa [a,b]$.
\end{stm*}
\begin{proof}
Рассмотрим произвольное $[a,b]$ и $f_{c}(x)=c, x\in[a,b], c\in\R$.
Для любого $T_\zeta$,
${x_0,\dots,x_n},\;\zeta=(\zeta_1,\dots,\zeta_n)$,
$\si(f_c;T_\zeta)=\sum\limits_{k=1}^{n} c \De
x_k=c\sum\limits_{k=1}^{n} \De x_k=c(b-a)=\intl{a}{b}
c\,dx.$
\end{proof}

\subsubsection{Критерий Коши существования интеграла Римана}

\begin{thh}
\begin{itshape}
Функция $f$, определённая на $[a,b]$, имеет $\intl{a}{b}
f(x)\,dx \Lra$ для
$\fa \ep>0\;\exi \de>0$ такое, что на элементе
$B_\de$ \boxed{d(T)\ra\de} колебание
$\om(\Phi_f;\;B_\de)<\ep$, то есть для любого
$\ep>0\;\exi \de>0$ такое, что для любых
$T'_{\zeta'}$ и $T''_{\zeta''}$, имеющих $d(T'_{\zeta'})<\de$,
$d(T''_{\zeta''})<\de$, выполнено условие
$$\hm{\Phi_\de(T'_{\zeta'})-\Phi_\de (T''_{\zeta''})}=\hm{\si(f;T'_{\zeta'})-\si(f;T''_{\zeta''})}<\ep$$
то есть
$$\hm{\sum\limits_{k=1}^{n'} f(\zeta'_{k})\;\De x'_k - \sum\limits_{k=1}^{n''} f(\zeta''_{k})\;\De x''_k}<\ep.$$
\end{itshape}
\end{thh}

\subsubsection{Необходимое условие существования определённого
интеграла}

\begin{theorem}
Всякая функция, интегрируемая на отрезке, ограничена на этом
отрезке.
\end{theorem}
\begin{proof}
(От противного) Пусть $\exi [a,b]$ и функция $f$ интегрируема и
неограничена на $[a,b]$. Тогда существует предел
$I=\liml{d(T)\ra0}\si(f;T_\zeta)$. Рассмотрим $\ep_0=1>0$. $\exi
\de_0>0\colon \hm{I-\si(f;T_\zeta)}<\ep_0=1$ для всех $T_\zeta\colon
d(T_\zeta)<\de_0 \Rightarrow$
\eqa{5}{\hm{\si(f;T_\zeta)}\le\hm{\si(f;T_\zeta)-I}+\hm{I}<\hm{I}+1
\;\fa  T_\zeta\colon d(T_\zeta)<\de_0.}

Рассмотрим $n\in\N, n>\frac{b-a}{\de_0}$ и $\widetilde T$ отрезка
$[a,b]$ на $\De_k=[x_{k-1},x_k],k=\ol{1,n}$, одинаковой
длины $\hm{\De_k}=\De x_k=\frac{b-a}n, k=\ol{1,n}$.
Тогда $d(\widetilde T)<\de_0$. Так как, по условию, $f$
неограничена на $[a,b]$, она будет неограничена на некотором $\De
x_l$, $1\le l\le n$. Рассмотрим $\zeta_k=x_{k-1}\in\De_k$, для
всех $k\in[1;n]$ таких, что $k\ne l$, и неполный набор
$\zeta^{*}={\zeta_1,\dots,\zeta_{l-1},\zeta_{l+1},\dots,\zeta_n}$ и
неполную сумму $\si^{*}=\sum\limits_{k=1,k\ne l}^{n}
f(\zeta_k)\De x_k$. Так как $f$ неограничена на $\De x_l$ и
$\De x_l>0$, то найдётся $\zeta_l\in\De x_l$, в которой
$\hm{f(\zeta_l)}\De x_l>\hm{\si^{*}}+\hm{I}+1$. Добавляя
$\zeta_l$ к $\zeta^{*}$, получим полный набор
$\zeta=\hc{\zeta_1,\dots,\zeta_{l-1},\zeta_l,\dots,\zeta_n}$ и
полную сумму $\si(f,\widetilde T_\zeta)=\sum\limits_{k=1}^n
f(\zeta_k)\De x_k$ с $d(\widetilde T_\zeta)=d(\widetilde
T)<\de_0$, для которого справедливо:
\eqa{6}{\hm{\si(f;\widetilde
T_\zeta)}=\hm{\si^{*}+f(\zeta_l)\De
x_l}\ge\hm{f(\zeta_l)\De
x_l}-\hm{\si^{*}}=\hm{f(\zeta_l)}\De x_l -
\hm{\si^{*}}>\hm{\si^{*}}+\hm{I}+1-\hm{\si^{*}}=\hm{I}+1,}
для $\widetilde T_\zeta, d(\widetilde T_\zeta)<\de_0$.

Неравенства~(5) и~(6) взаимоисключающие, так как~(5) справедливо для
всех $T_\zeta\colon d(T_\zeta)<\de_0$, а в~(6) для некоторого
$\widetilde T_\zeta$ получим противоречие.
\end{proof}

\subsubsection{Контрпример}

Свойство ограниченности подинтегральной функции не является
достаточным условием существования интеграла Римана.

{\bfseries Пример: } Функция Дирихле: $D(x)=\case{1, x\in\Q \\ 0,
x\in\R\wo\Q}$ неинтегрируема на $\fa [a,b]$.

\begin{proof}
Рассмотрим произвольное $T$ отрезка $[a,b]; T\colon
a=x_0<\dots<x_n=b$. По свойству плотности множеств $\Q$ и
$\R\wo\Q$ на
$\forall\De_k=[x_{k-1},x_k],k=\ol{1,n}$, существует
$\zeta^{'}_k\in\De_k\cap\Q$ и
$\zeta^{''}_k\in\De_k\cap(\R\wo\Q)$. Обозначим наборы
$\zeta'={\zeta^{'}_1,\dots,\zeta^{'}_n}$ и
$\zeta''={\zeta^{''}_1,\dots,\zeta^{''}_n}$ и рассмотрим
$T_{\zeta'}$ и $T_{\zeta''}$.

$\si(D;T_{\zeta'})=\sum\limits_{k=1}^{n} f(\zeta'_k)\De
x_k=b-a>0$ и $\si(D;T_{\zeta''})=\sum\limits_{k=1}^{n}
f(\zeta''_k)\De x_k=\sumkun 0\De_k=0$.

Итак, для произвольного $\de>0$ найдутся разбиения $T_{\zeta'}$ и $T_{\zeta''}$ такие, что
$d(T_{\zeta'})<\de$ и
$d(T_{\zeta''})<\de$ и $\si(D;T_{\zeta'})=b-a>0$ и
$\si(D;T_{\zeta''})=0$, так что $\om(D;B_{\de}) =
\hm{\si(D;T_{\zeta'})-\si(D;T_{\zeta''})}=b-a>0$. Таким
образом, для $D(x)$ не выполнен критерий Коши существования
интеграла.
\end{proof}

\subsubsection{Свойство линейности определённого интеграла}

\begin{theorem}
Если $f,g\in \Rc[a,b]$, то для $\fa \la_1,\la_2\in\R$
функция $(\la_{1}f+\la_{2}g)\in \Rc[a,b]$ и справедливо
\eqa{7}{\intl{a}{b}
(\la_{1}f(x)+\la_{2}g(x))\,dx=\la_{1}\intl{a}{b}
f(x)\,dx+\la_{2}\intl{a}{b} g(x)\,dx.}
\end{theorem}
\begin{proof}
Рассмотрим функции $\Phi_f\colon\Ps\ra\R$, $\Phi_g\colon\Ps\ra\R,
\Phi_{\la_{1}f+\la_{2}g}\colon\Ps\ra\R$. Тогда, по~(2),
\begin{multline*}
\Phi_{\la_{1}f+\la_{2}g}=\si(\la_{1}f+\la_{2}g;
T_\zeta)=\sum\limits_{k=1}^n
(\la_{1}f(\zeta_k)+\la_{2}g(\zeta_k))\De
x_k=\la_{1}\sum\limits_{k=1}^n f(\zeta_k)\De
x_k+\la_{2}\sum\limits_{k=1}^n g(\zeta_k)\De x_k=
\\
=\la_1\si(f;T_\zeta)+\la_2\si(g;T_\zeta)=
\la_1\Phi_f (T_\zeta)+\la_{2}\Phi_g (T_\zeta)
\;\;\;\;\;\;\;\;\;(8)
\end{multline*} для любого $T_\zeta$ отрезка $[a,b]$.

Так как $f,g\in \Rc[a,b]$, то $\exi \liml{d(T)\ra
0}{\Phi_f}=\intl{a}{b} f(x)\,dx$ и $\liml{d(T)\ra
0}\Phi_g=\intl{a}{b} g(x)\,dx$ и, в силу~(8) и свойств
линейности предела функции по базе, существует предел
$$\liml{d(T)\ra 0} \Phi_{\la_{1}f+\la_{2}g}=\intl{a}{b}(\la_{1}f(x)+\la_{2}g(x))\,dx$$
и справедливо~(7).
\end{proof}

\subsection{Критерии интегрируемости функций}

\subsubsection{Нижние и верхние суммы Дарбу}

\label{par::UpperLowerDarbouxSum}

Пусть функция $f$ определена и ограничена на $[a,b]$ (условие
ограниченности функции необходимо для её интегрируемости).
Рассмотрим произвольное $T\colon a=x_0<x_1<\dots<x_n=b$ и на каждом
$\De_k=[x_{k-1},x_k], k=\ol{1,n}$, рассмотрим числа
$m_k=\infl{x\in\De_k} f(x); M_k=\supl{x\in\De_k}
f(x)$ и обозначим: \eqa{1}{s(f;T)=\sum\limits_{k=1}^n m_k\De x_k
\mbox{\bfseries{ (нижняя) }}; S(f;T)=\sum\limits_{k=1}^n M_k\De
x_k \mbox{\bfseries{ (верхняя) }}, \De
x_k=\hm{\De_k},k=\ol{1,n}} --- нижняя и верхняя суммы Дарбу
функции $f$, порождённой разбиением $T$ отрезка $[a,b]$.

Так как $0\le M_k-m_k=\om(f;\De_k),k=\ol{1,n}$, то

\eqa{2}{S(f;T)-s(f;T)=\sum\limits_{k=1}^n \om(f;\De_k)\De
x_k.}

Так как для любого $\zeta_k\in\De_k$ справедливо $m_k\le
f(\zeta_k)\le M_k$, то из~(1), для любой интегральной суммы
$\si(f;T_\zeta), \zeta=(\zeta_1,\dots,\zeta_n), \zeta_k=\De_k,
k=\ol{1,n}$, справедливо \eqa{3}{s(f;T)\le\si(f;T_\zeta)\le
S(f;T)} для любых $T$ и $\zeta$.

\begin{thn}{(основное свойство сумм Дарбу)}
Если $f$ определена и ограничена на $[a,b]$, то для $\fa T$
отрезка $[a,b]$ верхняя (нижняя) сумма Дарбу $S(f;T) \; (s(f;T))$
равна точной верхней (точной нижней) грани множества интегральных
сумм $\hc{\si(f;T_\zeta)}$, в котором $T$ --- фиксировано и
меняются всевозможные наборы точек $\zeta$; т.е.
$S(f;T)=\supl{\zeta} \si(f;T_\zeta)$ и
$s(f;T)=\infl{\zeta} \si(f;T_\zeta)$.
\end{thn}
\begin{proof}
(Для верхней суммы). Рассмотрим произвольное $\ep>0$. По
(3), $\si(f;T_\zeta)\le S(f;T)$ для любых наборов $\zeta$. Так
как $M_k=\supl{x\in\De_k} f(x), k=\ol{1,n}$, то
$\exi \zeta'_k\in\De_k$, в котором $M_k - \frac{\ep}
{b-a}<f(\zeta'_k)\le M_k\Rightarrow$ получим набор
$\zeta'_k=(\zeta'_1,\dots,\zeta'_n)$ и размеченное разбиение
$T_{\zeta'}$, для которого
$$\si(f;T_{\zeta'})=\sum\limits_{k=1}^n f(\zeta'_k)\De x_k > \sum\limits_{k=1}^n \hr{M_k-\frac{\ep}{b-a}}\De x_k=S(f;T)-\ep$$
(так как $\sum\limits_{k=1}^n\De x_k=b-a$ и $\De x_k>0$).

Последнее неравенство вместе с $\si(f;T_{\zeta})\le S(f;T)$ для
всех $\zeta$ означают, что $S(f;T)=\supl{\zeta}
\si(f;T_{\zeta})$.
\end{proof}
Аналогично для нижней суммы.

\subsubsection{Свойство монотонности суммы Дарбу}

\label{par::DarbouxSumsProperty}

\begin{thh}
Если $f$ определена и ограничена на $[a,b]$ и разбиение $T_1$
отрезка $[a,b]$ получено из разбиения $T$ добавлением любого
конечного множества точек, то $s(f;T_1)\ge s(f;T)$ и $S(f;T_1)\le
S(f;T)$.
\end{thh}

\begin{proof}
Утверждение достаточно доказать для случая, когда к $T$ добавляется
единственная новая точка $\ol{x}$. В этом случае
$\exi k\in[1,n]\colon \ol{x}\in(x_{k-1},x_k)$ и
$\De_k=[x_{k-1},x_k]=[x_{k-1},\ol{x}]\cup[\ol{x},x_k]=\De_k
^1\cup \De_k ^2$.

К числам $m_k=\infl{x\in\De_k} f(x)$ и
$M_k=\supl{x\in\De_k} f(x)$ добавим числа
$m_k^i=\infl{x\in\De^i_k} f(x)$ и
$M_k^i=\supl{x\in\De^i_k} f(x)$. Тогда известно (из
первого семестра), что $m^i_k=\infl{x\in\De^i_k} f \ge
m_k, M^i_k\le M_k, i=1,2$. Поэтому, \ml{S(f;T_1)-S(f;T)=M^1_k
(\ol{x}-x_{k-1})+M^2_k (x_k-\ol{x})-M_k (x_k-x_{k-1})\le M_k
(\ol{x}-x_{k-1}) + M_k (x_k-\ol{x})-M_k (x_k -
x_{k-1})=0} и \ml{s(f;T_1)-s(f;T)=m^1_k (\ol{x}-x_{k-1})+m^2_k
(x_k-\ol{x})-m_k (x_k-x_{k-1})\ge m_k (\ol{x}-x_{k-1}) +
m_k (x_k-\ol{x})-m_k (x_k - x_{k-1})=0.}
\end{proof}

\begin{note}
Так как $M_k-M^i_k\le\om(f,[a,b])$, то
$$0<S(f,T)-S(f,T_1)=(M_k-M^1_k)(\ol{x}-x_{k-1})+(M_k-M^2_k)(x_k-\ol{x}) \le \om(f;[a,b])\De x_k\le \om(f;[a,b]) d(T).$$

Аналогично: $0\le s(f,T_1)-s(f,T)\le\om(f;[a,b]) d(T)$.

Оба неравенства справедливы в случае, когда в отрезке разбиения
$\De_k$ содержится несколько новых точек разбиения.
\end{note}

\subsubsection{Свойство отделимости множеств суммы Дарбу}

\begin{thh}
Если $f$ определена и ограничена на $[a,b]$, то для \textbf{любых}
разбиений $T_1$ и $T_2$ отрезка $[a,b]$ справедливо $s(f;T_1)\le
S(f;T_2)$.
\end{thh}

\begin{proof}
Рассмотрим $T=T_1\cup T_2$. По теореме пункта \ref{par::DarbouxSumsProperty} и неравенству~(3), имеем:

$s(f;T_1)\le s(f;T)\le S(f;T) \le S(f;T_2).$
\end{proof}

\subsubsection{Нижний и верхний интегралы Дарбу}

Так как множества $\hc{s(f;T)}$ и $\hc{S(f;T)}$ обладают свойством
отделимости, то, по принципу отделяющего отрезка,
$\exi \supl{T} \hc{s(f;T)}=\ul{I}$ и $\infl T
\hc{S(f;T)}=\ol{I}$ и справедливо
\eqa{4}{s(f;T)\le\ul{I}\le\ol{I}\le S(f;T)} для любого
разбиения $T$.

$\ul{I}$ --- нижний интеграл Дарбу функции $f$,
$\ol{I}$ --- верхний интеграл Дарбу функции $f$.

\subsubsection{Теорема Дарбу}

\label{par::DarbouxTheorem}

Рассмотрим на множестве $\Ps$ функции $\psi_f, \Psi_f$ вида
$\psi_f\colon\Ps\ra\R, \psi_f(T_\zeta)=s(f;T), T_{\zeta}\in\Ps$ и
$\Psi_f\colon\Ps\ra\R, \Psi_f(T_{\zeta})=S(f;T), T_{\zeta}\in\Ps$.
(Таким образом, $\psi_f, \Psi_f$ --- постоянные для каждого
фиксированного $T$ относительно всевозможного выбора наборов точек
$\zeta$). По неравенству~(3) справедливо \eqa{5}{\psi_f
(T_{\zeta})\le\Phi_f (T_{\zeta}) \le \Psi_f (T_{\zeta})} для любого
$T_{\zeta}\in\Ps$.

\begin{denote}

$$\liml{d(T)\ra 0}\psi_f=\liml{d(T)\ra 0} s(f;T),$$

$$\liml{d(T)\ra 0} \Psi_f=\liml{d(T)\ra 0} S(f;T),$$

$$\liml{d(T)\ra 0} \Phi_f=\liml{d(T)\ra 0} \si(f;T_{\zeta}).$$
\end{denote}

Следующую теорему называют теоремой Дарбу.

\begin{thh}
Если $f$ определена и ограничена на $[a,b]$, то $$\ul{I}=\liml{d(T)\ra 0}\psi_f, \quad \ol{I}=\liml{d(T)\ra 0} \Psi_f,$$
или же
$$\ul{I}=\liml{d(T)\ra 0} s(f;T), \quad \ol{I}=\liml{d(T)\ra 0} S(f;T).$$
\end{thh}

\begin{proof}
(Для $\ol{I}$). По неравенству~(4), $\ol{I}\le
S(f,T),\fa T$, и $\ol{I}=\infl T S(f;T)$.

Рассмотрим произвольное $\ep>0$. Тогда $\exi T_1$
отрезка $[a,b]$: \eqa{6}{\ol{I}\le S(f;T_1) <
\ol{I}+\frac{\ep}2 .}

Пусть $T_1$ имеет $m$ точек разбиения. Так как $f$ ограничена на
$[a,b]$, то $\om=\om(f;[a,b])<+\infty$ (конечно) и $\om\ge
0$. Если $\om=0$, то $f=\const$, т.е. $f(x)=c, x\in[a,b]$ и тогда
$m_k=M_k=c, k=\ol{1,n} \Rightarrow s(f;T)=S(f;T)=c(b-a)$ и
$\exi \liml{d(T)\ra 0} s(f,T)=\ul{I}=c(b-a)$ и
$\exi \liml{d(T)\ra 0} S(f;T)=\ol{I}=c(b-a)$. Пусть
$\om>0$ и $\de=\frac{\ep}{2m\om}>0$ ($m$ ---
множество точек разбиения $T_1$). Рассмотрим произвольное
$T_{\zeta}, d(T_{\zeta})<\de$, которое определяет разбиение $T,
d(T)=d(T_{\zeta})< \de$. Рассмотрим $T_2=T\cup T_1$. По
свойству монотонности верхних сумм: $S(f;T_2)\le S(f;T)$ и
$S(f;T_2)\le S(f;T_1)$. По замечанию теоремы пункта \ref{par::DarbouxSumsProperty},

\eqa{7}{0\le S(f;T)-S(f;T_2) \le \om md(T)<\om
m\de=\frac{\ep}2.}

По неравенствам~(6) и~(7) имеем:
$$\ol{I}\le S(f;T)\le S(f;T_2)+\frac{\ep}2\le S(f;T_1)+\frac{\ep}2<\ol{I}+\frac{\ep}2+\frac{\ep}2=
\ol{I}+\ep, \fa T_{\zeta}\colon d(T_{\zeta})<\de,$$
то есть
$$\hm{\ol{I}-S(f;T)} <\ep, \fa T_{\zeta}\colon d(T_{\zeta})<\de$$
или $\ol{I}=\liml{d(T)\ra 0} S(f;T)$.
\end{proof}

\subsubsection{Критерий Дарбу существования интеграла Римана}

\begin{thh}
$f$ определена и интегрируема на $[a,b]\; \Lra\; f$ ---
ограничена на $[a,b]$ и $\ul{I}=\ol{I}$, при этом
$\ul{I}=\ol{I}=I=\intl{a}{b} f(x)\,dx$.
\end{thh}

\begin{proof}
($\Leftarrow$) По условию, $f$ --- ограничена и
$\ul{I}=\ol{I}=I$. По теореме Дарбу (пункт \ref{par::DarbouxTheorem}),
$\ul{I}=\liml{d(T)\ra 0} \psi_f$ и
$\ol{I}=\liml{d(T)\ra 0} \Psi_f$. По неравенству~(5) и
оценочному признаку существования предела функции по базе,
существует предел $\liml{d(T)\ra 0} \Phi_f=\ul{I}=\ol{I}=I$ и $I=\intl{a}{b} f(x)\,dx$.

($\Rightarrow$) По условию, существует $\intl{a}{b} f(x)\,dx=I$.
Рассмотрим произвольное $\ep>0$. Так как
$I=\liml{d(T)\ra 0} \Phi_f=\liml{d(T)\ra 0}
\si(f;T_{\zeta})$, то $\exi \de>0\colon
\fa T_{\zeta}, d(T_{\ze})<\de$, справедливо:
\eqa{8}{I-\frac{\ep}3<\si(f;T_{\zeta})<I+\frac{\ep}3,
\fa T_{\zeta}, d(T_{\zeta})<\de.}

По неравенству~(8) и основному свойству сумм Дарбу, справедливо неравенство
$$I-\frac{\ep}3\le s(f;T)=\infl{\zeta}\si(f;T_{\zeta})\le \supl{\zeta} \si(f;T_{\zeta})=S(f;T)\le I+\frac{\ep}3,$$
откуда, с учётом~(4), получаем \eqa{9}{I-\frac{\ep}3 \le s(f;T) \le
\ul{I}\le\ol{I}\le S(f;T)\le I+\frac{\ep}3.}

Из~(9) следует, что $\hm{\ul{I}-I}\le
\frac{2\ep}3<\ep,
\hm{\ol{I}-I}\le\frac{2\ep}3<\ep$ и
$I=\ul{I}=\ol{I}$ в силу произвольности
$\ep>0$.
\end{proof}

\subsubsection{Критерий интегрируемости в терминах колебаний $f$}

\begin{thh}
$f\in \Rc[a,b]\;\Lra\;f$ --- ограничена на $[a,b]$ и
$$\liml{d(T)\ra 0}\hr{S(f;T)-s(f;T)}=0 \text{  или  } \liml{d(T)\ra 0}\sum\limits_{k=1}^n \om(f;\De_k) \De x_k=0,$$
то есть для любого $\ep>0$ найдётся $\de>0$ такое, что для произвольного разбиения $T$ отрезка $[a,b]$, у которого
$d(T)<\de$, справедливо неравенство
$$0\le S(f;T)-s(f;T)=\sum\limits_{k=1}^n \om(f;\De_k)\De x_k<\ep.$$
\end{thh}

\begin{proof}
  По~(2) пункта \ref{par::UpperLowerDarbouxSum}, $S(f;T)-s(f;T)=\sum\limits_{k=1}^n
  \om(f;\De_k)\De x_k$. По теореме Дарбу,
  $\liml{d(T)\ra 0}
  \hr{S(f;T)-s(f;T)}=\ol{I}-\ul{I}\ge0$. По критерию
  Дарбу, $f\in \Rc[a,b] \Lra f$ --- ограничена на $[a,b]$ и
  $\ol{I}=\ul{I}$.
\end{proof}

\subsubsection{Третий критерий интегрируемости функции}

\begin{thh}
$f\in \Rc[a,b]\Lra \fa \ep>0\;\exists$ разбиение $T$
отрезка $[a,b]$, для которого $S(f;T)-s(f;T)<\ep$.
\end{thh}

\begin{proof}
Так как множества $\hc{s(f;T)}$ и $\hc{S(f;T)}$ обладают свойствами
отделимости, то, по принципу отделяющего отрезка,
$\ul{I}=\ol{I}\Lra$ для любого
$\ep>0\;\exists$ разбиения $T_1$ и $T_2$ отрезка $[a,b]$, для
которых выполнено: $0\le S(f;T_1)-s(f;T_2)<\ep$. Тогда $T_1\cup
T_2=T$ --- некоторое разбиение отрезка $[a,b]$, для которого
$s(f;T)\ge s(f;T_2)$ и $S(f;T)\le S(f;T_1)$ и, следовательно, $0\le
S(f;T)-s(f;T)\le S(f;T_1)-s(f;T_2) < \ep$.

По формуле~(4): $s(f;T)\le \ul{I} \le \ol{I} \le
S(f;T)$ для любого $T$ отрезка $[a,b]$.
\end{proof}

\subsection{Классы интегрируемых функций}

\subsubsection{Интегрируемость функций}

\begin{thh}
Если $f\in \Cc[a,b]$, то $f\in \Rc[a,b]$.
\end{thh}
\begin{proof}
Так как $f\in \Cc[a,b]$, то, по теореме Вейерштрасса, $f$ ограничена
на $[a,b]$. Более того, $f$ --- равномерно непрерывна на $[a,b]$ и,
следовательно, для любого
$\ep>0\;\exi \de>0\colon\om(f;\De)<\frac{\ep}{b-a}$ для
любого $\De\subset[a,b], \hm{\De}<\de$. Рассмотрим
произвольное $T$ отрезка $[a,b]$ на $\De_k = [x_{k-1};x_k],
k=\ol{1,n}, a=x_0, x_n=b, d(T)<\de.$ Тогда $\hm{\De_k}\le
d(T) < \de,\;\forall k=\ol{1,n}$, и
$\om(f;\De_k)<\frac{\ep}{b-a}, k=\ol{1,n}$. Поэтому
$$\sum\limits_{k=1}^n \om(f;\De_k)\De x_k<\frac{\ep}{b-a}\sum\limits_{k=1}^n\De x_k=\frac{\ep}{b-a}(b-a)=\ep,
\quad \fa  T\colon d(T)<\de.$$
По второму критерию интегрируемости, $f\in \Rc[a,b]$.
\end{proof}

\subsubsection{Интегрируемость функций с конечным множеством
точек разрыва}

\begin{thh}
Всякая функция, определённая и ограниченная на $[a,b]$ и имеющая
только конечное число $l$ точек разрыва, интегрируема на $[a,b]$.
\end{thh}

\begin{proof}
(По индукции). Если $l=0$, то есть $f$ непрерывна на $[a,b]$, то
используем теорему предыдущего пункта.

Пусть теорема доказана для всех $l<n$, и $f$ удовлетворяет её
условиям для $l=n$. Рассмотрим сначала случай, когда $f$ имеет точку
разрыва $c$, внутреннюю для $[a,b], a<c<b$.

Так как $f$ ограничена на $[a,b]$, то $\exi C>0\colon
\hm{f(x)}\le C,\;\fa x\in[a,b]$. Рассмотрим произвольное
$\ep>0$ и точки $c_1\in(a,c), c_2\in(c,b)$, для которых
$c_2-c_1<\frac{\ep}{6C}$. Пусть $m=\infl{x\in[c_1,c_2]}f(x)$
и $M=\supl{x\in[c_1,c_2]}f(x)$. Тогда $M\le C$, а $m\ge -C
\Rightarrow M-m\le 2C$ и $(M-m)\cdot(c_2-c_1)<\frac{\ep}3$.

На отрезках $[a,c_1]$ и $[c_2,b]$ функция может иметь менее, чем $n$
точек разрыва, и, по предположению индукции, функция интегрируема на
каждом из этих отрезков.

$f\in \Rc[a,c_1]$ и $f\in \Rc[c_2,b]$. По третьему критерию
интегрируемости, $\exi T_1$ отрезка $[a,c_1]$ и $T_2$ отрезка
$[c_2,b]$, для которых $S(f;T_1)-s(f;T_1)<\frac{\ep}3$ и
$S(f;T_2)-s(f;T_2)<\frac{\ep}3$. Образуем разбиение $T$ отрезка
$[a,b]$ добавлением к отрезкам разбиений $T_1$ и $T_2$ отрезка
$[c_1,c_2]$, для которого
$$S(f;T)-s(f;T)=S(f;T_1)-s(f;T_1)+(M-m)(c_2-c_1)+S(f;T_2)-s(f;T_2)<\frac{\ep}3+ \frac{\ep}3+\frac{\ep}3=\ep.$$
По третьему критерию, $f\in \Rc[a,b]$.

Пусть $n=1$ и функция имеет единственную точку разрыва в одной из
концевых точек.

Пусть $x=a$ --- точка разрыва функции $f$. Рассмотрим
$c_2\in(a,b)\colon c_2-a<\frac{\ep}{4C}$. На $[c_2,b]$ функция
непрерывна $\Rightarrow f\in \Rc[c_2,b]$. Обозначим
$M=\supl{x\in[a,c_2]}f(x)$ и
$m=\infl{x\in[a,c_2]}f(x)$. Тогда
$(M-m)(c_2-a)<\frac{\ep}{4C}\cdot2C=\frac{\ep}2$. По третьему
критерию, $\exi T_2$ отрезка $[c_2,b]$, для которого
$S(f;T_2)-s(f;T_2)<\frac{\ep}2$. Образуем разбиение $T$ отрезка
$[a,b]$ добавлением к отрезку разбиения $T_2$ отрезка $[a,c_2]$, для
которого
$S(f;T)-s(f;T)=(M-m)(c_2-a)+S(f;T_2)-s(f;T_2)<\frac{\ep}2+\frac{\ep}2=\ep$.
По третьему критерию, $f\in \Rc[a,b]$.

Аналогично рассматривается случай, когда $x=b$ --- единственная
точка разрыва.
\end{proof}

\subsubsection{Применение теоремы пункта 3.2}

\begin{theorem}
Если $f$ определена на $[a,b]$ и $f(x)=0$ для любого $x\in[a,b]$ за
возможным исключением некоторого конечного множества $K$ на отрезке
$[a,b]$, то $\intl{a}{b} f(x)\,dx=0$.
\end{theorem}

\begin{proof}
По условию теоремы, $f$ ограничена на $[a,b]$ и, следовательно, по
теореме пункта 3.2 $\exi \intl{a}{b} f(x)\,dx=I$. Рассмотрим
произвольное $p\in\N$ и размеченное разбиение $T_{\ze}^{(p)}$
отрезка $[a,b]$ с $d(T_{\ze}^{(p)})<\frac1p$ с набором точек
$\zeta=(\ze_1,\dots,\ze_n)$, в которых все $\ze_k\notin K$, что
возможно в силу конечности множества $K$. Тогда
\equ{\si_p=\si(f;T_{\ze}^{(p)})=\sum\limits_{k=1}^n
f(\ze_k)\De x_k=\sum\limits_{k=1}^n 0\cdot\De
x_k=0,\;d(T_{\ze}^{(p)})<\frac1p, p\in\N.}

Так что $\liml{p\ra\infty} \si_p=0$. Так как
$\exi \liml{d(T)\ra 0} \si(f;T_{\ze})=I$, то $I=0$.
\end{proof}

\begin{theorem}
Если $f\in \Rc[a,b]$; $g$ --- определена и ограничена на $[a,b]$ и
$g(x)=f(x)$ для любого $x\in[a,b]$ за возможным исключением
некоторого конечного множества $K\subset[a,b]$, то $g\in \Rc[a,b]$ и
$\intl{a}{b} g(x)\,dx=\intl{a}{b} f(x)\,dx$.
\end{theorem}

\begin{proof}
$h=g-f$ ограничена на $[a,b]$ (так как $f\in \Rc[a,b]$ и,
следовательно, $f$ --- ограничена на $[a,b]$) и $h(x)=g(x)-f(x)=0$
для любого $x\in[a,b]\wo K$. По теореме 1, $\intl{a}{b}
h(x)\,dx=0 \; \Rightarrow \; \intl{a}{b}
g(x)\,dx=\intl{a}{b} f(x)\,dx$ (по линейному свойству).
\end{proof}

\subsubsection{Свойство монотонности определённого интеграла}

\begin{thh}
Если $f,g\in \Rc[a,b]$ и $f(x)\ge g(x)$, $x\in[a,b]$, то
$\intl{a}{b} f(x)\,dx \ge \intl{a}{b} g(x)\,dx$.
\end{thh}

\begin{proof}
Для любого размеченного разбиения $T_{\ze}$ отрезка $[a,b]$ на
$\De_k=[x_{k-1},x_k],k=\ol{1,n}, a=x_0, x_n=b$ и набора
$\ze=\hc{\ze_1,\dots,\ze_n},\ze_k\in\De_k,k=\ol{1,n}$,
справедливо: \equ{\si(f;T_{\ze})=\sum\limits_{k=1}^n f(\ze_k)\De
x_k \ge \sum\limits_{k=1}^n g(\ze_k)\De x_k=\si(g;T_{\ze}) \mbox{
(так как }\De x_k>0,k=\ol{1,n}),} или $\Phi_f(T_{\ze}) \ge
\Phi_g(T_{\ze}),\;\fa T_{\ze}\in\Ps$.

Так как $\intl{a}{b} f(x)\,dx=\liml{d(T)\ra
0}\Phi_f;\;\intl{a}{b} g(x)\,dx=\liml{d(T)\ra 0}\Phi_g$ и
предел по базе обладает свойством монотонности $\liml{d(T)\ra
0}\Phi_f \ge \liml{d(T)\ra 0}\Phi_g$, то $\intl{a}{b}
f(x)\,dx \ge \intl{a}{b} g(x)\,dx$.
\end{proof}

\subsubsection{Интегрируемость произведения функций}

\begin{lem}
Если $f\in \Rc[a,b]$, то $f^2\in \Rc[a,b]$.
\end{lem}

\begin{proof}
По условию, $\exi C>0\colon \hm{f(x)}\le C,\;\fa x\in[a,b]$
и, следовательно, $f^2$ ограничена на $[a,b]$. Для произвольных $x'$
и $x''\in[a,b]$ справедливо:
\eqa{1}{\hm{f^2(x')-f^2(x'')}=\hm{f(x')-f(x'')}\cdot\hm{f(x')+f(x'')}
\le \hr{\hm{f(x')}+\hm{f(x'')}}\cdot\hm{f(x')-f(x'')} \le 2C\cdot
\hm{f(x')-f(x'')}.}

Рассмотрим произвольное $T$ отрезка $[a,b]$ на отрезке
$\De_k=[x_{k-1}; x_k],k=\ol{1,n}$. По~(1):
\eqa{2}{\om(f^2;\De_k)\le 2C\cdot\om(f;\De_k),
k=\ol{1,n}\mbox{ и } 0\le\sum\limits_{k=1}^n
\om(f^2;\De_k) \De x_k \le \sum\limits_{k=1}^n
\om(f;\De_k)\cdot \De x_k.}

Так как $f\in \Rc[a,b]$, то, по второму критерию, $\liml{d(T)
\ra 0} \sum\limits_{k=1}^n \om(f;\De_k)\cdot \De x_k=0$,
откуда, с учётом~(2),

$\liml{d(T)\ra 0} \sum\limits_{k=1}^n
\om(f^2,\De_k)\cdot\De x_k=0\;\Rightarrow\;f^2\in \Rc[a,b]$
(по второму критерию).
\end{proof}

\begin{thh}
Если $f$ и $g \in \Rc[a,b]$, то $(f\cdot g)\in \Rc[a,b]$.
\end{thh}

\begin{proof}
Так как $f\cdot g=\frac14\hr{(f+g)^2-(f-g)^2}$ и $(f\pm g)\in
\Rc[a,b], (f\pm g)^2 \in \Rc[a,b]$ (по лемме), то $fg\in \Rc[a,b]$.
\end{proof}

\subsubsection{Интегрируемость монотонной функции}

\begin{thh}
Всякая монотонная функция, определённая на $[a,b]$, интегрируема на
$[a,b]$.
\end{thh}

\begin{proof}
Пусть, например, $f\uparrow$ на $[a,b]$. Тогда для произвольного $T$
отрезка $[a,b]$ на
$\De_k=[x_{k-1},x_k],k=\ol{1,n};a=x_0,x_n=b;$ на каждом
$\De_k,k=\ol{1,n}$ имеем: $M_k=f(x_k), m_k=f(x_{k-1}),
k=\ol{1,n}$ и, следовательно,
\ml{S(f;T)-s(f;T)=\sum\limits_{k=1}^n (M_k-m_k)\De x_k=
\sum\limits_{k=1}^n \hr{f(x_k)-f(x_{k-1})} \De x_k \le
\sum\limits_{k=1}^n \hr{f(x_k)-f(x_{k-1})}\cdot d(T)=\\=
\hs{f(x_1)-f(x_0) + f(x_2) - f(x_1) + \dots + f(x_n) - f(x_{n-1})}
d(T)=\hr{f(x_n)-f(x_0)} d(T)=\hr{f(b)-f(a)} d(T)} и $S(f;T)-s(f;T) <
\ep$, если $d(T)<\frac{\ep}{f(b)-f(a)+1}$.
\end{proof}

\subsubsection{Свойство аддитивности определённого интеграла}

\begin{thh}
Если $f\in\Rc[a,b]$ и $c, a<c<b$ --- произвольное, то $f\in\Rc[a,c]$
и $f\in\Rc[c,b]$. Обратно, если функция $f\in\Rc[a,c]$ и
$f\in\Rc[c,b], a<c<b$, то $f\in\Rc[a,b]$. В обоих случаях
справедливо \eqa{1}{\intl{a}{b} f(x)\,dx=\intl{a}{c}
f(x)\,dx + \int\limits_c^b f(x)\,dx.}
\end{thh}

\begin{proof}
Рассмотрим сначала случай, когда $f\in\Rc[a,b]$ и $a<c<b$ и проверим
справедливость третьего критерия интегрируемости функции на $[a,c]$
и $[c,b]$. Для этого рассмотрим произвольное $\ep>0$. Так как
$f\in\Rc[a,b]$, то по третьему критерию интегрируемости существует
такое разбиение $\widetilde T$ отрезка $[a,b]$, для которого
$S(f;\widetilde T)-s(f;\widetilde T)<\ep$. Добавляя точку $c$ к
точкам разбиения $\widetilde T$, получим новое разбиение $T$, для
которого $s(f;T)\ge s(f;\widetilde T)$ и $S(f;T)\le S(f;\widetilde
T)$ и, следовательно, $S(f;T)-s(f;T)\le S(f;\widetilde
T)-s(f;\widetilde T)<\ep$. Разбиение $T$ есть $T'\cup T''$
некоторого разбиения $T'$ отрезка $[a,c]$ и $T''$ отрезка $[c,b]$,
для которого справедлива формула
\eqa{2}{S(f;T)-s(f;T)=S(f;T')-s(f;T')+S(f;T'')-s(f;T'').}

Так как все разности в формуле~(2) неотрицательные и разность в
левой её части меньше $\ep$, то $S(f;T')-s(f;T')<\ep$ и
$S(f;T'')-s(f;T'')<\ep$. Так что по третьему критерию
интегрируемости, функция $f\in\Rc[a,c]$ и $f\in\Rc[c,b]$.

Пусть теперь $f\in\Rc[a,c]$ и $f\in\Rc[c,b]$ и число $\ep>0$ ---
произвольное. Согласно третьему критерию интегрируемости, существует
такое разбиение $T'$ отрезка $[a,c]$, что
$S(f;T')-s(f;T')<\frac{\ep}2$, и такое разбиение $T''$ отрезка
$[c,b]$, что $S(f;T'')-s(f;T'')<\frac{\ep}2$. Объединение $T'\cup
T''$ образует некоторое разбиение $T$ отрезка $[a,b]$, для которого
справедливо~(2). Поэтому,
$S(f;T)-s(f;T)<\frac{\ep}2+\frac{\ep}2=\ep$. Так что по третьему
критерию интегрируемости функция $f\in\Rc[a,b]$.

Чтобы доказать формулу~(1), обозначим $\intl{a}{b} f(x)\,dx=I,
\int\limits_a^c f(x)\,dx=I_1, \int\limits_c^b f(x)\,dx=I_2$ и
рассмотрим произвольное $\ep>0$. Согласно определению интеграла
Римана, существует такое $\de_1>0$, что
\eqa{3}{\hm{I-\si(f;T_{\ze})}<\frac{\ep}3} для всех размеченных
разбиений $T_{\ze}$ отрезка $[a,b]$ с $d(T_{\ze})<\de_1$. Существует
такое $\de_2>0$, что \eqa{4}{\hm{I_1-\si(f;T_{\ze'}')}<\frac{\ep}3}
для всех размеченных разбиений $T'_{\ze'}$ отрезка $[a,c]$ с
$d(T'_{\ze'})<\de_2$ и существует такое $\de_3>0$, что
\eqa{5}{\hm{I_2-\si(f;T_{\ze''}'')}<\frac{\ep}3} для всех
размеченных разбиений $T''_{\ze''}$ отрезка $[c,b]$ с
$d(T''_{\ze''})<\de_3$. Положим $\de=\min(\de_1,\de_2,\de_3),\de>0$
и рассмотрим такие разбиения $T'_{\ze'}$ с
$d(T'_{\ze'})<\de\le\de_2$ и размеченные разбиения $T''_{\ze''}$
отрезка $[c,b]$ с $d(T''_{\ze''})<\de\le\de_3$, для которых точка
$c$ не входит в наборы $\ze'$ и $\ze''$.

Тогда $T'_{\ze'}$ и $T''_{\ze''}$ образует некоторое размеченное
разбиение $T_{\ze}$ отрезка $[a,b]$ с $d(T_{\ze})<\de\le\de_1$, для
которого справедливы формулы $\si(f;T_{\ze})=\si(f;T'_{\ze'})+
\si(f;T''_{\ze''})$. Поэтому, с учётом формул~(3)---(5), имеем
\mla{6}{\hm{I-(I_1+I_2)}=\hm{I-\si(f;T_{\ze})-(I_1+I_2 - \si(f;T_{\ze}))} \le \\ \le
\hm{I-\si(f;T_{\ze})} + \hm{I_1 + I_2 - \si(f;T_{\ze})} =
\hm{I - \si(f;T_{\ze})} + \hm{I_1 - \si(f;T'_{\ze'}) + I_2 - \si(f;T''_{\ze''})} \le \\ \le
\hm{I-\si(f;T_{\ze})} + \hm{I_1 - \si(f;T'_{\ze'})} + \hm{I_2 - \si(f;T''_{\ze''})} < \frac{\ep}3 + \frac{\ep}3 + \frac{\ep}3 = \ep.}

В силу произвольного выбора числа $\ep$, число, стоящее в левой
части~(6), равно нулю, то есть $I=I_1+I_2$, что равносильно формуле
(1).
\end{proof}

\begin{note}
Формально можно рассмотреть случай, когда $a>b$ и в качестве
размеченного разбиения $T^{*}$ считать точки $a=x_0^* > x_1^* >
\dots > x_{k-1}^* > x_k^* > \dots > x_{n-1}^* > x_n^*=b$, где
$\De x_k^*=x_k^* - x_{k-1}^* < 0, k=\ol{1,n}$. В качестве
формального набора $\ze^*=\hr{\ze_1^*,\dots,\ze_n^*},x_{k-1}^* \ge
\ze_k^* \ge x_k^*,k=\ol{1,n}$, и формальное размеченное
разбиение $T_{\ze^*}^*$, которому отвечает формальная интегральная
сумма $\si^*(f; T^*_{\ze^*})=\sum\limits_{k=1}^n f(\ze^*_k) \De
x^*_k$. Если обозначить $l=n-k, x_l=x^*_{n-k}$, то $b=x_0<x_1<\dots
<x_l<x_{l-1}< \dots < x_{n-1} < x_n = a$; при этом $\ze^*_k=
\ze_{n-l},l=\ol {1,n}$, и $T_{\ze}$ --- размеченное разбиение
отрезка $[b,a]$ с интегральной суммой
$\si(f;T_{\ze})=\sum\limits_{k=1}^n f(\ze_l) \De x_l$. $\De
x_l=x_l-x_{l-1} = x^*_{n-k} - x^*_{n-k+1} = -\De x^*_{n-k} >
0,l=\ol{1,n}$. Таким образом, $\si(f;T^*_{\ze^*})= -
\si(f;T_{\ze})$.

Если функция $f\in\Rc[b,a]$, то $\intl{b}{a} f(x)\,dx= \liml{d(T)\ra 0} \si(f;T_{\ze})$. Поскольку предел функции линеен, то
существует предел
$$I^* = \liml{d(T)\ra 0} \si^*(f; T^*_{\ze^*}) = - \liml{d(T)\ra 0} \si(f;T_{\ze}) = -\intl{b}{a} f(x)\,dx.$$
Число $I^*$ обозначается символом $\intl{b}{a} f(x)\,dx, a>b$ и по определению
\eqa{7}{\intl{a}{b} f(x)\,dx = -\intl{b}{a} f(x)\,dx, a>b.}

Таким образом, $0=\intl{a}{b} f(x)\,dx + \intl{b}{a}
f(x)\,dx = \int\limits_a^a f(x)\,dx$.
\end{note}

\begin{thh}
Если функция $f$ интегрируема на наибольшем из отрезков с концевыми
точками $a,b,c$ (или $f$ интегрируема на двух отрезках, объединение
которых образует максимальный отрезок), то \eqa{8}{\int\limits_a^c
f(x)\,dx + \int\limits_c^b f(x)\,dx + \intl{b}{a} f(x)\,dx = 0.}
\end{thh}

\begin{proof}
Не ограничивая общности, считаем $a<c<b$. Согласно предыдущей
теореме,
$$\int\limits_a^c f(x)\,dx + \int\limits_c^b f(x)\,dx = \intl{a}{b} f(x)\,dx,$$
откуда
$$0 = \int\limits_a^c f(x)\,dx + \int\limits_c^b f(x)\,dx - \intl{a}{b} f(x)\,dx =
\int\limits_a^c f(x)\,dx + \int\limits_c^b f(x)\,dx + \intl{b}{a} f(x)\,dx,$$ что есть~(8).
\end{proof}

\begin{imp*}
Если функция $f\in\Rc[a,b]$, то $f\in\Rc[c,d]$ для любых $c,d,a\le
c<d\le b$.
\end{imp*}

\begin{proof}
$f\in\Rc[a,b]\;\Rightarrow\;f\in\Rc[c,b]\;\Rightarrow\;
f\in\Rc[c,d]$.
\end{proof}

\subsubsection{Оценка модуля определённого интеграла}

\begin{lemma}
Если функция $f\in\Rc[a,b]$, то $\hm{f}\in\Rc[a,b]$.
\end{lemma}

\begin{proof}
Так как $f\in\Rc[a,b]$, то $f$ ограничена на $[a,b]$ и $\exi C>0
\colon \hm{f(x)}\le C, x\in[a,b]$. Так как $\hm{\hm{f(x')} -
\hm{f(x'')}} \le \hm{f(x') - f(x'')}$ для всех $x',x''\in[a,b]$, то
для любого множества $E\subset [a,b]$ справедлива оценка
$\om(\hm{f},E)=\supl{x',x''\in E} \hm{\hm{f(x')} -
\hm{f(x'')}} \le \supl{x',x''\in E}
\hm{f(x')-f(x'')}=\om(f,E)$.

Поэтому для любого разбиения $T$ отрезка $[a,b]$ на отрезке
$\De_k=[x_{k-1},x_k],k=\ol{1,n},a=x_0,\dots,x_n=b$ $\sumkun
\om(\hm{f},\De_k) \De x_k \le \sum\limits_{k=1}^n \om(f;\De_k)
\De x_k (\De x_k>0,k=\ol{1,n})$.

Так как $f\in\Rc[a,b]$, то по второму критерию,
$\liml{d(T)\ra 0} \sum\limits_{k=1}^n \om(f,\De_k) \De
x_k=0$ и по свойству монотонности предела функции по базе
$\liml{d(T)\ra 0} \sum\limits_{k=1}^n \om(\hm{f},\De_k)
\De x_k=0$, так что $\hm{f}\in\Rc[a,b]$ (по второму критерию).
\end{proof}

\begin{note}
Функция $h(x)=\case{+1\mbox{, если }x\in\Q,\\-1\mbox{, если
}x\in\R\wo\Q.}$ не интегрируема ни на каком отрезке $[a,b]$ в
$\R$, в то время как $\hm{h(x)}=1,x\in\R$ интегрируема на любом
отрезке $[a,b]$. Кроме того, $h^2(x)=1$ и $h^2\in\Rc[a,b]$.
\end{note}

\begin{thh}
Если $f\in\Rc[a,b]$, то справедлива оценка
\eqa{9}{\hm{\intl{a}{b} f(x)\,dx} \le \intl{a}{b}
\hm{f(x)}\,dx.}
\end{thh}

\begin{proof}
Так как $-\hm{f(x)}\le f(x) \le \hm{f(x)}$ для любого $x\in[a,b]$,
то, согласно свойствам монотонности и линейности определённого
интеграла и лемме, справедливо $-\intl{a}{b} \hm{f(x)}\,dx \le
\intl{a}{b} f(x)\,dx \le \intl{a}{b} \hm{f(x)}\,dx$, что
равносильно~(9).
\end{proof}

\begin{imp*}
Если $f\in\Rc[a,b], \hm{f(x)}\le C$ для любого $x\in[a,b]$, то
\eqa{10}{\hm{\intl{a}{b} f(x)\,dx} \le C(b-a).}
\end{imp*}

\begin{proof}
Согласно~(9), $\hm{\intl{a}{b} f(x)\,dx} \le \intl{a}{b}
\hm{f(x)}\,dx \le \intl{a}{b} C\,dx$ (свойство монотонности)
$=C(b-a)$.
\end{proof}

\begin{note}
Оценки типа~(9) и (10) справедливы также для случая $a>b$, так как
$$\hm{\intl{a}{b} f(x)\,dx}=\hm{-\intl{b}{a} f(x)\,dx}=\hm{\intl{b}{a} f(x)\,dx}$$
и имеют вид \eqa{9'}{\hm{\intl{a}{b}
f(x)\,dx} \le \hm{\intl{a}{b} \hm{f(x)}\,dx},}
\eqa{10'}{\hm{\intl{a}{b} f(x)\,dx} \le C\hm{b-a}.}
\end{note}

\subsubsection{Первая теорема о среднем значении для неопределённого интеграла}

\begin{thh}
Если функции $f,g\in\Rc[a,b]$ и функция $g$ сохраняет знак на
$[a,b]$ (то есть $g(x)\ge 0,x\in[a,b]$, либо $g(x)\le 0,x\in[a,b]$)
и $m=\infl{x\in[a,b]} f(x),M=\supl{x\in[a,b]} f(x)$,
то существует некоторое $\mu\in[m;M],m\le\mu\le M$, что
\eqa{11}{\intl{a}{b} f(x) g(x)\,dx = \mu \intl{a}{b}
g(x)\,dx.} Если $f\in\Cc[a,b]$, то существует некоторое
$\xi\in[a,b]$, в которой \eqa{12}{\intl{a}{b} f(x) g(x)\,dx =
f(\xi) \intl{a}{b} g(x)\,dx.}
\end{thh}

\begin{proof}
Рассмотрим сначала случай, когда $g(x)\ge 0,x\in[a,b]$. Тогда
$m\le f(x)\le M$ и $mg(x)\le f(x)g(x)\le Mg(x)$ при $x\in[a,b]$
и по свойству монотонности и линейности интеграла
$$m\intl{a}{b} g(x)\,dx \le \intl{a}{b} f(x)g(x)\,dx \le M\intl{a}{b} g(x)\,dx.$$
Поскольку $\intl{a}{b} g(x)\,dx \ge \intl{a}{b} 0\,dx=0$, то
формула (11) в случае $\intl{a}{b} g(x)\,dx=0$ справедлива для
всех $\mu$.

Пусть $\intl{a}{b} g(x)\,dx \ne 0$ (то есть $\intl{a}{b}
g(x)\,dx >0$). Тогда
$$m\le \frac{\intl{a}{b} f(x)g(x)\,dx}{\intl{a}{b} g(x)\,dx} \le M \quad\text{ и }
\mu = \frac{\intl{a}{b} f(x)g(x)\,dx}{\intl{a}{b} g(x)\,dx}, m\le \mu \le M.$$
Если $g(x)\le 0$, то $-g(x)\ge 0,x\in [a,b]$ и по
доказанному $\intl{a}{b} f(x)(-g(x))\,dx= \mu\intl{a}{b}
(-g(x))\,dx$, что равносильно (11) в силу свойства линейности
интеграла.

Если $f\in\Cc[a,b]$, то $m=\infl{x\in[a,b]}
f(x)=\min\limits_{[a,b]} f(x)$, а $M=\supl{x\in[a,b]} f(x) =
\max\limits_{[a,b]} f(x)$ (по теореме Вейерштрасса). Так как $m\le
\mu \le M$, то по теореме Коши о промежуточных значениях непрерывной
функции на $[a,b],\exi \xi\in[a,b]$, в которой $\mu=f(\xi)$ и
формула (11) переходит в (12).
\end{proof}

\begin{imp*}
Если $f\in\Rc[a,b]$ и $m=\infl{x\in[a,b]}f(x),
M=\supl{x\in[a,b]} f(x)$, то существует $\mu\in[m,M]$, что
$\intl{a}{b} f(x)\,dx=\mu (b-a)$ $(11')$. Если $f\in\Cc[a,b]$,
то $\exi \xi\in[a,b]$, в котором \eqa{12'}{\intl{a}{b}
f(x)\,dx=f(\xi)(b-a).}
\end{imp*}

\begin{proof}
Полагая $g(x)=1,x\in[a,b]$, получим (11$'$) и (12$'$) из (11) и (12)
соответственно, так как
$$\intl{a}{b} g(x)\,dx = \intl{a}{b} 1\,dx = b-a.$$
\end{proof}

\subsection{Интеграл и производная}

\subsubsection{Непрерывность интеграла с переменным верхним пределом}

\label{par::ContIntegral}

Рассмотрим произвольную $f\in\Rc[a,b]$. По свойству аддитивности
интеграла, для произвольного $x\in[a,b]$ существует функция $F(x)$:
\eqa{1}{F(x)=\int\limits_a^x f(t)\,dt, x\in[a,b], F(a)=0} ---
\textbf{интеграл с переменным верхним пределом}.

\begin{thh}
Для произвольной $f\in\Rc[a,b]$ функция $F$, определяемая~(1),
непрерывна на $[a,b]$.
\end{thh}

\begin{proof}
Фиксируем $\forall x\in[a,b]$ и рассмотрим все $h, x+h\in[a,b]$. По
свойству аддитивности, \eqa{2}{F(x+h)-F(x)=\intl{a}{x+h}
f(t)\,dt - \int\limits_a^x f(t)\,dt = \int\limits_x^{x+h} f(t)\,dt.}

Так как $f\in\Rc[a,b]$, то $\exi  M>0\colon \hm{f(x)}\le M,
x\in[a,b]$, в силу оценки модуля интеграла,
\eqa{3}{\hm{\intl{x}{x+h} f(t)\,dt} \le
\hm{\intl{x}{x+h} \hm{f(t)}\,dt} \le \hm{\intl{x}{x+h}
M\,dt} = M\cdot \hm{h}.}

Так как $\liml{h\ra 0} M\hm{h}=0$, то на основании~(3) и~(2),
$\liml{h\ra 0} (F(x+h)-F(x))=0$ или $F(x) = \liml{h\ra
0} F(x+h), x\in[a,b]$, т.е. $F\in\Cc[a,b]$.
\end{proof}

\subsubsection{Дифференцируемость интеграла по верхнему пределу}

\begin{thh}
Если $f\in\Rc[a,b]$ и $f\in\Cc[a,b]$, то $F$, определяемая~(1),
дифференцируема в точке $x$, и
$$F'(x)=f(x), \quad \hr{\int\limits_a^x f(t)\,dt}' = f(x).$$
\end{thh}

\begin{proof}
Так как $\intl{x}{x+h} f(x)\,dt = f(x)(x+h-x)=f(x)\cdot h$, то
на основании~(2) и свойства линейности интеграла, получим \eqa{4}{
F(x+h) - F(x) = f(x)\cdot h - \intl{x}{x+h} f(x)\,dt +
\intl{x}{x+h} f(t)\,dt = f(x)\cdot h - \intl{x}{x+h}
(f(t)-f(x))\,dt,} откуда \equ{\frac{F(x+h)-F(x)}h = f(x) - \frac1h
\intl{x}{x+h} (f(t)-f(x))\,dt,h\ne 0.}

Так как $f$ непрерывна в точке $x$, то для любого $\ep>0\;\exists
\de>0\colon \hm{f(t)-f(x)} < \frac{\ep}2,\;\forall t\colon \hm{t-x}
\le \hm{h} < \de$ или $\hm{f(t)-f(x)}<\ep$ для $\forall
\hm{h}<\de;\;\hm{t-x} \le \hm{h} < \de$.

Согласно оценке модуля интеграла, \eqa{5} {\hm{\frac1h
\intl{x}{x+h} (f(t)-f(x))\,dt} \le \frac1{\hm{h}}
\hm{\intl{x}{x+h} \hm{f(t)-f(x)}\,dt} \le \frac{1}{\hm{h}}
\hm{\supl{\hm{t-x}\le\hm{h}} \hm{f(t)-f(x)} \hm{x+h-x}} =
\supl{\hm{t-x}\le\hm{h}} {f(t)-f(x)}<\ep} для $\fa
0<\hm{h} < \de\;\Rightarrow~(5)\Lra \liml{h\ra 0}
\frac1h \intl{x}{x+h} (f(t)-f(x))\,dt = 0$, тогда, на
основании~(4), $f(x)=\liml{h\ra 0} \frac{F(x+h)-F(x)}h =
F'(x)$.
\end{proof}

\begin{imp}
$f\in\Cc[a,b]\;\Rightarrow\;F$, определяемая~(1) --- точная
первообразная для $f$ на $[a,b]$.
\end{imp}

\begin{proof}
Так как $F\in\Cc[a,b]$, то по теореме пункта \ref{par::ContIntegral} (так как
$f\in\Rc[a,b]$), существует $F'(x)=f(x)$ для любого $x\in (a,b)$ по
предыдущей теореме.
\end{proof}

\begin{imp}
Если $f$ --- ограничена на $[a,b]$ и имеет только конечное множество
точек разрыва на $[a,b]$, то $F$, определяемая~(1), образует
первообразную функцию для $f$ с конечным исключительным множеством.
\end{imp}

\begin{proof}
По условию, $F$ непрерывна в каждой точке $x\in(a,b)$, за возможным
исключением некоторого конечного множества (по предыдущей теореме),
и $f\in\Rc[a,b]$. По теореме пункта 4.1, $f\in\Cc[a,b]$ и, по
предыдущей теореме, $F'(x)=f(x)$ для $\fa x\in(a,b)\wo K$.
\end{proof}

\subsubsection{Основная формула интегрального исчисления}

\begin{thh}
Если $f$ ограничена на $[a,b]$ и имеет только конечное множество
точек разрыва на $[a,b]$, то \eqa{6}{\intl{a}{b} f(x)\,dx =
\Phi(b)-\Phi(a),} где $\Phi(x)$ --- произвольная первообразная
функция для $f$ на $[a,b]$.
\end{thh}

\begin{proof}
По следствию 2, функция $F$, определяемая~(1), есть первообразная
для $f$ на $[a,b]$. Произвольная первообразная
$\Phi(x)=F(x)+c,\;x\in[a,b],\;c=\const$. Так как значение $F$ в
точке $a=0\colon F(a)=0$, то $\Phi(a)=c\;\Rightarrow$
\equ{\int\limits_a^x f(t)\,dt=\Phi(x)-\Phi(a),} для любого
$x\in[a,b]$. В частности, (6).
\end{proof}

(6) --- формула Ньютона--Лейбница.

\subsubsection{Интегрирование по частям в определённом интеграле}

\begin{thh}
Если $u,v\in\Cc[a,b];\;u',v'\in\Rc[a,b]$ (в концевых точках ---
односторонние производные), то \eqa{7}{\intl{a}{b}
u(x)v'(x)\,dx=(u(x)v(x))\biggl|_a^b - \intl{a}{b}
v(x)u'(x)\,dx,} где \equ{(u(x)v(x))\biggl|_a^b = u(b)v(b)-u(a)v(a).}
\end{thh}

\begin{proof}
Если $f\in\Cc[a,b]$, то \eqa{6'}{\intl{a}{b} f(x)\,dx = \Phi(b)
- \Phi(a),} для любой точной первообразной $\Phi(x)$ для $f(x)$ на
$[a,b]$.

Так как $u,v\in\Cc[a,b]$ и $u',v'\in\Rc[a,b]$, то оба интеграла в
(7) существуют. \eqa{8}{(uv)'=u'v+v'u,\;x\in(a,b).} Таким образом,
$u(x)v(x)$ --- точная первообразная для своей производной на $[a,b]$
и её представления~(8).

По (6'), \equ{\intl{a}{b} (u(x)v(x))'\,dx = u(b)v(b)-u(a)v(a) =
(u(x)v(x))\biggl|_a^b.} С другой стороны, \equ{(u(x)v(x))\biggl|_a^b
= \intl{a}{b} (u(x)v(x))'\,dx = \intl{a}{b}
(u(x)v'(x)+u'(x)v(x))\,dx \Lra (7) \mbox{ в силу свойства
линейности интеграла.}}
\end{proof}

\subsubsection{Замена переменной интегрирования в определённом
интеграле}

\begin{thh}
Пусть функция $x=\om(t)\in\Cc[\al,\be],\;\al<\be$ и имеет
производную $\om'(t)>0$ и $\om'(t)\in\Rc[\al,\be]$, так что образ
$\om([\al,\be])=[a,b],\;a=\om(\al),b=\om(\be)$. Если функция
$f\in\Cc[a,b]$, то \eqa{10}{\intl{a}{b} f(x)\,dx =
\intl{\al}{\be} (f(\om(t)))\cdot\om'(t)\,dt.}
\end{thh}

\begin{proof}
Сложная функция $f(\om(t))\in\Cc[\al,\be]$ как композиция
непрерывных функций, следовательно, $f(\om(t))\cdot\om'(t)$
интегрируема на $[\al,\be]$ как произведение интегрируемых функций.
Таким образом, оба интеграла из (10) определены. По (6'):
\equ{\intl{a}{b} f(x)\,dx = \Phi(b)-\Phi(a),} где $\Phi(x)$ ---
некоторая точная первообразная для функции $f$ на $[a,b]$, то есть
$\Phi\in\Cc[a,b]$ и $\Phi'(x)=f(x),x\in(a,b)$.

Сложная функция $F(t)=\Phi(\om(t))\in\Cc[\al,\be]$ как композиция
непрерывных функций и
$$F'(t)=\Phi'(x) \cdot \om'(t)=f(x)\cdot\om'(t)=f(\om(t))\cdot\om'(t),t\in(\al,\be),$$
таким образом, $F$ --- точная первообразная для $f(\om(t))\cdot\om'(t)$ на
$[\al,\be]$ и \equ{\intl{\al}{\be}
(f(\om(t)))\cdot\om'(t)\,dt = F(\be)-F(\al) =
\Phi(\om(\be))-\Phi(\om(\al)) = \Phi(b)-\Phi(a) = \intl{a}{b}
f(x)\,dx.}
\end{proof}

\subsubsection{Формула Тейлора с остаточным членом в интегральной
форме}

Пусть $f$ определена на невырожденном промежутке $I$ и имеет в $I$
производную (непрерывную) до порядка $n+1,n\in\N$ включительно и
$x,a\in I$; согласно (6')
\equ{\begin{aligned}
f(x)-f(a) &= \int\limits_a^x f'(t)\,dt = -\int\limits_a^x f'(t)(x-t)'\,dt = \\
&= -\left.\hr{f'(t)(x-t)}\right|_a^x + \int\limits_a^x f''(t)(x-t)\,dt = f'(a)(x-a) - \frac12 \int\limits_a^x f''(t)((x-t)^2)'\,dt = \\
&=\frac{f'(a)}{1!}(x-a) + \left.\hr{-\frac12 f''(t)(x-t)^2}\right|_a^x + \frac1{2!} \int\limits_a^x f'''(t)(x-t)^2\,dt = \\
&=\frac{f'(a)}{1!}(x-a) + \frac{f''(a)}{2!}(x-a)^2 - \frac1{2\cdot3} \int\limits_a^x f'''(t)((x-t)^3)'\,dt = \\
&=\frac{f'(a)}{1!}(x-a) + \frac{f''(a)}{2!}(x-a)^2 + \left.\hr{\frac1{3!}f''(t)(x-t)^3}\right|_a^x + \frac{1}{3!} \int\limits_a^x f^{(4)}(t)(x-t)^3\,dt = \\
&=\frac{f'(a)}{1!}(x-a) + \frac{f''(a)}{2!}(x-a)^2 + \frac{f'''(a)}{3!}(x-a)^3 + \frac1{3!} \int\limits_a^x f^{(4)}(t)(x-t)^3\,dt = \\
&=\frac{f'(a)}{1!}(x-a) + \frac{f''(a)}{2!}(x-a)^2 + \dots + \frac1{n!}\int\limits_a^x f^{(n+1)}(t)(x-t)^n\,dt.
\end{aligned}}

Итак, \equ{f(x)=f(a)+\sumkun \frac{f^{(k)}(a)}{k!}(x-a)^k +
\frac1{n!} \int\limits_a^x f^{(n+1)}(t)(x-t)^n\,dt.} Так как
$f^{(n+1)}\in\Cc(I)$, то по первой теореме о среднем значении
\equ{r_n(x,f,a) = \frac1{n!} \int\limits_a^x f^{(n+1)}(t)
(x-t)^n\,dt = \frac{1}{n!} f^{(n+1)}(\ze) \int\limits_a^x
(x-t)^n\,dt = \frac{f^{(n+1)}(\ze)}{(n+1)!}(x-a)^{n+1},} где
\equ{\ze=a+ \ta (x-a),\;0<\ta <1.}

\subsubsection{Вторая теорема о среднем значении}

\begin{thh}
Если $f\in\Rc[a,b]$, $g$ --- монотонна на $[a,b]$, то
\eqa{*}{\intl{a}{b} f(x)g(x)\,dx = g(a) \int\limits_a^{\xi}
f(x)\,dx + g(b) \int\limits_{\xi}^b f(x)\,dx,\;\xi\in[a,b].}
\end{thh}

\begin{proof}
Так как $g$ монотонна на $[a,b]$, то $g\in\Rc[a,b]$ и,
следовательно, существуют все три интеграла в (*), а
$g\in\Dc([a,b])$ и $g'\in\Rc[a,b]$. Тогда $g'$ сохраняет знак на
$(a,b)$. Так как, по предположению, $f\in\Cc[a,b]$, то функция
$F(x)$, такая, что \equ{F(x)=\int\limits_a^x
f(t)\,dt,\;x\in[a,b],\;F(a)=0,} также непрерывна на $[a,b]$ и
$F'(x)=f(x), x\in(a,b)$. Кроме того, по предположению,
$g'\in\Rc[a,b]$ и $g'$ сохраняет знак в $(a,b)$, так как по условию
$g$ --- монотонная функция.

По первой теореме о среднем значении для интеграла,
\eqa{1}{\intl{a}{b} F(x) g'(x)\,dx=F(\xi)\intl{a}{b}
g'(x)\,dx = F(\xi) [g(b)-g(a)] = [g(b) - g(a)] \int\limits_a^{\xi}
f(t) \, dt.}

На основании~(1) и свойства аддитивности интеграла, имеем
\ml{\intl{a}{b} f(x)g(x)\,dx = \intl{a}{b} g(x)F'(x)\,dx =
\hs{g(x)F(x)}_a^b - \intl{a}{b} F(x)g'(x)\,dx = \\ =
g(b)F(b)-f(a)F(a) - (g(b)-g(a)) \int\limits_a^{\xi} f(x)\,dx =
g(b)\hr{F(b)-\int\limits_a^{\xi} f(x)\,dx} + g(a)\int\limits_a^{\xi} f(x)\,dx = \\ =
g(a) \int\limits_a^{\xi} f(x)\,dx + g(b) \hr{\intl{a}{b} f(x)\,dx - \int\limits_a^{\xi} f(x)\,dx} =
g(a) \int\limits_a^{\xi} f(x)\,dx + g(b) \int\limits_{\xi}^b
f(x)\,dx.}
\end{proof}

\subsubsection{Формула суммирования Эйлера--Маклорена (слабая версия)}

Известно, что $[x]$ удовлетворяет неравенствам $[x]\le x < [x]+1,
x\in\R$ и $x-[x]=\{x\}$.

Рассмотрим функцию $\rho(x)=\frac12-\{x\},x\in\R$.

Функция $\rho(x)$ имеет разрывы только в $x\in\Z$ и $\forall
x\notin\Z\;\exi  \rho'(x)=-1$. Если $k\in\Z$, то
$\liml{x\ra k-0} \rho(x)=-\frac12$, $\liml{x\ra k+0}
\rho(x)=\frac12=\rho(k)$, то есть функция $\rho(x)$ непрерывна
справа $\forall k\in\Z$.

На произвольном $[a,b], a<b$ рассмотрим произвольную функцию $f$ и
образуем $F(x)$ по формуле $F(x)=\sum\limits_{a\le k \le x} f(k)$.
Доопределим $F(a)=0$, тогда $x\in[a,b]$. Для всех $\case{ x\ge a,\\
x<[a]+1}$ справедливо $F(x)=F(a)=0$.

Для $x=[a]+1=k_0-1\in\Z\colon F(k_0-1)=f(k_0-1)$ и
$F(x)=F(k_0-1)=f(k_0-1) \forall x\in[k_0-1,k_0)$. В точке $x=k_0\;
F(x)=F(k_0-1) + f(k_0) = F(k_0)$. Для всех $x\in[k_0,k_0+1)\;
F(x)=F(k_0)$.

Итак, $F(x)=F(k-1) \forall x\in[k-1,k);\; F(k)=F(k-1)+f(k)$ и
$F(x)=F(k) \forall x\in[k,k+1), k\in\Z$.

Для произвольного $k\in[a,b]\cap\Z$ справедливо
$\liml{x\ra k-0} F(x)=F(k-1), \liml{x\ra k+0}
F(x)=F(k)=F(k-1)+f(k)$.

Для всех $x\in(a,b)\wo\Z$ существует $F'(x)=0$.

\begin{thh}
Если функция $f$ имеет $f'\in\Rc[a,b]$, то
\eqa{1}{\sum\limits_{a<k\le b} f(k) = \intl{a}{b} f(x)\,dx +
\rho(b)f(b) - \rho(a)f(a) - \intl{a}{b} \rho(x)f'(x)\,dx.}
\end{thh}

\begin{proof}
Рассмотрим функцию $\Phi(x)=F(x)-\rho(x)f(x), x\in[a,b]$. Если
$x\in(a,b)\wo\Z$, то
$\Phi'(x)=F'(x)-\rho'(x)f(x)-f'(x)\rho(x)=f(x) - \rho(x)f'(x)$. И
$\Phi'$ не существует $\forall x\in(a,b)\cap \Z$, которые
образуют конечное множество.

Функция $\Phi(x)$ может иметь разрывы только в $x\in[a,b]\cap \Z$. Пусть $k\in(a,b)\cap \Z$.

\eqa{2}{\liml{x\ra k-0} \Phi(x)=\liml{x\ra k-0}F(x) -
\liml{x\ra k-0}\rho(x)f(x) = F(k-1) + \frac12 f(k);}

\eqa{3}{\liml{x\ra k+0} \Phi(x) = \liml{x\ra k+0} F(x)
- \liml{x\ra k+0} \rho(x)f(x) = F(k-1) + f(k) - \frac12 f(k)
= F(k-1) + \frac12 f(k).}

\equ{\begin{aligned}
\liml{x\ra k-0} \Phi(x) = \liml{x\ra k+0} \Phi(x) = F(k-1) + \frac12 f(k), \\
\Phi(x)=F(x) - \rho(x)f(x), \Phi(k) = F(k) - \rho(k)f(k) = F(k-1) + f(k) - \frac12 f(k) = F(k-1) + \frac12 f(k).
\end{aligned}}

Итак, $\liml{x\ra k-0} \Phi(x) = \liml{x\ra k+0}
\Phi(x) = \Phi(k)$. Если $x=k\in\Z$, то, согласно~(2), функция $\Phi(x)$
непрерывна справа в точке $x=k$.

Если $a\in\Z$, то, по определению, $F(x)=0, x\in[a,[x]+1], F(a)=0$.
Таким образом, $\liml{x\ra a+0} F(x) =0 = F(a)$.

$\liml{x\ra a+0} \Phi(x) = \liml{x\ra a+0} F(x) -
\liml{x\ra a+0} \rho(x)f(x) = 0 - \frac12 f(a) = -\frac12
f(a) = \Phi(a)$

$\hr{\Phi(a) = F(a)-\rho(a)f(a) = 0-\frac12 f(a) = -\frac12 f(a)}$.

Итак, $\Phi(x)$ непрерывна на $[a,b]$ и $\Phi'(x)=f(x)-\rho(x)f'(x)$
на $[a,b]$ с конечным исключительным множеством. Согласно формуле
Ньютона-Лейбница, \equ{\intl{a}{b} \hs{f(x)-\rho(x)f'(x)}\,dx =
\Phi(b) - \Phi(a) = F(b)-\rho(b)f(b) - F(a) + \rho(a)f(a) =
\sum\limits_{a<k\le b} f(k) - \rho(b)f(b) + \rho(a)f(a)
\Lra (1)} согласно свойству линейности интеграла.
\end{proof}

\subsection{Функции ограниченной вариации}

\subsubsection{Определение и обозначение}

Рассмотрим произвольную функцию $f$ на $[a,b]$, произвольное
разбиение $T$ отрезка $[a,b]$ точками $a=x_0<x_1<\dots<x_n=b$ и
число \equ{\bigvee(f,T)=\sum\limits_{k=1}^n \hm{f(x_k) - f(x_{k-1})}
\ge 0.}

Обозначим $\Ps_0$~--- множество всех разбиений отрезка $[a,b]$.

\begin{df*}
Функция $f$ называется функцией ограниченной вариации (изменения) на
$[a,b]$, если \\ $\exi \sup
\hc{\bigvee(f,T)\;|\;T\in\Ps_0}<+\infty$ и неограниченной
вариации~--- в противном случае.

Число $\bigvee\limits_a^b f = \sup\hc{\bigvee(f,T)\;|\;T\in\Ps_0}$
называется полной вариацией функции $f$ на $[a,b]$.
\end{df*}

\subsubsection{Лемма 1}

\begin{lemma}
Если функция $f$ возрастает на $[a,b]$, то она имеет ограниченную
вариацию на $[a,b]$ и $\bigvee\limits_a^b f = f(b)-f(a)$.
\end{lemma}

\begin{proof}
Для любого разбиения $T$, $a=x_0<x_1<\dots<x_n=b$ справедливо
$$\bigvee(f,T)=\sumkun\hm{f(x_k)-f(x_{k-1})}=\sumkun \hr{f(x_k)-f(x_{k-1})}=f(x_1) - f(x_0) + f(x_2) - f(x_1) + \dots +
f(x_n) - f(x_{n-1}).$$
Итак, $\hc{\bigvee(f,T)\;|\;T\in\Ps_0} =
\hc{f(b)-f(a)}$ и, следовательно, $\bigvee\limits_a^b f=f(b)-f(a)$.
\end{proof}

\subsubsection{Лемма 2}

\begin{lemma}
Если функции $f$ и $g$ возрастают на $[a,b]$, то их разность $h=f-g$
имеет ограниченную вариацию на $[a,b]$.
\end{lemma}

\begin{proof}
Для произвольного разбиения $T\colon a=x_0<\dots<x_n=b$ справедливо
\ml{\bigvee(h,T)=\sumkun\hm{h(x_k)-h(x_{k-1})}=\sumkun
\hm{f(x_k)-g(x_k) - f(x_{k-1}) + g(x_{k-1})} \le \sumkun \hm{f(x_k)
- f(x_{k-1})} +\\+ \sumkun\hm{g(x_k) - g(x_{k-1})} =
\bigvee(f,T)+\bigvee(g,T)\le \bigvee\limits_a^b f +
\bigvee\limits_a^b g.}

Таким образом, $\hc{\bigvee(h;T)\;|\;T\in\Ps_0}$ ограничена сверху
числом $\bigvee\limits_a^b f+\bigvee\limits_a^b g$, так что
$\bigvee\limits_a^b h \le \bigvee\limits_a^b f+\bigvee\limits_a^b
g$.
\end{proof}

\subsubsection{Лемма 3}

\begin{lemma}
Если функция $f$ имеет ограниченную вариацию на $[a,b]$, то $\forall
c; a<c<b$, функция $f$ имеет ограниченную вариацию на $[a,c]$ и
$\bigvee\limits_a^c f \le \bigvee\limits_a^b f - \hm{f(b)-f(c)}$.
\end{lemma}

\begin{proof}
Для любого $T$, $a=x_0<\dots<x_{n-1}=c$ отрезка $[a,c]$ рассмотрим
разбиение $\widetilde T$ отрезка $[a,b]$, получившегося из $T$
добавлением точки $x_n=b$. Тогда \equ{\bigvee(f,\widetilde T) =
\sumkun \hm{f(x_k) - f(x_{k-1})} = \sum\limits_{k=1}^{n-1}
\hm{f(x_k) - f(x_{k-1})} + \hm{f(x_n) - f(x_{n-1})} = \bigvee(f,T) +
\hm{f(b)-f(c)} \le \bigvee\limits_a^b f.}

Таким образом, $\hc{\bigvee(f,T)\;|\;T\in\Ps_0}$ ограничено сверху
числом $\bigvee\limits_a^b f - \hm{f(b)-f(c)}$, так что
$$\bigvee\limits_a^c f \le \bigvee\limits_a^b f - \hm{f(b)-f(c)}.$$
\end{proof}

\subsubsection{Леммы 4, 5}

Согласно лемме 3, $\forall x$, $a\le x\le b$ определена функция
$P(x)=\bigvee\limits_a^x f \ge 0$. Положим $P(a)=0$.

\begin{lemma}
Функция $P(x)$ возрастает на $[a,b]$.
\end{lemma}

\begin{proof}
Рассмотрим произвольные $a\le x_1 < x_2 \le b$. Согласно лемме 3,
\equ{P(x_1)=\bigvee\limits_a^{x_1} f \le \bigvee\limits_a^{x_2} f -
\hm{f(x_2) - f(x_1)} \le \bigvee\limits_a^{x_2} f = P(x_2).}
\end{proof}

\begin{lemma}
Функция $N(x)=P(x)-f(x)$ возрастает на $[a,b]$.
\end{lemma}

\begin{proof}
Согласно леммам 3 и 4, \equ{N(x_2)-N(x_1) = P(x_2) - P(x_1) -
\hs{f(x_2) - f(x_1)} \ge \hm{f(x_2) - f(x_1)} - \hs{f(x_2)-f(x_1)}
\ge 0.}
\end{proof}

\subsubsection{Основная теорема}

\begin{thh}
Для того, чтобы функция $f$ была ограниченной вариации на $[a,b]$,
необходимо и достаточно, чтобы $f$ представлялась в виде
возрастающих на $[a,b]$ функций.
\end{thh}

\begin{proof}
Прямое следствие леммы 1 (достаточность) и лемм 4, 5.
\end{proof}

\subsubsection{Примеры}

\begin{thh}
Всякая функция, удовлетворяющая условиям Липшица на отрезке, имеет
на этом отрезке ограниченную вариацию.
\end{thh}

\begin{proof}
Пусть функция $f$ определена на $[a,b]$ и $\exi M>0$, что
$\hm{f(x')-f(x'')} \le M\hm{x'-x''} \;\forall x',x''\in [a,b]$.
тогда $\forall T$, $a=x_0<x_1<\dots<x_n=b$ справедливо
\equ{\bigvee(f;T) = \sumkun \hm{f(x_k) - f(x_{k-1})} \le M \sumkun
\hm{x_k - x_{k-1}} = M\sumkun (x_k - x_{k-1}) = M(b-a).}
\end{proof}

\subsection{Приложение к определённому интегралу}
\subsubsection{Площадь криволинейной трапеции}

Рассмотрим $f\in\Rc[a,b],\;a<b$ и $f(x)>0,\;x\in[a,b]$. На
$\Pi\colon Oxy$ фигура, ограниченная графиком $\Ga_f,\;y=f(x), \;
Ox, \;x=a, \;x=b$ --- \textbf{криволинейная трапеция (подграфик $f$
на $[a,b]$)}.

Рассмотрим произвольное $T[a,b]\colon a=x_0<\dots<x_n=b$ и
$\De_k=[x_{k-1},x_k],\;k=\ol{1,n}$. Для любых
$k=\ol{1,n}\;\exi m_k=\infl{\De_k}f$ и
$M_k=\supl{\De_k}f\colon 0<m_k\le M_k$. Число $m_k\cdot\De
x_k$ равно площади $R_k$, где $R_k$ --- прямоугольник с основанием
$\De_k$, высотой $m_k$, вписанным в подграфик функции $f$ на
$\De_k$.

$R'(T) = \cupl{k=1}n R_k'$ --- ступенчатая прямоугольная
фигура, вписанная в подграфик $f$ на $[a,b]$.

Площадь $R'(T) = \sumkun m_k\De x_k=s(f;T)$.

Аналогично, число $M_k\De x_k$ равно площади $R_k''$ ---
прямоугольника с основанием $\De x_k$ и высотой $M_k$, которая
описывает подграфик $f$ на $\De_k$.

$R''(T)=\cupl{k=1}n R_k''$ --- ступенчатая прямоугольная
фигура, описанная подграфиком $f$ на $[a,b]$.

Площадь $R''(T)=\sumkun M_k\cdot\De x_k=S(f;T)$.

Если $T'$ получено из $T$ добавлением конечного множества точек
(условно, $T'\ge T$), то, по свойству монотонности суммы Дарбу,
$s(f,T')\ge s(f,T)$, а $S(f,T')\le S(f,T)$. Так как $f\in\Rc[a,b]$,
то для $\forall\ep>0\;\exi T_{\ep}\colon S(f,T)-s(f,T)<\ep$ для
всех $T\ge T_{\ep}$. По теореме Дарбу, \equ{\liml{d(T)\ra 0}
s(f,T) = \liml{d(T)\ra 0} S(f,T) = \intl{a}{b} f(x)\,dx =
I = \mbox{ площади подграфика $\Pi_f$.}}

$R(T)=R''(T)\wo R'(T)$, площадь $R(T) = \mbox{ площади }
R''(T) - \mbox { площадь } R'(T) = S(f,T)-s(f,T)<\ep$.

\subsubsection{Плоские кривые}

\begin{dfn}{1}
\textbf{Плоская кривая} $\Zc$ --- график функции, заданной
параметрически $\case{x=\ph(t),\\y=\psi(t).}$

$t\in[\al,\be],\al<\be$ и $\ph(t)$ и $\psi(t) \in\Cc[\al,\be]$.

$\forall t\in[\al,\be]$, соответственно, $P(\ph(t),\psi(t))\in\Zc$.

Точка $P\in\Zc$ --- \textbf{двойная точка}, если $\exi t_1,t_2
\in [\al,\be],\;t_1\ne t_2$ и $\ph(t_1)=\ph(t_2)$ и
$\psi(t_1)=\psi(t_2)$.
\end{dfn}

\begin{dfn}{2}
\textbf{Простая кривая (Жорданова) $\Zc$} --- если она не содержит
двойных точек, за возможным исключением значений $t=\al$ и $t=\be$.
Если $A(\ph(t),\psi(t))=B(\ph(t),\psi(t))$, то $\Zc$ ---
\textbf{замкнутая жорданова кривая} (например --- окружность). Иначе
--- \textbf{незамкнутая}.
\end{dfn}

\subsubsection{Спрямляемые кривые}

Рассмотрим жорданову кривую $\Zc\colon \case{x=\ph(t),\\y=\psi(t)}
t\in[\al,\be]$ и произвольное разбиение $T[\al,\be]\colon
\al=t_0<\dots<t_n=\be$. На $\Zc$ заданы точки
$P_k=P_k(\ph(t_k),\psi(t_k)),\;k=\ol{0,n}$ и отрезки
$[P_{k-1},P_k],\;k=\ol{1,n}$, образуем ломаную $\La(T)=P_0P_1
\dots P_{k-1}P_k\dots P_{n-1}P_n$. Длина этой ломаной \eqa{1}{l(\La)
= \sumkun \hm{P_{k-1}P_k} = \sumkun \sqrt{(\ph(t_k)-\ph(t_{k-1}))^2
+ (\psi(t_k) - \psi(t_{k-1}))^2}.}

\begin{lemma}
Если $T'\ge T$, то $l(\La(T'))\ge l(\La(T))$.
\end{lemma}

\begin{proof}
Утверждение достаточно проверить для случая, когда $T'$ получено
добавлением одной точки $t'\in(\al,\be)$. В этом случае
$\exi k\colon 1\le k\le n$ и $t'\in(t_{k-1},t_k),\;t_{k-1}<t'
<t_k$. Тогда, по~(1), \equ{l(\La(T'))-l(\La(T)) = \hm{P_{k-1}P'} +
\hm{P'P_k} - \hm{P_{k-1}P_k},} где $P'=P'(\ph(t'),\psi(t'))$. Но в
$\De P'P_{k-1}P_k$ справедливо: \equ{\hm{P'P_{k-1}} + \hm{P'P_k} \ge
\hm{P_{k-1}P_k} \;\Rightarrow\; l(\La(T')) \ge l(\La(T)).}
\end{proof}

\begin{dfn}{3}
Кривая $\Zc$ --- \textbf{спрямляемая} (имеющая длину), если числовое
множество длин ломаных $\hc{l(\La(T))\;|\;T\in\Ps_0}$ ограничено
сверху ($\Ps_0$ --- множество всех разбиений $T$). $\sup
\hc{l(\La(T))\;|\;T\in\Ps_0} = l(\Zc)$ --- \textbf{длина кривой
$\Zc$}.
\end{dfn}

\begin{stm}
Если $\Zc$ спрямляема, то её длина $l(\Zc) = \sup \hc{l(\La(T))\;
|\; \fa  T\ge T_0,\;\exi T_0\in\Ps_0}$.
\end{stm}

\begin{proof}
Следствие леммы 1.
\end{proof}

\subsubsection{Критерий спрямляемости кривой}

\begin{thh}
(Жордана) Кривая $\Zc$, задаваемая $\case{x=\ph(t)\\y=\psi(t)}
t\in[\al,\be]$ спрямляема тогда и только тогда, когда $\ph,\psi$
имеют ограниченную вариацию на $[\al,\be]$.
\end{thh}

\begin{proof}
\textbf{Необходимость.} \eqa{2}{\max(\hm{a},\hm{b}) \le
\sqrt{a^2+b^2} \le \sqrt{(\hm{a}+\hm{b})^2} = \hm{a}+\hm{b}}

По~(2) и~(1),
\ml{\max\hr{\hm{\ph(t_k)-\ph(t_{k-1})}; \hm{\psi(t_k) - \psi(t_{k-1})}} \le \sqrt{\hr{\ph(t_k)-\ph(t_{k-1})}^2 +
\hr{\psi(t_k) - \psi(t_{k-1})}^2} \le \\ \le
\hm{\ph(t_k)-\ph(t_{k-1})} + \hm{\psi(t_k) - \psi(t_{k-1})},\quad k=\ol{1,n}}
откуда
\ml{\max\hr{\bigvee(\ph,T);\bigvee(\psi,T)} =
\max\hr{\sumkun\hm{\ph(t_k) - \ph(t_{k-1})};\; \sumkun \hm{\psi(t_k)
- \psi(t_{k-1})}} \le \\ \le l\hr{\La(T)} \le \sumkun \hm{\ph(t_k) -
\ph(t_{k-1})} + \sumkun \hm{\psi(t_k) - \psi(t_{k-1})} =
\bigvee(\ph;T) + \bigvee(\psi;T).}

Если кривая $\Zc$ спрямляема, то $l\hr{\La(T)}\le l(\Zc)$ для
$\forall T\in\Ps_0$ и, следовательно,
$$\bigvee(\ph;T) \le l(\Zc), \quad \bigvee(\psi,T) \le l(\Zc), \forall T\in\Ps_0,$$
так что
$\bigvee\limits_{\al}^{\be} \ph \le l(\Zc),
\bigvee\limits_{\al}^{\be} \psi \le l(\Zc)$, то есть функции имеют
ограниченную вариацию.

\textbf{Достаточность.}

Если $\bigvee\limits_{\al}^{\be} \ph < +\infty,
\;\bigvee\limits_{\al}^{\be} \psi < +\infty$, то $\bigvee(\ph,T) \le
\bigvee\limits_{\al}^{\be} \ph;\; \bigvee(\psi,T) \le
\bigvee\limits_{\al}^{\be} \psi$ и, согласно~(3), $l\hr{\La(T)} \le
\bigvee\limits_{\al}^{\be} \ph + \bigvee\limits_{\al}^{\be} \psi$
для $\forall T\in\Ps_0$, то есть $\Zc$ --- спрямляема по определению
3.
\end{proof}

\subsubsection{Вычисление длины кривой}

\begin{dfn}{4}
Функция $f$, имеющая на $[a,b]$ непрерывную производную $f'$,
относится к классу $\Cc^1[a,b]$.
\end{dfn}

\begin{dfn}{5}
$\Zc$ --- кривая класса $\Cc^1$, если её параметрические функции
$\ph(t)$ и $\psi(t)\in\Cc^1[\al,\be]$.
\end{dfn}

\begin{thh}
$\forall \Zc\colon \case{x=\ph(t),\\y=\psi(t)} t\in[\al,\be]$ класса
$\Cc^1$ спрямляема и её длина \eqa{4}{l(\Zc) =
\intl{\al}{\be} \sqrt{\ph'^2(t) + \psi'^2(t)}\,dt.}
\end{thh}

\begin{proof}
1) Интеграл $I$ в правой части~(4) существует, так как
$\ph',\psi'\in\Cc[\al,\be]$. Кроме того, $\exi M>0\colon
\hm{\ph'(t)}\le M$, $\hm{\psi'(t)} \le M, t\in[\al,\be]$. По теореме
Лагранжа, $\hm{\ph(t_1) - \ph(t_2)} =
\hm{\ph'(\xi)}\cdot\hm{t_1-t_2} \le M\hm{t_1-t_2}$ и $\hm{\psi(t_1)
- \psi(t_2)} = \hm{\psi'(\xi)}\cdot\hm{t_1-t_2} \le M\hm{t_1-t_2}
\fa t_1,t_2\in[\al,\be]$. Так что $\ph$ и $\psi$ принадлежат
классу Липшица на $[\al,\be]$ и, следовательно, имеют ограниченные
вариации на $[\al,\be]$.

По теореме Жордана, $\Zc$ спрямляема, то есть $\exi l(\Zc)$.

2) \begin{lemma} \eqa{2}{\sqrt{a^2+b^2} - \sqrt{c^2+d^2} \le
\sqrt{(a-c)^2 + (b-d)^2} \le \hm{a-c} + \hm{b-d}
\;\fa a,b,c,d\in\R.}
\end{lemma}

\begin{proof}
На $\Pi\colon Oxy$ рассмотрим точки $A(a,b)$ и $B(c,d)$. Тогда
$\hm{OA}=\sqrt{a^2+b^2},\hm{OB}=\sqrt{c^2+d^2}$ и
$\hm{AB}=\sqrt{(a-c)^2 + (b-d)^2}$. В $\De OAB\colon \hm{AB} \ge
\hm{\hm{OA} - \hm{OB}} \Lra (5)$ (с учётом~(2)).
\end{proof}

Рассмотрим произвольное разбиение $T[\al,\be] \colon \al=t_0<\dots <
t_n = \be ; \; \De_k = [t_{k-1},t_k],\;k=\ol{1,n}$.
Обозначим $a_k=\infl{\De_k} \hm{\ph'},b_k =
\infl{\De_k} \hm{\psi'},\;c_k = \supl{\De_k}
\hm{\ph'}$ и $d_k = \supl{\De_k}
\hm{\psi'},\;k=\ol{1,n}$.

По теореме Лагранжа, $\hm{\ph(t_k) - \ph(t_{k-1})} =
\hm{\ph'(\xi_k)} \cdot \hm{t_k - t_{k-1}} = \hm{\ph'(\xi_k)} \De
t_k$, $\hm{\psi(t_k) - \psi(t_{k-1})} = \hm{\psi'(\ze_k)} \cdot
\De t_k,\;k=\ol{1,n}$.

Тогда \eqa{6}{l(\La(T)) = \sumkun \sqrt{\hm{\ph'(\xi_k)}^2 +
\hm{\psi'(\ze_k)}^2} \cdot \De t_k.}

Так как $a_k \le \hm{\ph'(\xi_k)} \le c_k$ и $b_k \le
\hm{\psi'(\ze_k)} \le d_k,\;\fa  t\in
[t_{k-1},t_k]=\De_k,\;k=\ol{1,n}$. Откуда $\sqrt{ a_k^2 +
b_k ^ 2} \le \hm{\ph'^2 (t) + \psi'^2 (t)} \le \sqrt{c_k^2 +
d_k^2},\; k=\ol{1,n}$.

Интегрируя по $\De_k$, получим \eqa{8}{\sqrt{a_k^2 + b_k^2}
\De t_k \le \intl{t_{k-1}}{t_k} \sqrt{ \ph'^2 (t) +
\psi'^2 (t)}\,dt \le \sqrt{c_k^2 + d_k^2} \cdot \De
t_k,\;k=\ol{1,n}.}

Суммируя~(8) по $k=\ol{1,n}$, получим:

\eqa{9}{\sumkun \sqrt{a_k^2 + b_k^2} \De t_k \le \sumkun
\intl{t_{k-1}}{t_k} \sqrt{\ph'^2 (t) + \psi'^2 (t)}\,dt =
\intl{\al}{\be} \sqrt{\ph'^2 (t) + \psi'^2 (t)}\,dt = I \le \sumkun
\sqrt{c_k^2 + d_k^2} \De t_k.}

На основании~(7) и~(9), имеем оценку~(5) \ml{\hm{l(\La(T)) - I} =
\sumkun \sqrt{c_k^2 + d_k^2} \cdot \De t_k - \sumkun \sqrt{a_k^2
+ b_k^2} \cdot \De t_k \le \sumkun (c_k-a_k) \cdot \De t_k +
\sumkun (d_k-b_k)\De t_k = \\ = S(\hm{\ph'};T) - s(\hm{\ph'};T) +
S(\hm{\psi'};T) - s(\hm{\psi'};T). \mbox{ (10)}}

Так как $\hm{\ph'},\hm{\psi'}\in\Rc[a,b]$, то для
$\fa \ep>0\;\exists T_{\ep} \in \Ps_0$, что правая часть в (10)
$<\ep$ для $T=T_{\ep}$ (по третьему критерию интегрируемости) и она,
тем более, $<\ep$ для всех разбиений $T\ge T_{\ep}$ (по свойству
монотонности сумм Дарбу). Итак, \eqa{11}{I - \ep < l(\La(T)) < I +
\ep,\;\fa T \ge T_{\ep}.}

Переходя в (11) к точной верхней грани $\sup \hc{l(\La(T))\;|\; T
\ge T_{\ep}} = l(\Zc)$ по утверждению 1, получим: $I-\ep < l(\Zc)
\le I+\ep$ и $I=l(\Zc)$ в силу произвольности $\ep>0$.
\end{proof}

Если $\Zc$ --- график $f\in\Cc^{1}[a,b]$, то её параметрические
функции $x=t,y=f(t), t\in[a,b]$ из класса $\Cc^1[a,b]$, так что
$\Zc$ --- спрямляемая кривая. Вычислим её длину. Так как $x'_t =
1;\;y'_t = f'(t)$ и $\sqrt{x'^2_t + y'^2_t} = \sqrt{ 1 + f'^2 (t)}$,
то, по формуле~(5), длина $l(\Zc)$ \eqa{5'}{l(\Zc) = \intl{a}{b}
\sqrt{x'^2_t + y'^2_t}\,dt = \intl{a}{b} \sqrt { 1+ f'^2
(t)}\,dt = \intl{a}{b} \sqrt{ 1+f'^2(x)}\,dx.}

\subsubsection{Свойство аддитивности спрямляемых кривых}

\begin{thh}
Если спрямляемая кривая $\Zc = \case{x=\ph(t),\\y=\psi(t)}
t\in[\al,\be]$, то для любого $\ga,\;\al<\ga<\be$, кривые $\Zc_i,
(i=1,2)$ с теми же параметрическими функциями, то на $[\al,\ga]$ и
$[\ga,\be]$, соответственно, спрямляемы и $l(\Zc_1) + l(\Zc_2) =
l(\Zc)$. Обратно, если кривые $\Zc_i, (i=1,2)$, заданные
параметрически $\case{x=\ph_1(t),\\y=\psi_1(t)}, t\in[\al_1,\be_1]$
и $\case{x=\ph_2(t),\\ y = \psi_2(t)} t\in[\al_2,\be_2]$
%уточнить буковки и индексы
спрямляемы и $\ph_1(\be_1) = \ph_2(\al_2);\;\psi_1(\be_1) =
\psi_2(\al_2)$, то $\Zc_1 \cup \Zc_2 = \Zc$ есть спрямляемая
кривая $\Zc\colon l(\Zc) = l_1(\Zc_1) + l_2(\Zc_2)$.
\end{thh}

\begin{proof}
При дополнительном предположении кривых класса $\Cc^1$, утверждение
теоремы, с учётом~(5), есть прямое следствие свойства аддитивности
определённого интеграла (хотя утверждение теоремы справедливо и без
дополнительного предположения).
\end{proof}

\section{Обобщение интеграла Римана}

\subsection{Несобственные интегралы}

\subsubsection{Интегралы по промежутку $[a,b)$}

Рассмотрим $f$, определённую на $[a,b), -\infty < a < b \le +\infty$
и $f\in\Rc[a,b]$ для $\fa t\colon a<t<b$, так что на $[a,b)$
определена непрерывная функция \eqa{1}{F_f(t) = \int\limits_a^t %или ^b?
f(x)\,dx,t\in[a,b),\;F_f(a)=0.} %а что значит интеграл принадлежит..?

Условимся символом $\boxed{t\rightarrow b-}$ обозначать как базу
$\boxed{t\ra b-0}$, если $b$ --- число, так и базу $\boxed{t\ra +
\infty}$, если $b=+\infty$.

\begin{dfn}{1}
Если $\exi \liml{t\ra b-} F_f(t) = I$, то число $I$ называют
несобственным интегралом по промежутку $[a,b)$.
\end{dfn}

Обозначение: $I=\int\limits_a^{b-} f(x)\,dx$. Говорят, что при этом
интеграл \textbf{сходится}, а $f$ \textbf{интегрируема на $[a,b)$.}
Если $F_f(t)$ не имеет $\liml{t\ra b-}$, то будем говорить,
что интеграл $\int\limits_a^{b-}f(x)\,dx$ \textbf{расходится.}

\begin{theorem}
Если в условии определения 1 $b$ --- число, а $f$ ограничена на
$[a,b)$, то $f$ будет интегрируема на промежутке $[a,b)$ (в смысле
определения 1) $\Lra\;f$ интегрируема по Риману на
$[a,b]\;(f\in\Rc[a,b])$, какое бы значение не придать $f(b)$. При
этом \eqa{*}{\int\limits_a^{b-} f(x)\,dx = \intl{a}{b} f(x)\,dx,}
где $\intl{a}{b} f(x)\,dx$ --- интеграл Римана $f$ на $[a,b]$.
\end{theorem}

\begin{proof}
($\Leftarrow$) Придадим произвольное значение $f(b)$. По условию,
$f\in\Rc[a,b]$ и $\intl{a}{b} f(x)\,dx$ не зависит от выбора
значения $f(b)$. Функция $F_f(t)$, определяемая~(1), в этом случае
непрерывна на $[a,b]$ и, в частности, $x=b$, так что
$\exi \liml{t\ra b-0} F_f(t) = F_f(b) = \intl{a}{b}
f(x)\,dx$. Таким образом, справедливо определение 1, то есть
$\liml{t\ra b-0} F_f(t) = I$ и $I=F_f(b)$.

($\Rightarrow$) По условию, $\exi \int\limits_a^{b-} f(x)\,dx =
\liml{t\ra b-0} F_f(t)$, где $F_f(t)$ определено~(1) и $f$
--- ограничена на $[a,b]$ и $\exi M>0$, что $\hm{f(x)} \le M,
x\in[a,b]$, так что $\om=\om(f;[a,b]) = \supl{x_1,x_2 \in
[a,b]}\hm{f(x_1) - f(x_2)} \le \supl{x_1,x_2 \in [a,b]}
\hm{f(x_1)} + \hm{f(x_2)} \le 2M$.

Свойство $f\in\Rc[a,b]$ проверим на основании третьего критерия
интегрируемости. Рассмотрим $\ep>0$ и выберем точку $c,\;a<c<b$
такую, что $b-c < \frac{\ep}{4M}$. Функция, по условию, интегрируема
на $[a,c]$ и для $\ep>0$ существует такое разбиение $T_1:\;[a,c]$
отрезками $\De_k=[x_{k-1},x_k],k=\ol{1,n-1},\; a=x_0, c =
x_{n-1}$, что $S(f;T_1) - s(f;T_1) = \sum\limits_{k=1}^{n-1}
\om(f;\De_k) \De x_k < \frac{\ep}2$. Добавляя точку $b=x_n$ к
$T_1$, получим $T[a,b]$, для которого
$$S(f;T) - s(f;T) = \sum\limits_{k=1}^{n-1} \om(f;\De_k) \De x_k + (b-c) \om (f;[c,b])
< \frac{\ep}2 + (b-c)\om \le \frac{\ep}2 + (b-c) \cdot 2M < \frac{\ep}2 + \frac{\ep}2 = \ep.$$
Итак, $f\in\Rc[a,b]$ по третьему критерию интегрируемости, следовательно, $F_f(t)$, определённая~(1),
непрерывна на $[a,b]$ и, в частности, $\liml{t\ra b-0} F_f(t)
= F(b) = \intl{a}{b} f(x)\,dx$. С другой стороны,
$\liml{t \ra b-0} F_f(t) = \int\limits_a^{b-} f(x)\,dx$, так
что справедливо (*).
\end{proof}

Таким образом, определение вводит новые объекты лишь если $b$ ---
число, а $f$ --- неограничена на $[a,b)$, либо если $b=+\infty$. В
этих случаях $\int\limits_a^{b-} f(x)\,dx$ --- \textbf{несобственный
интеграл}, в отличие от интегралов в прежних случаях ---
\textbf{собственных}. Если $b$ --- число, то вместо
$\int\limits_a^{b-} f(x)\,dx$ принято обозначать $\intl{a}{b}
f(x)\,dx$, если $b=+\infty$, то $\int\limits_a^{b-} f(x)\,dx =
\int\limits_a^{+\infty} f(x)\,dx$.

\begin{ex}
$f(x) = \frac1{x^s}$, на $[a,+\infty),\;a>0$. Если $s\ne 1$, то
\equ{F_f(t) = \int\limits_a^t \frac{dx}{x^s} =
\hr{\frac{x^{-s+1}}{-s+1}}\biggl|_a^t = \frac{t^{-s+1}}{-s+1} -
\frac{a^{-s+1}}{-s+1}} и \equ{\liml{t\ra + \infty}
F_f(t) = \bcase{&\frac{1}{(s-1)a^{s-1}}, & s>1,\\ &+\infty, &s<1 %там уточнить
.}}

Если $s=1$, то $F_f(t) = \int\limits_a^t \frac{dx}{x} = \ln t - \ln
a$  $\liml{t\ra + \infty} F_f(t) = +\infty$.

Таким образом, несобственный интеграл $\int\limits_a^{+\infty}
\frac{dx}{x^s}$, где $a>0$, сходится при всех $s>1$ и расходится при
$\fa s\le 1$. При этом, если $s>1$, то $\int\limits_a^{+\infty}
\frac{dx}{x^s} = \frac1{(s-1)a^{s-1}}$, в частности,
$\int\limits_1^{+\infty} \frac{dx}{x^s} = \frac1{s-1}$.
\end{ex}

\subsubsection{Интегралы по промежуткам $(a,b]$ и $[a,b)$}

Рассмотрим $f$, определённую на $(a,b],\; -\infty \le a < b <
+\infty$, и $f\in\Rc[t,b]$ для любого $t\colon a<t<b$, так что на
$(a,b]$ справедлива непрерывность $\Phi_f(t)\colon$
\eqa{2}{\Phi_f(t) = \int\limits_t^b f(x)\,dx,\;t\in(a,b],\;\Phi_f(b)
= 0.}

\begin{dfn}{2}
Если $\exi \liml{t\ra a+} \Phi_f(t) = I$, то $I$
--- \textbf{интеграл $f$ по $(a,b]$.}
\end{dfn}
Обозначение: $\int\limits_{a+}^b f(x)\,dx$. Интеграл
\textbf{сходится}, а $f$ --- \textbf{интегрируема.}

Если $\Phi_f(t)$ не имеет $\liml{t\ra a+}$, то
$\intl{a}{b} f(x)\,dx$ --- \textbf{расходится}.

Как и в предыдущем пункте, убеждаемся, что определение 2 вводит
новые объекты лишь если $a$ --- число, $f$ --- неограничена на
$(a,b]$ или если $a=-\infty$. В этих случаях $\int\limits_{a+}^b
f(x)\,dx$ --- \textbf{несобственный} и обозначается
$\int\limits_{a+}^b f(x)\,dx = \intl{a}{b} f(x)\,dx$ ($a$ ---
число) и $\int\limits_{-\infty}^b f(x)\,dx$ ($a=-\infty$).

\begin{ex}
$f(x) = \frac1{x^s}$ на $(0,a]$, где $a>0$ --- число.

Если \equ{\Phi_f(t) = \int\limits_t^a \frac{dx}{x^s} =
\frac{a^{-s+1}}{-s+1} - \frac{t^{-s+1}}{-s+1} \mbox{ и }
\liml{t\ra 0+0} \Phi_f(t) = \case{+\infty, s>1,\\
\frac{a^{1-s}}{1-s}, s<1.}}
\end{ex}

Таким образом, несобственный интеграл $\int\limits_0^a
\frac{dx}{x^s},\;a>0$ --- число, сходящееся при $s<1$ и расходящееся
$s\ge 1$. Интегралы в примерах 1 и 2 --- \textbf{стандартные
интегралы}. При этом $\int\limits_0^a \frac{dx}{x^s} =
\frac{a^{1-s}}{1-s},\;s<1$, в частности, $\int\limits_0^1
\frac{dx}{x^s} = \frac1{1-s}$.

Пусть теперь $f$ задана на промежутке $(a,b),\;-\infty \le a < b \le
+ \infty$ и $f$ интегрируема на каждом отрезке, лежащем в $(a,b)$.

\begin{stm}
Если несобственные интегралы $\int\limits_a^c f(x)\,dx$ и
$\int\limits_c^b f(x)\,dx$ сходятся для какого-либо $c\colon a<c<b$,
то они сходятся для всех $c\colon a<c<b$, причём их сумма
\eqa{3}{\int\limits_a^c f(x)\,dx + \int\limits_c^b f(x)\,dx} не
зависит от выбора $c$.
\end{stm}

\begin{proof}
Для произвольного $c'\colon a<c'<b$ имеем (согласно определениям 1 и
2 и условию) \eqa{4}{\int\limits_{a}^{c'} f(x)\,dx =
\liml{t'\ra a+} \int\limits_{t'}^{c'} f(x)\,dx =
\liml{t' \ra a+} \hr{\int\limits_{t'}^c f(x)\,dx +
\int\limits_c^{c'} f(x)\,dx} = \int\limits_a^c f(x)\,dx +
\int\limits_c^{c'} f(x)\,dx,} так как, по условию, существует предел
на $t\ra a+$ первого интеграла в формуле~(4), а второй интеграл ---
постоянная функция на базе $t'\ra a+$. Аналогично,
\mla{5}{\int\limits_{c'}^{t''} \dots = \int\limits_{c'}^c \dots +
\int\limits_c^{t''} \dots \mbox { и } \int\limits_{c'}^b =
\liml{t''\ra b-} \int\limits_{c'}^{t''} \dots =
\liml{t''\ra b-} \hr{\int\limits_{c'}^c +
\int\limits_c^{t''}} = \int\limits_{c'}^c + \int\limits_c^b \quad \Ra \\ \Ra \quad \int\limits_a^{c'} \dots + \int\limits_{c'}^b =
\int\limits_a^c + \int\limits_c^{c'} + \int\limits_{c'}^c \dots +
\int\limits_c^b = \int\limits_a^c f(x)\,dx + \int\limits_c^b
f(x)\,dx.}
\end{proof}

\begin{dfn}{3}
Пусть $-\infty\le a < b \le +\infty$ и $f$ определена на $(a,b)$ и
интегрируема на каждом отрезке из $(a,b)$. Если интегралы
$\int\limits_a^c f(x)\,dx$ и $\int\limits_c^b f(x)\,dx$ сходятся для
некоторого $c\colon a<c<b$, то сумма~(3) (по доказанному,
независимая от $c$) --- \textbf{интеграл $f$ по $(a,b)$.}
\end{dfn}

Обозначение: $\intl{a}{b} f(x)\,dx$. Говорят, что
\textbf{интеграл сходится и $f$ интегрируема на $(a,b)$.} В
противном случае --- \textbf{расходится}.


Таким образом, по определению, \eqa{6}{\intl{a}{b} f(x)\,dx =
\int\limits_a^c f(x)\,dx + \int\limits_c^b f(x)\,dx,} если интегралы
$\int\limits_a^c f(x)\,dx,\; \int\limits_c^b f(x)\,dx$ сходятся
($a<c<b$). Формула~(6) определяет свойство аддитивности
несобственных интегралов по промежутку $(a,b)$. Аналогичная формула
справедлива и для $[a,b)$ и $(a,b]$. При этом один из интегралов в
правой части~(6) --- интеграл Римана, а другой --- несобственный.

%тут ничего не пропущено? :)

\begin{exx}
Вычислим $\int\limits_{-\infty}^{\infty}\frac{dx}{1+x^2}$. По
свойству аддитивности,
\ml{\int\limits_{-\infty}^{+\infty}\frac{dx}{1+x^2} = \int\limits_{-\infty}^0 \frac{dx}{1+x^2} + \int\limits_0^{+\infty}
\frac{dx}{1+x^2} = \liml{t'\ra -\infty} \int\limits_{t'}^0 \frac{dx}{1+x^2} + \liml{t''\ra +\infty} \int\limits_0^{t''}
\frac{dx}{1+x^2} = \\ =
\liml{t'\ra -\infty} \hs{-\arctg{t'}} + \liml{t''\ra + \infty} \hs{\arctg{t''}} = -\hr{-\frac{\pi}2} + \frac{\pi}2 = \pi.}
\end{exx}

\subsubsection{Критерий Коши сходимости несобственного
интеграла}

\begin{thh}
Несобственный интеграл $\intl{a}{b} f(x)\,dx$ по промежутку
$[a,b)$ (по $(a,b]$) сходится $\Lra$ для любого $\ep>0$
можно указать $b_{\ep}$, $a<b_{\ep}<b$ ($a_{\ep}\colon a<a_{\ep} <
b)$, что неравенство \eqa{1}{\hm{\int\limits_{t_1}^{t_2} f(x)\,dx} <
\ep} справедливо для любого $t_i,\;b_{\ep} < t_i < b, i=1,2$ (для
любого $t_i,\; a<t_i < a_{\ep},i=1,2$).
\end{thh}

\begin{proof}
Пусть $a<t_1 < t_2 < b$. По свойству аддитивности определённого
интеграла, \equ{\int\limits_{t_1}^{t_2} f(x)\,dx =
\int\limits_a^{t_2} f(x)\,dx - \int\limits_a^{t_1} f(x)\,dx =
F_f(t_2) - F_f(t_1) \hr{ = \Phi_f(t_1) - \Phi_f(t_2)},} где $F_f$ и
$\Phi_f$ определены посредством формул~(1) (пункт 1.1) и~(2) (пункт
1.2), следовательно, неравенство~(1) $\Lra\; \hm{F_f(t_2)
- F_f(t_1)} < \ep$ для любого $t_i\colon b_{\ep} < t_i < b,\;i=1,2\;
(\hm{\Phi_f(t_2) - \Phi_f(t_1)} < \ep,\;\forall t_i\colon a<t_i <
a_{\ep}, i=1,2$), что, в свою очередь, есть критерий Коши
существования $\liml{t\ra b-} F_f(t) \hr{\liml{t\ra
a+} \Phi_f(t)}$, что равносильно исходимости несобственного
интеграла $\intl{a}{b} f(x)\,dx$.
\end{proof}

\subsubsection{Остаток несобственного интеграла}

Пусть $f$ интегрируема по $[a,b)$ (по $(a,b]$), то есть сходится
несобственный интеграл $\intl{a}{b} f(x)\,dx$. По свойству
аддитивности, для любого $t\colon a<t<b$, справедливо
\eqa{2}{\intl{a}{b} f(x)\,dx = \underbrace{\int\limits_a^t
f(x)\,dx}_{=F_f(t)} + \underbrace{\int\limits_t^b
f(x)\,dx}_{=r_f(t)},}
или же
\eqa{2'}{\intl{a}{b} f(x)\,dx = \int\limits_t^b f(x)\,dx + \int\limits_a^t f(x)\,dx = \Phi_f(t) +
r_f(t),} где $r_f(t) = \int\limits_t^b f(x)\,dx \hr{r_f(t)=\int\limits_a^t f(x)\,dx}$ --- несобственный интеграл с
особ. в $x=b-$ ($x=a+$) %что-то здесь --- остаток несобственного интеграла
%$\intl{a}{b} f(x)\,dx$.

и для любого $t\colon a<t<b\;r_f(t)$ сходится, так как
$\liml{t\ra b-} F_f(t) = \intl{a}{b} f(x)\,dx
\hr{\liml{t\ra a+} \Phi_f(t) = \intl{a}{b} f(x)\,dx}$,
то, на основании~(2) (($2'$)), \equ{\liml{t\ra b-} r_f(t) = 0
\hr{\liml{t\ra a+} r_f(t)=0}.}

\textbf{Обратно}, пусть $f$ определена на $[a,b)$ (на $(a,b]$),
$f\in\Rc[a,t]$ ($f\in\Rc[t,b]$) для любого $t\colon a<t<b$ и для
некоторого $t\colon a<t<b$, справедливо не только для несобственного
интеграла $\int\limits_t^b f(x)\,dx \hr{\int\limits_a^t f(x)\,dx}$
этого значения $t$, но и для остальных $t\colon a<t<b$ и,
следовательно, $\liml{t\ra b-} r_f(t)=0 \hr{\liml{t\ra
a+} r_f(t)=0}$.

\begin{thh}
Несобственный интеграл $\intl{a}{b} f(x)\,dx$ по $[a,b)$
($(a,b]$) сходится $\Lra$ $\liml{t\ra b-}
r_f(t)=0\;\hr{\liml{t\ra a+} r_f(t)=0}$.
\end{thh}

\begin{proof}
Доказательство изложено выше.
\end{proof}

\subsection{Основные свойства несобственных интегралов}

\subsubsection{Свойства, аналогичные свойствам определённого
интеграла}

\begin{thn}{1}
Пусть $f,g$ интегрируемы по промежутку $[a,b)$. Тогда:

1) для любого $k\in\R,k\ne0$ функция $k\cdot f$ интегрируема по
$[a,b)$ и $\intl{a}{b} kf(x)\,dx = k\intl{a}{b} f(x)\,dx$;

2) $(f+g)$ интегрируема по $[a,b)$ и $\intl{a}{b} (f+g)\,dx =
\intl{a}{b} f(x)\,dx + \intl{a}{b} g(x)\,dx$;

3) если $f(x)\ge g(x),x\in[a,b)$, то $\intl{a}{b} f(x)\,dx \ge
\intl{a}{b} g(x)\,dx$;

4) $f(x)\ge0,x\in[a,b) \Rightarrow \intl{a}{b} f(x)\,dx \ge 0$.
\end{thn}

\begin{proof}
1) Так как $F_{kf}(t) = kF_f(t), a<t<b$ (по свойству линейности
определённого интеграла) и $\exists \liml{t\ra b-} F_f(t) =
\intl{a}{b} f(x)\,dx$, то, по свойству линейности предела
функции по базе, $\exists \liml{t\ra b-} F_{kf}(t) =
k\liml{t\ra b-} F_f(t)= k\intl{a}{b} f(x)\,dx =
\intl{a}{b} kf(x)\,dx.$

2) Так как $F_{f+g}(t) = F_f(t) + F_g(t),a<t<b$ и $\liml{t\ra
b-} F_f(t) = \intl{a}{b} f(x)\,dx,\; \liml{t\ra b-}
F_g(t) = \intl{a}{b} g(x)\,dx$, то
\equ{\exi \liml{t\ra b-} F_{f+g}(t) = \liml{t\ra
b-} F_f(t) + \liml{t\ra b-} F_g(t) = \intl{a}{b} f(x)\,dx
+ \intl{a}{b} g(x)\,dx = \intl{a}{b} (f+g)\,dx.}

3) Так как для любого $t, a<t<b,\;F_g(t) = \int\limits_a^t f(x)\,dx
\ge \int\limits_a^t g(x)\,dx = F_g(t)$ и предел функции по базе
обладает свойством монотонности, то $\intl{a}{b} f(x)\,dx =
\liml{t\ra b-} F_f(t) \ge \liml{t\ra b-} F_g(t) =
\intl{a}{b} g(x)\,dx$.

4) По пункту 3), $\intl{a}{b} f(x)\,dx = \liml{t\ra b-}
\int\limits_a^t f(x)\,dx \ge \liml{t\ra b-} \int\limits_a^t
0\,dx = \liml{t\ra b-} 0= 0$. %верно?))
\end{proof}

\subsubsection{Теорема о существовании и оценке модуля
несобственного интеграла}

\begin{thn}{2}
(Критерий существования несобственного интеграла от положительной
функции). Если $f\in\Rc[a,t)$, для любого $a<t<b$ и $f(x)\ge
0,x\in[a,b)$, то $f$ интегрируема по $[a,b)$ (то есть сходится
несобственный интеграл $\intl{a}{b} f(x)\,dx$) $\Lra
\; F_f(t) = \int\limits_a^t f(x)\,dx$ ограничен сверху на $[a,b)$.
\end{thn}

\begin{proof}
По свойству определённых интегралов, $0\le \int\limits_a^t f(x)\,dx
= F_f(t), \forall t\colon a<t<b$. Для $\forall a<t_1<t_2<b$
справедливо \equ{F_f(t_2) = \int\limits_a^{t_2} f(x)\,dx =
\int\limits_a^{t_1} f(x)\,dx + \int\limits_{t_1}^{t_2} f(x)\,dx \ge
\int\limits_a^{t_1} f(x)\,dx = F(t_1) \ge0,} то есть $F_f(t)
\uparrow$ на $[a,b)$. По свойству монотонности функций,
$\liml{t\ra b-}F_f(t)$ существует $\Lra$ $F_f(t)$
ограничена сверху на $[a,b)$. Следовательно, $\intl{a}{b}
f(x)\,dx = \liml{t\ra b-} F_f(t)$ сходится $\Lra$
$F_f(t)$ ограничена сверху на $[a,b)$.
\end{proof}

\begin{thn}{3}
(Признак существования несобственного интеграла от положительной
функции). Если $f,g\in\Rc[a,t]$ для $\forall t\colon a<t<b$ и $0 \le
f(x) \le g(x),\;x\in[a,b)$, то из сходимости несобственного
интеграла $\intl{a}{b} g(x)\,dx$ следует сходимость
несобственного интеграла $\intl{a}{b} f(x)\,dx$ (и,
следовательно, из расходимости несобственного интеграла
$\intl{a}{b} f(x)\,dx$ следует расходимость $\intl{a}{b}
g(x)\,dx)$.
\end{thn}

\begin{proof}
Пусть сходится $\intl{a}{b} g(x)\,dx =
\liml{t\ra} %к чему?
F_g(t) = \lim\int\limits_a^t g(x)\,dx$.

Так как $0\le F_f(t) = \int\limits_a^t f(x)\,dx \le \int\limits_a^t
g(x)\,dx = F_g(t), \;\fa t\colon a<t<b$, то, по теореме 2,
$F_g(t)$ ограничена сверху на $[a,b)$ и, следовательно, $F_f(t)$
ограничена сверху на $[a,b)$ и, по теореме 2, сходится
$\intl{a}{b} f(x)\,dx$.
\end{proof}

\begin{note}
Утверждение теоремы 3 остаётся справедливым, если $0\le f(x) \le
g(x)$ справедливо на некотором $[c,b)\; a<c<b$. Действительно, в
этом случае сходится любой интеграл $\int\limits_c^b f(x)\,dx$,
являющийся остатком несобственного интеграла $\intl{a}{b}
f(x)\,dx$ и, следовательно, $\intl{a}{b} f(x)\,dx$ сходится.
\end{note}

\begin{exx}
Так как $e^{-x^2} \le e^{-x} \;\forall x\ge1$,
$\int\limits_a^{+\infty} e^{-x}\,dx = \liml{t\ra +\infty}
\int\limits_a^t e^{-x}\,dx = \liml{t\ra +\infty} \hs{-e^{-t}
+ e^{-a}} = e^{-a}$ для любого $a\in\R$, то, по теореме 3,
собственный интеграл $\int\limits_0^{+\infty} e^{-x^2}\,dx$ сходится
при любом $a\in\R$. %чо за а? где оно?

\equ{\int\limits_0^{+\infty} e^{-x^2}\,dx \mbox{ --- интеграл
Эйлера--Пуассона (третий семестр).}}
\end{exx}

\begin{thn}{4}
Если $f\in\Rc[a,t],\;\fa t\colon a<t<b$ и сходится
несобственный интеграл $\intl{a}{b} \hm{f(x)}\,dx$ по промежутку
$[a,b)$, то сходится несобственный интеграл $\intl{a}{b}
f(x)\,dx$ и справедливо: $\hm{\intl{a}{b} f(x)\,dx} \le
\intl{a}{b} \hm{f(x)}\,dx$.
\end{thn}

\begin{proof}
Рассмотрим функции $0\le g = \frac{\hm{f}+f}2 \le \hm{f}, \; 0 \le h
= \frac{\hm{f}-f}2 \le \hm{f}$ на $[a,b)$, тогда $f,\hm{f},g,h
\in\Rc[a,t]$ для $\fa  a<t<b$. Так как сходится несобственный
интеграл $\intl{a}{b} \hm{f(x)}\,dx$, то, по теореме 3, сходится
несобственный интеграл $\intl{a}{b} g(x)\,dx, \; \intl{a}{b}
h(x)\,dx$. Так как $f=g-h$, то, по теореме 1, $\intl{a}{b}
f(x)\,dx = \intl{a}{b} g(x)\,dx - \intl{a}{b} h(x)\,dx$ ---
сходится.

Учитывая, что $\hm{f}=g+h$, имеем:  \equ{\hm{\intl{a}{b}
f(x)\,dx} \le \hm{\intl{a}{b} g(x)\,dx} + \hm{\intl{a}{b}
h(x)\,dx} = \intl{a}{b} g(x)\,dx + \intl{a}{b} h(x)\,dx =
\intl{a}{b} \hm{f(x)}\,dx.}
\end{proof}

\begin{df}
Несобственный интеграл $\intl{a}{b} f(x)\,dx$ ---
\textbf{абсолютно сходящийся}, если сходится $\intl{a}{b}
\hm{f(x)}\,dx$.
\end{df}

\begin{imp*}
(К теореме 4). Всякий абсолютно сходящийся несобственный интеграл
сходится.
\end{imp*}

\begin{exx}
Несобственные интегралы $\int\limits_a^{+\infty}
\frac{\sin{kx}}{x^s}\,dx,\;\int\limits_a^{+\infty}
\frac{\cos{kx}}{x^s}\,dx$, где $a>0$ --- постоянная, абсолютно
сходятся для $s>1$ и $\fa k\ne0\in\R$.
\end{exx}

\begin{proof}
$\hm{\frac{\sin{kx}}{x^s}} \le \frac1{x^s}$,
$\hm{\frac{\cos{kx}}{x^s}} \le \frac1{x^s}$, $\fa  x\ge a$ и
$\int\limits_a^{+\infty} \frac{dx}{x^s}$ сходится для любого $s>1$.

По теореме 3, сходятся $\int\limits_a^{+\infty}
\frac{\hm{\sin{kx}}}{x^s}\,dx$, $\int\limits_a^{+\infty}
\frac{\hm{\cos{kx}}}{x^s}\,dx$, $s>1$.
\end{proof}

Утверждения, аналогичные данным, остаются справедливыми для $(a,b]$
и $[a,b)$ с аналогичными доказательствами.

\subsection{Интегрирование по частям и подстановкой в несобственном
интеграле}

\subsubsection{Интегрирование по частям}

\begin{thh}
Если $f,u,v$ принадлежат $\Cc^1$ на $[a,b)$, где $-\infty < a < b \le +\infty$, произведение $u(x)v(x)$ имеет предел
$\liml{x\ra b-} u(x)v(x)$, и сходится интеграл $\intl{a}{b} v(x)u'(x)\,dx$, то
сходится другой интеграл $\intl{a}{b} u(x)v'(x)\,dx$ и
справедлива формула:
\eqa{1}{\intl{a}{b} u(x)v'(x)\,dx = \hs{u(x)v(x)}_a^{b-} - \intl{a}{b} u'(x)v(x),} где
$\hs{u(x)v(x)}_a^{b-} = \liml{x\ra b-} u(x)v(x) - u(a)v(a).$
\end{thh}

\begin{proof}
Для произвольного $t\colon a<t<b$, на $[a,t]$ справедлива формула
интегрирования по частям в определённом интеграле (в слабой форме):
\eqa{2}{\int\limits_a^t u(x)v'(x)\,dx = u(t)v(t) - u(a)v(a) -
\int\limits_a^t v(x)u'(x)\,dx.}

Так как существует $\liml{t\ra b-} \int\limits_a^t
v(x)u'(x)\,dx = \intl{a}{b} v(x)u'(x)\,dx$, то остаётся перейти
к $\liml{t\ra b-}$ в~(2) и воспользоваться свойством
линейности предела функции по базе.
\end{proof}

\begin{exx}
$\int\limits_a^{+\infty}
\frac{\sin{kx}}{x^s}\,dx,\;\int\limits_a^{+\infty}
\frac{\cos{kx}}{x^s}\,dx,\;a>0,k\ne0$ --- фиксированы, сходятся (но
не абсолютно) для любого $s\colon 0<s\le1$.
\end{exx}

\begin{proof}
Применяя формально формулу~(1), получим для второго интеграла:
\equ{\int\limits_a^{+\infty} \frac{\cos{kx}}{x^s}\,dx =
\int\limits_a^{+\infty} \frac1{x^s} \hr{\frac{\sin{kx}}{k}}'\,dx =
\liml{x\ra +\infty} \frac{\sin{kx}}k - \frac{\sin{ka}}{ka^s}
+
\frac{s}k \int %без пределов интегрирования?!
\frac{\sin{kx}}{x^{s+1}}\,dx = -\frac{\sin{ka}}{ka^s} + \frac{s}k
\int\limits_a^{+\infty} \frac{\sin{kx}}{x^{s+1}}\,dx.} Так как
$s+1>1$, то последний интеграл сходится (абсолютно), так что, по
предыдущей теореме, сходится и исходный интеграл.

Покажем, что он не сходится абсолютно. Если бы это было так, то есть
сходился бы $\int\limits_a^{+\infty} \frac{\hm{\cos{kx}}}{x^s}\,dx$,
то, поскольку $\hm{\cos{kx}} \ge \cos^2{kx} \ge 0$, сходился бы
$\int\limits_a^{+\infty} \frac{\cos^2{kx}}{x^s}\,dx =
\int\limits_a^{+\infty} \frac{1+\cos{2kx}}{2x^s}\,dx.$ Так как
$\int\limits_a^{+\infty} \frac{\cos{2kx}}{2x^s}\,dx$ сходится, по
доказанному, то сходился бы интеграл $\int\limits_a^{+\infty}
\frac{dx}{2x^s}$, но последний расходится для $s\le1\;\Rightarrow$
противоречие.
\end{proof}

\begin{df}
Несобственный интеграл --- \textbf{сходящийся условно}, если он
сходится, но не сходится абсолютно.
\end{df}

\begin{exx}
Интеграл Дирихле $\int\limits_0^{+\infty} \frac{\sin{kx}}x\,dx$
сходится условно.
\end{exx}

\begin{proof}
Этот интеграл рассматривается на промежутке $(0;+\infty)$. Поэтому,
формально, \eqa{3}{\int\limits_a^{+\infty} \frac{\sin{kx}}x\,dx =
\int\limits_0^1 \frac{\sin{kx}}x\,dx + \int\limits_1^{+\infty}
\frac{\sin{kx}}x \,dx.}

Функция $\frac{\sin{kx}}x$ --- непрерывна на $(0,1]$ и имеет
$\liml{x\ra 0+} \frac{\sin{kx}}x = k < +\infty$, %меньше чего?
следовательно, она ограничена на $(0,1]$, следовательно, по теореме
пункта 1.1, интеграл $\int\limits_0^1 \frac{\sin{x}}x\,dx$ сходится.

Второй интеграл в правой части~(3) --- сходящийся условно, так как
$s=1$ (по предыдущему примеру), следовательно, сходится условно
интеграл Дирихле (материал третьего семестра)
$\int\limits_0^{+\infty} \frac{\sin{kx}}x\,dx = \frac{\pi}2
\sgn{k}$.
\end{proof}

\subsubsection{Интегрируемость заменой переменной интегрирования
или подстановкой в несобственном интеграле}

\begin{thh}
Пусть $f$ непрерывна на $[a,b)$, $-\infty < a < b \le + \infty$, а
$\ph \in \Cc^1[\al,\be)$, $-\infty<\al<\be\le +\infty$, строго
возрастает на $[\al,\be)$ и $\ph(\al)=a;\;\ph(\be-)=b$; то есть
$\liml{\tau\ra \be-} \ph(\tau) = \ph(\be-)=b$. Если сходится
интеграл $\int\limits_{\al}^{\be} f(\ph(\tau))\ph'(\tau)d\tau$, то
сходится $\intl{a}{b} f(x)\,dx$ и справедливо: $\intl{a}{b}
f(x)\,dx = \int\limits_{\al}^{\be} f(\ph(\tau))\ph'(\tau)d\tau$.
\end{thh}

\begin{proof}
По условию, образ $\ph([\al,\be))=[a,b)$ и на $[\al,\be)$ $\ph$
удовлетворяет условию теоремы о существовании и дифференцируемости
обратной функции, по которой на $[a,b)$ существует $\ph^{-1} =
\psi$, возрастающая на $[a,b)$, $\psi([a,b))=[\al,\be);\;\psi(a) =
\al, \; \ph(b-)=\be$,%а разве не пси от b слева?
то есть $\liml{x\ra b-} \psi(x) = \be$.

Поэтому, для произвольного $t\colon a<t<b$, на $[a,t]$ справедлива
теорема о замене переменной в определённом интеграле, по которой
\eqa{5}{\int\limits_a^t f(x)\,dx = \int\limits_{\al}^{\psi(t)}
f(\ph(\tau))\ph'(\tau)d\tau.}

Так как $\liml{t\ra b-} \psi(t) = \be$ и интеграл в правой
части~(4) сходится, то \equ{\liml{t\ra b-}
\int\limits_a^{\psi(t)} f(\ph(\tau))\ph'(\tau)d\tau =
\int\limits_{\al}^{\be} f(\ph(\tau))\ph'(\tau)d\tau,} и, по~(5),
существует $\liml{t\ra b-} \int\limits_a^t f(x)\,dx$ и
справедливо~(4).
\end{proof}

\begin{exx}
%непонятное слово
$\int\limits_0^{+\infty} \sin(x^2)\,dx,\; \int\limits_0^{+\infty}
\cos(x^2)\,dx$ сходятся условно.
\end{exx}

\begin{proof}
\eqa{6}{\int\limits_0^{+\infty} \sin(x^2)\,dx = \int\limits_0^1
\sin(x^2)\,dx + \int\limits_1^{+\infty} \sin(x^2)\,dx.}

Так как $\sin(x^2)$ непрерывна на $[0,1]$, то первый интеграл ---
интеграл Римана.

Во втором интеграле: $x^2=t, \;dx = \frac1{2\sqrt{t}}\,dt
\Rightarrow \;\int\limits_1^{+\infty} \sin(x^2)\,dx = \frac12 \int
\frac{\sin{t}}{\sqrt{t}}\,dt$ --- сходится условно
($s=\frac12;\;k=1).$

Аналогично для $\int\limits_0^{+\infty} \cos(x^2)\,dx$.
\end{proof}

\subsection{Признак сходимости несобственного интеграла. Приложения}

\subsubsection{Признак сходимости}

\begin{thh}
Если $f$ определена и ограничена на $[a,+\infty),\;a>0$, и на каждом
отрезке $[a,t],\;t>a$ имеет только конечное множество точек разрыва,
а $F(t)=\int\limits_a^t f(x)\,dx$ ограничена на $[a,+\infty)$, то
$\int\limits_a^{+\infty} \frac{f(x)}{x^{\al}}\,dx$ сходится при всех
$\al>0$.
\end{thh}

\begin{proof}
По условию, $\exi M>0\colon \hm{F(t)}\le M \;\fa
t\in[a;+\infty)$, и также $F(a)=0$. Кроме того, для любого $t>a$ на
$[a,t]$ справедлива теорема об интегрировании по частям в
определённом интеграле в полной своей формулировке, согласно которой
\eqa{7}{\int\limits_a^t \frac{1}{x^{\al}}f(x)\,dx =
\hs{\frac{F(x)}{x^{\al}}}_a^t + \al \int\limits_a^t
F(x)\frac1{x^{\al+1}}\,dx = \frac{F(t)}{t^{\al}} -
\frac{F(a)}{a^{\al}} + \al\int\limits_a^t \frac{F(x)}{x^{\al+1}}\,dx
= \frac{F(t)}{t^{\al}} + \al\int\limits_a^t
\frac{F(x)}{x^{\al+1}}\,dx.}

$\hr{F'(x)=f(x);\;x\in(a,t)}.$

Так как $\liml{t\ra +\infty} \frac{F(t)}{t^{\al}} = 0$
($\al>0,\;\hm{F(t)}\le M,\; t\ge a$) и $\int\limits_a^{+\infty}
\frac{F(x)}{x^{\al+1}}\,dx$ сходится абсолютно (поскольку
$\hm{\frac{F(x)}{x^{\al+1}}} \le \frac{M}{x^{\al+1}}$ при $x\ge a$ и $\al+1>1$), то существует $\liml{t\ra +\infty}
\hs{\frac{F(t)}{t^{\al}} + \al \int\limits_a^{+\infty}
\frac{F(x)}{x^{\al+1}}\,dx} = \al\int\limits_a^{+\infty}
\frac{F(x)}{x^{\al+1}}\,dx$, то, по~(7), сходится изучаемый
несобственный интеграл.
\end{proof}

\subsubsection{Неполная формула Стирлинга}

\begin{thh}
Существует число $c>0\colon\; n! = cn^{n+\frac12} \cdot e^{-n}
\hs{1+o(1)}, n\ra\infty$.
\end{thh}

\begin{proof}
Применим формулу Эйлера--Маклорена к
$f(x)=\ln{x},\;x\in[1,n],\;n\in\N$:
\mla{8}{\ln(n!) = \sumkun \ln{k} = \int\limits_1^n \ln{x}\,dx + \rho(n)\ln{n} - \rho(1)\ln{1} -
\int\limits_1^n \frac1x \rho(x)\,dx = \\ =
\hs{x\ln{x} - x}_1^n + \frac12 \ln{n} - \int\limits_1^n\frac1x \rho(x)\,dx = n\ln{n} + \frac12
\ln{n}-n + 1 - a_n,}
где $a_n = \int\limits_1^n \frac1x
\rho(x)\,dx$. Потенцируя, получим: \eqa{9}{n! = n^{n+\frac12} e^{-n}
e^{1-a_n}.}

Покажем, что $\exists \liml{n\ra\infty} a_n =a$. Для этого
проверим, что сходится несобственный интеграл
$\int\limits_1^{+\infty} \frac1x \rho(x)\,dx = a$. Рассмотрим
$\mu(t) = \int\limits_0^p \rho(x)\,dx$. $\rho(x)$ имеет лишь
конечное множество точек разрыва на $[1,n]$ и ограничена на $[0;1)$,
$\rho(x) = \frac12 - \hc{x} = \frac12 - x$, следовательно, для
$t\colon 0\le t \le 1$, \equ{\mu(t) = \int\limits_0^t \hr{\frac12 -
x}\,dx = \hs{\frac12 x - \frac12 x^2}\biggl|_0^t = \frac12 t -
\frac12 t^2,\; t\in[0,1]} и $\mu$ имеет период $T=1$.

По теореме предыдущего пункта, сходится $\int\limits_1^{+\infty}
\frac1x \rho(x)\,dx = a$ и $a=\liml{n\ra\infty} a_n$ по
определению. Итак, $\exists \liml{n\ra\infty} e^{1-a_n}$ и
$e^{1-a_n} = c\hs{1+o(1)},\; n\ra\infty,\;c>0$. Следовательно, (8)
доказана.
\end{proof}
\subsection{Интеграл Стилтьеса}

\subsubsection{Интегральные суммы Стилтьеса}

Рассмотрим $f$ и $g$, определённые на $[a,b]$ и произвольное
размеченное разбиение $T_{\ze}$ отрезка $[a,b]$ на
$\De_k=[x_{k-1},x_k],\;\hm{\De_k}=\De_k,\;k=\ol{1,n},\; a =
x_0, x_n = b$ и набор $\ze=\hc{\ze_1,\dots,\ze_n},\;\ze_k\in\De_k,\;
\ol{1,n}=k$.

Числа $\si(f;g;T_{\ze}) = \sumkun f(\ze_k) \hs{g(x_k) - g(x_{k-1})}$
--- \textbf{интегральные суммы $f$ по $g$, отвечающие размеченному разбиению
$T_{\ze}$}.

Обозначим $\Ps$ --- множество всех размеченных разбиений
$T_{\ze}[a,b]\colon T\colon a=x_0 < x_1 < \dots < x_n = b$ и набор
$\ze=\hc{\ze_1,\dots,\ze_n}, \; \ze_k \in \De_k,\;
k=\ol{1,n},\; \De_k = [x_{k-1};x_k]$.

Для $f$ и $g$, определённых на $[a,b]$, рассмотрим функцию
(отображение) $\Psi_f^g\colon \Ps\ra\R$, определяемую формулой:

\eqa{1}{\Psi_f^g (T_{\ze}) = \si(f;g;T_{\ze}) = \sumkun f(\ze_k)
\hs{g(x_k) - g(x_{k-1})},\; \forall T_{\ze} \in \Ps.}

Функция $\Psi_f^g$ обладает свойствами:

\begin{prop}
Для произвольных $f_1, f_2, g$, определённых на $[a,b]$ и
произвольных чисел $\la_1,\la_2\in\R$ справедливо $\Psi_{\la_1 f_1 +
\la_2 f_2}^g = \la_1 \Psi_{f_1}^g + \la_2 \Psi_{f_2}^g$.
\end{prop}

\begin{proof}
Все $\Psi$ определены на $\Ps$. Для $\fa  T_{\ze} \in \Ps$, по~(1), справедливо:
\ml{\Psi_{\la_1 f_1 + \la_2 f_2}^g (T_{\ze}) = \si(\la_1 f_1 + \la_2 f_2; g; T_{\ze}) = \sumkun \hr{\la_1 f_1
(\ze_k) + \la_2 f_2 (\ze_k)} \hs{g(x_k) - g(x_{k-1})} = \\ =
\la_1 \sumkun f_1(\ze_k) \hs{g(x_k) - g(x_{k-1})}  + \la_2 \sumkun
f(\ze_k) \hs{g(x_k) - g(x_{k-1})} = \la_1 \si(f_1;g;T_{\ze}) + \la_2 \si(f_2;g;T_{\ze}) = \la_1 \Psi_{f_1}^g + \la_2 \Psi_{f_2}^g.}
\end{proof}

\begin{prop}
Для произвольных $f,g_1,g_2$, определённых на $[a,b]$ и произвольных
чисел $\la_1,\la_2\in\R$, справедливо $\Psi_f^{\la_1 g_1 + \la_2
g_2} = \la_1 \Psi_f^{g_1} + \la_2 \Psi_f^{g_2}$.
\end{prop}

\begin{proof}
Для $\forall T_{\ze}\in\Ps$, по~(1), справедливо:
\ml{\Psi_f^{\la_1 g_1 + \la_2 g_2} (T_{\ze}) = \si(f;\la_1 g_1 + \la_2 g_2;T_{\ze}) =
\sumkun f(\ze_k) \hs{\la_1 g_1 (x_k) + \la_2 g_2 (x_k) - \la_1 g_1 (x_{k-1}) - \la_2 g_2(x_{k-1})} = \\ = \sumkun f(\ze_k) \hs{\la_1
g_1 (x_k) - \la_1 g_1 (x_{k-1})} + \sumkun f(\ze_k) \hs{\la_2 g_2 (x_k) - \la_2 g_2 (x_{k-1})} =  \\ =
\la_1 \si(f;g_1;T_{\ze}) + \la_2 \si(f;g_2; T_{\ze}) = \la_1 \Psi_f^{g_1} + \la_2 \Psi_f^{g_2}.}
\end{proof}

\subsubsection{Определение интеграла Стилтьеса}

\begin{dfn}{1}
Число $I\in\R$ --- \emd{интеграл Стилтьеса} функции $f$ по $g$ отрезка $[a,b]$, если $I=\liml{d(T)\ra0} \Psi_f^g$.

Обозначение:
$$I=\liml{d(T)\ra0} \Psi_f^g = \intl{a}{b} f(x)\,dg(x).$$
\end{dfn}

\begin{dfn}{2}
Говорят, что функция $f$ \emd{интегрируема (по Стилтьесу)} на отрезке $[a,b]$ по $g$,
если существует интеграл $\intl{a}{b} f(x)\,dg(x)$, где $f\in S_g[a,b]$.

Если $g(x)=x$ на $[a,b]$, то $\si(f;g;T_{\ze}) = \si(f,T_{\ze})$ и
$\intl{a}{b} f(x)\,dg(x) = \intl{a}{b} f(x)\,dx$.
\end{dfn}

\subsubsection{Свойство линейности интеграла Стилтьеса}
\begin{theorem}
Если $f_1$ и $f_2$ интегрируемы по $g$ на $[a,b]$, то для
произвольных чисел $\la_1,\la_2\in\R$ $f=\la_1 f_1 + \la_2 f_2$
интегрируема по $g$ на $[a,b]$ и справедливо:

\eqa{2}{\intl{a}{b} \hr{\la_1 f_1 (x) + \la_2 f_2 (x)}\, dg(x) =
\la_1 \intl{a}{b} f_1(x)\,dg(x) + \la_2 \intl{a}{b}
f_2(x)\,dg(x).}
\end{theorem}

\begin{proof}
По условию и определению 1, \eqa{3}{\exists \;
\liml{d(T)\ra0} \Psi_{f_1}^g = \intl{a}{b} f_1(x)\,dg(x)
\mbox{ и } \liml{d(T)\ra0} \Psi_{f_2}^g = \intl{a}{b}
f_2(x)\,dg(x).} Так как, по свойству 1 (пункт 5.1), $\Psi_f^g =
\la_1 \Psi_{f_1}^g + \la_2 \Psi_{f_2}^g$, то, в силу~(3) и свойства
линейности предела функции по базе, $\exists \;
\liml{d(T)\ra0} \Psi_f^g = \la_1 \liml{d(T)\ra0}
\Psi_{f_1}^g + \la_2 \liml{d(T)\ra0} \Psi_{f_2}^g = \la_1
\intl{a}{b} f_1(x)\,dg(x) + \la_2 \intl{a}{b} f_2(x)\,
dg(x)$. По определению 1, $\intl{a}{b} f(x)\,dg(x) =
\liml{d(T)\ra0} \Psi_f^g = \la_1 \intl{a}{b}
f_1(x)\,dg(x) + \la_2 \intl{a}{b} f_2(x)\,dg(x)$.
\end{proof}

\begin{theorem}
Если $f$ интегрируема по $g_1$ и $g_2$ на отрезке $[a,b]$, то для
произвольных чисел $\la_1,\la_2\in\R$ $f$ будет интегрируемой по
$g=\la_1g_1+\la_2g_2$ и справедливо:

\eqa{4}{\intl{a}{b} f(x)\,d(\la_1g_1(x) + \la_2g_2(x)) = \la_1
\intl{a}{b} f(x)\,dg_1(x) + \la_2 \intl{a}{b}
f(x)\,dg_2(x).}
\end{theorem}

\begin{proof}
По условию и определениям 1 и 2, \eqa{5}{\exists \;
\liml{d(T)\ra0} \Psi_f^{g_1} = \intl{a}{b} f(x)\,dg_1(x)
\mbox{ и } \liml{d(T)\ra0} \Psi_f^{g_2} = \intl{a}{b}
f(x)\,dg_2(x).}

Так как $\Psi_f^g = \la_1 \Psi_f^{g_1} + \la_2 \Psi_f^{g_2}$ (по
свойству 2) и предел по базе $\boxed{d(T)\ra0}$ обладает свойством
линейности, то, с учётом~(5): \equ{\exi  \liml{d(T)\ra0}
\Psi_f^g = \la_1 \liml{d(T)\ra 0} \Psi_f^{g_1} + \la_2
\liml{d(T)\ra0} \Psi_f^{g_2} = \la_1 \intl{a}{b}
f(x)\,dg_1(x) + \la_2 \intl{a}{b} f(x)\,dg_2(x).}

По определению 1, $\intl{a}{b} f(x)\,dg(x) =
\liml{d(T)\ra0} \Psi_f^g = \la_1 \intl{a}{b}
f(x)\,dg_1(x) + \la_2 \intl{a}{b} f(x)\,dg_2(x)$.
\end{proof}

\subsubsection{Существование интеграла Стилтьеса}

\begin{thh}
Всякая функция, непрерывная на отрезке $[a,b]$, интегрируема по
любой функции $g$ ограниченной вариации на $[a,b]$.
\end{thh}

Так как $g=g_1-g_2,\;g_i,\; i=(1,2) \uparrow$ на $[a,b]$, то, в силу
свойства линейности интеграла Стилтьеса,

\begin{thh}
Всякая функция, непрерывная на $[a,b]$, интегрируема по любой
функции $g$, возрастающей на $[a,b]$.
\end{thh}

\begin{proof}
Доказательство --- дополнительный материал на распечатках.
\end{proof}

\subsubsection{Интегрируемость по частям в интеграле Стилтьеса}

\begin{thh}
Если $f$ интегрируема по $g$ на $[a,b]$, то $g$ будет интегрируема
по $f$ и справедливо: \eqa{6}{\intl{a}{b} f(x)\,dg(x) = f(b)g(b)
- f(a)g(a) - \intl{a}{b} g(x)\,df(x).}
\end{thh}

\begin{proof}
Рассмотрим произвольное $T_{\ze}[a,b]\colon a=x_0<\dots<x_n=b$ и
$\ze=\hc{\ze_1,\dots,\ze_n}$, где $\ze_k\in\De_k=[x_{k-1},x_k],\;
k=\ol{1,n}$. По формуле~(1),
\ml{\si(g;f;T_{\ze}) = \sumkun g(\ze_k) \hs{f(x_k) - f(x_{k-1})} = \\=
g(\ze_1) \hs{f(x_1) - f(x_0)} + g(\ze_2) \hs{f(x_2) - f(x_1)} + \ldots + g(\ze_n) \hs{f(x_n) -
f(x_{n-1})} = \\ = -f(x_0)g(x_0) - f(x_0) \hs{g(\ze_1) - g(x_0)} -
f(x_1) \hs{g(\ze_2) - g(\ze_1)} - \ldots - f(x_n)\hs{g(x_n) -
g(\ze_n)} + f(x_n) g(x_n) = \\ = -f(x_0)g(x_0) + f(x_n)g(x_n) -
\sum\limits_{k=1}^{n+1} f(x_{k-1}) \hs{g(\ze_n) -
g(\ze_{n-1})},\mbox{ (7) }} где переобозначено $\ze_0=a$ и
$\ze_{n+1}=b$.

Рассмотрим $T'_{\mu}\colon a=\ze_0 \le \dots \le \ze_{n+1}=b$ и
набор $x=\hc{x_0,\dots,x_n}$, тогда~(7) переходит в~(8):

\eqa{8}{\si(g;f;T_{\ze}) = f(b)g(b) - f(a)g(a) - \si(f;g;T'_{\mu}).}

Диаметр $d(T')=\max\limits_{0\le k\le n+1} \hm{\ze_k - \ze_{k-1}} =
(\ze_l - \ze_{l-1}) \le \De x_l + \De x_{l+1} \le 2d(T_{\ze})$. Таким
образом, $d(T'_{\mu}) \le 2d(T_{\ze})$.

По условию теоремы, $\exi  I = \intl{a}{b} f(x)\,dg(x) =
\liml{d(T)\ra0} \si(f;g;T_{\ze})$. Так что для любого $\ep>0
\exi \de>0\colon$ \eqa{9}{\hm{I - \si(f;g;T_{\ze})} < \ep, \;
\fa T_{\ze} \colon d(T_{\ze}) < \de.}

Пусть $\de'=\frac{\de}2 > 0$ и рассмотрим произвольное размеченное
разбиение $T_{\ze}\colon d(T_{\ze}) < \de'$. Тогда $d(T'_{\mu}) \le
2d(T_{\ze}) < 2\de' = \de$, следовательно (по формуле~(9)), $\hm{I -
\si(f;g;T'_{\mu})} < \ep$.

По~(8), $\hm{\si(g;f;T_{\ze}) - \hs{f(b)g(b)-f(a)g(a)-I}} =
\hm{\si(f;g;T'_{\mu})-I} < \ep$ для всех разбиений $T_{\ze} \colon
d(T_{\ze}) < \de', \; \de'=\de'(\ep)>0$, то есть $f(b)g(b) -
f(a)g(a) - I = \liml{d(T)\ra0} \si(g;f;T_{\ze}) = \intl{a}{b}
g(x)\,df(x)$, что равносильно~(6).
\end{proof}

\begin{imp*}
Если $g\in\Rc[a,b]$, то $\intl{a}{b} x\,dg(x) = bg(b)-ag(x) -
\intl{a}{b} g(x)\,dx$.
\end{imp*}

\subsubsection{Вычисление интеграла Стилтьеса}

\begin{thh}
Если $g$ имеет на $[a,b]$ ограниченную производную $g'\in\Rc[a,b]$,
то для любой функции $f\in\Cc[a,b]$, справедливо
\eqa{10}{\intl{a}{b} f(x)\,dg(x) = \intl{a}{b}
f(x)g'(x)\,dx.}
\end{thh}

\begin{proof}
Так как $g'\in\Rc[a,b]$, то $g'$ --- ограничена на $[a,b]$, и,
следовательно, $g$ удовлетворяет условию Липшица на $[a,b]$ и $g$
имеет ограниченную вариацию на $[a,b]$, следовательно, интегралы в
формуле (10) существуют.

Так как $g'\in\Rc[a,b]$, то $\exi M>0\colon \hm{g'(x)} \le M,\;
x\in[a,b]$. Рассмотрим произвольное размеченное разбиение
$T_{\ze}[a,b]$ точками $a=x_0 < \dots < x_n = b$ и набор
$\ze=\hc{\ze_1,\dots, \ze_n}$, $\ze_k \in
\De_k=[x_{k-1},x_k],\;\hm{\De_k} = \De x_k = x_k - x_{k-1}, \;
k=\ol{1,n}$. Тогда, с учётом теоремы Лагранжа о конечных
приращениях, имеем \ml{\si(f;g;T_{\ze}) = \sumkun f(\ze_k) [g(x_k) -
g(x_{k-1})] = \sumkun f(\ze_k) \cdot g'(c_k)(x_k - x_{k-1}) = \\ =
\sumkun f(\ze_k) g'(c_k) \De x_k +  \sumkun g'(c_k) [f(c_k) -
f(c_k)] \De x_k = \si(f;g;T_{\ze}),\mbox{ (1) }} где $\ze_k,c_k \in
\De_k$.

Рассмотрим $T_c[a,b]\colon a=x_0 < \dots < x_n = b$ и
$c=\hc{c_1,\dots c_n},\; c_k\in\De_k, \; k=\ol{1,n}$. Тогда
$$\sumkun f(c_k) g'(c_k) \De x_k = \si(fg';T_c),$$
и так как произведение $fg'\in\Rc[a,b]$, то \eqa{2}{\liml{d(T)\ra0}
\sumkun f(c_k) g'(c_k) \De x_k = \liml{d(T)\ra0}
\si(fg';T_{\ze}) = \intl{a}{b} f(x)g'(x)\,dx.}

Оценка второй суммы в правой части~(1) даёт неравенство:
\eqa{3}{\hm{\sumkun g'(c_k) [f(\ze_k) - f(c_k)]\De x_k} \le \sumkun
\hm{g'(c_k)} [f(\ze_k) - f(c_k)] \De x_k \le M \sumkun \om(f;\De_k)
\De_k.}

Так как $f\in\Rc[a,b]$, $\liml{d(T)\ra0} \sumkun \om(f;\De_k)
\De x_k = 0$, следовательно, с учётом~(3), получаем, что
\eqa{4}{\liml{d(T)\ra0} \sumkun g'(c_k) [f(\ze_k) - f(c_k)]
\De x_k = 0.}

На основании~(1), (2) и~(4), заключаем существование
$\liml{d(T)\ra0} \si(f;g;T_{\ze}) = \intl{a}{b}
f(x)g'(x)\,dx = \intl{a}{b} f(x)\,dg(x)$.
\end{proof}

\begin{imp*}
Если $g\in\Rc[a,b]$, $f$ имеет $f'\in\Rc[a,b]$, то
\eqa{5}{\intl{a}{b} f(x)\,dg(x) = f(b)g(b)-f(a)g(a) -
\intl{a}{b} g(x)f'(x)\,dx.}
\end{imp*}

\begin{proof}
Так как $f'\in\Rc[a,b]$, то $f'$ ограничена на $[a,b]$ некоторым
числом $M>0,\; \hm{f'(x)} \le M, \; x\in[a,b]$, и, следовательно,
$f$ удовлетворяет условию Липшица с $L=M$, и, следовательно, $f$
имеет ограниченную вариацию на $[a,b]$, так что существует
$\intl{a}{b} g(x)\,df(x).$

По свойству интегрирования по частям, существует
\equ{\intl{a}{b} f(x)\,dg(x) = f(b)g(b)-f(a)g(a) -
\intl{a}{b} g(x)\,df(x) = f(b)g(b) - f(a)g(a) - \intl{a}{b}
g(x)f'(x)\,dx \mbox{ (по предыдущей теореме).}}
\end{proof}


\pagebreak

\hrule
\begin{center}\LARGE \bf Часть 4.\hfill М н о г о м е р н ы й\quad а н а л и з\end{center}
\hrule

\section{Непрерывные отображения нескольких действительных переменных}

\subsection{ Многомерное евклидово пространство}

\subsubsection{ Векторное пространство в $\R^m$}

Символ $\R^m =
\ub{\R\times\R\st\R}_{m \text{ раз}}$
($m$ экземпляров множества $\R$ действительных чисел, где $m\in\N$
--- фиксировано).

Множество $\R^m$ состоит из всех упорядоченных $m$--наборов
$(x^1,x^2,\ldots,x^m)$ чисел $x^i\in\R,\;i=\ol{1,m}$. Элементы
$(x^1,\ldots,x^m)$ множества $\R^m$ принято называть точками, а
числа $x^1,\ldots,x^m$ --- соответственно первой, второй, $\ldots$,
$m$--ой координатами точки $(x^1,x^2,\ldots,x^m)$. Точки в $\R^m$
часто будем обозначать одной буквой: в аналитических рассмотрениях
строчной буквой --- $x=(x^1,\ldots,x^m)$; в геометрических
рассмотрениях прописной --- $M$ или с указанием координат
$M(x^1,\ldots,x^m)$. При $m=1,2,3$ и иногда 4 индексация не
применяется: для $\R^1=\R$ обычна запись точки одной буквой; для
$\R^2=\R\times\R$ запись вида $(x,y)$; для $\R^3$ --- $(x,y,z)$; для
$\R^4$ --- $(x,y,z,t)$.

$\R^2$ отождествляем с координатной плоскостью, $\R^3$ отождествляем
с координатным пространством.

Но $\R^m$ есть не только множество. Оно наделяется некоторыми
математическими операциями, а именно:
\begin{nums}{-2}
\item в $\R^m$ вводится операция покоординатного сложения;
\item в $\R^m$ вводится умножение на действительные числа (называемые
скалярами).
\end{nums}

По определению полагают
\eqa{1}{(x^1,x^2,\ldots,x^m)+(y^1,y^2,\ldots,y^m):=(x^1+y^1,\ldots,x^m+y^m),\quad
\la\cdot(x^1,x^2,\ldots,x^m):=(\la x^1,\ldots,\la x^m), \la\in\R.}

Формула~(1) превращает $\R^m$ в линейное пространство.

Итак, $\R^m$, $m\ge 1$ --- \emd{векторное пространство}
над $\R$ (или действительные векторные пространства). В векторном
пространстве $\R^m$ с нулём $0=(0,\ldots,0)$ имеется стандартный
базис, образованный векторами $\ol{e}_1=(1,0,\ldots,0),\;
\ol{e}_2=(0,1,\ldots,0),\ldots,\ol{e}_m=(0,0,\ldots,1)$,
в котором для любой точки $x=(x^1,\ldots,x^m)\in \R^m$ справедливо
представление: $x=x^1 \ol{e}_1+x^2\ol{e}_2+\ldots + x^n
\ol{e}_n =: x^i \ol{e}_i$ и это представление
единственно. Так как $\R^m$ --- $m$--мерное векторное пространство,
отметим также, что элементы пространства $\R^m$ будем называть как
точками, так и векторами.

% Следовало бы пояснить, как понимать эту запись... Прим. ред.
\begin{note}
Запись $x = x^i\ol e_i$, в которой индексы у координат верхние, а у векторов\т нижние, подразумевает
суммирование по повторяющемуся индексу (тензорная запись, придуманная Эйнштейном).
\end{note}

Поскольку в $\R^2$ и $\R^3$ точки с координатами $(x,y)\hs{(x,y,z)}$
можно называть радиус-векторами, то возникает понятие векторного
пространства.

\subsubsection{ Скалярное произведение}
\begin{dfn}{1}
\emd{Скалярным произведением} в действительном векторном
пространстве $X$ называют функцию, относящую каждой упорядоченной
паре векторов $(u,v)$ из $X$ действительное число. Обозначим его
$\ha{u,v}$ так, что выполнены следующие условия (аксиомы скалярного
произведения):
\begin{nums}{-2}
\item $\ha{x+y,z}=\ha{x,z}+\ha{y,z}$ для всех $(x,y,z)\in X^3$;
\item $\ha{\la x,z} = \la\ha{x,z}$ для всех $(x,y,\la)\in X^2\times\R$;
\item $\ha{x,y} = \ha{y,x}$ для всех $(x,y)\in X^2$;
\item если $x\in X$ и $x\ne0$, то $\ha{x,x}>0$.
\end{nums}

Первое условие распространяется по индукции на любые конечные суммы:

$\ha{x_1+x_2+\ldots+x_k,z} = \ha{x_1,z}+\ha{x_2,z} \spl
\ha{x_k,z}, \; k\in\N$.

Из первого и второго условия следует линейность скалярного
произведения по первому множителю при каждом фиксированном значении
второго: $\ha{\la_1 x_1 \spl \la_k x_k,y} = \la_1 \ha{x_1,y} +
\ldots + \la_k \ha{x_k,y},\; k\in\N, \; (x_1,\ldots,x_k,y)\in
X^{k+1}, \; (\la_1,\ldots,\la_k) \in \R^k$.

В соединении с третьим условием последнее влечёт линейность
скалярного произведения по второму множителю при каждом
фиксированном значении первого. Таким образом, скалярное
произведение $\ha{u,v}$ --- билинейная форма.

Беря во втором условии $\la=0$ получим, что $\ha{0,z}=0\;\forall
z\in X$, что влечёт $\ha{z,0}=0$ для всех $z\in X$; в частности,
$\ha{0,0}=0$. Таким образом, четвёртое условие означает, что
$\ha{x,x}\ge 0$ для всех $x\in X$, причём $\ha{x,x}=0$ тогда и
только тогда, когда $x=0$.
\end{dfn}

Легко проверяется, что формула \eqa{2}{\ha{x,y}=\sumium x^i y^i}
определяет скалярное произведение векторов $(x^1,\ldots,x^m)$ и
$(y^1,\ldots,y^m)$.

\begin{proof}
Условие~(3) очевидно выполнено.

Проверим условие~(1): $x=(x^1,\ldots,x^m), y=(y^1,\ldots,y^m),
z=(z^1,\ldots,z^m)$.

$x+y=(x^1+y^1,\ldots,x^m+y^m)$.

$\ha{x+y,z}=\sumium (x^i+y^i)z^i = \sumium x^iz^i + \sumium y^iz^i =
\ha{x,z} + \ha{y,z}$.

Условие~(2) проверяется аналогично.

Проверим условие~(4): если $x\in\R^m$ и $x\ne0$, то
$x=(x^1,\ldots,x^m)$ и $x^j\ne0$ для некоторого $j,\; 1\le j \le m$,
тогда $\ha{x,x} = \sumium x^i x^i = (x^1)^2 + (x^2)^2 \spl
(x^m)^2 \ge (x^j)^2 > 0.$
\end{proof}

На самом деле, в $\R^n$ существует бесконечное множество скалярных
произведений: $\ha{x,y} = \sumium \rho_i x^i y^i$, где все
$\rho_i>0, \; i=\ol{1,m}$, поэтому скалярное произведение,
определяемое формулой~(2) называют стандартным скалярным
произведением в $\R^m$ (евклидовым скалярным произведением) и $\R^m$
становится евклидовым векторным пространством.

\subsubsection{ Неравенство Коши\ч Буняковского}

\begin{thh}
Скалярное произведение $\ha{x,y}$ в действительном векторном
пространстве $X$ удовлетворяет неравенству \eqa{3}{\ha{x,y}^2 \le
\ha{x,x}\cdot\ha{y,y}.}
\end{thh}

\begin{proof}
Для произвольного числа $\la\in\R \ha{\la x - y, \la x - y} \ge0$.
Следовательно,

\eqa{4}{\la^2\ha{x,x} - 2\la \ha{x,y} + \ha{y,y} \ge 0.}

Если $x\ne0$, то $\ha{x,x}\ne0 (>0)$ и для
$\la=\frac{\ha{x,y}}{\ha{x,x}}$ неравенство имеет вид:
$-\frac{\ha{x,y}^2}{\ha{x,x}} + \ha{y,y} \ge 0$, откуда следует~(3).

Если $x=0$, то $\ha{x,x}=0$ и $\ha{x,y}=0$ для любого $y\in X$,
следовательно, (3) превращается в равенство.
\end{proof}

\begin{note}
Знак равенства в~(3) справедлив тогда и только тогда, когда векторы
$x$ и $y$ линейно зависимы.
\end{note}

\begin{proof}
\emph{Необходимость}. Если $x=0$, то доказано равенство~(3) для
любого $y$.

Если $x\ne0$, то $\ha{x,x}\ne0$ и $y=\la x$, так что
$\ha{x,y}=\la\cdot\ha{x,x}$ и $\la=\frac{\ha{x,y}}{\ha{x,x}}$.

Далее, $\ha{y,y}=\la\ha{x,y}=\frac{\ha{x,y}^2}{\ha{x,x}}$, итак
$\ha{y,y}=\frac{\ha{x,y}^2}{\ha{x,x}}$, то есть~(3) --- равенство.

\emph{Достаточность}. Обратно, пусть $\ha{x,y}^2=\ha{x,x}\cdot
\ha{y,y}$. Если $x\ne0$, то $\ha{x,x}\ne0$ и для
$\la=\frac{\ha{x,y}}{\ha{x,x}}$ соотношение~(4) имеет вид
$$\frac{\ha{x,y}^2}{\ha{x,x}}-2\frac{\ha{x,y}^2}{\ha{x,x}}+\ha{y,y} =
\frac{\ha{x,y}^2}{\ha{x,x}} - 2\frac{\ha{x,y}^2}{\ha{x,x}} +
\frac{\ha{x,y}^2}{\ha{x,x}}=0,$$ то есть $\ha{\la x-y,\la x-y}=0$,
откуда $\la x-y=0,\;y=\la x$. При $x=0$ линейная зависимость
очевидна.
\end{proof}
\begin{points}{-2}
\item $X=\R^m, \; m\ge1$ и $x=(x^1,\ldots,x^m), \;
y=(y^1,\ldots,y^m)\in\R^m$ и $\ha{x,y}=\sumium x^i y^i$. Тогда
$\ha{x,x}=\sumium (x^i)^2$, $\ha{y,y}=\sumium (y^i)^2$ и~(3) имеет
вид \eqa{3_a}{\hr{\sumium x^iy^i}^2 \le \sumium (x^i)^2 \sumium
(y^i)^2 \mbox { --- неравенство Коши.}}

\item $X=\Cc[a,b], \; a<b$ --- пространство относительно операции
сложения и умножения на $\la$.
\end{points}

Покажем, что для любых функций $x,y\in\Cc[a,b]$ формула
$$\ha{x,y} := \intl{a}{b} x(t)y(t)\,dt$$
задаёт скалярное произведение в $X=\Cc[a,b]$.

Скалярное произведение удовлетворяет следующим аксиомам:

\begin{nums}{-2}
\item $\ha{x+y,z}=\intl{a}{b} (x(t)+y(t))z(t)\,dt= \intl{a}{b}
x(t)z(t)\,dt + \intl{a}{b} y(t)z(t)\,dt = \ha{x,z}+\ha{y,z}$.
\item $\ha{\la x,z}=\intl{a}{b} \la x(t)z(t)\,dt = \la \ha{x,z}$.
\item $\ha{x,y}=\intl{a}{b} x(t)y(t)\,dt = \intl{a}{b}
y(t)x(t)\,dt = \ha{y,x}$.
\end{nums}

% 4) Это не аксиома, а свойство! Прим. ред.
Заметим, что нулевым вектором в пространстве $X=\Cc[a,b]$ служит
функция $\ph(t)=0$, $t\in[a,b]$. Пусть теперь $x\in\Cc[a,b]$ и
$x\ne0$, то есть функция $x(t)$ непрерывна на $[a,b]$ и $\exists\;
t_0\in[a,b]$, в котором $x(t_0)\ne0$. Предположим сначала, что
$t_0\in(a,b), \; a<t_0<b$. Согласно теореме о сохранении знака,
$\exists\;\de>0\colon x(t)\ne0 \; \forall t\in[t_0-\de,t_0+\de]
\subset [a,b]$. Тогда $\ha{x,x}=\intl{a}{b} x^2(t)\,dt =
\int\limits_a^{t_0-\de}x^2(t)\,dt +
\int\limits_{t_0-\de}^{t_0+\de}x^2(t)\,dt + \int\limits_{t_0+\de}^b
x^2(t)\,dt \ge \int\limits_{t_0-\de}^{t_0+\de} x^2(t)\,dt = x^2(\xi)
2\de > 0$, \par\noindent так как $\xi\in\hs{t_0-\de,t_0+\de}$.
Случаи $t_0=a,t_0=b$ проверяются аналогично.

\eqa{3_b}{\hr{\intl{a}{b} x(t)y(t)\,dt}^2 \le
\intl{a}{b} x^2(t)\,dt \cdot \intl{a}{b} y^2(t)\,dt.}

Неравенство принадлежит В.\,Я.\,Буняковскому\footnote{А также Шварцу, Коши и многим другим математикам,
получившим его независимо от Буняковского. В англоязычной литературе пишут <<Schwartz inequality>>. (прим. ред.)}.
\subsubsection{ Метрика в $\R^m$}

Рассмотрим в $\R^m$ стандартное скалярное произведение $\ha{x,y} =
\sumium x^iy^i$. Тогда $x-y=(x^1-y^1,\ldots,x^m-y^m)$ и
$\ha{x-y,x-y} = \sumium(x^i-y^i)^2$.

%%%% Давайте назовем всё своими именами! :)
Функция $d_m(x,y)=\sqrt{\ha{x-y,x-y}} = \sqrt{\sumium (x^i-y^i)^2}$, называемая расстоянием между векторами $x$ и $y$,
обладает свойствами:

\begin{nums}{-2}
\item $d_m(x,y)=0 \; \Lra \; x=y$;
\item $d_m(x,y) \le d_m(x,z)+d_m(y,z)$.
\end{nums}

\ml{d_m(x,y)=\hr{\sumium(x^i-y^i)^2}^\frac12 =
\hr{\sumium\hs{x^i-z^i+z^i-y^i}^2}^\frac12 \le \hr{ \sumium
\hs{\hm{x^i-z^i} + \hm{y^i - z^i}}^2}^\frac12 \le\\\le \hr{\sumium
(x^i-z^i)^2}^\frac12 + \hr{\sumium (y^i-z^i)^2}^\frac12 =
d_m(x,z)+d_m(y,z) \mbox { --- неравенство Минковского при $p=2$}.}

Если $m=1, \; \R^1=\R$, то $d_1(x,y)=\hm{x-y}$.

Если $m=2, \; \R^2=\R\times\R$, то
$d_2(x,y)=\sqrt{(x_1-x_2)^2+(y_1-y_2)^2}.$ %почему так? должно получаться
%\sqrt{(x_1-y_1)^2+(x_2-y_2)^2}...

\begin{thh}
$\R^m$ --- $m$--мерное евклидово и метрическое пространство.
\end{thh}

Для произвольной $x=(x^1,\ldots,x^m)\in\R^m$ число
$d_m(x,0)=\sqrt{\sumium (x^i)^2}$ называют \emph{длиной вектора}
$x\in\R$ (нормой $x$) и обозначают $\hn{x}_m =
\hr{\sumium(x^i)^2}^\frac12, \; \hn{x}_m = \sqrt{\ha{x,x}}$.

\subsubsection{ Углы. Ортогональность}

Рассмотрим произвольные $x,y\in\R^m, \; x\ne0,y\ne0$. Тогда
$\hn{x}_m>0, \; \hn{y}_m>0$ и из~(3) следует $\hm{\ha{x,y}} \le
\hn{x}_m \cdot \hn{y}_m$.

\equ{\hm{\frac{\ha{x,y}}{\hn{x}_m\hn{y}_m}} \le 1.}

Всякое число $\hm{r}\le1$ равно значению $\cos\al$ для некоторого
$\al\in[0;\pi]$.

\eqa{5}{\ha{x,y}=\hn{x}_m \hn{y}_m \cos\ph, \; \ph\in[0,\pi].}

Число $\ph$ называют \emd{углом} между векторами
$x,y\in\R^m$.

При $m=2,3$ формула~(5) была определением скалярного произведения.

Если $X=\Cc[a,b]$, то функции $x,y\in X$ ортогональны, если
$\intl{a}{b} x(t)y(t)\,dt=0$.

\subsection{ Множества в метрическом пространстве}

\subsubsection{ Метрические пространства}

\begin{dfn}{1}
\emd{Метрикой} (\emd{расстоянием}) в непустом множестве
$X$ называют функцию, ставящую в соответствие каждой упорядоченной
паре $(x,y)$ элементов из $X$ действительное число (назовём его
расстоянием от $x$ до $y$ и обозначим символом $d(x,y)$), такое, что
выполнены следующие условия (аксиомы метрики):
\end{dfn}
\begin{nums}{-2}
\item $d(x,y) \le d(x,z)+d(y,z)$ для произвольных $(x,y,z)\in X^3$;
\item $d(x,y)=0 \; \Lra \; x=y$.
\end{nums}


Если положить $z=x$ в неравенстве аксиомы~(1) и воспользоваться
аксиомой~(2), то получим $d(x,y) \le d(x,x) + d(y,x) = d(y,x)$ и
аналогично $d(y,x) \le d(x,y)$, то есть $d(x,y)=d(y,x)$ (расстояние
от $x$ до $y$ равно расстоянию от $y$ до $x$).

Если положить $y=x$ в неравенстве аксиомы~(1) и воспользоваться~(2),
то получим $0=d(x,x) \le 2d(x,z)$, откуда заключаем, что $d(x,z)
\ge0$ для произвольных $(x,z)\in X^2$.

$(1)\Rightarrow d(x,y) \le d(x,z) + d(z,y)$, то есть неравенство
треугольника.

\begin{dfn}{2}
Непустое множество с определённой на нём метрикой называется
\emph{метрическим пространством} и обозначается $(X;d)$.
\end{dfn}
Вместо $(\R^m,d_m)$ пишем $\R^m$, $m\ge1$ (как для множества
$\R^m$).

На $\R^m$ можно определить бесконечно много метрик, из которых
выделим одну:
\\
а) для произвольных $x=(x^1,\ldots,x^m), \; y=(y^1,\ldots,y^m)$ из
$\R^m$ функция $d_m^1(x,y) = \max\limits_{1\le i \le m}
\hm{x^i-y^i}$ определяет метрику.

\subsubsection{ Шары, $\de$---окрестности}

Рассмотрим произвольное метрическое пространство $(X;d)$. Множество
$\Uc(a;r)=\hc{x\in X \; | \; d(x,a)<r}$ называется
\emd{открытым шаром} в $X$ с центром $a\in X$ и радиусом
$r$.

Если $X=\R^1$, то $\Uc(a;r)=\hc{x\in\R^1 \; | \; \hm{x-a}<r} =
(a-r;a+r) \subset \R$.

При $m=2$, $X=\R^2$, $\Uc((a;b),r) = \hc{(x,y)\in\R^2 \; | \;
(x-a)^2 + (y-b)^2 < r^2}$.

При $m=3$, $X=\R^3$, $\Uc((a,b,c),r) = \hc{(x,y,z)\in\R^3 \; | \;
(x-a)^2 + (y-b)^2 + (z-c)^2 < r^2 }$ --- шар с центром в точке
$(a,b,c)$ и радиусом $r$.

\begin{dfn}{1}
\emd{$\de$---окрестностью} ($\de>0$) точки $a\in(X;d)$
называют открытый шар $\Uc(a,\de)$.
\end{dfn}
Если $(X;d)=(\R^m,d_m)=\R^m$, то при $m=1$
$\Uc(a,\de)=(a-\de,a+\de)$.

При $m=2$ имеем $\Uc((a,b),\de)$ --- окрестность радиуса $\de$.

\subsubsection{ Общее понятие окрестности}

\begin{dfn}{2}
\emd{Окрестностью точки} в метрическом пространстве назовём любое
подмножество метрического пространства, содержащее некоторую
$\de$---окрестность этой точки.
\end{dfn}

Так, множество $\Uc$ является окрестностью точки $a\in X$, если $\Uc
\subset X$ и $\Uc \supset\Uc(a,\de)$. Заметим, что $a\in\Uc(a,r)$
(ибо $d(a,a)=0<r$) и если $0<r_1<r_2$, то шар $\Uc(a,r_1) <
\Uc(a,r_2)$ (в силу неравенства $d(x,a)<r_1<r_2$). Любая
$\de$---окрестность точки есть её окрестность в смысле определения
2.

$B(a;r)=\hc{x\in X \; | \; d(x,a) \le r}$ --- замкнутый шар.

Также всё множество $X$ есть окрестность каждой своей точки $a\in
X$, так как $\Uc(a,\de)\subset X$.

В метрическом пространстве $(X;d)$ рассмотрим $\Uc(a,r)$ ---
открытый шар с центром $a$ и радиуса $r>0$. Тогда $a\in\Uc(a,r)$.

\begin{thn}{1}
Любой открытый шар в метрическом пространстве есть окрестность
каждой своей точки.
\end{thn}

\begin{proof}
Рассмотрим произвольную $\Uc(a,r), a\in X, r>0$ и рассмотрим
произвольную точку $x\in\Uc(a,r)$, так что $d(x,a)<r$ и
$\de=r-d(x,a)>0$. Докажем, что $\Uc(x;\de) \subset \Uc(a;r)$.

Действительно, для любой $y\in \Uc(x,\de) (d(y,x)<\de)$ справедливо
неравенство $d(y,a) \le d(y,x) + d(x,a) < r-d(x,a) + d(x,a) = r$,
так что $y\in\Uc(a,r)$ и, следовательно,
$\Uc(x,\de)\subset\Uc(a,r)$.
\end{proof}

\subsubsection{ Свойства окрестностей точек метрического
пространства}

\begin{thh}
Окрестности \textbf{точки} в метрическом пространстве обладают
свойствами:
\\
1) всякое множество метрического пространства, содержащее
окрестность точки --- окрестность.
\\
2) пересечение любых двух окрестностей точки образует окрестность
этой точки.
\\
3) точка принадлежит всем своим окрестностям.
\\
4) любая окрестность точки метрического пространства содержит такую
окрестность этой точки, которая будет окрестностью каждой своей
точки.
\\
5) (свойство отделимости): для любых различных точек метрического
пространства можно указать такие их $\de$---окрестности, которые не
пересекаются.
\end{thh}

\begin{proof}
Рассмотрим метрическое пространство $(X;d)$ и произвольную $a\in X$.
Обозначим символом $\Uc(a)$ систему всех окрестностей точки $a$ в
$(X;d)$.

1) Рассмотрим произвольное множество $\Uc\in\Uc(a)$ и произвольное
множество $\Vc\subset X$ такое, что $\Vc\supset\Uc$. По определению,
существует $\de>0$ такое, что $\Uc(a,\de)\subset\Uc\subset\Vc$,
следовательно, $\Uc(a,\de)\subset \Vc$, следовательно, $\Vc$ ---
окрестность по определению.

2) Рассмотрим произвольные $\Uc_1,\Uc_2\in\Uc(a)$. По определению,
$\exists\; \de_1>0, \; \de_2>0$ такие, что $\Uc(a,\de_i) \subset
\Uc_i (i=1,2)$. Положим $\de=\min(\de_1,\de_2)$. Тогда $\Uc(a,\de)
\subset \Uc(a,\de_i)$ и, следовательно, $\Uc(a,\de)\subset \Uc_i$.
Тогда $\Uc\subset\Uc_1\cap\Uc_2$, следовательно,
$\Uc_1\cap\Uc_2\in\Uc(a)$.

3) Рассмотрим произвольное $\Uc\in\Uc(a)$. По определению,
$\exists\; \de>0$ такое, что $\Uc(a,\de) \subset \Uc$. Так как точка
$a\in\Uc(a,\de) (d(a,a)=0)$, то $a\in\Uc$.

4) Рассмотрим произвольное $\Uc\in\Uc(a)$. По определению,
$\exists\;\de>0$ такое, что $\Uc(a;\de)\subset\Uc$. По теореме 1,
$\Uc(a,\de)$ удовлетворяет условиям утверждения.

5) Пусть $a,b\in X$ и $a\ne b$, так что $d(a,b)>0$. Положим
$\de=\frac{d(a,b)}3$ и предположим, что $\Uc(a,\de)\cap \Uc(b,\de) \ne \es$, так что $\exists\; x\in\Uc(a,\de) \cap \Uc(b,\de)$, для которого $d(x,a)<\de,\;d(x,b)<\de$, следовательно,
$d(a,b) \le d(a,x)+d(x,b) < \de+\de = \frac23 d(a,b) < d(a,b)$,
следовательно, $\Uc(a,\de) \cap \Uc(b,\de)=\es$.
\end{proof}

\begin{imp*}
Система окрестностей $\Uc(a)$ любой точки метрического пространства
образует базу в метрическом пространстве.
\end{imp*}

\begin{proof}
Прямое следствие свойств 2, 3 и определения базы.
\end{proof}

\subsubsection{ Открытые и замкнутые множества метрического
пространства}

\begin{dfn}{1}
Множество $G\subset(X;d)$ называется \emd{открытым множеством}, если каждая его точка входит в $G$ вместе с некоторой
своей $\de$\д окрестностью.
\end{dfn}

\textbf{Примеры:}
\\
1) $\Uc(a;r); \; a\in X, \; r>0$ --- открытое множество.
\\
2) $X$ --- открытое множество.
\\
3) $\es$ --- открытое множество.
\\
4) дополнение замкнутого шара $G(a;r)=X \wo B(a;r) = \hc{x\in
X \; | \; d(x;a)> r}$ --- открытое множество.

Действительно, рассмотрим произвольное $y \in G(a;r)$, то есть
$d(y,a)>r$ и положим $\de=d(y;a)-r>0$. Если $\Uc(y,\de)\cap
B(a;r) \ne \es$, то для $z\in\Uc(y,\de) \cap B(a;r)$ справедливо:
$d(z;y)<\de$ и $d(z;a)\le r$, тогда $d(y;a) \le d(z;y)+d(z;a) <
\de+r = d(y;a)$, следовательно, $\Uc(y,\de)\cap B(a;r)=\es$,
следовательно, $\Uc(y,\de)\subset G(a;r)$.


\begin{thh}
Открытые множества метрического пространства обладают свойствами:
\\
1) объединение любого непустого набора открытых множеств есть
открытое множество.
\\
2) пересечение двух (а следовательно, и любого конечного набора)
открытых множеств образует открытое множество.
\\
3) $\es$ --- открытое.
\\
4) всё $X$ --- открытое.
\end{thh}

\begin{proof}
Свойства~(3) и~(4) установлены в примерах (примеры 2.3).

1) Рассмотрим произвольный непустой набор $\hc{G_{\al}},\; \al\in A$
открытых множеств $G_{\al}$ в пространстве $X$ и множество
$G=\bigcup\limits_{\al\in A} G_{\al}$. Рассмотрим произвольное $x\in
G$, тогда $x\in G_{\al_0}$ для некоторого $\al_0\in A$ и $G_{\al_0}$
--- окрестность точки $x$ в пространстве $(X;d)$, ибо $G_{\al_0}$
--- открытое множество. Тогда $G_{\al_0}\subset G$, следовательно,
$G$ --- открытое множество.

2) Рассмотрим произвольные открытые множества $G_1,G_2$ в $(X;d)$ и
множество $G=G_1\cap G_2$. Если $G=\es$, то $G$ --- открытое.
Если $G\ne\es$, то для любой его точки $x\in G = G_1\cap G_2$, то
есть $x\in G_1$ и $x\in G_2$. Так как $G_i,\; i=1,2$ --- открытое,
то по определению $G_i$ --- окрестности точки $x$. Тогда $G_1\cap G_2$ также окрестность точки $x$, то есть $G$
--- открытое множество.
\end{proof}

\begin{dfn}{2}
Множество $F$ метрического пространства $(X;d), \; F\subset X$,
называется замкнутым множеством, если его дополнение $X\wo F$
--- открытое множество.
\end{dfn}

\textbf{Примеры:}
\\
1) Замкнутый шар $B(a;r), \; a\in X, \; r>0.$
\\
2) $X$ --- замкнутое множество.
\\
3) $\es$ --- замкнутое.

\begin{thh}
Замкнутые множества в $(X;d)$ обладают свойствами:
\\
1) пересечение произвольного непустого набора замкнутых множеств
образует замкнутое множество.
\\
2) объединение любых двух замкнутых множеств есть замкнутое
множество.
\\
3) $\es$ --- замкнуто.
\\
4) $X$ --- замкнуто.
\end{thh}

\begin{proof}
Свойства 3 и 4 установлены выше.

1) Рассмотрим произвольный набор $\hc{F_{\al}}, \; \al\in A$
замкнутых множеств в $(X;d)$ и $F=\bigcap\limits_{\al\in A}
F_{\al}$. Тогда $\Cs F = X \wo F = \bigcup\limits_{\al\in A}
(X\wo F_{\al})$ и $X\wo F_{\al}=G_{\al}$ --- открытые
множества. Согласно предыдущей теореме --- свойство 1 --- множество
$\bigcup\limits_{\al\in A} (X\wo F_{\al}) = \Cs F$ ---
открытое множество, следовательно, по определению 2, множество $F$
--- замкнутое.

2) Рассмотрим произвольные замкнутые множества $F_1$ и $F_2$ в
$(X;d)$ и $F=F_1\cup F_2$. Тогда $G_i=X\wo F_i (i=1,2)$,
согласно определению 2 --- открытые множества и $G_1\cap G_2 =
X\wo (F_1\cup F_2)$ --- открытое множество, следовательно,
$F$ --- замкнутое множество.
\end{proof}

Рассмотрим произвольное метрическое пространство $(X;d)$, точку
$a\in X$ и произвольное число $r>0$. Множество $S(a,r)=\hc{x\in X \;
| \; d(x,a)=r}$ называют сферой с центром $a$ и радиусом $r$. Так
как $X\wo S(a;r) = \Uc(a;r)\cup G(a;r)$ и множества
$\Uc(a;r)$ и $G(a;r)$ --- открытые, то $S(a;r$) --- замкнутое
множество.

\subsubsection{ Критерий замкнутости множества}

\begin{dfn}{1}
Точка $a$ метрического пространства $(X;d)$ называется точкой
прикосновения для множества $E\subset X$, если любая её окрестность
$\Uc(a)$ пересекает множество $E$, то есть $\Uc(a)\cap E \ne
\es$.
\end{dfn}

\begin{thh}
Множество $F$ метрического пространства $(X;d)$ замкнуто тогда и
только тогда, когда оно содержит все свои точки прикосновения.
\end{thh}

\begin{proof}
\emph{Необходимость.} Рассмотрим произвольное замкнутое множество
$F$ в метрическом пространстве $(X;d)$, так что его дополнение $G$
--- открытое множество. Следовательно, $G$ --- окрестность каждой
своей точки $x\in G$ и $G\cap F = \es$. Согласно обращению
определения 1, каждая точка $x\in G$ не является точкой
прикосновения множества $F$, другими словами --- $F$ содержит все
свои точки прикосновения.

\emph{Достаточность.} Пусть множество $F\subset X$ содержит все свои
точки прикосновения. Обозначим $X\wo F = G$. Так как $F\cap G =
\es$, то все точки $x\in G$ не являются точками прикосновения
множества $F$. Согласно определению 1, для любой $x\in G$ существует
такая окрестность $\Uc_1(x)$, что $\Uc_1(x)\cap F=\es$. Но $\Uc_1(x)
\subset X$ и, следовательно, $\Uc_1(x) \subset G$, так что $G$ ---
окрестность каждой своей точки, то есть открытое множество,
следовательно, $F$ --- замкнутое множество.
\end{proof}
\subsubsection{ Кубические окрестности в пространстве $\R^m$}

В метрическом пространстве $(X;d)=(\R^m,d_m)=\R^m, \; m\ge1$, кроме
открытых шаров используются открытые кубы. Рассмотрим произвольную
точку $a=(a^1,\ldots,a^m) \in \R^m$ и произвольное число $h>0$.
Множество \equ{Q(a;h) = \hc{x=(x^1,\ldots,x^m)\in\R^m \; | \;
\hm{x^i;a^i}<h, \; i=\ol{1,m}} = \hc{x\in\R^m \; | \;
d_m^1(x;a)<h}} называют открытым кубом с центром в точке $a$ и
стороной $2h$. Если $m=1$, это интервал $(a-h,a+h)$. Если $m=2$, то
$Q((a,b),h) = \hc{(x,y)\in\R^2 \; | \; a-h < x < a+h; \; b-h < y <
b+h }$ --- квадрат на $\R^2$. Если $m=3$, то $Q((a,b,c),h)$ --- куб.

Так как справедлива формула \eqa{1}{d_m^1(x,y) \le d_m(x,y) \le
\sqrt{m} d_m^1 (x,y),} где $x=(x^1,\ldots,x^m)\in\R^m, \;
y=(y^1,\ldots,y^m)\in\R^m$, то $Q(a,h)\supset \Uc(a,h)$, так как
$d_m(x,h)$ влечёт $d_m^1(x,a)<h$. Следовательно, каждый открытый куб
$Q(a,h)$ есть окрестность каждой своей точки $a\in X$.

Так же, как и в случае открытого шара, можно доказать, что каждый
открытый куб есть окрестность каждой своей точки. Неравенство~(1)
определяет свойство эквивалентности $d_m$ и $d_m^1$ в $\R^m$, так
что каждый открытый шар $Q(a,r)$ содержится в открытом шаре
$Q(a,\sqrt{m}h)$, так как неравенство $d_m^1(x,a)<h$ влечёт
$d_m(x,a) < \sqrt{m}\cdot h$.

\subsubsection{ Компакты в метрическом пространстве}

Рассмотрим метрическое пространство $(X;d)$ и непустой набор
$\hc{G_{\al}},\;\al\in A$ открытых множеств $G_{\al} \subset X$.
Набор $G_{\al},\;\al\in A$ называют \emd{открытым покрытием} множества $E\subset X$, если $E\subset
\bigcup\limits_{\al\in A}G_{\al}$, то есть для любой точки $x\in E
\; \exists \; G_{\al},\; \al \in A$, что $x\in G_{\al}$. Если
множество индексов $A$ --- конечно, то открытое покрытие $G_{\al},\;
\al\in A$ множества $E$ называют \emd{конечным открытым покрытием} множества $E$.

\begin{dfn}{2}
Множество $C$ из $(X;d)$ называют \emd{компактом} (или
компактным множеством), если любое открытое покрытие $G_{\al},\;
\al\in A$ множества $C$ содержит конечное множество множеств
$G_1,\ldots,G_n$, которое образует конечное покрытие множества $C$,
то есть $C\subset \bigcup\limits_{k=1}^n G_k$.
\end{dfn}

\textbf{Примеры:}
\\
1) любое конечное множество в пространстве $(X;d)$ --- компакт.
\\
2) если $(X;d)=\R^1$, то любой отрезок $[a,b]$ --- компакт в $\R^1$.
\\
Если $(X;d)=\R^m$ и существуют точки $a=(a^1,\ldots,a^m),\;
b=(b^1,\ldots,b^m)$, где $a^i<b^i$, $i=\ol{1,m}$, то множество
$I=\hc{x\in\R^m \; | \; a^i \le x^i \le b^i,\; i=\ol{1,m}}$ называют
\emd{$m$--мерным параллелепипедом} (или $m$--мерным \emd{брусом}).

\begin{stm}
Любой $m$--мерный брус --- компакт.
\end{stm}

\begin{proof}
Предположим, что существует брус $I$ и бесконечное открытое покрытие
его множествами $G_{\al},\; \al\in A$, то есть $I\subset
\bigcup\limits_{\al\in A} G_{\al}$, такие, что $I$ не допускает
конечного покрытия множествами $G_{\al}$.

Разделим каждый $[a^i,b^i]$ на два равных отрезка, так что брус $I$
разобьётся на $2^m$ равных $m$--мерных бруса, из которых выберем
такой брус, который обладает свойством исходного бруса $I$ и
обозначим его $I_1\subset I$. С брусом $I_1$ поступим так же, как и
с брусом $I$ и выберем брус $I_2$ такой, что $I_2\subset I_1\subset
I$ и $I_2$ обладает свойством брусом $I$ и $I_1$. Продолжая эту
процедуру, получим набор брусов $I\supset I_1\supset I_2\supset
\ldots \supset I_n \supset \ldots$, обладающих свойством бруса $I$ и
имеющих диаметры $\diam I_n \ra 0$ при $n\ra +\infty$. Если
обозначим $I_n = \hc{x\in\R^m\; | \; a_n^i \le x^i \le b_n^i, \;
i=\ol{1,m}},\; n\in \N$, то для каждого $i,\; 1\le i\le m$
отрезки $[a_n^i;b_n^i],\; n\in\N$ образуют систему стягивающихся
отрезков. По теореме о системе стягивающихся отрезков, для каждого
$i, \; 1\le i\le m$ существует $\ze^i \in \bigcap\limits_{n\in\N}
[a_n^i;b_n^i], \; 1\le i \le m$. Рассмотрим
$\ze=(\ze^1,\ldots,\ze^m)$ и убеждаемся, что $\ze\in
\bigcap\limits_{n\in\N} I_n$. Так как $\ze\in I$, то существует
открытое множество $G_{\al}, \; \al\in A$ такое, что $\ze\in
G_{\al}$. Открытое множество $G_{\al}$ содержит открытый шар
$\Uc_m(\ze,\de), \; \de>0$ такое, что $\ze\in \Uc(\ze,\de) \subset
G_{\al}$. Так как $\diam I_n\ra0$, то для $\de>0$ существует
$N=N_{\de},\; N\in\N$ такое, что $I_N \subset \Uc(\ze;\de)$ для
любого $n\ge N$. Так что $I_n \subset \Uc(\ze;\de) \subset G_{\al},
\; n\ge N$, что противоречит свойству бруса $I_n$.
\end{proof}

\subsubsection{ Ограниченные множества метрического пространства}
Рассмотрим произвольное метрическое пространство $(X;d)$ и
произвольное множество $E\subset X$. Обозначим $d(E)=\sup\hc{
d(x_1,x_2) \; | \; x_1,x_2\in E}$ и назовём $d(E)$ диаметром
множества $E$, если $d(E)<+\infty$.

\begin{dfn}{3}
Множество $E$ метрического пространства $(X;d)$ называют
\emd{ограниченным}, если $E$ имеет конечный диаметр.
\end{dfn}

\begin{stm*}
$E$ ограничено в метрическом пространстве $(X;d)$ тогда и только
тогда, когда $\exists \; M>0$ такое, что $d(x,a) \le M$ для всех
$x\in E$ и некоторого $a\in X$.
\end{stm*}

\begin{proof}
Фиксируем $a\in X$. Тогда $d(x_1,x_2) \le d(x_1,a) + d(x_2,a) \le
2M$ для любых $x_1,x_2\in E$ и применяем определение 3.
\end{proof}

\begin{thh}
Любой компакт в метрическом пространстве есть ограниченное
множество.
\end{thh}

\begin{proof}
Рассмотрим произвольный компакт $C$ в $(X;d)$ и фиксируем
произвольную точку $a\in X$. Для любой $x\in C$ число $d(x;a) \ge 0$
и $\exists \; n\in\N$ такое, что $n>d(x,a)$, так что точка
$x\in\Uc(a,n)$. Следовательно, $C\subset \bigcup\limits_{n\in\N}
\Uc(a;n)$. Так как $C$ --- компакт, то $\exists \;
n_1,n_2,\ldots,n_p \in \N$, что $C\subset \bigcup\limits_{j=1}^p
\Uc(a;n_j)$. Обозначим $q=\max(n_1,\ldots,n_p)\in\N$. Тогда
$\Uc(a;n_j)\subset \Uc(a,q)$, $j=\ol{1,p}$ и, следовательно,
$C\subset \bigcup\limits_{j=1}^p \Uc(a;n_j) \subset \Uc(a;q)$, то
есть $C$
--- ограниченное множество.
\end{proof}

\subsubsection{ Компактность и замкнутость}

\begin{thn}{1}
Всякий компакт метрического пространства есть замкнутое множество.
\end{thn}

\begin{proof}
Рассмотрим произвольный компакт $C$ в $(X;d), \; C\subset X$ и
рассмотрим произвольную точку $a\in X\wo C$. Так как $x\ne a$
для всех $x\in C$ и метрическое пространство $(X;d)$ --- отделимое
пространство, то существуют открытые шары $\Uc(x)$ и $\Oc(a)$, не
пересекающиеся. Поскольку $C\subset \bigcup\limits_{x\in C} \Uc(x)$,
то система шаров образует покрытие компакта $C$, которое обязано
содержать некоторое конечное покрытие $\Uc_1(x_1),\ldots,\Uc(x_n),
x_j\in C, \; j=\ol{1,n}$ и $C=\bigcup\limits_{j=1}^n
\Uc(x_j)$.

Для каждого $\Uc(x_j)$ рассмотрим $\Oc_j(a)$, что $\Uc(x_j)\cap \Oc_j(a) = \es$. Так как $\bigcap\limits_{j=1}^n \Oc_j(a)=\Oc(a)$
--- окрестность точки $a$ (по свойству окрестностей) и $\Oc(a) \cap \Uc(x_j) = \es$, то $\Oc(a)\cap \bbr{\bigcup\limits_{j=1}^n
\Uc(x_j)} = \es$ и, следовательно, $\Oc(a)\cap C = \es.$ Итак,
$\Oc(a) \subset X\wo C$ и, следовательно, $X\wo C$ --- открытое, тогда $C$ --- замкнутое множество.
\end{proof}

\begin{thn}{2}
Всякое замкнутое подмножество любого компакта в метрическом
пространстве есть компакт.
\end{thn}

\begin{proof}
Рассмотрим произвольный компакт $C$ в метрическом пространстве
$(X;d)$, произвольное замкнутое множество $F\subset C$ и
произвольное открытое покрытие $\hc{G_{\al}},\al\in A$ множества
$F$.

Добавляя к $G_{\al},\al\in A$ открытое $X\wo F=G$, получим
открытое покрытие всего множества $X$, и, в частности, компакта
$C\subset X$. Так как $C$ --- компакт, существует конечное его
покрытие множествами $G_1,\ldots,G_n; \; C\subset
\bigcup\limits_{k=1}^n G_k$.

Если множество $G$ входит в набор этих множеств $G_1,\ldots,G_n$, то
после его удаления из этого набора получим конечное покрытие
множества $F\subset C$, так что $F$ --- компакт по определению.
\end{proof}

\begin{thn}{3 (критерий компакта в $\R^m$)}
Множество $C$ метрического пространства $\R^m,m \ge 1$ является
компактом только и только тогда, когда $C$ ограничено и замкнуто в
$\R^m$.
\end{thn}

\begin{proof}
\textbf{Необходимость}. Если $C$ --- компакт в метрическом
пространстве $(X;d)$, то по теореме 1 и теореме пункта 2.9,
множество $C$ замкнуто и ограничено.

\textbf{Достаточность}. Рассмотрим произвольное замкнутое и
ограниченное множество $C$ в $\R^m,m\ge1$. Согласно свойствам метрик
$d_m$ и $d_m^1$ в $\R^m$, существует $m$--мерный брус $I$, который
содержит ограниченное множество $C$, $C\subset \Uc(a;r)\subset I$.
Так как брус $I$ --- компакт в $\R^m$, то $C$ --- компакт в $\R^m$
по теореме 2.
\end{proof}

\begin{ex}
Все замкнутые шары и кубы в $\R^m$ --- компакты.
\end{ex}

\begin{ex}
Все сферы в $\R^m$ --- компакты.
\end{ex}

\subsection{ Предел и непрерывность}
\subsubsection{ Сходящиеся последовательности в метрическом пространстве}
Рассмотрим произвольное метрическое пространство $(X;d)$. Всякое
отображение $\ph\colon\N\ra X$ определяет
\emd{последовательность} $(x_n)$ точек $x_n=\ph(n),n\in\N$
в $(X;d)$. Последовательность $(x_n)$ назовём
\emd{сходящейся} в $(X;d)$, если существует некоторый
элемент $a\in X$, что $\liml{n\ra\bes} d(x_n,a)=0$. Точка
$a\in X$ называется пределом последовательности $(x_n)$ и
обозначается $a=\liml{n\ra\bes} x_n$. Если $(X;d)=\R^1$,
то $(x_n)$ становится числовой последовательностью --- $a\in\R$ ---
число и $d(x,y)=\hm{x-y}, \; x,y,\in\R^1$, так что свойство
$\liml{n\ra\bes} d(x_n,a)=0$ принимает вид
$\liml{n\ra\bes}\hm{x_n-a}=0$, что равносильно свойству
$a=\liml{n\ra\bes} x_n$ (число $a$ есть предел числовой
последовательности ($x_n$)).

\begin{thh}
Всякая сходящаяся последовательность в метрическом пространстве
имеет единственный предел.
\end{thh}

\begin{proof}
Предположим, что в метрическом пространстве $(X;d)$
последовательность ($x_n$) имеет $\liml{n\ra\bes} x_n=a$,
$\liml{n\ra\bes} x_n=b$ и $a\ne b$. Так как $d(a,b)>0$, то
для числа $\de=\frac12 d(a,b)>0$, по определению, существуют
натуральные числа $N_i\in\N,i=1,2$, что $d(x_n,a)<\de$ для всех
$n\in\N,n\ge N_1$ и $d(b,x_n)<\de$ для всех $n\in\N, n\ge N_2$. В
частности, для $N=\max(N_1,N_2)$ справедливо $d(x_N,a)<\de$ и
$d(x_N,b)<\de$.

Поэтому расстояние $d(a;b)\le d(x_N,a)+d(x_N,b) < 2\de=d(a,b)$, что
невозможно.
\end{proof}

\subsubsection{ Сходящиеся последовательности в $\R^m$}

Пусть $(X;d)=\R^m,m\ge1$. Тогда для любой последовательности $(x_n)$
точек $x_n=(x_n^1,\ldots,x_n^m)\in\R^m,n\in\N$ определены
координатные числовые последовательности $(x_n^i),i=\ol{1,m}$.
Последовательность $(x_n)$ будет сходящейся в $\R^m$, если
существует точка $a=(a^1,\ldots,a^m)\in\R^m$, что
$\liml{n\ra\bes} d_m(x_n,a)=0$, то есть
$$\liml{n\ra\bes} \sqrt{\sumium (x^i-a^i)^2}=0.$$

\begin{thn}{1 (критерий сходимости последовательности)}
Последовательность $(x_n)$,
$x_n=(x_n^1,\ldots,x_n^m)\in\R^m$, $n\in\N$, сходится к точке
$a=(a^1,\ldots,a^m)\in\R^m$ тогда и только тогда, когда каждая
координатная числовая последовательность $(x_n^i),i=\ol{1,m}$,
имеет $\liml{n\ra\bes} x_n^i=a^i$.
\end{thn}

\begin{proof}
\textbf{Необходимость}. Пусть $a=\liml{n\ra\bes}x_n$, то
есть $\liml{n\ra\bes} d_m(x_n,a)=0$. Так как $d_m^1\le
d_m$, то
$$\hm{x_n^i-a^i} \le d_m^1(x_n;a) \le d_m(x_n,a)$$
для каждого $1\le i \le m$. Следовательно, $\liml{n\ra\bes}
\hm{x_n^i-a^i}=0,i=\ol{1,m}$, или
$a^i=\liml{n\ra\bes} x_n^i$.

\textbf{Достаточность}. Пусть $\liml{n\ra\bes}
x_n^i=a^i,\;i=\ol{1,m}$. Тогда $\liml{n\ra\bes} d_m^1
(x_n,a)=0$. Так как $d_m\le \sqrt{m}d_m^1$, то $d_m(x_n,a) \le
\sqrt{m} d_m^1 (x_n,a)$, и, следовательно,
$\liml{n\ra\bes} d_m(x_n,a)=0$ или
$a=\liml{n\ra\bes} x_n$.
\end{proof}

Последовательность $(x_n^i), \; 1\le i \le m$, называется
фундаментальной (или последовательностью Коши), если для
произвольного $\ep>0$ существует $N_i\in\N, \; N_i=N_i(\ep), \; 1\le
i \le m $, что $\hm{x_p^i-x_q^i}<\ep$ для всех $p,q\in\N, \; p,q\ge
N_i, \; 1\le i \le m$.

\begin{dfn}{1}
Последовательность $(x_n), \; x_n\in\R^m, \; n\in\N$. называется
\emd{фундаментальной} последовательностью (или
последовательностью Коши), если для произвольного $\ep>0$ существует
$N\in\N,\; N=N(\ep)$, что $d_m(x_p,x_q)<\ep$ для всех $p,q\in\N, \;
p,q\ge N$.
\end{dfn}

\begin{thn}{2 (критерий фундаментальной последовательности в $\R^m$)}
Последовательность $(x_n), \; x_n=(x_n^1,\ldots,x_n^m)\in\R^m, \;
n\in\N$, будет фундаментальной в $\R^m$ тогда и только тогда, когда
каждая её координатная последовательность $(x_n^i)$ ---
фундаментальная.
\end{thn}

\begin{proof}
\textbf{Необходимость}. Пусть $(x_n), \; x_n\in\R^m, \; n\in\N$ ---
фундаментальная в смысле определения 1, то есть, для произвольного
$\ep>0$ существует $N\in\N, \; N=N(\ep)$, что $d_m(x_p,x_q)<\ep$ для
всех $p,q\in\N, \; p,q\ge N$.

Так как $d_m^1\le d_m$, то $d_m^1(x_p,x_q)<\ep$ для всех $p,q\in\N,
\; p,q\ge N$ и, следовательно, $\hm{x_p^i-x_q^i} \le d_m^1(x_p,x_q)
< \ep$ для всех $p,q\in\N, \; p,q\ge N$; то есть, каждая $(x_n^i),
\; i=\ol{1,m}$ --- фундаментальная.

\textbf{Достаточность}. Пусть каждая $(x_n^i), \; i=\ol{1,m}$ ---
фундаментальная. Рассмотрим произвольное $\ep>0$. Тогда существует
$N_i\in\N, \; N_i=N_i(\ep), \; 1\le i \le m$, что $\hm{x_p^i-x_q^i}
\le \frac{\ep}{\sqrt{m}}$ для всех $p,q\in\N, \; p,q\ge N_i$.

Обозначим $N=\max(N_1,\ldots,N_m)$. Тогда неравенство
$\hm{x_p^i-x_q^i} < \frac{\ep}{\sqrt{m}}$ справедливо для всех
$p,q\in\N, \; p,q\ge N$, и всех $i, \; 1\le i \le m$, или
$d_m^1(x_p,x_q) < \frac{\ep}{\sqrt{m}}$ для всех $p,q\in\N, \;
p,q\ge N$. Поэтому $d_m(x_p,x_q) \le \sqrt{m} d_m^1 (x_p,x_q) < \ep$
для всех $p,q\in\N, \; p,q\ge N$, то есть $(x_n)$ ---
фундаментальная в $\R^m$.
\end{proof}

\begin{thn}{3 (критерий сходящейся последовательности в $\R^m$)}
Последовательность $(x_n)$ точек $x_n\in\R^m, \; n\in\N$ сходится
тогда и только тогда, когда $(x_n)$ --- фундаментальная (или
последовательность Коши).
\end{thn}

\begin{proof}
Прямое следствие теорем 1, 2 и критерия Коши сходимости числовой
последовательности.
\end{proof}

\subsubsection{ Полные метрические пространства}
\begin{dfn}{2}
Метрическое пространство называется \emd{полным}, если в
нём сходится каждая последовательность Коши.
\end{dfn}

\begin{exx}
Пространства $\R^m,\; m\ge1$ --- полные.
\end{exx}

\begin{dfn}{3}
Последовательность $(x_n)$ точек $x_n, \; n\in\N$ из метрического
пространства $(X;d)$ называют \emd{фундаментальной} (или
последовательностью Коши), если для любого числа $\ep>0$ существует
$N\in\N, \; N=N(\ep)$, что $d(x_p,x_q)<\ep$ для всех $p,q\in\N, \;
p,q\ge N$.
\end{dfn}

\begin{stm*}
Всякая сходящаяся последовательность в метрическом пространстве ---
фундаментальная.
\end{stm*}

\begin{proof}
Пусть $(X;d)$ --- метрическое пространство и последовательность
$(x_n), \; x_n\in X, \; n\in\N$ --- фундаментальная, имеет
$\liml{n\ra\bes} x_n=a, \; a\in X$. Рассмотрим произвольное $\ep>0$.
Согласно определению, существует $N\in\N, \; N=N(\ep)$, что
$d(x_n,a)<\frac{\ep}2$ для всех $n\ge N$, тогда $d(x_p,x_q) \le
d(x_p,a)+d(x_q,a) < \frac{\ep}2 + \frac{\ep}2 = \ep$ для всех
$p,q\in\N, \; p,q\ge N$.
\end{proof}

\subsubsection{ Предел отображений метрических пространств}
Рассмотрим произвольные метрические пространства $(X;d)$ и
$(Y,\rho)$ и произвольную базу $\B$ в $X$.

\begin{dfn}{4}
Элемент $b\in Y$ есть \emd{предел отображения} $f$ из $X$
в $Y$ по базе $\B$, если для произвольного $\ep>0$ существует
элемент $B_{\ep}$ базы $\B$, что $\rho(f(x),b)<\ep$ справедливо для
всех точек $x\in D_f\cap B_{\ep}$.
\end{dfn}

Напомним, что в $(X;d)$ система $\Uc(a)$ окрестностей точки $a\in X$
образует базу.

\begin{dfn}{5}
Отображение из $\R^m, \; m\ge1$ в $\R^1$ называют \emd{функцией}
(от) $m$ действительных переменных.
\end{dfn}

Если $D_f \subset \R^m$ --- область определения функции $f$, то для
любой точки $x=(x^1,\ldots,x^m)\in D_f \subset \R^m$ число
$f(x)=f(x^1,\ldots,x^m)$ называют значением функции $f$ в точке $x$.

Если $m=2$, обозначают $f(x,y)$; если $m=3$, обозначают $f(x,y,z)$.

Отображение $f$ из $\R^m, \; m\ge 1$ в $\R^1$.

$f(x)=f(x^1,\ldots,x^m), \; D_f\subset\R^m$.

$R_f$ --- числовое множество.

\begin{df}
Число $l=\liml{\B} f(x)$, где $\B$ --- база в $\R^m$
\par\noindent тогда и только тогда, когда для любого $\ep>0$ существует $B_{\ep}\in\B$, что
$\hm{f(x)-l}<\ep$ для всех $x\in B_{\ep}$. \par\noindent тогда и
только тогда, когда для любой окрестности $\Vc$ числа $l$ в
$\R^1=\R$ существует $B\in\B$, что $f(B\cap D_f)\subset\Vc$.
\end{df}

\textbf{Примеры баз}

1. Система $\Uc(a),\; a=(a^1,\ldots,a^m)\in\R^m$ --- база в $\R^m$;

2. Если $a$ --- точка прикосновения для множества $E\subset\R^m$ и
$a$ --- не изолированная точка множества $E$, то есть для любой
$\os{\circ}{\Uc}(a,r)\cap E \ne \es, \; r>0, \; \hc{ E \cap \os{\circ}{\Uc} (a;r) \; | \; r>0 }$ --- база в $\R^m$, обозначаемая
$E\owns x\ra a$.

3. Если существует $\os{\circ}{\Uc}(a;r_0), \;r_0>0$, что
$\os{\circ}{\Uc}(a;r_0) \subset E$, то есть $\os{\circ}{\Uc}(a;r_0)
\cap E = \os{\circ}{\Uc} (a;r_0)$, то база $E\owns x\ra a$
обозначается $\boxed{x\ra a} = \hc{ \os{\circ}{\Uc} (a;\de); \;
\de>0}$.

4. $\hc{G(a;r); \; r>0}$ --- база в $\R^m$, обозначается
$\boxed{x\ra+\infty},\;m\ge2$. %тут точно G(a;r)?

$l=\liml{E\owns x\ra a} f(x)$ тогда и только тогда, когда для
любого $\ep>0$ существует $\de>0$ такое, что для всех $x\in E = D_f$
и $0<d_m(x,a)<\de$ справедливо $\hm{f(x)-l}<\ep$ (или
$d_1(f(x),l)<\ep$).

\begin{thn}{(локальные свойства функций, имеющих предел по базе)}
Если функции $f(x),\;g(x),\; x=(x^1,\ldots,x^m)\in E = D_f=D_g$
имеют пределы $\liml{E\owns x\ra a} f(x)=l_1, \;
\liml{E\owns x\ra a} g(x)=l_2$, то

\noindent 1) существует $\liml{E\owns x\ra a} (\la_1 f(x) +
\la_2 g(x)) = \la_1 l_1 + \la_2 l_2$ для любых $\la_1,\la_2\in\R$;

\noindent 2) существует $\liml{E\owns x\ra a} f(x)g(x)=l_1
l_2$;

\noindent 3) если $l_2\ne0$, то существует $\liml{E\owns x\ra
a} \frac{f(x)}{g(x)} = \frac{l_1}{l_2}$;

\noindent 4) функции $f(x)$ и $g(x)$ ограничены в некоторой
$\Uc(a;r_0)\; r_0>0$, то есть $\hm{f(x)} \le M, \; \hm{g(x)} \le M$
с некоторым $M>0$ для всех $x\in E=D_f=D_g$ и $x\in\Uc(a;r_0)$;

\noindent 5) если $f(x)\le g(x)$ для всех $x\in E=D_f=D_g$ и $x\in
\os{\circ}{\Uc}(a;r_0)$ для некоторого $r_0>0$, то
$l_1=\liml{E\owns x\ra a} f(x) \le \liml{E\owns x\ra
a} g(x) = l_2$;

\noindent 6) если $l_2\ne0$, то существует
$\os{\circ}{\Uc}(a;\de_0),\; \de_0>0$, что $g(x)\ne0, \; x\in
E\cap \os{\circ}{\Uc}(a,\de_0)$ и $\sgn g(x) = \sgn l_2$ для всех
$x\in E\cap \os{\circ}{\Uc}(a;\de_0),\; E=D_g$;

\noindent 7) если $f(x)\le h(x) \le g(x)$ для всех $x\in
\os{\circ}{\Uc}(a;\de_0)$ для некоторого $\de_0>0$ и $x\in
E=D_f=D_g=D_h$ и $l_1=l_2$, то существует $\liml{E\owns x\ra a}
h(x)=l$ и $l=l_1=l_2$.
\end{thn}

\subsubsection{ Непрерывность функции нескольких переменных}
\begin{dfn}{1}
Функция $f(x)=f(x^1,\ldots,x^m)$ называется
\emd{непрерывной} в $x_0=(x_0^1,\ldots,x_0^m)$, если

\noindent 1) $x_0\in D_f\subset \R^m$;

\noindent 2) для произвольного $\ep>0$ существует такое $\de>0$, что
$\hm{f(x)-f(x_0)}<\ep$ для всех $x\in D_f$ и $d_m(x,x_0)<\de$.
\end{dfn}

\begin{stm*}
Если $x_0\in D_f, \; f(x)=f(x^1,\ldots,x^m), \;
x_0=(x_0^1,\ldots,x_0^m)$ и $x_0$ --- не изолированная точка
множества $D_f=E$, то функция $f(x)$ непрерывна в $x_0 \;
\Lra \; f(x_0)= \liml{E\owns x\ra x_0} f(x) \;
\Lra \;$существует $\liml{E\owns x\ra x_0} f(x)=l$
и $l=f(x_0)$.
\end{stm*}

Если $D_f=G$ --- открытое множество в $\R^m$, то любая точка $x_0\in
G$ вместе с некоторой своей окрестностью $\Uc(x_0;r_0), \; r_0>0$, и
тогда вместо $\liml{E\owns x\ra x_0} f(x)$ пишут $\liml{x\ra x_0}
f(x)$.

Итак, если $D_f=G$ --- открытое множество в $\R^m,\; x_0\in G$, то
$f(x)$ непрерывна в $x_0 \; \Lra \;$для любого $\ep>0$ существует
$\de>0$ такое, что $\hm{f(x)-f(x_0)}<\ep$ справедливо для всех $x,\;
d_m(x,x_0)<\de$ $\hr{0\le\de\le r_0, \; \Uc(x_0;r_0)\subset D_f} \;
\Lra \;$для любой окрестности $\Vc$ числа $f(x_0)$ в $\R^1$
существует такая окрестность $\Uc$ точки $x_0$ в $\R^m, \;
x_0\in\Uc\subset G$, что образ $f(\Uc)\subset\Vc$.

\begin{thn}{(критерий непрерывности функции в точке)}
Функция $f(x)$ непрерывна в $x_0\in E=D_f\; \Lra \;$для
любой последовательности $(x_n)$, $x_n=(x_n^1,\ldots,x_n^m)\in E, \;
n\in\N, \; \liml{n\ra\bes} x_n=x_0$, числовая
последовательность $(f(x_n))$ имеет $\liml{n\ra\bes}
f(x_n)=f(x_0)$.
\end{thn}

\begin{thn}{(локальные свойства непрерывных функций)}
Если функции $f(x),\; g(x)$ непрерывны в
$x_0=(x_0^1,\ldots,x_0^m)\in E=D_f=D_g\subset\R^m$, то:

\noindent 1) $\la_1 f(x) + \la_2 g(x)$ непрерывна в $x_0$ для всех
$\la_1,\la_2\in\R$;

\noindent 2) $f(x)g(x)$ непрерывна в $x_0$;

\noindent 3) функции $f(x)$ и $g(x)$ ограничены в некоторой
окрестности $\Uc(x_0)$;

\noindent 4) если $g(x_0)\ne0$, то $\frac{f(x)}{g(x)}$ непрерывна в
$x_0$;

\noindent 5) если $g(x_0)\ne0$, то существует $\Uc(x_0,\de_0), \;
\de_0>0$, что $\sgn g(x)=\sgn g(x_0)$ для всех $x\in E=D_g\cap
\Uc(x_0,\de_0)$.
\end{thn}

\subsubsection{ Предел отображения из $\R^m$ в $\R^n$}
Опишем структуру отображения $f$ из $\R^m$ в $\R^n$. Область его
определения $D_f\subset\R^m$, то есть аргумент
$x=(x^1,\ldots,x^m)\in D_f\subset\R^m$.

Значение $f(x)=(f^1(x),\ldots,f^n(x))\in\R^n$, где каждая функция
$f^k(x)=f^k(x^1,\ldots,x^m),\; D_f=D_{f^k}, \; D_{f^k}\subset\R^m,
\; k=\ol{1,n}$. В частности, если $m=1$, имеем отображение из
$\R^1=\R$ в $\R^n$, у которого $D_f$ --- числовое множество
($D_f\subset\R)$, аргумент $x\in\R$ --- число, а значение
$f(x)=(f^1(x),\ldots,f^n(x))$, где $f^k(x), \; k=\ol{1,n}$ ---
числовые функции и $D_{f^k}=D_f$ --- числовое множество.

Отображение $f(x)=(f^1(x), \ldots, f^n(x)), \; x\in E = D_f =
D_{f^k}, \; k=\ol{1,n}$ --- числовое множество, называется
\emd{векторной функцией}.

\begin{dfn}{2}
Точка $A=(A^1,\ldots,A^n)\in\R^n$ называется \emd{пределом отображения} $f$ из $\R^m$ в $\R^n\subset D_f=D_{f^k} = E\subset
\R^m, \; k=\ol{1,n}$ по базе  $E\owns x\ra a, \;
a=(a^1,\ldots,a^m)\in\R^m$, если для любого $\ep>0$ существует
$\de>0$, что $d_n(f(x),A)<\ep$ для всех $x\in E=D_f$ и
$0<d_m(x,a)<\de$.
\end{dfn}

Обозначается $A=\liml{E\owns x\ra a} f(x)$.

\begin{thh}
$A=\liml{E\owns x\ra a}f(x)$ тогда и только тогда, когда для каждого
$k,\; k=\ol{1,n}$ справедливо равенство
$$A^k=\liml{E\owns x\ra a} f^k(x) = \liml{E\owns x\ra a} f^k (x^1,\ldots,x^m).$$
\end{thh}

\begin{proof}
\textbf{Необходимость}. Дано, что $A=\liml{E\owns x\ra a} f(x)$.
Согласно определению 2, для любого $\ep>0$ существует $\de>0$, что
$d_n(f(x),A)<\ep$ для всех $x\in D_f=E$ и $0<d_m(x,a)<\de$. Так как
$f(x)=(f^1(x),\ldots,f^n(x)), \; f^k(x) = f^k(x^1,\ldots, x^m), \;
k=\ol{1,n}$ и $A=(A^1,\ldots,A^n)$ и метрика $d_n^1 \le d_n$, то
$d_n^1(f(x),A) \le d_n(f(x),A)<\ep$ для всех $x\in D_f=D_{f^k}, \;
k=\ol{1,n}$, и $0<d_m(x,a)<\de$, так что $\hm{f^k(x)-A^k} \le d_n^1
\hm{f(x);A} < \ep$ для всех $x\in D_{f^k}, \; k=\ol{1,n}$, и
$0<d_m(x,a)<\de$.

Иными словами, $A^k=\liml{E\owns x\ra a} f^k(x), \;
k=\ol{1,n}$.

\textbf{Достаточность}. Дано, что существует $\liml{E\owns x\ra a}
f^k(x)=A^k, \; k=\ol{1,n}$. Рассмотрим произвольное $\ep>0$.
Согласно определению предела функции нескольких переменных по базе
$E\owns x\ra a$, существует $\de_k>0, \; k=\ol{1,n}$, что
\eqa{1}{\hm{f^k(x)-A^k}<\ep} для всех $x\in D_{f^k}=D_f$ и
$0<d_m(x,a)<\de_k, \; k=\ol{1,n}$. Положим
$\de=\min(\de_1,\ldots,\de_n), \; \de>0$. Для всех $x\in D_f =
D_{f^k}, \; k=\ol{1,n}$, и $0<d_m(x,a)<\de$ неравенство~(1)
справедливо для любого $k, \; k=\ol{1,n}$. Поэтому, $d_n^1(f(x),A) <
\frac{\ep}{\sqrt{n}}$ для всех $x\in D_f$ и $0<d_m(x,a)<\de$.

Так как $d_n\le \sqrt{n} d_n^1$, то $d_n(f(x),A) \le \sqrt{n} d_n^1
(f(x),A) < \sqrt{n} \cdot \frac{\ep}{\sqrt{n}} = \ep$ для всех $x\in
D_f$ и $0<d_m(x,a)<\de$, то есть $A=\liml{E\owns x\ra a}
f(x)$.
\end{proof}

\subsubsection{ Непрерывные отображения из $\R^m$ в $\R^n$}

\begin{dfn}{3}
Отображение $f\colon E\ra\R^n, \; E\subset\R^m, \; E=D_f$,
называется \emd{непрерывным} в неизолированной точке
$x_0\in E$, если существует $\liml{E\owns x\ra x_0} f(x)=A$ и
$A=f(x_0)$.
\end{dfn}

Если множество $E=D_f$ --- открытое в $\R^m, \; D_f=G$, и $x_0\in
G$, то отображение $f\colon G\ra\R^n$ непрерывно в $x_0$, если для
произвольного $\ep>0$ существует такое $\de>0$, что
$d_n(f(x),f(x_0))<\ep$ для всех $x, \; d_m(x,x_0)<\de(x\in G)$.

\begin{thn}{(критерий непрерывного отображения из $\R^m$ в $\R^n$)}
Отображение $f$ из $\R^m$ в $\R^n$ непрерывно в точке
$x_0=(x_0^1,\ldots,x_0^m) \; \Lra \; $1) $x_0\in D_f
\subset \R^m$ и 2) каждая функция $f^k(x)=f^k(x^1,\ldots,x^m), \;
k=\ol{1,n}$, где $f(x)=(f^1(x),\ldots,f^n(x))$, непрерывна в $x_0$.
\end{thn}

\begin{proof}
Прямое следствие теоремы пункта 3.6 и определения непрерывной
функции, $f^k(x_0)=\liml{E\owns x\ra x_0} f^k(x), \;
k=\ol{1,n}$.
\end{proof}

Рассмотрим функцию $f(x)=f(x^1,\ldots,x^m)$, определённую на
открытом множестве $G\subset\R^m$ и некоторую точку
$x_0=(x_0^1,\ldots,x_0^m) \in G$. Функция $f(x)$ называется \emd{непрерывной} в $x_0 \;
(f\in\Cc(x_0)) \; \Lra \;$для произвольного $\ep>0$
существует $\de>0$, что $\hm{f(x)-f(x_0)}<\ep$ для всех $x,\;
d_m(x,x_0) < \de \; (x=(x^1,\ldots,x^m)\in G)$.

Фиксируем $x^k=x_0^k, \; k=\ol{1,m}, \; k\ne i, \; i=\ol{1,m}$.
Тогда $f(x_0^1,\ldots,x_0^{i-1},x^i, x_0^{i+1}, \ldots, x_0^m) =
\ph_i(x^i), \; i=\ol{1,m}$.

\begin{stm*}
Если $f\in\Cc(x_0)$, то каждая $\ph_i(x^i), \; i=\ol{1,m}$,
непрерывна в $x_0^i$.
\end{stm*}

\begin{proof}
Рассмотрим произвольное $\ep>0$, находим $\de>0$ такое, что
$\hm{f(x)-f(x_0)}<\ep$ для всех $x,\; d_m(x,x_0) < \de$. В
частности, для
$x=(x_0^1,\ldots,x_0^{i-1},x^i,x_0^{i+1},\ldots,x_0^m)$, имеем
$$d_m(x,x_0) = \sqrt{\sum\limits_{k=1}^m (x_0^k-x_0^k)^2 + (x^i-x_0^i)^2} = \hm{x^i-x_0^i} < \de,$$
а $\hm{f(x)-f(x_0)} = \hm{\ph_i(x^i) - \ph_i(x_0^i)} < \ep$, то есть $\ph_i(x^i)$ непрерывна в $x_0^i, \;
i=\ol{1,m}$.
\end{proof}

\begin{exx}
$$f(x,y)=\bcase{
&\frac{xy}{x^2+y^2}, \mbox{ если } (x,y)\ne(0,0); \\
&0, \mbox{ если } (x,y)=(0,0).}$$
\end{exx}

$f(x,0)=0, \,x\in\R; \; f(0,y)=0, \,y\in\R$.

$f(0,0)=0$. Функции $f(x,0)$ и $f(0,y)$ непрерывны соответственно в
$x=0$ и $y=0$. С другой стороны, предел $\liml{(x,y)\ra(0,0)} f(x,y)$ не существует, так как
$$\liml{(x,0)\ra(0,0)} f(x,y) = \liml{x\ra0} f(x,0) = 0 = f(0,0), \quad \liml{(0,y)\ra(0,0)} f(x,y) = \liml{y\ra0} f(0,y) =0 = f(0;0),$$
однако
$$\liml{(x,y)\ra(0,0)} f(x,y)=\liml{\substack{x\ra0\\x\ne0}} f(x,x)=\liml{x\ra0}\frac12 = \frac12 \ne f(0,0).$$

По определению, $f\in \Cc(x_0), \; x_0\in G\subset \R^m$ ---
открытое множество $\Lra$ для любого $\ep>0$ найдётся $\de>0$ такое,
что $\hm{f(x)-f(x_0)}<\ep$ (или $d_1(f(x),f(x_0))<\ep$) для всех
$x\in G, \; d_m(x,x_0)<\de$ $\Lra$ для любой окрестности $\Vc$ точки
$f(x_0)$ в $\R^1$ существует окрестность $\Uc$ точки $x_0$ в $\R^m,
\; \Uc\subset G$, что образ $f(\Uc)\subset\Vc$.

\subsubsection{ Непрерывные отображения открытых множеств
метрических пространств}

Рассмотрим произвольные метрические пространства $(X;d)$ и
$(Y;\rho)$ и множество $G\subset X$ --- открытое в $(X;d), \; x_0\in
G$ --- произвольное.

\begin{dfn}{1}
Отображение $f\colon G\ra Y$ называют \emd{непрерывным} в
$x_0\in G$, если для произвольного $\ep>0$ существует $\de>0$ такое,
что $\rho(f(x),f(x_0))<\ep$ для всех $x\in G, \; d(x,x_0)<\de$.
\end{dfn}

\begin{dfn}{1'}
Отображение $f\colon G\ra Y$ называют \emd{непрерывным} в
$x_0\in G$, если для любой окрестности $\Vc$ точки $f(x_0)$ в
$(Y,\rho)$ существует окрестность $\Uc$ точки $x_0$ в $(X;d), \;
\Uc\subset G$, что образ $f(\Uc)\subset\Vc$.
\end{dfn}

\begin{thn}{(критерий непрерывности отображения открытого
множества)} Отображение $f\colon G\ra Y$ открытого множества $G$ в
$(X;d)$ непрерывно в каждой точке множества $G$ (то есть непрерывно
на $G$) тогда и только тогда, когда прообраз $f^{-1}(\Vc)$ любого
открытого множества $\Vc$ в $(Y,\rho)$ есть открытое множество в
$(X;d)$.
\end{thn}

\begin{proof}
\textbf{Необходимость}. Дано, что $f\colon G\ra Y$ непрерывно в
каждой точке $x_0\in G$ --- открытого множества. Рассмотрим
произвольное открытое $\Vc$ в $(Y,\rho)$. Если $f^{-1}(\Vc)=\es$, то
$f^{-1}(\Vc)$ --- отображение в $(X;d)$. Пусть $f^{-1}(\Vc)\ne\es$ и
$x_0\in f^{-1}(\Vc)$. Тогда $x_0\in G$, и так как $f(x_0)\in\Vc$ и
$\Vc$ --- открытое множество, то согласно определению $1'$,
существует шар $\Uc(x_0,r)$, что $f(\Uc(x_0;r))\subset\Vc$. Тогда
$\Uc(x_0,r)=f^{-1} (f(\Uc(x_0;r))\subset f^{-1}(\Vc)$. Таким
образом, $x_0$ входит в $f^{-1}(\Vc)$ вместе с некоторым своим
открытым шаром, так что $f^{-1}(\Vc)$ --- окрестность (каждой) своей
точки $x_0$, то есть $f^{-1}(\Vc)$ --- открытое.

\textbf{Достаточность}. Дано, что $f^{-1}(\Vc)$ каждого открытого
множества $\Vc$ в $(Y,\rho)$ есть открытое множество в $(X;d)$.
Рассмотрим произвольное $x_0\in G$ и $y_0=f(x_0)\in Y$. Рассмотрим
произвольную окрестность $\Vc$ точки $f(x_0)$ в $(Y,\rho)$. Тогда
$f^{-1}(\Vc)$ --- открытое, $x_0\in f^{-1}(\Vc)$ и $x_0\in
f^{-1}(\Vc)\cap G$. Множество $\hr{f^{-1}(\Vc)\bigcap G}$ ---
открытое (как пересечение открытых множеств). Следовательно,
$\Uc=f^{-1}(\Vc)\bigcap G$ --- некоторая окрестность точки $x_0$ и
$f(f^{-1}(\Vc)\bigcap G) \subset f(f^{-1}(\Vc))=\Vc$. Согласно
определению 1$'$, отображение $f$ непрерывно в точке $x_0$.
\end{proof}

\subsubsection{ Непрерывность композиции}

Рассмотрим функции $x^1=\ph^1(t)=\ph^1(t^1,t^2,\ldots,t^k), \ldots,
x^m=\ph^m(t) = \ph^m(t^1,\ldots,t^k)$, где $t=(t^1,\ldots,t^k) \in
E^* \subset \R^k, \; E^*$ --- множество в $\R^k$. Эти функции задают
отображение $x=\ph(t),\;\ph(t) = (\ph^1(t),\ldots,\ph^m(t))$
множества $E^*$ в $\R^m$. Обозначим образ $\ph(E^*)=E$ --- множество
в $\R^m$.

Рассмотрим на $E\subset\R^m$ функцию $f(x)=f(x^1,\ldots,x^m)$ и её
композицию $(f\circ\ph)(t)=F(t)=F(t^1,\ldots,t^k)$.

\begin{thn}{(о непрерывности сложной функции)}
Если отображение $x=\ph(t)$ непрерывно на множестве $E^*$, а функция
$f(x)$ непрерывна на $E=\ph(E^*)$, то сложная функция
$F(t)=(f\circ\ph)(t)$ непрерывна на $E^*$.
\end{thn}

\begin{proof}
Рассмотрим произвольную точку $t_0\in E^*$ и произвольную
последовательность $(t_n)$ точек $t_n\in E^*$ такую, что
$\liml{n\ra\bes} t_n=t_0 \; (t_0=(t_0^1,\ldots,t_0^k), \;
t_n=(t_n^1,\ldots,t_n^k), \; n\in\N)$. Так как отображение
$x=\ph(t)$ непрерывно в $t_0$, то каждая функция $\ph^j(t)=\ph^j
(t^1,\ldots,t^k), \; j=\ol{1,m}$ непрерывна в $t_0$, следовательно,
$\liml{n\ra\bes} \ph^j(t_n) = \ph^j (t_0), \; j=\ol{1,m}$.

Таким образом, если $x_n = \ph(t_n) = (x_n^1,\ldots,x_n^m) =
(\ph^1(t_n), \ldots, \ph^m(t_n))$, то $\liml{n\ra\bes} x_n =
\liml{n\ra\bes} \ph^j(t_n) = \ph^j(t_0) = x_0^j, \;
j=\ol{1,m}$, и $x_0=(x_0^1,\ldots,x_0^n)\in E$. Так как функция
$f(x)$ непрерывна в $x_0\in E$, то $f(x_0) = \liml{n\ra\bes}
f(x_n)$, где $x_n=(x_n^1,\ldots,x_n^m)$. Поэтому
$\liml{n\ra\bes} F(t_n) = \liml{n\ra\bes}
(f\circ\ph)(t_n) = \liml{n\ra\bes} f(\ph(t_n)) =
\liml{n\ra\bes} f(x_n) = f(x_0) = (f\circ\ph)(t_0) = F(t_0)$
для любой последовательности $(t_n)$ точек $t_n\in E^*, \;
\liml{n\ra\bes} t_n = t_0$. Согласно критерию, $F$
непрерывна в $t_0\in E^*$.
\end{proof}

Рассмотрим отображение $f$ множества $E=\ph(E^*)\subset \R^m$ в
$\R^n$, то есть $f(x)=(f^1(x),\ldots,f^n(x))$, где
$$f^j(x)=f^j(x^1,\ldots,x^m), \; j=\ol{1,n}, \; x=(x^1,\ldots,x^m)\in E \subset \R^m.$$

\begin{thh}
Если отображение $x=\ph(t)$ непрерывно на множестве $E^*$, и
отображение $f$ множества $E=\ph(E^*)\subset\R^m$ в пространство
$\R^n$ непрерывно на множестве $E$, то их композиция $f$ ограничена, % ?
являющаяся отображением множества $E^* \subset \R^k$ в пространство
$\R^n$, непрерывна на множестве $E^*\subset\R^k$.
\end{thh}

\begin{proof}
Отображение $(f\circ\ph)(t), \; t\in E^*$, имеет компоненты
$(f\circ\ph)(t) = (F^1(t),\ldots,%какая тут степень?
F^k(t))$, где
$$F^j(t) = F^j(\ph^1(t),\ldots,\ph^k(t)),
%какая тут степень?
\; j=\ol{1,n}, \; t=(t^1,\ldots,t^k).$$
Согласно предыдущей теореме, каждая функция $F^j(t), \; j=\ol{1,n}$ непрерывна на $E^*$. Таким
образом, непрерывно и отображение $(f\circ\ph)(t)$ на $E^*$.
\end{proof}

\subsubsection{ Равномерно непрерывные отображения из $\R^m$ в
$\R^n$}
\begin{df}
Отображение $f(x)$ из $\R^m$ в $\R^n$ называется
\emd{равномерно непрерывным} на множестве $E\subset \R^m$,
если $E\subset D_f$ и для любого $\ep>0$ существует $\de>0$ такое,
что $d_n(f(x'),f(x''))<\ep$ для всех $x',x''\in E$, для которых $d_m(x',x'')<\de$.
\end{df}

\begin{thh}
Всякое равномерно непрерывное отображение множества непрерывно в
каждой точке множества.
\end{thh}

\begin{proof}
Фиксируем произвольное $x_0\in E$. Согласно определению, для любого
$\ep>0$ существует $\de>0$, что $d_n(f(x),f(x_0))<\ep$ для всех
$x\in E, \; d_n(f(x),f(x_0))<\ep$ для всех $x\in E, \; d_m(x,x_0) <
\de$, то есть $f$ непрерывна в $x_0\in E$.
\end{proof}

Аффинное отображение $A(x)=y$ пространства $\R^m$ в $\R^n, \; m>1,
\; n\ge1$ задаётся матрицей $A=\hr{a_i^j}, \; i=\ol{1,m}, \;
j=\ol{1,n}$ и вектором $b=(b^1,\ldots,b^n)\in\R^n$, так что для
любого $x=(x^1,\ldots,x^m)\in\R^m$ его образ $y=A(x), \;
y=(y^1,\ldots,y^n)\in \R^n$ и $y^j=\sumium a_i^j x^i + b^j, \;
j=\ol{1,n}$.

В матричной форме

$$\rbmat{y^1 \\ \vdots \\ y^n} =
\rbmat{a_1^1 & a_2^1 & \hdots & a_m^1\\
\vdots & \vdots & \ddots & \vdots
\\ a_1^n & a_2^n & \hdots & a_m^n}
\cdot\rbmat{x^1 \\ \vdots \\ x^m} + \rbmat{b^1 \\ \vdots \\ b^n}$$

Таким образом, $y=A(x)=L(x)+b$, где $L(x)$ --- линейное отображение
из $\R^m$ в $\R^n$, задаваемое матрицей $A$.

\begin{stm*}
Любое аффинное отображение $A(x)$ равномерно непрерывно на $\R^m$.
\end{stm*}

\begin{proof}
Рассмотрим произвольные точки $x_1=(x_1^1,x_1^2,\ldots,x_1^m), \;
x_2=(x_2^1,\ldots,x_2^m)$ и $y_1=A(x_1), \; y_2=A(x_2)$. Тогда
$y_2-y_1 = L(x_2) + b - L(x_1) - b = L(x_2) - L(x_1) = L(x_2-x_1)$.
Если $y_1=(y_1^1,\ldots,y_1^n), \; y_2=(y_2^1,\ldots, y_2^n)$, то
$y_2^j-y_1^j = \sumium a_i^j (x_2^i-x_1^j), \; j=\ol{1,m}$.
Обозначим $\al=\max\limits_{i,j} \hm{a_i^j}>0$.

Если $d_m(x_1,x_2)<r$, то $\hm{x_2^i-x_1^i}\le d_m(x_1,x_2)<r, \;
i=\ol{1,m}$, и $\hm{y_2^j-y_1^j}\le m\cdot d \cdot r, \;
j=\ol{1,n}$, так что $d_n(y_1,y_2) \le \sqrt{n} \cdot \max\limits_j
\hm{y_2^j-y_1^j} < m\cdot \sqrt{n} \cdot d \cdot r$. Рассмотрим
произвольное $\ep>0$ и положим $\de=\frac{\ep}{m\sqrt{n}\cdot d}>0,
\; \de=\de(\ep)$. Тогда, если $d_m(x_1,x_2)<\de$, то $d_n(y_1,y_2)<
m \sqrt{n} \al \cdot \de = \ep$, то есть $y=A(x)$ равномерно
непрерывно в $\R^m$.
\end{proof}

В частности, любое линейное отображение $L(x)$ пространства $\R^m$ в
$\R^n$ равномерно непрерывно на $\R^m$.

При $n=1$ заключаем, что любая линейная функция $L(x)=a_1x^1+\ldots
+ a_m x^m$ равномерно непрерывна на $\R^m$.

\subsection{ Глобальные свойства непрерывных отображений}
\subsubsection{ Линейно связные множества в $\R^m$}

\begin{dfn}{1}
\emd{Непрерывной кривой} $\Ga$ в $\R^m$ называют всякую
непрерывную функцию (вектор--функцию) $x=\ph(t)$, областью
определения $D_{\ph}$ которой служит некоторый отрезок
$[\al,\be]\subset\R$. То есть \equ{\Ga =
\hc{(\ph^1(t),\ldots,\ph^m(t)) \; | \; \ph^i(t)\in\Cc[\al,\be], \;
[\al,\be]\subset\R, \; i=\ol{1,m}}.}
\end{dfn}

Точки с координатами $(\ph^1(\al), \ldots,\ph^m(\al))$ и $(\ph^1(\be), \ldots, \ph^m(\be)) \in \R^m$ называются концевыми
точками кривой~$\Ga$.

Прямолинейный отрезок $[a,b]$ с концевыми точками
$a=(a^1,\ldots,a^m), \; b=(b^1,\ldots,b^m)\in\R^m$, задаваемый
вектор--функцией $x=x(t)=(x^1(t),\ldots,x^m(t))$ и $x^i(t) = tb^i +
(1-t)a^i, \; t\in[0,1], \; i=\ol{1,m}$. \\ $x(0)=a, \; x(1)=b$, и
все $x^i(t)m, \; i=\ol{1,m}$ непрерывны на $[0,1]$.

\begin{dfn}{2}
Множество $E\subset\R^m$ называют \emd{линейно связным},
если для любых его точек $x_1,x_2\in E$ можно указать непрерывную
кривую $\Ga$, лежащую в $E$ и соединяющую $x_1$ и $x_2$.
\end{dfn}

\begin{dfn}{3}
Всякое открытое и линейно связное множество в $\R^m$ называется
\emd{областью} в $\R^m$.
\end{dfn}

\begin{dfn}{4}
Множество $E\subset\R^m$ называется \emd{выпуклым}, если
вместе с любыми своими точками $x_1,x_2\in E$ оно содержит
прямолинейный отрезок $[x_1,x_2]$.
\end{dfn}

\begin{stm*}
Любой открытый шар $\Uc(a;r), \; a=(a^1,\ldots,a^m)\in\R^m, \; r>0$,
есть выпуклая область в $\R^m$.
\end{stm*}

\begin{proof}
Шар $\,\Uc(a;r)$ --- открытое множество в $\R^m$. Рассмотрим
произвольные точки $x_1=(x_1^1,\ldots,x_1^m)$  и $x_2=(x_2^1,\ldots,x_2^m)$, лежащие в $\Uc(a;r)$,
то есть $d_m(x_k,a)<r, \;k=1,2$.

Прямолинейный отрезок $[x_1,x_2]$ задаётся вектор\д функцией
$$x(t)=\br{x^1(t),\ldots,x^m(t)}, \quad  x^i(t) = tx_2^i + (1-t)x_1^i, \quad t\in[0,1], \quad  i=\ol{1,m}.$$

С учётом $d_m(x_k,a)<r, \;k=1,2$, и неравенства Минковского, имеем
оценку (в этой выкладке все суммы по индексу $i$ от $1$ до $m$):
\equ{\begin{aligned}
d_m(x(t),a)&=\sqrt{\sum (x^i(t)-a^i)^2} =
\sqrt{\sum \hs{t x_2^i + (1-t) x_1^i - ta^i - (1-t)a^i}^2} =\\
&=\sqrt{\sum\hs{t(x_2^i-a^i) + (1-t)(x_1^i-a^i)}^2} \le \\
&\le  \sqrt{\sum t^2(x_2^i - a^i)^2} + \sqrt{\sum (1-t)^2 (x_1^i-a^i)^2} =
t\cdot d_m(x_2,a) + (1-t)\,d_m(x_1,a) < t\,r+(1-t)\,r=r.\end{aligned}}

Итак, отрезок $[x_1,x_2]\subset \Uc(a,r)$, то есть $\Uc(a,r)$ ---
\emd{выпуклая область} в $\R^m$.
\end{proof}

\subsubsection{ Непрерывный образ линейно связного множества в
$\R^m$}

\begin{thh}
Если $E$ --- линейно связное множество в $\R^m$, то для любого
непрерывного по $E$ отображения $f$ в $\R^m$ множество $f(E)$ ---
связное в $\R^n$.
\end{thh}

\begin{proof}
Рассмотрим произвольные $y_1,y_2\in f(E)\subset \R^n$ и выберем
\textbf{некоторые} $x_1,x_2\in E$ такие, что $f(x_1)=y_1, \,
f(x_2)=y_2$.

Так как $E$ --- линейно связное, то существует непрерывная кривая
$\Ga$, соединяющая $x_1$ и $x_2$ и лежащая в $E$; то есть,
существует непрерывная вектор--функция $x=\ph(t) =
(\ph^1(t),\ldots,\ph^m(t)), \; t\in[\al,\be] \subset \R$, что
$\ph(t)\in E, \; t\in[\al,\be]$ и $\ph(\al)=x_1, \; \ph(\be)=x_2$.

Композиция $(f\circ\ph)(t)$ непрерывных отображений $f$ и $\ph$ ---
непрерывная вектор--функция, определённая на $[\al,\be]$. При этом
$F(\al) = f(\ph(\al)) = f(x_1) = y_1$ и $F(\be) = f(\ph(\be)) =
f(x_2) = y_2$, а также $F([\al,\be]) = f(\Ga) \subset f(E)$. Таким
образом, $f(\Ga)$ непрерывно в $f(E)$ и $f(E)$ --- связное в $\R^n$.
\end{proof}

\subsubsection{ Непрерывные отображения компактов в $\R^n$}

Рассмотрим произвольный компакт $C$ в $\R^m$.

\begin{thn}{1}
Если отображение $f\colon C\ra\R^n$ непрерывно на $C$, то $f$
равномерно непрерывно на $C$.
\end{thn}

\begin{proof}
Так как отображение $f$ непрерывно в произвольной точке $x\in C$, то
для любого $\ep>0$ существует $\de=\de(x,\ep)>0, \; \de=\de(x)$, что
$d_n(f(x'),f(x))<\frac{\ep}{3}$ для всех $x'\in\Uc(x,\de(x))\bigcap
C \subset \R^m$. Тогда
$$d_n(f(x'),f(x'')) \le d_n(f(x'),f(x)) + d_n(f(x''),f(x)) < \frac{\ep}3+\frac{\ep}3 < \ep$$
для любых
$x',x''\in \Uc(x,\de(x)) \bigcap C$. Обозначим $\Vc(x) =
\Uc\hr{x,\frac{\de(x)}2}, \; x\in C$. Тогда $\bigcup\limits_{x\in C}
\Vc(x) \supset C$ есть открытое покрытие компакта $C$, которое
обязано содержать некоторое конечное покрытие
$\Vc(x_1),\ldots,\Vc(x_k)$ компакта $C$, то есть
$\bigcup\limits_{j=1}^k \Vc(x_j) \supset C$. Положим $\de=\min
\hr{\frac{\de(x_1)}2,\ldots, \frac{\de(x_k)}2}, \;
\de>0,\de=\de(\ep)$.

Рассмотрим произвольные $x',x''\in C$, для которых $d_m(x',x'') <
\de$. Существует $\Vc(x_j)\owns x'$, то есть $d_m(x',x_j) <
\frac{\de(x_j)}2$. Тогда
$$d_m(x'',x_j) \le d_m(x'',x') + d_m(x',x_j) < \de + \frac{\de(x_j)}2 \le \frac{\de(x_j)}2 + \frac{\de(x_j)}2 =
\de(x_j),$$ то есть $x'' \in \Uc(x_j,\de(x_j))$ и $x'\in\Uc(x_j,\de(x_j))$, так что $d_n(f(x'),f(x'')) < \ep$.
\end{proof}

\begin{thn}{2}
Если отображение $f\colon C\ra\R^n$ непрерывно на компакте $C$, то
отображение $f$ ограничено на $C$.
\end{thn}

\begin{proof}
Как и в доказательстве теоремы 1, заключаем, что для произвольной
точки $x\in C$ существует такая её окрестность $\Uc(x)\subset\R^m$,
в которой непрерывной отображение $f$ локально ограничено, то есть
существует $M=M(x)>0$, что $d_n(f(x'),0) \le M(x)$ для всех
$x'\in\Uc(x)\cap C$. Так как $\bigcup\limits_{x\in C} \Uc(x)
\supset C$, то это открытое покрытие компакта $C$ содержит некоторое
конечное покрытие $\Uc(x_1),\ldots,\Uc(x_k)$, то есть $C\subset
\bigcup\limits_{j=1}^k \Uc(x_j)$.

Положим $M=\max(M(x_1),\ldots,M(x_k)), \; M>0$. Для любой $x\in C$
существует $\Uc(x_j)\owns x$, и, следовательно, $d_n(f(x),0) \le
M(x_j) \le M$.
\end{proof}

\begin{thn}{3}
Если функция $f(x)=f(x^1,\ldots,x^m)$ непрерывна на компакте
$C\subset \R^m$, то существуют такие точки $x_1,x_2\in C$, в которых
$f(x_1) \le f(x) \le f(x_2)$ для всех $x\in C$.
\end{thn}

\begin{proof}
Согласно теореме 2, существует число $m=\inf\limits_{x\in C} f(x),
\; M=\sup\limits_{x\in C} f(x)$ и $m\le f(x) \le M$ для всех $x\in
C$.

Если предположить, что $f(x)<M$ для всех $x\in C$, то для
произвольного $\ep>0$ существует $x_{\ep}\in C$ такое, что $M-\ep <
f(x_{\ep})$ и функция $\frac1{M-f(x)}$ определена и непрерывна на
$C$. Согласно теореме 2, функция $\frac1{M-f(x)}$ ограничена на $C$,
но её значение $\frac1{M-f(x_{\ep})} > \frac1{\ep}$, что ведёт к
противоречию в силу произвольности $\ep>0$.
\end{proof}

\subsubsection{Непрерывные отображения линейно связных множеств}

\begin{thh}
Если множество $E\subset\R^m$ линейно связно, то для любой
непрерывной функции $f(x)=f(x^1,\ldots,x^m)$ на $E$, и любых других
точек $a=(a^1,\ldots,a^m), \; b=(b^1,\ldots,b^m)\in E$, в которых
$f(a) = A; \; f(b)=B$, на $E$ существует точка $c\in E$, в которой
$f(c)=C$ для любого числа $C$, лежащего между $A$ и $B$.
\end{thh}

\begin{proof}
Так как $E$ --- линейно связно, существует непрерывная
вектор--функция $x=\ph(t)=(\ph^1(t),\ldots,\ph^m(t)), \;
t\in[\al,\be]$, что $\ph(\al)=a, \; \ph(\be)=b$ и $\ph(t)\subset C,
\; t\in[\al,\be]$, композиция $(f\circ\ph)(t)$ непрерывной функции
$f$ и непрерывного отображения $\ph$ непрерывна на $[\al,\be]$ и
$(f\circ\ph)(\al)$ и $(f\circ\ph)(\al)=f(\ph(\al))=f(a)=A; \;
(f\circ\ph)(\be)=f(\ph(\be))=f(b)=B$.

По теореме Коши о промежуточных значениях, существует
$\ga\in[\al,\be]$ такое, что $f(\ph(\ga))=C$, при этом $\ph(\ga)=c\in E$ (так как $\ph(t)\in E,
\; t\in[\al,\be]$).

Получаем, что $f(c)=C$.
\end{proof}

\section{Дифференцируемые функции нескольких переменных}

\subsection{ Частные производные}

\subsubsection{ Основные определения и обозначения}
Рассмотрим функцию $f(x)=f(x^1,\ldots,x^m)$, область определения
$D_f$ которой есть окрестность каждой своей точки; то есть,
множество $D_f$ --- открытое в $\R^m$. Фиксируем точку
$x=(x^1,\ldots,x^m)\in D_f$. Для любой точки
$x^1=(x_1^1,\ldots,x_1^m)\in D_f$ разность $x^1-x=\De x$ называется
(полным) \emd{приращением аргумента} функции $f$ в точке
$x$. На самом деле, $\De x$ --- вектор в $\R^m$ и $\De x =
(x_1^1-x^1,\ldots,x_1^m-x^m) = (\De x^1,\ldots, \De x^m)$. При этом
$\De x = \sum\limits_{k=1}^m \De x^k\,e_k$, где $e_k,\; k=\ol{1,m}$
--- стандартный базис.

$e_k=(0,\ldots,0,\us{k}{1},0,\ldots,1), \; k=\ol{1,m}$, и
$\De x^k \, e_k = (0,\ldots,\De x^k,\ldots,0), \; k=\ol{1,m}$ и
$f(x+\De x^k \, e_k) - f(x) = \De_k f(x)$ --- \emd{частное приращение функции} $f$ в точке $x$ по $k$\д й переменной $x^k$.

Разность $f(x+\De x) - f(x) = \De f(x)$ --- (полное)
\emd{приращение функции} $f$ в точке $x$, отвечающее
приращению $\De x$ аргумента в точке $x$.

\begin{dfn}{1}
\emd{Частной производной} функции $f(x)=f(x^1,\ldots,x^m)$
по $k$--ой переменной (\лк по $x^k$\пк) назовём функцию --- будем
обозначать её $\pd_k f(x)$ --- задаваемую следующими условиями:

а) областью определения функции $\pd_k f(x)$ служат все те точки
$x\in D_f$, для которых разностное отношение $\frac{f(x+\De x^k e_k)
- f(x)}{\De x^k}$ имеет предел по базе $\De x^k\ra0$ (при $\De
x^k\ra0)$

б) в каждой такой точке значение $\pd_k f(x)$ равно этому пределу,
то есть \eqa{1}{\pd_k f(x) = \liml{\De x^k \ra 0}\frac {f(x+
\De x^k e_k) - f(x)}{\De x^k}.}

В координатной форме формула~(1) имеет вид \eqa{1'}{\pd_k f(x) =
\liml{\De x^k\ra0} \frac {f(x^1,\ldots,x^{k-1},x^k+\De
x^k,x^{k+1},\ldots,x^m) - f(x^1,\ldots,x^m)}{\De x^k}.}
\end{dfn}

Для $f(x,y)$ справедливо
$$\pd_1 f(x,y) = \liml{h\ra0} \frac{f(x+h,y)-f(x,y)}{h}, \quad
\pd_2 f(x,y) = \liml{k\ra0} \frac{f(x,y+k)-f(x,y)}{k}\quad (h=\De x, \; k=\De y).$$

Рассмотрим частную функцию $\ph_k(z) = f(x^1,\ldots, x^{k-1}, z,
x^{k+1}, \ldots, x^m)$. Тогда
$$\ph_k(x^k)=f(x^1,\ldots, x^m), \quad \ph_k(x^k + \De x^k) = f(x^1,\ldots, x^{k-1}, x^k+\De x^k, x^{k+1},
\ldots, x^m)$$
и формула (1') принимает вид
$$\pd_k f(x) = \liml{\De x^k\ra0} \frac{\ph_k(x^k + \De x^k) - \ph_k (x^k)}{\De x^k} = \ph_k'(x^k); \qquad
\pd_k f := \frac{\pd f}{\pd x^k} = f'_{x^k};$$

В частности, для $f(x,y)$ имеем $f'_x = \frac{\pd f}{\pd x}, f'_y =
\frac{\pd f}{\pd y}$.

\begin{exx}
Функция $$f(x,y)=\bcase{\frac{xy}{x^2+y^2},\text{ если } x^2+y^2\ne0; \\ 0,
\text{ если } x^2+y^2=0 \; (x=y=0).}$$
имеет $f(x,0)=0, \; f(0,y)=0$
и, следовательно, $f'_x(0,0)=0, \; f'_y(0,0)=0$, но $f(x,y)$
разрывна в $(0,0)$ (не имеет предела).
\end{exx}

\subsubsection{ Производная по направлению}

Единичный орт (вектор единичной длины) $e$ задаёт направление в
$\R^m$. Когда говорят, что точка $x'$ находится от точки $x$ в
направлении орта $e$, то имеют в виду, что $x'=x+\rho e$, где
$\rho>0$.

Орты $e_k, \; k=\ol{1,m}$. Для любых $x,y\in\R^m$ однозначно
определён угол $\ph$ между векторами $x$ и $y$ по формуле
$\cos\ph=\frac{\ha{x,y}}{\hn{x}_m\cdot \hn{y}_m}$ или
$\ha{x,y}=\hn{x}_m\cdot \hn{y}_m \, \cos\ph$.

Если $x=(x^1,\ldots,x^m)$, то $x^k=\ha{x,e_k}$, где
$e_k,\;k=\ol{1,m}$ --- стандартный базис.

Если орт $e=(e^1,\ldots,e^m)$, то $e^k=\ha{e,e_k}, \; k=\ol{1,m}$.
Обозначим $\al_k$ --- угол между $e$ и $e_k, \; k=\ol{1,m}$. Тогда
$e^k=\ha{e,e_k} = \hn{e}_m \cdot \hn{e_k}_m \cdot \cos \al_k = \cos
\al_k, \; k=\ol{1,m}$.

Итак, $e=(\cos \al_1,\ldots,\cos \al_m)$ и числа $\cos
\al_1,\ldots,\cos\al_m$ --- направляющие косинусы орта $e$.

$\sum\limits_{k=1}^m \cos^2 \al_k=1.$

\begin{dfn}{2}
\emd{Производной по направлению} орта $e$ функции
$f(x)=f(x^1,\ldots,x^m)$ называют функцию --- будем обозначать её
$\pd_e f(x)$ --- заданную следующими условиями:

а) областью её определения служит множество тех точек $x\in D_f$,
для которых $\frac{f(x+te)-f(x)}{t}$ имеет предел при $t\ra0$;

б) в каждой такой точке значение $\pd_e f(x)$ равно этому пределу,
т.е. \eqa{2}{\pd_e f(x)=\liml{t\ra0} \frac{f(x+te)-f(x)}{t}.}
\end{dfn}

В частности, $\pd_k f(x) = \pd_{e_k} f(x), \; k=\ol{1,m}$.

Так как $e=(\cos \al_1,\ldots,\cos\al_m)$, то
$te=(t\cos\al_1,\ldots, t\cos\al_m)$. \eqa{2'}{\pd_e f(x) =
\liml{t\ra0}\frac{f(x^1+t\cos\al_1,\ldots, x^m+t\cos\al_m ) -
f(x^1, \ldots, x^m)}{t}.}

Рассмотрим функцию $\ph(t)=f(x^1+t\cos\al_1, \ldots, x^m+t\cos
\al_m)$, считая фиксированной точку $x=(x^1,\ldots,x^m)$ и орт $e$ с
направляющими косинусами.

\eqa{2''}{\pd_e f(x) = \liml{t\ra0}
\frac{\ph(t)-\ph(0)}{t}=\ph'(0).}

\begin{exx}
Функция $f(x,y)=\case{1, \mbox{ если } 0<y<x^2; \\ 0, \mbox{ если }
y\le0 \mbox { или } y\ge x^2.}$

$\pd_e f(0,0)$ существует (равна нулю). $f(x,y)$ в любой окрестности
точки имеет значение 0 и 1.
\end{exx}

\subsection{ Дифференцируемость функций нескольких переменных}
\subsubsection{ Понятие дифференцируемости функции}

\begin{dfn}{1}
Функцию $f(x)=f(x^1,\ldots,x^m)$ называют
\emd{дифференцируемой} в точке $x\in D_f$, если $D_f$ ---
окрестность точки $x$ и справедливо представление: \eqa{1}{f(x+\De
x) - f(x) = l(\De x) + \al(\De x) \cdot \hn{\De x}_m,} где $l(h) =
\sum\limits_{k=1}^m a_k h^k, \; h=(h^1,\ldots,h^m)\in \R^m$ ---
некоторая линейная функция в $\R^m$, а функция $\al(\De x) \bw= \al(\De
x^1,\ldots, \De x^m)$ непрерывна в нуле, и $\liml{\De x\ra0} \al(\De x) = 0 = \al(0)$.
\end{dfn}

Отметим, что функция $\hn{x}_m$ непрерывна в каждой $x\in\R^m$ как
композиция непрерывных функций ---
$$\hn{x}_m = \sqrt{\sum\limits_{k=1}^m (x^k)^2}.$$

\begin{thn}{1}
Функция $f(x)=f(x^1,\ldots,x^m)$ дифференцируема в точке $x\in D_f$
тогда и только тогда, когда справедлива формула \eqa{1'}{f(x+\De x)
- f(x) = l(\De x) + \sum\limits_{k=1}^m \al_k (\De x)\cdot \De x^k,}
в которой $l(\De x)$ --- некоторая линейная функция в $\R^m$, а
функции $\al_k(\De x) = \al_k(\De x^1,\ldots, \De x^m)$ непрерывны в
$\De x = 0$ и $\liml{\De x\ra0} \al_k(\De x)=0=\al_k(0), \;
k=\ol{1,m}.$
\end{thn}

\begin{proof}
Пусть $f\in\Dc(x)$ в смысле определения 1, то есть справедливо~(1).
Но \equ{\al(\De x) \, \hn{\De x}_m = \bcase{\sum\limits_{k=1}^m
\frac{\al(\De x)}{\hn{\De x}_m}\cdot (\De x^k)^2, \mbox { если } \De
x\ne0; \\ 0,\mbox{ если } \De x=0}} и~(1) переходит в (1') с
\equ{\al_k(\De x) = \bcase{\frac{\al(\De x)}{\hn{\De x}_m}\cdot (\De x_k)^2,\mbox { если } \De x\ne0, \; k=\ol{1,m};\\ 0, \mbox{ если } \De
x=0.}}

При этом $\hm{\al_k (\De x)} = \frac{ \hm{\al(\De x)}}{\hn{\De x}_m}
\cdot \hm{\De x^k} \le \hm{\al(\De x)}$ и, следовательно,
$\liml{\De x\ra0} \al_k(\De x) = 0, \; k=\ol{1,m}$, так как
$\liml{\De x\ra0} (\De x)=0$.

Пусть выполнено (1'). Положим \equ{\al(\De x) = \bcase{
\frac1{\hn{\De x}_m} \cdot \sum\limits_{k=1}^m \al_k (\De x) \cdot
\De x^k, \mbox{ если } \De x\ne0; \\ 0, \mbox { если } \De x=0.}}
Согласно неравенству Коши, \equ{\hm{\al(\De x)} \le \frac1{\hn{\De
x}_m} \sqrt{\sum\limits_{k=1}^m \al_k^2 (\De x)} \cdot
\sqrt{\sum\limits_{k=1}^m \hr{\De x^k}^2} =
\sqrt{\sum\limits_{k=1}^m \al_k^2 (\De x)},} и, следовательно,
$\liml{\De x\ra0} \al(\De x)=0$, так как $\liml{\De
x\ra0} \al_k(\De x) = 0, \; k=\ol{1,m}$.
\end{proof}

\begin{thn}{2}
Если функция $f(x)$ дифференцируема в точке $x\in D_f$, то $f$
непрерывна в $x$.
\end{thn}

\begin{proof}
Так как $f\in\Dc(x)$, то справедлива формула~(1), в которой функции
$l(\De x), \; \al(\De x)$ и $\hn{\De x}_m$ непрерывны на $\R^m$, и
следовательно, $\liml{\De x\ra0} l(\De x)=l(0)=0, \;
\liml{\De x\ra0} \al(\De x)=0, \; \liml{\De x\ra0}
\hn{\De x}_m = \hn{0}_m=0$. Поэтому
$$\liml{\De x\ra0} (f(x+\De x)-f(x))=0, \text{ то есть } f(x)=\liml{\De x\ra0} f(x+\De x).$$
\end{proof}

\subsubsection{ Дифференцируемость и частные производные}
\begin{thn}{3}
Функция $f$, дифференцируемая в $x=(x^1,\ldots,x^m)\in D_f\subset
\R^m$, имеет в $x$ \textbf{все} частные производные $\pd_k f(x), \;
k=\ol{1,m}$.
\end{thn}

\begin{proof}
По условию теоремы и теореме 1, справедливо представление
\equ{f(x+\De x) - f(x) = \sum\limits_{k=1}^m a_k \De x^k +
\sum\limits_{k=1}^m \al_k (\De x) \De x^k.}

Фиксируем $k, \; 1\le k \le m$, и рассмотрим $\ul{\De x = \De x^k
e_k}$. Тогда
$$f(x+\De x^k e_k) - f(x) = a_k \De x^k + \al_k(\De x^k) \De x^k \quad \text{и}\quad
\frac{f(x+\De x^k e_k)-f(x)}{\De x^k}=a_k+\al_k(\De x^k).$$

Так как $\liml{\De x^k\ra0} \al_k (\De x^k)=0$, то
$$a_k = \liml{\De x^k\ra0} \frac{f(x+\De x^k e_k)-f(x)}{\De x^k} = \pd_k f(x), \quad k=\ol{1,m}.$$
\end{proof}

\begin{imp*}
Если функция $f(x)=f(x^1,\ldots,x^m)$ дифференцируема в
$x=(x^1,\ldots,x^m)\in D_f$, то \eqa{1}{f(x+\De x) - f(x) =
\sum\limits_{k=1}^m \frac{\pd f}{\pd x^k}(x) \cdot \De x^k + \al(\De
x) \hn{\De x}_m,} где $\liml{\De x\ra0} \al(\De x)=0=\al(0)$
и \eqa{1'}{f(x+\De x) - f(x) = \sum\limits_{k=1}^m \frac{\pd f}{\pd
x^k} (x)\cdot \De x^k + \sum\limits_{k=1}^m \al_k(\De x) \cdot \De
x^k,} где $\liml{\De x\ra0} \al_k(\De x) = 0 = \al_k(0), \;
k=\ol{1,m}$.

\eqa{1''} {f(x+\De x) - f(x) = \sum\limits_{k=1}^m \frac{\pd f}{\pd
x^k}(x) \De x^k + \ol{\ol{o}} (\De x), \; \De x\ra0,} где
$\ol{\ol{o}} (\De x)$ непрерывна в $\De x=0.$
\end{imp*}

\begin{thn}{4}
Если функция $f(x)$ дифференцируема в точке $x\in D_f$ --- открытое
множество в $\R^m$, то в $x$ существуют все $\pd_e f(x)$ по любому
направлению $e$ и если $e=(\cos \al_1,\ldots,\cos \al_m)$, то
\eqa{2}{\pd_e f(x) = \sum\limits_{k=1}^m \frac{\pd f}{\pd
x^k}(x)\cos\al_k.}
\end{thn}

\begin{proof}
Согласно условию теоремы, справедливо (1') и если $\De x=t\,e=(t\cos
\al_1,\ldots, t\cos\al_m)$, то
\ml{f(x+t\,e)-f(x)=\sum\limits_{k=1}^m \frac{\pd f}{\pd x^k}(x)\cdot t\cos\al_k +
\sum\limits_{k=1}^m \al_k (t\, e)t\cos \al_k = \\ =
t\hs{\sum\limits_{k=1}^m \frac{\pd f}{\pd x^k} (x) \cos \al_k + \sum\limits_{k=1}^m \al_k (t\, e) \cos\al_k} =
\sum\limits_{k=1}^m \frac{\pd f}{\pd x^k} (x) \cos \al_k.}
\end{proof}

\subsubsection{ Градиент}

Рассмотрим функцию $f(x)=f(x^1,\ldots,x^m)$, дифференцируемую в
$x=(x^1,\ldots,x^m)\in D_f$. Согласно теореме 3, существует
$\frac{\pd f}{\pd x^k}(x), \; k=\ol{1,m}$. Вектор $\hr{\frac{\pd
f}{\pd x^1}(x), \ldots, \frac{\pd f}{\pd x^m}(x)}$ называют
\emd{градиентом} функции $f$ в точке $x$ и обозначают
$\grad f(x)$. Тогда, согласно~(2), $\pd_e f(x) = \ha{\grad f(x),e}$.

\begin{stm*}
Если $\grad f(x)\ne0$, то функция $=\pd_e f(x)$ имеет наибольшее
значение тогда, когда $e=\frac{\grad f(x)}{\hn{\grad f(x)}_e}$, то
есть когда $e$ --- направляющий орт градиента.
\end{stm*}

\begin{proof}
Согласно неравенству Коши\ч Буняковского и~(2), \equ{\hm{\pd_e f(x)}
= \hm{\ha{\grad f(x),e}} \le \hn{\grad f(x)}_m \cdot \hn{e}_m =
\hn{\grad f(x)}_m.}

С другой стороны,
$$\ha{\grad f(x), \frac{\grad f(x)}{\hn{\grad f(x)}_m}} = \hn{\grad f(x)}_m.$$
Значит, число $\hn{\grad f(x)}_m$ --- наибольшее значение для $\pd_e f(x)$, которое достигается для $e=\frac{\grad f(x)}{\hn{\grad f(x)}_m}$.
\end{proof}

Истолковывая производную $f(x)$ по направлению $e$ в точке $x$ как
скорость изменения функции $f$ в этом направлении, можно сказать,
что $\grad$ функции в точке есть вектор, указывающий направление и
скорость наибольшего роста функции в этой точке.

\equ{\hn{\grad f(x)}_m = \sqrt{\sum\limits_{k=1}^m \hr{\frac {\pd
f}{\pd x^k} (x)}^2}.}

\subsubsection{ Достаточное условие дифференцируемости}

\begin{thh}
Функция двух переменных $f(x,y)$ будет дифференцируемой в точке
$M(x,y)$, если частная производная по $y \; f'_y$ определена и
конечна в точке $M$, а $f'_x$ определена в некоторой окрестности
$\Uc$ точки $M(x,y)$ и непрерывна в $M(x,y)$. Функции $f'_x$ и
$f'_y$ можно поменять местами.
\end{thh}

\begin{proof}
Рассмотрим $(\De x, \De y)$ такие, что $(x+\De x,y+\De y)\in\Uc$, а
также точка $(x+\De x,y)\in \Uc$ и разность $f(x+\De x,y+\De
y)-f(x,y)$ представим в виде \eqa{3}{\De f(M) = f(x+\De x, y + \De
y) - f(x,y) = \hs{f(x+\De x, y + \De y) - f(x,y+\De y)} +
\hs{f(x,y+\De y) - f(x,y)}.}

Так как существует $f'_y(x,y)$, то \eqa{4}{\hs{f(x,y+\De y) -
f(x,y)} = f'_y(x,y) \De y + \al_2(\De y) \De y,} где $\liml{\De
y\ra0} \al_2(\De y)=0$ и~(4), определённую первоначально для $\De y
\ne 0$, можно доопределить в $\De y=0$, положив $\al_2(0)=0$, так
что $\al_2(\De y)$ --- непрерывная бесконечно малая функция
аргумента $\De y$.

$\liml{(\De x,\De y)\ra(0,0)} \al_2(\De y) = 0 = \al(0)$.

Согласно теореме о среднем значении, \eqa{5}{f(x+\De x, y + \De y) -
f(x,y+\De y) = f'_x (x+\theta\Delta x, y+ \De y)\cdot \De x.}

Так как $f'_x$ непрерывна в точке $(x,y)$, то
$$f'_x(x+\theta\Delta x, y+\De y) = f'_x(x,y) + \al_1(\De x, \De y),$$
где
$$\liml{(\De x, \De y)\ra(0,0)} \al_1 (\De x, \De y) = 0 = \al_1(0,0).$$

Подставляем~(4),(5),(6) в~(3); получим \equ{\De f(M) = f'_x(x,y) \De
x + f'_y (x,y) \De y + \al_1 \De x + \al_2 \De y,} где $\liml{(\De
x,\De y)\ra(0,0)} \al_1=0$, то есть, согласно (1'), $f(x,y)$
дифференцируема в $(x,y)$.
\end{proof}

Доказанная выше теорема справедлива при более сильных предположениях
о существовании обеих $f'_x$ и $f'_y$ в некоторых $\Uc$ точки
$M(x,y)$ и непрерывности их в точке $M(x,y)$.

\begin{thh}
Функция $f(x)=f(x^1,\ldots,x^m)$ будет дифференцируемой в
$x=(x^1,\ldots,x^m)\in D_f$ --- открытое множество в $\R^m$, если
все частные производные $\frac{\pd f}{\pd x^k}(x), \; k=\ol{1,m}$
существуют в некоторой окрестности $\Uc$ точки $x\in \Uc \subset
D_f$ и непрерывны в $x$.
\end{thh}

\begin{df}
Функцию $f(x)=f(x^1,\ldots,x^m)$ назовём \emd{принадлежащей классу $C^1$} на открытом множестве $G\subset D_f$, если в каждой
$x\in G$ существуют все $\frac{\pd f}{\pd x^k}(x), \; k=\ol{1,m}$ и
все $\frac{\pd f}{\pd x^k}$ непрерывны на $G$.
\end{df}

\subsubsection{ Частные производные сложной функции}

\begin{thh}
Пусть функция $f(x)=f(x^1,\ldots,x^m)$ имеет областью определения
$D_f=G$ открытое множество в $\R^m$ и $f$ дифференцируема в точке
$x=(x^1,\ldots,x^m)\in D_f$. Пусть функции $x^i=\ph^i(t), \;
\ph^i(t) = \ph^i(t^1,\ldots,t^k), \; i=\ol{1,m}$ образуют
непрерывное отображение $x=\ph(t)$ из $\R^k$ в $\R^m$, имеющее
$D_f=\ph^{-1}(G)$ и все функции $\ph^i(t), \; i=\ol{1,m}$
дифференцируемы в точке $t=(t^1,\ldots,t^k) \in \ph^{-1}(G)$. Тогда
композиция $F(t) = (f\circ\ph)(t) = F(t^1,\ldots,t^k)$
дифференцируема в $t=(t^1,\ldots,t^k)$ и справедливы формулы:

\eqa{*}{\frac{\pd F}{\pd t^1} = \frac{\pd f}{\pd x^1}\cdot \frac{\pd
x^1}{\pd t^1} \spl \frac{\pd f}{\pd x^m} \cdot \frac{\pd x^m}{\pd
t^m},} \equ{\ldots \ldots \ldots \ldots \ldots \ldots \ldots \ldots
\ldots \ldots \ldots \ldots \ldots} \equ{\frac{\pd F}{\pd t^k} =
\frac{\pd f}{\pd x^1}\cdot \frac{\pd x^1}{\pd t^k} + \ldots +
\frac{\pd f}{\pd x^m}\cdot \frac{\pd x^m}{\pd t^k}.}

Все $\frac{\pd F}{\pd t^j}, \; \frac{\pd x^i}{\pd t^j}, \;
j=\ol{1,k}, \; i=\ol{1,m}$ берутся в точке $t$, а все $\frac{\pd
f}{\pd x^i}$ --- в точке $x_k$.
\end{thh}

Так как отображение $x=\ph(t)$ непрерывно, то множество
$\ph^{-1}(G)$ --- открытое (как прообраз открытого множества при
непрерывном отображении).

Рассмотрим окрестность $\Uc$ точки $x\in \Uc \subset G$ и приращение
$\De x = (\De x^1,\ldots, \De x^m)$ такое, что $x+\De x\in \Uc$.
Тогда \eqa{7}{f(x+\De x) - f(x) = \sum\limits_{k=1}^m \frac{\pd
f}{\pd x^k}(x)\De x^k + \ol{\ol{o}}(\De x), \; \De x\ra0,} где
$\ol{\ol{o}}(\De x)$ непрерывна в $\De x = 0$.

Обозначим $\Vc=\ph^{-1}(\Uc) \subset D_{\ph}$ и $\Vc$ ---
окрестность точки $t\in\Vc\subset D_{\ph} \subset D_{\ph}^j,%верно?
\;j=\ol{1,k}$.

\eqa{8}{\De x^i=\ph^i(t+\De t) - \ph^i(t) = \sum\limits_{j=1}^k
\frac{\pd \ph^i}{\pd t^j}(t) \cdot \De t^j + \al_i(\De t) \cdot
\hn{\De t}_k,} где $\liml{\De t\ra0} \al_i(\De t) = 0 =
\al_i(0), \; i=\ol{1,m}$.

Имеем, на основании неравенства Коши, \eqa{9}{\hm{\De x^i} \le
\sqrt{\sum\limits_{j=1}^k \hr{\frac{\pd \ph^i}{\pd t^j}(t)}^2} \cdot
\sqrt{\sum\limits_{j=1}^k \hr{\De t^j}^2} + \hm{\al_i(\De t)} \cdot
\hn{\De_t}_k \le M\cdot \hn{\De t}_k + \hm{\al_i (\De t)} \hn{\De
t}_k, \; i=\ol{1,m}.} Следовательно, справедлива формула
\eqa{10}{\De x^i=O(\De t), \; \De t \ra0, \; i=\ol{1,m}} и все
функции $O(\De T)$ непрерывны в $\De t=0$.

Теперь представление для $F(t+\De t) - F(t) = (f\circ\ph)(t+\De t) -
(f\circ\ph)(t)$ получается подстановкой формул~(8), (9) и (10) (в
формуле~(7) $f(x+\De x)-f(x) = \sumium \frac{\pd f}{\pd x^i} (x) \De
x^i + \ol{\ol{o}}(\De x), \; \De x\ra0$) в формулу~(7), так что
\eqa{11}{\begin{aligned}
F(t+\De t) - F(t) &= \sum\limits_{k=1}^m \frac{\pd f}{\pd x^i}
\hs{\sum\limits_{j=1}^k \frac{\pd \ph^i}{\pd t^j} \De t^j +
\al_i(\De t) \hn{\De t}_k} + \ol{\ol{o}}(O(\De t)) =\\
&= \sumium \frac{\pd f}{\pd x^i} \sum\limits_{j=1}^k \frac{\pd \ph^i}{\pd t^j}
\De t^j + \sumium \frac{\pd f}{\pd x^i} \al_i(\De t) \hn{\De t}_m
+ \ol{\ol{o}}(O(\De t)) =\\&= \sum\limits_{j=1}^k \hr{\sumium
\frac{\pd f}{\pd x^i} \cdot \frac{\pd \ph^i}{\pd t^j}} \De t^j +
\ol{\ol{o}}(\De t) + \ol{\ol{o}}(O(\De t)) =\\
&= \sum\limits_{j=1}^k \hr{\sumium \frac{\pd f}{\pd x^i}\cdot \frac{\pd \ph^i}{\pd t^j}}\De t^j + \ol{\ol{o}}(\De t).
\end{aligned}}

Функция $F(t)$ --- дифференцируема, и, следовательно, справедлива
$(*)$. $\frac{\pd F}{\pd t^j} = \sumium \frac{\pd f}{\pd x^i}\cdot
\frac{\pd \ph^i}{\pd t^j}, \; j=\ol{1,k}$, что равносильно $(*)$.

\subsubsection{ Дифференциал функции нескольких переменных}

Рассмотрим функцию $f(x)=f(x^1,\ldots,x^m)$, дифференцируемую в
точке $x=(x^1,\ldots,x^m)\in D_f$. Тогда существует $\grad f(x) =
\hr{\frac{\pd f}{\pd x^1}(x), \ldots, \frac{\pd f}{\pd x^m}(x)}$ и
определена линейная функция $L(h) = \frac{\pd f}{\pd x^1}(x)\cdot
h^1 \spl \frac{\pd f}{\pd x^m}(x)\cdot h^m; \;
h=(h^1,\ldots,h^m)\in\R^m$, называемая дифференциалом функции $f$ в
точке $x$ и обозначаемая $df(x)(h) = \frac{\pd f}{\pd x^1}(x)\cdot
h^1 \spl \frac{\pd f}{\pd x^m}(x)\cdot h^m$. Если
$f_k(x)=x^k$, то $\frac{\pd f_k}{\pd x^k}(x) = 1, \; x\in\R^m$ и
$df_k(x)(h) = dx^k(h) = h, \; k=\ol{1,m}$.

$df(x)(h) = \frac{\pd f}{\pd x_1}(x) dx^1(h) \spl \frac{\pd f}{\pd
x^m}(x)dx^m(h); \; h\in\R$, или \equ{df(x) = \frac{\pd f}{\pd
x^1}(x)dx^1 \spl \frac{\pd f}{\pd x^m}(x)dx^m;} \eqa{1}{d f(x) =
\sumium \frac{\pd f}{\pd x^i} (x)dx^i;}

Эта величина является инвариантной относительно того, являются ли
$x^i, \; 1\le i \le m$, независимыми переменными или функциями
$x^i=x^i(t) = x^i(t^1,\ldots,t^k), \; 1\le i\le m$,
дифференцируемыми в соответствующей точке $t=(t^1,\ldots,t^k)\in
D_x$, где $x=x(t) = (x^1(t),\ldots,x^m(t))$ и
$x(t)=x=(x^1,\ldots,x^m)$. Действительно, полагая $(f\circ x)(t) =
f(x^1(t),\ldots,x^m(t)) = F(t)$ и используя формулу (1) и теорему о
дифференцировании сложной функции, имеем \equ{d(f\circ x)(t) = dF(t)
= \sum\limits_{j=1}^k \frac{\pd F}{\pd t^j}(t) dt^j =
\sum\limits_{j=1}^k \hr{\sumium \frac{\pd f}{\pd x^i}\cdot \frac{\pd
x^i}{\pd t^j}}dt^j = \sumium \frac{\pd f}{\pd x^i}
\hr{\sum\limits_{j=1}^k \frac{\pd x^i}{\pd t^j} dt^j} = \sumium
\frac{\pd f}{\pd x^i}dx^i=df(x).}

\textbf{Свойства дифференциалов}

Если функции $u,v$ дифференцируемы в одной и той же точке, то
справедливо:
\\ 1) $d(u\pm v) = du\pm dv; \; 2) d(cu)=c\,du, c\in\R.
\\ 3) d(uv)=v\,du+u\,dv; \; 4) d\hr{\frac{u}{v}}=\frac{v\,du-u\,dv}{v^2}; \; 5)
d\,c=0, \; c\in\R$.

\begin{proof}
Приведём доказательство свойства 3). Рассмотрим функцию $z=uv$.
Согласно свойству инвариантности первого дифференциала,
$dz=\frac{\pd z}{\pd u}\,du + \frac{\pd z}{\pd v}\,dv = vdu+udv.$
\end{proof}

\subsubsection{ Геометрический смысл дифференциала функции двух
переменных}

\begin{df}
Пусть функция $f(x,y)$ непрерывна в точке $P_0(x_0,y_0)\in D_f$
($D_f$ --- открытое множество в $\R^2)$. \emph{\textbf{Касательной
плоскостью}} к графику функции $f$ в точке $M_0(x_0,y_0,z_0), \;
z_0=f(x_0,y_0)$ называют такую плоскость, проходящую через $M_0$,
что расстояние $MN$ от точки $M(x,y,f(x,y)), P(x,y)\in D_f$, графика
до этой плоскости бесконечно мало по сравнению с $M_0M$ при
$P(x,y)\ra P_0(x,y)$.
\end{df}

\begin{thh}
Если функция $f(x,y)$ дифференцируема в точке $P_0(x_0,y_0)$, то
график $z=f(x,y)$ этой функции обладает в точке
$M_0(x_0,y_0,f(x_0,y_0))$ касательной плоскостью, которая задаётся
уравнением \eqa{2}{z-z_0=f'_x(x_0,y_0)(x-x_0)+f'_y(x_0,y_0)(y-y_0),}
так что значение дифференциала функции $f$ в точке $P_0$ при
приращениях  $x-x_0,y-y_0$ аргументов равно приращению $z-z_0$
аппликаты точки касательной плоскости (текущие координаты
касательной плоскости $x,y,z$ в отличие от текущих координат $x,y,z$
поверхности $z=f(x,y)$).
\end{thh}

\begin{proof}
Плоскость, задаваемая уравнением~(2), проходит через
$M_0(x_0,y_0,z_0), \; z_0=f(x_0,y_0)$. Обозначим
$\rho=P_0P=\sqrt{(x-x_0)^2 + (y-y_0)^2}$ и, в силу
дифференцируемости функции $f$ в точке $(x_0,y_0)$, справедливо
\eqa{3}{z-z_0 = f'_x(x_0,y_0)(x-x_0) + f'_y(x_0,y_0)(y-y_0) +
\al(\De x, \De y)\rho,} где $\liml{P(x,y)\ra P_0(x_0,y_0)} \al(\De
x,\De y)=0$.

Обозначим $\Pi$ плоскость, задаваемую уравнением~(2). Вычитая~(2) из
(3), получим $z-z_0=\al\rho$. Пусть $N$ --- основание
перпендикуляра, опущенного на $\Pi$ из точки $M(x,y,z), \; z=f(x,y)$
и $M'$ --- точка плоскости $\Pi$, имеющая ту же абсциссу $x$ и
ординату $y$, что и точка $M$. Тогда, $MN\le \hm{MM'}$ и $M_0M\ge
PP_0$. Поэтому, $0\le \frac{MN}{M_0M} \le \frac{\hm{MM'}}{P_0P} =
\frac{\hm{z-z}}{\rho}=\hm{\al}$. Следовательно,
$\frac{MN}{M_0M}\ra0$ при $\rho\ra0$, так как
$\liml{\rho\ra0}\hm{\al}=0$.

Итак, плоскость, задаваемая уравнением~(2) --- касательная к графику
в точке $M_0$.
\end{proof}

\subsection{ Частные производные и дифференциалы высших порядков}

\subsubsection{ Частные производные высших порядков}

Рассмотрим функцию $f(x)=f(x^1,\ldots,x^m)$, определённую на
открытом множестве $D_f$ в $\R^m$, и предположим, что частная
производная $\frac{\pd f}{\pd x^i}(x) = \ph_i(x), \; 1\le i \le m$,
определена на некотором открытом множестве $G\subset D_f$. Если в
точке $x\in G\subset D_f$ существует $\frac{\pd \ph_i}{\pd x^j}(x),
\; 1\le j \le m$, то её называют частной производной второго порядка
функции $f$ по аргументам $x^i$ и $x^j$ и обозначают $\frac{\pd^2
f}{\pd x^i \pd x^j}(x)$, то есть $\frac{\pd^2 f}{\pd x^i \pd x^j}(x)
= \frac{\pd}{\pd x^j}\hr{\frac{\pd f}{\pd x^i}(x)}, \; k\in\N$.

\subsubsection{ Достаточное условие равенства смешанных
производных}

\begin{thh}
Если в окрестности точки $(x,y)$ функция $f(x,y)$ обладает частными
производными $f'_x,f'_y$ и $f'_{xy}$ и $f''_{xy}$, непрерывными в
$(x,y)$, то другая смешанная производная $f''_{yx}$ существует в
этой точке и совпадает с $f''_{xy}$.
\end{thh}

\begin{proof}
По условию, существует такое $\de_0>0$, что функции $f,f'_x,f'_y$ и
$f''_{xy}$ определены для всех $(x+h,y+k)$ с $\hm{h} < \de_0, \;
\hm{k}<\de_0$. Положим $\ph(x)=f(x,y+k)-f(x,y)$. Тогда, применяя
дважды формулу конечных приращений, для всех $h,\hm{h}<\de_0$ и
$k,\hm{k}<\de_0$, имеем \eqa{1}{\ph(x+h)-\ph(x) = h\ph'(x+\ta,h) =
h\hs{f'_x(x+\ta_1h,y+k) - f'_x(x+\ta_1h,y)} =
hkf''_{xy}(x+\ta_1h,y+\ta_2k),} $0<\ta_1<1, \; 0<\ta_2<1$.

Так как функция $f''_{xy}$ непрерывна в $(x,y)$, то \eqa{2}{f''_{xy}
(x+\ta_1 h,y+\ta_2 k) = f''_{xy}(x,y)+\al(h,k),} где
$\liml{(h,k)\ra(0,0)}\al(h,k) = 0 = \al(0,0)$.

Подставляя~(2) и в~(1), получим \eqa{3}{\frac1h
\hs{\frac{\ph(x+h)}{k} - \frac{\ph(x)}{k}} - f''_{xy}(x,y) =
\al(h,k)} для всех $0<\hm{h} < \de_0, \; 0<\hm{k} < \de_0$.

Так как $\liml{(h,k)\ra(0,0)} \al(h,k)=0$, то для произвольного
числа $\ep>0$ существует $\de>0,\; 0<\de \le \de_0$, что
$\hm{\al(h,k)}<\ep$ для всех $0<\hm{h}<\de, \; 0<\hm{k} < \de$, и
следовательно, на основании~(3), получим оценку
\eqa{4}{\hm{\frac1h\hs{\frac{\ph(x+h)}{k} - \frac{\ph(x)}{k}} -
f''_{xy}(x,y)} < \ep, \; 0<\hm{h}<\de, \; 0<\hm{k}<\de.}

Так как $\liml{k\ra0} \frac{\ph(x)}{k} = \liml{k\ra0}
\frac{f(x,y+k)-f(x,y)}{k} = f'_y(x,y)$ и
$\liml{k\ra0}\frac{\ph(x+h)}{k} = f'_y(x+h,y)$, то, переходя в~(4) к
пределу при $k\ra0$, получим \eqa{5}{\hm{\frac{f'_y(x+h,y) -
f'_y(x,y)}{h} - f''_{xy}(x,y)} \le \ep, \; 0<\hm{h}<\de.}
\equ{f''_{xy}(x,y) = \liml{h\ra0} \frac{f'_y(x+h,y) - f'_y(x,y)}{h}
= f''_{yx} (x,y).}
\end{proof}

\begin{note}
Поскольку в этой теореме смешанную производную $f''_{xy}$ можно
заменить другой смешанной производной $f''_{yx}$, поскольку они
взаимозаменимы, то утверждение теоремы будет справедливо, если обе
смешанные производные непрерывны в точке $(x,y)$.
\end{note}

\begin{df}
Функцию $f(x)=f(x^1,\ldots,x^m)$ называют принадлежащей классу
$\Cc^n, \; n\in\N$, на открытом множестве $G\subset \R^m$, если все
её частные производные порядка $n$ (а следовательно, и все частные
производные низших порядков) непрерывны на $G$.
\end{df}

\begin{thh}
Если функция $f(x)=f(x^1,\ldots,x^m)$ принадлежит классу $\Cc^n$ в
своей области $D_f$ ($D_f$ --- открытое множество), то для любого
$k, \; 2\le k \le n$, у которой частной производной $k$--го порядка
любые входящие в неё дифференцирования по различным аргументам
перестановочны.
\end{thh}

\subsubsection{ Дифференциалы высших порядков}
Пусть $f(x,y)$ дифференцируема в $(x,y)$. Тогда её дифференциал
$$df(x,y)(h,k) = \pd_1 f(x,y)h + \pd_2f(x,y)k, \quad \pd_1f(x,y)=f'_x(x,y), \; \pd_2 f(x,y) = f'_y(x,y),$$
а $(h,k)$ --- независимые переменные, $(h,k)\in \R^2$. Фиксируем $h$ и $k$, можно
рассматривать $d$ как операцию $d=h\pd_1+k\pd_2$, применяемую к
$f(x,y)$. Такая операция линейна, то есть $d(f+g)=df+dg, \;
d(cf)=cdf$.

Функция $f(x,y)$ дифференцируема в $(x,y)\in D_f$ и $D_f$ ---
открытое множество в $\R^2$. Дифференциал функции $f$ в $(x,y)$
имеет вид:

$df(x,y)(h,k) = \frac{\pd f}{\pd x}(x,y)h + \frac{\pd f}{\pd
y}(x,y)k = h\pd_1f(x,y) + k\pd_2f(x,y); \; (h,k)\in\R^2$ ---
произвольные.

Формула справедлива, если $f$ принадлежит классу $\Cc^1$ в $D_f$.

Фиксируем $h$ и $k$. Операция $d=h\pd_1+k\pd_2$ применит к $f(x,y)$
сост. в дифор-ах $\pd_1,\pd_2$ функции $f$ в $(x,y)$, умнож. частн.
произв. соотв. на $h$ и $k$ и сложения результатов. %что за чушь?
Тогда операция $d=h\pd_1+k\pd_2$ линейна, то есть
$d\,(f+g)=d\,f+d\,g, \; d\,(cf)=cd\,f, \; c\in\R$.

Если операция $d$ применима к $f'_x(x,y)$ и $f'_y(x,y)$, то можно
рассмотреть её повторение $d^2\,f=d\,(d\,f)$ по правилам
\ml{d^2\,f(x,y)(h,k) = (h\pd_1+k\pd_2)(h\pd_1+k\pd_2)f(x,y) =
(h\pd_1+k\pd_2)(h\pd_1+f(x,y) + k\pd_2f(x,y)) = \\=h^2\pd_1^2 f(x,y)
+ hk\pd_1 \pd_2 f(x,y) + kh\pd_2\pd_1 f(x,y) + k^2\pd_2^2 f(x,y).}

Если функция $f\in\Cc^2$ в $D_f$, то дифференцирование $\pd_1$ и
$\pd_2$ перестановочно и формула примет вид (в более обычных
обозначениях): \equ{d^2\,f(x,y)(h,k) = \frac{\pd^2 f}{\pd
x^2}(x,y)h^2 + 2\frac{\pd^2 f}{\pd x \pd y}(x,y)hk + \frac{\pd^2
f}{\pd y^2}(x,y)k^2.}

Если функция $f$ принадлежит классу $\Cc^n$ в $D_f, \; n\in\N$, то
\eqa{2}{d^n f(x,y)(h,k) = (h\pd_1+k\pd_2)^n f(x,y) =
\sum\limits_{i=0}^n C_n^i \cdot \frac{\pd^{n-i}f}{\pd x^{n-i}} \cdot
\frac{\pd f^i}{\pd y^i}h^{n-i}k^i.} Действительно, так как
дифференцирование $\pd_1,\pd_2$ перестановочно, то справедлива
формула \equ{(h\pd_1+k\pd_2)^n = \sum\limits_{i=0}^n C_n^i
\pd_1^{n-i} \cdot \pd_2^i h^{n-i}k^i,} как если бы $\pd_1$ и $\pd_2$
были бы числами.

Если функция $f(x)=f(x^1,\ldots,x^m)$ принадлежит классу $\Cc^n$ на
открытом множестве $D_f\subset\R^m$, то \equ{d^n f(x)(h) = \hr{h^1
\frac{\pd}{\pd x^1} \spl h^m\frac{\pd}{\pd x^m}}^nf(x), \;
h=(h^1,\ldots,h^m)\in\R^m.}

\subsection{ Формула Тейлора}
\subsubsection{ Вспомогательные леммы}

Рассмотрим $f(x^1,x^2)$ класса $\Cc^k$ на открытом множестве
$D_f\subset\R^2$ и произвольное $x=(x^1,x^2)\in D_f$, так что
существует открытый шар $\Uc(x;r)\subset D_f, \; r>0$.

Рассмотрим произвольное приращение $h=(h^1,h^2)$, чтобы
$x+h\in\Uc(x;r)$ и функцию $\ph(t)=f(x+th)=f(x^1+th^1,x^2+th^2), \;
t\in[0,1]$.

Тогда \ml{\forall\;t\in[0,1] \; \exists \; \ph'(t) = \frac{\pd
f}{\pd x^1}(x^1+th^1,x^2+th^2) \frac{d}{dt}(x^1+th^1) + \frac{\pd
f}{\pd x^2}(x^1+th^1,x^2+th^2)\frac{d}{dt}(x^2+th^2) =\\= \frac{\pd
f}{\pd x^1}(x+th)h^1 + \frac{\pd f}{\pd x^2}(x+th)h^2 =
(h^1\pd_1+h^2\pd_2)f(x+th)} \equ{\ph''(t) = \frac{d}{dt}\ph'(t) =
h^1(\pd_1^2h^1+\pd_1\pd_2 h^2) f(x+ht) + h^2(\pd_2\pd_1h^1 +
\pd_2^2h^2) f(x+ht) = (h^1\pd_1+h^2\pd_2)^2 f(x+th).}

По индукции: \equ{\ph^{(k)}(t) = \hr{h^1\frac{\pd}{\pd x^1} +
h^2\frac{\pd}{\pd x^2}}^k f(x+th), \; t\in[0,1].}

\begin{lem}
Если функция $f(x)=f(x^1,\ldots,x^m)$ принадлежит классу $\Cc^k$ на
открытом множестве $D_f\subset\R^m$, то $\forall\;
x=(x^1,\ldots,x^m)\in D_f$ и $\forall\; h=(h^1,\ldots,h^m)$, для
которых $x+h\in\Uc(x;r) \subset D_f$, функция $\ph(t)=f(x+th)$ на
[0,1] имеет \eqa{1}{\ph^{(k)}(t) = \hr{h^1\frac{\pd}{\pd x^1} +
\ldots + h^m\frac{\pd}{\pd x^m}}^k f(x+ht).}
\end{lem}

\subsubsection{ Формула Тейлора с остаточным членом в форме
Лагранжа}

\begin{thh}
Если функция $f(x)=f(x^1,\ldots,x^m)\in\Cc^{n+1}, \; n\in\N$ на
открытом множестве $D_f\subset \R^m$, то для любой
$x=(x^1,\ldots,x^m)\in D_f$ и любого $h=(h^1,\ldots,h^m)$, что
$x+h\in \Uc(x,r) \subset D_f, \; r>0$, справедливо \ml{f(x+h)-f(x) =
f(x^1+h^1,\ldots, x^m+h^m) - f(x^1,\ldots,x^m) =
\\=\sum\limits_{k=1}^m \frac1{k!} \hr{h^1 \frac{\pd}{\pd x^1} + \ldots
+ h^m \frac{\pd}{\pd x^m}}^k f(x) + \frac1{(n+1)!} \hr{h^1
\frac{\pd}{\pd x^1} \spl h^m\frac{\pd}{\pd x^m}}^{n+1}f(x+\ta
h), \; 0<\ta<1.\;\;\;\;\;\;\;(2)}
\end{thh}

\begin{proof}
Как и в лемме, рассмотрим $\ph(t)=f(x+th), \;t\in[0,1]$, так что
$\ph(0)=f(x), \; \ph(1)=f(x+h)$. Применением к $\ph(t)$ формулы
Тейлора на $[0,1]$ с остаточным членом в форме Лагранжа, получим
\eqa{3}{\ph(1)-\ph(0) = \sumkun \frac1{k!}\ph^{(k)}(0) +
\frac1{(n+1)!}\ph^{(n+1)}(\ta), \; 0<\ta<1.}

Подставим в~(3) формулы~(1), $t=0$, получим~(2), т.к.
$\ph(1)=f(x+h), \; \ph(0)=f(x)$.
\end{proof}

\subsection{ Локальные экстремумы функций нескольких переменных}
\subsubsection{ Необходимое условие экстремума}

\begin{thn}{1 (Ферма)}
Если функция $f(x)=f(x^1,\ldots,x^m)$ принимает во внутренней точке
$(x_0^1,\ldots,x_0^m)$ множества $E\subset D_f$ наибольшее или
наименьшее значение на $E$ и дифференцируема в этой точке, то все
частные производные первого порядка в этой точке равны нулю.
\end{thn}

\begin{proof}
Пусть, например, $x_0$ --- точка наибольшего значения на $E$. Тогда
существует $r>0$, что открытый шар $\Uc(x_0,r)\subset E$ и
$f(x_0)\ge f(x)$ для всех $x\in \Uc(x_0,r)$.

Рассмотрим $i$--ую частную функцию $\ph_i(z) =
f(x_0^1,\ldots,x_0^{i-1},z,x_0^{i+1},\ldots,x_0^m), \; i=\ol{1,m}$,
определённую на $(x_0^i-r,x_0^i+r)$.

Мы знаем, что существует $\ph_i'(x_0^i)$ и $\ph_i(x_0^i) = f(x_0)
\ge \ph_i(z), \; z\in(x_0^i-r,x_0^i+r)$. По теореме Ферма:
$\ph_i'(x_0^i)=0$ или $\frac{\pd f}{\pd x^i}(x_0) = \ph_i'(x_0^i)=0,
\; i=\ol{1,m}$.
\end{proof}

\begin{df}
Точку $x=(x^1,\ldots,x^m)$, в которой функция
$f(x)=f(x^1,\ldots,x^m)$ дифференцируема и все $\frac{\pd f}{\pd
x^i}(x)=0, \; i=\ol{1,m}$, называется \emd{стационарной}
точкой функции $f$.
\end{df}

\begin{note}
В этом определении функция $f$ предполагается дифференцируемой в
точке (а не просто существование частных производных).
\end{note}

Для того, чтобы во \emph{внутренней} точке $x$ множества $E\subset
D_f$ функция $f$ принимала наибольшее или наименьшее значение на
$E$, \emph{необходимо}, чтобы $x$ была стационарной точкой функции
$f$.

\begin{exx}
Функция $f(x,y)=xy$, точка (0,0) --- стационарная, $f(0,0)=0$, но в
любой окрестности точки (0,0) существуют точки $(x,y)$, в которых
$f(x,y)>0$ и точки $(x,y)$, в которых $f(x,y)<0$.
\end{exx}

\subsubsection{ Достаточное условие локального экстремума}

Пусть функция $f(x)=f(x^1,\ldots,x^m)$ определена на открытом
множестве $G=D_f\subset\R^m$ и $x_0=(x_0^1,\ldots,x_0^m)$ ---
стационарная точка функции $f$, так что $\frac{\pd f}{\pd
x^i}(x_0)=0, \; i=\ol{1,m}$.

Рассмотрим открытый шар $\Uc(x_0,r) \subset G, \; r>0$ и
предположим, что функция $f(x)\in\Cc^2$ в $\Uc(x_0,r)$. Тогда для
произвольного $h=(h^1,\ldots,h^m)$, что $x_0+h\in \Uc(x_0,r)$.
Согласно формуле Тейлора: \eqa{1}{f(x_0+h)-f(x_0) = \frac1{1!}
\hr{h^1\frac{\pd f}{\pd x^1}(x_0) \spl h^m \frac{\pd f}{\pd
x^m}(x_0)} + \frac1{2!} \hr{h^1\frac{\pd}{\pd x^1} \spl h^m
\frac{\pd}{\pd x^m}}^2f(x_0+\ta h), \; 0<\ta<1.}

Так как $\frac{\pd f}{\pd x^i}(x_0)=0, \; i=\ol{1,m}$, то~(1) примет
вид \eqa{1'}{f(x_0+h)-f(x_0) = \frac12 \hr{h^1\frac{\pd}{\pd x^1} +
\ldots + h^m\frac{\pd}{\pd x^m}}^2 f(x_0+\ta h).}

Так как $\frac{\pd^2 f}{\pd x^i \pd x^j}(x_0+\ta h) = \frac{\pd^2
f}{\pd x^i \pd x^j}(x_0) + \al_{ij}(h)$, где $\liml{h\ra0}
\al_{ij}(h)=0= \al_{ij}(0), \; i,j=\ol{1,m}$.

Поэтому, из ($1'$) следует \eqa{2}{\frac12 \hr{h^1 \frac{\pd}{\pd
x^1} \spl h^m\frac{\pd}{\pd x^m}}^2 f(x_0+\ta h) = \frac12 \hr{h^1
\frac{\pd}{\pd x^1} \spl h^m\frac{\pd}{\pd x^m}}^2 f(x_0) + \frac12
\hr{h^1\be_1(h) \spl h^m\be_m(h)}^2,} где все функции $\be_i(h), \;
i=\ol{1,m}$ явно выражаются через $\al_{ij}(h)$ и $\liml{h\ra0}
\be_i(h)=0=\be_i(0), \; i=\ol{1,m}$.

Кроме того, \eqa{3}{\hm{h^1\be_1(h) \spl h^m\be_m(h)}^2 \le
\sum\limits_{i=1}^n (h^i)^2 \sumium (\be_i(h))^2 =
\hn{x}^2\hn{\be(h)}^2 = O(\hn{h}^2), \; h\ra0.}

Подставляя~(2) и~(3) в $(1')$, получим \eqa{4}{f(x_0+h)-f(x_0) =
\frac12 \hr{h^1\frac{\pd}{\pd x^1} \spl h^m \frac{\pd}{\pd
x^m}}^2 f(x_0) + O(\hn{h}^2), \; h\ra0,} где
\eqa{5}{\hr{h^1\frac{\pd}{\pd x^1} \spl h^m\frac{\pd}{\pd
x^m}}^2f(x_0) = \sum\limits_{i,j=1}^m \frac{\pd^2 f}{\pd x^i \pd
x^j}(x_0) h^ih^j.}

\begin{thn}{2}
Если функция $f(x)=f(x^1,\ldots,x^m)\in\Cc^2$ на открытом множестве
$G=D_f\subset \R^m$ и $x_0\bw=(x_0^1,\ldots,x_0^m)$ --- стационарная
точка функции $f$, то $f$ имеет в точке $x_0$ локальный экстремум
тогда и только тогда, когда квадратичная форма~(5) ---
знакопостоянная. При этом $x_0$ --- точка строгого максимума, если
форма~(5) --- положительная, и $x_0$ --- точка строгого минимума,
если форма~(5) --- отрицательная.
\end{thn}

\begin{proof}
Объединяя формулы~(1)---(5), имеем \eqa{6}{f(x_0+h) - f(x_0) =
\frac12 \hr{\sum\limits_{i,j=1}^m \frac{\pd^2 f}{\pd x^i \pd
x^j}(x_0) h^i h^j + O(\hn{h}^2)} = \frac{\hn{h}^2}2
\hr{\sum\limits_{i,j=1}^m \frac{\pd^2 f}{\pd x^i \pd x^j}(x_0)
\frac{h^i}{\hn{h}\frac{h^j}{\hn{h}} + O(1)}}, \; h\ra0\in \R^m.}

Вектор $e(h) = \hr{\frac{h^1}{\hn{h}},\ldots,\frac{h^m}{\hn{h}}} =
\frac{h}{\hn{h}}$ --- непрерывное отображение множества $\R^m_0 := \R^m\wo 0$ на сферу $S(0;1)$ единичного
радиуса с центром в начале координат, так как $\hn{e(h)}=1$.

Сфера $S(0,1)$ --- компакт в $\R^m$. Функция
$$F(e(h)) = \sum\limits_{i,j=1}^m \frac{\pd^2 f}{\pd x^i \pd x^j}(x_0)
\frac{h^i}{\hn{h}}\frac{h^j}{\hn{h}}$$
непрерывна на $S(0;1)$ и непрерывна на $\R^m_0$ как сложная функция. Так как $S(0,1)$ ---
компакт, то функция $F(e(h))$ достигает на $S$ своего наименьшего
значения $m$ и своего наибольшего значения $M$, $m\le M$. Более
того, числа $m$ и $M$ --- наименьшее и наибольшее значение сложной
функции $F(e(h))$ на множестве $\hc{h\in\R^m \; | \; 0 < \hn{h} <
r}$, где $r>0$ --- радиус шара $\Uc(x_0,r)$, в котором действительна
формула~(6).

Предположим, что форма~(5) положительна. Тогда $0<m\le M$, и из~(6)
следует: \equ{f(x_0+h)-f(x_0) > \frac{\hn{h}^2}2 (m+O(1)), \;
h\ra0.} Рассмотрим такое $\de_1, \; 0<\de_1 \le r$, чтобы $\hm{O(1)}
< \hm{m} =m$. Тогда
$$f(x_0+h)-f(x_0) \ge \frac{\hn{h}^2}2(m-\hm{O(1)}) > 0, \; h, 0<\hn{h}<\de_1,$$
то есть $x_0$ --- точка строгого минимума функции $f$.

Пусть квадратичная форма~(5) отрицательна. Тогда $m\le M<0$ и,
согласно~(6),
$$f(x_0+h)-f(x_0) < \frac{\hn{h}^2}2(M+O(1)).$$

Выберем такое число $\de_2, \; 0<\de_2 \le r$, чтобы
$\hm{O(1)}<\hm{M}$ для всех $h, \; o<\hn{h}<\de_2$, тогда
$f(x_0+h)-f(x_0)<0$ для всех $h,\; 0<\hn{h}<\de_2$ и $x_0$ --- точка
строгого максимума функции $f$.

Пусть форма~(5) знакопеременна. Тогда $m<0<M$. Рассмотрим на
$S(0;1)$ точку $e_m$, в которой $F(e_m)=m$ и произвольное $h=te_m,
\; t>0$. Тогда $\hn{h} = t\hn{e_m}=t$. Существует $\de_1, \; 0<\de_1
\le r$, что $\hm{O(1)} < \hm{m}$ для всех $h, \; 0<\hn{h} < \de_1$.
Тогда для $h=te_m, \; 0<t<\de_1$, справедливо $f(x_0+te_m) - f(x_0)
< 0$ для всех $t, \; 0<t<\de_1$.

Существует $\de_2, \; 0<\de_2\le r$ такое, что для произвольного
$h=te_M, \; 0<t<\de_2$ и $F(e_M)=M$ имеем $\hn{h}=t>0$ и
$f(x_0+te_M)-f(x_0) > 0$ для всех $t, \; 0<t<\de_2$. Итак, каждая
проколотая окрестность точки $x_0$ содержит $x_0+te_m, \;
0<t<\de_1$, в которой $f(x_0+te_m) - f(x_0) < 0$ и точки $x_0+te_M,
\; 0<t<\de_2$, в которой $f(x_0+te_M) - f(x_0) > 0$, то есть $x_0$
не есть точка локального экстремума функции $f$.
\end{proof}

\newpage

\section{Дифференцируемые отображения конечномерных евклидовых пространств}

\subsection{ Дифференцируемые отображения из $\R^m$ в $\R^n$}

\subsubsection{ Предварительные определения и обозначения}

Рассмотрим произвольное открытое множество $G$ в пространстве $\R^m$
и отображение $f\colon G\ra\R^n$. Отображение $f(x), \; x\in
G=D_f\subset \R^m$ задаётся набором $(f^1(x),\ldots,f^n(x))$ функций
$f^j(x) = f^j(x^1,\ldots,x^m), \; j=\ol{1,n}$. Фиксируем
произвольное $x=(x^1,\ldots,x^m)\in G$ и рассмотрим открытый шар
$\Uc(x,r)\subset G$. Пусть вектор $h=(h^1,\ldots,h^m)$ такой, что
$x+h = (x^1+h^1,\ldots,x^m+h^m) \in \Uc(x,r)$.

\emph{Отображение} $f(x+h)-f(x)$ называется
\emd{приращением} отображения $f(x)$ в точке $x$,
отвечающим приращению $h$ отрезка $x$.

В координатной форме $f(x+h)-f(x) = (f^1(x+h)-f^1(x),\ldots,
f^n(x+h) - f^n(x))$, где $f^j(x+h)-f^j(x) = f^j(x^1+h^1,\ldots,
x^m+h^m) - f(x^1,\ldots,x^m), \; j=\ol{1,n}$.

\begin{df}
Отображение $f(x)$ называется \emd{дифференцируемым} в
точке $x\in G=D_f\subset \R^m$, если существует линейное отображение
$L(x;h)\colon \R^m \ra \R^n$ ($x\in G$ --- фиксировано, $h\in\R^m$
--- произвольное).
\end{df}

Отображение $\al(x;h)$ из $\R^m$ в $\R^n$ ($x\in G$ --- фиксировано,
$h\in\R^m$ --- произвольное), которое недифференцируемо в точке
$h=(0,\ldots,0)\in\R^m$ и удовлетворяет условию $\liml{h\ra0}
\frac{\hn{\al(x;h)}_n}{\hn{h}_m}=0 \in \R$, так что для любого $h,
\; x+h\in \Uc(x;r)$ справедливо \eqa{1}{f(x+h) - f(x) = L(x;h) +
\al(x;h)} или \eqa{$1'$}{f(x+h) - f(x) = L(x;h) + o(\hn{h}_m), \;
h\ra0\in\R^m.} Условие $\liml{h\ra0}
\frac{\hn{\al(x;h)}_n}{\hn{h}_m}=0$ запишем в виде
$\al(x;h)=o(\hn{h}_m), \; h\ra0\in\R^m$.

Линейное отображение $L(x;h)$ задаётся матрицей \eqa{2}{A(x)=\rbmat{
a^1_1(x) & a_2^1 (x) & \ldots & a_m^1(x) \\ a_1^2(x) & a_2^2(x) &
\ldots & a_m^2(x) \\ \hdotsfor{4} \\ a_1^n(x) & a_2^n(x) & \ldots &
a_m^n(x)},} $x\in G, \; a_j^i(x) = a_j^i(x^1,\ldots,x^m), \;
i=\ol{1,m}, \; j=\ol{1,n}, \; x\in G$.

\eqa{3}{\rbmat{ f^1(x+h)-f^1(x) \\ \vdots \\ \vdots \\
f^n(x+h)-f^n(x)} = A(x) \rbmat{h^1 \\ \vdots \\ \vdots \\ h^m} +
\rbmat{\al^1(x;h) \\ \vdots \\ \vdots \\ \al^n(x;h)},} ($x\in G$
--- фиксировано, $h\in\R^m$ --- произвольное), где $o(x;h) =
(\al^1(x;h), \ldots, \al^n(x;h))$ и $\al^j(x;h)$ = \\ =
$\al^j(x^1,\ldots,x^m,h^1,\ldots,h^m), \; j=\ol{1,n}$.

\eqa{1}{f(x+h)-f(x) = L(x;h) + \al(x;h),} или \eqa{$1'$}{f(x+h)-f(x)
= L(x;h) + o(\hn{h}_m), \; h\ra0\in\R^m.}

\subsubsection{ Матрица Якоби}

Пусть отображение $f(x), \; x\in G=D_f$ --- открытое множество в
$\R^m$ дифференцируемо в $x\in G$, то есть справедливы~(1) и ($1'$),
в которых $\hn{(x;h)}_n=o(\hn{h}_m), \; h\ra0 \in \R^m$.

Так как $\hm{\al^j(x;h)} \le \hn{\al(x;h)}_n, \; j=\ol{1,n}$, то
$\al^j(x;h) = o(\hn{h}_m), \; h\ra0, \; j=\ol{1,n}$.

\eqa{4}{f^j(x+h) - f^j(x) = a_1^j(x)h^1 \spl a_m^j(x)h^m +
\al^j(x,h), \; j=\ol{1,n}.} Так как $\al(x;h)$ непрерывна в $h=0$,
то каждая $\al^j(x;h)$ непрерывна в $h=0, \; j=\ol{1,n}$.

$a_i^j(x) = \frac{\pd f^j}{\pd x^i}(x), \; i=\ol{1,m}, \;
j=\ol{1,n}$ и матрица~(2) принимает вид

\eqa{5}{A(x) = \rbmat{
\frac{\pd f^1}{\pd x^1}(x) & \frac{\pd f^1}{\pd x^2}(x) & \ldots &
\frac{\pd f^1}{\pd x^m}(x) \\ \frac{\pd f^2}{\pd x^1}(x) & \frac{\pd
f^2}{\pd
x^2}(x) & \ldots & \frac{\pd f^2}{\pd x^m}(x) \\ \hdotsfor{4} \\
\frac{\pd f^n}{\pd x^1}(x) & \frac{\pd f^n}{\pd x^2}(x) & \ldots &
\frac{\pd f^n}{\pd x^m}(x)}.}

\subsubsection{Критерий дифференцируемости отображения}

\begin{thn}{1}
Любое отображение $f\colon G\ra\R^n, \; G=D_f\subset \R^m$ ---
открытое множество, дифференцируемое в точке $x\in G$, непрерывно в
$x$.
\end{thn}

\begin{proof}
Если $f(x)$ дифференцируемо в $x\in G$, то справедливо~(1) или
$(1')$. Так как $L(x;h)$ непрерывно в $h=0$, то $\liml{h\ra0} L(x;h)
= L(x;0) = 0\in\R^m$. Кроме того, $\liml{h\ra0} \al(x;h) = \al(x;0)
= 0\in \R^m$. Поэтому, из~(1) следует, что $\liml{h\ra0}
(f(x+h)-f(x)) = 0\in \R^m$, или $\liml{h\ra0} f(x+h) = f(x)$, то
есть отображение $f$ непрерывно в $x$.
\end{proof}

\begin{thn}{2 (необходимое и достаточное условие дифференцируемости
отображений)} Отображение $f(x), \; x\in G=D_f\subset \R^m$ ---
открытое множество, задаваемое функциями $f^j(x) =
f^j(x^1,\ldots,x^m), \; j=\ol{1,n}$, дифференцируемо в точке $x\in
G$ тогда и только тогда, когда каждая функция $f^j(x), \;
j=\ol{1,n}$ дифференцируема в $x\in G$.
\end{thn}

\begin{proof}
\textbf{Необходимость}. Условно проверена в п. 1.2.

\textbf{Достаточность}. По условию, каждая $f^j(x) =
f^j(x^1,\ldots,x^m), \; j=\ol{1,n}$, дифференцируема в $x$, то есть
справедливо~(4).

Из~(4) следует~(3), в котором матрица $A(x)$ задаётся в виде~(2). Из
представления~(3) следует~(1) или ($1'$), в котором отображение
$\al(x;h)$ имеет компоненты $\al(x;h) = (\al^1(x;h), \ldots,
\al^n(x;h))$ и отображение $\al(x;h)$ непрерывно в $h=0$, так как
непрерывна каждая $\al^j(x;h)$ и $\al^j(x;h) = o(\hn{h}_m), \;
h\ra0, \; j=\ol{1,n}$.

Поскольку $\hn{\al(x;h)}_n \le \sqrt{n} \max\limits_{1\le j \le n}
\hm{\al^j(x;h)}$ и $\al^j(x;h) = o(\hn{h}_m), \; h\ra0, \;
j=\ol{1,n}$, то $\al(x;h) = o(\hn{h}_m), \; h\ra0$. Таким образом,
из~(3) следует ($1'$) и отображение $f(x)$ дифференцируемо в $x$.
\end{proof}

\subsection{Неявные отображения}

\subsubsection{ Предварительные замечания}

Функцию $u=f(x)=f(x^1,\ldots,x^m)$ назовём заданной неявно на
множестве $E\subset \R^m$, если существует функциональное уравнение
$F(x^1,\ldots,x^m,u) = 0, \; (x^1,\ldots,x^m,u) \in \R^{m+1}$, что в
каждой $x\in E$ функция $u=f(x)$ есть его единственное решение.

$F(x,y,u) = x^2+y^2+u^2-1$.

$F(x,y,u)=0$ --- уравнение сферы $S$ единичного радиуса с центром в
начале координат для декартовой системы. $(x,y,u)\in \R^3$.

$u=f_1(x,y) = \sqrt{1-x^2-y^2}, \; u=f_2(x,y) = - \sqrt{1-x^2-y^2},
\; (x,y)\in \Dc$, где $\Dc$ --- круг $x^2+y^2 \le 1$ на $\R^2\colon
Oxy$. Фиксируем точку $(x_0,y_0)\in \Dc$ и ищем $u=f(x,y)$, чтобы
$(x_0,y_0,r_0)\in S, \; u_0 = f(x_0,y_0)$.

Если $(x_0,y_0)\in \pd \Dc \; \fa
\Uc((x_0,y_0,u_0),\ep)\bigcap S = \Dc_{\ep}. \; \hm{u-u_0}<\ep$.

$\frac{\pd F}{\pd u} = 2u_0=0;$

$\frac{\pd F}{\pd u_0} - 2u_0>0$.

\subsubsection{ Существование и дифференцируемость неявной
функции}

\begin{thh}
Пусть функция $F(x,u)=F(x^1,x^2,u)$ дифференцируема в некоторой
окрестности $\Uc(M_0)$, \\$M_0(x^1_0,x^2_0,u_0)\in\R^3$ и частная
производная $F_u(x^1,x^2,u)$ непрерывна в $\Uc(M_0)$. Если
а)$F(x_0^1,x_0^2,u_0) = 0$ и \\ б)$F_u(x_0^1,x_0^2,u_0) \ne 0$, то
для любого числа $\ep>0$ существует такая окрестность
$\Uc(x_0)\subset \R^2, \; x_0=(x_0^1,x_0^2)$, в которой определена
единственная функция $u=\ph(x^1,x^2)$, обладающая свойствами:

1) функция $u=\ph(x^1,x^2)$ есть решение уравнения $F(x^1,x^2,u)=0$
в каждой точке $x=(x^1,x^2) \in \Uc(x_0)$, то есть
$F(x^1,x^2,\ph(x^1,x^2))=0$ для всех $x=(x^1,x^2) \in \Uc(x_0)$;

2) $u_0=\ph(x_0^1,x_0^2)$;

3) $\hm{u-u_0} < \ep$ для всех $x=(x^1,x^2) \in \Uc(x_0), \;
u=\ph(x^1,x^2)$;

4) функция $u=\ph(x^1,x^2)$ непрерывна в $\Uc(x_0)$;

5) функция $u=\ph(x^1,x^2)$ дифференцируема в $x_0=(x^1,x^2)$ и

6) $\frac{\pd \ph}{\pd x^1} (x_0^1,x_0^2) = - \frac{F'_{x^1}
(x_0^1,x_0^2,u_0)}{F_u' (x_0^1,x_0^2,u_0)}$ и $\frac{ \pd \ph}{\pd
x^2} (x_0^1,x_0^2) = - \frac{F'_{x^2} (x_0^1,x_0^2,u_0)}{
F'_u(x_0^1,x_0^2,u_0)}$.
\end{thh}

\begin{note}
Утверждение теоремы справедливо для функции $F(x,u)$ класса $\Cc^1$
в $\Uc(M_0)$.
\end{note}

\begin{proof}
Считаем, что $F'_u(x_0^1,x_0^2,u_0) > 0$. Так как $F_u'$ непрерывна
в $M_0(x_0^1,x_0^2,u_0)$, то существует открытый шар $\Uc(M_0;r), \;
r>0$, в котором $F'_u(x^1,x^2,u) > 0$ для всех $M(x^1,x^2,u) \in
\Uc(M_0;r)$. Считаем, что $\Uc(M_0;r) = \Om \subset \Uc(M_0)$.

Функция $F(x^1,x^2,u)$, будучи дифференцируемой в $\Om$, непрерывна
в $\Om$ и $f(u) = F(x_0^1,x_0^2,u)$ непрерывна на $[u_0-r, u_0+r]$,
дифференцируема в $(u_0-r,u_0+r)$ и $f'(u_0) = F'_u(x_0^1,x_0^2,u_0)
> 0$.

Рассмотрим произвольное $\ep>0, \; 0<\ep < r$. Тогда $f(u)$
непрерывна на $[u_0-\ep, u_0+\ep]$, дифференцируема в $(u_0 - \ep,
u_0 + \ep)$ и $f'(u_0) = F'_u(x_0^1,x_0^2,u_0)>0$. Тогда $f(u)
\uparrow\uparrow$ на $[u_0-\ep; u_0+\ep]$ и $f(u_0) =
F(x_0^1,x_0^2,u_0)=0$. Поэтому $f(u_0-\ep) < 0, \; f(u_0+\ep) > 0$
или $F(x_0^1,x_0^2, u_0 - \ep) < 0, \; F(x_0^1,x_0^2,u_0 + \ep)>0$.

Так как функция $F(x^1,x^2,u)$ непрерывна по аргументу $x=(x^1,x^2)$
при любом фиксированном $u \in [u_0-\ep, u_0+\ep]$, то по теореме о
сохранении знака непрерывной функции, существует такая окрестность
$\Uc(x_0) \in \Om$, в которой \eqa{1}{ F(x^1,x^2, u_0 - \ep)<0, \;
F(x^1,x^2,u_0 + \ep) > 0} для всех $x=(x^1,x^2) \in \Uc(x_0)$.

Выберем $\Uc(x_0)$ такую, что $\hm{x^1 - x_0^1} < \de, \; \hm{x^2 -
x_0^2} < \de, \; \de>0$.

\begin{points}{-1}
\item Существование неявной функции

По построению, параллелепипед $\Pi\colon \hm{x^1-x^1_0} < \de, \;
\hm{x^2 - x^2_0} < \de, \; \hm{u-u_0} < \ep$ лежит в шаре $\Om$.

Рассмотрим и фиксируем произвольное $x=(x^1,x^2) \in \Uc(x_0)$.
Функция $F=(x^1,x^2,u)$ ($(x^1,x^2)$ --- фиксировано) имеет $F>0$.
$u\in(u_0 - \ep, u_0 + \ep)$, и справедливо~(1). Следовательно, на
$( u_0 - \ep, u_0 + \ep)$ существует точка $u$ (выбор $u$ зависит от
$(x^1,x^2)$; то есть $u=\ph(x^1,x^2))$, в которой $F(x^1,x^2,u) =
0$. Другими словами, в $\Uc(x_0)$ определена некоторая функция
$u=\ph(x), \; F(x^1,x^2, \ph(x^1,x^2))=0$ для любой $x=(x^1,x^2) \in
\Uc(x_0)$.

\item Единственность неявной функции

Предположим, что в $\Uc(x_0)$ определены две различные функции $u_1
= \ph_1(x^1,x^2), \; u_2=\ph_2(x^1,x^2)$, что $$F(x^1,x^2,
\ph_1(x^1,x^2)) = F(x^1,x^2, \ph_2(x^1,x^2)) = 0$$ для всех
$x=(x^1,x^2) \in \Uc(x_0)$. Следовательно, существует $x=(x^1,x^2)
\in \Uc(x_0)$, в которой $u_1\ne u_2$. Так как отрезки с концевыми
точками $u_1$ и $u_2$ лежат на $[u_0 - \ep,u_0+\ep]$, то по теореме
Лагранжа о конечных приращениях, \equ{ 0 = F(x^1,x^2, u_1) -
F(x^1,x^2,u_2) = F'_u(x^1,x^2,u') (u_1 - u_2)} и $(x^1,x^2,u')
\in\Om$. Так как $u_1-u_2\ne0$, то $F'_u(x^1,x^2, u')=0$, что
противоречит построению шара $\Om$.

\item Непрерывность неявной функции

Так как $\hm{u-u_0} = \hm{\ph(x^1,x^2) - \ph(x_0^1,x_0^2)} < \ep$
для всех $x=(x^1,x^2) \in \Uc(x_0)$, то есть $\hm{x^1 - x_0^1} <
\de, \; \hm{x^2 - x_0^2}<\de$, то $\ph(x^1,x^2)$ непрерывна в $x_0 =
(x_0^1,x_0^2)$. Рассмотрим произвольную $\ol{x} = (\ol{x}^1,
\ol{x}^2) \in \Uc(x_0)$. Тогда существует такая окрестность
$\Uc(\ol{x}) \subset \Uc(x_0)$ и для всех $x\in \Uc(\ol{x}), \;
x=(x^1,x^2)$ справедливо $\hm{x^1 - x^1_0}<\de, \; \hm{x^2 - x^2_0}
< \de$, а также $\hm{\ol{x}^1 - x_0^1} < \de, \; \hm{\ol{x}^2 -
x_0^2} < \de$. Поэтому, $$\hm{\ph(x^1,x^2) - \ph(\ol{x}^1,\ol{x}^2)}
\le \hm{ \ph(\ol{x}^1,\ol{x}^2) - \ph(x_0^1,x_0^2)} +
\hm{\ph(x^1,x^2) - \ph(x_0^1,x_0^2)} < \ep+\ep = 2\ep.$$

Другими словами, $\ph(x^1,x^2)$ непрерывна в $\ol{x} =
(\ol{x}^1,\ol{x}^2) \in \Uc(x_0)$ и, следовательно, для
произвольного приращения $(\De x^1,\De x^2), \; \De x^1 = x^1 -
x_0^1, \; \De x^2 = x^2 - x_0^2$ и $\De u = \ph(x_0^1 + \De x^1,
x_0^2 + \De x^2) - \ph(x_0^1,x_0^2)$.

Справедливо $\lim\De u=0$ и $\De u$ --- непрерывна функция от $(\De
x^1,\De x^2), \; (\De x^1, \De x^2) \ra (0,0)$.

\item Дифференцируемость неявной функции в точке $(x_0^1,x_0^2)$

Рассмотрим произвольное $(\De x^1, \De x^2) \subset \hm{\De
x^1}<\de, \; \hm{\De x^2} < \de$ и приращение $\De u, \; \hm{\De u}
< \ep$. Точки $(x_0^1 + \De x^1, x_0^2 + \De x^2, u_0 + \De u)$ и
$(x_0^1,x_0^2,u_0)$ лежат на графике функции $u=\ph(x^1,x^2)$ и,
следовательно, $F(x_0^1 + \De x^1, x_0^2 + \De x^2, u_0 + \De u) =
F(x_0^1,x_0^2, u_0) = 0$.

Так как $F(x^1,x^2,u)$ дифференцируема в $(x_0^1,x_0^2,u_0)$, то, по
определению,
\ml{0 = F(x_0^1 + \De x^1, x_0^2 + \De x^2, u_0 + \De u) - F(x_0^1,x_0^2,u) = \\ =
\De F = F'_{x^1} (x_0^1,x_0^2, u_0) \De x^1 + F'_{x^2} (x_0^1,x_0^2, u_0) \De x^2 + F'_u (x_0^1,x_0^2, u_0) \De
u + \\ + \al_1(\De x^1, \De x^2, \De u) \De x^1 + \al_2( \De x^1,
\De x^2, \De u) \De x^2 + \al_3 (\De x^1, \De x^2, \De u) \De u_1,}
в котором $\liml{(\De x^1,\De x^2, \De u)\ra(0;0;0)} (\De x^1,\De
x^2, \De u) = 0, \; j=1,2,3$, и все $\al_j$ непрерывны в $(0;0;0)$.

Так как $F'_u$ непрерывна в $(x_0^1,x_0^2,u_0)$ и $F'_u(x_0^1,x_0^2,
u_0) > 0$, то для всех достаточно малых $(\De x^1, \De x^2, \De u)$
$$F'_u(x_0^1,x_0^2,u_0) + \al_3(\De x^1, \De x^2, \De u) > 0.$$
Следовательно, из~(2) получим \eqa{3}{\De u = -\frac{F'_{x^1}
(x_0^1, x_0^2, u_0) \De x^1}{F'_u(x_0^1, x_0^2, u_0) + \al_3} -
\frac{ F'_{x^2} (x_0^1,x_0^2, u_0) \De x^2}{F'_u(x_0^1, x_0^2, u_0)
+ \al_3} - \frac{\al_1}{F'_u(x_0^1, x_0^2, u_0) + \al_3} \De x^1 -
\frac{\al_2}{F'_u(x_0^1, x_0^2, u_0) + \al_3} \De x^2.}

Сложение функции $\al_j( \De x^1, \De x^2, \De \ph(\De x^1,\De x^2))
\ra 0$ при $(\De x^1,\De x^2) \ra 0$ и непрерывна в $(0;0)$ для всех
$j=1,2,3$. Так как $\frac1{1+\om} = 1-\om + \ol{\ol{o}}(\om), \;
\om\ra0$, то из~(3) следует \eqa{4}{\De u = - \frac{F'_{x^1}
(x_0^1,x_0^2,u_0)}{F'_u(x_0^1,x_0^2,u_0)}\De x^1 - \frac{F'_{x^2}
(x_0^1,x_0^2, u_0)}{F'_u(x_0^1,x_0^2, u_0)} \De x^2 + \al^1(\De
x^1,\De x^2) \De x^1 + \al^2 (\De x^1, \De x^2) \De x^2,} где
$\al^i(\De x^1,\De x^2) \ra 0$ при $(\De x^1,\De x^2) \ra 0$ и
$\al^i$ непрерывны в $(0,0), \;  i=1,2$, то есть $u=\ph(x^1,x^2)$
--- дифференцируема в $(x_0^1,x_0^2)$ и $$\frac{\pd \ph}{\pd x^1}
(x_0^1,x_0^2) = -\frac{F'_{x^1}(x_0^1, x_0^2, u_0)}{F'_u(x_0^1,
x_0^2, u_0)}, \; \frac{ \pd \ph}{\pd x^2}(x_0^1,x_0^2) = -
\frac{F'_{x^2} (x_0^1,x_0^2, u_0)}{F'_u(x_0^1,x_0^2, u_0)}.$$
\end{points}
\end{proof}

\begin{note}
Из доказательства следует, что производная $F'_u(x^1,x^2,u) \ne 0$
для всех $(x^1,x^2,u) \in \Om$, так что единственная неявная функция
$u=\ph(x,y)$ дифференцируема всюду в $\Uc(x_0)$ и
$$\ph'_{x^1}(x^1,x^2) = -
\frac{F'_{x^1}(x^1,x^2,u)}{F'_u(x^1,x^2,u)}, \; \ph'_{x^2} (x^1,x^2)
= -\frac{ F'_{x^2} (x^1,x^2,u)}{F'_u(x^1,x^2,u)}.$$
\end{note}

$F(u,x,y)=0$ имеет решение $u=\ph(x,y)$ в $\Uc(x_0,y_0)$,
удовлетворяющее теореме.

Рассмотрим $\Ph(x,y) = F(\ph(x,y),x,y)$. $$\frac{\pd \Ph}{\pd x} =
\frac{\pd F}{\pd u} \frac {\pd \ph}{\pd x} + \frac{\pd F}{\pd x}, \;
\frac{\pd \Ph}{\pd y} = \frac{\pd F}{\pd u} \frac{\pd \ph}{\pd y} +
\frac{\pd F}{\pd y}.$$

\begin{thn}{1}
Пусть функция $F(x,u) = F(x^1,\ldots,x^m,u)$ дифференцируема в
некоторой окрестности точки $(x_0,u_0) = (x_0^1,\ldots,x_0^m, u_0)
\in \R^{m+1}$ и частная производная $F'_u$ непрерывна в точке
$(x_0,u_0)$. Если $F(x_0,u_0)=0$ и $F'_u(x_0,u_0)\ne0$, то для
любого числа $\ep>0$ существует такая окрестность $\Uc(x_0)$ точки
$x_0 = (x_0^1,\ldots, x_0^m)\in \R^m$, в которой определена
единственная неявная функция $u=\ph(x) = \ph(x^1,\ldots,x^m)$,
обладающая свойствами:

1) $u=\ph(x)$ есть решение уравнения $F(x,u)=0$ в $\Uc(x_0)$, то
есть $F(x,\ph(x))=0$ для всех $x\in \Uc(x_0)$;

2) $u_0=\ph(x_0) = \ph(x_0^1,\ldots,x_0^m)$;

3) $\hm{u-u_0} < \ep$ для всех $x=(x^1,\ldots,x^m) \in \Uc(x_0)$;

4) функция $u=\ph(x) = \ph(x^1,\ldots,x^m)$ дифференцируема (а
значит и непрерывна) в $\Uc(x_0)$;

5) $$\frac{ \pd \ph}{\pd x^i}(x^1,\ldots, x^m) = - \frac{F'_{x^i}
(x^1,\ldots,x^m,u)}{F'_u(x^1,\ldots,x^m,u)}, \; i=\ol{1,m}.$$
\end{thn}

\begin{note}
Утверждение теоремы 1 справедливо, если $F(x,u)\in \Cc^1(G)$, $G$
--- открытое множество в $\R^{m+1}$ и $(x_0,u_0)\in G$ ---
произвольное.
\end{note}

\subsubsection{ Отображения, заданные неявно}

Рассмотрим систему из $n, \; n\in\N$, функциональных уравнений
\eqa{1}{\mat{F_1(x,u) = F_1(x^1,\ldots,x^m,u^1,\ldots,u^n)=0, \\
\hdotsfor{1} \\ F_n(x,u) = F_n(x^1,\ldots,x^m, u^1,\ldots,u^n)=0},}
где $F_j(x,u) = F_j(x^1,\ldots,x^m,u^1,\ldots,u^n), \; j=\ol{1,n}$,
дифференцируемы на некотором открытом множестве $E\subset \R^{m+n}$.

\begin{dfn}{1}
Функции \eqa{2}{\mat{u^1 = \ph_1(x^1,\ldots,x^m) \\ \hdotsfor{1} \\
u^n = \ph_n(x^1,\ldots,x^m)}} называются решением системы~(1) на
некотором непустом открытом множестве $G\subset \R^m$, если при
подстановке их в уравнения системы все уравнения системы переходят в
тождества на множестве $G$.

Функции~(2) образуют некоторое отображение $u=\ph(x)$ множества $G$
на некоторое множество $G^* = \ph(G) \subset \R^n, \;
u=(u^1,\ldots,u^n), \; x=(x^1,\ldots,x^m)$. Это отображение и
называют \emd{неявным отображением}.
\end{dfn}

\begin{dfn}{2}
Определитель \eqa{3}{\mbmat{\frac{ \pd F_1}{\pd u^1} & \hdots &
\frac{ \pd F_1}{\pd u^n} \\ \hdotsfor{3} \\ \frac{ \pd F_n}{\pd u^1}
& \hdots & \frac{ \pd F_n}{\pd u^n}}} называется определителем Якоби
(или якобианом) системы функций $F_j(x,u), \; j=\ol{1,n}$ (или
системы уравнений~(1)) по переменным $u^1,\ldots,u^n$ и обозначается
$\frac{D(F_1,\ldots,F_n)}{D(u^1,\ldots, u^n)}$. Понятно, что якобиан
(3) есть функция от $m+n$ переменных $(x^1,\ldots,x^m,u^1, \ldots,
u^n)$, определённая на множестве $E$.
\end{dfn}

\begin{thn}{2}
Пусть функции \equ{\mat{ F_1(x^1,\ldots,x^m,u^1, \ldots, u^n) \\
\hdotsfor{1} \\ F_n(x^1,\ldots,x^m,u^1,\ldots,u^n)}} дифференцируемы
в некоторой окрестности $\Uc(x_0,u_0)$ точки $(x_0,u_0) =
(x_0^1,\ldots,x_0^m, u_0^1, \ldots, u_0^n)$ из пространства
$\R^{m+n}$ и пусть первые производные этих функций по аргументам
$u^1,\ldots, u^n$ непрерывны в $(x_0,u_0)$. Если а) $F_j(x_0,u_0)=0,
\; j=\ol{1,n}$; б) якобиан $\frac{D(F_1,\ldots,F_n)}{D(u^1,\ldots,
u^n)} \ne 0$ в точке $(x_0,u_0)$, то для произвольного набора чисел
$\ep_1>0, \ldots, \ep_n>0$ существует такая окрестность $\Uc(x_0)$
точки $x_0$ в $\R^m$, в которой определено единственное неявное
отображение $u=\ph(x)$, порождаемое системой уравнений~(1) и
обладающей свойствами:

1) $u_0^j = \ph_j(x_0^1,\ldots, x_0^m)$;

2) $\hm{u^j - u_0^j} < \ep^j, \; j=\ol{1,n}$, где $u^j =
\ph_j(x^1,\ldots,x^m), \; u_0^j = \ph_j(x_0^1, \ldots, x_0^m)$ для
всех $x=(x^1,\ldots,x^m) \in \Uc(x_0)$;

3) отображение $u=\ph(x)$ дифференцируемо (а значит и непрерывно) в
окрестности $\Uc(x_0)$'

4) якобиан $\De = \frac{\pd F_1}{\pd u_1}, \; F_1(x,u) =
F_1(x^1,\ldots,x^m,u^1)$.
\end{thn}

\subsection{ Отображения с ненулевыми якобианами}
\subsubsection{ Существование локальных диффеоморфизмов}

\begin{thn}{1}
Пусть отображение $y=f(x), \; x=(x^1,\ldots, x^n), \;
y=(y^1,\ldots,y^n), \; y^j = f_j(x^1,\ldots,x^n), \; j=\ol{1,n}$ или
\eqa{1}{\mat{y^1 = f_1(x^1,\ldots,x^n) \\ \hdotsfor{1} \\
y^n=f_n(x^1,\ldots,x^n)}} из $\R^n$ в $\R^n$ дифференцируемо на
открытом множестве $G\subset \R^n$ и его якобиан
$\frac{D(y^1,\ldots,y^n)}{D(x^1,\ldots,x^n)}$ отличен от нуля в $x_0
\in G$. Тогда существуют окрестности $\Uc(x_0)$ и $\Uc(y_0), \;
y_0=f(x_0)$, в которых отображение $y=f(x)$ есть дифференцируемая
биекция множеств $\Uc(x_0)$ и $\Uc(y_0)$ и обратная биекция
$x=f^{-1}(y)$ множеств $\Uc(y_0)$ и $\Uc(x_0)$ также дифференцируема
в $\Uc(x_0)$. Такие отображения называют \emd{диффеоморфизмами}
(дифференцируемые биекции множеств).
\end{thn}

\begin{proof}
Перепишем систему~(1) в виде \eqa{2}{\mat{ f_1(x^1,\ldots,x^n) -
y^1=0 \\ \hdotsfor{1} \\ f_n(x^1,\ldots, x^n) - y^n = 0}} или
\eqa{2'}{F_j(x,y) = f_j(x^1,\ldots,x^n) - y^j, \; F_j(x,y)=0, \;
j=\ol{1,n}.}

$$\frac{D(F_1,\ldots, F_n)}{D(x^1,\ldots,x^n)} =
\frac{D(f_1,\ldots,f_n)}{D(x^1,\ldots,x^n)} \ne 0$$ в $x_0 =
(x_0^1,\ldots,x_0^n) \in G$.

Согласно теореме 2, существует окрестность $\Uc^* (x_0)$ и
$\Uc(y_0), \; \Uc^*(x_0) \subset G$, что в $\Uc(y_0)$ существует
отображение $x=g(y)$ множества $\Uc(y_0)$ в $\Uc^*(x_0)$,
дифференцируемое в $\Uc(y_0)$, при это для любого $y\in \Uc(y_0)$
справедливо $f(g(y))=y$, то есть $g=f^{-1}$, определённое в
$\Uc(y_0)$. Если $f^{-1}(\Uc(y_0)) \bigcap \Uc^*(x_0) = \Uc(x_0)$,
то $x_0\in \Uc(x_0)$ и отображения $f$ и $g=f^{-1}$ --- биекция
множеств $\Uc(x_0)$ и $\Uc(y_0)$. Так как отображение $f$
дифференцируемо в $G$, то $f$ непрерывно в $G$, и согласно
характеристическому свойству, непрерывное отображение
$f^{-1}(\Uc(x_0))$ --- открытое множество в $\R^n$. Поэтому
$f^{-1}(\Uc(y_0)) \bigcap \Uc^*(x_0) = \Uc(x_0) \subset G$ ---
открытое множество или окрестность точки $x_0$.
\end{proof}

Область переходит в область.

\begin{note}
Пусть
$\frac{D(f_1,\ldots,f_n)}{D(x_1^1,\ldots,x_n^n)} \ne 0$ и
$\frac{D(g_1,\ldots,g_n)}{D(y^1,\ldots,y^n)} = \frac{D(f_1^{-1},\ldots,f_n^{-1})}{D(y^1,\ldots,y^n)}\ne0$. Тогда
$$\frac{D(f_1,\ldots,f_n)}{D(x^1,\ldots,x^n)}\cdot \frac{D(f_1^{-1},
\ldots, f_n^{-1})}{D(y^1,\ldots,y^n)} = \hm{E} = \mbmat{1 & \hdots &
0 \\ \vdots & \ddots & \vdots \\ 0 & \hdots & 1}.$$
\end{note}

\subsubsection{ Принцип сохранения области}

\begin{thh}
Если $y=f(x)$ --- дифференцируемое отображение области $\Dc \subset
\R^n$ в $\R^n$, то её образ $f(\Dc)$ будет также областью в $\R^n$.
\end{thh}

\begin{proof}
Проверим сначала, что $\Dc^* = f(\Dc)$ --- открытое множество. По
условию, $\Dc$ --- открытое множество.

Рассмотрим произвольное $y\in D^*$ и в прообразе $f^{-1}(y) \in \Dc$
точки $y$ выберем некоторое $x\in f^{-1}(y) \subset \Dc$, так что
$y=f(x)$.

Для точек $x\in \Dc$ и $y=f(x) \in \Dc^*$ справедлива теорема 1
пункта 3.1., согласно которой, существуют окрестности $\Uc(x)
\subset \Dc$ и $\Uc(y)\subset \Dc^*$, что $\Uc(y) = f(\Uc(x))$. Но
$f(\Uc(x)) \subset f(\Dc) = D^*$, то есть $y$ входит в $\Dc^*$
вместе со своей окрестностью $\Uc(y)$, так что $\Dc^*$ --- открытое
множество.

По условию, область $\Dc$ --- связное множество в $\R^n$ и
дифференцируемое отображение $f$ области $\Dc$ непрерывно в $\Dc$ и
непрерывный образ $f(\Dc)$ связного множества $\Dc$ есть связное
множество в $\R^n$. Таким образом, $\Dc^* = f(\Dc)$ --- область в
$\R^n$.
\end{proof}

\subsubsection{ Зависимость функций}

Пусть на открытом множестве $G$ пространства $\R^m, \; m\ge1$,
заданы $n$ дифференцируемых функций ($n\in\N$). \eqa{3}{\mat{ u^1 =
\ph_1(x^1,\ldots,x^m), \\ \hdotsfor{1} \\ u^n =
\ph_n(x^1,\ldots,x^m)}.}

\begin{dfn}{1}
Некоторая функция из~(3), скажем, функция $u^k =
\ph_k(x^1,\ldots,x^m)$ называется \emd{зависимой} от остальных
функций из~(3), если на некотором открытом множестве $\Dc$
пространства $\R^n, \; n\ge1$, существует такая дифференцируемая
функция $\Ph = \ph(u^1,\ldots,u^{k-1},u^{k+1}, \ldots, u^n)$, что
$$u^k = \Ph(u^1,\ldots, u^{k-1}, u^{k+1}, \ldots, u^n) =
\Ph(\ph_1(x^1,\ldots, x^m), \ldots, \ph_{k-1}(x^1,\ldots,x^m),
\ph_{k+1} (x^1,\ldots,x^m), \ldots, \ph_n(x^1,\ldots,x^m))$$ для
всех $x=(x^1,\ldots,x^m) \in G$.
\end{dfn}

Если ни одна из функций в системе~(3) не является зависимой от
остальных на $G$, то система~(3) называется независимой системой на
$G$.

\begin{exx}
Функции $u_1=x+y, \; u_2=x-y$ функционально независимы в $\R^2$.
\end{exx}

\begin{proof}
Если $u_1$ и $u_2$ функционально зависимы в $\Uc(x_0,y_0)$, то
существует $u_1=\Ph(u_2)$, дифференцируемая на $(a,b) \subset \R$.

Если $u_2=\const$, то $u_1=\const$. С другой стороны, если $x-y=c$,
то $y=x-c$ и $u_1=2x-c$, что не является константой.
\end{proof}

\begin{exx}
Функции $u_1=x+y, \; u_2 = x^2+y^2, \; u_3=2xy, \; (x,y)\in\R^2$.
Тогда $u_2=u_1^2-u_3$.
\end{exx}

\begin{thh}
Если в системе функций \eqa{1}{\mat{ u^1=f_1(x^1,\ldots,x^m) \\
\hdotsfor{1} \\ u^n = f_n(x^1,\ldots,x^m)}}, дифференцируемых на
некотором открытом множестве $G\subset \R^m$, число $n<m$ и
якобианом системы~(1) по каким-либо $n$ аргументам отличен от нуля в
некоторой точке $x_0 = (x_0^1,\ldots,x_0^m) \in G$, то система~(1)
функционально независима в любой окрестности точки $x_0$.
\end{thh}

\begin{proof}
Не ограничивая общности, считаем, что \equ{\mat{u^1 =
f_1(x^1,\ldots,x^n,x^{n+1}, \ldots, x^m) \\ \hdotsfor{1} \\ u^n =
f_n(x^1,\ldots,x^n,x^{n+1}, \ldots, x^m)}} и $\frac{D(u^1,\ldots,
u^n)}{D(x^1,\ldots,x^n)}(x_0) \ne 0$. Допустим, что система~(1)
функционально зависима в некоторой окрестности $\Uc(x_0) \subset G$.
Тогда существует $\Ph = \Ph(u^1,\ldots,u^{k-1}, u^{k+1}, \ldots,
u^n)$, дифференцируемая на некотором открытом множестве $\Dc \subset
\R^n$ и $u^k = \Ph(u^1, \ldots, u^{k-1}, u^{k+1}, \ldots, u^n), \;
(u^1,\ldots,u^{k-1}, u^{k+1}, \ldots, u^n) \in \Dc$.

Сложная функция $\Ph = \Ph(f_1(x^1,\ldots,x^m), \ldots, f_{k-1}
(x^1,\ldots,x^m), f_{k+1} (x^1,\ldots, x^m), \ldots,
f_n(x^1,\ldots,x^m))$ дифференцируема на $G$ и $u^k=\Ph'$ и
\eqa{2}{\mat{ \frac{ \pd u^k}{\pd x^1} = \frac{\pd \Ph}{\pd u^1}
\frac{\pd u^1}{\pd x^1} \spl \frac{\pd \Ph}{\pd u^n} \frac{\pd
u^n}{\pd x^1}; \\ \hdotsfor{1} \\ \frac{ \pd u^k}{\pd x^m} =
\frac{\pd \Ph}{\pd u^1} \frac{\pd u^1}{\pd x^m} \spl \frac{\pd
\Ph}{\pd u^n} \frac{\pd u^n}{\pd x^m}}.}

Соотношение~(2) показывает, что в якобиане
$$\frac{D(u^1,\ldots,u^n)}{D(x^1,\ldots, x^n)} = \mbmat{\frac{\pd
u^1}{\pd x^1} & \hdots & \frac{\pd u^1}{\pd x^n} \\ \hdotsfor{3} \\
\frac{\pd u^k}{\pd x^1} & \hdots & \frac{\pd u^k}{\pd x^n} \\
\hdotsfor{3} \\ \frac{\pd u^n}{\pd x^1} & \hdots & \frac{\pd
u^n}{\pd x^n}}$$ $k$--ая строка есть линейная комбинация остальных.
Противоречие.
\end{proof}

\begin{exx}
$$\frac{D(u^1,u^2)}{D(x,y)} = \mbmat{ 1 & 1 \\ 1 & -1 } = -2 \ne 0.$$
\end{exx}

\subsection{ Относительные (или условные) экстремумы}
\subsubsection{ Понятие относительных экстремумов}

\emd{Относительный} (или \emd{условный})
\emd{экстремум} нескольких действительных переменных ---
это её максимум или минимум с учётом только тех точек области
определения, координаты которых связаны одним или несколькими
заданными уравнениями (так называемыми уравнениями связи).

Функция $u=xy$ не имеет экстремума в $(0,0)$, но если $y=x$, то
$u=x^2$ имеет в $(0,0)$ минимум, а если $y=-x$, то $u=-x^2$ имеет в
$(0,0)$ максимум.

Пусть функция $f(x,y)$ дифференцируема на открытом множестве
$G\subset \R^2$ и её аргументы $x,y$ связаны уравнением $F(x,y)=0$,
в котором функция $F(x,y)$ дифференцируема на $G$. Тогда функция $f$
есть функция одного аргумента, скажем, $x$, если $F(x,y)$
удовлетворяет условию теоремы о существовании и дифференцировании
неявной функции $y=y(x)$. Сложная функция $f(x,y(x))$
дифференцируема на некотором интервале.

Полная производная функции $f$ равна нулю, то есть $$\frac{\pd
f}{\pd x} + \frac{\pd f}{\pd y} \frac{dy}{dx} = 0.$$

Дифференцируя уравнение $F(x,y)=0, \; y=y(x)$, получим $$\frac{\pd
F}{\pd x} + \frac{\pd F}{\pd y} \frac{dy}{dx} = 0.$$

Имеем систему, $$\bcase{\frac{\pd f}{\pd x} + \frac{\pd f}{\pd y} \frac{dy}{dx} &= 0; \\
\frac{\pd F}{\pd x} + \frac{\pd F}{\pd y} \frac{dy}{dx} &= 0.}$$
$$\frac{\pd f}{\pd x} \frac{\pd F}{\pd y} - \frac{\pd f}{\pd y} \frac{\pd F}{\pd x} = 0.$$

Координаты стационарных точек находятся в виде решений: $$\bcase{
&\frac{\pd f}{\pd x} \frac{\pd F}{\pd y} - \frac{\pd f}{\pd y}
\frac{\pd F}{\pd x} = 0;\\
&F(x,y)=0.}$$

\subsubsection{ Общий случай}

Пусть задана функция $f$ из $\R^{m+n}$ в $\R$ и отображение $F$ из
$\R^{m+n}$ в $\R^n$. Тогда $F=(F_1,\ldots,F_n)$.

$F_j(x,y) = F_j(x^1,\ldots,x^m, y^1, \ldots, y^n), \; j=\ol{1,n}, \;
(x,y)\in D_F \subset \R^{m+n};$

$f(x,y) = f(x^1,\ldots,x^m, y^1, \ldots, y^n), \; (x,y) \in D_f
\subset \R^{m+n}$.

Точку $(a,b) = (a^1,\ldots,a^m, b^1,\ldots, b^n) \in D_f \bigcap
D_F$ называют точкой \emd{относительного максимума}
(\emd{относительного минимума}) функции $f$ при уравнении
связи $F_j(x,y) = 0, \; j=\ol{1,n}$, если она удовлетворяет этим
уравнениям, то есть $F_j(a,b) = 0, \; j=\ol{1,n}$ (или $F(a,b) =
0$), и обладает такой окрестностью $\Qc \subset D_f\bigcap D_F$, что
$f(a,b) \ge f(x,y)$ (соответственно, $f(a,b) \le f(x,y)$) для всех
$(x,y)\in \Qc$, удовлетворяющих уравнениям связи $F(x,y) = 0$, или
$F_j(x,y) = 0, \; j=\ol{1,n}$.

Считаем функцию $f(x,y)$ дифференцируемой на некотором открытом
множестве $G \subset D_f \subset \R^{m+n}$, и все функции, входящие
в уравнения связи \eqa{1} {F_j(x,y) = F_j(x^1,\ldots, x^m, y^1,
\ldots, y^n)=0, \; j=\ol{1,n},} дифференцируемы на $G\subset D_F
\subset \R^{m+n}$. Предположим, что $\frac{ D(F_1,\ldots, F_n)}
{D(y^1, \ldots, y^n)} \ne 0$ на $G$. Тогда $f(x,y)$ есть сложная
функция от аргументов $x^1,\ldots, x^m$.

Полная производная функции $f(x^1,\ldots, x^m, y^1, \ldots, y^n), \;
y^j = y^j(x^1, \ldots, x^m), \; j=\ol{1,n}$, равна нулю, то есть
\eqa{2}{\frac{ \pd f}{\pd x^1}\,dx^1 + \frac{\pd f}{\pd x^2} \,dx^2
\spl \frac{\pd f}{\pd x^m} \,dx^m + \frac{\pd f}{\pd y^1}
\,dy^1 \spl \frac{\pd f}{\pd y^n} \,dy^n=0,} где
$dx^1,\ldots,dx^m$ --- дифференциалы независимых переменных, а
$dy^1, \ldots, dy^n$ --- дифференциалы функций от аргументов
$x^1,\ldots,x^m$. \eqa{3}{ \frac{\pd F_j}{\pd x^1} \, dx^1 + \ldots
+ \frac{\pd F_j}{\pd x^m} \, dx^m + \frac{\pd F_j}{\pd y^1} \, dy^1
\spl \frac{\pd F_j}{\pd y^n} \, dy^n, \; j=\ol{1,n}.}

Имеем $n+1$ уравнение для нахождения стационарных точек. Так как
якобиан $\ne0$, то $dy^1, \ldots, dy^n$ линейно выражаются из~(3).

\eqa{4} {M_1 \, dx^1 + M_2 \, dx^2 \spl M_m\, dx^m=0,} $M_i,
\; i=\ol{1,m},$ явно выражаются через частные производный функций
$f$ и $F_j, \; j=\ol{1,n}$.

Так как $dx^1, \ldots, dx^m$, то из~(4) следует, что $M_i=0, \;
i=\ol{1,m}$.

\subsubsection{ Метод множителей Лагранжа}

Умножим уравнения~(3) на постоянные неизвестные множители $\la_1, \ldots, \la_n$, соответственно, и сложим с~(2).

\eqa{5}{A_1 \, dx^1 + A_m \, dx^m + B_1\, dy^1 + \B_n\, dy^n=0,} где
$A_i, \; i=\ol{1,m}$ и $B_j, \; j=\ol{1,m}$ --- функции, явно
выражающиеся через частные производные.

Допустим, что $\la_1, \ldots, \la_n$ можно выбрать так, чтобы $B_1=
\ldots = B_n=0$. Тогда~(5) примет вид $$ A_1 \, dx^1 \spl A_m
\,dx^m=0$$ и все $A_1=\ldots=A_m=0$ в силу произвольности
дифференциалов $dx^1,\ldots,dx^m$. Получаем $2n+m$ уравнений для
нахождения $x^1,\ldots,x^m,y^1,\ldots,y^n$ и $\la_1,\ldots,\la_n$.

$$\bcase{\frac{\pd f}{\pd x^1} &+ \la_1 \frac{\pd F_1}{\pd x^1} \spl \la_n \frac{\pd F_1}{\pd x^1}=0; \\
\ldots & \\
\frac{\pd f}{\pd x^m} &+ \la_1 \frac{\pd F_1}{\pd x^m} \spl \la_n \frac{\pd F_n}{\pd x^m}=0; \\
\frac{\pd f}{\pd y^1} &+ \la_1 \frac{\pd F_1}{\pd y^1} \spl \la_n \frac{\pd F_1}{\pd y^1}=0; \\
\ldots &
\\ \frac{\pd f}{\pd y^n} &+ \la_1 \frac{\pd F_1}{\pd y^n} \spl \la_n \frac{\pd F_n}{\pd y^n} = 0.}$$

Рассмотрим функцию $\Ph = f = \la_1 F_1 + \ldots \la_nF_n$.

$\Ph(x,y,\la) = \Ph(x^1,\ldots,x^m,y^1,\ldots, y^n, \la_1, \ldots,
\la_n)$. $\boxed{2n+m}$

(6) эквивалентно \eqa{7}{\frac{ \pd \Ph}{\pd x^i} = 0, \;
i=\ol{1,m}, \; \frac{\pd \Ph}{\pd y^j} = 0, \; j=\ol{1,n}. \;
F_j(x,y) = 0, \; j=\ol{1,n}.}

\ml{d^2\Ph = \frac12 \hr{ \frac{\pd}{\pd x^1}\, dx^1 +
\frac{\pd}{\pd x^m}\, dx^m + \frac{\pd}{\pd y^1}\, dy^1 \spl
\frac{\pd}{\pd y^n}\, dy^n}^2 \Ph + \frac{\pd F}{\pd y^1} d^2 y^1 +
\ldots + \frac{\pd F}{\pd y^n} d^2 y^n = \\ = \frac12 \hr{
\frac{\pd}{\pd x^1}\, dx^1 + \frac{\pd}{\pd x^m}\, dx^m +
\frac{\pd}{\pd y^1}\, dy^1 \spl \frac{\pd}{\pd y^n}\, dy^n}^2
\Ph.}

\subsection{ Дополнительные результаты}

\subsubsection{ Теорема Эйлера о дифференцируемости однородных
функций}

Функция $f(x^1, \ldots, x^m)$ называется \emd{однородной},
если формула \eqa{1} {f(tx^1, tx^2, \ldots, tx^n) = t^n
f(x^1,\ldots,x^m)} выполняется для всех $t$. Число $n\in\N$
называется \emd{степенью однородности} функции $f$.

Рассмотрим функцию $f(x,y,z)$. Тогда~(1) имеет вид \eqa{1'}
{f(tx,ty,tz)= t^n f(x,y,z).}

Положим $t=\frac1x$ в (1'). Тогда $$f(x,y,z) = x^n f
\hr{1,\frac{y}{x}, \frac{z}{x}},$$ или \eqa{2}{f(x,y,z) = x^n \ph
\hr{\frac{y}{x}, \frac{z}{x}}.}

Таким образом, если разделить однородную функцию $n$--ой степени на
$n$--ую степень одной из переменных, то частное будет зависеть
только от отношения переменных между собой.

Обратно, если справедлива формула~(2), то $f(tx,ty,tz) = t^n x^n \ph
\hr{\frac{y}{x}, \frac{z}{x}} = t^n f(x,y,z)$, то есть справедливо
(1'). Дифференцируя (1') по $t$, получим \eqa{3}{x f'_x (tx,ty,tz) +
yf'_y (tx,ty,tz) + zf'_z (tx,ty,tz) = nt^{n-1} f(x,y,z),} и полагая
в~(3) $t=1$, имеем \eqa{4}{xf'_x (x,y,z) + yf'_y (x,y,z) +
zf'_z(x,y,z) = nf(x,y,z).} В формуле~(4) и заключается теорема
Эйлера.

\begin{thn}{Эйлера}
Сумма произведений частных производных однородной функции на
соответствующие переменные равна произведению самой функции на её
степень однородности.
\end{thn}

Дифференцируя~(3) по $t$, получим \ml{ x^2 f''_{xxx} (tx,ty,tz) + xy f''_{xy} (tx,ty,tz) + xzf''_{xz} (tx,ty,tz) +
yx f''_{yx}(tx,ty,tz) + y^2 f''_{yy} (tx,ty,tz) + yz f''_{yz} (tx,ty,tz) + \\ +
zx f''_{zx} (tx,ty,tz) + zy f''_{zy} (tx,ty,tz) + z^2 f''_{zz} (tx,ty,tz) = n(n-1) t^{n-2} f(x,y,z).}

Для любого $k\in\N$ $$ \hr{x^1 \frac{\pd}{\pd x^1} \spl x^m
\frac{\pd}{\pd x^m}}^k f(x^1,\ldots,x^m) = n(n-1) \cdot \ldots \cdot
(n-k+1) f(x^1,\ldots,x^m).$$

\subsubsection{ Теорема о пределе частных производных}
\begin{thh}
Если $f'_x(x+h,y,z)$ имеет предел при $h\ra0$, то $\liml{h\ra0}
f'_x(x+h,y,z) = f'_x(x,y,z)$.
\end{thh}

\begin{proof}
По формуле конечных приращений, $$\frac{f(x+h,y,z)-f(x,y,z)}{h} =
f'_x(x+\ta h,y,z), \; 0<\ta<1$$ и $$\liml{h\ra0} f'_x (x+\ta h,y,z)
= \liml{h\ra0} f'_x(x+h,y,z) = \liml{h\ra0}
\frac{f(x+h,y,z)-f(x,y,z)}{h} = f'_x(x,y,z).$$
\end{proof}

\subsubsection{ Преобразование Лежандра}
Пусть $z=f(x,y)$. Положим $p=\frac{\pd z}{\pd x}, \; q=\frac{\pd
z}{\pd y}$ и считаем, что функции $p$ и $q$ функционально
независимы.

$\boxed{p \mbox{ и } q}$ --- независимые переменные. Рассмотрим
новую функцию \eqa{6}{u=px+qy-z.}

Так как $dz=p\,dx+q\,dy$, то $$du=p\,dx + x\,dp + q\,dy + y\,dq - dz
= x\,dp + y\,dq + (p\,dx + q\,dz) - (p\,dx + q\,dy) = x\,dp +
y\,dq.$$ Следовательно, $$\frac{\pd u}{\pd p}=x; \; \frac{\pd u}{\pd
q} = y.$$

Дифференцируя последние две формулы, получим \eqa{8}{dx =
\frac{\pd^2 u}{\pd p^2}\,dp + \frac{\pd^2 u}{\pd p \pd q}\,dq, \; dy
= \frac{\pd^2 u}{\pd q \pd p}\,dp + \frac{\pd^2 u}{\pd q^2}\, dq.}

Так как в уравнениях~(8) $dx$ и $dy$ --- произвольные, то
определитель системы обязан быть отличен от $0$.

\eqa{9}{ H = \frac{\pd^2 u}{\pd p^2} \frac{\pd^2 u}{\pd q^2} -
\hr{\frac{\pd^2 u}{\pd p \pd q}}^2 \ne 0.}

Разрешая систему~(8) относительно $dp$ и $dq$, получим $$dp =
\frac1H \hr{\frac{\pd^2 u}{\pd q^2}\,dx - \frac{\pd^2 u}{\pd p \pd
q}\,dy}, \; dq = \frac1H \hr{\frac{\pd^2 u}{\pd p^2}\,dy -
\frac{\pd^2 u}{\pd p \pd q}\,dx}.$$

$$d^2z = d(dz) = d(p\,dx+q\,dy)=dx\,dp+dy\,dq.$$

$$d^2z=\frac1H \hr{\frac{\pd^2 u}{\pd q^2}\,dx^2 - 2\frac{\pd^2 u}{\pd p
\pd q}\,dx\,dy + \frac{\pd^2 u}{\pd p^2}\,dy^2},$$ откуда
$$\frac{\pd^2 z}{\pd x^2} = \frac1H \frac{\pd^2 u}{\pd q^2}, \;
\frac{\pd^2 z}{\pd x \pd y} = -\frac1H \frac{\pd^2 u}{\pd p \pd
y};$$ $$\frac{\pd^2 z}{\pd y^2} = \frac1H \frac{\pd^2 u}{\pd p^2}.$$

\subsubsection{ Теорема С. Банаха о неподвижной точке}
Пусть $X=(X;p)$ --- полное метрическое пространство. Отображение
$f\colon X\ra X, \; D_f=X$, называется \emph{\textbf{сжимающим
отображением}}, если существует $q, \; 0<q<1$, что $\rho
(f(x'),f(x'')) \le q \rho(x',x'')$.

Из определения следует, что сжимающее отображение $f$ равномерно
непрерывно на $X$ $(\de=\ep)$ и, следовательно, непрерывно на $X$.

\begin{thn}{(принцип сжимающего отображения)}
Если $f\colon X\ra X$ сжимающее, то существует единственная $x_0 \le
x$, для которой $f(x_0)=x_0$. Точка $x_0 \in X$ называется
неподвижной точкой отображения $f$.
\end{thn}

\begin{proof}
Рассмотрим произвольное $x_1 \in X$ и образуем точки $x_2 = f(x_1),
\; x_3=f(x_2), \ldots, x_{n+1} = f(x_n), \; n\in \N$.

Покажем, что $(x_n)$ --- фундаментальная. Обозначим
$\rho_n=\rho(x_n,x_{n+1}), \; \rho_1=\rho(x_1,x_2)$. Тогда $\rho_n =
\rho(x_n,x_{n+1}) = \rho( f(x_n-1),f(x_n)) \le q \rho(x_{n-1},x_n) =
q\rho_{n-1}, \; n\in \N, \; n\ge2.$ Таким образом, $\rho_n \le
q^{n-1} \rho_1, \; n\ge2.$

Используя неравенство треугольника для метрики $\rho$, получим для
всех $n,m\in \N$ оценки \ml{\rho(x_n,x_{n+m}) \le \rho(x_n,x_{n+1})
+ \rho(x_{n+1},x_{n+2}) \spl \rho(x_{n+m-1},x_{n+m}) = \rho_n
+ \rho_{n+1} \spl \rho_{n+m-1} \le \\ \le \hr{q^{n-1} +
q^{n-2} \spl q^{n+m-2}}\rho_1 = q^{n-1}
(1+q+\ldots+\rho_{n+m-1})\rho_1 < q^{n-1} \frac1{1-q} \rho_1, \;
n\ge2.} Так как $\liml{n\ra\bes} q^{n-1}=0$, то для любого $\ep>0$
существует $N\in\N, \; N=N(\ep)$, что $q^{n-1} \frac1{1-q}\rho_1 <
\ep$ для всех $n \ge N = N(\ep)$ и, следовательно,
$\rho(x_n,x_{n+m}) < \ep$ для всех $n \ge N=N(\ep)$ и всех $m\in\N$,
то есть $(x_n)$ --- фундаментальная в $X$. Так как $X$ --- полное
пространство, то последовательность $(x_n)$ имеет предел (существует
$\liml{n\ra\bes}x_n=x_0$ и $x_0\in X$).

Так как $f$ непрерывна в $x_0\in X$, то $\liml{n\ra\bes}
f(x_n)=f(x_0)$.

Так как $\liml{n\ra\bes} x_{n+1} = \liml{n\ra\bes} x_n=x_0$, то
переходя в $x_{n+1}=f(x_n), \; n\in\N$ к пределу по $n\ra\bes$,
получим $x_0=f(x_0)$, то есть $x_0$ --- неподвижная точка
отображения $f$.


Если $y=f(y)$ для некоторого $y\in X, \; y\ne x_0$, то $\rho(y,x_0)
= \rho(f(y),f(x_0)) \le q \rho(y,x_0)$ и $\rho(y,x_0)>0$, что
невозможно, так как $0<q<1$.
\end{proof}

\subsubsection{Усиленная форма теоремы Банаха}
\begin{thh}
Если отображение $f\colon X\ra X$ полного метрического пространства
$X$ для некоторого $n\in N$ порождает отображение $f^n =
\ub{f\circ f\circ \ldots \circ f}_{\mbox{n раз}}$, сжимающее
в $X$, то $f$ имеет в $X$ единственную неподвижную точку.
\end{thh}

\begin{proof}
Согласно теореме Банаха, существует единственная точка $x_0 \in X$,
в которой $f^n(x_0) = x_0$. Тогда $f^n(f(x_0)) = f(x_0)$, то есть
точка $f(x_0)$ --- неподвижная точка отображения $f^n$.

Но $f^n$ имеет единственную неподвижную точку $x_0$ и,
следовательно, $f(x_0)=x_0$, то есть $x_0$ --- неподвижная точка
отображения $f$.

Если $y=f(y)$ для некоторого $y\in X$, то $f^n(y) = f^{n-1}(f(y)) =
f^{n-1}(y) = f^{n-2}(f(y)) = f^{n-3}(y) = \ldots = f(y) = y$, то
есть $y$ --- неподвижная точка отображения $f^n$.

Но $f^n$ имеет единственную неподвижную точку $x_0$, то есть
$x_0=y$.
\end{proof}


\end{document}
