\documentclass[draft]{article}
 \usepackage[utf8]{inputenc}
\usepackage[russian]{babel}
%\usepackage{amsmath}
\usepackage{dmvn}

\tocsubsectionparam{4em}

\begin{document}

%------------------------------------------------------------------------------------
% Макроопределения
%------------------------------------------------------------------------------------

% разное



\def\abs#1{\left|#1\right|}
\def\reals{\mathbb{R}}
\def\wholes{\mathbb{Z}}
\def\complex{\mathbb{C}}
\def\ph{\varphi}
\def\mes{\mu^* }

%\def\lq{<}
%\def\rq{>}

\def\re{\mathrm{Re}}
\def\im{\mathrm{Im}}
\def\inat{i\in\N}
\def\inatn{i \in \N}
\def\islip{\in\mathrm{Lip}}
\def\isvb{\in\mathrm{VB}}
\def\isc{\in\mathrm{C}}
\def\J{\mathcal{J}}
\def\O#1{\Omega_{#1}}

%\def\sectionname#1#2{\bigskip\bigskip\hrule\bigskip\bigskip{\large\bf \hfil§#1. #2\hfil}\bigskip\bigskip\hrule\bigskip\bigskip}
\def\sectionname#1#2{\bigskip\bigskip\hrule\bigskip\section{#2}\bigskip\hrule\bigskip\bigskip}
\def\lead{\leaders\hbox to 1em{\hss.\hss}\hfill\hfil}

% отрезки и~интервалы

\def\seg[#1,#2]{\left[#1, #2\right]}
\def\segab{\seg[a,b]}
\def\ab{$\segab$}
\def\li{\ell_{ i}}
\def\si{\overline\ell_{ i}}
\def\lj{\ell_{ j}}
\def\l(#1,#2){\ell_{ #1}^{#2}}
\def\lij{\l(j,i)}

% эф от икс

\def\at(#1){ \left(#1\right)}
\def\atx{\at(x)}
\def\set#1{\left\{#1\right\}}

%  эпсилон - дельта

\def\de{\delta}
\def\dexs{{\de\,\at( x )}}
\def\dex{\de\atx}
\def\ep{\varepsilon}
\def\degz{\de > 0}
\def\dexgz{\dex > 0}
\def\epgz{\ep > 0}
\def\dx{\Delta x}

% те состоит из пар дельта итое, кси итое

\def\pair(#1;#2){\left({#1};{#2}\right)}
\def\T{\mathbb{T}}
\def\Tt{\widetilde\T}

\def\dei{\Delta_i}
\def\ldei{\abs\dei}
\def\gdei{g\at(\dei)}
\def\xii{\xi_i}
\def\dixi{\pair(\dei;\xii)}
\def\dixif{\set{\dixi}}
\def\Tdixif{$\T = \dixif$}

% те штрих состоит из пар дельта штрих житое, кси штрих житое

\def\dej{\Delta'_j}
\def\ldej{\abs\dej}
\def\gdej{g\at(\dej)}
\def\xij{\xi'_j}
\def\djxj{\pair(\dej;\xij)}
\def\djxjf{\set{\djxj}}
\def\Tdjxjf{$\T' = \djxjf$}

% интегральные суммы, интегралы, вариации и~проч.

\def\isum(#1,#2){\sigma \left(#1,#2\right)}
\def\isfT{\isum(f,\T)}
\def\isfdgT{\isum(fdg,\T)}
\def\isfTp{\isum(f,\T')}
\def\isfTt{\isum(f,\Tt)}
\def\isgT{\isum(g,\T)}
\def\isgTp{\isum(g,\T')}
\def\isfdgTp{\isum(fdg,\T')}

\def\sumi{\sum\limits_i}
\def\sumj{\sum\limits_j}
\def\sumk{\sum\limits_k}

\def\fxildi{f\at(\xii)\ldei}
\def\fxigdi{f\at(\xii)\gdei}
\def\fxjldj{f\at(\xij)\ldej}
\def\fxjgdj{f\at(\xij)\gdej}

\def\intd[#1,#2]#3,#4{\int\limits_{#1}^{#2} #3\,d#4}
\def\intab#1,#2{\intd[a,b]#1,#2}
\def\intabfdx{\intab f,x}
\def\intabgdx{\intab g,x}
\def\intabfdg{\intab f,g}
\def\intdei#1,#2{ \int\limits_{\dei} #1\,d#2}
\def\intdeifdx{\intdei f,x}

\def\osc{\mathop{\mathrm{osc}}\limits}
\def\var{\mathop{\mathrm{Var}}\limits}

% в~одном из трех смыслов

\def\base{\mathfrak{B}}

\def\rimanletter{\Rc}
\def\baseriman{\base^\rimanletter}
\def\riman{\,\at(\rimanletter) }
\def\isriman[#1,#2]{ \in \rimanletter \seg[#1,#2]}

\def\macshletter{\Mc}
\def\basemacsh{\base^\macshletter}
\def\macsh{\,\at(\macshletter) }
\def\ismacsh[#1,#2]{ \in \macshletter \seg[#1,#2]}

\def\kurzhletter{\Hc}
\def\basekurzh{\base^\kurzhletter}
\def\kurzh{\,\at(\kurzhletter) }
\def\iskurzh[#1,#2]{ \in \kurzhletter \seg[#1,#2]}

\def\stletter{\Sc}
\def\rimanst{\,\at(\rimanletter-\stletter) }
\def\macshst{\,\at(\macshletter-\stletter) }
\def\kurzhst{\,\at(\kurzhletter-\stletter) }

% определение, аксиома, теорема

\def\df{\smallskip{\bf Определение}. }
\def\dfn#1{\smallskip{\bf Определение #1}. }
\def\pr{\smallskip{\bf Доказательство}. }
\def\prs{\smallskip{\bf Доказательства}. }
\def\rem{\smallskip{\bf Замечание}. }
\def\rems{\smallskip{\bf Замечания}. }
\def\lm{\smallskip{\bf Лемма}. }
\def\lmn#1{\smallskip{\bf Лемма #1}. }
\def\lmd#1{\smallskip{\bf Лемма}. {\sl(#1)}}
\def\lmnd#1#2{\smallskip{\bf Лемма #1}. {\sl(#2)}}
\def\prlm{\smallskip{\bf Лемма доказана}. }
\def\ut{\smallskip{\bf Утверждение}. }
\def\uts{\smallskip{\bf Утверждения}. }
\def\utn#1#2{\smallskip{\bf Утверждение #1}. {\sl(#2)}}
\def\prut{\smallskip{\bf Утверждение доказано}. }
\def\pruts{\smallskip{\bf Утверждения доказаны}. }
\def\imp{\smallskip{\bf Следствие}. }
\def\impn#1{\smallskip{\bf Следствие #1}. }
\def\imps{\smallskip{\bf Следствия}. }
\def\impd#1{\smallskip{\bf Следствие}. {\sl(#1)}}
\def\primp{\smallskip{\bf Следствие доказано}. }
\def\primps{\smallskip{\bf Следствия доказаны}. }
\def\tm#1#2{\smallskip{\bf Теорема #1}. {\sl(#2)}}
\def\prtm{\smallskip{\bf Теорема доказана}. }
\def\prtms{\smallskip{\bf Теоремы доказаны}. }
\def\eqdf{\smallskip{\bf Докажем эквивалентность определений}. }
\def\preqdf{\smallskip{\bf Эквивалентность определений доказана}. }
\def\ex{\smallskip{\bf Пример}. }

%------------------------------------------------------------------------------------
% НАЧАЛО ТЕКСТА
%------------------------------------------------------------------------------------

\tolerance=1600

%------------------------------------------------------------------------------------

\subsection{Предисловие.}

\bigskip

Вот уже который год Тарас Павлович Лукашенко продолжает радовать
мехматянских первокурсников интегралами Мак-Шейна и~Курцвейля --
Хенстока. Этот текст призван помочь в~восстановлении и~расшифровке
конспектов по первой половине второго семестра, в~которой излагаются
основные положения теории определенного интеграла. С~другой
стороны, вполне может оказаться и~так, что текст этот увидят обычные
любители математики, не связанные с~Тарасом Палычем, и~им тоже,
может быть, удастся что-нибудь понять. Для них поясню: интеграл
Мак-Шейна --- это такое определение интеграла Лебега, достаточно
узкое, но для его введения не требуется понятие меры множества,
которое так в~курсе и~не появляется. Интеграл Курцвейля -- Хенстока
является более простым определением интеграла Данжуа, поэтому этот
математический мутант, с~одной стороны, на действительной прямой не
слабее интеграла Лебега, а~с~другой --- интегрирует в~собственном
смысле все, что можно проинтегрировать в~любом из трех несобственных
смыслов на конечных отрезках (теорема Хейка). Впечатляет и
формулировка формулы Ньютона-Лейбница.

\smallskip

Этот конспект составлен по лекциям, читавшемся весной 2006 года.
Предыдущие известные конспекты датируются, вроде бы, 2003 годом, и~с
тех пор курс немного перестроился: например, во второй семестр
переехали теорема Хейка и~неравенство Чебышева, поменялись
обозначения. Логично ожидать, что и~при следующем приходе Тараса
Палыча в~аудиторию 16-24 курс перестроится, и~конспект этот снова
устареет. Но, к~сожалению, заранее писать конспекты не получается.

\smallskip

Была мысль сделать этот текст возможно более подробным, потому что
доказательства все непростые и~нетривиальные. С~другой стороны, это
может быть не вполне удобным, если китайский язык нужно сдавать
завтра. И, разумеется, никто не спасет нас почти всюду от ошибок,
глюков и~очепяток, поэтому в~случае чего пишите замечания и
предложения туда, откуда вы это скачали (это такая защита копирайта
спамом).

\medskip

Наш одноэлементный наборщицкий коллектив желает успехов в~изучении
матана!

\medskip
\dmvntrail

\bigskip



\subsection{Обозначения.}

\bigskip

Если $X$ --- множество, $E\subseteq X$, то $\chi_E\atx$ ---
характеристическая функция множества $E$, определенная на $X$ и
равная единице при $x\in E$ и~нулю при $x\in X\setminus E$.

\medskip

Если $f$ --- отображение из множества $X$ в~множество $Y$ и
$E\subseteq X$, то $f\at(E)=\set{f\atx: x\in E}\subseteq Y$ ---
образ множества $E$ при отображении $f$. Не путайте это обозначение
со следующим: если $I=\segab$ --- отрезок $\R$, то
$F\at(I)=F\at(b)-F\at(a)$ --- приращение $F$ на $I$.

\medskip

Написание всяческих рукописных буковок местами отличается от
оригинала. Например, интегралы для единообразия обозначены буквами
$\rimanletter$, $\macshletter$ и~$\kurzhletter$, хотя правдоподобнее
выглядят варианты $\mathscr{R}$, $\mathfrak{M}$ и~$\mathscr{H}$.
Знаменитую Лукашенковскую букву $\base$, быть может, точнее было бы
изобразить как $\mathfrak{P}$.

\medskip

$\delta$-окрестность точки $\xi$ обозначается $B_{\delta}\at(\xi)$.
Проколотая $\delta$-окрестность $\xi$ --- это
$B'_{\delta}\at(\xi)=B_{\delta}\at(\xi)\setminus\set{\xi}$.

\medskip

Длина отрезка $I=\segab$ обозначается $\abs{I}=b-a$.

\medskip

Остальные обозначения будем вводить по мере их появления.

\eject

\tableofcontents

\eject


%------------------------------------------------------------------------------------

\sectionname1{Определение интегрирования. }

\subsection{Основные определения.}

\df Отрезки $I$ и~$J$ называются неперекрывающимися, если их
пересечение $I \cap J$ пусто или состоит из одной точки, являющейся
концом каждого из них.

\df Система неперекрывающихся отрезков --- это такая система
отрезков, любые два отрезка которой неперекрывающиеся.

\df Разбиением $T$ отрезка \ab\ называют конечную систему
неперекрывающихся отрезков, объединение которых равно \ab.

\df Любая конечная система неперекрывающихся отрезков, лежащих в
\ab, называется подразбиением отрезка \ab.

\df Отмеченным разбиением (или разбиением Мак-Шейна) $\T$ отрезка
\ab\ называют конечный набор пар $\dixi$, где отрезки
$\set{\Delta_i}$ образуют разбиение отрезка \ab, а~точки $\xi_i$
лежат на отрезке \ab. При этом точки $\xii$ называются отмеченными
точками, соответствующими отрезкам разбиения $\dei$.

\df Разбиение Мак-Шейна \Tdixif\ называется разбиением Хенстока,
если все отмеченные точки лежат в~своих отрезках, то есть для всех
пар $\dixi$ выполняется условие $\xii\in\dei$.

\df Пусть дано число $\degz$. Будем говорить, что разбиение \Tdixif\
отрезка \ab\ мельче $\de$, если длина любого отрезка $\dei$ строго
меньше $\de$, или, что то же самое, $\max \set{\abs\dei}<\de$.

\df Масштабом на множестве $E$ называется любая строго положительная
функция на $E$.

\df Пусть $\dex$ --- масштаб на \ab. Тогда будем говорить, что
разбиение \Tdixif\ согласовано с~масштабом $\dex$, если во всех
парах $\dixi$ имеет место включение $\dei\subset
B_{\de\,\at(\xi_{ i} )}\at(\xii)$.

\bigskip

Пусть теперь действительнозначная или комплекснозначная функция $f$
определена на отрезке \ab.

\bigskip

\df Интегральной суммой от $f$, соответствующей отмеченному
разбиению \Tdixif\ отрезка \ab, называется выражение
$$\isfT=\sumi\fxildi.$$

\df Функция $f$ интегрируема на \ab\ в~смысле Римана и~$I$ --- ее
интеграл Римана, если для любого $\epgz$ найдется такое число
$\degz$, что для любого разбиения \Tdixif\ Хенстока отрезка \ab\
мельче $\de$ выполняется оценка $\left|\isfT-I\right|<\ep$.

В этом случае пишут $f\isriman[a,b]$ и~$$\riman\intabfdx=I.$$

\df Функция $f$ интегрируема на \ab\ в~смысле Мак-Шейна и~$I$ --- ее
интеграл Мак-Шейна, если для любого $\epgz$ найдется такой масштаб
$\dex$ на \ab, что для любого разбиения \Tdixif\ Мак-Шейна отрезка
\ab, согласованного с~$\dex$, выполняется оценка
$\left|\isfT-I\right|<\ep$.

В этом случае пишут $f\ismacsh[a,b]$ и~$$\macsh\intabfdx=I.$$

\df Функция $f$ интегрируема на \ab\ в~смысле Курцвейля --- Хенстока
и $I$ --- ее интеграл Курцвейля -- Хенстока, если для любого $\epgz$
найдется такой масштаб $\dex$ на \ab, что для любого разбиения
\Tdixif\ Хенстока отрезка \ab, согласованного с~$\dex$, выполняется
оценка $\left|\isfT-I\right|<\ep$.

В этом случае пишут $f\iskurzh[a,b]$ и~$$\kurzh\intabfdx=I.$$

\eject

\subsection{Базы Римана, Мак-Шейна и Курцвейля -- Хенстока.}

База Римана $\baseriman$ на отрезке \ab\ состоит из множеств
$B_\de$, каждое из которых есть множество всевозможных разбиений
\Tdixif\ Хенстока отрезка \ab\ мельче $\de$.

База Мак-Шейна $\basemacsh$ на отрезке \ab\ состоит из множеств
$B_\dexs$, каждое из которых есть множество всевозможных разбиений
\Tdixif\ Мак-Шейна отрезка \ab, согласованных с~$\dex$.

База Курцвейля -- Хенстока $\basekurzh$ на отрезке \ab\ состоит из
множеств $B_\dexs$, каждое из которых есть множество всевозможных
разбиений \Tdixif\ Хенстока отрезка \ab, согласованных с~$\dex$.

\bigskip

Несложно доказать, что множества $\baseriman$, $\basemacsh$ и
$\basekurzh$ действительно являются базами, то есть выполняются
следующие свойства базы:

\bigskip {\sl

1. \ Ни один элемент базы не является пустым множеством;

2. \ В~пересечении любых двух элементов базы содержится третий.

} \bigskip

Второе свойство проверяется элементарно: в~пересечении элементов
базы, заданных мелкостями $\de_1$ и~$\de_2$ (масштабами $\de_1\atx$
и $\de_2\atx$), содержится элемент, заданный мелкостью
$\min \set{\de_1,\de_2 }$ (соответственно, масштабом
$\min \set{\de_1\atx,\de_2\atx}$). Для интеграла Римана первое
свойство также очевидно: найдем по аксиоме Архимеда такое
натуральное число $n$, что ${b-a\over n}<\de$ , и~разобьем \ab\ на
$n$ равных отрезков, выбирая произвольно отмеченные точки в~них;
полученное разбиение будет лежать в~$B_\de\in\baseriman$, чем и
показано, что $B_\de$ не пусто. Поскольку во всяком элементе базы
Мак-Шейна содержится элемент базы Курцвейля -- Хенстока, заданный
тем же масштабом, осталось только доказать следующее утверждение:

\bigskip

\lmd{о существовании разбиений}

Для любого масштаба $\dexgz$, заданного на отрезке \ab, найдется
согласованное с~ним разбиение Хенстока отрезка \ab.

\pr

Пусть для некоторого масштаба $\dex$ не существует согласованного с
ним разбиения отрезка $\segab=I_1$.

\smallskip

Разделим отрезок $I_1$ пополам, и~пусть $I_2$ --- та его половина,
для которой не существует разбиения, согласованного с~$\dex$.
Действительно, если бы для каждой половины существовало отмеченное
разбиение, согласованное с~$\dex$, то объединение разбиений
половинок было бы разбиением всего отрезка $I_1$, согласованным с
$\dex$, что противоречит нашему предположению.

\smallskip

Разделим отрезок $I_2$ пополам, и~пусть $I_3$ --- та его половина,
для которой не существует разбиения, согласованного с~$\dex$.
Действительно, если бы для каждой половины существовало отмеченное
разбиение, согласованное с~$\dex$, то объединение разбиений
половинок было бы разбиением всего отрезка $I_2$, согласованным с
$\dex$, что противоречит нашему предположению.

\smallskip

Разделим отрезок $I_3$ пополам, и~пусть $I_4$ --- та его половина,
для которой не существует разбиения, согласованного с~$\dex$ \dots

\smallskip

Продолжая этот процесс, получим последовательность вложенных
отрезков
$$I_1\subset I_2\subset I_3\subset I_4\subset\cdots\,,$$
причем длины отрезков стремятся к~нулю. Пусть $\xi$ --- общая точка
всех отрезков. Поскольку длины отрезков стремятся к~нулю, найдется
натуральное число $n$, такое, что $\left|I_n\right|<\de\at(\xi)$. Но
тогда $\set{\left(I_n,\xi \right)}$ --- отмеченное разбиение отрезка
$I_n$, согласованное с~масштабом $\dex$, что противоречит выбору
$I_n$.

\prlm

\bigskip

Таким образом, сравнивая определения интегрирования и~предела по
базе, можем переправить определение интегрируемости:

$$\riman\intabfdx=\lim_{\baseriman}\isfT,$$
$$\macsh\intabfdx=\lim_{\basemacsh}\isfT,$$
$$\kurzh\intabfdx=\lim_{\basekurzh}\isfT.$$

\eject

\subsection{Простейшие свойства интегралов.}

Из этих утверждений сразу следуют некоторые простейшие свойства
интегралов Римана, Мак-Шейна и~Курцвейля -- Хенстока.

\bigskip

\utn1{о взаимосвязи интегралов}

Если функция $f\isriman[a,b]$ или $f\ismacsh[a,b]$, то
$f\iskurzh[a,b]$ и~значения интегралов совпадают.

\pr

{\it Для интеграла Римана:} каждый элемент базы Римана $B_\de$
 содержит внутри себя элемент базы Курцвейля~-- Хенстока, задаваемый масштабом
 $\dex\equiv{\de/2}$. Таким образом, если найдется число $\de$, для которого оценка
выполнена, то найдется и~масштаб.

{\it Для интеграла Мак-Шейна:} всякое разбиение Хенстока является
разбиением Мак-Шейна, поэтому если оценка выполняется для всех
разбиений Мак-Шейна, то, в~частности, и~для разбиений Хенстока.

\prut

\bigskip

{\small\rems

\smallskip

При доказательстве этого утверждения можно ссылаться на
соответствующие теоремы о~пределах по базе.

\smallskip

Всегда полезно напомнить, что в~этом курсе не бывает \lq\lq
разбиений Римана\rq\rq\ или \lq\lq разбиений Курцвейля --
Хенстока\rq\rq, а~бывают только разбиения Мак-Шейна и~их
разновидность --- разбиения Хенстока.

\smallskip

Утверждение о~том, что из интегрируемости по Риману следует
интегрируемость по Мак-Шейну, будет доказано позднее, как следствие
из критерия Лебега.

\smallskip

В дальнейшем будем считать, что все на свете интегралы понимаются
всего в~трех смыслах, и~все они перечислены выше. Правда, потом
появятся еще три интеграла Стилтьеса и~три несобственных интеграла,
и тогда надо будет соображать, о~каких трех смыслах идет речь.}

\bigskip

\utn2{о линейности по функции}

Если функция $f$ интегрируема на отрезке \ab\ в~каком-либо смысле,
то для любого числа $\lambda$ функция $\lambda f$ тоже интегрируема
на \ab\ в~том же смысле, и~выполняется равенство
$$\intd[a,b]\lambda f,x=\lambda\cdot\intabfdx.$$

Если две функции $f$ и~$g$ интегрируемы на \ab\ в~одном и~том же
смысле, то функция $f+g$ тоже интегрируема на \ab\ в~том же самом
смысле, и~выполняется равенство
$$\intd[a,b](f+g),x=\intabfdx+\intabgdx.$$

\pr

Очевидно, что для любого разбиения $\T$ отрезка \ab\ выполняются
равенства
$$\isum(\lambda f,\T)=\lambda\cdot\isfT,$$
$$\isum(f+g,\T)=\isfT+\isgT.$$

Теперь для доказательства утверждения достаточно сослаться на
линейность по функции предела по базе. То есть интегральная сумма
линейна по функции, а~предел по базе линеен по интегральной сумме.

\prut

\bigskip

\utn3{о сохранении неравенств}

Если $f$ и~$g$ интегрируемы на \ab, возможно, в~разных смыслах, и
$f(x)\le g(x)$ для всех $x\in\segab$, то
$$\intabfdx\le\intabgdx.$$

\pr

Поскольку из интегрируемости в~любом смысле следует интегрируемость
по Курцвейлю -- Хенстоку с~тем же значением интеграла, можно
считать, что оба интеграла понимаются в~смысле Курцвейля --
Хенстока. Очевидно, для любого разбиения $\T$ Хенстока отрезка \ab\
выполняется неравенство $\isfT\le\isgT$. Таким образом, утверждение
следует из сохранения неравенства при переходе к~пределу по базе
Курцвейля -- Хенстока.

\prut

\eject

\utn4{критерий Коши интегрируемости}

\smallskip

Функция $f$ интегрируема на отрезке \ab\ в~смысле Римана тогда и
только тогда, когда f определена на \ab\ и~для любого $\epgz$
найдется такое число $\degz$, что для любых двух разбиений $\T$ и
$\T'$ Хенстока отрезка \ab\ мельче $\de$ выполняется оценка
$\abs{\isfT-\isfTp}<\ep$.

\smallskip

Функция $f$ интегрируема на отрезке \ab\ в~смысле Мак-Шейна тогда и
только тогда, когда $f$ определена на \ab\ и~для любого $\epgz$
найдется такой масштаб $\dex$ на \ab, что для любых двух разбиений
$\T$ и~$\T'$ Мак-Шейна отрезка \ab, согласованных с~$\dex$,
выполняется оценка $\abs{\isfT-\isfTp}<\ep$.

\smallskip

Функция $f$ интегрируема на отрезке \ab\ в~смысле Курцвейля --
Хенстока тогда и~только тогда, когда $f$ определена на \ab\ и~для
любого $\epgz$ найдется такой масштаб $\dex$ на \ab, что для любых
двух разбиений $\T$ и~$\T'$ Хенстока отрезка \ab, согласованных с
$\dex$, выполняется оценка $\abs{\isfT-\isfTp}<\ep$.

\smallskip

\pr

Утверждение представляет собой переформулировку критерия Коши
существования предела по базе. Так что в~качестве доказательства
напомним формулировку: функция $h$ имеет предел по базе $\base$
тогда и~только тогда, когда $h$ определена на некотором элементе
базы и
$$\forall\epgz\ \exists B\in\base:\ \forall t_1\in B,\ \forall t_2\in
B\quad \abs{h\at(t_1)-h\at(t_2)}<\ep.$$

\prut

\bigskip

%------------------------------------------------------------------------------------

\sectionname2{Дальнейшие свойства интегралов.}

\subsection{Интегрируемость на подотрезках.}

\tm1{об интегрируемости на подотрезках}

Если функция $f$ интегрируема на отрезке \ab\ в~каком-либо смысле,
то $f$ интегрируема и~на любом подотрезке
$\seg[a',b']\subseteq\segab$ в~том же смысле.

\pr

Поскольку о~значении интегралов мы ничего не утверждаем, можно
воспользоваться критерием Коши. Возьмем любое $\epgz$ и~найдем такое
число $\degz$ (а в~случае интегрирования в~смысле Мак-Шейна или
Курцвейля~-- Хенстока --- масштаб $\dex$ на \ab), чтобы для любых
отмеченных разбиений $\T$ и~$\T'$ Хенстока (Мак-Шейна, Хенстока)
отрезка \ab\ выполнялась оценка $\abs{\isfT-\isfTp}<\ep$.

\smallskip

Пусть $\Tt$ и~$\Tt'$ --- разбиения отрезка $\seg[a',b']$ Хенстока
(Мак-Шейна, Хенстока) мельче $\de$ (согласованные с~$\dex$).

\smallskip

Если $a\neq a'$, то пусть $\T_a$ --- разбиение Хенстока (Мак-Шейна,
Хенстока) отрезка $\seg[a,a']$ мельче $\de$ (согласованное с
$\dex$), в~противном случае $\T_a=\varnothing$ .

\smallskip

Если $b\neq b'$, то пусть $\T_b$ --- разбиение Хенстока (Мак-Шейна,
Хенстока) отрезка $\seg[b',b]$ мельче $\de$ (согласованное с
$\dex$), в~противном случае $\T_b=\varnothing$ .

\smallskip

Положим
$$\T=\Tt\sqcup\T_a\sqcup\T_b,$$
$$\T'=\Tt'\sqcup\T_a\sqcup\T_b.$$

Тогда, очевидно,
$$\isum(f,\T)=\isum(f,\Tt)+\isum(f,\T_a)+\isum(f,\T_b),$$
$$\isum(f,\T')=\isum(f,\Tt')+\isum(f,\T_a)+\isum(f,\T_b).$$

$\T$ и~$\T'$ --- тоже разбиения Хенстока (Мак-Шейна, Хенстока)
отрезка \ab\ мельче $\de$ (согласованные с~$\dex$), поэтому
доказательство завершается просто и~элегантно:
$$\abs{\isum(f,\Tt)-\isum(f,\Tt')}=\abs{\isum(f,\T)-\isum(f,\T')}<\ep.$$

Все $\isum(f,\T_a)$ и~$\isum(f,\T_b)$ сократились. Мы показали, что
для любых двух разбиений $\Tt$ и~$\Tt'$ Хенстока (Мак-Шейна,
Хенстока) отрезка $\seg[a',b']$ мельче $\de$ (согласованных с
$\dex$) выполняется оценка $\abs{\isum(f,\Tt)-\isum(f,\Tt')}<\ep$.
Таким образом, выполнен критерий Коши.

\prtm

\eject

\subsection{Аддитивность по отрезку.}

\lmd{об ограниченности интегрируемых по Риману функций}

Если функция $f$ интегрируема на отрезке \ab\ в~смысле Римана, то
$f$ ограничена на \ab.

\pr

Если некоторая функция имеет предел по некоторой базе, то она
ограничена на некотором элементе этой базы. В~данном случае роль
функции играет интегральная сумма, поэтому требуется показать, что
какое бы число $\degz$ мы ни выбрали, всегда найдется разбиение
Хенстока отрезка \ab\ мельче $\de$, интегральная сумма от которого
по модулю больше любого наперед заданного числа.

\smallskip

Действительно, пусть $T=\set{\dei}$ --- разбиение \ab, причем
$\max \set{\abs\dei}<\de$. Тогда непременно найдется такой отрезок
$\Delta_j$, на котором функция $f$ неограничена. Зафиксируем
какие-нибудь точки $\xii\in\dei$ при $i\neq j$, и~представим
интегральную сумму в~виде
$$f\at(\xi_j)\abs{\Delta_j}+\sum\limits_{i\neq j}\fxildi,$$
где точку $\xi_j\in\Delta_j$ мы еще не выбрали. Теперь, поскольку
функция $f$ неограничена на \ab, а~$\abs{\Delta_j}\neq0$ по
определению, интегральная сумма оказывается неограниченной даже как
функция от одной точки $\xi_j\in\Delta_j$, что и~приводит нас к
противоречию с~интегрируемостью функции $f$ по Риману.

\prlm

\bigskip

{\small \rems

\smallskip

1. Вместо свойства ограниченности можно ссылаться на более сильное
свойство --- Критерий Коши: мы нашли два разбиения мельче $\de$,
которые отличаются на сколь угодно большую величину.

\smallskip

2. В~доказательстве существенно использовался тот факт, что
интегрируемость понимается в~смысле Римана, а~именно, в~тот момент,
когда мы произвольно выбирали точку $\xi_j\in\Delta_j$. С~другими
интегралами такой прием не проходит, поскольку, например, масштаб
может быть выбран так, что есть всего одна точка, которая может быть
отмечена для этого отрезка (в остальных точках масштаб может быть
таким, что отрезок просто не уместится в~окрестность). При этом
можно привести пример функции, интегрируемой по Мак-Шейну (и,
следовательно, по Курцвейлю -- Хенстоку), но неограниченной на любом
интервале.

\smallskip

3. Пример функции Дирихле $\mathfrak{D}\atx=\chi_\Q\atx$
показывает, что ограниченность не является достаточным условием
интегрируемости по Риману. Можно привести примеры ограниченных
функций, неинтегрируемых даже по Курцвейлю~-- Хенстоку, но для этого
потребуются более продвинутые свойства интегралов и, вообще говоря,
аксиома выбора.}

\bigskip

\tm2{об аддитивности интегралов по отрезку}

Если функция $f$ интегрируема на отрезках $\seg[a,b]$ и~$\seg[b,c]$
в одном из трех смыслов, то $f$ интегрируема и~на отрезке
$\seg[a,c]$ в~том же смысле и~выполняется равенство
$$\intd[a,c]f,x=\intd[a,b]f,x+\intd[b,c]f,x.$$

\pr

Введем обозначения:
$$\intd[a,b]f,x=I_1\quad\hbox{и}\quad\intd[b,c]f,x=I_2.$$

Будем доказывать теорему отдельно для интеграла Римана и~отдельно
для интегралов Мак-Шейна и~Курцвейля -- Хенстока.

\medskip

{\it Для интеграла Римана:}

\smallskip

Возьмём любое $\epgz$ и~подберем такие положительные числа $\de_1$ и
$\de_2$, чтобы для любого разбиения $\T_1$ Хенстока отрезка
$\seg[a,b]$ мельче $\de_1$ выполнялась оценка
$\abs{\isum(f,\T_1)-I_1}<{\ep\over4}$ и~для любого разбиения $\T_2$
Хенстока отрезка $\seg[b,c]$ мельче $\de_2$ выполнялась оценка
$\abs{\isum(f,\T_2)-I_2}<{\ep\over4}$. Вспоминая только что
доказанную лемму, обнаруживаем, что функция $f$ ограничена на
$\seg[a,b]$ и~$\seg[b,c]$, и, следовательно, на $\seg[a,c]$. Поэтому
можем взять такое число $\degz$, что выполняются сразу три условия:
$$\de<\de_1,\qquad\de<\de_2,\eqno(1)$$
$$\de\cdot\sup_{\left[a ,c\right]}\abs{f}<{\ep\over4}.\eqno(2)$$

Пусть \Tdixif\ --- любое разбиение Хенстока отрезка $\seg[a,c]$
мельче $\de$. Рассмотрим два случая.

\eject

{\sl Первый случай:} Точка $b$ не является внутренней точкой ни
одного из отрезков $\dei$. В~этом случае разбиение $\T$ распадается
в объединение непересекающихся множеств: $\T=\T_1\sqcup\T_2$, где
$$\T_1=\set{\dixi\in\T:\dei\subseteq\seg[a,b]},$$
$$\T_2=\set{\dixi\in\T:\dei\subseteq\seg[b,c]}.$$

Здесь очевидно, что $$\isum(f,\T)=\isum(f,\T_1)+\isum(f,\T_2),$$
причем $\T_1$ и~$\T_2$ --- тоже разбиения Хенстока отрезков
$\seg[a,b]$ и~$\seg[b,c]$ мельче $\de$, и, следовательно, мельче
$\de_1$ и~$\de_2$. Поэтому можно, применяя оценку
$\abs{x+y}\le\abs{x}+\abs{y}$ и~условия $\left(1\right)$,
утверждать, что
$$\abs{\isum(f,\T)-\left(I_1+I_2\right)}\le\abs{\isum(f,\T_1)-I_1}+
\abs{\isum(f,\T_2)-I_2}<{\ep\over4}+{\ep\over4}={\ep\over2}.$$

{\sl Второй случай:} Точка $b$ является внутренней точкой отрезка
$\Delta_j$ разбиения $\T$. Пусть $\Delta_j$=$\seg[u,v]$. Тогда
заменим пару $\pair(\Delta_j;\xi_j)$ двумя парами
$\pair(\seg[u,b];b)$ и~$\pair(\seg[b,v];b)$. Получится новое
разбиение $\T'$ Хенстока отрезка $\seg[a,c]$ мельче $\de$. При этом
интегральная сумма изменится на величину
$$\abs{\isum(f,\T)-\isum(f,\T')}=
\abs{f\at(\xi_j)\at(v-u)-f\at(b)\at(b-u)-f\at(b)\at(v-b)}=
\abs{f\at(\xi_j)\at(v-u)-f\at(b)\at(v-u)}=$$
$$=\abs{f\at(\xi_j)-f\at(b)}\at(v-u)\le
2\sup_{\left[a ,c\right]}\abs{f}\cdot\at(v-u)\le
2\de\sup_{\left[a ,c\right]}\abs{f}<{\ep\over2}.$$

Но теперь мы добились того, что разбиение $\T'$ имеет вид,
рассмотренный в~первом случае, и~ссылаясь на проведенные выше
рассуждения, утверждаем, что
$\abs{\isum(f,\T')-(I_1+I_2)}<{\ep\over2}$. Осталось только
заметить, что $$\abs{\isum(f,\T)-(I_1+I_2)}\le
\abs{\isum(f,\T)-\isum(f,\T')}+\abs{\isum(f,\T')-(I_1+I_2)}<
{\ep\over2}+{\ep\over2}=\ep.$$

Таким образом, заканчивая рассмотрение {\sl случаев}, мы доказали,
что для любого $\epgz$ нашлось такое число $\degz$, что для любого
разбиения $\T$ Хенстока отрезка $\seg[a,c]$ мельче $\de$ выполняется
оценка $\abs{\isum(f,\T)-(I_1+I_2)}<\ep$, что и~требовалось
доказать.

\medskip

{\it Для интегралов Мак-Шейна и~Курцвейля -- Хенстока:}

\smallskip

Возьмём любое $\epgz$ и~подберем такие масштабы $\de_1\atx$ и
$\de_2\atx$ на $\seg[a,b]$ и~$\seg[b,c]$ соответственно, чтобы для
любого разбиения $\T_1$ Мак-Шейна (Хенстока) отрезка $\seg[a,b]$,
согласованного с~$\de_1\atx$, выполнялась оценка
$\abs{\isum(f,\T_1)-I_1}<{\ep\over2}$ и~для любого разбиения $\T_2$
Мак-Шейна (Хенстока) отрезка $\seg[b,c]$, согласованного с
$\de_2\atx$, выполнялась оценка
$\abs{\isum(f,\T_2)-I_2}<{\ep\over2}$. Теперь определим масштаб
$\dex$ на $\seg[a,c]$ следующим образом:
$$\dex=\begin{cases}\min \set{\de_1\atx,b-x},\quad x\in\left[a,b\right);\cr
\min \set{\de_1\atx,\de_2\atx},\quad x=b;\cr
\min \set{\de_2\atx,x-b},\quad
x\in\left(b,c\right].\cr\end{cases}$$

Этим мы добились того, что если, например,
$\xii\in\left[a,b\right)$, то непременно $\dei\subset
B_{\de\left(\xii\right)}  \left(\xii\right)\subset\left[a,b\right)$,
аналогично для $\left(b,c\right]$. То есть если какой-то отрезок
отмеченного разбиения, согласованного с~масштабом $\dex$, содержит
точку $b$, то она и~будет его отмеченной точкой. Рассмотрим любое
разбиение \Tdixif\ Мак-Шейна (Хенстока) отрезка $\seg[a,c]$,
согласованное с~масштабом $\dex$. Снова возникают два случая.

\medskip

{\sl Первый случай:} Точка $b$ не является внутренней точкой ни
одного из отрезков $\dei$. Этот случай разбирается аналогично
первому случаю для интеграла Римана. Разбиение $\T$ распадается в
объединение непересекающихся множеств: $\T=\T_1\sqcup\T_2$, где
$$\T_1=\set{\dixi\in\T:\dei\subseteq\seg[a,b]},$$
$$\T_2=\set{\dixi\in\T:\dei\subseteq\seg[b,c]}.$$

Разбиения $\T_1$ и~$\T_2$ и~теперь являются разбиениями отрезков
$\seg[a,b]$ и~$\seg[b,c]$ Мак-Шейна (Хенстока), согласованными с
масштабом $\dex$, и, следовательно, с~масштабами $\de_1\atx$ и
$\de_2\atx$. Маленькая тонкость: в~случае интеграла Мак-Шейна точки
по определению не обязаны лежать в~своих отрезках, и~поэтому,
например, отмеченные точки определенного таким образом разбиения
$\T_1$ могли бы выскочить за отрезок $\seg[a,b]$. Но мы специально
подобрали масштаб таким образом, чтобы этого быть не могло. Поэтому
можно, снова применяя оценку $\abs{x+y}\le\abs{x}+\abs{y}$, смело
утверждать, что
$$\abs{\isum(f,\T)-\left(I_1+I_2\right)}\le\abs{\isum(f,\T_1)-I_1}+
\abs{\isum(f,\T_2)-I_2}<{\ep\over2}+{\ep\over2}=\ep.$$

{\sl Второй случай:} Если точка $b$ является внутренней точкой,
например, отрезка $\Delta_j=\seg[u,v]$,  то, как мы уже заметили,
разбиение $\T$ содержит пару $\pair(\Delta_j;b)$. То есть точка $b$
является отмеченной точкой отрезка $\Delta_j$. Поэтому если мы
заменим пару $\pair(\Delta_j;b)$ двумя парами $\pair(\seg[u,b];b)$ и
$\pair(\seg[b,v];b)$, обозначив новое разбиение через $\T'$, то
интегральная сумма вообще не изменится:
$$f\at(b)\at(v-u)=f\at(b)\at(b-u)+f\at(b)\at(v-b).$$

А разбиение $\T'$ уже рассмотрено в~первом случае, поэтому
$$\abs{\isum(f,\T)-(I_1+I_2)}=
\abs{\isum(f,\T')-(I_1+I_2)}<\ep.$$ Таким образом, заканчивая
рассмотрение {\sl случаев}, мы доказали, что для любого $\epgz$
нашлось такой масштаб $\dexgz$ на $\seg[a,c]$, что для любого
разбиения $\T$ Мак-Шейна (Хенстока) отрезка $\seg[a,c]$,
согласованного с~$\dex$, выполняется оценка
$\abs{\isum(f,\T)-(I_1+I_2)}<\ep$, что и~требовалось доказать.

\medskip

\prtm

\subsection{Формула Ньютона -- Лейбница.}

\tm3{формула Ньютона -- Лейбница}

Пусть функция $f$ определена на отрезке \ab, функция $F$ непрерывна
на \ab, и~$F'\atx = f\atx$ всюду на \ab, кроме, быть может, не
более чем счетного множества точек (в точках которого функция $F$
либо не имеет конечной производной, либо имеет конечную производную,
отличную от $f\atx$). Тогда функция $f$ интегрируема на отрезке \ab\
по Курцвейлю -- Хенстоку и~выполняется равенство
$$\kurzh\intabfdx=F\at(b)-F\at(a)=F\at(\segab).$$

\pr

Возьмем любое число $\epgz$. Пусть $E$ --- упомянутое в~условии
\lq\lq исключительное\rq\rq\ множество точек, где не выполняется
равенство $F'\atx = f\atx$. Раз уж оно не более чем счетное,
занумеруем его натуральными числами: $E=\set{x_i}$, $\inat$.
Начинаем строить масштаб $\dex$ на \ab. В~точках множества $E$
выберем $\de\at(x_i)$ так, чтобы выполнялись следующие два условия:
$$F\at({B_{\de\,\at(x_i)}\at(x_i)\cap\segab})\subseteq
B_{\ep\over2^{i+3}}\at({F\at(x_i)}),\eqno(1)$$
$$\abs{f\at(x_i)}\cdot\de\at(x_i)<{\ep\over2^{i+2}}.\eqno(2)$$

Первое условие выполнимо, поскольку функция $F$ непрерывна на \ab\
по \ab\ (напомним, что через $\ph\at(X)$ обозначается образ
множества $X$ при отображении $\ph$). Если же $x\in E$, то выберем
$\dex$ так, чтобы для любого числа $\dx$, по модулю меньшего $\dex$,
выполнялась оценка
$$\abs{F\at(x+\dx)-F\atx-f\atx \cdot \dx}<
{\ep\cdot\abs{\dx}\over2\at(b-a)}.\eqno(3)$$

Это достижимо потому, что в~точке $x$ функция $F$ дифференцируема и
$F'\atx=f\atx$, откуда следует, что выражение под модулем есть
$o\at(\abs{\dx})$. Все, масштаб построен. Возьмем любое разбиение
\Tdixif\ Хенстока отрезка \ab, согласованное с~вновь построенным
масштабом $\dex$, и~начинаем оценивать интегральную сумму. Напомним,
что $F\at(\seg[u,v])$ означает $F\at(v)-F\at(u)$. Очевидно, что
$F\at(\segab)=\sum\limits_i F\at(\dei)$, поэтому
$$\abs{\isfT-F\at(\segab)}=\abs{\sumi\at({\fxildi-F\at(\dei)})}\le
\sum_{\xii\in E}\abs{\fxildi-F\at(\dei)}+
\sum_{\xii\in\reals\setminus E}\abs{\fxildi-F\at(\dei)}.$$

\smallskip

Оцениваем отдельно обе суммы.

\smallskip

{\sl Первая сумма}, где $\xii \in E$, при сделанных предположениях
оценивается очень грубо. A~именно, если $\xii = x_j$,
$\dei = \seg[u,v]$, то при помощи условий $\at(1)$ и~$\at(2)$
каждое слагаемое можно оценить так:
$$\abs{\fxildi-F\at(\dei)}\le\abs{F\at(v)-F\at(u)}+\abs\fxildi<
2\cdot{\ep\over2^{j+3}}+{\ep\over2^{j+2}}={\ep\over2^{j+1}}.$$

Здесь первый модуль оценивается первой оценкой (и $F\at(u)$, и
$F\at(v)$ лежат в~$\ep\over2^{i+3}$ --- окрестности одного и~того же
числа), а~второй модуль --- второй оценкой (поскольку
$\ldei<\de\at(x_j)$). Таким образом, вся первая сумма не превышает
суммы арифметической прогрессии
$\sumj{\ep\over2^{j+1}}<{\ep\over2}$.

\eject

{\sl Вторая сумма}: тоже считаем, что $\dei=\seg[u,v]$, и~при этом
нам понадобится тот факт, что $u\le\xii\le v$ (обратите внимание на
то, что для интеграла Мак-Шейна это предположение было бы неверным!)
Теперь можно каждое слагаемое оценить при помощи условия $\at(3)$:
$$\abs{\fxildi-F\at(\dei)}\le \abs{f\at(\xii)\at(\xii-u)-F\at(\seg[u,\xii])}+
\abs{f\at(\xii)\at(v-\xii)-F\at(\seg[\xii,v])}
<{\ep\at(\xii-u)\over2\at(b-a)}+{\ep\at(v-\xii)\over
2\at(b-a)}={\ep\ldei\over2\at(b-a)}.$$

И тогда всю вторую сумму можно оценить величиной
$$\sumi{\ep\ldei\over2\,\at(b-a)}={\ep\over2}.$$

В итоге получаем что
$\abs{\isfT-F\at(\segab)}\le{\ep\over2}+{\ep\over2}=\ep$ То есть для
любого $\epgz$ нашелся такой масштаб $\dex$ на \ab, что для любого
разбиения $\T$ Хенстока отрезка \ab, согласованного с~этим
масштабом, выполнена соответствующая оценка.

\prtm

\bigskip

\impn1

Если $f$ интегрируема на отрезке \ab\ в~любом из смыслов, $F$ ---
обобщенная первообразная $f$ на \ab, то
$$\intabfdx=F\at(b)-F\at(a).$$

В этом ограниченном смысле формула Ньютона -- Лейбница выполняется
для интегралов Римана и~Мак-Шейна. Для них из существования
первообразной не обязательно следует интегрируемость. Утверждение
очевидным образом следует из взаимосвязи трех наших интегралов.

\bigskip

\impn2

Если функции $F_1$ и~$F_2$ непрерывны на промежутке $I$, а~конечные
производные $F_1'$ и~$F_2'$ равны всюду на $I$, кроме не более чем
счетного множества точек (где хотя бы одна из производных не
существует или бесконечна, либо обе существуют и~конечны, но не
равны), то функция $F_1-F_2$ постоянна на $I$.

\pr

Напомним, что промежуток --- это такое подмножество $\reals$,
которое вместе с~любыми двумя точками $a$ и~$b$ содержит весь
отрезок \ab. Поскольку и~$F_1$, и~$F_2$ сойдут за $F$ в~формуле
Ньютона -- Лейбница для, скажем, $F_1'$, то для любых двух точек $a$
и $b$ промежутка $I$, $a>b$ (если промежуток --- пустой или
одноточечный, то утверждение очевидно), получаем выражения
$$\kurzh\intd[a,b]F_1',x=F_1\at(b)-F_1\at(a),$$
$$\kurzh\intd[a,b]F_1',x=F_2\at(b)-F_2\at(a).$$

Приравняв правые части и~перекинув пару слагаемых туда-сюда, видим,
что $F_1\at(x_2)-F_2\at(x_2)=F_1\at(x_1)-F_2\at(x_1)$, что и
требовалось увидеть.

\primp

\eject

%------------------------------------------------------------------------------------

\sectionname3{Внешняя мера Лебега и~критерий Лебега.}

\subsection{Определение и свойства внешней меры.}

\df Пусть дано множество $E\subseteq\reals$. Тогда верхняя (она же
внешняя) мера Лебега $\mes E$ множества $E$ есть точная нижняя грань
сумм длин не более чем счетных систем интервалов, покрывающих
множество $E$:
$$\mes E = \inf\limits_{E\subseteq \bigcup\limits_{\inat}
  \li}\sum\limits_{\inat}\abs\li.$$

\df Множество, внешняя мера Лебега которого равна нулю, называется
множеством меры нуль по Лебегу.

\bigskip

\uts

\smallskip

1. Внешняя мера Лебега  любого множества существует и~заключена в
промежутке $\seg[0,{+\infty}]$.

\smallskip

2. В~определении внешней меры Лебега можно разрешить бесконечные
интервалы.

\smallskip

3. Пустое и~одноточечное множество есть множества меры нуль по
Лебегу.

\prs

\smallskip

1. Само множество $\reals$ можно покрыть системой интервалов
$\set{\,\at(i-1,i+2)}, i ~ \in ~ \wholes$. Этой же системой
можно покрыть и~любое подмножество $E\subseteq\reals$. Таким
образом, множество систем интервалов, покрывающих $E$, не пусто, и,
значит, множество $G \subset \reals$ сумм длин интервалов таких
систем не пусто и~ограничено снизу нулем. Поэтому по принципу
полноты Вейерштрасса множество $G$ имеет точную нижнюю грань,
которая по определению и~будет внешней мерой Лебега множества $Е$.
Поскольку ноль является нижней гранью множества $G$, точная нижняя
грань G не может быть меньше нуля.

\smallskip

2. Ясно, что от введения бесконечных интервалов мера никакого
множества не уменьшится --- ведь, покрывая множество бесконечными
интервалами, мы никак не получим меру меньше, чем была
(бесконечность не меньше никакого числа). В~то же время мера не
может увеличится, потому что никто не заставляет использовать
бесконечные интервалы, если и~без них хорошо.

\smallskip

3. Пустое множество можно покрыть, например, пустой системой
интервалов, а~точку --- ее окрестностью радиуса меньше $\ep$.

\pruts

\bigskip

\ut

Если в~определении внешней меры Лебега заменить интервалы отрезками,
то получится эквивалентное определение.

\pr

Обозначим меру по отрезкам через $\mu^*_s$, а~меру по интервалам ---
через $\mu^*_i$. Пусть $E\subseteq\reals$. Требуется доказать две
вещи.

\medskip

1. $\mu^*_s E \le \mu^*_i E$. Действительно, пусть $\set{\li}$,
${\inat}$
--- любая система интервалов, покрывающих $E$. Тогда, добавив к
каждому интервалу его концы, получим систему отрезков $\set{\li'}$,
${\inat}$, покрывающих $E$. Таким образом, отрезки ничуть не хуже
интервалов.

\medskip

2. $\mu^*_i E \le \mu^*_s E$. Возьмем любое $\epgz$. Пусть
$\set{\overline\ell_{ i}}$, $\inat$ --- любая система отрезков,
покрывающих $E$. Тогда построим систему интервалов $\set\li$,
$\inat$, где $\si \subset \li$ для всех $\inat$ и~при этом
$\abs\li\le\abs\si+{\ep\over2^i}$. Тогда система $\set\li$, $\inat$
покрывает $E$, и~при этом
$$\sumi\abs\li\le\sumi\at(\abs{\si}+{\ep\over2^i})\le
\sumi\abs{\si}+\ep.$$

В силу произвольности $\epgz$, переходя к~точным нижним граням,
получаем, что мера по интервалам будет не меньше меры по отрезкам.

\prut

\eject

\ut

Если $\set{E_i}$, $\inat$ --- не более чем счетная система множеств,
$E=\bigcup\limits_i E_i$, то $\mes E \le \sumi\mes E_i$.

\pr

Возьмем любое $\epgz$, и~пусть для всех $\inat$ выбрана покрывающая
$E_i$ система интервалов $\set\lij$, такая, что
$$\sum\limits_j\abs\lij<\mes{E_i}+{\ep\over2^i}.$$

Тогда совокупность всех интервалов $\set\lij$, $j\in\N$, $\inat$
покрывает $E$, и~при этом
$$\sum\limits_{i,j}\abs\lij\le\sum\limits_i\sum\limits_j\abs\lij\le
\sumi\at(\mes{E_i}+{\ep\over2^i})\le\sumi\mes{E_i}+\ep.$$

Устремляя $\ep$ к~нулю, получаем требуемое утверждение. В~этой
выкладке первое неравенство требует отдельного рассмотрения:
требуется показать, что бесконечная сумма действительно не зависит
слишком сильно от порядка суммирования, то есть мы в~этот момент
утверждаем, что сумма при любом порядке суммирования не превышает
написанную справа. Действительно, для любой конечной суммы можно
проделать такую формальную выкладку:
$$\sum\limits_{k=1}^n\abs{\l({j\,\at(k)},{i\,\at(k)})}\le
           \sum\limits_{{i\le\max\set{i\,\at(k)}}\atop{j\le\max\set{j\,\at(k)}}}
           \abs\lij\le\sumi\sum\limits_j\abs\lij.$$

Таким образом, неравенство сохранится и~для бесконечной суммы как
предела конечных сумм. Когда-нибудь мы сможем ссылаться на теорему
Коши о~перестановке членов абсолютно сходящегося ряда, которая
утверждает, что на самом деле эти суммы равны, но пока достаточно
более слабого утверждения.

\prut

\bigskip

\uts

\smallskip

1. Если $E\subseteq F\subseteq\reals$, то $\mes{E}\le\mes{F}$. В
частности, если $\mes{F}=0$, то и~$\mes{E}=0$.

\smallskip

2. Если $\set{E_i}$, $\inat$ --- не более чем счетная система
подмножеств $\reals$, все $E_i$ имеют меру нуль, и
$E\subseteq\bigcup\limits_i E_i$, то $\mes{E}=0$.

\smallskip

3. Не более чем счетные множества имеют меру нуль по Лебегу.

\prs

\smallskip

1. Множество $E$ содержится в~объединении не более чем счетной
системы множеств, состоящей из одного множества $\set{F}$, поэтому
$\mes{E}\le\mes{F}$. В~частности, если $\mes{F}=0$, то
$0\le\mes{E}\le\mes{F}=0$.

\smallskip

2. $0\le\mes{E}\le\sumi\mes{E_i}=0$.

3. Если $E=\set{e_i}\subset\reals$, то $E=\bigcup\limits_i{e_i}$.

\pruts

\bigskip

\df Если какое-то свойство выполняется для всех точек множества
$E\subseteq\reals$, кроме некоторого подмножества $F\subseteq E$, и
при этом $\mes{F}=0$, то будем говорить, что это свойство
выполняется почти всюду на $E$.


\subsection{Достаточное условие интегрируемости.}

\lm

Если $\set{\dei}_{i=1}^n$ --- разбиение отрезка \ab, отрезок
$\seg[c,d\,]\subseteq\segab$, то невырожденные отрезки
$\seg[c,d\,]\cap\dei$ образуют разбиение отрезка $\seg[c,d\,]$, и
сумма их длин равна $d-c$.

\pr

Пусть $\dei=\seg[a_{i-1},a_i]$, где $i=1,\dots,n$, и
$$a=a_0<a_1<a_2<\cdots<a_n=b.$$

Тогда найдутся такие числа $k$ и~$l$, что $a_{k-1}\le c<a_k$ и
$a_{l-1}<d\le a_l$. В~этом случае невырожденными отрезками будут
(только) отрезки
$$\seg[c,a_k],\Delta_{k+1},\dots,\Delta_{l-1},\seg[a_{l-1},d\,].$$

Они образуют разбиение $\seg[c,d\,]$, и~сумма их длин равна\
$\at(a_k-c)+\at(a_{k+1}-a_k)+\ \cdots\ +\at(d-a_{l-1})=d-c.$

\prlm

\eject

\tm1{достаточное условие интегрируемости}

Если $f$ ограничена на отрезке \ab\ и~непрерывна почти всюду на \ab,
то $f$ интегрируема на  \ab\ по Риману и~Мак-Шейну (и,
следовательно, по Курцвейлю--Хенстоку).

\pr

Возьмем любое $\epgz$. Найдем не более чем счетную систему
интервалов $\set{\li}$, $\inat$, покрывающую все точки разрыва
функции $f$, такую, что
$$\sumi\abs\li\le{\ep\over{8\cdot\sup\limits_{\seg[a,\,b]}\abs{f}}}.\eqno(1)$$

Здесь мы воспользовались ограниченностью функции $f$. A~для любой
точки отрезка \ab, в~которой функция $f$ непрерывна, найдем такое
число $\gamma\atx$, чтобы $$f\at(B_{3\gamma\,\atx}\atx)\subseteq
B_{\ep\over4\,\at( b-a )}\at(f\atx)\eqno(2)$$

Таким образом, выбрана система интервалов
$\set{\li}\,\cup\set{B_{\gamma\,\atx}\atx}$, покрывающая отрезок
\ab. По теореме Гейне~-- Бореля можно выбрать конечное подпокрытие
отрезка \ab, состоящее, например, из таких интервалов:
$$\ell_{i_1},\dots,\ell_{i_p},\quad
B_{\gamma\,\at(x_1)}\at(x_1),\dots,B_{\gamma\,\at(x_q)}\at(x_q).$$

Снова пользуясь ограниченностью функции $f$, выберем число $\degz$
так, чтобы выполнялись два условия:
$$2\cdot p\cdot\de<{\ep\over8\cdot\sup\limits_{\seg[a,\,b]}\abs{f}},\eqno(3)$$
$$\de<\min\limits_{1\le k\le q} \set{\gamma\at(x_k)}.\eqno(4)$$

А вот теперь проверим критерий Коши. Пусть \Tdixif\ и~\Tdjxjf\
--- два разбиения Хенстока (Мак-Шейна) отрезка \ab\ мельче $\de$
(согласованные с~масштабом $\dex\equiv\de$). Тогда, пользуясь
леммой, производим следующую оценку:
$$\abs{\isfT-\isfTp}=\abs{\sumi f\at(\xii)\abs\dei-\sumj f\at(\xij)\abs\dej}=
\abs{\sumi f\at(\xii)\sumj\abs{\dei\cap\dej}-
\sumj f\at(\xij)\sumi\abs{\dei\cap\dej}}=$$
$$=\abs{\sumi\sumj\at({f\at(\xii)-f\at(\xij)})\abs{\dei\cap\dej}}\le
\sumi\sumj\abs{f\at(\xii)-f\at(\xij)}\abs{\dei\cap\dej}.$$

Разобьем двойную сумму на две части по принадлежности точки $\xii$ к
системе интервалов $\ell_{i_1},\dots,\ell_{i_p}$. Тогда та часть
суммы, где $\xii$ лежит в~объединении $\ell_{i_k}$, грубо
оценивается следующим образом (пользуемся леммой и~условиями
$\at(1)$ и~$\at(3)$):
$$\sum\limits_{\xii\in\bigcup\limits_{k=1}^p
\ell_{i_k}}
\sumj\abs{f\at(\xii)-f\at(\xij)}\abs{\dei\cap\dej}\le
2\cdot\sup\limits_{\seg[a,\,b]}\abs{f}
\cdot        \sum\limits_{\xii\in\bigcup\limits_{k=1}^p
\ell_{i_k}}      \abs\dei<
2\cdot\sup\limits_{\seg[a,\,b]}\abs{f}\cdot
\sum_{k=1}^p\at(\abs{\ell_{i_k}}+2\de)\le$$$$\le
2\cdot\sup\limits_{\seg[a,\,b]}\abs{f}\cdot\at(\sumi\abs\li+2p\de)
<2\cdot\sup\limits_{\seg[a,\,b]}\abs{f}\cdot
\at({\ep\over{8\cdot\sup\limits_{\seg[a,\,b]}\abs{f}}}+
{\ep\over{8\cdot\sup\limits_{\seg[a,\,b]}\abs{f}}})={\ep\over2}.$$

Вторая же сумма оценивается более хитро. Если $\xii$ не лежит ни в
одном из $\ell_{i_k}$, то $\xii\in B_{\gamma\,\at(x_m)}\at(x_m)$.
Поэтому, учитывая, что $\dei\cap\dej\neq\varnothing$, можно
утверждать, что $\xij\in B_{3\gamma\,\at(x_m)}\at(x_m)$. Подробнее:
для интеграла Римана расстояние между $\xii$ и~$\xij$ не превышает
суммы длин отрезков $\dei$ и~$\dej$, которая меньше
$2\de<2\gamma\at(x_m)$ (пользуемся условием $\at(4)$), а~так как от
$x_m$ до $\xii$ расстояние меньше $\gamma\at(x_m)$, то в~сумме как
раз и~получается меньше $3\gamma\at(x_m)$; в~случае интеграла
Мак-Шейна отрезки $\dei$ и~$\dej$ пересекаются, значит, пересекаются
и содержащие их окрестности $B_\de\at(\xii)$ и~$B_\de\at(\xij)$, и~в
сумме снова получаем $2\de+\gamma\at(x_m)$, что меньше, чем
$3\gamma\at(x_m)$. Отсюда, используя до сих пор остававшуюся
невостребованной оценку $\at(2)$, получаем, что
$$\abs{f\at(\xii)-f\at(\xij)}\le\abs{f\at(\xii)-f\at(x_m)}+
\abs{f\at(x_m)-f\at(\xij)}<{\ep\over2\at(b-a)}.$$

Таким образом, вторая сумма оценивается величиной
$$\sum\limits_{\xii\notin \bigcup\limits_{k=1}^p \ell_{i_k}}    \sumj
\abs{f\at(\xii)-f\at(\xij)}\abs{\dei\cap\dej}\le{\ep\over2\at(b-a)}
\sumi\sumj\abs{\dei\cap\dej}={\ep\over2}.$$

\eject

Подведем итог: мы показали, что для любого $\epgz$ можно найти такое
число $\degz$ (оно же масштаб $\dex\equiv\de$), что для любых двух
разбиений $\T$ и~$\T'$ Хенстока мельче $\de$ (Мак-Шейна,
согласованных с~масштабом $\dex$) разность соответствующих им
интегральных сумм меньше, чем $\ep/2+\ep/2=\ep$. Выполнен критерий
Коши.

\prtm

\bigskip

\imps

\smallskip

1. Непрерывные на отрезке \ab\ функции интегрируемы по Риману и
Мак-Шейну на \ab, поскольку любая непрерывная на отрезке функция
ограничена.

\smallskip

2. Монотонные функции тоже интегрируемы по Риману и~Мак-Шейну, так
как все разрывы монотонных функций --- первого рода, и~им
соответствуют конечные промежутки (не пустые и~не одноточечные) в
области значений функции, которых может быть не более чем счетное
множество --- ведь каждому можно поставить в~соответствие
какое-нибудь рациональное число, которое в~нем лежит. Значит, и
множество точек разрыва не более чем счетно. К~тому же функция,
монотонная на отрезке \ab, ограничена на нем значениями $f\at(a)$ и
$f\at(b)$.

\subsection{Необходимое условие интегрируемости по Риману. Критерий Лебега.}

\df Если функция $f$ определена на произвольном множестве $E$, то
осцилляцией $f$ на $E$ --- это $$\osc_E f=\sup\limits_{x,x'\in E}
\abs{f\atx-f\at(x')}.$$

\bigskip

\ut

Если $f$ --- действительнозначная функция на множестве $E$, то
$$\osc_E f=\sup\limits_E f-\inf\limits_E f.$$

\pr

Если $x,x'\in E$ и~$f\at(x')\le f\atx$, то, очевидно,
$f\atx-f\at(x')\le\sup\limits_E f - \inf\limits_E f$, значит, то же
верно и~для осцилляции. Докажем обратное: $\osc_E f\ge\sup\limits_E
f-\inf\limits_E f$ . Если $\sup\limits_E f=+\infty$
 или $\inf\limits_E f=-\infty$ , то утверждение верно, так как в~этом случае $\osc_E
f=+\infty$. В~противном случае для любого $\epgz$ найдутся такие $x$
и $x'$ из $E$, что $f\atx>\sup\limits_E f-{\ep\over2}$ и
$f\at(x')<\inf\limits_E f+{\ep\over2}$ Но тогда
$$\osc_E f\ge f\atx-f\at(x')>\sup\limits_E f-\inf\limits_E f-\ep.$$

Устремляя $\ep$ к~нулю, получаем требуемое утверждение.

\prut

\bigskip

\lm

Пусть действительнозначная функция $f$ определена на отрезке \ab,
$\set{\dei}$ --- некоторое разбиение отрезка \ab. Тогда
$$\osc_{\xii\in\dei} \isfT=\sumi\at(\osc_{\dei}f\cdot\ldei),$$

где через $\T$ обозначено разбиение $\dixif$ при данном выборе
$\xii\in\dei$.

\pr

Если функция $f$ неограничена на \ab, то оба выражения равны
$+\infty$ и~утверждение верно. Если же $f$ ограничена на \ab, то
можно произвести оценку:
$$\osc_{\xii^{}\in\dei} \isfT=    \sup\limits_{\xii^{},\,\xii'\in\dei}
\abs{\sumi\Bigl(
f\at(\xii)\ldei-f\at(\xii')\ldei\Bigr)}\le\sumi\at(\sup\limits_{\xii^{},\,\xii'\in\dei}
\abs{f\at(\xii)-f\at(\xii')}\cdot\ldei)=\sumi\at(\osc_{\dei}f\cdot\ldei).$$

В последнем равенстве мы пользуемся действительнозначностью. Докажем
теперь, что
$$\osc_{\xii\in\dei} \isfT\ge\sumi\at(\osc_{\dei}f\cdot\ldei).$$

\eject

Возьмем любое $\epgz$. Тогда найдутся такие пары точек
$\xii,\xii'\in\dei$, что
$f\at(\xii)-f\at(\xii')>\osc_{\dei}f-{\ep\over b-a}$ (мы по-прежнему
считаем функцию ограниченной). Модуль здесь не играет роли, так как
всегда можно поменять местами $\xii$ и~$\xii'$. Теперь оцениваем:
$$\sumi f\at(\xii)\ldei-\sumi f\at(\xii')\ldei=
\sumi\at({f\at(\xii)-f\at(\xii')})\ldei>
\sumi\at(\osc_{\dei}f-{\ep\over b-a})\ldei=
\sumi\at(\osc_{\dei}f\cdot\ldei)-\ep.$$

Устремляя $\ep$ к~нулю, получаем требуемое утверждение.

\prlm

\bigskip

\tm2{необходимое условие интегрируемости по Риману}

Если действительнозначная функция $f$ интегрируема на отрезке \ab\ в
смысле Римана, то $f$ ограничена и~непрерывна почти всюду на \ab.

\pr

Ограниченность уже доказана. В~соответствии с определением
интегрируемости по Риману для любого $\epgz$ найдется такое
разбиение $\set{\dei}$ отрезка \ab, что при любом выборе точек
$\xii\in\dei$ выполнялась оценка
$$\abs{\isfT-\riman\intabfdx}\le{\ep\over3},$$
где \Tdixif. Тогда $\osc_{\xii\in\dei}\isfT\le{2\ep\over3}<\ep$,
откуда по лемме $\sumi\at(\osc_{\dei}f\cdot\ldei)<\ep$

\smallskip

Вот теперь мы фиксируем $\epgz$, и~проделываем следующую процедуру.
Для всех чисел $\inat$ находим такое разбиение
$T_i=\set{\Delta^{ i}_j}$, что выполняется оценка
$$\sumj\at(\osc_{\Delta^{ i}_j}f\cdot\abs{\Delta^{ i}_j})<{\ep\over2^{2j}}.$$

Теперь среди всех этих разбиений найдем такие отрезки
$\Delta^{ i}_j$, на которых $\osc f\ge{1/2^i}$. Тогда ясно, что
сумма длин отрезков, выбранных из разбиения $T_i$, будет меньше
$\ep/2^i$
--- иначе оценка никак не выполнится. Таким образом, сумма длин всех
выбранных отрезков будет меньше $\sumi{\ep/2^i}=\ep$.

\smallskip

И завершающее утверждение: если какая-то точка $x$ не принадлежит ни
одному из выбранных отрезков, то $f$ непрерывна в~точке $x$.
Действительно, если нужно получить $\de$-окрестность точки $x$, в
которой $\forall\,t \in B_{\de}\atx\ \abs{f\atx-f\at(t)}<1/2^i$,
то достаточно рассмотреть разбиение $T_i$, и~если точка $x$ лежит
внутри отрезка разбиения, то упихиваем окрестность в~этот отрезок, а
если на краю двух отрезков (тогда они оба невыбранные)~--- то в
объединение этих двух отрезков, и~выполнение условия следует из
оценки осцилляции в~выбранных отрезках. Следовательно, нам удалось
покрыть все точки разрыва функции $f$ системой отрезков, сумма длин
которых меньше $\ep$. Отсюда ввиду произвольности $\ep$ получаем,
что $f$ непрерывна почти всюду на отрезке \ab.

\prtm

\bigskip

\impd{критерий Лебега интегрируемости по Риману}

Функция $f$, определенная на отрезке \ab, интегрируема на \ab\ в
смысле Римана тогда и~только тогда, когда $f$ ограничена на \ab\ и
непрерывна почти всюду на \ab.

\pr

Для действительнозначных функций утверждение только что доказано,
поэтому покажем, что происходит в~комплекснозначном случае, а~заодно
и выведем важное утверждение.

\smallskip

Пусть $f$ --- комплекснозначная функция на \ab, $I$ --- ее интеграл
на \ab\ (в любом из смыслов). Тогда, очевидно, $$\isum(\re
f,\T)=\sumi\re\fxildi=\re\sumi\fxildi=\re\,\isfT.$$ Аналогичная
оценка работает и~для мнимой части. Поэтому, с~учетом неравенств
$\abs{\re z}\le\abs{z}$, $\abs{\im z}\le\abs{z}$ и~$\abs{z}=\abs{\re
z}+\abs{\im z}$, выполняются неравенства
$$\abs{\isum(\re f,\T)-\re
I}=\abs{\re\at(\isfT-I)}\le\abs{\isfT-I},$$
$$\abs{\isum(\im f,\T)-\im
I}=\abs{\im\at(\isfT-I)}\le\abs{\isfT-I},$$
$$\abs{\isfT-I}\le\abs{\isum(\re f,\T)-\re
I}+\abs{\isum(\im f,\T)-\im I}$$

\eject

Таким образом, комплекснозначная функция интегрируема в~любом из
смыслов на отрезке \ab\ тогда и~только тогда, когда интегрируемы в
том же смысле ее действительная и~мнимая часть, и~в~случае
интегрируемости выполняется равенство
$$\intabfdx=\intab{\re f},x+i\cdot \intab{\im f},x.$$

Применяя вышесказанное к~интегралу Римана и~теореме $2$, получаем
следующее утверждение: если комплекснозначная функция интегрируема
по Риману на отрезке \ab, то интегрируемы её действительная и~мнимая
часть, которые и~будут ограничены и~непрерывны почти всюду, делая
таковой и~саму функцию. Обратное утверждение мы доказывали для любых
функций, поэтому критерий Лебега теперь полностью доказан.

\primp

\bigskip

\imps

\smallskip

1. Если функция $f$ интегрируема по Риману на отрезке \ab, то $f$
интегрируема и~по Мак-Шейну на \ab, и~значения интегралов совпадают.

\smallskip

2. Если функции $f$ и~$g$ интегрируемы по Риману на отрезке \ab, то
и $f  \cdot g$ интегрируема по Риману на \ab.

\smallskip

3. Если значения интегрируемой по Риману функции изменить в~конечном
числе точек, то получится функция, также интегрируемая по Риману, и
значение интеграла не изменится.

\smallskip

4. Если функция $f$ интегрируема по Риману на отрезке \ab, функция
$\ph$ непрерывна на отрезке $\seg[c,d]\subseteq f\at(\segab)$, то
$\ph\at(f)$ тоже интегрируема на \ab\ по Риману.

\smallskip

5. Интеграл Римана --- абсолютный, то есть если функция $f$
интегрируема по Риману на отрезке \ab, то $\abs{f}$ тоже
интегрируема по Риману на \ab\ и~выполняется равенство
$$\abs\intabfdx\le\intab{\abs f},x$$

\smallskip

6. Если функция $f$ интегрируема по Риману на отрезке \ab\ и
$f\atx\ge0$ на \ab, а~так же $f\atx>0$ в~некоторой точке
непрерывности функции $f$, то и~интеграл функции $f$ на \ab\ строго
больше нуля.

\prs

\smallskip

1. Если $f$ интегрируема по Риману, то она ограничена и~непрерывна
почти всюду, а~значит интегрируема по Мак-Шейну. Равенство
интегралов следует, например, из того, что они оба равны интегралу
Курцвейля -- Хенстока.

\smallskip

2. Если $f$ и~$g$ ограничены, то и~$f \cdot g$ тоже. $f \cdot g$
непрерывна почти всюду, поскольку объединение множеств точек разрыва
$f$ и~$g$, которые меры нуль, тоже будет множеством меры нуль.

\smallskip

3. Понятно, что функция останется интегрируемой, поскольку не теряет
ограниченности и~непрерывности почти всюду. Покажем, что не
изменится значение интеграла. Пусть $\widetilde f$ --- новая
функция, отличающаяся от $f$ в~$n$ точках. Тогда для любого
разбиения $\T$ Хенстока отрезка интегрирования \ab\ мельче $\de$
выполняется оценка $$\isum(f - \widetilde
f,\T)\le2n\de\cdot\sup\limits_{\seg[a,\,b]}\at(f - \widetilde
f\,),$$ с~учетом того, что каждая точка может оказаться в~двух
отрезках. Из этого рассуждения видно, что
$$\riman\intab{\at(f - \widetilde f)},x=0,$$
а так как $f=\widetilde f+\at(f - \widetilde f)$, то как раз и
получается требуемый результат.

\smallskip

4. Поскольку $\ph$ непрерывна на отрезке, она ограничена на нем, а
значит и~функция $\ph\at(f)$ тоже ограничена. Если $f$ непрерывна в
точке $x$, то и~$\ph\at(f)$ тоже, а~значит $\ph\at(f)$ непрерывна
почти всюду.

\smallskip

5. Функция $\abs{x}$ непрерывна на $\reals$ (и даже на $\complex$),
следовательно, можно сослаться на предыдущее следствие. Неравенство
для интегралов непосредственно следует из неравенства для
интегральных сумм ввиду свойств предела по базе:
$$\abs{\riman\intabfdx}=\abs{\lim\limits_{\baseriman}\isfT}=\lim\limits_{\baseriman}\abs\isfT\le
\lim\limits_{\baseriman}\isum(\abs{f},\T)=\riman\intab\abs{f},x$$

6. Неотрицательность интеграла следует из неотрицательности
интегральных сумм. Если $f\at(x_0)>0$ и~$f$ непрерывна в~$x_0$, то
найдется такая замкнутая $\de$-окрестность $x_0$, в~которой
$f\atx>{1\over2}f\at(x_0)$ . Введем функцию
$$g\atx=\begin{cases}{f\at(x_0)\over2},\quad x\in
\overline{B}_{\de}\at(x_0)\cap\segab;
\cr\quad0\qquad\hbox{в остальных случаях.}\end{cases}$$

Тогда проделывается следующая выкладка:
$$\riman\intabfdx\ge\riman\intab
g,x={f\atx\over2}\cdot\abs{\overline{B}_{\de}\at(x_0)\cap\segab}>0$$

\primps

\bigskip

{\small \rems

Свойство $2$ не выполняется для интегралов Мак-Шейна и~Курцвейля --
Хенстока. Контрпримером является функция $f\atx={1\over\sqrt{x}}$,
которая интегрируема на отрезке $\seg[0,1]$ по Мак-Шейну, но квадрат
которой не интегрируем даже по Курцвейлю~-- Хенстоку. Свойство 3
нельзя обобщить даже на случай не более чем счетного множества
точек, контрпример --- функция Дирихле, которая отличается от
тождественно нулевой функции на счетном множестве точек. Однако
интегралы Мак-Шейна и~Курцвейля -- Хенстока выдерживают даже
изменение на множестве меры нуль, что мы сейчас и~докажем. Как будет
показано ниже, свойство 5 выполняется также и~для интегралов
Мак-Шейна, но не для интегралов Курцвейля -- Хенстока, хотя,
конечно, в~случае интегрируемости, неравенство на интегралы
выполняется, поскольку нам безразлично, которая из трех баз
упоминалась в~доказательстве.}

\subsection{Интегрирование функций, определенных почти всюду.}

Докажем теперь полезное утверждение о~том, что значение функции на
множестве меры нуль не влияет на интегралы Мак-Шейна и~Курцвейля --
Хенстока.

\bigskip

\lm

Если функция $f$ определена на \ab\ и~$f\atx=0$ почти всюду на \ab,
то $f$ интегрируема на \ab\ по Мак-Шейну и~ее интеграл равен нулю.

\pr

Возьмем любое $\epgz$. Для всех $\inat$ введем обозначение $$E_i =
\set{x\in\segab:i - 1<\abs{f\atx}\le i}.$$

Ясно, что $\forall\ \inatn\ \mes E_i = 0$, поскольку их
объединение $E$ имеет по условию меру нуль. Это означает, что для
любого $\inatn$ можно выбрать такую систему интервалов $\set{\lij}$,
$j \in \N$, покрывающих $E_i$, сумма длин которых меньше, скажем,
$\ep\over{i\cdot2^i}$. Возьмем масштаб $\dex$, равный, например,
единице (хотя это и~не важно), если $x\notin E$, a~при $x\in E_i$
потребуем только, чтобы $B_{\de\,\atx}\atx$ лежало целиком в~одном
из интервалов $\lij$, что приемлемо, поскольку $x$ лежит в~одном из
этих интервалов, a~интервал --- множество открытое. Пусть \Tdixif\
--- любое отмеченное разбиение отрезка \ab, согласованное с~$\dex$.
Оценка делается просто:
$$\abs\isfT=\abs{\sumi\fxildi}=\abs{\sum_{i=1}^\infty
\sum_{\xi_j\in E_i}{f\at(\xi_j)\abs{\Delta_j}}}\le
\sum_{i=1}^\infty\at(i\cdot\sum_{\xi_j\in
E_i}{\abs{\Delta_j}})\le\sum_{i=1}^\infty\at(i\cdot\sumj\abs\lij)<
\sum_{i=1}^\infty i\cdot{\ep\over{i\cdot2^i}}=\ep$$

\prlm

\bigskip

\ut

Если функции $f$ и~$g$ определены на отрезке \ab, $f=g$ почти всюду
на \ab, то $f$ и~$g$ одновременно интегрируемы или не интегрируемы
по Мак-Шейну (Курцвейлю -- Хенстоку), и~в~случае интегрируемости
значения их интегралов равны.

\pr

По предыдущей лемме имеем: $$\at(g-f)\ismacsh[a,b] \hbox{, и~}
\macsh \intab{(g-f)},x=0.$$

\eject

Поэтому если $f\ismacsh[a,b]$, то и~$g\ismacsh[a,b]$, и~выполняется
равенство $$\macsh\intab
g,x=\macsh\intabfdx+\macsh\intab{(g-f)},x=\macsh\intabfdx.$$

В частности из этого следует, что если одна из функций
неинтегрируема, то неинтегрируема и~вторая, иначе проинтегрировалась
бы первая. Для интеграла Курцвейля -- Хенстока все аналогично.

\prut

\bigskip

Это утверждение позволяет корректно определить интегралы Мак-Шейна и
Курцвейля -- Хенстока от функций, определенных лишь почти всюду.

\bigskip

\df Функция $f$, определенная почти всюду на отрезке \ab,
интегрируема в~смысле Мак-Шейна (Курцвейля -- Хенстока) на \ab, и~ее
интеграл равен $I$, если при некотором (а значит, и~при любом)
доопределении $f$ на весь отрезок \ab\ получается интегрируемая в
соответствующем смысле (по старому определению) функция, интеграл
которой на \ab\ равен $I$.

\bigskip

{\small\rem

Именно в~этом смысле недавно понималась интегрируемость функции
$1\over\sqrt{x}$ на отрезке $\seg[0,1]$.}

\bigskip

%------------------------------------------------------------------------------------

\sectionname4{Измеримые функции.}

\subsection{Два определения измеримости функций.}

\dfn1 Функция $f$, определенная почти всюду на \ab, называется
измеримой на \ab, если для любого $\epgz$ найдется такая функция
$g$, непрерывная на \ab, что $\mes\set{x\in\segab:f\atx\neq
g\atx}<\ep$.

\dfn2 Функция $f$, определенная почти всюду на \ab, называется
измеримой на \ab, если для любого $\epgz$ найдется такое множество
$E\in\segab$, что $\mes E=0$ и~$f$ непрерывна на множестве
$\segab\setminus E$ по $\segab\setminus E$.

\bigskip

\eqdf

Ясно, что если функция $f$ измерима по первому определению, то она
измерима и~по второму: достаточно положить $E=\set{x\in\segab:
f\atx\neq g\atx}$. Чтобы вывести из второго определения первое, и
тем самым установить эквивалентность определений, потребуются
дополнительные утверждения.

\bigskip

\lmnd1 {о структуре открытых множеств}

Любое открытое множество на $\reals$ представляется в~виде
дизъюнктного объединения не более чем счетной системы интервалов
(возможно, с~бесконечными концами), концы которых не лежат в~самом
множестве.

\pr

Пусть $G\subseteq\reals$ --- открытое множество. Если оно пусто, то
утверждение очевидно, поэтому будем считать, что оно не пусто. Пусть
$x\in G$. Тогда положим
$$\alpha_x=\inf\set{s\in\reals:\seg[s,x]\subseteq G},$$
$$\beta_x=\sup\set{s\in\reals:\seg[x,s]\subseteq G}.$$

Получаем, что $x\in\at(\alpha_x,\beta_x)\subseteq G$. Действительно,
если $\alpha_x<t<x$, то в~интервале $\at(\alpha_x,t)$ найдется такое
число $s$, что $\seg[s,x]\subseteq G$, откуда $t\in G$. Аналогично,
если $x<t<\beta_x$, то тоже $t\in G$, при этом мы брали $x\in G$,
поэтому весь интервал $\at(\alpha_x,\beta_x)\subseteq G$. При этом
$\alpha_x\notin G$, поскольку иначе, так как множество $G$ открыто,
нашлась бы окрестность точки $\alpha_x$, лежащая в~$G$, что
противоречило бы выбору $\alpha_x$. Аналогично, $\beta_x\notin G$.

\smallskip

Описанное построение можно произвести для всех точек множества $G$,
причем построенные интервалы покрывают $G$, так как мы их строили
вокруг каждой точки $G$. Любые два из построенных для разных точек
интервалов не пересекаются либо совпадают. Действительно, если
$\at(a,b)\cap\at(c,d\,)\neq\varnothing$, то возьмем
$x \in\at(a,b)\cap\at(c,d\,)$. Тогда $a\le c<x$ по построению
$\at(a,b)$, a~$c\le a<x$ по построению $(c,d\,)$, поэтому
$a = c$\,; аналогично $b = d$. Поэтому, взяв по одному интервалу
из каждого класса равных друг другу интервалов, получим требуемое
разбиение множества $G$. Ну и, наконец, как уже отмечалось выше,
множество непересекающихся интервалов не может быть более чем
счетным, так как в~каждом содержится рациональное число.

\prlm

\eject

\lmn2

Пусть $F\subseteq\reals$ --- непустое замкнутое множество, функция
$f$ определена на $F$ и~непрерывна на $F$ по $F$. Пусть множество
$\reals\setminus F$ представляется (в смысле предыдущей леммы) в
виде объединения интервалов $\at(\alpha_i,\beta_i),
\alpha_i,\beta_i\in F$. Доопределим $f$ на конечных отрезках
$\seg[\alpha_i,\beta_i]$ линейно, a~на бесконечных --- константами.
Тогда получится непрерывная на $\reals$ функция.

\pr

На интервалах $\at(\alpha_i,\beta_i)$ функция действительно
непрерывна. В~точках множества $F$ она непрерывна по $F$ по условию.
Остается показать, что $f$ непрерывна на $F$ по $\reals$. Возьмем
любое $x\in F$, и~покажем, что $f\at(x+0)=f\at(x)$ и
$f\at(x-0)=f\at(x)$. Причем второе утверждение аналогично первому,
поэтому мы его не будем отдельно доказывать. Если точка $x$ является
одной из $\alpha_i \neq -\infty$, то требуемое утверждение следует
из того, как мы доопределяли функцию на соответствующем интервале. В
противном случае можно утверждать, что $\forall\ \degz \at(x,x+\de)\cap F\neq\varnothing$.

\smallskip

Теперь возьмем любое $\epgz$. Так как $f$ непрерывна на $F$ по $F$,
то найдется $\degz$, такое, что $\forall\ t \in B_{\de}\atx\cap F\quad f\at(t) \in B_{\ep}\at(f\atx)$. Поскольку $\at(x,x+\de)\cap
F\neq\varnothing$, найдется число $\gamma \in\at(0,\de)$, такое,
что $x+\gamma\in F$. Тогда, очевидно, для любого
$t\in\at(x,x+\gamma)$ имеем $f\at(t)\in B_{\ep}\at(f\atx)$. Это и
показывает, что $f\at(x+0)=f\at(x)$.

\prlm

\bigskip

Докажем теперь, что из второго определения измеримости следует
первое. Действительно, возьмем любое $\epgz$, и~пусть $E$ --- то
самое множество из определения $2$. Поскольку его внешняя мера
строго меньше $\ep$, найдется система $\set{\li}$ интервалов,
покрывающая $Е$, сумма длин которых меньше $\ep$. Положим
$$G=\at(\bigcup\limits_i\li)\cup\at(-\infty,a)\cup\at(b,+\infty).$$

Тогда $G$ --- открытое множество, a~$F=\reals\setminus G$ ---
замкнутое. По определению, $f$ непрерывна на
$F\subseteq\segab\setminus E$. Пусть $g$ --- доопределенная c~$F$ на
$\reals$ по только что доказанной лемме функция $f$. Тогда $g$
непрерывна на $\reals$, и~$$H=\set{x\in\segab:f\atx\neq
g\atx}\subseteq\bigcup\limits_i\li,$$ причем внешняя мера
объединения меньше $\ep$ (его можно покрыть самим собой), значит и
$\mes H$ меньше $\ep$.

\smallskip

\preqdf

\subsection{Интегрируемость ограниченных измеримых функций.}

\tm1 {об интегрируемости ограниченных измеримых функций}

Если функция $f$ определена, ограничена и~измерима на отрезке \ab,
то $f\ismacsh[a,b]$.

\pr

Поскольку $f$ ограничена, есть такое число $C$, что $\abs{f}<C$ на
\ab. Возьмем любое $\epgz$. Поскольку $f$ измерима, найдется такая
функция $g\in С\segab$, что $\mes E<{\ep\over8C}$, где
$E=\set{x\in\segab: f\atx\neq g\atx}$. При этом можно считать, что
$g$ тоже ограничена тем же числом $C$: в~конце концов, всегда можно
срезать функцию $g$ в~тех точках, где она по модулю вылезает за $C$.
Выберем систему интервалов $\set{\li}$, покрывающую $E$, с~суммой
длин меньше вышеупомянутого ${\ep\over8C}$.

\smallskip

Поскольку $g$ непрерывна на \ab, то она интегрируема на нем по
Мак-Шейну. Пусть $I$ --- интеграл $g$ на \ab.

\smallskip

Найдется такой масштаб $\de_0\atx$ на \ab, что для любого
отмеченного разбиения $\T$ отрезка \ab, согласованного с
$\de_0\atx$, будет верно неравенство $\abs{\isgT-I}<\ep/4$. Построим
новый масштаб $\dex$, который будет совпадать с~$\de_0\atx$ на
$\segab\setminus E$, a~на множестве $E$ будет выбран так, чтобы для
всех точек $x\in E$ окрестность $B_{\de\,\atx}\atx$ лежала бы
целиком в~одном из интервалов $\li$. Проверим критерий Коши для
функции $f$.

\smallskip

Пусть $\T$ и~$\T'$ --- два отмеченных разбиения \ab, согласованных с
$\dex$. Тогда
$$\abs{\isfT-\isfTp}\le\abs{\isfT-\isgT}+\abs{\isgT-I}+\abs{I-\isgTp}+\abs{\isgTp-\isfTp}.$$

Второе и~третье слагаемые уже меньше $\ep/4$ каждое. Два других
слагаемых оцениваются одинаково, поскольку для любого отмеченного
разбиения \Tdixif\ отрезка \ab, согласованного с~$\dex$,
$$\abs{\isfT-\isgT}=\abs{\sumi\at({f\at(\xii)-g\at(\xii)})\ldei}
=\abs{\sum_{\xii\in E}\at({f\at(\xii)-g\at(\xii)})\ldei}\le
2C\cdot\sum_{\xii\in E}\ldei<2C\cdot{\ep\over8C}={\ep\over4},$$
поскольку все отрезки $\dei$ содержатся в~интервалах $\li$ и~не
перекрываются. Мы показали, что
$\abs{\isfT-\isfTp}<{\ep\over4}+{\ep\over4}+{\ep\over4}+{\ep\over4}=\ep$.
Выполнен критерий Коши интегрируемости.

\prtm

\eject

{\small \rem

Условие ограниченности можно заменить следующим  условием \lq\lq
существенной ограниченности\rq\rq: функция $f$, определенная почти
всюду на множестве $E$, называется существенно ограниченной на $E$,
если найдется такая константа $C$, что $\abs{f\atx}\le C$ почти
всюду на $E$. Таким образом, если функция измерима на отрезке \ab\ и
существенно ограничена на \ab, то ее можно подкорректировать,
доопределив нулем там, где она не определена, и~изменив значение на
0 там, где оно по модулю больше $C$. При этом функция станет всюду
определенной и~ограниченной, a~значит, интегрируемой по Мак-Шейну,
следовательно, этим же свойством обладает и~первоначальная функция,
поскольку, как было отмечено в~предыдущем параграфе, значение
функции на множестве меры нуль влияет на интегрируемость по
Мак-Шейну и~значение интеграла.}

\bigskip

%------------------------------------------------------------------------------------

\sectionname5{Неопределенный интеграл.}

\df Если функция $f$ интегрируема на отрезке \ab\ в~каком-либо
смысле и~$c\in\segab$, то будем считать, что
$$\intd[b,a]f,x=-\intabfdx\qquad\hbox{и}\qquad\intd[c,c]f,x=0.$$

\bigskip

\ut

Не сложно обобщить теорему об~аддитивности интеграла по отрезку на
случай, когда в~равенстве
$$\intd[a,c]f,x=\intd[a,b]f,x+\intd[b,c]f,x=\at(\int\limits_a^b + \int\limits_b^c)f\,dx$$
(встречается и~такое обозначение) допустить интегралы, у~которых
верхний предел меньше нижнего или ему равен. Все сводится к
применению определений и~основной теоремы об~аддитивности по отрезку
при переборе всевозможных расположений точек a, b и~с на числовой
прямой.

\bigskip

\df Если функция $f$ интегрируема на отрезке \ab\ в~каком-либо
смысле, то определенную на \ab\ функцию $$F\atx=\intd[x_0,x]f,t+C$$
при любых фиксированных $x_0\in\segab$ и~$C\in\reals$ будем называть
неопределенным интегралом (интегралом с~переменным верхним пределом)
от функции $f$ на отрезке \ab\ в~соответствующем смысле.

\bigskip

\ut

Любые два неопределенных интеграла от одной и~той же функции в~любом
смысле отличаются на константу.

\pr

Пусть
$$F_1\atx=\intd[x_1,x]f,t+C_1\qquad\hbox{и}\qquad F_2\atx=\intd[x_2,x]f,t+C_2.$$

Тогда
$$F_2\atx-F_1\atx=\at(\int\limits_{x_2}^x - \int\limits_{x_1}^x)f\,dx
+\at(C_1-C_2)=\intd[x_2,x_1]f,x+\at(C_1-C_2),$$

что действительно является константой.

\prut

\eject

\df Функция $f$ принадлежит классу Липшица (Гельдера) на множестве
$E$, если $f$ определена на $E$, и~найдется такая постоянная $C$,
что $$\forall\ x,t\in E\quad\abs{f\at(x)-f\at(t)}\le C \abs{x-t}.$$

Обозначение: $f\islip{\,\at( E)}$.

\bigskip

Ясно, что из принадлежности функции $f$ классу Липшица на $E$
следует непрерывность и~даже равномерная непрерывность $f$ на $E$,
поскольку модуль непрерывности $\omega_f\at(\de)\le C\cdot\de$.

\bigskip

\tm1{о свойствах неопределенного интеграла}

Пусть функция $f$ интегрируема на \ab\ в~каком-либо смысле и
$$F\atx=\intd[x_0,x]f,t+C.$$

1. Если $f$ ограничена на \ab, то ее неопределенный интеграл
$F\islip \segab$.

2. Если функция $f$ непрерывна в~некоторой точке $x_0\in\segab$, то
$\exists F'\at(x_0)=f\at(x_0)$.

\pr

\medskip

1. Пусть $f\atx\le D$ на \ab. Покажем, что $$\intd[x_1,x_2]f,x\le
D\at(x_2-x_1).$$ Действительно, для каждой интегральной суммы
$\isum(f,\dixif)$ на отрезке $\seg[x_1,x_2]$ (можно считать, что
$x_1<x_2$; при $x_1=x_2$ утверждение очевидно) можно произвести
оценку $$\abs{\sumi\fxildi}\le C\sumi\ldei=C\at(x_2-x_1),$$ значит,
то же самое верно и~для интеграла как предела по соответствующей
базе. Для интегралов Мак-Шейна и~Курцвейля -- Хенстока все
переносится на случай существенной ограниченности так же, как в
замечании в~конце предыдущего параграфа.

\medskip

2. Пусть функция $f$ непрерывна в~точке $x_0$. Покажем, что
$$\lim\limits_{\dx\rightarrow0}
\at({{F\at(x_0+\dx)-F\at(x_0)\over\dx}-f\at(x_0)})=0$$

Действительно, первое слагаемое можно записать в~виде
$${1\over\dx}\cdot    \intd[x_0,x_0+\dx]    f\at(t),t,$$
второе --- в~весьма искусственном виде
$${1\over\dx}\cdot    \intd[x_0,x_0+\dx]    f\at(x_0),t,$$
и все выражение примет вид
$${1\over\dx}\cdot    \intd[x_0,x_0+\dx]    \at({f\at(t)-f\at(x_0)}),t.$$

Применяя первый пункт теоремы (в некоторой окрестности точки $x$,
где функция ограничена), оцениваем это выражение сверху по модулю
величиной
$${1\over\dx}\ \cdot    \sup\limits_{\seg[x_0,\,x_0+\dx]}
        \abs{f\at(t)-f\at(x_0)}\abs{\dx
}=        \sup\limits_{\seg[x_0,\,x_0+\dx]}
        \abs{f\at(t)-f\at(x_0)}$$

Но эта величина бесконечно мала, так как функция $f$ непрерывна в
точке $x_0$, что нам и~требовалось. {\sl Замечание:}
$\seg[x_0,\,x_0+\dx]$ означает отрезок с~концами $x_0$ и~$x_0+\dx$
(а вдруг $\dx < 0$?)

\prtm

\bigskip

\imp

Если $f\isriman[a,b]$, то неопределенный интеграл $F\islip \segab$,
и $\exists\ F'\atx=f\atx$ почти всюду на \ab.

\eject

%------------------------------------------------------------------------------------

\sectionname6{Леммы Колмогорова -- Сакса -- Хенстока и
Витали.}

Пусть функция $f$ интегрируема на отрезке \ab\ в~каком-либо из трех
смыслов, $I$ --- ее интеграл. Пусть также дано некоторое $\epgz$ и
выбрано такое число $\degz$ (или масштаб $\dex$), что для любого
разбиения $\T$ Хенстока (Мак-Шейна, Хенстока) отрезка \ab\ мельче
$\de$ (согласованного с~$\dex$) выполняется неравенство
$\abs{\isfT-I}<\ep$. Две следующие леммы позволят нам уже в~этой
ситуации делать некоторые выводы.

\bigskip

\lmnd1{слабая лемма Колмогорова -- Сакса -- Хенстока}

При описанных выше условиях для любого разбиения
$\T = \dixif_{i\in\J}$ Хенстока (Мак-Шейна, Хенстока) отрезка \ab\
мельче $\de$ (согласованного с~$\dex$) и~для любого подмножества
индексов разбиения $J\subseteq\J$ выполняется оценка
$$\abs{\sum_{i\in J}\fxildi-\intdeifdx}\le\ep.$$

\pr

Для всех отрезков, не попадающих в~сумму, сейчас построим маленькие
разбиения. A~именно, если $i\notin J$, то для отрезка $\dei$ найдем
такое число $\de_i$, $0 < \de_i < \de$ (масштаб $\de_i\atx$,
$0 < \de_i\atx < \dex$), чтобы для любого разбиения $\T_i$
Хенстока (Мак-Шейна, Хенстока) отрезка $\dei$ мельче $\de_i$
(согласованного с~$\de_i\atx$) была верна оценка
$$\abs{\isum(f,\T_i)-\intdeifdx}<{\gamma\over n},$$
где $n = \abs{\J}$ --- количество пар в~разбиении $\T$, $\gamma$
--- некоторое наперед заданное положительное число. Теперь соберем
из $\T_i$ новое разбиение:
$$\Tt=\at(\bigcup\limits_{i\in\J\setminus J}   \T_i)\cup\dixif_{i\in J}$$

Это разбиение тоже мельче $\de$ (согласованное с~$\dex$), поскольку
$\de_i < \de$ (соответственно, $\de_i\atx < \dex$). Поэтому
$\abs{\isfTt-I}<\ep$. В~то же время
$$\isfTt-I=  \sum\limits_{i\in\J\setminus
J}\at({\isum(f,\T_i)-\intdeifdx})+ \sum\limits_{i\in
J}\at(\fxildi-\intdeifdx)$$

Левая часть по модулю меньше $\ep$, первая сумма в~правой части по
модулю меньше $\gamma$. Перенося ее вправо и~беря модуль, получаем
искомую оценку для второй суммы:
$$\abs{\sum\limits_{i\in
J}\at(\fxildi-\intdeifdx)}<\ep+\gamma$$

Ввиду произвольности $\gamma > 0$ как раз и~получилось
доказываемое утверждение. Заметим, что неравенство при таком
переходе вообще говоря становится нестрогим, что мы предвидели еще в
формулировке.

\prlm

\bigskip

\lmnd2{сильная лемма Колмогорова -- Сакса -- Хенстока}

При тех же условиях для любого разбиения $\T=\dixif_{i\in\J}$
Хенстока (Мак-Шейна, Хенстока) отрезка \ab\ мельче $\de$
(согласованного с~$\dex$) справедлива следующая оценка --- теперь
уже на сумму модулей:
$$\sum_{i\in\J}\abs{\fxildi-\intdeifdx}\le
\begin{cases}2\ep\hbox{ в~действительнозначном случае;}\cr4\ep\hbox{
в комплекснозначном случае.}\end{cases}$$

\eject

\pr

Рассмотрим сначала случай, когда $f$ --- действительнозначная
функция. Обозначим
$$J=\set{i\in\J:\fxildi-\intdeifdx\ge0}$$

Тогда по слабой лемме Колмогорова -- Сакса -- Хенстока имеем
$$\abs{\,\sum\limits_{i\in J}\at(\fxildi-\intdeifdx)}=
\sum\limits_{i\in J}\abs{\fxildi-\intdeifdx}\le\ep$$

Поскольку все слагаемые неотрицательны, модуль суммы, очевидно,
равен сумме модулей. Теперь так же обработаем отрицательные
слагаемые:
$$\abs{\,\sum\limits_{i\in\J\setminus J}\at(\fxildi-\intdeifdx)}=
\sum\limits_{i\in\J\setminus J}\abs{\fxildi-\intdeifdx}\le\ep$$

Складывая две оценки, получаем как раз $2\ep$. В~комплекснозначном
случае оценка производится отдельно для действительной и~мнимой
части. Вот так:
$$\sum_{i\in\J}\abs{\re\fxildi- \int\limits_{\dei} \re f\,dx}\le2\ep,$$
$$\sum_{i\in\J}\abs{\im\fxildi- \int\limits_{\dei} \im f\,dx}\le2\ep.$$

Итого в~сумме $4\ep$, поскольку $\abs{z}\le\abs{\re z}+\abs{\im z}$.

\prlm

\bigskip

\df Множество $E\subseteq\reals$ покрыто системой отрезков $\Omega$
в смысле Витали, если для любой точки $x\in E$ и~для любого числа
$\degz$ найдется такой отрезок $I\in\Omega$, что $x\in I$ и
$\abs{I} < \de$.

\bigskip

\tm1{первая теорема Витали}

Если ограниченное множество $E\subset\reals$ покрыто системой
отрезков $\Omega$ в~смысле Витали, то найдется не более чем счетная
система непересекающихся отрезков из $\Omega$ с~конечной суммой
длин, покрывающая почти все множество $E$.

\pr

Если множество $E$ пусто, то утверждение верно. Если $E$ не пусто,
то и~$\Omega$ тоже. По условию множество $Е$ ограничено, то есть
лежит, скажем, внутри отрезка $\seg[a+1,\,b-1]$. Для начала выкинем
из $\Omega$ все ненужные отрезки, обозначив
$\O1=\set{I\in\Omega:I\subseteq\segab}$. Тогда $\O1$ тоже не пусто и
покрывает $E$ в~смысле Витали (достаточно, например, ограничиться
отрезками, подходящими под определение при $0 < \de < 1$).
Сейчас мы продемонстрируем процедуру, выбирающую искомую систему
отрезков, не глядя на множество $E$.

\smallskip

Возьмем отрезок $I_1\in\O1$, да такой, что
$\abs{I_1}>{1\over2} \sup\limits_{I\in\O1}  \abs{I}$. Положим
$\O2=\set{I\in\O1:I\cap I_1=\varnothing}$. Если $\O2$ пусто, то
построение закончено.

\smallskip

Возьмем отрезок $I_2\in\O2$, такой, что
$\abs{I_2}>{1\over2} \sup\limits_{I\in\O2}  \abs{I}$. Положим
$\O3=\set{I\in\O2:I\cap I_2=\varnothing}$. Если $\O3$ пусто, то
построение закончено.

\smallskip

Возьмем отрезок $I_3\in\O3$, такой, что
$\abs{I_3}>{1\over2} \sup\limits_{I\in\O3}  \abs{I}$. Положим
$\O4=\set{I\in\O3:I\cap I_3=\varnothing}$. Если $\O4$ пусто, то
построение закончено.

\smallskip

Возьмем отрезок $I_4\in\O4$, такой, что
$\abs{I_4}>{1\over2} \sup\limits_{I\in\O4}  \abs{I}$. Положим
$\O5=\set{I\in\O4:I\cap I_4=\varnothing}$. Если $\O5$ пусто, то
построение закончено \dots

\smallskip

И так далее. Теперь возможны два случая.

\eject

{\sl Первый случай:} Полученная система конечна, то есть какая-то из
систем $\O{n+1}$ оказалась пустой, и~построение остановилось. Тогда
система отрезков $I_1,\dots,I_n$ покрывает вообще все множество $Е$.
Действительно, обозначим через $F$ объединение отрезков $I_i$,
$i = 1,\dots,n$, и~предположим, что есть точка $x\in E$, не
принадлежащая множеству $F$. Но $F$ замкнуто как конечное
объединение отрезков. Значит, найдется такое $\degz$, что
$B_{\de}\atx\cap F=\varnothing$. A~в эту $\de$-окрестность можно
запихнуть отрезок $J$ из системы $\O1$ длины меньше $\de$,
содержащий точку $x$. При этом этот отрезок не пересекается ни с
одним из $I_i$, a~значит, по построению, содержится в~$\O{n+1}$,
которая по предположению пустая. Противоречие возникло из-за того,
что мы предположили, что есть точка $x\in E\setminus F$.

\medskip

{\sl Второй случай:} Пусть процесс так и~не оборвался, и~мы получили
бесконечную систему отрезков $I_1,\dots,I_n,\dots\,$ Тогда их длины
стремятся к~нулю, поскольку сумма длин конечна, так как не
превосходит длины всего отрезка \ab\ (ссылаемся на следствие из
критерия Коши сходимости ряда: члены ряда должны стремиться к~нулю).

\smallskip

Обозначим через $I_i^*$ растянутый в~пять раз относительно центра
отрезок $I_i$. Покажем, что
$$E\setminus\bigcup\limits_{i=1}^n
I_i\subseteq  \bigcup\limits_{i=n+1}^\infty  I_i^*$$

Действительно, возьмем точку $x$, не покрытую первыми $n$ отрезками
(если таких точек не найдется, то не совсем ясно, что мы
доказываем). Тогда в~системе $\O1$ найдется такой отрезок $J$, что
$x\in J$ и~$J\cap I_i=\varnothing$ для $i=1,\dots,n$. Найдется такой
наименьший номер $k > n$, что $J\cap I_k\neq\varnothing$: иначе
$J\in\O{i}$ для всех $i$, и, следовательно,
$\abs{J}<\sup\limits_{I\in\O{i}}\abs{I}<2\abs{I_i}$ для всех $i$, но
длины $I_i$ стремятся к~нулю, и~получится, что $\abs{J}=0$. A~как
только $J\cap I_k\neq\varnothing$, так сразу $\abs{J}<2\abs{I_{k}}$
(потому что $k$ мы брали наименьшее, значит $J\in\O{k}$).
Следовательно, имеем $x\in J\subset I_k^*$. Таким образом, мы
действительно уложили все, что не покрыто первыми $n$ отрезками, в
объединение оставшихся отрезков, снабженных звездочками. Осталось
заметить, что
$$\sum_{i=m}^\infty\abs{I_i^*}=5\cdot\sum_{i=m}^\infty
\abs{I_i}\mathop{\to}\limits_{m\rightarrow\infty}0.$$

То есть для любого $\epgz$ мы можем найти такое число $n\in\N$, что
не более чем счетная система отрезков $\set{I_i^*}_{i=n}^\infty$,
покрывающая все непокрытые участки $E$, имеет сумму длин меньше
$\ep$, что с~нас и~требовалось.

\prtm

\bigskip

\impd{вторая теорема Витали}

Если ограниченное множество $E\subset\reals$ покрыто системой
отрезков $\Omega$ в~смысле Витали, то для любого $\epgz$ можно
выбрать на этот раз уже конечную систему непересекающихся отрезков
$\set{I_i}_{i=1}^n\subset\Omega$, что
$$\mes\at(E\setminus\bigcup\limits_{i=1}^n I_i)<\ep.$$

\pr

Пусть нам не повезло, и~сразу выбрать конечную систему, покрывающую
почти все $E$, не удается. Выберем сначала счетную систему
$\set{I_i}_{i=1}^\infty$. Тогда
$$E\setminus\bigcup\limits_{i=1}^n I_i\subseteq
\at(\bigcup\limits_{i=n+1}^\infty I_i)\cup
\at(E\setminus\bigcup\limits_{i=1}^\infty I_i) .$$

И по свойствам меры Лебега получаем

$$\mes\at(E\setminus\bigcup\limits_{i=1}^n I_i)\le
  \sum_{i=n+1}^\infty  \abs{I_i}+\mes\at
(E\setminus\bigcup\limits_{i=1}^\infty
I_i)=  \sum_{i=n+1}^\infty  \abs{I_i}
\mathop{\to}\limits_{n\rightarrow\infty}0.$$

Значит, выбором $n$ действительно можно добиться сколь угодно
хорошего покрытия.

\primp

\eject

%------------------------------------------------------------------------------------

\sectionname7{Еще несколько теорем об~интегралах.}

\subsection{Непрерывность и дифференцируемость неопределенного
интеграла.}

\bigskip

\tm1{непрерывность неопределенного интеграла Курцвейля -- Хенстока}

Если функция $f\iskurzh[a,b]$, то ее неопределенный интеграл $F\atx$
непрерывен на \ab.

\pr

Возьмем любое $\epgz$. Поскольку $f\iskurzh[a,b]$, найдется масштаб
$\dex$, такой, что для любого разбиения $\T$ Хенстока отрезка \ab,
согласованного с~$\dex$,
$$\abs{\isfT-\kurzh\intabfdx}<{\ep\over2}.$$

Теперь, пусть требуется показать, что интеграл непрерывен в~точке
$\xi\in\segab$. Возьмем положительное число
$$\gamma<\min\set{\de\at(\xi),{\ep\over2\abs{f\at(\xi)}+1}}.$$

Причем единицу мы добавили только затем, чтобы избежать вопросов про
деление на ноль. Пусть $\dx$ --- такое, что $\xi+\dx\in
B_\gamma\atx\cap\segab$. При этом мы добились того, что
$\seg[\xi,\,\xi + \dx]\subset B_{\de\,\at(\xi)}\at(\xi)$, где, как
всегда, $\seg[\xi,\,\xi+\dx]$ обозначает отрезок с~концами $\xi$ и
$\xi + \dx$. Таким образом, можно собрать разбиение $\T$ Хенстока
отрезка \ab, согласованное с~$\dex$ и~содержащее пару
$\pair(\seg[\xi,\,\xi + \dex];\xi)$. Применяя слабую лемму
Колмогорова -- Сакса -- Хенстока к~одному слагаемому
$\set{\pair(\seg[\xi,\,\xi + \dx];\xi)}\subseteq\T$, имеем
$$\abs{f\at(\xi)\dx-   \intd[\xi,\xi+\dx]   f,t}\le{\ep\over2}.$$

С другой стороны,
$$\abs{f\at(\xi)\cdot\dx}<\abs{f\at(\xi)}\cdot\gamma<
\abs{f\at(\xi)}\cdot{\ep\over2\abs{f\at(\xi)}+1}<{\ep\over2}.$$

Значит,
$$\abs{F\at(\xi+\dx)-F\at(\xi)}=\abs{\intd[\xi,\xi+\dx]   f,t}<\ep$$

Что и~требовалось доказать: для любого $\epgz$ нашлось
$\gamma > 0$, такое, что как только $0 < \abs\dx < \gamma$,
выполняется оценка $\abs{F\at(\xi+\dx)-F\at(\xi)}<\ep$.

\prtm

\bigskip

{\small \rem

Отсюда сразу следует непрерывность неопределенных интегралов Римана
и Мак-Шейна, хотя для интеграла Римана мы это уже доказали.}

\bigskip

\tm2{дифференцируемость почти всюду интеграла Курцвейля -- Хенстока}

Если $f\iskurzh[a,b]$, $F$ --- неопределенный интеграл $f$ на \ab,
то $F'\atx=f\atx$ почти всюду на \ab.

\pr

Перепишем доказываемое утверждение в~таком виде: для почти всех
$x_0\in\segab$
$$\lim\limits_{\dx\rightarrow0}\abs{{F\at(x_0+\dx)-F\at(x_0)\over\dx}-f\at(x_0)}=0.$$

\eject

Для любого $\gamma > 0$ положим
$$E_\gamma=\set{x_0\in\segab:\limsup\limits_{\dx\rightarrow0}
\abs{{F\at(x_0+\dx)-F\at(x_0)\over\dx}-f\at(x_0)}>\gamma}=$$
$$=\set{x_0\in\segab:\exists\ t_k\rightarrow0,t_k\neq0:\forall\
k\in\N\abs{{F\at(x_0+t_k)-F\at(x_0)\over{t_k}}-f\at(x_0)}>\gamma}$$

Требуется показать, что внешняя мера любого из множеств $E_\gamma$
равна нулю. Зафиксируем какую-нибудь из вышеупомянутых
последовательности $t_k$ для каждой точки множества $E_\gamma$.
Возьмем любое $\epgz$ и~найдем такой масштаб $\dex$ на \ab, что для
любого разбиения $\T$ Хенстока отрезка \ab, согласованного с~$\dex$,
выполняется оценка
$$\abs{\isfT-\kurzh\intabfdx}<{\ep\cdot\gamma\over8}.$$

Определим множество $\Omega=\set{\seg[x_0,\,x_0+t_k]:x_0\in
E_\gamma,\abs{t_k}<\de\at(x_0)}$. Как всегда, $\seg[x_0,\,x_0+t_k]$
--- это отрезок с~концами $x_0$ и~$x_0 + t_k$. Система $\Omega$ покрывает
$E_\gamma$ в~смысле Витали. Пользуясь второй теоремой Витали, найдем
конечную систему отрезков $\seg[x_j,\,x_j+t_j]\subseteq\segab$
покрывающую все $E_\gamma$, кроме множества меры меньше $\ep/2$.
Построим разбиение \Tdixif\ Хенстока отрезка \ab, согласованное с
масштабом $\dex$, и~содержащее все пары $\pair(\seg[x_j,\,x_j+t_j];
x_j)$. Применяя к~этому разбиению сильную лемму Колмогорова -- Сакса
-- Хенстока, получаем оценку
$$\sumj\abs{f\at(x_j)\abs{t_j}-   \intd[x_j,x_j+t_j]   f,x}\le
\sumi\abs{\fxildi-\intdeifdx}\le{\ep\cdot\gamma\over2}.$$

В действительнозначном случае оценка, разумеется, в~два раза лучше.
То есть видим, что
$$\sumj\abs{f\at(x_j)-{F\at(x_j+t_j)-F\at(x_j)\over
t_j}}\abs{t_j}\le{\ep\cdot\gamma\over2}.$$

По выбору $t_j$, первый множитель в~слагаемых получившейся суммы
больше $\gamma$. Но тогда
$$\sumj\gamma\abs{t_j}\le{\ep\cdot\gamma\over2}\quad
\Rightarrow\quad\sumj\abs{t_j}\le{\ep\over2}.$$

Вспоминаем, что в~$E_\gamma$ осталась еще не покрытая отрезками
часть, но ее внешняя мера меньше $\ep/2$. A~все, что покрыто, тоже
оказывается не больше $\ep/2$. Значит, для любого $\epgz$ мы
показали, что $\mes E_\gamma<\ep$. То есть $\mes E_\gamma=0$. А
значит и~мера всего множества точек, где производная $F\atx$ не
равна $f\atx$, тоже имеет меру нуль, поскольку содержится внутри
счетного объединения  множеств $E_{1\over k}$ по $k\in\N$, каждое из
которых есть множество меры нуль.

\prtm

\bigskip

{\small \rem

Из теоремы сразу следует дифференцируемость почти всюду
неопределенных интегралов Римана и~Мак-Шейна, хотя для интеграла
Римана мы это уже доказывали.}

\bigskip

\imp

Если $f\iskurzh[a,b]$, $f\atx\ge0$ почти всюду на \ab\ и
$$\kurzh\intabfdx=0,$$

то $f\atx=0$ почти всюду на \ab.

\pr

Можно считать, что $f\atx\ge0$ всюду на \ab. Тогда имеем следующее
рассуждение:
$$0\le F\atx=\intd[a,x]f,t\le\intd[a,x]f,t+\intd[x,b]f,t=\intabfdx=0.$$

Значит, $F\atx=0$ всюду на \ab, и~$F'\atx=0$ всюду на \ab, значит,
по только что доказанной теореме $f\atx=F'\atx=0$ почти всюду на
\ab.

\primp

\eject

\subsection{Неравенство Чебышева.}

\tm3{неравенство Чебышева}

Если $f$ --- действительнозначная неотрицательная функция,
интегрируемая в~смысле Курцвейля -- Хенстока на отрезке \ab, то для
любого числа $\lambda > 0$ имеем неравенство
$$\mes\set{x\in\segab:f\atx \ge \lambda}\ \le\ {1\over\lambda}\cdot\kurzh\intabfdx$$

\pr

Обозначим $\set{x\in\segab: f\atx \ge \lambda}=E$. Возьмем любое
$\epgz$ и~найдем масштаб $\dex$ на \ab, такой, что для любого
разбиения \Tdixif\ Хенстока отрезка \ab выполняется неравенство
$$\abs{\isfT-\kurzh\intabfdx}<\ep.$$

Пусть $\Omega$ --- система всех отрезков вида $\seg[x,t]$ (или
$\seg[t,x]$), где $x\in E$ и~$\abs{x-t}<\dex$. Очевидно, система
$\Omega$ покрывает множество $E$ в~смысле Витали. Согласно второй
теореме Витали, выберем конечную систему непересекающихся отрезков
$\seg[\alpha_i,\beta_i]\in\Omega$, $i=1,\dots,n$, покрывающую все
множество $E$, кроме множества меры меньше $\ep$. При этом для
удобства можно считать, что отрезки расположены по порядку, то есть
$a \le \alpha_1 < \beta_1 < \alpha_2 < \beta_2 < \cdots
 < \beta_n \le b$. Обозначим через $\xii$ тот конец отрезка
$\seg[\alpha_i,\beta_i]$, который лежит в~множестве $E$. На еще не
покрытых невырожденных отрезках
$\seg[a,\alpha_1],\seg[\beta_1,\alpha_2],\dots,\seg[\beta_n,b]$
построим разбиение Хенстока,согласованное с~масштабом $\dex$, и
пусть $\T$ --- разбиение, состоящее из разбиений этих отрезков и~пар
$\pair(\seg[\alpha_i,\beta_i];\xii)$, $i=1,\dots,n$. Тогда $\T$
согласовано с~масштабом $\dex$, и, применяя слабую лемму Колмогорова
-- Сакса -- Хенстока, получаем, что
$$\abs{\sum_{i=1}^n\at({f\at(\xii)\at(\beta_i - \alpha_i)-
\kurzh\intd[\alpha_i,\beta_i]f,x})}\le\ep.$$

Вспоминая, что функция $f$ у~нас неотрицательная вместе со~всеми ее
интегральчиками, демонстрируем следующую достаточно слабую оценку:
$$0\le\sum_{i=1}^n f\at(\xii)\at(\beta_i - \alpha_i)\le
\sum_{i=1}^n\kurzh\intd[\alpha_i,\beta_i]f,x+\ep\le\kurzh\intabfdx+\ep$$

А поскольку все $\xii\in E$, все $f\at(\xii)\ge\lambda$. Отсюда
получаем, что
$$0\le\lambda\sum_{i=1}^n\at(\beta_i - \alpha_i)\le\kurzh\intabfdx+\ep,$$
$$\sum_{i=1}^n\at(\beta_i - \alpha_i)\le
{1\over\lambda}\cdot\kurzh\intabfdx+{\ep\over\lambda}.$$

Значит, мера множества $E$, состоящего из отрезков
$\seg[\alpha_i,\beta_i]$ и~непокрытого множества меры меньше $\ep$,
имеет меру
$$\mes E<{1\over\lambda}\cdot\kurzh\intabfdx+
\ep\cdot\at({1\over\lambda}+1).$$

В силу произвольности выбора числа $\epgz$ получаем как раз
доказываемое утверждение. Заметим, что здесь нас никто не заставляет
получать в~конце оценки ровно $\ep$. A~еще заметим, что строгое
неравенство заменяется на нестрогое.

\prtm

\subsection{Абсолютность интеграла Мак-Шейна.}

\tm4{об абсолютности интеграла Мак-Шейна}

Если функция $f$ интегрируема по Мак-Шейну на отрезке \ab, то и
функция $\abs{f}$ тоже интегрируема на отрезке \ab\ по Мак-Шейну и
модуль интеграла от $f$ не превосходит интеграла от $\abs{f}$.

\eject

\pr

Возьмем любое $\epgz$ и~найдем масштаб $\dex$ на \ab, такой, что для
любого разбиения \Tdixif\ Мак-Шейна отрезка \ab\ выполняется
неравенство
$$\abs{\isfT-\macsh\intabfdx}<{\ep\over9}$$

Проверим критерий Коши. Возьмем два отмеченных разбиения \Tdixif\ и
\Tdjxjf, согласованных с~масштабом $\dex$, и~соберем из них два
новых разбиения:

$$\Tt=\set{\pair(\dei\cap\dej;\xii):\dei\cap\dej\hbox{ --- невырожденный отрезок}},$$
$$\Tt'=\set{\pair(\dei\cap\dej;\xij):\dei\cap\dej\hbox{ --- невырожденный отрезок}}.$$

Здесь мы вспомнили старую лемму о~том, что это действительно будут
разбиения отрезка \ab. И~именно здесь используется, что интеграл
понимается в~смысле Мак-Шейна: в~новоиспеченных разбиениях точки
могут выскочить за отрезки. Разбиения $\Tt$ и~$\Tt'$, очевидно,
согласованы с~масштабом $\dex$, поэтому можно применить к ним
сильную лемму Колмогорова -- Сакса -- Хенстока:
$$\sumi\sumj\abs{f\at(\xii)\abs{\dei\cap\dej}-\macsh
    \int\limits_{\dei\cap\dej}     f\,dx}\le{4\ep\over9},$$
$$\sumi\sumj\abs{f\at(\xij)\abs{\dei\cap\dej}-\macsh
    \int\limits_{\dei\cap\dej}     f\,dx}\le{4\ep\over9}.$$

Приступаем к~оценке разности интегральных сумм от функции $\abs{f}$.
Нам понадобится известное неравенство о~том, что
$\abs{\abs{a}-\abs{b}}\le\abs{a-b}$. Оно становится очевидным, если
на комплексной плоскости нарисовать два круга с~центрами в~нуле и
радиусами $\abs{a}$ и~$\abs{b}$: тогда справа будет стоять
расстояние между двумя точками с~разных окружностей, a~слева ---
минимальное из таких расстояний, a~именно разность радиусов
окружностей. Итак,

$$\abs{\isfT-\isfTp}=\abs{\sumi\abs{f\at(\xii)}\ldei-
\sumj\abs{f\at(\xij)}\ldej}=\abs{\sumi\sumj\abs{f\at(\xii)}\abs{\dei\cap\dej}-
\sumj\sumi\abs{f\at(\xij)}\abs{\dei\cap\dej}}=$$
$$=\abs{\sumi\sumj\at({\abs{f\at(\xii)}-
\abs{f\at(\xij)}})\cdot\abs{\dei\cap\dej}}\le
\sumi\sumj\abs{\abs{f\at(\xii)}-\abs{f\at(\xij)}}
\cdot\abs{\dei\cap\dej}\le\sumi\sumj\abs{f\at(\xii)-f\at(\xij)}\cdot\abs{\dei\cap\dej}=$$
$$=\sumi\sumj\abs{f\at(\xii)\abs{\dei\cap\dej}-
f\at(\xij)\abs{\dei\cap\dej}}\le\sumi\sumj\abs{f\at(\xii)\abs{\dei\cap\dej}-\macsh
    \int\limits_{\dei\cap\dej}     f\,dx}+$$
$$+\sumi\sumj\abs{\macsh
    \int\limits_{\dei\cap\dej}     f\,dx-
f\at(\xij)\abs{\dei\cap\dej}}\le{4\ep\over9}+{4\ep\over9}<\ep$$

Таким образом, критерий Коши выполняется, и~интегрируемость
доказана. Неравенство на интегралы доказывается так же, как и~для
интегралов Римана и~вообще по любой базе:
$$\abs{\macsh\intabfdx}=\lim\limits_{\basemacsh}\abs\isfT\le
\lim\limits_{\basemacsh}\isum(\abs f,\T)=\macsh\intab\abs{f},x$$

\prtm

\bigskip

\ex Функция $f\atx$, которая равна $\hr{-1}^{k+1}2^k/k$ при
$x\in\hrs{2^{-k},2^{-k+1}}$, интегрируема по Курцвейлю~-- Хенстоку
на $\seg[0,1]$, и~интеграл равен
$\sum\limits_{k=1}^{\infty}\hr{-1}^{k+1} /k$. Это доказывается
через непрерывность неопределенного интеграла и~формулу Ньютона --
Лейбница. Однако из только что доказанной теоремы следует, что
интеграла в~смысле Мак-Шейна не существует, поскольку ряд
$\sum\limits_{k=1}^{\infty}1/k$ расходится.

\eject

%------------------------------------------------------------------------------------

\sectionname8{Интегралы Стилтьеса.}

Пусть действительнозначные или комплекснозначные функции $f$ и~$g$
определены на отрезке \ab.

\bigskip

\df Интегральной суммой функции $f$ по функции $g$, соответствующей
отмеченному разбиению \Tdixif, называется выражение
$$\isfdgT=\sumi\fxigdi$$

\bigskip

\df Функция $f$ интегрируема на \ab\ по функции $g$ в~смысле Римана
-- Стилтьеса и~$I$ --- ее интеграл, если для любого $\epgz$ найдется
такое число $\degz$, что для любого разбиения \Tdixif\ Хенстока
отрезка \ab\ мельче $\de$ выполняется оценка $\abs{\isfdgT-I}<\ep$.

В этом случае пишут $$\rimanst\intabfdg=I.$$

\bigskip

\df Функция $f$ интегрируема на \ab\ по функции $g$ в~смысле
Мак-Шейна -- Стилтьеса и~$I$ --- ее интеграл, если для любого
$\epgz$ найдется такой масштаб $\dex$ на \ab, что для любого
разбиения \Tdixif\ Мак-Шейна отрезка \ab, согласованного с~$\dex$,
выполняется оценка $\abs{\isfdgT-I}<\ep$.

В этом случае пишут $$\macshst\intabfdg=I.$$

\bigskip

\df Функция $f$ интегрируема на \ab\ по функции $g$ в~смысле Римана
-- Стилтьеса и~$I$ --- ее интеграл, если для любого $\epgz$ найдется
такой масштаб $\dex$ на \ab, что для любого разбиения \Tdixif\
Хенстока отрезка \ab, согласованного с~$\dex$, выполняется оценка
$\abs{\isfdgT-I}<\ep$.

В этом случае пишут $$\kurzhst\intabfdg=I.$$

\bigskip

Ясно, что интегралы Римана, Мак-Шейна и~Курцвейля -- Хенстока
являются частными случаями соответствующих интегралов Стилтьеса при
$g\atx\equiv x$. Также ясно, что
$$\rimanst\intabfdg=\lim_{\baseriman}\isfdgT,$$
$$\macshst\intabfdg=\lim_{\basemacsh}\isfdgT,$$
$$\kurzhst\intabfdg=\lim_{\basekurzh}\isfdgT.$$

Обратите внимание на то, что никаких новых баз мы не вводим.
Интегралы Стилтьеса отличаются от обычных только структурой
интегральной суммы, a~предел берется по тем же самым базам.

\eject

Из этих утверждений сразу же выводятся знакомые свойства.

\bigskip

\utn1{о взаимосвязи интегралов Стилтьеса}

Если функция $f$ интегрируема по функции $g$ в~смысле Римана --
Стилтьеса или Мак-Шейна -- Стилтьеса, то $f$ интегрируема по $g$ и в
смысле Курцвейля -- Хенстока -- Стилтьеса с~тем же значением
интеграла.

\pr

Здесь, как и~раньше, все следует из свойств пределов по базам.
Заметим, что и~тут мы не утверждаем, что из интегрируемости по
Риману -- Стилтьеса следует интегрируемость по Мак-Шейну --
Стилтьеса.

\prut

\bigskip

\utn2{о линейности по функциям}

Если функция $f$ интегрируема по функции $g$ в~одном из трех смыслов
(Стилтьеса, разумеется; дальше не будем упоминать явно, если и~так
понятно), то для любого числа $\lambda$ функция $\lambda f$
интегрируема по $g$ в~том же смысле, a~так же $f$ интегрируема по
$\lambda g$ в~том же смысле и~выполняется равенство
$$\intab{\lambda f},g=\intab f,{\lambda g}=\lambda\cdot\intabfdg.$$

Если функции $f_1$ и~$f_2$ интегрируемы по $g$ в~одном из трех
смыслов, то функция $f_1 + f_2$ тоже интегрируема по $g$ в~этом
смысле и~выполняется равенство $$\intab\at(f_1+f_2),g=\intab
{f_1},g+\intab {f_2},g.$$

Если функция $f$ интегрируема по функциям $g_1$ и~$g_2$ в~одном из
трех смыслов, то $f$ интегрируема и~по $g_1 + g_2$ в~этом смысле и
выполняется равенство $$\intab f,{\,\at(g_1+g_2)}=\intab
f,{g_1}+\intab f,{g_2}.$$

\pr

Все, как обычно, следует из \lq\lq билинейности\rq\rq\ интегральной
суммы по обоим функциям:
$$\lim_\base\isum(\lambda fdg,\T)=\lim_\base\isum(fd\lambda
g,\T)=\lambda\cdot\lim_\base\isfdgT,$$
$$\lim_\base\isum(\,\at(f_1+f_2)dg,\T)=\lim_\base\isum(f_1dg,\T)+
\lim_\base\isum(f_2dg,\T),$$
$$\lim_\base\isum(fd\,\at(g_1+g_2),\T)=\lim_\base\isum(fdg_1,\T)+
\lim_\base\isum(fdg_2,\T).$$

\prut

\bigskip

\utn3{о сохранении неравенств}

Если $f\atx\le h\atx$ на отрезке \ab, $g$ -- неубывающая функция на
\ab, и~функции $f$ и~$h$ интегрируемы по функции $g$ на \ab,
возможно, в~разных смыслах, то выполняется неравенство $$\intab
f,g\le\intab h,g.$$

\pr

$$
\intab f,g =
\lim_{\basekurzh}\isum(fdg,\T)\le\lim_{\basekurzh}\isum(hdg,\T)=\intab
h,g$$

Как и~в~прошлый раз, все сводится к~интегралу Курцвейля -- Хенстока
-- Стилтьеса. Условие, что функция $g$ --- неубывающая, нужно, когда
утверждаем, что $\isum(fdg,\T)\le\isum(hdg,\T)$.

\prut

\eject

\utn4{критерий Коши}

\smallskip

Функция $f$ интегрируема по функции $g$ на отрезке \ab\ в~смысле
Римана -- Стилтьеса тогда и~только тогда, когда $f$ и~$g$ определены
на \ab\ и~для любого $\epgz$ найдется такое число $\degz$, что для
любых двух разбиений $\T$ и~$\T'$ Хенстока отрезка \ab\ мельче $\de$
выполняется оценка $\abs{\isum(fdg,\T)-\isum(fdg,\T')}<\ep$.

\smallskip

Функция $f$ интегрируема по функции $g$  на отрезке \ab\ в~смысле
Мак-Шейна -- Стилтьеса тогда и~только тогда, когда $f$ и~$g$
определены на \ab\ и~для любого $\epgz$ найдется такой масштаб
$\dex$ на \ab, что для любых двух разбиений $\T$ и~$\T'$ Мак-Шейна
отрезка \ab, согласованных с~$\dex$, выполняется оценка
$\abs{\isum(fdg,\T)-\isum(fdg,\T')}<\ep$.

\smallskip

Функция $f$ интегрируема по функции $g$  на отрезке \ab\ в~смысле
Курцвейля -- Хенстока -- Стилтьеса тогда и~только тогда, когда $f$ и
$g$ определены на \ab\ и~для любого $\epgz$ найдется такой масштаб
$\dex$ на \ab, что для любых двух разбиений $\T$ и~$\T'$ Хенстока
отрезка \ab, согласованных с~$\dex$, выполняется оценка
$\abs{\isum(fdg,\T)-\isum(fdg,\T')}<\ep$.

\pr

Ну тут доказывать нечего, мы просто привели критерий Коши
существования пределов по нашим базам в~соответствие с~действующими
определениями.

\prut

\bigskip

\utn5{интегрируемость на подотрезках}

Если функция $f$ интегрируема по функции $g$ на отрезке \ab\ в
каком-либо смысле, то $f$ интегрируема по $g$ и~на любом подотрезке
$\seg[a',b']\subseteq\segab$ в~том же смысле.

\pr

В точности повторяет доказательство для обычных интегралов, с
заменой $\isfT$ на $\isfdgT$ (проверьте!).

\prut

\bigskip

\utn6{аддитивность по отрезку}

Если функция $f$ интегрируема по функции $g$ в~смысле Римана --
Стилтьеса на отрезке $\seg[a,b]$, на отрезке $\seg[b,c]$, и~еще на
отрезке $\seg[a,c]$, то выполняется равенство
$$\intd[a,c]f,g=\at(\int\limits_a^b+\int\limits_b^c)f\,dg.$$

Если функция $f$ интегрируема по функции $g$ на отрезке $\seg[a,b]$
и на отрезке $\seg[b,c]$ в~смысле Мак-Шейна -- Стилтьеса или
Курцвейля -- Хенстока -- Стилтьеса, то $f$ интегрируема по $g$ на
отрезке $\seg[a,c]$ в~том же смысле и~выполняется то же равенство.

\pr

Для интегралов Мак-Шейна -- Стилтьеса и~Курцвейля -- Хенстока --
Стилтьеса доказательство в~точности повторяет доказательство для
обычных интегралов. A~для интеграла Римана -- Стилтьеса
доказательство не проходит: например, если
$f\atx=\chi_{\hrs{0,1}}\atx$ и~$g\atx=\chi_{\hs{0,1}}\atx$, то
интеграл от $f$ по $g$ существует как на $\hs{-1,0}$, так и~на
$\hs{0, 1}$, но не на $\hs{-1,1}$. Так что формулировка становится
слабее, a~доказательство проще.

\smallskip

Обозначим через $\base^b$ базу, элементы которой $B^b_{ \de}$ (где
$\degz$) состоят только из тех разбиений Хенстока соответствующего
отрезка мельче $\de$, у~которых точка $b$ не является внутренней
точкой ни одного из отрезков разбиения. Очевидно, что это
действительно база, и~что если интеграл существует по базе Римана,
то он существует и~по $\base^b$, и~значение интегралов совпадают.
Теперь пусть $\T$ --- любое разбиение отрезка $\seg[a,c]$, такое,
что точка $b$ не является внутренней точкой ни одного из отрезков
разбиения. В~этом случае разбиение $\T$ распадается в~объединение
разбиений $\T_1\sqcup\T_2$, где

$$\T_1=\set{\dixi\in\T:\dei\subseteq\seg[a,b]},$$
$$\T_2=\set{\dixi\in\T:\dei\subseteq\seg[b,c]}.$$

Тогда

$$\isum(fdg,\T_1)\mathop{\to}\limits_{\base^b}\rimanst\intd[a,b]f,g,
\qquad\isum(fdg,\T_2)\mathop{\to}\limits_{\base^b}\rimanst\intd[b,c]f,g,$$
$$\isum(fdg,\T)\mathop{\to}\limits_{\base^b}\rimanst\intd[a,c]f,g.$$

Значит, переходя к~пределам по базе $\base^b$ в~равенстве
$\isum(f,\T)=\isum(f,\T_1)+\isum(f,\T_2)$, получим то что надо.

\prut

\eject

%------------------------------------------------------------------------------------

\sectionname9{Вариация. Функции ограниченной вариации.}

\subsection{Два определения вариации.}

\dfn1 Пусть $E$ --- множество на $\reals$, $D=\set{a_i}_{i=0}^n$ ---
любой конечный набор точек из $E$, занумерованный в~порядке
возрастания: $a_0 < a_1 <\cdots<~ a_n$. Если функция $f$
определена на $E$, то вариацией $f$ на $E$ называется величина
$$\var_E f=\sup_{D\subseteq E}\sum_{i=1}^n\abs{f\at(\seg[a_{i-1},a_i])}=
\sup_{D\subseteq E}\sum_{i=1}^n\abs{f\at(a_i)-f\at(a_{i-1})}$$

\dfn2 Пусть $E$ --- множество на $\reals$, $\Omega=\set{I_i}$ --- не
более чем счетная система неперекрывающихся отрезков с~концами в
$E$. Если функция $f$ определена на $E$, то вариацией $f$ на $E$
называется величина
$$\var_E f=\sup_\Omega\sumi\abs{f\at(I_i)}$$

\bigskip

\eqdf

\smallskip

Временно будем обозначать вариации в~смысле различных определений
через, соответственно,
$$\var_E{}^{ (1)}f\quad\hbox{и}\quad\var_E{}^{ (2)}f.$$

\smallskip

Во-первых, если дано какое-то $D=\set{a_i}_{i=0}^n$, то обозначим
$I_i=\seg[a_{i-1},a_i]$. Тогда станет ясно, что
$$\sum_{i=1}^n\abs{f\at(\seg[a_{i-1},a_i])}=\sumi\abs{f\at(I_i)}\le\var_E{}^{ (2)}f.$$

Поэтому то же выполнено и~для точной верхней грани сумм, то есть
$$\var_E{}^{ (1)}f\le\var_E{}^{ (2)}f.$$

Во-вторых: сначала рассмотрим случай, когда система
$\Omega=\set{I_i}$ из второго определения конечна. Пусть
$\set{a_i}_{i=0}^n$ --- концы этих отрезков, расположенные как
следует, по возрастанию. Тогда среди отрезков $\seg[a_{i-1},a_i]$
содержатся все отрезки $I_i$, поэтому
$$\sumi\abs{f\at(I_i)}\le\sum_{i=1}^n\abs{f\at(\seg[a_{i-1},a_i])}\le
\var_E{}^{ (1)}f.$$

Но раз это неравенство выполнено для конечных сумм, то оно выполнено
и для бесконечных, поскольку при переходе к~пределу неравенства
сохраняются. A~теперь, переходя к~точной верхней грани, получаем
требуемое утверждение
$$\var_E{}^{ (2)}f\le\var_E{}^{ (1)}f.$$

\preqdf

\bigskip

\df Если вариация функции $f$ на множестве $E$ конечна, то функцию
$f$ называют функцией ограниченной вариации, или VB-функцией, на $E$
и пишут $f\isvb\at(E)$.

\bigskip

\df
$\var_a^bf=\var_{\seg[a,\,b]}f$.\quad$\var_b^af=-\var_a^bf$.\quad
$\var_a^af=0$.

\bigskip

\ex функция $f\atx=x\sin\at(1/x)$, $f\at(0)=0$, непрерывная на
отрезке $\seg[0,1]$, не является на нем функцией ограниченной
вариации --- несложно указать счетную систему отрезков, приращения
на которых составляют расходящийся ряд $\sum\limits_{k\in\N}{1/k}$.

\eject

\subsection{Свойства вариации.}

\uts

\smallskip

Пусть дано множество $E\subseteq\reals$, функции $f$ и~$g$
определены на $E$.

\smallskip
\smallskip

1. Если $H\subseteq E$, то $\var_Hf\le\var_Ef$.

\smallskip

2. Если $\lambda$ --- любое число, то $\var_E\at(\lambda
f)=\abs{\lambda}\var_Ef$ (где $0\cdot\infty=0$).

\smallskip

3. $\var_E\at(f+g)\le\var_Ef+\var_Eg$.

\smallskip

4. Если $f\isvb\at(E)$, то $f$ ограничена на $E$.

\smallskip
\smallskip

5. Если $f,g\isvb\at(E)$, то и~$f \cdot g\isvb\at(E)$ и
выполняется неравенство
$$\var_E\,\at(f \cdot g)\le\sup_E\abs{f}\cdot\var_Eg+
\sup_E\abs{g}\cdot\var_Ef.$$

6. Если $f\isvb\at(E)$, $\inf\limits_E\abs{f}=\gamma>0$, то
${1/f} \isvb\at(E)$ и
$$\var_E{1\over f}\le{1\over\gamma^2}\var_Ef.$$

7. Если $f\isvb\at(E)$, $\psi\islip\at({f\at(E)})$ с~константой $C$,
то $\psi\at(f)\isvb\at(E)$ и~$$\var_E\psi\at(f)\le C\var_Ef.$$

8. Если $f\isvb\at(\seg[a,b])$, $f\isvb\at(\seg[b,c])$, то
$f\isvb\at(\seg[a,c])$ и~$$\var_a^cf=\var_a^bf+\var_b^cf.$$

\prs

\smallskip

Первые два утверждения ну совсем уж очевидны, поэтому оставим их в
качестве упражнения и~начнем доказывать третье.

\smallskip

3. Воспользуемся вторым определением вариации.
$$\abs{\,\at(f+g)\at(I_i)}=\abs{f\at(I_i)+g\at(I_i)}\le\abs{f\at(I_i)}+\abs{g\at(I_i)},$$
$$\sumi\abs{\,\at(f+g)\at(I_i)}\le\sumi\abs{f\at(I_i)}+\sumi\abs{g\at(I_i)}\le\var_Ef+\var_Eg,$$

откуда наше утверждение и~следует.

\smallskip

4. Пусть $x,x_0 \in E$ (а если $E$ пусто, то, очевидно, $f$
ограничена на $E$), тогда имеем оценку
$$\abs{f\atx}\le\abs{f\atx-f\at(x_0)}+\abs{f\at(x_0)}\le\var_Ef+\abs{f\at(x_0)}.$$


5. Снова пользуемся вторым определением вариации, a~так же четвертым
утверждением. Оценим приращение функции $f \cdot g$ на отрезке
$I=\seg[\alpha,\beta]$:
$$\abs{\,\at(f \cdot g)\at(I)}
=\abs{f\at(\beta)g\at(\beta)-f\at(\alpha)g\at(\alpha)}\le$$
$$\le\abs{f\at(\beta)\at({g\at(\beta)-g\at(\alpha)})}+
\abs{g\at(\alpha)\at({f\at(\beta)-f\at(\alpha)})}\le$$
$$\le\sup_E\abs{f}\cdot\abs{g\at(I)}+
\sup_E\abs{g}\cdot\abs{f\at(I)}.$$

Теперь воспользуемся этой оценкой для оценки самой вариации:
$$\sumi\,\at(f \cdot g)\at(I_i)\le\sup_E\abs{f}\cdot\var_Eg+\sup_E\abs{g}\cdot\var_Ef,$$

и, как всегда, при переходе к~верхним граням неравенство сохранится.

\smallskip

6. Для любого отрезка $I=\seg[\alpha,\beta]$ снова имеем оценку:
$$\abs{{1\over f}\at(I)}=\abs{{1\over f\at(\beta)}-
{1\over f\at(\alpha)}}=\abs{f\at(\alpha)- f\at(\beta)\over
f\at(\alpha)f\at(\beta)}\le{\abs{f\at(I)}\over\gamma^2}.$$

Значит, аналогично можно оценить и~суммы, a~значит и~вариацию:
$$\sumi\abs{{1\over
f}\at(I_i)}\le{1\over\gamma^2}\sumi\abs{f\at(I_i)}\le{1\over\gamma^2}\var_Ef.$$

\eject

7. Опять оцениваем приращение на $I=\seg[\alpha,\beta]$:
$$\abs{\psi\at(f)\at(I)}=\abs{\psi\at({f\at(\beta)})-\psi\at({f\at(\alpha)})}
\le C\abs{f\at(\beta)-f\at(\alpha)}=C\abs{f\at(I)}.$$

Значит, на суммы приращений имеем оценку
$$\sumi\abs{\psi\at(f)\at(I)}\le C\sumi\abs{f\at(I_i)}\le C\var_Ef.$$

8. Пусть $\set{I_i}$ --- какая-нибудь система неперекрывающихся
отрезков на $\seg[a,b]$, $\set{I'_j}$ --- система неперекрывающихся
отрезков на $\seg[b,c]$. Тогда $\set{I_i}\cup\set{I'_j}$ --- система
неперекрывающихся отрезков на $\seg[a,c]$, и~получаем первое
неравенство:
$$\sumi\abs{f\at(I_i)}+\sumj\abs{f\at(I'_j)}\le\var_a^cf,$$
$$\var_a^bf+\sumj\abs{f\at(I'_j)}\le\var_a^cf,$$
$$\var_a^bf+\var_b^cf\le\var_a^cf.$$

Теперь, пусть $\set{I_i}$ --- какая-нибудь система неперекрывающихся
отрезков на $\seg[a,c]$. Если в~ней нет отрезка, для которого $b$
--- внутренняя точка, то
$$\sumi\abs{f\at(I_i)}=    \sum_{I_i\subseteq\seg[a ,\,b]}   \abs{f\at(I_i)}+
    \sum_{I_i\subseteq\seg[b ,\,c]}   \abs{f\at(I_i)}\le\var_a^bf+\var_b^cf.$$

Если же $b$ оказалась внутри, скажем, отрезка $I_j$, то заменим его
на два отрезка $I'_j=I_j\cap\seg[a,b]$ и~$I''_j=I_j\cap\seg[b,c]$.
При этом имеет место неравенство
$$\abs{f\at(I_j)}=\abs{f\at(I'_j)+f\at(I''_j)}\le\abs{f\at(I'_j)}+\abs{f\at(I''_j)},$$

так что при такой замене сумма модулей приращений может только
увеличиться. Но, поскольку мы при этом пришли к~уже рассмотренному
случаю, она все равно не превзойдет того чего не надо:
$$\sumi\abs{f\at(I_i)}\le\sum_{i\neq
j}\abs{f\at(I_i)}+\abs{f\at(I'_j)}+\abs{f\at(I''_j)}\le\var_a^bf+\var_b^cf$$

Что и~требовалось: мы доказали неравенства в~обе стороны, a~значит и
доказываемое утверждение.

\pruts

\bigskip

{\small \rem

Последнее утверждение остается в~силе даже если не выполняется
соотношение $a<b<c$. Доказывается это перебором различных взаимных
расположений точек $a$, $b$ и~$c$.}

\subsection{Представление VB-функций.}

\lm

Если функция $f$ определена на промежутке $I$, точка $x_0\in I$, то
$\var_{\hs{x_0,x}} f$ --- неубывающая функция на $I$. Если функция
$f$ к тому же действительнозначная, то неубывающими являются также
функции $\var_{\hs{x_0,x}} f+f\atx$ и~$\var_{\hs{x_0,x}} f-f\atx$.

\pr

Если даны две точки $x$ и~$y$ из промежутка $I$, $x<y$, то,
очевидно,
$$\var_{x_0}^yf-\var_{x_0}^xf=\var_x^yf\ge\abs{f\at(y)-f\at(x)}\ge0.$$

При этом в~действительнозначном случае для, например, первой
функции, имеем аналогичную оценку
$$\var_{x_0}^yf+f\at(y)-\var_{x_0}^xf-f\at(x)\ge
f\at(y)-f\at(x)+\abs{f\at(y)-f\at(x)}\ge0.$$

\prlm

\eject

\bigskip

\ut

Действительнозначная функция $f$, определенная на интервале $I$,
является на нем функцией ограниченной вариации тогда и~только тогда,
когда $f$ представляется на $I$ в~виде разности двух неубывающих
ограниченных функций. При этом если функция $f$ является-таки
функцией ограниченной вариации, то можно подобрать такие неубывающие
ограниченные функции $f_1$ и~$f_2$ на $I$, что
$f\atx=f_1\atx-f_2\atx$ и~$$\var_If=\var_If_1+\var_If_2.$$

\pr

Во-первых, покажем, что неубывающие ограниченные функции являются
функциями ограниченной вариации. Для этого воспользуемся первым
определением вариации, в~котором ввиду монотонности функций можно
убрать модули:
$$\sum_{i=1}^n\,\at({f\at(a_i)-f\at(a_{i-1})})=f\at(a_n)-f\at(a_0).$$

Отсюда видно, что $$\var_If=\sup\limits_If-\inf\limits_If.$$

Значит, $f\isvb\at(I)$. Поэтому и~разность двух таких функций будет
функцией ограниченной вариации, так что в~одну сторону утверждение
доказано. Заодно мы написали явное выражение вариации для монотонных
ограниченных функций, и~в~ближайшее время нам это понадобится. Еще
можно отметить, что если $\alpha$ и~$\beta$ --- концы промежутка
$I$, то
$$\sup_If=\begin{cases}f\at(\beta),\beta\in I;\cr f\at(\beta-0),
\beta\notin I;\end{cases}\quad
\inf_If=\begin{cases}f\at(\alpha),\alpha\in I;\cr
f\at(\alpha+0),\alpha\notin I.\end{cases}$$

Теперь докажем обратное утверждение. Возможность представления
следует из леммы: если $x_0\in I$, то
$$f\atx=\,\at(\var_{x_0}^xf+f\atx)-\var_{x_0}^xf=
\var_{x_0}^xf-\,\at(\var_{x_0}^xf-f\atx).$$

А еще можно записать так:
$$f\atx={1\over2}\at(\var_{x_0}^xf+f\atx)-{1\over2}\at(\var_{x_0}^xf-f\atx).$$

Проверим, что в~последней записи сумма вариаций этих двух функций
действительно равна вариации $f$. Действительно,
$$\var_I\at(\var_{x_0}^xf\pm f\atx)=\sup_I\at(\var_{x_0}^xf\pm f\atx)-
\inf_I\at(\var_{x_0}^xf\pm f\atx)=$$
$$=\at({\var_{x_0}^{\beta\,(-\,0)}f\pm f\at({\beta\,(-\,0)})})-
\at({\var_{x_0}^{\alpha\,(+\,0)}f\pm f\at({\alpha\,(+\,0)})})=$$
$$=\var_If\pm\bigl(f\at({\beta\,(-\,0)})-f\at({\alpha\,(+\,0)})\bigr).$$

Значит, при сложении плюсминусы сократятся и~получится $\var_If$.

\prut

%------------------------------------------------------------------------------------

\eject

\sectionname{10}{И еще несколько теорем об~интегралах.}

\subsection{Интегрируемость непрерывных функций по VB-функциям.}

\lm

Если $\set{\dei}_{i=1}^n$ --- разбиение отрезка \ab, отрезок
$\seg[c,d\,]\subseteq\seg[a,b]$, то для любой функции $f$,
определенной на \ab, имеем равенство
$$f\at(\seg[c,d\,])=\sum_{i=1}^nf\at(\seg[c,d\,]\cap\dei).$$

\pr

Обозначим $\Delta=\seg[c,d\,]$ и~$\dei=\seg[a_{i-1},a_i]$. Без
ограничения общности можно считать, что $$a=a_0<a_1<\cdots<a_n=b.$$

Найдутся такие числа $k$ и~$l$, что $a_{k-1}\le c<a_k$ и
$a_{l-1}<d\le a_l$. В~этом случае невырожденными отрезками будут
(только) отрезки
$$\seg[c,a_k],\Delta_{k+1},\dots,\Delta_{l-1},\seg[a_{l-1},d\,].$$

Теперь проводим простую выкладку:
$$f\at(\Delta)=f\at(d)-f\at(c)=f\at(a_k)-f\at(c)+
\sum_{i=k+1}^{l-1}\at({f\at(a_i)-f\at(a_{i-1})})+f\at(d)-f\at(a_{l-1})=$$
$$=f\at(\Delta\cap\Delta_k)+  \sum_{i=k+1}^{l-1}
f\at(\Delta\cap\dei)+f\at(\Delta\cap\Delta_l)=\sum_{i=1}^nf\at(\Delta\cap\dei)$$

\prlm

\bigskip

\tm1{интегрируемость непрерывных функций по функциям ограниченной
вариации}

Если $f\isc\segab$, a~$g\isvb\segab$, то $f$ интегрируема по $g$ на
отрезке \ab\ в~смысле Римана -- Стилтьеса и~Мак-Шейна -- Стилтьеса,
и, следовательно, в~смысле Курцвейля -- Хенстока -- Стилтьеса,
и~имеет место оценка $$\abs{\intab
f,g}\le\max_{\seg[ a,\,b]}\abs{f\,}\cdot\var_a^bg.$$

\pr

Поскольку $f$ непрерывна на \ab, $f$ равномерно непрерывна на \ab.
Поэтому, зафиксировав любое число $\epgz$, можем взять такое число
$\degz$, чтобы для любых двух точек $x,x'\in\segab$, находящихся на
расстоянии $\abs{x-x'}<\de$, выполнялась оценка
$$\abs{f\atx-f\at(x')}<{\ep\over\var_a^bg}.$$

\smallskip

Проверим критерий Коши. Пусть \Tdixif\ и~\Tdjxjf\ --- два отмеченных
разбиения Хенстока (Мак-Шейна) мельче $\de$ (согласованных с
$\dex\equiv\de$). Оцениваем разность интегральных сумм.
$$\abs{\isfdgT-\isfdgTp}=\abs{\sumi\fxigdi-\sumj\fxjgdj}=$$
\vskip -0.2cm
$$=\abs{\sumi\sumj f\at(\xii)g\at(\dei\cap\dej)-\sumj\sumi f\at(\xij)g\at(\dei\cap\dej)}\le$$
$$\le\sumi\sumj\abs{f\at(\xii)-f\at(\xij)}\cdot\abs{g\at(\dei\cap\dej)}<
{\ep\over\var_a^bg}\cdot\sumi\sumj\abs{g\at(\dei\cap\dej)}\le\ep.$$

\eject

Здесь мы воспользовались вторым определением вариации:
$$\sumi\sumj\abs{g\at(\dei\cap\dej)}\le\var_a^bg.$$

Выполнен критерий Коши интегрируемости. Оценка на интеграл понятна:
$$\abs{\sumi\fxigdi}\le\sumi\abs{f\at(\xii)}\abs{\gdei}\le\max_{\seg[ a,\,b]}\abs{f\,}\sumi\abs{\gdei}\le\max_{\seg[ a,\,b]}\abs{f\,}\cdot\var_a^bg.$$

\prtm

\subsection{Интегрирование по частям в интеграле Римана --
Стилтьеса.}

\tm2{интегрирование по частям в интеграле Римана -- Стилтьеса}

Если функция $g$ интегрируема по функции $f$ в~смысле Римана --
Стилтьеса, то и~$f$ интегрируема по $g$ в~смысле Римана --
Стилтьеса, и~выполняется равенство
$$\rimanst\intabfdg=f\at(b)g\at(b)-f\at(a)g\at(a)-\rimanst\intab g,f.$$

\pr

Пусть $\T=\dixif_{i=1}^n$ --- разбиение отрезка \ab, где
$\xii\in\dei=\seg[a_{i-1},a_i]$, $i=1,\dots,n$, и~все расположено по
порядку: $$a=a_0\le\xi_1\le a_1\le\xi_2\le a_2\le\cdots\le
a_{n-1}\le\xi_n\le a_n=b.$$


Тогда преобразуем интегральную сумму следующим хитрым способом:
$$\isfdgT=\sum_{i=1}^nf\at(\xii)\at({g\at(a_i)-g\at(a_{i-1})})=
\sum_{i=1}^nf\at(\xii)g\at(a_i)-\sum_{i=0}^{n-1}f\at(\xi_{i+1})g\at(a_i)=$$
$$=-\sum_{i=0}^ng\at(a_i)\at({f\at(\xi_{i+1})-f\at(\xii)})+
f\at(\xi_{n+1})g\at(a_n)-f\at(\xi_0)g\at(a_0).$$

Здесь при переходе от первой строчки ко~второй мы сдвинули нумерацию
на второй сумме, a~при переходе от второй строчки к~третьей ---
прибавили и~вычли две величины, то есть добавили к~сумме, чтобы
нумерация выровнялась, a~потом вычли. Причем мы не знаем, что такое
$\xi_0$ и~$\xi_{n+1}$, но их значение на равенство не влияет,
поэтому положим $\xi_0=a$, $\xi_{n+1}=b$. Получится занятное такое
равенство
$$\isfdgT=f\at(b)g\at(b)-f\at(a)g\at(a)-\sum_{i=0}^ng\at(a_i)\at({f\at(\xi_{i+1})-f\at(\xii)}).$$

При этом, как и~раньше, все расположено по порядку:
$$a=\xi_0=a_0\le\xi_1\le a_1\le\cdots\le a_{n-1}\le\xi_n\le
a_n=\xi_{n+1}=b.$$

Поэтому, меняя местами точки и~отрезки, то есть вводя обозначение
$\Tt=\set{\pair(\seg[\xi_i,\xi_{i+1}];a_i)}$, получаем вот такое
красивое выражение:
$$\isfdgT=f\at(b)g\at(b)-f\at(a)g\at(a)-\isum(gdf,\Tt).$$

При этом имеем такую оценку: если все $\dei$ имеют длину меньше
некоторого положительного числа $\de$, то длина отрезка
$\seg[\xi_i,\xi_{i+1}]$ оказывается меньше $2\de$, ведь точки
$\xi_i$ и~$\xi_{i+1}$ лежат в~соседних отрезках. Поэтому, переходя к
пределу по базе Римана, получаем как раз требуемое утверждение, ведь
правая часть по условию стремится куда надо, значит и~левой некуда
деваться.

\prtm

\bigskip

\imp

Если $f\isvb\segab$, $g\isc\segab$, то $f$ интегрируема по $g$ в
смысле Римана -- Стилтьеса, и, следовательно, в~смысле Курцвейля --
Хенстока -- Стилтьеса,

\eject

\subsection{Cведение интеграла Римана -- Стилтьеса к~интегралу Римана.}

\tm3{сведение интеграла Римана -- Стилтьеса к~интегралу Римана}

Если функция $f$ ограничена на \ab, функции $g$ и~$f\cdot g$
интегрируемы по Риману на \ab, $G$ --- неопределенный интеграл $g$
на \ab, то существует интеграл
$$\rimanst\intab f,G=\riman\intab fg,x.$$

\pr

Возьмем любое $\epgz$. Поскольку $g\isriman[a,b]$, имеем право взять
такое число $\de_1 > 0$, что для любого разбиения \Tdixif\
Хенстока отрезка \ab\ мельче $\de_1$ получится оценка
$$\abs{\isgT-\riman\intab g,x}<{\ep\over8\cdot\sup\limits_{\seg[a,\,b]}\abs{f}}.$$

Тогда, применяя сильную лемму Колмогорова -- Сакса -- Хенстока, для
любого разбиения \Tdixif\ Хенстока отрезка \ab\ мельче $\de_1$
получим оценку $$\sumi\abs{g\at(\xii)\ldei-\riman\intdei
g,x}\le{\ep\over2\cdot\sup\limits_{\seg[a,\,b]}\abs{f}}.$$

При этом не забываем ключевое соображение о~том, что интегральчики
от $g$ на $\dei$ есть всего лишь $G\at(\dei)$.

\smallskip

Теперь зайдем с~другой стороны: поскольку $f\cdot g\isriman[a,b]$,
то можем найти такое число $\de_2 > 0$, чтобы для любого разбиения
\Tdixif\ Хенстока отрезка \ab\ мельче $\de_2$ выполнялась оценка
$$\abs{\isum(f\cdot g,\T)-\riman\intab f\cdot g,x}<{\ep\over2}.$$

Так, кажется, все готово. Берем мелкость
$\de=\min\set{\de_1,\de_2}$, и~для любого разбиения \Tdixif\
Хенстока отрезка \ab\ мельче $\de$ получаем оценку
$$\abs{\isum(fdG,\T)-\riman\intab f\cdot g,x}\le\abs{\sumi\bigl(
f\at(\xii)G\at(\dei)-f\at(\xii)g\at(\xii)\ldei\bigr)}+\abs{\sumi
f\at(\xii)g\at(\xii)\ldei-\riman\intab f\cdot g,x}.$$

Вторая сумма сразу меньше $\ep/2$. Но и~первая сумма оценивается той
же величиной:
$$\sumi\abs{f\at(\xii)}\cdot\abs{g\at(\xii)\ldei-G\at(\dei)}\le
\sup\limits_{\seg[a,\,b]}\abs{f}\cdot\sumi\abs{g\at(\xii)\ldei-G\at(\dei)}\le
\sup\limits_{\seg[a,\,b]}\abs{f}\cdot{\ep\over2\cdot
\sup\limits_{\seg[a,\,b]}\abs{f}}={\ep\over2}$$

Вот и~оказалось, что суммы $\isum(fdG,\T)$ стремятся к~интегралу от
$f\cdot g$, a~раз у~них есть предел, то он должен был бы быть равен
интегралу $f$ по $G$, значит, остается сделать вывод, что интегралы
эти равны.

\prtm

\bigskip

\impd{интегрирование по частям для интеграла Римана}

Если функции $u$ и~$v$ интегрируемы по Риману на отрезке \ab, $U$ и
$V$
--- их неопределенные интегралы, то $u\cdot V$ и~$v\cdot U$
интегрируемы по Риману на \ab, и~выполняется равенство
$$\riman\intab uV,x=U\at(b)V\at(b)-U\at(a)V\at(a)-\riman\intab
Uv,x.$$

\pr

Функция $V$ ограничена на \ab, и~даже интегрируема на нем по Риману,
поскольку она вообще на нем непрерывна. Функция $u$ интегрируема по
Риману по условию, $u\cdot V$
--- тоже интегрируема по Риману, поскольку $V$ непрерывна, a~у~нас есть критерий
Лебега. Поэтому $$\riman\intab uV,x=\rimanst\intab V,U.$$

\eject

 По
совершенно симметричным соображениям имеем второе равенство
$$\riman\intab Uv,x=\rimanst\intab U,V.$$

Тогда после замены в~требуемом утверждении интегралов Римана на
равные им интегралы Римана -- Стилтьеса получится как раз
утверждение самой первой теоремы об~интегрировании по частям.

\primp

\subsection{Сведение интеграла Курцвейля -- Хенстока к~интегралу Римана --
Стилтьеса.}

\tm4{сведение интеграла Курцвейля -- Хенстока к~интегралу Римана --
Стилтьеса}

Если функция $f$ интегрируема на \ab\ в~смысле Курцвейля --
Хенстока, $F$ --- ее неопределенный интеграл, a~функция
$g\isvb\segab$, то $f\cdot g$ тоже интегрируема по Курцвейлю --
Хенстоку на \ab, и~выполняется равенство $$\kurzh\intab
fg,x=\rimanst\intab g,F.$$

\pr

Прежде всего, отметим, что интеграл Римана -- Стилтьеса в~правой
части существует, поскольку функция $F$ непрерывная, a~$g$ ---
функция ограниченной вариации. A~тогда можно, взяв любое $\epgz$,
подобрать такое число $\degz$, что для любого разбиения \Tdixif\
Хенстока отрезка \ab\ мельче $\de$ получится оценка
$$\abs{\isum(gdF,\T)-\rimanst\intab g,F}<{\ep\over2}.$$

С другой стороны, функция $f$ интегрируема по Курцвейлю -- Хенстоку,
значит, найдется масштаб $\dex$ на \ab, такой, что для любого
разбиения \Tdixif\ Хенстока отрезка \ab, согласованного с~$\dex$,
будет
$$\abs{\isfT-\kurzh\intabfdx}<{\ep\over8\cdot\sup\limits_{\seg[a,\,b]}\abs{g}}.$$

Тогда по сильной лемме Колмогорова -- Сакса -- Хенстока для любого
разбиения \Tdixif\ Хенстока отрезка \ab, согласованного с~$\dex$,
будет
$$\sumi\abs{\fxildi-\kurzh\intdeifdx}\le{\ep\over2\cdot\sup\limits_{\seg[a,\,b]}\abs{g}}.$$

Ну и~снова оцениваем. Пусть \Tdixif\ --- отмеченное разбиение
Хенстока отрезка \ab\ мельче $\de$ и~одновременно согласованное с
масштабом $\dex$. Тогда $$\abs{\isum(fg,\T)-\rimanst\intab
g,F}\le\abs{\sumi f\at(\xii)g\at(\xii)\ldei-\sumi
g\at(\xii)F\at(\dei)}+\abs{\sumi g\at(\xii)F\at(\dei)-\rimanst\intab
g,F}<$$
$$<\sup\limits_{\seg[a,\,b]}\abs{g}\cdot\sumi\abs{\fxildi-F\at(\dei)}+{\ep\over2}<{\ep\over2}+{\ep\over2}=\ep.$$

\prtm

\bigskip

\impd{интегрирование по частям в~интеграле Курцвейля -- Хенстока}

Если функция $f\iskurzh[a,b]$, $g\isvb\segab$, то $f\cdot
g\iskurzh[a,b]$ и
$$\kurzh\intab fg,x=g\at(b)F\at(b)-g\at(a)F\at(a)-\rimanst\intab
F,g.$$

В частности, умножение на VB-функцию не влияет на интегрируемость по
Курцвейлю -- Хенстоку.

\eject

\subsection{Замена переменной под знаком интеграла.}

\tm5{замена переменной под знаком интеграла}

Пусть функция $f$ интегрируема на отрезке \ab\ в~одном из трех
смыслов, $\ph$ --- строго возрастающая непрерывно дифференцируемая
функция отрезке $\seg[\alpha,\beta]$, где $\ph\at(\alpha)=a$ и
$\ph\at(\beta)=b$. Тогда функция $f\at(\ph)\ph'$ будет интегрируема
на отрезке $\seg[\alpha,\beta]$ в~соответствующем смысле, и
выполнится равенство
$$\intabfdx=\intd[\alpha,\beta]f\at(\ph)\ph',x.$$

\pr

Зафиксируем любое $\epgz$, и~обозначим интеграл от $f$ на \ab\ через
$I$. Функция $\ph'$ по условию непрерывна на \ab. В~случае интеграла
Римана мы пользуемся даже равномерной непрерывностью функции $\ph'$,
которая следует из обычной непрерывности, a~так же ограниченностью
функции $f$, и~находим такое число $\de_1 > 0$, что как только
точки $t,s\in\seg[\alpha,\beta]$ находятся на расстоянии меньше
$\de_1$, получается оценка
$$\abs{\ph'\at(t)-\ph'\at(s)}<{\ep\over2\at(\beta-\alpha)\sup\limits_{\seg[a,\,b]}\abs{f}}.\eqno(1)$$

А в~случае интегралов Мак-Шейна и~Курцвейля -- Хенстока можно и
нужно найти такой масштаб $\de_1\atx$ на отрезке
$\seg[\alpha,\beta]$, что для любых точек
$t,s\in\seg[\alpha,\beta]$, таких, что $\abs{t-s}<\de_1\at(t)$,
выполняется оценка
$$\abs{\ph'\at(t)-\ph'\at(s)}<{\ep\over2\at(\beta-\alpha)\abs{f\at({\ph\at(t)})}}.\eqno(1')$$

Теперь воспользуемся интегрируемостью функции $f$ и~найдем такое
число $\de_2 > 0$ (масштаб $\de_2\atx$ на \ab), чтобы для любого
разбиения \Tdixif\ Хенстока (Мак-Шейна, Хенстока) отрезка \ab\
мельче $\de_2$ (согласованного с~$\de_2\atx$), выполнялась оценка
$$\abs{\isfT-I}<{\ep\over2}.\eqno(2)$$

Опять разберем отдельно два интеграла. Для интеграла Римана,
пользуясь равномерной непрерывностью на этот раз уже функции $\ph$,
найдем число такое $\degz$, меньшее $\de_1$, чтобы для любых точек
$t,s\in\seg[\alpha,\beta]$, находящихся на расстоянии меньше $\de$,
выполнялась оценка $$\abs{\ph\at(t)-\ph\at(s)}<\de_2.\eqno(3)$$

А для интегралов Мак-Шейна и~Курцвейля -- Хенстока выбираем масштаб
$\dexgz$, $\dex<\de_1\atx$, на $\seg[\alpha,\beta]$, что при
$t,s\in\seg[\alpha,\beta]$, $\abs{t-s}<\de\at(t)$, будет
$$\abs{\ph\at(t)-\ph\at(s)}<\de_2\at({\ph\at(t)}).\eqno(3')$$

Вот теперь уже рассмотрим произвольное разбиение \Tdixif\ Хенстока
(Мак-Шейна, Хенстока) отрезка $\seg[\alpha,\beta]$ мельче $\de$
(согласованное с~$\dex$). Припоминая формулу конечных приращений
Лагранжа, обнаруживаем, что найдутся такие точки $\theta_i\in\dei$,
что $\ph\at(\dei)=\ph'\at(\theta_i)\ldei$. Все, теперь все готово.

$$\abs{\isum(f\at(\ph)\ph',\T)-I}=\abs{\sumi
f\at({\ph\at(\xii)})\ph'\at(\xii)\ldei-I}\le\abs{\sumi
f\at({\ph\at(\xii)})\,\at({\ph'\at(\xii)-\ph'\at(\theta_i)})\ldei}+\abs{\sumi
f\at({\ph\at(\xii)})\ph\at(\dei)-I}.$$

Во-первых, $\abs{\xii-\theta_i}<\de<\de_1$ (или
$<\de\at(\xii)<\de_1\at(\xii)$), поэтому можно сослаться на оценки
$\at(1)$ и~$\at(1')$ и~оценить первую сумму величиной $\ep/2$. A~для
второй суммы оценки $\at(3)$ и~$\at(3')$ утверждают, что
$\set{\pair({\ph\at(\dei)};{\ph\at(\xii)})}$ --- разбиение отрезка
\ab\ Хенстока (Мак-Шейна, Хенстока) мельче $\de_2$ (согласованное с
$\de_2\atx$), значит вся вторая сумма по оценке $\at(2)$ тоже
оценивается величиной $\ep/2$, что не может не радовать.

\prtm

\bigskip

{\small \rem

Аналогичную теорему можно сформулировать и~для случая строго
убывающей функции $\ph$, при этом перед интегралом появится минус.
Примерно так: функция $f\at(-x)$ будет интегрируемой на
$\seg[-b,-a]$, $-\ph\at(t)$
--- возрастающей, $-\ph\at(\alpha)=-b$, $-\ph\at(\beta)=-a$, тогда
$f\at({-\at({-\ph\at(t)})})\ph'\at(t)$ будет интегрируема в
соответствующем смысле, и~т.д.}

\eject

\subsection{Формула Тейлора с~остаточным членом в~интегральной форме.}

\tm6{формула Тейлора с~остаточным членом в~интегральной форме}

Если функция $f$ дифференцируема $n+1$ раз на отрезке с~концами
$x_0$ и~$x$, то имеет место равенство

$$f\atx=\sum_{k=0}^n{f^{(k)}\at(x_0)\over k!}\,\at(x-x_0)^k+
{1\over n!}\cdot\kurzh\intd[x_0,x]\,\at(x-t)^nf^{(n+1)}\at(t),t.$$

\pr

Сразу заметим, что интеграл существует, поскольку $f^{(n+1)}\at(t)$
--- интегрируемая функция, a~$\,\at(x-t)^n$ --- функция ограниченной вариации.
Будем действовать по индукции. При $n=0$ утверждение верно:
$$f\atx=f\at(x_0)+f\at(x)-f\at(x_0)=f\at(x_0)+\kurzh\intd[x_0,x]f'\at(t),t.$$

Пусть утверждение верно для $n=m$. Докажем его для $n=m+1$. Для
этого проделаем такую выкладку:

$${1\over(m+1)!}\kurzh\intd[x_0,x]\,\at(x-t)^{m+1}f^{(m+2)}\at(t),t=
{1\over(m+1)!}\rimanst\intd[x_0,x]\,\at(x-t)^{m+1},f^{(m+1)}\at(t)=$$
$$={1\over(m+1)!}\at({\,\at(x-x)^{m+1}f^{(m+1)}\at(x)-\,\at(x-x_0)^{m+1}f^{(m+1)}\at(x_0)-
\rimanst\intd[x_0,x]f^{(m+1)}\at(t),\,\at(x-t)^{m+1}})=$$
$$=-{f^{(m+1)}\at(x_0)\over(m+1)!}\,\at(x-x_0)^{m+1}+{1\over m!}\riman\intd[x_0,x]\,\at(x-t)^mf^{(m+1)}\at(t),t.$$

Здесь мы сначала сводим интеграл Курцвейля -- Хенстока к~интегралу
Римана -- Стилтьеса, потом интегрируем по частям, и~в~конце сводим
интеграл Римана -- Стилтьеса к~интегралу Римана.

\smallskip

Заменяя интеграл Римана на равный ему интеграл Курцвейля --
Хенстока, полученное равенство можно переписать так:

$${1\over m!}\kurzh\intd[x_0,x]\,\at(x-t)^mf^{(m+1)}\at(t),t=
{f^{(m+1)}\at(x_0)\over(m+1)!}\,\at(x-x_0)^{m+1}+
{1\over(m+1)!}\kurzh\intd[x_0,x]\,\at(x-t)^{m+1}f^{(m+2)}\at(t),t.$$

Но несложно видеть, что именно это и~требовалось доказать:
подставляя выражение слева в~формулу для $n=m$, как раз получаем
формулу для $n=m+1$.

\prtm

\subsection{Теоремы о среднем.}

\tm7{первая теорема о~среднем}

Пусть действительнозначная функция $f$ ограничена на отрезке \ab,
функция $g$ неотрицательна на \ab, и~функции $g$ и~$f\cdot g$
интегрируемы на \ab\ в~одном из трех смыслов. Тогда найдется число
$\mu\in\seg[{\smash{\inf\limits_{\seg[a,\,b]}f}\vphantom{\inf}},\
{\smash{\sup\limits_{\seg[a,\,b]}f}\vphantom{\sup}}]$:
$$\intab fg,x=\mu\cdot\intabgdx.$$

\pr

Для начала напишем для всех $x\in\segab$ бесспорную оценку
$\inf\limits_{\seg[a,\,b]}f\le f\atx\le\sup\limits_{\seg[a,\,b]}f$.

Теперь умножим ее на $g\atx$ (имеем право, поскольку $g$
неотрицательная), и~возьмем интеграл от $a$ до $b$ (имеем право,
поскольку при интегрировании неравенства сохраняются). Получится
примерно следующее:
$$\inf\limits_{\seg[a,\,b]}f\cdot\intabgdx\le\intab fg,
x\le\sup\limits_{\seg[a,\,b]}f\cdot\intabgdx.$$

\eject

Тогда если интеграл от $g$ равен нулю, то любое $\mu$ нам подойдет,
поскольку, оказывается, в~этом случае интеграл от $f\cdot g$ тоже
равен нулю. Если же интеграл от $g$ больше нуля, то $\mu$ выбирается
единственным образом, поскольку можно поделить неравенство на
интеграл от $g$.

\prtm

\bigskip

\tm8{первая теорема о~среднем}

Пусть действительнозначная функция $f$ ограничена на отрезке \ab,
функция $G$ --- неубывающая на \ab, и~$f$ интегрируема по $G$ на
\ab\ в~одном из трех смыслов. Тогда найдется число
$\mu\in\seg[{\smash{\inf\limits_{\seg[a,\,b]}f}\vphantom{\inf}},\
{\smash{\sup\limits_{\seg[a,\,b]}f}\vphantom{\sup}}]$:
$$\intab f,G=\mu\cdot\intab ,G=\mu\,\at({G\at(b)-G\at(a)}).$$

\pr

Снова начинаем с~простых и~понятных вещей:

$$\inf\limits_{\seg[a,\,b]}f\le f\atx\le\sup\limits_{\seg[a,\,b]}f.$$

Поскольку функция $G$ --- неубывающая, имеем право проинтегрировать
по ней полученное неравенство:

$$\inf\limits_{\seg[a,\,b]}f\cdot\intab ,G\le\intab f,G\le\sup\limits_{\seg[a,\,b]}f\cdot\intab ,G.$$

Как и~в~прошлый раз, если интеграл единицы по $dG$ равен нулю, то
берем любое $\mu$, a~если он положителен, то можно неравенство на
него поделить --- и~найти $\mu$ единственным образом.

\prtm

\bigskip

{\small \rem

Если функция $f$ непрерывна, то в обоих теоремах на отрезке \ab\
можно найти такую точку $c$, что $\mu=f\at(c)$.}

\bigskip

\tm9{вторая теорема о~среднем}

Пусть функция $f\iskurzh[a,b]$, $g$
--- монотонная функция на \ab, тогда функция $f \cdot g$ интегрируема на \ab\
в смысле Курцвейля -- Хенстока , и~найдется точка
$\xi\in\segab$:$$\kurzh\intab
fg,x=g\at(a)\cdot\kurzh\intd[a,\xi]f,x+g\at(b)\cdot\kurzh\intd[\xi,b]f,x.$$

При этом если функция $g$ невозрастающая и~неотрицательная на \ab,
то найдется точка
$\zeta\in\segab$:$$\kurzh\intd[a,b]fg,x=g\at(a)\cdot\kurzh\intd[a,\zeta]f,x.$$

А если функция $g$ неубывающая и~неотрицательная на \ab, то найдется
точка $\zeta\in\segab$:
$$\kurzh\intd[a,b]fg,x=g\at(b)\cdot\kurzh\intd[\zeta,b]f,x.$$

\tm{10}{вторая теорема о~среднем}

Пусть функция $F$ непрерывна на \ab, $g$
--- монотонная функция на \ab, тогда $g$ интегрируема по $F$ на \ab\
в смысле Римана -- Стилтьеса и~найдется точка $\xi\in\segab$:
$$\rimanst\intab g,F=g\at(a)\cdot\rimanst\intd[a,\xi] ,F+g\at(b)\cdot\rimanst\intd[\xi,b] ,F
=g\at(a)\hr{F\at(\xi)-F\at(a)}+g\at(b)\hr{F\at(b)-F\at(\xi)}.$$

При этом если функция $g$ невозрастающая и~неотрицательная на \ab,
то найдется точка
$\zeta\in\segab$:$$\rimanst\intd[a,b]g,F=g\at(a)\cdot\rimanst\intd[a,\zeta] ,F
=g\at(a)\hr{F\at(\zeta)-F\at(a)}.$$

А если функция $g$ неубывающая и~неотрицательная на \ab, то найдется
точка $\zeta\in\segab$:
$$\rimanst\intd[a,b]g,F=g\at(b)\cdot\rimanst\intd[\zeta,b] ,F
=g\at(b)\hr{F\at(b)-F\at(\zeta)}.$$

\prs

Существование интеграла следует в~теореме $9$ --- из сохранения
интегрируемости по Курцвейлю -- Хенстоку при умножении на функции
ограниченной вариации (в частности, на монотонные), в~теореме $10$
--- из интегрируемости функций ограниченной вариации (в частности,
монотонных) по непрерывным функциям.

\smallskip

Теорема $9$ следует из теоремы $10$, так как если $F$ ---
неопределенный интеграл $f$, то
$$\kurzh\intab fg,x=\rimanst\intab g,F,$$
поскольку $g$ --- функция ограниченной вариации. Так что доказываем
теорему $10$.

\smallskip

Проинтегрируем по частям:
$$\rimanst\intab g,F=g\at(b)F\at(b)-g\at(a)F\at(a)-\rimanst\intab
F,g.$$

Пусть, скажем, $g$ неубывающая. Если на самом деле она
невозрастающая, то перейдем к~$-g$ --- и~все сведется к~нашему
случаю. A~в нашем случае имеет место оценка
$$\min_{\seg[a,\,b]}F\cdot\intab ,g\le\rimanst\intab
F,g\le\max_{\seg[a,\,b]}F\cdot\intab ,g.$$

Тогда, раз уж $F$ непрерывна, на \ab\ найдется такая точка $\xi$,
что
$$F\at(\xi)\cdot\intab ,g=\rimanst\intab F,g.$$

Теперь легко выводится первое утверждение нашей теоремы:
$$\rimanst\intab g,F=g\at(b)F\at(b)-g\at(a)F\at(a)-F\at(\xi)\hr{g\at(b)-g\at(a)}=
g\at(a)\hr{F\at(\xi)-F\at(a)}+g\at(b)\hr{F\at(b)-F\at(\xi)}.$$

Два других утверждений симметричны, поэтому достаточно доказать
одно. Здесь тоже удобно рассмотреть случай, когда $g$ неубывающая и
неотрицательная. Запишем уже знакомую оценку
$$\min_{\seg[a,\,b]}F\cdot\intab ,g\le\rimanst\intab
F,g\le\max_{\seg[a,\,b]}F\cdot\intab ,g.$$

Еще больше усилим неравенства, прибавив соответственно
$\min\limits_{\seg[a,\,b]}F\cdot g\at(a)$, $F\at(a)g\at(a)$ и
$\max\limits_{\seg[a,\,b]}F\cdot g\at(a)$:

$$\min_{\seg[a,\,b]}F\cdot g\at(b)\le\rimanst\intab
F,g+F\at(a)g\at(a)\le\max_{\seg[a,\,b]}F\cdot g\at(b).$$

Теперь воспользуемся непрерывностью $F$ и~найдем такую точку
$\zeta\in\segab$, что
$$F\at(\zeta)g\at(b)=\rimanst\intab F,g+F\at(a)g\at(a).$$

Но тогда, возвращаясь из интегрирования по частям, получим как раз
то что надо.

\prtms

\eject

%------------------------------------------------------------------------------------

\sectionname{11}{Несобственные интегралы.}

\subsection{Сведение несобственных интегралов к собственным. Теорема Хейка.}

\df Пусть функция $f$ определена на полуотрезке $\hsr{a, b}$, где
$a\in\reals$, $b\in\bar{\reals}$, $b>a$, и~интегрируема в~одном из
трех смыслов на каждом отрезке $\seg[a,b']\subset\hsr{a, b}$. Тогда
если существует предел
$$\int\limits_{\hsr{a,b}}   f\,dx\eqdef\lim_{b'\rightarrow b-0}\intd[a,b']f,x,$$
то он называется несобственным интегралом функции $f$ на промежутке
$\hsr{a, b}$ (с особенностью в~точке $b$) в~соответствующем смысле,
а сама функция $f$ называется интегрируемой в~соответствующем
несобственном смысле на $\hsr{a, b}$. Аналогично определяется
несобственный интеграл с~особенностью на левом конце промежутка.

\bigskip

\ut

Если функция $f$ интегрируема по Риману в~несобственном смысле на
$\hsr{a, b}$, $b\in\reals$, и~ограничена на $\hsr{a, b}$, то при
любом доопределении $f$ в~точке $b$ получим интегрируемую по Риману
на \ab\ функцию, и
$$\riman\intabfdx=\riman \int\limits_{\hsr{a,b}}   f\,dx.$$

\pr

Это следует из критерия Лебега. Во-первых, доопределенная функция
$f$ ограничена на \ab:
$$\abs{f}\le\max\set{\smash{\sup_{\hsr{a,b}}\abs{f\at(x)}},\abs{f\at(b)}}.$$

Обозначим через $E_k$ множество точек разрыва $f$ на отрезке
$\seg[a,\ b-1/k]$. Поскольку $f$ интегрируема по Риману на всех
таких отрезках, мера любого такого множества равна нулю. Тогда и
$$\mes\at(\bigcup_kE_k)=0.$$

Возможно, добавится еще разрыв в~точке $b$, но одна точка на меру не
влияет. Значит, функция $f$ после доопределения останется
непрерывной почти всюду на \ab, и~можно применить критерий Лебега.
Равенство интегралов следует из непрерывности неопределенного
интеграла.

\prut

\bigskip

{\small\rem

Аналогичное утверждение верно и~для интеграла с~особенностью в~левом
конце. Дальше мы это отдельно замечать не будем, поскольку и~так
ясно.

}

\bigskip

Сейчас мы собираемся доказать аналогичную теорему для интеграла
Курцвейля -- Хенстока.

\bigskip

\lm

Пусть функция $f$ определена на $\hsr{a, b}$, $b\in\reals$, и
интегрируема в~смысле Курцвейля -- Хенстока на любом отрезке
$\seg[a,b']\subset\hsr{a, b}$. Тогда для любого числа $\epgz$
найдется такой масштаб $\dex$ на $\hsr{a, b}$, что для любого
согласованного с~ним отмеченного разбиения $\T$ Хенстока любого
отрезка $\seg[a,b']\subset\hsr{a, b}$ выполнится оценка
$$\abs{\isfT-\kurzh\intd[a,b']f,x}<\ep.$$

\eject

\pr

Зафиксируем какую-нибудь последовательность $\set{b_k}_{k=1}^\infty$
точек полуинтервала $\hsr{a, b}$, монотонно возрастающую и
стремящуюся к~$b$, и~обозначим $b_0=a$. Поскольку функция $f$
интегрируема по Курцвейлю -- Хенстоку на любом отрезке
$\seg[a,b']\subset\hsr{a, b}$, и, следовательно, на любом отрезке
$\seg[b_{k-1},b_k]$, можно выбрать такие масштабы $\de_k\atx$ на
$\seg[b_{k-1},b_k]$, $k\in\N$, что для любого согласованного с~ними
разбиения $\T$ Хенстока соответствующего отрезка будет выполнена
оценка
$$\abs{\isum(f,\T)-\kurzh \intd[b_{k-1},b_k]{ f},x}<{\ep\over2^k}.$$

Тогда соберем требуемый универсальный масштаб $\dex$ на
$\hsr{a, b}$ следующим образом:
$$\dex=\begin{cases}\min\set{\de_1\at(b_0),b_1-b_0}&\qquad\hbox{при}\quad x=b_0;\cr
\min\set{\de_k\at(b_k),\de_{k+1}\at(b_k),b_k-b_{k-1},b_{k+1}-b_k}&\qquad\hbox{при}\quad
x=b_k;\cr \min\set{\de_k\atx,x-b_{k-1},b_k-x}&\qquad\hbox{при}\quad
b_{k-1}<x<b_k.\end{cases}$$

Здесь мы добились того, что если отмеченная точка разбиения лежит
внутри одного из отрезков $\seg[b_{k-1},b_k]$, то и~ее отрезок тоже
там целиком лежит. A~если точка случайно совпала с~одной из $b_k$,
то ее влияние будет незначительным: ведь ее отрезок не вылезет за
промежуток $\seg[b_{k-1},b_{k+1}]$. И~масштаб $\de\atx$ еще вдобавок
не превосходит ни одного из масштабов $\de_k\atx$ в~тех точках,
когда последний определен. Теперь, чтобы доказать работоспособность
нашего масштаба, выберем любую точку $b'\in\hr{a, b}$ и~любое
отмеченное разбиение \Tdixif\ Хенстока отрезка $\seg[a,b']$,
согласованное с~$\dex$. Новое соображение: без ограничения общности
можно считать, что точка $\xii$ является граничной точкой отрезка
$\dei$, поскольку если это не так, то пару $\dixi$ можно разбить на
две пары: $\hr{\dei\cap\hrs{-\infty,\xii};\xii}$ и
$\hr{\dei\cap\hsr{\xii,+\infty};\xii}$. Это соображение существенно,
и в~случае интегрирования по Мак-Шейну не работает. A~в нашем же
случае наше разбиение $\T$ вообще хорошо укладывается в~отрезки
$\seg[b_{k-1},b_k]$ --- ни один отрезочек $\dei$ не накрывает
никакую точку $b_k$. Найдем такое натуральное число $n$, что
$b'\in\seg[b_{n},b_{n+1}]$, и~приступим к~оценке.
\begin{multline*}
\abs{\isum(f,\T)-\kurzh\intd[a,b']f,x} = \abs{\sumi\fxildi-\kurzh\intd[a,b']f,x} \le\\ \le
\sum_{k=1}^n\abs{\sum_{\dei\subseteq\seg[b_{k-1},\,b_k]}
\fxildi-\kurzh  \intd[b_{k-1},b_k]{  f},x} + \abs{\sum_{\dei\subseteq\seg[b_n,\,b']}
\fxildi-\kurzh\intd[b_n,b']f,x}.
\end{multline*}

Все первые слагаемые вместе оцениваются суммой геометрической
прогрессии $\sum\ep/2^k=\ep\hr{1-1/2^n}$, в~силу того, что мы хорошо
выбрали масштабы на маленьких отрезках, a~потом хорошо выбрали
масштаб на всем промежутке. A~самое последнее слагаемое оценивается
числом $\ep/2^{n+1}$ по слабой лемме Колмогорова~-- Сакса --
Хенстока: ведь на отрезке $\seg[b',b_{n+1}]$ можно дорисовать еще
несколько отрезочков с~точками, и~получится разбиение Хенстока
отрезка $\seg[b_{n},b_{n+1}]$, согласованное с~масштабом $\dex$, a~к
этому разбиению мы слабую лемму и~применим. Итого получится меньше
$\ep$.

\prlm

\bigskip

\tm1{Хейка}

Если функция $f$ определена на $\hsr{a, b}$, $b\in\R$, и
интегрируема по Курцвейлю -- Хенстоку в~несобственном смысле на
$\hsr{a, b}$, то $f$ интегрируема на \ab\ по Курцвейлю -- Хенстоку
в собственном смысле, и~значения собственного и~несобственного
интегралов совпадают.

\pr

Доопределим произвольно функцию $f$ в~точке $b$, если она там была
неопределена. Возьмем любое $\epgz$. Поскольку $f$ интегрируема на
$\hsr{a, b}$ в~несобственном смысле, можем найти такую точку
$\bar{b}\in\hsr{a, b}$, что выполнено условие
$\abs{f\at(b)}\cdot\hr{b-\bar{b}}<\ep/3$ и~для любой точки
$b'\in\hr{\bar{b}, b}$ выполняется оценка
$$\abs{\kurzh\intd[a,b']f,x-\kurzh \int\limits_{\hsr{a,b}}   f\,dx}<{\ep\over3}.$$

Пользуясь леммой, найдем для нашего $\ep$ такой масштаб $\degz$ на
$\hsr{a, b}$, что $\dex<b-x$, $\de\at(b)<b-\bar{b}$ и~для любого
согласованного с~ним отмеченного разбиения $\T$ Хенстока любого
отрезка $\seg[a,b']\subset\hsr{a, b}$ выполнится оценка
$$\abs{\isfT-\kurzh\intd[a,b']f,x}<{\ep\over3}.$$

\eject

Ну и~пусть \Tdixif\ --- любое разбиение Хенстока отрезка \ab,
согласованное с~масштабом $\dex$. По выбору этого масштаба если
точка $\xii$ отлична от $b$, то отрезок $\dei$ оказывается тоже
внутри $\hsr{a, b}$, и, следовательно, в~$\T$ непременно должна
найтись пара $\hr{\xi_j;\Delta_j}$, где $\xi_j=b$ и, следовательно,
$\Delta_j\subset\seg[\bar{b},b]$. Теперь обозначим левый конец
отрезка $\Delta_j$ через $b'$, и~оценка проведется совсем легко:
$$\abs{\sumi\fxildi-\kurzh \int\limits_{\hsr{a,b}}   f\,dx}\le
\abs{\sum_{i\neq j}\fxildi-\kurzh\intd[a,b']f,x}+
\abs{f\at(\xi_j)\abs{\Delta_j}-\kurzh  \int\limits_{\hsr{b' ,b}}   f\,dx}.$$

Первый модуль оценивается величиной $\ep/3$ благодаря лемме. А
второй --- просто суммой модулей: величина
$f\at(\xi_j)\abs{\Delta_j}$ ввиду малости отрезка $\Delta_j$ будет
меньше $f\at(b)\hr{b-\bar{b}}<\ep/3$, a~интеграл будет благодаря
близости $\bar{b}$ к~$b$ тоже меньше $\ep/3$. Итого, как всегда,
меньше $\ep$.

\prtm

\bigskip

\df Пусть функция $f$ определена на промежутке $\hsr{a, b}$ и
интегрируема в~одном из трех смыслов на любом подотрезке
$\seg[a,b']\subset\hsr{a, b}$. Будем говорить, что $f$
удовлетворяет условию Коши несобственной интегрируемости в
соответствующем смысле на $\hsr{a, b}$, если для любого числа
$\epgz$ найдется такое число $\degz$, что для любых двух точек $b'$
и $b''$ из промежутка $B_{\de}\at(b)\cap\hsr{a, b}$ выполняется
неравенство
$$\abs{\intd[a,b'']f,x-\intd[a,b']f,x}=\abs{\,\intd[b',b'']f,x}<\ep.$$

\bigskip

\ut

Функция $f$, определенная на промежутке $\hsr{a, b}$
и~интегрируемая в одном из трех смыслов на любом подотрезке
$\seg[a,b']\subset\hsr{a, b}$, интегрируема в~соответствующем
несобственном смысле на промежутке $\hsr{a, b}$ тогда и~только
тогда, когда удовлетворяет условию Коши несобственной
интегрируемости в~этом смысле на этом промежутке.

\pr

Доказывать тут нечего, это просто критерий Коши существования в
точке $b$ левого предела функции $$F\atx=\intd[a,x]f\at(t),t.$$

\prut

\bigskip

\df Будем говорить, что несобственный интеграл от функции $f$ на
$\hsr{a, b}$ в~данном смысле сходится абсолютно, если $f$
интегрируема в~этом смысле на любом подотрезке
$\seg[a,b']\subset\hsr{a, b}$ и~существует несобственный интеграл
от $\abs{f}$ на $\hsr{a, b}$.

\bigskip

\ut

Если несобственный интеграл от функции $f$ сходится абсолютно на
$\hsr{a, b}$, то он сходится.

\pr

Утверждение следует из критерия Коши несобственной интегрируемости и
неравенства $$\abs{\intd[b',b'']f,x}\le\intd[b',b'']\abs{f},x.$$

\prut

\subsection{Признаки сходимости несобственных интегралов.}

\tm2{признак сравнения 1}

Если функции $f$ и~$g$ определены на $\hsr{a, b}$ и~интегрируемы на
любом отрезке $\seg[a,b']\subset\hsr{a, b}$, может быть, в~разных
смыслах, и~при этом на $\hsr{a, b}$ выполняется неравенство $0\le
f\atx \le g\atx$, то из существования несобственного интеграла от
функции $g$ на $\hsr{a, b}$ будет следовать существование
несобственного интеграла от функции $f$ на $\hsr{a, b}$, и
наоборот, если несобственный интеграл от $f$ на $\hsr{a, b}$ не
существует, то несобственный интеграл от $g$ --- тем более не
существует.

\eject

\pr

Из сохранения неравенства при интегрировании в~любом из трех смыслов
следует, что на любом отрезке $\seg[b',b'']\subset\hsr{a, b}$ имеет
место неравенство $$0\le\intd[b',b'']f,x\le\intd[b',b'']g,x.$$

Таким образом, из выполнения условия Коши для $g$ следует его
выполнение и~для $f$, и~наоборот.

\prtm

\bigskip

\tm3{признак сравнения 2}

Если неотрицательные функции $f$ и~$g$ определены на $\hsr{a, b}$,
и интегрируемы на любом отрезке $\seg[a,b']\subset\hsr{a, b}$,
может быть, в~разных смыслах, и~при этом на $\hsr{a, b}$
выполняется оценка $0<c_1<{f\,\atx/g\,\atx}\le c_2<+\infty,$ то
несобственные интегралы на $\hsr{a, b}$ функций $f$ и~$g$ сходятся
или расходятся одновременно.

\pr

Снова пользуемся критерием Коши --- и~замечаем, что
$$0\le\abs{\intd[b',b'']f,x}\le c_2\abs{\intd[b',b'']g,x},\qquad
0\le\abs{\intd[b',b'']g,x}\le{1\over c_1}\abs{\intd[b',b'']f,x}.$$

\prtm

\bigskip

\lm

Пусть $\ph$ --- функция ограниченной вариации на полуотрезке
$\hsr{a, b}$, где, возможно, $b=+\infty$. Тогда для любого числа
$\epgz$ найдется такая точка $\bar{b}\in\hsr{a, b}$, что
$\var_{\hsr{\bar{b},b}}\ph<\ep$.

\pr

Существует такой набор точек $\set{a_i}_{i=0}^n$, упорядоченных по
возрастанию: $a\le a_0<a_1<\cdots<a_n<b$, что
$$\var_{\hs{a,a_n}}\ph\ge\sum_{i=1}^n\abs{\ph\at(a_i)-\ph\at(a_{i-1})}>\var_{\hsr{a,b}}\ph-\ep.$$

Здесь и~первое неравенство, и~существование набора точек следует из
определения вариации как точной верхней грани. Тогда благодаря
аддитивности вариации по множеству имеем оценку
$$\var_{\hsr{a_n,b}}\ph=\var_{\hsr{a,b}}\ph-\var_{\hs{a,a_n}}\ph<\ep.$$

\prlm

\bigskip

\tm4{признак Абеля}

Если функция $f$ интегрируема на $\hsr{a, b}$ по Риману (или по
Курцвейлю -- Хенстоку), $\ph$ --- функция ограниченной вариации на
$\hsr{a, b}$, то функция $f\cdot\ph$ также интегрируема на
$\hsr{a, b}$ по Риману (Курцвейлю -- Хенстоку).

\smallskip

\tm5{признак Дирихле}

Если неопределенный интеграл $F$ от функции $f$ существует на
промежутке $\hsr{a, b}$ в~смысле Римана (Курцвейля -- Хенстока) и
ограничен на $\hsr{a, b}$, $\ph$
--- функция ограниченной вариации на $\hsr{a, b}$, и
$\ph\atx\rightarrow0$ при $x\rightarrow b-0$, то произведение
$f\cdot\ph$ также интегрируемо на $\hsr{a, b}$ в~смысле Римана
(Курцвейля -- Хенстока).

\prs

Будем проверять критерий Коши. Возьмем любое $\epgz$ и~проведем
небольшое преобразование.

$$\abs{\,\intd[b',b'']f\ph,x}=\abs{\rimanst\intd[b',b'']\ph,F}=
\abs{F\at(b'')\ph\at(b'')-\rimanst\intd[b',b'']F,\ph}.$$

Здесь мы воспользовались теоремой об~интегрировании по частям,
которой не было для интегралов Мак-Шейна, поэтому для Мак-Шейна мы
эти признаки не доказываем. Слагаемое $F\at(b')\ph\at(b')$ можно
отбросить, потому что неопределенный интеграл $F$ мы можем в~каждом
случае отсчитывать от точки $b'$, и~тогда $F\at(b')=0$.

\eject

{\it Признак Абеля:}

\smallskip

По критерию Коши найдется такая точка $\bar{b}\in\hsr{a, b}$, что
для любой точки $b_1\in\hsr{\bar{b}, b}$ будут верны оценки
$$\abs{F\at(b_1)}=\abs{\,\intd[b',b_1]f,x}<{\ep\over2\sup\limits_{\hsr{a,b}}\abs{\ph}}\qquad\hbox{и}\qquad
\abs{F\at(b_1)}<{\ep\over2\var_{\hsr{a,b}}\ph}.$$

Все величины справа конечны, поскольку $\ph$ --- функция
ограниченной вариации, и, в~частности, ограничена. A~теперь можно
оценить каждое слагаемое под модулем. Поскольку $F\at(b'')$ мы
только что сделали достаточно маленьким, величина
$F\at(b'')\ph\at(b'')$ по модулю не превосходит $\ep/2$. А
оставшийся интегральчик оцениваем произведением максимума $F\atx$ на
$\hsr{\bar{b}, b}$ на вариацию $\ph$. Вторая оценка выполняется для
всех $b_1$ из $\hsr{\bar{b}, b}$, поэтому она есть верхняя грань
для $F$ на $\seg[b',b'']\subset\hsr{\bar{b}, b}$, значит
интегральчик не превосходит $\ep/2$. Итого меньше $\ep$.

\medskip

{\it Признак Дирихле:}

\smallskip

Величина интегральчика $F\at(b'')$ оценивается следующим образом:
$$\abs{F\at(b'')}=\abs{\,\intd[b',b'']f\atx,x}=\abs{\,\intd[a,b'']f\atx,x-\intd[a,b']f\atx,x}\le2\sup_{a\le
x\le b}\abs{\,\intd[a,x]f\at(t),t}=C<+\infty.$$

Пользуясь леммой и~стремлением $\ph$ к~нулю, найдем такую точку
$\bar{b}\in\hsr{a, b}$, чтобы выполнялись оценки

$$\sup_{\hsr{\bar{b},b}}\abs{\ph}<\ep/2C\qquad\hbox{и}\qquad\var_{\hsr{\bar{b},b}}\abs{\ph}<\ep/2C.$$

Тогда на весь искомый интегральчик получаем оценку
$C\cdot(\ep/2C)+C\cdot(\ep/2C)=\ep$.

\smallskip

Выполнен критерий Коши несобственной интегрируемости.

\prtms

\eject

%------------------------------------------------------------------------------------
% КОНЕЦ ТЕКСТА
%------------------------------------------------------------------------------------

\end{document}
