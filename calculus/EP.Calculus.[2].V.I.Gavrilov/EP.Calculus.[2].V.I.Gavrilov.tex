\documentclass[a4paper]{article}
\usepackage[utf8]{inputenc}
\usepackage[russian]{babel}
\usepackage[simple]{dmvn}

\title{Математический анализ}
\author{Лектор Валериан Иванович Гаврилов}
\date{2 семестр, 2005--2006 г.}

\begin{document}
\maketitle


\medskip\dmvntrail

\begin{nums}{-1}
\item Точная первообразная и неопределённый интеграл на промежутке.
Первообразная и неопределённый интеграл с исключительным множеством
на промежутке. Свойство линейности, замена переменной
интегрирования, интегрирование по частям.

\item Определение интеграла Римана. Необходимое условие
интегрируемости функции. Линейное свойство интеграла Римана.

\item Свойства верхних и нижних сумм Дарбу.

\item Критерии интегрируемости функции по Риману.

\item Интегрируемость непрерывной функции, ограниченной функции с
конечным множеством точек разрыва и монотонной функции на отрезке.

\item Свойство монотонности интеграла Римана. Оценка модуля
интеграла Римана. Интегрируемость произведения интегрируемых
функций.

\item Первая теорема о среднем значении для определённого интеграла.
Вторая теорема о среднем значении для определённого интеграла
(доказательство при дополнительных предположениях).

\item Свойства непрерывности и дифференцируемости интеграла с
переменным верхним пределом. Формула Ньютона--Лейбница.

\item Интегрирование по частям в определённом интеграле. Формула
Тейлора с остаточным членом в интегральной форме, в форме Лагранжа.

\item Формула суммирования Эйлера--Маклорена.

\item Функции ограниченной вариации. Основная теорема.

\item Критерий Жордана спрямляемости кривой. Вычисление длины
гладкой кривой.

\item Несобственные интегралы. Свойства линейности, аддитивности и
монотонности несобственного интеграла. Критерий Коши сходимости
несобственного интеграла.

\item Остаток несобственного интеграла от положительной функции.
Признак сравнения несобственных интегралов.

\item Замена переменной интегрирования в интеграле Римана и в
несобственном интеграле. Оценка модуля несобственного интеграла.

\item Интегрирования по частям в несобственном интеграле. Признак
сходимости несобственного интеграла $\displaystyle{\int_a^{+\bes}
\frac{f(x)}{x^{\al}}\,dx, \; a>0}$. Неполная формула Стирлинга.

\item Свойства интегральных сумм Стилтьеса. Интеграл Стилтьеса.
Линейное свойство интеграла Стилтьеса. Оценка модуля интеграла
Стилтьеса.

\item Интегрирование по частям в интеграле Стилтьеса. Вычисление
интеграла Стилтьеса.

\item Метрическое пространство $\R^m, \; m\ge1$. Свойства открытых и
замкнутых множеств в метрическом пространстве. Критерий замкнутого
множества. Свойства компактов в $\R^m$.

\item Характеристическое свойство предела последовательности точек в
$\R^m,\;m\ge1$. Фундаментальные последовательности. Базы в $\R^m, \;
m\ge1$. Локальные свойства функций нескольких действительных
переменных, имеющих предел в точке и непрерывных в точке.
Отображения из $\R^m$ в $\R^n$. Характеристические свойства
отображения из $\R^m$ в $\R^n$, имеющего предел в точке и
непрерывного в точке.

\item Непрерывность композиции непрерывных отображений,
непрерывность сложной функции нескольких действительных переменных.
Характеристическое свойство непрерывного отображения открытого
множества. Равномерно непрерывные отображения метрических
пространств и конечномерных евклидовых пространств.

\item Свойства непрерывных отображений и непрерывных функций на
компактах из $\R^m$. Непрерывный образ связного множества из $\R^m$.
Теорема о промежуточных значениях непрерывной функции на связном
множестве из $\R^m$.

\item Частные производные, производные по направлениям и
дифференцируемость функций нескольких действительных переменных.
Градиент; основное свойство.

\item Достаточное условие дифференцируемости функций нескольких
действительных переменных в точке. Теорема о дифференцируемости
сложной функции (формулировка). Дифференциал функции нескольких
действительных переменных. Свойство инвариантности формы
дифференциала при замене переменных; свойства дифференциала.
Геометрический смысл дифференциала функции двух переменных.

\item Частные производные высших порядков функции нескольких
действительных переменных. Теорема о перестановке порядка
дифференцирования. Формула для дифференциала $n$--го порядка.

\item Формула Тейлора функции нескольких действительных переменных с
остаточным членом в форме Лагранжа.

\item Необходимое условие локального экстремума функции нескольких
действительных переменных. Достаточное условие локального экстремума
функции нескольких действительных переменных.

\item Дифференцируемые отображения из $\R^m$ в $\R^n$. Матрица
Якоби. Характеристическое свойство дифференцируемого отображения из
$\R^m$ в $\R^n$. Теорема о существовании и дифференцируемости
неявного отображения (без доказательства).

\item Теорема о существовании локального диффеоморфизма. Принцип
сохранения области при отображениях с неравным нулю якобианом.
Достаточный признак независимости функций.

\item Условный экстремум. Необходимый признак условного экстремума.
Метод неопределённых множителей Лагранжа.
\end{nums}

\medskip\dmvntrail
\end{document}
