\documentclass[a4paper]{article}
\usepackage[utf,simple]{dmvn}

\title{Программы экзамена по математическому анализу}
\author{Лектор Т.\,П.\,Лукашенко}
\date{1--4 семестры, 2003--2004 г.}

\begin{document}
\maketitle

\section*{1 семестр}

\begin{nums}{0}
\item Множества и операции над ними. Свойства операций. Законы Моргана. Декартово  произведение множеств и его свойства.
\item Натуральные, целые и рациональные числа, их свойства. Аксиоматика действительных чисел. Бесконечные десятичные
дроби как модель действительных чисел.
\item Принципы полноты действительных чисел. Их эквивалентность.
\item Эквивалентные множества. Счетные множества и их свойства. Несчётные множества. Сравнение мощностей. Теорема Кантора Бернштейна.
\item Открытые и замкнутые множества и их свойства.
\item Теоремы о конечных подпокрытиях и о существовании предельной точки.
\item Предел последовательности и его свойства.
\item Предел монотонной ограниченной последовательности. Число $e$. Критерий Коши сходимости последовательности.
\item Частичные пределы последовательности, их свойства. Числовые ряды.
\item Два определения предела функции, их эквивалентность. Свойства предела функции.
\item Критерий Коши существования предела функции. Односторонние пределы и их свойства.
\item Непрерывность функции в точке. Классификация точек разрыва.
\item Предел функции по базе и его свойства.
\item Функции, непрерывные на отрезке, и их свойства (теоремы Больцано Коши, Вейерштрасса, Кантора). Теорема об обратной функции.
Модуль непрерывности.
\item Элементарные функции, их свойства. Замечательные пределы.
\item Производная, касательная, дифференциал и их связи.
\item Правила вычисления производных. Производные элементарных функций. Производные и дифференциалы высших порядков.
\item Теоремы Ферма. Ролля, Лагранжа, Коши и Бонне. Следствия теоремы Лагранжа.
\item Свойства производной. Правила Лопиталя.
\item Формула Тейлора с различными формами остаточного члена Ряды Тейлора. Разложения некоторых элементарных функций.
\item Достаточные условия локального экстремума. Глобальные экстремумы функции на отрезке.
\item Выпуклость, точки перегиба. Свойства выпуклых функций. Неравенство Иенсена.
\item Свойства односторонних производных выпуклых функций. Условия выпуклости.
\item Первообразная. Неопределенный интеграл и его свойства. Основные
неопределенные интегралы. Интегрирование рациональных дробей, различных
иррациональностей. тригонометрических и некоторых других выражений.
\end{nums}


\section*{2 семестр}

\begin{nums}{-2}
\item Определенные интегралы Римана, Мак Шейна и Курцвейля Хенстока. Основная лемма
о существовании разбиений. Простейшие свойства интегралов. Критерии Коши интегрируемости.
Интегрируемость на подотрезках.

\item Необходимое условие интегрируемости по Риману. Аддитивность интегралов по отрезкам.
Интегрируемость производных по Курцвейлю Хенстоку. Формула Ньютона
Лейбница и следствие из нее.

\item Верхняя мера Лебега и ее свойства. Множества меры нуль по Лебегу. Интегрируемость
ограниченных и непрерывных почти всюду функций по Риману и по Мак Шейну.

\item Ограниченность и непрерывность почти всюду интегрируемых по Риману функций.
Связь интегралов Римана и Мак Шейна. Критерий Лебега интегрируемости по Риману
и следующие из него дополнительные свойства интеграла Римана.

\item Интеграл с переменным верхним пределом. Принадлежность к классу Липшица при
условии ограниченности. Дифференцируемость в точке. Существование первообразных.
Интегрируемость по Мак Шейну функции, равной нулю почти всюду.

\item Два определения измеримых на отрезке функций, их эквивалентность. Интегрируемость
по Мак Шейну ограниченных измеримых функций.

\item Слабая и сильная леммы Хенстока. Непрерывность интеграла Курцвейля Хенстока с
переменным верхним пределом. Интегрируемость по модулю функций, интегрируемых
по Мак Шейну.

\item Покрытие в смысле Витали. Теоремы Витали. Дифференцируемость почти всюду интеграла
Курцвейля Хенстока с переменным верхним пределом.

\item Определенные интегралы Римана Стилтьеса, Мак Шейна Стилтьеса и Курцвейля
Хенстока Стилтьеса; их простейшие свойства. Критерии Коши интегрируемости. Интегрируемость
на подотрезках.

\item Аддитивность интегралов Стилтьеса по отрезкам. Функции ограниченной вариации их
свойства. Функции ограниченной вариации, как разность неубывающих функций.

\item Интегрируемость в смысле Римана Стилтьеса непрерывных функций по функциям
ограниченной вариации. Интегрирование по частям в интеграле Римана Стилтьеса.

\item Сведение интегралов Римана Стилтьеса к интегралу Римана. Интегрирование по частям
и замена переменной в интеграле Римана. Формула Тейлора с остаточным членом
в интегральной форме.

\item Первая и вторая теоремы о среднем.

\item Несобственные интегралы. Критерий Коши сходимости несобственных интегралов.
Абсолютная и условная сходимости. Признаки сходимости.

\item Метрические пространства. Нормированные пространства. Пространство $\R^n$, норма и
метрика в нем. Открытые и замкнутые множества, их свойства.

\item Компакты, их свойства. Критерий компактности в $\R^n$. Теорема Больцано Вейерштрасса
о существовании предельной точки.

\item Последовательности в метрических, нормированных пространствах и в $\R^n$, их пределы,
свойства. Полные метрические пространства. Принцип вложенных шаров. Полнота $\R^n$.

\item Предел функции и его свойства (в метрических и нормированных пространствах).

\item Непрерывные функции и их свойства (в метрических и нормированных пространствах).
Принцип сжимающих отображений.

\item Связные множества в метрических и нормированных пространствах и их свойства.

\item Кривые, длина кривой и ее свойства в метрических, нормированных пространствах и
в $\R^n$.

\item Дифференцируемость отображений нормированных пространств. Дифференцируемость
функций нескольких переменных. Дифференциал. Частные производные. Достаточные
условия дифференцируемости. Геометрический смысл дифференцируемости функций
нескольких переменных.

\item Производная по направлению. Градиент. Правила дифференцирования.
Частные производные и дифференциалы высших порядков. Равенство смешанных производных.

\item Формула Тейлора функции нескольких переменных с остаточным членом в форме
Лагранжа, интегральной и Пеано.

\item Локальный экстремум функции нескольких переменных. Необходимое и достаточное
условие его существования.

\item Теоремы о существовании и дифференцируемости неявной функции.

\item Условный экстремум. Метод неопределенных множителей Лагранжа его отыскания.

\end{nums}

\section*{3 семестр}

\begin{nums}{-4}
\item Числовые ряды. Критерий Коши. Операции над рядами. Абсолютная и условная сходимости.
Ряды с неотрицательными членами. Признаки сходимости: ограниченность частичных сумм, сравнения.
\item Признаки Д'Аламбера, Коши, интегральный Коши Маклорена, Куммера, Раабе и Гаусса.
\item Ряды с членами произвольных знаков и ряды комплексных чисел. Признак Лейбница. Преобразование Абеля.
Последовательности ограниченной вариации и их свойства. Признаки Абеля и Дирихле.
\item Теоремы Коши и Римана о перестановках членов ряда. Умножение рядов. Теоремы Коши и Мертенса.
\item Бесконечные произведения. Условия сходимости. Разложение функции $\sin x$ в бесконечное произведение.
\item Метод суммирования Чезаро (средних арифметических), его вполне регулярность и необходимое условие суммируемости.
Метод суммирования Абеля. Теорема Фробениуса о суммируемости методом Абеля рядов, суммируемых по
Чезаро. Вполне регулярность метода Абеля.
\item Критерий Маркова Гордона перестановки предельных переходов. Функциональные последовательности и ряды.
Равномерная сходимость и операции с нею. Критерий Коши равномерной сходимости.
\item Признаки Вейерштрасса, Дини. Лейбница, Абеля и Дирихле равномерной сходимости.
\item Теорема об изменении порядка пределов и следствия из неё. Полнота пространства $\Cb(K)$ непрерывных на компакте функций.
Почленное дифференцирование и интегрирование функциональных последовательностей и рядов.
\item Критерий компактности Хаусдорфа. Равностепенная непрерывность. Теорема Арцела Асколи.
\item Степенные ряды. Теорема Коши Адамара. Непрерывность, дифференцируемость и интегрируемость суммы степенного ряда.
\item Степенной ряд как ряд Тейлора своей суммы. Теорема единственности. Теорема Абеля. Функции комплексного переменного.
Формула Эйлера. Пример непрерывной нигде не дифференцируемой функции.
\item Функции, зависящие от параметра; равномерное стремление к пределу; связь с равномерной сходимостью последовательностей.
Критерий Коши. Свойства равномерной сходимости.
\item Перестановка пределов, дифференцирование и интегрирование пределов функций, зависящих от параметра.
\item Собственные интегралы с параметром. Их свойства: переход к пределу, непрерывность, дифференцируемость и интегрируемость.
\item Несобственные интегралы с параметром, их равномерная сходимость. Критерий Коши. Признаки равномерной сходимости Вейерштрасса,
Дини, Абеля и Дирихле.
\item Свойства несобственных интегралов с параметром: переход к пределу, непрерывность, дифференцируемость, интегрируемость
(собственная и несобственная).
\item Интеграл Дирихле. Интегралы Эйлера и их свойства. Формула Эйлера и формула дополнения для гамма функции. Интеграл Пуассона.
\item Связь функций Эйлера $\Be(x)$ и $\Ga(x)$. Формула Стирлинга.
\item Пространства со скалярным произведением. Ортогональные системы. Экстремальное свойство коэффициентов Фурье. Тождество
Бесселя и неравенство Бесселя.
\item Ортогональные системы и ряды Фурье. Сходимость рядов Фурье. Замкнутость, равенство Парсеваля, полнота; связь этих понятий.
Пространство $\ell_2$, его полнота.
\item Неравенство Чебышёва. Измеримые функции и их свойства. Теорема Егорова. Измеримость интегрируемых по Курцвейлю Хенстоку функций.
\item Эквивалентность интегралов Мак-Шейна и Курцвейля Хенстока на ограниченных функциях. Теорема Б.\,Леви для интегралов
Мак-Шейна и Курцвейля Хенстока.
\item Критерий интегрируемости неотрицательных измеримых функций и следствия из него. Эквивалентность интегралов Мак-Шейна и
Курцвейля Хенстока на неотрицательных функциях. Лемма Фату и теорема Лебега для интегрируемых по Мак-Шейну функций.
\item Гильбертовы пространства $L_2[а,b]$ и $L_2(\R)$.
\item Свёртка и её свойства. Аппроксимативная единица ($\de$ образная последовательность) и теорема о ней. Примеры. Теоремы
Вейерштрасса о приближении полиномами и тригонометрическими многочленами. Замкнутость тригонометрической системы в $L_2[-\pi,\pi]$.
\item Тригонометрические ряды Фурье и их свойства: линейность, инвариантность относительно сдвигов, симметрии, сжатий,
дифференцирования; ряд Фурье свёртки, равенство Парсеваля. почленная интегрируемость. Стремление к нулю коэффициентов Фурье
интегрируемых по Мак-Шейну функций.
\item Представление частичных сумм. Ядро Дирихле. Признак Дини и следствия из него. Принцип локализации Римана.
\item Признак Дирихле Жордана. Суммирование тригонометрических рядов методами Чезаро Фейера и Абеля Пуассона.
Преобразование Фурье.
\end{nums}

\section*{4 семестр}

\begin{nums}{-2}
\item Брусы и простые множества в $\R^n$, их мера и ее свойства.
\item Мера Жордана. Измеримые множества и их свойства.
\item Кратный интеграл Римана, его определение и простейшие свойства. Связь интегрируемости по Риману и ограниченности.
\item Суммы Дарбу и их свойства. Критерий интегрируемости Дарбу.
\item Множества меры нуль по Лебегу. Критерий интегрируемости Лебега.
\item Некоторые свойства кратного интеграла Римана.
\item Теоремы о связи интеграла Римана и меры Жордана.
\item Теоремы о сведении кратных интегралов к повторным.
\item Замены переменных в кратном интеграле: одновременная интегрируемость
$f(x)$ и $f\br{x(t)} \cdot \hm{\det\hr{\pf{x_i}{t_j}}}$.
\item Общая теорема о замене переменных в кратном интеграле.
\item Несобственный кратный интеграл.
\item Криволинейные интегралы I и II рода, их свойства.
\item Формула Грина.
\item Потенциальные векторные поля. Условия независимости криволинейного интеграла II
рода от пути интегрирования.
\item Поверхности в $\R^3$, их площадь. Поверхностные интегралы I и II рода, их свойства.
\item Кусочно-гладкие поверхности. Формула Остроградского Гаусса.
\item Ротор векторного поля. Формула Стокса.
\item Пространство, сопряженное к $\R^n$. Антисимметричные билинейные и полилинейные формы
и их свойства. Внешнее произведение.
\item Касательное пространство. Касательное отображение. Дифференциальные формы.
Внешнее дифференцирование. Замена переменных.
\item Интеграл от дифференциальной формы по цепи. Обобщенная формула Стокса и её
частные случаи.
\end{nums}

\medskip\dmvntrail
\end{document}
