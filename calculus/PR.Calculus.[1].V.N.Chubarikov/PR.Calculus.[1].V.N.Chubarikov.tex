\documentclass[a4paper]{article}
\usepackage[utf8]{inputenc}
\usepackage[russian]{babel}
\usepackage[simple]{dmvn}

\newcommand{\lima}{\uliml{n\ra\infty}}
\newcommand{\limb}{\lliml{n\ra\infty}}

\title{Задачи для подготовки к коллоквиуму №1\\ по математическому анализу}
\author{Лектор В.\,Н.\,Чубариков}
\date{I семестр, 2004~г.}

\begin{document}
\maketitle
\centerline{\footnotesize Набор текста: П. Рахмонов, вёрстка: DMVN Corporation.}

\begin{nums}{-1}
\item Пусть $x$, $y\in[a, b]$. Тогда $|x-y|\le b-a$.
\item  Доказать равенство
  \begin{equation*}
    \frac{x+y+|x-y|}{2}=\max\hc{x, y}.
  \end{equation*}
\item Пусть $F(1)=2$, $F(n)=F(n-1)+\frac{1}{2}$ для всех
натуральных чисел $n$, больших единицы. Тогда
$F(n)=2+\frac{n-1}{2}$.

\item Построить такие множества $B\subset A\subset X$ и
отображение $F: X\ra X$, что
$$
F(A\wo B)\neq F(A)\wo F(B)
$$

\item Пусть $F: X\ra Y$. Тогда следующие утверждения
эквивалентны:\\
а) $F$ -- вложение (инъективное отображение);\\
б) $F^{-1}(F(A))=A$ для любого подмножества $A\subset X$;\\
в) $F(A\cap B)=F(A)\cap F(B)$ для любой пары подмножеств $A$, $B$ в $X$.\\
г) $F(A)\cap F(B)=\varnothing$ для любой пары подмножеств $A$, $B$ в $X$ для которой $A\cap B=\varnothing$.\\
д) $F(A\wo B)=F(A)\wo F(B)$ для любой пары подмножеств $A$, $B$ в $X$ для которой $B\subset A$.

\item Рассмотрим отображения $f\cln A\ra B$, $g\cln B\ra C$ и $h\cln C\ra D$. Пусть отображения $f\circ g$ и $g\circ h$
биективны. Доказать, что отображения $f$, $g$ и $h$ сами биективны.

\item  Множество всех конечных подмножеств натуральных чисел $\N$ счетно.

\item  Для того, чтобы $X$ было бесконечно, необходимо и
достаточно, чтобы для каждого отображения $f\cln X\ra X$
существовало такое непустое подмножество $A\subset X$, что $A\neq
X$ и $f(A)\subset A$.

\emph{Указание:} Если бы $f$ обладало этим свойством и $X$ было
бы бесконечным, то $X$ было бы счетным. Тогда можно считать, что
$X=\N$ и $f(n)>n$ при $n\ge 0$; это приводит к
противоречию).

\item Пусть $E$ --- бесконечное множество, $D\subset E$, $D$
--- не более чем счетное множество и $E\wo D$. Тогда
множества $E\wo D$ и $D$ равномощны.

\item Множество всех иррациональных чисел равномощно
множеству всех вещественных чисел $\mathbb{R}$.

\item Доказать, что $[a, b]\sim(a, b)$, $[a, b]\sim[a, b)$.

\item Доказать, что $\sup A=-\inf(-A)$, $\sup A\cup
B=\max(\sup A, \sup B)$.

\item Пусть определены выражения в правых частях соотношений.
Тогда справедливы следующие утверждения:
\begin{enumerate}
\item $ \inf\limits_{x\in A}(-f(x))=-\supl{x\in A} f(x)$;
\item $ \supl{x\in A} (f(x)+g(x))\leq \supl{x\in A} f(x)+\supl{x\in A} g(x)$;
\item $ \supl{x\in A}(f(x)+g(x))\ge \supl{x\in A}f(x)+\inf\limits_{x\in A}g(x)$, если $\sup g(x)$ существует;
\item $ \supl{x\in A}(f(x)+c)=c+\supl{x\in A} f(x)$ для любого вещественного числа $c$;
\item $\supl{x_1\in A_1}\hr{\supl{x_2\in A_2}f(x_1, x_2)}=\supl{(x_1, x_2)\in A_1\times A_2}f(x_1, x_2)$;
\item $\supl{(x_1, x_2)\in A_1\times A_2}\br{f(x_1)+f(x_2)}=\supl{x_1\in A_1}f(x_1)+\supl{x_1\in A_1}f(x_2)$.
\end{enumerate}

\item Пусть $B$ --- непустое ограниченное множество
вещественных чисел, $b =\sup B$ и $b\notin B$. Тогда $b$ является
предельной точкой множества $B$.

\item Пусть $\{x_n\}$ --- бесконечно малая последовательность
отрицательных вещественных чисел. Тогда для каждого натурального
числа $m$ существует бесконечно много номеров $n\ge m$ таких, что
$x_n\le x_m$.

\item Доказать, что а) $\liml{n\ra
\infty}\frac{n^k}{2^n}=0$, где $k$~--- постоянная; б)
$\liml{n\ra\infty}n(a^{1/n}-1)=\ln a$, $a$>0.

\item Пусть $\liml{n\ra\infty}x_n=+\infty$.
Тогда $\lim\limits\frac{x_1+\ldots +x_n}{n}=+\infty$.

\item Пусть $p_n$>0 для всех $n\in \N$ и
$\liml{n\ra\infty}(p_1\cdot\ldots\cdot
p_n)^{1/n}=p$.

\item Исходя из равенства
$\liml{n\ra\infty}\hr{1+\frac{1}{n}}^n=e$, доказать,
что $\liml{n\ra\infty}\frac{n}{(n!)^{1/n}}=e$.

\item Доказать, что последовательность
$a_n=\hr{1+\frac{1}{n}}^{n+p}$ строго убывает тогда и только тогда,
когда  $p\ge \frac{1}{2}$.

\item Для любого рационального числа $r$ с условием $|r|<1$
справедливо неравенство $1+r\le e^r\le 1+\frac{r}{1-r}$.

\item Доказать, что
$\liml{n\ra\infty}\left(\frac{1}{n+1}+\frac{1}{n+2}+\ldots
+\frac{1}{2n}\right) =\ln 2$.

\item Пусть $x_n$ --- последовательность с ограниченным изменением, т.е. существует такое $c>0$, что для всех $n\in
\N$ справедливо неравенство
$$\sum_{k=1}^{n-1}|x_{k+1}-x_{k}|<c.$$
Тогда последовательность $x_n$ сходится.

\item Пусть $0\le x_{m+n}\le x_m+x_n$. Тогда существует предел $\liml{n\ra\infty}\ \frac{x_n}{n}$.

\item а) $\lima(a_n+b_n)\leq \lima a_n+\lima b_n$, если последние пределы существуют.

б) Если $\liml{n\ra\infty}a_n=a$ и $\lima b_n=b$, то $\lima a_nb_n=ab$.

в) $\lima a_n=-\limb (-a_n)$

\item Пусть $\liml{n\ra\infty} a_n=+\infty$. Тогда существует $\min\limits_{n\in\N}a_n$.

\item Пусть $\liml{n\ra\infty} a_n=a$. Тогда последовательность $\{a_n\}$ имеет либо наибольший, либо
наименьший элемент, либо тот и другой.

\item Пусть $s_n=a_1+a_2+\ldots +a_n\ra+\infty$, $a_k>0$, $\liml{n\ra\infty} a_n=0$. Тогда множество
предельных точек дробных частей $\{s_n\}$ совпадает с отрезком
$[0; 1]$.

\item Пусть $\liml{n\ra\infty}
(s_{n+1}-s_n)=0$ и не существует ни конечного, ни бесконечного предела $\liml{n\ra\infty} s_n$, и пусть
$$l=\limb s_n, \quad L=\lima s_n.$$

Тогда последовательность $s_m$ расположена всюду плотно на отрезке $[l; L]$.

\item a) Пусть $a_n>0$ и $\liml{n\ra\infty}
a_n=0$. Тогда существует бесконечно много номеров $n$ таких, что
$$a_n >\max(a_{n+1}, a_{n+1}, a_{n+1}, \ldots ).$$

б) Пусть $a_n>0$ и $\limb a_n=0$. Тогда существует бесконечно много номеров $n$ таких, что
$$a_n<\min(a_{1}, a_{2}, \ldots a_{n-1}).$$
\end{nums}

\medskip
\dmvntrail

\end{document}
