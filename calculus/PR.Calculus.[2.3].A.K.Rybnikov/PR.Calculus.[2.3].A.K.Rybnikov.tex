\documentclass[a4paper]{article}
\usepackage{dmvn}

\title{Материалы с контрольных и зачетов\\ по математическому анализу}
\author{Преподаватель\т А.\,К.\,Рыбников}
\date{II---III семестр. 2005 г.}

\def\dx{\,dx}

\begin{document}
\maketitle
\centerline{\small Набрано П. Рахмоновым, отредактировано и свёрстано DMVN Corporation.}

\medskip
\dmvntrail

\section{II семестр}

\subsection{Контрольная 11 марта 2005 г.}

Найти следующие интегралы:

\eqn{\int\frac{x^{11}\dx}{1+x^8}}
\eqn{\int \sqrt{1-x^2}\dx}
\eqn{\int \frac{\sin^2 x}{\cos^2 x}\sqrt{\tg x}\dx}
\eqn{\int\frac{x+\sqrt{1+x-x^2}}{x+2}\dx}
\eqn{\int\frac{dx}{\sqrt[3]{2x^2+3}}}
\eqn{\int\frac{\cos 2x}{\sin x-2\cos x}\dx}
\eqn{\int\frac{dx}{x^4(x^3+1)^2}}
\eqn{\int(|1+x|-|1-x|)\dx}

\subsection{Контрольная 16 мая 2005 г.}

\subsubsection{Вариант 1}

1. Исследовать на абсолютную и условную сходимость:
$$
\int\limits_0^{+\infty}\frac{\sqrt[3]{x}\sin x}{1+\sqrt x}\dx.
$$

2. Фигура, ограниченная линиями
$$
\left\{
\begin{array}{lll}
x&=&a(t-\sin t) \\
y&=&a(1-\cos t)
\end{array}
\right. \ \ \text{(циклоида)}\qquad\text{и}\qquad
y=4a-\frac{2}{\pi}x.
$$
вращается вокруг оси $Ox$. Найти объем тела вращения.

3. Найти наибольшее и наименьшее значения функции
$u=x^2-2ax+y^2-2ay+z^2-2az\ \ (a>0)$ в полушаре $x^2+y^2+z^2\le
4a^2\ \ (z\ge 0)$.

4. Преобразовать уравнение $\displaystyle
\frac{\partial^2z}{\partial x^2}+\frac{\partial^2z}{\partial
x\partial y}+\frac{\partial z}{\partial x}=z$ приняв
$u=\displaystyle\frac{x+y}{2}$, $v=\displaystyle\frac{x-y}{2}$ за
новые аргументы, а $w=ze^y$ за новую функцию.

\subsubsection{Вариант 2}

1. Исследовать на абсолютную и условную сходимость:
$$
\int\limits_0^{+\infty}\sqrt[3]{x}\sin x^2dx.
$$

2. Найти площадь фигуры, ограниченной линиями $x^2+y^2=6$ и
$x^2+y^2=2x+2y\ \ (x^2+y^2\ge 6)$.

3. Имеет ли функция $u=xy+yz+xz$ условный экстремум в точке
$M(1;1;1)$, если
$$2x^3y^2z+4x^2+5y^2+6z^2-17=0.$$

4. Преобразовать уравнение
$\displaystyle\frac{\partial^2z}{\partial
x^2}-2\frac{\partial^2z}{\partial x\partial y}+\frac{\partial^2
z}{\partial y^2}=0$ приняв $u=x+y$, $v=\displaystyle\frac{x}{y}$
за новые аргументы, а $w=\frac{z}{x}$ за новую функцию.

\subsubsection{Вариант 3}

1. Исследовать на сходимость интеграл:
$$
\int_e^{+\infty}\frac{dx}{\sqrt[3]x\ln x}.
$$

2. Найти площадь поверхности фигуры, образованной вращением кривой
$r^2=a^2\cos 2\varphi$ вокруг полярной оси.

3. Найти условный экстремум функции $u=xyz$, если $x^2+y^2+z^2=3$.

4. Приняв $u$ и $v$ за новые аргументы, преобразовать уравнение
$$
(1+x^2)\frac{\partial^z}{\partial
x^2}+(1+y^2)\frac{\partial^z}{\partial y^2}+x\frac{\partial
z}{\partial x}+y\frac{\partial z}{\partial y}=0
$$
если $u=\ln (x+\sqrt{1+x^2})$, $v=\ln (y+\sqrt{1+y^2})$

\subsection{Зачёт \No 1}

1. Исследовать на абсолютную и условную сходимость:
$$
\int_0^{+\infty}x^2\cos (e^x)dx
$$

2. Найти объем тела, ограниченного поверхностью, образованной
вращением $x^2-y^2=3a^2$ и $y^2=2ax$\ $(a>0)$ вокруг оси $Ox$.

3. Найти точки условного экстремума функции $u=xy^2z^3$, если
$x+y^2+z^3=1$?

4. Преобразовать уравнение $\displaystyle\frac{\partial^2
z}{\partial x^2}-2\frac{\partial^2z}{\partial x\partial
y}+\frac{\partial^2z}{\partial y^2}=0$ приняв $u=x+y$,
$v=\displaystyle\frac{y}{x}$ за новые аргументы, а
$w=\displaystyle\frac{z}{x}$ за новую функцию.

\subsection{Зачёт \No 2}

1. Найти длину дуги кривой, заданной в полярной системе координат
уравнением
$$
\varphi =\frac{1}{2}\left(r+\frac{1}{r}\right),\quad (1\le r\le
3)
$$

2. Имеет ли функция $u=xyz$ условный экстремум в точке $M(2;1;2)$,
если $x+y+z=5$, $xy+yz+zx=8$?

3. Преобразовать уравнение $\displaystyle
\frac{\partial^2z}{\partial x^2}-2\frac{\partial^2z}{\partial
x\partial y}+\left(1+\frac{y}{x}\right)\frac{\partial^2z}{\partial
y^2}=0$ приняв $u=x$, $v=x+y$ за новые аргументы, а $w=x+y+z$ за
новую функцию.

\section{III семестр}

\subsection{Контрольная работа 26 сентября 2005г.}

\subsubsection{Вариант 1}


1. Исследовать на сходимость:

%\begin{problem}
$$
\sumnui\left(\frac{n}{3n-1}\right)^{2n-1}\sqrt[n]{\frac{n+1}{n}}
$$
%\end{problem}
%\begin{problem}
$$
\sumnui\left(\sqrt[3]{n+3}-\sqrt[3]{n}\right)\sqrt[3]{\ln^2\left(\cos\frac{1}{n}\right)}
$$
%\end{problem}

2. Исследовать на абсолютную и условную сходимость:
%\begin{problem}
$$
\sum_{n=2}\frac{\sqrt{n}-1}{\sqrt{n}}\ln\left(1+\frac{(-1)^n}{\sqrt[4]{n^3}}\right)
$$
%\end{problem}
%\begin{problem}
$$
\sum_{n=2}^\infty\frac{\sin\frac{n\pi}{4}}{n\ln n-\sqrt{\ln^3n}}
$$
%\end{problem}
%\begin{problem}

3. Определить область абсолютной и условной сходимости:
$$
\sumnui\frac{3+(-1)^n}{n}x^n
$$
%\end{problem}

\subsubsection{Вариант 2} \setcounter{problem}{0}

1. Исследовать на сходимость:
%\begin{problem}
$$
\sumnui\frac{n^{n+\frac{1}{n}}}{(n!)^2}
$$
%\end{problem}
%\begin{problem}
$$
\sumnui\frac{e^\frac{2}{\sqrt n+1}-1}{\sqrt[5]{n+3}-\sqrt[5]{n-3}}
$$
%\end{problem}

2. Исследовать на абсолютную и условную сходимость:
%\begin{problem}
$$
\sum_{n=2}^\infty\frac{\tg(\pi\sqrt{n^2+2})}{\sqrt{\sqrt{n}+(-1)^n}}
$$
%\end{problem}
%\begin{problem}
$$
\sumnui\frac{(-1)^n}{n}\cos\frac{1}{n}
$$
%\end{problem}
%\begin{problem}

3. Определить область абсолютной и условной сходимости:
$$
\sumnui\frac{3^n+(-2)^n}{n}(x+1)^n
$$
%\end{problem}

\subsection{Контрольная работа 31 октября 2005г.}

\subsubsection{Вариант 1}

1. Исследовать на равномерную сходимость:
$$
f_n(x)=1-x^{1/2n},\quad (0<x<1)
$$
$$
f_n(x)=nx(1-x)^n,\quad (0\le x\le 1)
$$
$$
\sumnui \frac{(-1)^nx}{n^2+x},\quad (0<x<+\infty)
$$
$$
\sumnui \frac{(-1)^n}{n^x}\sin\frac{1}{\sqrt n}, \quad (0\le x<
+\infty)
$$

2. Исследовать на непрерывность:
$$
f(x)=\sumnui ne^{-nx}, \quad (x>0)
$$

3. Вычислить сумму ряда
$$
\sumnui n(n+2)x^n
$$

\subsubsection{Вариант 2}

1. Исследовать на равномерную сходимость:
$$
f_n(x)=\frac{\sqrt{x+n}-\sqrt x}{n},\quad (0<x<+\infty)
$$
$$
f_n(x)=n(x^{1/n}-1),\quad (1\le x\le 2)
$$
$$
\sumnui \frac{(-1)^n}{n}\arctg (x^n),\quad (0<x<+\infty)
$$
$$
\sumnui \frac{xe^{-nx^2}}{\sqrt{n\ln^3(n+1)}}
$$

2. Исследовать на непрерывность:
$$
f(x)=\sumnui \frac{\ln(1+nx)}{nx^n}, \quad (x>1)
$$

3. Вычислить сумму ряда
$$
\sumnui (-1)^{n-1}\frac{x^{2n}}{n(2n-1)}
$$

\subsection{Контрольная работа 9 декабря 2005г.}

\subsubsection{Вариант 1}

1. Исследовать на равномерную сходимость:
$$
\int\limits_1^{+\infty}e^{-\alpha^2x}\frac{\cos x}{\sqrt x}dx,
\quad \alpha\in [-1, 1]
$$
$$
\int\limits_0^{+\infty}e^{-\alpha^2(x^2+4)}\arctg \alpha dx,\quad
\alpha \in \mathbb{R}
$$
$$
\int\limits^{+\infty}_0\frac{\sin x^3}{1+x^\alpha}dx,\quad \alpha
>0
$$

2. Вычислить:
$$
\int\limits_0^{+\infty}\frac{\arctg \alpha x}{x(1+x^2)}dx
$$
$$
\int\limits_0^1\frac{dx}{\sqrt[3]{1-x^3}}
$$

\subsubsection{Вариант 2}

1. Исследовать на равномерную сходимость:
$$
\int\limits_1^{+\infty}\frac{\cos \alpha x}{\sqrt x}dx,\quad
\alpha >0
$$
$$
\int\limits_0^{+\infty}\frac{x\arctg
\frac{\alpha}{x}}{1+x^2}dx,\quad \alpha\in (0, 2)
$$
$$
\int\limits_0^1\frac{1}{x^\alpha}\sin\frac{1}{x^2}dx, \quad
\alpha\in (0, 3)
$$

2. Вычислить:
$$
\int\limits_0^{+\infty}e^{-\sqrt[3]{x}}dx
$$
$$
\int\limits_0^1\frac{\ln(1-\alpha ^2x^2)}{\sqrt{1-x^2}}dx
$$

\subsection{Зачет \No 1}

Исследовать на абсолютную и условную сходимость:
$$
\sumnui \frac{\ln (e^n+n^2)}{n^2\ln^2(n+1)}
$$
$$
\sumnui \frac{(-1)^n}{\sqrt[4]{n}+\frac{(-1)^n}{2\sqrt[4]{n}}}
$$
$$
\sumnui \frac{(-1)^n}{n}\arctg \left(1+\frac{1}{\sqrt{n}}\right)
$$
$$
\sumnui \left( e^\frac{\cos n}{\sqrt n}-1\right)
$$
$$
\sumnui \frac{\sin n}{\sqrt n +\sin n}
$$

\end{document}
