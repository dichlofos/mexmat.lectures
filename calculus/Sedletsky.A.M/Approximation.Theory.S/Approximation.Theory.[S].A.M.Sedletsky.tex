\documentclass[a4paper]{article}
\usepackage[utf]{dmvn}
\usepackage{wrapfig}

\def\defin{\noindent\textbf{Опр:}\ }

\begin{document}
\dmvntitle{Введение}
{в теорию аппроксимации}
{лектор -- Анатолий Мечиславович Седлецкий}
{}
{Москва, 2003\,г.}

\centerline{\epsfbox{mmlogo.2}}

\renewcommand{\abstractname}{Предисловие}
\begin{abstract}

Эти лекции основаны на моих собственных рукописных конспектах спецкурса, читавшегося в период с 01.10.2003 по
01.05.2004. Материал здесь изложен так, как читал его сам Анатолий Мечиславович. Изменения, внесенные мною,
минимальны (в основном это дополнительные пояснения). На мой взгляд, изложение ведется довольно ровно и
понятно. Но все же доказательства не разжевываются до конца, и некоторые моменты (явно или неявно)
предлагается додумать читателю.

Начинаются лекции с того материала, что должен проходиться студентами
механико-математического факультета в третьем семестре (первая
глава). Далее идет материал, тесно связанный с программой по
комплексному и функциональному анализу 5--6 семестров (преобразование
Лапласа, классы Харди это одни из основных инструментов теории
аппроксимации). В 3--4 главах приведены результаты, полученные
сравнительно недавно (1950--1980~гг). Не все они снабжены
доказательствами по причине сложности этих доказательств (учитывая то,
что спецкурс читался для студентов 2-го курса не знакомых еще ни с
комплексным, ни с функциональным анализом). 5-я глава состоит из
материала, читавшегося уже на третьем курсе и называется так, как
предложил называть её Анатолий Мечиславович, а именно <<Ликбез>>.

Разумеется, я не могу смело
заявлять, что данные лекции не содержат ошибок (очипаток), поэтому, если таковые будут замечены Вами, прошу
немедленно сообщать мне об этом (или сам исправлю, или спрошу у Анатолия Мечиславовича).

\end{abstract}


\rightline{\small \copyright~Юхименко Александр (401~гр). E-mail: \texttt{alesandro1985@mail.ru.}}
\rightline{\small Размещено на сайте DMVN Corporation (\dmvnwebsite).}
\rightline{\small По всем вопросам пишите на \dmvnmail.}


\begin{center}
\textbf{Используемые обозначения}
\end{center}

Часто, говоря о пространстве $L^p$, $1\le p<\infty$, возникает
необходимость говорить и о сопряженном пространстве $L^q$, где $p$
и $q$ связаны соотношением
\begin{equation}\label{100}
\frac1p+\frac1q=1.
\end{equation}
Поэтому если где-либо в лекциях фигурирует $p$, а потом появляется
$q$, то нужно понимать, что $q$ находится из~\eqref{100}.

\medskip

У человека, читающего эти лекции, может возникнуть вопрос: почему
не рассматриваются пространства $L^p$, $0\le p<1$? Ответ: на таких
пространствах нельзя ввести норму, поэтому и рассматривать их нам
бесполезно.

\medskip

$\wh{f}$ стандартное обозначение для преобразования Фурье (е
$\wh{f}(x)=\frac1{2\pi}\int_\R e^{-ixt}f(t)dt$).

\newpage

\tableofcontents
\newpage

\section{Часть I}

\subsection{Предмет теории аппроксимации}

\begin{df}
  \emph{Метрическое пространство} это совокупность элементов, для каждой пары $(x_1,x_2)$ которой
  определена функция расстояния $\rho(x_1,x_2)$, обладающая следующими свойствами:
  \begin{enumerate}
  \item Неотрицательная определенность и невырожденность:
    $$\rho(x_1,x_2) \ge 0,\quad \rho(x_1,x_2)=0\Lra x_1=x_2;$$
  \item Неравенство треугольника: $\rho(x_1,x_3)\le\rho(x_1,x_2)+\rho(x_2,x_3)$;
  \item Симметричность: $\rho(x_1,x_2)=\rho(x_2,x_1)$.
  \end{enumerate}
\end{df}

\begin{df}
  Подпространство $Y$ метрического пространства $X$
  называется \emph{плотным} (в $X$), если $\fa x\in X$, $\fa\ep>0$ $\exi y\in Y\cln \rho(x,y)<\ep$
  (в этом случае говорят, что $y$ приближает элемент $x$ с точностью~$<\ep$).
\end{df}

Теория аппроксимации изучает плотные множества в различных метрических пространствах.

Частный случай метрического пространства пространство нормированное.

\begin{df}
  \emph{Нормированное пространство} это линейное пространство, снабженное функцией
  <<норма>> (обозначается~$\hn{\cdot}_X$), обладающей свойствами:
  \begin{enumerate}
  \item
    $\fa x\in X\quad\hn{x}\ge 0$, $\hn{x}=0\Lra x=0$;
  \item
    $\fa x\in X\quad\hn{kx}=\hm{k}\hn{x},\quad k\in\R;$
  \item
    $\fa x\in X$, $y\in Y\quad\hn{x+y}\le\hn{x}+\hn{y}.$
  \end{enumerate}
\end{df}

\noindent\textbf{Примеры нормированных пространств:}

\begin{denote}
  $C[a,b]$ -- пространство непрерывных на $[a,b]$ функций с нормой
  $\hn{f}_{C[a,b]}=\maxl{x\in[a,b]}|f(x)|$;
\end{denote}

\begin{denote}
  $L^p(a,b),\quad 1\le p<\infty$ пространство функций, интегрируемых в
  степени $p$ по Лебегу на $(a,b)$ (е
  $\exi\int_a^b|f(x)|^pdx<\infty$).
  $$\hn{f}_{L^p}=\hr{\int_a^b|f(x)|^pdx }^{1/p}.$$
\end{denote}

\begin{df}
  Если имеется последовательность
  $\{x_n\}_{n=1}^{\infty}\subset X$, говорят, что она
  \emph{сходится} к элементу $x$ ($x_n\xra{n\ra\infty} x$), если
  $\hn{x_n-x}_X\xra{n\ra\infty} 0$.
\end{df}

\subsection{Полиномы Бернштейна, плотность алгебраических
многочленов}

\begin{df}
  \emph{Полиномом Бернштейна} степени $n$ для функции
  $f$ называется следующий многочлен (степени $\le n$)
  $$B_n(x)=\sum\limits_{k=0}^n C_n^k f\hr{\frac{k}{n}}x^k(1-x)^{n-k}.$$
\end{df}
\begin{theorem}[Бернштейн]
$\fa \;f\in C[0,1]\quad B_n(x)\xra{n\ra\infty}
f(x).$
\end{theorem}
\textbf{Доказательство:} Воспользуемся всем известной формулой
\begin{equation}\label{1}
\sum_{k=0}^n C_n^k a^k b^{n-k}=(a+b)^n,\quad n\in\mathbb{N}.
\end{equation}

Положим $a=x$, $b=1-x$, получим

\begin{equation}\label{2}
\sum_{k=0}^n C_n^k x^k (1-x)^{n-k}=1.
\end{equation}

Продифференцируем (\ref{1}) по $a$, получим
\begin{equation}\label{3}
\sum_{k=0}^n C_n^k ka^{k-1} b^{n-k}=n(a+b)^{n-1}.
\end{equation}

Поделим левую и правую часть этого выражения на $n$, на $a$ умножим и снова положим $a=x$, $b=1-x$:
\begin{equation}\label{4}
\sum_{k=0}^n C_n^k \frac{k}{n}x^k(1-x)^{n-k}=x.
\end{equation}

Умножим (\ref{3}) на $a$ и снова продифференцируем
\begin{equation*}
\sum_{k=0}^n C_n^k
k^2a^{k-1}b^{n-k}=n(a+b)^{n-1}+n(n-1)a(a+b)^{n-2}.
\end{equation*}

Опять делаем замену переменных и получаем
\begin{equation}\label{5}
\sum_{k=0}^n C_n^k
\hr{\frac{k}{n}}^2x^k(1-x)^{n-k}=\frac{x}{n}+\hr{1-\frac1{n}}x^2.
\end{equation}

Воспользуемся формулами (\ref{2}), (\ref{4}), (\ref{5}) и получим
\begin{equation}\label{6}
\sum_{k=0}^n C_n^k
\hr{\frac{k}{n}-x}^2x^k(1-x)^{n-k}=\frac{x}{n}+\hr{1-\frac1{n}}x^2-2x^2+x^2=\frac{x(1-x)}{n}.
\end{equation}

Попытаемся оценить разность между $B_n$ и $f(x)$:
\begin{equation*}
|B_n(x)-f(x)|\le\sum\limits_{k=0}^n
C_n^k\hm{f\hr{\frac{k}{n}}-f(x)}x^k(1-x)^{n-k}=\sum_1+\sum_2.
\end{equation*}

В $\sum_1$ суммирование ведется по $k$ таким, что
$\hm{\frac{k}{n}-x}\le\frac1{\sqrt[4]{n}}$, а в $\sum_2$ по остальным. По-отдельности рассмотрим $\sum_1$ и $\sum_2$.

Т.к. $f$ непрерывна на отрезке, то она ограничена, т.е. $\exi
M:\quad|f(x)|\le M$, поэтому, пользуясь (\ref{6}):
\begin{multline}\label{8}
\sum_2\le 2M\sum_{|k/n-x|\ge \frac1{\sqrt[4]{n}}}
\frac{|k/n-x|^2}{|k/n-x|^2}C_n^k
x^k(1-x)^{n-k}\le\\
\le2M\sqrt{n}\sum\limits_{k=0}^n C_n^k
x^k(1-x)^{n-k}\hm{\frac{k}{n}-x}^2=2M\sqrt{n}\frac{x(1-x)}{n}\le\frac
{M}{2\sqrt{n}}.
\end{multline}

\begin{equation}\label{9}
\sum_1\le\omega_f\hr{\frac1{\sqrt[4]{n}}}\sum_{k=0}^{\infty}C_n^k
x^k(1-x)^{n-k}=\omega_f\hr{\frac1{\sqrt[4]{n}}}.
\end{equation}

Соединяя оценки (\ref{8}) и (\ref{9}), получаем, что
\begin{equation*}
|B_n(x)-f(x)|\le\omega_f\hr{\frac1{\sqrt[4]{n}}}+\frac
{M}{2\sqrt{n}}\xra{n\ra\infty} 0\qquad\fa
x\in[0,1].
\end{equation*}
$\blacksquare$

Из этой оценки сразу следует нужная нам теорема.
\begin{theorem}[Вейерштрасс]
Алгебраические многочлены плотны в пространстве $C[a,b]$.
\end{theorem}

\noindent\textbf{Замечание:} Теорема доказана для случая
$[a,b]=[0,1]$, но общий случай сводится к этому линейной заменой
переменных, переводящей $[a,b]$ в $[0,1]$.

\begin{theorem} Алгебраические многочлены плотны в
пространстве $L^p[a,b]$.
\end{theorem}
\textbf{Доказательство:} Пусть $f\in L_p[a,b]$. Воспользуемся
широко известным фактом из действительного анализа: $C[a,b]$
плотно в $L^p[a,b]$. Поэтому можно найти $\ph\in C[a,b]:\quad
\hn{f-\ph}_{L^p[a,b]}\le\frac{\ep}{2}.$ Для $\ph(x)$
найдем многочлен $P(x)$ такой, что
$\hn{\ph-P}_{C[a,b]}\le\frac{\ep}{2(b-a)^{1/p}}$. Тогда
$\hn{\ph-P}_{L^p[a,b]}\le\frac{\ep}2.$ И поэтому
$\hn{f-P}_{L^p[a,b]}\le\hn{f-\ph}_{L^p[a,b]}+\hn{\ph-P}_{L^p[a,b]}\le\frac{\ep}2+\frac{\ep}2\le\ep.\quad\blacksquare$

\medskip Введем новое пространство: $C_0[a,b]=\{f\in C[a,b]:\quad f(0)=0\}$.

\begin{theorem}
Многочлены без свободного члена плотны в $C_0[0,1]$ и в
$L^p[0,1]$.
\end{theorem}
\textbf{Доказательство:} Пусть $f\in C[0,1]\Ra\exi
P_n(x)\ra f.$
$$P_n(x)=a_0^{(n)}+a_1^{(n)}x+\ldots+a_{k_n}^{(n)}x^{k_n}.$$
$P_n(0)\ra f(0)\Ra a_0^{(n)}\ra 0$ Пусть
$\wt{P_n}(x)$ тот же $P_n(x)$ только без $a_0^{(n)}$.
$$f\in
C_0[0,1]\Ra\hn{f-\wt{P_n}}=\hn{f-P_n+a_0^{(n)}}\le\hn{f-P_n}+\hm{a_0^{(n)}}.$$
Таким образом, показано, что многочлены без свободного члена
плотны в $C_0[0,1].$

Для доказательства второго утверждения теоремы остается лишь
заметить, что $C_0[0,1]$ плотно в $L^p[0,1].\quad\blacksquare$

\subsection{Скорость приближения функций многочленами}
\textbf{Опр:} \emph{Ядром Дирихле} порядка $n$ называется
$$D_n(t)=\frac12+\cos t+\ldots+\cos nt=\frac{\sin\hr{n+\frac12}t}{2\sin \frac t2}.$$

\begin{df}
  \emph{Ядром Фейера} порядка $n$ называется
  \begin{multline*}
    F_n(t)=\frac{D_0(t)+D_1(t)+\dots+D_{n-1}(t)}{n}=\frac12+\hr{1-\frac1n}\cos
    t+\dots+\hr{1-\frac{n-1}{n}}\cos
    (n-1)t=\frac1{2n}\left(\frac{\sin \frac{nt}2}{\sin\frac t2}\right)^2.
  \end{multline*}
\end{df}

\begin{df}
  \emph{Ядром Джексона} порядка $n$
  называется
  $$J_n(t)=\frac{6n}{2n^2+1}F_n^2(t).$$
\end{df}

\noindent\textbf{Свойства ядра Джексона:}
\begin{enumerate}
\item
Ядро Джексона - четный тригонометрический полином порядка $2n-2.$ Ну
действительно,
\begin{equation}\label{10}
J_n(t)=\frac{6n}{2n^2+1}\hr{\frac14+\sum_{j=1}^{n-1}
\hr{1-\frac jn}^2\cos^2 jt+\sum_{j=1}^{2n-2}C_j^{(n)}\cos
jt}.
\end{equation}
\item
$\frac1{\pi}\int_{-\pi}^{\pi}J_n(t)dt=1.$ Действительно, из
(\ref{10}) видно, что
\begin{multline*}
a_0^{(n)}=\frac{6n}{2n^2+1}\hr{\frac14+\frac12\sum_{j=1}^{n-1}
\hr{1-\frac jn}^2}=\frac{3n}{2n^2+1}\hr{\frac12+\frac1{n^2}\sum_{j=1}^{n-1}j^2}=\\
=\frac{3n}{2n^2+1}\hr{\frac12+\frac{n(n-1)(2n-1)}{6n^2}}=\frac12.
\end{multline*}
Дальше остается лишь воспользоваться тем, что
$\int_{-\pi}^{\pi}\cos nt \;dt=0\quad(n\in\mathbb{N}).$
\item
$$\frac1{\pi}\int\limits_0^{\pi} tJ_n(t)\;dt\le\frac{3\pi}{4n}.$$

Для доказательства этого факта воспользуемся двумя известными
неравенствами: $|\sin nt|\le n|\sin t|$ (легко доказывается по
индукции), $\sin t\ge \frac2{\pi}t\quad (0\le t\le\pi/2).$
$$\frac1{\pi}\int_0^{\pi}
tJ_n(t)\;dt=\frac1{\pi}\hr{\int\limits_0^{\frac{\pi}{n}}+\int\limits_{\frac{\pi}{n}}^{\pi}}=I_1+I_2.$$

\begin{equation}\label{11}
I_1\le\frac1{\pi}\frac{3}{2n(2n^2+1)}\int\limits_0^{\frac{\pi}{n}}
t\left(\frac{n\sin t/2}{\sin t/2}\right)^2dt=
\left.\frac1{\pi}\frac{3n^3}{2(2n^2+1)}\frac{t^2}2\right|_0^{\frac{\pi}2}<\frac38\frac{\pi}n.
\end{equation}
\begin{equation}\label{12}
I_2\le\frac1{\pi}\intl{\frac{\pi}{n}}{\pi}
t\hr{\frac1{t/\pi}}^4dt=\frac{3\pi^3}{4n^3}\intl{\frac{\pi}{n}}{\pi}t^{-3}dt<\frac{3\pi^3}{4n^3}\frac12\hr{\frac{\pi}n}^{-2}=\frac{3\pi}{8n}
\end{equation}

Соединяя оценки (\ref{11}) и (\ref{12}), получаем требуемое.
\end{enumerate}

\subsection{Приближение тригонометрическими полиномами. Теорема Джексона}
\textbf{Опр:} \emph{Тригонометрическим полиномом} порядка n
называется $T_n(x)=\frac{a_0}2+\sum_{k=1}^n (a_k\cos kx+b_k\sin
kx).$

\medskip Введем новое пространство: $C_{2\pi}=\{f\in
C[-\pi,\pi]:\;f(\pi)=f(-\pi)\}.$ Очевидно,
$\hn{f-T_n}_{C[-\pi,\pi]}\ra 0$ только если $f\in
C_{2\pi}.$

\begin{df}
  Пусть $f(x)\in C_{2\pi}$, тогда
  $E_n(f,T)=\inf_{T_n}\hn{f-T_n}_{C[-\pi,\pi]}$ \emph{наилучшее
    приближение} функции $f(x)$ тригонометрическими полиномами порядка
  $n.$
\end{df}

\begin{theorem}[Джексона] Если $f\in C_{2\pi}$ с модулем непрерывности $\omega_f(\de)$, то $E_n(f,T)\le
C\omega_f\hr{\frac1n}.$
\end{theorem}

\begin{proof}
  Воспользовавшись свойством 2 ядра
  Джексона, можно записать
  \begin{equation}\label{14}
    f(x)=\frac1{\pi}\intl{-\pi}{\pi}f(x)J_n(t)dt.
  \end{equation}

  Пусть $Q_n(x,f)=\frac1{\pi}\int_{-\pi}^{\pi}f(x)J_n(x-t)dt.$ Мы
  знаем, что
  $$J_n(t)=\frac12+\sum_{j=1}^{2n-2}a_j^{(n)}\cos jt\Ra
  J_n(x-t)=\frac12+\sum_{j=1}^{2n-2}a_j^{(n)}(\cos jx\cos jt+\sin
  jx\sin jt).$$

  Поэтому $Q_n(x,f)$ тригонометрический полином порядка $2n-2.$

  Поскольку интеграл от $2\pi$-периодической функции по любому
  отрезку длины $2\pi$ имеет постоянное значение, то можно записать
  $Q_n(x,f)$ так:
  \begin{equation}\label{13}
    Q_n(x,f)=\frac1{\pi}\intl{x-\pi}{x+\pi}f(x-t)J_n(t)dt=\frac1{\pi}\intl{\pi}{\pi}f(x-t)J_n(t)dt.
  \end{equation}

  Соединяя вместе (\ref{14}) и (\ref{13}), получаем
  \begin{equation}\label{15}
    f(x)-Q_n(x,f)=\frac1{\pi}\intl{-\pi}{\pi}(f(x)-f(x-t))J_n(t)dt.
  \end{equation}

  Заметим, что
  \begin{equation*}
    |f(x)-f(x-t)|\le\omega_f(|t|)=\omega_f\hr{2n|t|\frac1{2n}}\le(2n|t|+1)\omega_f\hr{\frac1{2n}}.
  \end{equation*}

  и подставим эту оценку в (\ref{15}). Пользуемся свойствами 2 и 3
  ядра Джексона, получаем:
  \begin{equation*}
    |f(x)-Q_n(x,f)|\le\omega_f\hr{\frac1{2n}}\hr{\frac{4n}{\pi}\int\limits_0^{\pi}tJ_n(t)dt+\frac1{\pi}\intl{-\pi}{\pi}J_n(t)dt}\le4\omega_f\hr{\frac1{2n}}.
  \end{equation*}

  Таким образом,
  $$\left\{\begin{aligned}
  &E_{2n}(f,T)\le E_{2n-2}(f,T)\le C\omega_f\hr{\frac1{2n}}\\
  &E_{2n-1}(f,T)\le E_{2n-2}(f,T)\le
  C\omega_f\hr{\frac1{2n}}\le
  C\omega_f\hr{\frac1{2n-1}}
  \end{aligned}
  \right.
  $$
  и теорема доказана.
\end{proof}

Введем новое пространство $C_{2\pi}^r$ подпространство в
$C_{2\pi}$, состоящее из функций, имеющих $r$ непрерывных
производных.
\begin{theorem}[Обобщенная теорема Джексона] $f\in
C_{2\pi}^r\Ra\quad E_n(f,T)\le
\frac{C(r)}{n^r}\;\omega_{f^{(r)}}\hr{\frac1n}.$
\end{theorem}
Мы не будем доказывать эту теорему.

\begin{theorem}[Вейерштрасс] Тригонометрические полиномы плотны в
$C_{2\pi}.$
\end{theorem}
\textbf{Доказательство:} Т.к. для непрерывной на отрезке функции
$f$ $\omega_f(\de)\xra{\de\ra +0} 0$, то,
пользуясь оценкой, приведенной в предыдущей теореме, получаем
$E_n(f,T)\xra{n\ra\infty} 0.$ Отсюда
непосредственно и следует плотность тригонометрических полиномов.

\begin{theorem} Тригонометрические полиномы плотны в
$L^p(-\pi,\pi),\quad1\le p<\infty$.
\end{theorem}
\textbf{Доказательство:} По функции $f(x)\in L^p(-\pi,\pi)$
построим функцию $f_{\de}(x)$:

\begin{wrapfigure}[8]{l}{140pt}
\epsfbox{pictures.6}
\end{wrapfigure}

$$f_{\de}(x)=
\case{f(x),& |x|<\pi-\de\\0, & \pi-\de<|x|,}$$
где $\de$ настолько мало, что
$\hn{f-f_{\de}}_{L^p(-\pi,\pi)}<\ep$. Далее $f_{\de}$
на отрезке $[-\pi+\de,\pi-\de]$ приблизим (с
точностью~$\ep$) непрерывной функцией $\ph(x)$. Затем
построим $\wt{\ph}(x)$, совпадающую с $\ph(x)$ на
$[-\pi+\de,\pi-\de]$, а дальше определенной по непрерывности
и линейности таким образом, чтобы
$\wt{\ph}(-\pi)=\wt{\ph}(\pi)=0$ и
$\hn{\ph-\wt{\ph}}_{L^p[-\pi,\pi]}<\ep$. Тогда
получим, что $\wt{\ph}$ приближает $f$ с точностью менее
$3\ep$. Ну а для $\wt{\ph}$ уже известна теорема
Вейерштрасса. $\blacksquare$

\subsection{Приближение алгебраическими полиномами. Теорема Джексона}
\textbf{Опр:} Пусть $f(x)\in C[-\pi,\pi]$. \emph{Наилучшим
приближением} функции $f(x)$ алгебраическими многочленами порядка
$n$ назовем
$$E_n(f,P)=\inf_{P_n}\hn{f-P_n}_{C[-\pi,\pi]}.$$

\begin{theorem}[Джексон] $E_n(f,P)\le
C\omega_f\hr{\frac1n}.$
\end{theorem}
\textbf{Доказательство:}
\begin{enumerate}
\item
Сделаем замену переменных $x=\cos t$ и рассмотрим функцию
$\ph(t)=f(\cos t)\in C[0,\pi].$
\item
$\omega_{\ph}$ оценим через $\omega_f$:
\begin{equation*}
|\ph(t_2)-\ph(t_1)|=|f(\cos t_2)-f(\cos t_1)|\le\omega_f(|\cos
t_2-\cos
t_1|)=\omega_f\hr{2\sin\frac{t_2-t_1}2\sin\frac{t_2+t_1}2}\le\omega_f(|t_2-t_1|).
\end{equation*}
Т.е. $\omega_{\ph}(\de)\le\omega_f(\de).$
\item
Продолжим $\ph(t)$ на отрезок $[-\pi,0]$ четным образом и новую
функцию обозначим за $\Phi(t).$ $\Phi(t)\in C_{2\pi}$. Докажем,
что $\omega_{\Phi}\le\omega_{\ph}.$ Возьмем $t_1$ и $t_2$. Если
они одного знака, то
$$|\Phi(t_2)-\Phi(t_1)|=|\ph(t_2)-\ph(t_1)|.$$ А если разного
(пусть для определенности $t_1<0<t_2,\; t_2>|t_1|$), то
$$|\Phi(t_2)-\Phi(t_1)|=|\ph(t_2)-\ph(|t_1|)|\le\omega_{\ph}(|t_2-t_1|).$$
\item
Для $\Phi$ мы уже знаем, что $E_n(\Phi,T)\le
C\omega_{\Phi}\hr{\frac1n}\le
C\omega_f\hr{\frac1n}.$ Вспомним, что в роли приближающего
полинома в теореме Джексона выступал
\begin{equation*}
Q_n(t,\Phi)=\frac1{2\pi}\intl{-\pi}{\pi}\Phi(u)J(t-u)du.
\end{equation*}
Но
\begin{equation*}
J_n(t-u)=\frac{a_0^{(n)}}2+\suml{k=1}{2n-2} a_k^{(n)}\cos
k(t-u)=\frac{a_0^{(n)}}2+\suml{k=1}{2n-2} a_k^{(n)}(\cos
kt\cos ku+\sin kt\sin ku).
\end{equation*}

Но $\int_{-\pi}^{\pi}\Phi(u)\sin ku \;du=0$ (т.к. $\Phi$ четная,
а $\sin$ нечетная функции). Поэтому $Q_n(t,\Phi)$ тригонометрический полином порядка $(2n-2)$, состоящий лишь из
косинусов. Ну а $\cos kt$ можно выразить через $\cos t=x$. Выразим
так все косинусы и получим многочлен $P_{2n-2}$. Он, очевидно, и
будет искомым: $|f(x)-P_{2n-2}(x)|\le
C\omega_f\hr{\frac1n},\qquad x\in[-1,1]$.
\end{enumerate}
\begin{theorem} [Обобщенная теорема Джексона]
$f\in C^r[-\pi,\pi]\Ra E_n(f,P)\le
C(r)\frac1{n^r}\omega_{f^{(r)}}\hr{\frac1n}.$
\end{theorem}

\subsection{Прямые и обратные теоремы теории приближений}
\begin{df}
  Теоремы, в которых скорость убывания наилучших приближений оценивается через дифференциальные
  свойства функции, называются \emph{прямыми} (теоремы Джексона - прямые).
\end{df}

\begin{df}
  \emph{Обратными} называются теоремы, в
  которых на основе скорости убывания наилучшего приближения делают
  вывод о дифференциальных свойствах функции.
\end{df}

\begin{theorem}[Бернштейн] Пусть $E_n(f,P)\le C
\frac1{n^{r+\al}},\; r\in\mathbb{Z}^{+},\;\al\in(0,1).$
Тогда $f\in C^r[-1,1]$ и \\\hangindent=5cm$f^{(r)}\in
Lip\;\al.$
\end{theorem}
Эту теорему мы доказывать не будет, заметим лишь, что она
относится к обратным теоремам.

\subsection{Элементы наилучшего приближения (ЭНП)}

Пусть $X$ нормированное пространство, $e_1\sco e_n\in X$ конечная линейно независимая система. Пусть также $x\in X$.
Обозначим $E_n(x)=\inf_{\{\al_k\}}\hn{x-\sum_{k=1}^n \al_k e_k}.$

\begin{df}
  Если $\exi\sum_{k=1}^n \al_k^0
  e_k:\quad\hn{x-\sum_{k=1}^n \al_k^0 e_k}=E_n(x)$, то эта
  линейная комбинация называется \emph{ЭНП}.
\end{df}

\begin{theorem}
ЭНП существует.
\end{theorem}
\begin{proof}

  $f(\al_1\sco \al_n)=\hn{x-\sum_{k=1}^n \al_k e_k},\quad
  f:\R ^n\ra \R .$ $f$ непрерывна:
  \begin{multline*}
    \hm{f(\al_1\sco \al_n)-f(\al_1'\sco \al_n')}=\left|\;\;\left\|x-\sum_{k=1}^n
    \al_k e_k\right\|-\left\|x-\sum_{k=1}^n \al_k'
    e_k\right\|\;\;\right|\le\\
    \le\left\|\left(x-\sum_{k=1}^n \al_k
    e_k\right)-\left(x-\sum_{k=1}^n \al_k'
    e_k\right)\right\|=\left\|\sum_{k=1}^n (\al_k-\al_k')
    e_k\right\|\xra{\al'\ra\al}0.
  \end{multline*}

  $\ph(\al_1\sco \al_n)=\hn{\sum_{k=1}^n \al_k e_k}$
  - непрерывная функция, $\al=(\al_1\sco \al_n),\quad|\al|=\sqrt{\sum_{k=1}^n\al_k^2}.$

  Пусть $S={\al\in\R ^n:\;|\al|=1}$ компакт, поэтому
  $\ph$ достигает на $S$ своего минимума (обозначим его $m$). $m>0$
  (иначе $e_i$ зависимы).

  $$|f(\al_1\sco \al_n)|\ge\left\|\sum_{k=1}^n \al_k
  e_k\right\|-\hn{x}=|\al|\left\|\sum_{k=1}^n
  \frac{\al_k}{|\al|} e_k\right\|-\hn{x}\ge|\al| m-\hn{x}.$$

  Пусть $|\al|>R$, тогда $|f(\al_1\sco \al_n)|\ge
  Rm-\hn{x}$ и поэтому нижняя грань $f$ достигается (точнее сказать,
  найдется последовательность к ней стремящаяся) в каком-то шаре
  радиуса $R_0$. Но шар - компакт, поэтому $f$ достигает значения
  нижней грани.
\end{proof}

\begin{df}
  Нормированное пространство $X$ называется \emph{строго нормированным}, если в
  неравенстве треугольника $\hn{x+y}\le\hn{x}+\hn{y}$ равенство имеет место лишь когда $y=cx, \quad c\ge0$.
\end{df}

\begin{theorem} В строго нормированном пространстве ЭНП
единственен.
\end{theorem}
\begin{proof}
  Пусть $x\in X$ и существуют два ЭНП:
  $\sum\al_k e_k$ и $\sum\beta_k e_k.$ Тогда:
  \begin{equation}\label{16}
    E_n\le\hn{x-\sum\frac{\al_k+\beta_k}2e_k}\le\frac12\hn{x-\sum\al_k e_k}+\frac12\hn{x-\sum\beta_k e_k}=E_n.
  \end{equation}
  Очевидно, что в (\ref{16}) знак равенства имеет место всюду,
  поэтому $x-\sum\al_k e_k=c(x-\sum\beta_k e_k).$ Если $c\neq1$,
  то $x\in<e_1\sco e_n>$ и поэтому разложение единственно. Если
  $c=1$, то $\al_k=\beta_k.$
\end{proof}

\subsection{Многочлены наилучшего приближения в пространстве $C[a,b]$}
\begin{theorem}[Чебышев]
Пусть $f\in C[a,b]$. $P_n(x)$ МНП для $f(x)$ $\Lra$
функция $r_n(x)=f(x)-P_n(x)$ обладает свойствами:
\begin{enumerate}
\item
Существуют $n+2$ точки на $[a,b]$ $x_1\sco x_n$:
$|r_n(x_k)|=\max_{[a,b]}|r_n(x)|.$
\item
Знаки $r_n(x_k)$ чередуются.
\end{enumerate}
\end{theorem}
\begin{df}
  Совокупность точек $x_k$ называется \emph{чебышевским альтернансом}.
\end{df}

\begin{proof}
  \emph{Необходимость}
  Пусть $P_n(x)$ МНП для $f(x)$. Тогда, очевидно,
  $\hn{r_n(x)}_{C[a,b]}=E_n(f).$ Для простоты точки $x_k$ из
  формулировки теоремы будем называть $e$-точками и будем разделять
  их на $e_+$-точки и $e_-$-точки.

  По определению наилучшего приближения существует хотя бы одна
  $e$-точка. Пусть, для определенности это $e_+$-точка. Докажем, что
  тогда существует и $e_-$-точка. Предположим противное - ее нет.
  Это означает, что $\min_{[a,b]}r_n(x)>-E_n$, т.е. $r_n(x)\ge
  \de-E_n$. Пусть $Q_n=P_n(x)+\frac{\de}2$. Тогда
  $$|f(x)-Q_n(x)|=\hm{f(x)-P_n(x)-\frac{\de}2}=\hm{r_n(x)-\frac{\de}2}\Ra
  \max|f(x)-Q_n(x)|\le E_n-\frac{\de}2.$$ Но это означает, что
  $Q_n$ приближает $f$ лучше, чем $P_n$, что невозможно по
  определению $P_n$ противоречие. Значит $e_-$-точки таки есть.

  Разобьем отрезок $[a,b]$ на конечное число отрезков
  $[t_i,t_{i+1}]$ таких, что
  \begin{enumerate}
  \item
    На каждом отрезке деления существует хотя бы одна $e$-точка.
  \item
    На каждом отрезке деления $e$-точки одного знака.
  \item
    На соседних отрезках знаки $e$-точек противоположные.
  \end{enumerate}

  Докажем, что таких отрезков конечное число. Предположим это не
  так, тогда из $e_+$-точек можно выделить сходящуюся
  подпоследовательность $x_{k_i}^+\ra x.$ Но тогда
  $r_n(x_{k_i}^+)\ra r_n(x)=E_n(f)$. Аналогично
  $r_n(x_{k_i}^-)\ra r_n(x)=-E_n(f)$ противоречие. Значит
  отрезков таки конечное число. Пусть $(m+2)$ число таких
  отрезков.

  Надо доказать, что $m+2\ge n+2$. Предположим противное:
  $m+2<n+2\Ra m+1\le n$. Возьмем многочлен
  $p_{m+1}=h(x-t_1)\dots(x-t_{m+1})$, где $h$ некоторая константа,
  которую мы сейчас подберем нужным нам образом.

  Пусть минимальное значение $r_n(x)$ на совокупности отрезков
  $[t_i,t_{i+1}]$, содержащих только $e_+$-точки, не меньше, чем
  $\de_1-E_n(f),\quad\de_1>0$. И пусть максимальное значение
  $r_n(x)$ на совокупности отрезков $[t_i,t_{i+1}]$, содержащих
  только $e_-$-точки, не больше, чем
  $E_n(f)-\de_2,\quad\de_2>0$. Положим
  $\de=\min\{\de_1,\de_2\}$. Теперь $h$ выберем так, чтобы:
  \begin{enumerate}
  \item
    Знак $p_{m+1}$ на $[t_i,t_{i+1}]$ совпадал со знаком $e$-точек,
    лежащих на этом отрезке.
  \item
    $\hn{p_{m+1}(x)}_{C[a,b]}<\frac{\de}2.$
  \end{enumerate}

  Пусть $Q_n(x)=P_n(x)+p_{m+1}(x)$. Тогда
  $|f(x)-Q_n(x)|=|r_n(x)-p_{m+1}(x)|\le E_n(f)-\frac{\de}2$ противоречие, завершающее доказательство этой части теоремы.

  \emph{Достаточность} Предположим $P_n$ не многочлен наилучшего
  приближения. Тогда пусть $Q_n\neq P_n$ МНП. Тогда очевидно, что
  $|f(x_k)-Q_n(x_k)|\le|f(x_k)-P_n(x_k)|.$
  $$Q_n(x_k)-P_n(x_k)=(f(x_k)-P_n(x_k))-(f(x_k)-Q_n(x_k)).$$
  Поэтому $sign(Q_n(x_k)-P_n(x_k))=sign\; r_n(x_k)$. Значит
  многочлен $Q_n(x)-P_n(x)$ имеет $n+1$ смену знака. Учитывая то,
  что это многочлен степени не выше $n$, получаем $Q_n(x)=P_n(x)$ противоречие с предположением.
\end{proof}

\subsection{Многочлены наименее уклоняющиеся от нуля (многочлены Чебышева)}
Поставим задачу: найти многочлен $P_n(x)$ степени $n$ наименее
уклоняющийся от нуля на отрезке $[-1,1]$, т.е. такой, что значение
$\hn{P_n}_{C[-1,1]}$ минимально. Будем искать его в виде
$P_n(x)=x^n+\wt{P}_{n-1}(x).$ Тогда нашу задачу можно
переформулировать: найти МНП ($\wt{P}_{n-1}(x)$) функции
$x^n$. По теореме Чебышева из предыдущего пункта
$\wt{P}_{n-1}(x)$ МНП $\Lra$
$\wt{P}_{n-1}(x)$ на $[-1,1]$ имеет чебышевский альтернанс
в количестве $n+1$.

Докажем что искомым многочленом $P_n$ является многочлен:
$$T_n(x)=\frac1{2^n}\hr{\hr{x+\sqrt{x^2-1}}^n+\hr{x-\sqrt{x^2-1}}^n}.$$
$T_n$ очевидно, алгебраический многочлен степени $n$ (расписать
по биному Ньютона и увидеть, что все нецелые степени $x$
сократятся). Вычислим коэффициент перед $x^n$, он равен
$$\lim_{x\ra\infty}\frac{T_n(x)}{x^n}=\lim_{x\ra\infty}\frac{x^n\hr{\hr{1+\sqrt{1-\frac1{x^2}}}^n+\hr{1-\sqrt{1-\frac1{x^2}}}^n}}{2^nx^n}=1.$$
Положим $x=\cos t,\quad t\in[0,\pi]$. Тогда
$$T_n(x)=\frac1{2^n}((\cos t+i\sin t)^n+(\cos t-i\sin
t)^n=\frac1{2^n}(e^{int}+e^{-int})=\frac1{2^{n-1}}\cos nt.$$

$|\cos nt|=1\Lra nt=k\pi\ra t=\frac{k\pi}n,\quad k=0,1\sco n$ $(n+1)$ точка. Но эти
точки и составляют чебышевский альтернанс. Таким образом, действительно $T_n$ многочлен наименее
уклоняющийся от нуля. Он называется \emph{многочленом Чебышева}. Сделав обратную замену переменной $t=\arccos
x$, получаем более простую формулу для вычисления $T_n$:
$$T_n(x)=\frac1{2^{n-1}}\cos(n\arccos x),\quad -1\le x\le1.$$

Есть также простой способ вычисления многочленов Чебышева, который
чаще всего и применяется при решении практических задач (в
частности в программировании). Поскольку
$\cos\al\cos\beta=\frac12(\cos(\al+\beta)+\cos(\al-\beta))$,
то $\cos(n+1)\theta=2\cos\theta\cos n\theta-\cos(n-1)\theta.$
Применяя эти выражения к многочленам Чебышева, получаем:
$$T_0(x)=1,\;T_1(x)=x,\;T_{n+1}(x)=2xT_n(x)-T_{n-1}(x),\;n>0.$$

Рассмотрим пространство
$\wt{L}^2=L^2\hr{(-1,1),\frac{dx}{\sqrt{1-x^2}}}$
интегрируемых на $(-1,1)$ с весом $\frac{dx}{\sqrt{1-x^2}}$
функций, т.е. функций $f$:
$$\hn{f}_{\wt{L}^2}=\hr{\int\limits_{-1}^1|f(x)|^2\frac{dx}{\sqrt{1-x^2}}}^{1/2}<\infty.$$
\begin{theorem}
Система многочленов Чебышева $T_0,\;T_1,\dots$ ортогональна в
$\wt{L}^2.$
\end{theorem}
\begin{proof}
  Если $m\neq n$, то
  $$(T_n,T_m)=2^{2-n-m}\int\limits_{-1}^1\cos(n\arccos
  x)\cos(m\arccos
  x)\frac{dx}{\sqrt{1-x^2}}=2^{2-n-m}\int\limits_0^{\pi}\cos(nt)\cos(mt)dt=0.$$
  $$\hn{T_n}^2=2^{2-2n}\hr{\int\limits_0^{\pi}\cos^2nt\;dt}=\frac{\pi}{2^{2n-1}}.$$
\end{proof}
\begin{theorem}
Нормированная система многочленов Чебышева образует базис
пространства $\wt{L}^2.$
\end{theorem}
\begin{proof}
  Достаточно проверить полноту этой системы
  (раз нормированность уже есть). Доказывать будем по-шагам:
  \begin{enumerate}
  \item
    $$\fa\ep>0\;\exi\de>0:\quad\int_{1-\de<|x|<1}\frac{|f(x)|^2}{\sqrt{1-x^2}}dx<\ep.$$
    Пусть
    $$f_{\ep}(x)=\left\{\begin{aligned}f(x),&\quad|x|<1-\de\\
    0,&\quad1-\de<|x|<1.\end{aligned}\right.$$

    Тогда
    \begin{equation}\label{a}
      \hn{f-f_{\ep}}_{\wt{L}^2}<\ep.
    \end{equation}
  \item
    $$\int_{|x|<1-\de}|f(x)|^2dx=\int_{|x|<1-\de}\frac{|f(x)|^2}{\sqrt{1-x^2}}\sqrt{1-x^2}dx\le\hn{f}_{\wt{L}^2}.$$
    Поэтому $f_{\ep}\in L^2(|x|<1-\de)$. Значит найдется $\ph_{\ep}(x)\in
    C[\de-1,1-\de]:\quad\hn{f_{\ep}-\ph_{\ep}}_{L^2(|x|<1-\de)}<\ep.$ А~из этого
    следует, что
    \begin{multline}\label{b}
      \hn{f_{\ep}-\ph_{\ep}}_{\wt{L}^2}=\hr{\int\limits_{-1}^1|f_{\ep}-\ph_{\ep}|^2\frac{dx}{\sqrt{1-x^2}}}^{1/2}=\\
      =\hr{\int\limits_{|x|<1-\de}|f_{\ep}-\ph_{\ep}|^2\frac{dx}{\sqrt{1-x^2}}}^{1/2}\le\frac{\hn{f_{\ep}-\ph_{\ep}}_{L^2(|x|<1-\de)}}{\sqrt{1-(1-\de)^2}}<\ep.
    \end{multline}

  \item
    Можно линейно доопределить $\ph_{\ep}$ на отрезке
    $[-1,1]$ до функции $\psi$ так, чтобы
    \begin{equation}\label{c}
      \hn{\ph_{\ep}-\psi}_{\wt{L}^2}<\ep.
    \end{equation}
  \item
    $\psi\in C[-1,1]\Ra P_n(x):\quad
    \hn{\psi-P_n}_{C[-1,1]}<\ep.$ Но тогда
    \begin{equation}\label{d}
      \hn{\psi-P_n}_{\wt{L}^2}=\hr{\int\limits_{-1}^1|\psi-P_n|^2\frac{dx}{\sqrt{1-x^2}}}^{1/2}<\ep\hr{\int\limits_{-1}^1\frac{dx}{\sqrt{1-x^2}}}^{1/2}=\ep\sqrt{\pi}.
    \end{equation}
  \item
    Теперь соединим все полученные оценки:
    $$\hn{f-P_n}_{\wt{L}^2}=\hn{(f-f_{\ep})+(f_{\ep}-\ph_{\ep})+(\ph_{\ep}-\psi)+(\psi-P_n)}_{\wt{L}^2}\le|(\ref{a})+(\ref{b})+(\ref{c})+(\ref{d})|\le C\ep.$$
    Таким образом, нами показано, что алгебраические многочлены плотны
    в $\wt{L}^2$. Для завершения доказательства остается только
    проверить, что
    $$\fa n\in\mathbb{N}\quad\exi
    c_1,c_2\sco c_{n-1}:\quad
    x^n=T_n(x)+\sum_{k=0}^{n-1}c_kT_k(x).$$

    Докажем это по индукции. $T_0(x)=1,\;T_1(x)=x$ база индукции.
    $$T_n(x)=x^n+\sum_{k=0}^{n-1}b_kx^k=|\text{по предположению индукции}|=x^n+\sum_{k=0}^{n-1}b_k\sum_{j=0}^{k-1}d_kT_k(x).\quad$$

  \end{enumerate}
\end{proof}

\newpage
\section{Часть II}
\subsection{От теоремы Вейерштрасса к теореме Мюнца}
Пусть $f\in C[0,1]$. Рассмотрим функцию $f(\sqrt{t})\in C[0,1]$, тогда по теореме Вейерштрасса
$\fa\ep>0\quad\exi P_n(t):\quad\hn{f(\sqrt{t})-P_n(t)}_{C[0,1]}<\ep.$ Сделаем замену
$\sqrt{t}=x$ и получим $\hn{f(x)-P_n(x^2)}_{C[0,1]}<\ep.$ Т.е. в $C[0,1]$ полна система
$\{x^{2n},\quad n\in\mathbb{Z}^{+}\}$. аналогично можно доказать и полноту системы, скажем, $\{x^{100n},\quad
n\in\mathbb{Z}^{+}\}$. Рассмотрим более общий случай: систему
$\{x^{\la_n},\quad(\la_n)_1^{\infty}\subset\R ^{+}\}$*. Наша задача - описать полные в
$C[0,1]$ системы *.

Одним из результатов решения этой задачи станет следующая теорема.
\begin{theorem}[Мюнц 1914г.]
Пусть $0<\la_1<\la_2<\ldots$ Тогда для того, чтобы * была
полна в $C[0,1]$, необходимо и достаточно расходимости ряда
$\sum_{n=1}^{\infty}\frac1{\la_n}$.
\end{theorem}
Как видно, теорема Вейерштрасса - частный случай теоремы Мюнца
(здесь возникает гармонический ряд, который, как известно,
расходится).

\subsection{Вспомогательные сведения из комплексного анализа}
Для дальнейшего изучения материала нам просто необходим аппарат комплексного анализа, поэтому придется
привести основные определения и теоремы (без доказательств) из ТФКП.

Пусть в $\mathbb{C}$ есть область $G$ и пусть на ней определена
функция $f(z)$.

\begin{df}
  \emph{Производной} функции $f(z)$ в точке
  $z_0$ называется следующий предел (если он существует):
  $$f'(z_0):=\lim_{h\ra0}\frac{f(z_0+h)-f(z_0)}{h}.$$
\end{df}

Существование предела можно переписать так: $\De
f=f'(z_0)\De z+o(\De z)$ при $\De z\ra0.$ Это
означает, что приращение функции есть комплексный дифференциал.
Такие функции называются \emph{комплексно дифференцируемыми}.

\begin{df}
  Говорят, что функция \emph{аналитична в
    точке} $z_0$, если она комплексно-дифференцируема в некоторой
  окрестности этой точки.
\end{df}

\begin{df}
  Функция называется \emph{аналитической в
    области}, если она аналитична в каждой точке области.
\end{df}

\begin{theorem}[Эквивалентное определение аналитичности]
$f(z)$ аналитична в $G\Lra\;\fa z_0\in G\;\exi
U_{z_0}:\;f(z)=\sum_{k=0}^{\infty}c_k(z-z_0)^k$ сходится
равномерно в $U_{z_0}.$
\end{theorem}


\begin{theorem}
Если функция аналитична в области, то во всех точках этой области
у нее существуют производные всех порядков.
\end{theorem}

\begin{theorem}[Единственности]
Пусть $f(z)$ аналитична в области $G$ и множество ее нулей имеет
предельную точку в $G$, тогда $f(z)\equiv0.$
\end{theorem}

\begin{theorem}
Пусть $f(z)$ и $g(z)$ аналитичны в некоторой окрестности $U_{z_0}$
точки $z_0$. И пусть $z_0$ ноль порядка $m$ для $f(z)$ и порядка
$n$ для $g(z)$ ($m\ge n$) и других нулей в $U_{z_0}$ нет. Тогда
функция $\frac {f(z)}{g(z)}$ аналитична в $U_{z_0}$.
\end{theorem}

\begin{theorem}[Принцип максимума модуля]
Если $f(z)$ аналитична в ограниченной области $G$ и непрерывна на
ее замыкании, тогда $\max_{z\in \ol{G}}|f(z)|$ достигается
на границе области.
\end{theorem}

\begin{theorem}[Вейерштрасс]
Если последовательность аналитических в области $G$ функций
сходится равномерно на любом компакте из этой области, то
предельная функция также аналитична в $G.$
\end{theorem}

\subsection{Условие и произведение Бляшке для круга}
\begin{theorem}
Пусть $f(z)$ аналитична и ограничена в $|z|<1$ и $f(0)=0$.
$\{z_n\}_1^{\infty}$ последовательность нулей $f(z)$,
занумерованная в порядке неубывания модулей. Тогда
$\sum_{n=1}^{\infty}(1-|z_n|)<\infty\quad(B).$
\end{theorem}
Условие (B) называют условием Бляшке.

\noindent\textbf{Доказательство:} Рассмотрим конечное произведение
$$B_n=\prod\limits_{k=1}^n\frac{z_k-z}{1-\ol{z_k}z}.$$

Знаменатель $B_n$ в единичном круге в нуль не обращается, поэтому
$B_n$ аналитична в $|z|<1+\de,\quad\de>0$.

Пусть $z=e^{i\ph}$. Тогда
$$\hm{\frac{z_k-z}{1-\ol{z_k}z}}=\hm{\frac{z_k-e^{i\ph}}{1-\ol{z_k}e^{i\ph}}}=\hm{\frac{z_k-e^{i\ph}}{\ol{z_k-e^{i\ph}}}}=1.$$

Рассмотрим функцию $\frac{f(x)}{B_n(z)}$. Она аналитична в круге
$|z|<1$. Пусть $|f(z)|\le M$ в $|z|<1$, тогда по принципу
максимума модуля для аналитических функций ($f(z)$ можно по
непрерывности продолжить на $|z|\le 1$):
\begin{equation}\label{17}
\hm{\frac{f(z)}{B_n(z)}}\le\frac M1=M
\end{equation}
Теперь положим $z=0$ и, воспользовавшись оценкой (\ref{17}),
получаем:
$$0<|f(0)|\le M|B_n(0)|=M\prod\limits_{k=1}^n |z_k|.$$

Поэтому $P_n=\prod_{k=1}^n |z_k|$ ограничены снизу. К тому же они,
очевидно, убывают ($|z_k|<1$) поэтому $\prod_{k=1}^{\infty} |z_k|$
сходится, а значит и сходится
$\sum_{n=1}^{\infty}(1-|z_n|)<\infty.\blacksquare$

\medskip\noindent\textbf{Замечание:} В случае кратных корней
условие Бляшке надо понимать так: пусть $z_k$ корень кратности
$m_k$, тогда
$$\sum_{n=1}^{\infty}(1-|z_n|)m_k<\infty.$$

\noindent\textbf{Замечание:} Условие $f(0)\neq0$ не снижает
общности, т.к. если $0$ корень кратности $m$, то можно перейти к
функции $\frac{f(z)}{z^m}.$

\begin{theorem}
Пусть $\{z_k\}_1^{\infty}$ последовательность ненулевых точек
единичного круга и выполнено условие Бляшке, тогда
$$B(z)=\prod\limits_{k=1}^{\infty}\frac{z_k-z}{1-\ol{z_k}z}\frac{\ol{z_k}}{|z_k|}$$
сходится равномерно на каждом круге $|z|\le r<1$ и задает ограниченную аналитическую функцию в этом круге,
множество нулей которой совпадает с $\{z_k\}_1^{\infty}$.
\end{theorem}
$B(z)$ называют произведением Бляшке.

\noindent\textbf{Доказательство:} Обозначим общий член
произведения Бляшке за $f_n(z).$
\begin{multline*}
1-f_n(z)=\frac1{|z_n|}-\frac{z_n-z}{1-\ol{z_n}z}\frac{\ol{z_n}}{|z_n|}+1-\frac1{|z_n|}=\frac1{|z_n|}\hr{1-\frac{|z_n|^2-z_nz}{1-\ol{z_n}z}}-\frac1{|z_n|}(1-|z_n|)=\\
=\frac1{|z_n|}\hr{\frac{1-|z_n|^2}{1-\ol{z_n}z}-(1-|z_n|)}=\frac{1-|z_n|}{|z_n|}\hr{\frac{1+|z_n|}{1-\ol{z_n}z}-1}.
\end{multline*}

Поэтому
$|1-f_n(z)|\le\frac{1-|z_n|}{|z_n|}\hr{\frac2{1-r}-1},\quad\fa
|z|\le r<1$.

Далее, поскольку нули аналитической функции не могут стремиться к внутренней точке области, где функция
аналитична, то найдется $M:\quad \frac1{|z_n|}\le M$. Пользуясь этим и условием Бляшке получаем, что ряд
$\sum_{k=1}^{\infty}|1-f_n(z)|$ сходится равномерно в круге $|z|\le r<1$. Аналитичность $B(z)$ берется из
теоремы Вейерштрасса о пределе (в A(G)) последовательности аналитических функций. $|B(z)|\le1$ просто по
построению.$\blacksquare$

\noindent\textbf{Замечание:} Если $\{z_n\}$ последовательность
точек единичного круга кратности $m_k$, то условие Бляшке
принимает вид $\sum_{k=1}^{\infty} m_k(1-|z_k|)<\infty$, а
произведение Бляшки принимает вид
$$B(z)=z^{m_0}\prod\limits_{k=1}^{\infty}\hr{\frac{z_k-z}{1-\ol{z_k}z}\frac{\ol{z_k}}{|z_k|}}^{m_k}.$$

\subsection{Условие и произведение Бляшке для полуплоскости}
Как известно из курса комплексного анализа, отображение
$z=\frac{w-1}{w+1}$ переводит правую полуплоскость ($Re\;w>0$) в
единичный круг. Пусть у нас есть последовательность точек
$\{w_k\}_1^{\infty}$ в правой полуплоскости. Рассмотрим образ этой
последовательности при отображении в единичный круг. Это будет
последовательность $\{z_k\}_1^{\infty}$. Условие Бляшке для нее
можно записать так
\begin{equation}\label{18}
\suml{k=1}{\infty}(1-|z_k|^2)<\infty.
\end{equation}
(очевидно, умножение каждого слагаемого на $(1+|z_k|)\in[1,2]$ на
сходимость не влияет). Делая замену переменных, получаем, что
(\ref{18}) эквивалентно
\begin{equation}
\suml{k=1}{\infty}\hr{1-\hm{\frac{w_k-1}{w_k+1}}^2}<\infty.
\end{equation}

Пусть $w_k=u_k+iv_k$, тогда
$$\hm{\frac{w_k-1}{w_k+1}}^2=\frac{(u_k-1)^2+v_k^2}{(u_k+1)^2+v_k^2}.$$
И общий член ряда будет иметь вид
$$\frac{(u_k+1)^2+v_k^2-(u_k-1)^2-v_k^2}{(u_k+1)^2+v_k^2}=\frac{4u_k}{(u_k+1)^2+v_k^2}.$$

Далее, поскольку
$$\frac12\frac{u_k}{1+u_k^2+v_k^2}\le\frac{u_k}{(u_k+1)^2+v_k^2}\le\frac{u_k}{1+u_k^2+v_k^2},$$
то условие Бляшке для полуплоскости примет вид
\begin{equation}\label{19}
\suml{k=1}{\infty} \frac{Re\;w_k}{|w_k|^2+1}<\infty.
\end{equation}

Пусть $f(z)$ аналитична в $Re\;w>0$. Рассмотрим ее образ при
отображении на единичный круг: $z=\frac{w-1}{w+1}\Ra
w=\frac{1+z}{1-z}$. $F(z)=f\hr{\frac{1+z}{1-z}}$ аналитична в единичном круге и нулями ее будут являться $z_k$
(кратность сохраняется).

Произведение Бляшке для полуплоскости будем обозначать
$\wt{B}(w)=B\hr{\frac{w-1}{w+1}}.$

Из всех этих соображений очевидным образом вытекают две теоремы.

\begin{theorem}
Пусть $f(w)$ аналитична и ограничена в $Re\;w>0$, $\{w_n\}_1^{\infty}$ ее нули. Тогда выполнено условие
Бляшке для полуплоскости (\ref{19}).
\end{theorem}

\begin{theorem}\label{B}
Пусть $w_k: \quad Re\;w_k>0$ и выполнено условие Бляшке для
полуплоскости. Тогда $\wt{B}(w)$ аналитична, ограничена по
модулю единицей в $Re\;w>0$ и множество ее нулей совпадает с
$\{w_n\}_1^{\infty}$.
\end{theorem}

\subsection{Классы Харди}
\textbf{Опр:} Будем говорить, что аналитическая в $Re\;w>0$
функция $f(w)$ лежит в классе $H^p$, если
\begin{equation}
\sup_{u>0}\int_{\R } |f(u+iv)|^p\;dv<\infty,\qquad1\le
p<\infty.
\end{equation}

Ниже приведены несколько теорем, которые будут полезны нам в
дальнейшем, но доказывать их мы не будем.
\begin{theorem}
Если $f(w)\in H^p$, тогда для последовательности ее нулей
выполнено условие Бляшке для полуплоскости.
\end{theorem}

\begin{theorem}
Если $f(w)\in H^p$, тогда почти в каждой точке $iv$ существует
угловой предел, т.е.
$$\exi \lim_{r\ra
0}f(iv+re^{i\ph}).$$

при этом функция граничных значений $f(iv)\in L^p(\R ).$
\end{theorem}

\begin{theorem}[Единственности]
Если $f(w)\in H^p$ и $f(iv)=0$ на множестве положительной меры,
тогда $f(w)\equiv0.$
\end{theorem}
Таким образом, функция класса $H^p$ полностью определяется своей
функцией граничных значений.

\subsection{Преобразование Лапласа}
\textbf{Опр:} \emph{Преобразованием Лапласа} локально суммируемой
на $\R ^{+}$ функции $f(t)$ называется
\begin{equation}\label{200}
F(w)=\int\limits_0^{\infty} e^{-wt}f(t)\;dt.
\end{equation}

Тот факт, что $F(w)$ преобразование Лапласа для $f(t)$,
обозначается $F(w)\risingdotseq f(t)$.

\begin{theorem}\label{2.9}
$f(t)\in L^p(\R ^{+}),\quad1\le p\le2\Ra F(w)\in
H^{q},\quad q:\frac1p+\frac1q=1$.
\end{theorem}

\begin{theorem}\label{2.10}
$F(w)\in H^p,\quad1\le p\le2\Ra F(w)$ представляется в
виде (\ref{20}) с $f\in L^q(\R ^{+}),\quad
q:\frac1p+\frac1q=1$ (и такая $f$ единственна).
\end{theorem}

\begin{imp}[Пэли Винер (Paley Wiener)]
Класс $H^2$ совпадает с классом функций, представимых в виде (\ref{200}) с $f\in L^2(\R ^{+})$ (и
такая $f$ единственна).
\end{imp}

\begin{theorem}
Если $f\in L^p(\R ^{+}),\quad1\le p\le2\Ra F(w)$
аналитична в $Re\;w>0$.
\end{theorem}
\textbf{Доказательство:} По одной из теорем комплексного анализа
для аналитичности в области $G$ интеграла вида
$$\int\limits_0^{\infty}f(w,t)dt,\quad f\in A(G)\;\fa t\in\R ^{+}$$
достаточно его равномерной сходимости на каждом компакте из $G$.

\begin{wrapfigure}[7]{l}{155pt}
\epsfbox{pictures.5}
\end{wrapfigure}

В нашем случае $f(w,t)=e^{-wt}f(t)$ очевидно, аналитична
$\fa t\in\R ^{+}$. Возьмем какой-нибудь компакт $K$ из
правой полуплоскости. Пусть абсцисса его самой <<левой>> точки
есть $u_0>0$, тогда

\begin{multline*}
\hm{\int\limits_R^{\infty}e^{-wt}f(t)\;dt}\le\hr{\int\limits_R^{\infty}|e^{-qwt}|dt}^{1/q}\hn{f}_{L^p(R,\infty)}\le\\
\le\hr{\int\limits_R^{\infty}e^{-qu_0t}dt}^{1/q}\hn{f}_{L^p(R,\infty)}=\hr{\frac{e^{-qu_0R}}{qu_0}}^{1/q}\hn{f}_{L^p(R,\infty)}.
\end{multline*}

Отсюда и следует равномерная сходимость на $K.\quad\blacksquare$

\begin{theorem}[Единственности]
Если в $Re\;w>0\quad0\equiv F(w)\risingdotseq f(t),\quad f\in
L^q\;(1\le q<\infty)$, тогда почти всюду $f(t)=0.$
\end{theorem}

\subsection{Преобразование Лапласа Стилтьеса}
Обозначим за $V[0,\infty)=\{\si(t):\quad var_{[0,\infty)}
\si(t)<\infty\}.$ Для функций из такого класса определим
преобразование Лапласа-Стилтьеса как
$$F(w)=\int\limits_0^{\infty}e^{-wt}d\si(t).$$

\begin{theorem}
Если $\si(t)\in V[0,\infty)$, то ее преобразование
Лапласа-Стилтьеса $F(w)$ аналитично и ограничено в $Re\;w>0$ и
непрерывно в $Re\;w\ge0.$
\end{theorem}
\textbf{Доказательство:} Докажем равномерную сходимость интеграла
на каждом компакте из $Re\;w>0$ (как и в случае преобразования
Лапласа, считаем, что абсцисса самой <<левой>> точки компакта есть
$u_0$):

$$\hm{\int\limits_R^{\infty}e^{-wt}d\si(t)}\le e^{-u_0 R}var_{[0,\infty)}\si(t).$$
$F(w)$ ограничена, поскольку $F(w)\le
var_{[0,\infty)}\si(t).\blacksquare$

\begin{theorem}[Единственности]
Пусть $F(w)$ преобразование Лапласа Стилтьеса от $\si(t)$, причём $\si\in V[0,\infty).$ Если
$F(w)\equiv0,\quad Re\;w>0$, то $\si(t)\equiv const.$
\end{theorem}


\newpage
\section{Часть III}
\subsection{Проблема Мюнца Саса (Muntz Szasz)}
\begin{theorem}[Сас 1916]\label{sas}
Пусть $Re\;\mu_n>-\frac12$, тогда для того, чтобы
$\{x^{\mu_n}\}_1^{\infty}$ была полна в $L^2(0,1)$ необходимо и
достаточно сходимости ряда
$$\suml{k=1}{\infty}\frac{Re\;\mu_k+\frac12}{1+(Re\;\mu_k+\frac12)^2}<\infty.$$
\end{theorem}

Условие $Re\;\mu_n>-\frac12$ требуется для того, чтобы
$x^{\mu_n}\in L^2(0,1).$

$$(*)\left\{
\begin{aligned}
&x^{\mu_n}\in L^p(0,1)&\Lra \quad&Re\;\mu_n>-\frac1p\\
&x^{\mu_n}\in C_0[0,1]&\Lra \quad&Re\;\mu_n>0.
\end{aligned}
\right.
$$
Проблема Мюнца-Саса заключается в описании полных систем
$\{x^{\mu_n}\}_1^{\infty}$ в пространствах $L^p$ и $C_0$ (в
предположении (*)).

\subsection{Переформулировка проблемы}
Сделаем замену переменных: $x=e^{-t}$. Рассмотрим функцию $f(x)\in
C_0[0,1].$ Пусть $Tf:=f(e^{-t})\in C_0[0,\infty)$, $T\colon
C_0[0,1]\ra C_0[0,\infty).$ Очевидно, что $\hn{T}=1.$
Пусть $g(t)\in C_0[0,\infty)$, тогда $g(-\log(x))\in C_0[0,1].$
Таким образом, $T$ задает изоморфизм, а изоморфизмы сохраняют
полноту систем. Поэтому эквивалентной постановкой проблемы
Мюнца-Саса для пространства $C_0[0,1]$ можно считать такую:
описать полные системы $\{e^{-\mu_n t}\}_1^{\infty},\quad
Re\;\mu_n>0$ в пространстве $C_0[0,\infty).$ Заметим, что
$C_0[0,\infty)$ это вовсе не пространство непрерывных функций
$f(t)$, таких, что $f(0)=0$ (как могло бы показаться), а
пространство непрерывных функций, таких, что
$f(t)\xra{t\ra\infty}0.$

Для случая пространства $L^p(0,1)$
$$\int\limits_0^1|f(x)|^pdx=\int\limits_0^{\infty}|f(e^{-t})|^pe^{-t}dt.$$
Поэтому здесь изоморфизм $T$ будет действовать по правилу
$T:f(x)\ra f(e^{-t})e^{-t/p}.$ И в частности $T:
x^{\mu_n}\ra e^{-\mu_n t-t/p}=e^{-(\mu_n+1/p)t}.$

Поэтому эквивалентной постановкой проблемы Мюнца-Саса для
пространства $L^p(0,1)$ можно считать такую: описать полные
системы $\{e^{-\la_n t}\}_1^{\infty},\quad
Re\;\la_n>0\quad(\la_n=\mu_n+\frac1p)$ в пространстве
$L^p[0,\infty).$

\subsection{Аналитический эквивалент неполноты}
По критерию неполноты (следствие из теоремы Хана-Банаха) неполнота
системы $\{e^{-\la_n t}\}_1^{\infty}$ в $L^p[0,\infty)$
(или $C_0[0,\infty)$) равносильна существованию непрерывного
функционала на $L^p[0,\infty)$ (или $C_0[0,\infty)$), аннулирующего
систему, т.е.
$$\exi \Phi:L^p[0,\infty)\; (C_0[0,\infty))\ra
\R ,\quad \Phi(e^{-\la_n t})=0\quad\fa
n\in\mathbb{N}.$$

Применим теорему Рисса об общем виде линейного непрерывного
функционала на~$L^p[0,\infty)$ или $C_0[0,\infty).$

$$L^p[0,\infty):\quad\Phi,\ph\in
L^q[0,\infty)\qquad\Phi(f)=\int_0^{\infty} f(x)\ph(x)dx,\quad
f(x)\in L^p[0,\infty).$$

$$C_0[0,\infty):\quad\Phi,\si\in V[0,\infty)\qquad\Phi(f)=\int_0^{\infty} f(x)d\si(x),\quad
f(x)\in C_0[0,\infty).$$

Здесь, как видно, и находят свое применения преобразования Лапласа и Лапласа-Стилтьеса. Окончательная теорема
формулируется так:
\begin{theorem}[Аналитический эквивалент неполноты]
Неполнота системы $\{e^{-\la_n t}\}_1^{\infty}$ в
$L^p[0,\infty)$\\ ($C_0[0,\infty)$) равносильна существованию
нетривиальной функции $F(w)$ вида:
$$L^p[0,\infty):\quad F(w)=\int\limits_0^{\infty}
e^{-wt}\ph(t)dt,\quad\ph\in L^q,$$
$$C_0[0,\infty):\quad F(w)=\int\limits_0^{\infty}e^{-wt}d\si(t),\quad\si\in
V[0,\infty),$$

обращающегося в $0$ в точках $\la_n.$
\end{theorem}

\subsection{Начальные результаты}
Ниже для простоты обозначений вместо <<система $\{e^{-\la_n
t}\}_1^{\infty}$>> будем писать <<система $*$>>. А вместо
$L^p[0,\infty)$ и $C_0[0,\infty)$ будем писать соответственно
просто $L^p$ и $C_0.$

\begin{theorem}\label{22}
Пусть $\La=\{\la_n\}_1^{\infty}$. Если $\La$ имеет
конечную предельную точку ($\la$) в $Re\;w>0$, тогда система *
полна и в $L^p$ и в $C_0$.
\end{theorem}

\begin{wrapfigure}[8]{l}{90pt}
\epsfbox{pictures.4}
\end{wrapfigure}

\textbf{Доказательство:} Предположим противное - система неполна,
тогда, согласно аналитическому эквиваленту, найдется нетривиальная
аналитическая в правой полуплоскости функция, обращающаяся в $0$ в
точках последовательности $\La$. Но в этом случае по теореме
единственности для аналитических функций она тождественно равна
нулю - противоречие с нетривиальностью. $\blacksquare$

Таким образом, про случай, разобранный в теореме нам все известно
и он становится нам неинтересен. Интерес приковывают случаи, когда
предельные точки расположены либо в бесконечности (в правой
полуплоскости), либо на мнимой оси.

\begin{theorem}\label{S}
Если множество предельных точек последовательности $\La$ на
мнимой оси имеет положительную линейную меру, то система * полна в
$C_0$.
\end{theorem}
\textbf{Доказательство:} Предположим противное - система неполна, тогда найдется нетривиальная $F(w)$, такая что
$F(\la_n)=0$. Пусть $Y$ множество предельных точек последовательности $\La$ на мнимой оси. Вспомним,
что $F(w)$ это преобразование Лапласа-Стилтьеса, а про него мы знаем, что оно непрерывно на $Re\;w\ge0$.
Поэтому $F(Y)=0$. Но $F(w)\in H^{\infty}$ (т.к. ограничена в правой полуплоскости), и по теореме
единственности для функций данного класса получаем, что $F\equiv0$ противоречие с
нетривиальностью.$\blacksquare$

\begin{theorem}
Рассмотрим условие Саса:
$$\suml{n=1}{\infty}\frac{Re\;\la_n}{1+|\la_n|^2}=\infty\quad(S).$$

1) Условие $(S)$ достаточно для полноты системы * в $L^p,\quad
2\le p<\infty$ и в $C_0$.

2) Условие (S) необходимо для полноты * в $L^p,\quad 1\le
p\le2.$
\end{theorem}
\textbf{Доказательство:} 1)Предположим противное - * неполна в
$L^p$ ($C_0$). Значит некоторая нетривиальная функция $F(w)$
обращается в $0$ в точках $\La$. Но по Теореме \ref{2.9}
$F(w)\in H^p \;(H^{\infty}).$ Но для функций из классов Харди
выполнено условие Бляшке для полуплоскости, которое является
отрицанием $(S).$

2) Предположим $(S)$ не выполнено, а значит выполнено $(B)$. Тогда
по Теореме \ref{B} найдется аналитическая и ограниченная в правой
полуплоскости функция $\wt{B}(w)$, такая, что
$\wt{B}(\la_n)=0.$ Рассмотрим
$F(w)=\wt{B}(w)/(1+w)^2.$ Пусть $|\wt{B}(w)|<C$,
тогда
$$\int_{\R }|F(u+iv)|^pdv\le\int_{\R }\frac{C}{(1+v^2)^p}\le\infty\quad,
u>0.$$ Поэтому $F(w)\in H^p.$ А значит, по Теореме \ref{2.10},
$F(w)$ представляется в виде преобразования Лапласа от функции из
$L^q$ а это означает неполноту * - противоречие.$\blacksquare$

\begin{imp}[Теорема Саса]
Полнота * в $L^2$ равносильна условию $(S).$
\end{imp}

Такая формулировка теоремы Саса эквивалентна приведенной в части
$3.1.$

\subsection{Случай пространства $C_0$ (1972, Siegel)}
\begin{theorem}
Условие $(S)$ не является необходимым для полноты * в $C_0$.
\end{theorem}
\textbf{Доказательство:} Положим
$$\La=\left\{\left\{\al_k+\frac{in}{\beta_k}\right\}_{n\in
\mathbb{Z}}\right\}_{k\in\mathbb{N}},\quad\al_k,\;\beta_k>0,\quad\al_k\ra0,\quad\beta_k\ra\infty,\quad\suml{k=1}{\infty}\al_k\beta_k<\infty.$$
Видно, что у $\La$ множество предельных точек - вся мнимая
ось, поэтому по Теореме \ref{S} система * полна в $C_0.$ Теорема
будет доказана, если мы покажем, что для * выполняется $(B).$

$$\sum\limits_0^{\infty}\frac1{n^2+b^2}\le\frac1{b^2}+\int\limits_0^{\infty}\frac{dx}{x^2+b^2}=\frac1{b^2}+\frac{\pi}{2b}.$$
Применим эту оценку ниже:
\begin{multline*}
\sum\limits_{\la\in\La}\frac{Re\;\la}{1+|\la|^2}=\sum\limits_{k,n}\frac{\al_k}{1+\al_k^2+\frac{n^2}{\beta_k^2}}=\sum\limits_k\al_k\beta_k^2\sum\limits_n\frac1{\beta_k^2(1+\al_k^2)+n^2}\le\\
\le2\sum\limits_k\al_k\beta_k^2\hr{\frac1{\beta_k^2(1+\al_k^2)}+\frac{\pi}{2\beta_k\sqrt{1+\al_k^2}}}=2\sum\limits_k\frac{\al_k}{1+\al_k^2}+\pi\sum\limits_k\frac{\al_k\beta_k}{\sqrt{1+\al_k^2}}<\infty.
\end{multline*}
Условие $(B)$ выполнено, а * полна.$\blacksquare$

\subsection{Необходимое условие Саса}
\begin{theorem}\label{neob}
Если * полна в $C_0$ или в $L^p,\quad p\ge2$, тогда
$$\fa\de>0\qquad\suml{k=1}{\infty}\frac{Re\;\la_k+\de}{1+|\la_k|^2}=\infty.$$
\end{theorem}
\textbf{Доказательство:} Вспомним, что вначале изучения проблемы
Мюнца-Саса мы имели дело с системой
$\{x^{\mu_n}\}_1^{\infty},\quad Re\;\mu_n>-\frac1p\quad **$. Мы
показывали, что полнота ** в $L^p(0,1)\quad(C_0[0,1])$ равносильна
полноте системы * в $L^p(0,\infty)\quad(C_0[0,\infty)).$ Из
условий теоремы следует, что ** полна в $L^p(0,1),\quad p\ge2$
или в $C_0[0,1]$. Но тогда она полна и в $L^2(0,1)$ (поскольку
отрезок конечен, можно применить неравенство Гельдера). Ну а тогда
можно воспользоваться теоремой Саса (Теорема \ref{sas}) и получить
(не забывая, что $\la_k=\mu_k+\frac1p$), что
$$\suml{k=1}{\infty}\frac{Re\;\mu_k+\frac12}{1+|\mu_k+\frac12|^2}=\infty\Ra\suml{k=1}{\infty}\frac{Re\;\la_k+\frac12-\frac1p}{1+|\la_k+\frac12-\frac1p|^2}=\infty.$$
Пусть $\de_0=\frac12-\frac1p$, тогда
$$\suml{k=1}{\infty}\frac{Re\;\la_k+\de_0}{1+|\la_k+\de_0|^2}=\infty.$$
$\de_0$ в знаменателе не влияет на расходимость, поскольку
$1+|\la_k|^2\le1+|\la_k+\de_o|^2\le
C(1+|\la_k|^2).$ Таким образом,
$$\suml{k=1}{\infty}\frac{Re\;\la_k+\de_0}{1+|\la_k|^2}=\infty.$$
Но это означает, что либо
$$\suml{k=1}{\infty}\frac{Re\;\la_k}{1+|\la_k|^2}=\infty.$$
либо
$$\suml{k=1}{\infty}\frac1{1+|\la_k|^2}=\infty.$$
Отсюда, очевидно, и следует утверждение теоремы.$\blacksquare$

\subsection{Случай, когда $\La$ отделена от мнимой оси}
\begin{theorem}\label{331}
Пусть $\fa n\in\mathbb{N}\quad Re\;\la_k\ge h>0$. Тогда
для полноты * в $L^p$ или $C_0$ необходимо достаточно выполнение
$(S).$
\end{theorem}
\textbf{Доказательство:} \emph{Необходимость} В эту сторону нужно
доказывать только для $L^p\quad p>2,\quad C_0$. Предположим *
полна, тогда по необходимому условию Саса (Теорема \ref{neob})
$$\suml{k=1}{\infty}\frac{Re\;\la_k+h}{1+|\la_k|^2}=\infty.$$
Но тогда, по теореме сравнения для рядов
$$2\suml{k=1}{\infty}\frac{Re\;\la_k}{1+|\la_k|^2}=\infty.$$
А это и есть $(S).$

\emph{Достаточность} В эту сторону нужно доказывать только для
$L^p\quad 1\le p<2,\quad C_0$. Предположим противное - система *
неполна, тогда по аналитическому эквиваленту найдется
нетривиальная аналитическая $\Phi(w)$, такая, что
$\Phi(\La)=0.$
\begin{equation}\label{30}
|\Phi(w)|=\hm{\int\limits_0^{\infty}e^{-wt}\ph(t)dt}\le\hn{e^{-wt}}_{L^p}\hn{\ph}_{L^q}=C\frac1{(pu)^{1/p}}.
\end{equation}

$$\hm{\Phi\hr{w+\frac{h}2}}\le C_p\frac1{(\frac{h}2)^{1/p}}<\infty$$

Поэтому для $\Phi(w+\frac{h}2)$ выполнено $(B)$ (следует из
принадлежности к $H^{\infty}$) и пользуясь тем, что нулями для
этой функции являются $\la_k-\frac{h}2$, получаем
\begin{equation}\label{20}
\suml{k=1}{\infty}\frac{Re\;\la_k-\frac{h}2}{1+|\la_k-\frac{h}2|^2}<\infty.
\end{equation}
Заменяя в последнем неравенстве $Re\;\la_k$ на $h$, получаем
\begin{equation}\label{21}
\suml{k=1}{\infty}\frac1{1+|\la_k-\frac{h}2|^2}<\infty.
\end{equation}
Из (\ref{20}) и (\ref{21}) получаем
$$\suml{k=1}{\infty}\frac{Re\;\la_k}{1+|\la_k-\frac{h}2|^2}<\infty\Ra\suml{k=1}{\infty}\frac{Re\;\la_k}{1+|\la_k|^2}<\infty.\blacksquare$$

\begin{imp}[Теорема Мюнца]
Пусть $0<\la_1<\la_2<\ldots$, тогда для полноты * в
$L^p,\quad 1\le p<\infty$ или $C_0$ необходимо и достаточно чтобы
$$\suml{k=1}{\infty}\frac1{\la_k}=\infty. \quad(M)$$
\end{imp}
\textbf{Доказательство:} в роли $h$ из предыдущей теоремы
выступает $\la_1$, значит полнота равносильна условию~$(S)$.
Поскольку $\la_k\in\R $ и если считать, что
$\la_k\ra\infty$, то
$$\frac{\la_k}{1+\la_k^2}\sim \frac1{\la_k},\qquad
(n\ra\infty).$$

А если $\la_k\nrightarrow\infty$, тогда * полна в $L^p$ и в
$C_0$ (по Теореме \ref{22}) и ряд, очевидно,
расходится.$\blacksquare$

\noindent\textbf{Замечание:} Поскольку сходимость ряда
$\sum\frac1{\mu_n+\frac1p}$ эквивалентна сходимости ряда
$\sum\frac1{\mu_n}$, то эта формулировка теоремы Мюнца никак не
противоречит приведенной в части $2.1.$


\subsection{Случай, когда $\La\in\R $}

Приведем для начала несколько лемм (некоторые - без
доказательства). Читателю поначалу может показаться, что они не
найдут своего применения впоследствии, но это не так - все они
понадобятся для доказательства необходимости в Теореме Шварца.

\begin{lemma}\label{35}
$$tg(t)\in L^2(1,\infty)\Ra g(t)\in
L^q(1,\infty)\quad\fa q\in[1,2).$$
\end{lemma}
\textbf{Доказательство:} Положим
$r=\frac2q,\;\frac1r+\frac1s=1\Ra \frac1s=\frac{2-q}2.$
Тогда
$$\int\limits_1^{\infty}|g(t)|^qdt=\int\limits_1^{\infty}|g(t)t|^q\frac1{t^q}dt\le\hr{\int\limits_1^{\infty}|g(t)t|^2dt}^{1/r}\hr{\int\limits_1^{\infty}\frac1{t^{\frac{2q}{2-q}}}dt}^{1/s}<\infty.\qquad\blacksquare$$

\begin{lemma}\label{34}
Пусть $g\in L^2(\R )$, пусть $\wh{g}$ дифференцируемо
всюду за исключением конечного числа точек и $\wh{g}\;'\in
L^2(\R )$, тогда $tg(t)\in L^2(\R ).$
\end{lemma}

\begin{lemma}\label{37}
Пусть
\begin{enumerate}
\item
$F(w)\in H^2.$
\item
$F(w)$ непрерывна в $Re\;w\ge0$ за исключением конечного числа
точек $T$ на мнимой оси.
\item
Вне $T$ функция $F(iv)$ дифференцируема и $F'(iv)\in
L^2(\R ).$

\end{enumerate}
Тогда, если $\La$ множество корней $F(w)$ в $Re\;w>0$, то
система $\{e^{-\la_nt}\}$ неполна в $L^p,\; p>2$ и в $C_0.$
\end{lemma}
\textbf{Доказательство:} Из условия 1 по теореме Пэли-Винера
следует, что
$$F(w)=\int\limits_0^{\infty}e^{-wt}f(t)dt,\quad f\in
L^2(\R ^{+}),\quad Re\;w>0.$$

Поэтому $F(iv)=\int\limits_0^{\infty}e^{-ivt}f(t)dt$. Пользуясь
леммой (\ref{34}) получаем, что $tf(t)\in L^2(\R ^{+})$ и в
частности $tf(t)\in L^2(1,\infty).$ Пользуемся леммой (\ref{35}) и
получаем, что $f(t)\in L^q(1,\infty),\;1\le q<2$. Поскольку $f\in
L^2(\R ^{+})$, то $f\in L^2(0,1)\Ra \;f\in
L^q(0,1),\;1\le q<2.$ Итак, $f(t)\in L^q(0,\infty),\;1\le q<2.$
Теперь остается лишь заметить, что наша $F(w)$ и есть та самая
функция, что фигурировала в аналитическом эквиваленте
неполноты.$\blacksquare$

\begin{theorem}[Шварц, 1943 необходимость, Грамм, 1957 достаточность]
Пусть $\la_n>0,\;n\in\mathbb{N}.$ Тогда для полноты системы *
в $L^p$ или в $C_0$ необходимо и достаточно
$$\suml{k=1}{\infty}\frac{\la_k}{1+\la_k^2}=\infty.$$
\end{theorem}
\textbf{Доказательство:} \emph{Достаточность} Очевидно,
интересовать нас может лишь случай $1\le p<2.$ Доказывать будем
от противного - предположим * не полна. Тогда по аналитическому
эквиваленту найдется нетривиальная аналитическая в $Re\;w>0$
функция $\Phi(w):\quad\Phi(\La)=0.$ Для $\Phi(w)$ верна оценка
(\ref{30}), приведенная в предыдущей части. Возьмем
$\La_1=\{\la\in\La:\quad\la>1/2\}$, тогда по
Теореме \ref{331} $(S)$ выполнено для $\La_1.$ Осталось
доказать, что оно выполнено и для
$\La_2=\{\la\in\La:\quad\la\le1/2\}$. Заметим,
что сходимость ряда
$\sum_{0<\la_k\le1/2}\frac{\la_k}{1+\la_k}$
эквивалентна сходимости ряда $\sum_{0<\la_k\le1/2}\la_k.$
Положим $G(w)=w^2\Phi(w)$.
\begin{equation}\label{31}
\hr{u-\frac12}^2+v^2\le\frac14\Ra\;u^2+v^2<u\Lra\;|w|<u.
\end{equation}
Из оценок (\ref{30}) и (\ref{31}) следует, что $|G(w)|\le
u^2\frac{C}{u^{1/p}}\le C$ в $|w-\frac12|\le\frac12.$ Т.е. $G$ аналитическая и ограниченная, поэтому для нулей функции
$G\hr{\frac12(w+1)}$ выполнено $(B)$, т.е.
$$\sum\hr{1-\hm{2\la_k-1}}<\infty\Ra\sum\la_k<\infty.$$
\emph{Необходимость} Как всегда доказывать будем от противного -
предположим не выполнено $(S)$, т.е. выполнено $(B)$. Докажем
неполноту системы * - это и будет противоречием. Из предыдущего
пункта доказательства возьмем $\La_1$ и $\La_2$ (только
теперь граничной точкой будет не $1/2$, а $1$). $(B)$ для
$\La_1$ таково: $\sum\frac1{\la_n}<\infty.$

Возьмем
$$B_1(w)=\prod\limits_{\la_n\in\La_1}\frac{\la_n-w}{\la_n+w}=\prod\limits_{\la_n\in\La_1}\hr{1-\frac{2w}{\la_n+w}}.$$

Докажем, что $B_1(w)$ аналитическая в правой полуплоскости,
включая мнимую ось. Для этого возьмем компакт $K$ такой, что его
самая <<левая>> точка имеет абсциссу $u_0>-1.$ Тогда в $K$ для
элементов $\La_1$ выполнено:
$\hm{\frac{2w}{\la_n+w}}\le\frac{M}{\la_n}$
(поскольку $\la_n<2|\la_n+w|,\;Re\;w>-1$). А это означает,
что $B_1(w)$ мажорируется абсолютно сходящимся рядом. Поэтому ряд,
определяющий $B_1(w)$, сходится равномерно на каждом компакте и
утверждение об аналитичности $B_1(w)$ доказано.

Оценим $B_1'(w)$ на мнимой оси:
$$\frac{iB_1'(iv)}{B_1(iv)}=(\log
B_1(iv))'=\sum(\log(\la_n-iv)-\log(\la_n+iv))'=-i\sum\hr{\frac1{\la_n-iv}+\frac1{\la_n+iv}}=-2i\sum\frac{\la_n}{\la_n^2+v^2}.$$
Поэтому, пользуясь $(B)$, получаем:
\begin{equation}\label{36}
\hm{\frac{B_1'(iv)}{B_1(iv)}}<\infty.
\end{equation}
Заметим, что $|B(iv)|=1$, поэтому, пользуясь (\ref{36}),
$|B_1'(iv)|<\infty.$ Очевидно также, что $B_1(w)$ ограничена в
правой полуплоскости (просто из ее явного вида:
$|\la_n-w|<|\la_n+w|$ в правой полуплоскости).

Теперь перейдем к $\La_2$. Условие $(B)$ для нее записывается
так: $\sum\la_n<\infty.$ Построим последовательность
$\La_3=\{z_n=\frac1{\la_n}\}.$
$\sum\frac1{z_n}<\infty,\quad z_n>1$, поэтому по $\La_3$ можно
построить функцию $B_2(z):$
$$B_2(z)=\prod\limits_{z_n\in\La_3}\frac{z_n-w}{z_n+w}=\prod\limits_{z_n\in\La_3}\hr{1-\frac{2w}{z_n+w}}.$$

$B_2(z)$ аналитична и ограничена в $Re\;z\ge0$,
$B_2(\La_3)=0.$ И $|B_2'(iy)|\le M.$ Рассмотрим
$A(w)=B_2\hr{\frac1w}.$ $A(w)$ аналитична в $Re\;w>0$ и
ограничена и непрерывна в $Re\;w\ge0$, $A(\La_2)=0$,
$$A'(iv)=-\frac1{v^2}B_2'\hr{\frac1{iv}}\Ra
|A'(iv)|\le\frac{C}{v^2}.$$

Пусть $F(w)=\frac{w^2}{(1+w)^3}B_1(w)A(w)$. Далее автор лекций
предлагает нам наслово поверить тому, что $F(w)\in H^2.$ Несложно
проверить, что для $F(w)$ остальные условия Леммы \ref{37} тоже
выполнены и поэтому * неполна в $L^p,\quad p>2$ и в
$C_0$.$\blacksquare$

\begin{wrapfigure}[10]{l}{100pt}
\epsfbox{pictures.3}
\end{wrapfigure}

Ниже приведены некоторые интересные и важные результаты, которые
мы не будем доказывать.

\begin{theorem}[Грамм]
Если $\La$ лежит в объединении конечного числа кругов из
правой полуплоскости, касающихся мнимой оси, то условие $(S)$
является необходимым и достаточным для полноты $e(\La)$ в
$L^p,\;1\le p<\infty$ и в $C_0$.
\end{theorem}

\begin{theorem}[Седлецкий]
Пусть
\begin{equation}\label{33}
\intl{-\infty}{\infty}\frac{\log
\dist(iv,\La)}{1+v^2}\;dv>-\infty.
\end{equation}

Тогда $(S)$ является необходимым для полноты * в $L^p, \;p>2$ и в
$C_0.$
\end{theorem}

\begin{theorem}[Ладыгин]
Если выполнено условие (\ref{33}), тогда условие $(S)$ достаточно
для полноты системы * в $L^p,\;1\le p<2.$
\end{theorem}

\begin{imp}
В классе последовательностей $\La$ со свойством (\ref{33})
условие $(S)$ является необходимым и достаточным для полноты * в
$L^p,\;1\le p<\infty$ и в $C_0.$
\end{imp}

Теперь рассмотрим некоторые частные случаи.

\begin{enumerate}
\item
Если $Re\;\la_n\ge h>0$, тогда $dist(iv,\La)\ge h$ и
поэтому
$$\intl{-\infty}{\infty}\frac{\log
\dist(iv,\La)}{1+v^2}\;dv>h\intl{-\infty}{\infty}\frac{dv}{1+v^2}>-\infty.$$
Т.е. (\ref{33}) выполнено.
\item
Если $\la_n\in\R ^{+}$, тогда
$$\intl{-\infty}{\infty}\frac{\log
\dist(iv,\La)}{1+v^2}\;dv>\intl{-\infty}{\infty}\frac{\log
|v|}{1+v^2}\;dv>-\infty.$$ Т.е. (\ref{33}) выполнено. Таким
образом, теорема Шварца Грама является следствием теоремы
Седлецкого Ладыгина.
\end{enumerate}


\newpage
\section{Введение в негармонический анализ}
\textbf{Опр:} \emph{Гармонический анализ} - это часть
математического анализа, изучающая ряды и преобразования Фурье.

\defin \emph{Негармонический анализ} занимается
изучением аппроксимационных свойств систем $e^{i\la_nt},\;
(\La=\{\la_n\}_1^{\infty}\subset \mathbb{C})$ в
функциональных пространствах \underline{\texttt{на конечном}}
интервале или отрезке . В основном это пространства $L^p(a,b)$ и
$C[a,b]$.

\subsection{Аналитический эквивалент. Свойство инвариантности}

Пусть у нас есть система
$e(\La)=\{e^{i\la_nt}\}_{n=1}^{\infty}.$ Будем считать,
что $|\la_{n+1}|>|\la_n|.$ В роли $(a,b)$ у нас будет
выступать $(-\pi,\pi).$

\begin{theorem}[Левинсон, N.Levinson (1936), Аналитический эквивалент неполноты]
Неполнота системы $e(\La)$ в $L^p(-\pi,\pi),\;1\le p<\infty$
(в $C(-\pi,\pi)$) равносильна существованию нетривиальной функции
вида

\begin{equation}\label{40}
L^p(-\pi,\pi):\quad
\Phi(z)=\intl{-\pi}{\pi}e^{izt}\ph(t)dt,\quad\ph\in
L^q(-\pi,\pi).
\end{equation}

\begin{equation}\label{41}
C(-\pi,\pi):\quad
\Phi(z)=\intl{-\pi}{\pi}e^{izt}d\si(t),\quad
var\si(t)<\infty,
\end{equation}
обращающейся в $0$ в точках $\La.$


\end{theorem}

Заметим, что $\Phi(z)$ является целой функцией, т.е. аналитической
во всей комплексной плоскости.

\begin{lemma}\label{ew}
Если $\la$ корень функции вида (\ref{40}) (или (\ref{41})),
то $\frac{\Phi(z)}{z-\la}$ тоже имеет вид (\ref{40}) (или
(\ref{41})).
\end{lemma}
\textbf{Доказательство:} По условию

$$\intl{-\pi}{\pi}e^{i\la u}\ph(u)du=0.$$

Поэтому, воспользовавшись формулой интегрирования по частям, можно
записать

$$\Phi(z)=\intl{-\pi}{\pi}e^{i(z-\la)t}d\hr{\intl{-\pi}{t}e^{i\la u}\ph(u)du}=-i(z-\la)\intl{-\pi}{\pi}e^{izt}\left(e^{-i\la t}\intl{-\pi}{t}e^{i\la
u}\ph(u)du\right)dt.$$

Т.е.

$$\frac{\Phi(z)}{z-\la}=\intl{-\pi}{\pi}e^{izt}\ph_{\la}(t)dt,\quad\text{где  }\ph_{\la}(t)=-ie^{-i\la t}\intl{-\pi}{t}e^{i\la u}\ph(u)du\in L^q(-\pi,\pi).\blacksquare$$

\begin{theorem}[Свойство инвариантности полноты]
Полнота $e(\La)$ в $L^p(-\pi,\pi)$ ($C[-\pi,\pi]$) не
нарушится при замене конечного числа точек $\la_n$ таким же
количеством точек не из $\La.$
\end{theorem}
\textbf{Доказательство:} Пусть $e(\La)$ неполна. Заменим
точку $\la_1$ точкой $\mu_1$ и покажем, что новая система
$e(\wt{\La})$ неполна.

Раз $e(\La)$ неполна, то $\exi \Phi(z)$ вида (\ref{40}):
$\Phi(\La)=0.$ Тогда
$\wt{\Phi}(z)=\frac{z-\mu_1}{z-\la_1}\Phi(z)$
обращается в $0$ в точках новой последовательности
$\wt{\La}.$ Остается лишь доказать представимость
$\wt{\Phi}(z)$ в виде (\ref{40}) :

$$\frac{z-\mu_1}{z-\la_1}\;\Phi(z)=\hr{1+(\la_1-\mu_1)\frac1{z-\la_1}}\Phi(z)=\Phi(z)+(\la_1-\mu_1)\frac{\Phi(z)}{z-\la_1}.$$

Для завершения доказательства неполноты $e(\wt{\La})$
остается лишь воспользоваться предыдущей леммой. Таким образом,
нами показано, что при замене конечного числа точек неполнота
сохраняется. Сохранение полноты отсюда очевидным образом следует.
Действительно: предположим $e(\La)$ полна, а
$e(\wt{\La})$ неполна, но тогда применяем уже
доказанную часть теоремы и получаем, что $e(\La)$ также
неполна - противоречие полностью завершающее доказательство
теоремы.$\blacksquare$

\subsection{Условие полноты в терминах считающей функции}

\begin{wrapfigure}[10]{l}{162pt}
\epsfbox{pictures.2}
\end{wrapfigure}

\defin \emph{Считающей функцией} последовательности
$\La$ называется функция $n_{\La}(t)=$ (число точек
последовательности $\La$ в круге $|z|<t).$

Обозначим
$$N_{\La}(r)=\int\limits_0^r\frac{n_{\La}(t)}{t}dt,\qquad0\notin\La.$$

Обозначим $n_{\Phi}(t)$ считающую функцию для нулей функции
$\Phi(z).$ Тогда
$$N_{\Phi}(r)=\int\limits_0^r\frac{n_{\Phi}(t)}{t}dt,\qquad \Phi(0)\neq0.$$

Приведем ниже теорему без доказательства. Ею мы позже
воспользуемся.
\begin{theorem}[Формула Йенсена]
Пусть $F(z)$ целая функция и $|F(0)|=1.$ Тогда
$$\fa r>0\qquad N_F(r)=\frac1{2\pi}\intl{-\pi}{\pi}\log|F(re^{i\theta})|d\theta.$$
\end{theorem}



\begin{theorem}\label{50}
Если $0\notin\La$ и
$$\ol{\lim}_{r\ra\infty}\left(N_{\La}(r)-2r+\frac1p \log
r\right)>-\infty,\quad 1<p<\infty\qquad(=+\infty,\quad
p=1,\infty),$$ то $e(\La)$ полна в $L^p(-\pi,\pi).$
\end{theorem}
\textbf{Доказательство:} Предположим противное система неполна.
Тогда существует нетривиальная $\Phi(z)$ класса (\ref{40}):
$\Phi(\La)=0.$ Воспользовавшись неравенством Гельдера,
получим:
\begin{multline*}
|\Phi(z)|\le\intl{-\pi}{\pi}e^{|y| |t|}|\ph(t)|dt=\int_{\pi-\de<|t|<\pi}+\int_{|t|<\pi-\de}
\le\hn{\ph}_{L^q(\pi-\de<|t|<\pi)}\hr{2\intl{\pi-\de}{\pi}e^{p|y|t}dt}^{1/p}+\\
+\hn{\ph}_{L^q(-\pi,\pi)}\hr{2\int\limits_0^{\pi-\de}e^{p|y|t}dt}^{1/p}\le\ep\frac{e^{\pi|y|}}{|y|^{1/p}}+
C_p\frac{e^{|y|(\pi-\de)}}{|y|^{1/p}}
=\frac{e^{\pi|y|}}{|y|^{1/p}}\hr{C_pe^{-\de|y|}+\ep}.
\end{multline*}

Если $\Phi(0)\neq0$, то можно домножить $\Phi(z)$ на такую
константу, чтобы попасть в условия формулы Йенсена. А если $0$ корень кратности $m$ для $\Phi(z)$, то заменим $\Phi(z)$ на
$\frac{(z-1)^m}{z^m}\Phi(z).$

Ясно, что $n_{\La}(t)\le n_{\Phi}(t)$. Значит
$N_{\La}(t)\le N_{\Phi}(t).$ И поэтому
\begin{multline*}
N_{\La}(t)\le\frac1{2\pi}\intl{-\pi}{\pi}\log|\Phi(re^{i\theta})|d\theta\le
\frac1{2\pi}\intl{-\pi}{\pi}\pi
r|\sin\theta|d\theta-\frac1{2\pi}\intl{-\pi}{\pi}\frac1p(\log
r+\log|\sin\theta|)d\theta+\\
+\frac1{2\pi}\intl{-\pi}{\pi}\log(C_pe^{-\de
r|\sin\theta|}+\ep)d\theta=C+2r-\frac1p\log
r+\frac1{2\pi}\intl{-\pi}{\pi}\log(C_pe^{-\de
r|\sin\theta|}+\ep)d\theta.
\end{multline*}

Обозначим последнее слагаемое в этой сумме за $I_r$ и докажем, что
$I_r\xra{r\ra\infty} -\infty.$

Разобьем интеграл $I_r$ на два: $I_1$ и $I_2$. В $I_1$
интегрирование ведется по $\frac{\pi}4<|t|<\frac{3\pi}4$, а в
$I_2$ по всему остальному.
$$I_1\le\frac1{2\pi}2\pi\log\left(e^{-\de
r\frac1{\sqrt{2}}}+\ep\right).$$

Поэтому, за счет выбора $\ep$, $I_1$ может быть сделан
$\le-M,\quad\fa M>0$ при $r>r_0$. Очевидно также, что
$I_2<C<+\infty.$ Поэтому $I_r\xra{r\ra\infty}
-\infty.$

Подведем итоги. Мы получили, что $\left(N_{\La}(r)-2r+\frac1p
\log r\right)\xra{r\ra\infty}-\infty$ противоречие с условием.

Фактически мы разобрали лишь случай $1<p<\infty.$ Оставшийся
случай $p=1,\infty$ доказывается аналогично (только исчезает
слагаемое, стремящееся к $-\infty$, но здесь оно нам и не нужно).
Единственное, не лишне было бы напомнить, что под $L^{\infty}$
подразумевается пространство с нормой $\inf\{M:\quad|f(x)|\le
M,\quad x\in\R \backslash E,\quad\mu(E)=0\}\blacksquare.$

\begin{imp}[Теорема Левинсона]
Если $|\la_n|\le|n|+\frac1{2p},\quad1<p<\infty$
($|\la_n|\le|n|+\frac1{2p}-\ep,\quad p=1,\infty$),
тогда $\{e^{i\la_n t}\}_{n=-\infty}^{+\infty}$ полна в
$L^p(-\pi,\pi).$
\end{imp}
\begin{proof}
  Можно считать, что $\la_0>0$ (в силу
  теоремы об инвариантности полноты при конечных заменах). Если
  выполнены условия теоремы и $n+\frac1{2p}<t<n+1+\frac1{2p}$, то
  $n_{\La}(t)=2n+1.$

  Тогда
  \begin{multline*}
    N_{\La}(r)=\int\limits_0^r\frac{n_{\La}(t)}tdt=\intl{\la_0}{1+\frac1{2p}}\frac{dt}t+\sum\limits_{n=1}^N\intl{n+\frac1{2p}}{n+1+\frac1{2p}}\frac{2n+1}t
    dt=C+\sum\limits_{n=1}^N(2n+1)\log\frac{n+1+\frac1{2p}}{n+\frac1{2p}}\le\\
    \le
    C+2\sum\limits_{n=1}^N\hr{n+\frac1{2p}+\frac12-\frac1{2p}}\log\hr{1+\frac1{n+\frac1{2p}}}=\\
    =C+2\sum\limits_{n=1}^N\hr{\hr{n+\frac1{2p}}+\hr{\frac12-\frac1{2p}}}\hr{\frac1{n+\frac1{2p}}-\frac1{2\hr{n+\frac1{2p}}^2}+O\hr{\frac1{n^3}}}=\\
    =O(1)+2N-\sum\limits_{n=1}^N\frac1p\frac1{n+\frac1{2p}}=O(1)+2N-\frac1p\log
    N=O(1)+2r-\frac1p\log r=N_{\La}(r).
  \end{multline*}
  И остается лишь воспользоваться Теоремой \ref{50}
\end{proof}

\begin{note}
  Константа $\frac1{2p}$ в Теореме
  Левинсона точная, т.е. не может быть уменьшена. Можно построить
  пример, иллюстрирующий это. Но мы не станем этим заниматься.
\end{note}

\subsection{Устойчивость полноты системы экспонент}
\begin{theorem}
Пусть есть две последовательности: $\La$ и $M$. И пусть
$\sum|\la_n-\mu_n|<\infty$, тогда системы $e(\La)$ и
$e(M)$ полны в $L^p(-a,a)$ и в $C[-a,a]$ одновременно.
\end{theorem}
\begin{proof}
  Предположим условия теоремы выполнены, но
  $e(M)$ полна, а $e(\La)$ неполна. Тогда по аналитическому
  эквиваленту существует нетривиальная целая функция $F$ вида
  $$F(z)=\int\limits_{-a}^a e^{izt}f(t)dt,\qquad f\in L^q,$$
  обращающаяся в $0$ в точках $\La.$ Покажем, что тогда
  существует нетривиальная целая $G(z)$, обращающаяся в $0$ в точках
  M.

  Пусть $f_0(t)=f(t)$. Тогда, пользуясь леммой (\ref{ew}),
  \begin{equation*}
    F_1(z)=\frac{z-\mu_1}{z-\la_1}F(z)=F(z)+\frac{\la_1-\mu_1}{z-\la_1}F(z)=\int\limits_{-a}^ae^{izt}f_1(t)dt,\quad
    f_1(t)=f(t)+i(\mu_1-\la_1)\int\limits_{-a}^te^{i\la_1(u-t)}f_0(u)du.
  \end{equation*}
  Аналогично
  \begin{equation*}
    F_n(z)=\prod\limits_{k=1}^n\hr{\frac{z-\mu_k}{z-\la_k}}F(z)=\int\limits_{-a}^ae^{izt}f_n(t)dt,\quad
    f_n(t)=f_{n-1}(t)+i(\mu_n-\la_n)\int\limits_{-a}^te^{i\la_n(u-t)}f_{n-1}(u)du.
  \end{equation*}

  Рассмотрим $\la_n:\quad
  Im\;\la_n\le0\Ra|e^{i\la_n(u-t)}|=e^{Im\;\la_n(t-u)}\le1.$
  Тогда

  $$\hm{\int\limits_{-a}^te^{i\la_n(u-t)}f_{n-1}(u)du}\le\int\limits_{-a}^t\hm{e^{i\la_n(u-t)}f_{n-1}(u)}du\le(2a)^{1/p}\hn{f_{n-1}}_{L^q}.$$

  И значит
  $\hn{f_n}_{L^q}\le(1+C_q|\mu_n-\la_n|)\hn{f_{n-1}}_{L^q}\le\hn{f_0}_{L^q}\prod\limits_{k=1}^n(1+C_q|\mu_k-\la_k|).$

  Бесконечное произведение $\prod\limits_{k=1}^{\infty}(1+C_q|\mu_k-\la_k|)$ сходится по условию теоремы,
  поэтому все частные произведения ограничены. Вывод: $\hn{f_n}_{L^q}\le C<\infty.$

  $$\hn{f_n-f_{n-1}}_{L^q}\le
  A|\mu_n-\la_n| \hn{f_{n-1}}_{L^q}\le B|\mu_n-\la_n|.$$

  Поэтому в $L^q(-a,a)$ сходится ряд
  $\sum_{n=1}^{\infty}(f_n-f_{n-1})$. Обозначим за $g(t)$ предел
  этого ряда. Легко заметить, что
  $g(t)=-f_0(t)+\lim_{n\ra\infty}f_n(t).$ Поэтому у
  последовательности $f_n$ есть предел в $L^q$. Обозначим его $g_0$
  $(g_0=g+f_0).$

  Обозначим $G(z)=\intl{-a}{a}e^{izt}g_0(t)dt$. Тогда
  $$|G(z)-F_n(z)|\le\intl{-a}{a}e^{izt}|g_0(t)-f_n(t)|dt.$$
  И поэтому $F_n(z)\rra G(z)$ на каждом круге конечного
  радиуса. $G(z)$ целая по построению. К тому же ясно, что
  $G(\mu_n)=0$, если $Im\;\mu_n\le 0.$ Построим функцию
  $\wt{G}(z)$:
  $$\wt{G}(z)=\left\{\begin{aligned}
  G(z),\qquad Im\;z\le0\\
  \ol{G(\ol{z})},\qquad Im\;z\ge0\end{aligned}\right.$$
  Полученная функция по одной из теорем комплексного анализа будет
  целой. Кроме того, $\wt{G}(M)=0.$
\end{proof}

\subsection{Радиус полноты системы экспонент}
\begin{df}
  \emph{Радиусом полноты} системы экспонент $e(\La)$ называют
  $$r(\La)=\sup\hc{a>0:\;
  e(\La)\text{ полна в }L^2(-a,a)\}=\inf\{a>0:\;
  e(\La)\text{ не полна в }L^2(-a,a)}.
  $$
\end{df}

\begin{df}
  \emph{Плотностью} последовательности
  $\La$ называют $\displaystyle \De(\La)=\lim\limits_{n\ra\infty}\frac
  n{|\la_n|}.$ Предела может и не существовать, тогда можно
  рассматривать \emph{верхнюю плотность}
  $\displaystyle \ol{\De}(\La)=\varlimsup\limits_{n\ra\infty}\frac
  n{|\la_n|}.$
\end{df}

\begin{theorem}[Картрайт, Cartwright] Пусть
$F(z)=\int_{-a}^ae^{izt}f(t)dt,\quad f\in L^q(-a,a),\quad f\neq0$
п.в. в любой окрестности точки $a.$ Тогда $\fa\quad
0<\ep<\pi/2$ сужение корней $F(z)$ в угле
$|arg\;z|<\ep$ имеет плотность $\De=\frac a{\pi}$, вне
его $\De=0.$ Одним словом, почти все нули функции $F(z)$
расположены вблизи действительной оси.
\end{theorem}


\begin{theorem}
Пусть $\La\in \R ^{+}$ и
$\ol{\De}(\La)>\frac a{\pi}.$ Тогда $e(\La)$
полна в $L^2(-a,a).$
\end{theorem}
\begin{proof}
  Предположим противное - система неполна.
  Тогда существует нетривиальная целая функция как в условиях
  предыдущей теоремы с $q=2.$ Поэтому плотность ее нулей в секторе
  $|arg\;z|<\ep$ равна $a/\pi$. В частности
  $\ol{\De}(\La)\le a/\pi$ противоречие с
  условием.
\end{proof}

\begin{hypotetic}[Шварца]
  Пусть $\exi\De(\La),\quad\La\subset\R ^{+}.$ Тогда
  $r(\La)=\pi\De(\La).$
\end{hypotetic}

Эта гипотеза оказалась неверна. В 1950-е годы Ж.П.Кахан построил
такой опровергающий ее пример:
$\La\subset\R ^{+},\quad\De(\La)=0,\quad
r(\La)=+\infty.$

Окончательно задачу описания $r(\La)$ в каких-либо терминах
решили Бъерлинг и Мальявен (1962 - для вещественных $\La$,
1967 - $\La\subset\mathbb{C}$).

Введем новый класс функций. $M_a$ класс непрерывных кусочно
дифференцируемых на $\R ^{+}$ функций $\ph(t)$:
$0\le\ph'(t)\le a$ во всех точках дифференцируемости.

\begin{wrapfigure}[7]{l}{5.4cm}
\epsfbox{pictures.1}
\end{wrapfigure}

Пусть у нас есть последовательность
$\La\subset\R ^{+}$, $n_{\La}(t)$ ее считающая
функция. Обозначим за $S_a(t)$ <<тень-функцию>> этой считающей
функции. Что это такое? $S_a(t):=$наименьшая мажоранта функции
$n_{\La}(t)$ из класса $M_a$.

\begin{df}
  \emph{Эффективной плотностью} называется
  величина
  \begin{equation*}
    A_e(\La):=\inf\hc{a>0:\frac{S_a-n_{\La}(t)}{1+t^2}\notin
    L^1(\R ^{+})}.
  \end{equation*}
\end{df}

\begin{theorem}[Бъерлинг и Мальявен]
$r(\La)=\pi A_e(\La).$
\end{theorem}

\subsection{Минимальные системы экспонент}

\begin{df}
  Система $\{e_n\}_{n=1}^{\infty}$ элементов Банахова пространства
  называется \emph{минимальной}, если ни один ее элемент не
  принадлежит замыканию линейной оболочки остальных, т.е.
  $dist(e_n,\;clos \{e_k\}_{k\neq n})>0$. Другими словами, нет лишних
  элементов.
\end{df}

\begin{denote}
  Обозначим за $B'$ пространство линейных непрерывных функционалов
  на $B.$
\end{denote}

\noindent\textbf{Эквивалентное определение минимальности:}
$\{e_n\}_{n=1}^{\infty}$ минимальна в $B\Lra$
существует система сопряженных функционалов $f_n\in B'$, т.е.
такая, что $f_n(e_m)=\de_{mn}.$ Система
$\{f_n\}_{n=1}^{\infty}$ называется биортогональной к системе
$\{e_n\}_{n=1}^{\infty}.$

\begin{df}
  Пусть $F(z)$ целая функция. Говорят, что
  $F(z)$ \emph{конечного порядка}, если при некотором $\mu>0$
  $|F(z)|\le e^{|z|^{\mu}},\quad\fa |z|\ge R_0$. Инфимум таких
  $\mu$ называют \emph{порядком} (будем обозначать $\rho$) целой
  функции.
\end{df}

\begin{df}
  Пусть $F(z)$ целая порядка $\rho>0.$
  Говорят, что $F(z)$ конечного типа при порядке $\rho$, если
  $\exi A:\quad|F(z)|\le e^{A|z|^{\rho}},\quad|z|>R_1.$ Инфимум
  таких $A$ называют типом функции $F(z)$ (будем обозначать
  $\si$) при порядке $\rho.$
\end{df}

\begin{ex}
  \begin{enumerate}
  \item
    $F(z)=e^{az^n},\quad\rho=n,\;\si=a$
  \item
    $F(z)=\sin\;z,\quad\rho=1,\;\si=1.$
  \end{enumerate}
\end{ex}

\begin{df}
  Целой функцией экспоненциального типа $a$
  будем называть функцию с порядком $1$ и типом $a$.
\end{df}

Введем новое пространство $W^2_a$ целых функций, экспоненциального
типа не выше, чем $a$, таких, что
$$\int_{\R }|F(x)|^2dx<\infty.$$

\begin{theorem}[Пэли-Винер]
$W^2_a$ совпадает с классом функций, представимых в виде
$$F(z)=\int\limits_{-a}^ae^{izt}f(t)dt,\quad f\in L^2(-a,a).$$
\end{theorem}

\begin{theorem}
Для того, чтобы $e(\La)$ была минимальной в $L^2(-a,a)$
необходимо и достаточно, чтобы существовала нетривиальная целая
$F(z)$: $F(\La)=0$, $\frac {F(z)}{z-\la_1}\in W^2_a.$
\end{theorem}
\begin{proof}
  \emph{Необходимость} Пусть $e(\La)$
  минимальна в $L^2(-a,a)$. Воспользовавшись фактом из функционального анализа, что $(L^2)'\simeq
  L^2$, имеем, что найдется биортогональная система
  $\{f_n\}_{n=1}^{\infty}\subset L^2(-a,a)$. $f_n(e^{i\la_k t})=(f_n(t),e^{i\la_k
    t})=\de_{nk}$ ($(\cdot\;,\cdot)$ скалярное произведение в
  $L^2(-a,a)$).
  $$F_1=\int\limits_{-a}^ae^{izt}f_1(t)dt,\quad f_1\in L^2(-a,a).$$
  Но, в силу биортогональности системы $\{f_n\}_{n=1}^{\infty}$,
  $F_1(\la_n)=\de_{1n}.$ Возьмем в качестве $F(z)$ из
  формулировки этой теоремы функцию $(z-\la_1)F_1(z).$

  \emph{Достаточность} Рассмотрим
  $\frac{F(z)}{F'(\la_n)(z-\la_n)}\in W^2_a$. Поэтому
  $$\frac{F(z)}{F'(\la_n)(z-\la_n)}=\int\limits_{-a}^ae^{izt}f_n(t)dt,\quad f_n\in L^2(-a,a).$$

  Из последнего равенства видно, что $f_n(t)$ образуют
  биортогональную систему, поскольку
  $$\lim_{z\ra\la_n}\frac{F(z)}{F'(\la_n)(z-\la_n)}=1,\qquad F(\la_k)=0\quad(k\neq n).$$
\end{proof}

\begin{df}
  $$f_n(t)=\frac1{2\pi}\int_{\R }e^{-ixt}\frac{F(x)}{F'(\la_n)(x-\la_n)}dx,\text{ где  } F(x)=\int\limits_{-a}^ae^{ixt}f(t)dt.$$
  \emph{Негармоническим рядом Фурье} функции $f(t)\in L^2(-a,a)$
  называется ряд
  $$\suml{n=1}{\infty}(f_n,f)e^{i\la_n t}.$$
\end{df}

\section{<<Ликбез>>}
\subsection{Преобразование Фурье}
Преобразование Фурье первоначально определяется для функций из $L^1$. В этом параграфе мы постараемся
распространить его и на некоторые другие классы функций.

\begin{stm}
 $\hat{f}(x)\in\Cb(\R ),\;\hn{\wh{f}}_{\Cb(\R )}\le\frac1{2\pi}\hn{f}_{L^1},\;\lim_{x\ra\pm\infty}\wh{f}(x)=0.$
 \end{stm}

Доказательство этого утверждения входит в стандартный курс функционального анализа. Оно не очень сложное, но,
тем не менее, на нем останавливаться мы не будем. Заметим лишь простое следствие: оператор, действующий из
$L^1$ в $L^{\infty}$ как преобразование Фурье, является ограниченным.

\begin{theorem}[Планшерель] Если $f\in L^2(\R )$, тогда существует предел (в $L^2)$ $\lim_{A\ra\infty}\wh{f_A}(x)$, где $\wh{f_A}(x)=\int_{-A}^A
e^{-ixt}f(t)dt$. Этот предел и назовем преобразованием Фурье для
функции $f$. Если кроме того $f\in L^1(\R )$, то этот предел совпадает
с обыкновенным преобразованием Фурье.
\end{theorem}

\begin{theorem}[Равенство Парсеваля]
  $\hn{\wh{f}}_{L^2}=\sqrt{2\pi}\hn{f}_{L^2}.$
\end{theorem}

Опираясь на результаты этих теорем, можно сказать, что мы распространили преобразование Фурье и на функции из
$L^2$:
$$\wh{f}:\;L^2(\R )\ra L^2(\R ).$$

\begin{theorem}[1930г, Рисс-Торин] Пусть $T$ линейный оператор, $T:L^{p_1}(X,d\mu)\ra
L^{q_1}(Y,d\eta)$ и $T:L^{p_2}(X,d\mu)\ra L^{q_2}(Y,d\nu)$. Пусть $1\le
p_1,p_2,q_1,q_2\le\infty,\;p_1\neq p_2,\;q_1\neq q_2$ и
\begin{equation}\label{102}
\frac1p=\frac{1-\theta}{p_1}+\frac{\theta}{p_2},\quad
\frac1q=\frac{1-\theta}{q_1}+\frac{\theta}{q_2},\;0<\theta<1.
\end{equation}

Тогда $T:L^{p}(X,d\mu)\ra L^{q}(Y,d\nu).$
\end{theorem}

Применим эту теорему к следующему случаю: $X=\R
,\;d\mu=d\nu=dx,\;p_1=1,\;p_2=2,\;q_1=\infty,\;q_2=2.$ Пусть
$p\in(1,2)$, тогда, очевидно, $\exi\theta\in(0,1):\;\frac{1-\theta}1+\frac{\theta}2=\frac1p.$ $q$ в
этом случае определяется из равенства
$\frac{1-\theta}{\infty}+\frac{\theta}2=\frac1q.$ Легко заметить, что
$\frac1p+\frac1q=1.$ И мы пришли к

\begin{theorem}[Хаусдорф-Юнг]
Пусть $p\in[1,2],\;q\in[2,\infty],\;\frac1p+\frac1q=1$. Тогда $\wh{f}:L^p(\R )\ra
L^q(\R ).$ Правильнее будет сформулировать теорему так: $f\in L^p(\R ),\;1\le
p\le2\Ra \wh{f}(x)=\lim_{A\ra\infty} \wh{f_A}(x)$ (предел здесь в
$L^q(\R )$).
\end{theorem}

Вот так мы распространили преобразование Фурье с $L_1$ на $L_p,\;1\le p \le 2$. На $p>2$ распространить
нельзя.

Пусть $T$ ограниченный оператор, $T:L^p(X,d\mu)\ra L^q(Y,d\nu)$. Пусть
$$E_{\si}=\{y\in Y:|Tf(y)|\ge\si\},\;mes E_{\si}=\eta(E_{\si}).$$ Ограниченность $T$ означает, что
$\hn{Tf}_{(q,d\nu)}\le M\hn{f}_{(p,d\mu)}.$ Значит

$$M\hn{f}_{(p,d\mu)}=M\hr{\int_X |f|^pd\mu}^{1/p}\ge \hr{\int_Y |Tf|^qd\nu}^{1/q}\ge
\hr{\int_{E_{\si}} |Tf|^qd\nu}^{1/q}\ge\si(mes(E_{\si}))^{1/q}.$$ Поэтому
\begin{equation}\label{*}
\nu(E_{\si})\le\hr{\frac{M\hn{f}_{(p,d\mu)}}{\si}}^q.
\end{equation}

\begin{df}
  Говорят, что оператор $T$ имеет \emph{слабый тип} $(p,q)$, если $\fa\si>0$
  выполнено (\ref{*}), где $M$ не зависит от $f$ и $1\le p,q\le\infty.$
\end{df}

\begin{df}
  Говорят, что оператор $T$ имеет \emph{сильный тип} $(p,q)$, если он действует как
  ограниченный из $L^p(X,d\mu)$ в $L^q(Y,d\nu).$
\end{df}

\begin{theorem}[Марцинкевич] Пусть $T$ имеет слабый тип
$(p_1,q_1)$ и $(p_2,q_2),\;1\le p_i\le q_i<\infty$. Тогда $T$ имеет сильный тип $(p,q)$, где $p$ и $q$
определяются из (\ref{102}).
\end{theorem}

Применим эту теорему к такому случаю: $X=Y=\R ,\;d\mu=dx,\;d\nu=\frac{dx}{x^2},\;Tf=x\wh{f}(x)$
\begin{enumerate}
\item
$$p=q=2:\quad\hn{Tf}_{2,d\nu}=\hr{\int_{\R }|y\wh{f}(y)|^2\frac{dy}{y^2}}^{1/2}=\hn{\wh{f}}_2=\sqrt{2\pi}\hn{f}_{2,d\mu}.$$
Таким образом оператор $T$ имеет сильный тип $(2,2).$
\item
$p=1: f\in L^1\Ra\hn{f}_{\infty}\le\frac1{2\pi}\hn{f}_1.$ Поэтому очевидно, что
$$E_{\si}=\{y:\;|Tf(y)|\ge\si\}\subseteq\left\{y:\;\frac1{2\pi}\hn{f}_1|y|\ge\si\right\}=\R \backslash\hr{-\frac{2\pi\si}{\hn{f}_1},\frac{2\pi\si}{\hn{f}_1}}=\wt{E}_{\si}.$$

$$\nu(\wt{E}_{\si})=\int_{\wt{E}_{\si}}\frac{dx}x^2=2\intl{\frac{2\pi\si}{\hn{f}_1}}{\infty}\frac{dx}{x^2}=\frac{\hn{f}_1}{\pi\si}.$$

Поэтому $\nu(E_{\si})\le\frac{\hn{f}_1}{\pi\si}$, а это в свою очередь есть условие (\ref{*}) для
$q=1.$ Поэтому оператор $T$ имеет слабый тип $(1,1).$
\end{enumerate}
Объединяем вместе результаты пунктов 1 и 2 и получаем (по теореме Марцинкевича), что $x\wh{f}(x)$ имеет
сильный тип $(p,p),\;\fa p\in(1,2).$ Это просто-напросто означает, что если $f\in
L^p(\R ),\;1<p<2,$ то $x\wh{f}(x)\in L^p\hr{\R ,\frac{dx}{x^2}}.$

$$\int_{\R }|x\wh{f}(x)|^p\frac{dx}{x^2}\le M\int_{\R }|f(x)|^p
dx.$$

\begin{theorem}
$$f\in L^p(\R ),\;1<p<2\Ra
\int_{\R }|\wh{f}(x)|^px^{p-2}dx\le M_p\int_{\R }|f(x)|^pdx.$$ Т.е.
\begin{enumerate}
\item
$\wh{f}:L^p(\R )\ra L^p_{p-2}(\R ),\;1<p\le2$
\item
$\wh{f}:L^p_{p-2}(\R )\ra L^p(\R ),\;2\le p<\infty.$
\end{enumerate}
\end{theorem}

Введем новое пространство $L^p_{\al}(\R )$:
$$\hn{f}_{p,\al}=\hr{\int_{\R }|f(t)|^p|t|^{\al}dt}^{1/p},\;1\le p<\infty,\;\al\in \R .$$
\begin{theorem}[Питт]
Пусть

\begin{center}$\left\{\begin{aligned}
&1\le p\le q<\infty\\
&0\le\al<p-1\\
&-1<\beta\le0\\
&\frac{1+\al}p+\frac{1+\beta}q=1
\end{aligned}\right.$
\end{center}
Тогда $\wh{f}:L^p_{\al}(\R )\ra L^q_{\beta}(\R ).$
\end{theorem}

\begin{wrapfigure}[10]{l}{220pt}
\epsfbox{pictures.8}
\end{wrapfigure}

Легко заметить, что частными случаями этой теоремы будут Теорема Хаусдорфа-Юнга ($1<p\le2,\;\al=\beta=0$)
и Теорема Харди-Литтлвуда ($p\ge2,\;\al=p-2,\;q=p,\;\beta=0$).

Можно ввести еще более общий класс пространств $L^p_{\omega(t)}$:
$$\hn{f}_{p,\omega(t)}=\hr{\int_{\R }|f(t)|^p\omega(t)dt}^{1/p}.$$

Теперь можно поставить еще более общую задачу: какими должны быть $p,\;q,\;u(t)$ и $w(t)$ чтобы
$\wh{f}:L^p_{u(t)}\ra L^q_{w(t)}.$ Эта задача решалась в 70-80 года и решил ее Mucken Houp. Мы
не будем здесь приводить его результаты.

\subsection{Преобразования Фурье на полупрямой и первая теорема Пэли Винера}
С классами Харди мы уже встречались в \S2.5, но сейчас мы слегка видоизменим определение, данное раньше.
$H^p$ это класс аналитических в верхней полуплоскости функций, таких, что
$$\hn{f}_p=\sup_{y>0}\hr{\int_{\R }|f(x+iy)|^pdx}^{1/p}<\infty. $$
Получается, что мы просто <<перевернули>> определение, данное ранее. Материал, излагаемый далее можно найти в
следующих книгах:

\medskip
Titchmarsh <<Введение в  теорию аппроксимации>>,

Зигмунд <<Тригонометрические ряды>>,

Гоффман <<Банаховы пространства аналитических функций>>,

Гарнет <<Ограниченные аналитические функции>>.

\medskip
Все те свойства функций из класса Харди, что были изложены в \S2.5 остаются и для нового определения (только
теперь, очевидно, роль мнимой оси будет играть действительная).

Еще одно новое свойство класса $H^p$: если $f\in H^p,\;0<p<\infty$, то $\fa
y_0>0\;f(z)\xra{|z|\ra\infty}0$ в полуплоскости $y\ge y_0.$

$$F(z)=\int\limits_0^{\infty}e^{izt}f(t)dt\quad (*).$$

\begin{theorem}\label{104}
1. $f(x)\in L^p(\R ^+),\;1\le p\le2\Ra F(z)\in H^q.$ \\\hangindent=2.3cm 2. $F(z)\in
H^p,\;1\le p\le2\Ra F(z)$ представима в виде (*) с $f\in L^q(\R ^+).$
\end{theorem}
\begin{proof}

  1.
  \begin{equation}\label{101}
    F(z)=\int\limits_0^{\infty}e^{izt}f(t)dt,\quad f\in L^p(\R ^+),\;1\le p\le2.
  \end{equation}
  $|e^{izt}f(t)|=e^{-yt}|f(t)|$, поэтому интеграл в (\ref{101}) сходится абсолютно и равномерно, значит можно
  дифференцировать и значит есть аналитичность.

  $$F(x+iy)=\int\limits_0^{\infty}e^{ixt}(e^{-yt}f(t))dt.$$
  Но $e^{-yt}f(t)\in L^p(\R ^+)$ как произведение функций из $L^p(\R ^+)$. Теперь можно применить
  теорему Хаусдорфа-Юнга, согласно которой $\hn{F(x+iy)}_q\le c_p\hn{f}_p$. Отсюда и следует принадлежность $F$
  к классу $H^q.$

  2. Пусть $F\in H^p,\;1\le p\le 2.$ Фиксируем $y>0.$

  \begin{wrapfigure}[10]{l}{200pt}
    \epsfbox{pictures.7}
  \end{wrapfigure}

  $$\wh{F}_R(x+iy)=\frac1{2\pi}\int_{-R}^RF(t+iy)e^{-ixt}dt.$$

  По одному из свойств функций из класса $H^p$, $F(z)\xra{|z|\ra{\infty}}0.$ И мы попадаем в
  условия леммы Жордана, которая утверждает, что тогда
  $$\frac1{2\pi}\int_{C_R}e^{-iwx}F(w+iy)dw\xra{R\ra{\infty}}0. $$
  Ну а по теореме Коши о вычетах:
  $$\wh{F}_R(x+iy)=\int_{-R}^R e^{-ixt}F(t+iy)dt=\int_{C_R}e^{-iwx}F(w+iy)dw\xra{R\ra{\infty}}0. $$
  Т.е. $x<0\Ra \wh{F}_R(x+iy)\xra{R\ra\infty}0$, т.е.
  $\wh{F}(x+iy)=0\;\fa x<0.$

  Поскольку $F\in H^p,$ то $F(x+iy)\in L^p$ и значит $\wh{F}(x+iy)\in L^q.$ Но
  $\wh{F}(x+iy)\xra{y\ra0}\wh{F}(x)$ (сходимость в $L^q$). Поэтому $\wh{F}(x)=0$
  на $\R ^-.$

  \begin{equation}\label{103}
    \int\limits_0^{\infty}e^{izt}\wh{F}(t)dt=|\text{равенство
      Парсеваля}|=\int_{\R }\wh{e^{izt}}F(t)dt=\ldots
  \end{equation}

  $$\wh{e^{izt}}=\frac1{2\pi}\int\limits_0^{\infty}e^{-iut}e^{izt}dt=\frac1{2\pi}\int\limits_0^{\infty}e^{-it(z-u)}dt=\left.\frac1{2\pi
    i(z-u)}e^{it(z-u)}\right|_{t=0}^{\infty}=\frac1{2\pi i}\frac1{u-z}.$$ Продолжаем равенство (\ref{103}):
  $$\ldots =\frac1{2\pi i}\int_{\R }\frac{F(t)}{t-z}dt=F(z),\quad Im\;z>0.$$
  Последнее равенство - известная теорема из комплексного анализа.
\end{proof}

\begin{theorem}[первая, Пэли-Винер]
Класс $H^2$ совпадает с классом функций, представимых в виде \\\hangindent=6.5cm (*) с $f\in
L^2(\R ^+),\;Im\;z>0.$
\end{theorem}

\subsection{Вторая и третья теоремы Пэли-Винера}
В этом параграфе мы вернемся к теореме Пэли-Винера, изложенной в \S4.5 и докажем ее.

$$F(z)=\int\limits_{-a}a e^{izt}f(t)dt.\quad(*)$$

\begin{theorem}
1. $f\in L^p(-a,a),\;1\le p \le2\Ra F(z)\in W^q_a.$ \\\hangindent=2.5cm 2. $F(z)\in W^p_a,\;1\le
p\le2\Ra F(z)$ представляется в виде (*) с $f\in L^q(-a,a).$
\end{theorem}
\begin{proof}

  \begin{enumerate}
  \item
    По неравенству Гельдера:
    $$|F(z)|\le\hn{e^{izt}}_q\hn{f}_p=C\hr{\int\limits_{-a}^a e^{-qyt}dt}^{1/q}\le C_1\left(\int\limits_0^a
    e^{q|y|t}dt\right)^{1/q}=C_1\hr{\frac{e^{q|y|a}-1}{q|y|}}\sim e^{a|y|}\le e^{a|z|}.$$

    Поэтому экспоненциальный тип у $F(z)$ не превосходит $a.$ Ну а то, что $F(x)\in L^q$ следует просто из
    теоремы Хаусдорфа-Юнга.
  \item
    Для доказательства второй части теоремы воспользуемся утверждением, которое не будем доказывать.

    \begin{stm}[Boas <<Entire functions>>, 1954г.]
      $F(z)\in B^p_a\Ra F(z)e^{iaz}\in H^p (y>0).$
    \end{stm}

    Применяем это утверждение и теорему \ref{104} из предыдущего параграфа, получаем:
    $$F(z)e^{iaz}=\int\limits_0^{\infty}e^{izt}f_+(t)dt.$$

    Поэтому

    $$F(z)=\int\limits_0^{\infty}e^{iz(t-a)}f_+(t)dt=\intl{-a}{\infty}e^{izt}f_+(t+a)dt,\quad f_+(t+a)\in L^q(-a,\infty).$$

    Аналогично $F(-z)e^{iaz}\in H^p$ и поэтому
    $$F(z)=\int\limits_{-\infty}^a e^{izt}f_-(a-t)dt,\quad f_-(a-t)\in L^q(-\infty,a).$$

    Теперь только остается воспользоваться теоремой единственности для преобразования Фурье, откуда следует, что
    $f\in L^q(-a,a).$

  \end{enumerate}
\end{proof}
Введем новый класс функций $H^p(-a,a)$. Он состоит из функций, аналитических в полосе $|y|<a$ и таких, что
$$\sup_{|y|<a}\int_{\R }|f(x+iy)|^pdx<\infty.$$

Пусть
$$F(z)=\int_{\R }e^{izt}e^{-a|t|f(t)dt}.\quad(**)$$

\begin{theorem}
1. $f\in L^p(\R ),\;1\le p\le2\Ra F(z)\in H^q(-a,a).$\\\hangindent=2.5cm 2. $F(z)\in
H^p,\;1\le p\le2\Ra F(z)$ представляется в виде (**) с $f\in L^q(\R ).$
\end{theorem}

\begin{theorem}[третья, Пэли-Винер]
$H^2(-a,a)$ совпадает с классом функций, представимых в виде (**) с $f\in L^2(\R).$
\end{theorem}

\end{document}


%% Local Variables:
%% eval: (setq compile-command (concat "latex  -halt-on-error -file-line-error " (buffer-name)))
%% End:
