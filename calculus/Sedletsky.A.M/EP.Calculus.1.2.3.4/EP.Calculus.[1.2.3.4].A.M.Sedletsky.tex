\documentclass[a4paper]{article}
\usepackage[utf,simple]{dmvn}

\title{Математический анализ}
\author{Лектор Анатолий Мечиславович Седлецкий\\
{\footnotesize Решения задач: DMVN Corporation, UrKud}}
\date{1--4 семестры, 2002--2004 г.}

\begin{document}
\maketitle

\section{1 семестр}

\subsection{Программа экзамена}

\begin{nums}{-2}
\item Множества. Операции над множествами и их свойства. Отображения. Простейшая классификация отображений.
Обратное отображение. Композиция отображений. Композиция биекций.
\item Аксиоматика и свойства вещественных чисел. Верхняя и нижняя грани числового множества. Лемма о верхней грани.
\item Натуральные числа. Принцип математической индукции. Неравенство Бернулли. Бином Ньютона.
Принцип Архимеда и его следствия. Геометрическая интерпретация вещественных чисел. Модуль числа и его свойства.
Множества точек на прямой.
\item Лемма о вложенных отрезках. Лемма о конечном покрытии. Лемма о предельной точке.
\item Эквивалентные множества. Счётные множества и их свойства. Несчётность множества $[0,1]$ и её следствия.
Мощность континуума.
\item Предел последовательности. Определения и примеры. Ограниченность сходящейся последовательности.
Предел и предельная точка. Единственность предела. Переход к пределу в неравенстве. Арифметические операции
над пределами.
\item Критерий Коши. Теорема о пределе монотонной последовательности. Число $e$.
\item Частичный предел последовательности. Теорема Больцано Вейерштрасса. Верхний и нижний пределы последовательности.
Их существование у ограниченной последовательности. Условия сходимости ограниченной последовательности. Бесконечно
большие последовательности. Расширение множества $\R$. Расширенный вариант теоремы Больцано Вейерштрасса. Верхний
и нижний пределы произвольной последовательности. Условия сходимости произвольной последовательности в широком смысле.
\item Предел функции в точке. Определение, примеры отрицание. Локальная ограниченность функции, имеющей предел.
Предел функции в точке по Гейне. Эквивалентность понятий предела по Коши и по Гейне. Единственность предела.
Бесконечно малые функции и их свойства. Арифметические операции над пределами.
\item Переход к пределу (функции) в неравенстве. Предел промежуточной функции. Первый замечательный предел.
Критерий Коши существования предела функции в точке. Предел монотонной функции.
\item Предел функции по базе. Наиболее употребительные базы. Бесконечно большие функции и их связь с бесконечно малыми.
Односторонние пределы. Предел композиции функций. Второй замечательный предел.
\item Непрерывность функции в точке. Определения, примеры ($у = \const$, $у = х$, $у = \sin x$). Односторонняя
непрерывность. Классификация точек разрыва, примеры Локальные свойства непрерывных функций. Непрерывность многочлена,
рациональной и тригонометрических функций.
\item Глобальные свойства непрерывной функции на отрезке. Теорема о нуле непрерывной функции, промежуточные значения.
Теоремы Вейерштрасса об ограниченности и достижимости точных граней.
\item Точки разрыва монотонной функции, их характер и мощность. Критерий непрерывности монотонной функции.
Теорема об обратной функции. Обратные тригонометрические функции.
\item Построение показательной функции на основе теории пределов и непрерывности. Логарифмическая и степенная функция.
Гиперболические функции.  Обратные гиперболические функции.
\item Понятие равномерной непрерывности. Примеры. Равномерная непрерывность функции, непрерывной на отрезке.
Модуль непрерывности функции и его свойства.
\item Сравнение функций. Символы <<$O$>> и <<$o$>>, их свойства. Примеры. Критерий
эквивалентности функций. Таблица эквивалентных бесконечно малых. Замена эквивалентных при вычислении пределов. Примеры.
\item Понятие производной функции. Механический и геометрический смысл. Дифференцируемость функции в точке,
необходимое и достаточное условие дифференцируемости. Непрерывность функции, имеющей производную.  Дифференциал и
его геометрический смысл. Производная композиции функций.  Инвариантность формы дифференциала. Производная
обратной функции. Правила дифференцирования.
\item Таблица производных. Логарифмическое дифференцирование. Производные и дифференциалы высших порядков.
Формула Лейбница. Параметрическое дифференцирование. Пример.
\item Теоремы Ферма, Ролля, их геометрический смысл. Односторонние производные, геометрический смысл, связь с
односторонней непрерывностью. Бесконечные производные. Теорема Дарбу.
\item Теорема Лагранжа и её следствия: постоянство функции с нулевой производной, равномерная непрерывность функции
с ограниченной производной, достаточное условие строгой монотонности. Доказательства неравенств, предел производной,
характер точек разрыва производной.
\item Теорема Коши. Правило Лопиталя (раскрытие неопределенностей вида $\frac{0}{0}$ и $\frac{\infty}{\infty}$).
Сравнение роста показательной, степенной и логарифмической функций.
\item Формула Тейлора. Остаточный член в общей форме. Остаточный член в форме Коши, Лагранжа, Пеано. Разложения
элементарных функций но формуле Тейлора Маклорена. Применения.
\item Исследование монотонности и экстремумов функции с помощью первой производной. Применение к доказательству
неравенств. Достаточное условие существования обратной функции.
\item Выпуклые функции. Эквивалентные определения. Существование односторонних производных. Два критерия выпуклости
дифференцируемой функции. Вогнутые функции.
\item Точки перегиба функции, необходимое условие точки перегиба, достаточное условие. Исследование функции
с помощью высших производных. Асимптоты графика функции.
\item Классические неравенства (Йенсена, Юнга, Гёльдера, Минковского, сравнение среднего геометрического
со средним арифметическим).
\item Первообразная. Неопределенный интеграл. Таблица интегралов. Интегрирование заменой переменной и по частям.
Обобщенная первообразная.
\end{nums}

\subsection{Вопросы и задачи коллоквиумов}

\subsubsection{Коллоквиум №1}

Теоретическая часть совпадает с вопросами 1 -- 14 экзаменационной программы. Задачи:

\begin{problem}
Установить биекцию $[0, 1] \lra (0, 1)$.
\end{problem}
\begin{problem}
Доказать, что в любой окрестности рациональной точки есть иррациональная точка.
\end{problem}
\begin{problem}
Доказать, что если $x_n \ra a$, то и $\frac{x_1 + x_2 + \ldots + x_n }{n} \ra a$ (здесь и далее $n \ra \infty$).
\end{problem}
\begin{problem}
Доказать, что если $x_n \ra a$, то и $\sqrt[n]{x_1 x_2 \ldots x_n } \ra a$ при $a > 0$.
\end{problem}
\begin{problem}
Доказать, что $x_n = \frac{1}{1} + \frac{1}{2} + \ldots + \frac{1}{n}$ не является фундаментальной.
\end{problem}
\begin{problem}
Доказать, что $x_n = \frac{1}{1^2} + \frac{1}{2^2} + \ldots + \frac{1}{n^2}$ является фундаментальной.
\end{problem}
\begin{problem}
Доказать, что $\sqrt[n]{a} \ra 1$ при $a > 0$.
\end{problem}
\begin{problem}
Доказать, что $\frac{n}{q^n} \ra 0$ при $q > 1$.
\end{problem}
\begin{problem}
Доказать, что $\sqrt[n]{n} \ra 1$.
\end{problem}
\begin{problem}
Доказать, что $\frac{q^n}{n!} \ra 0$.
\end{problem}
\begin{problem}
Доказать, что $\ulim x_n = \liml{n\ra \infty} \supl{k\ge n} x_k$.
\end{problem}
\begin{problem}
Доказать, что $\ulim\hr{x_n + y_n} \le \ulim x_n + \ulim y_n$.
\end{problem}
\begin{problem}
Доказать, что если одна из последовательностей сходится, то $\ulim \hr{x_n + y_n} = \ulim x_n + \ulim y_n$.
\end{problem}
\begin{problem}
Доказать, что функция Дирихле $\Dc(x) = \case{1, & x \in \Q; \\ 0, & x \in \R \wo \Q}$ разрывна всюду.
\end{problem}
\begin{problem}
Доказать сходимость и найти предел: $x_1 = a, \; x_2 = b, \; x_{n+2} = \frac{1}{2}\hr{x_n + x_{n + 1}}$.
\end{problem}
\begin{problem}
Доказать сходимость и найти предел: $x_0 > 0, \; x_{n + 1} = \frac{1}{2}\hr{x_n + \frac{1}{x_n}}$.
\end{problem}

\subsubsection{Коллоквиум №2}

Теоретическая часть совпадает с вопросами 15 -- 23 экзаменационной программы. Задачи:
\setcounter{problem}{0}
\begin{problem}
Найти модуль непрерывности функции $\sqrt{x}$, $x \in [0,1]$.
\end{problem}

\begin{problem}
Доказать, что если $f$ периодична, то $f'$ периодична с тем же периодом.
\end{problem}

\begin{problem}
$f(x) = \hm{x}^3$. Найти $f'(x)$, $f''(x)$, $f'''(0)$.
\end{problem}

\begin{problem}
Доказать, что если $\hm{f(x)-f(y)} \le C \hm{x-y}^2$ для $\fa x,y \in [a,b]$, то $f = \const$.
\end{problem}

\begin{problem}
Доказать, что если все корни многочлена вещественны, то и все корни его производной вещественны.
\end{problem}

\begin{problem}
Доказать, что если $f$ дифференцируема на $\R_+$ и $f'(x) \ra 0, \; x \ra \infty$, то
$f(x+h)-f(x) \ra 0$ для $\fa h > 0$ при $x \ra \infty$.
\end{problem}

\begin{problem}
Доказать, что если $f'$ непрерывна на $[a,b]$, то $f$ равномерно дифференцируема на $[a,b]$, е
для $\fa \ep > 0$ найдётся $\de > 0$, такое, что если $x,y \in [a,b]$ и $\hm{x-y} < \de$,
то $\hm{f'(x)-\frac{f(x)-f(y)}{x-y}} < \ep$.
\end{problem}

\begin{problem}
Пусть $y(x)=\case{0, & x \le 0; \\ e^{-1/x}, & x > 0.}$ Показать, что $y \in \Cb^\infty(\R)$. Как выглядит график
функции $y(x)$?
\end{problem}

\begin{problem}
Доказать неравенства и дать их геометрическую иллюстрацию:
\eqn{\cos x > 1 - \frac{x^2}{2}, \; x \neq 0; \quad e^x > 1 + x + \frac{x^2}{2}, \; x \ne 0.}
\end{problem}

\begin{problem}
Пусть
\eqn{y(x)=\case{0, & \hm{x} \ge 1, \\ \exp\hr{\frac{1}{x^2-1}}, & \hm{x} < 1.}}
Показать, что $y \in \Cb^\infty(\R)$. Как выглядит график функции $y(x)$?
\end{problem}

\begin{problem}
Пусть
\eqn{y(x) = \case{0, & x = 0, \\ x^{2n}\sin\frac{1}{x}, & x \neq 0.}}
Найти $y^{(i)}(0)$ при $i=\ol{1,\, n+1}$.
\end{problem}

\begin{problem}
Доказать, что если $f''(x)$ существует, то
\eqn{f''(x) = \liml{h\ra 0} \frac{f(x+2h)-2f(x+h)+f(x)}{h^2}.}
Как выглядит аналогичная формула для $f'''(x)$?
\end{problem}

\begin{problem}
Пусть $f \in\Cb^\infty[-1,1]$, $f^{(n)}(0)=0$ при любом $n \ge 0$, и
\eqn{\supl{x\in[-1,1]} \bm{f^{(n)}(x)} \le C^n \cdot n! \text{ при } n \in \N.}
Доказать, что $f \equiv 0$.
\end{problem}

\begin{problem}
Написать многочлен Тейлора Маклорена $P_n(x)$ для функции $e^x$.

\pt{1} Оценить $\hm{e^x - P_{10}(x)}$ при $x \in[0,1]$.

\pt{2} При каком $h > 0$ верно $\hm{e^x - P_{10}(x)} \le 10^{-7}$ при $x \in [0,h]$?

\pt{3} При каком $n$ верно $\hm{e^x - P_n(x)} \le 10^{-7}$ при $x \in [0,1]$?
\end{problem}

\begin{problem}
Подобрать коэффициенты $a,b $ так, что разность
\eqn{\cos x - \frac{1+ax^2}{1+bx^2}}
была бесконечно малой при $x \ra 0$ по возможности наивысшего порядка.
\end{problem}

\pagebreak

\section{2 семестр}

\subsection{Программа экзамена}

\begin{nums}{-2}
\item Интеграл Римана как предел по базе. Необходимое условие интегрируемости.
\item Суммы Дарбу.  Критерий интегрируемости в предельной  Форме.
\item Интегралы Дарбу как пределы сумм Дарбу. Следствия. Критерий Дарбу.
\item Классы интегрируемых функций.
\item Свойства интеграла Римана. Первая теорема о среднем.
\item Интеграл с переменным верхним пределом; его непрерывность и дифференцируемость.
Существование обобщённой первообразной, формула Ньютона Лейбница. Интегрирование по частям. Замена
переменной.
\item Формула Тейлора с остаточным членом в интегральной форме.
\item Вторая теорема о среднем.
\item Интегрируемость композиции. Неравенства Гёльдера и Минковского для интегралов.
\item Функции ограниченной вариации и их свойства. Критерий ограниченности вариации.
\item Отображения отрезка в $\R^n$. Интегрирование отображений. Вычисление вариации
непрерывно дифференцируемого отображения.
\item Понятие кривой. Спрямляемость.  Критерий спрямляемости. Вычисление длины кривой.
\item Квадрируемость плоской фигуры. Первый критерий квадрируемости. Свойства меры Жордана.
Второй критерий квадрируемости. Квадрируемость Фигуры, ограниченной конечным числом числом спрямляемых кривых.
Квадрируемость криволинейной трапеции и вычисление её площади.
\item Несобственные интегралы 1 и 2 рода. Простейшие свойства, критерий Коши. Теоремы сравнения.
Признаки Дирихле и Абеля. Абсолютная сходимость.
\item Замена переменной и интегрирование по частям в несобственном интеграле.
\item Интеграл Римана Стилтьеса. Достаточное условие существования, свойства, интегрирование по частям, вычисление.
\item Множества в $\R^m$. Выделение конечного покрытия из покрытия компакта открытыми множествами.
\item Функции нескольких переменных (ФНП). Предел и непрерывность скалярных ФНП. Локальные и
глобальные свойства непрерывных функций.
\item Частные производные и их геометрический смысл. Дифференцируемость ФНП. Необходимое условие.
Достаточное условие, геометрический смысл дифференцируемости (при $m=2$).
\item Дифференцируемость композиции функций. Инвариантность формы дифференциала.
\item Производная по направлению. Градиент.
\item Частные производные высших порядков. Теорема о равенстве смешанных производных.
\item Дифференциалы высших порядков. Формула Тейлора для ФНП. Экстремум ФНП. Необходимое условие. Достаточное условие
экстремума.
\item Последовательности в $\R^m$, предел. Критерий Коши. Теорема Больцано Вейерштрасса. Векторнозначные отображения.
Предел и непрерывность. Локальные свойства непрерывных отображений. Глобальные свойства.
\item Дифференцируемые отображения, производная, дифференциал.  Матрица Якоби. Дифференцируемость
отображения и его координат.
\item Необходимое условие дифференцируемости. Достаточное условие. Линейность операции дифференцирования.
\item Дифференцируемость композиции отображений. Дифференцируемость обратного отображения.
\item Неявные функции одной переменной. Теорема о неявной функции.
\item Неявные ФНП. Теорема о неявной функции. Уравнение касательной плоскости.
\item Неявные отображения. Теорема о неявном отображении.
\item Зависимость функций. Примеры. Условие независимости функций. Условие зависимости.
\item Теорема об обратном отображении (существование диффеоморфизма).
\item Условный экстремум. Метод множителей Лагранжа.
\end{nums}

\subsection{Вопросы и задачи коллоквиума}

Теоретическая часть совпадает с вопросами 1 -- 16 экзаменационной программы. Задачи:

\begin{problem}
Доказать, что функция Римана
\eqn{\Rc(x) = \case{0, & x \in [0,1]\wo \Q; \\ \frac{1}{n}, & x = \frac{m}{n}, \; \GCD(m,n)=1}}
интегрируема по Риману на отрезке $[0,1]$.
\end{problem}
\begin{solution}
Докажем, что $\int_0^1\Rc(x)dx=0$. Пусть $0=x_0<x_1<\ldots<x_N=1$
разбиение с отмеченными точками $\xi_1,\ldots,\xi_N$ и диаметром разбиения
$\la<\frac{1}{n^3}$. Тогда среди чисел $f(\xi_1),\ldots,f(\xi_N)$ не более
чем $1+2+\cdots+n\bw<n^2$ превосходят $\frac{1}{n}$. Следовательно,
$$\sum_{i=1}^N \xi_i (x_i- x_{i-1})\le n^2\la+n^{-2}<n^{-1}+n^{-2}.$$
Переходя к пределу, получаем требуемое равенство.
\end{solution}

\begin{problem}
Доказать, что изменение значений интегрируемости функции в конечном числе точек не меняет ни её интегрируемости, ни её
интеграла.
\end{problem}
\begin{solution}
Пусть $\De$  максимальное число, на которое изменялось значение функции,
$n$ количество точек, в которых изменялось значение функции. Тогда
интегральные суммы старой и новой функции отличаются не более, чем на
$\la n\De$. Далее по определению интеграла.
\end{solution}

\begin{problem}
\label{Pro3}Доказать, что если $f \in \Cb[a,b]$, $f \ge 0$, и $f(c) \ge 0$ в некоторой точке $c \in [a,b]$, то $\intl{a}{b}f(x)\,dx > 0$.
\end{problem}
\begin{solution}
В силу непрерывности $\exi \de > 0\cln \fa x\in [a,b] \cap [c-\de,c+\de]$ имеем $f(x) > f(c)/2$.
Тогда, так как $f(x)\ge 0$, то $\int_a^b f(x)\,dx\ge \delta f(c)/2>0$.
\end{solution}

\begin{problem}
Доказать, что если $f \in \Cb[a,b]$, $f \ge 0$, и $\intl{a}{b}f(x)\,dx = 0$, то $f \equiv 0$.
\end{problem}
\begin{solution}
Пусть только попробует функция $f$ вылезти выше нуля в некоторой точке. Тогда по задаче \ref{Pro3} интеграл будет
положительным.
\end{solution}

\begin{problem}
Пусть $f \in \Cb[a,b]$. Тогда в теореме о среднем $\xi \in (a,b)$.
\end{problem}

\begin{problem}
Какой из интегралов больше:
\eqn{\intl{0}{\pi/2} \exp\hr{-\cos x}\,dx \text{ или } \intl{0}{\pi/2}\exp\hr{-\frac{2x}{\pi}}\,dx?}
\end{problem}
\begin{solution}
Заметим сначала, что
\eqn{\intl{0}{\pi/2} e^{-2x/\pi}\, dx=\intl{0}{\pi/2} e^{\frac{-2(\pi/2-x)}\pi}\,dx= \intl{0}{\pi/2} e^{-1+2x/\pi}\,dx.}
Так как при $x\in [0,\pi/2]$ выполнено неравенство $\cos x\ge 1-2x/\pi$,
то $e^{-\cos x}\le e^{-1+2x/\pi}$. Далее осталось проинтегрировать неравенство.
\end{solution}

\begin{problem}
Доказать, что если $f \in\Rb[a,b]$, то $\hm{f}^p \in \Rb[a,b]$ при $ p \in (0,1)$.
\end{problem}
\begin{solution}
Без ограничения общности, $a=0$ и $b=1$. Докажем сначала, что при $u\ge v\ge 0$, $p \in (0,1)$ выполнено
неравенство $ u^p-v^p\le (u-v)^p$. Действительно, при таких $p$ функция $g(x)=x^p$ выпукла вверх.
Следовательно, при $x_1\le x_2, x_3\le x_4, x_2+x_3=x_1+x_4$ выполнено неравенство $g(x_1)+g(x_4)\le
g(x_2)+g(x_3)$. При $x_1=0, x_2=v, x_3=u-v,x_4=u$ получаем $0^p+u^p\le v^p+(u-v)^p$, откуда $u^p-v^p\le
(u-v)^p$.

Таким образом, при $|f(x_1)|\le|f(x_2)|$ верно неравенство
\eqn{|f(x_2)|^p-|f(x_1)|^p\le (|f(x_2)|-|f(x_1)|)^p\le |f(x_2)-f(x_1)|^p.}

Пусть теперь у нас есть разбиение диаметра $\la$ такое, что $$ \sumiun \om_i(f)\De_i<\ep.$$

Тогда по доказанному выше и неравенству Йенсена,
$$\sum_{i=1}^n\om_i(|f|^p)\De_i\le \sum_{i=1}^n(\om_i(f))^p\De_i\le (\sum_{i=1}^n\om_i(f)\De_i)^p<\ep^p.$$
Осталось воспользоваться соответствующим признаком существования интеграла.
\end{solution}

\begin{problem}
Доказать, что если $\Vb_f[a,b] < \infty$ и $\hm{f(x)} \ge \de > 0$ при $x \in [a,b]$, то и $\Vb_{1/f}[a,b]
\le \infty$.
\end{problem}
\begin{solution}
Так как $f\ge\de>0$, то
$$\hm{\frac{1}{f(x_1)}-\frac{1}{f(x_2)}}=\hm{\frac{f(x_2)-f(x_1)}{f(x_1)f(x_2)}} \le \frac{|f(x_2)-f(x_1)|}{\delta^2}.$$
Следовательно, $V_{1/f}[a,b]<V_f[a,b]\delta^{-2}.$
\end{solution}

\begin{problem}
Пусть $f, g \in \Cb[a, +\infty)$, $f, g >0$ и $f,g$ возрастают. Пусть $f(x) = o(g(x)), \; x \ra \infty$.
Тогда \eqn{\intl{a}{x}f(t)\,dt = o\hr{\intl{a}{x}g(t)\,dt}, \; x \ra \infty.}
\end{problem}
\begin{solution}
Достаточно применить один раз правило Лопиталя.
\end{solution}

\begin{problem}
Доказать, что \eqn{\hm{\intl{x}{2x}\cos\hr{t^3}\,dt} \le \min\hr{x, \frac{2}{3x^2}}.}
\end{problem}

\begin{problem}
Доказать, что \eqn{\intl{1}{x^2}\frac{e^t}{t}\,dt \sim \frac{e^{x^2}}{x^2}, \; x \ra \infty.}
\end{problem}
\begin{solution}
По правилу Лопиталя, достаточно доказать, что
$$\frac{d}{dx}\intl{1}{x^2}\frac{e^t}{t}dt=\frac{d}{dx}\frac{1}{x^2}e^{x^2},$$
что проверяется прямым подсчётом:
$$\frac{d}{dx}\intl{1}{x^2}\frac{e^t}{t}dt=\frac{e^{x^2}}{x^2}\cdot 2x=
\frac{2e^{x^2}}x, \quad \frac{d}{dx}\frac{1}{x^2}e^{x^2}=\frac{2xe^{x^2}}{x^4}.$$
\hfill\end{solution}
\begin{problem}
Доказать, что \eqn{\intl{x}{x+1}\sin\hr{t^2}\,dt = \frac{\cos\hr{x^2}}{2x} - \frac{\cos\hr{x+1}^2}{2(x+1)} +
O\hr{\frac{1}{x^2}}.}
\end{problem}

\begin{problem}
Придумать функцию $f \in \Cb[a; +\infty)$, такую, что $\intl{a}{\infty}f(x)\,dx$ сходится, $f > 0$ и $f(x_k)
\ra +\infty$ для некоторой последовательности $x_k \ra +\infty$.
\end{problem}
\begin{solution}
Построим <<пилу>> c достаточно узкими, но высокими <<зубьями>>. Пусть $f(x)=8^n(x-n+4^{-n})$, если
$x\in [n-4^{-n},n]$; $f(x)=8^n(n+4^{-n}-x)$, если $x\in [n,n+4^{-n}]$; $f(x)=0$ в остальных случаях. Тогда
$F(x)=e^{-x}+f(x)$~--- искомая функция.
\end{solution}

\pagebreak

\section{3 семестр}

\subsection{Программа экзамена}

\begin{nums}{-2}
\item Числовой ряд, сходимость и сумма. Пример: $\sum q^n$. Критерий Коши и простейшие свойства. Необходимое
условие сходимости. Теоремы сравнения для знакоположительных рядов.
\item Эквивалентное определение $\ol{\lim}$ и $\ul{\lim}$. Признаки Коши и Даламбера.  Интегральный
признак, оценка остатка ряда.
\item Признак Раабе. Знакоположительные ряды с монотонными членами: необходимое условие сходимости,
необходимое и достаточное условие сходимости.
\item Знакочередующиеся ряды, признак Лейбница. Преобразование Абеля. Знакопеременные ряды: признаки Дирихле
и Абеля. Пример: $\sum b_n \sin nx, \; b_n \downarrow 0$.
\item Абсолютная сходимость ряда. Сходимость абсолютно сходящегося ряда. Связь между поведением ряда и поведением его
положительных и отрицательных членов. Признаки Даламбера и Коши абсолютной сходимости.
\item Сочетательное свойство сходящихся рядов. Перестановки рядов. Переместительное свойство абсолютно
сходящихся рядов. Теорема
Римана о перестановках условно сходящегося ряда.
\item Умножение рядов; Теорема Коши. Суммирование методом Чезаро. ряды с комплексными членами.
Бесконечные произведения, связь
с рядами. Абсолютная сходимость. Теорема Абеля о расходящихся знакоположительных рядах.
\item Двойной ряд. Критерий Коши, необходимое условие сходимости. Условие сходимости двойного знакоположительного ряда.
Повторный ряд. Линейная перестановка двойного ряда. Связь между сходимостью двойного ряда, повторного ряда и линейной
перестановки ($a_{mn} \ge 0$). То же самое для абсолютной сходимости. Изменение порядка суммирования.
\item Функциональные последовательности и ряды. Сходимость, равномерная сходимость, признаки неравномерной сходимости.
Критерий Коши, признаки Вейерштрасса, Дирихле и Абеля равномерной сходимости.
\item Признак Дини равномерной сходимости. Равномерная сходимость и предельный переход.  Непрерывность суммы
функционального ряда.
\item Почленное интегрирование и дифференцирование функциональных рядов.
\item Степенные ряды. Радиус и интервал сходимости. Свойства. Теорема Абеля. Пример $\sum \frac{(-1)^n}{n}$.
Суммирование методом
Абеля. Степенные ряды в $\Cbb$, радиус и круг сходимости.
\item Единственность разложения функции в степенной ряд. Ряд Тейлора. Критерий разложимости в ряд Тейлора,
достаточное условие.
Элементарные функции в $\Cbb$.
\item Равномерная сходимость функции 2 переменных по базе. Сведение к функциональной последовательности, критерий Коши.
Свойства равномерной сходимости (непрерывность предельной функции, аналог теоремы Дини, переход к пределу под
знаком интеграла).
\item Собственные интегралы, зависящие от параметра. Непрерывность интеграла по параметру, дифференцирование и
интегрирование по параметру под знаком интеграла.
\item Более сложная зависимость собственного интеграла от параметра, непрерывность и дифференцируемость.
\item Несобственные интегралы с параметром. Равномерная сходимость, критерий Коши, признаки равномерной сходимости
Вейерштрасса, Абеля, Дирихле, Дини.
\item Предельный переход под знаком несобственного интеграла. Непрерывность по параметру.
\item Интегрирование несобственного интеграла по параметру (случай отрезка и полупрямой). Дифференцирование по параметру.
\item Гамма функция. Сходимость соответствующего произведения. Функциональное соотношение для $\Ga(s)$. Формула
дополнения. Интегральное представление $\Ga(s)$.
\item $\Ga$ функция и $\Be$ функция. Исследование $\Ga(s), s > 0$. Связь между $\Ga$ и $\Be$. Примеры:
\eqn{\intl{0}{1}x^{p-1}(1-x^m)^{q-1} \, dx, \; \intl{0}{\pi/2}\sin^{a-1}t \cos^{b-1}t \,dt.}
\item Ортогональная система непрерывных функций на отрезке. Тригонометрическая система, её ортогональность на отрезке
длины $2\pi$. Единственность разложения функции в тригонометрический ряд. Ряд Фурье. Оценки коэффициентов Фурье.
Простейшие результаты о сходимости ряда Фурье. Разложение $\sin$ в бесконечное произведение.
\item Ядра Дирихле и Фейера. Теорема Фейера и её следствия.
\item Аппроксимация в нормированных пространствах. Теоремы Вейерштрасса о плотности тригонометрических
многочленов в $\Cb_{2\pi}$ и о плотности алгебраических многочленов в $\Cb[a,b]$.  Плотность
тригонометрических многочленов в $\Rb^2[-\pi,\pi]$.
\item ОНС в предгильбертовом пространстве. Ряд Фурье. Наилучшее приближение. Минимальное свойство коэффициентов Фурье.
Неравенство Бесселя. Стремление к 0 коэффициентов Фурье. Условия разложимости произвольного элемента в ряд
Фурье. Критерий базиса для ОНС.
\item Тригонометрический ряд Фурье, сходимость в среднем. Равенство Парсеваля. Ряд Фурье по синусам и
косинусам на $(0, \pi)$. Пространство $\Rb^2_c[-\pi,\pi]$. Ряд Фурье в комплексной форме.
\item Принцип локализации и признак Дини. Признак Жордана.
\item Ряд Фурье в интервале произвольной длины. Интеграл Фурье. Преобразование Фурье и его свойства. Обращение
преобразования Фурье. Аналог признака Жордана (без доказательства). Синус- и косинус-преобразования Фурье.
\item Операции над рядами Фурье. Почленное интегрирование и дифференцирование, свёртка функций и её ряд Фурье.
\end{nums}

\subsection{Вопросы и задачи коллоквиума}
Теоретическая часть совпадает с вопросами 1 -- 13 экзаменационной программы. Задачи:

\begin{problem}
Пусть $a_n, b_n > 0$ и $a_n=o(b_n), n \ra \infty$. Тогда $\sumui b_n < +\infty \Ra \sumui a_n < + \infty$.
\end{problem}

\begin{proof}
Так как $a_n=o(b_n)$, то $\exi N_0:  \frac{a_n}{b_n} < 1, n > N_0$. Значит, для достаточно больших~$n$
имеем $a_n < b_n \Ra$ по теореме сравнения из сходимости $\sumui b_n$ следует сходимость $\sumui a_n$.
\end{proof}

\begin{problem}
$a_n \rightarrow 0, n \ra \infty \Ra \exi \hc{n_k}\cln \suml{k=1}{\infty}
\hm{a_{n_k}} < \frac{1}{2}$.
\end{problem}

\begin{proof}
Так как $a_n \ra 0$, то  можно выделить подпоследовательность $a_{n_k}$ такую, что $a_{n_k} <
\frac{1}{2^{k+2}}$. Имеем \eqn{\sumui a_{n_k} < \sumui \frac{1}{2^{k+2}} =
\frac{1/4}{1-\frac{1}{2}} = \frac{1}{2}.}
\hfill\end{proof}

\begin{problem}
Построить сходящийся знакоположительный ряд
$\sumui a_n$, для которого $\uliml{n \ra \infty} \hr{\frac{a_{n+1}}{a_n}} > 1$.
\end{problem}

\begin{proof}
Положим $a_{n} = \frac{1}{2^n}$, если $n$ чётное, и $a_{n} = \frac{\ln n}{2^n}$, если $n$ нечётное. Очевидно, что
ряд сходится, и
$$
  \uliml{n \ra \infty} \hr{\frac{a_{2n+1}}{a_{2n}}} = \frac{1}{2}\ln (2n+1)= +\infty.
$$
\hfill\end{proof}

\begin{problem}
При $\al > 0$ биномиальный ряд сходится в точках $\pm 1$.
\end{problem}

\begin{proof}
При $\al \in \N$ в биномиальном ряде конечное число ненулевых членов и, следовательно, он сходится. Далее
будем считать $\al\notin \N$. Тогда в точке $1$ получим ряд $\suml{k=0}{\infty}\binom{\al}k$. При $k>\al$ этот ряд~---
знакочередующийся, причём $\frac{\hm{a_{k+1}}}{\hm{a_k}}=\frac{k-\al}{k+1}<1$, а следовательно, сходится.

В точке $-1$ получаем ряд $\suml{k=0}{\infty}(-1)^k\binom{\al}k$. При $k>\al$ этот
ряд~--- знакопостоянный. По признаку Гаусса
$$\frac{a_k}{a_{k+1}}=\frac{k+1}{k-\al}=1+\frac{\al+1}{k-\al}=1+\frac{\al+1}k+\frac{\gamma_k}{k^2},$$
где $\ga_k$~--- ограниченная последовательность. Так как $\al+1>1$, то рассматриваемый ряд сходится.
\end{proof}

В последующих задачах потребуется следующее утверждение:
\begin{stm}\label{stat}
Пусть функция~$f(x)$ невозрастает.
Тогда $$\intl{n}{n+m} f(x)\,dx \ge \suml{k=n+1}{n+m}f(k)\ge \intl{n+1}{n+m+1} f(x)\,dx.$$
\end{stm}

Формальное его доказательство было в курсе. Вспомним, что интеграл~--- <<площадь под графиком>>, и оно становится очевидным.

\begin{imp}\label{a}
Пусть функция~$f(x)$ невозрастает и $\liml{n \ra \infty}f(n)=0$.  Тогда существует конечный предел
$$\liml{n\ra\infty}\hr{\intl{0}{n}f(x)\,dx-\sumkun f(k)}.$$
\end{imp}

\begin{imp}\label{b}
Пусть функция~$f(x)$ невозрастает интеграл $\int f(x)\,dx$ сходится. Тогда
$$\intl{n}{\infty} f(x)\,dx \ge \suml{k=n+1}{\infty}f(k)\ge \intl{n+1}{\infty} f(x)\,dx.$$
\end{imp}

\begin{problem}\label{c}
Доказать оценку остатка обобщённого гармонического ряда:
$$\frac1{\al-1}\frac1{(n+1)^{\al-1}} \le \suml{k=n+1}{\infty}\frac1{k^\al}\bw \le \frac1{\al-1}\frac1{n^{\al-1}}\qquad \mbox{при $\al>0$.}$$
\end{problem}

\begin{proof}
Эта задача является частным случаем следствия~\ref{b}.
\end{proof}

\begin{problem}
Доказать оценку частичной суммы обобщённого гармонического ряда при $\al \in (0,1)$:
$$\frac{1}{1-\al}\left((n+1)^{1-\al}-1\right)< \sumkun \frac{1}{k^\al} < \frac{1}{1-\al}n^{1-\al}.$$
\end{problem}

\begin{proof}
Эта задача является частным случаем утверждения~\ref{stat}.
\end{proof}

\begin{problem}
Доказать оценку частичной суммы гармонического ряда: $\ln(n+1) < \sumkun \frac{1}{k} < \ln{n}+1$.
\end{problem}

\begin{proof}
Первое из неравенств этой задачи сразу следует из утверждения~\ref{stat} (подставим в утверждение $n=1$). Второе
следует из утверждения~\ref{stat} после вычитания~1 из обеих частей.
\end{proof}

\begin{problem}
Доказать сходимость последовательности $\sumkun \frac{1}{k}- \ln{n}$.
\end{problem}

\begin{proof}
Частный случай следствия~\ref{a}.
\end{proof}

\begin{problem}
Пусть $a_n>0$ и $\sumui a_n$ сходится. Тогда
найдутся $0< b_1 \le b_2 \le \ldots \le b_n \le \ldots$ такие,
что $b_n \ra \infty$ и $\sumui a_nb_n<\infty$.
\end{problem}

\begin{proof}
Пусть $c_n=\suml{n+1}{\infty} a_k$. Тогда
достаточно положить $b_n=\frac{1}{\sqrt{c_{n-1}}}$. Очевидно, что
$b_n$ возрастает и стремится к бесконечности.
Докажем, что ряд $a_nb_n$ сходится. Действительно,

$$2(\sqrt{c_{n-1}}-\sqrt{c_n})=2\cdot\frac{c_{n-1}-c_n}{\sqrt{c_{n-1}}+\sqrt{c_n}}>\frac{a_n}{\sqrt{c_{n-1}}}=a_nb_n.$$

Следовательно, по оценочному признаку сходимости, ряд $a_nb_n$ сходится.
\end{proof}

\begin{problem}\label{cosnx}
Доказать, что $\suml{n=1}{\infty}\cos nx$ расходится при всех действительных $x$.
\end{problem}

\begin{proof}
Докажем, что последовательность $d_n=\cos nx$ не стремится к нулю ни при каком действительном~$x$.  Предположим,
что для какого-то числа~$x$ выполнено ${\lim\limits_{n\ra\infty}d_n=0}$. Переходя к пределу в
равенстве $\cos(n+1)x \bw= \cos nx\cos x-\sin nx\sin x$, получим ${\lim\limits_{n\ra\infty} \sin nx\sin x=0}$.
Заметим далее, что $|\sin nx|=\sqrt{1-\cos^2 nx}\ra 1$ при $n\ra\infty$.
Следовательно, $\sin x = 0$, то~есть $x=2\pi k$ или $x=\pi+2\pi k$.  В каждом из этих случаев $|\cos nx|=1$.

Итак, ни при каком действительном~$x$ последовательность~$\cos nx$ не является бесконечно малой. Значит, для ряда из
условия задачи не выполнено необходимое условие сходимости, поэтому он расходится.
\end{proof}

\begin{problem}
Доказать, что $\suml{n=1}{\infty}\sin nx$ расходится при $x \neq k\pi$.
\end{problem}

\begin{proof}
Аналогично задаче~\ref{cosnx}.
\end{proof}

\begin{problem}
Исследовать на сходимость двойной ряд $\suml{m=2,n=1}{\infty}\frac{1}{m^n}$.
\end{problem}

\begin{proof}
Этот ряд содержит в качестве подряда (при $n=1$) гармонический ряд, а значит расходится.
\end{proof}

\begin{problem}
Найти сумму двойного ряда $\suml{\substack{m=2\\n=2}}{\infty}\frac{1}{m^n}$.
\end{problem}

\begin{proof}
$$\sum_{\substack{m=2\\n=2}}^{\infty}\frac1{m^n}= \sum_{m=2}^{\infty}\sum_{n=2}^{\infty}\frac1{m^n}=
\sum_{m=2}^{\infty}\frac{1}{m^2-m}= \sum_{m=2}^{\infty}\frac1{m-1}-\frac1m=1.$$
\hfill
\end{proof}

\begin{problem}
Доказать, что ряд $\suml{n=1}{\infty}\frac{\sqrt{x}\ln n}{1+n^3x}$ сходится равномерно на $[0, +\infty)$.
\end{problem}

\begin{proof}
По неравенству о средних, $1+n^3x\ge 2\sqrt{n^3x}$. Следовательно, $\frac{\sqrt{x}\ln n}{1+n^3x}\le\frac{\ln n}{2n^{3/2}}$, т.~е.
исходный ряд не превосходит сходящегося ряда с постоянными членами, а значит равномерно сходится.
\end{proof}

\begin{problem}
Ряд $\suml{n=1}{\infty}\frac{x^n}{n}$ сходится неравномерно на $(-1,1)$.
\end{problem}

\begin{proof}
Сходимость ряда следует, например, из того, что его радиус сходимости равен единице. Докажем, что сходимость
неравномерная. Для этого достаточно заметить, например, что сумма ряда равна $-\ln(1-x)-x$ и неограниченна на $(-1,1)$, а равномерный
предел ограниченных функций ограничен.
\end{proof}

\begin{problem} Ряд $\suml{n=1}{\infty}n^{-x}$ сходится
равномерно на $[1+\ep,+\infty)$ для $\fa \ep>0$ и неравномерно
на $(1, +\infty)$.  \end{problem}

\begin{proof}
Докажем сначала, что этот ряд сходится равномерно на $[1+\ep,\infty)$. Действительно, при таких $x$ ряд оценивается
сверху сходящимся рядом с постоянными членами, а именно $\suml{n=1}{\infty}n^{-(1+\ep)}$, а значит равномерно сходится.

Теперь докажем, что этот ряд сходится неравномерно на $(1,+\infty)$. Допустим, он сходится равномерно, е
$$\fa \ep>0\; \exi n_0\cln \fa n\ge n_0,\; \fa x\in (1,+\infty) \text{ имеем } r_n(x)=\sum_{k=n+1}^{\infty}n^{-x}<\ep.$$

По задаче~\ref{c}, $r_n(x)\ge\frac1{x-1}\frac1{(n+1)^{x-1}}$. Следовательно, подставляя $\ep=1, n=n_0$, получим, что
$\exi n_0$ такое, что для $\fa x>1$ имеем
\eqn{\frac1{x-1}\cdot\frac1{(n_0+1)^{x-1}}<1,}
откуда $(x-1)(n_0+1)^{x-1}>1$. Так как $(n_0+1)^{x-1}<n_0+1$ при $x\bw <2$, то при всех $1\bw<x\bw<2$ выполнено неравенство
$(x-1)(n_0+1)\bw >1$, что невозможно (например, в пределе при $x\ra 1$ получаем $0\ge1$).
\end{proof}

\pagebreak

\section{4 семестр}

\subsection{Программа экзамена}

\begin{nums}{-2}
\item Двойной интеграл по прямоугольнику. Критерии интегрируемости.
\item Двойной интеграл по измеримому множеству. Критерии интегрируемости. Эквивалентность двух понятий интегрируемости.
\item Измеримые по Жордану множества в $\R^3$. Критерий измеримости цилиндрического бруса.
\item Свойства двойного интеграла и его сведение к повторному.
\item Кратные интегралы. Сведение тройного интеграла к повторному. Объём $n$ мерного симплекса.
\item Леммы об отображениях класса $\Cb^1$.
\item Криволинейные координаты. Объём в криволинейных координатах. Замена переменных в кратном интеграле.
\item Множества лебеговой меры нуль в $\R^n$. Критерий Лебега интегрируемости функции по Риману на параллелепипеде.
\item Несобственные кратные интегралы. Критерий сходимости интеграла от неотрицательной функции. Теорема сравнения.
Эквивалентность понятий сходимости и абсолютной сходимости в $\R^n$ при $n \ge 2$.
\item Площадь поверхности.
\item Задача о вычислении массы материальной кривой. Криволинейный интеграл первого рода, его сведение к определённому.
Задача о вычислении работы силового поля. Криволинейный интеграл второго рода и его сведение к определённому.
\item Связь между интегралами первого и второго рода. Формула Грина.
\item Поверхностный интеграл второго рода, его сведение к двойному.
\item Ориентация поверхности. задача о вычислении количества жидкости, протекающего за единицу времени через
ориентированную поверхность. Интеграл по ориентированной поверхности, его вычисление.
\item Ориентированные кусочно гладкие поверхности. Формула Гаусса Остроградского. Вторая формула Грина.
\item Гармонические функции двух переменных. Ядро Пуассона. Интеграл Пуассона. Представление гармонической функции интегралом Пуассона.
Гармонические функции на плоскости.
\item Формула Стокса.
\item Условия независимости криволинейного интеграла второго рода от пути интегрирования. Скалярные и векторные поля.
Векторные линии и векторные трубки. Дивергенция векторного поля. Инвариантное определение дивергенции и его
физический смысл. Соленоидальное поле.
\item Ротор векторного поля. Инвариантное определение, физический смысл. Примеры. Потенциальное поле. Оператор
Гамильтона $\nabla$.
\item Дифференциальные операции теории поля второго порядка.
\item Ортогональные криволинейные координаты. Коэффициенты Ламе. Выражение градиента, дивергенции и оператора Лапласа в ортогональных
криволинейных координатах.
\item $k$-мерные кусочно-гладкие поверхности в $\R^n$. Согласование ориентации поверхности и её границы.
Дифференциальные формы. Замена переменных. Интеграл от дифференциальной формы.
\item Внешний дифференциал дифференциальной формы. Общая формула Стокса (без доказательства).
\item Метод Лапласа асимптотической оценки интегралов. Примеры: формула Стирлинга, асимптотика бесселевой
функции целого порядка.
\end{nums}

\medskip\dmvntrail
\end{document}
