\documentclass[a4paper]{article}
\usepackage[utf8]{inputenc}
\usepackage[russian]{babel}
\usepackage{dmvn}

\tocsubsectionparam{2.6em}

\newcommand{\sap}[2]{a_{{#1}}^+ \spl a_{{#2}}^+}
\newcommand{\sam}[2]{a_{{#1}}^- \spl a_{{#2}}^-}

\newcommand{\tpm}{\hr{t + \frac{\pi}{\mu}}}
\newcommand{\xtpm}{\hr{x + t + \frac{\pi}{\mu}}}

\newcommand{\intlab}{\intl{a}{b}}
\newcommand{\intlop}{\intl{0}{\pi}}
\newcommand{\intlox}{\intl{0}{x}}
\newcommand{\intlpp}{\intl{-\pi}{\pi}}
\newcommand{\intlii}{\intl{-\infty}{+\infty}}
\newcommand{\intloi}{\intl{0}{+\infty}}
\newcommand{\frpi}{\frac{1}{\pi}}

\newcommand{\bsbs}[2]{\biggl|_{{#1}}^{{#2}}}
\newcommand{\sbs}[2]{\bigl|_{{#1}}^{{#2}}}

\newcommand{\dx}{\De x}
\newcommand{\du}{\De u}
\newcommand{\Dv}{\De v}

\begin{document}
\dmvntitle{Курс лекций по}{математическому анализу}{Лектор Сергей Александрович Теляковский}
{II курс, 3 семестр, поток математиков}{Москва, 2004 г.}

\pagebreak
\tableofcontents
\pagebreak

\section*{Введение}

\subsection*{Предисловие}
Хотя это и не первое издание данного документа, ошибки, опечатки и неточности здесь, возможно, ещё
есть. Поэтому пишите на \dmvnmail{} о замеченных ошибках. Обновления документа
смотрите на сайте \dmvnwebsite{}. Последнее обновление: \today~года.

\subsection*{Слова благодарности}

Хочется отметить, что этот документ содержал бы очень много лажи,
если бы не эти люди: Сергей Захаров, Григорий Мерзон, Дмитрий Антонов, Дмитрий Котенко, Рудой Николай,
Дмитрий Колосов, Смирнов Ярослав, Панфёров Фёдор, Коробейник Роман.

\subsection*{Принятые в тексте соглашения и используемые сокращения}

\begin{points}{-3}
\item Аббревиатуры: <<КК>> = <<Критерий Коши>>, <<ОЛР>> = <<обобщённая лемма
      Римана>>, <<ФКПЛ>> = <<формула конечных приращений Лагранжа>>, <<ФНЛ>> = <<формула Ньютона
      Лейбница>>, <<НКБ>> = <<неравенство Коши Буняковского>>, <<ОНС>> = <<ортонормированная система>>,
      <<ЧПФ>> = <<частное преобразование Фурье>>, <<ОЧФЛ>> = <<формула Тейлора с остаточным членом в
      форме Лагранжа>>.
\item $\sum a_k \sim \sum b_k$ ряды сходятся или расходятся одновременно. Это обозначение не является
      стандартным и использовано только ради краткости записи. Знак $\sim$ также
      используется для обозначения ряда Фурье некоторой функции:
      $f(x) \sim \frac{a_0}{2} + \sum \br{a_k\cos kx + b_k\sin kx}$.
\item \begin{items}{-3}
      \item $f \in\Cb(x)$ функция $f$ непрерывна в точке $x$;
      \item $f \in\Cb(\Xc)$ $f$ непрерывна на множестве $\Xc$;
      \item $f \in\Db(\Xc)$ $f$ дифференцируема на множестве $\Xc$;
      \item $f \in\Rb[a,b]$ $f$ интегрируема по Риману на $[a,b]$;
      \item $f\in\SegC^n(\Xc)$ $f$ имеет $n-1$ непрерывную производную и кусочно-непрерывную~$n$-ю
            производную на $\Xc$.
      \end{items}
\item $\overline{\R}$ расширенная числовая прямая $\R \cup \hc{-\infty, +\infty}$.
\item $\Dc(x)$ функция Дирихле.
\item Если в обозначении суммы или произведения не указаны верхние и нижние пределы, чаще всего предполагается,
      что верхний предел равен $+\infty$, а нижний единице. Нижний предел иногда может быть нулём,
      но это всегда ясно из контекста. Других умолчаний для индексов не используется.
\item Ряды и функции, обозначенные буквами $a,b,\dots$, как правило, будем считать вещественнозначными,
      а обозначенные буквами $u,v,\dots$ комплекснозначными.
\item Там, где это не критично, скобки в повторных интегралах будут опущены.
      Выражение $\intlab \intl{c}{d}f(x,y)\,d x \,d y$ это не кратный интеграл, а повторный:
      $\intlab \bbr{\intl{c}{d}f(x,y)\,d x}\,d y$.
\item $f\cln \Xc$ функция вида $f\cln\Xc\ra\R$, иначе говоря, функции без указания области значений
      предполагаются вещественнозначными. Там, где это не критично,
      мы будем опускать аргументы функций. Если у подинтегральной функции не указан аргумент,
      то имеется в виду переменная, по которой производится интегрирование.
\item Запись $A \cdot E$ обозначает скалярное произведение векторов строк $A = (a_1\sco a_m)$ и $E = (e_1\sco e_m)$.
      Иначе говоря, $A\cdot E = \suml{k=1}{m} a_k e_k$.
\end{points}

\begin{thebibliography}{5}
    \setlength\itemsep{-3pt}
    \bibitem{lectures:telyakovsky} М.\,Н.\,Вельтищев. \emph{Конспекты лекций С.\,А.\,Теляковского
                                   по математическому анализу.} 2003.
    \bibitem{lectures:sedletsky}   Д.\,Н.\,Вельтищев. \emph{Конспекты лекций А.\,М.\,Седлецкого
                                   по математическому анализу.} 2003.
    \bibitem{zorich}               В.\,А.\,Зорич. \emph{Математический анализ.} М., МЦНМО, 2002.
    \bibitem{rudin}                W.\,Rudin. \emph{Основы математического анализа.} М., МИР, 1976.
    \bibitem{fiht}                 Г.\,М.\,Фихтенгольц. \emph{Основы математического анализа.} М., ГИФМЛ, 1960.
\end{thebibliography}

\pagebreak

\section{Числовые ряды}

\subsection{Ряды с неотрицательными членами, критерий сходимости. Признаки сравнения}

\begin{df}
\emph{Числовым рядом} называется формальная запись вида $\suml{k=1}{\infty} u_k$, где $u_k \in \Cbb$. Положим
$S_n\bw{:=}\suml{1}{n} u_k$. Если $\exi \liml{n} S_n = S \in \Cbb$, то говорят, что ряд \emph{сходится к} $S$,
иначе \emph{расходится}. Для сходящихся рядов
определены \emph{остатки} $R_n := \suml{n+1}{\infty} a_k,\; n=0,1,2,\dots$.
\end{df}

\begin{theorem}[КК]
Ряд $\sum u_k$ сходится $\Lra \fa \ep > 0 \exi N\cln \fa N \le m \le n$ имеем
$\hm{\suml{m}{n} u_k} < \ep$.
\end{theorem}
\begin{proof}
Следует из КК для последовательности частичных сумм ряда.
\end{proof}

\begin{theorem}[Необходимое условие сходимости]
Пусть ряд $\sum u_k$ сходится. Тогда $u_k \ra 0$.
\end{theorem}
\begin{proof}
Положим в КК $m=n$. Отсюда следует утверждение теоремы.
\end{proof}

\begin{theorem}
Пусть $\sum u_k$ и $\sum v_k$ сходятся. Тогда $\fa a,b \in \Cbb$ сходится $\sum(au_k+bv_k)$.
\end{theorem}
\begin{proof}
Для частичных сумм имеем $\suml{1}{n}(au_k+bv_k)=a\suml{1}{n}u_k + b\suml{1}{n}v_k$. Остаётся перейти к пределу.
\end{proof}

\begin{note}
Если ряд $\sum u_k$ сходится, а ряд $\sum (u_k+v_k)$ расходится, то $\sum v_k$ расходится.
\end{note}

\begin{theorem}[Сумма геометрической прогрессии]
Пусть $z \in \Cbb$ и $|z|<1$. Тогда $\sum Az^k = \frac{A}{1-z}$.
\end{theorem}
\begin{proof}
Имеем $\suml{0}{n} Az^k = \frac{A-Az^{n+1}}{1-z}=\frac{A}{1-z}-A\frac{z^{n+1}}{1-z}$. Остаётся перейти к пределу при $n\ra\infty$.
\end{proof}

\begin{theorem}
Пусть $a_k \ge 0$. Тогда сходимость ряда $\sum a_k$ равносильна ограниченности частичных сумм.
\end{theorem}
\begin{proof}
В самом деле, $S_n$ возрастающая последовательность.
\end{proof}

\begin{note}
В этом и следующих разделах будут рассмотрены некоторые признаки сходимости рядов. Очень часто можно требовать выполнения некоторых условий
для всех членов ряда, начиная с некоторого. Например, в только что доказанной теореме таково требование $a_k \ge 0$. В дальнейшем мы не будем
обращать на это внимание.
\end{note}

\begin{theorem}[I-й признак сравнения]
Пусть $|u_k| < ca_k$, где $c>0$. Если $\sum a_k$ сходится, то и $\sum u_k$ сходится. Если $\sum u_k$
расходится, то и $\sum a_k$ расходится.
\end{theorem}
\begin{proof}
Применим КК: $\fa \ep > 0 \exi N\cln \fa N \le m \le n$ имеем $\suml{m}{n} a_k < \frac{\ep}{c}$. Тогда
$\hm{\suml{m}{n} u_k} \le \suml{m}{n} |u_k| \le \suml{m}{n}ca_k < \ep$. Отсюда по КК следует сходимость. Вторая часть теоремы вытекает из первой.
\end{proof}

\begin{theorem}[II-й признак сравнения]
Пусть $\sum a_k$, $\sum b_k$ положительные ряды. Если $0 \bw< c \bw\le \frac{a_k}{b_k} \bw\le d \bw< \infty$, то $\sum a_k\sim\sum b_k$.
\end{theorem}
\begin{proof}
Применим предыдущую теорему в обе стороны.
\end{proof}

\begin{theorem}[III-й признак сравнения]
Пусть $a_k>0, b_k>0$. Если $\frac{a_{k+1}}{a_k} \le \frac{b_{k+1}}{b_k}$, то если $\sum b_k$ сходится, то и
$\sum a_k$ сходится, а если $\sum a_k$ расходится, то и $\sum b_k$ расходится.
\end{theorem}
\begin{proof}
Имеем $\frac{a_n}{a_1} = \prodl{1}{n-1} \frac{a_{k+1}}{a_k} \le \prodl{1}{n-1} \frac{b_{k+1}}{b_k} = \frac{b_n}{b_1}$. Осталось применить
I-й признак сравнения.
\end{proof}

\subsection{Признаки Даламбера и Коши сходимости рядов. Сравнение этих признаков}

\begin{theorem}[Признак Даламбера]
Пусть $\sum a_k$ положительный ряд. Если $\exi q < 1\cln \frac{a_{k+1}}{a_k} \le q$, то ряд сходится.
Если $\frac{a_{k+1}}{a_k} \ge 1$, то ряд расходится.
\end{theorem}
\begin{proof}
В первом случае рассмотрим $b_k := q^k$, отсюда $\frac{a_{k+1}}{a_k} \le q = \frac{b_{k+1}}{b_k}$.
Но $\sum b_k$ сходится. Тогда по признаку сравнения наш ряд сходится. Во втором случае имеем
$a_{k+1} \ge a_k > 0$, поэтому $a_k \nra 0$, и ряд расходится.
\end{proof}

\begin{theorem}[Предельный вариант признака Даламбера]
Пусть $a_k > 0, \lim \frac{a_{k+1}}{a_k}:=q$. Тогда при $q < 1$ ряд сходится, при $q > 1$ расходится. Если $q=1$, то ничего сказать нельзя.
\end{theorem}
\begin{proof}
Тривиально выводится из признака Даламбера. Пример для $q=1$: рассмотрим ряды $\sum \frac{1}{k}$ и $\sum \frac{1}{k^2}$.
\end{proof}

\begin{theorem}[Радикальный признак Коши]
Пусть $a_k \ge 0$. Если $\exi q < 1\cln \sqrt[k]{a_k} \le q$, то $\sum a_k$ сходится.
Если же $\sqrt[k]{a_k} \ge 1$, то ряд расходится.
\end{theorem}
\begin{proof}
В первом случае имеем $a_k \le q^k$, а $\sum q^k$ сходится. Тогда по признаку сравнения сходится и наш ряд. Во
втором случае имеем $a_k \ge 1$, значит, члены ряда не стремятся к 0, и он расходится.
\end{proof}

\begin{theorem}[Предельный вариант признака Коши]
Пусть $\lim \sqrt[k]{a_k} = q$. Тогда при $q < 1$ ряд сходится, при $q > 1$ расходится, при $q=1$ ничего сказать нельзя.
\end{theorem}
\begin{proof}
Тривиально выводится из признака Коши. Пример для $q=1$: рассмотрим $\sum\frac{1}{k}$ и $\sum\frac{1}{k\ln^2 k}$.
\end{proof}

\begin{theorem}
Если выполнены условия признака Даламбера, то выполнены условия признака Коши, но не наоборот.
\end{theorem}
\begin{proof}
Пусть выполнено $\frac{a_{k+1}}{a_k} \le q < 1$.
Рассмотрим $\prodl{1}{n-1} \frac{a_{k+1}}{a_k} = \frac{a_n}{a_1}$. Поскольку каждый множитель не превосходит $q$, всё произведение не превосходит
$q^{n-1}$, отсюда $a_n \le \frac{a_1}{q}q^n$. Положим $C:=\frac{a_1}{q}$, тогда имеем $a_n \le Cq^n$, и
отсюда $\sqrt[n]{a_n} \le q \sqrt[n]{C}$. Так как $\sqrt[n]{C}\ra 1$, то найдётся $p$ такое, что для всех достаточно больших
$n$ выполнено $q \sqrt[n]{C} < p < 1$. Поэтому выполнены условия признака Коши.

Предъявим теперь ряд, для которого выполнены условия признака Коши, но не выполнены условия признака Даламбера. Рассмотрим ряд $\sum a_k$, где
$a_k:=\frac{1}{2^k}$, если $k$ чётно, и $a_k := \frac{k}{2^k}$, если $k$ нечётно. Легко видеть, что $\sqrt[k]{a_k} \ra \frac{1}{2}$, но
при чётных $k$ имеем $\frac{a_{k+1}}{a_k}=\frac{k+1}{2} \ra \infty$.
\end{proof}

\subsection{Интегральный признак сходимости. Постоянная Эйлера}

\begin{theorem}[Интегральный признак Коши]
Пусть $f(x)>0$ и монотонно убывает на $[1, +\infty)$. Тогда ряд $\sum f(k)$ сходится или расходится
одновременно с интегралом $\intl{1}{+\infty} f\,d x$.
\end{theorem}
\begin{proof}
Имеем $f(k+1) \le \intl{k}{k+1}f\,d x \le f(k)$ в силу монотонного убывания $f$. Тогда
$\suml{1}{n} f(k+1) \le \intl{1}{n+1}f\,d x \le \suml{1}{n}f(k)$. В силу того, что $f > 0$,
сходимость интеграла равносильна его ограниченности. Отсюда, со ссылкой на теорему о пределах,
выводится утверждение теоремы.
\end{proof}

\begin{theorem}[Уточнение интегрального признака]
Пусть $f(x)>0$ и монотонно убывает на $[1, +\infty)$. Тогда существует неотрицательный предел
$\liml{n} \hr{ \suml{1}{n}f(k) - \intl{1}{n+1}f\,d x}$.
\end{theorem}
\begin{proof}
Имеем $0 \le \suml{1}{n}f(k) - \intl{1}{n+1}f\,d x=\suml{1}{n}\intl{k}{k+1}\br{f(k)-f(x)}\,d x
\le \lcomm$ в силу монотонного убывания функции $f$ $\rcomm \le
\suml{1}{n}\intl{k}{k+1}\br{f(k)-f(k+1)}\,d x=f(1)-f(n+1) < f(1)$. Легко видеть, что выражение под
знаком предела возрастает с ростом $n$. Оно неотрицательно и ограничено сверху числом $f(1)$.
Значит, его предел существует и неотрицателен.
\end{proof}

\begin{ex}
Оценим скорость расходимости гармонического ряда. Имеем $\suml{1}{n-1}\frac{1}{k+1} \le
\intl{1}{n}\frac{\,d x}{x} \le \suml{1}{n-1} \frac{1}{k}$. Отсюда $\ln n \in
\hs{\suml{1}{n}\frac{1}{k}- 1, \suml{1}{n}\frac{1}{k} - \frac{1}{n}}$. В частности, отсюда следует
представление $\ln n = \suml{1}{n}\frac{1}{k} - 1 + \alpha_n$, где $\alpha_n \in [0,1)$. Из нашего
уточнения следует, что существует неотрицательный предел
$$\liml{n}\bbr{ \suml{1}{n}\frac{1}{k} -
\frac{1}{n} - \ln n} = \liml{n}\bbr{ \suml{1}{n}\frac{1}{k} - \frac{1}{n} - \bbr{
\suml{1}{n}\frac{1}{k} - 1 + \alpha_n}} = \liml{n}\hr{ 1 - \frac{1}{n} - \alpha_n} =
1-\liml{n}\alpha_n =: \gamma.$$ Здесь $\gamma$ \emph{постоянная Эйлера}. Отсюда следует, что
$\ln n + \gamma + o(1) = \suml{1}{n}\frac{1}{k}$.
\end{ex}

\subsection{Ряды с монотонными членами. Теорема Коши. Необходимое условие сходимости}

\begin{theorem}[Необходимое условие сходимости]
Пусть $a_k \downarrow$, ряд $\sum a_k$ сходится. Тогда $a_k = o \hr{ \frac{1}{k}}$.
\end{theorem}
\begin{proof}
Из условия следует, что $a_k \ge 0$. Воспользуемся КК: $\fa \ep > 0 \exi N\cln \fa n, m > N$ имеем $\suml{m}{n} a_k < \frac{\ep}{2}$.
В силу монотонности имеем $\suml{m}{n}a_k \ge \suml{m}{n}{a_n} = (n-m+1)a_n$. Тогда $2(n-m+1)a_n < \ep$. Положим $m=\hs{\frac{n}{2}}$.
Так можно сделать, если $n$ достаточно велико. Тогда $na_n < \ep$, что и требовалось доказать.
\end{proof}

\begin{theorem}[Коши]
Пусть $a_k \downarrow 0$. Тогда ряды $\sum a_k$ и $\sum 2^ma_{2^m}$ сходятся или расходятся одновременно.
\end{theorem}
\begin{proof}
Пусть $f(x) > 0$. Пусть $f \downarrow [1,+\infty)$ и $f(k) = a_k$. Воспользуемся интегральным
признаком Коши. Имеем $\intl{1}{\infty} f\,d x=\suml{0}{\infty}\intl{2^m}{2^{m+1}}f\,d x \ge
\suml{0}{\infty}\intl{2^m}{2^{m+1}}f(2^{m+1})\,d x=
\suml{0}{\infty}f(2^{m+1})2^m=\frac{1}{2}\suml{0}{\infty}f(2^{m+1})2^{m+1}$. Таким образом, из
сходимости интеграла вытекает сходимость ряда. Обратно, пусть сходится ряд
$\suml{0}{\infty}2^ma_{2^m}$. Тогда $\intl{1}{2^n}f\,d x=\suml{0}{n-1}\intl{2^m}{2^{m+1}}f\,d x
\bw\le \suml{0}{n-1}f(2^m)2^m < M$ в силу сходимости ряда.
\end{proof}

\subsection{Признаки Раабе, Гаусса и Куммера сходимости рядов}

\begin{theorem}[Признак Раабе]
Пусть $a_k > 0$. Тогда если $\exi p > 1\cln \frac{a_{k+1}}{a_k} \le 1 - \frac{p}{k}$, то $\sum a_k$
сходится. Если же $\frac{a_{k+1}}{a_k} \ge 1 - \frac{1}{k}$, то ряд расходится.
\end{theorem}
\begin{proof}
Пусть $p>1$ удовлетворяет условию теоремы. Тогда найдём $r\cln 1<r<p$. Положим $b_k := \frac{1}{k^r}$, тогда ряд $\sum b_k$ сходится. Покажем, что
$\frac{a_{k+1}}{a_k} \le \frac{b_{k+1}}{b_k}$, тогда сходимость будет гарантирована. В самом деле, имеем $\frac{b_{k+1}}{b_k} =
\hr{ \frac{k}{k+1}} ^r=\hr{ 1+\frac{1}{k}} ^{-r}=1-\frac{r}{k}+o\hr{\frac{1}{k}}$. Следовательно, наша оценка справедлива, и первый случай доказан.
Во втором случае имеем $\frac{a_{k+1}}{a_k} \ge 1-\frac{1}{k}=\frac{k-1}{k}$. Рассмотрим $b_k :=\frac{1}{k-1}$, тогда имеем
$\frac{a_{k+1}}{a_k} \ge \frac{b_{k+1}}{b_k}$, а ряд $\sum b_k$ расходится, значит, расходится и наш ряд.
\end{proof}

Аналогично признакам Коши и Даламбера, справедлив предельный вариант признака Раабе. Положим $p\bw{:=}\liml{k} k\hr{ 1-\frac{a_{k+1}}{a_k}} $. Если $p > 1$, то
ряд сходится, если $p < 1$, ряд расходится. Если $p=1$, то ничего сказать нельзя. В качестве примера можно рассмотреть ряд $\sum \frac{1}{k(\ln k)^r}$,
сходящийся при $r > 1$.

\begin{note}
Признаки Раабе и Коши несравнимы. В самом деле, ряд $\sum k^{-r}$ можно взять по Раабе, но для него не работает признак Коши.
А если мы возьмём тот ряд, который различает признаки Даламбера и Коши, то пользуясь признаком Раабе, ничего сказать нельзя.
\end{note}

Рассмотрим \emph{скорость сходимости рядов}. Пусть даны сходящиеся ряды $\sum a_k$ и $\sum b_k$. Пусть $R_n$ и $U_n$
соответствующие остатки этих рядов. Если $\frac{R_n}{U_n} \ra 0$, то $\sum a_k$ сходится быстрее, чем $\sum b_k$.
Покажем, что для любого сходящегося положительного ряда существует ряд, который сходится медленнее. Пусть ряд
$\sum a_k$ сходится. Тогда $R_n \downarrow 0$. Рассмотрим $b_n := \sqrt{R_{n-1}} - \sqrt{R_n} \ge 0$. Тогда
остаток $U_n$ для ряда $\sum b_k$ будет равен, как легко видеть, $\sqrt{R_n}$.
Значит, отношение $\frac{R_n}{U_n}=\sqrt{R_n} \ra 0$. Значит, $\sum b_k$ сходится медленнее.

\begin{theorem}[Признак Гаусса]
Пусть дан положительный ряд $\sum a_k$. Если справедливо представление $\frac{a_k}{a_{k+1}}=1+\frac{p}{k} + b_k$,
и ряды $\sum b_k$ и $\sum b_k^2$ сходятся, то при $p > 1$ ряд $\sum a_k$ сходится, а при $p \le 1$ расходится.
\end{theorem}
\begin{proof}
Рассмотрим $\frac{a_1}{a_{n+1}}= \prodl{1}{n} \frac{a_k}{a_{k+1}} = \prodl{1}{n}\hr{ 1+\frac{p}{k} + b_k } =
\exp \hr{ \suml{1}{n} \ln \hr{ 1 + \frac{p}{k} + b_k }} = \lcomm$ разлагая $\ln$ по формуле Тейлора, получаем $\rcomm =
\exp \hr{ \suml{1}{n} \hr{ \frac{p}{k} + r_k }} = \exp \hr{ p\suml{1}{n} \frac{1}{k} + \suml{1}{n} r_k }$, где
$r_k = b_k + \frac{1}{2}\hr{\frac{p^2}{k^2} + \frac{2p}{k}b_k + b_k^2} + s_k$, где $\sum s_k$ сходится абсолютно. Ряд $\sum \frac{p^2}{k^2}$ сходится, ряд $\sum \frac{p}{k}b_k$ сходится по
признаку Дирихле, а ряды $\sum b_k$ и $\sum b_k^2$ сходятся по условию. Отсюда вытекает сходимость ряда $\sum r_k =: A$.
Обозначим $R_n := \suml{1}{n}r_k$.
Теперь покажем, что $a_{n+1} \bw\sim \frac{C}{n^p}$, где $C \neq 0$, е докажем, что $\lim \frac{1}{n^pa_{n+1}} \neq 0$. Действительно,
$\frac{1}{n^p a_{n+1}}=\frac{a_1}{a_1n^p a_{n+1}}=\frac{1}{a_1n^p} \exp \hr{p\suml{1}{n}\frac{1}{k} + R_n} = \lcomm$
пользуясь выведенной оценкой для $\suml{1}{n} \frac{1}{k}$, получаем
$\rcomm = \frac{1}{a_1n^p}\hr{e^{\ln n + \gamma + o(1)}}^p \cdot e^{R_n} \bw= \frac{n^p}{a_1n^p} \hr{ e^{\gamma + o(1)}}^p
\cdot e^{R_n}\bw=\frac{1}{a_1}e^{\gamma p + o(1)}\cdot e^{R_n}$. Перейдём к пределу в этом равенстве при $n \ra \infty$. Получим
$\liml{n} \frac{1}{n^pa_{n+1}}= \frac{1}{a_1}e^{\gamma p+A}$. Но это и означает, что $a_{n+1} \sim \frac{C}{n^p}$. Поскольку все члены исходного ряда
положительны, можно применить признак сравнения, откуда вытекает, что он сходится или расходится одновременно с рядом $\sum \frac{C}{n^p}$.
А этот ряд сходится при $p > 1$ и расходится при $p \le 1$.
\end{proof}

\begin{theorem}[Признак Куммера]
Пусть $a_k>0$. Если $\exi b_k>0$ и $C>0\cln a_k < C\hr{ \frac{a_k}{b_k}-\frac{a_{k+1}}{b_{k+1}}}$,
то $\sum a_k$ сходится.
\end{theorem}
\begin{proof}
В самом деле, $\suml{1}{n} a_k < C \suml{1}{n}\hr{ \frac{a_k}{b_k}-\frac{a_{k+1}}{b_{k+1}}} = C\hr{ \frac{a_1}{b_1}-\frac{a_{n+1}}{b_{n+1}}}
< C\frac{a_1}{b_1}$, е частичные суммы ограничены сверху. Следовательно ряд сходится, ведь его члены положительны.
\end{proof}

\begin{note}
В признаке Куммера в качестве $b_k$ имеет смысл брать только члены расходящихся рядов. Действительно, имеем
$\frac{a_k}{b_k}-\frac{a_{k+1}}{b_{k+1}} > 0$, е $\frac{b_{k+1}}{b_k}>\frac{a_{k+1}}{a_k}$. Тогда если $\sum b_k$ сходится, то
$\sum a_k$ сходится по признаку сравнения.
\end{note}

\subsection{Преобразование Абеля. Признаки Абеля, Дирихле и Лейбница сходимости рядов}

\begin{theorem}[Признак Лейбница]
Пусть $a_k \downarrow 0$. Тогда ряд $a_1-a_2+a_3-a_4+\dots$ сходится.
\end{theorem}
\begin{proof}
Сгруппируем члены ряда так: $(a_1-a_2)+(a_3-a_4)+\dots$. В силу монотонного убывания членов, каждая скобка здесь положительна,
следовательно, последовательность $S_{2k}$ возрастает. Покажем, что она ограничена. В самом деле,
$S_{2k}=a_1-(a_2-a_3)-\dots-(a_{2k-2}-a_{2k-1})-a_{2k}$. Здесь каждая скобка также положительна, значит, $S_{2k} < a_1$.
Значит, $S_{2k}$ имеет предел. Имеет место равенство $S_{2k+1}=S_{2k} + a_{2k+1}$. Перейдём к пределу при $k \ra \infty$. В силу
того, что $a_{2k+1} \ra 0$, получаем, что нечётные частичные суммы сходятся
туда же, что и чётные. Отсюда и вся последовательность $S_n$ имеет предел. Значит, исходный ряд сходится.
\end{proof}

\begin{imp}
Пусть $a_k \downarrow 0$. Тогда справедлива оценка $0 \le \hm{\suml{n}{\infty} (-1)^{k+1}a_k} < a_n$.
\end{imp}

\begin{theorem}[Преобразование Абеля]
Преобразуем $\sum u_kv_k$.
\end{theorem}
\begin{proof}
Положим $V_k := \suml{i=1}{k} v_i$, тогда $v_k= V_k-V_{k-1}$, причём $V_0:=0$.
Отсюда имеем $\suml{k=1}{n}u_k v_k\bw=\suml{1}{n}u_k(V_k-V_{k-1})\bw=(u_1V_1\spl u_nV_n)-(u_1V_0\spl u_nV_{n-1})=
V_1(u_1-u_2)\spl V_{n-1}(u_{n-1}-u_n) + u_nV_n$.
\end{proof}

\begin{note}
Аналогично, если положить $V_k:=\suml{i=p}{k}v_i$, получим формулу $\suml{k=p}{n}u_k v_k=\suml{k=p}{n-1}(u_k-u_{k+1})V_k + u_nV_n$.
\end{note}

\begin{theorem}[Лемма Абеля]Пусть последовательность $\hc{a_k}$ монотонна и $\fa n$ имеем $\bbm{\suml{1}{n}u_k} \le B$. Тогда
$\bbm{\suml{k=p}{n}a_ku_k} \le 4B\br{|a_p| + |a_n|}$.
\end{theorem}
\begin{proof}
Пусть $U_k := \suml{i=p}{k}u_i$. Заметим, что $|U_k|=\bbm{\suml{i=p}{k}u_i} =\bbm{\suml{i=1}{k}u_i-\suml{i=1}{p-1}u_i} \le 2B$.
Применим преобразование Абеля: $\bbm{\suml{k=p}{n}a_ku_k} = \bbm{\suml{k=p}{n-1}(a_k-a_{k+1})U_k +
a_nU_n} \le \suml{k=p}{n-1}|a_k-a_{k+1}|\cdot|U_k| + |a_n|\cdot|U_n| \le \lcomm$ в силу монотонности $\rcomm \le
2B\br{|a_p-a_n| + |a_n|} = 2B\br{|a_p +(-a_n)| + |a_n|} \le 2B\br{|a_p| + 2|a_n|} \le 4B\br{|a_p| + |a_n|}$.
\end{proof}

\begin{theorem}[Признак Дирихле]
Пусть $a_k \downarrow 0$, а $\hm{\suml{1}{n}u_k} \le B$. Тогда ряд $\sum a_ku_k$ сходится.
\end{theorem}
\begin{proof}
По лемме Абеля $\bbm{\suml{k=p}{n}a_ku_k} \le 4B\br{|a_p| + |a_n|} \ra 0$, поскольку $a_k \ra 0$. Тогда по КК ряд сходится.
\end{proof}

\begin{theorem}[Признак Абеля]
Пусть $a_k$ монотонны и ограничены. Пусть $\sum u_k$ сходится. Тогда сходится ряд $\sum a_ku_k$.
\end{theorem}
\begin{proof}
В силу монотонности и ограниченности, $a_k$ имеют предел $A$. Тогда последовательность $b_k := a_k-A$ монотонно сходится к 0.
Имеем $\sum b_ku_k = \sum (a_k -A)u_k$. Но этот ряд сходится по Дирихле. Тогда сходится и $\sum a_ku_k$, ведь ряд $\sum Au_k=A\sum u_k$
сходится по условию.
\end{proof}

\subsection{Перестановка членов абсолютно сходящихся рядов. Неравенства Гёльдера и Минковского}

\begin{df}
Дан ряд $\sum u_k$. Рассмотрим произвольную биекцию $f\cln \N \ra \N$. \emph{Перестановкой ряда} назовём
ряд $\sum u_k^*$, в котором $u_k^* :=u_{f(k)}$. В дальнейшем будем использовать символ $*$ для обозначения подобных перестановок.
\end{df}

\begin{theorem}[О безусловной сходимости]
Пусть $\sum |u_k|$ сходится. Тогда сходится любая перестановка $\sum |u_k^*|$, и $\sum u_k^* =\sum u_k$.
\end{theorem}
\begin{proof}
Положим $S_n := \suml{1}{n} u_k, S^*_n := \suml{1}{n} u_k^*$. Тогда в силу абсолютной сходимости, $\fa \ep > 0 \exi N\cln
\fa N \le n \le m$ имеем
$\suml{n}{m}|u_k| < \ep$. Пусть функция $f$ задаёт нашу перестановку. Тогда найдём $p = \max \{ f(1), f(2)\sco f(N) \}$. Тогда при
$n>p$ в разности $S_n - S_n^*$ уничтожатся все слагаемые $u_1\sco u_N$. Значит, там могут остаться только слагаемые с номерами, большими $N$.
Отсюда, в силу нашей оценки, получаем, что при $n>p$ имеем $|S_n-S_n^*| < \ep$. Значит, $S_n^*\ra \sum u_k$.

Покажем, что $\sum u_k^*$ сходится абсолютно. В самом деле, $\fa p \exi n\cln |u_1^*|\sco |u_p^*|$
содержатся среди $|u_1|\sco |u_n|$. Отсюда $\suml{1}{p}|u_k^*| \le \suml{1}{n}|u_k| \le \sum |u_k|$. Значит, частичные суммы
модулей перестановки ограничены, значит, они имеют предел.
\end{proof}

\begin{theorem}[Неравенство Гёльдера для рядов]
Пусть $p>1$, а $q$ сопряжено с $p$. Пусть $\sum |u_k|^p$ и $\sum |v_k|^q$ сходятся. Тогда $\sum |u_kv_k|$ сходится
и $\sum |u_kv_k| \le \br{\sum|u_k|^p}^{\frac{1}{p}} \cdot \br{\sum|v_k|^q} ^{\frac{1}{q}}$.
\end{theorem}
\begin{proof}
Частичные суммы ограничены по неравенств Гёльдера: $\suml{1}{n} |u_kv_k| \le \br{ \sum|u_k|^p }^{\frac{1}{p}}
\cdot \br{ \sum|v_k|^q} ^{\frac{1}{q}}$.
\end{proof}

\begin{theorem}[Неравенство Минковского для рядов]
Пусть $p>1$. Пусть $\sum |u_k|^p$ и $\sum |v_k|^p$ сходятся. Тогда ряд
$\sum |u_k+v_k|^p$ сходится, и справедливо неравенство $\br{\sum |u_k+v_k|^p}^\frac{1}{p} \le \br{\sum |u_k|^p}^{\frac{1}{p}}
+ \br{\sum |v_k|^p}^{\frac{1}{p}}$.
\end{theorem}
\begin{proof}
Аналогично неравенству Гёльдера следует из ограниченности частичных сумм.
\end{proof}


\subsection{Почленное умножение абсолютно сходящихся рядов}

\begin{theorem}[О произведении рядов]
Ряды $\sum u_k$ и $\sum v_k$ сходятся абсолютно. Пусть $\{w_p\}$ составлена из
произведений вида $u_kv_m$, причём каждое произведение встречается один и только один раз.
Тогда ряд $\sum w_p$ сходится абсолютно, и $\sum w_p= \sum u_k \cdot \sum v_k$.
\end{theorem}
\begin{proof}
Заметим, что какую бы конечную сумму $\sum u_kv_m$ мы не взяли, всегда можно указать $N$ так, что произведение
$\suml{1}{N} u_k \cdot \suml{1}{N} v_k$ содержит все слагаемые исходной суммы. Отсюда
$\hm{\sum u_kv_m} \le \sum |u_kv_m| \bw\le \suml{k,m=1}{N}|u_kv_m| \bw= \suml{k=1}{N} |u_k| \cdot \suml{m=1}{N} |v_m|
\le \suml{k=1}{\infty}|u_k| \cdot \suml{m=1}{\infty} |v_m|$. Отсюда следует абсолютная сходимость ряда
$\sum u_kv_m$. По теореме о безусловной сходимости его сумма однозначно определена независимо от порядка слагаемых.
Получим её как предел $\liml{n} U_nV_n$, где $U_n$ и $V_n$ частичные суммы рядов $\sum u_k$ и $\sum v_k$.
Но $\liml{n} U_nV_n=\sum u_k \cdot \sum v_k$.
\end{proof}

\subsection{Теорема Мертенса об умножении рядов}

\begin{theorem}[Мертенса]
Пусть ряд $\sum |u_k| =:W$ сходится, $U := \sum u_k$, $V \bw{:=} \sum v_k$. Пусть $y_n \bw{:=} \suml{0}{n}u_kv_{n-k}$.
Тогда $\sum y_n = UV$.
\end{theorem}
\begin{proof}
Пусть $U_n, V_n, Y_n$ соответствующие частичные суммы рядов. Положим $\xi_n = V_n-V$. Тогда
$Y_n\bw= (u_0v_0)+(u_0v_1+u_1v_0)\spl (u_0v_n+u_1v_{n-1}\spl u_nv_0)=u_0V_n+u_1V_{n-1}\spl u_nV_0=
u_0(V+\xi_n)+u_1(V+\xi_{n-1})\spl u_n(V+\xi_0)=U_nV+u_0\xi_n\spl u_n\xi_0$. Наша цель показать, что
$\ph_n=u_0\xi_n\spl u_n\xi_0$ стремится к~0 при $n \ra \infty$, ибо мы знаем, что $U_nV \ra UV$.
В силу сходимости $\sum v_k$, $\fa \ep > 0 \exi N\cln \fa n \ge N$ имеем $|\xi_n|<\ep$. Отсюда при $n \ge N$ имеем
$|\ph_n| \le |\xi_0u_n\spl \xi_Nu_{n-N}| + |\xi_{N+1}u_{n-N-1} \spl \xi_nu_0| \le |\xi_0u_n\spl \xi_Nu_{n-N}| + W\ep$.
Но и оставшийся модуль можно сделать маленьким, ибо там $N$ слагаемых (е фиксированное для каждого~$\ep$ число), а тогда в силу сходимости
$\sum u_k$ достаточно потребовать, чтобы $|u_k| < \frac{\ep}{N}$ для всех $k > M$. Такая оценка показывает, что $|\ph_n| \le \ep(1+W)$
при $n \ge \max \hc{M,N}$.
Значит, мы можем, выбирая достаточно малые $\ep$, сделать~$\ph_n$ сколь угодно малым.
\end{proof}

\subsection{Теорема Римана о перестановке членов условно сходящегося ряда}

\begin{theorem}[Римана]
Пусть $\sum a_k$ условно сходится. Тогда $\fa A \in \overline{\R} \exi \sum a_k^*\cln A=\sum a_k^*$.
\end{theorem}
\begin{proof}
Положим $$a_k^+ := \case{a_k,& a_k \ge 0, \\ 0, & a_k < 0,} \quad
a_k^- := \case{0, a_k \ge 0, \\ a_k, a_k < 0.}$$
Тогда $a_k = a_k^+ + a_k^-$. Рассмотрим ряды
$\sum a_k^+$ и $\sum a_k^-$. Покажем, что первый ряд расходится к~$+\infty$, а второй к~$-\infty$. В самом деле, если бы они оба сходились,
то сходилась бы их разность. Но это невозможно, ибо, как легко видеть, $\sum (a_k^+ - a_k^-) = \sum |a_k|$, а такой ряд по условию расходится.
Если бы один из рядов расходился, а второй сходился, то их сумма расходилась бы, а мы знаем, что ряд $\sum a_k$ сходится. Таким образом,
оба ряда расходятся, а из определения $a_k^+$ и $a_k^-$ следует, что $\sum a_k^+ = +\infty$, а $\sum a_k^-=-\infty$.

Вычеркнем из наших рядов $\sum a_k^+$ и $\sum a_k^-$ те и только те нули, которые появились в них за счёт условий $a_k <0$ или $a_k \ge 0$.
Тогда в совокупности эти два ряда будут содержать все члены исходного ряда, и никаких других.
Рассмотрим произвольное $A \in \R$,
и предъявим перестановку, которая сходится к числу $A$. Возьмём столько первых членов ряда $\sum a_k^+$, чтобы
$\sap{1}{n_1} > A$, но $\sap{1}{n_1-1} \le A$. Теперь к полученной сумме добавим столько первых членов ряда $\sum a_k^-$, чтобы
получилось так: $(\sap{1}{n_1}) + (\sam{1}{m_1}) < A$, но $(\sap{1}{n_1}) + (\sam{1}{m_1-1}) \ge A$. Посмотрим, на какую величину
может отличаться полученная сумма от числа $A$. Ясно, эта разница не превосходит $\max \{|a_{n_1}^+|, |a_{m_1}^-| \}$. Тогда будем снова
добавлять к полученной сумме следующие члены ряда $\sum a_k^+$, пока не превысим величины $A$, затем добавим несколько членов ряда $\sum a_k^-$,
пока не получим сумму меньше, чем $A$. Полученная сумма будет иметь вид $(\sap{1}{n_1})+(\sam{1}{m_1})+(\sap{n_1+1}{n_2})+(\sam{m_1+1}{m_2})$.
Тогда получим, что $A$ отличается от этой суммы не более, чем на $\max \{|a_{n_2}^+|, |a_{m_2}^-| \}$. Продолжим этот процесс. Это всегда возможно
в силу расходимости рядов $\sum a_k^+$ и $\sum a_k^-$.

Покажем, что сконструированный таким образом ряд сходится к $A$. В самом деле, в силу сходимости исходного ряда, имеем
$a_k^+ \ra 0, a_k^- \ra 0$, значит, частичные суммы полученного ряда в силу приведённых выше оценок можно сделать сколь угодно мало
отличающимися от $A$. Действительно, если начиная с какого-то номера $N$ имеем $|a_k^+| < \ep$, и $|a_k^-| < \ep$, то любая частичная
сумма вида $(a_1^+ + \dots)\spl (\dots + a_{m_i}^-)$, где $m_i > N, n_i > N$, будет отличаться от $A$ меньше, чем на $\ep$. А из алгоритма построения следует,
что любая другая частичная сумма отличается от $A$ ещё меньше, если только взять достаточно много слагаемых.

Покажем теперь, что существует перестановка, сходящаяся к $+\infty$. Будем складывать $a_k^+$, пока не получим числа, большего 1. Затем прибавим
столько отрицательных, чтобы далеко не уехать от 1. Затем добавим так много положительных, чтобы заехать за число 2. Затем добавим ещё отрицательных...
Ясно, что в силу стремления к 0 членов исходного ряда, мы можем позволить себе не отходить далеко от очередного целого числа. Так мы построим ряд,
который, очевидно, расходится к $+\infty$.
\end{proof}

\subsection{Суммирование рядов методом средних арифметических}

\begin{df}
Пусть $S_n$ частичные суммы ряда $\sum a_k$. Положим $\si_n := \frac{1}{n}\suml{1}{n}S_k$ и назовём $\si_n$ частными
\emph{среднеарифметическими} суммами. Говорят, что ряд \emph{суммируется методом (C,1) (методом Чезаро) к числу} $S \in \ol{\R}$, если
$\lim \si_n=S$.
\end{df}

Выразим $\si_n$ через члены ряда: $\si_n = \frac{1}{n}\bs{(a_1) + (a_1 + a_2) \spl (a_1 \spl a_n)}=
\frac{1}{n}\suml{1}{n}(n+1-k)a_k= \suml{1}{n}\hr{1-\frac{k-1}{n}} a_k$. Заметим, что расходящийся в обычном смысле ряд
$1-1+1-1+\dots$ суммируется методом (С,1) к $\frac{1}{2}$. Метод (С,1) обладает
линейностью, е если ряды $\sum a_k$ и $\sum b_k$ суммируются к $A$ и $B$ соответственно, то ряд $\sum (\lambda a+\mu b)$ суммируется
к $\lambda A + \mu B$, и это легко доказывается исходя из выражения $\si_n$ через члены ряда.

\begin{theorem}[О вполне регулярности метода (С,1)]
Пусть $\sum a_k = S \in \overline{\R}$. Тогда ряд суммируется методом (С,1) к $S$.
\end{theorem}
\begin{proof}
\pt{1} Пусть $S \in \R$, тогда в силу сходимости частичные суммы будут ограничены: $|S_n| < L$ и $|S - S_n| < 2L$. Тогда
$\fa \ep \bw> 0 \exi N\cln \fa k \in \N$ имеем
$|S - S_{N+k}| \bw< \ep$. Пусть $p \in \N$. Отсюда
$$|S - \si_{N+p}| = \hm{\frac{1}{N+p}\suml{1}{N+p}S -\frac{1}{N+p}\suml{1}{N+p}S_k}
\le \frac{1}{N+p}\suml{1}{N}|S-S_k| +\frac{1}{N+p}\suml{N+1}{N+p}|S-S_k|
\le \frac{2LN}{N+p}+\ep\frac{p}{N+p} \le \frac{2LN}{p} + \ep.$$
Отсюда видно, что взяв $\ep$ достаточно маленьким, а $p$ достаточно большим,
эти два слагаемых можно сделать сколь угодно малыми.

\pt{2} $S=+\infty$. Тогда $\fa M >0 \exi N\cln \fa n > N$ имеем $S_n>M$. Имеем
$$\si_n=\frac{1}{n}\suml{1}{n}S_k=
\frac{1}{n}\suml{1}{N}S_k+\frac{1}{n}\suml{N+1}{n}S_k > \frac{1}{n}\suml{N+1}{n}M-\hm{\frac{1}{n}\suml{1}{N}S_k} =
M\frac{n-N}{n}-\hm{\frac{1}{n}\suml{1}{N}S_k} > \frac{M}{2}$$
при достаточно больших $n$, ибо второе слагаемое можно сделать
сколь угодно малым. Отсюда следует, что $\si_n \ra +\infty$.
\end{proof}

\begin{theorem}
Пусть ряд $\sum a_k$ суммируется к числу $S \in \R$. Тогда $S_n=o(n)$, $a_n=o(n)$.
\end{theorem}
\begin{proof}
Легко видеть, что $S_n= n(\si_n-\si_{n-1}) + \si_{n-1}$. Имеем
$\si_n - \si_{n-1} \ra 0$, а $\si_{n-1}$ имеет конечный предел. Отсюда $\lim\frac{S_n}{n} = \lim\hr{\si_n-\si_{n-1}+\frac{\si_{n-1}}{n}} = 0$,
откуда $S_n=o(n), n \ra \infty$. Используя эту оценку, получаем оценку для членов ряда: $a_n=S_n-S_{n-1}=o(n)-o(n)=o(n)$.
\end{proof}

\begin{theorem}
Если неотрицательный ряд $\sum a_k$ суммируется к числу $A \in \R$, то он сходится к числу $A$.
\end{theorem}
\begin{proof}
В силу того, что $a_k \ge 0$, достаточно доказать ограниченность частичных сумм $S_n$ нашего ряда. Тогда он гарантированно сойдётся к некоторому
числу $S$. Но по теореме о регулярности метода Чезаро, он суммируется к этому же числу $S$. Значит, $A=S$. Покажем ограниченность.
Докажем, что $S_n\le 2\si_{2n}$. Поскольку $a_k\ge 0$, имеем $2\si_{2n}=\frac{1}{n}\suml{k=1}{2n}S_k
\ge \frac{1}{n}\suml{k=n+1}{2n}\suml{i=1}{k}a_i \ge \frac{1}{n}\suml{k=n+1}{2n}\suml{i=1}{n}a_i=
\frac{1}{n}\suml{k=n+1}{2n}S_n=S_n$. Поскольку $\lim \si_n < \infty$, $S_n$ ограничены сверху.
\end{proof}

\subsection{Бесконечные произведения. Связь сходимости произведений и рядов}

Пусть $p_k \in \R$. Если существует предел $\liml{n}\prodl{1}{n}p_k = P \in \overline{\R}$, то
говорят, что произведение сходится, если $P \in \R \wo \hc{0}$, и расходится, если $P \in \{0,+\infty,-\infty\}$. Всегда будем предполагать,
что $p_k \neq 0$, иначе произведение равно 0. Частичные произведения будем обозначать через $P_n$.

\begin{theorem}[Необходимое условие сходимости]
Пусть $\prod p_k$ сходится. Тогда $\liml{k}p_k=1$.
\end{theorem}
\begin{proof}
Имеем $p_n=\frac{P_{n}}{P_{n-1}}$. Поскольку $\liml{n} P_n \neq 0$, можно перейти к пределу и получить требуемое.
\end{proof}

\begin{theorem}
Пусть $p_k > 0$. Тогда сходимость $\prod p_k$ равносильна сходимости ряда $\sum \ln p_k$, и в случае сходимости, $\prod p_k = e^{\sum \ln p_k}$.
\end{theorem}
\begin{proof}
Имеем $\ln P_n= \suml{1}{n} \ln p_k$. Тогда $P_n=e^{\suml{1}{n} \ln p_k}$. Из непрерывности $\exp$ всё следует.
\end{proof}

\begin{theorem}
Если все $a_k$ имеют один знак, сходимость $\prod (1+a_k)$ равносильна сходимости ряда $\sum a_k$.
\end{theorem}
\begin{proof}
Из необходимого условия сходимости получаем, что $a_k \ra 0$. Тогда для всех номеров, начиная с некоторого,
имеем $1 + a_k > 0$. Применим предыдущую теорему, тогда сходимость произведения $\prod (1+a_k)$ равносильна
сходимости $\sum \ln (1 + a_k)$. Если все $a_k$ имеют один и тот же знак, $\ln(1+a_k)$ тоже знакопостоянны. Тогда имеем
$\lim \frac{\ln(1+ a_k)}{a_k}=1$, а в силу знакопостоянности можно применить II-й признак сравнения.
\end{proof}

\begin{theorem}
Пусть $a_k > -1, a_k \neq 0$. Если один из рядов $\sum a_k$ (1) или $\sum a_k^2$ (2) сходится, то для сходимости
$\prod (1+a_k)$ необходимо и достаточно сходимости второго ряда.
\end{theorem}
\begin{proof}
Если (1) или (2) сходится, то $a_k \ra 0$. Имеем $\ln (1+ a_k) = a_k - \frac{a_k^2}{2}+o(a_k^2)$, $k \ra \infty$.
Отсюда $\lim \frac{a_k - \ln(1+a_k)}{a_k^2}\bw=\frac{1}{2}$. Заметим, что $a_k > \ln(1+a_k)$, поэтому по второму признаку
сравнения имеем $\sum (a_k - \ln(1+a_k)) \sim \sum a_k^2$. Заметим также, что сходимость $\prod (1+a_k)$ равносильна
сходимости $\sum \ln(1+a_k)$. Но из полученной выше равносильности следует, что если сходится (1), то (2) $\sim \sum \ln(1+a_k)$.
Аналогично, если сходится (2), то (1) $\sim \sum \ln(1+a_k)$.
\end{proof}

\section{Функциональные ряды. Равномерная сходимость}

\subsection{Равномерная сходимость. Критерий Коши. Признак Вейерштрасса}

\begin{df}
Пусть $\Xc$ некоторое множество, $\Fs(\Xc)$ множество функций $f\cln \Xc \ra \Cbb$. \emph{Последовательностью функций} на $\Xc$ называется
отображение $\mu\cln \N \ra \Fs(\Xc), \mu\cln n \mapsto f_n$. Если $\fa x \in \Xc \exi \liml{n} f_n(x) =: f(x) \in \Cbb$, то говорят, что
$f_n$ сходится к $f$ поточечно на $\Xc$ и пишут $f_n \overset{\Xc}{\ra} f$. Говорят, что $f_n$ \emph{сходится к $f$ равномерно} на $\Xc$,
если $\fa \ep > 0 \exi N\cln \fa n > N$ и $\fa x \in \Xc$ имеем $|f_n(x)-f(x)| < \ep$.
Обозначение: $f_n \convu{\Xc} f$. Из равномерной сходимости следует поточечная. Введём \emph{равномерную норму}
на $\Xc$: $\|f\|_\Xc := \underset{\Xc}{\sup} |f(x)|$.
\end{df}

\begin{theorem}[Равномерная сходимость в терминах нормы]
$f_n \convu{\Xc} f \Lra \liml{n}\|f_n(x)-f(x)\|_\Xc = 0$.
\end{theorem}
\begin{proof}
Распишем всё по определению и убедимся в справедливости данного утверждения.
\end{proof}

\begin{note}
Существуют неравномерно сходящиеся последовательности функций. Рассмотрим $f_k := x^k$ на множестве $[0,1]$. Эти функции
сходятся поточечно к 0 на $[0,1)$ и к 1 при $x=1$. Если бы сходимость была равномерной, то $\fa \ep > 0$ начиная с некоторого номера $n$
значения всех функций $f_n$ на множестве $[0,1)$ не превосходили бы $\ep$. Но все $f_k$ непрерывны, в частности, в точке $x=1$. Значит,
для каждой функции $f_k$ всегда найдётся такая левая окрестность точки $x=1$, что в ней значения функции близки к $f_k(1)=1$.
\end{note}

\begin{theorem}[Критерий Коши равномерной сходимости]
$f_n \convu{} f \Lra \fa \ep > 0 \exi N\cln \fa n,m > N$ имеем $|f_n(x) - f_m(x)| < \ep$.
\end{theorem}
\begin{proof}
Из условия равномерной сходимости, $\fa \ep > 0 \exi N\cln \fa n, m > N$ имеем $|f_n(x) - f(x)| < \ep$ и $|f_m(x) - f(x)| < \ep$.
для всех $x \in \Xc$. Отсюда следует, что $|f_n(x) - f_m(x)| < 2\ep$ для всех $x \in \Xc$.

Обратно, пусть выполнено условие Коши. Тогда $\fa \ep > 0 \exi N\cln \fa n,m > N$ имеем $|f_n(x) - f_m(x)| < \ep$ на $\Xc$. Таким образом,
предел существует для любого $x \in \Xc$ по КК для последовательностей. Положим $f(x) \bw{:=} \liml{m} f_m(x)$. Перейдём к пределу при $m \ra \infty$ в неравенстве,
получим $|f_n(x) - f(x)| \le \ep$ для $\fa x \in \Xc$, что и требовалось доказать.
\end{proof}

Говорят, что ряд $\sum u_k(x)\convu{} S(x)$, если $S_n(x)\convu{}S(x)$. Аналогично числовым рядам, справедив КК для
функциональных рядов. Если ряд $\sum u_k(x)$ сходится равномерно, то $u_k(x)\convu{}0$. В самом деле, положим в КК $m=n$.
Ясно, что если $\sum u_k(x) \convu{\Xc}$ и $\sum v_k(x)\convu{\Xc}$, то для любых ограниченных функций
$\alpha, \beta \in \Fs(\Xc)$ имеем $\sum \bs{\alpha(x)u_k(x) + \beta(x)v_k(x)} \convu{\Xc}$.

\begin{df}
Пусть $\ph\cln \Xc \times \Tc \ra \Cbb$, $t_0$ предельная точка $\Tc$. Говорят, что $\ph(x,t) \convu{\Xc} f(x)$ при
$t \overset{\Tc}{\ra} t_0$, если $\fa \ep > 0 \exi U_\de(t_0)\cln \fa x \in \Xc, \fa t \in \dot{U}_\de(t_0) \cap \Tc$ имеем $|\ph(x,t)-f(x)| < \ep$.
\end{df}

\begin{theorem}[Признак Вейерштрасса]
Пусть на $\Xc$ заданы ряды $\sum u_k(x)$ и $\sum a_k(x)$. Если $\fa x \in \Xc$ имеем $|u_k(x)| \le a_k(x)$,
то из равномерной сходимости $\sum a_k(x)$ на $\Xc$ вытекает абсолютная и равномерная сходимость $\sum u_k(x)$ на $\Xc$, а если
$\sum u_k(x)$ не является равномерно сходящимся, то и $\sum a_k(x)$ не сходится равномерно.
\end{theorem}
\begin{proof}
Пусть $\sum a_k(x)$ сходится равномерно. Тогда $\fa \ep > 0 \exi N\cln \fa x \in \Xc,\fa n \ge m > N$ имеем $\suml{m}{n} a_k(x) < \ep$. Тогда
$\hm{\suml{m}{n} u_k(x)} \le \suml{m}{n}|u_k(x)|\le \suml{m}{n} a_k(x) < \ep$. Отсюда следует сходимость.
\end{proof}

\subsection{Признаки Дирихле и Абеля равномерной сходимости рядов}

\begin{theorem}[Признак Дирихле]
Пусть $a_k(x)\convu{\Xc}0$, и $\fa x \in \Xc$ последовательность $a_k(x)$ монотонна. Пусть $\suml{1}{n} u_k(x)$ равномерно
ограничены на $\Xc$ числом $B$. Тогда $\sum u_k(x)a_k(x)\convu{\Xc}$.
\end{theorem}
\begin{proof}
Положим $U_n(x) := \suml{1}{n}u_k(x)$. Так как $|U_n| \le B$, по лемме Абеля $\fa x \in \Xc$ имеем
$\hm{\suml{m}{n}u_k(x)a_k(x)} \bw\le 4B\br{|a_m(x)| + |a_n(x)|}$.
В силу первого условия теоремы, $\fa \ep > 0 \exi N\cln \fa m>N, \fa x \in \Xc$ имеем $|a_m(x)| < \ep$. Тогда, продолжая нашу оценку,
имеем $\hm{\suml{m}{n}u_k(x)a_k(x)} < 4B(\ep + \ep) = 8B\ep$. Значит, выбрав достаточно малые $\ep$, можно эту сумму сделать
сколь угодно малой для всех достаточно больших $n,m$, и по КК ряд сходится равномерно.
\end{proof}

\begin{theorem}[Признак Абеля]
Пусть $a_k(x)$ равномерно ограничены на $\Xc$ числом $B$. Пусть $\fa x \in \Xc$ последовательность
$a_k(x)$ монотонна, а $\sum u_k(x)\convu{\Xc}U(x)$. Тогда $\sum u_k(x)a_k(x)$ сходится равномерно на $\Xc$.
\end{theorem}
\begin{proof}
В силу равномерной сходимости $\sum u_k(x)$, имеем $\fa \ep > 0 \exi N\cln \fa n \ge m > N$ имеем $\hm{\suml{m}{n} u_k(x)} <\ep$ для
$\fa x \in \Xc$. По лемме Абеля $\hm{\suml{m}{n} u_k(x)a_k(x)} < 4\ep\br{|a_m(x)| + |a_n(x)|}< 4\ep \cdot 2B=8B\ep$. Осталось применить КК.
\end{proof}

\subsection{Признак Дини равномерной сходимости рядов}

\begin{theorem}[Признак Дини]
Пусть $\Kc$ компакт, на котором задана последовательность непрерывных функций $f_n$. Если она монотонно сходится к
непрерывной функции $f$, то эта сходимость равномерная.
\end{theorem}
\begin{proof}
Пусть для определённости $f_n(x)$ монотонно возрастают при каждом $x \in \Kc$. Тогда рассмотрим произвольное $\ep > 0$ и для любой точки
$x \in \Kc$ найдём такой номер $n_x$, что $0 \le f(x)-f_{n_x}(x) < \frac{\ep}{2}$. Тогда в силу непрерывности функций $f_{n_x}$ и $f$
найдётся такая окрестность $U(x)$, что в ней выполнена оценка $0 \le f(x)-f_{n_x}(x) < \ep$. Теперь рассмотрим все такие окрестности,
пробежав по всем точкам $\Kc$. Они образуют открытое покрытие $\Kc$, значит, из него можно выделить конечное
подпокрытие $U(x_1)\sco U(x_k)$. Тогда выберем $N(\ep) := \max \hc{n_{x_1}\sco n_{x_k}}$. Тогда $\fa n>N$ в силу возрастания
$\hc{f_n(x)}$ при каждом $x$ будем иметь $0 \le f(x) - f_n(x) < \ep$. Это означает равномерную сходимость.
\end{proof}

\begin{theorem}[Признак Дини для рядов]
Если члены ряда $\sum a_k(x)$ есть неотрицательные непрерывные на компакте $\Kc$ функции, и ряд сходится
к непрерывной функции, то эта сходимость равномерная.
\end{theorem}
\begin{proof}
В самом деле, частичные суммы $s_n(x)$ нашего ряда удовлетворяют условиям признака Дини.
\end{proof}

\subsection{Теорема о предельном переходе в равномерно сходящихся рядах}

\begin{theorem}[О предельном переходе]
Пусть $\sum u_k(x)$ сходится равномерно на $\Xc$, и $x_0$ предельная точка $\Xc$. Пусть $\Bc$ база
$x \ra x_0, x \in \Xc$. Пусть $\fa k$ существует $\liml{\Bc} u_k(x) =\alpha_k$. Тогда $\sum \alpha_k$ сходится и $\liml{\Bc}
\sum u_k(x) = \sum \alpha_k$.
\end{theorem}
\begin{proof}
В силу КК и равномерной сходимости имеем $\fa \ep > 0 \exi N\cln \fa x \in \Xc, \fa n \ge m>N$ имеем $\hm{\suml{m}{n}u_k(x)} < \ep$.
Перейдём к пределу по базе $\Bc$ в этом неравенстве. Имеем $\hm{\suml{m}{n}\liml{\Bc}u_k(x)} = \hm{\suml{m}{n} \alpha_k}
\le \ep$. Значит, ряд $\sum a_k$ сходится, обозначим его сумму числом $A$. Положим $S(x) := \sum u_k(x)$, $S_p(x) := \suml{1}{p} u_k(x)$.
Тогда $\fa \ep > 0 \exi N\cln \fa p > N, \fa x \in \Xc$ имеем $|S(x)-S_p(x)|<\ep$ и, кроме того, $\hm{A-\suml{1}{p}\alpha_k} < \ep$.
Тогда
\begin{multline*}
|S(x)-A| = \bbm{S(x)-S_p(x)+S_p(x)-\suml{1}{p}\alpha_k+\suml{1}{p}\alpha_k-A} \le\\
\le|S(x)-S_p(x)|+\bbm{S_p(x)-\suml{1}{p}\alpha_k} +\bbm{\suml{1}{p}\alpha_k-A} <\ep+\bbm{S_p(x)-\suml{1}{p}\alpha_k} +\ep.
\end{multline*}
Тогда для каждого фиксированного $p>N$ по условию $\exi U(x_0)\cln \fa x \in U(x_0) \cap \Xc$ выполнено
$\bbm{S_p(x)-\suml{1}{p}\alpha_k} < \ep$. В самом деле, достаточно потребовать, чтобы в этой окрестности каждая из функций $u_k(x)$ отличалась
по модулю от $\alpha_k$ меньше, чем на $\frac{\ep}{p}$, для $k=1\sco p$. Тогда для $\fa x \in U \cap \Xc$ оценка перепишется в виде $|S(x)-A| < 3\ep$.
Но это и означает, что $\liml{\Bc} \sum u_k(x) = A$.
\end{proof}

\begin{imp}
Пусть $\sum u_k(x)\convu{\Xc}$. Пусть $u_k(x) \in\Cb(x_0)$ по $\Xc$.
Тогда $\sum u_k(x) \in\Cb(x_0)$ по $\Xc$.
\end{imp}

\subsection{Почленное интегрирование равномерно сходящихся рядов}

\begin{theorem}[Об интегрировании]
Пусть $f_k\convu{[a,b]}f$, и $f_k \in\Rb[a,b]$. Тогда $f \in\Rb[a,b]$ и $\intlab f\,d
x=\liml{k}\intlab f_k\,d x$.
\end{theorem}
\begin{proof}
В силу равномерной сходимости имеем $\fa \ep >0 \exi N\cln \fa p>N, \fa x \in [a,b]$ имеем
$|f_p(x)-f(x)|<\ep$. Значит, функция $f$ ограничена. В самом деле, имеем $f_p -\ep < f < f_p+\ep$
$(*)$. Все функции $f_p$ интегрируемы, и, следовательно, ограничены. Кроме того, $\fa p \exi
T_p[a,b]\cln \usd{T_p}{f_p}-\lsd{T_p}{f_p} < \ep$. Отсюда, используя неравенство $(*)$, получаем
$\usd{T_p}{f} \le \usd{T_p}{f_p}+\ep(b-a)$ и $\lsd{T_p}{f_p}-\ep(b-a) \le \lsd{T_p}{f}$. Вычтем из
первого неравенства второе, получим $\usd{T_p}{f}-\lsd{T_p}{f} \le
\usd{T_p}{f_p}-\lsd{T_p}{f_p}+2\ep(b-a) < \ep\bs{1+2(b-a)}$. Тем самым доказано, что $f
\in\Rb[a,b]$. Теперь можно проинтегрировать неравенство $(*)$. Получим $\intlab f_p\,d x
-\ep(b-a)\le \intlab f\,d x\le \intlab f_p\,d x+\ep(b-a)$. Данное неравенство верно для всех
$p>N(\ep)$. Но это и означает, что $\intlab f\,d x=\liml{k}\intlab f_k\,d x$.
\end{proof}

\begin{imp}
Пусть $\sum u_k(x)\convu{[a,b]}S(x)$, $u_k(x)\in\Rb[a,b]$. Тогда $S(x) \in\Rb[a,b]$, и $\intlab
\sum u_k(x)\,d x=\sum \intlab u_k(x)\,d x$.
\end{imp}

\begin{note}
Равномерная сходимость существенное условие. Пусть $\hc{r_k}$ нумерация $\Q$. Рассмотрим
$$u_k(x) =\case{1,&x=r_k,\\0,&x \neq r_k,}$$
Легко видеть, что $u_k \in\Rb[0,1]$ и $\sum u_k(x)$ сходится,
но $\sum u_k(x) = \Dc(x) \notin \Rb[0,1]$.
\end{note}

\subsection{Почленное дифференцирование равномерно сходящихся рядов}

\begin{theorem}[О дифференцировании]
Пусть $\Xc := [a,b]$, $f_k(x) \in\Db(\Xc), \exi x_0 \in \Xc\cln \hc{f_k(x_0)}$ сходится, $f'_k(x)\convu{\Xc}$. Тогда
$f_k\convu{}f$, $f \in\Db(\Xc)$ и $f'(x)=\liml{k}f'_k(x)$.
\end{theorem}
\begin{proof}
Воспользуемся КК: $\fa \ep > 0 \exi N\cln \fa m,n>N$ имеем $|f_m(x_0)-f_n(x_0)|<\ep$ и $\fa x \in \Xc$ имеем
$|f'_m(x)\bw-f'_n(x)| < \ep$. Здесь мы одновременно пользуемся КК для последовательностей чисел и
для равномерно сходящихся функций. Тогда $|f_m(x)-f_n(x)| \le \hm{f_m(x)-f_n(x)-\bs{f_m(x_0)-f_n(x_0)} } + |f_m(x_0)-f_n(x_0)| \le \lcomm$ по ФКПЛ $\rcomm \le |f'_m(\xi)-f'_n(\xi)||x-x_0|+\ep
\le \ep(b-a)+\ep$. Тем самым равномерная сходимость $f_k$ доказана.

Теперь пусть $x \in \Xc$, рассмотрим $g_k(t) := \frac{f_k(t)-f_k(x)}{t-x}$, причём $g_k$ определены на множестве $\dot{\Xc} := \Xc \wo \hc{x}$.
Докажем, что $g_k$ равномерно сходятся на $\dot{\Xc}$. Рассмотрим $|g_n(t)-g_m(t)|=\frac{1}{|t-x|}\hm{f_n(t)-f_m(t)-\bs{f_n(x)-f_m(x)}} =
\lcomm$ по ФКПЛ $\exi \xi$ между $t$ и $x$ $\rcomm=\frac{1}{|t-x|}|f'_n(\xi)-f'_m(\xi)||t-x|=|f'_n(\xi)-f'_m(\xi)|$, а в силу равномерной сходимости
производных $\fa \ep>0 \exi N\cln \fa n,m>N$ имеем $|f'_n(\xi)-f'_m(\xi)|< \ep$. Это доказывает равномерную сходимость функций $g_k$.
Имеем $\liml{t\ra x} g_k(t)=f'_k(x)$, $\liml{k} g_k(t)=\frac{f(t)-f(x)}{t-x}$. В силу равномерной сходимости два предела
можно поменять местами. Тогда имеем $\liml{k}\hr{\liml{t\ra x} g_k(t)} =\liml{t\ra x}\hr{ \liml{k}g_k(t)} $. Отсюда
$\liml{k}f'_k(x)=\liml{t\ra x} \frac{f(t)-f(x)}{t-x}=f'(x)$.
\end{proof}

\begin{imp}
Пусть $u_k(x) \in\Db(\Xc), \exi x_0 \in \Xc\cln \sum u_k(x_0)$ сходится, $\sum u'_k(x)\convu{\Xc}$. Тогда
$\sum u_k\convu{\Xc}u$, $u \in\Db(\Xc)$ и $u'(x)=\sum u'_k(x)$.
\end{imp}

\subsection{Пример непрерывной нигде не дифференцируемой функции}
Рассмотрим $f(x) := \sum \frac{\sin(8^kx)}{2^k}$. Эта функция, очевидно, непрерывна, ибо каждое слагаемое непрерывно,
а ряд сходится равномерно по признаку Вейерштрасса. Более того, каждое слагаемое бесконечно дифференцируемо. Докажем, что $f$ нигде
не дифференцируема.

Рассмотрим $|f(t)-f(x)|$, где $t=x \pm 2^{-3n-1}\pi$. Имеем тогда $t-x=\pm 2^{-3n-1}\pi$. Распишем по определению:
$|f(t)-f(x)|=\hm{\suml{k=1}{\infty} 2^{-k}\br{\sin (8^kt)-\sin(8^kx)}} $. Покажем, что в этой сумме все слагаемые при $k>n$ равны 0.
В самом деле, пусть $k>n$. Тогда $\sin(8^kt)=\sin\br{8^k(x \pm 2^{-3n-1}\pi)}=\sin(8^k x \pm 2^{3(k-n)-1}\pi)$. Тогда при
$k>n$ число $m=2^{3(k-n)-1} \in 2\Z$, значит, $\sin(8^kt)=\sin(8^kx\pm m\pi)=\sin(8^kx)$. Значит, мы имеем право написать
$|f(t)-f(x)|=\hm{\suml{k=1}{n} 2^{-k}\br{\sin (8^kt)-\sin(8^kx)}} $. Мы знаем, что $|\sin \alpha - \sin \beta| \le |\alpha - \beta|$,
ибо $\sin$ удовлетворяет условию Липшица с константой 1. Мы хорошо знаем, что $\sin \alpha - \sin \beta =
2\cos\hr{ \frac{\alpha+\beta}{2}} \sin\hr{ \frac{\alpha-\beta}{2}} $. Теперь оценим снизу наш модуль:
\begin{multline*}
|f(t)-f(x)| \ge 2^{-n}|\sin (8^nt)-\sin(8^nx)| - \suml{k=1}{n-1}2^{-k}|\sin (8^kt)-\sin(8^kx)|=\\
=2^{-n} \cdot 2\hm{\cos\hr{ \frac{8^n(t+x)}{2}} \sin\hr{ \frac{8^n(t-x)}{2}}} - \suml{k=1}{n-1} 2^{-k}\cdot 8^k|t-x|=\\
=2^{-n} \cdot 2\hm{\cos\hr{ 8^nx\pm \frac{\pi}{4}}} \cdot\hm{\sin\hr{ \pm \frac{\pi}{4}}} - 2^{-3n-1}\pi\suml{k=1}{n-1} 4^k=\\
=2^{-n} \sqrt{2}\hm{\cos\hr{ 8^nx\pm \frac{\pi}{4}}} - 2^{-3n}\pi\frac{4^n-4}{6} \ge
2^{-n} \sqrt{2}\hm{\cos\hr{ 8^nx\pm \frac{\pi}{4}}} - 2^{-n}\frac{\pi}{6}.
\end{multline*}
Для любого $n$ всегда можно так выбрать знак $+$ или $-$, чтобы $\hm{\cos\hr{ 8^nx \pm \frac{\pi}{4}}} > \frac{1}{2}$.
\textbf{Указание:} нарисуйте график $|\cos \alpha|$ и убедитесь в справедливости этого утверждения.
Тогда при правильно выбранном знаке справедлива оценка
$|f(t)-f(x)| \ge 2^{-n}\hr{ \frac{\sqrt{2}-1}{2}} = 2^{-n}C, C > 0$. Значит, можно оценить выражение
$\frac{|f(t)-f(x)|}{|t-x|} \ge C\frac{2^{-n}}{2^{-3n-1}}\bw=2C \cdot 4^n \ra \infty$. Это означает, что
в точке $x$ производная функции $f$ не существует, ибо мы выделили подпоследовательность, сходящуюся к $x$, для
которой выражение под знаком предела в определении производной стремится к $\infty$.

\subsection{Первая теорема Абеля о степенных рядах. Радиус сходимости, формула Коши Адамара}

Степенным рядом называется ряд $\sum c_k(z-z_0)^k$, где $c_k, z,z_0 \in \Cbb$. Ясно, что сделав подходящую замену,
можно свести всё к изучению степенных рядов вида $\sum c_kz^k$. Их мы и будем изучать.
\begin{theorem}[Первая теорема Абеля]
Пусть $\sum c_kz^k$ сходится в $z_0 \neq 0$. Тогда $\fa q \in (0, |z_0|)$ ряд $\sum c_kz^k$ сходится абсолютно и равномерно
при $|z| \le q$.
\end{theorem}
\begin{proof}
В самом деле, $|c_kz^k|=|c_kz_0^k|\cdot\bm{\frac{z}{z_0}} ^k \le |c_k z_0^k| \cdot \bm{\frac{q}{z_0}} ^k <
M\bm{\frac{q}{z_0}} ^k$, а в силу того, что $\bm{\frac{q}{z_0}} < 1$, ряд $\sum M \bm{\frac{q}{z_0}} ^k$
сходится как геометрическая прогрессия. Осталось применить признак Вейерштрасса.
\end{proof}

\begin{note}
Условие $|z| \le q < |z_0|$ существенно. Рассмотрим ряд $\sum \frac{z^k}{k}$, $z_0=1$.
Допустим, что он сходится равномерно при всех $z\cln |z|<1$ и запишем условие
Коши: $\fa \ep > 0 \exi N\cln \fa n,m>N, \fa z\cln |z| < 1$ имеем
$\hm{\suml{m}{n} \frac{z^k}{k}} < \ep$. Перейдём к пределу при $z \ra 1$. Тогда
$\hm{\suml{m}{n} \frac{1}{k}} \le \ep$, а
это противоречит расходимости $\sum \frac{1}{k}$.
\end{note}

\begin{note}
Требование сходимости ряда в точке $z_0$ можно ослабить, заменив его равномерной ограниченностью
членов ряда в точке $z_0$ константой $M$.
\end{note}

\begin{imp}
Если есть сходимость в $z_0$, то $\fa z\cln |z| < |z_0|$ ряд сходится абсолютно, но не обязательно равномерно.
Если при $z_0 \neq 0$ имеет место расходимость, то $\fa z\cln |z| > |z_0|$ ряд расходится.
\end{imp}

\begin{df}
Число $R \in \overline{\R}$ называется \emph{радиусом сходимости степенного ряда}, если он
сходится при $\fa z \cln |z| < R$ и расходится при
$\fa z\cln |z| > R$.
Покажем, что $R = \sup |z|$, где $\sup$ берётся по всем $z$, для которых ряд сходится.
Допустим, что $R \neq \sup |z| = R'$. Пусть сначала $R > R'$. Рассмотрим $z \in (R', R)$. Для
$z$ ряд сходится, поскольку $|z| < R$. Но это противоречит определению $R'$. Наоборот, пусть
$R'=R+\ep$, где $\ep > 0$. Тогда по определению точной верхней грани $\exi z\cln R'-|z| < \frac{\ep}{2}$, для
которого ряд сходится. Значит, мы нашли $z\cln |z| >R$, для которого ряд сходится, а это противоречит
определению $R$. Значит, остаётся единственная возможность $R=R'$. Итак, наши определения эквивалентны.
\end{df}

\begin{theorem}[Формула Коши Адамара]
Радиус сходимости степенного ряда $\sum c_k z^k$ вычисляется по формуле $R = \frac{1}{A}$, где
$A=\overline{\liml{k}} \sqrt[k]{|c_k|}$. Если $A = +\infty$, то $R = 0$, а если $A=0$, то $R = +\infty$.
\end{theorem}
\begin{proof}
Усовершенствуем радикальный признак Коши, а с помощью него докажем нашу формулу. Пусть дан ряд $\sum u_k$, и
$\alpha = \overline{\liml{k}}\sqrt[k]{|u_k|}$. Тогда ряд сходится при $\alpha < 1$ и расходится при $\alpha > 1$.
В самом деле, если $\alpha < 1$, то выберем $q$ так, что $\alpha < q < 1$. В силу того, что $\alpha < q$,
а $\alpha$ верхний предел, найдётся $N \cln \fa k > N$ имеем $\sqrt[k]{|u_k|} < q$ (это легко доказать от противного).
Тогда для $k > N$ будем иметь $|u_k| < q^k$, а геометрическая прогрессия сходится.
Тогда по признаку сравнения сходится и исходный ряд. Если же $\alpha > 1$, то выберем
подпоследовательность $\sqrt[k_i]{|u_{k_i}|}$, сходящуюся
к верхнему пределу. Тогда $\exi N \cln \fa i > N$ имеем $|u_{k_i}| > 1$, поэтому
$u_{k_i} \nra 0$, значит, ряд расходится.

Теперь применим этот признак к ряду $\sum c_k z^k$. Имеем
$\overline{\liml{k}}\sqrt[k]{|c_kz^k|}=|z|\cdot\overline{\liml{k}}\sqrt[k]{|c_k|}$.
Положим $A := \overline{\liml{k}}\sqrt[k]{|c_k|}$. Ряд сходится, когда $|z|A < 1$, и
расходится, когда $|z|A>1$. Отсюда следует утверждение теоремы.
\end{proof}

\begin{theorem}
Ряды $\sum c_k z^k$ и $\sum k c_kz^k$ имеют одинаковые радиусы сходимости.
\end{theorem}
\begin{proof}
Положим $A_1 := \overline{\liml{k}}\sqrt[k]{|c_k|}$, $A_2 := \overline{\liml{k}}\sqrt[k]{k|c_k|}=
\overline{\liml{k}}\sqrt[k]{k}\cdot\sqrt[k]{|c_k|}$. Поскольку первый сомножитель имеет
предел, равный 1, по теореме о верхнем пределе произведения
можно перейти к произведению пределов. Тогда получим:
$A_2 = 1 \cdot \overline{\liml{k}}\sqrt[k]{|c_k|} = A_1$, что и требовалось доказать.
\end{proof}

Заметим, что поведение ряда на границе круга сходимости может быть неоднородным, е ряд
может сходиться в одной точке границы и расходиться
в другой. Исследуйте самостоятельно ряды $\sum z^k$, $\sum \frac{z^k}{k}$ и $\sum \frac{z^k}{k^2}$. Все они
имеют радиус сходимости, равный 1. Настоятельно рекомендуется изучить поведение этих рядов в точках $z=1$ и $z=-1$.

\subsection{Почленное дифференцирование степенных рядов. Теорема единственности}
Научимся дифференцировать функции вида $f\cln \Cbb \ra \Cbb$, чтобы иметь возможность
дифференцировать комплексные ряды. Как и в случае вещественнозначных функций, скажем, что
функция $f$ имеет производную в точке $z$, если $\exi$ конечный предел
$A=\liml{h \ra 0} \frac{f(z+h)-f(z)}{h}$. Тогда число $A$ называется производной $f$ в
точке $z$ и обозначается $f'(z)$.

Найдём $\hr{z^k}'$, где $k \in \N$. Рассмотрим
\begin{multline*}
\frac{(z+h)^k-z^k}{h}=\frac{1}{h} \cdot
\br{(z+h)^{k-1} + (z+h)^{k-2}z \spl (z+h)z^{k-2} + z^{k-1}}\br{(z+h)-z} =\\
=(z+h)^{k-1} + (z+h)^{k-2}z \spl (z+h)z^{k-2} + z^{k-1}.
\end{multline*}
В этой сумме ровно $k$ слагаемых. Поскольку $h$ в знаменателе
устранено, можно переходить к пределу при $h \ra 0$, тогда получим:
$\liml{h \ra 0} \frac{(z+h)^k-z^k}{h} = z^{k-1}+ z^{k-2}z \spl zz^{k-2} + z^{k-1} = kz^{k-1}$.

\begin{theorem}
Пусть ряд $f(z)=\suml{k=0}{\infty}c_kz^k$ имеет радиус сходимости $R>0$. Тогда $\fa z\cln |z| <R$
существует $f'(z)= \suml{k=0}{\infty} k c_k z^{k-1}$.
\end{theorem}
\begin{proof}
Мы уже показали, что ряды $\sum c_kz^k$ и $\sum kc_kz^k$ имеют одинаковые радиусы сходимости. Но
легко видеть, что если ряд $\sum kc_kz^k$ домножить на $\frac{1}{z}$, то радиус сходимости тоже
не поменяется. В самом деле, если ряд сходился при каком-то $z \neq 0$, то этот же ряд,
умноженный на константу $\frac{1}{z}$, тоже будет сходиться. Аналогично, если при каком-то~$z$
ряд расходится, то от умножения на ненулевую константу он не станет сходящимся. Но это и означает, что ряды
$\sum c_kz^k$ и $\sum kc_kz^{k-1}$ имеют одинаковые радиусы сходимости.

Фиксируем $z\cln |z| < \rho < R$. Возьмём $h\cln |z+h| < \rho<R$. Рассмотрим
\begin{multline*}
\frac{f(z+h)-f(z)}{h}= \frac{1}{h}\hr{ \sum c_k(z+h)^k-\sum c_kz^k} =\frac{1}{h}\sum c_k\hr{ (z+h)^k-z^k} =\\=
\frac{1}{h}\suml{k=0}{\infty}c_k\suml{m=0}{k-1}(z+h)^mz^{k-m-1}h=\suml{k=0}{\infty}c_k\suml{m=0}{k-1}(z+h)^mz^{k-m-1}.
\end{multline*}
Оценим нашу сумму по модулю, используя оценку для $|z|$. Имеем
$\hm{\suml{m=0}{k-1}(z+h)^mz^{k-m-1}} < \suml{m=0}{k-1}\rho^m\rho^{k-m-1}=k\rho^{k-1}$. Но
в силу того, что $\rho < R$, ряд $\sum c_kk\rho^k$ сходится. Тогда по первой теореме Абеля
ряд $\frac{f(z+h)-f(z)}{h}$ сходится абсолютно и равномерно при
$|z|\le q < \rho, |z+h| \le q < \rho$. Значит, в силу непрерывности можно перейти к
пределу $h \ra 0$ в каждом слагаемом, а потом просуммировать. Это и доказывает нашу теорему.
\end{proof}

\begin{theorem}[О единственности]
Даны ряды $\suml{k=0}{\infty} c_kz^k$, $\suml{k=0}{\infty}d_kz^k$. Пусть дана последовательность $\hc{z_n}$, причём
$z_n \neq 0, z_n \ra 0$. Если $\sum c_kz^k_n= \sum d_kz^k_n$ при $\fa n \in \N$, то $c_k=d_k$.
\end{theorem}
\begin{proof}
В силу равномерной сходимости рядов в некотором круге ($z_n \ra 0$) перейдём к пределу при
$n \ra \infty$ в каждом слагаемом ряда $\suml{k=0}{\infty} (c_k-d_k)z^k_n = 0$. Тогда
все слагаемые занулятся, кроме самых первых. Но тогда получим $c_0-d_0=0$. Тогда преобразуем
равенство, используя то, что $z_n \neq 0$ и $c_0=d_0$: вычеркнем первые слагаемые и поделим
на $z_n$. Получим тогда $\suml{k=1}{\infty} c_kz^{k-1}_n=\suml{k=1}{\infty} d_kz^{k-1}_n$. Тогда
опять пользуясь равномерной непрерывностью, переходим к пределу, и получаем, что $c_1=d_1$.
Ничто не мешает продолжить этот процесс. Так мы придём к утверждению теоремы.
\end{proof}

\subsection{Функции комплексной переменной $e^z$, $\sin z$, $\cos z$, $\sh z$, $\ch z$. Формулы Эйлера}

\begin{df}
Функция называется \emph{аналитической в точке} $z_0$, если её можно в некоторой
окрестности этой точки представить степенным рядом $f(z) = \sum c_k(z-z_0)^k$.
\end{df}

Как следует из доказанных выше теорем, наша функция будет в этой окрестности бесконечно дифференцируема, и
$f^{(m)}(z)=\suml{k=m}{\infty}c_kk(k-1)\sd (k-m+1)(z-z_0)^{k-m}$. Отсюда легко выводится, что
$f(z) =\suml{k=0}{\infty} \frac{f^{(k)}(z_0)}{k!}(z-z_0)^k$. Для этого достаточно заметить,
что ряд $f^{(m)}(z_0)$ содержит всего одно ненулевое слагаемое, равное
$c_m\cdot m\cdot (m\bw-1)\cdot(m-2)\sd 2\cdot 1= m!c_m$. Как обычно,
определим остаток ряда $R_n(z)= f(z)-\suml{k=0}{n}\frac{f^{(k)}(z_0)}{k!}(z-z_0)^k$. Таким
образом, аналитичность функции эквивалентна $R_n(z) \ra 0, n \ra \infty$.

Определим $e^z := \sum \frac{z^k}{k!}$. Как легко видеть, радиус сходимости такого ряда равен
$+\infty$. Тогда $e^z \in \Cb^\infty(\Cbb)$.
Покажем, что $e^z \cdot e^w = e^{z+w}$. Перемножим ряды $\sum \frac{z^k}{k!}$ и $\sum \frac{w^k}{k!}$,
используя теорему об умножении абсолютно сходящихся рядов и бином Ньютона:
$$e^z \cdot e^w=\suml{p=0}{\infty}\hr{ \suml{k=0}{p}\frac{1}{k!}z^k \frac{1}{(p-k)!}w^{p-k}} =
\suml{p=0}{\infty}\frac{1}{p!}\hr{ \suml{k=0}{p}\frac{p!}{k!(p-k)!}z^kw^{p-k}} =
\suml{p=0}{\infty}\frac{(z+w)^p}{p!}=e^{z+w}.$$

Теперь рассмотрим $e^{iz}=\sum \frac{1}{p!}(iz)^p=(1-\frac{z^2}{2!}+
\frac{z^4}{4!}-\dots)+i(z-\frac{z^3}{3!}+\frac{z^5}{5!}-\dots)$.
Это наблюдение позволяет определить тригонометрические функции комплексного аргумента:
$$\sin z := z-\frac{z^3}{3!}+\frac{z^5}{5!}-\dots, \quad \cos z :=1-\frac{z^2}{2!}+\frac{z^4}{4!}-\dots.$$
Тогда $e^{iz}=\cos z + i\sin z$. Кроме того, несложно показать, что
$\cos z = \frac{e^{iz}+e^{-iz}}{2}$, а $\sin z = \frac{e^{iz}-e^{-iz}}{2i}$. Это формулы Эйлера.
Для комплексной экспоненты есть ещё одно полезное представление. Пусть $z=x+iy, \; x,y \in \R$.
Тогда $e^z=e^{x+iy}=e^x\cdot e^{iy}=e^x(\cos y + i \sin y)$.

Совершенно аналогично действительным функциям, можно определить гиперболические функции комплексного аргумента:
$\sh z := \frac{e^z-e^{-z}}{2}$, $\ch z := \frac{e^z+e^{-z}}{2}$. Используя формулы Эйлера, получаем, что
$\cos (iz)=\ch z$ и $\sin (iz) = i\sh z$.

\subsection{Вторая теорема Абеля о рядах. Представление функций $\ln(1+x)$ и $\arctg x$ рядами Тейлора}

\begin{theorem}[Вторая теорема Абеля]
Ряд $\sum c_kz^k$ сходится в $w \neq 0$. Тогда он сходится равномерно на отрезке $\alpha w, \alpha \in [0,1]$,
и функция $f(\alpha) = \sum c_k (\alpha w)^k$ непрерывна по переменной $\alpha$ на $[0,1]$.
\end{theorem}
\begin{proof}
Покажем, что $f(\alpha)$ сходится равномерно на $[0,1]$. Действительно,
$\sum c_kw^k \cdot \alpha^k$ сходится равномерно по признаку Абеля:
$\sum c_kw^k$ сходится, а $\alpha^k$ монотонна и ограничена. Из равномерной сходимости следует непрерывность.
\end{proof}

Рассмотрим функцию $\ln(1+x)$ и покажем, что её ряд Тейлора сходится к ней на $(-1,1]$. Рассмотрим
ряд $\suml{m=0}{\infty}(-1)^mt^m=\frac{1}{1+t}$, что следует из формулы для геометрической
прогрессии. Таким образом, $\fa x \in [0,1)$ имеем равномерную сходимость ряда на $[-x,x]$.
Проинтегрируем: $\intl{0}{x}\frac{\,d t}{1+t}=\intl{0}{x}\sum (-1)^mt^m\,d t=\sum
\intl{0}{x}(-1)^mt^m\,d t$, отсюда $\ln(1+x)=\sum
(-1)^m\frac{x^{m+1}}{m+1}=\suml{k=1}{\infty}(-1)^{k+1}\frac{x^k}{k}$. В точке $x=1$ ряд сходится по
признаку Лейбница. Тогда по второй теореме Абеля ряд сходится равномерно на $[0,1]$ и к непрерывной
функции. Значит, в силу непрерывности функции $\ln (1+x)$, наше равенство верно на $(-1,1]$.

Разложим $\arctg x$ в ряд Тейлора. Для этого рассмотрим ряд
$\suml{m=0}{\infty}(-1)^mt^{2m}=\frac{1}{1+t^2}$. Эта формула справедлива на $(-1,1)$,
следовательно, на $[-x,x]$ имеем равномерную сходимость, если $x \in [0,1)$. Проинтегрируем:
$\intl{0}{x}\frac{\,d t}{1+t^2}\bw=\sum \intl{0}{x}(-1)^mt^{2m}\,d t=\sum
(-1)^m\frac{t^{2m+1}}{2m+1}\Bigl|_0^x$, откуда получаем, что $\arctg x = \sum
(-1)^m\frac{x^{2m+1}}{2m+1}$. Поскольку в точках $x=\pm 1$ ряд сходится по признаку Лейбница, он
равномерно сходится по второй теореме Абеля на $[0,1]$ и $[-1,0]$. Значит, он равномерно сходится и
на $[-1,1]$. Отсюда в силу непрерывности ряд сходится в точках $\pm 1$ к $\arctg x$.

\subsection{Представление степенной функции $(1+x)^m$ рядом Тейлора при $|x| < 1$}

Рассмотрим функцию $f(x) := (1+x)^m$. Будем считать, что $m \notin \Z_+$, в противном
случае $f$ многочлен. $f$ бесконечно дифференцируема при $x=0$. Очевидно, что $k$-я производная
этой функции имеет вид $f^{(k)}(x) \bw= m(m-1)\sd(m-k+1)(1+x)^{m-k}$. Отсюда $f^{(k)}(0) =
m(m-1)\sd(m-k+1)$. Тогда ряд Тейлора имеет вид $f(x)\bw=1 +
\suml{k=1}{\infty}\frac{m(m-1)\sd(m-k+1)}{k!} x^k$. Найдём радиус сходимости этого ряда по признаку
Даламбера, рассмотрев отношение соседних членов ряда: $\hm{\hr{
\frac{m(m-1)\sd(m-k)}{(k+1)!}x^{k+1}} : \hr{ \frac{m(m-1)\sd(m-k+1)}{k!}x^k}}
=\hm{\frac{m-k}{k+1}x} $. Видно, что при $k \ra \infty$ это отношение стремится к $|x|$. Значит,
при $|x| < 1$ ряд сходится, при $|x| > 1$ расходится, а $R=1$. Надо показать, что при $|x|<1$ ряд
сходится к $(1+x)^m$. Используем для этого формулу Тейлора с остаточным членом в интегральной
форме: $(1+x)^m=1 + \suml{k=1}{n}\frac{m(m-1)\sd(m-k+1)}{k!} x^k + R_n(x)$, где $R_n(x) =
\frac{1}{n!}\intl{0}{x}f^{(n+1)}(t)(x\bw-t)^n\,d t$. Покажем, что $\fa x\cln |x|<1$ имеем $R_n(x) \ra 0,
n \ra \infty$. В самом деле, $R_n(x)=\frac{m(m-1)\sd(m-n)}{n!}\intl{0}{x}\hr{ \frac{x-t}{1+t}}
^n(1\bw+t)^{m-1}\,d t$. Поскольку $t$ и $x$ имеют одинаковые знаки, $|t|\le |x|$, а $|x| < 1$, при $x
\neq 0$ имеет место оценка $\hr{ \frac{x-t}{1+t}} ^n \bw= x^n\hr{ \frac{1-\frac{t}{x}}{1+t}} ^n \le
x^n \hr{ \frac{1-\frac{|t|}{|x|}}{1-|t|}} ^n \le x^n \cdot 1^n = x^n$. Тогда $|R_n(x)| \le
\frac{|m(m-1)\sd(m-n)|}{n!}|x|^n \hm{\intl{0}{x}(1+t)^{m-1}\,d t} $. Последний множитель в этом
произведении есть константа при фиксированном $x$. С помощью признака Даламбера легко показать, при
$|x|<1$ ряд $\suml{1}{\infty}\frac{|m(m-1)\sd(m-k)|}{k!}|x|^k$ сходится. Поэтому
$\frac{|m(m-1)\sd(m-n)|}{n!}|x|^n \ra 0, n \ra \infty$. Отсюда следует, что $R_n(x) \ra 0$.

\subsection{Поведение остаточного члена формулы Тейлора для функции $(1+x)^m$ при $x=\pm 1$}

Положим $a_k := \frac{m(m-1)\sd(m-k+1)}{k!}x^k$. Мы хотим установить, при каких $m$ сходится ряд $\sum a_k$, если
$x =\pm 1$. Вначале рассмотрим абсолютную сходимость. Рассмотрим $\hm{\frac{a_{k+1}}{a_k}} = \hm{\frac{m-k}{k+1}}$.
Без ограничения общности можно считать, что $k > m$. Тогда $\hm{\frac{a_{k+1}}{a_k}} = \frac{k-m}{k+1}$, и по
признаку Раабе имеем $\liml{k} k\hr{ 1-\frac{k-m}{k+1}} =\liml{k} \frac{k}{k+1}(1+m) = 1+m$. Значит, при $m > 0$
имеет место абсолютная сходимость, а при $m < 0$ ряд будет абсолютно расходиться. Напомним, что $m=0$ нас не
интересует. Если $x=-1$, то $x^k$ меняет знак, но для всех достаточно больших $k$ произведение $m(m-1)\sd(m-k+1)$
тоже будет знакочередующимся, отсюда $a_k$ будут знакопостоянны. Поскольку для знакопостоянных рядов абсолютная
сходимость равносильна условной, получаем, что при $x=-1$ и $m < 0$ ряд расходится. Осталось выяснить вопрос
при $m < 0, x=1$. Имеем тогда $a_k =\frac{m(m-1)\sd(m-k+1)}{k!}$. Очевидно, при $m \le -1$ имеем $a_k \nra 0$.
Значит, при таких $m$ ряд расходится. Пусть теперь $m \in (-1,0)$. Тогда для достаточно больших $k$ получаем, что
$a_k$ знакочередующаяся последовательность, причём $|a_k|$ монотонно убывают. Покажем, что $|a_k| \ra 0$, тогда
сходимость будет вытекать из признака Лейбница. Имеем $|a_k| = \prodl{i=1}{k}\frac{i-m-1}{i}=
\prodl{i=1}{k}\hr{1-\frac{1+m}{i}} $. Здесь все множители положительны и меньше 1. Бесконечное
произведение такого вида сходится или расходится одновременно с рядом $\suml{i=1}{\infty} \frac{1+m}{i}$, а он,
очевидно, расходится. Значит, наше бесконечное произведение тоже расходится, причём к 0, в силу того, что
множители меньше 1. А это означает, что частные произведения, е наши $|a_k|$, стремятся к 0.

\subsection{Суммирование рядов методом Абеля Пуассона}

\begin{df}
Пусть ряд $\suml{0}{\infty} a_k x^k$ сходится поточечно на $[0,1)$. Пусть $\Bc := x \ra 1-0$. Если
$\exi \liml{\Bc} \sum a_k x^k\bw=S \in \overline{\R}$,
то говорят, что ряд $\sum a_k$ \emph{суммируется методом Абеля Пуассона к} $S$. Положим $S_n := \suml{0}{n}a_k$.
\end{df}

\begin{theorem}
Пусть $\fa x \in [0,1)$ сходится один из рядов $\sum a_kx^k$ (1) или $\sum S_kx^k$ (2). Тогда сходится и второй, и
$\sum a_k x^k = (1-x)\sum S_k x^k$.
\end{theorem}
\begin{proof}
Докажем тождество $\suml{0}{n} a_kx^k=(1-x)\suml{0}{n-1}S_kx^k + S_nx^n$. Воспользуемся преобразованием Абеля:
\begin{multline*}
\suml{0}{n} a_kx^k = a_0+\suml{1}{n}(S_k-S_{k-1})x^k=\suml{0}{n}S_kx^k-\suml{1}{n}S_{k-1}x^k=\\=
\suml{0}{n}S_kx^k-\suml{0}{n-1}S_kx^{k+1}=S_nx^n + \suml{0}{n-1}S_k(x^k-x^{k+1})=(1-x)\suml{0}{n-1}S_kx^k + S_nx^n.
\end{multline*}
Нам остаётся показать, что $S_nx^n \ra 0, n \ra \infty$.

\pt{1} Пусть (1) сходится. Фиксируем $x \in [0,1)$ и выберем $r\in(x,1)$. Ряд
$\sum a_kr^k$ сходится, значит, $\exi M\cln |a_kr^k|< M$. Отсюда $|a_k| < \frac{M}{r^k}$.
Значит, $|S_nx^n| \le x^n\suml{0}{n}|a_k| < Mx^n\suml{0}{n}\frac{1}{r^k}=M\frac{x^n}{r^n}\suml{0}{n}r^k<M
\frac{x^n}{r^n}\sum r^k = \frac{M}{1-r}\hr{\frac{x}{r}}^n \ra 0, n \ra \infty$.

\pt{2} Пусть (2) сходится. Тогда $S_nx^n \ra 0$. А нам только этого и надо. Теорема доказана.
\end{proof}

\begin{theorem}[Фробениуса]
Пусть ряд $\sum a_k$ суммируется методом (С,1) к $S \in \overline{\R}$. Тогда он
суммируется методом Абеля Пуассона к $S$.
\end{theorem}
\begin{proof}
\pt{1} Пусть $S \in \R$. Мы знаем, что если ряд суммируем по (С,1), то справедлива оценка $a_k = o(k)$.
Тогда $|a_k| \le C\cdot k$. По формуле Коши Адамара, $\frac{1}{R} = \overline{\liml{k}}\sqrt[k]{|a_k|} \le
\overline{\liml{k}}\sqrt[k]{C \cdot k}=1$, откуда $R\ge 1$. По предыдущей теореме
$\sum a_k x^k = (1-x)\sum S_k x^k$. Применим теперь это же равенство к ряду $\sum S_k x^k$. Получим
$\sum S_k x^k \bw= (1-x)\sum (k+1)\si_k x^k$. Отсюда $\sum a_k x^k = (1-x)^2\sum (k+1)\si_k x^k$.

Рассмотрим ряд $\suml{0}{} x^k = \frac{1}{1-x}$. Легко видеть, что для $x \in [0,1-\ep]$ имеет место равномерная
сходимость ряда его производных, откуда $\suml{1}{\infty}kx^{k-1}\bw=\frac{1}{(1-x)^2}$. Тогда
$\suml{0}{\infty}(k+1)x^k \bw= \frac{1}{(1-x)^2}$, е$(1-x)^2 \cdot \suml{0}{\infty}(k+1)x^k \bw= 1$.
Поскольку $\ep$ можно взять сколь угодно малым, это верно при $x \in[0,1)$.
Рассмотрим $\sum a_k x^k \bw- S \bw= (1\bw-x)^2\sum (k\bw+1)\si_k x^k \bw- 1\cdot S=(1\bw-x)^2
\sum (k\bw+1)(\si_k\bw-S) x^k$. Тогда
$\fa \ep \bw>0 \exi N\cln \fa k \bw>N$ имеем $|\si_n-S| < \ep$. Тогда
$\hm{\sum a_k x^k -S} \bw\le \bbm{(1-x)^2\suml{0}{N}(k+1)(\si_k-S)x^k} +\bbm{(1-x)^2\suml{N+1}{\infty}(k+1)(\si_k -S) x^k}
\bw\le (1-x)^2\bbm{\suml{0}{N}(k+1)(\si_k-S)x^k} +(1-x)^2\ep\bbm{\suml{N+1}{\infty}(k+1)x^k}$. Первое слагаемое можно
сделать малым за счёт выбора $x$, близких к 1, ибо там конечное количество слагаемых. Второе слагаемое оценивается
сверху числом $\ep$, поскольку $\suml{0}{\infty}(k+1)x^k = \frac{1}{(1-x)^2}$.

\pt{2} Пусть $S=+\infty$. В этом случае мы предполагаем, что ряд $\sum a_k x^k$ сходится при $x \in [0,1)$.
Тогда сходятся ряды $\sum S_k x^k$ и $\sum (k+1)\si_k x^k$. Кроме того, отсюда следует, что справедливо равенство
$\sum a_k x^k \bw= (1\bw-x)^2\sum (k\bw+1)\si_k x^k$.

Поскольку $\si_k \bw\ra \infty$, получаем $\fa M\bw>0 \exi N\cln \fa k\bw> N$ имеем $\si_k \bw> M$. Отсюда
$\sum a_k x^k = (1-x)^2\sum(k+1)\si_kx^k \bw\ge
(1-x)^2\hr{ \suml{0}{N}(k+1)\si_kx^k + \suml{N+1}{\infty}(k+1)Mx^k} =\lcomm$ добавим во
вторую сумму начальные слагаемые, а из первой их выкинем для компенсации
$\rcomm =(1-x)^2\suml{0}{N}(k+1)(\si_k\bw-M)x^k+M(1-x)^2\suml{0}{\infty}(k+1)x^k =
(1-x)^2\suml{0}{N}(k+1)(\si_k\bw-M)x^k+M \ge M \bw- (1\bw-x)^2\suml{0}{N}(k+1)|\si_k\bw-M|> \frac{M}{2}$
в силу того, что здесь лишь конечное число слагаемых, и, выбирая $x$ близким к 1, можно сделать
вычитаемое маленьким. Тем самым $\liml{\Bc}\sum a_k x^k = +\infty$.
\end{proof}

\begin{note}
По существу, доказанная теорема утверждает лишь то, что метод Абеля Пуассона регулярен.
В случае, когда $\exi x_0 \in [0,1)\cln \sum a_kx_0^k$ расходится, а ряд $\sum a_k$ суммируется к $+\infty$,
теорема Фробениуса ничего не утверждает. Покажем, что если $S_k \ra +\infty$, то $\sum a_kx_0^k = +\infty$.
В самом деле, $S_k>0$ для всех $k$, начиная с некоторого, поэтому частичные суммы ряда $\sum S_kx_0^k$
возрастают. Но этот ряд не может сходиться, поскольку в силу тождества
$\suml{0}{n} a_kx_0^k=(1-x_0)\suml{0}{n-1}S_kx_0^k + S_nx_0^n$ сходился бы и ряд $\sum a_k x_0^k$. Поэтому
$\sum S_k x_0^k$ расходится к $+\infty$, тогда в силу тождества $\sum a_k x^k_0 = +\infty$. Поэтому
метод Абеля Пуассона доопределяется так: если $\sum a_k = +\infty$ и $\exi x_0 \in [0,1)$,
для которого $\sum a_k x_0^k$ расходится, то исходный ряд по определению
\emph{суммируется методом Абеля Пуассона к} $+\infty$.
\end{note}

Покажем теперь, что этот метод сильнее, чем метод (С,1). Рассмотрим ряд $\sum (-1)^k(k+1)$.
Он не суммируется методом (С,1), поскольку для него не выполнено $a_k=o(k)$. Мы уже вывели
тождество $\sum (k+1)x^k = \frac{1}{(1-x)^2}$. Заменим в нём $x$ на $-x$,
получим $\sum (-1)^k(k+1)x^k=\frac{1}{(1+x)^2}$. Тогда, применяя суммирование методом Абеля Пуассона, получаем
$\liml{\Bc} \sum (-1)^k(k+1)x^k=\frac{1}{(1+1)^2}= \frac{1}{4}$.

\section{Интегралы, зависящие от параметра}

\subsection{Собственные интегралы, зависящие от параметра}

\begin{df}
Пусть на $[a,b]$ заданы функции $\ph(x)$ и $\psi(x)$, такие, что $\ph \le \psi$. Положим $G :=
[a,b]\bw\times\bs{\ph(x),\psi(x)}$. Рассмотрим $f\cln G \ra \Cbb$. Пусть $\fa x \in [a,b]$ имеем $f(x,y)
\in\Rb\hs{\ph(x), \psi(x)}$. Положим $I(x) \bw{:=} \intl{\ph(x)}{\psi(x)}f(x,y)\,d y$. Такой интеграл
называют \emph{зависящим от параметра} $x$.
\end{df}

\begin{theorem}
Пусть $f \in\Cb[a,b]\times[c,d]$. Тогда $J(x,u,v) := \intl{u}{v}f(x,y)\,d y \in\Cb
[a,b]\times[c,d]^2$.
\end{theorem}
\begin{proof}
Рассмотрим достаточно малые приращения $\dx, \du, \Dv$, чтобы все функции имели смысл. Заметим,
что в силу непрерывности $f$ ограничена некоторым числом $L$. Используя аддитивность интеграла, получаем
\begin{multline*}
\bm{J(x+\dx,u+\du,v+\Dv)-J(x,u,v)}=\bbm{\intl{u+\du}{v+\Dv}f(x+\dx,y)\,d y-\intl{u}{v}f(x,y)\,dy} \le\\
\le\bbm{\intl{u+\du}{u}f(x+\dx,y)\,d y} + \bbm{\intl{u}{v}\br{f(x+\dx,y)\bw-f(x,y)}\,d y} +
\bbm{\intl{v}{v+\Dv}f(x+\dx,y)\,d y} \le\\\le
\bbm{\intl{u+\du}{u}|f(x+\dx,y)|\,d y} +
\bbm{\intl{u}{v}\hm{f(x+\dx,y)-f(x,y)}\,d y} + \bbm{\intl{v}{v+\Dv}|f(x+\dx,y)|\,d y} \le\\\le L|\du|
+ \bbm{\intl{u}{v}\hm{\omega\br{f, |\dx|}}\,d y} + L|\Dv| \le L\br{|\du| + |\Dv|} +
|u-v|\omega\br{f,|\dx|} \ra 0
\end{multline*}
при $\dx, \Dv, \du \ra 0$ в силу равномерной непрерывности $f$ на
компакте $[a,b]\times[c,d]$. Это и доказывает непрерывность $J$.
\end{proof}

\begin{theorem}
Пусть $\ph, \psi \in\Cb[a,b]$ и $f \in\Cb(G)$. Тогда $I(x) \in\Cb[a,b]$.
\end{theorem}
\begin{proof}
Найдем $c,d$, для которых $G \subs [a,b]\times[c,d]$. Доопределим функцию $f$: положим
$$f(x,y) := \case{f\br{x,\ph(x)}, & y \in \big[c,\ph(x)\big), \\ f\br{x,\psi(x)}, & y \in \big(\psi(x), d\big].}$$
Теперь $f$ определена на всём множестве $[a,b]\times[c,d]$ и непрерывна на нём. Тогда
$I(x) = J\br{x,\ph(x),\psi(x)} \in\Cb[a,b]$ по теореме о композиции
непрерывных функций (по предыдущей теореме $J \in\Cb[a,b]\times[c,d]^2$).
\end{proof}

Обозначим базу $x \ra x_0$ через $\Bc$ для краткости. Мы знаем, что непрерывность
функции $I(x)$ в точке $x_0$ равносильна существованию предела
$\liml{\Bc} I(x) = I(x_0)$. Тогда по нашей теореме получаем
$$\liml{\Bc} \intl{\ph(x)}{\psi(x)}f(x,y)\,d y=\intl{\ph(x_0)}{\psi(x_0)}f(x_0,y)\,d y
=\intl{\liml{\Bc}\ph(x)}{\liml{\Bc}\psi(x)}\liml{\Bc}f(x,y)\,d y.$$ Таким образом, в условиях
нашей теоремы возможен переход к пределу под знаком интеграла.

\begin{theorem}
Пусть $\Xc \subs \R$, и $x_0$ предельная точка $\Xc$. Пусть $\ph, \psi\cln \Xc \ra \R$. Обозначим
через $\Bc$ базу $x \overset{\Xc}{\ra} x_0$. Пусть $\liml{\Bc}\ph(x)=a, \liml{\Bc}\psi(x)=b$. Пусть
$\fa x \in \Xc$ имеем $c \le \ph \le \psi \le d$. Пусть $f\cln \Xc\times[c,d]\ra \R$ и существует
$\intl{c}{d}f(x,y)\,d y$. Пусть $f(x,y)\convu{[c,d]}g(y)$ при базе $\Bc$. Тогда $g(y)\in\Rb[a,b]$
и $\liml{\Bc}\intl{\ph(x)}{\psi(x)}f(x,y)\,d y=\intlab g(y)\,d y$.
\end{theorem}
\begin{proof}
Мы знаем, что если последовательность функций сходится равномерно, то можно переходить к пределу
под знаком интеграла. Рассмотрим произвольную последовательность $\hc{x_n}$ из множества $\Xc$,
сходящуюся к $x_0$. Тогда получаем $\liml{n}\intlab f(x_n,y)\,dy\bw=\intlab g(y)\,dy$. Рассмотрим
$\intl{\ph(x_n)}{\psi(x_n)}f(x_n,y)\,dy\bw=\intlab f(x_n,y)\,dy\bw+\intl{\ph(x_n)}{a}f(x_n,y)\,dy\bw+
\intl{b}{\psi(x_n)}f(x_n,y)\,dy$. Имеем $\intlab f(x_n,y)\,dy \ra \intlab g(y)\,d y, n \ra
\infty$, а оставшиеся два интеграла можно оценить: в силу равномерной сходимости
$f(x,y)\convu{[c,d]}g(y)$ при $\Bc$ и интегрируемости $f(x_n,y)$, функции $f(x_n,y)$ равномерно
ограничены числом $M$. Следовательно, $\bbm{\intl{\ph(x_n)}{a}f(x_n,y)\,d y} \le M\bm{a-\ph(x_n)} \ra
0, n \ra \infty$. Аналогично оценивается третий интеграл. Осталось воспользоваться эквивалентностью
предела по Коши и Гейне.
\end{proof}

\begin{theorem}[Формула Лейбница]
Пусть $\ph, \psi \in\Db[a,b]$, а $f(x,y), f'_x(x,y) \in\Cb[a,b]\times[c,d]$. Тогда $I(x)
\bw\in\Db[a,b]$ и $I'(x)=\intl{\ph(x)}{\psi(x)}f'_x(x,y)\,d y +
f\br{x,\psi(x)}\psi'(x)-f\br{x,\ph(x)}\ph'(x)$.
\end{theorem}
\begin{proof}
\pt{1} Пусть сначала $\ph(x)\equiv c, \psi(x) \equiv d$, е пределы интегрирования не зависят от
$x$. Покажем, что $\liml{\dx\ra 0}\frac{I(x+\dx)-I(x)}{\dx}=\intl{c}{d}f'_x(x,y)\,d y$. В самом
деле, $\frac{I(x+\dx)-I(x)}{\dx}=\intl{c}{d}\frac{f(x+\dx,y)-f(x,y)}{\dx}\,d y = \lcomm$ ФКПЛ
$\rcomm = \intl{c}{d}f'_x(x\bw+\theta\dx,y)\,d y$. Поскольку $\hm{f'_x(x\bw+\theta\dx,y)-f'_x(x,y)} \le
\omega\br{f'_x, \theta|\dx|} \le \omega\br{f'_x, |\dx|} \ra 0, \dx \ra 0$. Но если функция $\ra 0$,
то и её интеграл $\ra 0$, отсюда выводим обещанное равенство. Тем самым частный случай доказан.

\pt{2} Рассмотрим $J(x,u,v):=\intl{u}{v}f(x,y)\,d y$. Покажем, что $J \in\Db[a,b]\times[c,d]^2$.
Мы знаем, что если частные производные по всем переменным непрерывны, то функция будет
дифференцируемой. Поскольку $J'_v=f(x,v)$, а $J'_u=-f(x,u)$, значит, $J'_v$ и $J'_u$ будут
непрерывными функциями. Поскольку $J'_x=\intl{u}{v}f'_x(x,y)\,d y$, она тоже будет непрерывной.
Отсюда $I(x)=J\br{x,\ph(x),\psi(x)}$ дифференцируема по теореме о производной композиции функций.
Продифференцируем её:
$I'(x)=J'_x\br{x,\ph(x),\psi(x)}+J'_u\br{x,\ph(x),\psi(x)}\ph'(x)+J'_v\br{x,\ph(x),\psi(x)}\psi'(x)\bw=
\intl{\ph(x)}{\psi(x)}f'_x(x,y)\,dy\bw-f\br{x,\ph(x)}\ph'(x)\bw+f\br{x,\psi(x)}\psi'(x)$, что и требуется.
\end{proof}

\begin{theorem}
Пусть $f\in\Cb[a,b]\times[c,d]$. Тогда $\intlab \intl{c}{d}f(x,y)\,d y\,d x=\intl{c}{d}\intlab
f(x,y)\,d x\,d y$.
\end{theorem}
\begin{proof}
Само существование интегралов следует из непрерывности $f$. Рассмотрим
$F(x,y):=\intl{c}{y}f(x,t)\,d t$. Она непрерывна, кроме того, непрерывна $F'_y(x,y)=f(x,y)$.
Рассмотрим $\intlab F(x,y)\,d x$. По предыдущей теореме его можно продифференцировать:
$\bbr{\intlab F(x,y)\,d x}_y'=\intlab F'_y(x,y)\,d x=\intlab f(x,y)\,d x$. По ФНЛ имеем
$\intl{c}{d}\intlab f(x,y)\,d x\,d y\bw=\intlab F(x,d)\,d x-\intlab F(x,c)\,d x$. Теперь,
поскольку $F(x,c)=0$, осталось подставить в полученную формулу выражение для $F$.
\end{proof}

\begin{note}
В этой теореме требование $f\in\Cb[a,b]\times[c,d]$ можно ослабить, однако если ничего
не требовать от $f$, то формула может не сработать. Рассмотрим $f(x,y):=0$ на
$[0,1]\times(0,1]$ и $f(x,0):=\Dc(x)$. Для такой функции один из внутренних интегралов не существует.
\end{note}

\subsection{Равномерная сходимость несобственных интегралов, зависящих от параметра}

\begin{df}
Пусть $f\cln\Xc\times[c,d)\ra\R$, где $d\in\ol{\R}$. Рассмотрим несобственный интеграл
$\intl{c}{d}f(x,y)\,dy\bw{:=}\liml{\de\ra d-0}\intl{c}{\de}f(x,y)\,d y$. Интеграл называется
\emph{равномерно сходящимся} по $\Xc$, если $\fa \ep >0 \exi \De < d\cln \fa x \in \Xc, \fa \de \bw\in
(\De, d)$ имеем $\bbm{\intl{\de}{d}f(x,y)\,d y} \bw< \ep$. В этом определении
предполагается, что $\fa x \in \Xc$ существует $\intl{c}{d}f(x,y)\,dy$.
\end{df}

\begin{theorem}[КК]
Пусть $f\cln\Xc\times[c,d)$. Тогда $\intl{c}{d}f(x,y)\,d y \convu{\Xc}$ равносильна условию Коши:
$\fa \ep \exi \De<d\cln \fa x \bw\in \Xc, \fa \de_1,\de_2 \in (\De,d)$ имеем
$\bbm{\intl{\de_1}{\de_2}f(x,y)\,d y} < \ep$.
\end{theorem}
\begin{proof}
Сразу следует из определения несобственных интегралов и КК для функций.
\end{proof}

\begin{theorem}[Признак Вейерштрасса]
Если $f,g\cln\Xc\times[c,d)$, $|f|\le g$ и $\intl{c}{d}g(x,y)\,d y\convu{\Xc}$, то
$\intl{c}{d}f(x,y)\,d y \convu{\Xc}$.
\end{theorem}
\begin{proof}
Применим КК: $\fa \ep >0 \exi \De<d\cln \fa x \in \Xc, \fa \de_1,\de_2 \in(\De,d)$ имеем
$\bbm{\intl{\de_1}{\de_2}g(x,y)\,d y}<\ep$. Отсюда $\bbm{\intl{\de_1}{\de_2}f(x,y)\,dy}\bw\le
\bbm{\intl{\de_1}{\de_2}|f(x,y)|\,d y}\le\bbm{\intl{\de_1}{\de_2}g(x,y)\,d y}<\ep$.
Следовательно, по КК имеем $\intl{c}{d}f(x,y)\,d y \convu{\Xc}$.
\end{proof}

\begin{ex}
Рассмотрим $\intl{-\infty}{+\infty}\frac{\,d y}{1+(y-x)^2}$. При каждом фиксированном $x$ можно
сделать замену и убедиться, что значение интеграла не зависит от $x$. Однако он не сходится равномерно
на $\R$, хотя сходится равномерно $\fa [A,B] \bw\subs \R$. Мораль: независимость значения интеграла
от параметра не гарантирует равномерную сходимость.
\end{ex}

\begin{theorem}[Признак Абеля Дирихле]
Пусть $f,g\cln\Xc\times[c,d)$. Пусть $\fa \de < d \exi \intl{c}{\de}f(x,y)\,d y$, а $g(x,y)$
монотонна по $y$ при каждом $x \in \Xc$. Тогда $\intl{c}{d}f(x,y)g(x,y)\,d y\convu{\Xc}$, если
выполнена хотя бы одна пара условий $(D_1),(D_2)$ или $(A_1),(A_2)$:

$(D_1)\quad \exi B\cln \fa x \in \Xc, \fa \de \in [c,d)$ имеем $\bbm{\intl{c}{\de}f(x,y)\,d y} \le B
\qquad (D_2)\quad g(x,y)\convu{\Xc}0$ при $y \ra d-0$,

$(A_1)\quad \intl{c}{d}f(x,y)\,d y\convu{\Xc}\qquad (A_2)\quad g(x,y)$ равномерно ограничена
числом $M$ на $\Xc\times[c,d)$.
\end{theorem}
\begin{proof}
Используя КК и применяя вторую теорему о среднем, получаем, что $\exi \xi(x) \in (\de_1,\de_2)$,
для которой $\intl{\de_1}{\de_2}f(x,y)g(x,y)=g(x,\de_1)\intl{\de_1}{\xi(x)}f(x,y)\,d y+
g(x,\de_2)\intl{\xi(x)}{\de_2}f(x,y)\,d y$. Если выполнены условия $(D_1),(D_2)$, то это выражение
оценивается по модулю числом $2B\br{|g(x,\de_1)|+|g(x,\de_2)|}$, а его можно сделать маленьким при
$y\bw\ra d-0$ в силу условия $(D_2)$. Если же выполнены условия $(A_1),(A_2)$, то его можно оценить по
модулю числом $M\hr{\bbm{\intl{\de_1}{\xi(x)}f(x,y)\,d y}+\bbm{\intl{\xi(x)}{\de_2}f(x,y)\,d
y}}$, а его можно сделать маленьким в силу условия $(A_1)$.
\end{proof}

\begin{theorem}
Пусть $f\cln\Xc\times[c,d)$ и $\intl{c}{d}f(x,y)\,d y \convu{\Xc}$. Пусть $x_0$ предельная точка
$\Xc$. Обозначим базу $x \overset{\Xc}{\ra} x_0$ через $\Bc$. Пусть $\fa [c,\de]$, где $\de < d$,
имеем $f(x,y)\convu{[c,\de]}\ph(y)$ при базе $\Bc$. Тогда $\liml{\Bc}\intl{c}{d}f(x,y)\,d y=
\intl{c}{d}\ph(y)\,d y\bw=\intl{c}{d}\liml{\Bc}f(x,y)\,d y$.
\end{theorem}
\begin{proof}
По теореме о равномерной сходимости интегрируемых функций получаем, что $\fa \de \in [c,d)$ существует
$\intl{c}{\de}\ph(y)\,d y$. Воспользуемся КК: $\fa \ep > 0 \exi \De \in [c,d)\cln \fa \de_1,\de_2 \in
(\De,d), \fa x\in\Xc$ имеем $\bbm{\intl{\de_1}{\de_2}f(x,y)\,d y}< \ep$. Мы знаем, что
$\intl{\de_1}{\de_2}f(x,y)\,d y$ существует. В силу равномерной сходимости можно перейти к пределу
по $\Bc$ (для собственных интегралов мы уже всё доказали), тогда получим
$\bbm{\intl{\de_1}{\de_2}\ph(y)\,d y}\le \ep$. Тем самым доказано существование интеграла
предельной функции.

Теперь, в силу сходимости соответствующих интегралов, $\fa \ep > 0 \exi \De\cln
\bbm{\intl{\De}{d}\ph(y)\,d y}<\ep$ и $\fa x \in\Xc$ имеем $\bbm{\intl{\De}{d}f(x,y)\,d y}<\ep$.
Отсюда получаем $\bbm{\intl{c}{d}f(x,y)\,d y-\intl{c}{d}\ph(y)\,d y}\le
\bbm{\intl{c}{\De}f(x,y)\,d y - \intl{c}{\De}\ph(y)\,d y}+ \bbm{\intl{\De}{d}f(x,y)\,d
y}\bw+\bbm{\intl{\De}{d}\ph(y)\,d y} < \bbm{\intl{c}{\De}f(x,y)\,d y - \intl{c}{\De}\ph(y)\,d y} +
2\ep$. Оставшийся интеграл $\intl{c}{\De}\br{f(x,y)-\ph(y)}\,d y$ можно сделать маленьким в силу
сходимости $f(x,y)\convu{\Xc}\ph(y)$ при базе $\Bc$. Следовательно, наши два интеграла равны.
\end{proof}

\begin{note}
Когда мы доказываем непрерывность функции на множестве, мы доказываем непрерывность в
каждой точке этого множества, и потому нас
интересует поведение этой функции только в некоторой окрестности этой точки.
\end{note}

\begin{imp}
Пусть $f \in\Cb[a,b]\times[c,d)$, где $a,b \in \ol{\R}$. Пусть $\intl{c}{d}f(x,y)\convu{[a,b]}$.
Тогда $\intl{c}{d}f(x,y)\,d y \in\Cb[a,b]$.
\end{imp}
\begin{proof}
Пусть сначала $[a,b]$ отрезок. Рассмотрим $x_0 \in [a,b]$. Имеем
$f(x,y)\in\Cb[a,b]\times[c,\de]$ для $\de < d$. Более того, она равномерно непрерывна на этом
множестве. Пусть $\Bc := x \overset{[a,b]}{\ra} x_0$. Отсюда $f(x,y)\convu{[c,\de]}f(x_0,y)$ при
базе $\Bc$. По предыдущей теореме $\liml{\Bc}\intl{c}{d}f(x,y)\,d y=\intl{c}{d}f(x_0,y)\,d y$.

Если же $[a,b]$ не отрезок, то рассмотрим $\fa x_0 \in [a,b]$ и возьмём отрезок
$[p,q] \subs [a,b]\cln x_0 \in (p,q)$. А для отрезка всё доказано.
\end{proof}

\begin{theorem}
Пусть $f\in\Cb[a,b]\times[c,d)$. Пусть $\intl{c}{d}f(x,y)\,d y\convu{[a,b]}$. Тогда $\intlab
\intl{c}{d}f(x,y)\,d y\,d x = \intl{c}{d}\intlab f(x,y)\,d x\,d y$.
\end{theorem}
\begin{proof}
Мы знаем, что $\fa \De \in [c,d)$ имеется свойство $\intlab \intl{c}{\De}f(x,y)\,d y\,d x =
\intl{c}{\De}\intlab f(x,y)\,d x\,d y~~(*)$. Рассмотрим функцию $g(x,\De) :=
\intl{c}{\De}f(x,y)\,d y$. Имеем $g(x,\De) \in\Cb[a,b]$ и
$g(x,\De)\convu{[a,b]}\intl{c}{d}f(x,y)\,d y$ при $\De \ra d-0$. Используя свойства собственных
интегралов, перейдём к пределу при $\De \ra d-0$ в интеграле $\intlab g(x,\De)\,d x$. Но поскольку
по определению $g$ имеем $\intlab g(x,\De)\,d x=\intlab \intl{c}{\De}f(x,y)\,d y\,d x$, из
существования предела следует возможность перехода к пределу в равенстве $(*)$.
\end{proof}

\begin{theorem}
Пусть $f\in\Cb[a,b)\times[c,d)$, и $\fa \xi < b$ имеем $\intl{c}{d}f(x,y)\,d y
\convu{[a,\xi]}$, а $\fa \De < d$ имеем $\intlab f(x,y)\,d x \convu{[c,\De]}$. Пусть сходится
один из интегралов $\intl{c}{d}\intlab |f(x,y)|\,d x\,d y$ или $\intlab \intl{c}{d}|f(x,y)|\,d
x\,d y$. Тогда имеет место равенство $\intl{c}{d}\intlab f(x,y)\,d x\,d y=\intlab \intl{c}{d}f(x,y)\,d y\,d x$.
\end{theorem}
\begin{proof}
Пусть сходится первый из интегралов. Тогда по предыдущей теореме
$\intl{c}{d}\intl{a}{\xi}f(x,y)\,d x\,d y=\intl{a}{\xi}\intl{c}{d}f(x,y)\,d y\,d x$. Рассмотрим
функцию $h(\xi,y) := \intl{a}{\xi}f(x,y)\,d x$. Имеем $|h(\xi,y)| \le \intl{a}{\xi}|f(x,y)|\,d x
\le \intlab |f(x,y)|\,d x$. Следовательно, $\intl{c}{d}h(\xi,y)\,d y \convu{[a,b]}$ по признаку
Вейерштрасса. Значит, возможен переход к пределу $\xi \ra b-0$ под знаком этого интеграла.
\end{proof}

\begin{theorem}
Пусть $f(x,y), f'_x \in\Cb[a,b]\times[c,d)$. Пусть $\fa x \in [a,b] \intl{c}{d}f(x,y)\,d y$
сходится, а $\intl{c}{d}f'_x(x,y)\,d y\convu{[a,b]}$. Тогда $I(x) := \intl{c}{d}f(x,y)\,d y
\in\Cb^1[a,b]$ и $I'(x)=\intl{c}{d}f'_x(x,y)\,d y$.
\end{theorem}
\begin{proof}
Имеем $\frac{I(x+h)-I(x)}{h}=\intl{c}{d}\frac{f(x+h,y)-f(x,y)}{h}\,d y = \lcomm$ по ФНЛ $\rcomm =
\frac{1}{h}\intl{c}{d}\intl{x}{x+h}f'_x(t,y)\,d t\,d y =
\frac{1}{h}\intl{x}{x+h}\intl{c}{d}f'_x(t,y)\,d y\,d t$. Здесь можно поменять порядок
интегрирования в силу равномерной сходимости интеграла $\intl{c}{d}f'_x(x,y)\,d y$. Тогда по
теореме о производной интеграла с переменным верхним пределом получаем
$\liml{h\ra0}\frac{1}{h}\intl{x}{x+h}\intl{c}{d}f'_x(t,y)\,d y\,d t = \intl{c}{d}f'_x(x,y)\,d
y$, а из непрерывности производной $f'_x$ и равномерной сходимости $\intl{c}{d}f'_x(x,y)\,d y$
следует непрерывность $I'(x)$.
\end{proof}

\subsection{$\Ga$-функция и $\Be$-функция}
\begin{note}
Этот раздел не является фрагментом лекций С.\,А.\,Теляковского. То, что было рассказано на лекциях,
содержало огромное количество выкладок и занимало много места. Попытка найти более простые обоснования
привела к книгам Зорича и Фихтенгольца. Единственный недостаток этого раздела хаотичность и недостаточно
подробное изложение, чего нельзя сказать о первоисточниках. Причина тому полное отсутствие времени
перед экзаменом, но всё это можно изложить абсолютно строго. Есть надежда,
что в последующих изданиях книги лектора всё будет написано как надо, поскольку после экзамена
автор данного документа сообщил лектору о наличии в природе более простых доказательств.
\end{note}

Пусть $f(x,t)=t^{x-1}e^{-t}$. $\Ga(x) := \intloi f \,d t$. Определён при $x>0$ и $\convu{[a,b]}$,
где $[a,b] \subs (0,+\infty)$, поэтому непрерывен. Поскольку $f^{(m)} = t^{x-1}e^{-t}\ln^m t$, а
интеграл от такой функции тоже $\convu{[a,b]}$, получаем возможность формального дифференцирования.
Имеем $\Ga''(x) > 0$, поэтому $\Ga$ выпукла. Пихая $e^{-t}$ под дифференциал и интегрируя по
частям, получаем формулу приведения $\Ga(x+1)=x\Ga(x)$. Имеем $\Ga(1)=1$, поэтому $\Ga(n+1) = n!$.
В силу непрерывности при $x\ra 0$ имеем $\Ga(x)=\frac{\Ga(x+1)}{x}\sim \frac{1}{x}$.

Изучим поведение при $x\ra +\infty$. Замена $t=x(1+s)$, тогда
$\Ga(x+1)\bw=x^{x+1}e^{-x}\intl{-1}{+\infty}e^{-x\br{s-\ln(1+s)}}\,ds\bw=x^{x+1}e^{-x}\cdot I$. В силу
хороших свойств, можно сделать замену $\frac{u^2}{2}=s-\ln(1+s)$, причём $\sgn u := \sgn s$, тогда
это гладкая биекция между $s\in(-1,+\infty)$ и $u\in\R$. Дифференцируя, получаем
$u=s'_u-\frac{s'_u}{1+s}$, откуда $s'_u=\frac{u}{s}+u$. ОЧФЛ с центром в~0: $\dots +
\frac{f^{(m)}(\theta x)}{m!}x^m$, поэтому $\ln(1+s)=s -\frac{s^2}{2(1+\theta s)^2}$, значит,
$\frac{u^2}{2}=s\bw-\ln(1+s)\bw=\frac{s^2}{2(1+\theta s)^2}$, и в силу $1+\theta s > 0$ и соглашения о
знаках извлекаем корень и получаем $u=\frac{s}{1+\theta s}$. Выражаем отсюда $s$ и подставляем в
$s'_u$, получаем $s'_u = 1 + (1-\theta)u = 1 + \mu u$, где $\mu \in (0,1)$. C производной в нуле
всё тоже хорошо, поскольку её значение в 0 есть по (6.2.10) предел производных при $s\ra0$.
Заметим, что $\theta, \mu$ есть некоторые функции от $s$, кроме того, имеет место $|s'_u-1| \le
|u|$.

Переходя к $u$, получаем $I \bw= \intlii e^{-x u^2/2}\,du \bw+ r(s)$, где $|r| \bw\le \intlii
e^{-x u^2/2}|u|\,du\bw= 2\intloi e^{-x u^2/2} u\,d u = 2\intloi e^{-xv}\,d v \bw=\frac{2}{x}$. Для первого
интеграла имеем $\intlii e^{-x u^2/2}\,d u= \sqrt{\frac{2}{x}}\intlii e^{-v^2}\,d v
=\sqrt{\frac{2\pi}{x}}$, поскольку последний интеграл есть $\Ga\hr{\frac{1}{2}}=\sqrt{\pi}$. Таким
образом, $I=\sqrt{\frac{2\pi}{x}} + R$, где $|R| \le \frac{2}{x}$. Отсюда
$\Ga(x+1)=\sqrt{2\pi x}\frac{x^x}{e^x}(1+T)$, где $|T| \le \sqrt{\frac{2}{\pi x}}$.

$\Be(p,q) := \intl{0}{1}t^{p-1}(1-t)^{q-1}\,d t$. Определена при $p,q>0$ и интеграл сходится
равномерно при $p \ge p_0 > 0$ и $q \ge q_0 > 0$. Свойство $\Be(p,q)=\Be(q,p)$ очевидно получается
с помощью замены.

Докажем \emph{формулу приведения}: $\Be(p,q+1)=\lcomm$ по частям $\rcomm =\frac{p}{q}\Be(p+1,q)$.
Далее, растаскиваем множитель $1-t$ в $\Be(p,q+1)=\Be(p,q)-\Be(p+1,q)$. Отсюда получаем
два уравнения и выражаем что надо.

\emph{Связь} $\Be$ и $\Ga$. Пусть $x=ty,t>0$, тогда $\frac{\Ga(a)}{t^a}=\intloi y^{a-1}e^{-ty}\,d
y$. Заменяя $a$ на $a+b$ и $t$ на $1+t$, получаем $\frac{\Ga(a+b)}{(1+t)^{a+b}}\bw-\intloi
y^{a+b-1}e^{-(1+t)y}\,d y$. Заметим, что $\Be$ приводится заменой $t=\frac{1}{1+s}$ к виду
$\intloi\frac{s^{a-1}\,d s}{(1+s)^{a+b}}$. Домножим полученное тождество на $t^{a-1}$ и
проинтегрируем по $t$. Тогда $\Ga(a+b)\Be(a,b)=\intloi \intloi t^{a-1}y^{a+b-1}e^{-(1+t)y}\,d y
\,d t=\lcomm$ переставим интегралы $\rcomm =\intloi y^{a+b-1}e^{-y}\intloi t^{a-1}e^{-ty}\,d t
\,d y=\intloi y^{a+b-1}e^{-y}\frac{\Ga(a)}{y^a}\,d y = \Ga(a)\intloi y^{b-1}e^{-y}\,d
y=\Ga(a)\Ga(b)$. Для остальных $a,b$ применяем формулы приведения.

Обоснуем перестановку. Пусть $a,b>1$, тогда подинтегральная функция $\ge 0$. Она непрерывна, и каждый
из интегралов под своим интегралом есть непрерывная функция: $\Ga(a+b)*\dots(t)$ и $\Ga(a)*\dots(y)$.
Покажем, что в этом случае внутренние интегралы сходятся равномерно на
каждом отрезке внутри.

\begin{theorem}
Равносильность определения равномерной сходимости при базе и сходимости последовательности.
\end{theorem}
\begin{proof}
В одну сторону доказывается тривиально, обратно от противного: допустим, что нет
равномерной сходимости, тогда
$\exi \ep_0\cln\fa \de \exi y_\de\in\dot{U}_\de(y_0)\cap Y, x_\de \in E\cln |f(x_\de,y_\de)-\ph(x_\de)|\ge \ep_0$.
Возьмём $\de := \frac{1}{n}$ и получаем плохую последовательность.
\end{proof}

\begin{theorem}[Признак Дини]
Пусть $y_0 := \sup Y$, $\fa x \in \Kc$ имеем $f(x,y)\uparrow(y)$, и $f(x,y)\ra\ph(x)$ при $y\ra y_0-0$ при
$\fa x \in \Kc$. Пусть $f(x,y), \ph(x) \in \Cb(\Kc)$ при $\fa x \in \Kc$. Тогда $f\convu{\Kc}\ph$ при $y\ra y_0-0$.
\end{theorem}
\begin{proof}
В самом деле, в силу указанной выше эквивалентности, всё верно для любой
последовательности, в частности, для монотонной. Фиксируем
$y_n\uparrow y_0-0$. Тогда выполнены все условия обычного признака Дини для $f(x,y_n)$,
поэтому $f(x,y_n) \convu{\Kc} \ph$. Далее,
в силу монотонности и этой равномерной сходимости имеем $\fa \ep \exi N\cln n>N$ имеем
$|f(x,y_n)-\ph(x)|<\ep$ для $\fa x \in \Kc$. Но,
кроме того, $f(x,y) \le \ph(x)$. Поэтому достаточно взять $y>y_{N+1}$, и тогда в такой
окрестности в силу монотонности всё будет хорошо.
\end{proof}

Теперь тривиально обосновывается возможность изменения порядка: условия основной теоремы
выполнены, поскольку в силу только что доказанного,
на любом конечном отрезке $\intl{a}{\xi} \convu{}$ и $\intl{c}{\De} \convu{}$.

\section{Ряды Фурье}

\subsection{Гильбертовы пространства}
Пусть задано линейное (векторное) пространство $\Ls$ над полем $K$. Мы будем
рассматривать $K=\Cbb$ или $K = \R$. Набор векторов $\Xc$ из $\Ls$ называется
\emph{линейно независимым}, если любое конечное подмножество
векторов из $\Xc$ линейно независимы. В качестве примера можно рассмотреть кольцо
многочленов от одной переменной на отрезке $[a,b]$.
Линейно независимой системой будут степени $1,x,x^2,\dots$.

\emph{Нормой} в $\Ls$ называется функция $\|\cdot\|\cln\Ls\ra \R$, удовлетворяющая свойствам:

\pt{1} $\fa x \in \Ls$ имеем $\|x\| \ge 0$, причём $\|x\|=0 \Lra x = 0$.

\pt{2} $\fa x \in \Ls, \fa \al \in K$ имеем $\|\al x\|=|\al|\cdot\|x\|$.

\pt{3} $\fa x,y\in \Ls$ имеем $\|x+y\| \le \|x\|+\|y\|$ \emph{неравенство треугольника}.

В этом случае $(\Ls, \|\cdot\|)$ называют \emph{линейным нормированным пространством}.

\begin{ex}
\pt{1} $\R^n$ с нормой $\|x\|:= |x|$.

\pt{2} $\Cb[a,b]$ с нормой $\|x(t)\| := \max |x(t)|$ \emph{равномерная} или Чебышёвская норма.

\pt{3} $\Cb[a,b]$ с нормой $\|x(t)\|_{L^p} := \bbr{\intlab |x(t)|^p\,d t}^{\frac{1}{p}}$. Здесь $p
\ge 1$, причём $p = 1$ представляет собой особый интерес. Заметим, что неравенство треугольника
выполнено благодаря неравенству Минковского для интегралов.
\end{ex}

\emph{Метрикой} на множестве $\Mc$ называется функция $\rho\cln\Mc \times \Mc \ra \R$, удовлетворяющая свойствам

\pt{1} $\fa x,y\in \Mc$ имеем $\rho(x,y) \ge 0$ и $\rho(x,y) = 0 \Lra x = y$.

\pt{2} $\fa x,y \in \Mc$ имеем $\rho(x,y) = \rho(y,x)$.

\pt{3} $\fa x,y,z \in \Mc$ имеем $\rho(x,y) \le \rho(x,z) + \rho(z,y)$.

В этом случае $(\Mc, \rho)$ называется \emph{метрическим пространством}.

Ясно, что если $\Ls$ обладает нормой, то можно ввести метрику следующим образом:
$\rho(x,y) := \|x - y\|$. Действительно, первая аксиома, очевидно, выполнена. Проверим
симметричность: $\|x-y\|=|-1|\cdot\|y-x\|$ по свойству линейности нормы,
а потому $\rho(x,y) = \rho(y,x)$. Неравенство треугольника проще всего доказать так:
возьмём $x := x' -z'$, $y := z'-y'$. Подставим $x$ и $y$ в выражение в свойстве \pt{3}
нормы. Получим $\|x'-y'\| \le \|x'-z'\|+\|z'-y'\|$. Но это и означает, что выполнено свойство \pt{3}
для метрики.

\emph{Скалярное произведение} в $\Ls$ функция $(\cdot,\cdot)\cln\Ls\times\Ls \ra K$, удовлетворяющая свойствам:

\pt{1} $(x,y) = \ol{(y,x)}$.

\pt{2} $(x+y,z)=(x,z)+(y,z)$.

\pt{3} $(\al x,y)=\al(x,y)$.

\pt{4} $(x,x)\ge 0$ и $(x,x)=0 \Lra x = 0$.

Из определения вытекают очевидные свойства $(x,y+z)=(x,y)+(x,z)$, $(x, \al y) = \ol{\al}(x,y)$,
$(x,0)=0$ и $(x,x) \in \R$.

Заметим, что если есть скалярное произведение, то норму можно определить так:
$\|x\| := \sqrt{(x,x)}$. Первые две аксиомы нормы очевидны.
Докажем \emph{НКБ}: $|(x,y)| \le \|x\| \cdot \|y\|$. Пусть $\al, \be \in K$. Тогда в силу
неотрицательности скалярного квадрата имеем
$(\al x + \be y, \al x + \be y) = \al \ol{\al}(x,x) + \al \ol{\be}(x,y) + \be\ol{\al}(y,x) + \be \ol{\be}(y,y) \ge 0$.
Поскольку $\al$ и $\be$ произвольные элементы поля $K$, это, в частности, верно и
для $\al = (y,y)$ и $\be = -(x,y)$. Тогда имеем
$(y,y)\bs{(y,y)(x,x) - \ol{(x,y)}(x,y) - (x,y)(y,x) + \ol{(x,y)}(x,y)} = (y,y)\bs{(y,y)(x,x)-(x,y)(y,x)} \ge 0$.
Заметим, что когда $y = 0$, доказываемое неравенство очевидно. Когда $y \neq 0$, можно разделить
неравенство на $(y,y)$, поскольку это не 0. Но тогда получаем
$\|x\|^2\cdot\|y\|^2 \ge |(x,y)|^2$, откуда следует доказываемое неравенство.

Осталось доказать само неравенство треугольника. Имеем $\|x+y\|^2 \bw= (x+y,x+y) \bw= (x,x) \bw+ (y,x) \bw+ (x,y) \bw+
(y,y) \bw\le \|x\|^2 \bw+ 2|(x,y)| \bw+ \|y\|^2 \le \lcomm$ НКБ $\rcomm \le \|x\|^2 \bw+ 2\|x\|\cdot\|y\| \bw+ \|y\|^2 \bw=
\br{\|x\| \bw+ \|y\|}^2$, что и требуется.

\begin{ex}
\pt{1} $(x,y) := x \cdot \ol{y}$.

\pt{2} $(x,y) := \intlab x(t)\ol{y(t)}\,d t$, откуда $\|x\| = \sqrt{(x,x)} =
\bbr{\intlab x(t)\ol{x(t)}\,d t}^{\frac{1}{2}} = \|x\|_{L^2}$.
\end{ex}

\begin{df}
Пусть $\hc{x_n} \subs \Ls$, тогда $x_n \ra x$ по норме $\|\cdot\|$, если $\liml{n}\|x_n - x\| = 0$.
Ясно, что если последовательность имеет предел, то он единствен.
\end{df}

Если $x_n \ra x$, то выполнено \emph{условие Коши}: $\fa \ep > 0 \exi N\cln \fa m,n \ge N$
имеем $\|x_n - x_m\| < \ep$. Доказательство этого факта стандартное и потому опущено.
Обратное неверно: фундаментальная последовательность может не иметь предела
по норме, поскольку предельная по норме точка может не лежать в том пространстве, в
котором мы рассматриваем последовательность.
Пример: рассмотрим $\Q$ в качестве линейного пространства и возьмём последовательность
рациональных чисел, являющуюся последовательными приближениями числа $\sqrt{2} \notin \Q$,
именно, $x_1 = 1,\; x_2 = 1.4,\; x_3 = 1.41,\; \dots$.
Фундаментальность очевидна, но предела в $\Q$ она не имеет!

\begin{df}
Пространство называется \emph{полным}, если в нём любая фундаментальная последовательность имеет предел. Полное нормированное пространство
называется \emph{банаховым}. Полное пространство со скалярным произведением называется \emph{гильбертовым}.
\end{df}

\begin{theorem}
Пусть $\Ls$ пространство со скалярным произведением и $\dim \Ls = m$. Тогда $\Ls$ полно.
\end{theorem}
\begin{proof}
Пусть $E := (e_1\sco e_m)$ ОНБ в $\Ls$. Пусть последовательность $\hc{x_n}$
фундаментальна. Тогда $x_n = \xi_n \cdot E$, где $\xi_n = (\xi_n^1\sco\xi_n^m)$.
Заметим, что $|\xi_n^k-\xi_p^k| = |(x_n,e_k)-(x_p,e_k)|=|(x_n-x_p,e_k)|\le \|x_n-x_p\|\cdot\|e_k\|=\|x_n-x_p\|$
в силу ортонормированности базиса. Отсюда, используя числовой КК, выводим, что по
каждой координате существует предел $\liml{n} \xi^k_n =: \xi^k$.
Положим $\xi := (\xi^1\sco\xi^m)$ и рассмотрим $x := \xi \cdot E$. Тогда
$\|x_n - x\| \bw= \|\xi_n \cdot E \bw- \xi \cdot E\| = \|(\xi_n \bw- \xi)\cdot E \| \bw\le
\sum \|(\xi_n^k-\xi^k)e_k\| \bw= \sum |\xi_n^k \bw- \xi^k|$.
Отсюда следует сходимость к $x$ по норме.
\end{proof}

\begin{theorem}
Пространство $\Cb[a,b]$ полно по равномерной норме.
\end{theorem}
\begin{proof}
Поскольку для функциональных последовательностей справедлив КК, это
переформулировка теоремы о том, что равномерный
предел последовательности непрерывных функций непрерывен.
\end{proof}

Далее в наших рассуждениях $\Ls$ будет пространством со скалярным произведением.

Векторы $x, y \in \Ls$ называются \emph{ортогональными}, если $(x,y) = 0$. Набор попарно
ортогональных векторов называется \emph{ортогональной} системой. Система
$\Xc \subs \Ls$ называется \emph{ортонормированной}, если она ортогональна и
$\fa x \in \Xc$ имеем $\|x\| = 1$, е $(x,x)=1$.

Рассмотрим \emph{тригонометрическую систему} $\hc{1, \sin kx, \cos kx}$ и покажем, что она
ортогональна. Для таких систем скалярное произведение выбирается таким: $(x,y) := \intlpp x y \,dt$.
Имеем $(\cos kt, \cos mt) \bw= \intlpp \cos kt \cdot \cos mt \,dt \bw=
\frac{1}{2}\intlpp\br{\cos(k-m)t \bw+ \cos(k+m)t}\,dt= \lcomm$ при $k \neq m$ получаем $\rcomm =
-\frac{1}{2}\hs{\frac{\sin(k-m)t}{k-m}+\frac{\sin(k+m)t}{k+t}}_{-\pi}^{\pi} = 0$. Если же $k=m$, то
получаем $(\cos kt, \cos kt) = \pi$. Совершенно аналогично рассматриваются случаи $(\sin, \sin)$ и
$(\sin, \cos)$.

Рассмотрим несчётную систему $e_\al = \sin \al t,\; \al \ge 0$. Определим $(e_\al,e_\be):=\liml{T
\ra +\infty} \frac{1}{T} \intl{-T}{T}e_\al \cdot e_\be \,d t$. Проверим корректность и покажем
ортогональность: $(e_\al,e_\be)=\liml{T \ra \infty}\frac{1}{2T}\intl{-T}{T}\br{\cos(\al-\be)t -
\cos (\al+\be)t}\,d t$. Если $\al \neq \be$, то интеграл получается ограниченным, и потому предел
равен 0. Если же $\al = \be$, то предел равен 1, поскольку первоообразная $\cos 2\al$ ограничена, а
$\lim\frac{1}{2T}\intl{-T}{T} 1 \,d t = 1$.

Пусть $\Ec = \hc{e_\al}$ ОНС. Покажем, что она линейно независима. В самом деле,
допустим противное, тогда $\exi e \in \Ec\cln e = A \cdot E$, где $A = (a_1\sco a_m)$ и
$E = (e_1\sco e_m)$. Рассмотрим $(e,e) = (A \cdot E, e) = \sum a_k (e_k,e) =0$ в силу
ортогональности системы. Но из свойств скалярного произведения следует, что тогда
$e = 0$, а это невозможно.

Рассмотрим задачу о наилучшем приближении элемента $x \in \Ls$ векторами из линейной оболочки конечного
числа элементов $E = (e_1\sco e_m)$ из некоторой ОНС. Именно, мы хотим найти $A = (a_1\sco a_m)$, для которого
$\|x - A\cdot E\|$ достигает минимума.
Рассмотрим
\begin{multline*}
\|x - A \cdot E\|^2 = (x - A\cdot E, x - A\cdot E) = (x,x) + (x, -A\cdot E) + (-A\cdot E, x) +
(A\cdot E, A\cdot E) =\\= (x,x)-\sum \ol{a}_k(x,e_k) - \sum a_k (e_k,x) + A \cdot \ol{A}.
\end{multline*}
Последнее слагаемое действительно имеет такой вид: если расписать
по линейности скалярное произведение $(A\cdot E, A \cdot E)$, в силу ортогональности поубиваются все
слагаемые, кроме диагональных.

Обозначим $c_k := (x, e_k)$, тогда
\begin{multline*}
\|x - A\cdot E\|^2 = \|x\|^2 + C \cdot \ol{C} - A \cdot \ol{C} - \ol{A}\cdot C + A \cdot \ol{A} - C \cdot \ol{C}=\\=
\|x\|^2 - C \cdot \ol{C} + \sum (c_k -a_k)(\ol{c}_k -\ol{a}_k) = \|x\|^2 - \sum |c_k|^2 + \sum|c_k-a_k|^2.
\end{multline*}
Заметим, что $c_k$ есть величины,
зависящие только от $x$. Теперь ясно, что сие выражение будет минимальным, когда последнее
неотрицательное слагаемое обратится в~0, а это будет в точности тогда, когда $a_k = c_k$. Итак, мы
нашли, что $\min \|x-A\cdot E\|^2 = \|x\|^2 - \sum |c_k|^2$.
Следовательно, можно написать $\|x - C\cdot E\|^2 = \|x\|^2 - \sum |c_k|^2$. Это выражение
называется \emph{тождеством Бесселя}, а $c_k$ \emph{коэффициентами Фурье}. Такое свойство
называется \emph{минимальным свойством коэффициентов Фурье}.

Из тождества Бесселя вытекает, что $\fa E \subs \Ec$ имеем $\|c - C\cdot E\|^2 \ge \|x\|^2 - \sum |c_k|^2$.
Отсюда следует, что для несчётных систем каждый элемент $x \in \Ls$ может иметь не более чем
счётное множество отличных от нуля коэффициентов Фурье. Действительно, $\fa n \in \N$ может
существовать лишь конечное число коэффициентов Фурье, для которых $|c_\al| \ge \frac{1}{n}$.
В самом деле, если бы из было бесконечно много, то можно было бы, взяв достаточно большое
количество базисных векторов, для которых $|c_\al| \ge \frac{1}{n}$,
нарушить неравенство Бесселя, ибо $\sum |c_\al|^2$ можно сделать сколь угодно большой. Итак,
$\fa n \in \N$ число коэффициентов, для которых
$|c_\al| > \frac{1}{n}$, конечно. Но если $c_\al \neq 0$, то $\exi n\cln |c_\al| \ge \frac{1}{n}$.
Поскольку счётное объединение конечных множеств счётно,
любой ненулевой коэффициент мы рано или поздно пересчитаем.

Из неравенства Бесселя для конечных сумм вытекает \emph{неравенство Бесселя}
$\|x\|^2 - \sum |c_k|^2 \ge 0$. Действительно, если бы ряд расходился, то для достаточно
больших $n$ его частные суммы нарушили бы неравенство Бесселя для конечных сумм.
Если знак неравенства обращается в равенство, то это \emph{равенство Парсеваля}.

\begin{note}
Сходимость векторных рядов подразумевается по норме пространства $\Ls$!
\end{note}

\begin{theorem}[Критерий полноты]
Пусть $\Ec$ ОНС. Тогда $\fa x \in \Ls$ равносильны утверждения:

\pt{1} $\fa \ep > 0 \exi$ полином $A\cdot E^*$, для которого $\|x - A^*\cdot E^*\|< \ep$, где $E^* \subs \Ec$.

\pt{2} $\exi \hc{e_k} \subs \Ec\cln x = \sum c_k e_k$, где $c_k$ коэффициенты Фурье по системе $\hc{e_k}$.

\pt{3} $\exi \hc{e_k} \subs \Ec\cln \|x\|^2 = \sum |c_k|^2$.
\end{theorem}
\begin{proof}
Пусть выполнено \pt{1}. Возьмём все ненулевые коэффициенты Фурье для $x$ и построим из них ряд $\sum c_k e_k$,
ведь их не более чем счётное множество. Если их конечное число, добавим нулевых, они нам не помешают.
Пусть $\|x - A^*\cdot E^*_n\| < \ep$. В силу минимального свойства имеем $\|x-C^*\cdot E^*_n\| \ep$.
Выберем частичную сумму $C\cdot E_N$ ряда, которая содержит все ненулевые слагаемые
полинома $C^*\cdot E^*_n$, ведь мы взяли в наш ряд все ненулевые слагаемые, а на нулевые нам
плевать. В силу тождества Бесселя получаем
$$\|x - C\cdot E\|^2 = \|x\|^2 - \suml{1}{N}|c_k|^2 \le \|x\|^2 - \suml{1}{n}|c_k^*|=
\|x-C^*\cdot E^*_n\|^2 < \ep^2.$$
Поэтому из \pt{1} следуют \pt{2} и \pt{3}.

Далее, из \pt{2} следует \pt{1}, поскольку если ряд сходится, то
$\fa \ep > 0 \exi n\cln \|x - C\cdot E\|^2 < \ep$. Полином $C\cdot E$
можно взять в качестве $A^*\cdot E^*$. Что касается \pt{2} и \pt{3}, то их эквивалентность
очевидно следует из тождества Бесселя: если сходится одно, то сходится и другое.
\end{proof}

Если $c_\al$ коэффициенты Фурье для $x \in \Ls$, то $\sum c_k e_k$ называют \emph{рядом Фурье}
элемента~$x$. Определение корректно благодаря тому, что ненулевых коэффициентов
существует лишь счётное множество.

\begin{theorem}
Пусть $\Ec = \hc{e_k}$ ОНС в $\Ls$ и ряд $\sum a_k e_k$ сходится к $x \in \Ls$. Тогда
$a_k$ есть коэффициенты Фурье для $x$ по $\Ec$.
\end{theorem}
\begin{proof}
Пусть $n = \dim E \ge m$, тогда $(x, e_m) = (x-A\cdot E, e_m) + (A\cdot E, e_m) = \lcomm$
ортогональность $\rcomm = (x \bw- A\cdot E, e_m) + a_m(e_m,e_m) = (x - A \cdot E, e_m) + a_m$.
Но первое слагаемое здесь стремится к нулю в силу НКБ:
$|(x - A\cdot E, e_m)| \le \|x - A\cdot E\| \cdot \|e_m\| = \|x - A\cdot E\| \ra 0, n \ra \infty$.
Следовательно, $a_m = c_m$.
\end{proof}

Следовательно, для сходимости ряда $\sum a_k e_k$ необходима сходимость $\sum |a_k|^2$, что
следует из неравенства Бесселя. Отсюда следует справедливость условия Коши для ряда $\sum a_ke_k$,
поскольку $\hn{\suml{m}{n} a_k e_k} = \suml{m}{n} |a_k|^2$.

\begin{theorem}[Рисса Фишера]
Пусть $\Ls$ гильбертово пространство. Пусть $\Ec = \hc{e_\al}$ ОНС в $\Ls$ и пусть сходится ряд
$\sum |a_k|^2$. Тогда $\sum a_k e_k$ сходится к некоторому элементу $x \in \Ls$, является
рядом Фурье своей суммы и $\|x\|^2 = \sum |a_k|^2$.
\end{theorem}
\begin{proof}
Только что мы установили выполнение условия Коши для ряда $\sum a_k e_k$. Но в силу полноты
пространства, этот ряд должен сходиться к некоторому $x \in \Ls$.
Остальное сразу следует из двух предыдущих теорем.
\end{proof}

Систему $\Ec = \hc{e_\al} \subs \Ls$ называют \emph{полной}, если $\fa x \in \Ls, \fa \ep > 0 \exi$
полином $A \cdot E$ по $\Ec$, для
которого $\|x - A \cdot E \| < \ep$.
Если $\Ls$ пространство со скалярным произведением, а $\Ec$ ОНС, то в силу критерия полноты тот
факт, что $\Ec$ полная система, равносильна тому, что каждый элемент есть сумма своего ряда
Фурье и для каждого элемента выполнено равенство Парсеваля.

\begin{theorem}
Полнота системы $\Ec = \hc{e_\al}$ равносильна условию $\fa x, y \in \Ls$ имеем
$(x,y) = \sum c_k \ol{d}_k$, где $c_k := (x,e_k)$ и $d_k := (y,e_k)$.
\end{theorem}
\begin{proof}
Сначала докажем в обратную сторону. При $x = y$ это соотношение обращается в равенство Парсеваля,
поэтому, если соотношение выполнено, то система полна в силу критерия полноты. Теперь докажем
прямую импликацию. Рассмотрим строки $C$ и $D$ мощности $N$. Тогда
имеем в силу свойств скалярного произведения и ортогональности
$(x - C \cdot E, y - D \cdot E) = (x,y) - C \cdot \ol{D} - C \cdot \ol{D} + C \cdot \ol{D} = (x,y) - C\cdot \ol{D}$.
Далее, согласно НКБ получаем $|(x - C\cdot E, y - D \cdot E)| \le \|x- C\cdot E\|\cdot\|y - D\cdot E\|$,
отсюда по доказанному выше $|(x,y) - C \cdot \ol{D}| \le \|x- C\cdot E\|\cdot\|y - D\cdot E\|$.
В силу полноты системы, нормы в правой части стремятся к 0 при $N \ra \infty$,
поэтому $C \cdot \ol{D} \ra (x,y), N \ra \infty$.
\end{proof}

ОНС $\Ec$ называется замкнутой в $\Ls$, если из условия $\fa e \in \Ec \; (x,e) = 0$ следует $x = 0$.

\begin{theorem}[Критерий замкнутости]
ОНС $\Ec$ замкнута в гильбертовом пространстве $\Ls$ тогда и только тогда, когда $\Ec$ полна в $\Ls$.
\end{theorem}
\begin{proof}
Для достаточности несущественно, полно $\Ls$, или нет: в силу условия \pt{3} критерия полноты из равенства~0
всех коэффициентов Фурье следует $\|x\|^2 = 0$, а потому $x = 0$. Докажем необходимость: пусть $\Ec$ замкнута,
но не полна, значит, $\exi x \in \Ls$, для которого не выполнено равенство Парсеваля, и потому
$\|x\|^2 > \sum |c_k|^2$, где $c_k$ все отличные от нуля коэффициенты Фурье для $x$ по $\Ec$.
Тогда по теореме Рисса Фишера $\exi y \in \Ls\cln y = \sum c_k e_k$. Рассмотрим $z := x - y$, и в силу того, что
у $x$ и $y$ коэффициенты Фурье одинаковы, все коэффициенты для $z$ равны 0. В силу замкнутости, $z = 0$, откуда $x = y$.
Но этого не может быть, ибо $\|x\|^2 > \|y\|^2$. Противоречие, значит, $\Ec$ полна.
\end{proof}

\subsection{Ряды Фурье по тригонометрической системе. Класс $\LR$}
Пусть $[a,b]$ отрезок. Будем говорить, что $f \in\LR[a,b]$, если $\exi T[a,b] = x_0\sco x_n$,
для которого существуют конечные пределы $I_i := \intl{x_i}{x_{i+1}}|f|\,d x := \liml{\ep \ra
+0}\intl{x_i+\ep}{x_{i+1}-\ep}|f|\,d x$. Естественно предполагать, что интеграл под знаком предела
при всех допустимых $\ep > 0$ имеет смысл. Положим $\intlab |f|\,d x := \suml{i=0}{n-1} I_i$.

Пусть $[\omega_1,\omega_2]$ бесконечный промежуток, и для определённости, пусть это будет вся
$\R$. Будем говорить, что $f \in\LR(\R)$, если $\fa [a,b] \subs \R$ имеем $f \in\LR[a,b]$ и
существует конечный предел $\intlii |f|\,d x := \liml{\substack{a \rightarrow -\infty\\b
\rightarrow +\infty}}\intlab |f|\,d x$.

Далее не будем различать бесконечные и конечные промежутки, иными словами, $[a,b]$ всё, что
угодно. Из определения следует, что если $f \in\LR[a,b]$, то $f$ интегрируема в несобственном
смысле на $[a,b]$. Легко видеть, что функции этого класса образуют линейное пространство. Введём
норму в $\LR[a,b]$. Положим $\|f\| := \intlab |f|\,d x$. Ясно, что все аксиомы нормы, кроме одной,
будут выполнены. Действительно, если $\|f\| = 0$, это ещё не означает, что $f \equiv 0$. Будем
считать, что $f\sim g$, если $\|f-g\| = 0$, иными словами, не будем различать такие функции. Это
означает, что на самом деле мы рассматриваем факторпространство $\wh{\LR} := \fact{\LR}{\sim}$,
проверяем согласованность линейных операций и все рассуждения проводим уже в терминах $\wh{\LR}$.
Но если не рассматривать крайние случаи, новое пространство «слабо» отличается от $\LR$, поскольку
все «нормальные» не равные между собой функции по этой норме всё-таки не совпадают.

Из определения $\LR$ можно вывести, что для отрезков имеет место следующее свойство. Пусть $f
\in\LR[a,b], g\bw\in\Rb[a,b]$. Тогда $fg \in\LR[a,b]$ и $\intlab |fg|\,d x \le B\intlab |f|\,d x$,
где $B$ ограничитель $g$ в силу её интегрируемости. Для бесконечных промежутков нужно
потребовать следующее: $f \in\LR[a,b]$, $|g| \le B$ и $g \in\LocR[a,b]$.

\begin{df}
Функция $\ph$ называется \emph{финитной}, если она не равна нулю лишь внутри некоторого конечного отрезка.
\end{df}

\begin{theorem}
Функцию $f \in\LR[a,b]$ можно сколь угодно точно приблизить в $\LR$ ступенчатой функцией.
\end{theorem}
\begin{proof}
Совершенно очевидно, что всё сводится к рассмотрению одного отрезка разбиения (того самого, которое
участвовало в определении функции класса $\LR$). Пусть $[a,b]$ конечный отрезок. Покажем, что
$\fa \ep > 0 \exi \ph$ ступенчатая функция, для которой $\|f - \ph\| < \ep$. В силу сходимости,
$\fa \ep > 0 \exi [c,d] \subs (a,b)$, для которого имеем $\intl{a}{c}|f|\,d x + \intl{d}{b}|f|\,d
x < \ep$. Мы знаем, что $f \in\Rb[c,d]$, поэтому [II семестр] $\exi \ph$ ступенчатая функция,
для которой $\|f - \ph\|_{[c,d]} < \ep$. Тогда доопределим $\ph$ вне $[c,d]$, а именно, положим
$\ph(x) := 0$ для $x \notin [c,d]$. Очевидно, что $\|f - \ph\| \le \ep + \ep = 2\ep$, что и
требуется. Для бесконечных промежутков рассуждения аналогичны.
\end{proof}

\begin{note}
При построении $\ph$ для бесконечных промежутков мы получаем финитную функцию, причём $\ph \in\LR[a,b]$.
\end{note}
\begin{note}
Аналогично доказывается теорема о приближении функции класса $\LR$ непрерывной финитной функцией.
\end{note}

\begin{df}
\emph{Характеристической функцией} отрезка $[\al,\be]$ назовём функцию $\chi(x) := \case{1, x \in [\al,\be],\\0, x \notin [\al,\be].}$
\end{df}

\begin{theorem}[Лемма Римана]
Пусть $[a,b]$ промежуток и $f \in\LR[a,b]$. Тогда $\intlab f \sin \mu x \,d x \ra 0, \mu \ra
\infty$.
\end{theorem}
\begin{proof}
\pt{1} $f$ характеристическая. Тогда $\intlab f \sin \mu x \,d x = \intl{\al}{\be}\sin \mu x
\,d x = - \frac{\cos \mu x}{\mu}\bsbs{\al}{\be} = \frac{1}{\mu}\br{\cos \mu \al - \cos \mu \be}
\ra 0, \mu \ra \infty$.

\pt{2} Пусть $f$ финитная ступенчатая функция. Тогда это некоторая конечная линейная комбинация
характеристических функций $\chi_k$. Отсюда $\intlab f\sin \mu x \,d x = \suml{i=1}{m}\al_k
\intlab \chi_k \sin \mu x \,d x \ra 0$.

\pt{3} Общий случай. Тогда $\exi \ph$ финитная ступенчатая функция, для которой $\|f - \ph\| <
\ep$. Тогда $$\bbm{\intlab f\sin \mu x \,d x} \bw\le \bbm{\intlab (f-\ph)\sin \mu x \,d x} +
\bbm{\intlab \ph \sin \mu x \,d x} < \ep + \ep(\mu) \ra 0,$$
поэтому теперь теорема доказана полностью.
\end{proof}

\begin{note}
Абсолютно то же самое можно доказать и про $\cos \mu x$.
\end{note}

\begin{theorem}
Пусть $f \in\LR[a,b]$. Тогда $\liml{t \ra 0}\intlab |f(x+t) - f(x)|\,d x = 0$.
\end{theorem}
\begin{proof}
В самом деле, $\fa \ep > 0 \exi \ph$ финитная ступенчатая функция: $\|f - \ph\| < \ep$. Тогда
$\intlab |f(x+t)-f(x)|\,d x \bw\le \intlab |f(x+t)-\ph(x+t)|\,d x + \intlab |\ph(x+t)-\ph(x)|\,dx
+ \intlab |\ph(x)-f(x)|\,d x < 2 \ep + \intlab |\ph(x+t)-\ph(x)|\,d x$. Но для финитных
ступенчатых функций утверждение очевидно, поэтому оставшийся интеграл можно сделать маленьким.
\end{proof}

Аналогично пространству $\LR$, можно ввести пространство $\LRp$, где $p > 1$. Разница состоит лишь
в том, что в определении этого класса функций вместо $|f|$ появляется выражение $|f|^p$, например,
норма задаётся интегралом $\|f\| := \intlab |f|^p\,d x$. Заметим, что когда мы говорили о
приближении функций $f \in\Rb[a,b]$, мы уточняли, что можно вместо $|f - \ph|$ поставить $|f -
\ph|^p$, и рассуждения от этого не сильно меняются. Поэтому несложно показать, что $\fa f \in
\LRp[a,b], \fa \ep > 0 \exi \ph, \psi\cln \|f-\ph\|< \ep, \|f - \psi\| < \ep$, где $\ph$ финитная
ступенчатая, а $\psi$ непрерывная функция. Здесь норма понимается, естественно, в смысле
пространства $\LRp$.

\subsection{Коэффициенты Фурье по тригонометрической системе}

Рассмотрим тригонометрическую систему $\hc{1, \cos kx, \sin kx}\; (*)$. Как мы знаем, она
ортогональна. Будем рассматривать $2\pi$-периодические функции класса $\LR$. Введём скалярное
произведение $(f,g) := \intlpp f\ol{g} \,d x$, тогда нашей системе не хватает нормированности.
Введём новую систему $\hc{\frac{1}{\sqrt{2\pi}}, \frac{\cos kx}{\sqrt{\pi}}, \frac{\sin
kx}{\sqrt{\pi}}} \;(**)$.

Мы хотим представить функцию $f$ в виде ряда $\frac{a_0}{2} + \sum(a_k \cos kx + b_k \sin kx)$.
Допустим, что этот ряд сходится равномерно, тогда проинтегрируем его почленно: $\intlpp f \,d x =
\intlpp \frac{a_0}{2}\,d x + \sum \intlpp (a_k\cos kx + b_k \sin kx) \,d x = 2\pi\cdot
\frac{a_0}{2}$. Отсюда $a_0 := \frpi \intlpp f \,d x$. Ясно, что равномерная сходимость
сохранится, если ряд умножить на $\cos nx$, тогда $\intlpp f \cos nx \,d x = \intlpp
\frac{a_0}{2}\cos nx \,d x + \sum \intlpp (a_k \cos kx + b_k \sin kx)\cos nx \,d x = a_n \intlpp
\cos^2 n x \,d x = a_n \pi$. Отсюда $a_n \bw= \frpi \intlpp f \cos nx \,d x$. Точно так же можно,
домножив на $\sin nx$, получить формулу $a_n = \frpi \intlpp f \sin nx \,d x$. Следовательно,
когда ряд равномерно сходится, эти \emph{коэффициенты Фурье} $a_k$ и $b_k$ можно просто
посчитать.

Пусть теперь по определению эти коэффициенты задаются своими формулами, е мы забываем про то, как мы их посчитали. Тогда мы не можем утверждать,
что ряд $s(x) := \frac{a_0}{2} + \sum (a_k \sin kx + b_k \sin kx)$ сходится к той функции, для которой были вычислены $a_k$ и $b_k$. Поэтому
пишут $f \sim s$, и ряд $s$ называется \emph{рядом Фурье функции} $f$ по тригонометрической системе.

Заметим, что мы раскладывали ряд по системе $(*)$. Теперь покажем, что для ОНС $(**)$ коэффициенты
Фурье останутся такими же. Возьмём вектор $(f,e_k) \cdot e_k = \hr{f, \frac{\cos
nx}{\sqrt{\pi}}}\cdot \frac{\cos nx}{\sqrt{\pi}} = \frpi \intlpp f \cos nx \,d x \cdot \cos nx$.
Мы видим, что коэффициент получился тот же самый.

\begin{note}
По лемме Римана, коэффициенты Фурье стремятся к 0.
\end{note}

\subsection{Сходимость ряда Фурье в точке}

Пусть $f \in\LR[-\pi,\pi]$ и $f \sim \frac{a_0}{2} + \sum (a_k \cos kx + b_k \sin kx)$. Положим
$S_n(f,x) := \frac{a_0}{2} + \suml{1}{n}(a_k \cos kx + b_k \sin kx)$ это \emph{частичные суммы
ряда Фурье}. Подставляя в $S_n$ значения коэффициентов Фурье, получаем
$$S_n(f,x) = \frpi \intlpp
f\cdot\hs{\frac{1}{2} + \suml{1}{n}(\cos kt \cos kx + \sin kt \sin kx)} \,d t= \frpi \intlpp
f\cdot\hs{\frac{1}{2} + \suml{1}{n}\cos k(t-x)}\,d t = \frpi \intlpp f \cdot D_n(t-x) \,d t,$$
где $D_n(t) = \frac{1}{2} + \suml{1}{n}\cos kt$ \emph{ядро Дирихле порядка} $n$.

С помощью домножения и деления $D_n(t)$ на $2\sin \frac{t}{2}$ легко выводится формула $D_n(t) = \frac{\sin\hr{n + \frac{1}{2}}t}{2\sin \frac{t}{2}}$.
Эта формула не определена в $t = 0$, однако заметим, что $\liml{t \ra 0} D_n(t) = D_n(0) = n + \frac{1}{2}$.

Сделав теперь замену, легко получить следующее выражение: $S_n(f,x) = \frpi \intlpp
f(x+t)D_n(t)\,d t$.

\begin{note}
Во всех теоремах из этой главы функции подразумеваются класса $\LR$.
\end{note}

\begin{theorem}[Принцип локализации Римана]
Сходимость или расходимость ряда Фурье в точке $x$ и значения суммы ряда определяются значениями функции
в сколь угодно малой окрестности $x$.
\end{theorem}
\begin{proof}
Рассмотрим произвольное $\de \in (0,\pi)$. Пусть $$g(t) := \case{0, & |t| < \de,\\f(x+t), & |t| \ge
\de.}$$ Тогда $$S_n(f,x) = \frpi \intl{-\de}{\de} f(x+t)D_n(t)\,d t + \rho_n(x), \quad \rho_n(x) :=
\frpi \intlpp g(t)D_n(t)\,dt.$$
Покажем, что $\rho_n(x) \ra 0, n \ra \infty$. Действительно, из
второго представления ядра Дирихле следует, что $\rho_n(x) \bw= \frpi \intlpp \frac{g(t)}{2\sin
\frac{t}{2}}\sin\hr{n+\frac{1}{2}}t \,d t$. Испортить дело мог только $\sin$ в знаменателе, но
поскольку при $|t| < \de$ имеем $g(t) = 0$, получается, что $\frac{g(t)}{2\sin \frac{t}{2}}
\in\LR[-\pi,\pi]$, откуда по лемме Римана $\rho_n(x) \ra 0, n \ra \infty$.

Заметим, что значения интеграла $\frpi \intl{-\de}{\de} f(x+t)D_n(t)\,d t$ зависят только от
значений функции в $\de$-окрестности точки $x$. Отсюда следует наше утверждение.
\end{proof}

\begin{imp}
Если $f = 0$ в некоторой окрестности точки $x$, то $S_n(f,x) \ra 0$.
\end{imp}
\begin{proof}
Возьмём такую $\de$-окрестность точки $x$, что в ней $f = 0$ и применим предыдущую теорему.
\end{proof}

В силу чётности ядра Дирихле имеет место представление $S_n(f,x) = \frpi \intlop\bs{f(x+t) +
f(x-t)}D_n(t)\,d t$. Отсюда, поскольку $\intlop D_n(t)\,d t = \frac{\pi}{2}$, для произвольного
$A$ получаем $S_n(f,x) - A = \frpi \intlop\bs{f(x+t)+f(x-t)-2A}D_n(t)\,d t$. Такое представление
может быть полезным, если мы ждём, что $S_n(f,x) \ra A$. На этот счёт существует

\begin{theorem}[Признак Дини]
Если для некоторого $A$ сходится $\intlop \frac{|f(x+t)+f(x-t)-2A|}{t}\,d t$, то $S_n(f,x) \ra A$.
\end{theorem}
\begin{proof}
Поскольку $\sin\frac{t}{2} \ge \frac{t}{\pi}$ при $t \in [0,\pi]$ в силу его выпуклости, имеем
$\frac{|f(x+t)+f(x-t)-2A|}{2\sin\frac{t}{2}} \le \frac{\pi}{2}\cdot\frac{|f(x+t)+f(x-t)-2A|}{t}$.
Тогда в силу леммы Римана $S_n(f,x) - A = \intlop
\ub{\frac{f(x+t)+f(x-t)-2A}{2\sin\frac{t}{2}}}_{\in\LR} \sin\hr{n+\frac{1}{2}}t\,d t \ra 0$.
\end{proof}

\begin{theorem}
Пусть для $f$ в точке $x$ существуют такие $\al > 0$ и $M>0$, что для всех достаточно малых~$t$ имеется
соотношение $|f(x+t)-f(x)|\le M|t|^\al$. Тогда $S_n(f,x) \ra f(x), n\ra \infty$.
\end{theorem}
\begin{proof}
Имеем $|f(x+t) + f(x-t) - 2f(x)| \le 2M|t|^\al$. Тогда, поскольку $\intlop\frac{2Mt^\al}{t}\,d t$
сходится, по признаку сравнения сходится и интеграл, участвующий в признаке Дини. Осталось
применить сам признак Дини.
\end{proof}

\begin{imp}\label{diff:dini}
Если $\exi f'(x)$, то условия предыдущей теоремы выполнены.
\end{imp}
\begin{proof}
Пусть $\exi f'(x)$, тогда $\fa \ep > 0 \exi \de > 0\cln \fa h\cln |h|<\de$ имеем $\hm{\frac{f(x+h)-f(x)}{h} - f'(x)}<\ep$.
Без ограничения общности можно рассмотреть только правую окрестность точки $x$, е при $h > 0$ имеем
$|f(x+h) - f(x)| \bw\in \br{f'(x)h-\ep h, f'(x)h + \ep h}$. Теперь всё очевидно, поскольку для этой $\de$-окрестности
достаточно взять $M \bw{:=} \max\hc{f'(x)-\ep, f'(x) + \ep}$ и $\al = 1$.
\end{proof}

\begin{imp}
Пусть $f$ имеет в $x$ разрыв первого рода. Если $\exi\al\bw>0, M \bw>0, U(x)\cln \fa t \in U(x)$ имеем $|f(x+t)-f(x\pm 0)|\le M|t|^\al$,
то $S_n(f,x) \ra \frac{f(x+0)+f(x-0)}{2} =: A$.
\end{imp}
\begin{proof}
Из условия имеем $|f(x+t)-f(x-t) - 2A| \le 2M|t|^\al$. Теперь всё следует из предыдущей теоремы.
\end{proof}

\begin{ex}
Пусть $f = \cos \al x$ где $\al \notin \Z$. Поскольку $f \in\Cb^1(\R)$, применимо следствие
\ref{diff:dini}. В силу чётности
\begin{multline*}
a_k = \frac{2}{\pi}\intlop\cos \al x \cos k x \,d x= \frpi
\intlop\br{\cos(\al + k)x + \cos(\al - k)x}\,d x = \frpi \hs{\frac{\sin(\al + k)x}{\al + k} +
\frac{\sin(\al-k)x}{\al - k}}_0^\pi=\\
=\frpi \hs{\frac{\sin(\al + k)\pi}{\al + k} +
\frac{\sin(\al-k)\pi}{\al - k}} = (-1)^k\cdot\frac{\sin \al \pi}{\pi}\hr{\frac{1}{\al + k} +
\frac{1}{\al - k}},
\end{multline*}
откуда
\eqn{\label{cos.decomposition}\cos \al x = \frac{\sin \al \pi}{\al\pi} + \frac{\sin \al
\pi}{\pi}\sum (-1)^k\cdot \hr{\frac{1}{\al + k} + \frac{1}{\al - k}}\cos k x.}

Подставим $x = 0$, получим
$1 = \frac{\sin \al \pi}{\al \pi} + \frac{\sin \al \pi}{\pi} \sum (-1)^k \hr{\frac{1}{\al + k} + \frac{1}{\al - k}}$.
Отсюда $\frac{\pi}{\sin \al \pi} = \frac{1}{\al} + \sum (-1)^k\hr{\frac{1}{\al + k} + \frac{1}{\al - k}}$.
Эта формула использовалась при доказательстве формулы дополнения для $\Ga$-функции.

Подставим в (\ref{cos.decomposition}) значение $x=\pi$. Получим
$$\cos\al\pi = \frac{\sin \al \pi}{\al\pi} + \frac{\sin \al \pi}{\pi}\sum (-1)^k\cdot
\hr{\frac{1}{\al + k} + \frac{1}{\al - k}}\cos k\pi=
\frac{\sin \al \pi}{\al\pi} + \frac{\sin \al \pi}{\pi}\sum \hr{\frac{1}{\al + k} + \frac{1}{\al - k}}.$$
Теперь разделим равенство на $\sin\al\pi$ и сделаем замену $t := \frac{\al}{\pi}$. Получаем
формулу \emph{разложения $\ctg$ по полюсам}:
$\ctg t = \frac{1}{t} + \frac{1}{t}\sum \hr{\frac{1}{t + k\pi} + \frac{1}{t - k\pi}}$.

Мы хотим проинтегрировать это равенство. Перенесём $\frac{1}{t}$ в левую часть и проинтегрируем:
$\intl{0}{x}\hr{\ctg t - \frac{1}{t}}\,d t  \bw= \liml{\ep\ra 0}\intl{\ep}{x}\hr{\ctg t -
\frac{1}{t}}\,d t = \liml{\ep\ra0}\ln \frac{\sin t}{t}\sbs{\ep}{x}= \ln \frac{\sin x}{x}$. Теперь
надо проинтегрировать ряд $\sum \hr{\frac{1}{t + k\pi} + \frac{1}{t - k\pi}}$. Это можно сделать
почленно в силу его равномерной сходимости при $|t| \le t_0 < \pi$. Тогда можно утверждать, что
$\fa |x| < \pi$ имеет место равенство $\ln \frac{\sin x}{x} = \sum\bs{\ln(x+k\pi) + \ln(x - k\pi) -
2\ln(k\pi)} = \sum\ln\hr{1 - \frac{x^2}{(k\pi)^2}}$. Беря $\exp$ от обоих частей, получаем
разложение $\sin$ в бесконечное произведение $\sin x = x \prod \hr{1 - \frac{x^2}{(k\pi)^2}}$.
Формула была выведена в предположении, что $|x| < \pi$, но, как легко видеть, при $x = \pm\pi$
формула тоже верна, поскольку первый множитель в бесконечном произведении равен 0. Итак, формула
верна для $|x| \le \pi$. Осталось доказать, что функция $\ph(x) := x \prod \hr{1 -
\frac{x^2}{(k\pi)^2}}$ $2\pi$-периодическая.

Рассмотрим $\ph_n(x) := x\prodl{1}{n} \hr{1 - \frac{x^2}{(k\pi)^2}}$, тогда
\begin{multline*}
\ph_n(x+\pi)=(x+\pi)(k\pi)^{-n}\prodl{1}{n}\br{(k\pi)^2-(x+\pi)^2}=
(x+\pi)(k\pi)^{-n}\prodl{1}{n}\br{(k+1)\pi+x}\br{(k-1)\pi-x} =\\
=(x+\pi)(k\pi)^{-n}\prodl{2}{n+1}(k\pi+x)\cdot \prodl{0}{n-1}(k\pi-x)=
\prodl{1}{n}(k\pi+x)\cdot\prodl{1}{n}(k\pi-x) \cdot\frac{(n+1)\pi + x}{n\pi -x}\cdot(-x)(k\pi)^{-n} =\\
=(-x)\prodl{1}{n}\hr{1-\frac{x^2}{(k\pi)^2}} \cdot \frac{(n+1)\pi + x}{n\pi -x} =
-\ph_n(x)\cdot\rho_n(x),
\end{multline*}
 где $\liml{n}\rho_n(x)=1$. Отсюда следует формула $\ph(x+\pi) = -\ph(x)$, а заодно
и $2\pi$-периодичность. Мы сделали нехорошо, когда поделили на $n\pi -x$, но при $x=n\pi$ формула и так верна.

При $x = \frac{\pi}{2}$ отсюда получается \emph{формула Валлиса}: $\frac{\pi}{2} = \prod \frac{(2k)(2k)}{(2k-1)(2k+1)}$. Сходится это произведение очень медленно:
10000 членов дают точность примерно в 4 десятичных знака.
\end{ex}

\subsection{Пример непрерывной функции, ряд Фурье для которой расходится в точке}

Рассмотрим $f(x) = \sum b_k \sin kx$, где $x\in [0,\pi]$ и $f$ чётна. Кроме того, пусть $\sum b_k$ сходящийся неотрицательный ряд. Ясно,
что $\sum b_k \sin kx$ сходится абсолютно и равномерно, а потому $f \in\Cb[0,\pi]$ и $f(0)=0$. Покажем, что можно подобрать коэффициенты $b_k$
так, что $S_n(f,0)$ не будет иметь конечного предела.

Используя равномерную сходимость, получаем $S_n(f,0) = \frpi \intlop 2f(t)D_n(t)\,d t = \frpi \sum
b_k\intlop 2\sin kt D_n(t) \,d t$. Введём обозначение $\be_{kn} := \intlop 2\sin kt \cdot
D_n(t)\,d t$. Пусть $j \in \N$, тогда $\intlop \sin jt \,d t = -\frac{\cos jt}{j}\sbs{0}{\pi} =
\frac{1 - (-1)^j}{j} \ge 0$. Покажем, что $\be_{kn} \ge 0$.

Имеем $2\sin kt \cdot D_n(t) = 2\sin kt\hr{\frac{1}{2}+\suml{m=1}{n}\cos mt} = \sin kt + \suml{m=1}{n}\br{\sin(k+m)t + \sin(k-m)t}$.
Пусть $k>n$. Тогда в $\be_{kn}$ под знаком интеграла стоит сумма выражений вида $\sin jt$, ибо $k - m \in \N$ при суммировании, поэтому $\be_{kn} \ge 0$.
Пусть $k \le n$. Надо показать, что среди аргументов синусов (если забить на $t$) не будет отрицательных чисел. Легко видеть, что это будут в точности
числа $k-n,k-n+1\sco,-1,0,1\sco k,k+1\sco k+n$ и, поскольку $\sin(-x) = -\sin x$, все отрицательные поубиваются, поскольку положительных больше.
Отсюда $\be_{kn} \ge 0$ при всех $k,n$.

Далее, $\be_{nn} = \intlop \suml{m=1}{2n}\sin mt \,d t = \suml{m=1}{2n}\frac{1-(-1)^m}{m} =
\suml{m=1}{n} \frac{2}{2m-1} > \suml{1}{m}\frac{1}{m} > \intl{1}{n}\frac{\,d x}{x} = \ln n$.
Поэтому $S_n(f,0) \ge \frpi b_n\ln n$.

Теперь если $n = 2^{i^3}$, пусть $b_n := \frac{1}{i^2}$, остальные равны 0. Тогда $S_n(f,0) \ge \frpi \frac{1}{i^2}\ln 2^{i^3} =
\frac{\ln 2}{\pi} i \ra \infty$, а $\sum b_k$ сходится.

\subsection{Равномерная сходимость рядов Фурье}

Напомним, что мы изучаем функции класса $\LR$. Сейчас нас будет интересовать класс $\LR(\R)$.
\begin{theorem}[ОЛР]
$f \in\LR(\R)$, $|g| \le G$, $g \in\LocR(\R)$. Тогда $\intlii f(x+t)g(t)\sin \mu t \,d t
\convu{[a,b]} 0, \mu \ra \infty$, где $[a,b]$ произвольный конечный отрезок.
\end{theorem}
\begin{proof}
Мы уже знаем, что $f(x+t)g(t) \in\LR(\R)$, поэтому сам факт стремления к 0 вытекает уже из леммы Римана.

Заметим, что если $h(t) \in\LR(\R)$, то $\intlii h(t) \sin \mu t \,d t = \intlii h \tpm \sin \mu
\tpm\,d t = -\intlii h\tpm \sin \mu t \,d t$. Поэтому $\frac{1}{2}\hs{2\intlii h(t)\sin \mu t
\,d t} = \frac{1}{2}\hs{\intlii h(t) \sin \mu t\,d t - \intlii h\tpm \sin \mu t \,d t}=
\frac{1}{2}\intlii\hs{h(t) - h\tpm}\sin \mu t \,d t$.

Зафиксируем отрезок $[a,b]$ и разрешим $x$ бегать только по нему.

Пусть сначала $f$ финитная ступенчатая функция. Тогда $|f| \le M$ и $\exi [A,B]\cln f(x+t) = 0$ вне $[A,B]$. Теперь можно считать, что
$g(t) = 0$ вне этого отрезка, поскольку всё равно мы будем умножать на $f(x+t)$. Поэтому можно считать, что $g \in\LR$. Применим формулу, доказанную выше, далее
вычтем и добавим средние слагаемые:
\begin{multline}
\label{common.riemann.lemma}
\bbm{\intlii f(x+t)g(t)\sin \mu t \,d t} = \frac{1}{2}\bbm{\intlii\hs{f(x+t)g(t)-f\xtpm g\tpm}\sin \mu t \,d t}=\\
=\frac{1}{2}\bbm{\intlii\hs{f(x+t)g(t) - f(x+t)g\tpm + f(x+t)g\tpm-f\xtpm g\tpm}\sin \mu t \,d t}\le\\
\le\frac{1}{2}\bbm{\intlii f(x+t)\hs{g(t) - g\tpm}\sin \mu t\,d t} + \frac{1}{2}\hm{\intlii g\tpm\hs{f(x+t)-f\xtpm}
\sin \mu t \,d t}\le\\
\le\frac{M}{2}\intlii \bbm{g(t) - g\tpm}\,d t + \frac{G}{2}\intlii \bbm{f(x+t)-f\xtpm}\,dt
\convu{[a,b]} 0, \mu \ra \infty.
\end{multline}
Тот факт, что последний интеграл зависит от $x$, несуществен, поскольку можно сделать замену.

Переходим к общему случаю, когда $f$ произвольная функция класса $\LR(\R)$. Рассмотрим $\ep >0$,
тогда $\exi \ph$ финитная ступенчатая функция, для которой $\|f - \ph\| < \ep$. Тогда
\begin{multline*}
\bbm{\intlii f(x+t)g(t)\sin \mu t \,d t} \le \bbm{\intlii \ph(x+t)g(t)\sin \mu t\,dt}+
\bbm{\intlii \br{f(x+t)-\ph(x+t)}g(t)\sin \mu t \,d t} \le\\\le \bbm{\intlii \ph(x+t)g(t)\sin \mu t \,dt}
+ G \intlii |f(x+t)-\ph(x+t)|\,d t \le \ep(\mu) + G\ep \convu{[a,b]} 0, \mu \ra \infty.
\end{multline*}
Стремление к нулю каждого из интегралов в обоих случаях вытекает из следствия леммы Римана.
\end{proof}

\begin{note}
Абсолютно то же самое можно доказать про $\cos \mu t$.
\end{note}

Нам понадобится ещё один специальный вариант этой теоремы.
\begin{theorem}[ОЛР II]
Пусть $f \in\LR(\R)$. Пусть
$$g(t) := \case{\frac{\sin \al t}{t}, & |t| < \de,\\0, & |t| \ge \de,} \quad \al \in [0,1],\quad \de \in [0,1].$$
Тогда $\intlii f(x+t)g(t,\al)\sin \mu t\,d t \convu{} 0$
относительно $x \in \fa [a,b] \subs \R$, $\al$ и $\de$ при $\mu \ra \infty$.
\end{theorem}
\begin{proof}
Для сокращения записи не будем писать аргумент $\al$ у функции $g$. Имеем $|g| \le 1$. Пусть
сначала $f$ финитная ступенчатая функция. Повторяя слово в слово рассуждения и выкладки
(\ref{common.riemann.lemma}) из предыдущей теоремы, получаем $\bbm{\intlii f(x+t)g(t)\sin \mu t
\,dt} \le \frac{M}{2} \intlii \hm{g(t)-g\tpm}\,d t + \frac{1}{2}\intlii \hm{f(x+t)-f\xtpm}\,d
t$. Второй интеграл $\convu{} 0$, разберёмся с первым интегралом. В силу свойств функции $g$ имеем
$$\intlii \hm{g(t)-g\tpm}\,dt = \intl{-\de-\frac{\pi}{\mu}}{\de}\hm{g(t) - g\tpm}\,d t.$$

Используя тождество $\intl{-\de-\frac{\pi}{\mu}}{\de} =
\intl{-\de-\frac{\pi}{\mu}}{-\de}+\intl{-\de}{\de-\frac{\pi}{\mu}} +
\intl{\de-\frac{\pi}{\mu}}{\de}$, свойство $g(t)=0$ вне $[-t,t]$ и её ограниченность,
получаем
\begin{multline*}
\intl{-\de-\frac{\pi}{\mu}}{\de}\hm{g(t)-g\tpm}\,dt =
\intl{-\de-\frac{\pi}{\mu}}{-\de}\hm{g\tpm}\,dt +
\intl{-\de}{\de-\frac{\pi}{\mu}}\hm{g(t)-g\tpm}\,d t +
\intl{\de-\frac{\pi}{\mu}}{\de}\hm{g(t)}\,dt\le\\\le \frac{2\pi}{\mu} +
\intl{-\de}{\de-\frac{\pi}{\mu}}\hm{g(t)-g\tpm}\,dt.
\end{multline*}
Вспомним, что $g$ зависит от $\al \in [0,1]$, поэтому
$$g'_t \bw= \frac{\al t\cos \al t - \sin \al t}{t^2}\bw=
\frac{\al t\br{1 + O\hr{\al^2t^2}} - \br{\al t + O\hr{\al^3t^3}}}{t^2} \bw= \frac{O\hr{\al^3t^3}}{t^2} = O(t)$$
и потому на отрезке $\hs{-\de,\de-\frac{\pi}{\mu}}$
производная абсолютно ограничена некоторым числом $D$. Используя ФКПЛ, получаем
$\intl{-\de-\frac{\pi}{\mu}}{\de}\hm{g(t)-g\tpm}\,d t \le \frac{2\pi}{\mu} +
\frac{\pi}{\mu}\intl{-\de}{\de-\frac{\pi}{\mu}}D\,d t \convu{} 0, \mu \ra \infty$. Итак, для
финитных функций всё доказано. Общий случай разбирается аналогично предыдущей теореме.
\end{proof}

\begin{theorem}[Лемма Римана для периодических функций]
Пусть $f\in\LR[-\pi,\pi]$, а $g \in\Rb[-\pi,\pi]$ и обе они $2\pi$-периодические. Тогда $\intlpp
f(x+t)g(t)\sin \mu t \,d t \convu{\R} 0$.
\end{theorem}
\begin{proof}
В силу периодичности функций, можно рассматривать $x \in [-\pi,\pi]$. Отсюда $x + t \in [-2\pi,2\pi]$. Поэтому можно обрубить функции $f$ и $g$,
положив их равными нулю вне отрезка $[-2\pi,2\pi]$. Применяем ОЛР и получаем требуемое.
\end{proof}

\begin{theorem}[Принцип локализации для равномерной сходимости]
Пусть $f \in\LR[-\pi,\pi]$ и $f \equiv 0$ на $[a,b]$. Тогда $\fa [\al,\be] \subs (a,b)$ имеем
$S_n(f,x) \convu{[\al,\be]} 0$ при $n \ra \infty$.
\end{theorem}
\begin{proof}
Пусть $x \in [\al,\be]$, тогда $\exi \de > 0\cln U_\de(x) \subs [a,b]$. Значит, когда $|t| \le \de$,
имеем $f(x+t)= 0$. Введём функцию
$$g(t) := \case{0, & |t| < \de,\\\frac{1}{2\sin\frac{t}{2}}, & |t| \ge \de,}$$
тогда $S_n(f,x) = \frpi \intlpp f(x+t)D_n(t)\,d t = \intlpp f(x+t)g(t)\sin\hr{n +
\frac{1}{2}}t \,d t \convu{[\al,\be]} 0$ по предыдущей теореме.
\end{proof}

\subsection{Наилучшие приближения функций}

Пусть $\Ls$ линейное нормированное пространство над полем $K$. Пусть $\Phi := (\ph_1\sco\ph_n) \subs \Ls$
система линейно-независимых приближающих функций. Пусть $f \in \Ls$. Мы хотим найти такой полином
$A \cdot \Phi$, где $A \bw= (a_1\sco a_n)$ и $a_i \in K$, для которого $\|f - A\cdot \Phi\| \ra \min$.
Это \emph{полином наилучшего приближения} $f$ по $\Phi$.

\begin{theorem}
$\infl{A} \|f-A\cdot \Phi\|$ достигается при некотором $A$.
\end{theorem}
\begin{proof}
Рассмотрим функцию векторного аргумента $g(A) := \|f-A\cdot \Phi\|$. Покажем, что $g$ непрерывна. В самом деле,
имеем $|g(A)-g(A')| = \bm{\|f-A\cdot \Phi\| - \|f - A'\cdot \Phi\|} \le \|A\cdot\Phi - A'\cdot\Phi\| = \|(A-A')\cdot\Phi\| \le \suml{1}{n}|a_k - a_k'|\cdot\|\ph_k\|$
и тем самым непрерывность доказана. Заметим, что $g \ge 0$. Положим $\rho := \infl{A} g(A)$.

Рассмотрим $h(A) := \|A\cdot \Phi\|$. Она, очевидно, тоже непрерывна и $h = 0$ только при $A = 0$ в силу линейной независимости $\Phi$.
Рассмотрим сферу $S$ единичного радиуса $|A|\cdot|A| = 1$. Тогда на этой сфере $h > 0$ и, в силу компактности $S$ получаем $\mu := \minl{A \in S} h(A) > 0$.
Рассмотрим $R := \frac{1}{\mu}\br{\rho + 1 + \|f\|}$. Положим $r(A) := \sqrt{\suml{1}{n}|a_k|^2}$, е это просто длина вектора $A$.
Обозначим через $B$ нормированный вектор $A$, е $B := \frac{1}{r(A)}A$. Тогда по определению функции $g$ имеем
$g(A) \ge \|A\cdot\Phi\| - \|f\| = \|B\cdot\Phi\| r(A) - \|f\| = h(B)r(A) - \|f\| \ge \mu r(A) - \|f\|$. Кроме того,
имеем $\rho + 1 = \mu R - \|f\|$. Сопоставляя полученные равенства, делаем вывод, что если $A$ такой вектор, что $r(A) > R$, то
$g(A) \ge \mu r(A) - \|f\| > \rho + 1$ и потому $\rho$ никак не может достигаться на векторах $A$ с такими координатами. Значит,
все вектора $A$, для которых $g(A) \le \rho + 1$ (а такие есть), расположены по крайней мере внутри шара радиуса $R$. А шар, как известно, компакт,
поэтому $\exi A\cln r(A) \le R$ и $g(A) = \rho$, что и требовалось доказать.
\end{proof}

\begin{note}
Далее везде речь идёт об $f$ как о непрерывной и $2\pi$-периодической функции.
\end{note}

Пусть $E_n(f) := \min \hn{f(x) - \suml{0}{n}(\al_k\cos kx + \be_k \sin kx)}$. Определение корректно
в силу доказанной выше теоремы. В данном случае имеется в виду равномерная норма, и для краткости
значок $C$ у нормы мы писать не будем. Положим $L_n := \frpi \intlpp|D_n(t)|\,d t$
\emph{константы Лебега}.

\begin{theorem}[Лебега]
Справедливо соотношение $\|f - S_n\| \le (L_n + 1)E_n(f)$.
\end{theorem}
\begin{proof}
Пусть $T_n(f,x)$ полином наилучшего приближения по тригонометрической системе. Несложно видеть,
что $T_n(f,x) = \frpi \intlpp T_n(f,x+t)D_n(t)\,d t$. Неудивительно, что частная сумма ряда Фурье
для тригонометрического многочлена с ним совпадает! Тогда $|f(x) - S_n(f,x)| \bw= \bm{f(x) - T_n(f,x)
- \frpi \intlpp\br{f(x+t)-T_n(f,x+t)}D_n(t)\,d t} \bw\le |f(x) - T_n(f,x)| + \frpi \intlpp|f(x+t) -
T_n(f,x+t)||D_n(t)|\,d t$. По определению равномерной нормы имеем $\|f - T_n\| \bw= \max |f(x) -
T_n(f,x)|$, но, поскольку $T_n$ полином наилучшего приближения, имеем $\|f - T_n\| = E_n(f)$.
Отсюда, продолжая наши выкладки, получаем $|f(x) - S_n(f,x)| \le E_n(f)\bbs{1 + \frpi
\intlpp|D_n(t)|\,d t}= E_n(f)(1 + L_n)$, что и требовалось доказать.
\end{proof}

\begin{theorem}
Для констант Лебега справедлива оценка $c_1\ln n \le L_n \le c_2 \ln n$, где $c_i > 0$.
\end{theorem}
\begin{proof}
Имеем $L_n = \frac{2}{\pi}\intlop\frac{\hm{\sin\hr{n+\frac{1}{2}}t}}{2\sin \frac{t}{2}}\,d t =
\lcomm$ в первом интеграле сверху $\sin(x) \rightsquigarrow (x)$, во втором $\sin \le 1$, а в
знаменателях подпираем линейной функцией $\rcomm \le \intl{0}{1/n}\hr{n + \frac{1}{2}}\,d t +
\intl{1/n}{\pi}\frac{1}{t}\,d t = 1 + \frac{1}{2n} + \ln n + \ln \pi \le \ln n + C$.

Теперь оценим снизу: $L_n = \frac{2}{\pi}\intlop\frac{\hm{\sin\hr{n+\frac{1}{2}}t}}{2\sin
\frac{t}{2}}\,d t > \lcomm$ заменим внизу $\sin$ на его аргумент, а вместо $\sin$ сверху напишем
его квадрат $\rcomm > \frac{2}{\pi}\intl{1/n}{\pi}\frac{\sin^2\hr{n + \frac{1}{2}} t}{2
\frac{t}{2}}\,d t = \frpi \intl{1/n}{\pi}\frac{\br{1 - \cos(2n+1)t}}{t}\,d t = \frpi \bbs{\ln n +
\ln \pi - \intl{1/n}{\pi}\frac{\cos(2n+1)t}{t}\,d t}$, а, сделав замену в последнем интеграле,
несложно по признаку Дирихле убедиться в его сходимости. Таким образом, $L_n > \frpi \ln n + D$.
\end{proof}

\begin{theorem}[Джексона]
$E_n(f) \le 6 \omega\hr{f,\frac{1}{n}}$.
\end{theorem}
\begin{proof}
Рассмотрим тригонометрический полином $U_n(t) := \frac{1}{2} + \suml{1}{n}r_k \cos kt$ и потребуем,
чтобы $U_n(t) \ge 0$. Рассмотрим приближения $u_n(f,x) := \frpi \intlpp f(x+t)U_n(t)\,d t$. Такое
выражение называется свёрткой функции. Заметим, что $U_n(t)$ с точностью до коэффициентов очень
похоже на ядро Дирихле и является чётной функцией, поэтому свёртка тригонометрический полином
степени $n$. Отсюда $E_n(f) \le \|f - u_n(f)\|$. Очевидно, что $\frpi \intlpp U_n(t)\,d t = 1$.
Следовательно, $|f(x) - u_n(f,x)| = \bbm{\frpi \intlpp\br{f(x)-f(x+t)}U_n(t)\,d t} \bw\le \frpi
\intlpp|f(x)-f(x+t)|U_n(t)\,d t \bw\le \frpi \intlpp \omega\br{f,|t|}U_n(t)\,d t \bw=
\frac{2}{\pi}\intlop\omega(f,t)U_n(t)\,d t$.

\begin{note}
В силу полуаддитивности $\omega$, имеем для $n \in \N$ свойство $\omega(n\de) \le n \omega(\de)$. Для нецелых
значений $\la > 0$ имеем $\omega(\la \de) \le (\la + 1)\omega(\de)$, что следует из свойства для натуральных чисел и
возрастания $\omega$ с ростом аргумента.
\end{note}

Воспользуемся этим фактом при $\la = nt$, получим
$$\intlop\omega(f,t)U_n(t)\,d t = \intlop\omega\hr{f,nt\cdot\frac{1}{n}}U_n(t)\,d t
\le \intlop(nt+1)\omega\hr{f,\frac{1}{n}}U_n(t)\,d t =
\omega\hr{f,\frac{1}{n}}\hr{\intlop (nt+1)U_n(t)\,d t}.$$
Следовательно, $E_n(f) \le \omega\hr{f,\frac{1}{n}}\hr{\frac{2n}{\pi}\intlop t
U_n(t)\,d t + 1}$.

Осталось показать, что $\exi r_k\cln U_n(t) \ge 0$ и $\frac{2n}{\pi}\intlop t U_n(t)\,d t \le 5$.

Так как $\frac{t}{\pi} \le \sin\frac{t}{2}$ на $[0,\pi]$, имеем

\begin{multline*}
\frac{2}{\pi}\intlop t \cdot U_n(t)\,dt
\le 2\intlop \sin\frac{t}{2}U_n(t)\,dt=
\sqrt{2}\intlop\sqrt{2}\cdot \sin\frac{t}{2}\sqrt{U_n(t)}\cdot\sqrt{U_n(t)}\,dt \stackrel{!}{\le}\\
\stackrel{!}{\le}\sqrt{2\intlop 2 \sin^2\hr{\frac{t}{2}}U_n(t)\,d t \cdot \intlop U_n(t)\,d t} =
\sqrt{2\intlop (1-\cos t)U_n(t)\,d t\cdot\frac\pi2\cdot\frac2\pi\intlop U_n(t)\,d t} =\\=
\sqrt{\pi\intlop (1-\cos t)U_n(t) \,d t}
=\frac{\pi}{\sqrt{2}}\sqrt{\frac{2}{\pi}\intlop U_n(t)\,d t-\frac{2}{\pi}\intlop \cos t \cdot U_n(t)\,d t} \stackrel{!!}{=}
\frac{\pi}{\sqrt{2}}\sqrt{1 - r_1}.
\end{multline*}
В этих формулах неравенство <<!>> следует из НКБ, а <<!!>> из ортогональности системы.

Осталось доказать, что $\frac{n\pi}{\sqrt{2}}\sqrt{1 - r_1} \le 5$. Мы выберем $U_n(t)$ так: пусть
$$U_n^*(t) := a \bbm{\suml{1}{n+1} a_k e^{(k-1)ti}}^2, \quad a = \frac{1}{2\sum a_k^2},
\quad a_k = \sin \frac{k\pi}{n+2}.$$
Поскольку $|z|^2 = z \ol{z}$, получаем
\begin{multline*}
U_n^*(t) = a\hr{a_1\spl a_{n+1}e^{nti}}\cdot\hr{a_1\spl a_{n+1}e^{-nti}} =\\=
a\hs{\br{a_1^2\spl a_{n+1}^2} + \br{a_1a_2\spl a_na_{n+1}}
\hr{e^{ti} + e^{-ti}}\spl \br{a_1a_{n+1}}\hr{e^{nti}+ e^{-nti}}}.
\end{multline*}
Если вспомнить про формулы Эйлера, то получается, что это некоторый
полином по $\cos kt$ степени $n$. Легко видеть, что $r_1 = \frac{a_1a_2\spl a_n a_{n+1}}{\sum a_k^2}$, поскольку $e^{ti}+e^{-ti} = 2\cos t$.
Осталось вычислить $r_1$ и убедиться в справедливости оценки. Имеем

$$a_k a_{k+1} = \sin\frac{k\pi}{n+2}\cdot\sin\frac{(k+1)\pi}{n+2}=
\sin\frac{k\pi}{n+2}\hr{\sin\frac{k\pi}{n+2}\cos\frac{\pi}{n+2}+\cos\frac{k\pi}{n+2}\sin\frac{\pi}{n+2}}=$$
$$=\sin^2\frac{k\pi}{n+2}\cos\frac{\pi}{n+2} +\frac{1}{2}\cdot\sin\frac{2k\pi}{n+2}\sin\frac{\pi}{n+2}=
\sin^2\frac{k\pi}{n+1}\cos\frac{\pi}{n+2} +\frac{1}{4}\hr{\cos\frac{(2k-1)\pi}{n+2} - \cos\frac{(2k+1)\pi}{n+2}}.$$ Теперь заметим, что
$a_{n+2} = \sin\frac{(n+2)\pi}{n+2} = 0$. Поэтому, хотя нам надо найти сумму $a_1a_2\spl a_na_{n+1}$, мы будем считать сумму $a_1a_2\spl a_{n+1}a_{n+2}$,
поскольку в ней всё равно последнее слагаемое нулевое.

Имеем $a_1a_2\spl a_{n+1}a_{n+2}=\suml{k=1}{n+1}\hs{\sin^2\frac{k\pi}{n+2}\cos\frac{\pi}{n+2} +\frac{1}{4}\hr{\cos\frac{(2k-1)\pi}{n+2} - \cos\frac{(2k+1)\pi}{n+2}}} =\lcomm$
все $\cos$ сокращаются, кроме первого и последнего $\rcomm = \suml{k=1}{n+1}\sin^2\frac{k\pi}{n+2}\cos\frac{\pi}{n+2} +\frac{1}{4}\hr{\cos\frac{\pi}{n+2} - \cos\frac{(2n+3)\pi}{n+2}}=
\sum a_k^2 \cos\frac{\pi}{n+2}$, поскольку оставшиеся $\cos$ равны. Отсюда $r_1 = \cos \frac{\pi}{n+2}$, поэтому
$$\frac{\pi}{\sqrt{2}}n\sqrt{1-r_1} =\frac{\pi}{\sqrt{2}}n\sqrt{2\sin^2 \frac{\pi}{2(n+2)}} = \pi n \sin \frac{\pi}{2(n+2)}
< \pi n \frac{\pi}{2(n+2)} < \frac{\pi^2}{2} < 5.$$
Итак, мы обосновали справедливость оценки. Теорема доказана.
\end{proof}

\begin{df}
\emph{Условие Дини Липшица}: $\omega\hr{f,\frac{1}{n}}\ln n \ra 0, n \ra \infty$.
\end{df}

\begin{theorem}[Признак Дини Липшица]
Если выполнено условие Дини Липшица, то $f \rightrightarrows S_n(f)$.
\end{theorem}
\begin{proof}
Очевидно, поскольку по теореме Лебега имеем $\|f-S_n(f)\| \le (L_n+1)E_n(f) \le \lcomm$ по теореме Джексона и теореме об
оценке констант Лебега $\rcomm \le (c_1\ln n + c_2)\cdot 6\omega\hr{f,\frac{1}{n}} \ra 0$ в силу условия теоремы.
\end{proof}

\begin{note}
Условие Дини Липшица выполнено, если $f \in\SegC^1$. Более того, можно требовать всего лишь липшицевость функции $f$.
\end{note}

\subsection{Почленное дифференцирование и интегрирование рядов Фурье}

Будем предполагать, что $f \in\SegC^1$ и $2\pi$-периодическая. Пусть $f \bw\sim \frac{a_0}{2} + \sum(a_k \cos kx + b_k \sin kx)$
и $f' \bw\sim \frac{\al_0}{2} + \sum (\al_k\cos kx + \be_k\sin kx)$.

Имеем $\al_0 = \frpi \intlpp f'(t)\,d t = \lcomm$ ФНЛ $\rcomm = \frpi \br{f(\pi)-f(-\pi)}=0$ в
силу $2\pi$-периодичности функции. Для остальных коэффициентов имеем $\al_k = \frpi \intlpp
f'(t)\cos kt \,d t = \frpi \bbs{f(t)\cos kt\sbs{-\pi}{\pi} - \intlpp f(t)(-k\sin kt)\,d t}=
\frac{k}{\pi}\intlpp f(t)\sin kt \,d t = k b_k$. Аналогично получаем $\be_k = -k a_k$. Таким
образом, ряд Фурье для $f'$ можно получить почленным дифференцированием ряда Фурье исходной
функции.

Аналогичные формулы можно получить и для ряда Фурье функции $f^{(m)}$, если предположить, что $f \in\SegC^m$ и $2\pi$-периодическая.
Кроме того, поскольку коэффициенты Фурье стремятся к 0, получаем $\al_k = o\hr{\frac{1}{k^m}}$, то же верно и для $\be_k$, если $\al_k$ и $\be_k$ коэффициенты Фурье для $f^{(m)}$.

Поговорим об интегрировании рядов Фурье. Как обычно, функция у нас $2\pi$-периодическая и $f \in\SegC^1$.

Рассмотрим функцию $\ph(x) = \intlox\hr{f(t) -\frac{a_0}{2}}\,dt$ на $[-\pi,\pi]$, тогда $\ph'(x)
= f(x) - \frac{a_0}{2}$. Покажем, что $\ph$ можно продолжить как непрерывную функцию на всю ось. В
самом деле, $\ph(\pi)-\ph(-\pi)=\intlop \hr{f(t)-\frac{a_0}{2}}\,dt \bw-
\intl{0}{-\pi}\hr{f(t)-\frac{a_0}{2}}\,d t= \intlpp \hr{f(t)-\frac{a_0}{2}}\,dt = \intlpp
f(t)\,d t - \pi a_0 = 0$, поскольку $\intlpp f \,d t = a_0\pi$.

В силу следствия из теоремы Джексона, ряд Фурье для $\ph$ равномерно к ней сходится. Пусть
$\ph \sim \frac{A_0}{2} + \sum(A_k\cos kx + B_k\sin kx)$,
тогда продифференцируем его почленно и получим
$\ph'(x) = f(x)-\frac{a_0}{2} \sim \sum (-kA_k\sin kx \bw+ kB_k\cos kx)$, поэтому
$\ph \sim \frac{A_0}{2} + \sum\hr{-\frac{b_k}{k}\cos kx \bw+ \frac{a_k}{k}\sin kx}$ и $\frac{A_0}{2} = \sum \frac{b_k}{k}$.

\begin{note}
На самом деле всё это справедливо для любой функции класса $\LR[-\pi,\pi]$.
\end{note}

Если проинтегрировать ряд для функции $f$ почленно, получим
$$\intlox\hr{\frac{a_0}{2}+\sum(a_k\cos
kt + b_k\sin kt)}\,d t=\frac{a_0}{2}x+\sum\frac{b_k}{k} + \sum\hr{-\frac{b_k}{k}\cos kx +
\frac{a_k}{k}\sin kx}= \frac{a_0}{2}x+\ph(x) = \intlox f \,d t.$$

\subsection{Ряды Фурье в комплексной форме}

Пусть $f \in\LR[-\pi,\pi]$, и $f \sim \frac{a_0}{2} + \sum (a_k\cos kx + b_k \sin kx)$.
Имеем $$a_k\cos kx + b_k\sin kx = a_k\frac{e^{ikx}+e^{-ikx}}{2} + b_k\frac{e^{ikx} -e^{-ikx}}{2i} =
\frac{a_k - ib_k}{2}e^{ikx} + \frac{a_k + ib_k}{2}e^{-ikx}.$$
Положим $c_k := \frac{a_k - ib_k}{2}$ и $c_{-k} := \frac{a_k + ib_k}{2}$, кроме того, $c_0 := \frac{a_0}{2}$.
Тогда $f(x) \sim \suml{-\infty}{+\infty}c_ke^{ikx}$.

Получим выражения для чисел $c_k$ через интегралы: $$c_k = \frac{a_k -
ib_k}{2}=\frac{1}{2\pi}\intlpp f(t)(\cos kt -i\sin kt)\,d t = \frac{1}{2\pi}\intlpp
f(t)e^{-ikt}\,d t.$$
Пока это верно для $k \in \N$, но легко убедиться в том, что и для $k < 0$ это
тоже верно.

Пусть теперь для нашей функции выполнены условия, при которых мы устанавливали формулы для
производных, именно, $f \in\SegC^m$. Тогда
$f^{(m)} \sim \suml{-\infty}{+\infty}(ik)^mc_ke^{ikx}$.

\subsection{Явление Гиббса}

Изучим поведение частичных сумм ряда Фурье в окрестности разрыва функции. Рассмотрим функцию $f(x)
:= \case{0, & x = 0,\\\frac{\pi-x}{2}, & x \in (0,2\pi),}$ продолженную на всю ось $2\pi$-периодическим
образом. Она нечётна, поэтому $a_k = 0$. Поскольку неважно, по какому отрезку интегрировать
$2\pi$-периодическую функцию, $b_k = \frpi \intl{0}{2\pi}f(x)\sin kx \,d x = \lcomm$ по частям
$\rcomm = \frac{1}{k}$. Отсюда получаем $f(x) = \sum \frac{\sin kx}{k}$. Тогда $$S_n(f,x) =
\suml{1}{n}\frac{\sin kx}{k} = \suml{1}{n}\intlox\cos kt\,d t = \intlox \suml{1}{n}\cos kt \,d t
= \intlox D_n(t) - \intlox \frac{\,d t}{2} = \intlox D_n(t)\,d t - \frac{x}{2}.$$

Мы хотим проинтегрировать это ядро Дирихле. Рассмотрим функцию $g(t) = \frac{1}{2}\ctg \frac{t}{2} - \frac{1}{t}$. Используя формулы Тейлора,
легко показать, что $g(t) = O(t)$ и потому $g \in\Cb[0,\pi]$, если доопределить $g(0) := 0$. Имеем
\begin{multline*}
D_n(t)=\frac{\sin\hr{n+\frac{1}{2}}t}{2\sin\frac{t}{2}} =
\frac{\sin nt \cos \frac{t}{2} + \cos nt \sin \frac{t}{2}}{2\sin\frac{t}{2}}=
\frac{1}{2}\sin nt \ctg \frac{t}{2} + \frac{1}{2}\cos nt =\\=
\sin nt \hr{\frac{1}{2}\ctg \frac{t}{2}-\frac{1}{t}} +\frac{\sin nt}{t} + \frac{1}{2}\cos nt=
\sin nt \cdot g(t) + \frac{\sin nt}{t} + \frac{1}{2}\cos nt.
\end{multline*}
Подставляя $D_n(t)$ в $S_n(f)$, получим $S_n(f,x) = \intlox \frac{\sin nt}{t}\,d t + \intlox
g(t)\sin nt \,d t + \frac{1}{2}\intlox \cos nt \,d t -\frac{x}{2}$.

Введём функцию $h(t) := \case{1, t \in [-\pi,0],\\0, t \in (0,\pi)}$ и продолжим её на всю ось
$2\pi$-периодическим образом. Тогда $\intlox g(t)\sin nt \,d t = \intlop h(t-x)g(t)\sin nt \,d t
\convu{} 0$ по обобщённой лемме Римана. Тогда $S_n(f,x) = \intl{0}{nx}\frac{\sin t}{t}\,d t +
\frac{1}{2n}\sin nx \bw- \frac x2 + r_n(x)$, где $r_n(x)\convu{}0$. Поэтому $S_n(f,x) - f(x) =
\intl{0}{nx}\frac{\sin t}{t}\,d t - \frac{\pi}{2} + R_n(x)$, где $R_n(x)\convu{}0$.

Поскольку ряд Фурье для $f$ сходится к $f$, получаем значение интеграла Дирихле:
$\intl{0}{+\infty}\frac{\sin t}{t}\,d t =\frac{\pi}{2}$. Изучим немного свойства интегрального
синуса $\Si(x) := \intlox\frac{\sin t}{t}\,d t$. Поскольку $\Si'(x) = \frac{\sin x}{x}$, эта
функция имеет локальные $\min$ и $\max$ в точках $x = k\pi$. Можно сделать это с помощью второй
производной, но это видно и из наглядных соображений, что график $\Si$ колеблется рядом с прямой $y
= \frac{\pi}{2}$, сокращая амплитуду колебаний. Мораль: первый горб $\Si$ самый большой, и
своего максимума функция достигает в точке $x=\pi$, е в точности тогда, когда $\sin t$ впервые
становится отрицательным. Поэтому $\liml{n\ra\infty}\supl{x\in[0,\pi]}|S_n(f,x) - f(x)| =
\liml{n\ra\infty} \max \hm{\Si(nx) - \frac{\pi}{2} + R_n(x)} = \hm{\Si(\pi)-\frac{\pi}{2}} \approx
0.28$. \emph{Явление Гиббса} как раз и заключается в том, что частные суммы ряда Фурье в окрестности точки
разрыва функции будут значительно превосходить значение самой функции (по равномерной норме).

Рассмотрим теперь общий случай. Пусть $f$ достаточно хорошая функция, у которой $f(-0) - f(+0) =: d \neq 0$, е в точке $0$ она
терпит разрыв первого рода. Пусть $g(x) := f(x) - \frac{d}{\pi}\ph(x)$, где $\ph$ та самая функция, которую мы рассматривали с самого начала.
Если $g \in\Cb(0)$ и ряд Фурье для $g$ равномерно сходится к ней в окрестности точки 0, то будет иметь место явление Гиббса для $f$.

\subsection{Суммирование рядов Фурье методом средних арифметических}

Пусть $\si_n(f,x) := \frac{1}{n+1}\suml{0}{n}S_k(f,x)$. Эти величины называются \emph{средними Фейера}.
Получим удобное представление для $\si_n$. Поскольку $S_n(f,x) = \frpi \intlpp f(t)D_n(x-t)\,d t$,
получаем $\si_n(f,x) = \frac{1}{\pi(n+1)}\intlpp f(t)\suml{0}{n}D_k(t-x)\,d t$. Преобразуем сумму
ядер Дирихле: $\suml{0}{n}D_k(t) = \suml{0}{n}\frac{\sin\hr{k +
\frac{1}{2}}t}{2\sin\frac{t}{2}}=\frac{1}{2\sin\frac{t}{2}}\suml{0}{n}\sin\hr{k+\frac{1}{2}}t$.
Будем преобразовывать $\suml{0}{n}\sin\hr{k+\frac{1}{2}}t =
\frac{1}{2\sin\frac{t}{2}}\suml{0}{n}2\sin\frac{t}{2}\sin\hr{k + \frac{1}{2}}t =
\frac{1}{2\sin\frac{t}{2}}\suml{0}{n}\br{\cos kt - \cos(k+1)t} =
\frac{1-\cos(n+1)t}{2\sin\frac{t}{2}} = \frac{\sin^2\frac{n+1}{2}t}{\sin\frac{t}{2}}$.

Отсюда $F_n(t) := \frac{1}{n+1}\suml{0}{n}D_k(t) =
\frac{1}{2(n+1)}\hr{\frac{\sin\frac{n+1}{2}t}{\sin\frac{t}{2}}}^2$. Такие тригонометрические
многочлены называются \emph{ядрами Фейера}. Имеем $\frpi \intlpp F_n(t)\,d t = 1$, поскольку это
усреднённая сумма $D_k(t)$, а $\frpi \intlpp D_k(t)\,d t = 1$. Кроме того, $F_n(t) \ge 0$.
Далее, имеем $f(x) - \si_n(f,x) = \frpi \intlpp\br{f(x)-f(t)}F_n(x-t)\,d t$.

\begin{theorem}[Фейера]
Пусть $f$ $2\pi$-периодическая функция и $f \in\Cb(\R)$. Тогда $\si_n(f,x) \convu{} f(x)$.
\end{theorem}
\begin{proof}
В силу равномерной непрерывности $f$ имеем $\omega(f,\de)\ra 0$. Рассмотрим равномерную норму $\|f
- \si_n(f)\|$. Имеем $\fa \ep > 0 \exi \de\cln \omega(f,\de) < \ep$. Будем игнорировать константные множители
для сокращения выкладок. Сделав в интеграле замену, получаем
\begin{multline*}
|f(x) -\si_n(f,x)| \le \intlpp\br{f(x+t)-f(x)}F_n(t)\,d t
\le\intlpp|f(x+t)-f(x)|F_n(t)\,d t \le \intlpp \omega\br{f,|t|}F_n(t)\,d t =\\= \lcomm
\text{ чётность подинтегральной функции }\rcomm =
\intlop \omega(f,t) F_n(t)\,d t = \intl{0}{\de}\omega(f,t)F_n(t)\,d t \bw+
\intl{\de}{\pi}\omega(f,t)F_n(t)\,d t \le\\ \le\lcomm \text{ $F_n \ge 0$ + возрастание $\om$ }
\rcomm \le \omega(f,\de)\intlop F_n(t)\,d t + \omega(f,\pi)\intl{\de}{\pi}\frac{\,dt}{2(n+1)\sin^2\frac{t}{2}}
\bw< \ep + \frac{C(\de)}{n+1} \rightrightarrows 0.
\end{multline*}
Теорема доказана.
\end{proof}

\begin{theorem}[Вейерштрасса]
Тригонометрическая система полна в $\Cb[-\pi,\pi]$ по норме $\|\cdot\|_{\Cb}$.
\end{theorem}
\begin{proof}
В качестве приближающих многочленов можно как раз взять средние Фейера.
\end{proof}

\begin{imp}
Тригонометрическая система полна в $\LR[-\pi,\pi]$ по норме $\|\cdot\|_{\LR}$.
\end{imp}
\begin{proof}
Мы знаем, что для любой функции $f$ класса $\LR$ можно подобрать такую непрерывную функцию $g$, что $\|f - g\|_{\LR} < \ep$.
Напомним, что у нас в пространстве $\LR$ была введена норма функции в виде интеграла от модуля функции. Ясно, что функция $g$
может не быть $2\pi$-периодической, поэтому её надо изогнуть с одного конца так, чтобы $\|f - h\|_{\LR} < 2\ep$, где $h$
изогнутая функция $g$. Теперь, поскольку функция $h$ есть $2\pi$-периодическая непрерывная функция, её можно по предыдущей
теореме приблизить по равномерной норме тригонометрическим полиномом. Пусть $T$ полином, для которого $\|T - h\|_{\Cb} < \ep$.
Заметим, что если функции близки в смысле равномерной нормы, то они близки и в смысле нормы $\LR$, поскольку интегрируем
мы на конечном отрезке. Следовательно, $\|T-h\|_{\LR} < C\ep$, где $C$ какая-то не имеющая значения константа.
Отсюда $\|f - T\| \le 2\ep + C\ep = \ep(2 + C)$, что и требовалось доказать.
\end{proof}

\begin{theorem}[Вейерштрасса II]
Система алгебраических многочленов полна в $\Cb[a,b]$.
\end{theorem}
\begin{proof}
Без ограничения общности можно считать, что мы приближаем на отрезке $[-1,1]$. Пусть в $\Cb[-1,1]$ живут функции параметра $t$,
тогда сделаем замену $\cos \theta = t$ и перейдём таким образом к пространству $\Cb[0,\pi]$. Пусть $\ph(\theta) := f(t) = f(\cos \theta)$. Продолжим
функцию $\ph$ чётным образом на $[-\pi,0]$ и применим теорему Вейерштрасса. В силу чётности функции можно считать, что взятый в качестве
приближения многочлен Фейера есть многочлен по $\cos$. Тогда $t_n(\theta) = \suml{0}{n}\al_k\cos k\theta$. Покажем, что
$\cos k\theta$ есть многочлен по $\cos\theta$. Обозначим $a := \cos\ta$, $b=i\sin\ta$. Тогда
$2\cos k\theta = e^{ik\theta} + e^{-ik\theta}=\hr{e^{i\theta}}^k + \hr{e^{-i\theta}}^k=
(a+b)^k + (a-b)^k$.
Теперь несложно увидеть, что при раскрытии скобок $b$ останутся только в чётных степенях, а это означает, что
результирующее выражение будет состоять из $\cos\theta$ в каких-то степенях и из $\sin^{2m}\theta$. Но поскольку $\sin^{2m}\theta = (1-\cos^2\theta)^m$,
весь многочлен можно записать в виде некоторого многочлена по $\cos \theta$. Вспомним, что $\cos\theta = t$, тогда можно вернуться к переменной $t$ и
получить алгебраический многочлен.
\end{proof}

\begin{note}
Возникающие в теореме многочлены $T_k(t) := \cos k\theta$, где $t=\cos\theta$, называются \emph{многочленами Чебышева первого рода}.
\end{note}

\subsection{Преобразования Фурье}

Рассмотрим $2\pi$-периодические функции из $\LR(\R)$. Пусть $s \ge 0$. Введём функции $a(s) :=
\frpi \intlii f(x)\cos sx \,d x$ и $b(s) := \frpi \intlii f(x)\sin sx \,d x$ соответственно
\emph{косинус-преобразование и синус-преобразование} Фурье.

\begin{theorem}
$a(s), b(s) \in\Cb[0,+\infty)$.
\end{theorem}
\begin{proof}
Для сокращения выкладок будем игнорировать константные множители и не будем писать пределы
интегрирования $\intlii$. $\fa \ep > 0 \exi$ финитная ступенчатая функция $\ph\cln \|f-\ph\| < \ep$.
Рассмотрим $a(s+h)-a(s) = \int (f-\ph)\cos(s+h)x\,d x - \int(f-\ph)\cos sx \,d x +
\int\ph\cos(s+h)x\,d x - \int\ph\cos sx \,d x$, откуда $|a(s+h)\bw-a(s)| \le \int|f\bw-\ph|\,dx \bw+
\int|f-\ph|\,dx + \int|\ph||\cos(s+h)x-\cos sx|\,d x \le 2\ep \bw+
\int|\ph|\hm{2\sin\frac{h}{2}\sin\hr{s+\frac{h}{2}}}\,d x \le\lcomm$ заменяем первый $\sin$ на его
аргумент, второй на 1 $\rcomm \le2\ep + \int|\ph||hx|\,d x \le 2\ep + |h|\int|\ph||x|\,d x \le
2\ep + C|h| \ra 0$ при $h\ra 0$. Мы оценили последний интеграл константой в силу финитности $\ph$.
\end{proof}

Аналогом частных сумм Фурье будет \emph{частное преобразование Фурье} $S_N(f,x) :=
\intl{0}{N}\br{a(s)\cos sx + b(s)\sin sx}\,d s$. Имеем $S_n(f,x) = \frpi \intl{0}{N}\intlii
f(t)\br{\cos st \cos sx + \sin st \sin sx}\,d t \,d s=\frpi \intl{0}{N}\intlii f(t)\cos(t-x)s\,dt
\,ds =\frpi \intl{0}{N}\intlii f(x\bw+t)\cos ts \,d t \,d s = \lcomm$ поменяем порядок
интегрирования $\rcomm = \frpi \intlii f(x\bw+t)\intl{0}{N}\cos ts \,ds \,dt = \frpi \intlii
f(x+t) \frac{\sin Nt}{t} \,d t$ аналог ядра Дирихле.

Мы не обосновали законность смены порядка. Докажем, что $\intl{0}{N}\intlii f(x+t)\cos st \,d t
\,d s = \intlii \intl{0}{N}f(x+t)\cos st \,d s \,d t$. Заметим, что в наших рассуждениях $N$
фиксированное число. Сначала покажем, что если в каждом из интегралов заменить $f$ на близкую к ней
по норме функцию, ошибка от такой замены будет маленькой. Имеем $\fa \ep > 0 \exi \ph$ финитная
непрерывная функция, для которой $\|f - \ph\| < \ep$. В наших выкладках мы везде $|\cos st|$ будем
оценивать сверху числом 1.

Для первого интеграла $\bbm{\intl{0}{N}\intlii \br{f(x+t)-\ph(x+t)}\cos st \,d t \,d s} \le
\intl{0}{N}\intlii|f(x+t)-\ph(x+t)|\,d t \,d s\le \intl{0}{N}\ep \,d s = N\ep$. Для второго
$\bbm{\intlii\intl{0}{N}\br{f(x+t)-\ph(x+t)}\cos st\,d s\,d t} \le
\intlii\intl{0}{N}|f(x+t)-\ph(x+t)|\,d s\,d t = \intlii |f(x+t)-\ph(x+t)|\intl{0}{N}1\,d s\,d t
\le N\ep$.

Для финитных функций возможность смены порядка интегрирования очевидна, поскольку интегрирование происходит в собственном смысле.
Значит, имеем верное равенство для ступенчатых функций. Заменив $\ph$ на $f$, мы ошибёмся не больше, чем на $2N\ep$. Но эту величину
можно сделать сколь угодно малой, значит, равенство верно и для $f$.

\begin{theorem}[Равносходимость ряда и интеграла Фурье]
Пусть $f \in\LR(\R)$, а $f_0 \in\LR[-\pi,\pi]$ и $2\pi$-периодична. Если $f(x)=f_0(x)$ на
$[a,b]$, то $\fa [c,d] \subs (a,b)$ имеет место $S_N(f,x) - S_{[N]}(f_0,x) \convu{[c,d]} 0$.
\end{theorem}
\begin{proof}
Рассмотрим $\de \in (0,1)$, для которого если $x\in[c,d]$, то при $|t|<\de$ имеем $x+t \in (a,b)$.
Легко видеть, что $S_N(f,x) = \frpi \intl{-\de}{\de}f(x+t)\frac{\sin Nt}{t}\,d t + R_N(x)$, где
$R_N(x)\convu{[c,d]}0$ при $N\ra\infty$, и $S_n(f_0,x) \bw= \frpi \intl{-\de}{\de}f_0(x+t)D_n(t)\,d t
\bw+ r_n(t)$, где $r_n(x)\convu{[c,d]}0$ при $n \ra \infty$. При изучении явления Гиббса мы вывели
формулу $D_n(t) \bw= \frac{\sin nt}{t} \bw+ \sin nt \cdot g(t) \bw+ \frac{1}{2}\cos nt$, где $g \in\Cb$.
Поэтому $S_n(f_0,x)=\frpi \intl{-\de}{\de}f(x+t)\frac{\sin nt}{t}\,d t +
\frac{1}{2\pi}\intl{-\de}{\de}f(x+t)\cos nt\,d t + \frpi \intl{-\de}{\de}f(x\bw+t)g(t)\sin nt\,d t \bw+
r_n(x)$. Второй интеграл сходится к 0 равномерно, поскольку это коэффициент Фурье, а
последний вследствие ОЛР. Осталось доказать, что
$\intl{-\de}{\de}f(x+t)\frac{\sin Nt -\sin nt}{t}\,d t \convu{[c,d]}0$. Поскольку $n = [N]$, можно
написать $N = \al + n$, где $\al \in [0,1)$. Отсюда $\sin Nt-\sin nt =
2\sin\frac{\al}{2}t\cos\hr{N-\frac{\al}{2}}t$, а
$2\intl{-\de}{\de}f(x+t)\frac{\sin\frac{\al}{2}t}{t}\cos\hr{N-\frac{\al}{2}}t\,d t \convu{[c,d]}0$
по ОЛР II.
\end{proof}

Если имеет место сходимость, то $f(x) = \frpi \intl{0}{\infty}\intlii f(x+t)\cos st \,d t\,d s$.

Выведем вид ЧПФ в комплексной форме:
\begin{multline*}
S_N(f,x) = \frpi \intl{0}{N}\intlii f(x+t)\frac{e^{ist}+e^{-ist}}{2}\,dt\,d s=\\=
\frac{1}{2\pi}\intl{0}{N}\intlii f(x+t)e^{ist}\,d t \,d s \bw+
\frac{1}{2\pi}\intl{0}{N}\intlii f(x+t)e^{-ist}\,d t \,d s=\frac{1}{2\pi}\intl{-N}{N}\intlii f(x+t)e^{-ist}\,d t \,d s.
\end{multline*}
Отсюда $f(x) = \frac{1}{2\pi}\vp \intlii \intlii f(x+t)e^{-ist}\,d t
\,d s$. Сделав замену, получаем выражение:
\begin{multline*}
f(x) = \frac{1}{2\pi}\vp \intlii \intlii
f(t)e^{-is(t-x)}\,d t\,d s = \frac{1}{2\pi}\vp \intlii \intlii f(t)e^{-ist}\,d t e^{isx}\,ds=\\=
\vp \frac{1}{\sqrt{2\pi}}\intlii \hr{\frac{1}{\sqrt{2\pi}}\intlii f(t)e^{-ist}\,d t} e^{isx}\,d s.
\end{multline*}

Пусть $h(t) \in\LR(\R)$, тогда назовём прямым преобразованием Фурье функцию $\wt{h}(s) :=
\frac{1}{\sqrt{2\pi}}\intlii h(t)e^{-ist}\,d t$, а обратным преобразованием Фурье функцию
$\wh{h}(s) := \frac{1}{\sqrt{2\pi}}\intlii h(t)e^{ist}\,d t$. В случае сходимости получаем $f =
\wh{\wt{f}}$.

\begin{theorem}
Пусть $f \in\SegC^1(\R)$ и $f, f' \in\LR(\R)$. Тогда $\wt{f'}(s) = is\wt{f}(s)$.
\end{theorem}
\begin{proof}
По ФНЛ имеем $f(x)=f(0) + \intlox f'(t)\,d t$. Устремим в этом равенстве $x$ к $+\infty$.
Поскольку $f' \in\LR$, предел в левой части равенства существует, значит, он существует и в правой
части. Отсюда $\liml{x\ra+\infty} f(x) = 0$, иначе бы интеграл расходился. Аналогично получаем
$\liml{x\ra-\infty}f(x)=0$. Интегрируя по частям, получаем
$$\wt{f'}(s)=\frac{1}{\sqrt{2\pi}}\intlii f'(t)e^{-ist}\,d t =
\frac{1}{\sqrt{2\pi}}\hr{f(t)e^{-ist}\sbs{t=-\infty}{+\infty} - \intlii f(t)(-is)e^{-ist}\,d t}=
\frac{is}{\sqrt{2\pi}}\intlii f(t)e^{-ist}\,d t = is\wt{f}(s),$$
что и требуется.
\end{proof}

\begin{theorem}
Пусть $f \in\Cb(\R)$ и $f(t), tf(t) \in\LR(\R)$. Тогда $\hr{\wt{f}(s)}'=-i\wt{tf(t)}(s)$.
\end{theorem}
\begin{proof}
Имеем $\wt{f}(s) = \frac{1}{\sqrt{2\pi}}\intlii f(t)e^{-ist}\,d t$. Отсюда
$$\hr{\wt{f}(s)}'
=\frac{1}{\sqrt{2\pi}}\intlii f(t)(-it)e^{-ist}\,d t=\frac{-i}{\sqrt{2\pi}}\intlii
tf(t)e^{-ist}\,d t=-i\wt{tf(t)}(s),$$
что и требовалось доказать.
\end{proof}

\end{document}
