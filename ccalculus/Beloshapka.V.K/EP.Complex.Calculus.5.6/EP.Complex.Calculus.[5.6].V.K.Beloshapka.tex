\documentclass[a4paper]{article}
\usepackage[utf,simple]{dmvn}

\title{Программа экзамена по  комплексному анализу}
\author{Лектор-- В.\,К.\,Белошапка}
\date{5--6 семестры, 2004--2005 г.}
\begin{document}
\maketitle

\section*{V семестр}

\begin{nums}{-2}
\item $\Cbb$-- дифференцируемость и $\R$-- дифференцируемость, уравнения Коши-- Римана.
\item Теорема об обратной функции,  теорема о неявной функции.
\item Конформность линейного отображения, конформность в точке, конформное отображение области,
      связь с $\Cbb$-- дифференцируемостью.
\item Определенный интеграл, первообразная, формула Ньютона-- Лейбница.
\item Лемма Гурса, теорема о существовании первообразной.
\item Интегральная теорема Коши.
\item Интегральная формула Коши.
\item Теорема о разложении в ряд, неравенства Коши, теорема Лиувилля.
\item $\Cbb$-- дифференцирование степенных рядов.
\item Цепочка из четырех эквивалентных определения голоморфной функции, теорема Мореры.
\item Свойства голоморфных функций: бесконечная дифференцируемость, теорема единственности,
      принцип максимума, теорема о среднем.
\item Сходимость в пространстве голоморфных функций, теорема Вейерштрасса.
\item Принцип компактности.
\item Ряды Лорана, теорема о разложении  функции в кольце, единственность, неравенства Коши.
\item Изолированные особые точки, теорема о классификации, теорема Сохоцкого.
\item Сфера Римана как комплексное многообразие, голоморфность и мероморфность на комплексном
      многообразии, бесконечность как особая точка,  теорема: мероморфность на сфере эквивалентна рациональности.
\item Теорема о вычетах, вычисление вычетов.
\item Принцип аргумента, теорема Руше.
\item Принцип открытости, теорема Гурвица, эквивалентность  однолистности и конформности.
\item Попарная неэквивалентность сферы, плоскости и круга. Вычисление групп автоморфизмов
      сферы и плоскости, лемма Шварца, автоморфизмы круга.
\item Теорема Римана о конформном отображении.
\item Достижимая граничная точка, формулировка теоремы Каратеодори, <<обратная>> теорема о
соответствии границ, нормировка конформного отображения односвязной области.
\item Лемма о склейке голоморфных функций, принцип симметрии.
\item Аналитическое продолжение ростка (непосредственно и по пути), единственность продолжения по пути,
теорема о монодромии.
\item Полная Аналитическая Функция (ПАФ), её ветвь в области, однозначная ветвь, теорема Вольтерра.
\item Изолированная особая точка ветви ПАФ, классификация (корректность).
\item Конструкция римановой поверхности ПАФ, голоморфность ПАФ на своей римановой  поверхности.
\item Замыкание римановой поверхности в точке ветвления, ряды Пюизо.
\item Алгебраическая функция, критерий алгебраичности, группа монодромии.
\item Риманова поверхность алгебраической функции-- это сфера с ручками, формула Римана-- Гурвица.
\end{nums}

\pagebreak
\section*{VI семестр}
\begin{nums}{-2}
\item Гармонические функции двух переменных: связь с аналитическими, бесконечная дифференцируемость,
      теорема о среднем, принцип максимума, теорема Лиувилля.
\item Теорема о классификации изолированных особых точек гармонических функций.
\item Решение задачи Дирихле в односвязной области. Интеграл Пуассона.
\item Модель стационарного течения, комплексный потенциал и его типичные особые точки (источник,
      сток, вихрь, диполь), физическое доказательство теоремы Римана.
\item Кратные степенные ряды: лемма Абеля, область сходимости, полидиск сходимости, сопряженные
      радиусы сходимости, логарифмическая выпуклость области сходимости.
\item Голоморфные функции нескольких переменных: условия Коши-- Римана,  теорема о неявной функции.
\item Кратная интегральная формула Коши, разложение голоморфной функции в кратный степенной ряд.
\item Почленное дифференцирование кратного степенного ряда, голоморфность суммы. Четыре эквивалентных
      определения голоморфной функции.
\item Свойства голоморфных функций: теорема единственности, принцип максимума, неравенства Коши.
\item Области голоморфности, продолжение функций с помощью логарифмической выпуклости и с помощью
      интеграла Коши, фигура Хартогса.
\item Продолжение вдоль аналитических дисков.
\item Биголоморфные отображения, теорема Анри Картана.
\item Обобщенная лемма Шварца,  неэквивалентность шара и полидиска.
\item Представление мероморфных функций-- теорема Миттаг-- Лёффлера, метод Коши.
\item Представление целых функций-- теорема Вейерштрасса.
\item Строение группы периодов мероморфной функции, эллиптические функции и их свойства.
\item Построение функции Вейерштрасса, ее кратные точки и критические значения.
\item Две формы записи дифференциального уравнения на функцию Вейерштрасса, униформизация кубической кривой.
\item Разрешимость группы монодромии алгебраической функции, представимой суперпозицией радикалов.
\item Теорема Абеля о неразрешимости в радикалах алгебраического уравнения степени выше четвертой.
\item Построение универсальной накрывающей, действие фундаментальной группы на универсальной накрывающей.
\item Представление римановой поверхности как фактора универсальной накрывающей по действию
      фундаментальной группы. Теорема о классификации римановых поверхностей.
\item Отсутствие непостоянных отображений параболического многообразия в гиперболическое. Теорема Пикара.
\end{nums}

\pagebreak

\section*{Программа VI семестра}

\begin{nums}{-2}
\item Гармонические функции двух переменных и гидродинамика (\cite{shabat-first}, \cite{lavr-shabat})
  \begin{items}{-2}
    \item Гармонические функции и аналитические, свойства гармонических функций
    \item Интегральное представление Пуассона и задача Дирихле
    \item Модель стационарного течения, комплексный потенциал и его особые точки
          (источник, сток, вихрь, диполь), физическое доказательство теоремы Римана
  \end{items}
\item Представление голоморфных функций (\cite{shabat-first}, \cite{sid-fed-shab})
  \begin{items}{-2}
    \item Мероморфные функции-- теорема Миттаг-- Леффлера, метод Коши
    \item Целые функции-- теорема Вейерштрасса
  \end{items}
\item Функции нескольких переменных (\cite{shabat-second})
  \begin{items}{-2}
    \item Кратные степенные ряды
    \item Цепочка эквивалентных определений: (производная, кратная формула Коши,
           представление суммой ряда, производная), интегрирование в $\Cbb^n$
    \item Свойства голоморфных функций
    \item Геометрия пространства $\Cbb^n$
    \item Области голоморфности: логарифмическая выпуклость, фигура Хартогса,
             голоморфная выпуклость,  аналитические диски
    \item Простейшие особенности голоморфных и аналитических функций
    \item Биголоморфные отображения
  \end{items}
\item Эллиптические функции (\cite{shabat-first}, \cite{lavr-shabat}, \cite{gurvitz-courant})
  \begin{items}{-2}
    \item Определение и свойства эллиптических функций
    \item Эллиптический синус
    \item Построение функции Вейерштрасса и ее свойства
    \item Униформизация кубической кривой
  \end{items}
\item Римановы поверхности (\cite{gurvitz-courant}, \cite{alexeev})
  \begin{items}{-2}
    \item Теорема Абеля о неразрешимости в радикалах
    \item Алгебраические кривые в $\CP^2$
    \item Анализ на римановой поверхности
    \item Униформизация римановых поверхностей
  \end{items}
\end{nums}
\medskip

\begin{thebibliography}{7}
\setlength{\itemsep}{-2pt}
\bibitem{shabat-first} Б.\,В.\,Шабат. \emph{Введение в комплексный анализ}, том\,1.-- Наука, 1976.
\bibitem{shabat-second} Б.\,В.\,Шабат. \emph{Введение в комплексный анализ}, том\,2.-- Наука, 1985.
\bibitem{lavr-shabat} М.\,А.\,Лаврентьев, Б.\,В.\,Шабат. \emph{Методы ТФКП}.-- Физматгиз, 1958.
\bibitem{gurvitz-courant} А.\,Гурвиц, Р.\,Курант. \emph{Теория функций}.-- Наука, 1968.
\bibitem{alexeev} В.\,Б.\,Алексеев, \emph{Теорема Абеля в задачах и решениях}.-- Наука, 1976.
\bibitem{sid-fed-shab} Ю.\,В.\,Сидоров, М.\,В.\,Федорюк, М.\,И.\,Шабунин. \emph{Лекции по ТФКП}.-- Наука, 1989.
\bibitem{kra} И.\,Кра. \emph{Автоморфные формы и клейновы группы}.-- Мир, 1975.
\end{thebibliography}

\medskip\dmvntrail

\end{document}


%% Local Variables:
%% eval: (setq compile-command (concat "latex  -halt-on-error -file-line-error " (buffer-name)))
%% End:
