\documentclass[a4paper]{article}
\usepackage[russian]{babel}
\usepackage[utf8]{inputenc}
\usepackage{dmvn}

\begin{document}

\dmvntitle{Курс лекций по}{комплексному анализу}{Лектор Евгений Прокофьевич Долженко}
{III курс, 5 семестр, поток математиков}{Москва, 2004 г.}
\pagebreak

\tableofcontents

\pagebreak

\section*{Предисловие}

Народ! Предупреждаем сразу: если вы хотите \emph{знать} комплексный анализ,
ботайте его по Шабату или Белошапке. Если вы хотите \emph{сдать} экзамен,
то можете рискнуть и заботать по этим лекциям.
Но помните, что бреда здесь ещё больше, чем у Скляренко.

В некоторых случаях <<затыкать дыры>> в доказательствах полностью было бессмысленно, поскольку без
нормальных и строгих определений это сделать просто невозможно. Например, в теореме Коши
невозможно обойтись без таких понятий, как фундаментальная группа и гомотопия. Ну, а такого
бредового <<определения>> римановой поверхности, которое было дано на лекциях, мы нигде
не слышали.

Огромное спасибо Лене Хиль за то, что она героически записывала эти лекции.

Вопросы, комментарии, замечания и предложения направляйте на \dmvnmail{}, обновления
электронной версии на сайте \dmvnwebsite{}.

\hfill Набор и вёрстка: \texttt{DMVN Corporation}

\hfill Последняя компиляция: \texttt{\today}

\hfill Версия: \texttt{0.82}

\pagebreak

\section{Введение. Основные понятия}

\subsection{Введение}

\subsubsection{Комплексные числа, комплексная плоскость}

\begin{df}
\emph{Комплексными числами} называются упорядоченные пары действительных чисел, на которых заданы операции:

\begin{points}{-2}
\item Сложение покомпонентно;
\item Умножение по правилу $(x_1, y_1)\cdot(x_2, y_2) := (x_1x_2 - y_1 y_2, x_1y_2 + x_2y_1)$.
\end{points}
\end{df}

\begin{stm}
Множество комплексных чисел $\Cbb$ образует поле.
\end{stm}

Рассмотрим пары вида $(х, 0) = х$. Обозначим $i := (0,1)$. Тогда $(x,y) = x+iy$. Это и есть алгебраическая
форма записи. Точнее говоря, это запись элемента $(x,y)$ в базисе $(1,i)$ множества $\Cbb$ как двумерной
алгебры над полем $\R$.

Заметим, что $i^2= -1$. Действительно, имеем $(0,1)\cdot(0,1) = (0 - 1, 0 + 0) = (-1,0)$.

Пусть $z = x + iy$, тогда $\Rea z := x$ \emph{действительная} часть, $\Img z := y$ \emph{мнимая} часть,
а $\ol z := x - iy$ число, \emph{сопряжённое к} числу~$z$. \emph{Модулем} числа $z$ называется число $|z| := \sqrt{x^2+y^2}$.

\begin{theorem}[Гаусса]
Поле комплексных чисел алгебраически замкнуто, то есть любой многочлен положительной степени
имеет хотя бы один комплексный корень.
\end{theorem}

Имеют место равенства:

\begin{points}{-2}
\item $\ol{z_1 + z_2} = \ol z_1 + \ol z_2$;
\item $\ol{z_1\cdot z_2} = \ol z_1 \cdot \ol z_2$;
\item $\ol{\ol z} = z$;
\item $z\cdot\ol z = \hm{z}^2$;
\item $\hm{z_1 \cdot z_2 } = \hm{z_1}\cdot \hm{z_2}$.
\end{points}

\subsubsection{Модуль и аргумент комплексного числа}

\begin{df}
\emph{Аргументом} числа $z\neq 0$ называется число $\arg z$, выражающее угол между осью $Oх$ и вектором $\vec z =(x,y) \in \R^2$.
Также аргументом называется множество
\eqn{\Arg z := \hc{\arg z + 2\pi k \vl k \in \Z}.}
\end{df}

\begin{note}
Аргумент нуля не определён!
\end{note}

Аргумент является простейшим примером многозначной функции. При его определении следует договориться, в каких
пределах меняется число $\arg z$. Разумно считать, что $\arg z \in [0,2\pi)$, то есть для
$\fa \al \in \Arg z$ имеем $\al \equiv \arg z \pmod{2\pi}$.


На комплексных числах норма вводится естественным образом: $\hn{z} := |z|$, то есть поле $\Cbb$ является нормированным.
Другие примеры нормированных полей доставляют $\R$ и $\Q$.

Неравенство треугольника $\bm{|z_1|-|z_2|}\le |z_1+z_2|\le|z_1|+|z_2|$, которое позволяет корректно ввести норму
и метрику на $\Cbb$, легко доказать, исходя из геометрической интерпретации комплексных чисел, о которой
речь пойдёт ниже.

\subsubsection{Геометрическая интерпретация комплексных чисел}

Поставим в соответствие числу $z = x + iy$ вектор $(x,y)$ на плоскости $\R^2$.

\begin{items}{-2}
\item Сложение комплексных чисел это сложение соответствующих векторов;
\item Операция сопряжения равносильна симметрии относительно оси $Oх$;
\item $\Rea(z_1\ol z_2) = x_1 x_2 + y_1 y_2$ скалярное произведение векторов;
\item $\Img(z_1\ol z_2) = x_1 y_2 - y_1 x_2 = \mbmat{x_1 & x_2 \\ y_1 & y_2}$ ориентированная площадь
параллелограмма на векторах $z_1$ и $z_2$.
\end{items}

Геометрический смысл умножения будет ясен позже.

Некоторые геометрические места точек на плоскости удобно записывать с помощью комплексных чисел.

\begin{ex}
\begin{items}{-2}
\item $\hm{z - z_0} = \const$ окружность с центром в точке $z_0$;
\item $\hm{z - z_1} = \hm{z - z_2 }$ серединный перпендикуляр к отрезку $z_1z_2$;
\item $\hm{z - z_1} + \hm{z - z_2 } = \const$ эллипс с фокусами $z_1$ и $z_2$.
\end{items}
\end{ex}

\subsubsection{Тригонометрическая и показательная форма записи комплексного числа}

Пусть $z = x+iy$. Тогда в полярных координатах имеем $x = r \cos \ph$ и $y = r\sin \ph$. Значит,
\eqn{z = x + i y = r \hr{\cos \ph + i\sin \ph}.}

\begin{df}
Положим $e^{i\ph} := \cos \ph + i\sin \ph$.
\end{df}

\begin{stm}
При умножении модули перемножаются, а аргументы складываются.
\end{stm}
\begin{proof}
Действительно,
\begin{multline*}
r_1 \hr{\cos \ph_1 + i\sin \ph_1}r_2 \hr{\cos \ph_2 + i\sin \ph_2} =\\=
r_1 r_2 \br{\hr{\cos \ph_1 \cos \ph_2 - \sin \ph_1 \sin \ph_2} +
i\hr{\cos \ph _1 \sin \ph _2 + \sin \ph _1 \cos \ph _2}} =\\=
r_1 r_2 \br{\cos(\ph_1 + \ph _2) + i\sin(\ph_1 + \ph_2)}.
\end{multline*}

Пусть $\arg z_1 = \ph_1$ и $\arg z_2 = \ph_2$. Тогда
\eqn{\Arg(z_1z_2) = \hc{\ph_1 + \ph_2 + 2\pi k}.}
Под сложением аргументов <<$\arg$>> мы понимаем их сложение по модулю $2\pi$.
\end{proof}

\begin{imp}
Геометрический смысл преобразования $z \mapsto az$, $a = r e^{i\ph}$ это композиция гомотетии с коэффициентом $r$
и поворота на угол $\ph$ вокруг начала координат.
\end{imp}

\begin{imp}
Геометрический смысл числа $\arg\frac{z_1 - z_0}{z_2 - z_0}$ ориентированный угол между векторами $z_1-z_0$ и $z_2-z_0$.
\end{imp}

\begin{imp}[Формула Муавра]
$\hr{re^{i\ph}}^n = r^ne^{in\ph}$.
\end{imp}

\begin{df}
\emph{Корнем $n$ й степени} из комплексного числа $z$ называется всякое число $w$, для которого $w^n = z$.
Множество корней из $z$ обозначается $\sqrt[n]z$.
\end{df}

\begin{stm}
Если $z = 0$, то и $\sqrt[n]{z} = 0$. Если же $z \neq 0$, то существует ровно $n$ различных корней
из этого числа, и
\eqn{\sqrt[n]{z} = \sqrt[n]{r}\cdot \exp\hr{i\frac{\ph + 2k\pi }{n}}, \quad k = 0 \sco n - 1.}
\end{stm}

\begin{ex}
$\sqrt[3]{1} = \cos \frac{2k\pi}{3} + i\sin \frac{2k\pi}{3}$, то есть
$\hc{\sqrt[3]{1}} = \hc{1,\, \frac{1}{2} + i\frac{\sqrt3}{2},\, \frac{1}{2} - i\frac{\sqrt3}{2}}$.
\end{ex}

\subsubsection{Многозначные функции и однозначные ветви на примере $\Arg z$}

Одной из самых простых многозначных функций является функция

\eqn{\Arg z := \hc{\arg z + 2\pi k \vl k \in \Z}.}

Для многозначных функций запись <<$f(z)$>> обозначает множество значений в точке $z$.

\begin{df}
Функция $f$ многозначна, если $\Card f(z) > 1$ для хотя бы одной точки $z\in \Dc(f)$.
\end{df}

\begin{df}
Многозначные функции $f$ и $g$ совпадают на множестве $E$, если $f(z) = g(z)$ для $\fa z \in E$.
\end{df}

\begin{df}
Функция $f_0(z)$ однозначная ветвь многозначной функции $f(z)$, если
$f_0(z) \in f(z)$ для $\fa z \in \Dc(f)$.
\end{df}

\begin{df}
Говорят, что $f_0(z)$ однозначная непрерывная ветвь многозначной функции $f(z)$, если
$f_0(z)$ однозначная ветвь и $f_0$ непрерывна на $\Dc(f)$.
\end{df}

\subsubsection{Метрика и топология $\Cbb$. Последовательности и пределы}

На множестве $\Cbb$ определено расстояние $\rh(z,w) := |z-w|$. Под сходимостью будем понимать
сходимость в этой метрике.

Определения открытых и замкнутых множеств, окрестностей тд полностью совпадают с определениями этих понятий,
данных в курсе математического анализа. Топология $\Cbb$, задаваемая метрикой, совпадает со стандартной топологией $\R^2$.
Таким образом, комплексная прямая\footnote{Нехорошо называть плоскостью одномерное векторное пространство над полем $\Cbb$.}
является хаусдорфовым топологическим пространством (любые две несовпадающие точки обладают непересекающимися окрестностями).

\begin{df}
\emph{Окрестностью} точки в $\Cbb^n$ называется открытый шар с центром в этой точке.
\end{df}

\begin{df}
Множество $K\subs \Cbb^n$ называется \emph{компактным}, если из любой последовательности
его точек можно выделить подпоследовательность, сходящуюся к некоторой точке $z \in K$.
\end{df}

Как следует из теоремы Больцано Вейерштрасса, компакты в $\Cbb^n$ это в точности замкнутые ограниченные множества.

\begin{df}
\emph{Расстоянием} между подмножествами $X$ и $Y$ метрического пространства называется число
\eqn{\rh(X,Y) := \inf\hc{\rh(x,y) \vl x\in X, \; y \in Y}.}
\end{df}

\begin{stm}
Пусть $F_1$ и $F_2$ непересекающиеся замкнутые множества в $\Cbb$, и хотя бы одно из них компактно.
Тогда $\rh(F_1, F_2) > 0$.
\end{stm}
\begin{proof}
Допустим, $\rh(F_1,F_2)=0$. Тогда найдутся последовательности $\hc{x_n}$ и $\hc{y_n}$ такие, что $\rh(x_n,y_n)\ra 0$.
Без ограничения общности $F_1$ компактно. Тогда перейдём к сходящейся подпоследовательности (и перенумеруем).
Теперь можно считать, что $x_n\ra x_0$ и $y_n\ra x_0$. Но тогда в силу замкнутости $x_0\in F_1$, а так как $x_0$ будет
предельной точкой для $F_2$, то $x_0\in F_2$. Противоречие.
\end{proof}

\begin{note}
Условие компактности здесь по существу. Если потребовать только замкнутости, то утверждение перестаёт быть верным.
Например, рассмотрим график функции $e^x$ и ось $Ox$. Эти множества замкнуты и не пересекаются, но расстояние
между ними равно нулю.
\end{note}

\begin{problem}
Найти множества предельных точек для последовательностей
\eqn{\exp \hr{2\pi i e \cdot n!}, \quad \exp \hr{\pi i \cdot n\sqrt 2}, \quad \prodl{k=1}{n}\hr{1 + \frac{i}{k}}.}
\end{problem}

Непрерывность функций определяется точно так же, как и в математическом анализе.

\begin{ex}
Непрерывными комплексными функциями являются
\begin{items}{-2}
\item Многочлены $f(z) \in \Cbb[z]$;
\item Дробно рациональные функции $w(z) \in \Cbb(z)$;
\item $e^z = e^{x+iy} := e^x(\cos y + i\sin y)$ периодическая функция с периодом $2\pi i$;
\item $\cos z := \frac{e^{iz} + e^{- i z}}{2}$ и $\sin z := \frac{e^{iz} - e^{ - iz}}{2i}$.
\end{items}
\end{ex}

Многие теоремы о рядах, доказанные в курсе математического анализа, очевидным образом обобщаются на
комплексный случай.

\begin{problem}
Исследовать на сходимость ряд
\eqn{\sumnui \frac{(-1)^{\hs{\sqrt n}}}{n} z^n.}
\end{problem}

Как мы потом увидим, можно пополнить пространство $\Cbb$ бесконечно удалённой точкой. Тогда, по определению,
$z_n\ra\bes \; \Lra \; |z_n|\ra\bes$. Более подробно о топологии расширения $\ol\Cbb$ будет сказано в главе
о стереографической проекции.

\subsubsection{Кривые, пути и области}

\begin{df}
\emph{Кривая} непрерывное отображение $\ga\cln [a,b] \ra \Cbb$.
Кривая называется \emph{жордановой}, если она не имеет самопересечений. Кривая называется \emph{замкнутой},
если $\ga(a) = \ga(b)$.
\end{df}

\begin{df}
Множество $E$ называется связным, если его нельзя разбить на два непустых
открытых относительно $Е$ подмножества.
Множество $Е$ называется \emph{линейно связным}, если любые две его точки можно соединить непрерывной кривой в $Е$.
\end{df}

\begin{note}
С тем же успехом в определении связного множества можно открытые множества заменить на замкнутые.
\end{note}

\begin{df}
\emph{Область} непустое открытое связное множество.
\end{df}

\begin{stm}
Если множество линейно связно, то оно связно.
\end{stm}
\begin{proof}
В самом деле, допустим, что множество $E$ линейно связно и несвязно, то есть разбивается на два открытых: $E = A \sqcup B$.
Пусть $x \in A$, а $y \in B$. Соединим эти точки непрерывной кривой $\ga\cln[0,1]\ra E$, причём $\ga(0)=x$, а $\ga(1)=y$.
Рассмотрим $t_A := \sup\hc{t \in [0,1] \cln \ga(t) \in A}$. Если $\ga\hr{t_A}$ лежит в $B$, то лежит там вместе с
окрестностью. Так как прообраз открытого множества при непрерывном отображении открыт, то точка $t_A$
отображается в~$B$ вместе со своей окрестностью в $[0,1]$, то есть интервалом. Но это противоречит определению~$t_A$.
Если же $t_A \in A$, то по аналогичным соображениям получаем противоречие с тем, что $t_A$ есть точная верхняя грань.
\end{proof}

\begin{problem}
Докажите, что замыкание связного множества $M$ связно.
\end{problem}
\begin{solution}
Допустим, что $\Cl M = A \sqcup B$. Положим $A_M := A \cap M$, а $B_M := B \cap M$.
В силу определения индуцированной топологии, множества $A_M$ и $B_M$ открыты в $M$.
Они, очевидно, не пересекаются, поэтому одно из них пусто, иначе $M$ не было бы связным.
Пусть, например, $A_M$ пусто. Тогда $A$ не содержит точек из $M$ и является открытым подмножеством
в замыкании. Но это противоречит определению замыкания.
\end{solution}


\begin{problem}
Рассмотрим замыкание графика функции $y = \sin \frac{1}{x},\; x > 0$ на плоскости. Это эквивалентно
добавлению отрезка $[-1,1]$ на оси $Oу$. В силу предыдущей задачи это множество связно.
Докажите, что оно но не линейно связно.
\end{problem}

Итак, мы видим, что линейная связность более сильное условие. Однако, как показывает
следующая теорема, для областей эти понятия равносильны.

\begin{theorem}
Область $D$ в $\Cbb^n$ является линейно связным множеством.
\end{theorem}
\begin{proof}
Рассмотрим произвольные точки $x$ и $y$ в области $D$ и докажем, что их можно соединить кривой.
Пусть~$E$ множество тех точек, до которых можно дойти из точки~$x$. Оно непусто, так как~${x \in E}$, и открыто,
так как если ${z \in E}$, то $U(z) \subs D$, а окрестность~$U$ линейно связна, поэтому $U(z) \subs E$. По
аналогичным соображениям $D \wo E$ открыто. Таким образом, мы разбили~$D$ на два
непересекающихся открытых множества, что противоречит связности~$D$.
\end{proof}

\begin{theorem}
Открытое множество в $\Cbb^n$ разбивается на не более чем счётное число непересекающихся областей.
\end{theorem}
\begin{proof}
Рассмотрим открытое множество $E$ и разобьём его на классы эквивалентности относительно линейной связности.
Очевидно, что такое отношение будет симметричным, рефлексивным и транзитивным. Каждый класс будет областью.
В каждом классе, очевидно, есть точка с рациональными координатами, значит, самих классов не более чем
счётное число.
\end{proof}

\begin{note}
Вообще говоря, следствие справедливо в полном сепарабельном метрическом пространстве.
\end{note}

\begin{df}
\emph{Континуумом} называется связный компакт, состоящий более чем из одной точки.
\end{df}

\begin{stm}
Континуум имеет мощность континуум, чем и оправдывает своё название.
\end{stm}
\begin{proof}
Рассмотрим континуум~$K$. Пространство $\Cbb^n$ является полным пространством,
а замкнутое подмножество полного пространства полно. Таким образом, $K$~является
полным метрическим пространством. Если~$K$ не более чем счётно, то его
можно представить в виде не более чем счётного объединения нигде не плотных множеств его точек.
Тогда~$K$ было бы множеством первой категории, но по теореме Бэра это невозможно.
\end{proof}

\begin{df}
Область $G$ называется \emph{$n$ связной}, если $\pd G$ распадается на $n$ континуумов, и не распадается на
большее их число.
\end{df}

Несложно придумать пример ограниченной области с бесконечным числом компонент границы. Возьмём единичный
круг и выкинем из него последовательность кругов радиуса $\frac{1}{10^n}$ с центрами в точках $\frac{1}{2^n}$, $n\in\N$.
Они заведомо не пересекаются, и граница каждого из них является континуумом.

\begin{theorem}[Жордана]
Любая замкнутая жорданова кривая разбивает плоскость на два непересекающихся множества, границей каждого
из которых она является.
\end{theorem}
Доказательства этой теоремы в нашем курсе не будет.

\subsubsection{Кривая Пеано и жорданова кривая положительной площади}

\emph{Кривая Пеано} является отображением отрезка $[0,1]$ на квадрат $[0,1]^2$. На рисунке показано 3 итерации её
построения. Каждая следующая итерация может быть получена из предыдущей таким образом: берём образ при $n$ й
итерации, уменьшаем вдвое по каждому измерению, а затем 4 копии этого множества располагаем во всех четвертях
единичного квадрата. При этом первая копия поворачивается на $-\frac{\pi}{2}$, последняя на $\frac{\pi}{2}$.
Затем соединяем отрезками соответствующие концы кусков кривой. Параметризуется эта кривая очевидным образом: отрезок
$[0,1]$ дробится на $4^n$ частей, каждая из которых параметризует прямолинейный кусочек кривой Пеано.
$$\epsfbox{pictures.10}\quad\quad\epsfbox{pictures.20}\quad\quad\epsfbox{pictures.30}$$

\begin{stm}
На плоскости существует жорданова кривая кривая положительной площади.
\end{stm}

\begin{petit}
Честно говоря, я не понял, как оно там строится. Кроме того, не совсем ясно, почему получаемое
множество вообще измеримо. Есть хорошая (но редкая) книжка <<Контрпримеры в анализе>>,
там есть этот пример, но её у меня в данный момент нет, и переписать доказательство неоткуда.
\end{petit}

\subsubsection{Спрямляемые кривые. Натуральная параметризация}

\begin{df}
\emph{Длина} кривой $\ga\cln [a,b]\ra\Cbb$ это число
\eqn{|\ga| := \sup \suml{k=1}{n} |\ga(t_k)-\ga(t_{k-1})|,}
где точная верхняя грань берётся по всем разбиениям отрезка $[a,b]$ точками $a=t_0,\sco t_n=b$.
\end{df}

\begin{df}
Говорят, что кривая \emph{спрямляема}, если её длина конечна.
\end{df}

\begin{df}
Говорят, что кривая $\ga$ имеет \emph{натуральную} параметризацию, если $[a,b] = [0,|\ga|]$, и для любых точек
$s_1$ и  $s_2$ выполнено неравенство
\eqn{|z(s_1) - z(s_2)| \le |s_1-s_2|.}
\end{df}

Для <<хороших>> кривых у этого неравенства есть очень простой смысл: $|\dot z(t)|\equiv 1$.

\begin{petit}
Куда более практичным является следующее определение натуральной параметризации:

\begin{df}
Пусть $\ga\cln [0,L] \ra \Cbb$. Параметризация называется \emph{натуральной}, если для всякого $s \in [0,L]$
длина ограничения кривой на отрезок $[0,s]$ равна $s$.
\end{df}
\end{petit}


\begin{df}
Две параметризации одной и той же кривой $z(t_1)\cln[\al_1,\be_1]\ra \Cbb$ и $z(t_2)\cln[\al_2,\be_2]\ra \Cbb$
назовём \emph{эквивалентными}, если существует непрерывная и строго монотонная
функция $\ta\cln [\al_1,\be_1]\mapsto [\al_2,\be_2]$, такая что $\ta(\al_1)=\al_2, \; \ta(\be_1)=\be_2$
и $z(t_2) = z\br{\ta(t_1)}$.
\end{df}

\begin{problem}
Доказать эквивалентное определение длины кривой:
\eqn{|\ga| = \liml{\la_T\ra 0} \suml{k=1}{n} |z(t_k)-z(t_{k-1})|.}
\end{problem}

\begin{problem}
Доказать, что спрямляемая кривая имеет касательную почти всюду.
\end{problem}
\begin{hint}
Здесь нужно вспомнить два факта: во первых, спрямляемая кривая это функция
ограниченной вариации, и во вторых, теорему из курса действительного анализа
о том, что функции ограниченной вариации дифференцируемы почти всюду.
Отсюда сразу следует утверждение задачи, но доказательство второго факта
отнюдь не тривиально и потому здесь не приводится.
\end{hint}

\begin{problem}
Доказать, что длина кривой не зависит от параметризации.
\end{problem}
\begin{hint}
Нужно воспользоваться тем, что для каждой частичной суммы при одной параметризации
можно найти такое разбиение второго отрезка параметризации, что эти суммы совпадут.
\end{hint}

\begin{df}
Кривая $\ga$ называется \emph{гладкой}, если функция $\ga$ дифференцируема
на $[a,b]$ и $\ga'(t) \ne 0$ для $\fa t \in [a,b]$.
\end{df}

\begin{problem}
Доказать, что длина гладкой кривой $\ga$ вычисляется по формуле
\eqn{|\ga| = \intl{a}{b}{\hm{\ga '(t)}}\,dt.}
\end{problem}

\begin{note}
В дальнейшем мы будем рассматривать только \emph{правильные} параметризации кривых, то есть $\ga(t)\neq\const$ ни на
каком интервале отрезка параметризации.
\end{note}

\begin{problem}
Доказать, что для любой параметризации кривой существует эквивалентная правильная параметризация.
\end{problem}

\subsection{Стереографическая проекция}

\subsubsection{Формулы стереографической проекции}

Рассмотрим сферу $S^2$ единичного радиуса с центром в начале координат и плоскость $\pi := Oxy$.
Пусть $P(0,0,1)$ северный полюс сферы. Возьмём на сфере любую точку $М$ и через точки $Р$ и $М$
проведём прямую, которая при условии $M \neq P$ пересекает $\pi$ в точке $М'$. Это так называемая
\emph{стереографическая} проекция сферы на плоскость. Очевидно, отображение взаимно
однозначное. Получим координатные формулы для стереографической проекции.

Пусть $M$ имеет координаты $(\xi, \eta, \ze)$, а $M'$ имеет координаты $(x,y)$ на плоскости.
Напишем уравнение прямой, проходящей через точки $P(0,0,1)$ и $M'(x,y,0)$:
\eqn{\frac{\xi}{x} = \frac{\eta}{y} = \frac{\ze-1}{-1}.}
Преобразуя его, получаем
\eqn{\xi = x(1-\ze), \quad \eta = y(1-\ze).}
Отсюда получаем выражения для координат точки $M'$:
\eqn{x = \frac{\xi}{1-\ze}, \quad y = \frac{\eta}{1-\ze}.}
Пусть $z=x+iy$. Тогда
\eqn{z = \frac{\xi+i\eta}{1-\ze}.}
Так как точка $M$ лежит на сфере, то $\xi^2+\eta^2+\ze^2=1$. Учитывая это и подставляя
выражения для $\xi$ и $\eta$, получаем выражения для координат точки $M$ через координаты точки $M'$:
\eqn{\xi  = \frac{2x}{x^2+y^2+1}, \quad \eta = \frac{2y}{x^2+y^2+1}, \ze = \frac{x^2+y^2-1}{x^2+y^2+1}.}
В комплексных числах совсем просто:
\eqn{\xi+i\eta = \frac{2z}{|z|^2+1}, \quad \ze = \frac{|z|^2-1}{|z|^2+1}.}

\begin{note}
Можно также рассматривать другие виды стереографической проекции. Например, можно положить сферу на плоскость так, чтобы
она касалась её южным полюсом.  При этом немного изменятся формулы, но все свойства останутся прежними.
\end{note}

\subsubsection{Расширенная комплексная плоскость}

При стереографической проекции одна из точек на сфере, а именно северный полюс, не имеет образа на плоскости.
Если точка на плоскости стремится к бесконечности (по модулю), то её прообраз на сфере, очевидно, стремится
к полюсу. Это наблюдение побуждает выполнить совершенно естественную операцию: добавить к~$\Cbb$ бесконечно
удалённую точку. Тогда стереографическая проекция будет биективным отображением $\Cbb \cup \hc{\bes}$ на~$S^2$.
Окрестностями $\bes$ будем назвать множества точек $\hc{z \in \Cbb \cln |z| > r}$. Построенное таким образом
топологическое пространство будем обозначать через $\ol\Cbb$. Отображение стереографической проекции, доопределённое
в точке $P$, будет гомеоморфизмом. Пополненную комплексную плоскость часто называют \emph{сферой Римана}.

\begin{note}
Операция, которая была проделана выше, называется \emph{компактификацией}, так как некомпактное многообразие
$\Cbb$ превращается в компактное $S^2$. Иногда его ещё называют \emph{проективизацией}, ибо полученное
множество, как легко видеть, есть не что иное, как комплексная проективная прямая $\CP^1$.
\end{note}

\begin{df}
\emph{Сферическим расстоянием} между двумя точками на $\ol{\Cbb}$ называется расстояние между соответствующими точками
на сфере (длина дуги большого круга). \emph{Хордальным расстоянием} между точками на $\ol\Cbb$ называется
длина хорды, соединяющей их образы на сфере.
\end{df}

\begin{note}
Хордальная метрика плоха тем, что она не является инвариантной относительно сдвигов: вообще говоря,
$\rh(x+z,y+z)\neq \rh(x,y)$.
\end{note}

\begin{note}
Любая рациональная функция непрерывно отображает $\ol{\Cbb}$ в $\ol{\Cbb}$.
\end{note}

\begin{ex}
Рассмотрим отображение $w = \frac{1}{z}$, доопределим его: $0 \lra \bes$. Тогда оно будет непрерывным.
\end{ex}

\begin{problem}
Вывести формулу для сферического расстояния.
\end{problem}

\subsubsection{Свойства стереографической проекции}

В следующей теореме прямую на плоскости мы считаем окружностью бесконечного радиуса.

\begin{theorem}[Круговое свойство]
Стереографическая проекция осуществляет биекцию между окружностями на сфере Римана и окружностями на плоскости.
\end{theorem}
\begin{proof}
Запишем общее уравнение окружности на плоскости:
\eqn{A(x^2+y^2)+Bx+Cy+D=0.}
Подставляя координаты стереографической проекции, получаем
\eqn{A\frac{\xi^2+\eta^2}{(1 -\ze)^2} + B\frac{\xi}{1 -\ze} + C\frac{\eta}{1 - \ze} + D = 0.}
Теперь вспомним, что $\xi^2+\eta^2 = 1-\ze^2$. Подставим это выражение в уравнение и преобразуем его:
\eqn{B\xi+C\eta+(A-D)\ze + (A+D) = 0.}
Получилось линейное уравнение, задающее некоторую плоскость. В пересечении этой плоскости со сферой
Римана получится окружность.

Остаётся заметить, что образом прямой на плоскости будет окружность на сфере, проходящая через северный полюс $P$.
Действительно, если $A=0$, то точка $(0,0,1)$ удовлетворяет полученному уравнению.
\end{proof}

\subsubsection{Постоянство растяжений}

Сейчас мы покажем, что при стереографической проекции имеет место так называемое постоянство растяжений:
бесконечно малые вектора $dz$ при проекции растягиваются <<одинаково по всем направлениям>>, то есть
$|dM(z)| = K(z)|dz|$, где функция $K(z)$ зависит только от точки $z$ и не зависит от направления вектора $dz$.

\begin{petit}
Предварим эти формулы одной фразой из курса дифференциальной геометрии: Именно, покажем, что стереографическая проекция
индуцирует на сфере конформно евклидову метрику.
\end{petit}

\begin{stm}
Для стереографической проекции функция $K(z)$ равна $\frac{2}{1+|z|^2}$.
\end{stm}
\begin{proof}
Имеем
\begin{align}
d\xi &= \pf{\xi}{x}\,dx+\pf{\xi}{y}\,dy =\phm\frac{2(y^2-x^2+1)}{(1+x^2+y^2)^2}\,dx - \frac{4xy}{(1+x^2+y^2)^2}\,dy,\\
d\eta &= \pf{\eta}{x}\,dx+\pf{\eta}{y}\,dy =- \frac{4xy}{(1+x^2+y^2)^2}\,dx+ \frac{2(x^2-y^2+1)}{(1+x^2+y^2)^2}\,dy,\\
d\ze &= \pf{\ze}{x}\,dx+\pf{\ze}{y}\,dy = \phm\frac{4x}{(1+x^2+y^2)^2}\,dx + \frac{4y}{(1+x^2+y^2)^2}\,dy.\\
\end{align}
Отсюда
\eqn{|d M(z)|^2 = (d\xi)^2+(d\eta)^2+(d\ze)^2 = \frac{4(dx^2+dy^2)}{(1+x^2+y^2)^2}\quad \Ra \quad |d M(z)| = \frac{2}{1+|z|^2}\cdot |dz|.}
\hfill\end{proof}

\subsubsection{Угол с вершиной в бесконечности}

Пусть на плоскости заданы две ориентированные прямые, проходящие через точку $a$, и угол между ними равен $\al$.
Напомним, что углом между прямыми называется такой угол, при повороте на который вокруг точки $a$ одна из
прямых переходит в другую (так, чтобы ориентации совпали).

Пусть $a= 0$. Сделаем стереографическую проекцию этих прямых. Они перейдут в два больших круга $\om_1$ и $\om_2$ сферы,
проходящих через северный полюс $P$. Очевидно, что угол между окружностями по модулю равен $\al$, а по ориентации
противоположен ему, то есть равен $-\al$. Это число мы и примем за определение угла с вершиной в
бесконечности. Покажем, что эта величина не зависит от точки $a$. В самом деле, осуществим параллельный перенос на
плоскости так, чтобы точка $a$ переехала в нуль. Угол и его ориентация при этом не изменится, а при проекции поменяется
только знак.

Разумно поставить вопрос: а чему равен угол между параллельными прямыми? Стереографические проекции параллельных прямых
имеют общую касательную, а потому угол между ними равен нулю.

\begin{df}
Углом между кривыми в бесконечности называется угол между их касательными в точке~$\bes$, если смотреть из центра сферы.
\end{df}

\begin{df}
Отображение двумерных  ориентируемых поверхностей называется \emph{конформным}, если оно обладает свойствами постоянства
растяжений и сохранения углов по модулю и ориентации.
\end{df}

Если меняется ориентация углов, то говорят, что это конформное отображение \emph{второго рода}.

\begin{note}
Из конформности в каждой точке не вытекает глобальная конформность. Пример: $e^z$ конформна в каждой точке $\Cbb$.
Но оно не является взаимно однозначным отображением в~силу периодичности.
\end{note}

\begin{theorem}
Стереографическая проекция сохраняет углы между кривыми.
\end{theorem}
\begin{proof}
Доказательство, которое было дано на лекциях, мне не понравилось, тем более, что мы в предыдущем пункте уже
вывели формулы для метрики на сфере, конформно эквивалентной евклидовой. Рассмотрим кривые $\vec r_1(t) = \br{x_1(t),y_1(t)}$ и
$\vec r_2(t) = \br{x_2(t),y_2(t)}$ на плоскости, пересекающиеся в точке $z$. Из постоянства растяжений касательных векторов очевидным образом следует сохранение углов:
$\cos\ph = \frac{\hr{dr_1, dr_2}}{|dr_1| \cdot |dr_2|}$. В силу постоянства растяжений
\eqn{\cos\ph = \frac{K(z)(dx_1 d x_2 + dy_1 dy_2)}{\sqrt{K(z)(dx_1^2+dy_1^2)}\cdot \sqrt{K(z)(dx_2^2+ dy_2^2)}} =
\frac{dx_1 dx_2 + dy_1 dy_2}{\sqrt{d x_1^2 + dy_1^2} \cdot \sqrt{dx_2^2 + dy_2^2}}.}
Правая часть последнего равенства есть в точности выражение для угла между кривыми на плоскости.
\end{proof}

\subsubsection{Симметрии сферы Римана и отображение $\frac{1}{z}$}

Рассмотрим пару точек $M$ и $M^*$ на сфере, симметричных относительно экватора (плоскости $O\xi\eta$). Пусть
они имеют координаты $M(\xi,\eta,\ze)$ и $M^*(\xi,\eta,-\ze)$. Пусть $z$ и $z^*$ их прообразы при стереографической
проекции. Тогда
\eqn{z = \frac{\xi+i\eta}{1-\ze}, \quad z^*=\frac{\xi+i\eta}{1+\ze} \quad \Ra \quad z^*\cdot \ol z  = \frac{\xi^2+\eta^2}{1-\ze^2}.}
Снова вспоминая, что $\xi^2+\eta^2 = 1-\ze^2$, получаем, что $z^*\cdot \ol z = 1$.
Это свойство можно положить в основу определения симметрии на расширенной плоскости.

\begin{df}
Точки $z$ и $z^*$ называются \emph{симметричными} (относительно единичной окружности),
если
\eqn{z^* \cdot \ol z = 1.}
\end{df}

В общем случае, для окружности произвольного радиуса $r$, справа от знака равенства стоит число $r^2$.

Точка $0$ симметрична точке $\bes$, так как $z^* = \frac{1}{\ol z}$.

Из свойств стереографической проекции получаем, что преобразование $\frac{1}{z}$ сохраняет углы. В самом деле,
мы показали, что оно соответствует зеркальной симметрии сферы, которая, разумеется, сохраняет углы, а при проекции углы
тоже не поменяются. По аналогичным соображениям она обладает круговым свойством.

В завершение этой темы отметим, что преобразование $z\mapsto z^*$ является конформным автоморфизмом сферы.

\section{Функции комплексного переменного}

\subsection{Голоморфные функции и их простейшие свойства}

\subsubsection{Предел функции, непрерывность. Модуль непрерывности}

Предел функции в точке определяется также, как в математическом анализе.

\begin{df}
Модулем непрерывности функции $f\cln D\ra \Cbb$ на множестве $D$ называется число
\eqn{\om_f(D,\de) := \sup\hc{|f(x)-f(y)| \text { по всем точкам } x,y \in D \cln |x-y|< \de}.}
\end{df}

Очевидно, модуль непрерывности есть неубывающая функция от параметра $\de$.
В дальнейшем, если из контекста понятно, о какой функции и о каком множестве идёт речь,
мы будем опускать индекс <<$f$>> и первый аргумент функции $\om$.

\begin{stm}
Если множество $D$ выпукло, то $\om(\de_1+\de_2) \le \om(\de_1)+\om(\de_2)$.
\end{stm}
\begin{proof}
Пусть $z',z''\in D$, и точка $z$ лежит на отрезке $[z',z'']$. Тогда, по определению выпуклого
множества, $z\in D$. Пусть $|z'-z|=\de_1$ и $|z-z''|=\de_2$. Имеем
\eqn{|f(z')-f(z'')| = |f(z')-f(z)+f(z) -f(z'')|\le |f(z')-f(z)| + |f(z)-f(z'')|\le \om(\de_1)+\om(\de_2).}
Остаётся перейти к точной верхней грани по всем точкам $z'$ и $z''$, для которых $|z'-z''|\le \de_1+\de_2$.
\end{proof}

\begin{imp}
Если множество $D$ выпукло, то $\om(n\cdot \de) \le n\cdot \om(\de)$.
\end{imp}

\begin{imp}
Если $\om_f(D,\de) = \ol o(\de)$, то $f\equiv \const$.
\end{imp}
\begin{proof}
В самом деле, если это условие выполнено, то
\eqn{\frac{|f(z')-f(z'')|}{|z'-z''|} \le \frac{\om_f(D,|z'-z''|)}{|z'-z''|} \ra 0, \quad \hm{z'-z''} \ra 0.}
Таким образом, $f' \equiv 0$. Этого, как мы скоро увидим, достаточно для того,
чтобы $f \equiv \const$.
\end{proof}

\begin{problem}
Доказать, что если множество $D$ выпукло и ограничено, а функция $f\cln D\ra\Cbb$ не постоянна, то найдётся
константа $C$ такая, что $\om_f(D,\de) \ge C\cdot \de$.
\end{problem}

\subsubsection{Комплексная дифференцируемость}

\begin{df}
Пусть функция $f$ определена в некоторой окрестности точки $z_0$.
\emph{Комплексной производной} функции $f$ в точке $z_0$ называется следующий предел (если он существует):
\eqn{f'(z_0) := \liml{h \ra 0} \frac{f(z_0 + h) - f(z_0)}{h}.}
\end{df}

Также, как и в курсе математического анализа, доказываются все свойства производной (арифметические операции,
дифференцируемость сложной и обратной функций).

Существование предела можно переписать так: $\De f = f'(z_0 )\De z + o(\De z)$ при $\De z \ra 0$. Это означает, что
приращение функции есть комплексный дифференциал. Такие функции называются комплексно дифференцируемыми или, короче,
$\Cbb$ дифференцируемыми.

\begin{df}
Говорят, что функция $f$ \emph{голоморфна (аналитична)} в точке $z_0$, если она $\Cbb$ дифференцируема
в некоторой окрестности этой точки.
\end{df}

\begin{df}
Функция $f(z)$, определённая в окрестности $\bes$, называется голоморфной в бесконечности, если функция
$f\hr{\frac1z}$ голоморфна в точке~$0$.
\end{df}

\begin{ex}
Найдём производную функции $f(z) = z$. Имеем $\frac{z + h - z}{h} = 1$, поэтому $f' = 1$.
\end{ex}

\begin{ex}
Попробуем теперь продифференцировать функцию $f(z) = \ol{z}$.
Имеем $\frac{\ol{z + h} - \ol{z}}{h} = \frac{\ol{h}}{h}$. Пойдем по разным направлениям:
на вещественной оси имеем $h = \De x$ и $\frac{\De x}{\De x} = 1$, а на мнимой
$h = i\De y$ и $\frac{ - i\De y}{i\De y} = - 1$, то есть предела не существует.
\end{ex}

\subsubsection{Условие Коши Римана}

\begin{theorem}
Функция $f(z) = u\hr{x,y} + iv\hr{x,y}$ имеет комплексную производную в точке $z_0 = x_0 + iy_0$ тогда
и только тогда, когда функции $u(x,y)$ и $v(x,y)$ дифференцируемы в точке $\hr{x_0, y_0}$ и в ней выполняется условие
Коши Римана (Даламбера Эйлера):

\eqn{\case{\pf{u}{x} = \phm \pf{v}{y}, \\ \pf{u}{y} = -\pf{v}{x}.}}
\end{theorem}

\begin{proof}
Необходимость: Пусть $\exi f'(z_0 ) = a + i b$, тогда
\eqn{\De f = f'(z_0 )\De z + o(\rh) = \De u + i\De v = \hr{a + i b}\hr{\De x + i\De y} + o(\rh), \text{ где } \rh = \hm{\De z} \ra 0,}
откуда
\eqn{\case{
\De u = a\De x - b\De y + o(\rh), \\
\De v = b\De x + a\De y + o(\rh).
}}

Отсюда следует, что функции $u$ и $v$ дифференцируемы. Полученные равенства представляют собой условие Коши Римана, к
\eqn{\case{
a\De x = u_x, & b\De x = \phm v_x, \\
a\De y = v_y, & b\De y = - u_y. \\
}}
Тем самым необходимость доказана. Попутно доказана и достаточность, так как все рассуждения обратимы.
\end{proof}

\begin{theorem}[Лумана Меньшова]
Если функция $f$ непрерывна в области $D$ и в каждой точке области выполняется условие
Коши Римана, то она $\Cbb$ дифференцируема в области $D$.
\end{theorem}

\begin{note}
На самом деле эта теорема верна и в ещё более слабых предположениях: условие Коши Римана выполняется почти всюду
(относительно лебеговой меры), а функция a priori может быть не дифференцируемой в счётном числе точек области.
\end{note}

\begin{note}
Из одного лишь условия Коши Римана не вытекает комплексная дифференцируемость.
Пример: $f(z)=0$ на осях, $f(z)=1$ иначе.
\end{note}

\subsubsection{Формальные частные производные}

Пусть $f$ $\R$ дифференцируемая функция. Напишем её дифференциал: $df = \pf{f}{x}dx + \pf{f}{y}dy$.
Имеем
\eqn{dz = dx + i dy, \quad d\ol{z} = dx - i dy \quad \Ra \quad
\case{dx = \frac{dz + d\ol{z}}{2}, \\
dy = \frac{dz - d\ol{z}}{2i}.}}
Подставляя $dx$ и $dy$ в выражение для дифференциала, получим
\eqn{\label{FormalPartialDerivatives}d f = \hr{\frac{1}{2} \pf{f}{x} + \frac{1}{2i}\pf{f}{y}} dz + \hr{\frac{1}{2}\pf{f}{x} - \frac{1}{2i}\pf{f}{y}} d\ol{z}.}

Введём специальные обозначения для дифференциальных операторов.

\begin{df}
\emph{Формальными частными производными} $\pf{f}{z}$ и $\pf{f}{\ol{z}}$ называются выражения в
скобках в формуле \eqref{FormalPartialDerivatives}.
\end{df}

Условие Коши Римана в терминах формальных частных производных записывается так: $\pf{f}{\ol{z}} = 0$.
Условие комплексной дифференцируемости означает, что функция не зависит от $\ol z$.

Если комплексная производная существует, то её можно вычислить по одной из двух формул: либо $\frac{df}{dx}$,
либо $\frac{df}{idy}$, так как по условию Коши Римана $\frac{df}{dx} = \frac{df}{idy}$:
\eqn{\frac{du}{dx} - \frac{d v}{i dx} = \frac{du}{i dy} + \frac{dv}{dy} \; \Ra  \;
\case{\pf{u}{x} = \phm \pf{v}{y}, \\ \pf{v}{x} = -\pf{u}{y}.}}

\subsubsection{Сопряжённые гармонические функции}

\begin{df}
Вещественная функция $u=u(x,y)$ называется \emph{гармонической} в области $D$, если $u \in \Cb^2(D)$ и в этой области
выполнено \emph{условие Лапласа:}
\eqn{\De u := \frac{\pd^2u}{\pd x^2} + \frac{\pd ^2u}{\pd y^2} = 0.}
\end{df}

\begin{stm}
Если $f(z) = u\hr{x,y} + iv\hr{x,y}$ голоморфна в области $D$, то функции $u$ и $v$ гармоничны в этой области.
\end{stm}
\begin{proof}
Напишем условие Коши Римана и два раза продифференцируем:
\eqn{\case{\pf{u}{x} = \phm \pf{v}{y}, \\ \pf{u}{y} = -\pf{v}{x}} \rightsquigarrow
\case{\frac{\pd^2u}{\pd x^2} &= \phm\frac{\pd ^2v}{\pd x\pd y}, \\
\frac{\pd^2v}{\pd x\pd y} &= -\frac{\pd^2u}{\pd y^2}.}}
Складывая равенства, получаем, что $\De u = \De v = 0$, что и требовалось. Остаётся только обосновать существование
второй производной, но это будет сделано позднее.
\end{proof}

\begin{df}
\emph{Сопряжёнными} гармоническими функциями называются гармонические функции,
являющиеся вещественной и мнимой частями некоторой голоморфной функции.
\end{df}

\begin{petit}
Вреде бы он этого не доказывал, но в прошлом году оно на лекциях было... Пусть остаётся для полноты, но мелким шрифтом.

\medskip

\begin{stm}
Если функция $u$ гармонична в односвязной области $D$, то найдётся функция $f = u + iv$, голоморфная в области $D$.
\end{stm}
\begin{proof}
В силу условия Коши Римана дифференциал искомой функции равен
\eqn{df = v'_x dx + v'_y dy = -u'_y dx + u'_x dy.}

Положим $P(x,y):= -u'_y$ и $Q(x,y)=u'_x$. Функции $P$, $Q$, $Q'_x$ и $P'_y$ непрерывны в
односвязной области $D$. Продифференцируем форму $Pdx + Qdy$:
\eqn{d\hr{P\,dx+Q\,dy} = (P'_x \,dx + P'_y \,dy) \wg dx + (Q'_x \,dx + Q'_y \,dy) \wg dy =
P'_y \,dy \wg dx + Q'_x \,dx \wg dy = (Q'_x - P'_y)\, dx \wg dy.}
Она будет полным дифференциалом, если $Q'_x - P'_y = 0$. Покажем, что верно и обратное.
Если $Q'_x - P'_y = 0$, то по формуле Грина имеем
\eqn{\ints{\pd D} P \,dx + Q \,dy = \ints{D} (Q'x-P'_y)\,dx \,dy = 0,}
а из критерия потенциальности поля (вспомните математический анализ!) следует, что
если интеграл по замкнутому контуру нулевой, то поле потенциально, то есть его компоненты
являются частными производными некоторой функции. Таким образом можно построить требуемую функцию~$v$:
\eqn{v(x,y) := \intl{(x_0, y_0)}{(x,y)}{P\,dx + Q\,dy}.}
Из того же критерия следует, что интеграл не зависит от выбора пути интегрирования.
\end{proof}

\begin{note}
В многосвязной области такое утверждение не верно. Пример: функция $u = \ln r$.
\end{note}
\end{petit}

\begin{problem}
Выразить оператор Лапласа через формальные частные производные $\pf{}{z},\pf{}{\ol{z}}$.
\end{problem}

\begin{petit}
Тут есть некоторое количество задач, которых на лекциях этого года их не было, но они были 2 года назад.
Возможно, Долженко любит давать их на экзаменах.

\medskip

\begin{problem}
Пусть функция $w$ $\R$ дифференцируема.
Доказать, что она $\Cbb$ дифференцируема, если существует один из следующих пределов:
\eqn{\liml{\De z \ra 0} \Rea \frac{\De w}{\De z}, \quad
\liml{\De z \ra 0} \hm{\frac{\De w}{\De z}}, \quad
\liml{\De z \ra 0} \arg \frac{\De w}{\De z}.}
\end{problem}

\begin{problem}
Пусть функция $f$ является $\R$ дифференцируемой.
Доказать, что множество предельных точек $\hc{\frac{\De f}{\De z}}$ есть либо точка, либо окружность.
Найти её центр и радиус.
\end{problem}

\begin{problem}
Функция $f = u + iv$ голоморфна в области $D$, и $f \neq 0$. Доказать, что линии уровня функций $u$ и $v$
(то есть $u=\const$, $v=\const$) пересекаются под прямым углом.
\end{problem}

\begin{problem}
Доказать инвариантность условия Коши Римана относительно вращения плоскости:

Тогда условие Коши Римана запишется следующим образом:
$\pf{u}{\ol{n}} = \pf{v}{\ol{s}},\pf{v}{\ol{n}} = - \pf{u}{\ol{s}}$, где $\hm{\ol{s}} = 1$ и $\hm{\ol{n}} = 1$.
В~частности, записать условие Коши Римана в полярных координатах.
\end{problem}
\end{petit}

\subsubsection{Полианалитические функции}

\begin{df}
Функция $f$ называется \emph{полианалитической} (точнее, \emph{$n$ аналитической}), если $\frac{\pd^n f}{\pd \ol z^n} = 0$.
\end{df}

Полианалитические функции имеют вид
\eqn{w(z) = \ph_0(z) + \ol z \ph_1(z) + \ol z^2 \ph_2(z) \spl \ol z^{n-1}\ph_{n-1}(z),}
где $\ph_i(z)$ аналитические функции.

\subsection{Конформность и дифференцируемость}

\subsubsection{Геометрический смысл комплексной производной}

Выясним геометрический смысл аргумента и модуля комплексной производной. Рассмотрим $\R$ дифференцируемую функцию
$w= f(z)$. У неё есть дифференциал $df = \rbmat{a & b \\c & d}$ некоторое линейное преобразование плоскости.
Функция будет $\Cbb$ дифференцируемой $\Lra \; df = \rbmat{a & -b \\ b & \phm a}$ вследствие условия Коши Римана.
Линейное преобразование такого вида, очевидно, есть умножение на число $a+ib$. Таким образом, <<в первом приближении>>
происходит растяжение в $\hm{f'(z_0)}$ раз и поворот на угол $\arg f'(z_0)$, если, конечно, $f'(z_0)\neq 0$.

Таким образом, голоморфные функции обладают свойством постоянства растяжений и сохранения углов.

Если же $f'(z_0) = f''(z_0) = \ldots = f^{(k-1)}(z_0)$, а $f^{(k)}\neq 0$, то по формуле Тейлора
\eqn{\De w = c_k (\De z)^k + c_{k+1} (\De z)^{k+1} + \ldots,}
поэтому происходит увеличение углов в $k$ раз.

\begin{problem}
Доказать, что если функция обладает свойством постоянства растяжений и сохраняет углы, то она аналитична.
\end{problem}
\begin{solution}
Откройте лекции В.\,К.\,Белошапки (см. \dmvnwebsite) на странице 15 и прочтите \emph{Предложение 1.3} и \emph{Утверждение 3.3}.
Мне лень их переписывать оттуда...
\end{solution}

\subsubsection{Основные теоремы о конформных отображениях}

\begin{theorem}[Х.\,Бор, 1918 г.]
Если функция $f$ в области $G$ является однолистным гомеоморфизмом и имеет постоянство растяжений в каждой точке, причём
коэффициент растяжения $k$ не равен нулю, то либо $f$, либо~$\ol f$ аналитична~в~$G$.
\end{theorem}
В 1919 году Радемахером было доказано ещё более сильное утверждение: было убрано требование $k\neq 0$.

\begin{theorem}[Д.\,Е.\,Меньшов, 19231926 г.]
Если функция $f$ в области $G$ является гомеоморфизмом и сохраняет углы (как по ориентации, так и по величине), то
она аналитична~в~$G$.
\end{theorem}

\begin{theorem}[Б.\,Риман]
Пусть область $G$ односвязна и граница $\pd G$ состоит более чем из одной точки. Тогда существует конформное
отображение этой области на единичный круг $\hc{z\cln |z|<1}$.
\end{theorem}

Таких отображений на самом деле существует бесконечно много. Анри Пуанкаре доказал, что если зафиксировать
прообраз нуля, то есть такую точку $a$, что $f(a)=0$, и число $\al = \arg f'(a)$, то такое отображение единственно.
Это и есть \emph{нормировка конформного отображения} (один из способов).

\begin{theorem}[Каратеодори]
Пусть $D_1$ и $D_2$ односвязные области с жордановыми границами, и
$f\cln D_1\ra D_2$ конформное отображение. Тогда $f$ продолжается до гомеоморфизма между $\ol{D}_1$ и $\ol{D}_2$.
\end{theorem}

Эта теорема позволяет ввести ещё одно условие нормировки. Фиксируем точки
$$a_1\in D_1, \quad b_1\in\pd D_1, \quad a_2\in D_2, \quad b_2\in \pd D_2.$$
Отображение $f$, такое что $f(a_1)=a_2$ и $f(b_1)=f(b_2)$, единственно.

\begin{theorem}[принцип симметрии]
Пусть граница области $D_1$ содержит участок прямой или окружности $\ga_1$, а $D_1^*$
симметричная относительно $\ga_1$ область, и $D \cap D^*=\es$;
Пусть $(D_2, D_2^*,\ga_2)$ набор с теми же свойствами. Пусть функция $f\cln D_1\ra D_2$ конформное
отображение, продолжаемое по непрерывности до взаимно однозначного соответствия границ: $f\cln \ga_1 \ra \ga_2$.
Тогда продолженное по симметрии отображение~$f$ даёт конформное отображение
$D_1\cup\ga_1\cup D_1^* \mapsto D_2\cup\ga_2\cup D_2^*$.
\end{theorem}

\subsection{Дробно линейные преобразования}

\subsubsection{Определение и свойства}

\begin{df} \emph{Дробно линейным преобразованием (ДЛП)} называется функция
вида
\eqn{w(z) = \frac{az + b}{cz + d}, \quad ad - bc \neq 0.}
\end{df}

\begin{note}
Мы хотим, чтобы ДЛП были гомеоморфизмами из $\ol{\Cbb}$ в $\ol{\Cbb}$. Если же $ad - bc = 0$, то числитель и
знаменатель дроби пропорциональны, следовательно, дробь сокращается и $w(z) = \const$. А постоянное отображение
не является гомеоморфизмом.
\end{note}

\begin{theorem}
Все ДЛП образуют группу относительно композиции.
\end{theorem}
\begin{proof}
Очевидно: перемножить и посмотреть, что получится.
\end{proof}

Можно доказать, что конформными автоморфизмами сферы Римана являются дробно линейные преобразования, и только они.
Поэтому мы будем группу ДЛП обозначать через $\Aut(\ol\Cbb)$.

Рассмотрим группу $\GL_2(\Cbb)$. Зададим отображение $\ph\cln \GL_2(\Cbb) \ra \Aut \ol{\Cbb}$ по правилу
\eqn{\rbmat{a & b \\ c & d} \mapsto w(z) = \frac{az+b}{cz+d}.} Легко видеть, что это гомоморфизм групп.
Найдём его ядро. $\Ac \in \Ker \ph \; \Lra \; \ph(\Ac) = \id$, то есть $\frac{az + b}{cz + d} \equiv z$.
При постоянном отображении $0 \mapsto 0$, поэтому $\frac{b}{d} = 0$, а так как $\bes \mapsto \bes$,
то $\frac{a}{c} = 0$. Значит, $b = 0, \; d \neq 0$, и $a = 0, \; c \neq 0$. Так
как $1 \mapsto 1$, то $\frac{a}{d}=1$, а значит, $a=d$.
Значит, $\Ker \ph = \hc{\la \Ec \vl \la \in \Cbb^*} \cong \Cbb^*$. По теореме о гомоморфизме
\eqn{\Aut \ol{\Cbb} \cong \PSL_2(\Cbb).}

\subsubsection{Простейшие ДЛП и их геометрический смысл}

\begin{df}
Простейшими ДЛП назовём:
\begin{nums}{-2}
\item Сдвиг на вектор $b$: $\Pi_1(z) := z + b$;
\item Поворот на угол $\arg a$ и растяжение в $|a|$ раз: $\Pi_2(z) := az$;
\item \emph{Инверсия}: композиция двух симметрий относительно единичной окружности и относительно вещественной оси:
$\Pi_3(z) := \frac{1}{z}$.
\end{nums}
\end{df}

Покажем, что преобразование $\frac{1}{z}$ действительно есть композиция двух симметрий. Пусть $z = re^{i\ph}$. Тогда
\eqn{\Pi_3(z) = \frac{1}{z} = \frac{1}{re^{i\ph}} = \frac{1}{r}e^{-i\ph}.}

\begin{theorem}
Любое ДЛП разлагается в композицию простейших.
\end{theorem}
\begin{proof}
Основная идея: поделить числитель на знаменатель нацело.
Пример: $w = \frac{z - 1}{z + 1} = \frac{\hr{z + 1} - 2}{z + 1} = 1 - \frac{2}{z + 1}$.
В общем виде всё аналогично, только формулы страшнее. Для нашего примера имеем
$z \mapsto z_1 := z + 1$, затем $z_1 \mapsto z_2 := \frac{1}{z_1}$, далее
$z_2 \mapsto z_3 := - 2z_2$, и наконец, $z_3 \mapsto w = z_3 + 1$ это
композиция простейших ДЛП.
\end{proof}

\subsubsection{Свойство конформности для ДЛП}

\begin{theorem}
ДЛП сохраняют углы между кривыми.
\end{theorem}
\begin{proof}
Достаточно доказать утверждение для простейших ДЛП. Для первых двух всё очевидно. Докажем для
инверсии. Мы уже знаем, что преобразование $\frac{1}{\ol z}$ соответствует зеркальной симметрии сферы  Римана, и потому сохраняет углы.
Значит, $\frac{1}{z}$ тоже сохраняет углы, так как сопряжение, очевидно, обладает этим свойством.
\end{proof}

\begin{note}
ДЛП сохраняют углы не только по величине, но и по направлению.
\end{note}

\subsubsection{Круговое свойство ДЛП}

\begin{theorem}
ДЛП переводят окружности в окружности (прямая тоже окружность).
\end{theorem}
\begin{proof}
Снова достаточно рассмотреть простейшие ДЛП. Для первых двух всё очевидно, а для инверсии достаточно снова вспомнить
круговое свойство стереографической проекции и симметрии сферы.
\end{proof}

\subsubsection{Свойство 3 точек для ДЛП}

\begin{df}
\emph{Двойным отношением 4 точек} называется число
\eqn{[z_1:z_2:z_3:z_4] := \frac{z_1 - z_3}{z_2 - z_3}:\frac{z_1 - z_4}{z_2 - z_4}.}
\end{df}

\begin{lemma}
Двойное отношение сохраняется при дробно линейных преобразованиях.
\end{lemma}
\begin{proof}
Не верите подставьте и убедитесь в том, что оно действительно сохраняется.
\end{proof}

\begin{theorem}
Для любых 3 различных точек $z_1, z_2, z_3$ и любых 3 различных точек $w_1, w_2, w_3$ существует единственное
ДЛП, переводящее $z_i$ в $w_i$.
\end{theorem}
\begin{proof}
По лемме двойное отношение инвариантно относительно ДЛП, то есть
\eqn{[z_1:z_2:z_3:z] = [w_1:w_2:w_3:w].}
Рассмотрим это выражение как уравнение на $z$ и $w$. Координаты точки $w$ однозначно определяются по координатам
точки $z$ (причём дробно линейной функцией), если $z_i$ и $w_i$ фиксированы. Поэтому, если мы знаем,
что $z_i\mapsto w_i$, то образ любой другой точки $z$ однозначно определяется.
\end{proof}

\begin{problem}
Четыре точки лежат на одной окружности $\Lra$ их двойное отношение вещественно.
\end{problem}

\subsubsection{Неподвижные точки ДЛП}

Линейное отображение вида $w(z) = az+b$ при $a\notin\hc{0,1}$, очевидно, имеет две
неподвижные точки: $\hc{\frac{b}{1-a},\bes}$. Если $a\ra 1$, то бесконечность становится <<двойной>> неподвижной точкой.
Если $a=1$ и $b=0$, то  все точки неподвижны.

Для дробно линейных отображений неподвижными являются корни уравнения $cz^2 + (d-a)z-b=0$.

\subsubsection{Автоморфизмы единичного круга}

\begin{petit}
На лекциях оно было, но без доказательств. Но они не повредят. Тут есть ссылка на принцип максимума,
который был доказан в конце курса. Заодно здесь доказана лемма Шварца.
\end{petit}

Через $\De$ обозначим открытый единичный круг.

\begin{lemma}[Шварц]
Пусть функция $f\cln\De\ra\De$ голоморфна в $\De$. Пусть $f(0)=0$ и $|f(z)| \le 1$.
Тогда $|f(z)|\le |z|$, причём если существует точка $z_0\in\De$ такая, что $|f(z_0)|=|z_0|$,
то $f(z)=e^{i\ta}z$.
\end{lemma}
\begin{proof}
Рассмотрим функцию $\ph(z) := \frac{f(z)}{z}$. Поскольку $f(0)=0$, то нуль будет устранимой точкой для $\ph(z)$.
Значит, $\ph$ голоморфна в круге $\De$.

Возьмём замкнутый круг радиуса $\rh < 1$. По принципу максимума функция $\ph$ достигает своего максимума на
границе этого круга. Но так как $|f(z)| \le 1$, то
\eqn{|\ph(z)| \le \hm{\frac{f(z)}{z}} \le \frac{1}{\rh}.}
Устремляя $\rh$ к единице, получаем $\bm{\frac{f(z)}{z}} \le 1$, следовательно, $|f(z)|\le|z|$.

Пусть теперь $|f(z_0)|=|z_0|$ в некоторой точке $z_0$. Из доказанного выше следует, что $|\ph| \le 1$.
В точке $z_0$ функция $|\ph|$ достигает значения $1$, а больше единицы быть не может. Значит, по принципу максимума $\ph=\const$
и $|\ph|=1$, то есть $\ph(z) = e^{i\ta}$. Тогда $f(z) = e^{i\ta}z$ поворот на угол $\ta$.
\end{proof}

Обозначим через $G_2$ группу отображений следующего вида:
\eqn{G_2 :=\hc{T(z) = e^{i\ta}\frac{z-a}{1-\ol{a}z}}, \text{ где } |a| < 1, \quad \ta \in \R.}

\begin{theorem}
$\Aut(\De) \cong G_2$. Эта группа действует на $\De$ транзитивно.
\end{theorem}
\begin{proof}
Пусть $\ph\cln\De\ra\De$ конформный автоморфизм круга. Пусть $\ph(0)=a$. Рассмотрим дробно линейное преобразование
$T\in G_2$, заданное формулой $T(z) = \frac{z-a}{1-\ol a z}$. Оно переводит точку $a$ в нуль, то есть
композиция $\Ph := T \circ \ph$ оставляет нуль на месте: $\Ph(0)=0$. Применим к отображению $\Ph$ лемму Шварца.
Она утверждает, что
\eqn{\label{LeftInequality}|\Ph(z)| \le |z|.}
Теперь рассмотрим обратное отображение $\Ph^{-1}$. Это тоже будет некоторый
автоморфизм круга, сохраняющий нуль, и к нему тоже можно применить лемму Шварца, откуда
\eqn{|\Ph^{-1}(w)|\le |w|.}
Подставим в эту формулу $\Ph(z)$ вместо $w$. Так как $\Ph^{-1} \circ \Ph = \id$, то
\eqn{\hm{\Ph^{-1}\br{\Ph(z)}} \le |\Ph(z)| \; \Ra \; |z| \le |\Ph(z)|.}
Тем самым мы получили обратное неравенство к \eqref{LeftInequality}. Значит, $|\Ph(z)|=|z|$ для любой точки $z$.
По второму утверждению леммы Шварца $\Ph(z)=e^{i\ta}z$. Значит, $\ph = T^{-1}\circ \Ph$ отображение, являющееся
композицией некоторого поворота и дробно линейного отображения, то есть $\ph \in G_2$. Следовательно, $\Aut(\De) \cong G_2$.

Транзитивность следует из того, что любую точку можно перевести в нуль соответствующим преобразованием,
а затем с тем же успехом можно нуль отправить куда угодно.
\end{proof}

Дробно линейные отображения верхней полуплоскости на единичный круг имеют вид
\eqn{\hc{T(z) = e^{i\ta}\frac{z-a}{z-\ol{a}}}, \text{ где } |a| < 1, \Img a > 0, \quad \ta \in \R.}

Для полноты картины сообщим без доказательства  тот факт, что группа конформных отображений $\Cbb$ (не~расширенной!)
есть группа линейных преобразований вида $w(z)=az+b$.

\subsection{Элементарные комплексные функции}

\subsubsection{Экспонента}

\begin{df}
\eqn{e^z = e^{x+iy} := e^x(\cos y + i\sin y) = \suml{n=0}{\bes}\frac{z^n}{n!}.}
\end{df}

Экспонента голоморфна во всём $\Cbb$ (но не расширенном). Так как $w'(z) = e^z$, то
отображение конформно в~каждой точке.

Из определения следует, что эта функция имеет период $2\pi i$.
Областью однолистности для $e^z$ будет любая область, не содержащая различных точек, отличающихся
на её период.

\begin{problem}
Рассмотрим $D = \hc{z = x+iy \cln x \in (0, \ln 2),\; y \in (0,\pi)}$.
Найти $\exp(D)$.
\end{problem}
\begin{solution}
$e^z = e^x(\cos y + i\sin y)$. Имеем $R = e^x, R \in (1,2)$, а угол меняется от $0$ до $\pi $.
\end{problem}

\subsubsection{Комплексный логарифм}

\begin{df}
\emph{Логарифм} комплексного числа определяется так: $w = \ln z \; \Lra \; e^w = z$.
\end{df}

Если в определении $z = 0$, то $\ln z$ не определён, так как $w = u + iv$ и $\hm{e^w} = e^u > 0$ для любого $w$.
Если $z \neq 0$, то $z = re^{i\ph}$ и $e^u e^{iv} = re^{i\ph}$. Поэтому $e^u = r, \; u = \ln r, \; v = \ph$.
Следовательно,
\eqn{\ln z = \ln r + i\Arg z.}

Определим функцию $\Ln z \cln \Cbb \ra \Cbb$. Тут есть три подхода.

\pt{1} Многозначные функции: каждому значению аргумента $z \in \Cbb^*$ ставим в соответствие множество

\eqn{\Ln z := \hc{\ln r + i\hr{\ph + 2n\pi} \vl n \in \Z}.}

\pt{2} Непрерывная ветвь: фиксируем некоторую область в $D \subs \Cbb^*$ такую, чтобы функция $\ph$ была непрерывной.
Любые две ветви отличаются на $2 \pi$. Существование ветви зависит от $D$.

\pt{3} Римановы поверхности: мы хотим построить новое топологическое пространство, такое, что на нём функция будет
однозначной. Для логарифма риманова поверхность строится так: берём счётное множество $\hc{s_n}_{n\in \Z}$ экземпляров
множества $\Cbb^* := \Cbb\wo\hc{0}$ (<<листов>>), вырезаем из каждого луч $(0,\bes)$, а потом склеиваем их так, чтобы получилось
что то вроде винтовой лестницы: к <<правому берегу>> разреза поверхности $s_n$ приклеиваем <<левый берег>> разреза
листа $s_{n+1}$.

На новой поверхности определим логарифм так: пусть $z \in s_n$. Положим
$\Ln z := \ln r + i\hr{\ph + 2n\pi}, \; \ph \in (-\pi; \pi]$. Такая функция будет непрерывной, так как
в любой точке склейки листов значения углов совпадают: $\ph = \pi + 2\pi n$ на $s_n$ и $\ph = - \pi + 2\pi(n + 1) = \pi+ 2\pi n$ на $s_{n+1}$.

\begin{petit}
Я понимаю, что без картинки сложно осознать, как это всё устроено. Но времени рисовать её у меня пока нет.
\end{petit}

\begin{df}
Степенная функция с комплексным показателем определяется так:
\eqn{z^a := e^{a\ln z}.}
\end{df}

\begin{problem}
Найти все значения $i^i$.
\end{problem}
\begin{solution}
Имеем
$i^i = e^{i\ln i} = \exp\hs{i\br{\ln 1 + i\hr{\frac{\pi }{2} + 2\pi n}}} = \exp\hs{-\frac{\pi}{2}-2\pi n}$.
\end{solution}

\subsubsection{Функция Жуковского}

\begin{df}
\emph{Функцией Жуковского} называется функция $w := \frac{1}{2}\hr{z + \frac{1}{z}}$.
\end{df}

Функция Жуковского осуществляет непрерывное отображение $\ol{\Cbb} \ra \ol{\Cbb}$.
Выясним, в каких точках отображение конформно. Если $z \ra 0$, то $w \sim \frac{1}{2z}$, а это конформное отображение.
Если $z \ra \bes$, то $w \simeq \frac{z}{2} - k$, так что в $0$ и на $\bes$ отображение конформно. Рассмотрим
остальные точки: $w' \simeq \frac{1}{2}\hr{1 - \frac{1}{z^2}}$, то есть $w' = 0$ при $z = \pm 1$ в этих точках конформность
нарушается.

Найдём области однолистности функции (биективности). Это любая область, не содержащая одновременно
точек $z$ и $\frac{1}{z}$. Открытый единичный круг и верхняя полуплоскость являются примерами таковых областей.

\begin{problem}
Найти такие области, что функция Жуковского является конформным отображением этих областей.
\end{problem}
\begin{solution}
Рассмотрим семейство концентрических окружностей внутри единичного круга. Выясним, куда они переходят.
Уравнение окружности радиуса $r$: $z = re^{i\ph}$ где $\ph \in [0,2\pi)$. Имеем
\eqn{w(z) = \frac12\hr{r e^{i\ph} + \frac{1}{r}e^{-i\ph}} =
\frac12\hs{\hr{r + \frac{1}{r}}\cos\ph + i \hr{r - \frac{1}{r}}\sin\ph}.}
Получился эллипс  с полуосями $a := \frac12\hr{r + \frac{1}{r}}$ и $b := \frac12\hr{\frac{1}{r} - r}$.
Найдём его фокусное расстояние: $c^2 = a^2 - b^2 = 1$, откуда получаем, что фокусы находятся в точках
$\pm 1$. Рассмотрим предельные случаи. Пусть $r \ra 0$, тогда $a \ra \frac{1}{2\pi}$ и $b \ra \frac{1}{2\pi}$
в пределе получаем окружность. Если же $r \ra 1$, то $a \ra 1$ и $b \ra 0$, то есть область стягивается к отрезку $[-1,1]$.
Таким образом, $G = \ol{\Cbb} \wo [-1,1]$.
\end{solution}

Фактически мы доказали следующую теорему.

\begin{theorem}
Функция Жуковского конформно отображает единичный круг на область $\ol{\Cbb} \wo [-1,1]$.
\end{theorem}

Функция $z(w) := w + \sqrt{w^2 - 1}$ является обратной к функции Жуковского. Знак выбран так, чтобы $z$ лежало
в единичном круге.

\begin{problem}
Куда функция Жуковского отображает верхнюю полуплоскость?
\end{problem}

\subsubsection{Тригонометрические и гиперболические функции}

\begin{df}
\eqn{\sin z := \frac{e^{iz}-e^{-iz}}{2}, \quad \cos z := \frac{e^{iz}+e^{-iz}}{2}.}
\end{df}

\begin{df}
\eqn{\sh z := \frac{e^{z}-e^{-z}}{2}, \quad \ch z := \frac{e^{z}+e^{-z}}{2}.}
\end{df}

\section{Интеграл по комплексному переменному}

\subsection{Основные свойства интеграла по кривой}

\subsubsection{Определение интеграла}

Пусть $\ga(t)\cln [a,b]\ra\Cbb$ непрерывная кривая. Введём обозначения: \emph{разбиение}
\eqn{T := \hc{a = t_0 < t_1 < \ldots  < t_n = b},}
его \emph{диаметр} $\la_T := \maxl{j} \De t_j$. Пусть $z_j := \ga(t_j)$, и $\De z_j = z_j - z_{j-1}$.

Пусть $f\cln[a,b] \ra \Cbb$ произвольная функция на отрезке $[a,b]$. Рассмотрим разбиение
с отмеченными точками $Q_j \in [t_{j - 1}, t_j]$. Составим интегральную сумму:
\eqn{S(T) := \suml{j = 1}{n} f(Q_j) \De z_j.}

\begin{df}
\emph{Интегралом} называется предел $S = \liml{\la_T \ra 0} S(T)$, если этот предел существует.
Обозначение:
\eqn{S = \ints{\ga} f(z)\,dz.}
\end{df}

\begin{note}
Мы рассматриваем разбиение именно отрезка параметризации, а не самой кривой, так как она может
иметь самопересечения.
\end{note}

\begin{problem}
\begin{nums}{-2}
\item Теорема существования: Если кривая $\ga$ спрямляема, и $f\in\Cb[a,b]$, то $\ints{\ga} f\,d\ga$ существует.
\item Если $f$ непрерывна, а $\ga$ гладкая кривая, то $\ints{\ga} f\,dz = \intl{a}{b}f(t)\ga'(t)\,dt$.
\end{nums}
\end{problem}

\subsubsection{Простейшие свойства интеграла}

\begin{theorem}
Если функция $f$ интегрируема вдоль кривой $\ga$, то она ограничена на ней.
\end{theorem}
\begin{proof}
Пусть функция неограничена, тогда существует последовательность точек $t_n$ такая, что ${|f\br{\ga(t_n)}|>n}$
при $t_n\ra\tau$. Рассмотрим разбиение $\hc{t_k}$. Пусть $\hc{t_n}\subs[t_{k_0},t_{k_0+1}]$.
Тогда одно из слагаемых интегральной суммы стремится к бесконечности:
\eqn{f\br{z(t_n)}\br{z(t_{k_0+1})-z(t_{k_0})}\ra\bes,\quad n\ra\bes.}
\hfill\end{proof}
\begin{note}
Может оказаться, что $z(t_{k_0+1})-z(t_{k_0})=0$ в силу того, что кривая может иметь самопересечения. Тогда добавим
ещё одну точку разбиения между $t_{k_0}$ и $t_{k_0+1}$ и выберем тот из двух отрезков, на который попала точка $\tau$.
\end{note}

\begin{theorem}
Значение интеграла не зависит от выбора параметризации кривой.
\end{theorem}
\begin{petit}
Я ни хрена не понял, что там он наговорил... По моему, доказательством там и не пахнет.
\end{petit}

Сформулируем некоторые очевидные свойства интеграла.

\begin{items}{-2}
\item Линейность: $\ints{\ga}(\al f + \be g)\,dz = \al\ints{\ga}f\,dz + \be\ints{\ga}g\,dz$.

\item Ориентированность: $\ints{-X}f\,dz = -\ints{X}f\,dz$.

\item Аддитивность: $\ints{\ga_1 + \ga_2}f\,dz = \ints{\ga_1}f\,dz + \ints{\ga_2}f\,dz$.
\end{items}

Свойство аддитивности мы берём в качестве определения интеграла по формальной сумме кривых.

\subsubsection{Достаточные условия интегрируемости}

\begin{df}
\emph{Полным колебанием} функции $f$ на множестве $E$ называется число
\eqn{\Om(f,E) := \sup\hc{|f(x')-f(x'')|, \text { где } x',x''\in E}.}
\end{df}

Рассмотрим спрямляемую кривую $\ga$ и разбиение $T$ отрезка параметризации этой кривой.
Пусть $f\cln\ga\ra\Cbb$ некоторая функция. Через $\Om_k$ будем обозначать колебание функции $f$ на
куске $\ga\br{[t_{k-1},t_k]}$ этой кривой, а через $L_k$ длину этого куска кривой.

\begin{stm}
Пусть кривая $\ga$ спрямляема, и функция $f(z) = u(x,y)+iv(x,y)$. Пусть $\ga_k$ куски разбиения кривой.
Тогда для каждого <<куска>> $\ga_k$ выполнено неравенство
\eqn{\frac{1}{2}\br{\Om(u) + \Om(v)} \le \Om(f) \le \Om(u) + \Om(v).}
\end{stm}
\begin{proof}
Имеем
\begin{multline}
\Om(f) = \sup |f(z')-f(z'')| = \sup \bm{\br{u(z')- u(z'')}+i\br{(v(z')-v(z'')}} \le\\
\le \sup |u(z')-u(z'')| + \sup |v(z')-v(z'')| = \Om(u) + \Om(v),
\end{multline}
и таким образом второе неравенство доказано.
После сложения очевидных неравенств
\eqn{\case{\Om(u) \le \Om(f)\\\Om(v)\le\Om(f)}}
получаем первое неравенство.
\end{proof}

\begin{theorem}[Первый критерий интегрируемости]
Функция $f$ интегрируема на спрямляемой кривой~$\ga$ тогда и только тогда, когда $\sum \Om_k L_k \ra 0$ при $\la_T\ra 0$.
\end{theorem}

\begin{theorem}[Второй критерий интегрируемости]
Функция $f$ интегрируема на спрямляемой кривой~$\ga$ тогда и только тогда, когда множество точек
разрыва функции $f$ имеет лебегову меру нуль.
\end{theorem}

\subsubsection{Примеры}

Пусть $\ga\cln[\al,\be]\ra\Cbb$ спрямляемая кривая.

\begin{ex}
Найдём интеграл $\ints{\ga}1\,dz$. Имеем
\eqn{\ints{\ga}1\,dz = \liml{\la_T\ra 0} \sum 1 \br{\ga(t_{k-1})-\ga(t_{k})}.}
Промежуточные слагаемые погибают, остаётся
\eqn{\ints{\ga}1\,dz = \ga(\be) - \ga(\al).}
\end{ex}

\begin{ex}
Найдём интеграл $\ints{\ga}z\,dz$.
Так как функция непрерывна и не важно, какую промежуточную точку брать на каждом кусочке кривой, будем брать среднее
арифметическое значений на концах отрезков разбиения. Пусть $z_i := \ga(t_i)$.
\eqn{\suml{j = 1}{n}\frac{(z_j + z_{j-1})}{2}(z_j - z_{j-1}) = \suml{j = 1}{n}\frac{(z_j^2 - z_{j-1}^2)}{2} = \lcomm \text{ всё сокращается } \rcomm
= \frac{1}{2}\hr{z_n^2 - z_0^2} = \frac{\ga(\be)^2 - \ga(\al)^2}{2}.}
\end{ex}

\begin{ex}
Вычислим интеграл
\eqn{I_n = \frac{1}{2\pi i}\oints{\hm{z - a} = r}{\frac{dz}{\hr{z - a}^{n + 1}}}, \; n \in \Z.}
Это гладкая кривая, параметризуем её и посчитаем: Уравнение окружности запишется в виде
$z = a + re^{i\ph}$, где $\ph \in [0, 2\pi)$. Имеем
\eqn{I_n = \frac{1}{2\pi i}\intl{0}{2\pi }{\frac{ire^{i\ph}}{\hr{re^{i\ph }}^{n + 1}}\,d\ph =
\frac{1}{2\pi }\frac{1}{r^n}\intl{0}{2\pi}{e^{ - in\ph}}}\,d\ph = \case{1, & n = 0; \\ 0, & n \ne 0,}}
так как это интеграл от $\sin$, $\cos$ по периоду.
\end{ex}

\subsubsection{Связь комплексного интеграла с вещественными интегралами}

\begin{stm}
Пусть $f(z) = u(x,y) + i v(x,y)$, а $\ga = z(t)$ спрямляемая кривая.
Тогда
\eqn{\ints{\ga}f(z)\,dz = \ints{\ga} u\,dx-v\,dy + i\ints{\ga}v\,dx+u\,dy,}
причём интегралы в правой части существуют тогда и только тогда, когда существует интеграл в левой части.
\end{stm}
\begin{proof}
Обозначим $f_k := f\br{z(\tau_k)}$. Тогда
\eqn{\sum f_k \De z_k = \sum (u_k+iv_k)(\De x_k + i \De y_k) =
\sum (u_k\De x_k - v_k\De y_k) + i \sum (v_k\De x_k + u_k \De y_k) \ra
\ints{\ga} u\,dx-v\,dy + i\ints{\ga}v\,dx+u\,dy.}
Второе утверждение очевидно.
\end{proof}

Теперь найдём аналог криволинейного интеграла первого рода.

\begin{df}
\eqn{\ints{\ga}f(z)\,|dz| := \liml{\la_T\ra 0} \sum f\br{z(\tau_k)}|z(t_{k-1})-z(t_k)|.}
\end{df}

\begin{stm}
\eqn{\ints{\ga}f(z)\,|dz| = \intl{0}{|\ga|}f\br{z(s)}\,ds.}
\end{stm}
\begin{proof}
Имеем
\eqn{\intl{0}{|\ga|}f\br{z(s)}\,ds = \liml{\la_T\ra 0}\sum f\br{z(\si_k)} \De s_k.}
Так как предел не зависит от разбиения и промежуточных точек, Положим $\si_k := \tau_k$, $s_k := \tau_k$.
Тогда $\De s_k = |\ga_k|$.
Тогда
\eqn{\Bm{\ints{\ga}f\,|dz| - \intl{0}{|\ga|}f\br{z(s)}\,ds} = \lim \sum f\br{z(\tau_k)}\br{|\ga_k| - |z(t_k)-z(t_{k-1})|}
\le \supl{\ga} |f| \cdot o(1) \ra 0.}
\hfill\end{proof}

\begin{stm}
Пусть $\ga$ спрямляемая кривая. Тогда интеграл $\int f\,dz$ существует тогда и только тогда, когда существует
$\int f\,|dz|$.
\end{stm}
\begin{proof}
Вытекает из критерия Лебега.
\end{proof}

\begin{stm}
Оценка интеграла: пусть $|f| \le M$ на $\ga$. Тогда
\eqn{\Bm{\ints{\ga}f\,dz} \le \ints{\ga}|f|\,dz \le \ints{\ga}|f|\,|dz| \le M \cdot |\ga|.}
\end{stm}
\begin{proof}
В самом деле,
\eqn{\Bm{\ints{\ga}f\,dz}  = \Bm{\liml{\la_T\ra0} \sum f_k\De z_k} \le \sum |f_k| \cdot |\De z_k| \le M \cdot |\ga|.}
\hfill\end{proof}

\subsubsection{Предельный переход под знаком интеграла}

\begin{theorem}
Пусть кривая $\ga$ спрямляема и существуют интегралы $\ints{\ga}f_n\,dz$.
Если $f_n \convu{\ga} f$, то
\eqn{\ints{\ga} f_n\,dz \ra \ints{\ga} f\,dz, \quad n \ra\bes.}
\end{theorem}
\begin{proof}
Пусть $\Bc(f)$ множество точек разрыва\footnote{От английского <<break point>>.
Обозначение $E(f)$, которое было на лекциях, крайне неестественно.}
функции $f$. Покажем, что $\Bc(f) \subs \bigcup \Bc(f_n)$, чтобы применить критерий Лебега.
Действительно, пусть $b\in \Bc(f)$.
Тогда в любой окрестности $U(b)$ найдутся точки $t,s$ такие, что $|f\br{z(t)}-f\br{z(s)}|>\ep$ для
некоторого $\ep > 0$. Но в силу равномерной сходимости найдётся такой номер $n_0$, что все функции $f_n$
отличаются от $f$ меньше, чем на $\frac{\ep}{3}$ при $n\ge n_0$. Значит, точка $b$ будет точкой разрыва
для функций $f_n$ при $n\ge n_0$. Тем самым доказано, что интеграл $\ints{\ga}f\,dz$ существует.

Остаётся показать, что этот интеграл равен пределу интегралов от $f_n$.
Имеем
\eqn{\Bm{\int f_n\,dz - \int f\,dz} = \Bm{\int(f_n-f)\,dz}\le \supl{\ga}|f_n-f|\cdot |\ga|\ra 0, \quad n\ra\bes.}
\hfill\end{proof}

Таким образом, пространство $\Rc(\ga)$ интегрируемых на спрямляемой кривой $\ga$ функций является банаховым относительно
чебышевской (равномерной) нормы. Полнота гарантируется только что доказанной теоремой.

Интеграл является ограниченным функционалом на этом пространстве. Очевидно, его норма не превосходит длины
кривой по оценочному свойству.

\begin{problem}
Доказать, что на самом деле норма интеграла равна длине кривой.
\end{problem}

\subsubsection{Вычисление интегралов. Первообразная. Формула Ньютона Лейбница}

\begin{theorem}
Пусть дана кривая $z(t)\cln[\al,be]\ra\Cbb$, и $z(t)\in\Cb^1[\al,\be]$. Тогда
\eqn{\ints{\ga}f(z)\,dz = \intl{\al}{\be}f\br{z(t)}z'(t).}
\end{theorem}
\begin{proof}
Имеем
\eqn{\lim\sum f_k \hr{\intl{t_{k-1}}{t_k}z'(t)\,dt - z'(t_k)\intl{t_{k-1}}{t_k}\,dt} = \lim\sum f_k \intl{t_{k-1}}{t_k} \br{z'(t)-z'(t_k)}\,dt.}
Так как кривая гладкая, то $|z'(t)-z'(t_k)|\ra 0$ при $\la_T\ra 0$. А так как функция $f$ ограничена, то и вся сумма стремится к нулю.
\end{proof}

\begin{df}
\emph{Производной} функции $F$ по множеству $E$ в точке $a$ называется число
\eqn{F'_E(a) := \liml{\substack{z\ra a,\\a,z\in E}} \frac{F(z)-F(a)}{z-a}.}
\end{df}

\begin{df}
\emph{Первообразной} функции $f$ называется такая функция $F$, что $F' =f$.
\end{df}

\begin{theorem}[Формула Ньютона Лейбница]
Пусть $\ga=z(t)$ гладкая кривая, $f\in\Cb(\ga)$ и $f = F'_\ga$. Тогда
\eqn{\ints{\ga}f(z)\,dz = F\br{z(\be)}- F\br{z(\al)}.}
\end{theorem}
\begin{proof}
Имеем
\eqn{\frac{d}{dt}F\br{z(t)} =
\liml{\tau\ra t}\frac{F\br{z(\tau)}- F\br{z(t)}}{\tau-t}\cdot
\frac{z(\tau)-z(t)}{z(\tau)-z(t)} = f\br{z(t)}z'(t).}
По предыдущей теореме и обычной формуле Ньютона Лейбница
\eqn{\ints{\ga}f(z)\,dz = \intl{\al}{\be}f\br{z(t)}z'(t)\,dt = F\br{z(\be)}-F\br{z(\al)}.}
\hfill\end{proof}



\subsection{Интегральная теорема Коши}

\subsubsection{Шаг 0: интеграл от линейных функций}

\begin{stm}
Пусть $\ga\cln[\al,\be]\ra\Cbb$ замкнутая спрямляемая кривая, то есть $\ga(\al)=\ga(\be)$. Тогда
\eqn{\ints{\ga} (az + b)\,dz = 0.}
\end{stm}
\begin{proof}
Мы знаем, что $\ints{\ga}1\,dz = \ga(\be)-\ga(\al)$ и $\intl{A}{B} z\,dz = \frac{\ga(\be)^2 - \ga(\al)^2}{2}$.
Но так как $\ga(\al) =\ga(\be)$, то оба интеграла равны нулю. В силу линейности и исходный интеграл равен нулю.
\end{proof}

Возникает следующая естественная гипотеза: поскольку интеграл от любой линейной функции по замкнутой спрямляемой кривой
равен нулю, то можно предположить, что оно верно для любой голоморфной функции.
Интегральная теорема Коши подтверждает эту гипотезу.

\subsubsection{Шаг 1: интегральная теорема Коши для односвязной области}

\begin{lemma}[Гурса]
Пусть $\tri$ треугольник, лежащий в области $D$ вместе с внутренностью, и $f(z)$ голоморфна в
окрестности $\tri$. Тогда
\eqn{\ints{\tri}f(z)\,dz=0.}
\end{lemma}
\begin{proof}
Обозначим $\tri_0 := \tri$, $P_0$ периметр $\tri_0$, а
\eqn{I_0:=\Bm{\ints{\tri}f(z)\,dz}.}
\rightpicture{pictures.40}

\hangafter=-4
\hangindent=-40mm
Поделив стороны $\tri_{0}$ пополам и соединив между собой точки деления, получим ещё четыре треугольника.
Обозначим через $\tri_1$ <<центральный>> треугольник. Интеграл по всему контуру равен сумме интегралов
по всем $4$ треугольникам. Значит, хотя бы одно слагаемое не меньше четверти интеграла $I_0$. Обозначим его $I_1$,
а соответствующий контур через $\tri_1$. Итак, имеем
\eqn{I_1 := \Bm{\ints{\tri_1}f(z)\,dz} \ge \frac{I_0}{4}.}
Продолжая процесс, получим последовательность вложенных треугольников
$\tri_0 \sups \tri_1 \sups \ldots \sups \tri_n \ra a$, где $a$ некоторая точка в треугольнике, причём
\eqn{I_n\ge \frac{I_0}{4^n}.}

Имеем $f(z)=f(a)+f'(a)(z-a)+o(z-a)  = f(a) + f'(a)(z-a) + \al(z)(z-a)$.
Как мы уже убедились, интеграл от линейной функции $f'(a)(z-a)$ по $\tri_n$ равен нулю. Значит, вклад
в интеграл может дать только нелинейное слагаемое $o(\dots)$. Далее заметим, что $(z-a)$ не превосходит периметра
треугольника, по которому мы интегрируем. Учитывая это, имеем

\eqn{\frac{I_0}{4^n} \le I_n = \Bm{\ints{\tri_n}o(z-a)\,dz} = \Bm{\ints{\tri_n}\al(z)(z-a)\,dz} \le \frac{P_0}{2^n} \cdot \frac{P_0}{2^n} \maxl{\tri_n}|\al(z)| \ra 0.}

Умножая неравенство на $4^n$, получаем $I_0 \le P_0^2\cdot\maxl{\tri_n}|\al(z)| \ra 0$. Значит, $I_0 = 0$.
\end{proof}
\begin{imp}
Если $D$ односвязная область, а $f(z)$ голоморфна в $D$ и $\ga$ замкнутая ломаная в $D$, то
\eqn{\ints{\ga}f(z)\,dz=0.}
\end{imp}
\begin{proof}
Разобьём многоугольник, образованный ломаной, на треугольники. По лемме Гурса интеграл по каждому из них равен нулю,
но ориентации на <<стыках>> разные, поэтому суммарный интеграл равен $0$.
\end{proof}
\begin{imp}[интегральная теорема Коши]
Если $D$ односвязная область, и $f(z)$ голоморфна в $D$, а $\ga$ спрямляемая кривая в $D$, то
\eqn{\ints{\ga}f(z)\,dz=0.}
\end{imp}

\begin{note}
Для многосвязной области эта теорема неверна. Пример: $\oints{\hm{z} = 1}{\frac{dz}{z}} = 2\pi i$.
\end{note}

\begin{imp}
Если граница области $D$ есть простая замкнутая спрямляемая кривая, а~функция $f(z)$ голоморфна в $\ol{D}$,
то $\ints{\pd D}f(z)\,dz = 0$.
\end{imp}
\begin{proof}
Если $f(z)$ голоморфна в $\ol D$, то она дифференцируема в некоторой окрестности $\ol D$. Значит, к ней
можно применить уже доказанную теорему. Остаётся убедиться в том, что эта область односвязна (докажите это!).
\end{proof}

Верен и более общий результат:
\begin{theorem}
Если граница области $D$ есть простая, замкнутая спрямляемая кривая, а функция $f(z)$ голоморфна в~$D$ и
непрерывна на $\ol{D}$, то $\ints{\ga}f(z)\,dz = 0$.
\end{theorem}

\begin{problem}
Доказать эту теорему для круга.
\end{theorem}

\subsubsection{Шаг 2: интегральная теорема Коши для многосвязной области}

\begin{df}
Область $D$ называется \emph{правильной}, если она ограничена и её граница $\pd D$ состоит
из конечного числа попарно не пересекающихся простых, замкнутых, спрямляемых кривых.
Будем говорить, что границы правильной области \emph{ориентированы положительно}, если при обходе
границы область остаётся слева.
\end{df}


\begin{theorem}
Если $D$ правильная область с положительно ориентированной границей $\ga$, и
$f(z)$ голоморфна в $\ol{D}$, то $\ints{\pd D}f(z)\,dz = 0$.
\end{theorem}
\begin{proof}
Мы не будем проводить строгое доказательство, а приведём лишь его основную идею. Она заключается в том, чтобы
перейти от многосвязного контура к односвязному. Пусть в области имеется $k$ <<дырок>>. Проведём разрез
от некоторой точки на внешней границе до некоторой точки внутренней границы (одной из её компонент).
Дырок станет на одну меньше, и так далее. В итоге получится односвязная область и $k$ лишних разрезов.
Но они не помешают: когда мы будем интегрировать по ним, то два раза пройдём по одним и тем же точкам
в противоположных направлениях, то есть интегралы по разрезам уничтожатся. А по границе полученной односвязной области
интеграл равен нулю. Значит, и интеграл по границе исходной области равен нулю.
\end{proof}

\begin{note}
Справедливо усиление теоремы, когда $f(z)$ голоморфна в $D$ и непрерывна в $\ol{D}$.
\end{note}



\subsection{Пример}

Вычислить интеграл: $C = \intl{0}{\bes }{\cos (x^2)dx} $.
\begin{proof}
Функция быстро
осциллирует $ \Ra $ интеграл сходится условно. Вычислим
одновременно: $S = \intl{0}{\bes}{\sin (x^2)dx} $. Интегралы C u S
называются \textbf{интегралами Френеля}. Объединяем эти 2 интеграла в одно
комплексное число: $E = C + iS = \intl{0}{\bes }{e^{ix^2}dx} $.
Рассмотрим $f(z) = e^{iz^2} - $ голоморфную во всей комплексной области С.
Обозначим замкнутый контур$C_R ,$ тогда


$0 = \hc {инт. теор.} = \ints{C_R}f(z)\,dz = \ints{OA} + \ints{AB} + \ints{BO}$.
Перейдём к пределу при $R \ra \bes$.

$\ints{B0} = [z = te^{\frac{\pi i}{4}}] \ra \hc{R \ra \bes} \ra
- \intl{0}{\bes}e^{-t^2}e^{\frac{\pi i}{4}}\,dt = e^{\frac{\pi i}{4}}\frac{\sqrt \pi }{2}$,
$\ints{OA} \ra E$. Докажем, что $\ints{AB} \ra 0$ при $R \ra \bes$,
тогда отсюда $E = e^{\frac{\pi i}{4}}\frac{\sqrt \pi }{2}$ или по
отдельности: $C = S = \frac{\sqrt \pi }{2\sqrt 2 }$. Итак, $\intl{AB}{} = [z
= Re^{i\ph }] = \intl{0}{\frac{\pi }{4}}{\exp \{iR^2e^{2i\ph }\}R}
e^{i\ph }id\ph $. Оценим его по модулю: $\hm{\intl{AB}{} } \le
\intl{0}{\frac{\pi }{4}}{\exp \{ - R^2\sin 2\ph \}Rd\ph .} $ Как
доказать, что это выражение$\mathop \ra \limits_{R \ra \bes } 0?$ Нарисуем
график синуса, отсюда видно, что $\sin 2\ph \ge \frac{4\ph }{\pi }$,
итак, продолжим: $\hm{\intl{AB}{} } \le \intl{0}{\frac{\pi }{4}}{\exp \{ -
R^2\frac{4\ph }{\pi }\}Rd\ph = \left. {\frac{R \cdot \exp \{ -
R^2\frac{4\ph }{\pi }\}}{\frac{ - 4R^2}{\pi }}} \right|_0^{\frac{\pi
}{4}} = \frac{\pi }{4R}(1 - e^{ - R^2}) < \frac{1}{R}\mathop \ra \limits_{R
\ra \bes } 0.} $
\end{proof}

\textbf{Напоминание.} Почему$I = \intl{0}{\bes }{e^{ - x^2}dx = }
\frac{\sqrt \pi }{2}$ (?) $I^2 = \intl{0}{\bes }{e^{ - \hr{x^2 +
y^2}}dxdy} = \intl{0}{\bes }{\intl{0}{\frac{\pi }{2}}{e^{ - r^2}drd\ph
= \frac{\pi }{2}\frac{1}{2}} } $.

\end{document}

% bsh
\begin{stm}
Если интеграл от непрерывной функции $f(z)$ по любому замкнутому контуру в области $D$ равен нулю, то
$f(z)$ обладает первообразной.
\end{stm}
\begin{proof}
Пусть $\oints{\ga}f(z)\,dz=0$ для $\fa \ga$. Рассмотрим функцию
\eqn{F(z) = \intl{a}{z}f(\ze)\, d\ze.}
Покажем, что такое задание функции корректно, то есть не зависит от выбора пути интегрирования.
Пусть мы пошли другим путём $\wt{\ga}$ из $z$ в $a$. Тогда интеграл по контуру
$\ga \cap \wt{\ga}$, то есть по $\ga$ (от $a$ до $z$) и по $\wt{\ga}$ (от $z$ до $a$) будет равен нулю.
Но это и значит, что
\eqn{(\ga)\intl{a}{z} f(\ze)\,d\ze = -(\wt{\ga})\intl{z}{a} f(\ze)\,d\ze = (\wt{\ga})\intl{a}{z} f(\ze)\,d\ze.}
Таким образом, в качестве пути интегрирования можно выбрать отрезок прямой.

Теперь покажем, что $F(z)$ действительно первообразная. Пусть $\De z$ столь мало, что $z + \De z$ не вылезает за пределы области.
Имеем
\eqn{\frac{F(z+\De z)-F(z)}{\De z} = \frac{1}{\De z}\hr{\intl{a}{z+\De z} - \intl{a}{z}}f(\ze)\, d\ze = \frac{1}{\De z} \intl{z}{z+\De z}f(\ze)\,d\ze = (*).}
Так как $f(z)$ непрерывна, то $f(\ze) = f(z) + \al(\ze)$. Тогда данное равенство преобразуется к виду
\eqn{(*)=\frac{1}{\De z}\hr{\intl{z}{z+\De z}f(z)\,d\ze + \intl{z}{z+\De z}\al(\ze)\,d\ze} = f(z) + \frac{1}{\De z}\intl{z}{z+\De z}\al(\ze)\,d\ze.}
Покажем, что второе слагаемое есть $o(1)$ при $\De z \ra 0$. По оценочному свойству интеграла
оно не превосходит длины пути интегрирования (то есть $|\De z|$), умноженной на $\max|\al|$. Но оба множителя стремятся к нулю
при $\De z \ra 0$. Значит, $F'(z)=f(z)$.
\end{proof}


% bsh





\section{ИНТЕГРАЛ КОШИ}


\subsection{Формула Коши}

\begin{theorem}
Если 1) $D$ правильная область с положительно ориентированной границей, и
$f(z)$ голоморфна в $\ol{D}$, то для $\fa z \in D$ такой, что
$f(z) = \frac{1}{2\pi i}\ints{\Ga}\frac{f(\xi)\,d\xi}{\xi - z}}$.
\end{theorem}
\begin{proof}
Фиксируем точку $z$. Обозначим через $U_\ep$ её $\ep$ окрестность. $\ga_\ep$ граница с
обходом против часовой стрелки. $\ep$ любое достаточно малое число. $\ol{U}_\ep \subs D$,

$D_\ep = D \wo \ol{U}_\ep$, $\Ga_\ep = \pd D_\ep = \Ga \wo \ga_\ep$. Рассмотрим
функцию $\ph(\xi) = \frac{f(\xi) - f(z)}{\xi - z}$ голоморфную в $D \wo \hc{z}$.
В частности, она голоморфна в $\ol{D}_\ep$. Тогда
$\ints{\Ga_\ep}\ph(\xi)\,d\xi = \ints{\ga_\ep}\ph(\xi)\,d\xi$. Заметим, что
$\exi \liml{\xi \ra z} ({\hm{\xi } \le M} \ph \hr{\xi } = f'\hr{z}$, где $M$ - некоторая проколотая окрестность
\quad .
\quad
\hm{\intl{\ga _\ep }{}{\ph \hr{\xi }d\xi } } \le M2\pi
\ep \ra 0(\ep \ra
0)$.
Переходя в (*) к пределу, получим, что $\intl{\Ga }{}{\ph \hr{\xi }d\xi } = 0$. Теперь вспомним определение этой функции: $\intl{\Ga }{}{\frac{f\hr{\xi
}}{\xi - z}d\xi } = f(z)\intl{\Ga }{}{\frac{1}{\xi - z}d\xi =
f(z)\intl{\ga _z }{}{\frac{1}{\xi - z}d\xi = f(z)2\pi i.} } $ Здесь
в первом равенстве мы применили те же рассуждения, что и для $\ph $.
Следовательно, $f(z) = \frac{1}{2\pi i}\intl{\Ga }{}{\frac{f\hr{\xi
}}{\xi - z}d\xi } . - $Формула Коши.

\end{proof}

Некоторые пояснения к переходам в теореме. ($\cong$ если окружность маленького радиуса):

$$\ints{\Ga}\frac{f\hr(\xi)}{\xi - z}\,d\xi = \ints{\ga_\ep}\frac{f(\xi)}{\xi - z}\,d\xi}
\cong \int \frac{f(z)}{\xi - z}\,d\xi = f(z) 2\pi i$$.

\textbf{Отступление.} Основными были 2 примера: 1) $\intl{A}{B}{dz}$, $\intl{A}{B}z\,dz$ $\Ra$ Интегральная теорема Коши.

2) $\ints{\hm{z - a} = R}\frac{1}{(z - a)^{n + 1}}\,dz$ $\Ra $ Формула Коши.

\subsection{Интеграл типа Коши}

\begin{df}
Пусть $\Ga$ простая, спрямляемая кривая. $F$ функция, непрерывная на $\Ga$, тогда \emph{интегралом типа Коши}
называется следующий интеграл, зависящий от параметра:
$F(z) = \frac{1}{2\pi i}\ints{\Ga}\frac{f(\xi)}{\xi - z}\,d\xi$, $z \in \Cbb \wo \Ga$.

\begin{df}
Интеграл типа Коши называется \emph{интегралом Коши}}, если $\Ga$ замкнутая, простая, спрямляемая кривая.
\end{df}

функция f -- граничное значение голоморфной в области D функции.}

Пусть мы имеем интеграл Коши: (D*- дополнение к D)$\frac{1}{2\pi
i}\intl{\Ga }{}{\frac{f\hr{\xi }}{\xi - z}d\xi } = \left\{
{{\begin{array}{*{20}c}
 {f(z),z \in D} \hfill \\
 {0,z \in D\ast } \hfill \\
\end{array} }} \right.{\begin{array}{*{20}c}
 {из} \hfill \\
 {из} \hfill \\
\end{array} }{\begin{array}{*{20}c}
 {формулы} \hfill \\
 {инт.теоремы} \hfill \\
\end{array} }$Коши.





\textbf{Теорема1.} $Для\fa z \in \Cbb\backslash Г$ интеграл типа Коши имеет комплексные производные любого порядка, которые вычисляются по формуле:} $\frac{F^{\hr{n}}\hr{z}}{n!}
= \frac{1}{2\pi i}\intl{\Ga }{}{\frac{f\hr{\xi }}{\hr{\xi - z}^{n +
1}}d\xi } ,n = 0,1,..$.
\begin{proof} $\be \hr{h} = \frac{F\hr{z + h} - F\hr{z}}{h} =
\frac{1}{2\pi i}\frac{1}{h}\intl{\Ga }{}{\hr{\frac{1}{\xi - z - h} -
\frac{1}{\xi - z}}f\hr{\xi }d\xi } = \frac{h}{2\pi ih}\intl{\Ga
}{}{\frac{f\hr{\xi }}{\hr{\xi - z - h}\hr{\xi - z}}d\xi } $. $\al \hr{h}
= \be \hr{h} - \frac{1}{2\pi i}\intl{\Ga }{}{\frac{f\hr{\xi }}{\hr{\xi
- z}^2}d\xi } = \frac{1}{2\pi i}\intl{\Ga }{}{\hr{\frac{1}{\xi - z - h} -
\frac{1}{\hr{\xi - z}}}\frac{f\hr{\xi }}{\xi - z}d\xi = \frac{h}{2\pi
i}\intl{\Ga }{}{\frac{f\hr{\xi }d\xi }{\hr{\xi - z - h}\hr{\xi - z}^2}} }
$. Докажем, что всё это стремится к 0. Введём обозначения: $\hm{f} \le M$ на
Г, $\hm{\Ga } = \La < + \bes ,\delta = dist\hr{z,\Ga } > 0 -
$расстояние. Нужны оценки: $\hm{\al \hr{h}} \le \frac{\hm{h}}{2\pi
}\frac{M}{\frac{\delta }{2}\delta ^2}\La = C\hm{h}\mathop \ra \limits_{h
\ra 0} 0,\hm{h} < \frac{\delta }{2} \Ra $

$\be \hr{h}\mathop \ra \limits_{h \ra 0} \frac{1}{2\pi i}\intl{\Ga
}{}{\frac{f\hr{\xi }}{\hr{\xi - z}^2}d\xi } (n =
1).Д}л}я}$n = 2, 3, \ldots доказательство аналогичное (по
индукции). \end{proof}



\begin{imp}.} Голоморфная функция бесконечно дифференцируема.}





На комплексной плоскости, если дифференцируема один раз в некоторой точке
комплексной плоскости, тогда она бесконечно дифференцируема. Если
голоморфна, то представима в виде интеграла Коши (частный случай интеграла
типа Коши). По теореме2 она бесконечно дифференцируема. \end{proof}





\begin{note} Устранили пробел в доказательстве гармоничности
вещественной и мнимой части голоморфной функции, более того доказали, что
гармонические также бесконечно дифференцируемы.





\textbf{3}$^{0}$\textbf{. Неопределённый интеграл.}





\begin{df} $Задана f:D \ra \Cbb. Её $\textbf{первообразной}} в области D называется такая Ф, что }$\Ph ' =
f$ в области D. }





\textbf{Утверждение1.} Если первообразная}$\exi ,$ то она определена с точностью до аддитивной константы.}





\begin{proof} Пусть$\Ph ' = 0\mathop \Ra
\limits^{\hr{?}} \Ph = const$. (доказать).





\textbf{Утверждение2}\textbf{.}} Если первообразная}$\exi ,$ то функция f -- голоморфна}.





\begin{proof} $\exi \Ph '$т.е. Ф -- голоморфна$ \Ra
\Ph \in \Cbb^\bes \Ra \Ph " = f' \Ra f - $ голоморфна
(только у голоморфной функции бывает первообразная).





\textbf{Теорема3 (Теорема о существовании первообразной).} Если функция f голоморфна в односвязной области D, то она имеет первообразную в этой области.}

\begin{proof} Рассмотрим$\Ph \hr{z} = \intl{z_0 }{z}{f\hr{\xi
}d\xi ,z_0 - \fa } $ фиксированная точка области, z -- переменная точка.
Интеграл берётся по любой спрямляемой кривой, соединяющей эти 2 точки и
лежащей в области D. По интегральной теореме Коши интеграл не зависит от
кривой. Докажем, что она будет первообразной.


\[
\be \hr{z} = \frac{\Ph \hr{z + h} - \Ph \hr{z}}{h} =
\frac{1}{h}\intl{z}{z + h}{f\hr{\xi }d\xi ,}
f(z) = \intl{z}{z + h}{f(z)d\xi ,}
\]




\[
\al \hr{h} = \be \hr{h} - f(z) = \frac{1}{h}\intl{z}{z +
h}{\hr{f\hr{\xi } - f(z)}d\xi ,}
н}а}д}о}д}о}к}а}з}а}т}ь}
\quad ,ч}т}о}э}т}о}
\quad
 \ra 0.Б}у}д}е}м}
\]



\noindent
интегрировать по отрезку. Функция f непрерывна. Напишем определение
непрерывности:


\[
\fa \ep > 0\exi \delta > 0:\hm{\xi - z} < \delta \Ra
\hm{f\hr{\xi } - f(z)} < \ep .
\]




\[
\hm{h} < \delta \Ra \hm{\al \hr{h}} \le
\frac{1}{\hm{h}}\ep \hm{h} = \ep
,т}а}к}и}м}о}б}р}а}з}о}м}
\quad ,
\quad
\al \hr{h} \ra 0(h \ra 0),т} \quad .е}.
\quad
\Ph ' = f.
\quad
\end{proof}
\]





\begin{imp} (Ньютона-Лейбница в комплексной плоскости).} Если f - голоморфна в односвязной области D и Ф -- первообразная, то}$\intl{A}{B}{f\hr{\xi
}} d\xi = \Ph \hr{B} - \Ph \hr{A},$ интеграл берётся по любой спрямляемой кривой}.





Это следствие самого доказательства теоремы.





\begin{note} В многосвязной области теорема не верна.





\begin{ex} $\frac{1}{z} - $ голоморфна в С*. Локально
первообразная$\ln z$ определена, а глобально нет (для всего С*).





\begin{note} При доказательстве теоремы3 мы пользовались только 2мя
свойствами голоморфной функции: 1) непрерывность

2) интеграл по замкнутому контуру равен 0.





\textbf{4}$^{0}$\textbf{. Теорема Мореры.}





\begin{df} Говорят, что функция }\textbf{удовлетворяет условию }}$\De -
$\textbf{ ка}} в области D, если для любого замкнутого }$\De - $ ка, лежащего в этой области:} $\intl{\pd \De }{}f dz = 0$.




\textbf{Теорема (Мореры).} Если 1) f~непрерывна в D}

$2) \quad f \in \hr{\De }$ в D, то f -- голоморфна в D.}

\begin{center}
Лекция 6(7 октября 2002 года).}
\end{center}


\textbf{Теорема (Мореры).} Если 1) f~непрерывна в D}

$2) \quad f \in \hr{\De }$ в D, то f -- голоморфна в D (обратное тоже верно).}


\begin{proof}:} Голоморфна в области -- достаточно доказать
голоморфность в круге или любой односвязной области. Будем считать, что D --
односвязная область, тогда из свойств 1) и 2) следует, что для любой
замкнутой спрямляемой кривой$\ga \subs D:\intl{\ga }{}{fdz = 0} $
(см. доказательство ИТК). $f$ имеет первообразную Ф в области D (см. теорему
о существовании первообразной + замечания к ней), тогда$f$ голоморфна --
необходимое условие существования первообразной. \end{proof}



\section{Последовательности и ряды}

ПРИНЦИПЫ МАКСИМУМА МОДУЛЯ





\textbf{1}$^{0}$\textbf{. Теорема о среднем.}





\begin{theorem} Если функция }$f$ голоморфна в круге }$U = \{z \in \Cbb\vert \hm{z - z_0 } \le R\},$ то её значение в центре круга = среднему арифметическому её значений на границе, т.е.} $f\hr{z_0 } =
\frac{\intl{\Ga }{}{fds} }{\hm{\Ga }}, где \Ga = \pd u = \{z \in
\Cbb\vert \hm{z - z_0 \le R}\}$ - окружность.}

\begin{proof} Напишем формулу Коши: $f\hr{z_0 } = \frac{1}{2\pi
i}\oint {\frac{f\hr{\xi }}{\xi - z_0 }d\xi } ,$ параметрический вид
окружности: ${\begin{array}{*{20}c}
 {\xi = z_0 + Re^{i\ph }} \hfill \\
 {0 \le \phi \le 2\pi } \hfill \\
\end{array} },$ тогда$f\hr{z_0 } = \frac{1}{2\pi i}\intl{0}{2\pi
}{\frac{fRe^{i\ph }id\ph }{Re^{i\ph }} = \frac{1}{2\pi
R}\intl{0}{2\pi }{fRd\ph } } = \frac{1}{\hm{\Ga }}\int{fds} $. \end{proof}





\textbf{2}$^{0}$\textbf{. Принципы максимального модуля.}





\begin{theorem} $Если $ 1) f -- голоморфна в D}

$2) \quad \hm{f}$ достигает мах в D, то}$f \equiv const$.




\begin{proof} $\exi z_0 \in D:\fa z \in D:\hm{f(z)} \le
\hm{f\hr{z_0 }} = m$. 2 случая: 1. m = 0, тогда f = 0, \end{proof}

2. m > 0. Рассмотрим множество Е: $E = \hc{z \in D\vert \hm{f(z)} = m}$.

Свойства Е: 1) Во-первых, оно не пусто: $E \ne \emptyset \hr{z_0 \in E}$.
2) Во-вторых, Е замкнуто относительно D (следует из того, что$\hm{f(z)}$
непрерывна относительно D).

3) В-третьих, Е -- открытое.

Докажем свойство 3): Пусть$z_\ast \in E,$ тогда$\exi U = U\hr{z_\ast }
\subs D$. Докажем: $U \subs E - $ открытое. Некоторое$\ga \in E$.
Доказать, что каждая такая$\ga \in U$предположим обратное, т.е. $\exi
z \in \ga :\hm{f\hr{z'}} < m.\hr{1}$


\[
\exi \ga '\exi \ep > 0:\hm{f} \le m - \ep
н}а}
\ga 'и}
\quad
\hm{f} \le mн}а}
\quad
\ga '' = \ga \backslash \ga
'.Н}а}п}и}ш}е}м}т}е}о}р}е}м}у}о}с}р}е}д}н}е}м}
\quad :
\]




\[
f\hr{z_\ast } = \frac{\intl{\ga }{}{fds} }{\hm{\ga }},m =
\hm{f\hr{z_\ast }} \le \frac{1}{\hm{\ga }}\intl{\ga }{}{\hm{f}ds} =
\frac{1}{\hm{\ga }}\hr{\intl{\ga '}{}{\hm{f}ds} + \intl{\ga
''}{}{\hm{f}ds} } \le
\]


\[
\frac{1}{\hm{\ga }}\{\hr{m - \ep }\hm{\ga '} + m\hm{\ga
''}\} = m - \ep ' < m \Ra
п}р}о}т}и}в}о}р}е}ч}и}е},т}а}к}и}м}о}б}р}а}з}о}м}
\quad ,
\]



\noindent
предположение (1) неверное. $ \Ra E - $ открытое.

Т.к. D -- связанное и из 1),2),3) $ \Ra E = D \Ra \hm{f} =
m$.
Далее рассмотрим: $\wp = \hc{z \in D\vert f(z) = \mu },\mu = f\hr{z_0 }$.

Свойства $\wp $: 1) Оно не пусто: $\wp \ne \emptyset \hr{z_0 \in \wp }$.
2) $\wp $ замкнуто относительно D (непрерывность f).

3) $\wp $-- открытое.

Докажем 3) свойство. Возьмём $z_\ast \in \wp .\xi \in Lnw$, где Ln
--непрерывная ветвь логарифма, а$w \in V$.$\exi U:f\hr{U} \subs V,\xi =
Lnf\hr{z},z \in U$. (композиция голоморфна). $\xi = u + iv,u = ReLnf\hr{z} =
\ln \hm{f(z)} = \ln m,$ поэтому $u'_x = u'_y = 0$. Тогда из условия
Коши Римана следует $v'_x = v'_y = 0. \Ra v = const$ в круге $U
\Ra f \equiv \mu $ в круге $U \Ra U \subs \wp \Ra
\wp - $открытое. Аналогично, $\wp = D,$ таким образом, $f \equiv \mu $ в D.
\end{proof}



\begin{imp}.} Если 1) D -- ограниченная область с границей Г.}

2) f - а) голоморфно в D б) непрерывно в D, тогда }$\hn{f}_\ol{D} = \hn{f}_\Ga ,\hn{f}_\Ga = \mathop {\sup }\limits_{z
\in \Ga } \hm{f(z)}$.
Т.е. здесь сказано, что мах достигается на границе.





\textbf{Вопрос:} Справедлив ли принцип минимального модуля?

\textbf{Ответ:} а) Вообще говоря -- нет.

\noindent
б) если$f \ne 0,$ то -- да.

Т.к. если функция имеет нули, то в нулях -- минимум следовательно а)
доказано, а если функция не имеет нулей, то можно рассмотреть
функцию$\frac{1}{f}$, а для неё работает теорема о принципе мах модуля.





\textbf{3}$^{0}$\textbf{. Лемма Шварца.}





\textbf{Теорема (об устранимой особенности).} Если 1) Функция f -- непрерывна в некоторой окрестности точки z}$_{0}$, т.е. в круге }$U = \{z \in \Cbb\vert
\hm{z - z_0 } \le R\}$.
2) Функция f -- голоморфна в }$U\ast = \{z \in \Cbb\vert 0 < \hm{z - z_0 } \le R\}$. То функция f -- голоморфна в }$U$.




\begin{proof} Пусть $\De - \fa $ замкнутый $\De ,$
лежащий в $U. $Рассмотрим случаи:

1. $z_0 \notin \De ,\intl{\pd \De }{}{fdz} $ (по ИТК) 2. $z_0 \in
\pd \De ,$ проведём дугу окружности радиуса $\ep \Ra
\Ga _\ep $ 3. $z_0 \in \De ^ \circ $


2. $\intl{\Ga _\ep }{}{fdz} = 0 \Ra \intl{\pd \De
}{}{fdz} = 0,\ep \ra 0$.
3. Сводится ко 2-му разбиением на 2 части. Далее f --непрерывна в U и $f \in
\hr{\De }$ в U. Следовательно по теореме Мореры f -- голоморфна в U.


\textbf{Лемма Шварца.} Если 1) f -- голоморфна в }$D = \{z \in \Cbb\vert \hm{z} < 1\} 2) \hm{f(z)} \le
1,z \in D 3) f\hr{0} = 0,$

То 1) выполняются неравенства: а) }$\hm{f(z)} \le \hm{z},z \in D б) \hm{f'\hr{0}} \le 1,$

2) Если в неравенстве а) достигается равенство хотя бы в одной точке отличной от 0, имеем: в неравенстве б) достигается равенство, то }$f(z) = cz,\hm{c} = 1$.
\begin{proof} Определим функцию $g\hr{z} = \frac{f(z)}{z},z
\in D\ast $. Очевидно $g\hr{z} - $ голоморфна в $D\ast . \quad \exi \mathop
{\lim }\limits_{z \ra 0} g\hr{z}\mathop = \limits^{def} f'\hr{0},g\hr{0} =
f'\hr{0} \Ra g - $ непрерывна в D. Тогда $g\hr{z} - $ голоморфна в D
по определению. Зафиксируем $z \in D,$ проведём окружность $\ga
\hr{0,r},\hm{z} < r < 1$. Применим принцип мах модуля: $\hm{g\hr{z}} \le
\hn{g}_\ga \mathop = \limits^{def} \frac{\hn{f}_\ga }{r} \le
\frac{1}{r} \Ra \hm{g\hr{z}} \le 1$. Мы доказали неравенства а) и б).
Теперь пусть выполняется посылка 2) утверждения, $ \Ra \hm{g\hr{z}}$
достигает мах в D, $ \Ra $ по принципу мах модуля $g \equiv const =
c. \quad \end{proof} $





\begin{imp}.} $D \leftrightarrow D,$ конформно, следовательно ДЛП.}

\begin{proof} $w = w\hr{z} - $ берём такое отображение. Тогда $0
\ra a\mathop \ra \limits^{ДЛП} 0,a \in D$ так можно сделать. Тогда без
ограничения общности считаем, что $0 \ra 0$. Тогда к w можно применить лемму
Шварца к обратному отображению $z = z\hr{w} \Ra \hm{z} \le \hm{w}$. А тогда $\hm{z} = \hm{w},z \in D$ и снова применим лемму Шварца, получим $w
= cz$, а это ДЛП. \end{proof}


ФОРМУЛА СОХОЦКОГО

\textbf{1}$^{0}$\textbf{. Постановка задачи.}

Пусть Г -- простая спрямляемая кривая, f -- функция непрерывная на Г.
$F\hr{z} = \frac{1}{2\pi i}\intl{\Ga }{}{\frac{f\hr{\xi }}{\xi - z}} d\xi
,z \in \Cbb\backslash \Ga $ (интеграл типа Коши). Зафиксируем точку $\xi _0
$ на Г. Пусть она отлична от концов кривой.

\begin{df} $Точка \xi _0 $ называется }\textbf{правильной точкой}} кривой Г, если в этой точке существует касательная к кривой.}


Считаем далее, что $\xi _0 $ - правильная. $\exi U = U_r (\xi _0 ):\Ga
_r = \Ga \cap U - $ простая дуга (без самопересечений). Кривая разбивает
U на 2 области: U+ лежит слева от кривой и U$^{ - }$ справа. Обозначим:
$F^\pm = F\vert _{U^\pm } - $ ограничения.


\textbf{Задача.} Вычислить граничные значения.


\begin{df} $z\mathop \ra \limits_\angle z_0 ,$ \textbf{не
касательным способом}}$, если z \ra z_0 , и z \in \fa fix$ сектору с углом < 180}$^{0}$.

\[
F^\pm \hr{\xi _0 } = \mathop {\lim }\limits_{z\mathop \ra \limits_\angle \xi
_0 } F^\pm докажем существование этих пределов и вычислим их \quad .
\]


\begin{df} Пусть число}$\mu : \quad 0 < \mu \le 1$. Будем говорить, что }\textbf{f удовлетворяет
условию Гёльдера с показателем }}$\mu $\textbf{ в точке }}$\xi _0
$\textbf{,}} если выполняется неравенство: }$\hm{f\hr{\xi } - f\hr{\xi _0 }} \le m\hm{\xi - \xi _0
}^\mu ,\xi \in \Ga _\delta ,\Ga _\delta - $ некоторая окрестность.}





\begin{note} На протяжении всего параграфа полагаем, что f
удовлетворяет условию Гёльдера.





\textbf{2}$^{0}$\textbf{. Главное значение интеграла.}





Обозначим главное значение по Коши: $F\hr{\xi _0 } = v.p.\frac{1}{2\pi
i}\intl{\Ga }{}{\frac{f\hr{\xi }}{\xi - \xi _0 }} d\xi \mathop =
\limits^{def} \mathop {\lim }\limits_{r \ra 0} \frac{1}{2\pi i}\intl{\Ga
\backslash \Ga _r }{}{\frac{f\hr{\xi }}{\xi - \xi _0 }} d\xi $, если этот
предел существует.





\textbf{Лемма1.} Пусть Г -- замкнутая кривая, тогда главное значение интеграла существует и вычисляется по формуле: }$F\hr{\xi _0 } = v.p.\ub{ {\frac{1}{2\pi i}\intl{\Ga
}{}{\frac{f\hr{\xi } - f\hr{\xi _0 }}{\xi - \xi _0 }} d\xi }}_{\hr{\ast }} +
\frac{1}{2}f(\xi _0 ),$причём интеграл справа сходится абсолютно.}

\begin{proof} $\frac{1}{2\pi i}\intl{\Ga \backslash \Ga _r
}{}{\frac{f\hr{\xi }}{\xi - \xi _0 }} d\xi = \frac{1}{2\pi i}\ub{
{\intl{\Ga \backslash \Ga _r }{}{\frac{f\hr{\xi } - f\hr{\xi _0 }}{\xi
- \xi _0 }} d\xi }}_{док.\exi } + f\hr{\xi _0 }\ub{ {\intl{\Ga
\backslash \Ga _r }{}{\frac{d\xi }{\xi - \xi _0 }} }}_{док. \ra \pi i},r
\ra 0$. Таким образом надо доказать, что (*) сходится следовательно
абсолютная сходимость. Т.е. $\intl{\Ga _r }{}{\hm{\frac{f\hr{\xi } -
f\hr{\xi _0 }}{\xi - \xi _0 }}} \mathop {\hm{d\xi }}\limits_{ = ds \le Adt}
\mathop \le \limits_{Гёльдер} 2\intl{0}{r}{\frac{mt^\mu }{t}} Adt =
c\intl{0}{r}{\frac{dt}{t^{1 - \mu }}} < \bes ,t = \hm{\xi - \xi _0 }$ т.е.
сходится. \end{proof}


\[
\intl{\xi '}{\xi ''}{\frac{d\xi }{\xi - \xi _0 }} d\xi \mathop = \limits^{Н.
- Л.} Ln\hr{\xi - \xi _0 }\left. \right|_{\xi '}^{\xi ''} = Ln\frac{\xi '' -
\xi _0 }{\xi ' - \xi _0 } = iArg\frac{\xi '' - \xi _0 }{\xi ' - \xi _0 } \ra
\pi
i,т}.к}.у}г}о}л}м}е}ж}д}у}к}а}с}а}т}е}л}ь}н}ы}м}и}в}п}р}е}д}е}л}е}
\quad .
\quad
\end{proof}
\]


\textbf{3}$^{0}$\textbf{. Формулы Сохоцкого.}


\begin{theorem} $F^\pm \hr{\xi _0 } = F\hr{\xi _0 }\pm \frac{1}{2}f\hr{\xi
_0 }$.


\textbf{Лемма2.} $\mathop {\lim }\limits_{z\mathop \ra \limits_\angle \xi _0
} \intl{\Ga }{}{\frac{f\hr{\xi } - f\hr{\xi _0 }}{\xi - z}} d\xi =
\intl{\Ga }{}{\frac{f\hr{\xi } - f\hr{\xi _0 }}{\xi - \xi _0 }} d\xi $.





\begin{proof} Напишем разность этих интегралов: $\intl{\Ga
}{}{\frac{\hr{z - \xi _0 }\hr{f\hr{\xi } - f\hr{\xi _0 }}}{\hr{\xi -
z}\hr{\xi - \xi _0 }}} d\xi = \ub{ {\intl{\Ga \backslash \Ga _r }{}
}}_{\hr{1} \ra 0} + \ub{ {\intl{\Ga _r }{} }}_{\hr{2}}$.

Т.е. надо доказать, что сходится равномерно по z, за счёт выбора r , 2
интеграл можно сделать сколь угодно малым. $\hm{\intl{\Ga _r }{} } \le
2\intl{0}{r}{B\frac{m}{t^{1 - \mu }}Adt = c\intl{o}{r}{\frac{dt}{t^{1 - \mu
}} \ra 0,r \ra 0} } $, берём $\ep > 0,\xi - \xi _0 \Ra $
\end{proof}

\end{document}

\begin{document}
\begin{center}
Лекция 7(14 октября 2002 года).}
\end{center}





\begin{theorem} $F^\pm \hr{\xi _0 } = F\hr{\xi _0 }\pm \frac{1}{2}f\hr{\xi
_0 }$.




\begin{proof} Без ограничения общности считаем Г -- замкнутой
кривой (иначе замкнём и доопределим f). $F\hr{z} = \frac{1}{2\pi
i}\int{\frac{f\hr{\xi }}{\xi - z}} d\xi = \ub{ {\frac{1}{2\pi
i}\intl{\Ga }{}{\frac{f\hr{\xi } - f\hr{\xi _0 }}{\xi - z}} d\xi
}}_{\hr{1}} + f\hr{\xi _0 }\ub{ {\frac{1}{2\pi i}\intl{\Ga }{}{\frac{d\xi
}{\xi - z}} }}_{\hr{2}}$ перейдём к пределу, когда $z\mathop \ra
\limits_\angle z_0 ,$тогда (1) $\mathop \ra \limits_{лем2} \frac{1}{2\pi
i}\intl{\Ga }{}{\frac{f\hr{\xi } - f\hr{\xi _0 }}{\xi - \xi _0 }} d\xi
\mathop = \limits_{лем1} F\hr{\xi _0 } - \frac{1}{2}f\hr{\xi _0 }$

(2): Рассмотрим 2 случая: $\hr{2} = \left\{ {{\begin{array}{*{20}c}
 {1,если} \hfill \\
 {0,если} \hfill \\
\end{array} }} \right.{\begin{array}{*{20}c}
 {z \in U^ + } \hfill \\
 {z \in U^ - } \hfill \\
\end{array} }$ (по ИТК) $ \Ra ч.т.д.\end{proof} $





\textbf{4}$^{0}$\textbf{. Приложения.}





Рассмотрим задачу: Пусть Г -- простая замкнутая кривая (по теореме Жордана
она разбивает плоскость на 2 области: D$^{ + }$ и D$^{ - })$. Предположим,
что кривая гладкая. Пусть на Г задана функция f. Предположим, что функция
гладкая. $Вопрос:$ Найти необходимое и достаточное условие, чтобы функция f -- была
граничным значением функции, голоморфной в области D$^{ + }$. Может подходит
любая функция?

Ни фига!!! Необходимое условие:} $\intl{\Ga }{}{fd\xi } \equiv 0$ (по ИТК).





\textbf{Пример1.} $f\hr{\xi } = \ol{\xi },\xi \in {\rm T} = \{z \in
\Cbb:\hm{z} = 1\}\int{\ol{\xi }d\xi = 2\pi i - } $функция не является
граничным значением, функции, голоморфной в области D$^{ + }$.





\textbf{Пример2.} $f\hr{\xi } = \ol{\xi }^2,\xi \in {\rm T},\intl{{\rm
T}}{}{\ol{\xi }^2d\xi = 0} - $ эта функция не является граничным значением
никакой голоморфной функции. Докажите это сами. Т.е. необходимое условие не
является достаточным (найти достаточное условие).





Найти необходимое и достаточное условие, чтобы интеграл типа Коши являлся
интегралом Коши. Вспомним чему равен интеграл Коши: Если F -- интеграл Коши,
то $F\hr{z} = \left\{ {{\begin{array}{*{20}c}
 {f(z),z \in D^ + } \hfill \\
 {\ub{ {0,z \in D^ - }}_{\hr{\ast }}} \hfill \\
\end{array} }} \right.{\begin{array}{*{20}c}
 {формула} \hfill \\
 {по} \hfill \\
\end{array} }{\begin{array}{*{20}c}
 {Коши} \hfill \\
 {ИТК} \hfill \\
\end{array} }$ (*) -- это не только необходимое, но и достаточное условие.
$F\hr{z} = \frac{1}{2\pi i}\intl{\Ga }{}{\frac{f\hr{\xi }}{\xi - z}d\xi }
$. Докажем это.





\begin{theorem} Интеграл типа Коши будет интегралом Коши }$ \Leftrightarrow F^ - \equiv 0$.




\begin{proof} В одну сторону мы знаем по ИТК. В другую: $\hr{
\Leftarrow }$ Пусть $F^ - \equiv 0$. Из формулы Сохоцкого следует, что $F^
+ + F^ - \equiv f$ на Г для любого интеграла. В нашем случае остаётся, что
$f = F^ + $ на Г ($F^ + - $ функция голоморфная в D$^{ + }) \Ra $
\end{proof}





ПОСЛЕДОВАТЕЛЬНОСТИ ГОЛОМОРФНЫХ ФУНКЦИЙ





\textbf{1}$^{0}$\textbf{. Пространство голоморфных функций.}





Обозначим $H\hr{D} - $ линейное пространство, состоящее из всех функций
голоморфных в области D.





\begin{df} Будем говорить, что последовательность }$f_n $ сходится к f }\textbf{равномерно}}$ внутри D\hr{f_n \mathop
\ra \limits_D f}$, если для любого компакта }$K \subs D:f_n \ra f$ на К (равномерная сходимость). Т.е. }$\hn{f - f_n }_k \ra 0,n \ra \bes
, где \hn{f}_K = \mathop {\sup }\limits_{z \in H} \hm{f(z)} - $\textbf{равномерная
норма}}$$.




\textbf{Пример1.} $z^n\mathop \ra \limits_D 0,D - $ единичный круг (внутри
его). Но при этом: $z^n\mathop \ra \limits_{равномерно} 0$ на D (на всём
круге).





\textbf{Утверждение.} Сходимость в пространстве H(D) метризуема (порождается некоторой метрикой).}





\begin{proof} Обозначим: $K_n = \left\{ {\left. {z \in D\vert
\hm{z} \le n,dist\hr{z,\pd D} \ge \frac{1}{n}} \right\} ,n =
1,2,...;dist - } \right$. расстояние. Тогда $K_n $ - компакт (ограниченное,
замкнутое множество).

Свойства $K_n $: 1) $K_n \subs K_{n + 1} $ возрастает. 2) $\cupl{n =
1}{\bes } K_n = D$.

Если какая-либо последовательность будет обладать такими 2 свойствами, то
будем называть $\hc{K_n }_{n = 1}^\bes - $ исчерпанием} D. Тогда $\hc{\hn{ \cdot
}_{K_n } }_{n = 1}^\bes - $ порождает сходимость в нашем пространстве, а
тогда $\rho \hr{f,g} = \sum\limits_{n = 1}^\bes {\frac{1}{2^n}\frac{\hn{f
- g}_{K_n } }{1 + \hn{f - g}_{K_n } }} - $ формула из действительного
анализа. То, что это метрика проверьте самостоятельно. Утверждение доказано
(предъявлена метрика). \end{proof}





\textbf{2}$^{0}$\textbf{. Первая теорема Вейерштрасса.}





\begin{theorem} $Если 1) f_n - $ голоморфна в области D. }

$2) f_n \mathop \ra \limits_D f - $ равномерно. То а) }$f - $ голоморфна в области D. }

$б) f_n '\mathop \ra \limits_D f' - $ равномерно}.

\begin{proof} а) Пусть $U \subs D$ произвольный круг.

\noindent
функция f -- непрерывна в U (как равномерный предел непрерывных функций).

$\fa \De \in U:\intl{\pd \De }{}{f_n dz} \ra \intl{\pd
\De }{}{fdz} $



\noindent
т.к. сходимость равномерная. Но $\intl{\pd \De }{}{f_n dz} = 0$ по
ИТК $ \Ra \intl{\pd \De }{}{fdz} = 0 \Ra f \in
\hr{\De }$ в U, т.е. по теореме Мореры f -- голоморфна в U. $ \Ra
f - $ голоморфна в D (т.к. U -- любой круг в D).

\noindent
б) Рассмотрим 2 круга: ${\begin{array}{*{20}c}
 {U_R \hc{z \in \Cbb,\hm{z - z_0 } < R}} \hfill \\
 {U_{2R} \hc{z \in \Cbb,\hm{z - z_0 } < 2R}} \hfill \\
\end{array} }$ обозначим $\ga = \pd U_{2R} $.
Предположим, что $U_{2R} \subs D$. Докажем, что $f_n '\mathop \ra \limits_
f' - $ равномерно на $U_R $.

Зафиксируем $z \in U_R $, тогда $\hm{f'\hr{z} - f_n '\hr{z}} =
\hm{\frac{1}{2\pi i}\intl{\ga }{}{\frac{f\hr{\xi } - f_n \hr{\xi
}}{\hr{\xi - z}^2}d\xi } } \le \frac{1}{2\pi }\frac{\hn{f - f_n }_\ga
}{R^2} \cdot 2\pi \cdot 2R. \quad \hn{f' - f_n '}_{U_R } \le \frac{2}{R}\hn{f -
f_n }_\ga \ra 0,n \ra \bes ,$ по условию на любом компакте. Пусть $K
\subs D$ - любой компакт. Покроем К конечным числом кругов типа $U_R $,
т.к. сходимость равномерная на каждом круге, то она равномерная на всём
компакте (круги произвольные). \end{proof}





\begin{imp}.} H(D) -- полное.}





\textbf{3}$^{0}$\textbf{. Вторая теорема Вейерштрасса.}





\begin{theorem} Если 1) D -- ограниченная область с границей }$\Ga = \pd D$.
$2) f_n - $ голоморфна в области D, непрерывна в }$\ol{D}$.
$3) f_n $ сходится равномерно на Г. То }$f_n $ сходится равномерно внутри D.}





\begin{proof} Мы докажем, что $f_n \mathop \ra \limits_\ol{D} f
- $ равномерно. Проверим критерий Коши: $\hn{f_n - f_m }_\ol{D} \mathop =
\limits_{\max }^{принцип} \hn{f_n - f_m }_\Ga \ra 0,n,m \ra \bes ,$ по
условию. \end{proof}





\section{СТЕПЕННЫЕ РЯДЫ}

\textbf{1}$^{0}$\textbf{. Область сходимости степенного ряда.}


Степенной ряд имеет вид: $\sum\limits_{n = 0}^\bes {a_n \hr{z - z_0 }^n} $
(все параметры комплексные). В дальнейшем без ограничения общности: $z_0 =
0$.

\textbf{Теорема (Первая теорема Абеля).} Если ряд }$\sum\limits_{n = 0}^\bes {a_n
\hr{z}^n} $ (*) сходится в точке }$z_1 ,$ то он сходится равномерно внутри круга: }$\hm{z} \le \hm{z_1 }$.




\begin{proof} Имеем $a_n z^n \ra 0$ (общий член ряда $ \ra 0$;
необходимое условие сходимости ряда). В частности, $\hm{a_n z_1 ^n} \le M$.
Фиксируем $0 < q < 1$. И рассмотрим: $\hm{z} \le q\hm{z_1 }$ (любой круг
строго меньший). Тогда $\hm{a_n z^n} = \hm{a_n z_1 ^n\hr{\frac{z}{z_1 }}^n}
\le Mq^n - $ убывающая геометрическая прогрессия (по признаку сравнения ряд
сходится тогда равномерно). \end{proof}





\begin{imp}.} Областью сходимости степенного ряда является некоторый круг. Назовём }$\hm{z} < R - $\textbf{радиусом сходимости}}$$.




\begin{note} Не исключаются случаи, когда R = 0 (расходится везде
кроме одной точки) или $R = \bes $ (везде сходится).





\begin{ex} 1) $\sum\limits_{n = 1}^\bes {\hr{nz}^n} ,R = 0$
расходится везде кроме 0.

2) $\sum\limits_{n = 1}^\bes {\hr{\frac{z}{\rho }}^n} ,R = \rho $

3) $\sum\limits_{n = 1}^\bes {\hr{\frac{z}{n}}^n} ,R = \bes $ по
признаку Даламбера, сходящийся ряд.





\begin{note} На границе области сходимости поведение ряда может быть
любым.





\begin{ex} R = 1 -- единичная окружность.

1) $\sum\limits_{n = 1}^\bes {\hr{z}^n} ,$ расходится на единичной
окружности.

2) $\sum\limits_{n = 1}^\bes {\frac{\hr{z}}{n}^n}
,{\begin{array}{*{20}c}
 {z = 1 - } \hfill \\
 {z \ne 1 - } \hfill \\
\end{array} }{\begin{array}{*{20}c}
 {расходится} \hfill \\
 {сходится} \hfill \\
\end{array} }{\begin{array}{*{20}c}
 \hfill \\
 {условно} \hfill \\
\end{array} }$ на единичной окружности.

3) $\sum\limits_{n = 1}^\bes {\frac{\hr{z}}{n^2}^n} ,$ сходится абсолютно
и равномерно на единичной окружности.

4) $\sum\limits_{n = 1}^\bes {\frac{\hr{z}}{n}^n\hr{ - 1}^{[\sqrt n ]}} ,$
сходится условно во всех точках на окружности.





\textbf{Теорема (Формула Коши-Адамара).} Радиус сходимости вычисляется так: }$\frac{1}{R} = \ol {\mathop {\lim
}\limits_{n \ra \bes } } \sqrt[n]{\hm{a_n }}$.




\begin{proof} 1) Пусть $\frac{1}{R} = \ol {\mathop {\lim
}\limits_{n \ra \bes } } \sqrt[n]{\hm{a_n }},\hm{z} < R$. Мы должны
доказать, что ряд тогда сходится. Напишем: $\hm{z} < r < R$, тогда
$\frac{1}{R} < \frac{1}{r} \Ra $ начиная с некоторого номера:
$\exi N\fa n \ge N:\hm{a_n } < \hr{\frac{1}{r}}^n,$ а тогда $\hm{a_n
z^n} < \hr{\frac{\hm{z}}{r}}^n < 1 \Ra $ по признаку сравнения ряд
сходится.

2) Пусть $\frac{1}{R} = \ol {\mathop {\lim }\limits_{n \ra \bes } }
\sqrt[n]{\hm{a_n }},\hm{z} > R$. Мы должны доказать, что ряд тогда
расходится. $\frac{1}{R} > \frac{1}{\hm{z}} \Ra \exi \La
\subs N:\fa n \in \La ,\La - $ подпоследовательность
натурального ряда. $\hm{a_n } < \hr{\frac{1}{\hm{z}}}^n \Ra \hm{a_n
z^n} > 1 \Ra $

\noindent
по признаку сравнения ряд расходится. \end{proof}





\textbf{Утверждение.} 1) Степенной ряд сходится к функции, голоморфной в круге сходимости.}

2) Степенной ряд является рядом Тейлора для своей суммы. }$a_n = \frac{f^{\hr{n}}\hr{0}}{n!}$.




\begin{proof} 1) По первой теореме Абеля ряд сходится равномерно
внутри круга сходимости. По первой теореме Вейерштрасса его сумма
голоморфна, следовательно всё клёво.

2) $f^{\hr{n}}\hr{z}$ По первой теореме Вейерштрасса можно ряд
дифференцировать почленно. Т.е. $f^{\hr{n}}\hr{z} = \sum\limits_{k =
n}^\bes {a_k k\hr{k - 1}...\hr{k - n + 1}z^{n - k}} $. Подставим $z = 0
\Ra f^{\hr{n}}\hr{0} = a_n n! \Ra \end{proof} $





\textbf{Утверждение (Переформулировка).} В степенной ряд можно разложить только голоморфную функцию, при этом единственным образом -- только в ряд Тейлора.}





\begin{ex} $f(z) = e^{ - \frac{1}{z^2}} - $ нельзя разложить в
степенной ряд в окрестности 0. Т.к. она не голоморфна в 0.





\textbf{Теорема (Вторая теорема Абеля).} Если степенной ряд }$\sum\limits_{n = 0}^\bes {a_n
\hr{z}^n} $ сходится в точке }$z_1 ,$ то он сходится равномерно на [0, z}$_{1}]$.




\begin{proof} Сведём теорему к признаку сходимости числовых
рядов. Без ограничения общности $z_1 = 1$ (замена переменных), тогда 1.
$\sum\limits_{n = 0}^\bes {a_n } $ - сходится.

2. $\hc{x^n}_{n = 1}^\bes - $ равномерно ограничена на [0, 1].

3. $\hc{x^n}_{n = 1}^\bes - $ не возрастающая, а тогда это признак Коши.

$\sum\limits_{n = 0}^\bes {a_n x^n}
с}х}о}д}и}т}с}я}р}а}в}н}о}м}е}р}н}о}н}а}$[0,
1]. \end{proof}



\textbf{2}$^{0}$\textbf{. Разложение голоморфной функции в степенной ряд.}





\textbf{Теорема (Коши).} Если f голоморфна в }$U = \hc{z \in \Cbb\vert \hm{z} < R},$ то её можно разложить в степенной ряд в этом круге.}





\begin{proof} $\hm{z} < r < R$ Пусть $\xi \in \ga - \fa $
точка. Тогда $\frac{1}{\xi - z} = \frac{1}{\xi }\frac{1}{1 - \frac{z}{\xi }}
= \sum\limits_{n = 0}^\bes {\frac{z^n}{\xi ^{n + 1}}} $ ряд сходится
равномерно на $\ga $, т.к. $\hm{\frac{z}{\xi }} = \frac{\hm{z}}{r} < 1$. Но равномерно сходящиеся ряды можно интегрировать почленно: $f(z) =
\frac{1}{2\pi i}\intl{\ga }{}{\frac{f\hr{\xi }}{\xi - z}d\xi =
\sum\limits_{n = 0}^\bes {z^n\frac{1}{2\pi i}} } \intl{\ga
}{}{\frac{f\hr{\xi }}{\xi ^{n + 1}}d\xi } = \sum\limits_{n = 0}^\bes
{z^na_n } $. \end{proof}





\begin{imp}.} Круг сходимости степенного ряда совпадает с мах кругом голоморфности для его суммы }$f(z) = \sum\limits_{n = 0}^\bes {a_n z^n} $, голоморфная функция = ряду Тейлора.}





\begin{proof} Круг сходимости $ \subs $ кругу голоморфности.
Если в каком-то круге функция голоморфна, то уж в нём-то она разлагается в
степенной ряд, а может быть и больше. \end{proof}





\begin{ex} $f(z) = \frac{1}{z^2 + 1} = \sum\limits_{n = 0}^\bes
{\hr{ - z^2}^2,\hm{z}} < 1$. Мах круг голоморфности


\[
D_{голом.} = \hc{\hm{z} < 1},z = \pm i -
о}с}о}б}ы}е}т}о}ч}к}и}
\quad .Р}а}з}р}ы}в} \quad
,никак не доопределить.
\]


\end{document}


%Лекция 8(21 октября 2002 года).}

\textbf{3}$^{0}$\textbf{. Пример.}

Найти сумму тригонометрического ряда: $S\hr{\ph } = \sum\limits_{n =
1}^\bes {\frac{\sin n\ph }{n}} ,\ph \in R,0 \le \ph < 2\pi ,$
$S\hr{0} = 0,0 < \ph < 2\pi ,E\hr{\ph }\sum\limits_{n = 1}^\bes
{\frac{e^{2n\ph }}{n}} , \quad S\hr{\ph } = ImE\hr{\ph }, \quad f(z)
= - Ln\hr{1 - z},\hm{z} < 1 - $ главная ветвь логарифма. $\hm{Argz} <
\frac{\pi }{2},f\hr{0} = 0, \quad f(z) = \sum\limits_{n = 1}^\bes
{\frac{z^n}{n},\hm{z} < 1\hr{\ast }} $. Исследовать сходимость ряда на
границе. Пусть $\hm{z} = 1,$ но $z \ne 1$ (т.к. при $z = 1$ ряд расходится).
Убедимся, что ряд сходится: 1) $\hm{\sum\limits_{n = 1}^N {z^n} } =
\hm{z\frac{1 - z^N}{1 - z}} \le \frac{2}{\hm{1 - z}} - $ограничена

2) $\frac{1}{n}$ - убывает и стремится к 0, следовательно по теореме Дирихле
ряд (*) сходится. По 2ой теореме Абеля ряд сходится равномерно на отрезке от
[0, z]. Таким образом, мы имеем равенство и на границе тоже, исключая $z =
1$. \textbf{Ответ:} $S\hr{\ph } = - ImLn\hr{1 - e^{i\ph }} = -
Arg\hr{1 - e^{i\ph }}$.

\textbf{Упражнение.} Сделать проверку, разложив в ряд Фурье.}

\section{Теорема единственности и теорема Лиувилля}

\textbf{1}$^{0}$\textbf{. Теорема единственности для голоморфных функций.}





\begin{theorem} Если 1) функции f и g голоморфны в области D.}

$2) f = g на E \subs D$.
3) Е имеет хотя бы одну предельную точку в D, то }$f = g$ всюду в области D.}





\begin{proof} Переформулируем для разности $f - g:$ 1) функция f
голоморфна в области D.

2) $f = 0$ на $E \subs D$. 3) Е имеет хотя бы одну предельную точку в D.

Доказать: $f = 0$ в D. Теперь 2 случая:

\noindent
а) Область -- это круг (докажем для этого случая).

Дано: 1) f -- голоморфна в $U = \{z \in \Cbb\vert \hm{z - z_0 } < R\}$ (это
снова переформулировка для этого случая)

2) $f\hr{z_n } = 0,z_n \in U$

3) $z_n \ne z_0 ,z_n \ra z_0 $. Доказать: $f = 0$ в U. Если функция
голоморфна в круге, то она раскладывается в степенной ряд: $f(z) =
\sum\limits_{n = 0}^\bes {a_n \hr{z - z_0 }^n,z \in U.}  \quad \ub{
{f\hr{z^n}}}_{ = 0} \ra \ub{ {f\hr{z_0 }}}_{ = 0} \Ra a_0 = 0$. Рассмотрим новую функцию (ряд): $f_1 \hr{z} = \sum\limits_{n = 1}^\bes
{a_n \hr{z - z_0 }^{n - 1},z \in U.} $ ряд сходится в круге. И тогда
запишем: $f_1 \hr{z} = \frac{f(z)}{z - z_0 },z \ne z_0 \hr{1}$ и $f_1
\hr{z} = a_1 ,z = z_0 \hr{2}$. В силу непрерывности $\ub{ {f_1 \hr{z_n
}}}_{по\hr{1} = 0} \ra f\hr{z_0 },$ тогда $f_1 \hr{z_0 } = 0 \Ra a_1
= 0\hr{по\hr{2}}$. Рассмотрим функцию $f_2 \hr{z} = \sum\limits_{n =
2}^\bes {a_n \hr{z - z_0 }^{n - 2},z \in U.} $ Ряд сходится в круге. $
\Ra a_2 = 0$ аналогично. И так далее по индукции $a_0 = a_1 = a_2 =
... = 0 \Ra f = 0. \quad \end{proof} $

\noindent
б) В общем случае. Обозначим $\ep = \{z \in D\vert f(z) =
0\},\ep \ast = \{z \in D\vert z - $ предельная точка для
$\ep \}$, тогда

1) $\ep \ast \ne \emptyset $ (по условию теоремы).

2) $\ep \ast - $ замкнуто относительно D (по общему свойству
множества предельных точек).

3) $\ep \ast - $ открыто (доказано в а)). $ \Ra $ по
определению связного множества $\ep \ast = D$.
Заметим, что $\ep - $ замкнуто относительно D (из непрерывности
функции f), но замкнутое множество содержит все свои предельные точки, кроме
того $\ep \subs D,\ep \ast \subs \ep \subs D
\Ra \ep = D. \quad \end{proof} $





\begin{note} Теорема говорит о том, что нули у голоморфной функции
могут быть только изолированными, если она не равна 0.

(Но к границе области нули скапливаться могут).





\begin{df} f -- голоморфна в области D. }$\wt {f}$ -- голоморфна в области }$\wt {D}. \quad D \subs \wt {D},f =
\wt {f}$ в D. Тогда говорят, что }$\wt {f}$\textbf{ -- аналитическое продолжение функции
f.}}





\begin{note} Из теоремы единственности следует, что определение
корректно, т.е. если аналитическое продолжение существует, то оно
единственно.





\begin{ex} $f(z) = \sum\limits_{n = 0}^\bes {\hr{z}^n,z \in D =
\{z\vert \hm{z} < 1\} - } $ одна голоморфная функция. $\wt {f}\hr{z} =
\frac{1}{1 - z},z \in \wt {D} = \Cbb\backslash \{1\} - $ другая голоморфная
функция. Имеем $D \subs \wt {D},D - $ единичный круг, $\wt {D} - $
вся плоскость без одной точки, $f = \wt {f}\left. \right|_D - $ имеем
аналитическое продолжение (мы аналитически продолжили на всю плоскость и это
продолжение !-но).





\textbf{2}$^{0}$\textbf{. Неравенства Коши и теорема Лиувилля.}





\begin{theorem} $Если f(z) = \sum\limits_{n = 0}^\bes {a_n \hr{z}^n - } $ сходится в круге }$U
= \{z \in \Cbb\vert \hm{z} < R\},$ то для любого }$0 < r < R$ выполняется неравенство Коши: }$\hm{a_n } \le
\frac{M\hr{r}}{r^n},n = 0,1,2... где M\hr{r} = \mathop {\max }\limits_{\hm{z} =
r} \hm{f(z)}$ (норма функции f).}





\begin{proof} Напишем формулу Коши для производной (для
коэффициентов ряда): $a_n = \frac{1}{2\pi i}\intl{\hm{z} <
r}{}{\frac{f\hr{\xi }d\xi }{\xi ^{n + 1}}} $

Оценим этот интеграл: $\hm{a_n } \le \frac{1}{2\pi }\frac{M\hr{r}}{r^{n +
1}}2\pi r = \frac{M\hr{r}}{r^n}. \quad \end{proof} $





\textbf{Теорема (Лиувилля).} Если 1) f -- голоморфна в С 2) f -- ограничена в С , то f = const.}





\begin{proof} Разложим функцию в ряд Тейлора: $f(z) =
\sum\limits_{n = 0}^\bes {a_n \hr{z}^n,}  \quad R = \bes $- радиус
сходимости (т.к. ряд сходится во всей плоскости, т.к. функция голоморфна).
$\hm{a_n } \le \frac{M\hr{r}}{r^n} \le \frac{M}{r^n} - $ по неравенству Коши
$0 < r < \bes - $ любое. Зафиксируем $\fa n \ge 1$ и перейдём к
пределу при $r \ra \bes \Ra a_0 = 0. \quad f(z) = a_0 ,т.е.f \equiv
const. \quad \end{proof} $





\textbf{Задача.} 1) f -- голоморфна в С 2) $f(z) = O\hr{\hm{z^\nu }},z
\ra \bes (\nu > 0)$. Тогда f -- многочлен степени не выше $\nu $ (или
целая часть $\nu $, что одно и тоже).





ОСОБЫЕ ТОЧКИ СТЕПЕННОГО РЯДА





\textbf{1}$^{0}$\textbf{. Особые точки на границе круга сходимости.





Дан степенной ряд: $f(z) = \sum\limits_{n = 0}^\bes {a_n z^n.} $
Обозначим R -- радиус сходимости этого ряда. Предположим: $0 < R < \bes $. Обозначим: U -- круг, где сходится, Г -- граница круга. Зафиксируем точку
$\xi $ на границе круга сходимости.





\begin{df} $Точка \xi $ называется }\textbf{правильной}} точкой степенного ряда, если функция }$f(z)$ аналитически продолжается в некоторую окружность этой точки.}

\begin{df} $Точка \xi $ называется }\textbf{особой}} точкой степенного ряда, если она не является правильной.}





\begin{note} Из определения следует, что правильные точки образуют
открытое

\noindent
множество на окружности Г. А особые точки -- замкнутое.





\textbf{Утверждение.} На границе круга сходимости есть хотя бы одна особая точка.}





\begin{proof} Предположим обратное. Т.е. все точки -- правильные.
Тогда у каждой точки $\fa \xi \in \Ga \exi U\hr{\xi } - $
окрестность: f -- аналитически продолжается в $U\hr{\xi }$. Из открытого
покрытия компакта Г выберем конечное подпокрытие: $U\hr{\xi _1
},...,U\hr{\xi _m }$. И рассмотрим: $D = U \cup U\hr{\xi _1 } \cup ... \cup
U\hr{\xi _m }$. Имеем 2 свойства: 1) f -- аналитически продолжается в
области D (по теореме единственности) 2) $D \sups \wt {U} = \{z \in
\Cbb\vert \hm{z} < \wt {R}\},\wt {R} > R,\wt {R} - $ ближайший к
началу координат. $ \Ra $ противоречие с тем, что круг сходимости
степенного ряда совпадает с максимальным кругом голоморфности его суммы
($D_{сх} = D_{гол} ) \Ra $ предположение не верно. $ \Ra $
утверждение. \end{proof}





\textbf{2}$^{0}$\textbf{. Теорема Прингсхейма.}





\begin{theorem} $Если 1) f(z) = \sum\limits_{n = 0}^\bes {a_n z^n.}  \quad R = 1 -
$ радиус сходимости. 2) }$a_n \ge 0$

$То: \xi = 1 - $ особая точка ряда.}

\begin{proof} $fix\fa \hr{ \cdot },0 < x_0 < 1$. И напишем
разложение функции f(z) в ряд Тейлора с центром в этой точке. $f(z) =
\sum\limits_{n = 0}^\bes {\frac{f^{\hr{n}}\hr{x_0 }}{n!}\hr{z - z_0 }^n.}
$ Радиус сходимости обозначим R$_{0}$. Он вычисляется по формуле
Коши-Адамара: $\frac{1}{R_0 } = \mathop {\ol {\lim } }\limits_{n \ra \bes
} \sqrt[n]{\hm{\frac{f^{\hr{n}}\hr{x_0 }}{n!}}}$ Утверждается, что R$_{0 }$>
x$_{0}$ , т.к. если предположить обратное, т.е. что 1 -- правильная точка,
тогда существует окрестность, где функция аналитически продолжается, то
функция голоморфна в области.


\[
 \Ra
а тогда она здесь сходится
\quad
,чтоневерно
\quad .
\quad
 \Ra \xi \in \Ga = \{z \in \Cbb\vert \hm{z} = 1\},x_1 = \xi x_0 .
\]



Напишем такой ряд: $f(z) = \sum\limits_{n = 0}^\bes
{\frac{f^{\hr{n}}\hr{x_1 }}{n!}\hr{z - x_1 }^n.} $ Обозначим $R_1 - $ радиус
сходимости

\noindent
такого ряда. Он равен: $\frac{1}{R_1 } = \mathop {\ol {\lim } }\limits_{n
\ra \bes } \sqrt[n]{\hm{\frac{f^{\hr{n}}\hr{x_1 }}{n!}}}$ Сравним
производные. Для этого: $\hm{f^{\hr{n}}\hr{x_1 }} = \hm{\sum\limits_{k =
n}^\bes {a_k k\hr{k - 1}...\hr{k - n + 1}x_1 ^{k - n}} } \le
\sum\limits_{k = n}^\bes {\ub{ {a_k }}_{ \ge 0}k\hr{k - 1}...\hr{k - n +
1}x_0 ^{k - n}} = f^{\hr{n}}\hr{x_0 } \Ra $ производная в точке
х$_{1}$ не превосходит производной в точке х$_{0} \quad  \Ra \ol {\lim }
\hr{x_1 } \le \ol {\lim } \hr{x_0 }$ , таким образом, $\frac{1}{R_1 } \le
\frac{1}{R_0 } \Ra R_1 \ge R_0 > 1 - \hm{x_1 }$

Следовательно наша функция продолжается в некоторую окрестность точки $\xi $
, т.е. мы доказали, что $\xi $ правильная точка. (А мы предполагали, что 1
-- правильная точка, $ \Ra $ получили что любая точка на границе
правильная) $ \Ra $ противоречие.





\textbf{Теорема (Фабри об отношениях).} Дан степенной ряд: }$f(z) = \sum\limits_{n =
0}^\bes {a_n z^n.} $

$Если \frac{a_n }{a_{n + 1} }$ имеет предел }$\xi _0 , то \xi _0 $ - особая точка степенного ряда }


\[
f(z) = \sum\limits_{n = 0}^\bes {a_n z^n.} Т \quad .е.утверждается
\quad ,чтовкругерадиуса
\quad
\xi _0 \quad - рядсходитсяиестьособаяточка \quad .
\]





\begin{proof} Без него!!!





\textbf{3}$^{0}$\textbf{. Пример Адамара.}





Рассмотрим степенной ряд: $f(z) = \sum\limits_{n = 0}^\bes
{\frac{z^2^n}{2^{n^2}} - } $ ``лакунарный ряд'' (в нём много 0
коэффициентов). Легко видеть, что R = 1. На единичной окружности ряд
сходится абсолютно и равномерно. Поэтому $f \in \Cbb_R^\bes \hr{\Ga
},\Ga - $ единичная окружность. Докажем, что все такие точки Г -
единичной окружности -- особые для этого ряда. (Пример функции, не
продолжаемой за пределы единичного круга).





\begin{proof} а) Рассмотрим $\xi = 1$ - она особая (по теореме
Принсхейма).

\noindent
б) Рассмотрим $\xi = \exp \hr{2\pi i\frac{l}{2^m}},l,m \in N$ Докажем, что
они особые. Для этого: $f(z) = \sum\limits_{n = 0}^{m - 1} +
\sum\limits_{n = m}^\bes = p\hr{z} + g\hr{z}$ Заметим, что $\xi $ - особая
для $f(z) \Leftrightarrow \xi - $ особая точка для $g\hr{z}$ (т.к. есть
аналитическое продолжение) $ \Leftrightarrow 1 - $ особая точка для
$g\hr{\xi z} = g\hr{z} - $замена переменных (т.к. если $\xi ^{2^n}$, то
получим 1) $ \Leftrightarrow 1 - $ особая для $f(z) - $ доказано в а).

\noindent
в) 1) Множество всех $\xi = \exp \hr{2\pi i\frac{l}{2^m}},l,m \in N$ всюду
плотно на единичной окружности Г.

2) Особые точки образуют замкнутое множество

$ \Ra
в}с}е}т}о}ч}к}и}Г}$--
особые. \end{proof}


\begin{center}
Лекция 9(28 октября 2002 года).}
\end{center}


РЯД ЛОРАНА

\textbf{1}$^{0}$\textbf{. Область сходимости ряда Лорана.}


\[
f(z) = \sum\limits_{n = - \bes }^\bes {\mathop {a_n \hr{z - z_0
}^n}\limits_{\hr{L}} = \sum\limits_{n = 0}^\bes {\mathop {a_n \hr{z - z_0
}^n}\limits_{\hr{P}} + \mathop {\sum\limits_{n = 1}^\bes {\frac{a_{ - n}
}{\hr{z - z_0 }^n}.} }\limits_{\hr{Q}} } }
\]



\begin{df} $Ряд $\textbf{(L)}} -- сходится }$ \Leftrightarrow $ сходится каждый из рядов }\textbf{(P)
и (Q).}}



Найдём области сходимости рядов (P) и (Q).

(P) -- степенной ряд. Его область сходимости некоторый круг с радиусом (по
формуле Коши-Адамара): $\hm{z - z_0 } < R,\frac{1}{R} = \mathop {\ol {\lim }
}\limits_{n \ra \bes } \sqrt[n]{\hm{a_n }}$. Теперь для ряда (Q). Сделаем
замену переменных: $\xi = \frac{1}{z - z_0 } \Ra $ степенной ряд
$\sum\limits_{n = 1}^\bes {\frac{a_{ - n} }{\hr{z - z_0 }^n} =
\sum\limits_{n = 1}^\bes {a_{ - n} \xi ^n} .} $ Его область сходимости
некоторый круг: $\hm{\xi } < \frac{1}{r},r = \ol {\mathop {\lim }\limits_{n
\ra \bes } } \sqrt[n]{\hm{a_{ - n} }}$ по формуле Коши-Адамара. Тем самым
область сходимости ряда (Q) $\hm{z - z_0 } > r$.
\noindent
а) Если $R \le r \Ra $ область сходимости ряда (L) $ = \emptyset $.

\noindent
б) Если $r < R \Ra $ область сходимости ряда (L) -- кольцо $K =
\hc{z \in \Cbb\vert r < \hm{z - z_0 } < R}$.




\begin{df} $Ряд $\textbf{(L)}} с непустой областью сходимости называется }\textbf{рядом Лорана}}$$.




\textbf{Утверждение.} 1) Ряд Лорана сходится равномерно внутри К (кольца сходимости).}

2) Сумма ряда Лорана -- функция, голоморфная в кольце сходимости (т.е. в ряд Лорана можно разложить только голоморфные функции).}

3) Разложение функцию, голоморфной в некотором кольце в ряд Лорана, единственно.}

\begin{proof} Пусть $f(z) = \sum\limits_{n = - \bes }^\bes
{\mathop {a_n \hr{z - z_0 }^n}\limits_ } ,z \in K = \hc{z \in \Cbb\vert r <
\hm{z - z_0 } < R}$.

Ряд сходится равномерно на $\ga $. Равномерно сходящийся ряд можно
интегрировать почленно. Проведём окружность $\ga $ радиуса $r < \rho < R$
в кольце. $\frac{1}{2\pi i}\intl{\ga }{}{\frac{f(z)}{\hr{z - z_0 }^{k
+ 1}}dz = \sum\limits_{n = - \bes }^\bes {a_n } } \ub{ {\frac{1}{2\pi
i}\intl{\ga }{}{\frac{dz}{\hr{z - z_0 }^{k - n + 1}} = a_k .} }}_{ =
\left\{ {{\begin{array}{*{20}c}
 {1,n = k} \hfill \\
 {0,n \ne k} \hfill \\
\end{array} }} \right.}$ Т.е. мы явно вычислили коэффициент, докажем
единственность.





\textbf{2}$^{0}$\textbf{. Разложение голоморфной функции в ряд Лорана.}





\begin{theorem} Если функция }$f(z)$ голоморфна в }$K = \hc{z \in \Cbb\vert r < \hm{z - z_0 } < R},$ то её можно разложить в этом кольце в ряд Лорана.}

\begin{proof} Зафиксируем точку $z \in K$. И проведём 2
окружности (меньшую обозначим $\ga _1 $ (радиуса $\rho _1 )$;
большую$\ga _2 $ (радиуса $\rho _2 ))$. $r < \rho _1 < \hm{z - z_0 } <
\rho _2 < R$ по построению. Пусть точка $\xi \in \ga _2 $. Рассмотрим
ядро Коши: $\frac{1}{\xi - z} = \frac{1}{\hr{\xi - z_0 } - \hr{z - z_0 }} =
\frac{1}{\xi - z_0 }\frac{1}{1 - \frac{z - z_0 }{\xi - z_0 }} =
\hr{{\begin{array}{*{20}c}
 {сумма} \hfill \\
 {геом.прогр.} \hfill \\
\end{array} }} = \sum\limits_{n = 0}^\bes {\frac{\hr{z - z_0 }^n}{\hr{\xi
- z_0 }^{n + 1}}} $ Оценим равномерную сходимость: $\hm{\frac{z - z_0 }{\xi
- z_0 }} = \frac{\hm{z - z_0 }}{\rho _2 } < 1$ оценка не зависит от $\xi $ и
меньше 1, следовательно

\noindent
ряд сходится равномерно по $\xi \in \ga _2 $ (по признаку сравнения),
следовательно его можно

\noindent
интегрировать почленно:


\[
\frac{1}{2\pi i}\intl{\ga _2 }{}{\frac{f\hr{\xi }}{\hr{\xi - z}}d\xi =
\sum\limits_{n = 0}^\bes {\hr{z - z_0 }^n \cdot } } \frac{1}{2\pi i}\ub{
{\intl{\ga _2 }{}{\frac{f\hr{\xi }d\xi }{\hr{\xi - z_0 }^{n + 1}}}
}}_{a_n }\hr{1}.
\]



Пусть $\xi \in \ga _1 $. Рассмотрим ядро Коши: $\frac{1}{\xi - z} =
\frac{1}{\hr{\xi - z_0 } - \hr{z - z_0 }} = \frac{ - 1}{z - z_0 }\frac{1}{1
- \frac{\xi - z_0 }{z - z_0 }} = \hr{{\begin{array}{*{20}c}
 {сумма} \hfill \\
 {геом.прогр.} \hfill \\
\end{array} }} = - \sum\limits_{n = 1}^\bes {\frac{\hr{\xi - z_0 }^{n -
1}}{\hr{z - z_0 }^n}.} $ Оценим члены этого ряда: $\hm{\frac{\xi - z_0 }{z -
z_0 }} = \frac{\rho _1 }{\hm{z - z_0 }} < 1 \Ra $ аналогично ряд
сходится равномерно по $\xi \in \ga _1 \Ra $ интегрируем
почленно: $\frac{1}{2\pi i}\intl{\ga _1 }{}{\frac{f\hr{\xi }}{\hr{\xi -
z}}d\xi = - \sum\limits_{n = 1}^\bes {\frac{1}{\hr{z - z_0 }^n} \cdot } }
\frac{1}{2\pi i}\ub{ {\intl{\ga _2 }{}{\frac{f\hr{\xi }d\xi }{\hr{\xi -
z_0 }^{ - n + 1}}} }}_{a_{ - n} }\hr{2}$. Из (1) вычтем (2): $\ub{
{\frac{1}{2\pi i}\intl{\ga _2 - \ga _1 }{}{\frac{f\hr{\xi }}{\hr{\xi -
z}}d\xi } }}_{f(z)} = \sum\limits_{n = - \bes }^\bes {a_n \hr{z - z_0
}^n} $ по формуле Коши. \end{proof}



ИЗОЛИРОВАННЫЕ ОСОБЫЕ ТОЧКИ





\textbf{1}$^{0}$\textbf{. Классификация изолированных особых точек.}





\begin{df} Если функция }$f(z)$ голоморфна в некоторой проколотой окрестности точки }$z_0 , то z_0 - $ \textbf{изолированная
особая точка}} однозначного характера функции }$f(z)$.




\begin{note} Существуют и другие типы особых точек.





\begin{ex} 1) $f(z) = \sqrt z ,z_0 = 0 - $ изолированная особая
точка многозначного характера, а именно: точка ветвления 2ого порядка.

2) $f(z) = \frac{1}{\sin \frac{1}{z}},z_0 = 0 - $ неизолированная особая
точка (у этой точки есть другие особые точки: $z_n = \frac{1}{\pi n}$ и они
сходятся к нулю при $z_n \ra 0.)$

Классификация изолированных особых точек:





\begin{df} $Пусть f(z)$ голоморфна в }$\mathop U\limits^ \circ \hr{z_0 }$ (проколотая окрестность точки }$z_0 $), тогда:}

$Если \exi $ конечный предел: }$\mathop {\lim }\limits_{z \ra z_0 } f(z) \in \Cbb, то z_0 $ называется }\textbf{устранимой
особой точкой}} функции }$f(z)$.
$Если \exi $ предел: }$\mathop {\lim }\limits_{z \ra z_0 } f(z) = \bes , то z_0 $ называется }\textbf{полюсом}} функции }$f(z)$.
$Если не \exi  : \mathop {\lim }\limits_{z \ra z_0 } f(z), то z_0 $ называется }\textbf{существенной
особой точкой}} функции }$f(z)$.
\begin{ex}


\begin{table}[htbp]
\begin{tabular}
{|p{168pt}|p{168pt}|p{168pt}|}
\hline
Особые точки&
$z_0 = 0$&
$z_0 = \bes $ \\
\hline
1) устранимая&
$\frac{\sin z}{z}$&
$\frac{1}{z}$ \\
\hline
2) полюс&
$\frac{1}{z}$&
$z$ \\
\hline
3) существенная&
$e^{\frac{1}{z}}$&
$e^z$ \\
\hline
\end{tabular}
\label{tab1}
\end{table}



\textbf{2}$^{0}$\textbf{. Ряд Лорана в окрестности изолированной особой
точки.}


\[
z_0 \in \Cbb,f(z) -
г}о}л}о}м}о}р}ф}н}а}в}
\quad
\mathop U\limits^ \circ \hr{z_0 } = \hc{z \in \Cbb\vert 0 < \hm{z - z_0 } < R}
\quad -
п}р}и}м}е}р}к}о}л}ь}ц}а}с}р}а}д}и}у}с}о}м}
\quad
r =
0.П}о}т}е}о}р}е}м}е}
\quad :
\]



$f(z) = \sum\limits_{n = - \bes }^\bes {\mathop {a_n \hr{z - z_0
}^n}\limits_{\hr{L}} = \sum\limits_{n = 0}^\bes {\mathop {a_n \hr{z - z_0
}^n}\limits_{\hr{P}} + \mathop {\sum\limits_{n = 1}^\bes {\frac{a_{ - n}
}{\hr{z - z_0 }^n}.} }\limits_{\hr{Q}} } } $(L) -- ряд Лорана в $\mathop
U\limits^ \circ \hr{z_0 }$.



\begin{df} Ряд (Р) называется }\textbf{правильной частью ряда Лорана}}. Ряд (Q) называется }\textbf{главной
частью ряда Лорана.}}


Рассмотрим $z_0 = \bes ,f(z) - $ голоморфна в $\mathop U\limits^ \circ
\hr{\bes } = \hc{z \in \Cbb\vert r < \hm{z} < \bes }$ - частный случай
кольца с радиусом $R = \bes $. По теореме $\exi :f(z) =
\sum\limits_{n = - \bes }^\bes {\mathop {a_n \hr{z}^n}\limits_{\hr{L}} =
\mathop {\sum\limits_{n = 1}^\bes {\frac{a_{ - n} }{\hr{z}^n} + }
}\limits_{\hr{P}} \sum\limits_{n = 0}^\bes {\mathop {a_n
\hr{z}^n.}\limits_{\hr{Q}} } } $(L) -- ряд Лорана в $\mathop U\limits^ \circ
\hr{\bes }$. (Р) -- правильная часть ряда Лорана (состоит из слагаемых,
ограниченных при $z \ra z_0 )$. (Q) -- главная часть ряда Лорана (состоит из
слагаемых, неограниченных при $z \ra z_0 )$.





\begin{ex} 1) $e^z = 1 + z + \frac{z^2}{2} + ..$. Указать главную,
правильную части в окрестности точки $z_0 = 0$. Тогда всё это -- правильная
часть. А главная часть $Q = 0$. Если рассматривать в окрестности точки $z_0
= \bes ,$ то $P = 1,$ а $Q = z + \frac{z^2}{2} + ..$. .





\begin{theorem} $Пусть z_0 \in \Cbb,$ (конечна) пусть }$f(z) - $ голоморфна в }$\mathop U\limits^ \circ \hr{z_0
}. Пусть Q\hr{z}$ - главная часть ряда Лорана в }$\mathop U\limits^ \circ \hr{z_0 }$. Тогда справедливы следующие утверждения:}

$1) z_0 $ устранимая особая точка функции }$f(z) \Leftrightarrow Q = 0$ все коэффициенты нулевые.}

$2) z_0 $ полюс функции }$f(z) \Leftrightarrow Q$ содержит конечное ненулевое число отличных от нуля слагаемых.}

$3) z_0 $ существенная особая точка функции }$f(z) \Leftrightarrow Q$ содержит много ненулевых слагаемых.}





\begin{proof} Докажем 1): $\hr{ \Leftarrow }$ Напишем ряд Лорана
в точке $z_0 $. $f(z) = p\hr{z} - $ степенной ряд, сходящийся в
окрестности точки $z_0 $ к функции $f(z)$ - голоморфной в этой
окрестности $\mathop U\limits^ \hr{z_0 }$ (в частности непрерывной). Тогда
$\exi :\mathop {\lim }\limits_{z \ra z_0 } f(z) \quad \end{proof} $

$\hr{ \Ra }Д}а}н}о}
\quad
z_0
у}с}т}р}а}н}и}м}а}я}о}с}о}б}а}я}т}о}ч}к}а}
\quad
 \Ra \exi \mathop {\lim }\limits_{z \ra z_0 } f(z) = a_0
.Д}о}о}п}р}е}д}е}л}и}м}ф}у}н}к}ц}и}ю}
\quad
f(z)в}т}о}ч}к}е}
\quad
z_0
п}о}н}е}п}р}е}р}ы}в}н}о}с}т}и}
\quad ,т}о}г}д}а} \quad
1.ф}у}н}к}ц}и}я}
\quad
f(z) \quad -
г}о}л}о}м}о}р}ф}н}а}в}
\quad
\mathop U\limits^ \circ \hr{z_0 } \quad
2.ф}у}н}к}ц}и}я}
\quad
f(z) \quad -
н}е}п}р}е}р}ы}в}н}а}в}
\quad
\mathop U\limits^ \hr{z_0 } \Ra
п}о}т}е}о}р}е}м}е}о}б}у}с}т}р}а}н}и}м}о}й}о}с}о}б}е}н}н}о}с}т}и}ф}у}н}к}ц}и}я}
\quad
f(z) \quad -
г}о}л}о}м}о}р}ф}н}а}в}
\quad
\mathop U\limits^ \hr{z_0 }$(если функция голоморфна в круге, то она
раскладывается в степенной ряд), то $f(z) = p\hr{z},$ а $Q = 0$ (из
единственности разложения в ряд Лорана). \end{proof}

Докажем 2): $\hr{ \Leftarrow }$ Дано $f(z) = \frac{a_{ - p} }{\hr{z - z_0
}^p} + \frac{a_{ - p + 1} }{\hr{z - z_0 }^{p - 1}} + ... = \frac{a_{ - p} +
a_{ - p + 1} \hr{z - z_0 } + ...}{\hr{z - z_0 }^p} = \frac{g\hr{z}}{\hr{z -
z_0 }^p} \ra \bes ,z \ra z_0 $. Мы считаем $a_{ - p} \ne 0,p = 1,2,3,..$. , и $g\hr{z} - $ голоморфна в точке $z_0 $ и $g\hr{z_0 } \ne 0$. \end{proof}


\[
\hr{ \Ra }
\quad
\exi п}р}е}д}е}л} \quad :
\quad
\mathop {\lim }\limits_{z \ra z_0 } f(z) = \bes
,р}а}с}с}м}о}т}р}и}м}ф}у}н}к}ц}и}ю}
\quad
g\hr{z} = \frac{1}{f(z)}.Т} \quad
.к}.п}р}е}д}е}л} \quad =
б}е}с}к}о}н}е}ч}н}о}с}т}и}
\quad
,т}о}в}н}е}к}о}т}о}р}о}й}о}к}р}е}с}т}н}о}с}т}и}
\quad
g\hr{z} \ne
0.Т}а}к}и}м}о}б}р}а}з}о}м}
\quad ,
\quad
g\hr{z} -
г}о}л}о}м}о}р}ф}н}а}в}н}е}к}о}т}о}р}о}й}п}р}о}к}о}л}о}т}о}й}о}к}р}е}с}т}н}о}с}т}и}т}о}ч}к}и}
\quad
z_0 .
\]



Чему равен: $\mathop {\lim }\limits_{z \ra z_0 } g\hr{z} = 0,$ таким образом
$z_0 $- устранимая особая точка. Доопределив её по непрерывности, получаем
функцию голоморфную в некоторой окрестности точки $z_0 $, разложим её в ряд
Тейлора, тогда $g\hr{z} = b_p \hr{z - z_0 }^p + ...,b_p \ne 0$.


\begin{df} $Номер $\textbf{р}} называется }\textbf{порядком нуля}} для функции }$g\hr{z}$.





Теперь: $g\hr{z} = \hr{z - z_0 }^p\hr{b_p + ...} = \hr{z - z_0
}^ph\hr{z},h\hr{z} - $ голоморфная в точке $z_0 $ и $h\hr{z_0 } \ne 0$. Вернёмся к функции $f(z) = \frac{1}{g\hr{z}} = \frac{1}{\hr{z - z_0
}^p}\ub{ {\frac{1}{h\hr{z}}}}_{\ast },$ * - голоморфна в некоторой
окрестности точки $z_0 \Ra $можем разложить в степенной ряд.
Получим: $f(z) = \frac{1}{\hr{z - z_0 }^p}\hc{c_{ - p} + c_{ - p + 1}
\hr{z - z_0 } + ...} = \frac{c_{ - p} }{\hr{z - z_0 }^p} + ...,$ а это ряд
Лорана, где Q содержит конечное число слагаемых. \end{proof}

Утверждение 3) -- следствие из 1) и 2) (исключением этих случаев).





\textbf{3}$^{0}$\textbf{. Поведение функции в окрестности устранимой особой
точки.}





\begin{theorem} Если: 1) }$f(z) - $ голоморфна в }$\mathop U\limits^ \circ \hr{z_0 }$.
$2) f(z) - $ ограничена в }$\mathop U\limits^ \circ \hr{z_0 }. То z_0 $ устранимая особая точка.}

\begin{proof} Напишем ряд Лорана: $f = P + Q$. Исследуем главную
часть: $\xi = \frac{1}{z - z_0 }$ - замена переменных, тогда $g\hr{\xi } =
Q\hr{z} = \sum\limits_{n = 1}^\bes {a_{ - n} \xi ^n - } $ сходится в С
(это ряд Лорана, т.к. 1/r -- радиус сходимости). Таким образом $g\hr{\xi } -
$ голоморфна в С. Далее $f(z) - $ ограничена в $\mathop U\limits^ \circ
\hr{z_0 }$ и $P - $ ограничена в $\mathop U\limits^ \circ \hr{z_0 }
\Ra Q$ ограничена в некоторой окрестности $\mathop U\limits^ \circ
\hr{z_0 } \Ra g - $ ограничена в $\mathop U\limits^ \circ \hr{z_0
}$, тогда g ограничена в С. Тогда по теореме Лиувилля $g = const = 0
\Ra $ все коэффициенты равны 0, тогда $Q = 0,$ т.е. это устранимая
особая точка. \end{proof}





\textbf{4}$^{0}$\textbf{. Поведение функции в окрестности существенной
особой точки.}





\textbf{Теорема (Сохоцкого).} $Если z_0 $ - существенная особая точка }$f(z), то f(z) при z \ra z_0 $ есть расширенная комплексная плоскость.}





\begin{proof} Т.е. надо доказать, что для $\fa A \in \ol \Cbb
\exi $ последовательность $z_n \ne z_0 , \quad z \ra z_0 $, т.ч. $f\hr{z_n }
\ra A$.
\noindent
а) Частный случай $A = \bes $. Предположим противное, тогда $f(z) - $
голоморфна и ограничена в $\mathop U\limits^ \circ \hr{z_0 }$, тогда по
теореме (3$^{0}) \quad z_0 $ устранимая особая точка. $ \Ra
$противоречие. \end{proof}

\noindent
б) Общий случай $\fa A \ne \bes $ 1) либо $\exi z_n \ne z_0 , \quad z
\ra z_0 $, т.ч. $f\hr{z_n } = A.$(по пункту а)).

2) либо $f \ne A$ в некоторой проколотой окрестности точки $z_0 $. $g\hr{z}
= \frac{1}{f(z)} - $ голоморфная в $\mathop U\limits^ \circ \hr{z_0 }$ и
$z_0 $ - существенная особая точка для g. f и g связаны ДЛП, тогда $\exi
\hc{z_n }$ т.ч. $g\hr{z_n } \ra \bes ,$ что равносильно тому, что
$f\hr{z_n } \ra A.\end{proof} $

\end{document}


\section{ВЫЧЕТЫ}

\textbf{1}$^{0}$\textbf{. Определение вычетов.}

\begin{df} $Пусть z_0 \in \Cbb,f(z) - $ голоморфна в проколотой }

окрестности точки }$z_0 . \quad \mathop U\limits^ \circ \hr{z_0 } = \hc{z_0 \in \Cbb,r < \hm{z - z_0 }
< R}$.
Обозначим }$\ga - $ окружность с центром в точке }$z_0 $. Радиусом }$\rho $ и обход -- против }

часовой стрелки, }$0 < \rho < R$.




\begin{df}
\textbf{Вычетом }}$функции f(z) \quad \mathop
{res}\limits_{z = z_0 } \hr{f(z)}$ в точке z = z}$_{0}$ называется интеграл: }$\frac{1}{2\pi i}\intl{\ga
}{}{f(z)dz} $.
\end{df}

\begin{note}
По интегральной теореме Коши интеграл не зависит от
выбора радиуса $\rho $ в указанных пределах.
\end{note}

\begin{stm}
Если $z_0 - $ устранимая особая точка $f(z),$ то
$\mathop {res}\limits_{z = z_0 } \hr{f(z)} = 0$.
\end{stm}
\begin{proof} 1) По теореме об устранимой особенности, функцию
$f(z)$ можно доопределить в точке z = z$_{0}$ как голоморфную. 2) По ИТК
интеграл = 0.
\end{proof}

\begin{ex} Найдём: $\mathop {res}\limits_{z = 0} \hr{\frac{1}{2^{n +
1}}} = \left\{ {{\begin{array}{*{20}c}
 {1,n = 0} \hfill \\
 {0,n \in {\rm Z}\backslash \{0\}} \hfill \\
\end{array} }} \right.$.

\begin{df} Пусть функция }$f(z)$ голоморфная в некоторой окрестности }$\bes - $ти, т.е. }$\mathop U\limits^ \circ
\hr{\bes } = \hc{z \in Z\vert r < \hm{z - z_0 } < \bes }$. Проведём окружность }$\ga $ с центром в точке 0 и радиусом }$\rho
$. Обход окружности по часовой стрелке.}

\begin{df} \textbf{Вычетом}} функции }$f(z)$ в точке }$z = \bes $ называется }

$\mathop {res}\limits_{z = \bes } \hr{f(z)} = \frac{1}{2\pi
i}\intl{\ga }{}{f(z)dz.}$

\begin{ex} $\mathop {res}\limits_{z = \bes } \hr{\frac{1}{z}} =
\frac{1}{2\pi i}\intl{\ga }{}{\frac{1}{z}dz = - 1.} $ т.к. по часовой
стрелке.

\begin{note} Функция $\frac{1}{z}$ имеет в точке $\bes $ устранимую
особую точку $\hr{\exi \lim  = 0},$ тем не менее $\mathop {res}\limits_{z
= \bes } \hr{\frac{1}{z}} \ne 0$.

\begin{note} $\mathop {res}\limits_{z = \bes } \hr{f(z)} \ne \pm
\mathop {res}\limits_{z = 0} \hr{f\hr{\frac{1}{z}}}$ -это так и не иначе,
типичная ошибка при замене переменных.

\textbf{Правильно:} $\mathop {res}\limits_{z = \bes } \hr{f(z)}\mathop
= \limits^{def} \frac{1}{2\pi i}\oints{\hm{z} = \rho }{f(z)} dz = [z =
\frac{1}{\xi }] = \frac{1}{2\pi i}\oints{\hm{\xi } = \frac{1}{\rho
}}{f\hr{\xi }} \hr{ - \frac{1}{\xi ^2}}d\xi \mathop = \limits^{def} -
\mathop {res}\limits_{\xi = 0} \hr{\frac{f\hr{\frac{1}{\xi }}}{\xi ^2}}$.

Вычет -- не функция, а дифференциальная форма !!!

\textbf{2}$^{0}$\textbf{. Вычисление вычетов.}

\textbf{Утверждение.} 1) Пусть }$z_0 \in \Cbb,f(z) - $ голоморфная в }$\mathop U\limits^ \circ \hr{z_0
}, \quad f(z) = \sum\limits_{n = - \bes }^\bes {a_n \hr{z - z_0 }} ^n$ -ряд Лорана, тогда }$\mathop
{res}\limits_{z = z_0 } \hr{f(z)} = a_{ - 1} $. 2) Пусть }$f(z) - $ голоморфная в }$\mathop
U\limits^ \circ \hr{\bes }, \quad f(z) = \sum\limits_{n = - \bes }^\bes
{a_n \hr{z}} ^n$ -ряд Лорана, тогда }$\mathop {res}\limits_{z = z_0 } \hr{f(z)} = - a_{ - 1}
$.

\begin{proof} Это одна из формул для коэффициентов ряда Лорана.

\begin{df} $Пусть z_0 - $ полюс функции }$f(z)$. Говорят, что }\textbf{полюс имеет порядок
р}}, если функция }$g\hr{z} = \frac{1}{f(z)}$ имеет в точке }$z_0 - $ нуль порядка р (после устранения особенности -- доопределим в нуле).}


\textbf{Утверждение.} $Пусть z_0 \in \Cbb. \quad Пусть z_0 - $ полюс 1ого порядка функции }$f(z). Тогда \mathop {res}\limits_{z
= z_0 } \hr{f(z)} = \mathop {\lim }\limits_{z \ra z_0 } \hr{z - z_0
}f(z)$.

\begin{proof} $f(z) = \frac{a_{ - 1} }{z - z_0 } + a_0 + ..$. Рассмотрим $g\hr{z} = \hr{z - z_0 }f(z) = a_{ - 1} + a_0 \hr{z - z_0 } +
... - $ степенной ряд, сходящийся в окрестности точки $z_0 $ к функции
голоморфной и непрерывной $ \Ra \mathop {\lim }\limits_{z \ra z_0 }
g\hr{z} = g\hr{z_0 } = a_{ - 1} . \quad \end{proof} $

\begin{imp}.} $Если f(z) = \frac{\phi \hr{z}}{\psi \hr{z}}, где \phi \hr{z}
- $ голоморфная в точке }$z_0 ,\phi \hr{z_0 } \ne 0,\psi \hr{z} - $ голоморфная в точке }$z_0 ,\psi \hr{z_0 } =
0,\psi '\hr{z_0 } \ne 0. То \mathop {res}\limits_{z = z_0 } \hr{f(z)} =
\frac{\phi \hr{z_0 }}{\psi '\hr{z_0 }}$.

\begin{proof} Заметим, что $z_0 - $ полюс 1ого порядка. Можно
применить доказанную формулу: $\mathop {res}\limits_{z = z_0 } \hr{f(z)}
= \mathop {\lim }\limits_{z \ra z_0 } \hr{z - z_0 }f(z) = \mathop {\lim
}\limits_{z \ra z_0 } \frac{\phi \hr{z}}{\frac{\psi \hr{z} - \psi \hr{z_0
}}{z - z_0 }}$ т.к. $\psi \hr{z_0 } = 0,\mathop {res}\limits_{z = z_0 }
\hr{f(z)} = \mathop {\lim }\limits_{z \ra z_0 } \frac{\phi \hr{z_0
}}{\frac{\psi \hr{z} - \psi \hr{z_0 }}{z - z_0 }} = \frac{\phi \hr{z_0
}}{\psi '\hr{z_0 }}$ по определению производной. \end{proof}

\begin{ex} Найдём вычеты функции $ctg\hr{z}$ во всех её особых точках.
$ctg\hr{z} = \frac{\cos \hr{z}}{\sin \hr{z}},\phi \hr{z} = \cos z,\psi
\hr{z} = \sin z$. Особые точки -- нули знаменателя, т.е. $z = \pi n,n \in Z
- $ полюс первого порядка. Тогда $\mathop {res}\limits_{z = \pi n}
\hr{ctg\hr{z}} = \frac{\cos \hr{\pi n}}{\cos \hr{\pi n}} = 1$. все вычеты
равны 1. $z = \bes $ неизолированная особая точка, предельная точка
полюсов, для неё вычет определять не стоит (он не определён для неё).

\textbf{Утверждение.} $Пусть z_0 \in \Cbb. \quad Пусть z_0 - $ полюс порядка р + 1 функции }$f(z). Тогда \mathop {res}\limits_{z
= z_0 } \hr{f(z)} = \mathop {\lim }\limits_{z \ra z_0 }
\frac{1}{p!}\hr{\frac{d}{dz}}^p\hr{\hr{z - z_0 }^{p + 1}f(z)}$.

\begin{proof} По определению res: $\mathop {res}\limits_{z = z_0
} \hr{f(z)} = \frac{1}{2\pi i}\intl{\ga }{}{f(z)dz = }
\frac{1}{2\pi i}\intl{\ga }{}{\frac{\hr{z - z_0 }^{p + 1}f(z)}{\hr{z -
z_0 }^{p + 1}}dz} $ Функция, стоящая в числителе имеет устранимую
особенность в точке $z_0 .$Устраним её, тогда интеграл -- есть формула Коши
для р-ой производной, утверждение доказано. \end{proof}

\textbf{3}$^{0}$\textbf{. Основная теорема Коши о вычетах.}

\begin{theorem} Если 1) D -- правильная область с положительно ориентированной границей}

$2) f(z) - $ голоморфная в}$\ol{D}\backslash \hc{a_1 ,...,a_p }, где a_1 ,...,a_p $ - конечное число особых точек }$ \in D. То \frac{1}{2\pi
i}\intl{\ga }{}{f(z)dz = } \sum\limits_{j = 1}^p {\mathop
{res}\limits_{z = a_j } \hr{f(z)}} $.

\begin{proof} Изобразим правильную область. Обозначим: $K_j - $
круг с центром в точке $a_j $ и радиусом $\ep . \quad \Ga _j =
\pd K_j $ граница $K_j $ (обходится против часовой стрелки).
$\ep $ возьмём достаточно малым, чтобы $\ol{K}_j \in D$ и не
пересекаются попарно. $D_\ep = D\backslash \cupl{j = 1}{p}{\ol{K}_j } - $ правильная область с положительно ориентированной границей
$\Ga _\ep = \pd D_\ep = \Ga - \hr{\ga _1 + ...
+ \ga _p }$. По ИТК для многосвязной области: $\intl{\Ga _e
}{}{f(z)dz = 0} $ или $\frac{1}{2\pi i}\intl{\Ga }{}{f(z)dz =
\sum\limits_{j = 1}^p {\frac{1}{2\pi i}\intl{\ga _j }{}{f(z)dz} } }
\mathop = \limits^{def} \mathop {res}\limits_{z = a_j } \hr{f\hr{s}} = 0$. \end{proof}

\begin{imp}
Если функция }$f(z) - $ голоморфная в
$\ol{\Cbb}\backslash \hc{a_0 = \bes,a_1 \sco a_p }, где a_1 \sco a_p $ - конечное число особых точек, то
}$\sum\limits_{j = 1}^p {\mathop {res}\limits_{z =
a_j } \hr{f\hr{s}}} = 0$.

\begin{proof} Обозначим Г -- окружность. Радиус R возьмём
достаточно большим, чтобы $a_1 ,...,a_p $ лежали внутри круга. Вычислим
$\frac{1}{2\pi i}\intl{\Ga }{}{f(z)dz = \left\{
{{\begin{array}{*{20}c}
 {\sum\limits_{j = 1}^p {\mathop {res}\limits_{z = a_j } \hr{f\hr{s}}} }
\hfill \\
 {\mathop { - res}\limits_{z = a_j } \hr{f\hr{s}}} \hfill \\
\end{array} }} \right.} $ по ${\begin{array}{*{20}c}
 {основной} \hfill \\
 {определению.} \hfill \\
\end{array} } \quad {\begin{array}{*{20}c}
 {теореме} \hfill \\
 \hfill \\
\end{array} } \quad {\begin{array}{*{20}c}
 о \hfill \\
 \hfill \\
\end{array} } \quad {\begin{array}{*{20}c}
 {вычетах.} \hfill \\
 \hfill \\
\end{array} } \quad \end{proof} $





\begin{note}
Основная теорема Коши о вычетах верна и для неограниченных областей.
\end{note}

\subsection{Применение вычетов к вычислению интегралов}

\begin{ex}
$J = \frac{1}{2\pi i}\oints{\hm{z} = 1}{\frac{dz}{\sin\frac{1}{z}}} $.

Первый способ получения J.

$f(z) = \frac{1}{\sin \frac{1}{z}} -
голоморфная в
\quad
\ol{\Cbb}\backslash \hc{\frac{1}{\pi n},n \in Z\backslash \{0\}} -
простыенулизнаменателя
\quad ,т} \quad
.е.полюсы \quad
1огопорядка
\quad .
\quad
z = 0 -
неизолированная особая точка
\quad ,
\quad
z = \bes - полюс \quad
1 огопорядка
\quad
.Вонешностикруга
\quad
\hm{z} =
1толькооднаособаяточка
\quad
z = \bes
Применим основную теорему Коши о вычетах к внешности круга
\quad ,тогда
\quad
J = \mathop { - res}\limits_{z = \bes } \hr{f\hr{s}}$(направление обхода
против часовой стрелки).

Напишем ряд Лорана: $f(z) = \frac{1}{\frac{1}{z} -
\frac{1}{6}\frac{1}{z^3} + ...} = \frac{z}{1 - \frac{1}{6}\frac{1}{z^2} +
...} = z\hr{1 + \frac{1}{6}\frac{1}{z^2} + ...} = \ub{ {\hr{z}}}_{\hr{Q}} +
\ub{ {\hr{\frac{1}{6}\frac{1}{z} + ...}}}_{\hr{P}}$. Нас интересует только
коэффициент перед $\frac{1}{z}$: $J = \mathop {res}\limits_{z = \bes }
\hr{f(z)} = - \hr{ - \frac{1}{6}} = \frac{1}{6}$. Вот он ответ.
\end{proof}

\textbf{2}$^{о}й}}$\textbf{ способ получения J.}

Обозначим $\ga _N - $ окружность с центром в точке 0 и радиусом
$\frac{1}{\pi N + \frac{\pi }{2}}$ и применим теорему о вычетах к кольцу.


\[
 \Ra
\quad
\frac{1}{2\pi i}\intl{\Ga - \ga _N }{}{f(z)dz = \sum\limits_{n =
\pm 1,...,\pm N}^ {\mathop {res}\limits_{z = z_n } \hr{f\hr{s}}} } .R_n =
\mathop {res}\limits_{z = z_n } \hr{f\hr{s}}.
\]

Перейдём к пределу $\mathop {\lim }\limits_{N \ra \bes } \Ra $
справа бесконечная сумма, а слева

\noindent
останется: $\frac{1}{2\pi i}\intl{\Ga }{}{f(z)dz} $. Для этого надо
доказать, что $\intl{\ga _N }{}{f(z)dz} \ra 0$. Это следует из того,
что функция $\hm{f} \le M - $ равномерно ограничена на всех $\ga _N $ (N
= 1, 2, \ldots ). Доказать это в качестве упражнения!!!


\[
R_n = \mathop {res}\limits_{z = z_n } \hr{f\hr{s}} = \frac{1}{\cos
\hr{\frac{1}{z_n }}\hr{ - \frac{1}{z_n ^2}}} = - \hr{\frac{1}{\pi n}}^2\hr{
- 1}^n, \Ra J = \sum {R_n } = 2\sum\limits_{n = 1}^\bes {\hr{ -
1}^{n + 1}\hr{\frac{1}{\pi n}}^2 = \frac{1}{6}}
.О}т}с}ю}д}а} \quad :
\quad
\sum\limits_{n = 1}^\bes {\hr{ - 1}^{n + 1}\frac{1}{n^2} = \frac{\pi
^2}{12}.}
\]

\subsection{Лемма Жордана}

\begin{lemma}
Пусть $f(z)$ - голоморфная в}$\hc{Imz \ge 0}\backslash \{a_1 ,..,a_p \}, где a_1
,...,a_p $ - конечное число особых точек принадлежащих }$\hc{Imz > 0}$, Обозначим: }$M\hr{R} = \max \hc{\hm{f(z)}\vert \hm{z} = R,Imz
> 0}$. Предположим, что }$\exi R_n \ra \bes $ (последовательность чисел), т.ч. }$M\hr{R_n } \ra 0. Пусть \la > 0$. Тогда утверждается, что интеграл Фурье: }$\intl{ -
\bes }{\bes }{f\hr{x}e^{i\la x}dx = 2\pi i\sum\limits_{j = 1}^p
{\mathop {res}\limits_{z = a_j } \hr{f(z)e^{i\la z}}} } ,$ если интеграл слева существует в смысле главного значения по Коши.}
\end{lemma}
\begin{proof}
Обозначим интеграл за J. А правую часть равенства
за А.

Обозначим $C_R - $ замкнутый контур. Пусть R -- достаточно большое, т.ч.

Все $a_1 ,..,a_p $ лежат внутри замкнутого контура. Тогда по основной
теореме Коши о вычетах: $\intl{C_R }{}{f(z)e^{i\la z}dz = A} =
\mathop {\intl{ - R}{R} }\limits_{ \ra J} + \intl{\ga _R }{} $ при $R =
R_n \ra \bes $. Если докажем, что $\intl{\ga _R }{} \ra 0,$ тогда лемма
доказана. Оценим:$\hm{\intl{\ga _R }{} } \le M\hr{R}\intl{\ga _R
}{}{Re^{ - \la R\sin \phi }d\phi \le 2RM\hr{R}\intl{0}{\frac{\pi
}{2}}{e^{ - \la R\frac{2\phi }{\pi }}d\phi = \left. {2RM\hr{R}\frac{e^{
- \la R\frac{2\phi }{\pi }}}{ - \la R\frac{2}{\pi }}}
\right|_0^{\frac{\pi }{2}} = \frac{\pi }{\la }M\hr{R}\hr{1 - e^{ -
\la R}} \le \frac{\pi }{\la }M\hr{R} \ra 0} } $ При $R = R_n \ra
\bes $. Важно, что $\la > 0$ иначе не верно. \end{proof}

Лемма верна для той полуплоскости, где exp убывает.

\begin{ex} $S\hr{\la } = \intl{0}{\bes }{\frac{x\sin \la
x}{x^2 + 1}dx.} $


\textbf{Решение.} $S\hr{\la } = \frac{1}{2}ImE\hr{\la },E\hr{\la
} = \intl{ - \bes }{\bes }{\frac{x}{x^2 + 1}e^{i\la x}dx.} $ Возьмём
$\la > 0$ и $f(z) = \frac{z}{z^2 + 1},z = \pm i$ - полюс 1ого
порядка. По лемме Жордана пишем ответ: $E\hr{\la } = 2\pi i \cdot
\mathop {res}\limits_{z = i} \hc{\frac{z}{z^2 + 1}e^{i\la z}} = \left.
{2\pi i\frac{ze^{i\la z}}{2z}} \right|_{z = i} = \pi ie^{ - \la }
\Ra S\hr{\la } = \frac{\pi }{2}e^{ - \la }$. Верно для
$\la > 0$.

Для $\la = 0 \quad  \Ra S\hr{\la } = 0$.
Для $\la < 0 \Ra S\hr{\la } = - \frac{\pi }{2}e^\la $ в
силу нечётности. Тогда $S\hr{\la } = \frac{\pi }{2}sign\hr{\la }e^{
- \hm{\la }}$. График:





\textbf{Задача.} Рассмотрим: $S\hr{\la } = \intl{0}{\bes }{\frac{\sin
\hr{\la x}}{x^2 + 1}dx} $. Исследовать поведение $S\hr{\la } =
\frac{\pi }{2}sign\hr{\la }e^{ - \hm{\la }}$. при $\la \ra 0$.
(можно дифференцировать по параметру $\la $. Обоснование!!!).

\section{ПРЕОБРАЗОВАНИЕ ЛАПЛАСА}


\textbf{1}$^{0}$\textbf{. Определение функции--оригинала и её изображения.}

\begin{df}
\textbf{Функцией--оригиналом }}называется комплексно-значная функция вещественного переменного}$f:R \ra \Cbb$, т.ч. 1) }$f\hr{t}
= 0,t < 0$.
$2) \fa R > 0$ функция }$f$ имеет конечное число точек разрыва 1го рода на отрезке [0, R].}

$3) \exi a \in R,\exi M \in \Cbb > 0:\fa t \in R:\hm{f\hr{t}} \le
Me^{at}$.
\end{df}

\begin{df}
\textbf{Показателем роста}} \textbf{функции
}}$f$ называется: }$a\hr{f} = \inf \hc{a\vert \hm{f\hr{t}} \le Me^{at},M = M\hr{a}}$
\end{df}

\begin{ex}
1) $f\hr{t} = e^{t^2},t \ge 0 - $ функция не является
функцией--оригиналом.

2) $f\hr{t} = e^{at},t \ge 0 - $ функция будет функцией--оригиналом с
показателем роста $a\hr{f} = Re\hr{a}$.
\end{ex}

\begin{df}
Пусть $f$ функция-оригинал. Её }\textbf{изображением (или преобразованием Лапласа)}} называется следующий
несобственный интеграл, зависящий от параметра: }$F\hr{p} = \intl{0}{\bes }{e^{ - pt}f\hr{t}dt.} $


\subsubsection{Необходимые условия изображения}

\begin{theorem} $Пусть f\hr{t}$ функция--оригинал с показателем роста }$a\hr{f} = a. Пусть F\hr{p}$ - её изображение. Тогда справедливы следующие утверждения: }

$1) F\hr{p}$ сходится равномерно внутри полуплоскости }$Re\hr{p} > a$.
$2) F\hr{p}$ голоморфна в этой полуплоскости }$Re\hr{p} > a$.
$3) F\hr{p} \ra 0, если Re\hr{p} \ra \bes $.




\begin{proof} 1)$p = x + iy$. Возьмём числа $a < a_0 < a_1 \le
x,$ где а -- задано, а $a_0 ,a_1 $ - взяли, тогда $\hm{f\hr{t}} \le Me^{a_0
t}$ - по определению показателя роста, $\hm{e^{ - pt}f\hr{t}} \le Me^{a_0
t}e^{ - xt} \le Me^{ - t\hr{a_1 - a_0 }} - $ функция интегрируется на
интервале (0, $\bes ) \quad  \Ra  \quad F\hr{p}$ сходится равномерно для
всех $x \ge a_1 $ (по признаку сравнения).

2) Следует из 1ой теоремы Вейерштрасса (несобственный интеграл -- предел
некоторой последовательности).

3) Оценим: $\hm{F\hr{p}} \le \intl{0}{\bes }{Me^{ - t\hr{x - a_0 }}dt} =
\frac{M}{x - a_0 } \ra 0,x \ra \bes  \quad \end{proof} $





\begin{ex} Пусть $f\hr{t} = Q\hr{t} = \left\{ {{\begin{array}{*{20}c}
 {1,t \ge 0} \hfill \\
 {0,t < 0} \hfill \\
\end{array} }} \right. - $ единичная функция Хевисайда. Это функция-оригинал
с $a\hr{f} = 0$. Найдём изображение$F\hr{p} = \intl{0}{\bes }{e^{ - pt}dt}
= \frac{1}{p}$. Все утверждения теоремы для этой функции выполнены.





\textbf{Задача.} Вывести формулы для изображения: 1) $f'\hr{t};$ 2)
$\intl{0}{t}{f\hr{\tau }d\tau ;} $ 3) $e^{\al t}f\hr{t};$ 4) $f\hr{t -
t_0 } - сдвиг;$

5) свёртка$\hr{f\ast g}\hr{t} = \intl{0}{t}{f\hr{\tau }g\hr{t - \tau }d\tau
;} $





\textbf{3}$^{0}$\textbf{. Достаточные условия изображения.}





\begin{theorem} $Если 1) F\hr{p}$ голоморфна в полуплоскости }$Re\hr{p} > a$.
$2) F\hr{p} \ra 0, если p \ra \bes $ в полуплоскости }$Re\hr{p} > a$.
$3) \fa x > a:\intl{x - i\bes }{x + i\bes }{\hm{F\hr{p}}\hm{dp} <
\bes } $ - интегрировать на любой вертикальной прямой.}

То: а) функция }$F\hr{p}$ является изображением некоторой функции-оригинала }$f\hr{t} с a\hr{f} \le a$.
б) справедлива формула обращения Меллина: }$f\hr{t} = \frac{1}{2\pi i}\intl{x - i\bes }{x + i\bes
}{e^{pt}F\hr{p}dp.} $





\begin{proof} 1)$p = x + iy$. Определим функцию $f\hr{t}$
интегралом Меллина при некотором $x > a$. Проверим, что интеграл сходится:
$\hm{e^{pt}F\hr{p}} = e^{xt}\hm{F\hr{p}} \Ra $ интеграл сходится по
признаку сравнения. Более того, он сходится равномерно по t на компактах
вещественной оси: $t \in K \subs \subs R \Ra $ функция $f\hr{t}$
- непрерывна.

2) Если $t < 0,$ то $f\hr{t} = 0$ по лемме Жордана, применённой к
полуплоскости $Rep \ge x$. ($e^{pt} - $ убывает при $t < 0)$.

3) Заметим, что функция $f\hr{t}$ не зависит от $x > a$ по ИТК.

Интеграл по контуру прямоугольника = 0.

Интегралы по основаниям $ \ra 0,$ т.к. функция стремится к 0.

Значит интегралы по вертикальным прямым равны 0.

4) Оценим: $\hm{f\hr{t}} = \frac{1}{2\pi }\intl{x - i\bes }{x + i\bes
}{e^{xt}\hm{F\hr{p}}\hm{dp}.} $ В этом интеграле -- всё есть некоторые
константы, кроме $e^{xt}$ (т.к. сходится к const). $a\hr{f} \le x
\Ra $ (можем так подобрать) $a\hr{f} \le a$.

5) Проверим, что $F\hr{t}$ является изображением этой функции: Зафиксируем
р$_{0}$: $Re\hr{p_0 } > a$ и будем вычислять интеграл Лапласа:
$\intl{0}{\bes }{e^{ - p_0 t}f\hr{t}dt = } \intl{0}{\bes }{e^{ - p_0
t}\frac{1}{2\pi i}\intl{x - i\bes }{x + i\bes }{e^{pt}F\hr{p}dp} dt = }
= $ В этой формуле х будем брать любым, большим а. Возьмём $a < x < Rep_0
\Ra $ абсолютно сходится двойной интеграл $ \Ra $ можем
изменить порядок интегрирования (по теореме Фубини), тогда $ = =
\frac{1}{2\pi i}\intl{x - i\bes }{x + i\bes }{F\hr{p}dp} \intl{0}{\bes
}{e^{ - t\hr{p_0 - p}}dt} = \frac{1}{2\pi i}\intl{x - i\bes }{x + i\bes
}{\frac{F\hr{p}}{p_0 - p}dp} = F\hr{p_0 }$ по теореме Коши о вычетах.
\end{proof}

Пояснения к последнему равенству (как применять теорему Коши): Интеграл по
замкнутому

Контуру равен $F\hr{p_0 }$, но когда распрямляем и $R \ra \bes $, то
интеграл $ \ra 0,$ по дуге окружности.

\subsection{Преобразования Бореля}

\begin{theorem}
Если функция $F\hr{p}$ голоморфна }$u = 0$ в бесконечности, то она является изображением некоторой функции.
\end{theorem}
\begin{proof}
Напишем ряд Лорана: $F\hr{p} = \suml{n = 0}{\bes} {\frac{C_n }{p^{n + 1}}} $ (только правильная часть без const,
т.к. $F\hr{\bes } = 0)$ обозначим $R_0 - $ радиус сходимости, т.е. ряд
сходится при $\hm{p} > R_0 $, где $R_0 $ вычисляется по формуле
Коши-Адамара: $R_0 = \mathop {\ol {\lim } }\limits_{n \ra \bes }
\sqrt[n]{\hm{C_n }}, \Ra \fa R > R_0 $ выполняются неравенства:
$\hm{C_n } \le MR^n,$ где М зависит от R (по неравенству Коши). Напишем
степенной ряд: $f\hr{t} = \sum\limits_{n = 0}^\bes {\frac{C_n }{n!}t^n,} $
его $R_{cx} = \bes ,$ т.к. в числителе степенная функция, а в знаменателе
факториал $ \Ra $ ряд сходится при любом $t \Ra f\hr{t}$
голоморфна в С. Оценим: $\hm{f\hr{t}} = \sum\limits_{n = 0}^\bes
{\frac{MR^n\hm{t}^n}{n!} = Me^{R\hm{t}} \Ra } $ функция
экспоненциального роста и $a\hr{f} \le R \Ra a\hr{f} \le R_0 $.
Теперь рассмотрим: $f\hr{t} \cdot Q\hr{t}$ и докажем, что её изображением
будет функция $F\hr{t}$. Зафиксируем р: $Rep > R_0 $. И вычислим:
$\intl{0}{\bes }{e^{ - pt}f\hr{t}dt = } \intl{0}{\bes }{e^{ -
pt}\sum\limits_{n = 0}^\bes {\frac{C_n }{n!}t^n} dt = } \sum\limits_{n =
0}^\bes {\frac{C_n }{n!}} \ub{ {\intl{0}{\bes }{e^{ - pt}}
t^ndt}}_{\frac{n!}{p^{n + 1}}} = \sum\limits_{n = 0}^\bes {\frac{C_n
}{p^{n + 1}} = F\hr{p}} $ Почему ряд можем интегрировать поточечно? По
теореме Б. Леви о монотонной сходимости и по теореме Лебега о мажорантной
сходимости. $\sum\limits_{n = 0}^\bes {\frac{C_n }{n!}} $





\textbf{Определение (Соответствие рядов).} $Ряду \sum\limits_{n = 0}^\bes
{\frac{C_n }{p^{n + 1}}} $ ставится в соответствие ряд }$\sum\limits_{n = 0}^\bes {\frac{C_n }{n!}t^n}
$ называющийся }\textbf{преобразованием Бореля}} (А смысл соответствия -- оно обратное к преобразованию Лапласа).}





\textbf{Утверждение.} $f\hr{t} = \frac{1}{2\pi i}\oints{\hm{p} = R > R_0
}{e^{pt}F\hr{p}dp} $ (по окружности) -- эта формула справедлива для преобразования Бореля.}





\begin{proof} $\frac{1}{2\pi i}\intl{\hm{p} =
R}{}{e^{pt}\sum\limits_{n = 0}^\bes {\frac{C_n }{p^{n + 1}}} dp = }
\sum\limits_{n = 0}^\bes {C_n } \ub{ {\frac{1}{2\pi i}\intl{\hm{p} =
R}{}{\frac{e^{pt}}{p^{n + 1}}dp} }}_{ = \frac{t^n}{n!}\hr{\ast }} =
\sum\limits_{n = 0}^\bes {\frac{C_n }{n!}t^n} $. (*) -- n коэффициент ряда
Тейлора. \end{proof}





ГАММА-ФУНКЦИЯ И ДЗЕТА-ФУНКЦИЯ





\subsection{Гамма функция}

\begin{df}
Гамма функция (или интеграл Эйлера 2ого рода) -- это следующий несобственный интеграл, зависящий от параметра:
$\Ga \hr{s} = \intl{0}{\bes }{e^{ - t}t^{s - 1}dt.}$
\end{df}

\begin{stm}
1)Интеграл }$\Ga \hr{s}$ сходится равномерно внутри правой полуплоскости }$Re\hr{s} > 0$.
2)Функция }$\Ga \hr{s}$ голоморфна в }$Re\hr{s} > 0$.
\end{stm}
\begin{proof}
1) Запишем: $s = \sigma + i\tau $. Возьмём числа:
$0 < \ep \le \sigma \le E < \bes $. Пусть $t \in (0;1]$. Тогда
$\hm{e^{ - t}t^{s - 1}} \le t^{\sigma - 1} \le t^{\ep - 1} - $
функция интегрируема на интервале (0,1) $ \Ra \intl{0}{1}{e^{ -
t}t^{s - 1}dt} $ сходится равномерно для $\sigma \ge \ep $ Пусть $t
\in \hr{1;\bes }.\hm{e^{ - t}t^{s - 1}} \le e^{ - t}t^{\sigma - 1} \le e^{
- t}t^{E - 1} - $ интеграл $\intl{1}{\bes }{e^{ - t}t^{s - 1}dt} $
сходится равномерно на $\hr{1,\bes }$ при $\sigma \le E$.\end{proof}

2) Следует из 1ой теоремы Вейерштрасса.
\end{proof}

\begin{lemma}
Справедлива формула приведения:
$\Ga \hr{s + 1} = s\Ga \hr{s};Res > 0$.
\end{lemma}
\begin{proof}
$\Ga \hr{s + 1}\mathop = \limits^{def}
\intl{0}{\bes }{e^{ - t}t^sdt} = \ub{ { - e^{ - t}t^s\vert _0^\bes
}}_{\hr{\ast }} - \intl{0}{\bes }{\hr{ - e^{ - t}}st^{s - 1}dt =
s\intl{0}{\bes }{e^{ - t}t^{s - 1}dt\mathop = \limits^{def} s\Ga
\hr{s}} } $. (*) = 0 т.к. на бесконечности всех убивает exp, а в 0 в силу
условия $Res > 0$. \end{proof}

\begin{imp}
$\Ga \hr{n + 1} = n!,n \in \Z_+ $.
\end{imp}
\begin{proof}
По индукции из леммы.
\end{proof}

\begin{theorem}
Гамма функция аналитически продолжается на всю комплексную плоскость за исключением точек }$s = - n,n = 0,1,2,..$. которые
являются её простыми полюсами. При этом вычет функции }$\Ga \hr{s} - \mathop
{res}\limits_{s = - n} \hr{\Ga \hr{s}} = \frac{\hr{ - 1}^n}{n!},n \in Z_
+ $.
\begin{proof} $fix\fa n \in Z_ + $. Из леммы по индукции
$\Ga \hr{s + n + 1} = \hr{s + n}...\hr{s + 1}s\Ga \hr{s},Res > 0$. Перепишем эту формулу: $\Ga \hr{s + n + 1} = \hr{s + n}...\hr{s +
1}s\Ga \hr{s},Res > 0$. А где эта функция голоморфна на самом деле? В
$\hc{Res > - n - 1}\backslash \hc{s = 0, - 1, - 2,..., - n}$ - полюсы
первого порядка. Аналитическое продолжение доказано. Осталось только вычеты
посчитать: $\mathop {res}\limits_{s = - n} \hr{\Ga \hr{s}} = \mathop
{\lim }\limits_{s \ra - n} \hr{s + n}\Ga \hr{s} = \left. {\frac{\Ga
\hr{s + n + 1}}{\hr{s + n - 1}...s}} \right|_{s = - n} = \frac{\Ga
\hr{1}}{\hr{ - 1}...\hr{ - n}} = \frac{\hr{ - 1}^n}{n!}. \quad \end{proof} $

\begin{imp}
Мах и Мin всё ближе и ближе приближаются к оси Ох со стремлением }$s \ra - \bes $. \end{imp}

\begin{problem}
Доказать формулу дополнения: $\Ga \hr{s}\Ga \hr{s - 1} = \frac{\pi }{\sin \hr{\pi s}},s \in \Cbb\backslash Z$.

\subsection{Дзета функция}

\begin{df}
Дзета функция Римана (Эйлера) сумма ряда:
$\zeta \hr{s} = \sum\limits_{n = 1}^\bes {\frac{1}{n^s}} $. \end{df}

\begin{stm}
1) Ряд $\zeta(s)$ сходится равномерно внутри полуплоскости $Res > 1$. 2) Функция }$\zeta \hr{s}$ голоморфна в этой полуплоскости.
\end{stm}
\begin{proof} 1) Запишем $s = \sigma + i\pi $. Напишем: $1 < q
\le \sigma < \bes $. Теперь: $\hm{\frac{1}{n^s}} = \frac{1}{n^\sigma } \le
\frac{1}{n^q}$ сходится т.к. q > 1. Тогда по признаку сравнения получаем
наше утверждение: $\zeta \hr{s}$ сходится равномерно для $\sigma \ge q$.

2) Следует из 1ой теоремы Вейерштрасса.
\end{proof}

\begin{lemma}
Справедлива формула: }$\zeta \hr{s} = \frac{1}{\Ga \hr{s}}\intl{0}{\bes
}{\frac{t^{s - 1}}{e^t - 1}dt} $ при условии, что }$Res > 1$. \end{lemma}
\begin{proof}
$\intl{0}{\bes }{e^{ - nt}t^{s - 1}dt} =
\frac{\Ga \hr{s}}{n^s}$ - по определению гамма функции, поэтому: $\zeta
\hr{s}\Ga \hr{s} = \sum\limits_{n = 1}^\bes {\intl{0}{\bes }{e^{ -
nt}t^{s - 1}dt} = } \intl{0}{\bes }{\ub{ {\hr{\sum\limits_{n = 1}^\bes
{e^{ - nt}} }}}_{\frac{1}{e^t - 1}}t^{s - 1}dt} $ по теореме Б Леви и по
теореме Лебега мы имеем право перейти к пределу под знаком интеграла.
\end{proof}

\begin{theorem}[об аналитическом продолжении]
Дзета-функция аналитически продолжается на всю комплексную плоскость за исключением точки 1,
которая является её простым полюсом с вычетом, равным единице.
\end{theorem}
\begin{proof}
Изначально функция определена только на
полуплоскости. На комплексной плоскости $C_t $ (С -- комплексная плоскость,
а за переменную взята t)возьмём контур $\ga $, изображённый на рисунке
(неограниченно приближаемся в ``+'' вещественной оси). $\Ph \hr{s} =
\ints{\ga }{\frac{t^{s - 1}}{e^t - 1}dt} ,0 < \arg t < 2\pi$.

Интеграл сходится для $\fa s \in \Cbb$. Знаменатель в ноль не обращается,
непрерывная функция экспоненциально убывает на $\bes$. Более того, равномерно
сходящаяся на компакте $ \Ra $ по 1ой теореме Вейерштрасса $\Ph \hr{s}$
голоморфна в $bb$.

Пусть $Re\hr{s} > 1$ (Стягиваем контур $\ga $ к разрезу от 0 до $\bes
)$. Интегрируем по верхнему берегу от 0 до $\bes $, а обратно -- по
нижнему: $\Ph \hr{s} = \intl{0}{\bes }{\frac{t^{s - 1}}{e^t - 1}dt} -
\intl{0}{\bes }{\frac{t^{s - 1}}{e^t - 1}dt} \cdot e^{2\pi is} = \xi
\hr{s}\Ga \hr{s}\hr{1 - e^{2\pi is}}$, т.к. на нижнем берегу $\arg  =
2\pi $. Выразим отсюда $\xi $ - функцию: $\xi \hr{s} = \frac{\Ph
\hr{s}}{\Ga \hr{s}\hr{1 - e^{2\pi is}}},Res > 1,s \ne 2,3,..$. - это
функция голоморфная во всей $\Cbb\backslash Z$. Теперь надо исследовать все
целые (особые) точки, если можно устранить -- доопределить $ \Ra $
голоморфна во всей плоскости.

Сначала рассмотрим $s = 2,3,\dots$ устранимые особые точки (это мы знаем).
Теперь: $s = - n,n = 0,1,2,..$.

Утверждается, что эти точки тоже устранимые. Почему? $\ga \hr{s}$ имеет
полюс 1ого порядка в этих точках. $\hr{1 - e^{2\pi is}}$ имеет 0 1ого
порядка в этих точках $ \Ra $ предел знаменателя $\exi $ и $ \ne
0 \Ra $ устранимые особые точки. Осталось: $s = 1$. В ней: $\Ga
\hr{1} = 1,\hr{1 - e^{2\pi is}} \approx - 2\pi i\hr{s - 1},s \ra 1$ - первый
член ряда Тейлора. Числитель = 0 $ \Ra $ устранимая особая точка,
если числитель $ \ne 0 \Ra $ полюс.

\begin{note}
Контур $\ga $ можно заменить на окружность.
\end{note}

Вычислим: $\Ph \hr{1} = \oints{{\begin{array}{l}
 \hm{t} = R, \\
 0 < R < 2\pi \\
 \end{array}}}{\frac{dt}{e^t - 1}} = - 2\pi i \cdot \mathop {res}\limits_{t
= 0} \hr{\frac{1}{e^t - 1}} = \left. { - 2\pi i\frac{1}{e^t}} \right|_{t =
0} = - 2\pi i \cdot 1 = - 2\pi i. \Ra \xi \hr{s} \approx \frac{1}{s
- 1},s \ra 1$ - это первый член ряда Лорана, таким образом, здесь имеем
полюс 1ого порядка -- это единственная особая точка для $\xi
\hr{s}.
\end{proof}


\begin{problem}
Нарисовать график $ \ne 0 \Ra \xi \hr{s},s \in R$.
\end{problem}
\begin{problem}
Задача Чебышева -- возьмём наугад n натуральных чисел. С какой
вероятностью они взаимно просты: $n_1 ,n_2 ,...,n_m \in N$.
\end{problem}

\section{Принцип компактности}

\subsection{Теорема Монтеля}

Пусть $F = \hc{f:D \ra \Cbb},D - $ область в С.

\begin{df}
Семейство $F$ называется предкомпактным, если из любой последовательности функций этого семейства можно
выбрать подпоследовательность, равномерно сходящуюся к некоторой функции $f$ внутри области D.
\end{df}

\begin{note}
Говорят, что $F$ компактно, если оно предкомпактное и замкнуто относительно этой сходимости.
\end{note}

\begin{df}
Семейство $F$ равномерно ограничено внутри области $D$, если \fa K \subs D \exi M > 0:\fa f \in F\cln \hn{f}_K \le M$. \end{df}


\begin{stm}
Если семейство }$F$ предкомпактное, то оно равномерно ограничено внутри D.
\end{stm}

Нас интересует $H\hr{D} - $ пространство функций, голоморфных в области D.
Оказывается необходимое условие (смотри утверждение) является и достаточным
$ \Ra $

\begin{theorem}[Монтеля]
Если $F \in H\hr{D}$ и $F$ равномерно ограничено внутри области D, то $F$ - предкомпактное.
\end{theorem}

\begin{proof} 1) Докажем теорему для круга. Пусть $U = \hc{z \in
\Cbb\vert \hm{z - z_0 } < R} - $ произвольный круг. Пусть $f_n \hr{z} - $
голоморфны в $\ol{U}$. Пусть $\hn{f_n }_\ol{U} \le M - $ равномерно
ограничены в замкнутом круге. Надо доказать, что существует сходящаяся
подпоследовательность. Напишем ряды Тейлора: $f_n \hr{z} = \sum\limits_{n =
1}^\bes {a_k^{\hr{n}} \hr{z - z_0 }^k - } $ ряды сходятся в $\ol{U}$.
Напишем неравенство Коши: $\hm{a_k^{\hr{n}} } \le \frac{M}{R^k}\fa
n\fa k$. Рассмотрим числовую последовательность: $\hc{a_0^{\hr{n}} }_{n
\in N} - $ она ограничена $ \Ra \exi $ сходящаяся
подпоследовательность: $\hc{a_0^{\hr{n}} }_{n \in N} \sups
\hc{a_0^{\hr{n}} }_{n \in \La _0 \subs N} \ra a_0 $. Далее рассмотрим:
$\hc{a_1^{\hr{n}} }_{n \in \La _0 } \sups \hc{a_1^{\hr{n}} }_{n \in
\La _1 \subs \La _0 } \ra a_1 $ и т.д. до $\bes $ (аналогично).

Обозначим $\La - $ последовательность, полученная диагональным методом
Кантора.

По построению: $\fa k \in Z_ + :a_k^{\hr{n}} \ra a_k ,n \in \La $. Перейдём к пределу в неравенстве

Коши: $\hm{a_k } \le \frac{M}{R^k}$. Напишем степенной ряд: $f(z) =
\sum\limits_{k = 0}^\bes {a_k \hr{z - z_0 }^k - } $ где этот ряд
расходится? По формуле Коши-Адамара: $R_{cx} \ge R$ в открытом круге и
сходится. Сумма ряда -- голоморфна в $U$. Будем доказывать, что $f_n \ra f,n
\in \La $ равномерно внутри $U$. Зафиксируем $r:0 < r < R$. Будем
считать, что $\hm{z - z_0 } \le r$. Возьмём $\fa \ep > 0,$
найдём $k_0 :\sum\limits_{k = k_0 + 1}^\bes {M\hr{\frac{r}{R}}^k <
\ep } $ т.к. остаток сходящегося ряда сходится к 0. Теперь:
$\begin{array}{l}
 \hm{f(z) - f_n \hr{z}} = \hm{\sum\limits_{k = 0}^{k_0 } {\hr{a_k -
a_k^{\hr{n}} }\hr{z - z_0 }^k} + \sum\limits_{k = k_0 + 1}^\bes {\hr{a_k
}\hr{z - z_0 }^k} - \sum\limits_{k = k_0 + 1}^\bes {\hr{a_k^{\hr{n}}
}\hr{z - z_0 }^k} } \le \\
 \le \sum\limits_{k = 0}^{k_0 } {\hm{a_k - a_k^{\hr{n}} }r^k} + \ub{
{2\sum\limits_{k = k_0 + 1}^\bes {M\hr{\frac{r}{R}}^k} }}_{ < 2\ep
}, \Ra \exi N:\fa n > N,n \in \La , \\
 \end{array}$ ,т.к. конечное число ограниченных слагаемых, то
$\sum\limits_{k = 0}^{k_0 } {\hm{a_k - a_k^{\hr{n}} }r^k} < \ep $.
Итого: $\hm{f(z) - f_n \hr{z}} < 3\ep $. Т.е. для круга теорему
доказали. Докажем для $\fa $ компакта.

2) Дано: $f_n - $ голоморфны в области D. $\{f_n \} - $ равномерно
ограничены внутри D. Возьмём: $\fa K \subs \subs D$. Покроем компакт
К, так чтобы: $K \subs \cupl{j = 1}{m}{U_j } ,\ol{U}_j \subs D$. (Выбираем конечное подпокрытие). Применим утверждение, доказанное в пункте
1. Утверждается, что $\hc{f_n }_{n \in N} \sups \hc{f_n }_{n \in \La
_1 \subs N} $ - сходится равномерно внутри $U_1 $.
$\hc{f_n }_{n \in \La _1 } \sups \hc{f_n }_{n \in \La _2 \subs
\La _1 } \quad -
сходится равномерно внутри U_2 и т.д. \quad
За конечное число шагов получим:
\quad
\La = \La _m :\hc{f_n }_{n \in \La } \quad -
сходится равномерно внутри
\quad
U_1 ,U_2 ,...,U_m . \Ra
сходится равномерно внутри
\quad
\cupl{j = 1}{m}{U_j } \Ra
сходится равномерно на $К$ это вспомогательное условие.

3) Построим исчерпание области D. Обозначим $K_v = \hc{z \in D\vert 1)\hm{z}
\le v,2)dist\hr{z,\pd D} \ge \frac{1}{v},v = 1,2,...}$ Тогда $K_v - $
компакты (может быть и $\emptyset )$.

\textbf{Свойства} $K_v :$ 1) $K_v \subs K_{v + 1} $ - монотонное возрастание. 2) }$\cupl{v = 1}{\bes
}{K_v } = D$. Любая последовательность компактов с этими 2 свойствами называется }\textbf{исчерпанием}} области D.}

Далее: $\hc{f_n }_{n \in N} \sups \hc{f_n }_{n \in \La _1 \subs N} $
сходится равномерно на $K_1 $. Доказано в 2).


$\hc{f_n }_{n \in \La _1 } \sups \hc{f_n }_{n \in \La _2 \subs
\La _1$ сходится равномерно на
$K_2$ тд


Теперь обозначим $\La$ - подпоследовательность, полученную диагональным
методом Кантора. Тогда $\hc{f_n }_{n \in \La } $ сходится равномерно на
$K_1 ,K_2 ,...$. Однако из свойства 2) $K_1 ,K_2 ,... \Ra \fa K
\subs \subs D\exi v$ т.ч. $K \subs K_v \Ra $ сходимость
равномерная на $K$, где К -- любое.
\end{proof}


\subsection{Теорема Витали}

\begin{theorem}[Витали]
Если 1) функция }$f_n - $ голоморфная в области D.}

$2) \hc{f_n } - $ равномерно ограничены внутри D.}

$3) e \subs D$ имеет в D хотя бы 1 предельную точку.}

$4) f_n $ сходятся на е, то }$\hc{f_n } - $ сходятся равномерно внутри D.}
\end{theorem}

\begin{proof}
По теореме Монтеля $f_n - $ предкомпактная $
\Ra $ она имеет предельные точки. Пусть f и g -- две предельные
точки для $\hc{f_n }$, это означает, что $f_n \mathop \ra \limits_{n \in
\La _f } f,f_n \mathop \ra \limits_{n \in \La _g } g$ - равномерная
сходимость внутри D. По 1ой теореме Вейерштрасса f и g -- голоморфные в D.
Из условия теоремы: $f \equiv g$ на множестве е. е имеет хотя бы одну
предельную точку (см. условие 3)). Тогда по теореме единственности для
голоморфной функции $ \Ra f = g$ всюду в D. Мы доказали, что
предельная точка единственна, а мы знаем, что предкомпактная
последовательность сходится $ \Leftrightarrow $ она имеет предельную точку.
\end{proof}


\section{Теорема Рунге и теорема Мергеляна}

\subsection{Теорема Рунге}

\begin{theorem}[Рунге]
Если 1) D -- односвязная область.}

$2) f - $ голоморфна в области D, то }$\exi $ последовательность множеств }$P_n ,$ которая сходится к }$f$ равномерно внутри области D.}

\begin{proof}.
А) Нанесём на плоскость сетку квадратов со
сторонами $\hr{\frac{1}{3}}^n$:

Пусть q -- квадраты, обладающие следующими свойствами: 1) $q \subs
\hc{\hm{x} < 3^n,\hm{y} < 3^n}$

2) Квадрат q и все 8 соседних с ним квадратов $ \subs D$. Обозначим $\ol{D}_n = \cup \ol{q} \Ra $


$\ol{D}_n$ замыкание области $D_n$.

1) $\ol{D}_n \subs D_{n + 1} $ 2) $\cupl{n = 1}{\bes }{\ol{D}_n } =
D \Ra \ol{D}_n - $ исчерпание. Области $D_n $ обладают следующими
свойствами: а) $D_n $ - объединение конечного числа односвязных областей. б)
$\Ga _n = \pd D_n - $ объединение конечного числа простых замкнутых
спрямляемых кривых, т.е. ломаных.

Б) Напишем формулу Коши: $f(z) = \frac{1}{2\pi i}\ints{\Ga _{n + 1}
}{\frac{f\hr{\xi }}{\xi - z}d\xi } ,z \in D_{n + 1} $. (в этом равенстве и
формула Коши и теорема Коши, т.к. интервалы в некоторых областях = 0).

Теперь: $S\hr{z} = \frac{1}{2\pi i}\sum\limits_{j = 1}^N {\frac{f\hr{\xi
}}{\xi _j - z}\De \xi _j - } $ взяли любое разбиение кривой, записали
интегральную сумму. Зафиксировали последовательность интегральных сумм,
любых, лишь бы параметр разбиения $ \ra 0$. Тогда $\hc{S\hr{z}} \ra f(z)$
(по теореме существования интегралов для непрерывной функции), но сходимость
поточечная, а мы хотим доказать, что она равномерная. Возьмём $\fa K
\subs \subs D_{n + 1} ,M = \mathop {\max }\limits_{\xi \in \Ga _{n +
1} } \hm{f\hr{\xi }} < \bes ,\delta = dist\hr{K,\Ga _{n + 1} } > 0$.
Оценим: $\hm{S\hr{z}} \le \frac{1}{2\pi }\frac{M}{\delta }\hm{\Ga _{n +
1} } - $ оценка не зависит от выбора интегральной суммы $S$ и от выбора $z
\in K$. $ \Ra $ равномерная оценка. Тогда по теореме Витали $\hc{S}
\ra f$ равномерно внутри $D_{n + 1} $, в частности, равномерно на $\ol{D}_n $. Это всё было для фиксированного n. Теперь возьмём $\fa $
последовательность положительных чисел $ \ra 0$. Т.е. $\ep _n >
0,\ep _n \ra 0$. И для каждого n: $R_n \hr{s} = S\hr{z} - $
обозначение. $\hn{R_n - f}_\ol{D} < \ep _n $. Заметим, что $R_n $
- рациональные функции. Теперь надо переходить к полиному. Мы научились
приближать рациональные функции.

\end{document}


\subsection{Свойство симметрии для ДЛП}

\begin{theorem}
Если ДЛП переводит окружность $\Ga$ в окружность $\Ga^*$, то точки, симметричные относительно $\Ga$ оно
переводит в точки, симметричные относительно $\Ga^*$.
\end{theorem}
\begin{proof}
Вспомним критерий симметричности: две точки $A_1$ и $A_2$ симметричны относительно окружности $\Ga$ тогда и
только тогда, когда любая окружность $\ga$, проходящая через эти точки, пересекает $\Ga$ под прямым углом
(докажите самостоятельно). Теорема становится очевидной после применения этого критерия. Через точки
$A_1$ и $A_2$ проведём $\Ga^*$. Прообраз $\ga$ есть окружность. По условию $A_1$, и $А_2$ симметричны, и
по критерию $\Ga$ и $\ga$ пересекаются под прямым углом, а значит, и их образы пересекаются под прямым углом.
А тогда снова по критерию точки $A_1^*$ и $A_2^*$ симметричны.
\end{proof}


\begin{problem}
Доказать, что нули производной многочлена лежат в выпуклой оболочке нулей самого многочлена.
\end{problem}
\end{document}

%% Local Variables:
%% eval: (setq compile-command (concat "latex -halt-on-error -file-line-error " (buffer-name)))
%% End:
