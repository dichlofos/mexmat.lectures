\documentclass[a4paper]{article}
\usepackage{dmvn}

\tocsubsubsectionparam{3em}

\def\mcomp#1{\mskip-10mu#1\mskip-10mu}

\newenvironment{petit}
{\par \smallskip \hrule \smallskip \footnotesize}
{\par \smallskip \hrule \smallskip}

\begin{document}
\dmvntitle{Курс лекций по}{комплексному анализу}{Лектор\т Валерий Константинович Белошапка}
{III курс, 5 семестр, поток математиков}{Москва, 2005 г.}

\tableofcontents

\pagebreak

\section*{Предисловие}

Этот документ представляет собой переработанный курс лекций по комплексному анализу,
первоначально набранный одним из студентов МехМата. В~оригинальном варианте не~было иллюстраций, что осложняло
восприятие материала, а~кроме того, имелось отличное от нуля количество опечаток и~прочих глюков
типографского характера. Данный вариант был создан на~основе авторских лекций, и, естественно, не~без
посторонней помощи.
Огромное, бесконечномерное спасибо Алисе Домбровской и~Анастасии Тимофеевой за
конспекты лекций и~замечания об ошибках. Также благодарность за обнаружение большого количества ошибок
выносится Александру Веремьёву, Алексею Прянишникову и~Анастасии Абрашитовой, а~особенно Илье Питерскому,
Алексею Басалаеву и~Кириллу Никитину.

В 2005 году текст был отредактирован Сергеем Шашковым.

Порядок изложения материала наиболее соответствует курсу 2004 г.

\medskip
\dmvntrail

\begin{thebibliography}{4}
\setlength\itemsep{-.5mm}
\bibitem{shabat}
    Б.\,В.\,Шабат. \emph{Введение в~комлексный анализ.}\т М.: Наука, 1985. 3-е изд.
\bibitem{fomenko}
    А.\,С.\,Мищенко, А.\,Т.\,Фоменко. \emph{Курс дифференциальной геометрии и~топологии.}\т М.: Факториал, 2000.
\bibitem{vinberg}
    Э.\,Б.\,Винберг. \emph{Курс алгебры.}\т М.: Факториал, 2002.
\end{thebibliography}

\pagebreak

\section{Дифференцирование комплексных функций}

\subsection{Введение}

В~процессе развития математики постепенно происходило расширение числовых множеств:

$$\N \ra \Z \ra \Q \ra \R \ra \Cbb \ra \Hbb \ra \Cbb_a$$

Было замечено, что комплексные числа\т это очень хорошие числа, а~комплексный анализ оказывается
намного более красивым, нежели вещественный. Именно поэтому мы будем изучать комплекснозначные функции.

Переход от $\R$ к~$\Cbb$ открывает неожиданную связь между элементарными функциями $\exp z$, $\sin z$, $\cos z$,
$\ln z$, $\arcsin z$.

Рассмотрим функцию $f(x) = \frac{1}{1+x^2} = 1 - x^2 + \ldots$. Её ряд сходится при $|x| < 1$ и расходится при
$|x| \ge 1$, но она определена при $\fa x \in \R$. А~теперь рассмотрим функцию $f(z) = \frac{1}{1+z^2}$. При
$z = \pm i$ имеем $f = \infty$.

Рассмотрим функцию $f\cln \Cbb \ra \Cbb$. С вещественной точки зрения $z = x + iy$,
и~$f(z) = u(x,y) + iv(x,y)$, \те $$\rbmat{x \\y} \corr{f} \rbmat{u(x,y) \\ v(x,y)}.$$

Все основные определения вещественного анализа\т предел последовательности и функции, непрерывность, открытость,
замкнутость, компактность, связность и односвязность\т автоматически переносятся в комплексный анализ.

\begin{df}
\emph{Область}\т открытое связное множество. \emph{Окрестностью} точки называется произвольный открытый шар, её содержащий.
\end{df}

\begin{stm}
Открытое множество $D \subs \Cbb^n$ связно тогда и только тогда, когда оно линейно связно.
\end{stm}
\begin{proof}
\pt{1} Докажем, что из линейной связности следует связность.
В самом деле, допустим, что множество $D$ линейно связно и несвязно, \те разбивается на два открытых: $D = A \sqcup B$.
Пусть $x \in A$, а $y \in B$. Соединим эти точки непрерывной кривой $\ga\cln[0,1]\ra D$, причём $\ga(0)=x$, а $\ga(1)=y$.
Положим
\eqn{t_A := \sup\hc{t \in [0,1] \cln \ga(t) \in A}.}
Если $t_A = 1$, то получаем противоречие с тем, что $y \in B$, так
как $\ga(t_A)=\ga(1)=y$. Значит, $t_A \in (0,1)$. Поскольку прообраз открытого множества при непрерывном
отображении $\ga$ открыт, то точка $t_A$ отображается в~$B$ вместе со своей окрестностью в $[0,1]$, \те интервалом. Но это противоречит
определению $t_A$. Если же $t_A \in A$, то по аналогичным соображениям получаем противоречие с тем, что $t_A$ есть точная верхняя грань.

\pt{2} Теперь докажем в обратную сторону.
Рассмотрим произвольные точки $x$ и $y$ в области $D$ и докажем, что их можно соединить кривой.
Пусть $E$\т множество тех точек, до которых можно дойти из $x$. Оно непусто, так как $x \in E$, и открыто,
так как если $z \in E$, то $U(z) \subs E$, а окрестность линейно связна (это круг).
По аналогичным соображениям множество $D \wo E$ открыто и непусто. Таким образом, мы разбили $D$ на два непересекающихся
открытых множества, что противоречит связности области $D$.
\end{proof}

\begin{note}
Вообще говоря, из связности не следует линейная связность. Например, рассмотрим замыкание графика функции $f(x)=\sin \frac1x$ при $x > 0$.
Оно связно, но не линейно связно (докажите это!)
\end{note}

\begin{note}
Прямая и плоскость топологически неэквивалентны: если из прямой выкинуть точку, она распадается на два куска,
а плоскость\т нет.
\end{note}

\begin{note}
Для комплексных функций теряет смысл понятие $\pm \bes$, ибо $\Cbb$\т неупорядоченное поле.
\end{note}

\subsection{Комплексная производная}

\subsubsection{Определение комплексной производной}

Пусть функция $f$ определена в некоторой окрестности точки $z$.

\begin{df}
Функция $f$ называется \emph{$\Cbb$\д дифференцируемой} в точке~$z$, если существует предел
\eqn{\lim_{\De z \ra 0}\frac{f(z + \De z) - f(z)}{\De z} =: f'(z).}
\end{df}

Обозначим $\al(z, \De z) := \frac{f(z + \De z) - f(z)}{\De z} - f'(z)$, и $\De f := f(z+\De z)-f(z)$.
Дадим еще одно определение комплексной дифференцируемости:

\begin{df}
Функция $f$ называется \emph{$\Cbb$\д дифференцируемой} в точке~$z$, если
\eqn{\De f = f'(z) \De z + \al (z, \De z)\De z.}
\end{df}

Равносильность этих определений очевидна.

\begin{df}
Функция $f=u(x,y)+iv(x,y)$, определённая в окрестности точки $z=x+iy$,
называется \emph{$\R$\д дифференцируемой} в точке $z$, если
\begin{align*}
u(x+\De x, y+\De y)&=u(x,y)+\pf{u}{x}(x,y)\De x + \pf{u}{y}(x,y)\De y + o\hr{\sqrt{{\De x}^{2} + {\De y}^2}}, \\
v(x+\De x, y+\De y)&=v(x,y)+\pf{v}{x}(x,y)\De x + \pf{v}{y}(x,y)\De y + o\hr{\sqrt{{\De x}^{2} + {\De y}^2}}.
\end{align*}
\end{df}

Складывая первое равенство со вторым, умноженным на $i$, получаем
\eqn{f(z + \De z) =f(z) + \hr{\pf{u(z)}{x} + i \pf{v(z)}{x}} \De x +
\hr{\pf{u(z)}{y} + i \pf{v(z)}{y}} \De y + o\hr{\hm{\De z}}.}
Введём обозначения:
\eqn{\pf{f}{x} := \pf{u}{x} + i \pf{v}{x}, \quad \pf{f}{y} :=  \pf{u}{y} + i \pf{v}{y}.}
Тогда последнее равенство ввиду того, что $o\hr{\hm{\De z}} \bw= o\hr{\De z}$, перепишется в виде
\eqn{f(z + \De z) =f(z) +
\pf{f(z)}{x}\De x + \pf{f(z)}{y}\De y + o(|\De z|),} где $\De z = \De x + i \De y$. Имеем
$\De x = \frac{\De z + \ol{\De z}}{2}$ и $\De y = \frac{\De z - \ol{\De z}}{2i}$. Тогда
\eqn{f(z+\De z)=f(z)+\frac12\hr{\pf{f(z)}{x}-i\pf{f(z)}{y}}\De z+ \frac12\hr{\pf{f(z)}{x} + i\pf{f(z)}{y}}\ol{\De z} +
o\hr{|\De z|}.}
Введём обозначения:
\eqn{\pf{f}{z}:= \frac12\hr{\pf{f}{x}- i\pf{f}{y}}, \quad \pf{f}{\ol{z}}:= \frac12\hr{\pf{f}{x}+ i\pf{f}{y}}.}
Тогда приращение $f$ запишется следующим образом:
\eqn{f(z + \De z) = f(z) + \pf{f}{z}(z)\De z + \pf{f}{\ol{z}}(z)\ol{\De z} + o(|\De z|).}

\begin{df}
Операторы $\pf{}{z}$ и $\pf{}{\ol z}$ называются \emph{формальными частными производными}.
\end{df}

\subsubsection{Условие Коши\ч Римана}

\begin{prop}
$\Cbb$\д дифференцируемость функции $f$ в точке $z$ эквивалентна $\R$\д дифференцируемости
при условии $\frac{\pd f}{\pd \ol{z}}(z) = 0$.
\end{prop}

Получается следующее правило дифференцирования: забыть, что $ \ol{z}$\т это функция от $z$, и считать,
что $z$ и~$\ol z$\т это две независимые буквы.

Условие $\pf{f}{\ol{z}}(z) = 0$ называется \emph{условием Коши\ч Римана}. Оно эквивалентно тому, что

\eqn{\frac12\hs{\pf{}{x}(u+iv)+  i\pf{}{y}(u+iv)} = 0, \text{ \те } \; \case{\pf{u}{x} = \phm \pf{v}{y},\\\pf{u}{y} = -\pf{v}{x}.}}

\begin{problem}
Пусть $\De z = r e^{i\ph}$ и пусть функция  $f$ является $\R$\д дифференцируемой в точке $z$. Запишем приращение в полярных координатах:
$\De z=re^{i\ph}$. Доказать, что для любого $\ph$ существует предел
\eqn{h(\ph) := \liml{r \ra 0}\frac{f(z + \De z) - f(z)}{\De z},}
а образом полуинтервала $[0, 2\pi)$ при отображении $h$ является окружность (которая может вырождаться в точку),
и функция $\Cbb$\д дифференцируема тогда и~только тогда, когда образом $h$ является одна точка.
\end{problem}

Операторы $\pf{}{z}$ и $\pf{}{\ol z}$ линейны и~удовлетворяют правилу Лейбница, \те $L(fg) = (Lf)g+f(Lg)$.

Все свойства производных (сумма, произведение, частное, сложная функция) сохраняются и доказываются аналогично
вещественному случаю.

\begin{df}
Функция $f$, определённая в области $D$ и имеющая $\Cbb$\д производную в~каждой точке области~$D$, называется
\emph{голоморфной} (синонимы: \emph{аналитической}, \emph{регулярной}) в~этой области.
\end{df}

Как мы потом узнаем, дифференцируемые функции\т это хорошие функции, и~название <<аналитические>> дано не~случайно.

\subsubsection{Примеры дифференцируемых функций}

Функция $f(z)=z$\т хорошая, $z'=1$, а вот $f(z) = |z|^2$\т плохая. Функция $f(z)=\ol z$ тоже плохая.

\begin{problem}
Пусть $\exi f'(a)$, и $\pf{f}{x}(a)=A$. Чему равно $\pf{f}{y}(a)$?
\end{problem}
\begin{solution}
Вспомним условие Коши\ч Римана. Пусть $A=X+iY$, тогда $\pf{f}{y}(a)=-Y+iX=iA$.
\end{solution}

Примерами хороших функций являются:

\begin{items}{-2}
\item Рациональная функция $f \in \Cbb(z)$;
\item Дробно\д линейная функция $f(z) = \frac{az+b}{cz+d}$;
\item Элементарные функции: $e^{z}$, $\sin z$, $\cos z$, \dots, задаваемые рядами или формулой.
\end{items}

\begin{note}
У элементарных комплексных функций есть и новые свойства. Например, неограниченность: $\sin z$ неограничен.
Также возникает проблема с обратными функциями в силу эффекта многозначности (например, $\sqrt{z}$ или $\ln z$).
\end{note}

\begin{problem}
Найдите, чему равно $i^i$.
\end{problem}
\begin{solution}
Имеем
$i^i = e^{i\ln i} = \exp\hs{i\br{\ln 1 + i\hr{\frac{\pi }{2} + 2\pi n}}} = \exp\hs{-\frac{\pi}{2}-2\pi n}$.
\end{solution}


\subsection{Функции нескольких переменных}

\subsubsection{$\Cbb$\д дифференцируемость ФНП}

Будем рассматривать вектор\д функции вида
\eqn{f(\vec z)=f(z_1 \sco z_n)=f(x_1 \sco x_n, y_1 \sco y_n).}
Аналогично вещественному случаю,
\eqn{f(z+\De z)=f(z_{1}+\De z_1 \sco z_n+\De z_n)= f(z_1 \sco z_n)+\suml{j=1}{n}\pf{f}{z_j}\De z_j
+\suml{j=1}{n}\pf{f}{\ol z_j}\ol{\De z}_j + o(\De z).}
Обозначим
\eqn{\pd f := \suml{j=1}{n}\pf{}{z_j}\De z_j, \quad \ol{\pd} f := \suml{j=1}{n}\pf{}{\ol z_j}\ol{\De z}_j,}
\eqn{\pf{}{z_j}:= \frac12\hr{\pf{}{x_j}- i\pf{}{y_j}}, \quad \pf{}{\ol z_j} := \frac12\hr{\pf{}{x_j}+ i\pf{}{y_j}}, \quad df=\pd f+\ol{\pd} f.}

Как и в одномерном случае, комплексная дифференцируемость эквивалентна $\R$\д дифференцируемости при условии $\ol{\pd}f=0$.
Если $D$\т область в $\Cbb^n$, то голоморфность в $D$ означает $\Cbb$\д дифференцируемость в каждой точке области~$D$,
и то же самое для отображений $f\cln \Cbb^n \ra \Cbb^m$.

\subsubsection{Комплексная линейность. Теоремы о дифференцируемых отображениях}

Как и в действительном анализе, верны теоремы о производной сложной функции (композиции отображений),
обратном отображении и неявной функции. Сведём их доказательства к вещественному случаю вдвое большей размерности.

\begin{theorem}[О неявной функции]
Пусть дана система голоморфных в окрестности точки $(\ol z_0, \ol \om_0)$
функций $F_i(z_1\sco z_m, \om_1\sco \om_n)$, $i = 1\sco m$,
таких что в этой точке они равны нулю и ранг якобиана $\hr{\frac{\pd F_i}{\pd z_j}}$
максимален. Тогда существуют голоморфные в окрестности точки $\ol \om_0$ функции
$z_i(\om_1\sco \om_n)$, $i = 1\sco m$, такие что
$F_i(z_1(\om_1\sco \om_n)\sco z_m(\om_1\sco \om_n), \om_1\sco \om_n)\equiv0$, $i = 1\sco m$.
\end{theorem}
\begin{proof}
Применив <<вещественную теорему>>, мы получим не обязательно голоморфные функции
$z_i(\om_1\sco \om_n)$, $i = 1\sco m$, такие что
$F_i\br{z_1(\om_1\sco \om_n)\sco z_m(\om_1\sco \om_n), \om_1\sco \om_n}\equiv0$, $i = 1\sco m$.
Проверим, что они голоморфны. Заметим, что каждого $i$ по формуле производной сложной функции
\eqn{0\equiv\frac{\pd F_i}{\pd \ol\om_j}=\sum_{k=1}^m\frac{\pd F_i}{\pd z_k} \frac{\pd z_k}{\pd \ol\om_j} +
\sum_{k=1}^m\frac{\pd F_i}{\pd \ol z_k} \frac{\pd \ol z_k}{\pd \ol \om_j} + \frac{\pd F_i}{\pd \ol \om_j} =
\sum_{k=1}^m\frac{\pd F_i}{\pd z_k} \frac{\pd z_k}{\pd \ol \om_j }}
в силу голоморфности $F_i$. Получаем, что произведение невырожденной матрицы $\hr{\frac{\pd F_i}{\pd z_j}}$
на столбец $\hr{\frac{\pd z_k}{\pd \ol \om_j}}$ нулевой. Но это бывает только тогда, когда сам столбец
нулевой. Но этого нам и надо.
\end{proof}


\begin{df}
Пусть $\K$\т поле. Отображение $f$ называется \emph{$\K$\д линейным}, если $f(a+b)=f(a)+f(b)$, и можно из под знака~$f$
выносить скаляры из $\K$, то есть $f(\al x)=\al f(x)$ для $\fa \al \in \K$.
\end{df}

\begin{lemma}
Любое $\R$\д линейное отображение $f\cln\Cbb\ra\Cbb$ имеет вид
\eqn{f(z)=az+b\ol z.}
\end{lemma}
\begin{proof}
Пусть $z=x+iy$, тогда $f(z)=xf(1)+yf(i)$. Обозначим $f(1) =: \al$ и $f(i) =: \be$. Положим $a := \frac12(\al-i\be)$ и
$b := \frac12(\al+i\be)$. Тогда $f(z) = az+b\ol z$.
\end{proof}

\begin{lemma}
Любая $\Cbb$\д линейная функция имеет вид $f(z)=az$.
\end{lemma}
\begin{proof}
Имеем $z = z \cdot 1$. Тогда $f(z)=zf(1)$. Остаётся обозначить $a := f(1)$.
\end{proof}

\begin{theorem}
Функция $f$ комплексно\д линейна тогда и только тогда, когда она $\R$\д линейна, и $f(iz)=if(z)$.
\end{theorem}
\begin{proof}
Слева направо \т очевидно (частный случай). Обратно: по первой лемме имеем $f(z)=az+b\ol z$, следовательно,
$f(iz)=i(az-b\ol z)$. С другой стороны, $i f(z)=i(az+b\ol z)$.
По условию теоремы эти выражения должны совпадать, значит, $b\ol z = 0$ при любом $z$, следовательно, $b=0$, что и
означает $\Cbb$\д линейность.
\end{proof}

Очевидно, композиция комплексно\д линейных отображений будет комплексно\д линейной, а обратное к невырожденному
комплексно\д линейному отображению будет также невырожденным и комплексно\д линейным.

Сопоставим произвольному комплексному отображению $f_\Cbb\cln \Cbb^n\ra\Cbb^n$ некоторое отображение $f_\R\cln \R^{2n}\ra\R^{2n}$.
Пусть $M_\Cbb$\т матрица $f_\Cbb$ в базисе $e_1 \sco e_n$, тогда $M_\Cbb = A+iB$, где $A$ и $B$\т
некоторые вещественные матрицы размера $n\times n$. Рассмотрим матрицу $M_\R$ размерности $2n\times 2n$ следующего вида:
\eqn{M_\R = \rbmat{A & -B \\ B & \phm A}.}
Это и будет матрица отображения $f_\R$ в базисе $e_1 \sco e_n, i e_1 \sco i e_n$. Покажем, что такое
сопоставление корректно. Пусть $\det M_\Cbb \neq 0$. Докажем, что в этом случае и $\det M_\R\neq 0$. Действительно,
\begin{multline}
\det M_\R = \mbmat{A & -B \\ B & A} = \mbmat{A+iB & -B+iA \\ B & A} = \mbmat{A + i B & 0 \\  B &  A-i B} =\\
=\det(A+iB)\det(A-i B)= \det M_\Cbb \cdot \det \ol{M}_\Cbb  = \hm{\det M_\Cbb}^2,
\end{multline}
то есть определитель вещественной матрицы также отличен от нуля. Далее заметим, что наше сопоставление
биективно, а кроме того, оно является гомоморфизмом, ибо произведению комплексных матриц соответствует
произведение соответствующих вещественных матриц специального вида. Стало быть, это изоморфизм.

\begin{note}
В теореме о неявной функции рассматриваются функции класса $\Cb^1$. Позже мы узнаем, что если функция
голоморфна (\те дифференцируема хотя бы один раз), то она лежит в классе $\Cb^\bes$.
\end{note}

\section{Интегрирование комплексных функций}

\subsection{Интеграл от комплексной функции и его простейшие свойства}

\subsubsection{Первообразные}

\begin{df}
Функция $F(z)$ называется \emph{первообразной} по отношению к $f(z)$, если $F'(z)=f(z)$.
\end{df}

\begin{lemma}
Пусть функция $F_1$ \т первообразная к $f$. Функция $F_2$ является первообразной
к $f$ тогда и только тогда, когда $F_2=F_1+C$.
\end{lemma}
\begin{proof}
Справа налево\т очевидно. Обратно: положим $\Ph := F_2-F_1$, тогда $\Ph' \equiv 0$.
Пусть $\Ph = u+iv$. Тогда имеем
\eqn{
\case{
\frac12\hr{\pf{\Ph}{x} - i\pf{\Ph}{y}} = 0, \\
\frac12\hr{\pf{\Ph}{x} + i\pf{\Ph}{y}} = 0.
}}
Значит, $du = dv=0$, то есть $\pf{u}{x} = \pf{u}{y} =0$ и $\pf{v}{x}=\pf{v}{y}=0$. Следовательно, $u=\const$ и $v=\const$.
\end{proof}

\subsubsection{Интеграл по кривой}

\begin{df}
\emph{Кривой} называется непрерывное отображение $\ga\cln [a,b]\ra \Cbb$.
Говорят, что кривая \emph{кусочно\д гладкая}, если существует разбиение отрезка $[a,b]$ такое, что на каждом
интервале разбиения отображение $\ga$ непрерывно дифференцируемо, и в концах отрезков разбиения
существуют односторонние пределы производной~$\ga'$ (быть может, различные).
\end{df}

\begin{df}
\emph{Криволинейным интегралом второго рода} (работой силы) называется интеграл
\eqn{\ints{\ga}P(x,y)\,dx+Q(x,y)\,dy := \intl{a}{b}\hs{P\br{x(t), y(t)}\frac{dx}{dt}(t)+Q\br{x(t),y(t)}\frac{dy}{dt}(t)}\,dt.}
\end{df}

\begin{df}
\emph{Интегралом} от комплекснозначной функции $w(t)$ называется число
\eqn{\int w(t)\,dt := \int\Rea w(t)\, dt+i\int \Img w(t)\,dt.}
\end{df}

\begin{df}
Кривая называется \emph{простой}, если она не имеет самопересечений. Кривая называется \emph{замкнутой}, если
$\ga(a)=\ga(b)$. \emph{Жордановой} будем называть простую замкнутую кривую. Длину спрямляемой кривой $\ga$ будем
обозначать через $|\ga|$.
\end{df}

\begin{theorem}[Жордана]
Любая плоская простая замкнутая кривая разбивает плоскость на $2$ связных компоненты и является границей каждой
из них. Одна из компонент гомеоморфна (открытому) кругу.
\end{theorem}
Доказательства этой теоремы в нашем курсе не будет.

\begin{note}
В трёхмерном пространстве эта теорема неверна: существует пример (\emph{сфера Александера})\т вложение
двумерной сферы в $\R^3$, при котором её образ разбивает пространство на две  открытые области. Одна из этих областей
гомеоморфна шару, а вторая не является односвязной.
\end{note}

Теперь определим \emph{интеграл от комплексной функции по кривой}.

\begin{df}
Пусть $f(z)=u(x,y)+iv(x,y)$\т непрерывная функция, а $\ga=\hc{z(t) \vl t \in [a,b]}$\т кусочно\д гладкий путь.
В силу кусочной гладкости достаточно дать определение для гладкой кривой.

\eqn{\ints{\ga}f(z)\,dz := \ints{\ga}\br{u(x,y)+iv(x,y)}(dx+idy)= \ints{\ga}\br{u(x,y)\,dx-v(x,y)\,dy}+i\ints{\ga}\br{u(x,y)\,dy+v(x,y)\,dx}.}

\eqn{\ints{\ga}f(z)\,dz := \intl{a}{b}f\br{z(t)}z'(t)\,dt, \quad dz := z'(t)\,dt.}
\end{df}

\subsubsection{Свойства интеграла}

\pt{1} Линейность (следует из свойств вещественных интегралов):

\eqn{\ints{\ga}\br{a f(z)+b g(z)}dz=a\ints{\ga}f(z)\,dz+b\ints{\ga}g(z)\,dz.}

\pt{2} Аддитивность относительно кривой: если $\ga=\ga_1\cup\ga_2$, то

\eqn{\ints{\ga}f(z)dz=\ints{\ga_1}f(z)\,dz + \ints{\ga_2}f(z)\,dz.}

\pt{3} Независимость от параметризации: если $\ga= \hc{z(t) \vl t \in [a,b]}$, а
$t=\ph(\tau)$ и $\wt{\ga}=\hc{z\br{\ph(\tau)} \vl \tau \in [\al, \be]}$, где $\ph$\т
непрерывная функция, такая что $\ph(\al)=a$ и $\ph(\be)=b$, то \eqn{\ints{\ga}f(z)\,dz=\ints{\wt{\ga}}f(z)\,dz.}

\pt{4} Смена ориентации кривой (параметризация в обратном направлении) меняет знак интеграла.

\pt{5} Оценка  интеграла:

\eqn{\bbm{\ints{\ga}f(z)\,dz} = \bbm{\intl{a}{b}f(z)z'(t)\,dt} \le \max |f|\ub{\intl{a}{b}|z'(t)|\,dt}_{\text{длина}} = \maxl{\ga} |f| \cdot |\ga|.}

\begin{ex}
$$\ints{\ga}1\,dz=\intl{a}{b}z'(t)\,dt=z(b)-z(a),$$
$$\ints{\ga}z\,dz=\intl{a}{b}z(t)z'(t)\,dt = \frac12\intl{a}{b}\br{z^2(t)}'dt=\frac{z^2(b)-z^2(a)}{2}.$$
\end{ex}

\subsubsection{Другой подход к определению интеграла}

\begin{df}
\dmvnpicrh{pictures.10}{-7}
Пусть $\ga(t)$\т непрерывная\dmvnpicra{pictures.10}{1}{-1.5pc} кривая, соединяющая точки $A$ и $B$, а $T$\т разбиение отрезка её
параметризации $a=t_0, t_1 \sco t_{n-1}, t_n=b$. Через $\la_T$ обозначим диаметр разбиения $T$.
Положим $z_j := \ga(t_j)$. Рассмотрим интегральную сумму
\eqn{S(T):=\suml{j=1}{n} f \br{\ga(t^*_j)}\hr{z_j-z_{j-1}},}
где $t^*_j\in [t_{j-1}, t_j]$ и $\ze_j = \ga(t^*_j)$. \emph{Интегралом} от функции $f$ по кривой $\ga$ называется число
\eqn{\ints{\ga}f(z)\,dz := \liml{\la_T\ra 0}S(T).}
\end{df}

\begin{df}
Кривая называется \emph{спрямляемой}, если её длина конечна, \те существует предел
\eqn{|\ga|:=\supl{T}\suml{j=1}{n} \hm{z_j - z_{j-1}}.}
\end{df}

Напомним некоторые свойства спрямляемых кривых.
\begin{theorem}[критерий спрямляемости]
Кривая $\ga = z(t)$ спрямляема тогда и только тогда, когда~$z(t)$\т функция ограниченной вариации.
\end{theorem}

\begin{theorem}
Если кривая $\ga=z(t)$ спрямляема, то $z(t)$ дифференцируема почти всюду.
\end{theorem}

\begin{stm}
Если функция $f$ непрерывна, а кривая $\ga$ спрямляема, то $\ints{\ga}f(z)\,dz$ существует.
\end{stm}

\begin{problem}
Докажите, что если $\ga$\т кусочно\д гладкая кривая, то второе определение интеграла даёт тот же ответ, что и исходное.
\end{problem}

Как в нашей ситуации реализуются интегралы вида $\ints{\ga}f\,ds$ (интегралы  II рода)? Это
$\ints{\ga}f(z)|dz|$. Дадим два эквивалентных определения.

\begin{df}
\eqn{\ints{\ga}f(z)|dz| := \intl{\al}{\be}f\br{z(t)}\hm{z'(t)}\,dt = \liml{\la T \ra 0}\suml{j=1}{n}f(\ze_j)\hm{z_j-z_{j-1}}.}
\end{df}

В частности, $\ints{\ga}|dz|= |\ga|$ и оценку интеграла можно записать так:

\eqn{\bbm{\ints{\ga}f(z)\,dz} \le \ints{\ga}|f(z)| |dz|.}

\subsection{Основные интегральные формулы и теоремы}

\subsubsection{Краткая экскурсия в топологию: гомотопия кривых}

\begin{df}
\dmvnpicrh{pictures.20}{-7}
Пусть\dmvnpicra{pictures.20}{2}{-1.5pc} $\al, \be$\т две кривые в области $D$ на плоскости. Пусть $\al(0)=\be(0)=A$ и
$\al(1)=\be(1)=B$, \те концы кривых совпадают. Будем говорить, что они \emph{гомотопны}, если существует
непрерывная деформация одной кривой в другую, \те отображение \eqn{\de\cln [0,1]\times[0,1] \ra D\quad
(t,\tau) \corr{\de} z,} такое, что $\de(t,0)=\al(t)$, а $\de(t,1)=\be(t)$, и при этом $\de(0,\tau)=A$,
$\de(1,\tau)=B$. Отображение $\de$ называют \emph{гомотопией}.
\end{df}

Смысл\dmvnpicr{pictures.30}{3} последних условий ясен: в начале деформации кривая совпадает с $\al(t)$, а в конце\т
с $\be(t)$. При этом концы кривых неподвижны.

Будем теперь рассматривать гомотопии замкнутых кривых.

\begin{df}
\dmvnpicrh{pictures.30}{-2}
Область называется \emph{односвязной}, если любая замкнутая кривая, целиком лежащая в ней, гомотопно
стягивается в точку.
\end{df}
На рисунке изображен пример неодносвязной области (гомеоморфной кольцу).

\dmvnpicrh{pictures.30}{-4}
Теперь фиксируем точку $A$ в области $D$ и рассмотрим отношение эквивалентности на
множестве путей в этой области с началом и концом в точке $A$. Будем говорить, что $\al \sim \be$, если
существует гомотопия $\de$, деформирующая один путь в другой. Легко видеть, что это действительно отношение
эквивалентности (оно симметрично, рефлексивно и транзитивно). Введём операцию умножения на множестве наших
путей. Произведением путей $\al$ и $\be$ назовём путь $\al\be$, являющийся объединением путей\д множителей
(пробегаемых последовательно с удвоенной скоростью\т сначала бежим по $\al$, потом по $\be$).

\begin{stm}
Множество классов эквивалентных с точностью до гомотопии путей с заданной операцией образует группу.
\end{stm}
\begin{proof}
Через $[\al]$ будем обозначать класс эквивалентности пути $\al$. У нас есть класс <<единичных>> путей
$[\ep]$, который состоит из всех путей, стягивающихся в точку $A$. Ассоциативность умножения
$[\al]\br{[\be][\ga]} \bw= \br{[\al][\be]}[\ga]$ очевидна, так как эти пути геометрически совпадают (а
значит, легко предъявить гомотопию, переводящую один из них в другой). Умножение на $[\ep]$ с любой стороны,
очевидно, даёт тот же класс эквивалентности. Таким образом, все аксиомы группы проверены.
\end{proof}

\begin{df}
Построенная группа называется \emph{фундаментальной} группой (или \emph{первой гомотопической} группой)
области $D$ и обозначается $\pi_1(D)$.
\end{df}

Можно показать, что $\pi_1(D)$ не зависит от выбора точки $A$ в области $D$.


\subsubsection{Формула Ньютона\ч Лейбница}

\begin{theorem}[формула Ньютона\ч Лейбница]
Пусть $f$ непрерывна в области $D$, и существует\footnote{То есть существует однозначная функция $F$.} первообразная $F$ к $f$ в области $D$;
$\ga=\hc{z(t) \vl t \in [a,b]}$\т кусочно\д гладкая кривая в $D$, соединяющая точки $A$ и $B$.
Тогда
\eqn{\ints{\ga}f(z)\,dz=F(B)-F(A).}
\end{theorem}
\begin{proof}
\eqn{\ints{\ga}f(z)\,dz = \intl{a}{b}f\br{z(t)}z'(t)\,dt= \intl{a}{b}\frac{d}{dt}\hs{F\br{z(t)}}\,dt=F\br{z(b)}-F\br{z(a)}=F(B)-F(A).}
\hfill\end{proof}
\begin{imp}
Если первообразная в односвязной области существует, то интеграл по замкнутому контуру, целиком лежащему в
этой области, равен нулю.
\end{imp}

\begin{ex}
Покажем, что односвязность важна:
\eqn{\ints{|z|=r}\frac{dz}{z}=\intl{0}{2\pi}\frac{ire^{it}}{re^{it}}\,dt=2\pi i\neq 0.}
\end{ex}


\subsubsection{Лемма Гурса и её следствия}

\begin{stm}
Если интеграл от непрерывной функции $f$ по любому замкнутому контуру в области~$D$ равен нулю, то
$f$ обладает первообразной.
\end{stm}
\begin{proof}
Пусть $\oints{\ga}f(z)\,dz=0$ для $\fa \ga$. Рассмотрим функцию
\eqn{F(z) := \intl{a}{z}f(\ze)\, d\ze.}
Интеграл здесь понимается как интеграл по кривой из точки $a$ в точку $z$.
Покажем, что такое задание функции корректно, \те не зависит от выбора пути интегрирования.
Пусть мы пошли другим путём $\wt{\ga}$ из $z$ в $a$. Тогда интеграл по контуру
$\ga \cup \wt{\ga}$, \те по $\ga$ (от $a$ до $z$), а потом по $\wt{\ga}$ (от $z$ до $a$) будет равен нулю.
Но это и значит, что
\eqn{(\ga)\intl{a}{z} f(\ze)\,d\ze = -(\wt{\ga})\intl{z}{a} f(\ze)\,d\ze = (\wt{\ga})\intl{a}{z} f(\ze)\,d\ze.}
Таким образом, в качестве пути интегрирования можно выбрать отрезок прямой.

Теперь покажем, что $F$ является первообразной для $f$. Пусть $\De z$ столь мало, что $z + \De z$ не вылезает за пределы области.
Имеем
\eqn{\frac{F(z+\De z)-F(z)}{\De z} = \frac{1}{\De z}\hr{\intl{a}{z+\De z} - \intl{a}{z}}f(\ze)\, d\ze = \frac{1}{\De z} \intl{z}{z+\De z}f(\ze)\,d\ze = (*).}
Так как $f$ непрерывна, то $f(\ze) = f(z) + \al(\ze)$. Тогда данное равенство преобразуется к виду
\eqn{(*)=\frac{1}{\De z}\hr{\intl{z}{z+\De z}f(z)\,d\ze + \intl{z}{z+\De z}\al(\ze)\,d\ze} = f(z) + \frac{1}{\De z}\intl{z}{z+\De z}\al(\ze)\,d\ze.}
Покажем, что второе слагаемое есть $o(1)$ при $\De z \ra 0$. По оценочному свойству интеграла
оно не превосходит длины пути интегрирования (\те $|\De z|$), умноженной на $\max|\al|$. Но оба множителя стремятся к нулю
при $\De z \ra 0$. Значит, $F'(z)=f(z)$.
\end{proof}

\begin{lemma}[E.\,Goursat]
Пусть $\tri$\т треугольник, лежащий в области $D$ вместе с внутренностью, и $f$ голоморфна в
окрестности $\tri$. Тогда
\eqn{\ints{\tri}f(z)\,dz=0.}
\end{lemma}
\begin{proof}
Обозначим $\tri_0 := \tri$, $P_0$\т периметр $\tri_0$, а
\eqn{I_0:=\Bm{\ints{\tri}f(z)\,dz}.}

\dmvnpicrh{pictures.40}{-8}
Поделив\dmvnpicra{pictures.40}{4}{-1.5pc} стороны $\tri_{0}$ пополам и соединив между собой точки деления, получим ещё четыре треугольника.
Интеграл по всему контуру равен сумме интегралов по всем четырём треугольникам. Значит, хотя бы одно слагаемое не меньше
четверти интеграла~$I_0$. Обозначим его $I_1$, а соответствующий контур\т через $\tri_1$. Итак, имеем
\eqn{I_1 := \Bm{\ints{\tri_1}f(z)\,dz} \ge \frac{I_0}{4}.}
Продолжая процесс, получим последовательность вложенных треугольников
$\tri_0 \sups \tri_1 \sups \ldots \sups \tri_n \ra a$, где $a$\т некоторая точка в треугольнике, причём
\eqn{I_n\ge \frac{I_0}{4^n}.}

Имеем $f(z)=f(a)+f'(a)(z-a)+o(z-a)  = f(a) + f'(a)(z-a) + \al(z)(z-a)$.
У линейной функции $f'(a)(z-a)$ есть первообразная, поэтому её интеграл по $\tri_n$ равен нулю. Значит, вклад
в интеграл может дать только нелинейное слагаемое $o(\dots)$. Далее заметим, что $(z-a)$ не превосходит периметра
треугольника, по которому мы интегрируем. Учитывая это, имеем

\eqn{\frac{I_0}{4^n} \le I_n = \Bm{\ints{\tri_n}o(z-a)\,dz} = \Bm{\ints{\tri_n}\al(z)(z-a)\,dz} \le \frac{P_0}{2^n} \cdot \frac{P_0}{2^n} \maxl{\tri_n}|\al(z)| \ra 0.}

Умножая неравенство на $4^n$, получаем $I_0 \le P_0^2\cdot\maxl{\tri_n}|\al(z)| \ra 0$. Значит, $I_0 = 0$.
\end{proof}
\begin{imp}
Если $D$\т односвязная область, а $f(z)$ голоморфна в $D$ и $\ga$\т замкнутая ломаная в $D$, то
\eqn{\ints{\ga}f(z)\,dz=0.}
\end{imp}
\begin{proof}
Разобьём многоугольник, образованный ломаной, на треугольники. По лемме Гурса интеграл по каждому из них равен нулю,
но ориентации на <<стыках>> разные, поэтому суммарный интеграл равен нулю.
\end{proof}
\begin{imp}
Если область $D$ односвязна, $f$ голоморфна в $D$, то у $f$ существует первообразная в $D$.
\end{imp}
\begin{proof}
Покажем, что интеграл по любому кусочно\д гладкому контуру равен нулю. Рассмотрим произвольную замкнутую
кривую в области, аппроксимируем её звеньями ломаных. Для замкнутой ломаной по первому следствию интеграл равен
нулю. Значит, для кривой он тоже равен нулю.
\end{proof}
\begin{imp}[интегральная теорема Коши]
Если $D$\т односвязная область, и $f(z)$ голоморфна в $D$, а $\ga$\т кусочно\д гладкий контур в $D$, то
\eqn{\ints{\ga}f(z)\,dz=0.}
\end{imp}

Если потребовать несколько большего, то подобное утверждение можно доказать и прямым использованием
формулы Грина (в комплексном случае она доказывается путём разбиения интеграла на мнимую и действительную части).
Напомним формулировку теоремы Грина: если $D$\т область с кусочно\д гладкой границей, и функции $P(x,y), Q(x,y) \in \Cb^1$,
то для формы $\om := P(x,y)\,dx + Q(x,y)\,dy$ выполняется равенство
\eqn{\label{stokes}\ints{\pd D} \om = \ints{D} d\om.}

Сформулируем ещё один вариант интегральной теоремы Коши:
\begin{theorem}[Коши]
Пусть $D$\т ограниченная область c кусочно\д гладкой границей; функция $f(z)$ голоморфна в $D$ и $f\in \Cb^1(\ol D)$.
Тогда
\eqn{\ints{\pd D}f(z)\,dz=0.}
\end{theorem}
\begin{note}
Компоненты границы имеют согласованную ориентацию: при обходе вдоль границы область остаётся слева.
\end{note}
\begin{proof}
По формуле \eqref{stokes} имеем
\eqn{\ints{\pd D}f(z)\,dz = \ints{D}d\br{f(z)\,dz}.}
Распишем подробнее дифференциал: $df = \pf{f}{z}dz + \pf{f}{\ol z}d\ol z$. Так как функция голоморфна, то
$\pf{f}{\ol z} = 0$, то есть $df = \pf{f}{z}dz$.
Значит, $d\br{f(z)\,dz} = f'(z) \cdot dz \wg dz = 0$, так как $dz \wg dz = 0$. Отсюда следует,
что $\ints{\pd D}\om = 0$.
\end{proof}
\begin{note}
Условия теоремы заведомо выполнены, если $f(z)$ голоморфна в окрестности $\ol{D}$.
\end{note}

\subsection{Интегральная формула Коши и её следствия}

\subsubsection{Формула Коши}

\begin{theorem}[Интегральная формула Коши]
Пусть $D$\т область с кусочно\д гладкой границей $\ga$, и $f(z)$ голоморфна в $D$ и $f \in \Cb^1(\ol D)$. Тогда
\eqn{\label{Cauchy}f(z)=\frac{1}{2\pi i}\ints{\pd D}\frac{f(\ze)}{\ze-z}\,d\ze.}
\end{theorem}
\begin{proof}
Пусть пока область $D$ односвязна. Фиксируем точку $z\in D$. Рассмотрим область
$D_\ep := D \wo \ol{С}_\ep$, где~$С_\ep$\т круговая окрестность
точки~$z$ радиуса $\ep$. Рассмотрим функцию $g(\ze) := \frac{f(\ze)}{\ze-z}$. Она будет голоморфной в
окрестности $D_\ep$ как частное двух голоморфных функций, так как мы исключили $z$ из области рассмотрения.

Граница $C_\ep$ ориентирована по\д другому, значит, \eqn{\ints{\pd D_\ep}g\, d\ze = \ints{\pd D}g\,d\ze -
\ints{\pd C_\ep}g\,d\ze.} Так как функция $g$ голоморфна в $D_\ep$, то применима обычная теорема Коши, и
следовательно, левая часть обращается в нуль.

Имеем $f(\ze) = f(z) + \al(\ze)$, следовательно,
\eqn{\ints{\pd D}\frac{f(\ze)}{\ze-z}\,d\ze =
\ints{\pd C_\ep}\frac{f(\ze)}{\ze-z}\,d\ze =
\ints{\pd C_\ep}\frac{f(z)}{\ze-z}\,d\ze + \ints{\pd C_\ep}\frac{\al(\ze)}{\ze-z}\,d\ze.}
Первое слагаемое равно $f(z) \cdot \ints{\pd C_\ep}\frac{1}{\ze-z}\,d\ze = f(z) \cdot 2\pi i$. Отметим, что
оно не зависит от $\ep$. Покажем, что второе слагаемое стремится к $0$ при $\ep \ra 0$.
Функция $\al(\ze)$ является бесконечно малой при $\ep \ra 0$, а длина контура интегрирования равна $2\pi \ep$.
Поэтому имеем
\eqn{\Bm{\ints{\pd C_\ep} \frac{\al(\ze)}{\ze - z} \, d\ze} = O(\ep) \ra 0.}

Как было замечено, предельный переход не испортит первого слагаемого, так как в нем никакого $\ep$ нет.

Таким образом, $\ints{\pd D}\frac{f(\ze)}{\ze-z}\,d\ze = f(z)\cdot 2\pi i$, что и требовалось.
\end{proof}

А теперь обобщим теорему на случай неодносвязной области, граница которой состоит из конечного числа замкнутых кривых.
Идея доказательства проста: перейти от многосвязного контура к односвязному, сделав достаточное количество <<разрезов>>.
На берегах разрезов ориентация противоположная, поэтому значение интеграла по границе не изменится.

\begin{df}
Пусть граница области состоит из конечного числа замкнутых кривых $\ga_0 \sco \ga_n$. Пусть внешняя граница $\ga_0$
ориентирована против часовой стрелки, а все остальные\т по часовой стрелке, то есть при обходе контура область
остаётся слева. Такую границу мы будем называть \emph{ориентированной}.
\end{df}

\dmvnpicrh{pictures.45}{-6}
Докажем\dmvnpicra{pictures.45}{5}{-1.5pc} индукцией по числу компонент $\ga_i$. Покажем, как можно уменьшить
их число, не изменив значение интеграла. Фиксируем точку $X$ на границе $\ga_0$ и точку $Y$ на границе~$\ga_1$.
Как мы уже знаем, область линейно связна. Поэтому существует путь из точки $X$ в точку~$Y$ внутри области.
Выкинем из области этот путь вместе с маленьким <<каналом>> сколь угодно малой ширины, тогда число компонент
границы уменьшится. Теперь посмотрим, что будет с интегралом. У нас добавится две кривые: от $X$ до $Y$ по одной
стороне канала, и от $Y$ до $X$ по другой. Направление обхода на них противоположное, поэтому никакого суммарного вклада
в интеграл они не дадут. Продолжая далее разрезать область, придём к односвязному контуру, для которого теорема верна.

\begin{note}
Конечно, <<дырок>> в области может быть много больше, чем на рисунке, да и разрез может быть существенно кривее.
Но суть дела от этого не поменяется.
\end{note}


\subsubsection{Следствия интегральной формулы}

\begin{theorem}[о разложении в ряд]
Пусть $f(z)$ голоморфна в области $D$. Фиксируем точку $a \in D$. Пусть $r < \dist(a, \pd D)$.
Тогда функция $f(z)$ разлагается в степенной ряд
\eqn{f(z)=\suml{n=0}{\bes}c_n(z-a)^n, \text{ где } c_{n}= \frac{1}{2\pi i}\ints{|\ze-a|=r}\frac{f(\ze)}{(\ze-a)^{n+1}}\,d\ze.}
Этот ряд сходится абсолютно и равномерно при $|z-a| < r$.
\end{theorem}
\begin{proof}
Представим дробь $\frac{1}{\ze-z}$ в виде суммы геометрической прогрессии:
\eqn{\frac{1}{\ze-z}=\frac{1}{(\ze-a)-(z-a)} = \frac{1}{\ze-a}\cdot\frac{1}{1-\frac{z-a}{\ze-a}}=
\frac{1}{\ze-a}\cdot\suml{n=0}{\bes}\hr{\frac{z-a}{\ze-a}}^n = \suml{n=0}{\bes}\frac{(z-a)^n}{(\ze-a)^{n+1}}.}

Обозначим через $C$ окружность радиуса $r$ с центром в точке $a$.
Пусть $|\ze-a|=r$, \те $\ze$ бегает по окружности~$C$. Если $|z-a| < r$, то
$\frac{|z-a|}{|\ze-a|}  = \frac{|z-a|}{r} < 1$. Значит, в открытом круге радиуса~$r$
ряд сходится равномерно и его можно почленно интегрировать.

Представим функцию $f(z)$ формулой Коши, взяв в качестве контура окружность $C$, а затем подставим в интеграл
выражение для дроби $\frac{1}{\ze-z}$:
\eqn{f(z) = \frac{1}{2\pi i} \ints{C}\frac{f(\ze)}{\ze-z}\,d\ze =
\frac{1}{2\pi i}\ints{C} \suml{n=0}{\bes}\frac{f(\ze)(z-a)^n}{(\ze-a)^{n+1}}\,d\ze =
\suml{n=0}{\bes} \bbs{\frac{1}{2\pi i} \ints{C} \frac{f(\ze)}{(\ze-a)^{n+1}}\,d\ze}\cdot (z-a)^n.}
Обозначая выражения в скобках через $c_n$, получаем утверждение теоремы.
\end{proof}

\begin{imp}[Неравенство Коши]
Пусть выполнены условия теоремы, и $C_r$\т окружность радиуса~$r$ с центром в точке $a$.
Пусть $\maxl{C_r}|f(z)| = M$. Тогда $|c_n| \le \frac{M}{r^n}$.
\end{imp}
\begin{proof}
Имеем
\eqn{|c_{n}|= \bbm{\frac{1}{2\pi i}\ints{C_r} \frac{f(\ze)}{(\ze-a)^{n+1}}\,d\ze}\le
\frac{1}{2\pi}\cdot \frac{M}{r^{n+1}} \cdot 2\pi r = \frac{M}{r^n}.}
\hfill\end{proof}

\begin{imp}[Теорема Лиувилля]
Если $f(z)$ голоморфна в $\Cbb$ и ограничена, тогда $f(z) \equiv \const$.
\end{imp}
\begin{proof}
Пусть $|f(z)| \le M$. Голоморфность в $\Cbb$ означает голоморфность в круге сколь
угодно большого радиуса~$R$. По предыдущему следствию $|c_n| \le \frac{M}{R^n}$ для $\fa R$, значит,
$c_n = 0$ для $\fa n > 0$. Следовательно, $f = c_0$.
\end{proof}
\begin{imp}[Основная теорема алгебры]
Любой многочлен $P(z) \in \Cbb[z]$ положительной степени имеет в $\Cbb$ хотя бы один корень.
\end{imp}
\begin{proof}
Допустим противное, \те $P(z)\neq 0$ для $\fa z$. Тогда $|P(z)| \ge m > 0$.
Тогда функция $f(z) := \frac{1}{P(z)}$ всюду определена, а значит, голоморфна в $\Cbb$ и ограничена по модулю
числом $\frac{1}{m}$. По теореме Лиувилля $f(z)=\const$. Значит, $P(z)=\const$. Противоречие.
\end{proof}

\section{Немного о конформных отображениях}

\subsection{Дифференцируемость и конформность}

\subsubsection{Геометрический смысл комплексной производной}

Рассмотрим произвольную функцию $f(z)$, голоморфную в некоторой окрестности точки $a$. Пусть $f(a)=b$.
Рассмотрим гладкую кривую $z(t)\cln [-\ep,\ep]\ra \Cbb$, такую, что $z(0)=a$. Рассмотрим также образ
этой кривой при отображении $f$, то есть $f\br{z(t)}$.
Обозначим через $e$ касательный вектор к кривой $z(t)$, \те $e := \frac{dz}{dt}(0)$.
По правилам дифференцирования сложной функции
\eqn{\frac{df\br{z(t)}}{dt}(0)= df\evn{a}e.}
В этой формуле все корректно: полный дифференциал есть комплексное число, и двумерный вектор\т тоже
комплексное число.

Таким образом, мы получаем геометрический смысл комплексной производной (точнее, комплексного дифференциала):
происходит гомотетия с коэффициентом $|df|$ и поворот на угол $\arg df$. При этом сохраняются углы и ориентация.

\begin{df}
Линейное отображение, сохраняющее углы и ориентацию, называется \emph{конформным}.
\end{df}

\subsubsection{Свойства конформных отображений}
\label{ConformMappings}
\begin{prop}
Невырожденное линейное преобразование $L\cln \Cbb\ra\Cbb$ сохраняет углы и ориентацию
тогда и только тогда, когда оно является комплексным умножением, то есть
$L(z) = \la z$, где $\la \neq 0$.
\end{prop}
\begin{proof}
\dmvnpicrh{pictures.50}{-8}
Справа\dmvnpicra{pictures.50}{6}{-1.5pc} налево утверждение очевидно. Докажем, что из конформности следует линейность.
Возьмём преобразование, удовлетворяющее этим свойствам. Найдём такое комплексное умножение
$M$, что $M \circ L = \id$, то есть найдём обратное отображение, тогда само $L$ будет комплексным умножением.
Рассмотрим положительно ориентированный ортонормированный репер
$\hc{e_1, e_2}$ и его образ ${e'_1, e'_2}$ при отображении $L$. Образ также будет положительно
ориентированным и ортогональным. Покажем, что длины обоих векторов изменились в одинаковое число раз.
Действительно, пусть один из векторов (для определённости $e_2$) растянулся больше.
Имеем $\angle (e_1, e_1+e_2) = \frac{\pi}{4}$. Если $e'_2$ стал
длиннее $e'_1$, то $\angle(e'_1,e'_1+e'_2) \neq \frac{\pi}{4}$, а углы должны сохраняться. Противоречие.

Преобразованием вида $\la z$ можно добиться того, что $e'_1 \mapsto e_1$. Тогда, очевидно, $e'_2 \mapsto e_2$.
Таким образом, мы нашли обратное преобразование, являющееся умножением. Значит, $L(z) = \frac{1}{\la}z$.
\end{proof}

Как уже отмечалось выше, каждому комплексно\д линейному преобразованию $L_\Cbb$ можно сопоставит вещественное
преобразование $L_\R\cln\R^2\ra\R^2$, называемое \emph{овеществлением}. Пусть $L_\Cbb(z) = \la z$, и $\la = \al+i\be$.
Тогда $L_\Cbb(x+iy) = (\al+i\be)(x+iy)=(\al x-\be y)+i(\be x+\al y)$, а значит, матрица овеществления имеет вид
\eqn{\label{Realize}M_\R = \rbmat{\al & -\be \\ \be & \al}.}

Подытожим всё вышесказанное о конформных отображениях.

\begin{stm}
Рассмотрим невырожденное линейное отображение $L\cln \R^2 \ra \R^2$ . Следующие утверждения эквивалентны:
\begin{items}{-2}
\item Матрица овеществления $L$ имеет вид \eqref{Realize};
\item $L$ является комплексным умножением;
\item $L$ есть композиция гомотетии и поворота;
\item $L$ сохраняет углы и ориентацию.
\end{items}
\end{stm}

\begin{df}
Отображение $f(z)$ называется \emph{конформным}, если существует его дифференциал $df$ и он является
конформным отображением. Говорят, что функция $f\cln D_1 \ra D_2$ осуществляет конформное отображение
областей $D_1$ и $D_2$, если она биективна и конформна в каждой точке.
\end{df}

\begin{note}
Чтобы можно было говорить о сохранении углов при произвольном отображении, нужно требовать
хотя бы $\R$\д дифференцируемости. В определении явно не указано, какая дифференцируемость требуется.
Однако следующее утверждение показывает, что $\R$\д дифференцируемости и конформности дифференциала
хватает для того, чтобы обеспечить существование комплексной производной.
\end{note}

\begin{stm}
Отображение $f$ конформно в точке $a$ тогда и только тогда, когда $\exi f'(a)$ и $f'(a) \neq 0$.
\end{stm}
\begin{proof}
Если есть производная в точке $a$, и она не равна нулю, то есть и дифференциал, который является комплексным умножением,
а потому сохраняет углы. Значит, отображение конформно по определению.

Наоборот: пусть отображение $f$ конформно в точке $a$. Рассмотрим его вещественный дифференциал. По условию он является
конформным отображением, значит, по доказанному выше предложению он является умножением на комплексное число,
отличное от нуля. Значит, комплексная производная и будет равна этому числу, \те  мы установили голоморфность
функции в точке $a$.
\end{proof}

Очевидно, что если имеется конформное отображение областей $D_1 \mapsto D_2$, то обратное отображение
также будет конформным: если $f'(z) \neq 0$ для $\fa z \in D_1$, то производная обратной функции
$g' =\frac{1}{f'}$ будет везде определена и также не будет нигде обращаться в нуль.
Значит, имеется отношение эквивалентности на множестве областей: $D_1 \sim D_2$, если найдётся конформное
отображение $D_1$ на $D_2$. Транзитивность следует из теоремы о производной сложной функции.

\subsection{Стереографическая проекция сферы}

Множество комплексных чисел обладает двумя недостатками: оно некомпактно и не имеет бесконечно удалённой
точки, которая иногда оказывается очень полезной. Сейчас мы устраним оба недостатка с помощью
операции, которую иногда называют \emph{проективизацией} (или \emph{компактификацией}).

Рассмотрим стереографическую проекцию комплексной плоскости на двумерную сферу: кладём сферу на плоскость так,
чтобы она касалась плоскости в нуле. Через каждую точку $z\in \Cbb$ проводим прямую, соединяющую её с
северным полюсом $N$. Эта прямая проткнёт сферу в некоторой точке $\Ph(z)$. Чем больше модуль числа $z$, тем
ближе к полюсу будет находиться $\Ph(z)$. Добавим к $\Cbb$ еще одну точку (бесконечно удалённую) и положим
$\Ph(\bes) := N$. Таким образом, мы получили биективное отображение множества
$\ol{\Cbb} := \Cbb \cup \hc{\bes}$ на сферу $S^2$. Введём топологию на расширенном множестве: окрестностями $\bes$
будем называть множества точек $U(\bes) := \hc{z \in \ol{\Cbb} \cln |z| > r}$. В такой топологии отображение $\Ph$
будет непрерывным (так как прообраз открытого множества открыт). Более того, очевидно, что это будет гомеоморфизм.

Полученное множество представляет собой не что иное, как комплексную проективную прямую $\CP^1$. Стереографическая
проекция даёт возможность построить две карты $\CP^1$, изоморфные обычной комплексной плоскости. А именно, рассмотрим
проекции из северного и южного полюсов. При проекции из северного полюса точка $\bes$ оказывается несобственной
по отношению к одной из карт, а при проекции из южного полюса роль бесконечно удалённой точки играет нуль.
Легко поверить, что функции перехода между картами будут $\Cbb$\д диффеоморфизмами. Тем самым мы показали, что $\ol{\Cbb} \cong \CP^1 \cong S^2$
как гладкое компактное комплексное одномерное многообразие.

\begin{problem}
Докажите, что стереографическая проекция сферы на плоскость сохраняет углы.
\end{problem}

К конформным отображениям мы еще вернёмся, а пока займёмся рядами\ldots

\section{Степенные ряды и голоморфные функции}

\subsection{Степенные ряды}

\subsubsection{Напоминание некоторых фактов о рядах}

Рассмотрим степенные ряды вида $\sumnzi c_n(z-a)^n$.
Сделав подходящую замену, можно  всё свести к рядам вида
\eqn{\label{CenteredSeries}\sumnzi c_n z^n.}

\begin{lemma}[Абеля]
Рассмотрим ряд \eqref{CenteredSeries}. Пусть $\hm{c_n z_0^n} \le M$, где $z_0 \neq 0$. Тогда $\fa q \in (0, |z_0|)$ этот
ряд сходится абсолютно и равномерно при $|z| \le q$.
\end{lemma}
\begin{proof}
В самом деле,
\eqn{\hm{c_n z^n}=\Bm{c_n z_0^n}\cdot\Bm{\frac{z}{z_0}}^n \le \Bm{c_n z_0^n} \cdot \Bm{\frac{q}{z_0}}^n < M\cdot \Bm{\frac{q}{z_0}}^n.}
Если $\bm{\frac{q}{z_0}} < 1$, ряд $\sum M \bm{\frac{q}{z_0}}^n$ сходится (геометрическая прогрессия).
Далее  применим признак Вейерштрасса.
\end{proof}

\begin{note}
Условие $|z| \le q < |z_0|$ существенно. Рассмотрим ряд $\sum \frac{z^k}{k}$, $z_0=1$.
Допустим, что он сходится равномерно при всех $z\cln |z|<1$ и запишем условие
Коши: $\fa \ep > 0$ найдётся $N\cln \fa n,m>N, \fa z\cln |z| < 1$ имеем
$\Bm{\suml{m}{n} \frac{z^k}{k}} < \ep$. Перейдём к пределу при $z \ra 1$. Тогда
$\Bm{\suml{m}{n} \frac{1}{k}} \le \ep$, а
это противоречит расходимости $\sum \frac{1}{k}$.
\end{note}

По \emph{формуле Коши\ч Адамара} радиус сходимости степенного ряда вычисляется так:
\eqn{R=\frac{1}{{\uliml{n\ra\bes}\sqrt[n]{|c_n|}}}.}

Как и в вещественном случае, справедливо
\begin{stm}
Равномерно сходящиеся степенные ряды можно почленно дифференцировать и интегрировать
без изменения радиуса сходимости. Если ряд из производных равномерно сходится, а ряд сходится хотя бы в одной точке, то эта сходимость равномерна,
ряд сходится к дифференцируемой функции, и производная суммы равна сумме производных.
\end{stm}

Из замечания уже должно быть понятно, что на границе круга сходимости может быть всё, что угодно.


\begin{theorem}
Сумма степенного ряда
\eqn{ f(z) = \sum_{n=0}^{\infty}c_n(z-a)^n\label{rjad}}
голоморфна в круге его сходимости.
\end{theorem}
\begin{proof}
Пусть радиус сходимости $R>0$, иначе нечего доказывать. Напишем ряд производных
\eqn{ \ph(z) = \sum_{n=1}^{\infty}nc_n(z-a)^{n-1}}

В силу формулы Коши\ч Адамара радиус сходимости этого ряда тоже равен $R$. На компактных подмножествах круга сходимости ряд \eqref{rjad} сходится равномерно, следовательно, функция $\ph$ непрерывна в нём и ряд можно интегрировать по границе любого треугольника $\tri$, лежащего в круге:
\eqn{\int\limits_{\pd\tri}\ph\,dz=\sum_{n=1}^{\infty}nc_n\int\limits_{\pd\tri}(z-a)^{n-1}dx=0}
(по Теореме Коши все интегралы в правой части равны нулю). Значит, у $\ph$ есть первообразная
\eqn{\int\limits_{[a,z]}\ph(\ze)\,d\ze=\sum\limits_{n=1}^{\bes}nc_n\int\limits_{[a,z]}(\ze-a)^{n-1}d\ze=\sum_{n=1}^{\bes}c_n(z-a)^n=\sum_{n=0}^{\bes}c_n(z-a)^n - c_0=f(z)-c_0}
Следовательно $f'(z) = \ph(z)$. \end{proof}




\subsection{Голоморфные функции и их свойства}

\subsubsection{Эквивалентные определения голоморфности. Теорема Мореры}

\begin{theorem}
Пусть $f(z)$ непрерывна в области $D$. Следующие 4 условия эквивалентны:
\begin{nums}{-1}
\item Для $\fa z\in D$ существует производная $f'(z)$.
\item $\oints{\ga}f(z)\,dz=0$ для любой замкнутой кривой $\ga$, содержащейся вместе с внутренностью в $D$.
\item Для каждого простого замкнутого контура $\ga$, содержащегося вместе с внутренностью в $D$ и для $\fa z\bw\in \Int \ga$
имеет место формула Коши $(2.\ref{Cauchy})$.
\item Для $\fa a \in D$ найдётся $r > 0$ такое, что в круге $\De(a,r)$ функция $f(z)$ разлагается в ряд
\eqn{f(z) = \sumnzi c_n(z-a)^n.}
\end{nums}
\end{theorem}
\begin{proof}
$1 \Ra 2$\т по теореме Коши, $2 \Ra 3$\т по формуле Коши, $3\Ra4$\т по теореме о разложении в ряд.

$4 \Ra 1$: по соответствующей теореме ряд можно дифференцировать почленно, и радиус сходимости не изменится.
Из матана известно, что у такого ряда существуют обе частные производные по $x$ и по $y$.
Вот они:
\eqn{\pf{f}{x} = \sumnui n c_n \br{(x+iy)-a}^{n-1}, \quad \pf{f}{y} = i\sumnui n c_n \br{(x+iy)-a}^{n-1}.}
Мы видим, что они отличаются только множителем $i$. Теперь вспомним, что $\pf{}{\ol{z}}= \frac12\hr{\pf{}{x}+ i\pf{}{y}}$.
Но в нашем случае
\eqn{\pf{f}{\ol{z}} = \frac12\hr{\pf{f}{x}+ i\pf{f}{y}} = \frac12\hr{\sum + i^2 \sum} = \frac12\hr{\sum-\sum}=0,}
а это и означает голоморфность функции $f$.
\end{proof}

\begin{note}
Импликация $2 \Ra 1$ называется \emph{теоремой Мореры}. Доказательство не требуется, так как по предыдущей теореме
можно получить эту импликацию обходным путём: $2 \Ra 3 \Ra 4 \Ra 1$.
\end{note}

\subsubsection{Множество голоморфных функций и его свойства}

Множество функций, голоморфных на множестве $D$, будем обозначать $\Oc(D)$.
Легко видеть, что это векторное пространство над полем $\Cbb$, а также коммутативное кольцо.
Стало быть, это коммутативная алгебра.

\begin{df}
Сходимость в $\Oc(D)$\т это равномерная сходимость на компактных подмножествах.
\end{df}

\begin{problem}
Задать топологию в $\Oc(D)$, соответствующую этой сходимости.
\end{problem}

\begin{stm}
Если $f \in \Oc(D)$, то $f \in \Cb^\bes(D)$.
\end{stm}
\begin{proof}
По доказанной выше теореме голоморфная функция представляется степенным рядом, производная которого
имеет тот же радиус сходимости. Значит, можно применить теорему к производному ряду, и снова получим
тот же радиус, а значит, $f' \in \Oc(D)$, $f'' \in \Oc(D)$, и так далее.
\end{proof}

\begin{stm}
Если $f$ голоморфна в окрестности точки $z_0$, то коэффициенты разложения в ряд по степеням $(z-z_0)$
определяются по формулам $c_n  = \frac{1}{n!} f^{(n)}(z_0)$.
\end{stm}
\begin{proof}
Имеем $f(z) = \sumnzi c_n (z-z_0)^n$. Дифференцируя ряд и подставляя $z=z_0$,
получаем $c_1 = f'(z_0)$, потом дифференцируем еще раз, получаем $c_2 = \frac12 f''(z_0)$, и так далее.
\end{proof}

\begin{note}
Если в разложении голоморфной функции $f$ в ряд все коэффициенты равны $0$, то $f \equiv 0$. Как мы знаем,
для $\R$\д дифференцируемых функций это неверно: функция $\ph(x) := \exp\hr{-\frac{1}{x^2}}$ бесконечно дифференцируема,
но при попытке разложения в ряд Тейлора в нуле получаются нулевые коэффициенты. И это не удивительно: у функции
$\ph(z) = \exp \hr{\frac1{z^2}}$ нуль является существенно особой точкой.
\end{note}

\begin{theorem}[Вейерштрасса]
Если $f_n \in \Oc(D)$ и $f_n \convu{D} f$, то
$f \in \Oc(D)$, $f_n' \convu{D} f'$ и
\eqn{\ints{\ga}f_n(z)\,dz \ra \ints{\ga}f(z)\,dz}
для любого кусочно\д гладкого контура $\ga \subs D$.
\end{theorem}

\begin{proof}
Так как $\ga$\т компакт, то сходимость интегралов следует из равномерной сходимости функций на $\ga$.
Но так как $f_n \in \Oc(D)$, то $\ints{\De} f_n\,dz=0$. Мы уже доказали, что интегралы сходятся равномерно.
Значит, и $\ints{\De} f\,dz=0$. Непрерывность $f$ очевидна. Значит, можно применить теорему Мореры,
и таким образом $f \in \Oc(D)$.

Теперь покажем, что $f'_n \rra f'$. Рассмотрим произвольный простой кусочно\д гладкий контур внутри $D$ и для
него запишем формулу Коши:
\eqn{f_n(z)=\frac{1}{2\pi i}\ints{\ga}\frac{f_n(\ze)}{\ze-z}d\ze.}

По определению равномерной сходимости, нам нужно показать, что $f'_n \rra f'$ на любом компактном подмножестве.
Рассмотрим какой\д нибудь компакт $K$, не пересекающийся с $\ga$ (если он пересекается, возьмём другую кривую $\ga$).
Возьмём точку $z \in K$. Поскольку кривая $\ga$ компактна, то расстояние от $z$ до $\ga$ не равно нулю.
В интегральной формуле в знаменателе находится разность $\ze-z$, где $\ze \in \ga$, но так как кривая не подходит
очень близко к компакту, то дробь $\frac{1}{\ze-z}$ ограничена, а значит, зависимость интеграла от
параметра $z$ равномерна. Поэтому интеграл можно продифференцировать по параметру $z$ и переходить к пределу.
Имеем:
\eqn{f_n'(z)= \frac{1}{2\pi i}\ints{\ga}\frac{f_n(\ze)}{(\ze-z)^2}\,d\ze \convu{\text{по } z} \frac{1}{2\pi i}\ints{\ga}\!\frac{f(\ze)}{(\ze-z)^{2}}d\ze = f'(z).}
\hfill\end{proof}

Теорема Вейерштрасса показывает, что множество $\Oc(D)$ замкнуто относительно этой сходимости.

\subsubsection{Теорема единственности и её следствия}

\begin{theorem}[единственности]
Пусть $f \in \Oc(D)$, $a_{n} \in D$, $a_n \ra a \in D$ и $f(a_n)=0$. Тогда $f(z) \equiv 0$.
\end{theorem}
\begin{proof}
Разложим $f(z)$ в ряд в окрестности точки $a$: $f(z)=\sumnzi c_n(z-a)^n$.
Пусть не все коэффициенты равны нулю. Тогда найдётся номер $k$ такой, что
\eqn{\label{UniquenessTheorem}f(z) = c_k(z-a)^k(1 + \ub{\wt c_{k+1}(z-a) + \ldots}_{\ph(z)}).}
Ряд для $\ph(z) = \wt c_{k+1}(z-a) + \wt c_{k+2}(z-a)^2 + \ldots$ сходится в таком же круге, как и исходный
по формуле Коши\ч Адамара. Имеем $\ph(a) = 0$, значит, $(1+\ph(a))=1$. В силу непрерывности $(1+\ph(z))\neq 0$ в
малой окрестности точки~$a$. Кроме того, множитель $(z-a)^k \neq 0$ при $z \neq a$.
Значит, нашлась такая (проколотая) окрестность точки~$a$, в которой $f(z) \neq 0$. Это противоречит тому, что
$f(z) = 0$ в точках $a_n$, находящихся в любой окрестности точки~$a$.

Таким образом, доказано, что если есть нули по некоторой последовательности, то будет
тождественный нуль в некоторой окрестности. Докажем, что в любой точке $b$ области $D$ функция
равна нулю. Область $D$ линейно связна, поэтому точки $a$ и $b$ можно соединить кривой $z(t)$.
Пусть $K := \hc{t \in [0,1]\cln f\br{z(t)} = 0}$. Оно, очевидно, не пусто ($0 \in K$), открыто (по доказанному)
и замкнуто (по непрерывности $f$), а значит, совпадает со всем отрезком $[0,1]$. Таким образом, $f(b) = f\br{z(1)} = 0$.
\end{proof}
\begin{imp}
Если $f \in \Oc(D)$ и $f(z)\equiv 0$ в некоторой окрестности, то $f(z)\equiv 0$ в $D$.
\end{imp}
\begin{imp}
Если $f \in \Oc(D)$ и $f(z) \not\equiv 0$, то все нули функции $f$ изолированные.
\end{imp}
\begin{imp}
Если функции $f,g \in \Oc(D)$ совпадают на последовательности точек $a_n \ra a \in D$, то $f \equiv g$.
\end{imp}
\begin{proof}
Рассмотрим разность $\ph=f-g$ и применим к ней теорему единственности.
\end{proof}

\begin{theorem}[принцип максимума]
Пусть $f(z)\in\Oc(D)$ и $a \in D$. Если $|f(z)| \le |f(a)|$ при $z \in U(a)$, то $f \equiv \const$.
\end{theorem}
\begin{proof}
\dmvnpicrh{pictures.60}{-10}
Если\dmvnpicra{pictures.60}{7}{-1.5pc} $f(a)=0$, то $f(z)\equiv 0$ по теореме единственности. Пусть $f(a) \neq 0$.
Разложим $f(z)$ в ряд в окрестности точки $a$. Пусть $f \neq \const$ и
\eqn{f(z)=A+B(z-a)^k\br{1 + (z-a)\ph(z)},}
где $B$\т первый ненулевой коэффициент при положительной степени $(z-a)$. Заметим, что $A = f(a) \neq 0$.
Если~$z$ лежит достаточно близко к~$a$, то выражение в скобках очень близко к $1$, то есть его аргумент близок к нулю.
Значит, при умножении $B(z-a)^k$ на скобку аргумент этого числа изменится совсем чуть\д чуть.
Теперь подберём число $z$ так, чтобы вектор $B(z-a)^k$ смотрел в ту же сторону, что и $A$.
Пусть $(z-a)=re^{i\ph}$. Угол $\ph$ надо взять так, чтобы $\arg B+k\ph=\arg A$, то есть $\ph = \frac{\arg A-\arg B}{k}$.

Из рисунка видно, что можно за счёт выбора числа $r$ сделать круг, в котором бегает $f(z)$, достаточно маленьким.
Значит, $|f(z)|$ будет больше, чем $|f(a)|$. Умножение на скобку, которая <<почти>> равна $1$, явно не помешает.
Но по условию $|f(z)| \le |f(a)|$. Противоречие, следовательно $f = \const$.
\end{proof}

\begin{imp}
Пусть $D$\т ограниченная область, $f \in \Oc(D) \cap \Cb(\ol{D})$. Тогда $|f|$ достигает максимума
на границе области, \те
\eqn{\maxl{\ol{D}}|f|=\maxl{\pd D}|f|.}
\end{imp}
\begin{imp}
Пусть $D$\т ограниченная область, и $f,g \in\Oc(D) \cap \Cb(\ol{D})$. Если функции совпадают на~границе области,
то они совпадают и во всей области.
\end{imp}
\begin{proof}
В самом деле, рассмотрим модуль разности $|f - g|$. На границе он равен нулю, значит, $\maxl{\ol D} |f-g| = 0$.
Значит, $|f-g|=0$, \те $f=g$.
\end{proof}

\begin{theorem}[о среднем]
Пусть $f \in \Oc(D)$. Значение функции $f$ в центре круга (целиком лежащего в~области $D$) равно среднему значению
этой функции на окружности.
\end{theorem}
\begin{proof}
Пусть функция $f$ голоморфна в области $D$. Рассмотрим окружность $C_r$ с центром в точке $a$ и радиусом~$r$,
целиком лежащую в $D$. По формуле Коши
\eqn{f(a) = \frac{1}{2\pi i}\ints{C_r} \frac{f(\ze)}{\ze-a}\, d\ze.}
Параметризуем окружность углом $\ta$, \те $(\ze - a) = r e^{i\ta}$. Тогда
\eqn{f(a) = \frac{1}{2\pi i}\intl{0}{2\pi} \frac{f\hr{a + re^{i\ta}}i r e^{i\ta}}{r e^{i\ta}}\,d\ta =
\frac{1}{2\pi} \intl{0}{2\pi}f\hr{a+re^{i\ta}}\,d\ta.}
Это и есть среднее значение функции на окружности.
\end{proof}

\subsection{Ряды Лорана}

\subsubsection{Определения и свойства}

Рассмотрим так называемые \emph{ряды Лорана} вида
\eqn{\suml{n=-\bes}{+\bes}c_n(z-a)^n.}

\begin{df}
Сумма $\sumnzi c_n(z-a)^n$ называется \emph{регулярной частью} ряда Лорана, а $\suml{n=-\bes}{-1}$\т
\emph{главной частью}. \emph{Сходимость} лорановских рядов понимается как сходимость (по отдельности)
главной и регулярной части.
\end{df}

Регулярная часть\т это обычный степенной ряд. У него есть круг сходимости $|z-a| < R$.
Главная часть становится степенным рядом после замены $\ze = \frac{1}{z-a}$. Значит,
он сходится вне некоторого круга $|z-a| > r$. Таким образом, ряд Лорана
имеет \emph{кольцо сходимости} $r < |z-a| < R$. Если $r\ge R$, то кольцо сходимости пусто, а иначе
имеется непустое кольцо сходимости $0\le r < R \le +\infty$ (крайние случаи возможны).
Будем обозначать кольцо сходимости через $\Kc(r,R)$.

Иногда для красоты и краткости будем писать $\sums{\Z}$ вместо $\suml{n=-\bes}{+\bes}$.

\begin{stm}
Пусть $\sums{\Z}c_n(z-a)^n$ имеет непустое кольцо сходимости $\Kc(r,R)$, тогда
\begin{nums}{-2}
\item Сумма ряда для $f(z)$ голоморфна в $\Kc(r,R)$;
\item Если $\ga$\т кусочно\д гладкий контур в $\Kc(r,R)$, то ряд можно почленно интегрировать;
\item Ряд можно почленно дифференцировать.
\end{nums}
\end{stm}
\begin{proof}
Это следует из соответствующего утверждения для степенных рядов, применённого к регулярной и
главной частям по отдельности. Наш ряд сходится в $\Kc(r,R)$ равномерно на компактных
подмножествах и утверждение следует из теоремы Вейерштрасса.
\end{proof}

\subsubsection{Теорема Лорана и единственность разложения}

\begin{theorem}[Лорана]
Пусть $\Kc(a;r,R)$\т непустое кольцо, и $f \in \Oc(\Kc)$. Пусть $C_\rh$\т окружность радиуса $\rh \in (r, R)$ с
центром в точке $a$. Тогда $f$ представляется сходящимся в $\Kc$ рядом Лорана
\eqn{f(z)=\sums{\Z}c_n(z-a)^n, \text{ где } c_n = \frac{1}{2\pi i} \ints{C_\rh} \frac{f(\ze)}{(\ze-a)^{n+1}}\,d\ze, \quad n \in \Z.}
\end{theorem}
\begin{proof}
\dmvnpicrh{pictures.70}{-6}
Возьмём\dmvnpicra{pictures.70}{8}{-1.5pc} кольцо поменьше $\Kc'(a;r',R')$, где $r < r' < R' < R$. Через $c'$ и $C'$ обозначим окружности
радиусов $r'$ и $R'$ соответственно. Для $z\in \Kc'$ имеет место формула Коши
(для \emph{многосвязной} области!):
\eqn{f(z)=\frac{1}{2\pi i}\ints{\pd \Kc'}\frac{f(\ze)}{\ze-z}\,d\ze=
\frac{1}{2\pi i}\ints{C'}\frac{f(\ze)}{\ze-z}\,d\ze - \frac{1}{2\pi i}\ints{c'}\frac{f(\ze)}{\ze-z}\,d\ze =: I_1 - I_2.}

\dmvnpicrh{pictures.70}{-2}
Рассмотрим каждое слагаемое в отдельности. Из $I_1$ получим регулярную часть, а из $I_2$\т главную.

\pt{1} В первом интеграле $\ze$ находится дальше от центра, чем $z$. Представим дробь под интегралом в виде ряда:
\eqn{\frac{1}{\ze-z} = \frac{1}{(\ze-a)-(z-a)} =
\frac{1}{\ze-a} \cdot \frac{1}{1 - \frac{z-a}{\ze-a}} = \sumnzi \frac{(z-a)^n}{(\ze-a)^{n+1}}.}
В силу равномерной сходимости этот ряд можно почленно интегрировать:
\eqn{I_1 = \frac{1}{2\pi i} \ints{C'}\frac{f(\ze)}{\ze-z}\,d\ze =
\frac{1}{2\pi i}\ints{C'} \suml{n=0}{\bes}\frac{f(\ze)(z-a)^n}{(\ze-a)^{n+1}}\,d\ze =
\frac{1}{2\pi i} \suml{n=0}{\bes}
\ints{C'} \frac{f(\ze)}{(\ze-a)^{n+1}}\,d\ze \cdot (z-a)^n.}
Осталось обозначить интеграл в скобках через $c_n$.

\pt{2} Во втором слагаемом $z$ находится дальше от центра, чем $\ze$, поэтому будем выносить множитель $\frac{1}{z-a}$:
\eqn{\frac{1}{z-\ze} = \frac{1}{(\ze-a)-(z-a)} =
 \frac{1}{z-a} \cdot \frac{1}{1 - \frac{\ze-a}{z-a}} = \sumnzi \frac{(\ze-a)^n}{(z-a)^{n+1}}.}
После почленного интегрирования получаем второе слагаемое:
\eqn{I_2 = \frac{1}{2\pi i} \sumnzi \ints{c'}f(\ze)(\ze-a)^n\,d\ze \cdot \frac{1}{(z-a)^{n+1}}.}
Обозначим $m = -(n+1)$. Тогда, обозначая интегралы в скобках через $c_m$, получим $I_2 = \suml{m=-\bes}{-1} c_m (z-a)^m$.

В силу того, что $f \in \Oc(\Kc)$, радиусы $R'$ и $r'$ можно брать любыми в пределах от $r$ до $R$.
\end{proof}
\begin{imp}[неравенство Коши]
Пусть $C_\rh$\т окружность радиуса $\rh$ с центром в точке $a$, и функция~$f(z)$ представляется рядом
Лорана: $f(z) = \sums{\Z}c_n(z-a)^n$. Тогда $|c_n| \le \frac{M(\rh)}{\rh^n}$ при $n \in \Z$,
где $M(\rh)=\maxl{C_\rh}|f(z)|.$
\end{imp}

\begin{theorem}[единственность разложения в ряд Лорана]
Пусть $f(z)$ имеет два представления $f(z)=\sums{\Z}c_n(z-a)^n = \sums{\Z}d_n(z-a)^n$ в непустом кольце $\Kc(a;r,R)$.
Тогда $c_n=d_n$ для $\fa n \in \Z$.
\end{theorem}
\begin{proof}
Покажем, что нулевая функция разлагается единственным образом. Пусть $\sums{\Z}c_n z^n \equiv 0$.
Рассмотрим окружность $C$ радиуса $\rh$ с центром в точке $a$, где $\rh \in (r, R)$.
Вспомним, что
\eqn{\ints{C} (z-a)^n\,dz = \case{0, & n \neq -1; \\ 2\pi i, & n = -1.}}
В силу равномерной сходимости можно проинтегрировать ряд почленно. Из предыдущей формулы следует, что выживет
только одно слагаемое с индексом $-1$:
\eqn{0 = \ints{C} \sums{\Z} c_n (z-a)^n\,dz = \sums{\Z}c_n \ints{C} (z-a)^n\,dz = 2\pi i \cdot c_{-1}.}
Значит, $c_{-1} = 0$. Домножая исходный ряд на $(z-a)$ в подходящей степени, можно сдвинуть любой
коэффициент~$c_k$ на место коэффициента $c_{-1}$. От домножения на фиксированную степень область сходимости
не изменится. Значит, повторяя ту же процедуру над <<сдвинутым>> рядом, получим, что все коэффициенты равны нулю.
\end{proof}

\begin{ex}
\begin{align*}
e^{\frac1z}&=\sumnzi\frac{z^{-n}}{n!}, \quad r=0, \quad R= \bes;\\
\frac{1}{z(z-1)}&=-\frac1z\cdot \frac{1}{1-z}= -\suml{n=-1}{\infty}z^n, \quad 0<|z|<1;\\
\frac{1}{z(z-1)}&=\frac{1}{z^2}\cdot \frac{1}{1-\frac1z}=  \frac{1}{z^2}\suml{n=0}{\bes}\frac{1}{z^n}= \suml{n=-2}{\bes}\frac{1}{z^n}, \quad 1<|z|< \bes.
\end{align*}
\end{ex}

\subsection{Изолированные особые точки голоморфных функций}

\subsubsection{Классификация особых точек}

\begin{df}
Говорят, что точка $a$ является \emph{изолированной особой точкой}, если $f(z)$ не определена в $a$,
но голоморфна в проколотой окрестности этой точки.
\end{df}

\begin{df}
Пусть точка $a$\т изолированная особая точка функции $f$. Рассмотрим проколотую
окрестность $\dot{U}(a)$ радиуса~$R$ такую, что $f \in \Oc\br{\dot{U}(a)}$. Тогда в кольце $\Kc(a;0,R)$
имеет место разложение в ряд $f(z)=\sums{\Z}c_n(z-a)^n$. При этом возможны три случая:

\begin{items}{-2}
\item[(У)]\т главной части нет: $c_n=0$ при $n<0$. Тогда говорят, что точка \emph{устранимая};
\item[(П)]\т главная часть является многочленом от $\frac{1}{z-a}$, \те число слагаемых в ней
конечно ($c_n=0$ при $n<-N$). Тогда точку называют \emph{полюсом}, а старшую отрицательную
степень\т \emph{порядком} полюса.
\item[(С)]\т главная часть бесконечна. В этом случае говорят, то это \emph{существенно особая} точка.
\end{items}
\end{df}

Если особая точка устранима, то ряд даёт её голоморфное продолжение в полную окрестность $U(a)$.

\begin{theorem}[Классификация изолированных особых точек]
Рассмотрим изолированную особую точку $a$ функции $f(z)$. Рассмотрим предел
\eqn{\liml{z\ra a}f(z).}
Имеет место соответствие между типом особой точки и наличием этого предела:
\begin{items}{-2}
\item[(У)] $\Lra$ предел конечен (или $f(z)$ ограничена в некоторой окрестности точки $a$);
\item[(П)] $\Lra$ предел равен $\bes$;
\item[(С)] $\Lra$ предел не существует.
\end{items}
\end{theorem}
\begin{proof}
\framebox{У} Если существует $\liml{z\ra a}f(z)$, то доопределим $f(z)$ в точке по непрерывности.
Так как $f \in \Oc\br{\dot U(a)}$, то интеграл по любому треугольнику в проколотой окрестности равен $0$.
Но это значит, что интеграл по любому треугольнику в полной окрестности тоже равен $0$ (если бы это было не так,
он был бы ненулевым и при малом шевелении треугольника). Следовательно, $f\in\Oc\br{U(a)}$.

Обратное очевидно: если главная часть нулевая, то $f(z)$ представляется степенным рядом, а это и означает голоморфность.

Покажем, что для устранимости достаточно ограниченности функции. По неравенству Коши $|c_n|\le \frac{M}{\rh^n}$.
Значит, при $n < 0$ имеем $c_n=0$ (устремляем $\rh$ к нулю, тогда $c_n \ra 0$).

\framebox{П} Пусть $\liml{z\ra a}f(z)=\bes$. Положим $g(z) := \frac{1}{f(z)}$. Тогда $\liml{z\ra a}g(z)=0$. Значит, $g(z)$ ограничена
в некоторой окрестности точки $a$. Тогда точка $a$ является устранимой для $g(z)$, поэтому $g(z)$ разлагается в ряд Тейлора.
Пусть $g(z) = (z-a)^k\br{1 + \ph(z)}$, причём в некоторой окрестности $1+\ph(z) \neq 0$. Тогда $\frac{1}{1+\ph(z)}$ разлагается
в ряд Тейлора. Следовательно,
\eqn{f(z) = \frac{1}{(z-a)^k} \cdot \frac{1}{1 + \ph(z)} = \frac{1}{(z-a)^k} \cdot \sumnzi c_n(z-a)^n = \suml{n=-k}{\bes}c_n(z-a)^n.}
Но это и означает, что $a$ есть полюс для $f(z)$.

Наоборот: $f(z) = P\hr{\frac{1}{z-a}} + \sumnzi c_n (z-a)^n$, где $P \in \Cbb[z]$. Второе слагаемое голоморфно, а $P\ra \bes$ при $z\ra a$.

\framebox{С} Это единственная оставшаяся возможность.
\end{proof}

Для существенных особых точек имеет место более сильное утверждение.

\begin{theorem}[Ю.\,В.\,Сохоцкого]
Если $a$\т существенно особая точка, то для любого $A\in \ol\Cbb$ существует последовательность
$z_n\ra a$ такая, что $f(z_n) \ra A$ при $n \ra \bes$.
\end{theorem}
\begin{proof}
Пусть при некотором конечном $A$ утверждение неверно, то есть найдётся такое $A$ и его окрестность $U(A)$, что
в эту окрестность не попадает ни одной точки из образа сколь угодно малой окрестности точки $a$ при отображении $f$.
Тогда функция $g(z) := \frac{1}{f(z)-A}$ будет голоморфной в $U(a)$, так как $f(z)-A \neq 0$. Тогда точка $a$ будет
устранимой для $g(z)$, и по соображениям, аналогичным доказательству пункта <<П>> предыдущей теоремы для функции
$f(z) = A + \frac{1}{g(z)}$ точка $a$ будет либо устранимой, либо полюсом. Противоречие.

Теперь рассмотрим случай $A = \bes$. Функция $f$, как следует из первого утверждения теоремы классификации,
не может быть ограниченной в окрестности точки $A$. Значит, найдётся последовательность $z_n\ra a$ такая,
что $|f(z_n)| > n$. Стало быть, $f(z_n)\ra\bes$.
\end{proof}

\begin{note}
На самом деле верно ещё более сильное утверждение (\emph{теорема Пикара}): в любой окрестности существенно
особой точки функция обязана принимать все значения, за исключением, может быть, одного.
\end{note}

\subsubsection{Кратности нулей и полюсов}

\begin{df}
Пусть $f(z)\in \Oc(D)$ и $f \not\equiv 0$. Пусть $a\in D$. Точка $a$ называется \emph{нулём кратности $k$},
если $f(a)=f'(a)= \ldots =f^{(k-1)}(a)=0$, а $f^{(k)}(a)\neq 0$.
Эквивалентная формулировка: $f(z)=(z-a)^k \ph(z)$, где $\ph(a)\ne 0$.
Если $f \equiv 0$, то говорят, что кратность бесконечна.
Точка $a$ называется \emph{полюсом кратности} $k$, если $f(z) = \frac{\ph(z)}{(z-a)^k}$, причём $\ph(a) \neq 0$.
\end{df}

\begin{df}
Пусть $f(z) = (z-a)^m\ph(z)$, где $\ph(a) \neq 0$. Число $\ord_a f := m$ называют \emph{порядком} точки $a$.
\end{df}

\begin{df}
Если функция $f$ голоморфна в области $D$ за исключением конечного множества особых точек, среди которых нет
существенных особенностей, то она называется \emph{мероморфной} в $D$.
\end{df}

\begin{stm}
Пусть функция $f(z)$ мероморфна в окрестности точки $a$. Порядок точки $a$ равен $n$ тогда и только тогда, когда
$f(z)=(z-a)^n\ph(z)$ где $\ph(a) \neq 0$.
\end{stm}
\begin{proof}
Если $n \ge 0$, то $f(z)=c_{n}(z-a)^n+c_{n+1}(z-a)^{n+1}\ldots = (z-a)^n\ph(z)$, где $c_n$\т первый отличный от $0$
коэффициент. Если $n < 0$, то $\frac{1}{f(z)}=(z-a)^{-n}\psi(z)$, \те $f(z)=(z-a)^n\frac{1}{\psi(z)}$.
\end{proof}

Порядок точки является аналогом степени многочлена, о чём свидетельствует
\begin{lemma}
Имеет место формула $\ord_a(f\cdot g) = \ord_a f + \ord_a g$.
\end{lemma}
\begin{proof}
В самом деле, пусть $f(z)=(z-a)^n\ph(z)$, а $g(z)=(z-a)^m\psi(z)$. Тогда $f(z)g(z) = (z-a)^{m+n}\ph(z)\psi(z)$.
\end{proof}


Теперь посмотрим, как можно определить тип особой точки $\bes$. Рассмотрим функцию $g(z)=f\hr{\frac1z}$ в
окрестности нуля. Если в ней она ограничена, то можно доопределить, и тем самым бесконечность будет устранимой.

Как связаны ряды Лорана для функций $f(w)$ и $f\hr{\frac1z}$? Проколотая окрестность $\infty$ на
плоскости~$w$\т это $|w|>R$ или $0<|z|<\frac1R$ на плоскости $z$. Пусть $f\hr{\frac1z}$ разлагается в
проколотой окрестности нуля в ряд Лорана:
\eqn{f\hr{\frac1z}=\sums{\Z}c_n z^n.}
Как этот ряд связан с лорановским разложением $f(w)$ в кольце $|w|>R$? Очень просто:
\eqn{f(w)=\sums{\Z}c_{-n}w^n,}
то есть меняются местами главная и регулярная части.

\begin{note}
С точки зрения проективной геометрии, точка $\bes$ ничем не отличается от любой другой точки $\ol \Cbb \cong \CP^1$.
Если выбрать другую карту (например, сделаем стереографическую проекцию из южного полюса), то на ней она
станет обыкновенной точкой. Преобразование $\frac{1}{\ol z}$ как раз и осуществляет замену карты.
\end{note}

Таким образом, можно дать обобщённое определение порядка нуля и полюса.

\begin{df}
Пусть $a$\т устранимая особая точка или полюс. Тогда порядок точки $a$ равен либо кратности нуля $f(z)$,
если $a$\т устранимая особая точка $f(z)$, либо кратности нуля $\frac{1}{f(z)}$, если $a$\т полюс $f(z)$.
\end{df}

У мероморфной функции $f(z)$ порядок определён для любой точки. Если $\ord_a f\ge 0$, то $a$\т
устранимая особая точка $f(z)$, а если  $\ord_a f < 0$, то $a$\т полюс $f(z)$, и число $(-\ord_a f)$
есть кратность полюса.

Совокупность мероморфных в $D$ функций обозначается $\Mc(D)$. Легко видеть, что это поле.
Определим сходимость в $\Mc(D)$ следующим образом:

\begin{df}
Пусть $f_n\in \Mc(D)$. Будем говорить, что ряд $\suml{n=1}{\bes}f_n$ сходится в $\Mc(D)$, если для любого компакта
$K \subs D$ найдётся такое $N$, что для любого $n>N$ функции $f_n$ не имеют полюсов на $K$, причём
$\suml{n=N+1}{\bes}f_n$ сходится равномерно.
\end{df}

\begin{problem}
Сформулировать и доказать теорему Вейерштрасса для $\Mc(D)$.
\end{problem}

\subsubsection{Куча примеров}

\begin{ex}
Функция $f(z)=\frac1z$ голоморфна в $\bes$, так как $f\hr{\frac1z}=z$, и в нуле эта функция хорошая.
\end{ex}

\begin{ex}
Функция $f(z)=z$ имеет полюс в $\bes$, так как $f\hr{\frac1z}=\frac1z$, и в нуле она имеет полюс.
\end{ex}

\begin{ex}
Полином степени $n$ имеет в $\infty$ полюс порядка $n$, а для $e^z$ бесконечность будет существенно особой
точкой, так как $e^z = e^{x+iy} = e^x(\cos y + i \sin y)$, и если идти по последовательностям $z_n\ra\bes$
с разными значениями мнимой части $z$, то получим разные значения пределов.
\end{ex}

\begin{ex}
Функция $\frac{\sin z}{1+z^2}$ имеет три особых точки: $\hc{\pm i, \; \bes}$. Очевидно, что $\pm i$\т полюса
первого порядка, а бесконечность\т существенная, так как если пойти по вещественной оси,
то синус будет бегать по отрезку $[-1, 1]$ и предела не будет.
\end{ex}

\begin{ex}
Функция $\tg z$ имеет счётное число однократных полюсов, являющихся нулями
функции $\cos z$, те $\hc{\frac{\pi}{2} + \pi k}$. Бесконечность\т неизолированная особая точка, так как полюса
неограниченно близко подходят к ней.
\end{ex}

\begin{ex}
Функция $\frac{e^z}{e^z + 1}$ имеет полюса первого порядка в точках $2\pi i k + \pi i$. Бесконечность
изолированной не является (к ней стремится последовательность полюсов).
\end{ex}


\subsubsection{Голоморфные функции на многообразиях}

\begin{df}
Хаусдорфово топологическое пространство $\Xc$ называется \emph{$n$\д мерным комплексным многообразием},
если каждая точка $x \in \Xc$ содержится в некоторой окрестности $U \subs \Xc$, гомеоморфной некоторой
области~$V$ пространства $\Cbb^n$. \emph{Атласом} многообразия $\Xc$ называется множество троек
$(U_\al, V_\al, \ph_\al)$, где $U_\al \subs \Xc$, $V_\al\subs \Cbb^n$, а $\ph_\al\cln U_\al \ra V_\al$\т гомеоморфизмы.
Пусть какие\д то две карты $U_\al$ и $U_\be$ пересекаются, и $U = U_\al \cap U_\be \neq \es$. \emph{Функцией перехода}
называется отображение $\psi\cln \ph_\al(U) \ra \ph_\be(U)$, заданное формулой $\psi = \ph_\be\circ \ph_\al^{-1}$.
Многообразие $\Xc$ называется \emph{комплексно\д аналитическим}, если все функции перехода конформны.
\end{df}

\begin{df}
Пусть $\Xc$\т одномерное комплексное многообразие. Говорят, что функция $f$ голоморфна (мероморфна) на $\Xc$,
если она голоморфна (мероморфна) в любой карте $\Xc$.
\end{df}

Аналогично определяется голоморфное отображение $f\cln\Xc_1 \ra \Xc_2$.

На многообразия почти автоматически переносятся теорема единственности, принцип максимума модуля и
классификация особых точек. Так как одномерное многообразие локально гомеоморфно $\Cbb$, то все
доказательства дословно повторятся.

Для того, чтобы показать корректность классификации, докажем лемму.
\begin{lemma}
Пусть $a$\т изолированная особая точка функции $f(z)$, а $\ph(z)$ голоморфна в окрестности точки $a$,
$\ph(a)=a$ и $\ph'(a)\ne0$, тогда $a$\т изолированная особая точка того же типа для функции $f\br{\ph(z)}$, и  если
$a$ является устранимой особой точкой либо полюсом, то $\ord_a f(\ph)=\ord_a f$.
\end{lemma}
\begin{proof}
Первое утверждение следует из непрерывности $\ph$ и из теоремы о классификации изолированных особых точек.
Второе утверждение очевидно.
\end{proof}
\begin{imp}
Тип изолированной особой точки и её порядок не зависят от карты.
\end{imp}

\subsubsection{Целые функции}

\begin{df}
Функция, голоморфная в $\Cbb$, называется \emph{целой}. (Целая функция имеет на $\ol{\Cbb}$ только
одну особую точку\т бесконечность).
\end{df}

\begin{stm}
Пусть функция $f(z)$ целая. Бесконечность является устранимой особой точкой тогда и только тогда, когда
$f(z) \equiv \const$, и полюсом порядка $n$ тогда и только тогда, когда $f(z)$\т полином  степени $n$.
\end{stm}
\begin{proof}
Если $\bes$ является устранимой для $f(z)$, то она ограничена и по теореме Лиувилля $f(z)\equiv \const$.
Обратное очевидно.
Если $\infty$ является полюсом порядка $n$, рассмотрим лорановское разложение в нуле. Главная часть у
этого разложения отсутствует, так как в нуле функция голоморфна, а регулярная часть является полиномом степени $n$.
Обратное очевидно.
\end{proof}

Напомним, что рациональной функцией называется функция, являющаяся отношением двух многочленов.

\begin{stm}
Функция $f(z)$ мероморфна в $\ol{\Cbb}$ тогда и только тогда, когда она рациональна.
\end{stm}
\begin{proof}
Пусть функция мероморфна в $\ol{\Cbb}$ и $\hc{a_1 \sco a_n}$\т её полюса.
Разложим её в ряд Лорана в каждой из особых точек $a_i$. В каждой из конечных точек $a_i$ главная часть имеет
вид $P_i\hr{\frac{1}{z-a_i}}$, а при разложении на $\bes$ главная часть равна $P(z)$ (по предыдущему утверждению),
где $P_i$ и $P$\т многочлены. Рассмотрим функцию
\eqn{g(z) := f(z) - \suml{i=1}{n}P_i\hr{\frac{1}{z-a_i}}- P(z).}
Она не имеет особых точек, а значит, голоморфна в $\ol{\Cbb}$, по теореме Лиувилля она постоянна. Значит, и функция~$f$
была рациональной (как сумма рациональных). Обратное утверждение очевидно.
\end{proof}

\subsection{Вычеты}

\subsubsection{Понятие вычета}

\dmvnpicrh{pictures.80}{-7}
Особые\dmvnpicra{pictures.80}{9}{-1.5pc} точки обладают весьма противным
свойством: они мешают применить формулу Коши. Пусть
у функции $f$ в ограниченной области $D$ конечное число особых точек $a_1 \sco a_n$, а
на кусочно\д гладкой границе $\pd D$ их нет. Будем вырезать эти точки из области вместе
с маленькими кругами $C_i$ радиуса $\ep$, получим область $D_\ep$. Тогда функция будет голоморфной
в~$D_\ep$ и интегральной формулой пользоваться можно.

Имеем $D_\ep= D \wo \cupl{i=1}{n}\Int C_j$. Тогда по формуле Коши
\eqn{0=\ints{\pd D_\ep}f(z)\,dz=\ints{\pd D}f(z)\,dz-\sum\ints{C_i}f(z)\,dz \quad \Lra \quad \ints{\pd D}f(z)\,dz=\sum\ints{C_i}f(z)\,dz.}

Пусть $\sums{\Z}c_n(z-a_i)^n$\т лорановское разложение функции $f(z)$ в точке $a_i$.
Этот ряд можно проинтегрировать почленно в силу равномерной сходимости. У всех степеней есть первообразная, кроме
$n=-1$. Значит, получаем
\eqn{\ints{С_i}f(z)\,dz=2\pi i c_{-1}.}

\begin{df}
Лорановский коэффициент разложения функции $f$ в изолированной особой точке $a$ с номером $-1$
называется \emph{вычетом функции} $f$ в точке $a$ и обозначается $\res_a f$.
\end{df}

Сформулируем доказанную теорему:

\begin{theorem}[о вычетах]
Пусть $D$\т ограниченная область с кусочно\д гладкой границей, и функция $f(z)$
голоморфна в окрестности $\ol{D}$ за исключением особых точек $a_1 \sco a_n \in D$. Тогда
\eqn{\ints{\pd D}f(z)\,dz=2\pi i\suml{i=1}{n}\res_{a_i} f.}
\end{theorem}

\subsubsection{Нахождение вычетов}

\pt{1} Если $a$\т полюс первого порядка, то
$$f(z)=c_{-1}(z-a)^{-1}+c_0+c_1(z-a)+\ldots \quad \Ra \quad f(z)(z-a)=c_{-1}+c_0(z-a)+c_1(z-a)^2 +\ldots \quad \Ra$$
\eqn{c_{-1}=\liml{z\ra a}\hs{f(z)(z-a)}.}

\pt{2} Если $a$ --- полюс порядка $p$, то по определению
\begin{align*}
f(z) &= \frac{c_{-p}}{(z-a)^p} \spl \frac{c_{-1}}{z-a}+c_0 + \ldots \quad \Ra\\
f(z)(z-a)^p &= c_{-p} \spl c_{-1}(z-a)^{p-1}+\ldots
\end{align*}
Так как функция голоморфна, продифференцируем её $(p-1)$ раз:
\eqn{\frac{d^{p-1}}{d z^{p-1}} \hs{f(z)(z-a)^p}=(p-1)! \cdot c_{-1}+\ldots}
Преобразовывая выражение, получаем
\eqn{c_{-1}=\frac{1}{(p-1)!}\liml{z\ra a}\frac{d^{p-1}}{dz^{p-1}}\hs{f(z)(z-a)^p}.}

\pt{3} Если $a$\т существенно особая точка, то нужно разложить функцию в ряд и явно найти коэффициент $c_{-1}$.

\begin{ex}
Найдём вычет функции $f(z) = \frac{1}{z^2+1}$:
\eqn{\res_i\frac{1}{z^2+1}= \liml{z\ra i}\frac{(z-i)}{(z-i)(z+i)} = \frac{1}{2i}.}
\end{ex}

\begin{ex}
\eqn{\sin\hr{\frac1z}=\frac1z-\frac{1}{3!}\frac{1}{z^3}+\ldots \quad \Ra \quad \res_0\sin\hr{\frac1z}=1.}
\end{ex}

Теорема о вычетах позволяет легко вычислять интегралы.

\begin{ex}
\begin{multline*}
\res_i\frac{\sin z}{z^2+1}=\frac{\sin i}{2i}, \quad \res_{-i}\frac{\sin z}{z^2+1}=\frac{\sin -i}{2 (-i)}= \frac{\sin i}{2 i} \quad \Ra\\
\Ra \quad \ints{|z|=2}\frac{\sin z}{z^2+1}\,dz = 2\pi i\hs{\res_i\frac{\sin z}{z^2+1}+\res_{-i}\frac{\sin z}{z^2+1}}=2\pi\sin i.
\end{multline*}
\end{ex}

Следующий пример показывает, как с помощью вычетов можно находить несобственные интегралы от рациональных функций.

\begin{ex}
\dmvnpicrh{pictures.90}{5}
Вычислим интеграл
\eqn{I:=\intl{-\bes}{+\bes}\frac{P(x)}{Q(x)}\,dx, \text{ где } \deg P \le \deg Q - 2,}
и на вещественной оси знаменатель не обращается в нуль. Тогда условие на степени обеспечивает сходимость интеграла,
и остаётся понять, чему он равен. Пусть $f(z) := \frac{P(z)}{Q(z)}$. Рассмотрим контур $\ga_R$ следующего вида:
верхняя полуокружность $C_R$ радиуса $R$, у которой концы соединены отрезком прямой. \dmvnpicra{pictures.90}{10}{-3.5pc} Имеем:
$$\ints{\ga_R}\frac{P(z)}{Q(z)}\,dz= \intl{-R}{R}\frac{P(z)}{Q(z)}\,dz+ \ints{C_R}\frac{P(z)}{Q(z)}\,dz.$$


Теперь покажем, что второе слагаемое стремится к нулю при $R \ra \bes$. Длина этого контура равна $\pi R$. А так как
функция на бесконечности должна убывать быстрее, чем $\frac{1}{x^2}$, то знаменатель убывает со скоростью $\frac{1}{R^2}$,
а потому $\Bm{\ints{C_R}\frac{P(z)}{Q(z)}\,dz} \le M \cdot \frac{\pi R}{R^2} \ra 0$.
Значит, наш интеграл равен
\eqn{I = 2\pi i \mcomp{\sums{\Img a_i > 0}} \res_{a_i} f.}
\end{ex}

\subsection{Принцип аргумента}

\subsubsection{Связь количества нулей с количеством полюсов}

Пусть функция $f(z)$ голоморфна в проколотой окрестности точки $a$, а сама точка\т не хуже, чем полюс, то есть $f \in \Mc\br{U(a)}$.

\begin{lemma}
Если $a$\т обычная точка или полюс для $f(z)$, то
\eqn{\res_a \hr{\frac{f'}{f}}=\ord_a f.}
\end{lemma}
\begin{proof}
Пусть $n = \ord_a f$. Тогда имеет место представление $f(z) = (z-a)^n \ph(z)$, где $\ph(a) \in \Oc\br{U(a)}$.
Продифференцируем: $f'(z)=n(z-a)^{n-1}\ph(z)+(z-a)^{n}\ph'(z)$. Тогда
\eqn{\frac{f'(z)}{f(z)}=\frac{n (z-a)^{n-1}\ph(z) + (z-a)^n \ph'(z)}{(z-a)^n \ph(z)} = \frac{n}{z-a}+\frac{\ph'(z)}{\ph(z)}.}

Так как $\ph(z)$ была голоморфной в полной окрестности точки $a$, и $\ph(a) \neq 0$, то $\frac{\ph'(z)}{\ph(z)}$
также голоморфна. Числитель первого слагаемого в выражении для $\frac{f'}{f}$ и есть значение вычета. Лемма доказана.
\end{proof}

\begin{note}
Выражение $\frac{f'}{f}$ есть логарифмическая производная функции $f$, ибо $\br{\ln f(z)}' = \frac{f'(z)}{f(z)}$.
\end{note}

\begin{note}
Под количеством нулей или полюсов понимается их количество с учётом кратностей!
\end{note}

\begin{note}
У логарифмической производной все полюса простые.
\end{note}

В следующей теореме нам нужно будет работать с многозначными функциями. В связи с этим нам потребуется одно важное
обозначение. Приращение непрерывной ветви многозначной функции $\ph$ на кривой $\ga$ будем обозначать $\Varl{\ga}\ph$.

\begin{theorem}
Пусть $D$ --- область из теоремы о вычетах, функция $f(z)$ мероморфна в окрестности $\ol{D}$, и на границе
$\pd D$ нулей и полюсов нет. Тогда
\eqn{\frac{1}{2\pi i}\ints{\pd D}\frac{f'(z)}{f(z)}\,dz= N - P,}
где $N$\т число нулей, а $P$\т число полюсов функции $f$.
\end{theorem}
\begin{proof}
Запишем теорему о вычетах для логарифмической производной:
\eqn{\frac{1}{2\pi i } \ints{\pd D} \frac{f'(z)}{f(z)}\,dz = N-P.}
\hfill\end{proof}

\dmvnpicrh{pictures.100}{-8}
Разобьём\dmvnpicra{pictures.100}{11}{-1.5pc} границу области на маленькие последовательные
куски $\ga_1 \sco \ga_n$ так, что
на каждом из них можно выделить однозначную ветвь функции $\ln f(z)$, и пусть $\al_i$ и $\be_i$\т
левые и правые концы кусочков $\ga_i$ соответственно. Тогда интеграл по каждому маленькому кусочку
можно вычислять по формуле Ньютона\ч Лейбница:
\eqn{\ints{\ga_i}\frac{f'(z)}{f(z)}\,dz=\ln f(\be_i)-\ln f(\al_i).}

\dmvnpicrh{pictures.100}{-2}
Теперь вспомним, что $\ln f = \ln |f| + i \Arg f$. Функция $\ln |f|$ однозначна, поэтому достаточно выделить
ветвь $\Arg f$, непрерывно меняющуюся вдоль границы области. Поэтому
\eqn{\ints{\ga_i}\frac{f'(z)}{f(z)}\,dz=\ln f(\be_i)-\ln f(\al_i) = \Varl{\ga_i} \ln f}

Следовательно,
\eqn{\ints{\ga}\frac{f'(z)}{f(z)}\,dz= \suml{i=1}{n}\Varl{\ga_i} \ln f = \Varl{\ga} (i\Arg f).}
Если $\ga$\т замкнутый контур (или набор контуров), то $\Varl{\ga}\ln|f|=0$.

\begin{df}
\dmvnpicrh{pictures.110}{-4}
Число\dmvnpicra{pictures.110}{12}{-1.5pc} оборотов вектора $w = f(z)$ при обходе
точкой $z$ замкнутого контура $\ga$ называется
\emph{индексом} пути $\ga$ относительно точки $w=0$.
\end{df}

\begin{imp}[принцип аргумента]
\dmvnpicrh{pictures.110}{-4}
Пусть выполнены условия теоремы, $N$ и $P$\т количества нулей и полюсов функции $f$. Тогда
\eqn{N - P = \frac{1}{2\pi} \Varl{\pd D} \Arg f=  \ind_0(\ga).}
\end{imp}
\begin{proof}
Действительно, если приращение аргумента поделить на $2\pi$, получится как раз количество
оборотов вектора $f(z)$, если $z$ пробегает по пути $\ga$.
\end{proof}

\subsubsection{Теорема Руше и принцип открытости. Теорема Гурвица}

Будем через $N_f$ обозначать количество нулей функции $f$ (с учетом кратности!).

\begin{theorem}[Руш\'е]
Пусть область $D$\т как в теореме о вычетах, функции $F$ и $g$ голоморфны в окрестности $\ol{D}$,
причём $|F(z)|>|g(z)|$ на границе $\pd D$. Тогда $F+g$ имеет столько же нулей, что и $F$:
\eqn{N_{F+g}=N_{F}.}
\end{theorem}
\begin{proof}
В силу условия $|F(z)| > |g(z)|$ у суммы $F+g$ также нет нулей на границе области.
По принципу аргумента
\eqn{N_{F+g}=\frac{1}{2\pi}\Varl{\pd D} \Arg(F+g)= \frac{1}{2\pi} \Varl{\pd D} \Arg \hs{F \hr{1+\frac{g}{F}}}=
\frac{1}{2\pi} \Varl{\pd D} \Arg F + \frac{1}{2\pi} \Varl{\pd D}\Arg \hr{1+\frac{g}{F}} = N_F,}
так как $\hm{\frac{g}{F}} < 1$, а потому $\Varl{\pd D} \Arg \hr{1+\frac{g}{F}}=0$ (функция $1 + \frac{g}{F}$ не может
сделать <<оборота вокруг нуля>>, поэтому аргумент может получить только нулевое приращение).
\end{proof}

\begin{theorem}[принцип открытости (сохранения области)]
Голоморфная функция всегда осуществляет открытое отображение, то есть если
$D$\т область, $f\in\Oc(D)$ и $f\neq\const$, то $f(D)$\т область.
\end{theorem}
\begin{proof}
Связность образа следует из непрерывности $f$. Надо доказать открытость образа. Пусть $b\in f(D)$,
то есть существует $a \in D$ такая, что $f(a)=b$. Поскольку $f\neq \const$, то $a$\т изолированная точка,
то есть найдётся $\ep > 0$ такое, что $|f(z) - b| \neq 0$, если $z \in U_\ep(a)$ (если бы такой окрестности
не нашлось, это означало бы, что $f(z) = b$ при $z\ra a$, а тогда по теореме единственности $f(z) \equiv b$).
Значит, $\mu := \minl{\pd U_\ep(a)} |f(z)-b| > 0$.
Рассмотрим круг радиуса $\frac\mu2$ с центром в точке $b$. Пусть $w$ лежит в этом круге.
Представим функцию $f(z)-w$ в виде
\eqn{f(z)-w = (\ub{f(z)-b}_F)+(\ub{b-w}_g).}
Возьмём такое $w$, что $|b-w|<\mu$. Тогда по теореме Руше (для функций $F(z) =f(z)-b$, $g(z)=b-w$ и этого круга) $N_{f-w}=N_{f-b}>0$, то есть функция $f(z)-w$ имеет
хотя бы один нуль, и найдётся $z$ такое, что $f(z)=w$. То есть у каждой точки круга есть прообраз.
\end{proof}
\begin{note}
Число прообразов в малом круге равно $\ord\hs{F(z)-F(a)}$. Из этой теоремы легко
следует принцип максимума, доказанный ранее.
\end{note}

\begin{df}
Функция $f$ называется \emph{однолистной}, если $f(z_1) \neq f(z_2)$ для $\fa z_1 \neq z_2$.
\end{df}

\begin{theorem}[Гурвица]
Если функции $f_n\in\Oc(D)$ однолистны, и $f_n \rra f$, то либо функция $f$ однолистна, либо $f \equiv \const$.
\end{theorem}
\begin{proof}
Предположим противное: пусть предельная функция не постоянна и у точки $b$ есть много прообразов~$a_i$,
\те $f(a_i)=b$. Тогда полный прообраз $\hc{a_i}$ точки $b$\т дискретное множество (все точки изолированные).
Значит, найдутся окрестности $U_i(a_i)$ такие, что на их границах функция $f(z)-b$ не обращается в нуль.
Тогда
\eqn{\mu := \minl{\pd U_i(a_i)} |f(z)-b|\ > 0.}
Имеем
\eqn{f_n(z)-b=\br{\ub{f(z)-b}_F}+ \br{\ub{f_n(z)-f(z)}_g}.}
Начиная с некоторого номера $n$ вторая скобка станет меньше $\frac\mu2$ по модулю в силу равномерной сходимости.
Тогда у функции $F +g = f_n(z)-b$ нулей должно быть столько же, сколько и у $F$, а у неё нулей столько же, сколько
прообразов $a_i$. Это противоречит однолистности функций $f_n$.
\end{proof}

\section{Конформные отображения}

Мы уже кое\д что знаем о конформных отображениях. Настало время доказать про них нечто содержательное.

\subsection{Формулировка теоремы Римана}

\subsubsection{Ещё одно свойство однолистных функций}

Как мы знаем, конформность отображения $f$ в точке $a$ равносильна существованию производной в этой точке, причём $f'(a)\neq 0$.

\begin{df}
Будем говорить, что голоморфная функция $f\cln D\ra\wt{D}$ осуществляет \emph{конформное отображение} области $D$ на область $\wt{D}$, если
функция $f$ однолистна в $D$ и $f(D) = \wt{D}$.
\end{df}

\begin{stm}
Производная голоморфной однолистной функции не обращается в нуль ни в одной точке.
\end{stm}
\begin{proof}
Будем действовать аналогично доказательству принципа открытости. Допустим, $f'(a)=0$. Тогда покажем,
что у точки $b:=f(a)$ будет несколько прообразов, и тем самым придём к противоречию с однолистностью.
Возьмём такую окрестность точки $a$, в которой нет других нулей производной, кроме самой точки $a$. Так как
$(f(a)-b) = (f(a)-b)'=f'(a)=0$, то по определению функция $f(z)-b$ имеет нуль как минимум второй кратности.
Но это и означает, что у точки $b$ будет более одного прообраза.
\end{proof}

\begin{ex}
Рассмотрим функцию $f(z) = z^k$. Она имеет нуль кратности $k$, а в окрестности нуля имеется ровно $k$ прообразов.
\end{ex}

\subsubsection{Группа конформных автоморфизмов}

Через $\Aut(D)$ будем обозначать множество конформных отображений области $D$ на себя.
Очевидно, это множество имеет структуру группы с операцией композиции.
Наличие обратного элемента гарантируется однолистностью: в силу биективности можно рассмотреть
формальное (или, как еще говорят, алгебраическое) обратное отображение\т сопоставить каждой точке
$w$ её прообраз $z$, но так как $f'(z)\neq 0$, то и $\hr{f^{-1}}'(w) \neq 0$, поэтому обратное отображение
также голоморфно и однолистно. Композиция однолистных отображений также однолистна, так что определение $\Aut(D)$
корректно.

\begin{df}
Говорят, что области $D$ и $\wt{D}$ \emph{конформно эквивалентны}, если существует конформное
отображение $D$ на $\wt D$. Обозначение: $D \sim \wt{D}$.
\end{df}

Очевидно, что если $D \sim \wt{D}$, то $\Aut(D) \cong \Aut(\wt{D})$. В самом деле, пусть $f\cln D\ra \wt{D}$\т конформное
отображение, и $h \in \Aut(D)$. Изоморфизм задаётся так: $h \mapsto f \circ h \circ f^{-1} = :\wt{h} \in \Aut(\wt D)$.

\subsubsection{Теорема Римана}

\emph{Теорема Римана} утверждает, что все односвязные области $D$ на сфере распадаются на три класса эквивалентности
относительно конформных преобразований:

\begin{items}{-2}
\item $D \sim \ol{\Cbb}$\т вся сфера;
\item $D \sim \Cbb$\т плоскость;
\item $D \sim \De := \hc{z\cln |z|<1}$\т единичный круг (будем использовать это обозначение в дальнейшем).
\end{items}

Докажем мы её немного позже, а пока займёмся вычислением групп автоморфизмов этих трёх классов.

Эти классы можно различать по мощности множества точек границы. Первому классу соответствует случай $\pd D = \es$,
второму\т $\Card \pd D = 1$, а третьему\т $\Card \pd D > 1$ (на самом деле, если точек на границе не меньше двух,
то их континуум: если из сферы выбросить лишь конечное число точек, граница  будет несвязной, и мы не получим
односвязной области).

\begin{note}
Сфера $\ol{\Cbb}$ неэквивалентна двум другим классам и с топологической точки зрения:
множество $\ol{\Cbb}$ компактно, а $\De$ и $\Cbb$\т нет.
\end{note}

\begin{stm}
Множества $\Cbb$ и $\De$ конформно не эквивалентны.
\end{stm}
\begin{proof}
Пусть существует конформное отображение $\ph$ плоскости на круг. Тогда оно голоморфно на всей плоскости и ограничено,
ибо $|\ph(z)|<1$. Тогда по теореме Лиувилля оно постоянно, то есть является отображением не на весь круг, а в точку.
Противоречие.
\end{proof}

\subsection{Вычисление групп $\Aut(\ol{\Cbb})$, $\Aut(\Cbb)$ и $\Aut(\De)$}

\subsubsection{Автоморфизмы сферы}

Рассмотрим группу дробно\д линейных преобразований
\eqn{G_0 := \hc{T(z) = \frac{az+b}{cz+d}}.}
Мы уже знаем, что они конформно отображают сферу на себя.

\begin{theorem}
$\Aut(\ol\Cbb) \cong G_0$.
\end{theorem}
\begin{proof}
Пусть $\ph\cln\ol\Cbb\ra\ol{\Cbb}$\т произвольный автоморфизм сферы. Пусть $\ph$ переводит $\bes$
в некоторую конечную точку. Тогда найдётся такое дробно\д линейное преобразование $T \in G_0$, которое возвращает $\bes$
на место, то есть $(T\circ \ph)(\bes)=\bes$. Обозначим $\Ph := T \circ \ph$. Покажем, что $\Ph$ является линейным
отображением. Так как~$\Ph(\bes)=\bes$, то можно рассматривать $\Ph$ как отображение $\Ph\cln\Cbb\ra \Cbb$.
Это голоморфная функция без особых точек. Поэтому, если $z\ra\bes$, то $\Ph(z)\ra\bes$, то есть функция $\Ph$ является
целой и имеет полюс на бесконечности. Но мы\д то знаем, что все такие функции\т это обычные многочлены, а кроме того,
степень не выше кратности полюса, которая равна единице в силу однозначности $\Ph$. Значит, $\Ph = az+b$.
Таким образом, функция $\ph = T^{-1}\circ \Ph$ есть композиция линейной и дробно\д линейной функции. Значит, она сама
является дробно\д линейной.
\end{proof}

\subsubsection{Автоморфизмы плоскости}

Как видно из формул, дробно\д линейные преобразования, переводящие бесконечность в себя, являются
на самом деле линейными. Однако из этого мы пока не можем сделать вывод о том, что и все конформные автоморфизмы
плоскости тоже линейные: тут нельзя пользоваться непрерывностью на бесконечности. Однако это нам не помешает.

Обозначим через $G_1$ группу линейных преобразований плоскости:
\eqn{G_1 := \hc{T(z) = az+b}.}

\begin{theorem}
$\Aut(\Cbb)\cong G_1$.
\end{theorem}
\begin{proof}
Пусть $\ph\cln\Cbb\ra\Cbb$\т автоморфизм плоскости, тогда $\bes$\т изолированная особая точка. Она не может
быть устранимой, ибо в этом случае функция $\ph$ была бы голоморфной на  сфере, и по теореме Лиувилля $\ph\equiv\const$.
Покажем, что она не может быть существенно особой точкой. Рассмотрим любую точку $a \neq \bes$ и её образ $b=\ph(a)$.
Тогда в силу конформности точка $a$ отображается в $b$ вместе с некоторой окрестностью:
\eqn{\label{AutMapping}U(a) \corr{\ph} U(b), \quad \ol\Cbb \wo U(a) \corr\ph \ol\Cbb\wo U(b).}
Но теорема Сохоцкого говорит о том, что если $\bes$\т существенная особая точка, то найдётся последовательность $\al_n \ra \bes$
такая, что $\ph(\al_n) \ra b$. Получается, что в окрестность $U(b)$ могут попасть точки из окрестности бесконечности, что
противоречит \eqref{AutMapping}: туда должны попадать только точки из окрестности $U(a)$.

Таким образом, бесконечность является полюсом, а $\ph$\т линейной функцией.
\end{proof}

\subsubsection{Лемма Шварца. Автоморфизмы единичного круга}

\begin{lemma}[Шварц]
Пусть функция $f\cln\De\ra\De$ голоморфна на $\De$. Пусть $f(0)=0$.
Тогда $|f(z)|\le |z|$, причём если существует точка $z_0\in\De$ такая, что $|f(z_0)|=|z_0|$,
то $f(z)=e^{i\ta}z$, $\ta\in\R$.
\end{lemma}
\begin{proof}
Рассмотрим функцию $\ph(z) := \frac{f(z)}{z}$. Поскольку $f(0)=0$, то нуль будет устранимой точкой для $\ph(z)$.
Значит, $\ph$ голоморфна в круге $\De$.

Возьмём замкнутый круг радиуса $\rh < 1$. По принципу максимума функция $\ph$ достигает своего максимума на
границе этого круга. Но так как $|f(z)| \le 1$, то
\eqn{|\ph(z)| = \hm{\frac{f(z)}{z}} \le \frac{1}{\rh}.}
Устремляя $\rh$ к единице, получаем $\bm{\frac{f(z)}{z}} \le 1$, следовательно, $|f(z)|\le|z|$.

Пусть теперь $|f(z_0)|=|z_0|$ в некоторой точке $z_0$. Из доказанного выше следует, что $|\ph| \le 1$.
В точке $z_0$ функция $|\ph|$ достигает значения $1$, а больше единицы быть не может. Значит, по принципу максимума $\ph=\const$
и $|\ph|=1$, то есть $\ph(z) = e^{i\ta}$. Тогда $f(z) = e^{i\ta}z$\т поворот на угол $\ta$.
\end{proof}

Обозначим через $G_2$ группу отображений следующего вида:
\eqn{G_2 :=\hc{T(z) = e^{i\ta}\frac{z-a}{1-\ol{a}z}}, \text{ где } |a| < 1, \quad \ta \in \R.}

\begin{theorem}
$\Aut(\De) \cong G_2$. Эта группа действует на $\De$ транзитивно.
\end{theorem}
\begin{proof}
Пусть $\ph\cln\De\ra\De$\т конформный автоморфизм круга. Пусть $\ph(0)=a$. Рассмотрим дробно\д линейное преобразование
$T\in G_2$, заданное формулой $T(z) = \frac{z-a}{1-\ol a z}$. Оно переводит точку $a$ в нуль, то есть
композиция $\Ph := T \circ \ph$ оставляет нуль на месте: $\Ph(0)=0$. Применим к отображению $\Ph$ лемму Шварца.
Она утверждает, что
\eqn{\label{LeftInequality}|\Ph(z)| \le |z|.}
Теперь рассмотрим обратное отображение $\Ph^{-1}$. Это тоже будет некоторый
автоморфизм круга, сохраняющий нуль, и к нему тоже можно применить лемму Шварца, откуда
\eqn{|\Ph^{-1}(w)|\le |w|.}
Подставим в эту формулу $\Ph(z)$ вместо $w$. Так как $\Ph^{-1} \circ \Ph = \id$, то
\eqn{\hm{\Ph^{-1}\br{\Ph(z)}} \le |\Ph(z)| \; \Ra \; |z| \le |\Ph(z)|.}
Тем самым мы получили обратное неравенство к \eqref{LeftInequality}. Значит, $|\Ph(z)|=|z|$ для любой точки $z$.
По второму утверждению леммы Шварца $\Ph(z)=e^{i\ta}z$. Значит, $\ph = T^{-1}\circ \Ph$\т отображение, являющееся
композицией некоторого поворота и дробно\д линейного отображения, \те $\ph \in G_2$. Следовательно, $\Aut(\De) \cong G_2$.

Транзитивность следует из того, что любую точку можно перевести в нуль соответствующим преобразованием,
а затем с тем же успехом можно нуль отправить куда угодно.
\end{proof}

Найдём размерности групп автоморфизмов.

\begin{points}{0}
\item $\dim_\R \Aut(\De) = 3$, так как отображение задаётся углом поворота $\ta$ (один вещественный параметр), а также
точкой, которая едет в нуль, \те имеется ещё два вещественных параметра.

\item $\dim_\Cbb \Aut(\Cbb) = 2$, ибо линейное комплексное преобразование $f(z)=az+b$ задаётся числами $a$ и $b$.

\item Покажем, что $\dim_\Cbb\Aut(\ol\Cbb)=3$. Дробно\д линейное преобразование задаётся четырьмя числами $\hc{a,b,c,d}$,
но это не независимые параметры. Чтобы они однозначно задавали автоморфизм, нужно наложить условие $ad-bc=1$, \те
матрица $\rbmat{a & b\\c & d}\in \SL_2$. Значит, остаётся только три (независимых) параметра.
\end{points}

\subsection{Доказательство теоремы Римана}

\subsubsection{Принцип компактности}

\begin{df}
Семейство функций $\hc{f_\al}$ на области $D$ называется \emph{локально равномерно ограниченным},
если для любого компакта $K \Subset D$ найдётся константа $M_K$ такая, что $|f_\al(z)| \le M_K$
для $\fa z \in K$ и $\fa \al$.
\end{df}

\begin{df}
Семейство функций $\hc{f_\al}$ называется \emph{локально равностепенно непрерывным}, если для $\fa \ep > 0$ и
любого компакта $K \Subset D$ найдётся $\de$ такое, что при $\fa z_1,z_2\in K$, для которых $|z_1 - z_2| <\de$,
выполняется условие $|f_\al(z_1)-f_\al(z_2)|< \ep$ для любого $\al$.
\end{df}

\begin{note}
Слово <<локально>> мы далее писать не будем для краткости.
\end{note}

\begin{df}
Пусть $K \subs \Cbb$. Тогда \emph{$\rh$\д раздутием} множества $K$ назовём множество
\eqn{K_\rh := \cups{z \in K}\ol U_\rh(z).}
\end{df}

Смысл определения понятен: множество $K$ раздувается на величину, не превосходящую $\rh$.

\begin{lemma}
Если семейство функций $\hc{f_\al}$, голоморфных в области $D$, равномерно ограничено, то оно и равностепенно
непрерывно в ней.
\end{lemma}
\begin{proof}
Очевидно, расстояние от любого компакта $K$ до границы области больше нуля. Тогда найдётся достаточно малое число $\rh$,
при котором $K_\rh \Subset D$. В силу равномерной ограниченности найдётся число $M$, для которого $|f_\al(z)|\le M$ для $\fa z \in K_\rh$.

Рассмотрим произвольные точки $a,b\in K$ такие, что $|a-b|<\rh$. Тогда имеем $U_\rh(a) \subs K_\rh$ по построению $K_\rh$.
Значит,
\eqn{|f_\al(z)-f_\al(a)| \le |f_\al(z)| + |f_\al(a)|\le 2M}
для любой точки $z \in U_\rh(a)$ и для $\fa \al$. Теперь переведём этот круг в единичный круг $\De$ с центром
в нуле, то есть сделаем линейную замену переменной $\ze = \frac{1}{\rh}(z-a)$. Тогда функция
\eqn{g_\al(\ze) := \frac{1}{2M}\br{f_\al(a + \rh \ze)-f_\al(a)}}
будет удовлетворять условиям леммы Шварца, а значит, $|g_\al(\ze)| \le |\ze|$ для $\fa \ze \in \De$. Возвращаясь
к функции $f_\al$, получаем
\eqn{|f_\al(z)-f_\al(a)|\le \frac{2M}{\rh}|z-a| \text { для } \fa z \in U_\rh(a) \text { и для } \fa \al.}
Значит, выбирая достаточно малое $\de$, можно добиться того, чтобы выполнялось неравенство $|f(a)-f(b)|\le \ep$ при $|a-b|<\de$.
Например, можно взять $\de \le \min \hr{\rh, \frac{\ep\rh}{2M}}$.
\end{proof}

Из функционального анализа читателю, должно быть, известно следующее

\begin{df}
Бесконечное множество функций называется \emph{компактным}, если из любой его последовательности можно
выделить подпоследовательность, сходящуюся к функции из этого семейства, и \emph{предкомпактным},
если его замыкание компактно.
\end{df}

\begin{theorem}[Принцип компактности, теорема Монтеля]
Если семейство голоморфных функций $f_\al$ локально равномерно ограничено в области $D$, то оно предкомпактно в $D$.
\end{theorem}
\begin{proof}
Рассмотрим счётную всюду плотную в $D$ последовательность точек $\hc{a_p}$ (скажем, точки с рациональными координатами).
Пусть $\hc{f_n}\subs\hc{f_\al}$\т произвольная последовательность функций. Рассмотрим числовую
последовательность $\hc{f_n(a_1)}$. Она ограничена, а потому содержит сходящуюся подпоследовательность.
Пусть $\hc{n^{(1)}}$\т подмножество её индексов. Далее, из этой последовательности выберем подпоследовательность,
сходящуюся на точке $a_2$, и так далее. В итоге получатся последовательности функций $f_n^{k}$, сходящиеся на первых $k$
точках из множества $\hc{a_p}$:
\begin{align*}
f_1^1, f_2^1, f_3^1, \ldots\\
f_1^2, f_2^2, f_3^2, \ldots\\
f_1^3, f_2^3, f_3^3, \ldots
\end{align*}
Тогда последовательность $f_n^n$ сойдётся на всех этих точках. Для краткости обозначим её снова через $f_n$.

Остаётся показать, что наша последовательность сходится на самом деле во всех точках компакта. В~силу
равностепенной непрерывности найдётся такое $\de$, что $|f_n(z_1)-f_n(z_2)|<\ep$ при
$|z_1-z_2|<\de$. Покроем компакт $K \Subset D$ блинами $\De_i$ радиуса $\frac{\de}{2}$ и выделим конечное
подпокрытие. Рассмотрим произвольную точку $z$ в одном из блинов, тогда в силу равностепенной непрерывности
для $\fa z_1,z_2\in \De_i$ будет выполняться $|f_n(z_1)-f(z_2)|<\ep$. Запишем критерий Коши:
\begin{multline*}
|f_m(z)-f_n(z)| = |f_m(z) - f_m(a_j) + f_m(a_j) - f_n(a_j) + f_n(a_j) - f_n(z)|\le\\\le
 |f_m(z) - f_m(a_j)| + |f_m(a_j) - f_n(a_j)| + |f_n(a_j) - f_n(z)|.
\end{multline*}
Первое и третье слагаемое сколь угодно малы, так как точка $z$ приближается точками $a_j$ ввиду их всюду плотности.
Второе же слагаемое тоже можно устремить к нулю, так как на точках $a_j$ имеется сходимость. Осталось заметить, что
эта сходимость будет равномерной на $K$.
\end{proof}

\subsubsection{Доказательство теоремы Римана}

\begin{theorem}[Римана о конформных отображениях]
Пусть $D$\т односвязная область. Тогда она конформно эквивалентна одному из следующих множеств:
\begin{items}{-2}
\item $\Card \pd D = 0 \; \Ra \; D \sim \ol{\Cbb}$;
\item $\Card \pd D = 1 \; \Ra \; D \sim \Cbb$;
\item $\Card \pd D > 1 \; \Ra \; D \sim \De := \hc{z\cln |z|<1}$.
\end{items}
\end{theorem}
\begin{proof}
В первом случае область $D$ просто совпадает с $\ol\Cbb$, и даже ничего отображать не надо, во втором достаточно загнать
единственную точку $a$ границы в бесконечность преобразованием $\frac{1}{z-a}$, и мы получим $\Cbb$.
Остался нетривиальный третий случай, когда точек на границе хотя бы две. Тогда загоним их в точки $0$ и $\bes$.
Теперь применим преобразование $\sqrt{z}$. Так как точки $0$ и $\bes$ не лежат в области, то отображение будет конформным
в силу того, что область $D$ односвязна и по теореме о монодромии (см. главу <<Аналитическое продолжение>>) функция $\sqrt{z}$
допускает выделение двух однозначных ветвей $\ph_1$ и $\ph_2$(отличающихся знаком). Значит, образы $\ph_1(D)$ и $\ph_2(D)$
не пересекаются (предположим противное, тогда $\ph_1(z_1)=\ph_2(z_2)$, а так как это ветви квадратного корня, то
$z_1=z_2$ и $\ph_1(z_1) = -\ph_2(z_1)$, чего быть не может, так как $\ph_i(z)\neq 0$ на области $D$).
Далее, рассмотрим область $\ph_2(D)$, и так как она открыта, то содержит некоторый круг с центром в точке $a$.
Сделаем преобразование $\frac{r^2}{z-a}$ (вывернем круг наизнанку), тогда образ $\ph_1(D)$ попадёт в этот круг.
Таким образом, можно считать, что область $D$ исходно содержалась
в некотором круге. Без ограничения общности можно считать, что это единичный круг с центром в нуле.

Теперь рассмотрим семейство $\Sc$ функций $f_\al$, однолистных в области $D$, причём  таких, что $|f_\al(z)| < 1$ $\fa z \in D$. Это множество не пусто. Найдём среди них функцию, у которой достигается максимум производной в нуле и покажем, что это та самая функция, которая отображает область $D$ на единичный круг.

Пусть $f \in \Sc$. По неравенству Коши $|f'(0)| \le \frac{1}{\rh}$, где $\rh$\т ненулевой радиус
сходимости ряда для $f$, а значит, множество $\bc{|f_\al'(0)|}$ ограничено и имеет точную верхнюю грань. Пусть она достигается на некоторой последовательности $\hc{f_n}$. По теореме Монтеля наше семейство предкомпактно, а потому можно выделить сходящуюся к некоторой (голоморфной) функции $F$ подпоследовательность. Функция $F$ не постоянна, так как $f'(0)\neq 0$ (функции\д то однолистные!). Остаётся показать, что она осуществляет отображение на весь круг.
Заметим сначала, что $F(0)=0$. В самом деле, пусть $F(0)=c\neq 0$. Тогда рассмотрим функцию
\eqn{g(z) := \frac{F(z) -c}{1 - \ol c F(z)}.}
Имеем
\eqn{|g'(0)| = \frac{1}{1- |c|^2} \cdot |F'(0)| \ge |F'(0)|.}
Это противоречит экстремальному свойству $F$, так как $|F(z)|< 1$ и стало быть, функция $g$ также попадёт в наше семейство.

Пусть нашлась точка $b$ в круге, для которой $F(z)\neq b$. Рассмотрим функцию
\eqn{\psi(z) := \sqrt{\frac{F(z) -b}{1 - \ol b F(z)}}.}
Это композиция конформного автоморфизма, переводящего точку $b$ в нуль, с корнем. Так как <<симметричное>> к $b$
значение $b^* = \frac{1}{\ol b}$ функцией $F$ не принимается (оно вообще вне круга лежит), то у функции $\psi$
выделяется однозначная ветвь. Она опять\д таки лежит в семействе $\Sc$. Пусть $\psi(0) = d$. Тогда
функция
\eqn{h(z) := \frac{\psi(z) - d}{1 - \ol d \psi(z)}}
будет иметь производную в нуле побольше, чем у $F$:
\eqn{|h'(0)| = \frac{1 + |b|}{2\hm{\sqrt{-b}}} \cdot |F'(0)| > |F'(0)|,}
ибо $|b|<1$ и $1+|b|>2\hm{\sqrt{-b}}$. Получилось противоречие. Значит, $F(D) = \De$.
\end{proof}

\begin{note}
Конечно, въедливый читатель спросит: а почему, собственно говоря, нуль лежит в области? Но понятно, как с этим
бороться: если не лежит, применим какой\д нибудь конформный автоморфизм, переводящий некоторую точку $x\in D$ в нуль.
\end{note}

\subsection{Нормировка конформных отображений. Теорема Каратеодори}

\subsubsection{Нормировка конформного отображения}

\begin{lemma}
Пусть $a,A\in\De$, $b\in\pd\De$, $\ph\in[0,2\pi]$.
Тогда существует и единственно конформное отображение $f\cln\De\ra\De$, такое что $f(a) = A$ и  $\Arg(f'(a))=\ph$ или $f(b)=B$.
\end{lemma}
\begin{proof}
Докажем, что существует автоморфизм круга с этими условиями. Действительно, мы знаем все автоморфизмы круга:
\eqn{\hc{T(z) = e^{i\ta}\frac{z-a}{1-\ol{a}z}}, \text{ где } |a| < 1, \quad \ta \in \R.}
Точку $a$ можно отправить в $0$, а затем в $A$. (например так: $\frac{z+A}{1+\ol{A}z}\circ\frac{z-a}{1-\ol{a}z}$).
Далее, домножая на $e^{i\ph}$, добьёмся второго условия. Очевидно, что такой автоморфизм единственен, а любое
конформное отображение круга на себя и есть автоморфизм.
\end{proof}

\begin{theorem}
Пусть $D_1,D_2$ \т области у которых хотя бы две точки на границах, $a\in D_1$, $A\in D_2$, $b\in\pd D_1$, $B\in\pd D_2$ и $\ph\in[0,2\pi]$. Тогда существует и единственно конформное отображение $f\cln D_1 \ra D_2$, такое что $f(a) = A$ и  $\Arg(f'(a))=\ph$ или $f(b)=B$.
\end{theorem}
\begin{proof}
По теореме Римана области конформно эквивалентны кругу. Пусть $\ph_1\cln D_1 \ra \De$, $\ph_2\cln D_2 \ra \De$ \т конформные отображения. Обозначим $a'=\ph_1(a)\in \De$, $A'=\ph_2(A)\in \De$, $b'=\ph_1(b)\in\pd \De$, $B'=\ph_1(B)\in\pd \De$, $\ph'=(\frac{\ph'_2(A)}{\ph'_1(a)}\ph$ mod $2\pi) \in[0,2\pi]$. По лемме существует и единственно конформное отображение $F$, что $F(a') = A'$ и  $\Arg(F'(a'))=\ph'$ или $F(b')=B'$ соответственно. Тогда $\ph_2^{-1}\circ F\circ \ph_1$ будет осуществлять искомое конформное отображение. Очевидно, что оно единственно.
\end{proof}


\subsubsection{Теорема Каратеодори. Соответствие границ при конформном отображении}

\begin{df}
\emph{Достижимой граничной точкой} называется пара $(a,\ga_a)$, где $a\in\pd D$\т граничная точка, а
$\ga_a$\т кусочно\д гладкая кривая из точки $A \in D$ в точку $a$ такая, что $\br{\ga_a \wo \hc{a}}\subs D$.
Причём две такие пары $(a,\ga_a)$ и $(b,\ga_b)$ считаем эквивалентными, если $a=b$
и существует окрестность $U$ такая, что $\ga_a \wo \hc{a}$ и $\ga_b \wo \hc{b}$ попадают в одну компоненту
связности $D\cap U$.
\end{df}

\begin{note}
Область $D$, объединённую со своими достижимыми точками, можно снабдить
структурой метрического пространства, взяв в качестве расстояния точную нижнюю грань
кусочно\д гладких кривых, соединяющих две точки.
На самой области $D$ эта метрика даёт ту же топологию, что и была.
\end{note}

Следующую теорему мы оставим без доказательства.
\begin{theorem}[Каратеодори]
Пусть $D_1$ и $D_2$\т области с жордановыми границами, и
$f\cln D_1\ra D_2$\т конформное отображение. Тогда $f$ продолжается до гомеоморфизма между $\ol{D}_1$ и $\ol{D}_2$.
\end{theorem}

\begin{note}
\dmvnpicrh{pictures.120}{2}
Для произвольных областей утверждение неверно. Рассмотрим, например, открытый полукруг и отображение $f(z)=z^2$.
Оно, очевидно, переводит его в круг с <<разрезом>>\т отрезком $[0,1]$. Это отображение не продолжается
до гомеоморфизма на границу.
\dmvnpicra{pictures.120}{13}{-1.5pc}
В самом деле, рассмотрим точки на отрезке $[0,1]$. Гомеоморфизм\т отображение, непрерывное
в обе стороны, поэтому если чуть\д чуть отступить вверх от этого отрезка (в образе), то прообраз точки тоже далеко не уедет.
Если мы отступим вверх, то так оно и будет, а если вниз, то прообраз будет лежать где\д то недалеко от отрезка $[-1,0]$.
Короче говоря, расстояние между прообразами будет очень большим, даже если мы отступим от отрезка на очень маленькое
расстояние.
\end{note}

\dmvnpicrh{pictures.130}{-1}
Ещё одна формулировка теоремы Каратеодори такова: условие жордановости
границ заменяется на условие достижимости всех граничных точек.

\begin{note}
\dmvnpicrh{pictures.130}{-3}
Теорема \dmvnpicra{pictures.130}{14}{-3pc} перестаёт быть верной и в том случае, когда границы областей не являются жордановыми. Например, если в качестве
одной из областей взять область, у которой один из кусков границы представляет собой график функции
$\sin \frac{1}{x}$, то гомеоморфное продолжение невозможно.
\end{note}

\begin{theorem}[<<обратная>> к теореме Каратеодори]
\dmvnpicrh{pictures.130}{-2}
Пусть $D_1$ и $D_2$\т две области, с кусочно\д гладкой границей, $f\in\Oc(D_1) \cap \Cb(\ol{D}_1)$,
причём функция $f\cln \ol{D}_1\ra \ol{D}_2$ устанавливает гомеоморфизм границ областей.
Тогда $f$ конформно отображает $D_1$ на $D_2$.
\end{theorem}
\begin{proof}
Нужно доказать однолистность функции $f$. Фиксируем точку $b \in D_2$.
Применим принцип аргумента. Покажем, что функция $f(z)-b$ имеет ровно один нуль кратности $1$. Имеем

\eqn{N_{f(z)-b} = \frac{1}{2\pi} \Varl{\pd D_1} \Arg (f(z)-b).}

Так как у нас есть непрерывное и взаимно\д однозначное соответствие границ, то когда $z$ пробегает всю границу, то
вектор $f(z)-b$ поворачивается вокруг себя на $2\pi$. Значит, $N_{f(z)-b} = \frac{1}{2\pi} \cdot2\pi =1$.
Значит, функция $f(z)-b$ имеет ровно один нуль, то есть прообраз точки $b$ единственный.
\end{proof}

\subsection{Принцип симметрии}

Как следует из заглавия, речь далее пойдёт о симметричных областях и об их конформных отображениях.
Если не указано, относительно чего производится симметричное отражение, под \emph{симметричной}
областью мы будем понимать область, симметричную данной относительно вещественной оси.

\subsubsection{Вспомогательные леммы}

\begin{lemma}
Если $f\in\Oc(D)$, то функция $\wt f := \ol{f(\ol{z})}$ голоморфна в симметричной области $D^*$.
\end{lemma}
\begin{proof}
Пусть $a \in D^*$, тогда $\ol a \in D$. Из голоморфности функции $f$ в окрестности $\ol{a}$ следует
разложимость в ряд:
\eqn{f(z)=\sum c_n(z-\ol{a})^n.}
Тогда
\eqn{\wt f = \ol{f(\ol{z})}= \ol{\sum c_n(\ol{z}-\ol{a})^n}= \sum \ol{c}_n(z-a)^n.}
Полученный ряд имеет тот же радиус сходимости, откуда и следует голоморфность.
\end{proof}

\begin{lemma}
Даны две непересекающиеся области, такие что $\pd D_1 \cap \pd D_2=\ga$, где $\ga$\т отрезок прямой
или дуга окружности, и  $f_1 \in \Oc(D_1) \cap \Cb(D_1 \cup \ga)$, а $f_2 \in \Oc(D_2) \cap \Cb(D_2 \cup\ga)$, причём $f_1$ и $f_2$ совпадают на $\gamma$.
Тогда функция
\eqn{f(z)= \case{f_1(z), & z \in D_1 \cup\ga;\\
f_2(z), & z \in D_2 \cup\ga}}
голоморфна в $D_1 \cup \ga \cup D_2$.
\end{lemma}
\begin{proof}
\dmvnpicrh{pictures.140}{-7}
Достаточно\dmvnpicra{pictures.140}{15}{-3pc} доказать, что для любого треугольного
контура $\ints{\tri} f(z)\,dz=0$. Если~$\tri$ не
пересекает $\ga$, то это верно, так как функция в каждой из областей голоморфна.
Если же треугольник пересекает кривую, то разобьём его на два замкнутых контура $C_1$ и $C_2$, а
участок, идущий по $\ga$, сместим внутрь области на $\ep$, получим $C_1^\ep$ и $C_2^\ep$.
Интеграл по всему треугольнику равен сумме интегралов по кускам, так как интегрирование по границе
идёт в противоположных направлениях. Интегралы по каждому из этих контуров в отдельности равны нулю.
Переходя к пределу при $\ep\ra 0$, получаем, что интегралы по $C_1$ и $C_2$  также равны нулю. Следовательно,
$\ints{\tri}f(z)\,dz=0$.
\end{proof}

\subsubsection{Доказательство принципа симметрии}

\begin{theorem}[принцип симметрии]
Пусть граница области $D_1$ содержит участок прямой или окружности $\ga_1$, а $D_1^*$\т
симметричная относительно $\ga_1$ область, и $D \cap D^*=\es$;
Пусть $(D_2, D_2^*,\ga_2)$\т набор с теми же свойствами. Пусть функция $f\cln D_1\ra D_2$\т конформное
отображение, продолжаемое по непрерывности до взаимно\д однозначного соответствия границ: $f\cln \ga_1 \ra \ga_2$.
Тогда продолженное по симметрии отображение~$f$ даёт конформное отображение
$D_1\cup\ga_1\cup D_1^* \mapsto D_2\cup\ga_2\cup D_2^*$.
\end{theorem}
\begin{proof}
Если $\ga$ является отрезком прямой, то движением можно перевести этот отрезок на вещественную ось, а далее воспользоваться
леммами. Конформность, очевидно, сохранится (движение ничего не испортит). Если же $\ga$ является дугой окружности, то
сначала дробно\д линейным преобразованием <<распрямим>> её, передвинем на вещественную ось, сведя тем самым к первому случаю.
\end{proof}

\begin{note}
В некоторых случаях продолжение функции может оказаться мероморфным. Например, точка, симметричная нулю относительно
единичного круга,\т это бесконечность. Поэтому функция,  для которой найдётся точка $a$, такая что $f(a)=0$, будет
иметь полюс в точке $a^*$. Это будет полюс первого порядка, так как функция однолистна.
\end{note}

\begin{note}
Если разрешить пересечение симметричных областей, то произойдёт потеря однолистности.
\end{note}

\begin{problem}
Доказать, что конформное отображение одного прямоугольника на другой, переводящее вершины в вершины,
может быть только линейным.
\end{problem}
\begin{solution}
Многократным применением принципа симметрии можно продолжить наше отображение до отображения плоскости на плоскость.
Как мы знаем, конформное отображение плоскости на себя может быть только линейным.
Но линейная функция не может переводить вершины в вершины (это может быть только в том случае, когда прямоугольники
подобны).
\end{solution}

\begin{problem}
Доказать, что неконцентрическое кольцо конформно эквивалентно концентрическому.
\end{problem}
\begin{solution}
Развернём внешнюю окружность неконцентрического кольца в прямую, затем сделаем сдвиг, а потом свернём обратно.
\end{solution}

\begin{ex}
Рассмотрим конформное отображение двух концентрических колец. Путём многократного применения принципа симметрии
можно продолжить его до отображения $\Cbb\wo\hc{0} \lra \Cbb \wo\hc{0}$. Точка $0$ будет устранимой,
так как при $z\ra0$ образ точки $z$ тоже стремится к нулю. Получилось конформное отображение плоскости на себя,
а оно линейно. Значит, конформное отображение двух колец (без <<выворачивания>>) возможно в том случае, когда
радиусы пропорциональны.
\end{ex}

%\begin{petit}
%Тут что\д то не очень понятно было в лекциях... Если у кого лучше записано, скажите!
%\end{petit}

Рассмотрим функцию $f$, мероморфную в единичном круге $\De$, причём $f(e^{i\ta})=1$. Пусть $\hc{a_i}$\т её нули,
а $\hc{b_j}$\т полюса, причём на границе круга их нет.
Тогда можно мероморфно продолжить функцию по симметрии на всю расширенную плоскость. Мы знаем, что это это будет
рациональная функция. Умножением её на дроби вида
\eqn{\frac{z-a_i}{1-\ol{a}_i z}}
можно убрать полюса, а делением на дроби такого же вида можно убрать нули. Но такая функция в $\Cbb$ может быть только
постоянной. Значит, исходная функция также была произведением дробей такого вида.

\section{Аналитическое продолжение}

\subsection{Росток функции}

\subsubsection{Многозначные функции. Понятие ростка и аналитического продолжения}

Примерами многозначных функций являются $\log z$, $\sqrt[n]{z}$.

Одним из самых важных понятий, которые нам будут нужны, является понятие ростка функции.

\begin{df}
Рассмотрим две функции $f_1$ и $f_2$, определённые в окрестностях $V_1(a)$ и $V_2(a)$ точки $a$. Введём отношение
эквивалентности: $(f_1, V_1) \sim (f_2, V_2)$, если функции совпадают в некоторой окрестности $U \subs V_1 \cap V_2$
точки $a$. \emph{Ростком функции} в точке $a$ называется класс функций $[f]_a$ по этому отношению.
\end{df}

У голоморфных функций есть канонический представитель ростка\т степенной ряд.
Каждая функция, определённая в окрестности $V$, порождает некоторый росток и является его представителем.

\begin{df}
Пусть $\ga\cln[\al,\be]\ra\Cbb$\т непрерывная кривая, и $\ga(\al)=a$, $\ga(\be)=b$.
Росток $[f]_b$ является \emph{продолжением ростка $[f]_a$ вдоль $\ga$}, если существует семейство ростков
$\hc{[\ph]_{\ga(t)}}$ таких, что для любой точки $t_0\in[\al,\be]$ найдётся $\de$ такое, что для $\fa t \in U_\de(t_0)$
ростки этого семейства порождены некоторой одной функцией $F$, \те $\exi (F,V)\cln$ росток $[\ph]_{\ga(t)}$ порождается
парой $(F, V)$, и выполнено условие
\eqn{\label{GermContinuation}[\ph]_{\ga(\al)} = [f]_a, \quad [\ph]_{\ga(\be)} = [f]_b.}
\end{df}

Применительно к голоморфным функциям это определение переформулируется так:

\begin{df}
Росток $[f]_b$ является \emph{аналитическим продолжением ростка} $[f]_a$, если существует семейство
$\hc{[\ph]_{\ga(t)}}$ ростков голоморфных функций, локально порождаемых некоторой голоморфной функцией~$F$
и выполняется условие \eqref{GermContinuation}.
\end{df}

\begin{stm}
Аналитическое продолжение вдоль заданной кривой $\ga$ единственно, то есть если имеется два продолжения $[f_1]_b$ и $[f_2]_b$
ростка $[f]_a$ функции $f$, то они совпадают.
\end{stm}
\begin{proof}
Пусть $\hc{[\ph_1]_{\ga(t)}}$ и $\hc{[\ph_2]_{\ga(t)}}$\т семейства ростков, породившие продолжения $[f_1]_b$ и $[f_2]_b$
соответственно. Рассмотрим множество
\eqn{E:=\hc{t\in [0,1] \cln [\ph_1]_{\ga(t)}=[\ph_2]_{\ga(t)}}.}
Оно не пусто, так как $0\in E$. Оно открыто, так как если ростки совпадают в точке $\ga(t_0)$, то их представители локально совпадают в окрестности
точки $\ga(t_0)$. Значит, точки из $U(t_0)$ также принадлежат множеству $E$. Но множество замкнуто, так как в силу теоремы
единственности, если функции совпадают на последовательности точек, стремящихся к $\ga(t_0)$, то они совпадают и в самой точке
$\ga(t_0)$. Следовательно, $E=[0,1]$.
\end{proof}

\subsubsection{Теорема о монодромии}

Как только дано определение аналитического продолжения, сразу может возникнуть вопрос: а зависит ли оно от кривой,
по которой мы осуществляем продолжение? Следующая теорема даёт на него ответ.

\begin{theorem}[о монодромии]
Пусть $D$\т область, и $a,b\in D$. Если две кривые с фиксированными концами $a$ и $b$ гомотопны (в области $D$),
то аналитическое продолжение ростка $[f]_a$ определено однозначно.
\end{theorem}
\begin{proof}
Пусть гомотопия задана как отображение $\ga=\ga(t,\tau) \in \Cb([0,1]^2)$, где $t$\т параметр кривой, а $\tau$\т параметр
деформации. Пусть $[f_1]_b$ и $[f_2]_b$\т два продолжения ростка $[f]_a$. Достаточно доказать, что они совпадают при
малом изменении параметра $\tau$ (так как из того, что функция локально не изменилась, будет следовать, что она не изменилась
при всех $\tau$). Будем действовать аналогично доказательству предыдущего утверждения, но теперь в качестве
параметра будет выступать $\tau$. Из каждой точки кривой $\ga(t)$ осуществим аналитическое продолжение вдоль кривой $\ga(t_0,\tau)$
при каждом фиксированном $t_0$. Так как продолжение вдоль кривой единственно, то найдётся система достаточно малых кружочков\т
окрестностей каждой точки $t_0\in\ga(t,0)$, таких, что в любой точке, лежащей в их объединении, продолженная функция совпадает
с исходной. Но если параметр гомотопии достаточно мал ($\tau \le \de$, где $\de>0$), то кривая $\ga(t,\tau)$ лежит в объединении
кружочков, так как кривая компактна. Значит, при $\tau\le\de$ аналитическое продолжение совпадает с исходной функцией
на кривой $\ga_\tau(t)= \ga(t,\tau)$. Продолжая делать маленькие шаги по параметру $\tau$, дойдём до $\tau=1$
(снова пользуемся компактностью).
\end{proof}

\begin{df}
\emph{Полной аналитической функцией (ПАФ)}, порождённой ростком  $[f]_a$, называется
совокупность ростков, полученных при продолжениях по всевозможным путям.
\end{df}

\begin{note}
Во всех проведённых выше рассуждениях совершенно не важно, где всё происходит: на плоскости или на сфере.
\end{note}

\begin{theorem}[Вольтерра]
Множество ростков одной ПАФ в фиксированной точке не более, чем счётно.
\end{theorem}
\begin{proof}
Кривые, вдоль которых мы осуществляем продолжение, можно заменить на ломаные с вершинами в рациональных точках.
Их счётное число. Значит, и ростков не более, чем счётное число.
\end{proof}

\begin{df}
Рассмотрим область $D$ и точку $a\in D$. Если росток $[f]_a$ можно продолжить по любому пути в $D$,
тогда говорим, что данная ПАФ аналитична в~$D$.
При этом совокупность полученных ростков называется \emph{ветвью} ПАФ $f$ в $D$.
\end{df}

\begin{imp}[теоремы о монодромии]
Если область $D$ односвязна, то совокупность ростков ветви порождена $f\in\Oc(D)$
 \end{imp}
 Тогда говорят, что ПАФ допускает выделение в $D$ односвязной ветви.

\begin{imp}
В односвязной области у многозначной функции можно выделить однозначную ветвь, так как все пути гомотопны
и продолжение по любому пути даст одну и ту же голоморфную функцию.
\end{imp}

Но если область не односвязна, то это ещё не значит, что всё так плохо. Подтвердим это примером:

\begin{ex}
Рассмотрим область с разрезом от $0$ до $\bes$ и функции $\sqrt z$ и $\ln z$.
Эта область не является односвязной, а однозначную ветвь выделить можно.
\end{ex}

\subsection{Римановы поверхности}

\subsubsection{Конструкция римановой поверхности}

Многие комплексные функции, если их рассматривать как отображения $f\cln \Cbb\ra\Cbb$, являются многозначными.
Это плохо. Мы хотим построить некоторое множество, такое, чтобы отображение на нём было однозначным. Это множество
мы и назовём римановой поверхностью.

Напомним, что одномерное комплексное многообразие\т это хаусдорфово сепарабельное топологическое пространство,
локальное гомеоморфное некоторой области в $\Cbb$. Функции перехода между картами\т конформные отображения.

\begin{df}
\emph{Римановой поверхностью} голоморфной функции $f$ будем называть топологическое пространство $X$,
точками которого являются пары $(a, [f]_a)$, где $a \in \Cbb$. Топология вводится следующим образом:
$\ep$\д окрестностью точки $(a, [f]_a)$ назовём такие пары $(z, [f]_z)$, что $z \in U_\ep(a)$ и
росток функции $[f]_b$ является непосредственным продолжением ростка $[f]_a$.
\end{df}

Можно определить проекцию пространства $\pi\cln X \ra \Cbb$ очевидным образом: $\pi\br{(z,[f]_z)} = z$.

Фактически, риманова поверхность локально ничем не отличается от множества $\Cbb$, разве что у каждой точки
есть дополнительный <<атрибут>>\т номер листа поверхности, определяемый ростком функции.


Далее мы будем рассматривать следующую ситуацию: имеется ПАФ, которая имеет в $\ol{\Cbb}$ лишь точки
ветвления конечного порядка и конечное число значений в каждой точке.

\subsubsection{Свойства римановой поверхности}

\begin{theorem}
Риманова поверхность является одномерным комплексным многообразием.
\end{theorem}
\begin{proof}
Проекция $\pi$, определённая выше, очевидно, есть локальный гомеоморфизм. Стало быть, он отображает
круговую окрестность точки на римановой поверхности на некоторый круг в $\Cbb$. Тогда функции перехода
будет тождественными отображениями (так как проекции пересечения двух окрестностей просто наложатся друг на друга).
\end{proof}

\begin{imp}
Риманова поверхность $X$\т это сфера с ручками.
\end{imp}
\begin{proof}
То, что это комплексное многообразие, уже доказано. Теорема классификации двумерных, замкнутых, компактных
связных гладких вещественных многообразий гласит, что любое такое многообразие, если оно, вдобавок ко всему,
ориентируемо, гомеоморфно сфере с некоторым количеством ручек (см. \cite[гл.~4, \S~5, п.~3]{fomenko}). Покажем, что
риманова поверхность ориентируема и компактна. В самом деле, матрица овеществления комплексной производной (то
есть матрицы Якоби для случая $\dim_\Cbb X = 1$) имеет положительный определитель (см. \ref{ConformMappings}).
Значит, наша поверхность (как вещественное многообразие) двумерно и ориентируемо.

Докажем, что риманова поверхность компктна. Рассмотрим произвольную последовательность точек $z_n \in X$ и
их проекций $\pi(z_n)$. В силу компактности сферы из последовательности $\pi{z_n}$ можно выделить
сходящуюся подпоследовательность $\pi{z_{n_k}}$. Но тогда прообразы точек этой последовательности <<прыгают>> по
конечному множеству листов, и из них, очевидно, тоже можно выделить сходящуюся (в топологии $X$) подпоследовательность.
\end{proof}

\subsection{Изолированные особые точки}

\subsubsection{Классификация изолированных точек и её корректность}

\begin{df}
Точка $a$ называется изолированной особой точкой аналитической функции $f$, если в некоторой проколотой
окрестности $V$ этой точки некоторый росток $[f]_a$ продолжается по всем путям, в окрестности~$V$.
\end{df}

Иными словами, найдётся такая окрестность точки $a$, что кроме самой точки $a$ там никаких особых точек нет
и аналитическое продолжение возможно.

\begin{df}
Рассмотрим замкнутый путь $\ga$ вокруг изолированной особой точки и продолжим функцию по нему. При этом возможно
три случая:

\begin{nums}{-2}
\item После однократного обхода мы возвращается в тот же класс эквивалентности (тот же росток);
\item После $n$ оборотов мы возвращается к тому же ростку, \те получается $n$ (неэквивалентных) ростков;
\item При очередном обходе точки каждый раз получаются новые ростки.
\end{nums}

Соответственно, эти случаи называются так: \emph{точка однозначного характера}, \emph{точка ветвления порядка $n$} и
\emph{логарифмическая особая точка}.
\end{df}

\begin{stm}
Данное выше определение корректно, то есть тип точки не зависит от:
\begin{items}{-2}
\item выбора направления обхода кривой;
\item выбора представителя в классе гомотопных кривых;
\item выбора точки на кривой, с которой начинается продолжение.
\end{items}
\end{stm}
\begin{proof}
В самом деле, если $\ga^+$ и $\ga^-$\т одна и та же кривая, но с противоположными направлениями обхода, то
кривая $(\ga^+ \circ \ga^-)$ гомотопна нулю. Поэтому первое и второе утверждение следует из теоремы о монодромии.
Что касается третьего то оно тоже из неё следует, так как можно <<сдвинуть>> параметризацию кривой так, чтобы
произвольная точка на кривой оказалась в начале (то есть соответствует значению $t=0$), а такие кривые гомотопны.
\end{proof}

Название точки третьего типа объясняется тем, что у функции $\ln z$ нуль является особой точкой именно такого типа.

\subsubsection{Примеры}

\begin{ex}
У функции $\sqrt{z}$ имеется две изолированных точки: $0$ и $\bes$. Они являются точками ветвления второго порядка.
\end{ex}

\begin{ex}
Рассмотрим функцию $\sqrt{1-\sqrt{z}}$.
Имеем $\Var\arg \br{1- (+\sqrt z)}=2\pi$ и $\Var\arg \br{1-(-\sqrt z)}=0$. Значит, точка $z=1$ является особой
не для всех ветвей.
\end{ex}

\begin{ex} У функции $\frac{\sin\sqrt{z}}{\sqrt{z}}$ нуль\т однозначная точка. Если взять ветвь корня со знаком <<минус>>, то
$$\sqrt{z} \rightsquigarrow -\sqrt{z} \quad \Ra \quad
\frac{\sin\sqrt{z}}{\sqrt{z}} \rightsquigarrow \frac{-\sin\sqrt z }{-\sqrt z}=\frac{\sin\sqrt z}{\sqrt z},$$
то есть вернулись к той же функции.
\end{ex}

\begin{petit}
Риманову поверхность для логарифма нарисуем позже. Пока см. \cite[стр. 195]{shabat}.
\end{petit}

\subsection{Алгебраические особые точки и алгебраические функции}

\subsubsection{Ряды Пюизо. Алгебраические особые точки}

Поймём, откуда берутся точки ветвления конечного порядка.
Пусть $a$\т точка ветвления порядка $n$ функции $f(z)$. Сделаем замену переменной
$z = a + \ze^n$. Пусть точка $\ze$ бегает по достаточно малой окружности радиуса $\rh$.
Тогда при одном обороте $\ze$ точка $z$ накрутит $n$ оборотов вокруг точки $a$.
Но так как точка $a$ имеет порядок ветвления $n$, то после $n$ оборотов мы приходим к тому же
ростку, с которого и начали. Значит, точка $a$ будет однозначной точкой относительно переменной
$\ze$. Но отсюда следует, что в проколотой окрестности $\ze = 0$ имеет место представление рядом Лорана:
\eqn{f(\ze) = \sums{\Z} c_k \ze^k.}
Переходя к переменной $z$, получаем представление функции $f(z)$ в виде так называемого \emph{ряда Пюиз\'о}:
\eqn{f(z) = \sums{\Z} c_k \br{\sqrt[n]{z-a}}^k = \sums{\Z} c_k (z-a)^\frac k n.}

\begin{df}
Если разложение функции в ряд Пюизо содержит конечное число отрицательных членов, то
соответствующая особая точка называется \emph{алгебраической}.
\end{df}

\begin{df}
Полная аналитическая функция называется \emph{алгебраической}, если она конечнозначна и имеет конечное число
особых точек, причём все они алгебраические.
\end{df}

Пусть $A = \hc{a_1\sco a_n}$\т особые точки некоторой алгебраической функции. Тогда над каждой точкой $\ol\Cbb\wo A$ имеется
одно и то же число ростков (<<листов>>). В самом деле, пусть нашлись две точки $z_1$ и $z_2$, над которыми висит
разное число листов, и для определённости над $z_1$ их больше. Теорема единственности не позволяет росткам склеиваться,
поэтому при продолжении их вдоль пути из точки $z_1$ мы не получим меньшее их число.

\begin{df}
Число <<листов>> называется \emph{порядком} алгебраической функции.
\end{df}

\subsubsection{Критерий алгебраичности}

\begin{theorem}Значения алгебраической функции порядка $n$ удовлетворяют уравнению
\eqn{p_0(z)w^n + p_1(z)w^{n-1} \spl p_n(z) = 0, \quad p_i \in \Cbb[z],}
причём многочлены $p_i$ взаимно просты.
\end{theorem}
\begin{proof}
\pt{1} Частный случай $n=1$ мы уже знаем: у однозначной алгебраической функции особенностями могут быть
лишь полюсы в конечном числе, а тогда она рациональна, то есть является отношением двух многочленов:
$w = -\frac{p_1(z)}{p_2(z)}$, что после умножения на знаменатель и даст искомый вид.

\pt{2} При $n > 1$ фиксируем точку $z\in \ol\Cbb\wo A$. Тогда функция в этой точке принимает значения $f_1(z)\sco f_n(z)$,
причём функции $f_i$ аналитически продолжаются по любому пути, не проходящему через особые точки. Посмотрим, что будет
при обходе особой точки. Каждая из функций $f_i$ может, вообще говоря, изменить своё значение. Но так как число листов
конечно и склеиваться они не могут, значения функций могут лишь переставляться, \те набор $\hc{f_1(z)\sco f_n(z)}$
перейдёт в некоторый набор $\hc{f_{\si(1)}(z)\sco f_{\si(n)}(z)}$, где $\si \in \Sb_n$. Значит, базисные
симметрические функции
\begin{align*}
\si_1 & := f_1 \spl f_n,\\
\si_2 & := f_1f_2 \spl f_{n-1}f_n,\\
\dots\\
\si_n & := f_1\sd f_n
\end{align*}
не изменятся ни при каком обходе. Значит, они являются однозначными и по первому пункту рациональны. Формулы
Виета говорят, что значения $f_1(z)\sco f_n(z)$ суть корни уравнения
\eqn{w^n - \si_1(z)w^{n-1}\spl (-1)^n\si_n(z) = 0.}
Каждая функция $\si_i$ есть отношение многочленов, поэтому домножая на наименьшее общее кратное знаменателей,
получаем уравнение нужного нам вида, а заодно завершаем доказательство теоремы.
\end{proof}

На самом деле теорема верна и в обратную сторону: решения всякого неприводимого многочлена
указанного вида (относительно $w$) есть значения некоторой алгебраической функции (см. \cite[гл. IV, \S 11]{shabat}).

\subsubsection{Формула Римана\ч Гурвица}

Пусть у нас имеется некоторое (конечное) число особых точек. Рассмотрим
триангуляцию сферы, такую, что вершины треугольников
попадут в особые точки. Теперь такое же разбиение сделаем и для римановой поверхности, которая
висит над сферой, так, чтобы вершины разбиения на поверхности висели над вершинами графа на сфере,
а рёбра разбиения поверхности проектировались в рёбра разбиения сферы. Будем считать их количество.
Пусть $В$, $Р$ и $Г$\т числа вершин, рёбер и граней (на поверхности) соответственно, а $в$, $р$ и $г$\т
соответствующие числа для графа на плоскости.

Пусть функция $m$\д значна. Тогда количество граней <<вверху>> равно количеству граней <<внизу>>, умноженному на $m$,
и для рёбер\т то же самое. А вот для вершин соотношение нарушается (листы\д то склеиваются),
и при подсчёте мы ошибёмся на число $\sum (k_j - 1)$, где $k_j$\т индекс (порядок) ветвления, \те
количество склеенных листов. Таким образом, имеем $Р=m\cdot р$, $Г = m \cdot г$ и $В = m\cdot в - \sum (k_j-1)$.

\begin{df}
Число $\chi(M) :=В-Р+Г$ называется \emph{эйлеровой характеристикой} графа $M$.
\end{df}

\begin{stm}
Эйлерова характеристика\footnote{Иногда она именно так и определяется. В общем случае она равна $\suml{k=0}{n} (-1)^k H^k(M^n,\R)$,
где $H^k$\т $k$\д я группа когомологий.} <<хорошего>> многообразия (двумерной поверхности)
$\chi(M)$ равна $2-2g$, где $g$\т число ручек на многообразии.
\end{stm}
\begin{proof}
Мы приведём только идею доказательства. Мы будем использовать тот факт, что $\chi(M)$ совпадает с эйлеровой
характеристикой произвольного связного графа на этом многообразии (но доказывать его не будем).
Вначале покажем, что эйлерова характеристика сферы равна $2$. Рассмотрим произвольный граф и начнём его преобразовывать,
следя за количеством вершин, рёбер и граней. Расплющим сферу так, чтобы она превратилась в два блина, склеенных по краям.
Сотрём одно ребро, тогда количество рёбер уменьшится на $1$, и количество граней тоже уменьшится на $1$. Число $В - Р+Г$
не поменяется. Аналогично, если к вершине идёт 2 ребра, можно стереть вершину и эти ребра, а вместо них поставить
одно длинное (<<убрать излом>>). При этом тоже ничего не поменяется. Рано или поздно останется два треугольника со склеенными
сторонами. Для них $\chi(S^2) = 3 - 3 + 2\bw=2$, что и требовалось.

Если на многообразии имеются ручки, то легко видеть, каждая ручка уменьшает $\chi$ на $2$. Это легко понять на примере
двумерного тора $S^1 \times S^1$, который есть не что иное, как сфера с одной ручкой, и потому его характеристика равна нулю.
\end{proof}

Таким образом, мы получаем \emph{формулу Римана\ч Гурвица:}
\eqn{\case{в -р+г = 2,\\В-Р+Г = 2 -2 g.} \quad \Ra\quad \text{Ф.Р\ч Г:} \quad g = \frac12 \sum(k_j-1) - (m-1).}

\end{document}

\pagebreak

\section{Ещё не исправленное}
\begin{problem}
 \# $\,p^{-1}(z)$ не более чем счётно и дискретно $\forall z\in\ol{\Cbb}$.
\end{problem}

Есть в топологии близкое понятие --- накрытие, но риманова поверхность не
 обязана быть накрытием.

\begin{ex}
 $X=\{\,(z,w):\,\mathcal{F}(z,w)=0\text{ полином  }%
 \frac{\pd\mathcal{F}}{\pd w}(z,w)\ne 0\,\}$. В частности, $w^{2}=z$.
 Эту поверхность можно использовать как риманову поверхность
 $\sqrt{z}$.
\end{ex}

 Структура римановой поверхности: связность, хаусдорфовость.\par\noindent
 $F$ переносится на $R(F)$ очевидным образом. Пусть $\ze\in R(F)$, то есть
 $\ze=(z,[f]_{z})$, тогда  $F_{R}(\ze):=F(z)$.
 \begin{stm}
 $F_{R}$ голоморфна на $R(F)$.
 \end{stm}

 В качестве римановой поверхности $\sqrt{z}$  можно взять график
 \[
 \ga=\{\,w=\sqrt{z}\,\}=\{\,(z,w):\,z\ne 0,\,w=\sqrt{z}\,\}\,.
 \]
 Точка $z=0$ выкалывается, так как через неё $\sqrt{z}$ продолжать нельзя.
 \[
 \ga=\{\,w^{2}=z\,\}\setminus (0,0).
 \]
 Но  $\ol{\ga}$ тоже является
 хорошей поверхностью (комплексным многообразием), точка $(0,0)$ ничем не хуже
 прочих. Такой переход (замыкание)  можно осуществить в окрестности любой точки
 ветвления.\par
 Рассмотрим отображение
 $\Cbb^{1}\ra\Cbb^{2}
 \quad t\mapsto (\,t^{2},t\,)\quad z=t^{2},\: w=t$.
 Это отображение задаёт конформную эквивалентность
 $\Cbb\thicksim\ol{\ga}.$ \par
 Отображение взаимно\д однозначное и голоморфное в обе стороны. Обратное
 отображение  $\ol{\ga}\ra\Cbb\quad(z,w)
 \mapsto w$. \par
 Аналогично: $\Cbb\ra\{\,z=w^{n}\,\}\quad (t^{n},t)$. Эта
 поверхность в нуле также не имеет особенностей (является комплексным
 многообразием).\par\noindent
 [\,В отличии от поверхностей $\{\,w^{2}=z^{2}\,\}$ или $\{\,w^{2}=z^{3}\,\}$\,]

\begin{problem}
 Доказать, что эти множества в окрестности $(0,0)$ не являются комплексными
 многообразиями.
\end{problem}

 Пусть $F$ --- ветвь $A$ в проколотой окрестности
 $\overset{\textbf{.}}{\mathcal{U}}_{a}$ точки $a$ и точка $a$ --- точка
 ветвления порядка $n$. Этой ветви соответствует n--листная риманова
 поверхность над $\:\overset{\textbf{.}}{\mathcal{U}}_{a}$ --- $R(F)$. Тогда
 отображение
 \begin{align} \label{E:otobr}
 z=\ze^{n}+a
 \end{align}
 конформно отображает проколотую окрестность точки 0 в плоскости $\ze$ на
 $R(F)$. Добавим к $R(F)$ одну точку и будем считать, что ей соответствует
 карта, определённая (\ref{E:otobr}). Обозначим полученное комплексное
 многообразие $\widehat{R(F)}$.

 Все наши определения изолированной особой точки, типа особой точки, замыкания
 римановой поверхности в окрестности точки ветвления работают и при $a=\infty$.
 Нужно лишь правильно понимать проколотую окрестность $\infty$ и т.д.\par
 Если смотреть $\sqrt{z}$ на  $\ol{\Cbb}$, то особых точек будет
 две --- 0 и $\infty$, обе являются точками ветвления второго порядка.\par\noindent
 В примере $\sqrt{z^{2}-1}$ $\quad-1$ и 1, $\infty$ для каждой из двух ветвей ---
 полюс первого порядка.\par\noindent
 $z+\sqrt{z^{2}-1}$ \,---\, тоже $\pm 1$, а $\infty$ у одной ветви полюс (первого
 порядка), а у второй ($z-\sqrt{z^{2}-1}$)\, $\infty$ является устранимой особой
 точкой.

 Рассмотрим следующую ситуацию. Имеется ПАФ, которая имеет в
 $\ol{\Cbb}$ лишь точки ветвления конечного порядка и конечное
 число значений в каждой точке. (Из второго условия первое следует.)\par
 Рабочий пример --- суперпозиция корней
 $\left(\text{например, }{\sqrt[3]{1-\frac{1}{\sqrt{z}}}-\sqrt[5]{z^{2}}}\right).$
 \par\noindent Тогда:
 \begin{itemize}
 \item[(1)] Замкнутая риманова поверхность --- компактна.
 \item[(2)] Это многообразие --- ориентируемо.
 \item[(3)] Над каждой неособой точкой одинаковое количество листов римановой
  поверхности.
 \end{itemize}
 \begin{proof}
 \begin{itemize}
 \item[(1)] Следует из того, что для выбора конечного подпокрытия достаточно
 взять конечное подпокрытие  $\Cbb$ и растиражировать его на все листы.
 \item[(2)] Ориентируемость (любого) комплексного многообразия следует из того,
 что $\det (df_{R})=|f'|^{2}>0$
 \item[(3)]  \textit{Просим прощения, ИЛЛЮСТРАЦИИ пока нет}
 \end{itemize}

 \end{proof}

 Если листы в точке пронумеровать $(1,\dots,n)$, то продолжение по замкнутой
 кривой  даёт перестановку
 ${\ph:\,\pi_{1}(\ol{\Cbb}\setminus\{\,a_{1},\dots,a_{m}\,\})}
 \ra S_{n}$. Это --- гомеоморфизм; образ --- группа монодромии
 $F\quad G(F).$ \par
 С точки зрения вещественной топологии, $\widehat{R(F)}$ --- это компактное
 двумерное ориентируемое многообразие. По известной теореме такое многообразие
 гомеоморфно сфере с g  ручками (род).  g --- единств. топ. инвариант.
 $\sqrt{z}$\quad 2 сферы с разрезами $\thicksim$ 2 диска $\quad\ra$
 склейка с сохранением ориентации.\\
 ---- сфера. g=0\\
 $\sqrt{z^{2}-1}$ --- то же самое \\
 $\sqrt{(z-1)(z-2)\cdot\!\!\dots\!\!\cdot(z-2g-1)(z-2g)}$
   --- род = g \par
 Схема римановой поверхности.


\vskip50pt



\subsection{Нормировка конформных отображений}

Пусть $D$ односвязная область $\thicksim\varDelta$. Конформное отображение
 $f:\, D\ra\varDelta$ не единственно.
 $g:\, D\ra\varDelta$ --- другое конформное отображение
 $\Longleftrightarrow \: g=\ph\circ f$, где
 $\ph : \,\varDelta\ra\varDelta$ --- конформный автоморфизм
 $\varDelta$.                                 \\
 \begin{center}
3 способа нормировки:
 \end{center}
 \begin{alignat*}{2}
 &1)\qquad%
 &&f(a)=A\,,\:\arg f'(a)=\al;\: a,A\in\varDelta\,,\,\al\in\R\\
 &2)\qquad &&f(a)=A\quad f(b)=B;\:a,A\in\varDelta\,,\,b,B\in\pd\varDelta\\
 &3)\qquad &&f(a)=A\,,\,f(b)=B\,,\,f(c)=C\quad%
 a,b,c,A,B,C\in\pd\varDelta,
 \end{alignat*}
 причём направление обхода, заданное $(a,b,c)$ совпадает с направлением обхода,
 заданным $(A,B,C)$. (Вопрос: что будет, если его нарушить?) \par
 Для $\varDelta$ существование и единственность проверяется непосредственным
 вычислением. На произвольную область $ D\thicksim\varDelta$ переносятся с помощью
 представления $g=\ph\circ f$.

\end{document}

 \subsection{Применение вычетов}

 \begin{alignat*}{1}
 |\ctg\,\pi z|=
 \left|\frac{e^{\pi y}-e^{-\pi y}}{e^{\pi y}+e^{-\pi y}}\right|\le 1\,,\\
 \text{так как на }C_{n}D_{n}\quad \pi z=\frac{\pi}{2}+\pi n+i\pi y
 \end{alignat*}
 На $[B_{n},C_{n}]\quad\pi z=\pi x+i\pi\al_{n}$
 \[
 |\ctg\,(\pi z)|=
 \left|\frac{1+e^{2\pi iz}}{1-e^{2\pi iz}}\right|=
 \left|\frac{1+e^{-2\pi \al_{n}}e^{2i\pi x}}%
                          {1-e^{-2\pi \al_{n}}e^{2i\pi x}}\right|\le
 \frac{1+e^{-2\pi \al_{n}}}{1-e^{-2\pi \al_{n}}}\le
 \frac{1+e^{-\pi}}{1-e^{-\pi}}
 \]
 Но $|\ctg\,(-\pi z)|=|\ctg\,\pi z|\,,$ следовательно,
 $|\ctg\,\pi z|\mid_{\ga_{n}}$  ограничен, следовательно,
 $\left|\ctg\,\pi z - \frac{1}{\pi z} \right|\le M\,.$\par
 $\ctg\,\pi z$ имеет в 0 простой нуль, $\res_{0}=\pi\,,$ следовательно,
 $f(z)=\ctg\,\pi z - \frac{1}{\pi z}$ --- голоморфна в 0 и нечётна,
 следовательно, $f(0)=0$ \par
 $d_{n}$ --- расстояние от $\ga_{n}$ до 0,
 $l_{n}$ --- длина $\ga_{n}$


Рассмотрим мероморфную на одномерном комплексном многообразии функцию $f(z)$.
 Она задаёт голоморфное отображение $X\ra\ol{\Cbb}$.
 Действительно, пусть $f(a)=b\,,\:a\in X\,,\:b\in\ol{\Cbb}$. Если
 $b$ --- конечно, то  $a$ --- обычная точка и голоморфность $f(z)$ в окрестности
 $a$ очевидна. Если же $b=\infty$, то после перехода к карте в окрестности
 $\infty$, то есть от $w$ к $\ol{w}=\frac{1}{w}$, отображение $w=f(z)$
 принимает вид $\ol{w}=\frac{1}{f(z)}$ --- эта функция ограничена в
 окрестности $a$ и, следовательно, голоморфна. И наоборот, любое голоморфное
 отображение $X\ra\ol{\Cbb}$ определяет мероморфную
 функцию, полюса которой --- это прообраз  $\infty$.
 \begin{stm}
 Пусть $X$ --- одномерное комплексное многообразие, $f\in\mathcal{O}(X)$. Тогда:
 \begin{itemize}
 \item[(i)] (теорема единственности) Если $\exists\, a_{n}\ra a\in X$ такая, что
 $f(a_{n})=0$, то  $f(z)\equiv 0$.
 \item[(ii)] (принцип максимума) Если существует точка нестрогого локального
 максимума модуля,то $f(z)\equiv\mathsf{const}$.
 \item[(iii)] Если $X$ компактно, то $f(z)\equiv\mathsf{const}$.
 \end{itemize}
 \end{stm}

\begin{ex}
 \[
 \frac{\sin z}{1+z^{2}}\,,\:\frac{e^{z}}{1-e^{z}}\,,\:\text{tg}(e^{z}).
 \]
 Исследовать все особые точки в $\ol{\Cbb}$.
\end{ex}


 \subsection{Метод Коши разложения мероморфной функции в ряд из простых дробей.}
 \begin{stm}
 Пусть $f(z)$ мероморфна в $\Cbb$, в 0 голоморфна, все её полюса
 \begin{alignat*}{1}
 &|z_{1}|\le|z_{2}|\le\ldots\text{ --- простые }\notin\pd
                                                                     D_{n},\\
 &0\in D_{1}\subset D_{2}\subset\ldots=\Cbb,\;
 |f(z)|\vert_{\pd D_{n}}\le M\,,
 \end{alignat*}
 тогда
 \[
 f(z)=f(0)+\sum_{k=1}^{\infty}A_{k}%
 \left(\frac{1}{z-z_{k}}+\frac{1}{z_{k}}\right),\text{ где }A_{k}=\res_{z_{k}}f
 \]
 \end{stm}
\begin{proof}
 Рассмотрим
 \begin{align} % alignat*
 &\frac{1}{2\pi i}\int_{\pd D_{n}}\frac{zf(\ze)}{\ze(\ze -z)}d\ze\,,
 \text{ где } z\in  D_{n},\:z\ne z_{k}.\\
 &\res_{0}F+\res_{z}F+\sum_{z_{k}\in D_{n}} \res_{z_{k}}F\\
 &
 \case{\res_{0}F=-f(0) \\
 \res_{z}F=f(z) \\
 \res_{z_{k}}F=\frac{A_{k}}{z_{k}(z_{k}-z)}=
 -A_{k}\left(\frac{1}{z-z_{k}}+\frac{1}{z_{k}}\right),
 }\\
 &
 \text{то есть }f(z)=f(0)+\sum_{z_{k}\in D_{k}}A_{k}%
                             \left(\frac{1}{z-z_{k}}+\frac{1}{z_{k}}\right)+
 \frac{1}{2\pi i}\int_{\pd D_{n}}\frac{zf(\ze)}{\ze(\ze -z)}d\ze.
 \end{align} % alignat*

 Пусть при $n>N$ и $r_{n}>R$
 \[
 \left|
 \frac{1}{2\pi i}\int_{\pd D_{n}}\!\!\frac{zf(\ze)}{\ze(\ze -z)}d\ze
 \right|\le\frac{S_{n}\!\!\cdot\!M}{2\pi r_{n}(r_{n}-R)}\ra 0
 \]
 Для ctg это даёт
 \[
 \pi\!\!\cdot\!\ctg\pi z=\frac{1}{z}+\sum_{k\ne 0,k=-\infty}^{+\infty}
 \left(\frac{1}{z-k}+\frac{1}{k}\right)=
 \frac{1}{z}+\sum_{k=1}^{\infty}\frac{2z}{z^{2}-k^{2}}
 \]
 Сходимость --- правильная,
$f(z)-\sum_{z_{k}\in D_{k}}A_{k}\left(\frac{1}{z-z_{k}}+\frac{1}{z_{k}}\right)$
 сходится на $\ol{ D}_{n}$ равномерно.
 \[
 \frac{\pi}{\sin\pi z}=\frac{1}{z}+\sum_{k=1}^{n}(-1)^{k}%
                               \left(\frac{1}{z-z_{k}}+\frac{1}{z_{k}}\right)
 \text{ --- аналогично.}
 \]
\end{proof}

\subsection{Суммирование рядов}
 \begin{stm}
 Пусть $Q(z)$ --- рациональная функция, такая, что степень числителя $+\, 2\,%
 \le$ степень знаменателя, $z_{1},\ldots,z_{k}$ --- её полюса,
 $z_{1},\ldots,z_{k}\notin\mathbb{Z}$, тогда
 \[
 \sum_{-\infty}^{+\infty}Q(k)=
                            -\pi\sum_{j=1}^{p}\res{z_{j}}[Q(z)\ctg\,\pi z]
 \]
 \end{stm}
 \begin{proof}
 Рассмотрим $I_{n}=\int_{\pd D_{n}}\!\!\!Q(z)\ctg\pi z dz$.\\ При
 $\max_{\pd D_{n}}|Q(z)|\le \frac{M_{1}}{r_{n}^{2}}$
 \[
 |\ctg(z)|\vert_{\pd D_{n}}\le M_{2}\:\ra\:
 |I_{n}|\le\frac{\ln\!\cdot\!M_{1}\!\cdot\!M_{2}}{r_{n}^{2}}\le
 \frac{CM_{1}M_{2}}{r_{n}^{2}}\ra 0
 \]
 Но по теореме о вычетах
 \begin{alignat*}{1}
 &I_{n}=2\pi i\left\{\frac{1}{\pi}%
 \sum_{k=-n}^{n}Q(k)+\sum_{j=1}^{p}\res_{z_{j}}%
 \left[Q(z)\ctg\,\pi z\right]\right\}\\
 &Q(z):=\frac{1}{(z-a)^{2}}\quad a\notin\mathbb{Z}\\
 &\sum_{n=-\infty}^{+\infty}\frac{1}{(n-a)^{2}}=
 -\pi\!\!\cdot\!\res_{a}\left[\frac{\ctg\pi z}{(z-a)^{2}}\right]=
 -\pi(\ctg\pi z)_{a}'=\frac{\pi^{2}}{\sin^{2}\pi a}\\
 &\sum_{n=1}^{\infty}\frac{1}{(n-a)^{2}}+\frac{1}{a^{2}}+
 \sum_{n=1}^{\infty}\frac{1}{(n+a)^{2}}\\
 &\frac{1}{2}\lim_{a\ra 0}\left[
 \frac{\pi^{2}}{\sin^{2}\pi a}-\frac{1}{a^{2}}\right]=
 \sum_{n=1}^{\infty}\frac{1}{n^{2}}=\frac{\pi^{2}}{6}\\
 &\frac{\pi^{2}}{(\pi a-\frac{\pi^{3}a^{3}}{6}+\dots)^{2}}-\frac{1}{a^{2}}=
 \frac{1}{a^{2}}\left[\frac{1}{1-\frac{\pi^{2}a^{2}}{6}+\dots}\right]^{2}
                                                          -\frac{1}{a^{2}}=\\
 &\frac{1}{a^{2}}\left[1+\frac{\pi^{2}a^{2}}{6}+\dots\right]^{2}
                                                          -\frac{1}{a^{2}}=
 \frac{1}{a^{2}}\left(1+\frac{\pi^{2}a^{2}}{3}+\dots\right)-\frac{1}{a^{2}}=
 \frac{\pi^{3}}{3}+\dots
 \end{alignat*}
\end{proof}




\begin{ex}
 \[
 f(z)=\sum_{n=0}^{\infty}z^{n!}=1
 \]
 Пусть $z=re^{2\pi i(\frac{p}{q})}$. Фиксируем $(\frac{p}{q})$, тогда
 \[
 f(z)=\sum_{n=0}^{q-1}z^{n!}+\sum_{n=q}^{\infty}r^{n!}\:\ra\:
 f(z)\ra\infty\text{ при } r\ra 1.
 \]
 То есть $f$ не продолжается из $\varDelta$ в большую область.
\end{ex}

\begin{ex}
\begin{alignat*}{1}
&\frac{1}{1-z}=1+z+\dots\qquad R_{0}=0\\
&\frac{1}{(1-i)-(z-i)}=\frac{1}{z-i}\left(1+\frac{z-i}{1-i}\dots\right)\qquad
R_{i}=\sqrt{2}
\end{alignat*}
\end{ex}







Пусть $a\in D$. Если $f_{1}$ и $f_{2}$ голоморфны в окрестности точки $a$,
 то они являются представителями одного ростка $[f]_{a}\:\Longleftrightarrow$
 совпадают их тейлоровские разложения в $a$.\par
 Если $f_{1},\,f_{2}\in\Oc(D)$ и $[f_{1}]_{a}=[f_{2}]_{a}$, то $f_{1}=f_{2}$.
 Это сразу следует из теоремы единственности.\par
 Для класса непрерывных или гладких функций в $D$ это неверно.
%******Ма-аленький**рисуночек*************************************************
 \begin{itemize}
 \item[А)]  Непосредственное продолжение ростка
 $[f]_{a}\ra \underset{ D}{f}\ra [f]_{b}$
%*************РИСУНОЧЕК*******************************************************
 \begin{center}
 \textit{РИСУНОЧЕК}
 \end{center}
 \item[B)] Продолжение ростка вдоль кривой.
%*************РИСУНОЧЕК*********************************************************
 $\ga=\{\,z=z(t)\,\}$\qquad
 \[
 [f]_{a}\ra [f]_{a_{2}}%
 \ra\dots\ra [f]_{b}
 \]
 \begin{center}
 \textit{РИСУНОЧЕК}
 \end{center}
 \end{itemize}
 Вспомнить логарифм.
