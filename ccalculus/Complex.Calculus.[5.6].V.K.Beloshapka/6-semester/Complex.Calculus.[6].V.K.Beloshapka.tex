\documentclass[a4paper]{article}
\usepackage[utf8]{inputenc}
\usepackage[russian]{babel}
\usepackage{dmvn}

\DeclareMathOperator{\Sk}{Sk}
\def\mcomp#1{\mskip-10mu#1\mskip-10mu}


\begin{document}
\dmvntitle{Курс лекций по}{комплексному анализу}{Лектор-- Валерий Константинович Белошапка}
{III курс, 6 семестр, поток математиков}{Москва, 2005 г.}

\pagebreak
\tableofcontents

\pagebreak

\section*{Предисловие}

Этот документ представляет собой переработанный курс лекций по комплексному анализу, первоначально набранный
одним из студентов МехМата. В~оригинальном варианте не~было иллюстраций, что осложняло восприятие материала,
а~кроме того, имелось отличное от нуля количество опечаток и~прочих глюков типографского характера. Спасибо
А.\,Басалаеву, А.\,Веремьёву, А.\,Шапиро и Андрею (\texttt{avolk07@mail.ru}) за ценные замечания и исправления.

Данная версия документа обладает одним важным свойством: она была отредактирована самим лектором, что снизило количество опечаток и неточностей.

Порядок изложения материала наиболее соответствует курсу 2005 г.

\section*{Release notes}

В настоящее время переработка дошла до теоремы Абеля.
Из раздела про эллиптические функции остался только синус Якоби.
В последней редакции внесены правки 2010 года от Андрея.

\medskip
\dmvntrail


\begin{thebibliography}{4}
\setlength\itemsep{-.5mm}
\bibitem{shabat}Б.\,В.\,Шабат. \emph{Введение в~комлексный анализ.}-- Наука, 1985. 3-е изд.
\bibitem{lavr} М.\,А.\,Лаврентьев, Б.\,В.\,Шабат. \emph{Методы ТФКП.}-- Физматгиз, 1958.
\bibitem{gurwitz} А.\,Гурвиц, Р.\,Курант. \emph{Теория функций.}-- Наука, 1968.
\bibitem{alexeev} В.\,Б.\,Алексеев, \emph{Теорема Абеля в задачах и решениях.}-- Наука, 1976.
\bibitem{sidorov} Ю.\,В.\,Сидоров, М.\,В.\,Федорюк, М.\,И.\,Шабунин. \emph{Лекции по ТФКП.}-- Наука, 1989.
\bibitem{kra} И.\,Кра. \emph{Автоморфные формы и клейновы группы.}-- Мир, 1975.
\bibitem{vinberg} Э.\,Б.\,Винберг. \emph{Курс алгебры.}-- М.: Факториал, 2002.
\end{thebibliography}

\pagebreak

\section{Гармонические функции. Гидродинамика}

\subsection{Связь гармонических и голоморфных функций}

\subsubsection{Гармонические функции}

\begin{df}
\emph{Оператором Лапласа} называется дифференциальный оператор $\De := \sum \frac{\pd^2}{\pd x_i^2}$.
Пусть $D$-- область в $\R^n$. Функция $u\in \Cb^2(D)$ называется \emph{гармонической} в области $D$,
если $\De u = 0$. Множество гармонических функций будем обозначать через $\Hc$.
\end{df}

Очевидно, множество $\Hc$ есть линейное пространство. В случае одной переменной гармоническими функциями
являются в точности функции вида $y(x) = ax+b$. Для двух переменных гармонические функции являются решениями
задачи о форме мембраны, натянутой на контур, и обладают экстремальным  свойством:
площадь поверхности графика гармонической функции с заданными значениями на границе области минимальна.
С физической точки зрения гармоническая поверхность минимизирует энергию поверхностного натяжения. Из
физических соображений можно сделать много полезных выводов о свойствах гармонических функций (таких, как
принцип максимума), но в дальнейшем мы докажем все эти свойства средствами комплексного анализа.

\subsubsection{Восстановление голоморфной функции по гармонической вещественной части}

Далее мы будем рассматривать гармонические функции двух переменных. Их можно трактовать как функции комплексного
переменного. Имеем
\eqn{\De = \pf{^2}{x^2} + \pf{^2}{y^2} =
\ub{\hr{\pf{}{x} - i\pf{}{y}}}_{2\pf{}{z}}\cdot \ub{\hr{\pf{}{x}+i\pf{}{y}}}_{2\pf{}{\ol z}} =
4 \frac{\pd^2}{\pd z \pd \ol z}.}

Рассмотрим голоморфную функцию $f=u+iv$ и её вещественную часть. Покажем, что функция $u = \Rea f$ гармонична.
Действительно, запишем условия Коши-- Римана для функции $f$: $u_x = v_y$ и $u_y=-v_x$. Тогда
$u_{xx} + u_{yy}=v_{yx}-v_{xy}=0$, так как вторые частные производные непрерывны.

Выясним теперь, обратимо ли это свойство.

\begin{stm}
Пусть $D$-- односвязная область, и дана функция $u\in \Hc(D)$. Тогда найдётся такая гармоническая функция $v$,
что функция $f = u+iv$ будет голоморфной в области $D$.
\end{stm}
\begin{proof}
Пусть $f$ существует. Из уравнений Коши-- Римана $g := f' = u_x-iu_y$. Пусть $F$-- первообразная к $g$. Имеем
$F'=g$, $u_x=\Rea g = \Rea f'= \Rea F'=(\Rea F)_x$ и $u_y=-\Img g= -\Img f'= -\Img F'=(\Rea F)_y$, то есть
функция $u - \Rea F$ имеет тождественно нулевые частные производные, значит, она постоянна. Таким образом, мы
нашли голоморфную функцию $F$, удовлетворяющую нашим требованиям.
\end{proof}

\begin{note}
Функция $f$ восстанавливается неоднозначно (например, к ней можно добавить мнимую константу). Если же область не
была односвязной, то мы лишаемся однозначности функции $f$ (хотя локально это свойство выполнено: достаточно
малая окрестность гомеоморфна кругу и потому односвязна, а в ней первообразная будет однозначной).
\end{note}

Явное выражение для функции $f$ в односвязной области можно получить следующим образом:
рассмотрим форму $\om := -u_y \, dx + u_x\, dy$. Тогда $d\om = \De u \, dx \wg dy = 0 \cdot dx \wg dy = 0$.
Фиксируем точку $a \in D$ и положим $v(z) := \intl{a}{z}\om$. Односвязность области гарантирует нам независимость
значения функции от пути интегрирования.

\begin{df}
Вещественная и мнимая часть голоморфной функции $f$ называются \emph{сопряжёнными} гармоническими функциями.
\end{df}

\subsection{Свойства гармонических функций}

\subsubsection{Аналоги свойств голоморфных функций}

Выведем некоторые свойства гармонических функций из свойств голоморфных функций.

Пусть $u\in \Hc(D)$. Так как $u = \Rea f$, где $f \in\Oc(D)$, то функция $u$ вещественно-- аналитична.

\begin{theorem}[о среднем]
Пусть $u \in \Hc(D)$. Тогда значение этой функции в центре круга $C_r$, целиком лежащего в области $D$, есть её
среднее значение на окружности:
\eqn{u(a) = \frac{1}{2\pi}\intl{0}{2\pi}u(a+re^{i\ph})\,d\ph = \frac{1}{\pi r^2}\ints{C_r}u\,dx \wg dy.}
\end{theorem}
\begin{proof}
Пусть $u = \Rea f$, а $C_r$-- круг радиуса $r$ с центром в точке $a$. По теореме Коши
\eqn{f(a) = \frac{1}{2\pi i} \ints{\pd C_r} \frac{f(\ze)}{\ze-a}\,d\ze =
     \frac{1}{2\pi}\intl{0}{2\pi}f(a+re^{i\ph})\,d\ph.}
Но выражение в условии теоремы есть в точности  вещественная часть написанного равенства.
\end{proof}

\begin{theorem}[обратная к теореме о среднем]
Если для некоторой функции $u(z) \in \Cb^2(D)$ выполнено утверждение теоремы о среднем, то функция $u(z)$ гармонична.
\end{theorem}
\begin{proof}
Напишем формулу Тейлора для функции $u(z)$ до членов второго порядка:
\eqn{u(x,y) =  A_0 + B_1x + B_2y + C_1x^2 + C_2xy + C_3y^2 + o(x^2+y^2).}
Пользуясь линейностью  пространства $\Hc$, отбросим линейную часть в силу ее
гармоничности. Функции $x^2-y^2$ и $xy$ тоже гармоничны, поэтому можно их тоже выкинуть (с подходящими
коэффициентами). Поэтому можно считать, что у нас исходно была функция вида $u(x,y) = C x^2 + o(x^2+y^2)$. Для
простоты будем считать, что мы пишем условие теоремы о среднем для точки $a=0$, так как иначе бы можно
было написать разложение Тейлора с центром в другой точке. По условию значение в центре равно нулю, поэтому
\eqn{0 = \intl{0}{2\pi}u(re^{i\ph})\,d\ph = Cr^2 \intl{0}{2\pi}\cos^2\ph\,d\ph + o(r^2).}
При $r \ra 0$ правая часть должна стремиться к нулю.  Но с
другой стороны, $\intl{0}{2\pi}\cos^2\ph\,d\ph > 0$. Значит, $C = 0$, а это и означает, что $\De u = 0$.
\end{proof}

\begin{theorem}[единственности]
Если гармоническая функция равна нулю в некоторой окрестности, то она тождественно равна нулю.
\end{theorem}
\begin{proof}
Пусть $u =\Rea f$. Из условий Коши-- Римана имеем $f(z) = iC$ в окрестности (так как
вещественная часть равна нулю), а по комплексной теореме единственности функция $f(z)$ равна
этой константе и во всей области. --е $u = \Rea f=0$ во всей области.
\end{proof}

\begin{note}
Для совпадения гармонических функций в области недостаточно их совпадения на последовательности точек:
функция $u(x,y) = x$ гармонична и постоянна на множестве прямых $\hc{x=\const}$.
\end{note}

\begin{theorem}[принцип максимума]
Если функция $u\in \Hc(D)$ в некоторой внутренней точке достигает нестрогого локального максимума, то она постоянна.
\end{theorem}
\begin{proof}
Пусть $u=\Rea f$. Рассмотрим функцию $g := \exp f$. Тогда у функции $g$ в данной точке будет максимум модуля,
а потому она постоянна. Значит, $f\equiv \const$ и $u \equiv \const$.
\end{proof}

\begin{theorem}[Лиувилля]
Если функция $u\in\Hc(\Cbb)$ ограничена во всей плоскости сверху или снизу, то $u\equiv\const$.
\end{theorem}
\begin{proof}
Пусть $u=\Rea f$ и $u \le M$. Рассмотрим функцию $g := \exp f$. Тогда функция $|g|$ ограничена,
и~по~комплексной теореме Лиувилля функция $g$ постоянна, а вместе с ней постоянна и функция $f$.
Случай $u \ge M$ разбирается аналогично.
\end{proof}

\begin{stm}
Композиция голоморфной и гармонической функции гармонична: $u\in \Hc(D)$ и $f \in \Oc(\Om)$, тогда
$u\br{f(\ze)} \in \Hc(\Om)$.
\end{stm}
\begin{proof}
Пусть $u = \Rea F$. Тогда функция $F(f)$ голоморфна, а потому $\Rea F(f)$ гармоническая.
\end{proof}

\subsubsection{Особые точки гармонических функций}

Рассмотрим гармоническую функцию $u(x,y)$ и голоморфную функцию $f = u+iv$. Пусть она имеет
особую точку $a$. Будем продолжать аналитически нашу функцию, обходя вокруг особой точки. При
возвращении в точку отправления, вообще говоря, может получиться другая функция. Но вещественная часть
аналитического продолжения есть функция $u$, которая однозначна. Следовательно, приращение будет чисто
мнимым. Это приращение равно некоторой мнимой константе $i\Ga$ (следует из уравнения Коши-- Римана).

Это наблюдение уже подсказывает нам, что всё это очень похоже на $\Ln z$. Рассмотрим функцию
\eqn{g(z) := \exp \hr{\frac{2\pi}{\Ga}f(z)}.}

Покажем, что она однозначна. В самом деле, после обхода вокруг точки имеем
\eqn{\wt g(z)=\exp\hr{\frac{2\pi}{\Ga} \wt f(z)}= \exp\hr{\frac{2\pi}{\Ga}(f(z)+i\Ga)}=
\exp\hr{\frac{2\pi}{\Ga}f(z)}\cdot\exp\hr{2\pi i}=\exp\hr{\frac{2\pi}{\Ga}f(z)},}
поскольку $\exp(2\pi i)=1$. Тогда
\eqn{f(z)=\frac{\Ga}{2\pi}\ln g(z),\quad u=\frac{\Ga}{2\pi}\Rea \ln g(z)=\frac{\Ga}{2\pi}\ln |g(z)|.}

Возможны три случая для поведения функции $u$ в окрестности особой точки $a$.

\begin{items}{-2}
\item Если функция $u$ ограничена, то она доопределяется в точке $a$ по непрерывности, тем самым имеем
устранимую особенность.
\item $u(z)\ra \bes$ при $z\ra a$. Вспомним, что $u(z) = \frac{\Ga}{2\pi}\ln|g(z)|$. Если функция $g$ имеет
нуль в точке $a$, то $u(z) \ra -\bes$, а если полюс, то $u(z)\ra +\bes$. Точнее говоря, пусть $g(z)=(z-a)^k\ph(z)$,
где $\ph$ голоморфна в полной окрестности точки $a$ и $\ph(a)\neq 0$. Тогда
\eqn{u(z)=\frac{\Ga}{2\pi}\ln|g(z)|=\frac{\Ga}{2\pi}(k \ln |z-a|+\ln |\ph(z)|)=
\frac{k\Ga}{2\pi}\ln|z-a|+\frac{\Ga}{2\pi}\ln|\ph(z)|.}
Очевидно, функция $\frac{\Ga}{2\pi}\ln|\ph(z)|$ имеет устранимую особенность в точке $a$.
\item Функция $u$ не имеет предела в точке $a$. Тогда $g(z)$ имеет существенную особенность и  применима
теорема Сохоцкого. Например, функция $\exp\hr{\frac1z}$ порождает гармоническую функцию
$u(x,y) = \frac{x}{x^2+y^2}$. В достаточно малой окрестности точки $a$ она <<заметает>> всю вещественную
ось (найдётся подпоследовательность $z_n\ra a$, для которой $u(z_n) \ra b$ для $\fa b \in \R$.
\end{items}

\subsubsection{Задача Дирихле}

Комплексный анализ, а точнее, формула Коши позволяет почти без выкладок выписать решение задачи Дирихле
в~виде интеграла. Напомним, что в этой задаче ищется функция $u \in \Hc(D) \cap \Cb(\ol D)$, такая,
что $u\evn{\pd D} = \ph$.

Докажем, что если решение задачи Дирихле существует (мы этого еще не знаем, но скоро узнаем), и область
ограничена, то оно единственно. Действительно, пусть $u_1$ и $u_2$-- два решения. Тогда их разность
$v := u_1-u_2$ гармонична и на границе равна нулю. По принципу максимума $v \equiv 0$.

Для неограниченных областей это не верно. пример: $D = \hc{\Img  z > 0}$ и $u_\la :=\la y$. При разных
$\la$ имеем разные решения, и функции $u_\la$ гармоничны.

Докажем теперь существование решения для ограниченной односвязной области $D$.
Переведём её конформно в единичный круг $\De$. Так как композиция конформного отображения и гармонической
функции гармонична, а конформное отображение продолжается до гомеоморфизма областей с границами, мы свели
задачу к аналогичной задаче для круга с некоторыми другими граничными данными $\psi$. Пусть $u$--
искомое решение. В силу односвязности круга найдётся голоморфная (однозначная) функция $f$ на этом круге,
для которой $u=\Rea f$. Запишем интегральную формулу Коши для точки $z\in \De$ и для точки $z^*$,
полученной из $z$ симметрией относительно единичной окружности по формуле $z^* = \frac{1}{\ol z}$.
Далее везде $\ze := re^{i\ph}$ для краткости. Имеем
\eqn{\label{Dirichlet}f(z)=\frac{1}{2\pi\,i}\ints{|\ze|=1}\frac{f(\ze)}{\ze-z}\,d \ze =
\frac{1}{2\pi}\intl{0}{2\pi}\frac{f(\ze)\ze}{\ze-z}\,d\ph, \text{ где } \ze = e^{i\ph}.}
Теперь запишем ту же формулу для точки $z^*$. Так как точка $z^*$ лежит вне круга и подинтегральная
функция будет голоморфной, имеем
\eqn{\frac{1}{2\pi\,i}\ints{|\ze|=1}\frac{f(\ze)}{\ze-z^*}\,d \ze = 0.}
Вычтем этот интеграл из первого (то есть на самом деле ничего не вычтем). Под интегралом будет выражение
$$
  \frac{\ze}{\ze-z}-\frac{\ze}{\ze - \frac{1}{\ol z}} =
  \frac{\ze(\ze - \frac{1}{\ol z} - \ze + z)} {(\ze - z)(\ze - \frac{1}{\ol z})} =
  \frac{\ze(z \ol z - 1)}{(\ze - z)(\ze \ol z - 1)} =
  \frac{1 - z \ol z} {(\ze - z)(\frac{1}{\ze} - \ol z)} =
  \frac{1 - z \ol z} {(\ze - z)(\ol \ze - \ol z)} =
  \frac{1 - z \ol z} {(\ze - z)\ol{(\ze - z)}}
$$
То есть
\eqn{\frac{\ze}{\ze-z}-\frac{\ze}{\ze - \frac{1}{\ol z}} = \frac{1-|z|^2}{|\ze-z|^2}.}
Тогда выражение \eqref{Dirichlet} перепишется в виде
\eqn{f(z) = \frac{1}{2\pi}\intl{0}{2\pi}f(\ze)\frac{1-|z|^2}{|\ze-z|^2}\,d\ph.}
Выделим вещественную часть. Имеем
\eqn{\Rea f(z) = u(z) = \frac{1}{2\pi}\intl{0}{2\pi}\psi(e^{i\ph})\frac{1-|z|^2}{|e^{i\ph}-z|^2}\,d\ph.}
Это и есть решение для круга (так называемая \emph{формула Пуассона}). Остаётся применить к полученной
функции~$u$ конформное отображение, переводящее~$\De$ в исходную область.

Кроме того, заметим, что
\eqn{\frac{1-|z|^2}{|\ze-z|^2} = \Rea\frac{\ze+z}{\ze-z}}
Это соотношение можно проверить, например, следующим способом:
\begin{multline*}
  \Rea z = \frac{z + \ol z}{2};\quad
  \Rea \frac{\ze + z}{\ze - z} =
  \frac12 \left(\frac{\ze + z}{\ze - z} + \frac{\ol \ze + \ol z}{\ol \ze - \ol z}\right) =
  \frac{1}{2(\ze - z) \ol{(\ze - z)}}((\ze + z)(\ol \ze - \ol z) + (\ze - z)(\ol \ze + \ol z)) = \\
  \frac{1}{2|\ze - z|^2}(\ze \ol \ze + z \ol \ze - \ze \ol z - z \ol z + \ze \ol \ze - z \ol \ze + \ze \ol z - z \ol z) =
  \frac{2 \ze \ol \ze - 2 z \ol z}{2|\ze - z|^2} = \frac{1 - |z|^2}{|\ze - z|^2}
\end{multline*}
Поэтому $f(z)$ можно определить формулой
\eqn{f(z) = \frac{1}{2\pi}\intl{0}{2\pi}\psi(e^{i\ph})\frac{\ze+z}{\ze-z}\,d\ph + iC,}
где $C$-- некоторая вещественная константа (две голоморфные функции с одинаковыми действительными
частями отличаются на мнимую константу). Это выражение называется \emph{формулой Шварца}.

\begin{theorem}
Полученная функция $u$ непрерывно выходит на границу области, --е если $z\ra \ze_0 \in \pd \De$, то
$u(z)\ra u(\ze_0)$.
\end{theorem}
\begin{proof}
Введём обозначение
\eqn{P(\ze,z):=\Rea \frac{1}{2\pi}\frac{\ze+z}{\ze-z} = \frac{1}{2\pi}\frac{1-|z|^2}{|\ze-z|^2}.}
Имеем
\eqn{\label{ZeroInt}\intl{0}{2\pi}P(\ze,z)\,d\ph=1,}
так как функция $u\equiv 1$ гармонична. Заметим, что
$P(\ze, z) \ra 0$ при $z\ra \ze_0 \neq \ze$ (это видно из предыдущей формулы: числитель стремится
к~нулю, а знаменатель-- нет), причём сходимость равномерна по $\ze$ на каждой дуге $\ga$, не
содержащей точку $\ze_0$.

Рассмотрим разность
\eqn{d:=\intl{0}{2\pi}u(\ze)P(\ze,z)\,d\ph - u(\ze_0) \stackrel{!}{=}
\intl{0}{2\pi}\br{u(\ze)-u(\ze_0)}P(\ze,z)\,d\ph.}
Переход <<!>> обеспечен равенством \eqref{ZeroInt}, от умножения на константу $u(\ze_0)$ хуже не будет.

В силу непрерывности функции $u$ на границе для $\fa \ep$ найдётся $\de$ такое,
что $|u(\ze)-u(\ze_0)|<\ep$ при $|\ph-\ph_0|\le 2\de$. Пусть
$\ga_1 := \hc{\ze = e^{i\ph}\cln |\ph-\ph_0|\le 2\de}$, а $\ga_2 := \pd \De \wo \ga_1$.
На дуге $\ga_{1}$ имеем
\eqn{\Bm{\ints{\ga_1}\br{u(\ze)-u(\ze_0)}P(\ze,z)\,d\ph} < \ep\ints{\ga_1}P(\ze,z)\,d\ph <
\ep\ints{\pd\De}P(\ze,z)\,d\ph = \ep \cdot 1 = \ep.}
Теперь разберёмся с дугой $\ga_2$. Пусть $z=re^{i\ta}$ и $|\ta-\ph_0|<\de$. Тогда
найдётся $\rh \in(0,1)$ такое, что если $r \in (1-\rh, 1)$, то $P(\ze,z)<\ep$ для
всех $\ze\in\ga_2$. Тогда
\eqn{\Bm{\ints{\ga_2}\br{u(\ze)-u(\ze_0)}P(\ze,z)\,d\ph} <
2M\ints{\ga_2}P(\ze,z)\,d\ph < 2M\ep \cdot 2\pi,}
где $M = \maxl{\pd \De}u$. Значит, $|d| \le (1+4\pi M)\ep$.
\end{proof}

\subsection{Гидродинамическое доказательство теоремы Римана}

\subsubsection{Векторные поля и голоморфные функции}

Будем рассматривать гладкие векторные поля на плоскости и интерпретировать их как
поля скоростей потока жидкости.
Пусть $\vec V = \br{P(x,y),Q(x,y)}$-- поле класса $\Cb^1$.
Рассмотрим некоторый гладкий контур $\ga$ и функции
\eqn{\begin{aligned}
\Pi(\ga) &:= \ints{\ga}(V, \vec n)\,ds = \ints{\ga}\ub{-Q\,dx+P\,dy}_{\om_1},\\
\Be(\ga) &:= \ints{\ga}(V,\vec \tau)\,ds = \ints{\ga}\ub{P\,dx+Q\,dy}_{\om_2},
\end{aligned}}
где $\vec n$-- единичная внешняя нормаль к контуру, а $\vec \tau$-- касательный вектор к $\ga$.

\begin{df}
Функция $\Pi$ называется \emph{потоком} поля через контур $\ga$, а $\Be$-- \emph{вихрем}.
\end{df}

Пусть $\ga = \pd D$. По формуле Грина имеем
\eqn{\Pi(\ga) = \iints{D}(P_x+Q_y)\,dx\,dy, \quad \Be(\ga) = \iints{D}(Q_x-P_y)\,dx\,dy.}

Пусть течение жидкости безвихревое и не имеет источников и стоков. Тогда $\Be=\Pi=0$.
Но если это верно для любой области $D$, это значит, что подынтегральные выражения равны нулю:
\eqn{\case{\Div V = P_x+Q_y=0,\\\rot V = Q_x-P_y = 0.}}
Это означает замкнутость задаваемых ими
дифференциальных форм. Пусть область $D$ односвязна, тогда эти формы точны, --е являются
чьими-- то дифференциалами. Тогда рассмотрим функции
\eqn{v(z) := \intl{z_0}{z}\om_1, \quad u(z) := \intl{z_0}{z}\om_2.}

\begin{df}
Функция $u$ называется \emph{потенциалом поля}, а $v$--
\emph{функцией тока}.
\end{df}

В силу односвязности путь интегрирования в определении не важен. Название функции~$v$
объясняется тем, что жидкость течёт по линиям уровня этой функции. Действительно,
жидкость течёт по решениям дифференциального уравнения
\eqn{\case{\dot x = P(x,y),\\\dot y = Q(x,y).}}
Тогда $\frac{d}{dt}v\br{x(t),y(t)} = \pf{v}{x}\dot x + \pf{v}{y} \dot y = -QP + PQ = 0$,
то есть функция $v$ постоянна на траекториях системы. Аналогично, для $u$ получаем
$\frac{d}{dt}u\br{x(t),y(t)} = \pf{u}{x}\dot x + \pf{u}{y} \dot y = PP + QQ \geqslant 0$ ---
потенциал растет на траекториях движения частиц жидкости.

\begin{df}
Функция $f := u + iv$ называется \emph{комплексным потенциалом} поля.
\end{df}

Комплексный потенциал будет голоморфной функцией, так как $u_x = P = v_y$ и $u_y = Q = -v_x$,
то есть условия Коши-- Римана выполнены. Отметим также, что $V = P+iQ = \ol{f'}$, то есть
зная потенциал~$f$, можно найти векторное поле~$V$. Отсюда
\eqn{\ints{\ga}f'\,dz= \ints{\ga}(P-i Q)(dx+i\,dy)=
\ints{\ga}P\,dx+Q\,dy+ i\ints{\ga}-Q\,dx+P\,dy=\Be(\ga)+i\Pi(\ga).}

\subsubsection{Примеры}

\begin{nums}{-2}
\item Рассмотрим функцию $f(z) = \ln z = \ln |z| + i \arg z$. Её вещественная часть
      $u(x,y)=\ln (x^2+y^2)^{\frac12} = \frac12\ln (x^2+y^2)$ будет гармонической.
\item Функция $f_1(z) = \ln z$ представляет собой комплексный потенциал
      векторного поля, соответствующего перетеканию жидкости из $0$ в $\bes$. Для
      неё $\Be = 0$, а $\Pi = 2\pi$. Функция $f_2(z) = i\ln z$ представляет собой
      потенциал поля, для которого линиями уровня являются линии тока функции~$f_1$
      (и наоборот). Для $f_2$ всё наоборот: $\Be=2\pi$, а $\Pi=0$-- жидкость
      крутится вокруг нуля (и бесконечности тоже).

      Теперь переведём конформным преобразованием $\ln \frac{z+h}{z-h}$ точки~$0$
      и $\bes$ в точки $h$ и $-h$, и устремим $h$ к нулю. При этом будем увеличивать
      поток: $\Pi \cdot 2h = m$. Тогда жидкость будет перетекать из точки $h$ в точку $-h$.
      В пределе получится диполь-- семейство окружностей, касающихся в точке $0$ по
      обе стороны от вещественной оси.
\item Постоянное векторное поле соответствует перетеканию жидкости из $\bes$ в $\bes$.
\end{nums}

\subsubsection{Теорема Римана}

Рассмотрим произвольную область $G$ (не обязательно односвязную), и поместим в некоторую
её внутреннюю точку диполь (см. пример выше). Из физических соображений следует, что рано
или поздно в области образуется установившееся течение жидкости. Пусть наше поле безвихревое,
то есть $\Be=0$. У него есть комплексный потенциал $w=f(z)=u(z)+iv(z)$. Это голоморфная
функция с единственным полюсом первого порядка в нашей области.

Из физических же соображений ясно, что если вода обтекает препятствие (то есть дырку
в области), то найдётся такая точка на границе дырки, что левее её жидкость обтекает препятствие
с одной стороны, а правее-- с другой. Линия тока, которая упирается в эту точку,
называется \emph{сепаратрисой} (от слова \emph{separate}-- разделять).

Комплексный потенциал постоянен вдоль траекторий, значит, эта функция переводит траектории
в семейство прямых, параллельных оси абсцисс. Условие $B=0$ гарантирует нам, что не возникнет
неоднозначности у мнимой части. Сепаратрисам деваться некуда, и они тоже перейдут в семейство
прямых, из которых выкинуто некоторое семейство отрезков. Таким образом, получилось конформное
отображение $f$ на плоскость c разрезами, параллельными вещественной оси. Но если область была
односвязной, то разрез будет единственным (он будет соответствовать одной сепаратрисе, упиравшейся во внешнюю
границу области, а такую область отобразить конформно на круг уже совсем просто.

\section{Многомерный комплексный анализ}

\subsection{Голоморфные функции многих переменных}

\subsubsection{Определения, простейшие свойства}

Рассмотрим функции вида $f\cln \Cbb^n\ra\Cbb$, $f(\vec z) = f(z_1\sco z_n)$.
Введём обозначения: $|z|^2 = |z_1|^2\spl|z_n|^2$. Шаром, как обычно, будем называть
множество $B(a,r) := \hc{z\cln |z-a|< r}$.

\begin{df}
Область в $\Cbb$ вида $\De_r(a)=\hc{|z_i-a_i| < r_i}$, где $a\in \Cbb^n$, а
$r\in (\R_+)^n$ называется \emph{полидиском}. Она является декартовым произведением
одномерных дисков. Подмножество границы $\hc{|z_i-a_i|=r_i}$
называется \emph{остовом} полидиска. Мы будем обозначать остов $\De$ через $\Sk \De$.
\end{df}


Далее везде, где это не указано, суммирование ведётся по всем переменным, то есть от~$1$ до~$n$.

Как и в одномерном случае, будем использовать обозначения
\eqn{\pf{}{z_j} := \frac12\hr{\pf{}{x_j}-i\pf{}{y_j}}, \quad
\pf{}{\ol z_j} := \frac12\hr{\pf{}{x_j}+i\pf{}{y_j}},}
\eqn{\pd f := \sum f_{z_j}\,dz_j, \quad \ol \pd f := \sum f_{\ol z_j}\,d\ol z_j.}

Пусть функция $f$ является $\R$-- дифференцируемой, тогда
\eqn{df = \suml{j=1}{n} \hr{f_{x_j}dx_j+f_{y_j}dy_j} =
\suml{j=1}{n} \hr{f_{z_j}dz_j + f_{\ol z_j}d\ol z_j} = \pd f + \ol \pd f.}

\begin{df}
Функция $f$ называется \emph{голоморфной} в области $D$, если $\ol \pd f \equiv 0$ всюду в $D$.
\end{df}

Очевидно, голоморфная функция является голоморфной по каждому аргументу при фиксированных
остальных, так как $f_{\ol z_j}=0$ при всех $j$.

\subsubsection{Кратная интегральная формула Коши}

Пусть $f\in \Oc(D)$, где $D = D_1 \st D_n$. Зафиксируем все переменные, кроме первой,
и применим обычную формулу Коши. Имеем
\eqn{f(z) = \frac{1}{2\pi i}\ints{\pd D_1} \frac{f(\ze_1, z_2\sco z_n)}{\ze_1-z_1}\,d\ze_1.}
Но на этом, мы, конечно, не остановимся. Продолжим равенство, применив формулу для переменной $z_2$:
\eqn{f(z) = \frac{1}{(2\pi i)^2}\ints{\pd D_1}
\ints{\pd D_2}\frac{f(\ze_1,\ze_2,z_3\sco z_n)}{(\ze_1-z_1)\cdot(\ze_2-z_2)}\,d\ze_2\,d\ze_1.}
Под интегралом у нас хорошая функция, поэтому применима теорема Фубини. В итоге имеем
\eqn{f(z) = \frac{1}{(2\pi i)^n} \ints{\pd D_1 \st \pd D_n}
\frac{f(\ze_1\sco\ze_n)}{\prod(\ze_j-z_j)}\,d\vec\ze.}
Это и есть многомерная формула Коши.

\medskip

Рассмотрим голоморфную форму степени $n$:
\eqn{\om = f(z_1\sco z_n)\,dz_1\sw dz_n, \quad f\in \Oc(D).}
Покажем, что её дифференциал $d\om$ равен нулю. Действительно, имеем
$d\om = df\wg dz_1\sw dz_n$, а $df = \sum f_{z_j}\,dz_j$ (коэффициенты при $d\ol z_j$
равны нулю в силу голоморфности). Следовательно, в выражении для дифференциала~$d\om$
тоже не будет слагаемых, содержащих $d\ol z_j$, зато в каждом слагаемом будет пара
одинаковых дифференциалов~$dz_j$. Но всем хорошо известно, что $dz\wg dz =0$. Значит, $d\om=0$.

Отсюда и из теоремы Стокса вытекает, что
\eqn{\ints{\pd \si}\om = \ints{\si}d\om = \ints{\si}0=0.}

\subsection{Свойства голоморфных функций}

\subsubsection{Степенные ряды для функций многих переменных}

Будем рассматривать кратные степенные ряды вида
\eqn{\label{MultivariateSeries}\sums{m}c_m(z-a)^m :=
\mcomp{\sums{m_1\sco m_n}} c_{m_1\dots m_n}(z_1-a_1)^{m_1} \sd (z_n-a_n)^{m_n},}
где $m=(m_1 \sco m_n)$, $|m|:=m_1 \spl m_n$, $z=(z_1\sco z_n)$, $a=(a_1 \sco a_n)$.

Рассмотрим кратную геометрическую прогрессию
\eqn{\sums{m_j\ge 0}q_1^{m_1}\sd q_n^{m_n}.}
Имеем
\eqn{\sums{m_j\ge 0}q_1^{m_1}\sd q_n^{m_n} =
\lim \hs{\suml{m_1 = 0}{N_1}q_1^{m_1} \sd \suml{m_n = 0}{N_n}q_n^{m_n}} =
\lim \hs{\frac{1 - q_1^{N_1+1}}{1-q_1} \sd \frac{1 - q_n^{N_n+1}}{1-q_n}} =
\frac{1}{1-q_1}\sd \frac{1}{1-q_n}.}

Аналогично случаю одной переменой, доказывается
\begin{lemma}[Абеля]
Если в точке $\wh z \neq a$ члены ряда равномерно ограничены, то есть $\hm{c_m(\wh z-a)^m} \le M$,
то при $z \in \De(a, r)$, где $r=(r_1\sco r_n)$, $r_j = |\wh z_j-a_j|$, ряд \eqref{MultivariateSeries}
сходится равномерно.
\end{lemma}
\begin{proof}
Как и в одномерном случае, имеем
\begin{multline}
\hm{c_{m_1\dots m_n}(z_1-a_1)^{m_1} \sd (z_n-a_n)^{m_n}} =\\=
\hm{c_{m_1\dots m_n}(\wh{z}_1-a_1)^{m_1} \sd (\wh z_n-a_n)^{m_n} \cdot
\hr{\frac{z_1-a_1}{\wh z_1-a_1}}^{m_1} \sd \hr{\frac{z_n-a_n}{\wh z_n-a_n}}^{m_n}} \le\\\le
M\cdot \hm{\frac{z_1-a_1}{\wh{z}_1-a_1}}^{m_1} \sd \hm{\frac{z_n-a_n}{\wh z_n-a_n}}^{m_n},
\end{multline}
то есть общий член мажорируется кратной геометрической прогрессией со знаменателем $q_j < 1$,
и ряд сходится равномерно по признаку Вейерштрасса.
\end{proof}

\subsubsection{Разложение голоморфной функции в ряд}

Рассмотрим область $D$ и полидиск $\De(\vec a, \vec r)\subs D$.
Как и в одномерном случае, представим дробь $\frac{1}{\ze_j-z_j}$ в виде суммы геометрической прогрессии:
\eqn{\frac{1}{\ze_j-z_j}=\frac{1}{(\ze_j-a_j)-(z_j-a_j)} =
\frac{1}{\ze_j-a_j}\cdot\frac{1}{1-\frac{z_j-a_j}{\ze_j-a_j}}=
\frac{1}{\ze_j-a_j}\cdot\suml{m_j=0}{\bes}\hr{\frac{z_j-a_j}{\ze_j-a_j}}^{m_j} =
\suml{m_j=0}{\bes}\frac{(z_j-a_j)^{m_j}}{(\ze_j-a_j)^{m_j+1}}.}

Рассмотрим функцию $f\in \Oc(D)$. Напишем для неё формулу Коши, интегрируя по полидиску
$\wt \De \Subset \De$ (чтобы точка $\ze$ не подбиралась слишком близко к остову большего полидиска).
\eqn{f(z) = \frac{1}{(2\pi i)^n} \ints{\Sk \wt \De} \frac{f(\ze_1\sco\ze_n)}{\prod(\ze_j-z_j)}\,d\vec\ze.}
Для каждой из дробей $\frac{1}{\ze_j-z_j}$ применим написанное разложение. Получим
\eqn{f(z) = \frac{1}{(2\pi i)^n} \ints{\Sk\wt \De}
\sums{m} f(\ze) \prodl{j=1}{n}\frac{(z_j-a_j)^{m_j}}{(\ze_j-a_j)^{m_j+1}} \,d\vec\ze.}
Поскольку $f\in\Oc(\De)$, то  $|f|\le M$ на $\wt \De$, а так как $\wt \De\Subset \De$,
то $\frac{|z_j-a_j|}{|\ze_j-a_j|} \le q_j < 1$, и ряд мажорируется прогрессией:
\eqn{\Bm{f(\ze)\cdot \frac{1}{\ze_j-a_j}\cdot
\prodl{j=1}{n} \frac{(z_j-a_j)^{m_j}}{(\ze_j-a_j)^{m_j}}} \le
\ub{M\cdot \prod r_j}_{\const} \cdot q_1^{m_1} \sd q_n^{m_n}.}

Значит, по лемме Абеля имеется равномерная сходимость, и можно поменять порядок интегрирования
и суммирования. Тогда
\eqn{f(z) = \frac{1}{(2\pi i)^n}\sums{m} \ints{\Sk \wt \De}
\frac{f(\ze)\,d\ze}{\prod(\ze_j-a_j)^{m_j+1}} \prodl{j=1}{n}(z_j-a_j)^{m_j}.}

\subsubsection{Область сходимости}

\begin{df}
Полидиск $\De(a,r)$ называется \emph{полидиском сходимости}, а $r=\hr{r_1 \sco r_n}$--
набором \emph{сопряжённых радиусов сходимости}, если в $\De(a,r)$ ряд сходится, и в
любом полидиске $\De(a,R)$, где $R_j > r_j$ для некоторого $j$ (а остальные радиусы
такие же), ряд расходится.
\end{df}

Иными словами, полидиск является полидиском сходимости, если его нельзя <<раздуть>>
по некоторой координате, сохраняя сходимость и не уменьшая при этом других радиусов.

\begin{df}
\emph{Область сходимости} степенного ряда-- внутренность множества точек сходимости.
\end{df}

Название <<область>> корректно: это множество открыто (по определению) и связно (почти очевидно).

Далее для простоты будем рассматривать ряды с центром в нуле.

Из леммы Абеля следует, что если точка $z$ принадлежит области сходимости, то вместе с
ней там лежат точки вида $(e^{i\ph_1}z_1\sco e^{i\ph_n}z_n)$. Таким образом, вместе с
каждой точкой в области сходимости лежит полидиск $\De\br{0,(|z_1|\sco |z_n|)}$,
и область сходимости есть объединение полидисков сходимости.

\begin{problem}
Написать ряд, областью сходимости которого является единичный шар.
\end{problem}

\begin{ex}
Рассмотрим ряд $\sum z_1^{m_1}z_2^{m_2}$. Его область сходимости есть множество
$\hc{|z_1| < 1, |z_2| < 1}$.
\end{ex}

\begin{ex}
Область сходимости ряда $\sum z_1^{m_1}z_2$ есть множество $\hc{|z_1|<1}\times\Cbb$.
\end{ex}

\begin{ex}
Область сходимости ряда $\sum (z_1 z_2)^m$ есть множество $\hc{|z_1z_2|<1}$.
\end{ex}

\begin{ex}
Пусть $c_{m_1,m_2} = \frac{(m_1+m_2)!}{m_1!\cdot m_2!}$. Рассмотрим ряд с коэффициентами $c_{m_1,m_2}$.
Если он сходится в некоторой точке, то суммировать можно в любом порядке, поэтому
\eqn{\sums{m_1,m_2} \frac{(m_1+m_2)!}{m_1!\cdot m_2!}z_1^{m_1}z_2^{m_2} = \suml{m=0}{\bes}(z_1+z_2)^m.}
Следовательно, если $|z_1+z_2| \ge 1$, то ряд заведомо расходится. Значит, условие на сопряжённые
радиусы будет таким: $r_1+r_2=1$.
\end{ex}

\subsubsection{Логарифмическая выпуклость}

Рассмотрим образ области сходимости ряда под действием преобразования
\eqn{\ph\cln\Cbb^n\ra\R^n, \quad z=(z_1 \sco z_n) \corr{\ph} (\ln |z_1| \sco \ln |z_{n}|).}

\begin{df}
Говорят, что область \emph{логарифимически выпукла}, если её образ при отображении~$\ph$
есть выпуклое множество.
\end{df}

Очевидно, что образ полидиска $\De(0,r)$ при отображении $\ph$ есть множество
\eqn{\hc{x\cln x_j < \ln r_j} = (-\bes,\ln r_1) \st (-\bes,\ln r_n).}

\begin{theorem}
Область $\Om$ сходимости степенного ряда логарифимически выпукла.
\end{theorem}
\begin{proof}
Пусть в точках $a$ и $b$ общий член ряда $\sum c_m z^m$ ограничен, то есть
\eqn{\hm{c_{m_1\ldots m_n}a_1^{m_1}\sd a_n^{m_n}}\le M, \quad
\hm{c_{m_1\ldots m_n}b_1^{m_1} \sd b_n^{m_n}}\le M.}
Рассмотрим точку $\ze$ такую, что
\eqn{\ln|\ze_j|= t\ln|a_j|+(1-t)\ln|b_j|, \quad t \in [0,1],}
то есть $\ph(\ze)$ лежит на отрезке $[\ph(a),\ph(b)]$. Покажем, что ряд сходится и в точке $\ze$. Имеем
$|\ze_j|=|a_j|^t \cdot |b_j|^{1-t}$. Следовательно,
\begin{multline}
\hm{c_{m_1\ldots m_n}\ze_1^{m_1} \sd \ze_n^{m_n}}=\bm{c_{m_1\ldots m_n}
\hr{a_1^{m_1} \sd a_n^{m_n}}^t \hr{b_1^{m_1} \sd b_n^{m_n}}^{1-t}} =\\=
|c_m|^t\cdot |c_m|^{1-t}\cdot |a^m|^t\cdot |b^m|^{1-t} = |c_m a^m|^t \cdot |c_m b^m|^{1-t} \le M^t M^{1-t} = M.
\end{multline}

Поскольку $a$ и $b$ входят в область $\Om$ с окрестностью, можно найти $\wt{a}$ и $\wt{b}$ из области
сходимости такие, что $|a_j| < |\wt a_j|$ и $|b_j| < |\wt b_j|$ для всех $j$.
Тогда найдётся точка $\wt \ze$ такая, что $\ph(\wt \ze) \in [\ph(a),\ph(b)]$,
$|\ze_j| < |\wt{\ze}_j|$, а значит, точка~$\ze$ принадлежит области сходимости по лемме Абеля.
\end{proof}

\begin{problem}
Описать логарифмически выпуклую оболочку области
$\hc{\De\br{0,(1,2)}\cup \De\br{0,(2,1)}}$.
\end{problem}

\subsubsection{Эквивалентные определения голоморфной функции}

Мы уже знаем, что из голоморфности следует представимость функции интегралом Коши, а из неё--
разложимость в ряд. Остаётся замкнуть круг и показать, что представимость степенным рядом
влечёт голоморфность. Действительно, степенной ряд $\R$-- дифференцируем (с сохранением области
сходимости). Точно также, как и в одномерном случае, можно показать, что $\pf{}{\ol z} = 0$.
Частные производные ряда по $x_j$ и по $y_j$ отличаются только множителем $i$, поэтому
\eqn{\pf{}{\ol z} =
\frac12\hr{\sum \pf{}{x_j} + i \sum \pf{}{y_j}} = \frac12\hr{\sum + i^2\sum}=\frac12\hr{\sum-\sum}=0.}

\begin{note}
Можно показать (теорема Хартогса), что для голоморфности достаточно так называемой сепаратной
аналитичности, то есть требования $\pf{f}{\ol z_j}=0$ для всех $j$, но мы этого делать не будем.
\end{note}

\subsubsection{Стандартные теоремы о голоморфных функциях}

Как и в одномерном случае, справедлива формула Коши-- Адамара
\eqn{\ulim\sqrt[m]{|c_{m_1\sco m_n}|r_1^{m_1}\sd r_n^{m_n}}=1, \text{ где } m=m_1\spl m_n.}
Доказательство ничем не отличается, так как мы переходим к модулям и все рассуждения повторяются.

Сходящиеся ряды можно почленно дифференцировать и интегрировать сколько угодно раз. Они сходятся
равномерно на каждом компакте внутри области сходимости, что гарантирует непрерывность и голоморфность по
каждому аргументу.

Из возможности почленного дифференцирования немедленно получаем, что всякий степенной ряд есть ряд Тейлора
для порождаемой им функции, и верна формула для коэффициентов:
\eqn{c_{m_1\sco m_n} = \frac{1}{m_1!\sd m_n!}\cdot \frac{\pd f(a)}{\pd z_1^{m_1}\ldots \pd z_n^{m_n}}.}

\begin{theorem}[единственности]
Если функция $f$ равна нулю в полномерной окрестности $U\subs \Om$, где $\Om$-- область сходимости,
то $f\equiv0$ во всей области сходимости.
\end{theorem}
\begin{proof}
Рассмотрим множество $E$ тех точек, где степенной ряд равен нулю. Это множество открыто,
так как в каждой точке есть полидиск сходимости (с ненулевым набором радиусов), и замкнуто,
так как это множество нулей непрерывной функции. Рассмотрим произвольную точку $z_0$ в области, соединим её
кривой с некоторой точкой множества $E$. Кривая компактна, поэтому расстояние $\rh$ от неё до границы положительно,
а значит, в каждой точке кривой нам гарантирован радиус сходимости ряда не меньше $r := \frac\rh2$. Кривую
можно накрыть конечным числом полидисков радиуса $r$, а на множестве $E$ коэффициенты разложения нулевые, значит, они нулевые
и в $r$-- окрестности кривой. Значит, $z_0\in E$, --е $\Om=E$.
\end{proof}

\begin{note}
Условие полномерности окрестности существенно: функция $f(z_1,z_2)=z_1z_2$ голоморфна и равна
нулю на объединении прямых $\hc{z_1=0}\cup \hc{z_2=0}$, но $f\not\equiv0$.
\end{note}

\begin{problem}
Если функция $f(z_1,z_2)$ равна нулю на множестве $\hc{z_1 = \ol z_2}$, то $f \equiv 0$.
\end{problem}


\begin{theorem}[Принцип максимума]
Если голоморфная функция достигает в некоторой точке нестрогого локального максимума модуля,
то $f\equiv\const$.
\end{theorem}
\begin{proof}
Рассмотрим точку $a\in \Cbb^n$, в которой достигается максимум, и произвольный вектор
$\vec v\in \Cbb^n$. Проведём прямую $z(t) = a+t\vec v$, где $t\in\Cbb$, и рассмотрим
функцию $g(t)=f(a+t\vec v)$. Она, очевидно, голоморфна и имеет максимум модуля.
Но это уже функция одной переменной, стало быть, она постоянна. Значит, $f=C$
на любой прямой, проходящей через точку $a$, и эта константа одинакова для всех прямых
и равна~$f(a)$.
\end{proof}

\begin{theorem}[Неравенство Коши]
Пусть функция $f$ ограничена по модулю константой $M$ в полидиске сходимости $\De(0,\vec r)$.
Тогда имеет место оценка коэффициентов её степенного ряда:
\eqn{|c_{m_1\sco m_n}|\le \frac{M}{r_1^{m_1}\sd r_n^{m_n}}.}
\end{theorem}
\begin{proof}
Мы знаем формулу для коэффициентов $c_m$:
\eqn{c_m = \frac{1}{(2\pi i)^n}\ints{\Sk \De} \frac{f(\ze)\,d\ze}{\prod(\ze_j-a_j)^{m_j+1}}.}
Заменяя в интеграле $f(\ze)$ на $M$, получаем
\eqn{|c_m| \le \frac{1}{(2\pi)^n} \cdot \frac{M}{\prod r_j^{m_j+1}} \cdot
\prod 2\pi r_j = \frac{M}{r_1^{m_1}\sd r_n^{m_n}}.}
\hfill\end{proof}

\begin{theorem}[Принцип открытости]
Голоморфная непостоянная функция осуществляет открытое отображение.
\end{theorem}
\begin{proof}
Пусть $f\cln D \ra \Cbb$-- голоморфная функция.
Пусть $b \in f(D)$, и $a \in f^{-1}(b)$. Поскольку множество $D$ открыто, найдётся окрестность $U(a) \subs D$.
Рассмотрим прямые, проходящие через точку $a$, а точнее, их пересечения с окрестностью $U$.
По условию, найдётся прямая $\ell$, на которой наша функция не постоянна. По одномерному принципу открытости,
образ множества $M := \ell \cap U$ открыт. Поэтому вместе с точкой $b$ в образе лежит её окрестность $f(M)$.
\end{proof}

\begin{note}
Для отображений $F\cln \Cbb^n \ra \Cbb^m$ это неверно. В самом деле, если $f(z_1,z_2)$-- голоморфная функция,
то образ отображения $F(z_1,z_2) := \br{f(z_1,z_2),f(z_1,z_2)}$ лежит на прямой $\hc{z = w}$, поэтому
не может быть открытым множеством.
\end{note}

\medskip

Скажем пару слов от том, как обстоит дело с ростками у функций многих переменных.
Если в одномерном случае были особые точки, то здесь бывают даже особые прямые. Например, прямая $\hc{z = w}$
является особой для функции $f(z,w) = \sqrt{z - w}$. Однако само понятие ростка переносится на
многомерный случай без изменений.

\subsubsection{Плюригармонические функции}

Пусть $f$-- голоморфная функция в $\Cbb^n$. Распишем её в виде $f = u + iv$.
Имеем $u = \frac12(f + \ol f)$. Заметим, что
$\frac{\pd^2u}{\pd z_j \pd \ol z_j} = 0$, потому что дифференцирование по переменной $z_j$
убьёт антиголоморфную часть, а дифференцирование по $\ol z_j$-- голоморфную. Записывая оператор
$\frac{\pd^2}{\pd z_j \pd \ol z_j}$ в переменных $x_j$ и $y_j$, получаем оператор Лапласа по
переменным $x_j$ и $y_j$. Обозначим этот оператор через~$\De_j$.

\begin{df}
Если для функции $u$ выполнено $\De_j u = 0$ при всех $j$, то функция $u$ называется
\emph{плюригармонической}.
\end{df}

Очевидно, что всякая плюригармоническая функция $u$ гармонична, то есть $\De u = 0$.
Запишем уравнения Коши-- Римана:
\eqn{\pf{u}{x_j} = \pf{v}{y_j},\quad \pf{u}{y_j} = - \pf{v}{x_j}.}
Из этих уравнений по функции $u$ можно найти сопряжённую плюригармоническую функцию $v$:
рассмотрим форму $\om = dv$, тогда $d\om = d^2v = 0$, поэтому в односвязной области можно
(однозначно с точностью до константы) восстановить функцию $v$ по формуле
\eqn{v(z) = \intl{a}{z}\om.}
Такое задание корректно по теореме Стокса: если $\ga$-- замкнутый контур, на котором
лежат точки $z$ и $a$, то
\eqn{\ints{\ga}\om = \ints{\Int\ga}d\om = \ints{\Int\ga} 0 = 0,}
поэтому интегралы по двум половинкам контура $\ga$ от $z$ до $a$ и от $a$ до $z$ отличаются
только знаком, а это и значит, что интеграл не зависит от пути.

\subsection{Устранимые особые множества. Фигуры Хартогса}

\subsubsection{Об устранимых особых множествах}

Довольно полезным следствием логарифмической выпуклости областей голоморфности является
следующая лемма.

\begin{lemma}[об устранимой особенности]
Изолированная особая точка является устранимой особенностью для голоморфной функции
нескольких переменных.
\end{lemma}
\begin{proof}
Пусть $a$-- изолированная особая точка. В силу её изолированности, найдутся два
полидиска~$\De_1$ и~$\De_2$, в которых функция голоморфна, и точка~$a$ лежит сколь
угодно близко к их пересечению. Множество $L \bw{:=} \ph^{-1}\br{\conv\ph(\De_1 \cup \De_2)}$
кроме исходных полидисков будет содержать ещё некоторое множество, ограниченное снаружи
поверхностью, напоминающей гиперболу. Функция $f$ голоморфна в $L$, поэтому достаточно
придвинуть наши полидиски столь близко к точке $a$, чтобы точка $a$ была заметена множеством~$L$.
\end{proof}

$$\epsfbox{pictures.20}$$

\begin{note}
С помощью аналогичной процедуры можно уничтожить любой компакт $K \subs \Cbb^n$, правда,
придётся потребовать, чтобы функция была голоморфна в $\Cbb^n \wo K$. Чтобы сделать это,
нужно надвинуть на этот компакт такие <<длинные>> полидиски, чтобы множество~$L$ поглотило
весь компакт~$K$.
\end{note}

\subsubsection{Аналитическое продолжение функции с фигуры Хартогса на полидиск}

Мы ограничимся рассмотрением функций двух переменных $z_1,z_2$. Суть происходящего понятна уже
и в этом случае, а рисовать удобнее. Полидиски (в нашем случае-- бидиски) удобно рисовать на плоскости
$(|z_1|, |z_2|)$ в виде прямоугольников, подразумевая под отрицательными значениями координат их модули
(для симметрии). Таким образом, почти все точки имеют 4 симметричных изображения (но на самом деле они
соответствуют бесконечному множеству <<настоящих>> точек, получаемых вращениями относительно осей координат).

Фигурой Хартогса называется множество вида
$$\epsfbox{pictures.30}$$
Поскольку оно напоминает катушку, мы будем называть его среднюю часть перемычкой.

Без ограничения общности рассмотрим фигуру Хартогса $H$ ширины 1 и высоты 1. Пусть $f \in \Oc(H)$.
Заметим, что если точка $(|z_1|,|z_2|)$ принадлежит $H$, то можно написать интегральную формулу Коши для
контура $|z_2| \bw= 1-\ep$. Рассмотрим функцию
\eqn{\label{HartogsInt}F(z_1,z_2) = \frac{1}{2\pi i}\ints{|\ze| = 1-\ep} \frac{f(z_1,\ze)}{\ze - z_2}\,d\ze.}
Сия формула имеет смысл, поскольку весь контур интегрирования содержится в области голоморфности функции $f$.
Далее, заметим, что когда второй аргумент функции $f$ близок по модулю к $1$, первому аргументу разрешается
принимать любые по модулю значения, меньшие $1$. Отсюда следует, что интеграл~\eqref{HartogsInt} определён
при всех $|z_2| < 1 - \ep$ и $|z_1| < 1$. Видно, что $F$ будет непрерывной и голоморфной по каждому аргументу.
Осталось заметить, что если ограничить $|z_1|$ шириной перемычки,
то этот интеграл совпадает с формулой Коши для исходной функции $f$: в перемычке (а точнее в
закрашенном бицилиндре) она голоморфна и потому для неё справедлива формула Коши. Значит, на
перемычке имеем $f = F$.

Радость состоит в том, что область определения функции $F$ шире, чем у исходной. Поскольку $\ep$ можно взять
произвольно малым, получаем, что функция $f$ продолжается до функции, голоморфной в~бидиске
$\hc{|z_1| < 1, |z_2| < 1}$.

\subsubsection{Принцип непрерывности и области голоморфности}

\begin{df}
Пусть $\De$-- диск в $\Cbb^r$, $r < n$. Образ диска $\De$ при голоморфном биективном
отображении $\ph\cln \De \ra \Cbb^n$ с всюду невырожденным дифференциалом называется
\emph{аналитической $r$-- мерной поверхностью}. При $r=1$ эти поверхности
часто называют \emph{аналитическими кривыми}.
\end{df}

Если теперь рассмотреть функции на аналитических кривых (одномерных комплексных многообразиях),
то для них очевидным образом справедлива теорема единственности
и принцип максимума. Для доказательства достаточно перетащить функцию на диск:
если $S = \ph(\De)$-- аналитическая кривая, а $f\cln S\ra \Cbb^n$-- голоморфная функция
на кривой $S$, то функция $g(t) = f\br{\ph(t)}$ уже будет обычной голоморфной функцией.


\begin{theorem}
Пусть $\De \subs \Cbb$-- единичный круг, $D \subs \Cbb^n$-- область,
а $\ph\cln \ol{\De}\ra D$-- аналитическая кривая.
Пусть $K \sups \ph(\pd\De)$-- компакт в области $D$, а $\rh := \dist(K, \pd D)$.
Пусть $f \in \Oc(D)$. Тогда $f$ аналитически продолжается в $\frac\rh2$-- окрестность множества $\ph(\ol\De)$.
\end{theorem}
\begin{proof}
Положим для краткости $C := \ph(\pd\De)$, а $S := \ph(\ol\De)$.
Поскольку $\frac\rh2$-- окрестность всякой точки $z \in C$
содержится в области~$D$, радиус сходимости степенного ряда функции $f$ в этих точках не меньше,
чем $\frac\rh2$. Запишем разложение функции $f$ в степенной ряд с центром в каждой точке поверхности~$S$.
Заметим, что коэффициенты этого ряда Тейлора являются голоморфными функциями в зависимости от центра точки
разложения:
\eqn{c_m(a) = \frac{1}{m!}f^{(m)}(a).}
Функция $f$ ограничена на множестве $K \cup S$ некоторой константой $M$, потому что $K$ и $S$-- компакты.
Поэтому можно написать неравенство Коши:
\eqn{|c_m(a)| \le \frac{M}{R^m(a)},}
так как радиус разложения, вообще говоря, зависит от точки.
Рассмотрим функции
\eqn{c_m\circ\ph\cln \De\ra\Cbb.}
Для них справедлив обычный принцип максимума, поэтому они могут достигать своего максимума
только на границе $\De$. Но чем больше коэффициент ряда Тейлора, тем хуже получается
оценка на радиус сходимости этого ряда. Таким образом, <<самые плохие>> оценки на радиус
сходимости получатся на границе, то есть в точках множества $C$. Но в этих точках
мы уже гарантировали себе радиус $\frac\rh2$. Следовательно, эту оценку снизу на радиус
можно распространить на всю поверхность~$S$.
\end{proof}

\begin{note}
Вся прелесть этой теоремы состоит в том, что поверхность $S$ может подходить достаточно
близко к границе области $D$, и $\frac\rh2$-- окрестности её точек могут вылезти за пределы $D$.
Это означает, что функция $f$ аналитически продолжается в большую область.
\end{note}

\begin{df}
Область $D$ называется областью голоморфности, если существует функция $f \in\Oc(D)$ такая,
что она не продолжается аналитически ни через одну точку границы $D$.
\end{df}

Область голоморфности обладает <<выпуклостью>> в некотором смысле, но мы не будем уточнять, в каком.

\subsection{Биголоморфные отображения}

При рассмотрении функций одной переменной у нас были хорошие конформные отображения.
В многомерном случае с конформностью придётся расстаться. Некоторым аналогом конформных отображений
являются так называемые биголоморфные отображения.

\begin{df}
Пусть $D_1, D_2$-- области в $\Cbb^n$.
Отображение $f\cln D_1 \xra{\text{на}} D_2$ называется \emph{биголоморфным}, если $f$ и $f^{-1}$ голоморфны.
\end{df}

Такое определение позволяет говорить о биголоморфной эквивалентности областей: $D_1 \sim D_2$, если
существует биголоморфное отображение $D_1$ на $D_2$.

\subsubsection{Теорема о неявном отображении}

\begin{theorem}[о неявном отображении]
Пусть задано уравнение $F(\vec z, \vec w) = 0$, где $F\cln \Cbb^n \times \Cbb^m \ra \Cbb^m$-- голоморфное
отображение. Пусть $F(a,b) = 0$ и
\eqn{\det \hr{\pf{F(a,b)}{w}} \neq 0.}
Тогда существует голоморфная функция $w = f(z)$, для которой  в некоторой окрестности
точки $a$ выполнено тождество
\eqn{\label{ImplicitFuncEq}F\br{z,f(z)} \equiv 0.}
\end{theorem}
\begin{proof}
Заметим, что выполнены условия вещественной теоремы о неявном отображении. Применив её, получаем, что
существует $\R$-- дифференцируемая функция $w = f(z)$, удовлетворяющая нашему уравнению.
Имеем
\eqn{d F = \pf{F}{z}\,d z + \pf{F}{w}\,d w,}
\eqn{d f = \pf{f}{z}\,d z + \pf{f}{\ol z}\,d \ol z.}
Продифференцируем уравнение~\eqref{ImplicitFuncEq}, получим:
\eqn{\pf{F}{z}\,dz + \pf{F}{w}\hr{\pf{f}{z}\,dz + \pf{f}{\ol z}\,d \ol z} \equiv 0.}
Приводя подобные слагаемые при $dz$ и $d \ol z$, получаем
\eqn{\hr{\pf{F}{z} + \pf{F}{w}\pf{f}{z}}dz + \pf{F}{w}\cdot\pf{f}{\ol z}\,d \ol z \equiv 0.}
Матрица $\hr{\pf{F}{w}}$ невырождена, поэтому обязан быть нулевым вектор $\pf{f}{\ol z}$. Но
это и означает голоморфность функции $f$.
\end{proof}

\subsubsection{Примеры биголоморфных отображений}

Через $\Aut(D)$ будем обозначать группу биголоморфных отображений области $D \subs \Cbb^n$ на себя.
Если $D_1 \sim D_2$, то $\Aut(D_1) \cong \Aut(D_2)$. Этот изоморфизм задаётся следующим образом.
Пусть $D_2 = \ph(D_1)$, а $f_1 \in \Aut(D_1)$. Сопоставим автоморфизму $f_1$ автоморфизм
$f_2 := \ph f_1 \ph^{-1} \in \Aut(D_2)$.

Примерами биголоморфных автоморфизмов пространства $\Cbb^n$ будут линейные преобразования,
то есть группа $\GL_n(\Cbb)$.

Далее мы будем рассматривать пространство $\Cbb^2(z,w)$.

\begin{df}
Преобразования вида
\eqn{\label{TriangleTransform}\case{\wt z = z + f(w),\\ \wt w = w,}}
где $f(w)$-- целая функция, называются \emph{треугольными} преобразованиями.
\end{df}

Очевидно, обратным к треугольному преобразованию~\eqref{TriangleTransform} является преобразование
\eqn{\case{z = \wt z - f(\wt w),\\ w = \wt w,}}
которое также является треугольным. Следовательно, они образуют группу.

\begin{note}
В отличие от одномерного случая, группа биголоморфных отображений $\Cbb^n$ на себя бесконечномерна,
так как подгруппа треугольных преобразований бесконечномерна (их не меньше, чем целых функций, а их, в
свою очередь, не меньше, чем многочленов произвольной степени).
\end{note}

Как мы знаем, множество $\Cbb$ нельзя отобразить конформно и однолистно на единичный диск. В многомерном
случае такое уже возможно (см. пример в~\cite{shabat}).

\subsubsection{Дробно-- линейные отображения в $\CP^2$}

В пространстве $\Cbb^2$ можно рассматривать дробно-линейные преобразования вида
\eqn{\wt z= \frac{Az + Bw+C}{pz+qw+r}, \quad \wt w = \frac{az + bw+c}{pz+qw+r}, \quad
\mbmat{A&B&C\\a&b&c\\p&q&r} \neq 0.}
Но на самом деле удобнее вложить аффинное пространство $\Cbb^2$ в проективное пространство $\CP^2$,
поскольку в однородных координатах всё выглядит более симметрично.

Напомним, что
\eqn{\CP^2 = \hc{(\ze_0:\ze_1:\ze_2)},\quad \ze_i \in \Cbb,\quad |\ze_0| + |\ze_1| + |\ze_2| \neq 0.}
При этом тройки $(\ze_0:\ze_1:\ze_2)$ рассматриваются с точностью до пропорциональности, то есть
\eqn{(\ze_0:\ze_1:\ze_2) \sim (\la \ze_0:\la \ze_1:\la \ze_2), \quad \la \neq 0.}
Смысл условия $|\ze_0| + |\ze_1| + |\ze_2| \neq 0$ вполне ясен: хотя бы одна из однородных
координат не равна нулю.

Построим какое-- нибудь вложение $\Cbb^2 \inj \CP^2$: положим
$z = \frac{\ze_1}{\ze_0}$ и $w = \frac{\ze_2}{\ze_0}$. Это позволяет отождествить
подмножество $\hc{(1:\ze_1:\ze_2)} \subs \CP^2$ с множеством $\Cbb^2$.

В алгебраических терминах имеем
$\CP^2 = \fact{\hr{\Cbb^3\wo\hc{0}}}{\Cbb^*}$, и ещё можно считать, что
\eqn{\CP^2 = \Cbb^2 \cup \CP^1 = \Cbb^2 \cup \Cbb^1 \cup \hc{\bes}.}

Запишем наше дробно-- линейное преобразование в однородных координатах:
\eqn{\frac{\wt \ze_1}{\wt \ze_0}=
\frac{A\frac{\ze_1}{\ze_0}+B\frac{\ze_2}{\ze_0}+C}{p\frac{\ze_1}{\ze_0}+q\frac{\ze_2}{\ze_0}+r},\quad
\frac{\wt \ze_2}{\wt \ze_0} =
\frac{a\frac{\ze_1}{\ze_0}+b\frac{\ze_2}{\ze_0}+c}{p\frac{\ze_1}{\ze_0}+q\frac{\ze_2}{\ze_0}+r}.}
Приводя дроби к общему знаменателю и записывая эти выражения  в матричной форме, получаем
\eqn{\rbmat{\wt\ze_0\\\wt\ze_1\\\wt\ze_2} = \rbmat{r&p&q\\C&A&B\\c&a&b}\rbmat{\ze_0\\\ze_1\\\ze_2}.}
В итоге получаем группу линейных преобразований проективного пространства $\PGL_3(\Cbb)$
комплексной размерности~8. Всякое преобразование из $\PGL_n$ задаётся образом
$\hr{n+1}$-- й точки.\footnote{Здесь индекс $n$ у группы $\PGL$ означает размерность объемлющего
комплексного пространства. Соответствующее проективное пространство
имеет на единицу меньшую размерность.-- \emph{Прим. наб.}}

\medskip

Построим дробно-- линейное отображение $\Cbb^2 \wo \Cbb$ на $\Cbb^2 \wo \Cbb$.
Именно, возьмём отображение
\eqn{\text{прямое: } \case{\wt z = \frac{2z}{w + i},\\ \wt w = \frac{w-i}{w+i};} \quad
\text{обратное: }\case{z = -i\frac{\wt z}{\wt w -1},\\ w = -i\frac{\wt w+1}{\wt w-1}.}}
Прямым подсчётом проверяется, что
\eqn{\case{|\wt w| < 1 &\;\Lra\; \Img w > 0,\\
|\wt z|^2 + |\wt w|^2 < 1 &\;\Lra\; \Img w > |z|^2.}}
Таким образом, получаем отображение $\Cbb^2 \wo \hc{w + i = 0} \lra \Cbb^2 \wo \hc{\wt w - 1 = 0}$.

\begin{petit}
Тут ещё были какие-то формулы. Но к чему они, осталось загадкой.
\end{petit}


\subsubsection{Обобщённый принцип максимума и лемма Шварца}

Скажем пару слов о нормах в $\Cbb^n$. Вообще, чтобы задать норму в линейном пространстве,
нужно задать некоторое множество и объявить, что это единичный шар в смысле этой нормы.
Разумеется, годятся не всякие множества, но мы будем работать только с шарами и полидисками,
которые для этой цели вполне подходят.

Единичный шар $B^n$ является шаром в норме $\hn{z}_1 = \sqrt{|z_1|^2\spl|z_n|^2}$, а
полидиск $\De^n$-- шаром в норме $\hn{z} \bw= \maxl{j} |z_j|$.

\begin{lemma}
Во всякой норме шар является выпуклым множеством.
\end{lemma}
\begin{proof}
Пусть $B = \hc{z\cln \hn{z} \le 1}$-- единичный шар. Пусть $a, b \in B$. Рассмотрим произвольную точку
$\ze$ на отрезке $[a,b]$. Она имеет вид $\ze = \la a + (1-\la) b$, где $\la \in [0,1]$.
Применим неравенство треугольника:
\eqn{\hn{\ze} = \hn{\la a + (1-\la) b} \le \la\cdot \ub{\hn{a}}_{\le1} +
(1-\la)\cdot\ub{\hn{b}}_{\le1} \le \la + (1-\la) = 1.}
Таким образом, $\hn{\ze} \le 1$ и потому $\ze \in B$.
\end{proof}

\begin{lemma}[Обобщённый принцип максимума]
Пусть $D$-- область в $\Cbb$. Пусть $f\cln D \ra \Cbb^m$, и на $\Cbb^m$ введена норма $\hn{\cdot}$.
Пусть в некоторой точке $a \in D$ достигается максимум нормы $\hn{f(z)}$, то
есть $\hn{f(a)} \ge \hn{f(z)}$ при всех $z \in D$. Тогда $\hn{f(z)} \equiv \const$.
\end{lemma}
\begin{proof}
Если $f(a) = 0$, то доказывать нечего, так как $f \equiv 0$. В противном случае
рассмотрим шар $B$ в $\Cbb^m$ радиуса $\hn{f(a)}$. Тогда $f(D) \subs B$,
а точка $f(a)$ лежит на границе этого шара.
Так как шар $B$ является выпуклым множеством, существует\footnote{Строгое доказательство
этого факта нетривиально. Его можно прочесть, например,
в~\cite[гл.\,7,~\S\,2,~теорема\,1]{vinberg}-- \emph{Прим. наб.}} опорная
гиперплоскость $\pi$ (то есть такая плоскость, что одно из полупространств,
на которые она разбивает всё пространство, не содержит точек шара).

Через $w = u + i v$, где $u,v \in \R^m$, будем обозначать точки в пространстве $\Cbb^m$.
Гиперплоскость $\pi$ задаётся в $\Cbb^m \cong \R^m \oplus i\R^m$ некоторым вещественным
линейным функционалом $\ell(u,v) = \al$. Его можно единственным образом продолжить до
комплексного линейного функционала по правилу
$\ell(i \vec x) = i\ell(\vec x)$, $x \in \R^m\oplus i\R^m$. Полученное продолжение
обозначим через $L$. При этом имеем $\ell = \Rea L$. Рассмотрим голоморфную
функцию $g(z) = \exp L\br{f(z)}$. Имеем
\eqn{|g(z)| = \hm{\exp L\br{f(z)}} = \exp \Rea L\br{f(z)} = \exp \ell(u,v).}
Для определённости считаем, что то полупространство, куда не попало ни одной точки из
множества~$f(D)$,
задаётся неравенством $\ell(u,v) > \al$. Тогда для всех $z \in D$ имеем
\eqn{|g(z)| = \exp \ell(u,v) \le \exp\al,}
причём в точке $a$ достигается равенство. Применив к функции $g$ обычный принцип максимума,
получаем, что $g$~обязана быть константой.

Отсюда следует, что на образе всего множества $D$ функционал $\ell$ постоянен. Значит,
$f(D) \subs \pi$. Но так как $\pi$-- опорная гиперплоскость, она
не имеет общих точек с внутренностью шара $B$. Поэтому множество $f(D)$ может лежать только
на границе шара, но это и означает, что $\hn{f(z)}=\hn{f(a)} = \const$.
\end{proof}

\begin{note}
В случае, если норма такова, что шар в ней является \emph{строго} выпуклым множеством,
можно утверждать большее, а именно то, что $f \equiv \const$, поскольку в этом случае
гиперплоскость пересекается с шаром ровно в одной точке.
\end{note}

\begin{lemma}[Шварца]
Пусть $X_1 := \br{\Cbb^{n_1},\hn{\cdot}_1}$ и $X_2 := \br{\Cbb^{n_2},\hn{\cdot}_2}$--
нормированные пространства. Пусть $B_1$-- единичный шар в~$X_1$, а $B_2$-- единичный
шар в пространстве~$X_2$.
Пусть $f\cln B_1 \ra B_2$-- голоморфное отображение, для которого $f(0) = 0$. Тогда
$\hn{f(z)}_2 \le \hn{z}_1$.
\end{lemma}
\begin{proof}
Пусть $z \in B_1$. Рассмотрим единичный вектор $\vec a := \frac{z}{\hn{z}_1}$ и прямую
$\ell := \hc{t\vec a\vl t \in \Cbb}$. Рассмотрим функцию $\ph(t) = \frac{f(ta)}{t}$.
Она голоморфна, поскольку $f(0) = 0$ и особенность устраняется. Пусть $|t| \le r < 1$,
тогда по принципу максимума $\hn{\ph(t)}_2 \le \frac1r$. Поскольку $r$ можно брать сколь
угодно близким к $1$, в пределе получаем неравенство
\eqn{\hn{\ph(t)} \le 1 \;\Lra\; \hn{\frac{f(ta)}{t}}_2 \le 1 \;\Lra\; \hn{f(ta)}_2 \le |t|.}
Осталось положить $t = \hn{z}_1$.
\end{proof}


\subsubsection{Биголоморфная неэквивалентность шара и полидиска}

Скажем пару слов о группах автоморфизмов шара и полидиска.
Очевидно, что $\Aut(\De^n) \bw\sups \Sb_n \times (\Aut \De)^n$, так как можно как угодно переставлять
координаты и осуществлять конформные автоморфизмы дисков по каждой координате
преобразованиями вида
\eqn{z_j \mapsto e^{i\ph_j}\frac{z_j-a_j}{1 - \ol a_j z_j}.}
Отсюда следует, что $\dim_\R \Aut(\De^n) \ge n+2n$ (умножение на дискретную группу $\Sb_n$
на размерность не влияет).
На самом деле можно показать, что группа $\Aut(\De^n)$ исчерпывается такими преобразованиями,
и таким образом $\dim_\R \Aut(\De^n) = 3n$.

Что же касается шара, то можно показать, что для него $\dim_\R \Aut(B^n) = n^2 + 2n$. Легко видеть, что
в группе автоморфизмов шара содержится унитарная группа $\Ub_n$, состоящая из таких комплексных
матриц $A$, что $\ol A \cdot A^t = E$. Её размерность равна $n^2$ (у матрицы $2n^2$ вещественных переменных,
и на них накладывается $n^2$~независимых соотношений). Тем самым мы показали, что $\dim_\R \Aut(B^n) \ge n^2$.

Если применить соображения о размерностях, то доказательство биголоморфной неэквивалентности очевидно:
при $n \ge 2$ имеем $3n < n^2 +2n$ и потому соответствующие группы автоморфизмов не могут быть изоморфными.
Но мы пойдём другим путём, который не использует <<тяжёлой артиллерии>>.

\begin{theorem}
При $n \ge 2$ шар и полидиск биголоморфно неэквивалентны друг другу.
\end{theorem}
\begin{proof}
Будем рассуждать от противного: пусть существует биголоморфное отображение $\ph\cln B^n \ra \De^n$.
Из сказанного выше следует, что полидиск однороден, то есть
группа автоморфизмов действует на нём транзитивно (любую точку можно перевести в любую другую
перестановкой координат и координатным автоморфизмом).

Если бы полидиск и шар были биголоморфно эквивалентными, то шар тоже был бы однородным.
В самом деле, пусть $x, y \in B$. Рассмотрим точки $z = \ph(x)$ и $w = \ph(y)$. В силу однородности
полидиска, существует $f \in \Aut(\De^n)$, для которого $f(z) = w$. Тогда автоморфизм
шара $\ph^{-1} f \ph$ переводит~$x$ в~$y$.

Поэтому можно считать, что $\ph$ сохраняет $0$, то есть $\ph(0) = 0$. Действительно, пусть $\ph(0) = z_0$.
Рассмотрим автоморфизм полидиска $f$, переводящий $z_0$ в точку $0$. Тогда $\Ph := f \circ\ph$ уже обладает
требуемым свойством. А раз $\Ph(0) = 0$, то применима лемма Шварца: пусть $w = \Ph(z)$, тогда
$\hn{w}_\De \le \hn{z}_B$. Но так как $\Ph$-- биголоморфизм, лемма Шварца применима и к $\Ph^{-1}$.
Значит, верно и обратное неравенство $\hn{z}_B \le \hn{w}_\De$. Следовательно,
\eqn{\hn{\Ph(z)}_\De = \hn{z}_B.}
Но отсюда, в частности, следует, что всякая сфера радиуса $r < 1$ переходит в границу полидиска
(которая при $n\ge 2$ не диффеоморфна сфере в силу
<<угловатости>>\footnote{Почему полидиск угловатый, и почему они правда не диффеоморфны, подумайте сами.}).
Получаем противоречие.
\end{proof}

\begin{petit}
На самом деле легко видеть, что граница двумерного полидиска представляет собой склейку двух полноторий.
Склейка производится по их общей части, то есть по тору $\hc{|z_1|=1}\times \hc{|z_2|=1}$.
\end{petit}

\subsubsection{Теорема Анри Картана}

\begin{df}
\emph{Область ограниченного вида}-- область, биголоморфно эквивалентная ограниченной.
\end{df}

\begin{theorem}[А.\,Картана]
Пусть $D$-- область ограниченного вида, а $f \in \Aut(D)$. Если $f(a) = a$ и $f'(a) = \id$, то
$f = \id$.
\end{theorem}
\begin{proof}
Как обычно, доказываем от противного. Без ограничения общности $a = 0$. Напишем разложение Тейлора:
\eqn{f(z) = z + P_m(z) + O(z^{m+1}),}
где $P_m(z)$-- ненулевой многочлен степени $m$, у которого равны нулю коэффициент при $z$ и свободный член.
Рассмотрим итерации отображения $f$ (далее под записью $f^\nu$ мы понимаем $\nu$-- ю степень композиции).
Имеем
\eqn{f^2(z) = f\br{f(z)} = f\br{z+P_m(z)} = \br{z + P_m(z) +\dots} + P_m\br{z + P_m(z)+\dots} +\dots = z + 2P_m(z) +\dots}
Аналогично получаем, что
\eqn{f^\nu(z) = z + \nu P_m(z)+\dots}
Впишем в область $D$ полидиск $\De_1$ c центром в нуле радиуса $r$ и опишем вокруг $D$ ещё один
полидиск $\De_2$ радиуса~$R$ тоже с центром в нуле. Тогда $\hm{f(z)} \le R$ при всех $z \in D$, поскольку
$f$ отображает $D$ в себя, а~следовательно, и внутрь большого полидиска.
К меньшему полидиску применимо неравенство Коши:
\eqn{|c_m| \le \frac{R}{r^m}.}
Соответственно, для итераций должна быть справедливой та же оценка, то есть
\eqn{|\nu c_m| \le \frac{R}{r^m}.}
Но при достаточно больших $\nu$ неравенство, очевидно, нарушится. Значит, на самом деле $P_m = 0$.
\end{proof}

\begin{imp}
Если имеется биголоморфизм $\ph_1\cln D_1 \ra D_2$, то он задаётся образом одной точки и значением
производной в этой точке.
\end{imp}
\begin{proof}
В самом деле, пусть имеется два отображения $\ph_1$ и $\ph_2$ с такими свойствами. Тогда
отображение $\ph_1\circ \ph_2^{-1}$ удовлетворяет условиям теоремы Картана и потому тождественно,
то есть $\ph_1 = \ph_2$.
\end{proof}

\begin{imp}
Группа автоморфизмов произвольной области в $\Cbb^n$ имеет размерность не более $2n^2+2n$.
\end{imp}
\begin{proof}
Дифференциал биголоморфизма-- это невырожденная комплексная матрица размера $n\times n$, а пространство
таких матриц имеет размерность $2n^2$ (это в точности полная линейная группа $\GL_n(\Cbb)$).
А чтобы задать образ одной точки, нам потребуется ещё $n$ комплексных координат, то есть $2n$ вещественных.
По предыдущему следствию, эти параметры полностью задают биголоморфизм.
\end{proof}

\section{Представление мероморфных и целых функций}

Вернёмся к функциям одной переменной.

\subsection{Представление мероморфных функций}

\subsubsection{Теорема Миттаг-- Леффлера}

Как мы знаем, для рациональной функции справедливо представление вида
\eqn{R(z) = P_0(z) + \sum P_j\hr{\frac1{z-a_j}},}
где $P_j$-- многочлены. В этом разделе мы докажем, что аналогичное представление
справедливо и для мероморфных функций.

Напомним, что мероморфная в $\Cbb$ функция-- это такая функция, у которой все особые точки не хуже,
чем полюса, и при этом полюсам разрешено накапливаться только вокруг $\bes$.

Для простоты будем всегда рассматривать функции, мероморфные во всей плоскости.

\begin{df}
Пусть $\hc{f_n}$-- последовательность мероморфных функций. Говорят, что ряд $\sum f_n$ сходится, если
для каждого компакта $K \subs \Cbb$ найдётся $N$ такое, что при $n > N$ имеем $f_n \in \Oc(K)$ (то есть
<<хвост>> не имеет полюсов на компакте $K$), и <<хвост>> ряда $\sums{n \ge N} f_n$ сходится на $K$ равномерно.
\end{df}

Из определения и теоремы Вейерштрасса следует, что <<хвост>> является голоморфной функцией.
Поэтому предел автоматически является мероморфной функцией.

\begin{lemma}
Пусть имеется последовательность дисков $\hc{\De_n(r_n)}$, радиусы которых возрастают к~$+\bes$,
и $\hc{F_n}$-- последовательность функций, мероморфных в $\Cbb$ и голоморфных на $\ol\De_n$.
Тогда существует последовательность так называемых поправочных многочленов $\hc{P_n}$,
что ряд
\eqn{\label{FuncCorrByPolynoms}\sum\br{F_n(z) - P_n(z)}}
сходится как ряд из мероморфных функций.
\end{lemma}
\begin{proof}
Зафиксируем произвольную последовательность $\ep_n > 0$, для которой ряд
$\sum \ep_n$ сходится.
В силу голоморфности $F_n$ на $\ol \De_n$ имеет место разложение $F_n$ в ряд
\eqn{F_n(z) = \sums{m} c_{mn}z^m,}
равномерно сходящийся на $\ol \De_n$. В силу этой сходимости можно приблизить
функцию $F_n$ отрезком её степенного ряда с точностью $\ep_n$, то есть найти $m_n$ такое,
что
\eqn{\hm{F_n - P_n} < \ep_n, \quad P_n = \suml{m=0}{m_n} c_{mn} z^m.}
Так как ряд из $\ep_n$ сходится, то ряд~\eqref{FuncCorrByPolynoms} будет равномерно сходиться
на каждом из $\ol \De_n$, что и требуется.
\end{proof}

Пусть $a_n$-- полюса некоторой мероморфной функции $f$. Через $g_n$ будем обозначать
главную часть её лорановского разложения в полюсе $a_n$, то есть
\eqn{\label{LoranMainPart}g_n(z) = Q_n\hr{\frac1{z-a_n}},\quad Q_n \in \Cbb[z],\quad \deg Q_n = p_n.}

\begin{theorem}[Миттаг-- Леффлера]
Пусть $\hc{a_n}$-- последовательность точек, не имеющая точек накопления в $\Cbb$, и $g_n$-- набор
главных частей $g_n$ вида~\eqref{LoranMainPart}. Тогда существует
мероморфная функция, имеющая полюса ровно в этих точках, причём при всех $n$ главная часть лорановского
разложения в полюсе $a_n$ совпадает с $g_n$.
\end{theorem}
\begin{proof}
Упорядочим полюсы по возрастанию их модулей (это всегда можно сделать, потому что
во всяком круге их лишь конечное число).
Будем сначала считать, что $a_0 \neq 0$. Положим $F_n(z) = g_n(z)$. Применим лемму и рассмотрим
функцию
\eqn{F(z) = \sumnui\hr{g_n(z) - P_n(z)},}
где $P_n$-- поправочные многочлены. Она, очевидно, и будет искомой.

Избавиться от ограничения $a_0 \neq 0$ можно очень просто: если $F(z)$-- функция, имеющая требуемые
главные части в ненулевых полюсах, то функция
\eqn{\wt F(z) = g_0(z) + F(z)}
будет иметь полюса во всех точках $a_0,a_1\etc$
\end{proof}

\begin{imp}
Всякая мероморфная функция с полюсами в точках $\hc{a_n}$ имеет представление вида
\eqn{f(z) = G(z) + \sumnzi\hr{g_n(z) - P_n(z)}, \quad G \in \Oc(\Cbb).}
\end{imp}

\subsubsection{Метод Коши}

\begin{theorem}
Пусть $\ga_n$-- последовательность контуров таких, что $r_n := \rho(\ga_n,0) \ra \bes$.
Пусть $|\ga_n| = L_n$ и $L_n \le C r_n$, где $C$-- константа, ни от чего не зависящая. (Последнее
условие означает, что контуры не слишком <<кривые>>).
Пусть мероморфная функция $f$ такова, что
\eqn{|f(z)| \le M|z|^m \text{ при } z \in \ga_n \text{ для всех } n.}
Тогда в качестве поправочных многочленов можно брать многочлены степени не выше $m$.
\end{theorem}
\begin{proof}
Обозначим $D_n := \Int\ga_n$.
Возьмём контур $\ga_n$ и зафиксируем точку $z$ внутри него. Запишем теорему Коши о вычетах:
\eqn{\label{CauchyFormula}\frac1{2\pi i} \ints{\ga_n} \frac{f(\ze)}{\ze - z}\,d\ze = f(z) +
\ub{\sums{a_k\in D_n}\res_{a_k} f}_S.}
Последнее равенство верно, поскольку полюсами подинтегральной функции как раз являются
все особые точки внутри контура~$\ga_n$ и точка $z$, вычет в которой по интегральной
формуле Коши равен $f(z)$.

Рассмотрим функцию
\eqn{\ph_n(z) = \mcomp{\sums{a_k\in D_n}}g_k(z).}
Заметим, что в формуле~\eqref{CauchyFormula} можно $f$ заменить на $\ph_n$, потому что вычитание
голоморфного слагаемого ничего не поменяет.

Функция $\ph_n$ состоит из правильных дробей. При делении на $\ze - z$ они станут <<ещё правильнее>>,
и степень знаменателя будет по крайней мере на $2$ больше, чем степень числителя. Отсюда следует,
что интеграл
\eqn{I := \frac1{2\pi i}\ints{\ga_n} \frac{\ph_n(\ze)}{\ze - z}\,d\ze = 0.}
В самом деле, если увеличивать номер контура, то интеграл не меняется, поскольку вне контура
у $\ph_n$ полюсов нет. С другой стороны, при увеличении радиуса контура интеграл стремится к нулю,
потому что длина контура растёт так же, как $r_k$, а функция убывает как $r_k^2$.

Таким образом, имеем
\eqn{0 = I = \ph_n(z) + S,}
поэтому $S = - \ph_n(z)$. Подставим полученное выражение для $S$ в формулу~\eqref{CauchyFormula}:
\eqn{\frac1{2\pi i} \ints{\ga_n} \frac{f(\ze)}{\ze - z}\,d\ze = f(z) - \mcomp{\sums{a_k\in D_n}} g_k(z).}

Разложим ядро Коши:
\eqn{\frac1{\ze-z} = \frac1\ze \cdot\frac1{1-\frac z\ze} = \sumkzi \frac{z^k}{\ze^{k+1}}.}
Теперь дифференцируем полученную формулу $k$ раз ($k = 0\sco m$) и подставляем $z = 0$:
\eqn{\frac1{2\pi i} \ints{\ga_n} \frac{f(\ze)}{\ze^{k+1}}\,d\ze = \frac{f^{(k)}(0)}{k!} -
\mcomp{\sums{a_j\in D_n}} \frac{g^{(k)}_j(0)}{k!}.}
Вычтем продифференцированные равенства из исходного, домножая их на $z^k$. Получим
\eqn{\frac1{2\pi i}\ints{\ga_n}\hr{\frac1{\ze - z}-\suml{k=0}{m}\frac{z^k}{\ze^{k+1}}}f(\ze)\,d\ze =
f(z) - \ub{\suml{k=0}{m}\frac{f^{(k)}(0)}{k!}z^k}_{G(z)} -
\sums{a_j\in D_n}\bbbr{g_j(z) - \ub{\suml{k=0}{m}\frac{g_j^{(k)}(0)}{k!}z^k}_{P_j(z)}}.}
Выражение в скобках под интегралом является хвостом геометрической прогрессии, поэтому
левая часть равенства может быть записана в виде
\eqn{J := \frac{z^{m+1}}{2\pi i}\ints{\ga_n} \frac{f(\ze)}{\ze^m\cdot \ze(\ze-z)}\,d\ze.}
Поэтому
\eqn{|J| \le \frac{C \cdot M |z|^{m+1}}{2\pi(r_n-|z|)} \ra 0,\quad n \ra \bes.}
Значит $J = 0$. Заметим, что если теперь перенести $f(z)$ в левую часть, то
справа останется целая функция $G$ и сумма подправленных дробей. При этом поправочные
многочлены имеют степень не выше $n$.
\end{proof}

В качестве полезного в приложениях следствия этой теоремы получим формулу для разложения
мероморфной функции, у которой все полюса простые и точка $z=0$ не является полюсом.

Пусть $|f(z)| \le M$ на системе контуров $\ga_n$ (чтобы так получилось, нужно аккуратно
провести контуры подальше от полюсов). Тогда можно взять $m=0$. Условие простоты полюсов
означает, что функции $g_k$ в разложении $f$ имеют вид
\eqn{g_k(z) = \frac{b_k}{z-a_k}.}
Фактически, здесь коэффициенты $b_k$-- это вычеты функции $f$ в её полюсах.
По теореме получаем разложение для $f$ следующего вида:
\eqn{f(z) = f(0) + \sumkui b_k\hr{\frac1{z-a_k}+\frac1{a_k}}.}

Если функция $f$ таки имеет полюс в нуле, его можно сначала убить, вычтя функцию
$g_0(z) := \frac{\res_0 f}{z}$. Тогда для функции $f - g_0$ уже работает написанная выше формула.


\smallskip
Здесь и далее под символом $\sum'$ будем понимать \textbf{суммирование по всем целым
индексам, кроме нуля}.

\begin{ex}
Рассмотрим функцию $f(z) = \ctg z$. Регуляризуем её в нуле и запишем для неё разложение:
\eqn{\label{ctgDecomposition}\ctg z - \frac1z = \sumsstr{k}\hr{\frac1{z-\pi k} + \frac1{\pi k}}.}
Если сгруппировать положительные и отрицательные слагаемые, то получится следующее:
\eqn{\ctg z = \frac1z + \sumkui \frac{2z}{z^2-(\pi k)^2}.}

С помощью почленного дифференцирования равенства~\eqref{ctgDecomposition} получается разложение
для функции $\frac1{\sin^2z}$:
\eqn{\frac1{\sin^2z} = -(\ctg z)' = \sums{\Z} \frac1{(z-\pi k)^2}.}
\end{ex}

\subsection{Представление целых функций}

Если у нас есть многочлен $P$, то ничего не стоит получить его разложение по его нулям:
\eqn{P(z) = A z^m \prodl{k=1}{n}\hr{1 - \frac{z}{a_k}}.}
Поскольку у многочленов много общего с целыми функциями, возникает естественное желание
получить аналогичное разложение
для целых функций. Правда, при этом получаемое произведение рискует стать бесконечным.

\subsubsection{О бесконечных произведениях}

Напомним, что бесконечное произведение
\eqn{\prodl{k=1}{\bes} (1+c_k)}
называется сходящимся, если $1 + c_k \neq 0$ при всех $k$ и существует конечный ненулевой предел
последовательности
\eqn{\Pi_n := \prodl{k=1}{n}(1+c_k).}

Очевидно, что если произведение сходится, то
\eqn{1 + c_n  = \frac{\Pi_n}{\Pi_{n-1}} \ra 1 \quad\Ra\quad c_n \ra 0.}
Отсюда следует, что при всех  достаточно больших $n$ все множители лежат в достаточно малой
окрестности $1$, а там у функции $\Ln(1+z)$ выделяется однозначная ветвь и можно рассмотреть
ряд
\eqn{\sumnui \ln(1 + c_n).}
Легко видеть, что сходимость произведения равносильна сходимости этого ряда.

\subsubsection{Теорема Вейерштрасса о заказанных нулях}

Для начала заметим, что если $f$-- целая функция без нулей, то функция $h(z) = \ln f(z)$ неограниченно
продолжается в $\Cbb$ и по теореме о монодромии тоже является целой функцией. Тогда
получаем $f(z) = \exp h(z)$.

Пусть у целой функции $F$ конечное число нулей $a_1\sco a_n$ с кратностями $p_1\sco p_n$. Тогда функция
\eqn{f(z) = \frac{F(z)}{(z-a_1)^{p_1}\sd(z-a_n)^{p_n}}}
уже не имеет нулей и тоже является целой функцией. По предыдущему рассуждению получаем разложение
\eqn{F(z) = (z-a_1)^{p_1}\sd(z-a_n)^{p_n}\exp h(z).}

В общем случае необходимо побороть кратные нули. Пусть целая функция $F$ имеет нули
в точках $a_k$ с кратностями $p_k$. Будем считать, что $F(0) \neq 0$.
Перейдём к логарифмической производной: рассмотрим функцию
\eqn{L(z) := \frac{F'(z)}{F(z)}.}
У неё все полюса простые, и вычеты функции $L$-- это в точности кратности нулей функции $F$.

Значит, она имеет вид
\eqn{L(z) = H(z) + \sumnui \hr{\frac1{z-a_n} + P_{a_n}(z)} =:
H(z) + K(z), \mbox{ где } P_{a_n}(z) - \mbox {поправочные многочлены}.}

Разложим дробь $\frac{1}{z-a}$
в геометрическую прогрессию:
\eqn{\frac{1}{z-a} = -\frac{1}{a}\cdot \frac{1}{1-\frac {z}{a}}= -\frac{1}{a} \cdot \sumkzi \hr{\frac{z}{a}}^k}

Теперь будем знать вид поправочных многочленов для дроби:

\eqn{L(z) = H(z) + \sumnui \hr{\frac1{z-a_n} + \frac1{a_n} + \frac{z}{a_n^2}\spl \frac{z^{m_n-1}}{a_n^{m_n}}} =:
H(z) + K(z).}

Здесь мы написали при каждом слагаемом коэффициент $1$, потому что можно повторить каждое слагаемое
столько раз, какова кратность нуля $a_k$ (то есть в этой сумме допускаются повторения~$a_k$).

Рассмотрим функцию
\eqn{\wt K(z) = \intl{0}{z} K(\ze)\,d\ze.}
Она уже может быть неоднозначной. Выясним, насколько отличаются продолжения по разным путям.
Так как первообразная от $\frac1z$-- это $\Ln z$, мы будем получать приращения, кратные $2\pi i$.
Интегрируя, получаем функцию
\begin{multline}\wt K(z) = \sumnui\hr{\br{\ln(z-a_n) - \ln(0-a_n)} +
\frac{z}{a_n} + \frac{z^2}{2a_n^2}\spl \frac{z^{m_n}}{m_n a_n^{m_n}}} =\\=
\sumnui\hr{\ln\hr{1-\frac{z}{a_n}} + \frac{z}{a_n} + \frac{z^2}{2a_n^2}\spl \frac{z^{m_n}}{m_n a_n^{m_n}}}.
\end{multline}
Теперь рассмотрим функцию $f(z) := \exp\wt K(z)$. Экспонента заглушит неоднозначность в силу
своей $2\pi i$-- периодичности. Тогда в окрестности каждой точки $a_k$ получаем разложение вида
\eqn{f(z) = (z-a_k)^{p_k}\exp \wh K(z),}
где $\wh K$ представляет собой слагаемые по всем полюсам, кроме $a_k$. Здесь множитель
$z-a_k$ имеет кратность $p_k$, поскольку слагаемые с полюсом $a_k$ встречаются ровно
$p_k$~раз в показателе $\exp$.

Тем самым мы построили общую формулу для целой функции, имеющей заданные нули $a_k$
требуемой кратности $p_k$. При этом предполагалось, что $f(0) \neq 0$, но это препятствие
легко обойти, добавив множитель $z^{m_0}$. Итак, произвольная целая функция имеет вид
\eqn{f(z) = z^{m_0} \prodl{n=1}{\bes}\hs{\hr{1 - \frac{z}{a_n}}
\exp\hr{\frac{z}{a_n} + \frac{z^2}{2a_n^2}\spl \frac{z^{m_n}}{m_n a_n^{m_n}}}}\exp h(z),}
где $h$-- некоторая целая функция. Таким образом, доказана теорема:
\begin{theorem}[Вейерштрасса]
Для любого набора точек $a_k$ (удовлетворяющего теореме Миттаг-- Леффлера) существует
целая функция с нулями в этих точках наперёд заданной кратности.
\end{theorem}

\begin{imp}
Всякая мероморфная функция~$F$ есть отношение двух целых функций.
\end{imp}
\begin{proof}
Домножим функцию $F$ на такую целую функцию $Q$, чтобы она убила все полюса $F$
(она существует по только что доказанной теореме).
Тогда $P := F\cdot Q$ будет некоторой целой функцией. Значит, $F = \frac P Q$.
\end{proof}

\begin{ex}
Рассмотрим функцию $f(z) = \sin \pi z$. Её логарифмическая производная равна
\eqn{(\ln\sin\pi z)' = \frac{(\sin\pi z)'}{\sin \pi z} = \pi\ctg\pi z.}
Имеем
\eqn{\ctg \pi z = \frac1{\pi z} + \sumstr \hr{\frac1{\pi z-\pi n} + \frac1{\pi n}} =
\frac1\pi\hs{\frac1z + \sumstr \hr{\frac1{z-n} + \frac1n}}.}
Применяя формулу, получаем разложение
\eqn{\sin\pi z = \pi z\prodsstr{n} \hr{1-\frac z n}\exp\hr{\frac z n}.}
Группируя множители с симметричными индексами, получаем более удобную формулу:
\eqn{\sin \pi z = \pi z \prodl{n=1}{\bes}\hr{1 -\frac{z^2}{n^2}}.}
\end{ex}

\section{Эллиптические функции}

\subsection{Двоякопериодические функции и их свойства}

\subsubsection{Периодические функции в $\Cbb$}

Будем рассматривать функции, обладающие свойством периодичности, то есть такие,
что некотором ненулевом значении $T \in \Cbb$ имеем $f(z + T) = f(z)$ при всех $z \in \Cbb$.

Несложно проверить, что совокупность всех периодов функции-- это аддитивная подгруппа в~$\Cbb$:
сумма периодов-- период, и если $T$-- период, то $(-T)$ тоже является периодом.

Далее мы будем рассматривать \textbf{только мероморфные функции}.

Напомним, что подгруппа $G \subs \R^n$ называется дискретной, если существует такое число $R>0$,
что для любого $g \in G$ окрестность $U_R(g)$ не содержит других элементов группы, отличных от $g$.

\begin{lemma}
Группа периодов непостоянной мероморфной функции $f$ дискретна.
\end{lemma}
\begin{proof}
Допустим противное, тогда множество периодов имеет в конечной части плоскости предельную точку.
Следовательно, функция принимает одно и то же значение на некоторой сходящейся к этой предельной
точке последовательности. По теореме единственности она обязана быть постоянной.
\end{proof}

\begin{imp}
Группа периодов непостоянной мероморфной функции изоморфна либо $\Z^2$, либо $\Z$.
\end{imp}
\begin{proof}
Из алгебры\footnote{По этому поводу см., например,
\cite[гл.\,9,~\S\,1,~теорема\,4]{vinberg}.-- \emph{Прим. наб.}} известно, что всякая дискретная
подгруппа в $\R^n$ изоморфна решётке $\Z^k$, где $k \le n$.
Случай $k = 0$ невозможен, потому что в этом случае функция не является периодической.
Кроме того, имеем $k \le 2$. Значит, функция либо имеет два линейно-- независимых периода,
либо один период.
\end{proof}

Далее будем рассматривать функции с двумя независимыми периодами $\om_1$ и $\om_2$, а
через $G$ будем обозначать группу периодов $\om_1\Z \oplus \om_2\Z \subs \Cbb$.
Будем говорить, что $z_1 \equiv z_2 \pmod G$, если точка $z_2$ получается из $z_1$ сдвигом на
целое кратное какого-- нибудь периода (или их суммы). Иначе говоря, точки $z_1$ и $z_2$-- это один
и тот же элемент факторгруппы $\Cbb/G$.

\begin{stm}
Множество мероморфных функций с заданно группой периодов образует поле, выдерживающее
дифференцирование.
\end{stm}
\begin{proof}
Очевидно.
\end{proof}

\begin{df}
Двоякопериодические мероморфные функции называются \emph{эллиптическими}.
\end{df}

\begin{df}
Множество $\hc{t_1\om_1 + t_2\om_2 \vl t_i \in [0,1)}$ называется \emph{параллелограммом периодов}.
Мы будем обозначать его буквой $\Pi$.
\end{df}



\subsubsection{Свойства эллиптических функций}

\begin{theorem}
Любая целая эллиптическая функция постоянна.
\end{theorem}
\begin{proof}
Ясно, что целая функция ограничена на своём параллелограмме периодов, поэтому она ограничена
во всей плоскости. По теореме она обязана быть постоянной.
\end{proof}

\begin{imp}
Непостоянная эллиптическая функция имеет в параллелограмме периодов хотя бы один полюс.
\end{imp}

\begin{df}
\emph{Порядком} функции называется количество полюсов на параллелограмме периодов с учётом кратности.
\end{df}

\begin{theorem}
Если на $\pd \Pi$ эллиптическая функция не имеет полюсов, то
\eqn{\frac{1}{2\pi i}\ints{\pd\Pi}f(\ze)\,d\ze = 0.}
\end{theorem}
\begin{proof}
В силу периодичности функции, интегралы по противоположным сторонам параллелограмма
будут лишь различаться знаком (поскольку направления интегрирования на них противоположны).
Значит, в итоге получится нуль.
\end{proof}

\begin{imp}
Если $\pd \Pi$ не содержит полюсов, то сумма всех вычетов внутри $\Pi$ равна нулю.
\end{imp}
\begin{proof}
Вытекает из предыдущей теоремы и теоремы Коши о вычетах.
\end{proof}

\begin{imp}
Эллиптических функций порядка $1$ не существует.
\end{imp}
\begin{proof}
Если полюс в параллелограмме периодов всего один, то можно так подвигать этот параллелограмм,
что этот полюс попадёт в $\Int \Pi$. По предыдущему следствию, вычет в этом полюсе обязан быть нулевым.
Но вычет-- это лорановский коэффициент $c_{-1}$. Если он равен нулю, то это значит,
что в этой точке либо вообще нет полюса, либо его кратность не меньше двух.
\end{proof}

\begin{note}
В наших рассуждениях часто приходится оговаривать, что на границе $\Pi$ полюсов нет.
Ясно, что этого всегда можно добиться, подвигав параллелограмм периодов.
Поэтому больше не будем заострять на этом внимание.
\end{note}

\begin{theorem}
Всякое уравнение $f(z) = c$ имеет на параллелограмме периодов ровно столько же решений,
каков порядок функции $f$.
\end{theorem}
\begin{proof}
По принципу аргумента имеем
\eqn{\frac{1}{2\pi i}\ints{\pd \Pi}\frac{\br{f(\ze) - c}'}{f(\ze) - c}\,d\ze = N - P, }
где $N$ и $P$-- количества нулей и полюсов функции $f(z) - c$ соответственно.
Заметим, что под интегралом стоит эллиптическая функция того же периода, что и функция $f$. Поэтому
этот интеграл равен нулю. Следовательно, $N = P$. Но полюса функции
$f(z) - c$ и функции $f(z)$, очевидно, совпадают, а количество полюсов функции $f$ в параллелограмме
$\Pi$ и есть $P$. Это и требовалось доказать.

Это рассуждение проходит для всех конечных значений $c$, а если $c = \bes$, то утверждение
теоремы вытекает сразу из определения порядка функции $f$.
\end{proof}

\begin{df}
Решения уравнения $f(z) = c$ иногда называют $c$-- точками.
\end{df}

\begin{note}
Пусть $\ph$-- произвольная функция на параллелограмме $\Pi$ с вершинами $A,B,C,D$. Заметим,
что когда точка $z$ пробегает одну сторону параллелограмма, точка $z +\om_1$ пробегает
противоположную сторону параллелограмма в обратном направлении (относительно направления
интегрирования по контуру). Поэтому имеет место формула
\eqn{\ints{\pd\Pi} \ph(\ze)\,d\ze = \ints{A B}\bs{\ph(\ze+\om_1) - \ph(\ze)}\,d\ze -
\ints{A D}\bs{\ph(\ze+\om_2) - \ph(\ze)}\,d\ze.}
\end{note}

\begin{theorem}
Пусть $b_1\sco b_n$-- нули функции $f$, а $a_1\sco a_n$-- её полюса (дублированные
с учётом кратности) внутри параллелограмма периодов. Тогда
\eqn{\sum a_k \equiv \sum b_k \pmod G.}
\end{theorem}
\begin{proof}
Рассмотрим функцию
\eqn{\ph(z) = z\frac{f'(z)}{f(z)}.}
Можно считать, что $f(0) \neq 0$, потому что можно всегда сделать сдвиг аргумента, но в силу периодичности
сути дела это не поменяет. Если это так, то функция $\ph$ имеет полюса первого порядка ровно в тех
точках, где у $f$ были либо нули, либо полюса. Поскольку вычет логарифмической производной есть кратность
нуля или полюса, получаем, что
\eqn{\res_{a_k} \ph = a_k \cdot \res_{a_k} \hr{\frac{f'}{f}}, \quad
\res_{b_k} \ph = b_k \cdot \res_{b_k} \hr{\frac{f'}{f}}.}
Отсюда следует, что
\eqn{\frac1{2\pi i}\ints{\pd \Pi}\ph(\ze)\,d\ze = \sum \res \ph = \sum b_n  - \sum a_n.}

Пусть $\om$-- период. Тогда
\eqn{(z + \om)\frac{f'(z+\om)}{f(z+\om)} = z\frac{f'(z)}{f(z)} + \om\frac{f'(z)}{f(z)} \quad\Ra\quad
\om\frac{f'(z)}{f(z)} = \ph(z+\om) - \ph(z).}
Теперь проинтегрируем это тождество по одной и по другой стороне параллелограмма, подставляя в него
$\om = \om_1$ и $\om = \om_2$:
\eqn{\begin{aligned}
\frac{\om_2}{2\pi i} \intl{z_0}{z_0 +\om_1}\frac{f'(\ze)}{f(\ze)}\,d\ze =
\frac1{2\pi i} \intl{z_0}{z_0+\om_1}\bs{\ph(\ze+\om_2) - \ph(\ze)}\,d\ze.\\
\frac{\om_1}{2\pi i} \intl{z_0}{z_0 +\om_2}\frac{f'(\ze)}{f(\ze)}\,d\ze =
\frac1{2\pi i} \intl{z_0}{z_0+\om_2}\bs{\ph(\ze+\om_1) - \ph(\ze)}\,d\ze.
\end{aligned}}
Вычитая из второго равенства первое и используя замечание перед теоремой, получаем
\eqn{
\frac{\om_1}{2\pi i} \intl{z_0}{z_0 +\om_2}\frac{f'(\ze)}{f(\ze)}\,d\ze-
\frac{\om_2}{2\pi i} \intl{z_0}{z_0 +\om_1}\frac{f'(\ze)}{f(\ze)}\,d\ze=
\frac1{2\pi i}\ints{\pd \Pi}\ph(\ze)\,d\ze.}

По формуле Ньютона-- Лейбница имеем
\eqn{\frac1{2\pi i} \intl{z_0}{z_0 +\om}\frac{f'(\ze)}{f(\ze)}\,d\ze = \frac1{2\pi i}\Var\ln f(z).}
В силу периодичности, когда точка $\ze$ пробегает по отрезку $[z_0,z_0+\om]$, точка $f(\ze)$ пробегает
замкнутую петлю. Как известно, приращение аргумента-- это количество оборотов петли, умноженное на $2\pi$.
После сокращения на $2\pi$ получим целое число. Значит, для некоторых $n_1,n_2\in\Z$ имеем
\eqn{\frac1{2\pi i}\ints{\pd \Pi}\ph(\ze)\,d\ze = n_1\om_1 + n_2\om_2 \equiv 0 \pmod G.}
Осталось вспомнить, что значение этого интеграла есть разность $\sum b_k - \sum a_k$.
\end{proof}

\subsection{Функция Вейерштрасса}

Мы уже знаем, что эллиптических функций порядка $r < 2$ не существует.
Сейчас мы предъявим функцию порядка $2$. Это будет так называемая функция Вейерштрасса.

Здесь через $G$, как обычно, будем обозначать группу периодов.

\subsubsection{Построение $\wp$-- функции Вейерштрасса}

Рассмотрим функцию
\eqn{\ze(z) :=\frac1z + \sumsstr{\om\in G} \hr{\frac{1}{z-\om} + \frac{1}{\om} + \frac{z}{\om^2}}.}
Покажем, что это определение корректно, то есть докажем, что этот ряд сходится как ряд мероморфных функций.
В качестве системы раздувающихся компактов будем брать параллелограммы $\Pi_n$ с вершинами в точках
$\hc{n(\pm \om_1 \pm \om_2)}$. Оценим каждое слагаемое исходного ряда по модулю:
\eqn{\hm{\frac1{z-\om}+\frac1\om+\frac z{\om^2}} = \hm{\frac{z^2}{\om^2(z-\om)}}\le
\frac{|z|^2}{|\om|^2\br{|\om| - |z|}}.}
Мы группируем слагаемые, лежащие на границе параллелограммов $\Pi_n$. Для $n$-- й группы справедлива
следующая оценка:
\eqn{S_n = \sums{\om \in \pd\Pi_n} \frac{|z|^2}{|\om|^2\br{|\om| - |z|}} \le
\sums{\om \in \pd\Pi_n} \frac{|z|^2}{C^2\cdot n^2\br{C\cdot n - |z|}} \le
 \frac{8n\cdot|z|^2}{C^2\cdot n^2\br{C\cdot n - |z|}} \sim \frac{M}{n^2}.}
Действительно, на границе $\Pi_n$ лежит не более $8n$ периодов, а каждый период удалён от нуля
не меньше, чем на $C\cdot n$. Поэтому при каждом фиксированном $z$ приведённая оценка верна,
и она гарантирует сходимость ряда $\sum S_n$, поскольку $\sum\frac1{n^2}$ сходится.

Итак, корректность проверена. Теперь рассмотрим функцию
\eqn{\wp(z) = -\ze'(z) = \frac1{z^2} + \sumsstr{\om \in G}\hr{\frac1{(z-\om)^2} - \frac1{\om^2}}.}

Отметим, что
\eqn{\wp'(z) = -2\sums{\om \in G}\frac1{(z-\om)^3}.}

Производная $\wp(z)$-- двоякопериодическая функция просто в~силу того, что она представляет
из себя честную сумму по всей решётке периодов. Сдвинув $z$ на~$w$, мы не поменяем сумму.
Это значит, что сама $\wp(z)$ будет отличаться в~соседних периодических точках на~одну и~ту же константу,
вне зависимости от~того, какие соседние точки мы брали:
$\forall z \ \wp(z+\om_j) - \wp(z) = c_j$.
Заметим, что $\wp$ является чётной, то есть $\wp(-z) = \wp(z)$.
Поэтому подставив в~предыдущее равенство $z = -\frac{\om_j}{2}$, получим
$\wp(\frac{\om_j}{2}) - \wp(-\frac{\om_j}{2}) = c_j = 0$, $j=1,2$, что значит, что $\wp$~двоякопериодична.



\begin{df}
Построенная функция $\wp$ называется \emph{функцией Вейерштрасса}.
\end{df}

Непосредственно из определения видно, что $\wp$ является функцией порядка $2$, так как элементы
группы периодов-- это в точности её полюса кратности $2$.

Рассмотрим уравнение $\wp(z) = c$. Из свойств эллиптических функций второго порядка
следует, что имеется в каждом параллелограмме периодов расположено ровно две $c$-- точки
$z_1$ и~$z_2$. Кроме того, легко видеть, что $z_1 + z_2 \equiv 0\pmod G$. Действительно,
сумма нулей сравнима с суммой полюсов, а так как полюса находятся в узлах решётки, их сумма
сравнима с нулём. Но между нулями и $c$-- точками никакой разницы нет, потому что можно
рассмотреть функцию $\wp(z) - c$, обладающую теми же полюсами.

\begin{stm}
Равенство $\wp(z) = \wp(w)$ выполняется тогда и только тогда, когда $z \equiv w \pmod G$
или $z \equiv -w \pmod G$.
\end{stm}
\begin{proof}
Если это равенство выполнено, то $w$ и $z$-- это две $\wp(z)$-- точки. Значит, их
сумма сравнима с нулём. Обратно, если выполнено первое из сравнений, то равенство
$\wp(z) = \wp(w)$ верно в силу периодичности $\wp$. Аналогично, из второго сравнения
вытекает, что $\wp(z) = \wp(-w)$, но в силу чётности $\wp$ верно и доказываемое.
\end{proof}

Рассмотрим точки
\eqn{z_0 \equiv 0,\quad z_1 \equiv \frac12\om_1,\quad z_2 \equiv
\frac12\om_2,\quad z_3 \equiv \frac12(\om_1+\om_2).}
Имеем $\wp(z_0) = \infty$. Введём обозначения:
\eqn{e_i := \wp(z_i),\quad i = 1,2,3.}

\begin{stm}
Значения $e_i$ попарно различны.
\end{stm}
\begin{proof}
Непосредственно вытекает из предыдущего утверждения: $z_i$ не сравнимы между
собой (чтобы они стали сравнимыми, их нужно удвоить).

Другое доказательство: если, например, $e_1 = e_3$, то функция $\wp(z) - e_1 = \wp(z) - e_3$,
имеющая один двойной полюс в $\Pi$, имела бы там $4$ нуля (точнее, $2$ двукратных нуля
в точках $z_1$ и $z_2$). Это невозможно.
\end{proof}

Дифференцируя уравнение $\wp(z) = c$, получаем, что оно будет иметь двойные корни
тогда и только тогда, когда $c=e_i$ для некоторого $i$. Это означает, что
уравнение $\wp'(z) = 0$ имеет решения $\frac12\om_1$, $\frac12\om_2$, $\frac12(\om_1+\om_2)$.


\subsubsection{Дифференциальное уравнение для $\wp$}

Рассмотрим функцию $\wp$ в малой окрестности $U(0)$. Имеем
\eqn{\wp'(z) = -\frac2{z^3} + H(z),}
где $H$-- голоморфная функция в $U$.
Функция $\wp'$ имеет порядок $r =3$, а потому имеет $3$ нуля в параллелограмме периодов.
Но мы уже знаем три её нуля-- это точки $\frac12\om_1$, $\frac12\om_2$, $\frac12(\om_1+\om_2)$.
Далее, функция
\eqn{f(z) = \br{\wp(z) - e_1}\br{\wp(z) - e_2}\br{\wp(z) - e_3}}
имеет в нуле полюс $6$-- го порядка (потому что $\wp$ имеет в нуле двойной полюс).

Следовательно, эллиптическая функция
\eqn{Q(z) = \frac{\br{\wp'(z)}^2}{f(z)}}
не имеет нулей и полюсов в $\Pi$, а потому является константой.
Найдём её: в окрестности нуля имеем
\eqn{\br{\wp'(z)}^2 = \frac{4}{z^6}+\dots, \quad f(z) = \frac1{z^6}+\dots,}
откуда следует, что $Q \equiv 4$. Таким образом, функция Вейерштрасса удовлетворяет
дифференциальному уравнению
\eqn{\label{WPDiffEquationOne}\br{\wp'(z)}^2 = 4\br{\wp(z) - e_1}\br{\wp(z) - e_2}\br{\wp(z) - e_3}.}

\medskip

Теперь выведем это уравнение другим способом.
Напишем разложение функции $\wp(z)$ в окрестности нуля. Разложим дробь $\frac{1}{z-\om}$
в геометрическую прогрессию:
\eqn{\frac{1}{z-\om} = -\frac{1}{\om}\cdot \frac{1}{1-\frac z\om}= -\frac1\om \cdot \sumkzi \hr{\frac z\om}^k}
и подставим полученный ряд в выражение для функции $\ze(z)$:
\eqn{\begin{aligned}
\ze(z) &= \frac{1}{z} + \sumsstr{\om\in G}\hr{\frac{1}{z-\om}+\frac{1}{\om}+\frac{z}{\om^2}} =
\frac1z -\sumsstr{\om\in G}\hr{\frac{z^2}{\om^3} + \frac{z^3}{\om^4} + \ldots} =
\frac1z - \suml{n=2}{\bes}\frac{c_n}{2n-1}z^{2n-1},\\
\quad c_n &= (2n-1)\sumsstr{\om\in G}\frac{1}{\om^{2n}}.
\end{aligned}}
Очевидно, коэффициенты $c_n$ с нечётными номерами равны нулю, так как, с одной стороны,
при замене $\om$ на $(-\om)$ сумма не меняется, поскольку все слагаемые те же,
а с другой стороны, должна поменять знак.

Продифференцируем полученное выражение:
\eqn{\wp(z) = \frac{1}{z^2} + \suml{n=2}{\bes}c_n z^{2n-2}, \quad
c_n = (2n-1)\sumsstr{\om\in G}\frac{1}{\om^{2n}}.}
Теперь напишем ещё несколько уравнений:
\eqn{
\begin{aligned}
\wp'(z) &= -\frac{2}{z^3} + 2c_2 z + 4c_3z^3 + \ldots,\\
\br{\wp'(z)}^2 &= \frac{4}{z^6}-\frac{8c_2}{z^2}-16c_3 + \ldots,\\
\wp^3(z) &= \frac{1}{z^6} + \frac{3c_2}{z^2} + 3c_3 + \ldots.
\end{aligned}}
Отсюда
\eqn{\br{\wp'(z)}^2 - 4\wp^3(z) = -\frac{20c_2}{z^2}-28c_3+\ldots.}
В этих уравнениях выписаны все члены, которые не стремятся к нулю при $z\ra 0$. Получаем уравнение:
\eqn{\br{\wp'(z)}^2 - 4\wp^3(z) + 20c_2\wp(z) = -28c_3 + \dots.}
В правой части стоит некоторая эллиптическая функция, не имеющая нулей и полюсов
в параллелограмме периодов при $z\neq 0$. Но и точка $z=0$ тоже не является полюсом.
Стало быть, эта функция постоянна (а значит, равна числу $-28c_3$).

Введём обозначения:
\eqn{g_2(z) := 20 c_2 = 60 \sumsstr{\om\in G}\frac{1}{\om^4}, \quad
g_3(z) := 28 c_3 = 140 \sumsstr{\om\in G}\frac{1}{\om^6}.}
В этих обозначениях получаем дифференциальное уравнение
\eqn{\label{WPDiffEquationTwo}\br{\wp'(z)}^2 = 4 \wp^3 - g_2\wp(z)-g_3.}
Теперь подставим полученное выражение для квадрата производной в первое дифференциальное
уравнение~\eqref{WPDiffEquationOne}. Получим, что при любом значении $x$ имеет место
равенство
\eqn{4x^3-g_2x-g_3  = 4(x-e_1)(x-e_2)(x-e_3).}
Значит, числа $e_i$ есть корни кубического уравнения $4x^3-g_2x-g_3=0$.

\begin{df}
Величину $\De := 16(e_1-e_2)^2(e_1-e_3)^2(e_2-e_3)^2$ называют \emph{дискриминантом}.
\end{df}

Из формул Виета получаем уравнения на $e_i$:
\eqn{\case{e_1+e_2+e_3=0,\\
e_1e_2+e_2e_3+e_3e_1=-\frac14 g_2,\\
e_1e_2e_3 = -\frac14g_3,\\
\De = g_2^3-27g_3^2.}}

Дискриминант многочлена, как известно, равен нулю тогда и только тогда, когда он имеет
кратные корни. Вывод: если периоды $\om_1$ и $\om_2$ таковы, что
$\De(g_2,g_3)= g_2^3-27g_3^2 \neq 0$, то можно найти функцию с такими периодами.

\subsubsection{Выражение эллиптических функций через функцию Вейерштрасса}

\begin{theorem}
Любую эллиптическую функцию с периодами $\om_1$ и $\om_2$ можно представить в виде
\eqn{f(z) = R\br{\wp(z)}+R_1\br{\wp(z)}\wp'(z),}
где $R$ и $R_1$-- рациональные функции.
\end{theorem}
\begin{proof}
Сначала докажем, что любую чётную эллиптическую функцию можно представить в таком виде.
Если точка $a$-- нуль (или полюс) функции $f$, то точка $(-a)$-- тоже нуль (или полюс)
этой функции. Если $a \equiv -a$, то кратность нуля (полюса) удваивается. Значит,
порядок функции $f$ есть чётное число. Пусть $b'_1\sco b'_{2n}$-- все нули функции~$f$
(с учётом кратностей), и $a'_1\sco a'_{2n}$-- все полюса, также с учётом кратностей.

Полюса $a'_1\sco a'_{2n}$ разбиваются на пары $\hc{a_{k_1},a_{k_2}}$, в каждой
из которых $a_{k_1} + a_{k_2}\equiv 0$. Возьмём от каждой пары по одному представителю,
получим набор $a_1\sco a_n$. Аналогичную процедуру проделаем с нулями.

Рассмотрим функцию
\eqn{Q(z) := \frac{\br{\wp(z)-\wp(b_1)}\sd \br{\wp(z)-\wp(b_n)}}
{\br{\wp(z)-\wp(a_1)}\sd \br{\wp(z)-\wp(a_n)}}.}
Её полюса есть точки $a_1\sco a_n$ и $(-a_1)\sco (-a_n)$, а нули-- точки
$b_1\sco b_n$ и $(-b_1)\sco (-b_n)$ ввиду чётности функции Вейерштрасса. Но с другой
стороны, наша функция тоже чётна, и имеет тот же набор нулей и полюсов с теми же
кратностями. Значит, функция $\frac{f(z)}{Q(z)}$ не имеет полюсов и нулей на параллелограмме
периодов, а стало быть, она постоянна. Таким образом, $f = R(\wp)$.

С нечётными функциями поступим так: поскольку функция $\wp'(z)$ нечётна (как
производная чётной функции), функция $\frac{f(z)}{\wp'(z)}$ уже будет чётной.
Значит, $f(z)$ представима в виде $R_1\br{\wp(z)}\wp'(z)$.

В общем случае произвольную функцию представим в виде суммы чётной и нечётной:
\eqn{f(z) = \frac12\hs{f(z)+f(-z)} + \frac12\hs{f(z)-f(-z)}.}
Значит, $f(z) = R\br{\wp(z)}+R_1\br{\wp(z)}\wp'(z)$.
\end{proof}

\subsubsection{Униформизация кубической кривой}

Пусть в пространстве $\Cbb^2$ была задана кривая $C := \hc{P_n(x,y)=0}\subs\Cbb^2$.
Вложим $\Cbb^2(x,y)$ в $\CP^2(x:y:z)$ (выше подробно объяснялось, как это
сделать) и рассмотрим однородный многочлен
\eqn{\wt P(x,y,z) := z^nP_n\hr{\frac x z, \frac y z}.}

\begin{df}
Кривая $\ol C := \hc{\wt P(x,y,z) =0}\subs \CP^2$ называется \emph{проективным
замыканием} кривой $C$.
\end{df}

Рассмотрим кривую
\eqn{C := \hc{y^2 = 4x^3-g_2 x -g_3}.}
Тогда её проективным замыканием будет кривая
\eqn{\ol C := \hc{zy^2 = 4x^3 - g_2 xz^2-g_3z^3}.}
Пусть $\De = g_2^3-27g_3^2 \neq 0$.  Покажем, что тогда $\ol C$ будет
комплексным многообразием.
Применим теорему о неявной функции. Для этого нужно показать,
что в каждой точке кривой либо $P'_x\neq 0$, либо $P'_y\neq 0$.
Имеем $P(x,y) = -y^2 + 4x^3-g_2x-g_3$, тогда $P'_x = 12x^2-g_2$, $P'_y = -2y$.
Обратимости не будет тогда, когда одновременно обнуляется и сам многочлен,
и его производная. Но у нас кратных корней нет, поэтому всё хорошо.

Рассмотрим функцию $y  = 2\sqrt{(x-e_1)(x-e_2)(x-e_3)}$. У неё $4$ особых
точки: $e_1$, $e_2$, $e_3$, $\bes$. Риманова поверхность этой функции гомеоморфна
сфере с одной ручкой (это следует из формулы Римана-- Гурвица: количество ручек
для римановой поверхности алгебраической функции $w^2 = P_n(z)$ есть $g = \hs{\frac{n-1}{2}}$, если
многочлен $P_n$ не имеет кратных корней-- а в нашем случае оно так и есть), то есть двумерному тору.
С другой стороны, многообразие $\Cbb/G$, где $G$-- группа периодов, также есть двумерный тор.

Кривую $C$ можно параметризовать, положив
\eqn{\case{x = \wp(z),\\
y = \wp'(z),}}
где $\wp(z)$-- функция, являющаяся решением уравнения~\eqref{WPDiffEquationTwo}.

Рассмотрим отображение
\eqn{\ph\cln \Cbb/G\ra \ol C, \text { определённое по правилу }  \ph(t) := \br{\wp(t):\wp'(t):1}.}
Заметим, что оно биективно: если функция Вейерштрасса принимает в каких-- то
точках $t_1$ и $t_2$ одинаковое значение, то $t_1 = -t_2$ (с точностью до периода),
но в этих точках значения производной различны (и если $\wp (t)=0$, то $\wp (t)\neq 0$).
Значит, образы точек $t_1$ и $t_2$ будут различными.

Говорят, что функция Вейерштрасса \emph{униформизует} данную кубическую кривую.


\section{Приложение}

\begin{problem}
Найти в бидиске $P$ последовательность $z_n\ra 0$ такую, что для всех голоморфных
в $P$ функций $f$ выполняется свойство $f = 0$ если и только если $f(z_n) = 0$ для всех $z_n$.
\end{problem}

\end{document}



%===========================================

\subsection{Степенные ряды многих переменных}


 --римеры
 \begin{itemize}
 \item[(1)] $f=\ln z=\ln |z|+i\arg z\quad\Vec{V}=\frac{1}{\overline{z}},\;
 B=0,\;\Pi=2\pi$.\par\noindent Перетекание жидкости из 0 в $\infty$, мощность
 источника равна $2\pi$.
%%*******Рысуночэк*************************************************************
 \begin{center}
 \textit{Рисуночек....}
 \end{center}
 \item[(2)] $f=i\ln z,\;B=2\pi,\;\Pi=0$\: --- вихрь в 0.
 \item[(3)] $f=\ln\frac{z+h}{z-h}$. Дробно--линейная замена
 $0\longrightarrow -h,\;\infty\longrightarrow h$.
 \item[(4)] Диполь (источник и сток в одной точке), $h\to 0$ так, что
 $\Pi\!\cdot\!2h\to m$
 \[
 \lim_{h\to 0}\frac{\Pi\!\cdot\!2h}{2\pi}\!\cdot\!
 \frac{\ln(z+h)-\ln(z-h)}{2h}\to\frac{m}{2\pi}\!\cdot\!\frac{1}{z}
 \]
 Аналогичная процедура --- слияние диполей --- даёт полюса более высокого
 порядка.
 \end{itemize}
 \section{Принцип Дирихле.}
%%*****************************************************************************
%%**********РиСУНОК***********************************************************
%%*****************************************************************************
 \begin{center}
 \textit{Рисунок.... Он обязательно здесь будет!}
 \end{center}
 Потенциал --- однозначная функция $w=f(z)$, она осуществляет конформное
 отображение \D на плоскость с горизонтальными разрезами.
 \section{Принцип компактности.}
 --римеры
 $\R$ \:(замкн. и огр.), $\mathcal{C}[0,1]\quad f_{n}=\sin nx$
 \begin{df}
 Семейство функций на $E\;\{\,f_{\alpha}\,\}$ называется рав\-но\-сте\-пен\-но
 непрерывным на $E$, если
 \[
 \forall\,\varepsilon>0\:\exists\,\delta>0:\,|z_{1}-z_{2}|<\delta,\:
 z_{1},z_{2}\in E\:\Longrightarrow\:\forall\,\alpha\:
 |f_{\alpha}(z_{1})-f_{\alpha}(z_{2})|<\varepsilon.
 \]
 \end{df}
 \begin{df}
 Семейство функций $\{\,f_{\alpha}\,\}$ называется равнерно ограниченным
 в области \D, если для любого компактного $K\subset\Dm\;\exists\,M(K)$ такое,
 что $\forall\,\alpha\;|f_{\alpha}(z)|\leqslant M(K)$ на $K$.
 \end{df}
 \begin{lemma}
 Пусть $f_{\alpha}$ --- семейство голоморфных в \D функций, равномерно
 ограниченное в \D, тогда $f_{\alpha}$ --- равностепено непрерывно.
 \end{lemma}
 --во
%%*******Рысуночэк*************************************************************
 \begin{center}
 \textit{Рисуночек:"Ждите меня здесь......"}
 \end{center}
 Пусть $\rho>0$ --- расстояние от компакта $K$ до $\pd\Dm$ и $r<\rho;\;
 K_{r}$ --- r--раздутие $K$ (\,то есть $\bigcup_{z\in K}\varDelta(z,r)$\,).
 \begin{alignat*}{1}
 &|f'_{\alpha}|=\left|\frac{1}{2\pi}\int_{|\ze-z|=r}
 \frac{f(\ze)d\ze}{(\ze-z)^{2}}\right|\leqslant\frac{M(K_{r})}{r}\\
 &\Longrightarrow\:|f_{\alpha}(z_{1})-f_{\alpha}(z_{2})|=
 \left|\int_{z_{1}}^{z_{2}}\negthickspace f'_{\alpha}(z)dz\right|
 \leqslant\frac{M(K_{r})}{r}\!\cdot\!\ell (z_{1},z_{2})
 \end{alignat*}
 \begin{theorem}[Монтель]
 Если $f_{\alpha}$ --- бесконечное семейство функций, голоморфных в \D, то из
 него можно выбрать последовательность, сходящуюся в \D (равномерно на
 компактах).
 \end{theorem}
 --во
 Выберем счётное всюду плотное в \D множество $\{\,a_{n},\:n=1,2,\dots\,\}$.
 Выберем последовательность функций семейства $\{\,f_{1n}\,\}$ такую, что
 $\{\,f_{1m}(a_{1})\,\}$ сходится (значения в точке --- ограничены). Из этой
 последовательности выберем подпоследовательность $\{\,f_{2n}\,\}$, сходящуюся
 в $a_{2}$, и так далее. Тогда диагональная последовательность $f_{mm}$
 сходится во всех точках $a_{n}$. Рассмотрим компакт
 $K\subset\Dm.\;\forall\varepsilon>0$ найдём $\delta>0$ такое, что из
 $|z-\overline{z}|<\delta$ следует
 $f_{\alpha}(z)-f_{\alpha}(\overline{z})<\frac{\varepsilon}{3}$, и выберем из
 покрытия компакта $K\supset\bigcup_{z\in K}\!\varDelta(z,\frac{\delta}{2})$
 конечное подпокрытие $\{\,\varDelta_{1},\dots,\varDelta_{p}\,\}$, в каждом
 $\varDelta_{j}$  выберем $b_{j}\in\{a_{n}\}$. Возьмём $N$ так, что
 \[
 \forall\,m,n>N\text{ и }\forall\,b_{j},\:j=1,\dots,p\quad
 |f_{mm}(b_{j})-f_{nn}(b_{j})|<\frac{\varepsilon}{3},
 \]
 тогда $\forall z\in K$ имеем
 \begin{alignat*}{1}
 &|f_{mm}(z)-f_{nn}(z)|\leqslant |f_{mm}(z)-f_{mm}(b_{j})|+
 |f_{mm}(b_{j})-f_{nn}(b_{j})|+\\&+|f_{nn}(b_{j})-f_{nn}(z)|\leqslant
 \frac{\varepsilon}{3}+\frac{\varepsilon}{3}+\frac{\varepsilon}{3}=\varepsilon
 \end{alignat*}

 \par\bigskip\textbf{Лекция++}\par
 \section*{Теорема Римана.}
 \emph{Напоминание:}
 $\varDelta\,,\:\Cm$ и \oC попарно неэквивалентны.
 \begin{alignat*}{1}
 &\Aut\oCm=\left\{\,z\longmapsto\frac{az+b}{cz+d}\,,\;
 \begin{vmatrix}
 \,a & b \,\\
 \,c & d \,
 \end{vmatrix}
 \ne 0\,\right\}  \quad \dim_{R}=6 \\
 &\Aut\Cm=\{\,z\longmapsto az+b\,,\; a\ne 0\,\}\quad \dim_{R}=4 \\
 &\Aut\varDelta=\left\{\,z\longmapsto e^{i\ph}\frac{z-a}{1-\overline{a}z}
 \,,\;|a|<1\,,\:\ph\in\Rm\,\right\}  \quad \dim_{R}=3
 \end{alignat*}
 Пусть $\Dm\subset\oCm$ --- область.
 \begin{itemize}
 \item[(1)] Если $\pd\Dm=\varnothing\:\Longrightarrow\:\Dm=\oCm$
 \item[(2)] Если $\pd\Dm=\{\,\text{1 точка}\,\}\:\Longrightarrow\:
                                      \Dm\overset{\text{др.--лин.}}{\sim}\oCm$
 \item[(3)]  \emph{Теорема Римана:} Если \D --- односвязная область, такая, что
 $\pd\Dm=\{\,>1\text{ точки}\,\}$, то
 $\Dm\overset{\text{конф.}}{\sim}\varDelta$
 \end{itemize}
 \emph{Напоминание:}
 \begin{itemize}
 \item[(1)] при этом $\Aut\,\Dm\simeq\Aut\varDelta$.
 Если $f:\,\Dm\longrightarrow\varDelta$, то изоморфизм строится так
%*****************************************************************************
%%*******Рысуночэк*************************************************************
%*****************************************************************************
 \begin{center}
 \textit{Это Вы узнаете из} рисуночка\\
 \textit{(когда он здесь появится)}
 \end{center}
 $f:\,\Dm\longrightarrow\varDelta\,,\:\ph\in\Aut\varDelta\,,\;
 F:\,\ph\longmapsto f^{-1}\negmedspace\circ\ph\circ f$
 \item[(2)] В частности $\forall\,A\in\Dm\,,\:\forall\,b\in\Rm,\:
 \forall\,a\in\varDelta$ конформное отображение
 $f:\,\Dm\longrightarrow\varDelta$ такое, что $f(A)=a\,,\:f'(a)=b$, единственно.
 \end{itemize}
 \section*{Доказательство теоремы Римана.}
 \begin{description}
 \item[Шаг 1]  $a\ne b\in\pd\Dm$, тогда каждая из двух ветвей
 $\sqrt{\frac{z-a}{z-b}}$ отображает \D на непересекающиеся области
 $\:\Longrightarrow\:$ каждая ветвь отображает \D на область, лежащую вне
 некоторого круга $\:\Longrightarrow\:$ каждая ветвь $\sim$ области внутри
 $\varDelta$.\par\noindent
 Теперь считаем, что $\Dm\subset\varDelta\,,\:0\in\Dm\,,\:c\in\Dm\,,\:
 c\ne 0$.
 \item[Шаг 2] Рассмотрим $S=\{\,\,f\in\ODm:\:f\text{ --- однолистна, }
 |f|<1\,,\:f(0)=0\,,\:f'(0)>0\,\}.\quad S\ne\varnothing$, так как $z\in S$.\par
 Функционал $\mathcal{J}(f)=|f(c)|$ имеет на $S\;\sup|f|=M$. Пусть  на
 последовательности $f_{n}$ этот $\sup$ реализуется, тогда, в силу равномерной
 ограниченности, можно выбрать подпоследовательность, сходящуюся к
 $f_{0}\in\ODm\,.\;|f_{0}|\leqslant 1\:\Longrightarrow\:|f_{0}|<1\,,\:f(0)=0$
 и по теореме Гурвица $f'_{0}(0)>0\;(\,\text{так как }f_{0}\ne\const\,)
 \:\Longrightarrow\:f_{0}\in S$.
 \item[Шаг 3] Пусть $f_{0}:\,\Dm\Longrightarrow\varDelta$ --- не "на", то есть
 $\exists\,d\in\varDelta\setminus f(\Dm)\,,\:d\ne 0$.
 \end{description}
%*****************************************************************************
%%*******Рысуночэк*************************************************************
%*****************************************************************************
 \begin{center}
 \textit{Здесь ожидается появление рисунка...}
 \end{center}
 У $\mathcal{J}(\ph(f(z)))$ выделим ветвь (позаботимся, чтобы для неё
 $\ph(f(0))=0,\linebreak%@@@@@@@@@@@@@@@@@@@@@@@@@@@@@@@@@@@@@@@@@@@@@@@@@@@@
 (\ph(f(z)))>0\:\in S$). Обратная $\ph^{-1}$ ---
 голоморфная в $\varDelta,\;0\longmapsto 0$, не линейная. По лемме Шварца
 \begin{alignat*}{1}
 &|\ph^{-1}(\tau)|<|\tau|\:\Longrightarrow\:\tau\ne 0\\
 &|\ph(t)|>|t|,\;t\ne 0\:\Longrightarrow\:
 |\ph(f(c))|>|f(c)|=M.
 \end{alignat*}
 Противоречие.--тд
 Теорема верна и для областей на комплексном многообразии $\dim=1$
 (риманова поверхность).\par
 В частности, все односвязные римановы поверхности $\sim\oCm$ (элл.),
 \C (параб.), $\varDelta$ (гип.).

 \section{Алгебраические функции.}
 \begin{df}
 Изолированная особая точка называется алгебраической, если это точка
 ветвления конечного порядка (возможно, однозначного характера), причём
 разложение в ряд Пюизо имеет лишь конечное число отрицательных слагаемых.
 \end{df}
 $\left(\frac{1}{\sqrt{z}}\text{ --- да},\;e^{\sqrt{z}}\text{ --- нет} \right)$
 \begin{df}
 ПАФ, имеющая в \C лишь конечное число алгебраических точек и конечное число
 значений, называется алгебраической.
 \end{df}
 \begin{lemma}
 Количество значений $m$ в каждой неособой точке одинаково.
 \end{lemma}
 То есть риманова поверхность --- $m$--листное накрытие над
 $\Cm\setminus\{\,a_{1},\dots,a_{n}\,\}$.
 \begin{lemma}
 Если $m$=1, то функция рациональна.
 \end{lemma}
 \begin{theorem}
 Все элементы алгебраической функции $w=f(z)$ удовлетворяют неприводимому
 полиномиальному соотношению $P(z,w)\equiv 0$.
 \end{theorem}
 --во
 Пусть $f_{1}(z),\dots,f_{m}(z)$ --- полный набор элементов в неособой точке.
 Рассмотрим все элементарные симметрические функции от\linebreak%@@@@@@@@@@@@@@@@@@@
 $f_{1}(z),\dots,f_{m}(z)$, то есть $\sigma_{1}=f_{1}+\dots+f_{m},\,
 \sigma_{2}=\sum f_{i}\!\cdot\! f_{j},\,\dots,\,\sigma_{m}=f_{1}\negmedspace
 \cdot\ldots\cdot\! f_{m}$. Все эти функции рациональны. Любой элемент $w=f(z)$
 удовлетворяет соотношению
 \[
 (w-f_{1})\cdot\dots\cdot(w-f_{m})=w^{m}-\sigma_{1}(z)w^{m-1}+\dots+
 (-1)^{m}\sigma_{m}(z)
 \]
 Домножая на общий знаменатель $\sigma_{1}(z),\dots,\sigma_{m}(z)$, получаем
 $P(z,w)\equiv 0$.\par Пусть $P$ приводим, то есть $P=P_{1}\!\cdot\!P_{2}$,
 тогда для произвольного элемента $f(z)$ имеем
 $P_{1}(z,f(z))\!\cdot\!P_{2}(z,f(z))\equiv 0$ в окрестности, следовательно,
 один из сомножителей $\equiv 0$. (\,Пусть это будет $P_{1}(z,f(z))$\,). Это
 соотношение выполняется при всех продолжениях
 $\:\Longrightarrow\:f_{1}(z),\dots,f_{m}(z)$ удовлетв. Вне дискретного множества
 $f_{i}\ne f_{j}\:\Longrightarrow\:$deg$P_{1}\geqslant m\:\Longrightarrow\:$
 deg$P_{1}=m,\;$deg$P_{2}=0\:\Longrightarrow\:F_{2}=F_{2}(z)$, но из условия
 отсутствия общего множителя следует, что $F_{2}\equiv\const$.--тд

 \par\bigskip\textbf{Лекция++}\par
 Из курса алгебры (Курош \S 54):
 \[
 f(x)=a_{n}x^{n}+\dots+a_{0},\;g(x)=b_{n}x^{n}+\dots+b_{0}\quad
 \{\,\alpha\,\}\text{ и }\{\,\beta\,\}\text{ --- корни.}
 \]
 Результант  --- определитель порядка $(m+n)$
 \begin{alignat*}{1}
 &R(f,g)=a_{n}^{m}b_{n}^{m}\prod(\alpha_{i}-\beta_{j})=
 \begin{vmatrix}
 \,a_{n} & \hdotsfor{2} & a_{0} & & & & \\
 & a_{n} & \hdotsfor{2} & a_{0} & & &  \\
 & & \hdotsfor{4} &  & \\
 & & & a_{n} & \hdotsfor{2} & a_{0} &  \\
 \,b_{m} & \hdotsfor{3} & b_{0} & & &  \\
 & & \hdotsfor{5} &   \\
 & & & b_{m} & \hdotsfor{3} & b_{0}\,   \\
 \end{vmatrix}  \\
 &R(f,g)=0\Longleftrightarrow
 \begin{cases}
 \text{ либо } a_{n}=0 \\
 \text{ либо } b_{m}=0 \\
 \text{ либо есть общий корень}
 \end{cases}
 \end{alignat*}
 Ещё рассм. дискриминант  $D(f)$, т. ч.
 \begin{alignat*}{1}
 &R(f,f')=(-1)^{\frac{n(n-1)}{2}}a_{0}D(f)\\
 &D(ax^{2}+bx+c)=b^{2}-4ac \\
 &P(z,w)=a_{n}(z)w^{n}+\dots+a_{0}(z),\; d(z)=R(P,P_{w}')
 \end{alignat*}
 Если $P$ --- неприводим и $a_{n}\ne 0$, то $d\not\equiv 0$.\par
 Вернёмся к алгебраическим функциям. Рассмотрим соотношение $P(z,w)=0$.
 $\forall\,A\notin\{\,d(z)=0\text{ или }a_{n}=0\,\}=\sigma$ уравнение $P(A,w)=0$
 имеет $n$ различных корней по $w$: $(B_{1},\dots,B_{n})$, причём в каждой
 точке $\frac{\pd P}{\pd w}\ne 0$, то есть применима комплексная
 теорема о неявной функции, то есть существует окрестность $\mathcal{U}_{a}$
 точки $a,\;\exists !\,f_{j}(z)\in\mathcal{O}(\mathcal{U}_{a})$ такая, что
 $f_{j}(A)=B_{j}$ и  $P(z,f_{j}(z))=0$ в $\mathcal{U}_{a}$.\par
 Каждый росток продолжается по любым путям, не проходящим через $\sigma$ (так
 как в каждой неособой точке применима теорема о неявной функции)\par
 Получаем $n$--значную функцию с конечным числом особых точек в $\overline{\Cm}$.
 \begin{theorem}[обратная]
Совокупность решений неприводимого ал\-ге\-бра\-и\-чес\-ко\-го уравнения
 $P(z,w)=0$ определяет алгебраическую функцию.
 \end{theorem}
 --во
 Осталось доказать, что все особые точки --- алгебраические. Для этого покажем,
 что если $c\in\sigma$, то $\exists\,k\in\mathcal{Z}$ такое, что
 $(z-c)^{k}f(z)\to 0$ при $z\to c$. (Считаем, что точка конечная, если $\infty$
 --- то же самое.) Положим $g=(z-c)^{k}f(z)$, тогда
 \[
 g^{n}+(z-c)^{k_{1}}\dfrac{a_{n-1}(z)}{a_{n}(z)}g^{n-1}+\dots+
 (z-c)^{k_{n}}\dfrac{a_{0}(z)}{a_{n}(z)}=0
 \]
 Все коэффициенты стремятся к 0 при $z\to c$, следовательно, все корни тоже
 стремятся к 0. \par
 Если $g^{n}+p_{n-1}g^{n-1}+\dots+p_{0}=0$, то
 \begin{alignat*}{1}
 &|g^{n}|\leqslant |g|^{n-1}\sum_{j=0}^{n-1}|p_{j}|\text{ при }\quad|g|\geqslant 1\\
 &|g^{n}|\leqslant \sum_{j=0}^{n-1}|p_{j}|\text{ при }\quad|g|<1\\
 &\Longrightarrow |g|\leqslant\max \left(\,\sum_{j=0}^{n-1}|p_{j}|,
 \sqrt[n]{\sum_{j=0}^{n-1}|p_{j}|} \,\right)\to 0
 \end{alignat*}
 Далее, наша функция не может принимать значения, не удовлетворяющие уравнению
 (при продолжении равенство нулю сохраняется). Если она принимает не все $n$
 значений, то она удовлетворяет полиномиальному уравнению меньшей (по $w$)
 степени $Q(z,w)=0$, причём $Q$ должен быть делителем $P$, что невозможно.
 \begin{note}
 Среди алгебраических функций есть класс функций, которые имеют представление
 в виде суперпозиции рациональных функций и корней. При этом, если мы пишем
 некую формулу, включающую корни, то подразумевается, что значения корней
 выбираются независимо. Однако, такая формула может распадаться на несколько
 алгебраических функций.
 \end{note}
 --римеры
 \[
 \sqrt{z}+\sqrt{z}=
 \begin{cases}
 2\sqrt{z}\qquad \\
 0
 \end{cases}\negthickspace\negthickspace\negthickspace\negthickspace
 \negthickspace\negthickspace\negthickspace\negthickspace\negthickspace,\qquad
 \sqrt{z^{2}}=
 \begin{cases}
 z \\
 -z
 \end{cases}
 \]
 \begin{note}
 Среди отмеченных точек могли оказаться не только особые точ\-ки нашей функции
 (точки ветвления и полюса), но и точки, в которых различные ветви принимают
 одинаковые значения.
 \end{note}
 --ример
 $w=\sqrt{z}(z-1)$, при $z=1\quad w=0$.
 \subsection{Римановы поверхности алгебраических функций}
 В конце первого семестра мы рассмотрели процедуру замыкания римановой
 поверхности в точке ветвления конечного порядка. Это достигалось введением в
 точке $a$ (ветвления порядка $n$) координаты $\ze$, связанной с $z$
 следующим образом: $\ze=\sqrt[n]{z-a}$ или $z=a+\ze^{n}$. В результате
 получаем конформное отображение проколотого диска
 ${\{\,0<|\ze|<\delta\,\}}$ на часть римановой поверхности, лежащей над
 ${\{\,0<|z-a|<\delta^{n}\,\}}$. После чего добавляем точку,
 соответствующую $\ze=0$. Эта процедура аналогична смене проекции: график
 $w=\sqrt{z}$ проецируется на $z$ как двулистное отображение с особенностью в 0,
 а на $w$ --- как однозначное.
 \begin{stm}
 Риманова поверхность после замыкания становится гомеоморфной сфере с
 конечным числом ручек ($g$ --- род поверхности).
 \end{stm}
 --во
 \begin{description}
 \item[(1) компактность] $\forall\,z\in\Cm\;\exists\,\varDelta_{z}$ такой, что
 $\pi^{-1}(\varDelta_{z})\simeq\,$кон. объединение дисков
 $\varDelta_{1},\dots,\varDelta_{m}$. Пусть $X_{n}$ --- последовательность
 точек римановой поверхности. Рассмотрим $z_{n}=\pi(X_{n})$, выберем сходящуюся
 подпоследовательность $z_{n_{j}}\to z$, разделим её на части таким образом,
 чтобы $X_{n_{j}}^{(m)}\subset\varDelta_{m}$. Одна из частей должна быть
 бесконечной, и, следовательно, сходящейся.
 \item[(2) ориентируемость] Любое комплексное многообразие после
 ове\-ществ\-ле\-ния превращается в  чётномерное вещественное ориентируемое, так
 как (якобиан овеществлённой функции перехода)
 $\mathcal{J}_{R}=|\mathcal{J}_{C}|^{2}$ (комплексный якобиан).
 \end{description}
 Таким образом, наша риманова поверхность --- комплексное двумерное
 ориентируемое многообразие, которое по известной теореме (Прасолов, Сосинский
 "Узлы, зацепления, косы и 3-х мерные многообразия") $\sim$ сфере с $g$ ручками.
 \par
 В топологии есть понятие триангуляции. Для двумерного случая --- это разбиение
 на конечное число криволинейных треугольников, таких, что любые два
 пересекаются либо по стороне, либо по точке, либо не пересекаются. Эйлеровой
 характеристикой многообразия $\chi(M^{2})$ называется величина $В-Р+Г$. Эта
 величина не зависит от триангуляции.\par
 Если многообразие получено склейкой $A$ и $B$ по $A\cap B$, то
 \[
 \chi(A\cup B)+\chi(A\cap B)=\chi(A)+\chi(B)
 \]
 Если двумерные многообразия склеены по одной или нескольким окружностям, то
 $\chi(A\cup B)=\chi(A)+\chi(B)$, так как для окружности $\chi=0$.
 \begin{stm}
 Если $M_{g}^{2}$ --- сфера с $g$ ручками, то $\chi(M_{g}^{2})=2-2g$.
 \end{stm}
 --во
 Если $F_{g}^{2}=M_{g}^{2}\setminus(D^{2}+O^{2})$, то
 $\chi(M_{g}^{2})=\chi(F_{g}^{2})+2$, но
 $M_{g+1}^{2}=F_{g}^{2}+$ручка
 $\Longrightarrow\:\chi(M_{g+1}^{2})=\chi(M_{g}^{2})-2,\:\chi(S^{2})=2$.\par
 \subsection{Формула Римана--Гурвица}
 Пусть  $m$--значная алгебраическая функция имеет точки ветвления с
 кратностями $k_{1},\dots,k_{n}$ (точки считаем на римановой поверхности, а не
 на \oC), тогда род
 \[
 g=\sum_{j=1}^{n}\dfrac{k_{j}-1}{2}-m+1
 \]
 --во
 Пусть есть две согласованные триангуляции $2g-2=В-Р+Г$ и $2=в-р+г$, тогда
 $Г=mг,\:Р=mр,\:В=mв-\sum(d_{j}-1)\:\Longrightarrow\:2-2g=2m-\sum(d_{j}-1)$.
 \section*{Алгебраические функции.}
 \begin{theorem}
 Число листов не зависит от точки.
 \end{theorem}
 \begin{theorem}
 Если $m=1$, то функция рациональна.
 \end{theorem}
 \begin{theorem} \label{T:три}
 Значения алгебраической функции удовлетворяют неприводимому алгебраическому
 уравнению
 \begin{align}   \label{Равенство}
 &F(z,w)=p_{m}(z)w^{m}+\ldots+p_{0}(z)=0,
 \end{align}
 где $(p_{0},\dots,p_{m})$ --- не имеют общего множителя
 \end{theorem}
 \begin{theorem}
 Корни $p_{1}(z),\dots,p_{m}(z)$ неприводимого алгебраического уравнения
 $\mathrm{(\ref{Равенство})}\;\,\forall\,z$ являются значениями одной
 алгебраической функции.
 \end{theorem}
 --во
 При $z\notin\{\,p_{0}(z)=0\text{ или }d(z)=0\,\}=\sigma\;(\ref{Равенство})$\;
 имеет $m$ различных корней. Возьмём $(z_{0},w_{0})$ --- решение при
 $z_{0}\notin\sigma;\;\frac{\pd F}{\pd w}(z_{0},w_{0})\ne 0
 \:\Longrightarrow\:$ применим теорему о неявной функции. Получаем $w=f(z)$
 такую, что $f(z_{0})=w_{0}$.\par
 Продолжение корня --- корень. Решение продолжается по всем путям вне $\sigma$
 (по той же теореме о неявной функции).
 \begin{itemize}
 \item число значений $\leqslant m$
 \item число особых точек $<\infty$
 \item все они алгебраические
 \end{itemize}
 Следовательно, результат продолжения --- алгебраическая функция, которая по
 теореме (\ref{T:три}) удовлетворяет соотношению $\Phi(z,w)=0$.\par
 $\Phi(z,w)=q_{k}(z)w^{k}+\dots+q_{0}(z)$
 \begin{alignat*}{1}
 &F(z,w)=p_{m}(z)(w-f_{1}(z))\dots(w-f_{m}(z))\\
 &\Phi(z,w)=q_{k}(z)(w-f_{1}(z))\dots(w-f_{k}(z))\\
 &(w-f_{k+1}(z))\dots(w-f_{m}(z))=
 \dfrac{1}{R_{m-k}(z)}\left[R_{m-k}(z)w^{m-k}+\dots+R_{0}(z)\right]\\
 &\Longrightarrow\:F(z,w)=\dfrac{p_{m}(z)}{q_{k}(z)R_{n-k}(z)}%
 [q_{k}(z)w^{k}+\dots+q_{0}(z)][R_{m-k}(z)w^{m-k}+\dots+R_{0}(z)]\\
 &\dfrac{p_{m}(z)}{q_{k}(z)R_{n-k}(z)}=\dfrac{P(z)}{Q(z)}
 \end{alignat*}
 $\deg P=0$, так как иначе $\exists\,a:\,P(a)=0\:\Longrightarrow\:F(a,w)
 \equiv 0\:\Longrightarrow\:$есть общий множитель $(z-a)$ у коэффициентов.\\
 $\deg Q=0$, так как иначе $\exists\,b:\,Q(b)=0\:\Longrightarrow\:F(b,w)
 =\infty\:\Longrightarrow\:$есть общий множитель либо у $Q$, либо у $R$.

 \par\bigskip\textbf{Лекция++}\par
 \section{О разрешимости алгебраических уравнений в радикалах}
 Для всякой $n$--значной алгебраической функции $f$, чьи особые точки
 проектируются в точки плоскости $a_{1},\dots,a_{m}$, элемент фундаментальной
 группы $\pi_{1}(\Cm\setminus\{a_{1},\dots,a_{m}\})$, то есть петля с началом
 и концом в неособой точке порождает перестановку значений. Возникает
 гомоморфизм \\
 $\pi_{1}(\Cm\setminus\{a_{1},\dots,a_{m}\})\longrightarrow S{n}$, его образ
 называется группой монодромии  $G(f)$. Образы образующих в $\pi_{1}$, то есть
 образы вокруг $a_{j}$, дают образующие в $G(f)$. Эта конструкция применима к
 любой формуле, составленной из радикалов, даже если она распадается на
 несколько функций.\par\noindent
 \textsf{Вопрос:} Как по группе узнать, сколько функций задаёт формула?\par
 Для уравнения второй степени $aw^{2}+bw+c=0$ есть формула, выражающая корни
 через коэффициенты. Такие же формулы есть для уравнений степени три и четыре
 (см. Алексеев , теорема Абеля $\dots$). Пусть $P(a,w)=a_{n}w^{n}+\dots+a_{0}$ ---
 общее уравнение степени $n$.  Если какой--то набор $(a,w)$, такой, что
 $P(a,w)=0,\;P_{w}'(a,w)=0$, то применима теорема о неявной функции и локально
 мы получаем $w=f(a)$ --- это алгебраическая функция коэффициентов, она
 {R--дифференцируема} и голоморфна по каждому переменному.\par\noindent
 Наш вопрос: является ли эта функция суперпозицией радикалов и арифметических
 действий?\par
 Если --- да, то зависимость решения от каждого коэффициента --- алгебраическая
 функция одного переменного.
 \begin{theorem}[Абеля] Если $n\geqslant 5$, то $f$ не является суперпозицией
 радикалов.
 \end{theorem}
 \par\noindent\emph{Схема доказательства.}
 Покажем это для $n=5$ (произвольная степень получается домножением на степень
 $w$). Для этого приведём пример уравнения пятой степени с одним переменным
 коэффициентом $z$ и покажем, что зависимость от этого коэффициента $w=f(z)$
 не является суперпозицией.  Чтобы установить это, мы покажем, что группа
 монодромии суперпозиции корней обладает неким свойством (разрешимость),
 которое не выполнено для нашей функции.
 \section*{Напоминание материала из алгебры.}
 $G$ --- группа; $a,b\in G$, коммутатор$\,(a,b)=aba^{-1}b^{-1}$.
 \par\noindent\emph{Коммутант $K(G)$} --- это подгруппа, порождённая всеми
 коммутаторами.
 \begin{stm}
 \textup{(}a\textup{)}\quad $K(G)$ --- нормальная подгруппа;\;
 \textup{(}b\textup{)}\quad $G\diagup_{\!\!K(G)}$\; коммутативна.
 \end{stm}
 Группа называется \emph{разрешимой}, если  ряд
 \[
 G\supset K(G)\supset K(K(G))\supset\ldots\text{ заканчивается } \{\,e\,\}.
 \]
 \begin{stm}          \label{утверждение:3}
 \begin{itemize}
 \item[]
 \item[(a)]  подгруппа разрешимой группы разрешима
 \item[(b)] образ (при гомоморфизме) разрешимой группы разрешим
 \item[(c)] $G\supset N$ --- нормальная подгруппа в разрешимой группе
 $\Longrightarrow\:G\diagup_{\!\!N}$ --- разрешима
 \item[(d)] $N$ и $G\diagup_{\!\!N}$  --- разрешимы
 $\Longrightarrow\:G$--- разрешима
 \item[(e)] $N$ и $F$  --- разрешимы
 $\Longrightarrow\:G\times F$--- разрешима
 \end{itemize}
 \end{stm}
 \begin{stm}
 $S_{n}$ при  $n\geqslant 5$ неразрешима.\par\noindent
 ($A_{n}\subset S_{n}$ некоммутативна и не имеет нормальной подгруппы.)
 \end{stm}
 \begin{lemma}
 Пусть $f$ и $g$ ---алгебраические функции, $G(f),\;G(g)$ --- разрешимы, тогда
 $G(f\,\Box\,g)$, где $\Box$ --- одна из арифметических операций
 --- тоже разрешима.
 \end{lemma}
 --во
 Пусть $f_{1},\ldots,f_{n}$ --- набор значений $f$, $g_{1},\ldots,g_{n}$ ---
 набор значений  $g$. И пусть $f_{i}\,\Box\,g_{i}$ попарно различны, тогда это
 набор значений для $f\,\Box\,g$ и обходы действуют на каждое "слагаемое"
 независимо. Тогда $G(f\,\Box\,g)=G(f)\times G(g)$ и, поэтому, разрешима. Если
 произошло совпадение элементов, то $G(f\,\Box\,g)$ является образом
 $G(f)\times G(g)$ при гомоморфизме, порождённом склейкой.\par\noindent
 Как строится схема римановой поверхности $\sqrt[n]{f(z)}$ на основе римановой
 поверхности $f$?
 \begin{itemize}
 \item[---] каждый лист заменить на "пачку" из $n$ листов
 \item[---] если при этом произошло совпадение --- склеить
 \end{itemize}
 Как действуют обходы?
 \begin{itemize}
 \item[---] если обходим точки $f$, то идёт перестановка между пачками, то есть
 пачки переставляются как целое, если обходим нули $f$, то переставляются
 листы в каждой пачке (цикл).
 \end{itemize}
 \begin{lemma}
 Если $G(f)$ --- разрешима, то $G(\sqrt[n]{f(z)})$ --- тоже разрешима.
 \end{lemma}
 --во
 Рассмотрим гомоморфизм
 $\ph:\,G(\sqrt{f(z)})\overset{\text{на}}{\longrightarrow}G(f)$
 (лист $\longrightarrow$ пачка). Его ядро коммутативно (подгруппа в прямой сумме
 циклических групп). Но
 \[
 G(f)\simeq G(\sqrt{f(z)})\diagup{\ker\ph}\:\Longrightarrow\:
 (\,\mbox{утверждение }\ref{утверждение:3}\,)\quad G(\sqrt{f(z)})
 \mbox{ --- разрешима.}
 \]
 ($\ker\ph$ --- это действие обходов, не переставляющих пачки, а
 переставляющих листы в пачке, то есть мы имеем подгруппу в прямой сумме
 циклических групп.)\par
 Если взять первое попавшееся уравнение пятой степени, то оно в радикалах
 неразрешимо.
 \[
 P=3w^{5}-25w^{3}+60w-z=0\,,\quad w=f(z)\text{ --- пятизначная}
 \]
 Решая уравнение $\frac{\pd P}{\pd w}=0$, получаем
 $15(w^{4}-5w^{2}+4)=0,\; w=\pm 1,\,\pm 2$, при этом $z=\pm 38,\,\pm 16$.\par\noindent
 Корни кратности 2, следовательно, $\forall\,z\ne\pm 38,\,\pm 16\:$ уравнение имеет
 5 корней.
 Вот схема:
%*******************************************************************************
%*******Рисунок******************************************************************
%*******************************************************************************
 \begin{center}
 \textit{Место для Вашего Рисунка\\(Entschuldigen Sie mir, bitte!)}
 \end{center}
 (Каждая точка --- ветвление второго порядка, сшиты все листы ---
 иначе уравнение приводимо.) \par
 $G(f)=S_{5}$ --- неразрешима.  \par
 Для произвольного $n>5$ можно рассмотреть $(3w^{5}-25w^{3}+60w-z)w^{n-5}=0$

 \begin{note}
 Можно поставить вопрос: а нет ли радикальной формулы, которая задаёт
 несколько функций, таких, что среди них нашлась бы $f$? Ответ --- нет, так
 как если бы такая формула нашлась, то возник бы гомоморфизм разрешимой группы
 (группы монодромии этой формулы) на $S_{5}$.
 \end{note}

 \par\bigskip\textbf{Лекция++}\par
 \begin{flushleft}
 \textbf{А. Домрин}
 \end{flushleft}
 \begin{theorem}[Миттаг-- Лёфлера]
 Пусть $\{a_{n}\}\in\Cm$ --- последовательность без предельных точек в \C, и пусть
 заданы функции
 \[
 R_{n}(z)=\sum_{k=1}^{k_{n}}c_{kn}(z-a_{n})^{-k},\quad\text{где }
 k_{n}\geqslant 1\,,\;c_{kn}\in\Cm.
 \]
 Тогда существует $f\in\mathcal{M}(\Cm)$ такая, что
 $(f-R_{n})\in\mathcal{O}(\text{окрестность}\,a_{n})$ при всех $n$.
 \end{theorem}
 \begin{imp}[разложение на простейшие дроби]
 Если $\{\,a_{1},\ldots,a_{n}\,\}$ --- множество полюсов функции
 $f\in\mathcal{M}(\Cm)$, и $R_{n}(z)$ --- главная часть ряда $f$ Лорана в точке
 $a_{n}$, то существуют полиномы $P_{n}(z)$ и функция $g$ такие, что
 \[
 f=g+\sum_{n=1}^{\infty}(R_{n}-P_{n}).
 \]
 \end{imp}
 --ример
 \begin{alignat*}{1}
 &\ctg z=\dfrac{1}{z}+\sum_{n\in\Zm\setminus\{0\}}%
 \left(\dfrac{1}{z-\pi n}+\dfrac{1}{\pi n}\right)\\
 &\left[\text{Доказано через }\int_{|z|=\pi(n+\frac{1}{2})}%
 \ctg\ze\!\cdot\!\dfrac{2z}{z^{2}-\ze^{2}}d\ze\right]
 \end{alignat*}
 \begin{theorem}[Вейерштрасса]
 Пусть $\{a_{n}\}\in\Cm\setminus\{0\}$ --- последовательность без предельных точек
 в \C и пусть целые числа $p_{n}>0$ таковы, что ряд
 $\sum_{n=1}^{\infty}\left|\frac{z}{a_{n}}\right|^{p_{n}+1}$ сходится при всех
 $z\in\Cm$ $($что заведомо верно при $p_{n}=n-1)$. Тогда
 \begin{alignat*}{1}
 &f(z)=\prod_{n=1}^{\infty}E_{p_{n}}\left(\frac{z}{a_{n}}\right)\,,
 \text{ где }E_{p}(z):=(1-z)\exp\left\{\,z+\dfrac{z^{2}}{2}+\ldots+\
 \dfrac{z^{p}}{p}\,\right\},\\
 &E_{0}(z):=1-z\text{ --- множители Вейерштрасса,}
 \end{alignat*}
 есть целая функция с нулями в $a_{1},a_{2},\ldots$ и только в них.
 \end{theorem}
 \begin{lemma}
 Пусть $\Dm\subset\Cm$ область, $f_{n}\in\ODm,\;f_{n}\not\equiv 0$ в \D и ряд\\
%!!!!!!!!!!!!!ПЕРЕНОС!!!!!!!!!!!!!!!!!!
 $\sum_{n=1}^{\infty}|1-f_{n}(z)|$ сходится равномерно на компактах в \D.
 Тогда:
 \begin{itemize}
 \item[1)] последовательность $P_{n}=f_{1}f_{2}\dots f_{n}$ сходится равномерно
 на компактах в \D к некоторой $f\in\ODm,\;f\not\equiv 0$ (причём это свойство
 сохраняется при любой перестановке $f_{n}$--тых);
 \item[2)] Ряд $\sum_{n=1}^{\infty}\frac{f_{n}'}{f_{n}}$ сходится равномерно
 на компактах к $\frac{f'}{f}$ в \D;
 \item[3)] $\forall a\in\Dm$  имеем
 \begin{alignat*}{1}
 &\ord_{a}f=\sum_{n=1}^{\infty}\ord_{a}f_{n}\,,\text{ где}\\
 &\ord_{a}f:=\max\{\,n\in\Zm\mid\exists\,g\in\mathcal{O}(\text{окрестность}\,a)
 \,:\,f(z)=(z-a)^{n}g(z)\,\}
 \end{alignat*}
 $($то есть порядок нуля $f$ в точке $a)$
 \end{itemize}
 \end{lemma}
 \begin{lemma}
 $|1-E_{p}(z)|\leqslant |z|^{p+1}$ при всех $p\geqslant 0\,,\;|z|\leqslant 1$.
 \end{lemma}
 \begin{imp}[из теоремы Вейерштрасса]
 Пусть $f\in\mathcal{O}(\Cm)\,,\;f(0)\ne 0$ и пусть $\{\,a_{1},a_{2}\ldots\,\}$
 --- список всех нулей $f$ с учётом кратностей. Тогда существует
 $g\in\mathcal{O}(\Cm)$ и $p_{1},\,p_{2},\ldots\in\Zm$ (неотрицательные) такие,
 что
 \[
 f(z)=e^{g(z)}\prod_{n=1}^{\infty}E_{p_{n}}\negthickspace\left(\dfrac{z}{a_{n}}\right)
 \]
 \end{imp}
 --ример
 \[
 \sin z=z\prod_{n=1}^{\infty}\left(1-\dfrac{z^{2}}{\pi^{2}n^{2}}\right)
 \]

 \par\bigskip\textbf{Лекция++}\par
 \begin{flushleft}
 \textbf{А. Домрин}
 \end{flushleft}
 \begin{theorem}
 Пусть $\Dm\subset\Cm$ --- область, $\{\,a_{1},a_{2},\ldots\,\}$ ---
 последовательность без предельных точек в \D. Тогда существует $f\in\ODm$
 такая, что множество нулей $f$ совпадает с $\{\,a_{1},a_{2},\ldots\,\}$.
 \end{theorem}
 \begin{imp}
 Если $f\in\MDm$, то $f=\frac{g}{h}$ для некоторых $g,\,h\in\ODm$.
 \end{imp}
 \begin{imp}
 Для всякой области $\Dm\subset\Cm$ существует $f\in\ODm$, не продолжаемая
 голоморфно ни через одну точку $\pd\Dm$ (то есть не существует области
 $\Dm_{1}\supset\Dm$ такой, что $\Dm_{1}\ne\Dm$ и $f\in\mathcal{O}(\Dm_{1})\,$).
 \end{imp}
 \begin{theorem}
 Пусть $\{\,a_{n}\,\}\subset\Cm$ --- последовательность без предельных точек в
 \C, и заданы $c_{n}\in\Cm$. Тогда существует $f\in\mathcal{O}(\Cm)$ такая, что
 $f(a_{n})=c_{n}$  при всех $n$.
 \end{theorem}
 \begin{theorem}
 Пусть $f\in\mathcal{M}(\Cm)$ не постоянна. Тогда множество всех периодов $f$
 $($то есть таких $w\in\Cm$, что $f(z+w)=f(z)$ для всех $z\in\Cm)$ есть
 дискретная абелева подгруппа \C. В частности, оно есть либо
 \begin{itemize}
 \item[(a)] $\{\,0\,\}$;
 \item[(b)] $\Zm w_{1}$  для некоторого $w_{1}\in\Cm\setminus\{\,0\,\}$;
 \item[(c)] $\Zm w_{1}+\Zm w_{2}$  для некоторых
 $w_{1},\,w_{2}\in\Cm\setminus\{\,0\,\}$ таких, что
 $\frac{w_{2}}{w_{1}}\notin\Rm$.
 \end{itemize}
 \end{theorem}
 \begin{df}
 $f\in\mathcal{M}(\Cm)$ называется эллиптической, если либо $f=\const$, либо
 множество периодов $f$ есть $\Lambda=\Zm w_{1}+\Zm w_{2}$, где
 $w_{1},\,w_{2}\in\Cm\setminus\{\,0\,\},\;\frac{w_{2}}{w_{1}}\notin\Rm$.
 \end{df}
 \subsection*{Свойства эллиптических функций}
 \begin{itemize}
% \item[]
 \item[(1)]  Эллиптические функции с данным $\Lambda$ образуют поле, замкнутое
 относительно дифференцирования.
 \item[(2)]  Если $f$ --- эллиптическая функция и $f\not\equiv\const$, то $f$ имеет полюсы
 в параллелограмме периодов $\Pi$.
 \item[(3)] Если $\ze_{1},\ldots,\ze_{k}$ --- полюсы, то
 \[
 \sum_{j=1}^{k}\res_{z=\ze_{j}}f=0.
 \]
 \item[(4)] Всякое значение $w_{0}\in\Cm$ в $\Pi$ принимается ровно $k$ раз
 (\,где $k$ --- это число полюсов $f$ в $\Pi$ с учётом кратностей; $k$ называется
 порядком $f$\,).
 \item[(5)] Не существует эллиптической функции порядка 1.
 \end{itemize}
 \begin{stm}
 Ряд
 \[
 \ga(z)=\dfrac{1}{z^{2}}+\sum_{w\in\Lambda\setminus\{0\}}%
 \left(\dfrac{1}{(z-w)^{2}}-\dfrac{1}{w^{2}}\right)
 \]
 сходится равномерно на компактах в \C и задаёт чётную эллиптическую функцию
 порядка 2. $(\,\ga$--функция Вейерштрасса.$\,)$
 \end{stm}

 \par\bigskip\textbf{Лекция++}\par
 \begin{flushleft}
 \textbf{А. Домрин}
 \end{flushleft}
 \begin{theorem}
 Всякая эллиптическая функция с решёткой $\Lambda$ есть
 $R_{1}(\ga)+R_{2}(\ga)$, где $R_{1},\,R_{2}$ --- рациональные функции.
 \end{theorem}
 \begin{theorem} \label{теорема:2}
 Положим $G_{n}:=\sum_{\Lambda\setminus\{0\}}w^{-n}$. Тогда
 ${{\ga\,}'}^{2}=4\ga^{3}-g_{2}\ga-g_{3}$, где
 $g_{2}=60G_{4},\:g_{3}=140G_{6}$.
 \end{theorem}
 \begin{imp}[из теоремы \ref{теорема:2}]
 $4x^{3}-g_{2}x-g_{3}=4(x-e_{1})(x-e_{2})(x-e_{3})$, где
 $e_{1}=\ga\left(\frac{w_{1}}{2}\right),\:
 e_{2}=\ga\left(\frac{w_{1}+w_{2}}{2}\right),\:
 e_{3}=\ga\left(\frac{w_{2}}{2}\right)$. В частности, $e_{1}+e_{2}+e_{3}=0$
 и $g_{2}^{3}-27g_{3}^{2}\ne 0$.
 \end{imp}
 \begin{imp}[из теоремы  \ref{теорема:2}]
 Функция $z=\ga(w)$ есть обратная к
 \[
 w=\int_{\infty}^{z}(4\ze^{3}-g_{2}\ze-g_{3})^{-\frac{1}{2}}d\ze .
 \]
 \end{imp}
 \begin{theorem} Пусть $b_{1},\,b_{2}\in\Cm$ таковы, что
 $b_{2}^{3}-27b_{3}^{2}\ne 0$, то есть все три корня $k_{1},\:k_{2},\:k_{3}$
 полинома $P(\ze)=4\ze^{3}-b_{2}\ze-b_{3}$ различны. Выберем
 $z_{0}\in\Cm\setminus\{\,k_{1},\,k_{2},\,k_{3}\,\}$ и начальное значение
 $P(\ze)^{-\frac{1}{2}}$ в окрестности $U$ точки $z_{0}$. Тогда:
 \begin{itemize}
 \item[1)] Элемент $(U,g)$, где
 \[
 g(z)=\int_{z_{0}}^{z}\negthickspace\negthickspace%
 P(\ze)^{-\frac{1}{2}}d\ze\;\text{ для }z\in U,
 \]
 продолжается по всем путям в $\Cm\setminus\{\,k_{1},\,k_{2},\,k_{3}\,\}$ и
 полученная аналитическая функция имеет счётное число точек ветвления второго
 порядка над каждой из точек $k_{1},\:k_{2},\:k_{3},\:\infty$, причем при
 обходе вокруг $k_{j}$ (неважно, в каком направлении)
 $g\longrightarrow E_{j}-g$, где
 \[
 E_{j}=\int_{z_{0}}^{k_{j}}\negthickspace\negthickspace%
 P(\ze)^{-\frac{1}{2}}d\ze,
 \]
 $($скажем, интеграл по прямолинейному отрезку$)$;
 \item[2)] аддитивная подгруппа $\Lambda\subset\Cm$, порождённая разностями
 $E_{j}-E_{k},\:(1\leqslant j\leqslant k\leqslant 3)$ не зависит от выбора
 $z_{0}$ и начального элемента $P(\ze)^{-\frac{1}{2}}$ и является двумерной
 решёткой;
 \item[3)] если $\ga(v)$ есть $\ga$--функция, отвечающая решётке
 $\Lambda$, то $$\ga(g(z)-g(\infty))\equiv z.$$
 \end{itemize}
 \end{theorem}
 \begin{imp}["задача обращения"]
 Для любых $b_{2},\,b_{3}\in\Cm$, таких, что $b_{2}^{3}-27b_{3}^{2}\ne 0$
 существует решётка $\Lambda$ такая, что $g_{2}(\Lambda)=b_{2},\:
 g_{3}(\Lambda)=b_{3}$.
 \end{imp}

 \par\bigskip\textbf{Лекция++}\par
 \begin{flushleft}
 \textbf{А. Домрин}
 \end{flushleft}
 \begin{theorem}[Эллиптический синус Якоби]
 Пусть $0<k<1$. Тогда:
 \begin{itemize}
 \item[1)]
 \[
 w=g(z)=\int_{0}^{z}\negthickspace\negthickspace%
 \left\{(1-\ze^{2})(1-k^{2}\ze^{2})\right\}^{-\frac{1}{2}}d\ze
 \]
 есть аналитическая функция на $\Cm\setminus\{\pm1,\,\pm\frac{1}{k}\}$ со
 счётным числом точек ветвления второго порядка над каждой из точек
 $\pm1,\,\pm\frac{1}{k}$ и со счётным числом устранимых особенностей над
 $\infty$, причём каждая ветвь $g(z)$ в верхней полуплоскости
 $\Pi=\{\,\text{Im}\,z>0\,\}$ конформно отображает $\Pi$ на некоторый
 прямоугольник плоскости $w$;
 \item[2)] обратная функция sn$\,z$ есть эллиптическая функция порядка 2 с
 ре\-шёт\-кой периодов $\Lambda:=4K\Zm+2iK\Zm$, где
 \begin{alignat*}{1}
 &K=\int_{0}^{1}\negthickspace\negthickspace%
 \left\{(1-x^{2})(1-k^{2}x^{2})\right\}^{-\frac{1}{2}}dx,\\
 &K'=\int_{0}^{\frac{1}{k}}\negthickspace%
                \left\{(1-x^{2})(1-k^{2}x^{2})\right\}^{-\frac{1}{2}}dx
 \end{alignat*}
 $($арифметическое значение корней$)$.
 \end{itemize}
 \end{theorem}
 \begin{theorem}[Формула Кристоффеля--Шварца]
 Пусть $w=f(z)$ есть конформное отображение $\Pi=\{\,\text{Im}\,z>0\,\}$ на
 ограниченный многоугольник $M\subset\Cm$, имеющий в вершине $A_{i}=f(a_{i})$
 $($где $-\infty<a_{1}<\ldots<a_{n}<+\infty$, то есть $f(\infty)$ не есть
 вершина$)$ внутренний угол $\pi\alpha_{i}\;(0<\alpha_{i}\leqslant 2)$. Тогда
 существуют константы $c_{1},\:c_{2},\:z_{0}$ такие, что
 \[
 f(z)=c_{1}\int_{z_{0}}^{z}\prod_{j=1}^{n}(\ze-a_{j})^{\alpha_{j}-1}d\ze+c_{2}.
 \]
 \end{theorem}

 \par\bigskip\textbf{Лекция++}\par
 \section{Функции нескольких переменных.}
 \D --- область в $\Cm^{n}$.
 \begin{df}
 $f$ называется голоморфной в \D $(\,f\in\ODm\,)$, если:
 \begin{itemize}
 \item[(1)] $f\quad R$--дифференцируема в \D
 \item[(2)] $\forall\,z\in\Dm,\;\forall\,j=1,\ldots,n\quad
 \frac{\pd f}{\pd \overline{z}_{j}}=0$
 \end{itemize}
 Для любой гладкой $f$
 \[
 df=\sum\frac{\pd f}{\pd z_{j}}dz_{j}+
 \sum\frac{\pd f}{\pd \overline{z}_{j}}d\overline{z}_{j}=
 \pd f+\pd\overline{f}
 \]
 Условие (2)$\:\Longleftrightarrow\:\pd\overline{f}=0\:\Longleftrightarrow\:
 df$ --- комплексно--линеен
 \end{df}
 \begin{stm}
 Если $f$ голоморфна в \D, то имеет место формула Коши
 \[
 f(z_{1},\ldots,z_{n})=\dfrac{1}{(2\pi i)^{n}}\int_{|\ze-a|=r}%
 \dfrac{f(\ze_{1}\dots\ze_{n})}{(\ze_{1}-z_{1})\dots(\ze_{n}-z_{n})}
 d\ze_{1}\dots d\ze_{n}
 \]
%*******************************************************************************
%*****Das**Bild*****************************************************************
%*******************************************************************************
 \begin{center}
 \textit{Sehr geehrte Damen und Herren!\\Hier soll das Bild sein.}
 \end{center}
 $\overline{\varDelta(a,r)}\subset\Dm,\;z\in\varDelta(a,r)$
 \end{stm}
 \begin{stm}
 Если $f$ голоморфна, то $f=\sum c_{n}(z-a)^{n}$, причём ряд абсолютно и
 равномерно сходится в любом $\overline{\varDelta(a,r)}\Subset\Dm$.\par
 Имеет место формула для коэффициентов
 \[
 c_{p}=\dfrac{1}{(2\pi i)^{n}}\int_{|\ze-a|=r}%
 \dfrac{f(\ze)d\ze}{(\ze-a)^{p+1}}
 \]
 \end{stm}
 \begin{stm}
 Сумма кр. степенного ряда голоморфна в области сходимости.
 \end{stm}
 \begin{center}
 \large{\emph{Цепочка эквивалентностей:}}
 \end{center}
 \begin{align*}
 &(1) &&\text{Непрерывность}+
 \frac{\pd f}{\pd \overline{z}_{j}}=0\:\Longrightarrow\;(2)\\
 &(2) &&f=\text{формула Коши}\:\Longrightarrow\;(3)\\
 &(3) &&f=\text{степенному ряду}\:\Longrightarrow\;(1)
 \end{align*}
 \begin{theorem}
 $f$ голоморфна по каждому переменному $\Longrightarrow\:f\in\ODm$.
 \end{theorem}
 Задача про многочлены.
 \begin{theorem}[Аналог теоремы Мореры]
 \begin{itemize}
 \item[]
 \item[(1)] Если $f$ голоморфна, то для любой $(n+1)$--цепи $\sigma$
 \begin{alignat*}{1}
 &\int_{\pd\sigma}\negthickspace\negthickspace\negthickspace%
 fdz_{1}\wedge\ldots\wedge dz_{n}=
 \int_{\sigma}\negthickspace df\wedge dz=0\qquad
 \begin{cases}
 \pd\overline{f}=0\\
 \pd f\wedge dz=0
 \end{cases}
 \end{alignat*}
 \item[(2)] Если для некоторой $f$ и для любой $\sigma$
 \begin{alignat*}{1}
 0=\int_{\pd\sigma}\negthickspace\negthickspace\negthickspace fdz=
 \int_{\sigma}\negthickspace df\wedge dz\:\Longrightarrow\:
 df\wedge dz=0\:\Longrightarrow\:\pd\overline{f}=0
 \end{alignat*}
 \end{itemize}
 \end{theorem}
 \section{Свойства голоморфных функций.}
 \begin{itemize}
 \item[(1)] Если $f=\sum c_{p}(z-a)^{p}$, то
 $c_{p}=\frac{1}{p!}D^{\,p}\!f(0)$, то есть это ряд Тейлора.
 \item[(2)] \emph{Теорема единственности: }
 ($f=z_{1}\quad\{z_{1}=0\}$ имеет предельные точки)\\ $f\in\ODm\quad
 \mathcal{U}\subset\Dm\quad\left.f\right|_{\mathcal{U}}=0\:\Longrightarrow\:
 f\equiv 0$
 \item[(3)] Принцип максимума (через сечения)
 \item[(4)] Неравенство Коши: $|c_{p}|\leqslant\frac{M(r)}{r^{p}}$, где
 $M(r)=\max_{|z-a|=r}|f|$
 \item[(5)] Теорема Лиувилля (следует из (4))
 \item[(6)] принцип открытости (по сечениям)
 \item[(7)] теорема Вейерштрасса о предельном переходе
 \item[(8)] принцип компактности
 \item[(9)] многозначные функции
 \end{itemize}

 \par\bigskip\textbf{Лекция++}\par
 \section{Функции голоморфного продолжения.}
 \begin{itemize}
 \item[(1)] В $\Cm_{1}$ для любой обл-ти  \D есть $f\in\ODm$ такая, что $f$ не
 продолжается никуда.
 \item[(2)] $\Cm_{2},\;\Dm=\varDelta(0,(1,10))\cup\varDelta(0,(10,1))$. В силу
 логарифмической выпуклости области сходимости степенного ряда все функции из
 \OD продолжаются в $\widehat{\Dm}$ --- логарифмически выпуклую оболочку \D.
 \end{itemize}
 \begin{df}
 \D называется областью голоморфности, если существует $f\in\ODm$ такая, что
 для неё реализуется следущая ситуация:
%*******************************************************************************
%*****************Das*Bildchen**************************************************
%*******************************************************************************
 \begin{center}
 \textit{Meine Damen und Herren!\\Ich bitte um Entschuldigung:
 hier soll ein Bildchen sein!}
 \end{center}
 \begin{align*}
 &\mathcal{U}\not\subset\Dm        &g\in\mathcal{O}(\mathcal{U})\\
 &\mathcal{U}\cap\Dm\ne\varnothing
                             &\left.f=g\right|_{V\subset\,\mathcal{U}\cap\Dm}
 \end{align*}
 \end{df}
 --римерь
 области, не являющейся областью голоморфности --- область из пункта (2):
 \[
 \Dm=\varDelta(0,(1,10))\cup\varDelta(0,(10,1)).
 \]
 \begin{theorem}[Хартогс]
 Пусть $f\in\mathcal{O}(\,\Cm^{n}\setminus\textup{компакт}\,)$, тогда $f$ имеет
 голоморфное продолжение на $\Cm^{n}\;(n>1)$.
 \end{theorem}
 \begin{stm}
 Пусть $f$ голоморфна в проколотой окрестности точки $a\in\Cm^{n}\;(n>1)$,
 тогда $f$ продолжается в полную окрестность.
 \end{stm}
 \begin{imp}
 Не бывает изолированных нулей.
 \end{imp}
{\large\emph{Фигура Хартогса.}}
 $f\in\ODm$, тогда $f$ продолжается в $\De$.
 \begin{center}
 \textit{Pardon,\\ иллюстрация скоро будет.}
 \end{center}
%********И***С***Ц**************************************************************
%*********Л*Ю*Т*А*И*************************************************************
%**********Л***Р***Я************************************************************
 \[
 F(z_{1},z_{2}):=\dfrac{1}{2\pi i}\int_{|\ze_{2}|=1-\varepsilon}%
 \dfrac{f(z_{1},\ze_{2})d\ze_{2}}{\ze_{2}-z_{2}}
 \]
 \begin{align*}
 &(1) &&F\text{ --- голоморфная в }
                          \varDelta(\,0;\,1-\varepsilon,\,1-\varepsilon\,) \\
 &(2) &&F\text{ совпадает с }f\text{ в }
                          \varDelta(\,0;\,\delta,\,1-\varepsilon\,)
 \end{align*}
 $\Longrightarrow\:f$ продолжается в $\varDelta(\,0;\,1,\,1\,)$.
 \begin{stm}
 Выпуклая область --- область голоморфности.
 \end{stm}
 \begin{df}
 Говорим, что в $\xi$ есть барьер, если
 \begin{alignat*}{1}
 \forall\,K\Subset\Dm\;\forall\,\varepsilon>0\;\,\exists\,g\in\ODm:\:
 \max |g|\leqslant 1\;\exists\,\ze\in B(\,\xi,\,\varepsilon\,):\,
 |g(\xi)|>1\,.
 \end{alignat*}
 \end{df}
 \begin{theorem}
 Пусть $E\subset\pd E$ и $\forall\,\xi\in E$ "существует барьер", тогда
 существует $f\in\ODm$, неограниченная в точках $E$.
 \end{theorem}
 --во
 \begin{itemize}
 \item[(1)] $E$ можно считать не более, чем счётным с $\{\,\xi_{p}\,\}$ (причём
  каждая точка повторяется бесконечное число раз).
 \item[(2)] Достаточно построить $f$ и $\{\,z_{p}\,\}\subset\Dm$ такие, что
 $(z_{p}-\xi_{p})\to 0$ при $p\to\infty$, и $f(z_{p})\to\infty$.
 \item[(3)]
 \begin{itemize}
 \item[(a)] $K_{p}$ --- комплексное исчерпание
 \[
 K_{p+1}\Supset K_{p}\text{ и }\bigcup_{p=1}^{\infty}K_{p}=\Dm,
 \]
 \item[(b)] $|z_{p}-\xi_{p}|\leqslant\frac{1}{p}$,
 \item[(c)] $\left\|f_{p}\right\|_{K_{p}}\leqslant 1\,,\;|f_{p}(z_{p})|>1$.
 \end{itemize}
 \begin{alignat*}{1}
 &K_{1}\text{ --- любой }\;|z_{1}-\xi_{1}|<1\text{ и }|f_{1}(z_{1})|>1
 \geqslant\left\|f_{1}\right\|_{K_{1}}\\
 &K_{p+1}=K_{p}\cup\{\,\rho(z,\,\pd\Dm)\geqslant\frac{1}{p}\,,\;
 |z|\leqslant p\,\}\cup\{\,z_{1},\ldots,z_{p}\,\}\\
 &|z_{p}-\xi_{p}|\leqslant\dfrac{1}{p}\quad
 |f_{p}(z_{p})|>1\geqslant\left\|f_{p}\right\|_{K_{p}}
 \end{alignat*}
 \item[(4)]
 \begin{alignat*}{1}
 &\exists\,\{\,m_{p}\,\}\subset\Nm:\:
 \dfrac{1}{p^{2}}|f_{p}(z_{p})|^{m_{p}}\geqslant
 \sum_{j=1}^{p-1}\dfrac{1}{j^{2}}|f_{j}(z_{j})|^{m_{j}}+p\\
 &f:=\sum_{p=1}^{\infty}\dfrac{(f_{p}(z_{p}))^{m_{p}}}{p^{2}}\in\ODm\\
 &\forall\,p\;f(z_{p})\geqslant\left(\dfrac{1}{p^{2}}
 |f_{p}(z_{p})|^{m_{p}}-
 \sum_{j=1}^{p-1}\dfrac{1}{j^{2}}|f_{j}(z_{j})|^{m_{j}}\right)-
 \sum_{j=p+1}^{\infty}\dfrac{1}{j^{2}}\geqslant p-\dfrac{\pi^{2}}{6}
 \end{alignat*}
 \end{itemize}
 \begin{imp}
 Если есть барьер в плотном множестве на $\pd\Dm$, то  \D --- область
 голоморфности.
 \end{imp}
 \begin{imp}       \label{следствие::22}
 Если \D --- выпукла, то \D --- область голоморфности.
 \end{imp}
 \begin{lemma}
 \begin{alignat*}{1}
 &\forall\;\Rm\text{--лин. }\Cm^{n}\simeq\Rm^{2n}\longrightarrow\Rm\quad l(x,y)\\
 &\exists\;\Cm\text{--лин. }\Cm^{n}\longrightarrow\Rm\quad L(z)
 \text{ такое, что }l(x,y)=2\!\cdot\!\textup{Re}\,L(x+iy)
 \end{alignat*}
 \end{lemma}
 --во
 Пусть
 \[
 l(x,y)=\sum(a_{j}x_{j}+b_{j}y_{j})=
 2\!\cdot\!\text{Re}\left\{\sum
 \left(\dfrac{a_{j}-ib_{j}}{2}\right)(x_{j}+iy_{j})\right\}
 \]
 Тогда
 \[
 L(z)=\sum\left(\dfrac{a_{j}-ib_{j}}{2}\right)z_{j}\,.
 \]
 \noindent\emph{Доказательство следствия \ref{следствие::22}: }
 $\forall\,\xi\in\pd\Dm$ есть опорная гиперплоскость
 \[
 \Gamma_{\xi}=\{\,2\!\cdot\!\textup{Re}\,L(z)-d=0\,\},
 \]
  следовательно,
  $f(z)=\dfrac{1}{L(z)-d}$ голоморфна в \D и даёт барьер в $\xi$.
 \begin{imp}
 Если $\Dm=\Dm_{1}\times\ldots\times\Dm_{n}$, где $\Dm_{j}\in\Cm$, то \D ---
 область голоморфности.
 \end{imp}
 --во
 \[
 \dfrac{1}{(z_{1}-\xi_{1})\ldots(z_{n}-\xi_{n})}\text{ даёт барьер в }
 (\,\xi_{1}\ldots\xi_{n}\,) .
 \]

 \par\bigskip\textbf{Лекция++}\par
 \section{Биголоморфные отображения.}
 \begin{df}
 Взаимно--однозначное отображение
 $f:\Dm_{1}\overset{\text{на}}{\longrightarrow}\Dm_{2}$   называется
 биголоморфным, если обратное
 $f^{-1}:\Dm_{2}\longrightarrow\Dm_{1}$ голоморфно.
 (Голоморфность отображения --- это голоморфность его координат.)
 \end{df}
 Так же, как и при $n=1$,  вводим понятие эквивалентности  и для любой \D
 определяем группу автоморфизмов $\Aut\Dm$.
 \begin{stm}
 Если $\Dm_{1}\simeq\Dm_{2}$, то $\Aut\Dm_{1}\simeq\Aut\Dm_{2}$.
 \end{stm}
 Наша ближайшая цель показать, что шар $B=\{\,|z|^{2}+|w|^{2}<1\,\}$ и бидиск
 $\varDelta=\{\,|z|<1,\,|w|<1\,\}$ не эквивалентны. То есть, теорема Римана совсем не выполняется.
 \begin{theorem}[А. Картан]
 Пусть $\Dm\subset\Cm^{\,n}$ --- ограниченная область, $\xi\in\Dm,\linebreak
%!!!!!!!!!!!!!ПЕРЕНОС!!!!!!!!!!!!!!!!!!!!!!!!!!!!!!!!!!!!!!!!!!!!!!!!!!!!!!!!!!!
 f\in\Aut\Dm$ такая, что $f(\xi)=\xi$ и $f'(\xi)=\mathrm{Id}$, тогда $f$ ---
 тождественное отображение.
 \end{theorem}
 --во
 Выберем координаты в $\Cm^{\,n}$  так, что $\xi$ --- начало. Положим
 $f^{(1)}(z)=f(z)\,$ и $\,f^{(m+1)}(z)=f\left(\,f^{(m)}(z)\,\right)$ --- $m$-тая
 итерация,
%!!!!!!!!!!!!!ПЕРЕНОС!!!!!!!!!!!!!!!!!!!!!!!!!!!!!!!!!!!!!!!!!!!!!!!!!!!!!!!!!!!
 $f^{(m)}\in\Aut\Dm,\\f^{(m)}(0)=0,\;(f^{(m)})'/_{0}=\mathrm{Id}$.  Рассмотрим
 разложение $f$ в ряд в 0.
 \[
 f(z)=z+P_{k}(z)+P_{k+1}(z)+\ldots
 \]
 $P_{j}=(P_{j}^{1},\ldots,P_{j}^{n}),\;P_{j}^{l}$ --- однородный многочлен
 степени j, k --- номер первого ненулевого нелинейного члена (допустим, что
 такой есть). Тогда
 \[
 f^{(m)}(z)=(z+P_{k}(z))+P_{k}(z+P_{k}(z))+\ldots=z+2P_{k}(z)+\ldots\:,
 \]
 и вообще
 \[
 f^{(m)}(z)=x+mP_{k}(z)+\ldots
 \]
 Пусть $P_{k}(z)$ содержит в одной из координат некий
 $c_{m_{1}\ldots m_{n}}\ne 0$. Пусть область $\Dm\Subset\varDelta(0,R)$ и ряд
 сходится в $\varDelta(0,r)$, тогда из неравенств Коши получаем
 \[
 \forall\,m\quad|m\!\cdot\! c_{ m_{1}\ldots m_{n} }|\leqslant\dfrac{R}{r^{k}}
 \]
 --- противоречие.
 \begin{df}
 Функция $\left\|\,\mathbf{\cdot}\,\right\|:\,\Cm^{\,n}\longrightarrow\Rm_{+}$
 называется \C--нормой, если
 \begin{align*}
 &(1) &&\|a+b\|\leqslant\|a\|+\|b\|\quad\forall\,a,b\in\Cm^{n} \\
 &(2) &&\|\lambda\|=|\lambda|\!\cdot\!\|a\| \quad\forall\,a\in\Cm^{n},\;\lambda\in\Cm\\
 &(3) &&\|a\|=0\Longleftrightarrow a=0
 \end{align*}
 \end{df}
 --ример
 \begin{align*}
 &(1) &&\|a\|_{1}= \sqrt{\sum_{j=1}^{n}|a_{j}|^{2}}\\
 &(2) &&\|a\|_{2}= \max |a_{j}|\\
 &(3) &&\|a\|_{3}= \sum_{j=1}^{n}|a_{j}|\\
 &(4) &&\|a\|_{4}= \max\{\text{Re}|a_{j}|,\,\text{Im}|a_{j}|\}
 \end{align*}
 \textsf{Вопрос:} Как выглядят соответствующие шары?
 \begin{theorem}[новая форма принципа максимума]
 Пусть $\Dm\subset\Cm^{\,n}$  область, $f:\,\Dm\longrightarrow\Cm^{\,m}$ ---
 голоморфное отображение,  $\left\|\,\mathbf{\cdot}\,\right\|$ --- $\Cm$--норма в
 $\Cm^{\,m},\;a\in\Dm$ и  $\left\| f \right\|$ имеет в точке $a$ локальный
 максимум. Тогда
 \begin{align*}
 &(1) &&\exists\,(\lambda_{1},\ldots,\lambda_{n})\ne 0\text{ такие, что }\:
 \sum_{j=1}^{n}\lambda_{j}f_{j}(z)\equiv\const \text{ в } \Dm ;\\
 &(2) &&\left\|f(z)\right\|\equiv\const \text{ в } \Dm .
 \end{align*}
 \end{theorem}
 --во
 $b=f(a),\; B=\{\,w\in\Cm^{\,n}:\,\|w\|<\|b\|\,\}$. Этот шар --- выпуклое
 множество (мы предполагаем $b\ne 0$, иначе всё ясно). Пусть точка
 $b\in\pd\Dm$. Проведём опорную плоскость
 $\Gamma\ni b,\;B\subset\Gamma_{-}$. Пусть $\Gamma=\{\,\text{Re}\,l(z)=d\,\}$,
 на $\Gamma\quad\text{Re}\,l(z)-d<0$. Рассмотрим на \D голоморфную функцию
 $\exp (\,l(f(z))\,)$. В точке $a$ её модуль равен $e^{d}$,  а на \D  её модуль
 меньше $e^{d}$, следовательно
 \[
 \exp (\,l(f(z))\,)=\const\:\Longrightarrow\:l(f(z))=\const\:\Longrightarrow\:
 f:\,\Dm\longmapsto\overline{B}\cap\Gamma\:\Longrightarrow\:\|f\|=\const
 \]
 \begin{theorem}[Новая версия леммы Шварца]
 \begin{alignat*}{1}
 &\Cm^{\,n}\qquad\|\quad\|_{1}\qquad B_{1}\text{ --- единичный шар}\\
 &\Cm^{\,m}\qquad\|\quad\|_{2}\qquad B_{2}\text{ --- единичный шар}\\
 &f:\,B_{1}\longrightarrow B_{2}\qquad 0\longrightarrow 0
 \end{alignat*}
 \end{theorem}
 --во
 Пусть $a\in\pd B_{1}$, проведём через 0 и $a$  комплексную прямую
 $z=at,\;t\in\Cm$. Пересечению соответствует $|\,t\,|<1$. Рассмотрим
 \[
 g:\,\Cm^{\,1}\longrightarrow\Cm^{\,m},\quad g(t)=\dfrac{f(at)}{t}
 \]
 в круге $\{\,|\,t\,|\leqslant r,\,0<r<1\,\}$. По принципу максимума
 \[
 \|g(t)\|_{2}\leqslant\dfrac{1}{r}\:\Longrightarrow\:
 \|g(t)\|_{2}\leqslant 1\:\Longrightarrow\:\|f(at)\|_{2}\leqslant |\,t\,|
 \quad\forall\,t,\,|\,t\,|<1.
 \]
 Пусть $z$ --- произвольная точка из $B_{1}$, тогда
 $\frac{z}{\,\,\|z\|_{1}\!\!}\in\pd B_{1}$, положим
 $a=\frac{z}{\:\|z\|_{1}\negmedspace}$, и $t=\|z\|_{1}$. Получаем
 $\|f(z)\|_{2}\leqslant\|z\|_{1}$.\par
 Рассмотрим подгруппу $\text{LF}\subset\Aut B$ , состоящую из дробно--линейных
 ($\sim$ проективных) автоморфизмов шара. Ясно, что при этом
 ${\pd B=\{\,|z|^{2}+|w|^{2}=1\,\}}$ переходит в себя
 \[
 \underset{(z,w)}{\Cm^{\,2}}\hookrightarrow
 \underset{(T,Z,W)}{\Cm\mathbb{P}^{2}}=
 \Cm^{\,3}\diagup_{(\ze\longmapsto\varepsilon\ze,\,
                                             \varepsilon\in\Cm\setminus\{0\})}.
 \]
 В проективных координатах сфера выглядит так:
 \[
 -\left|T\right|^{2}+\left|Z\right|^{2}+\left|W\right|^{2}=0\qquad
 \left(z=\dfrac{Z}{T},\,w=\dfrac{W}{T}\right)
 \]
 Этот конус делит $\Cm^{\,3}$ на $Q_{-},\: Q_{0}$ и $Q_{+}$.
 \begin{alignat*}{1}
 &Q_{-}=\{\,(T,Z,W)\in\Cm^{\,3}:
 -\left|T\right|^{2}+\left|Z\right|^{2}+\left|W\right|^{2}<0\,\}\\
 &Q_{0}=\{\,(T,Z,W)\in\Cm^{\,3}:
 -\left|T\right|^{2}+\left|Z\right|^{2}+\left|W\right|^{2}=0\,\}\\
 &Q_{+}=\{\,(T,Z,W)\in\Cm^{\,3}:
 -\left|T\right|^{2}+\left|Z\right|^{2}+\left|W\right|^{2}>0\,\}\\
 \end{alignat*}
 $Q_{-}$ --- это шар, $Q_{0}$ --- сфера, $Q_{+}$ --- внешность шара. \par
 Пусть $\ze=(T,Z,W),\quad\langle\ze,\overline{\ze}\rangle=
 -\left|T\right|^{2}+\left|Z\right|^{2}+\left|W\right|^{2}$.
 Всякое линейное преобразование $\Cm^{\,3}$, переводящее $Q_{0}$ в себя, имеет
 вид
 \[
 \lambda U,\text{ где }\lambda\in\Cm\setminus\{0\},\quad U\in\mathsf{U}(2,1)
 \]
 --- гр. преобразование, сохраняющее
 $\langle\mathbf{\cdot},\mathbf{\cdot}\rangle$ (тогда
 $\langle\lambda U\ze,\lambda\overline{U\ze}\rangle=
 |\lambda|^{2}\langle\ze,\overline{\ze}\rangle$). Однородное растяжение
 даёт на $\Cm\mathbb{P}^{2}$ тождественное действие, поэтому $c\lambda$ можно
 убрать. Более того, заменой $U\longmapsto\varepsilon U$ можно добиться условия
 $\det U =1$, то есть $U\in\mathsf{SU(2,1)}$. Теперь подгруппа в
 $U\in\mathsf{SU}(2,1)$, действующая  на $\Cm\mathbb{P}^{2}$ тождественно, это
 \[
 \Zm_{3}\simeq\{\,\ze\longmapsto\varepsilon\ze,\;\varepsilon^{3}=1\,\}
 \]
 Таким образом
 \[
 \text{LF}=\mathsf{SU}(2,1)\diagup_{Z_{3}}\text{ и }
 \dim\text{LF}=\dim\mathsf{SU}(2,1)=3^{2}-1=8\,.
 \]
 Есть простое описание действия $\mathsf{SU}(2,1)$: определим
 "ортонормированный"  репер как
 \[
 (e_{0},e_{1},e_{2}):\quad \langle e_{i},\overline{e}_{j}\rangle=
 \begin{cases}
 0\,, &\text{если}\quad i\ne j\\
 -1\,, &\text{если}\quad i=j=0\\
 1\,, &\text{если}\quad i=j\ne 0
 \end{cases}
 \]
 Тогда для любой пары таких реперов существует единственное преобразование из
 $\mathsf{U}(2,1)$, переводящее первый во второй. Если мы хотим получить
 аналогичное описание $\mathsf{SU}(2,1)$, то надо добавить условие
 $\det(e_{0},e_{1},e_{2})=1$. При этом ясно, что
 $e_{0}\in Q_{-};\;e_{1},\,e_{2})\in Q_{+}$ и что для любой пары
 $e_{1},\,\widetilde{e}_{1}\quad \exists\,U\in\mathsf{SU}(2,1)$, то есть
 $Ue_{1}=\widetilde{e}_{1}$ (в таком случае говорят, что действие LF на $B$
 транзитивно).
 \begin{lemma}
 $\Aut B=\mathrm{LF}$.
 \end{lemma}
 --во
 Рассмотрим $f\in\Aut B,\;f(0)=\xi$; так как существует $L\in\mathrm{LF}$ такое,
 что $L\xi=0$, то, заменяя $f$ на $L\circ f$, можно считать, что $f(0)=0$.
 Применяя лемму Шварца к $\eta=f(\ze)$, получаем $f(\ze)\leqslant\ze$,
 а применяя к $\ze=f^{-1}(\eta)$ получаем, что  $f^{-1}(\eta)\leqslant\eta$,
 или $\ze\leqslant f(\ze)$, то есть $|f(\ze)|\equiv|\ze|$. Пусть
 \[
 f(\ze)=A\ze+P_{k}(\ze)+\ldots\,,\qquad A\ze\text{ --- невырожденное
 линейное преобразование.}
 \]
 Записывая тождество $|A\ze+P_{k}(\ze)+\ldots|^{2}=|\ze|^{2}$
 покомпонентно, получаем
 \begin{alignat*}{1}
 &(x,\overline{y})=\sum x_{j}\overline{y}_{j}\\
 &(\,A\ze,\overline{A\ze}\,)+(\,A\ze,\overline{P_{k}\ze}\,)+
 (\,P_{k}\ze,\overline{A\ze}\,)+\ldots
 \end{alignat*}
 Отсюда, приравнивая компоненты одинаковых бистепеней, получаем
 \begin{align*}
 &(1,\overline{1}) &&(\,A\ze,\overline{A\ze}\,)=
                                                (\,\ze,\overline{\ze}\,)\\
 &(k,\overline{1}) &&(\,P_{k}\ze,\overline{A\ze}\,)=0\:\Longrightarrow\:
 A^{*}P_{k}\ze=0\:\Longrightarrow\:P_{k}\ze=0,
 \end{align*}
 то есть $f(\ze)\equiv A\ze\in\mathsf{U}(2)$.
 --тд
 Рассмотрим $\Aut\varDelta^{2}$. Эта группа содержит подгруппу
 \[
 \left\{\,z\longmapsto e^{i\theta}\dfrac{w-b}{1-\overline{b}w}\,\right\}=
 \Aut\varDelta_{z}\times\Aut\varDelta_{w},
 \]
 а также $\Zm_{2}=\{\,\text{Id};\;(z\longmapsto w,\,w\longmapsto z)\,\}$.
 \begin{lemma}
 $\Aut\varDelta^{2}=\Aut\varDelta\times\Aut\varDelta\times\Zm_{2}$.
 \end{lemma}
 --во
 Пусть $f\in\Aut\varDelta^{2}$; заменяя $f$ на $L\circ f$, где
 $L\in\Aut\varDelta_{z}\times\Aut\varDelta_{w}$, можно считать, что $f(0)=0$.
 Пусть $\|\ze\|$ --- это $\max\{\,|z|,\,|w|\,\}$; применяя лемму Шварца  к
 прямому и к обратному отображению, получаем $\|f(\ze)\|=\|\ze\|$. Разложим
 $f$ в ряд в 0; пусть он сходится в $\varDelta(0,r)$. Возьмём в
 $\varDelta(0,r)$ точку $\widetilde{\ze}$ такую, что
 \[
 \widetilde{\ze}=(\widetilde{z},\widetilde{w}) \,\text{ и }\,
 |\widetilde{z}|\ne|\widetilde{w}|,\text{ а также }\,
 f(\widetilde{\ze})=\left(f_{1}(\widetilde{\ze}),f_{2}(\widetilde{\ze})\right)
 \,\text{ и }\,|f_{1}(\widetilde{\ze})|\ne|f_{2}(\widetilde{\ze})| ,
 \]
 после преобразования из $\Zm_{2}$ можно считать, что
 \[
 |\widetilde{z}|>|\widetilde{w}|\,\text{ и }\,
 |f_{1}(\widetilde{\ze})|>|f_{2}(\widetilde{\ze})|\:\Longrightarrow\:
 |f_{1}(z,w)|^{2}\equiv|z|^{2}
 \]
 в окрестности $\ze$, следовательно, $f_{1}(z)\equiv z\!\cdot\!e^{i\ph}.$
 \par
 Аналогично покажем, что $f_{1}(z)\equiv z\!\cdot\!e^{i\ph}$ ($z$ не может
 быть, так как иначе необратимо).
 \[
 \dim\Aut\varDelta=3\:\Longrightarrow\:\dim\Aut\varDelta^{2}=6.
 \]
 \begin{imp}
 $\Aut B\not\simeq\Aut\varDelta^{2}$.
 \end{imp}
 Аналогичные рассуждения показывают, что для произвольного $n$
 \begin{align*}
 &\dim\Aut B^{n}=n^{2}+2n\\
 &\dim\Aut \varDelta^{n}=3n
 \end{align*}
 \begin{flushleft}
 \large{\emph{"Простое" доказательство.}}
 \end{flushleft}
 Пусть
 \[
 f:\,B^{n}\longrightarrow\varDelta^{n},\quad \|z\|_{1}=\sqrt{\sum|z_{j}|^{2}},
 \quad \|z\|_{2}=\max |z_{j}| .
 \]
 Используя дробно--линейные отображения, можно считать, что $f(0)=0$ и,
 применяя лемму Шварца к $f$ и к $f^{-1}$ получаем, что
 \[
 \forall\,z\in B^{n}\quad \|f(z)\|_{2}\equiv\|z\|_{1},
 \]
 следовательно, образ гладкой поверхности (сферы) $\|z\|_{1}=\frac{1}{2}$ есть
 негладкая поверхность $\|w\|_{2}=\frac{1}{2}$ (граница полидиска).
 Противоречие. --тд

 \par\bigskip\textbf{Лекция++}\par
 \section{Мероморфные функции на римановых поверхностях.}
 Риманова поверхность (здесь) = одномерное комплексное многообразие.
 --римеры
 \begin{align*}
 &(1) &&\Cm=\Cm\mathbb{P}^{1}\:\text{ --- сфера}\\
 &(2) &&\Cm\diagup_{\negmedspace\{w_{1},w_{2}\}}\quad
               (\,\mathrm{Im}\dfrac{w_{1}}{w_{2}}\ne 0\,)\:\text{ --- тор}\\
 &(3) &&\Cm\diagup_{\negmedspace\{z\longmapsto 2z\}}\:\text{ --- тор}
 \end{align*}
 --адача
 Указать карты, проверить их согласованность.
 \begin{lemma}[была раньше, лекция 6 первого семестра]
 Пусть $a$ --- изолированная особая точка,
 $f\in\mathcal{O}(\overset{\text{\huge{.}}}{\mathcal{U}}_{a}),\;
 \ph\in\mathcal{O}(\mathcal{U}_{a}),\;\ph(a)=a,\;\ph\,'(a)\ne 0$,
 тогда $a$ --- изолированная особая точка для $f\left(\ph(z)\right)$, того
 же типа, и если $a$ --- не существенно особая, то
 $\ord f\!\circ\!\ph=\ord f$.
 \end{lemma}
 Таким образом, тип особой точки и порядок в ней не зависит от карты, в
 частности, можно говорить о голоморфности, о мероморфности функции на
 римановой поверхности. \par
 Кроме функций, нас будут интересовать дифференциальные формы. \par
 С вещественной точки зрения, риманова поверхность --- двумерное вещественное
 гладкое многообразие, поэтому там есть формы степени 1 и 2. В локальной
 координате $z$ 1--формы имеют вид $Pdz+Qd\overline{z}$, а 2--формы ---
 $Rdz\wedge d\overline{z}$. Если 1--форма имеет вид $Pdz$, где $P$ ---
 голоморфная (мероморфная) функция, то форму называем голоморфной
 (мероморфной).\par
 Это определение не зависит от локальной координаты. Действительно, если
 $w=P(z)dz$ --- голоморфная (мероморфная) 1--форма, то, записывая её в
 координате $\ze,\; z=\ph(\ze)$, получаем
 \[
 w=P\left(\ph(\ze)\right)\ph\,'(\ze)d\ze
 \]
 --- голоморфную (мероморфную) форму. Более того, можно корректно определить
 \[
 \ord_{a} w:=\ord_{a} P=\ord_{a}(P\circ\ph)\!\cdot\!\ph'
 \]
 порядок мероморфной 1--формы в точке.\par
 Примером 1--формы служит дифференциал.  Дифференциал голоморфной функции ---
 голоморфная форма, дифференциал мероморфной функции --- мероморфная форма.\par
 Дифференциальные формы можно интегрировать $\int_{\ga}w$. Если $w$ ---
 голоморфная в проколотой окрестности точки $a$ 1--форма, то можно определить
 вычет
 \[
 \res_{a} w:=\dfrac{1}{2\pi i}\underset{\negmedspace\odot}{\int}w%_{\odot}w
 \]
 \begin{theorem}
 Пусть \D --- область, $X$ --- риманова поверхность, $\Dm\subset X,\;w$ ---
 голоморфная в \D 1--форма, кривая $\ga_{1}$ гомотопна в \D кривой
 $\ga_{2}$, тогда
 \[
 \int_{\ga_{1}}\negthickspace\negthickspace w=
 \int_{\ga_{2}}\negthickspace\negthickspace w.
 \]
 \end{theorem}
 --во
 Разбить на куски и свести к плоской теореме.
 \begin{theorem}
 Пусть \D --- область, $X$ --- риманова поверхность, $\Dm\subset X,\;w$ ---
 голоморфная и непрерывная  в \D 1--форма, тогда
 \[
 \res_{a} w=\res_{a} f .
 \]
 \end{theorem}
 --во
 Почленным интегрированием ряда
 \[
 f(z)=\sum_{n=-\infty}^{+\infty}c_{n}z^{n} .
 \]
 \begin{theorem}[о вычетах]
 Пусть  $X$ --- риманова поверхность, \D --- область с компактным замыканием и
 кусочно--гладкой границей, $f$ --- голоморфная в \D и непрерывная в
 $\overline{\Dm}$ функция, за исключением конечного числа точек
 $\{\,a_{j}\,\}\subset\Dm$, тогда
 \[
 \int_{\pd\Dm}\negthickspace\negthickspace w=2\pi i\sum_{j}\res_{a_{j}}w
 \]
 \end{theorem}
 \begin{imp}[принцип аргумента]
 \end{imp}
 Если  $f:\,X_{1}\longrightarrow X_{2}$, то голоморфность (мероморфность)
 означает голоморфность (мероморфность) в рамках карты.
 \begin{center}
 \textit{* * * * * КАРТИНОЧКА * * * * *}
 \end{center}
%*******************КАРТИНОЧКА**************************************************
 \begin{stm}
 $f$ --- мероморфна $\:\Longleftrightarrow\:\widehat{f}$ --- голоморфна.
 \end{stm}
 Интересный пример многозначной функции на торе
 $\Cm\diagup_{\{w_{1},w_{2}\}}\,$ без особых точек: $f(z)=z$. \par
 Как всякое двумерное ориентируемое компактное многообразие риманова
 поверхность $X\simeq Mg$ --- сфере с $g$ ручками. У $Mg$ есть каноническое
 рассечение   $\;(\,a_{1},\ldots,a_{g},b_{1},\ldots,b_{g}\,)$.
 \begin{center}
 \textit{ КАРТИНОЧКА }
 \end{center}
%*******************КАРТИНОЧКА**************************************************
 $Mg$ превращается в $4g$--угольник со сторонами
 $a_{1}^{+},b_{1}^{+},a_{1}^{-},b_{1}^{-},\ldots,
                                     a_{g}^{+},b_{g}^{+},a_{g}^{-},b_{g}^{-}$.
 Эти $2g$ циклов --- образующие $\pi_{1}(X)$.\par
 Для любой мероморфной формы $w$ определены периоды
 \[
 A_{j}=\int_{a_{j}}\negthickspace\negthickspace w\,,\quad B_{j}=
 \int_{b_{j}}\negthickspace\negthickspace w\,.
 \]
 (предполагается, что циклы не проходят через особые точки).

 --адача
 $w=df$, где $f$ мероморфна на $X\:\Longleftrightarrow\:A_{j}=B_{j}=0,\;
 j=1,\ldots,g$.
 \section{Дивизоры.}
 \begin{df}
 Дивизор \D --- это формальная конечная линейная комбинация
 \[
 \sum_{j}m_{j}a_{j},\:\text{ где }a_{j}\in X,\,m_{j}\in \Zm.
 \]
 Степень
 \[
 \deg\Dm:=\sum_{j}m_{j}
 \]
 \end{df}
 --римеры
 \begin{itemize}
 \item[(1)] $\quad(\,f\,)$ --- дивизор мероморфной функции на компактном $X$.
 \item[(2)] $\quad(\,w\,)$ --- дивизор мероморфной формы.
 \end{itemize}
 Дивизоры образуют группу $\mathnormal{Div}(X)$.
 \begin{stm}
 Для любой мероморфной $f\quad\deg(f)=0\\ (X$ --- компактное$)$.
 \end{stm}
 --во
 \[
 \deg(f)=\sum\text{нулей}-\sum\text{полюсов}=
 \dfrac{1}{2\pi i}\underset{\varnothing}{\int}\dfrac{df}{f}=0.
 \]
 \begin{stm}
 $\forall\,w_{1}\not\equiv 0,\,w_{2}$ существует мероморфная функция $f$
 такая, что $w_{2}=f\circ w_{1}$.
 \end{stm}
 --во
 Локально $w_{j}=f_{j}dz$; положим $f=\frac{f_{2}}{f_{1}}$; при замене
 координат выражение не изменится, то есть $f$ определена корректно.
 \begin{imp}
 $\forall\,w_{1},\,w_{2}\quad\deg w_{1}=\deg w_{2}$.
 \end{imp}
 --римеры
 \begin{align*}
 &(1) &&\oCm\quad w=dz\;\deg w=-2\;
 \text{ (в бесконечности полюс второго порядка)}\\
 &(1) &&\Cm\diagup_{\{w_{1},w_{2}\}}\quad w=dz\;\deg w=0\;
 \text{ (нет особенностей и нулей).}
 \end{align*}
 Можно доказать, что эта степень равна $2g-2$.

 \par\bigskip\textbf{Лекция++}\par
 --римерь\textsf{А}.
 Пусть $X=\oCm$, $a_{1},\ldots,a_{s}=\infty$ --- полюса кратностей
 $m_{1},\ldots,m_{s}$ соответственно. Тогда произвольная мероморфная функция с
 такими данными имеет вид
 \[
 f(z)=c_{0}+\sum_{j=1}^{m_{s}}c_{j}z^{j}+\sum_{i=1}^{s-1}\sum_{k_{i}=1}^{m_{i}}%
 \dfrac{ c_{ i k_{i} } }{ (z-a_{i})^{ k_{i} } }
 \]
 следовательно, размерность пространства равна
 $\left(\sum_{j=1}^{s}m_{j}\right)+1$.

 \begin{df}
 Назовём дивизор положительным, если $D=\sum m_{j}a_{j}$, где
 $m_{j}\geqslant 0$; пишем $D\geqslant 0$.
 \end{df}
 \begin{df}
 Для любого дивизора определим пространство у--ра
 \[
 \ell(D)=\left\{\,f\in\mathcal{M}(X):\,\text{либо }f=0,\,
 \text{либо }\left(f\right)+D\geqslant 0\,\right\}
 \]
 \end{df}
 \begin{df}
 Говорим, что дивизоры эквивалентны $(\,a\sim b\,)$, если
 $b=a+\left(f\right)$.
 \end{df}
 \begin{stm}
 Пусть $X$ --- компактное, $a$ и $b$ --- дивизоры, тогда
 \begin{itemize}
 \item[(1)] $\ell(a)$ --- линейное пространство,
 $\dim a= \dim\ell(a)<\infty$.
 \item[(2)] Если $a\sim b$, то $\ell(a)\simeq\ell(b)$.
 \end{itemize}
 \end{stm}
 --во
 Пусть $a=\sum m_{j}x_{j}-\sum n_{j}y_{j},\;m_{j}>0,\;n_{j}\geqslant 0$, тогда
 \begin{alignat*}{3}
 & &&(i) &&f\text{ голоморфна на }X-\{\,x_{1},\ldots,x_{p}\,\}\\
 &f\in\ell(a)\quad\Longleftrightarrow\qquad &&(ii)
                 &&f\text{ имеет в }x_{j}\text{ полюс порядка не выше }m_{j}\\
 & &&(iii)\quad &&f\text{ имеет в }y_{j}\text{ нуль порядка не ниже }n_{j}
 \end{alignat*}
 Ясно, что это линейное пространство. Конечномерность следует из того, что
 число свободных коэффициентов в главных частях равно $\sum m_{j}$ и того, что
 без полюсов есть только $\const$.\par
 Если $a\sim b$, то есть, $b=h a(h),\;h\in\mathcal{M}(X)$, то отображение
 $\psi:\,\ell(a)\longrightarrow\ell(b)$ вида $\psi(f)=h\!\cdot\!f$.
 (Если $\left(f\right)+a\geqslant 0$, то
 $\left(h f\right)+a-\left(h\right)\geqslant 0$.)
 \begin{theorem}[Римана--Роха]
 Пусть $X$ --- компактная рода $g$, $a$ --- произвольный дивизор, тогда
 $w$ --- форма
 \[
 \dim a=\deg a+1-g+\dim\,(\,(w)-a).
 \]
 \end{theorem}
 \begin{note}
 Если $w\longmapsto fw$, то
 \[
 (w)-a\,\sim\,\left(f w\right)-a,\text{ так как } =(w)-a+\left(f\right).
 \]
 \end{note}
 \begin{imp}
 Если $g=0$, то $X\sim\oCm$.
 \end{imp}
 --во
 Рассмотрим  $a=1\!\cdot\!x_{0},\;x_{0}\in X$, тогда
 %\begin{alignat*}{1}
 \[\dim a\geqslant 1+1-0=2\;\Longrightarrow\;\exists\,f\in\ell(a)\ne\const
 \;\Longrightarrow\;\ord_{x_{0}}f=-1\]
 %\end{alignat*}
 и $f:\,X\longrightarrow\oCm$ взаимнооднозначно.
 \begin{imp}
 Если $D:=\:$каноническому у--ру $C=C w$, то
 \[
 \dim C = ( 2 g - 2 ) - g + 1 + 1 = g.
 \]
 \end{imp}

 \end{document}

% ======================================================================
% Done part
  \begin{df}
 Пусть
 $\De=\frac{\pd^{2}}{\pd x_{1}^{2}}+\dots+
 \frac{\pd^{2}}{\pd x_{n}^{2}}$ --- оператор Лапласа, $u$ --- класса
 $C^{2}$ в $\Dm\subset\mathbb{R}^{n}$ называется гармонической, если всюду в \D
 $\De u=0$. Множество гармонических функций обозначается \HD.
 \end{df}
 Если $n=2$ и $\mathbb{R}^{n}\cong\mathbb{C}$, то между гармоническими функциями двух переменных и голоморфными
 функциями есть тесная связь.
 \begin{note}
 \HD --- линейное пространство.
 \end{note}
 \begin{note}
 Если $n=1$, то гармоничность --- это линейность.
 \end{note}
 \begin{note}
 Задача о форме мембраны.
 \end{note}
 Нетрудно заметить, что
 \[
 \De=\left(\frac{\pd}{\pd x}-i\frac{\pd}{\pd y}\right)
 \left(\frac{\pd}{\pd x}+i\frac{\pd}{\pd y}\right)=
 4\frac{\pd^{2}}{\pd z \pd \overline{z}}.
 \]
 \begin{theorem}
 \begin{itemize}
 \item[(1)] Если $f\in\ODm$, то $u=\text{Re}f\in\HDm$.
 \item[(2)] Если \D --- односвязна, $u\in\HDm$, то $\exists\,v\in\HDm$ такая,
 что $(u+iv)\in\ODm$.
 \end{itemize}
 \end{theorem}
 --во
 \begin{itemize}
 \item[(1)]  $4\frac{\pd^{2}}{\pd z \pd \overline{z}}(u)=
 2\frac{\pd^{2}}{\pd z \pd \overline{z}}(f+\overline{f})=0$
 \item[(2)] Пусть ${g=u_{x}-iu_{y}}$, тогда $g\in\ODm$, так как
 \[
 {\frac{\pd g}{\pd \overline{z}}}={\frac{1}{2}
 \left(\frac{\pd}{\pd x}-i\frac{\pd}{\pd y}\right)
 \left(\frac{\pd}{\pd x}+i\frac{\pd}{\pd y}\right)}=
 \frac{1}{2}\De u=0.
 \]
 В односвязной \D у $g$ есть первообразная $f$, то
 есть ${f'=g\,,}\:{f=U+iV}$. Но
 \[
 u_{x}-iu_{y}=\frac{1}{2} \left(\frac{\pd}{\pd x}-i\frac{\pd}{\pd y}\right)=
 \text{[вследствие уравнения Коши--Римана]}=U_{x}-iU_{y}
 \]
 $U$ отличается от $u$ на постоянную, которую можно выбрать равной 0.
 \end{itemize}
 (Другое доказательство: рассмотрим 1--форму
 $\om = -u_{y}dx+iu_{x}dy$, тогда ${d\om=\De u\,dz\wedge dy}=0$ и,
 следовательно, $\int_{A}^{B}\om$ не зависит от пути. Положим
 $v(z)=\int_{A}^{z}\om$. Вообще, $\Omega=Pdx+Qdy$ даёт решение
 $v_{x}=P,\:v_{y}=Q$ при условии $P_{y}=Q_{x}$.)
 \begin{note}
 Если $f$ голоморфна, то Im$f$ также является гармонической.
 \end{note}
 \begin{note}
 Если \D не односвязна, то возникает многозначность, то есть обычной
 гармоничности нет.
 \end{note}
 --ример
 $u=\ln |z|$ гармоническая в $\{\,0<|z|<1\,\}$. Если сделать разрез, то
 сопряжение имеет вид $v=$arg$z+C$ и среди них нет однозначной в кольце.
 \subsection{Свойства гармонических функций}
 \begin{description}
 \item[1] Если $u\in\HDm$, то $u\in\mathcal{C}^{\infty}(\Dm)$.
 \item[2. Теорема о среднем] Если $u\in\HDm$ и $\varDelta(z,R)\subset\Dm$, то
 \[
 u(z)=\frac{1}{2\pi}\int_{0}^{2\pi}u(z+Re^{i\ph})d\ph=
 \frac{1}{\pi R^{2}}\underset{\negthickspace\negthickspace\varDelta}%
 {\int\negthickspace\negthickspace\int}u\,d\sigma
 \]
 где $d\sigma=dx\wedge dy$.
 --адача
 \[
 u\in \mathcal{C}^{2}(\Dm)\:+\:\text{теорема о среднем}\:\Longrightarrow\:u\in\HDm
 \]
 \item[3. Теорема единственности] Пусть $u_{1},u_{2}\in\HDm$ и $V$ --- открытое
 подмножество \D. Если $u_{1}=u_{2}|_{V}$, то $u_{1}\equiv u_{2}$.\par
 Рассмотрим $W$ --- внутренние точки множества $\{\,u=u_{1}-u_{2}=0\,\}$.
 $W\ne\varnothing$ и замкнуто.
%************РИСУНОК**********************************************************
 \begin{center}
 \textit{Рисуночек....}
 \end{center}
 $\exists\,f\in\mathcal{O}(\varDelta)$ такая, что $u=$Re$f$ и на
 W$\cap\varDelta\quad f(z)\in iR\:\Longrightarrow\:u|_{\varDelta}=0,$ то есть $W$ ---
 замкнуто и, следовательно, $W=\Dm$.
 \begin{note}
 Если формулировать теорему как для голоморфных, то она неверна. Пример ---
 $u(x,y)=x\,.$
 \end{note}
 \item[4. Принцип максимума] Пусть $u\in\HDm$ и $a\in\Dm$ --- точка локального
 максимума, тогда $u\equiv\const$.
 --во\negthickspace\negthickspace
 Рассмотрим круг $\varDelta(a,R)$, где $a$ --- максимум и  $u=$Re$f$. Тогда
 $|e^{f(z)}|=e^{u}$ достигает максимума$\;\Longrightarrow\;f$ и $u$ ---
 постоянны. Тоже с (...) $u\longrightarrow -u.$
 \item[5. Теорема Лиувилля] Если $u\in\mathcal{H}(\mathbb{C})$ и ограничена сверху,
 то $u$ --- постоянна.
 --во
 Пусть $u=$Re$f$, и
 \[
 f:\,\mathbb{C}\longrightarrow\text{ полуплоскость }
 \overset{\Lambda}{\longrightarrow}\;\{\,|z|<1\,\},\;\Lambda
 \text{ --- дробно-линейное,}
 \]
 то есть $\Lambda\circ f$ --- целая и ограниченная; следовательно, $f$ и $u$ ---
 постоянные.
 \item[6] Если $f:\,\Omega\longrightarrow\Dm$ --- голоморфная и $u\in\HDm$, то
 $u(f(z))\in\mathcal{H}(\Omega)$.
 Достаточно локально\; $a\overset{f}{\longrightarrow}b$, в окрестности
 $b$\linebreak $u=$Re$F$ $F(f(z))$ --- голоморфна$\;\Longrightarrow\;u(f(z))=
 $Re$F(f(z))$ --- гармоническая.
 \end{description}
 Свойства 1--5 сохраняются для $n=2$.

%===========================

 \subsection{Задача Дирихле}
 \D --- область, $\widetilde{u}$ непрерывна на $\pd \Dm$. Требуется найти
 $u\in\HDm\cap\mathcal{C}(\overline{\Dm})$ такую, что
 $u=\widetilde{u}|_{\pd \Dm}$.
 \begin{itemize}
 \item[(1)] Если \D ограничена, то решение единственно, так как по принципу
 $\max$--$\min$ решение $u_{1}-u_{2}\quad\leqslant\,$ и $\,\geqslant 0$.\par\noindent
 Если \D неограничена, то это не так. Пример:
 $\Dm=\{\,\text{Im}z>0\,\},\:u=ky$.
 \item[(2)] Для односвязной области с простой замкнутой (жордановой) границей
 вопрос о построении решения сводится к вопросу о построении решения для круга.
%************РИСУНОК**********************************************************
 \begin{center}
 \textit{Рисуночек....}
 \end{center}
 Пусть $f:\,\Dm\longrightarrow\varDelta$ --- конформное (теорема Римана). По
 теореме Каратеодори $f$ непрерывно продолжается до гомеоморфизма $\pd\Dm$
 и $\pd\varDelta$. Перенесём граничные значения $\widetilde{u}$ с
 $\pd\Dm$ на $\pd\varDelta$ с помощью $f$, то есть
 $\widetilde{U}=\widetilde{u}\circ f^{-1}$. Пусть $U$ --- решение полученной
 задачи Дирихле в $\varDelta$. Тогда $u=U\circ f$ --- решение исходной задачи
 Дирихле в  \D.
 \item[(3)] Решение для круга $\varDelta=\{\,|z|<R\,\}$. Если бы решение и
 существовало, то можно построить голоморфную $f$ такую, что $u=$Re$f$, пусть
 $f$, к тому же, непрерывно выходит на $\pd\varDelta$. Тогда
 \[
 f(z)=\frac{1}{2\pi i}\int_{\pd\varDelta}\frac{f(\ze)}{\ze-z}d\ze
 \]
 Сделаем замену $\ze=Re^{i\ph},\:d\ze=i\ze d\ph$
 \[
 f(z)=\frac{1}{2\pi i}\int_{0}^{2\pi}\frac{f(\ze)\ze}{\ze-z}d\ph
 \]
 Пусть $z^{*}$ --- точка симм. $z$. По теореме Коши
 \[
 0=\frac{1}{2\pi i}\int_{\pd\varDelta}\frac{f(\ze)}{\ze-z^{*}}d\ze=
   \frac{1}{2\pi i}\int_{0}^{2\pi}\frac{f(\ze)\ze}{\ze-z^{*}}d\ph
 \]
 Но
 \begin{alignat*}{1}
 &\frac{\ze}{\ze-z}-\frac{\ze}{\ze-\frac{R^{2}}{\overline{z}}}=
 \frac{R^{2}-|z|^{2}}{|\ze-z|^{2}}\:\Longrightarrow\\ &\Longrightarrow
 f(z)=\frac{1}{2\pi i}\int_{0}^{2\pi}f(\ze)
 \frac{R^{2}-|z|^{2}}{|\ze-z|^{2}}d\ph\:\Longrightarrow\\
 &u(z)=\frac{1}{2\pi i}\int_{0}^{2\pi}u(\ze)
 \frac{R^{2}-|z|^{2}}{|\ze-z|^{2}}d\ph\text{ --- интеграл Пуассона.}
 \end{alignat*}
 Но
 \begin{alignat*}{1}
 &\frac{R^{2}-|z|^{2}}{|\ze-z|^{2}}=\text{Re}
 \frac{\ze+z}{\ze-z}\:\Longrightarrow\\
 &f(z)=\frac{1}{2\pi i}\int_{0}^{2\pi}u(\ze)
 \frac{\ze+z}{\ze-z}d\ph\text{ --- интеграл Шварца.}
 \end{alignat*}
 Следовательно, $u=$Re$f$ гармонична.
 \end{itemize}
 \emph{Доказательство непрерывности.}\quad
 \begin{itemize}
 \item[(1)] $P(\ze,z):=\frac{1}{2\pi i}\frac{\ze+z}{\ze-z}\;
 \int_{0}^{2\pi}P(\ze,z)d\ph=0$, так как $u\equiv 1$ --- гармоническая.
 \item[(2)] $\lim_{z\to\ze_{0}}P(\ze,z)=0$ при $\ze\ne\ze_{0}$, причём
 равномерно по $\ze$ на каждой дуге $\ga\not\ni\ze_{0}$.
 \item[(3)] Рассмотрим
 \[
 \int_{0}^{2\pi}\negthickspace\negthickspace\negthickspace u(\ze)P(\ze,z)d\ph-u(\ze_{0})=
 \int_{0}^{2\pi}\negthickspace\negthickspace\{u(\ze)-u(\ze_{0})\}P(\ze,z)d\ph
 \]
 Разобьём
 $\pd\varDelta=\ga_{1}\cup\ga_{2}\quad
 \ga_{1}=\{\,Re^{i\ph}:\,|\ph-\ph_{0}|<2\delta\,\},\:
 \ga_{2}=\pd\varDelta\setminus\ga_{1}.\;\delta$ выберем так, чтобы
 $|u(\ze)-u(\ze_{0})|<\varepsilon$ \par\noindent
 На $\ga_{1}$
 \[
 \int_{\ga_{1}}\negthickspace\{u(\ze)-u(\ze_{0})\}P(\ze,z)d\ph<
 \varepsilon\int_{\ga_{1}}\negthickspace\negthickspace P(\ze,z)d\ph<
 \varepsilon\int_{\pd\varDelta}\negthickspace\negthickspace\negthickspace P(\ze,z)d\ph
 \]
 Теперь $z:=re^{i\theta}$. Пусть $|\theta-\ph_{0}|<\delta.\;\exists\,\rho\!:\,$
 если $R-\rho<r<R$, то $P(\ze,z)<\varepsilon,\:\ze\in\ga_{2}$. Тогда
 \[
 \left|\int_{\ga_{2}}\negthickspace\{u(\ze)-u(\ze_{0})\}P(\ze,z)d\ph\right|<
 2\max |u|\int_{\ga_{2}}\negthickspace\negthickspace P(\ze,z)d\ph\leqslant
 2M\varepsilon 2\pi
 \]
 \end{itemize}



 \par\bigskip\textbf{Лекция++}\par
 \section{Гидродинамика.}
 В области \D на плоскости имеется установившееся течение жидкости,
 $\Vec{V}(x,y)=(\,p(x,y),q(x,y)\,)$ --- скорость течения в точке $(x,y)$.
%%*******Рысуночэк*************************************************************
 \begin{center}
 \textit{...........Рисуночек...........}
 \end{center}
 $\ga$ --- простой гладкий замкнутый контур, $(x(s),y(s))$ --- параметризация,
 $s$ --- длина, тогда $\Vec{n}=\left(\frac{\pd y}{\pd s},
 -\frac{\pd x}{\pd s}\right)$ --- внешняя единичная нормаль.
 \begin{df}
 \[
 \Pi=\int_{\ga}\negthickspace\Vec{V}\!\cdot\Vec{n}ds=
 \int_{0}^{L}\negthickspace\negthickspace\,\left(p\frac{\pd y}{\pd s}-
 q\frac{\pd x}{\pd s}\right)ds=
 \int_{\ga}\negthickspace-qdx+pdy=\int_{\ga}\om_{1}
 \text{ --- поток через $\ga$.}
 \]
 \end{df}
 \begin{df}
 \[
 B=\int_{\ga}\negthickspace\Vec{V}\!\cdot\Vec{\tau}ds=
 \int_{0}^{L}\,\negthickspace\negthickspace\left(p\frac{\pd x}{\pd s}+
 q\frac{\pd y}{\pd s}\right)ds=
 \int_{\ga}\negthickspace pdx+qdy=\int_{\ga}\om_{2}
 \text{ --- вихрь.}
 \]
 \end{df}
 Если нет источников (и стоков) $\:\forall\,\ga\quad\Pi(\ga)=0$, то есть
 \begin{align}   \label{divergencia}
 &0=\int_{\ga}(-qdx+pdy)=\underset{\Omega}{\int\negthickspace\negthickspace\int}
 (p_{x}+q_{y})dx\wedge dy\:\Longrightarrow\:p_{x}+q_{y}=0\\%\quad
 &(p_{x}+q_{y}=\text{div}\,\Vec{V})\notag
 \end{align}
 Если нет вихрей, то $\forall\,\ga\;B(\ga)=0$, то есть
 \begin{align}   \label{rotor}
 &0=\int_{\ga}(pdx+qdy)=\underset{\Omega}{\int\negthickspace\negthickspace\int}
 (q_{x}-p_{y})dx\wedge dy
 \:\Longrightarrow\:q_{x}-p_{y}=0\\
 &(q_{x}-p_{y}=\text{rot}\,\Vec{V})\notag
 \end{align}
 (\ref{divergencia}) --- это $d\om_{1}=0$,
 (\ref{rotor}) --- это $d\om_{2}=0$.\par
 В односвязной области замкнутость означает точность, так как интегралы от
 этих форм не зависят от пути и можно рассмотреть функции:
 \[
 u(z)=u(x,y)=\int_{z_{0}}^{z}\om_{1}\,,\:v(z)=\int_{z_{0}}^{z}\om_{2}\,,\;
 \text{ причём } du=\om_{2}\,,\:dv=\om_{1}\,.
 \]
 \begin{df}
 $v$ называется функцией тока, $u$ --- потенциалом.
 \end{df}
 Жидкость течёт вдоль линий уровня $v$, $\{\,v=\text{const}\,\}$.
 Действительно,
 \[
 \frac{\pd (\,V(x(t),y(t))\,)}{dt}=
 \frac{\pd v}{\pd x}\frac{\pd x}{\pd t}+
 \frac{\pd v}{\pd y}\frac{\pd y}{\pd t}=
 -q\!\cdot\!p+p\!\cdot\!q=0.
 \]
 Заметим, что $u_{x}=p=v_{y},\:u_{y}=q=-v_{x}\:\Longrightarrow\:f=u+iv$ ---
 голоморфна в \D --- комплексный потенциал. Зная $f$, векторное поле $\Vec{V}$
 можно найти следующим образом:
 \[
 \Vec{V}=p+iq=\frac{\pd u}{\pd x}+i\frac{\pd u}{\pd y}
 \:\text{ или }\:
 \frac{\pd v}{\pd y}-i\frac{\pd v}{\pd x}=\overline{f'}
 \]
 Далее:
 \[
 \int_{\ga}\negthickspace\negmedspace f'dz=
 \int_{\ga}\negmedspace(p=iq)(dx+idy)=
 \int_{\ga}\negthickspace pdx+qdy+
 \int_{\ga}\negthickspace-qdx+pdy=B(\ga)+i\Pi(\ga)
 \]


%% Local Variables:
%% eval: (setq compile-command (concat "latex  -halt-on-error -file-line-error " (buffer-name)))
%% End:
