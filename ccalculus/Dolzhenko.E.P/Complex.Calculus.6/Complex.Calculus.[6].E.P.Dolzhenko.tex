\documentclass[a4paper]{article}
\usepackage[utf]{dmvn}

\tocsubsectionparam{2.6em}

\DeclareMathOperator{\perim}{perim}

\def\dr{\,dr}
\def\ds{\,ds}
\def\du{\,du}
\def\dt{\,dt}
\def\dx{\,dx}
\def\dy{\,dy}
\def\dz{\,dz}
\def\dze{\,d\ze}
\def\dta{\,d\ta}
\def\dsi{\,d\si}

\def\marksign{$\maltese$}
\newcommand{\markr}{\vrule width0pt height1pt\vadjust{\vbox to 0pt{\vss\hbox to \textwidth{\hfil\raise.1pc\hbox to 0pt{\hskip.75em\marksign\hss}}}}}


\DeclareMathOperator{\Lip}{Lip}

\def\td{\triangledown}
\def\ort{\overrightarrow}

\title{Лекции по комплексному анализу}
\author{Лектор\т Е.\,П.\,Долженко}
\date{VI семестр}

\begin{document}
\dmvntitle{Курс лекций по}{комплексному анализу}{Лектор Евгений Прокофьевич Долженко}
{III курс, 6 семестр, поток математиков}{Москва, 2005 г.}

\pagebreak

\tableofcontents

\pagebreak

\section*{Предисловие}

\subsection*{Слова благодарности и прочие комментарии}

Огромное спасибо Роме и Маше Ждановым за героизм, проявленный при наборе этого текста и Диме Горяшину,
предоставившему свои лекции.
Правка, вёрстка текста, набор разделов про теоремы Пенлеве и стандартизация оформления были проведены
DMVN Corporation. В текущей версии уже поменьше глюков, часть которых относится к устранимым,
а некоторые лечатся только полной заменой плохих доказательств хорошими.

\medskip

Несколько теорем заменены их аналогами из лекций В.\,К.\,Белошапки (5 семестр) там они
доказаны красиво, правильно и строго. Вообще настоятельно рекомендуем почитать лекции второго
потока --  узнаете много нового!

Что касается благодарностей по поводу поиска опечаток и прочей лажи, мы направляем их
Паше Наливайко, Володе Филатову, Алексею Басалаеву, Мише Малинину, Shviller'у и особенно Ване Вегнеру за непреодолимое рвение
навести в этом тексте порядок.

Мощность множества здесь иногда обозначается значком $\Card$.

\subsection*{Release Notes}

Значком <<$\maltese$>> на полях отмечаются места гибели лажи, чтобы удобнее было следить.
Обозначение введено с 16 июня 2005 года. В данном издании уже есть кое-что про операционный метод.

\subsection*{От редактора русского перевода}

\begin{items}{-2}
\item Через $\Oc(G)$ здесь мы обозначаем множество функций, голоморфных в области $G$, а через $\Mc(G)$ -- множество
функций, мероморфных в области $G$ (то есть все особые точки -- не хуже, чем полюса).
\item Пространство гармонических в области $G$ функций обозначается через $\Hc(G)$.
\item Замыкание множества обозначается чертой сверху.
\item Для сокращения записи мы, в отличие от лектора, будем вместо тяжеловесной фразы <<функция $f(z)$
однозначна и аналитична в области $G$>> писать просто и понятно: $f\in \Oc(G)$.
\item Лектор часто использовал странную конструкцию $\ol{\Int \Ga}$, где $\Ga=\pd G$.
Мы будем писать более просто: $\ol G$.
\item Лектором было выбрано крайне неудобное обозначение для приращения функции на контуре: $\De_\ga(f)$.
Все нормальные люди пишут $\Var_\ga(f)$, но против маразма не попрёшь.
\item Слова <<однозначная и аналитическая>> далее по тексту заменены словом <<голоморфная>>.
\item Окружность с центром в точке $a$ радиуса $R$ обозначается $C_R(a)$.
\item В интегралах типа Коши будем просто указывать, по какому контуру ведется интегрирование, сократив тем самым
длинную запись <<$\ze \in \pd G$>> под значком интеграла.
\end{items}

\medskip
\dmvntrail

\pagebreak

\section{Принцип аргумента}

Как правило, везде, если явно не указано обратное, под словом \emph{<<область>>} мы будем понимать
область с простой или составной спрямляемой жордановой границей.

Нули и полюса функций всегда подразумеваются с учётом кратностей!

\subsection{Логарифмический вычет}

\begin{df}
Пусть $G$ область с границей $\ga$, и функция $f(z)\in \Mc(G)$. \emph{Логарифмический вычет} это интеграл
$$\frac{1}{2\pi i}\ints{\ga} \frac{f'(z)}{f(z)}\dz.$$
\end{df}

Название объясняется тем, что этот вычет есть интеграл по контуру $\ga$ от логарифмической производной
функции $f(z)$, то есть от функции
\eqn{\br{\Ln f(z)}'= \frac{f'(z)}{f(z)}.}
Разумеется, в определении предполагается, что в области $G$ у функции $f$ могут быть полюсы, а на
границе нулей и полюсов функции $f(z)$ нет.

\begin{df}
Через $\De_\ga(f)$ будем обозначать \emph{приращение функции при обходе контура} $\ga$.
Через $N_f$ будем обозначать \emph{количество нулей} в некоторой области, а через $P_f$ \emph{количество полюсов}.
\end{df}

\begin{theorem}[Принцип аргумента]
Пусть $G$ ограниченная область, $\ga=\pd G$, и $f \in \Oc(\ol G \wo \hc{b_1\sco b_p})$, где $b_i$ полюса
кратностей $m_i$. Пусть $a_1\sco a_n \in G$ нули функции $f$ кратностей $k_i$.
Тогда
\eqn{N_f-P_f = \suml{i=1}{n} k_i -\suml{i=1}{p} m_i = \frac{1}{2\pi} \De_\ga \Arg f.}
\end{theorem}
\begin{proof}
Поскольку $\Ln f(z)$ первообразная для $\frac{f'(z)}{f(z)}$, а функция $\ln |f(z)|$ однозначна, то
\eqn{\frac{1}{2\pi i} \ints\ga \frac{f'(z)}{f(z)}\dz=\frac{1}{2\pi i}\De_\ga \Ln f(z) =
\frac{1}{2\pi i}\De_\ga \br{\ln |f(z)|+i \Arg f(z)} =\frac{1}{2 \pi} \De_\ga \Arg f(z).}
С другой стороны, мы можем вычислить этот же интеграл с помощью теоремы Коши о вычетах:
\eqn{\frac{1}{2\pi i} \ints\ga \frac{f'(z)}{f(z)}\dz=
\suml{i=1}{n}\resl{a_i} \frac{f'(z)}{f(z)}+ \suml{i=1}{p}\resl{b_i} \frac{f'(z)}{f(z)}.}

Пусть $\ph \in \Oc(\ol G)$. Тогда
\eqn{\frac{1}{2\pi i} \ints\ga \ph(z) \frac{f'(z)}{f(z)}\dz=
\suml{i=1}{n}\resl{a_i} \ph \frac{f'}{f}+ \suml{i=1}{p}\res_{b_i}\ph \frac{f'}{f}.}
Пусть $c$ это нуль или полюс функции $f(z)$ кратности $k$ (если $k>0$, то точка $c$ есть нуль, иначе полюс),
е $f(z)=(z-c)^{k}g(z)$, причём $g(с) \neq 0$. Тогда
\eqn{\frac{f'(z)}{f(z)}= \frac{k(z-c)^{k-1}g(z)+(z-c)^k g'(z)}{(z-c)^kg(z)}= \frac{k}{z-c}+\frac{g'(z)}{g(z)}.}
Отсюда ясно, что $\resl{c} \frac{f'}{f}=k$, так как функция $\frac{g'(z)}{g(z)}$ аналитична в окрестности точки $c$.
Кроме того,
\eqn{\ph(z)-\ph(c)=(z-c)^s h(z)\quad \Ra \quad \ph(z)=\ph(c)+(z-c)^s h(z),\quad h(c)\neq 0,}
поэтому
\eqn{\ph(z)\frac{f'(z)}{f(z)}=\frac{k\ph(z)}{z-c}+\psi(z),}
где $\psi \in \Oc(\ol G)$. Следовательно, $\resl{c}\frac{f'}{f}\ph =k\ph(c)$.
Осталось подставить это в формулу для интеграла:
\eqn{\frac{1}{2\pi i} \ints{\ga} \ph(z)\frac{f'(z)}{f(z)}\dz= \suml{i=1}{n}k_i \ph(a_i) - \suml{i=1}{p} m_i\ph(b_i).}
Знак <<минус>> перед второй суммой объясняется тем, что у полюсов
кратность отрицательная. При $\ph(z)\equiv 1$ получаем искомую формулу.
\end{proof}

\begin{imp} Если $f \in \Oc(G)$, и $f(z)\neq 0$ при $z\in \ga=\pd G$, то
\eqn{N=\frac{1}{2\pi}\De_\ga \Arg f(z).}
\end{imp}

\begin{imp}
Конформные отображения областей сохраняют направление обхода границы ($N=1$).
\end{imp}

\begin{note}
Принцип аргумента верен, если $f \in \Cb(\ol G) \cap \Oc(G)$.
\end{note}

Вернёмся к равенству
\eqn{\frac{1}{2\pi i} \ints{\ga} \ph(z)\frac{f'(z)}{f(z)}\dz= \suml{i=1}{n} k_i \ph(a_i),}
где $f(z) \in \Oc(\ol G)$ (е полюсов нет). При $\ph(z)\equiv 1$ мы получаем равенство
\eqn{N=\frac{1}{2\pi}\De_\ga \Arg f(z).}
Пусть теперь $\ph(z)=z$. Тогда
\eqn{\frac{1}{2\pi i} \int z\frac{f'(z)}{f(z)}\dz= \suml{i=1}{n}k_i a_i.}
Если же $\ph(z)=z^2$, то
\eqn{\frac{1}{2\pi i} \int z^2\frac{f'(z)}{f(z)}\dz= \suml{i=1}{n}k_i a_i^2,}
и так далее. Пусть, например, в области $G$ два нуля функции $f(z)$. Тогда мы
можем найти значения $z_1+z_2=A$ и $z_1^2+z_2^2=B$. Но так как $z_1z_2=\frac{1}{2}
[(z_1+z_2)^2-(z_1^2+z_2^2)]$, то нули функции $f(z)$ это корни уравнения
$z^2-Az+\frac{1}{2}(A^2-B)=0$.

\subsection{Теорема Руше}

\begin{theorem}[Руше] Пусть $G$ ограниченная область с границей~$\ga$,
функции $f(z)$ и $g(z)$ аналитичны в $G$ и непрерывны на $\ol G$, причем $|f(z)|>|g(z)|$ на $\ga$.
Тогда функция $f(z)+g(z)$ имеет внутри $G$ столько же нулей, сколько их имеет функция $f(z)$.
\end{theorem}
\begin{proof}
Так как на границе $|f(z)+g(z)| \ge |f(z)|-|g(z)| > 0$ и $|f(z)|>|g(z)|\ge 0$, то на границе нулей у функций $f+g$ и $f$ нет.
Имеем
\eqn{\frac{1}{2\pi}\De_\ga \Arg\br{f(z)+g(z)}= \frac{1}{2\pi}\De_\ga \Arg \hs{f(z)\hr{1+\frac{g(z)}{f(z)}}}=
\frac{1}{2\pi}\De_\ga \Arg f(z)+\frac{1}{2\pi}\De_\ga \Arg \hr{1+\frac{g(z)}{f(z)}}.}
По условию $\hm{\frac{g(z)}{f(z)}}<1$. Значит, вектор $1+\frac{g(z)}{f(z)}$ никогда не выйдет
за пределы правой полуплоскости и, следовательно, не совершит ни одного оборота вокруг нуля
при обходе $\ga$. Значит,
\eqn{\De_\ga \Arg\hr{1+\frac{g(z)}{f(z)}}=0 \quad \text{ и } \quad N_{f+g} = \frac{1}{2\pi}\De_\ga \Arg \br{f(z)+g(z)}=\frac{1}{2\pi}\De_\ga \Arg f(z) = N_f.}
\hfill\end{proof}

\begin{ex}
Найдём область, в которой находятся все нули многочлена $P(z)=z^{10}-2z+1$. Пусть $f(z)=z^{10}$,
а $g(z)=-2z+1$. Возьмём круг $|z|=R$. Тогда $|f(z)|=R^{10}$, и $|g(z)|=|-2z+1|<2R+1$. Значит,
нужно найти такое $R$, что $R^{10}>2R+1$. Это можно сделать приближенно ($R\approx 1,2$).
\end{ex}

\subsubsection{Замечание о теореме Лагранжа}

В действительном анализе верна теорема (Лагранжа) о среднем:
\eqn{\exi c\in[a,b]\cln \quad f(a)-f(b)=f'(c)(b-a).}
В комплексном анализе эта формула неверна:
\eqn{\De=[0, 2\pi],  \quad f(t)=e^{it}.}
Тогда $f(2\pi)-f(0)=0$, но $f'(t)=ie^{it},  \quad |f'(t)|=1$ для всех $t$.

\begin{stm}[Аналог теоремы Лагранжа]
Пусть $f \in \Oc(G)$. Тогда
\eqn{|f(b)-f(a)|\le  (b-a)\cdot \maxl{\ze\in[a,b]} |f'(\ze)|.}
\end{stm}
\begin{proof}
По формуле Ньютона Лейбница имеем
\eqn{f(b)-f(a)=\intl{a}{b} f'(\ze)\dze.}
Отсюда всё следует.
\end{proof}


\section{Аналитическое продолжение}

\subsection{Теоремы Пенлеве}

\subsubsection{Хаусдорфова мера}

\begin{df}
Пусть $(X,\rh)$ метрическое пространство, $E \subs X$. Покроем множество $E$ кругами $\si_n$
диаметров $d_n < \ep$ и найдём точную нижнюю грань суммы их радиусов по всем покрытиям, устремив $\ep$ к нулю.
Этот предел и называется \emph{длиной по Хаусдорфу}:
\eqn{\mes^1(E):=\liml{\ep \ra 0} \Br{\infl{d_n< \ep} \BC{\sums{n} d_n: E \subs \cups{n}\si_n}}, \quad d_n := \diam \si_n.}
\end{df}

\begin{df}
Более общее понятие \emph{$\al$ мера Хаусдорфа} ($\al \ge 0$):
\eqn{\mes^\al(E)=\liml{\ep \ra 0} \Br{\infl{d_n<\ep} \BC{\sums{n} d_n^\al\cln E\subs \cups{n}\si_n}}, \quad d_n := \diam \si_n.}
\end{df}

Очевидно, при $\al=2$ это определение совпадает с определением площади, а при $d=3$ с определением объёма.
Хаусдорфова мера определена всегда, хотя может принимать и бесконечные значения (например, $\mes^1([0,1]^2)=\bes$).

Можно пойти ещё дальше в обобщении хаусдорфовых мер. Вместо функции $t^\al$ можно взять произвольную
неотрицательную возрастающую непрерывную в нуле функцию $\ph(t)$, для которой $\ph(0)=0$.
Тогда в определении $\ph$ меры Хаусдорфа вместо $\sum d_n^\al$ будет $\sum \ph(d_n)$.
Обозначается она иногда так: $(\ph)\mes$, но мы будем использовать более компактное обозначение $\mes_\ph$.


\subsubsection{Свойства хаусдорфовых мер}

Для краткости  обозначим $\mu := \mes_\ph$. Очевидно, хаусдорфова мера полуаддитивна, то есть
$\mu(E_1\cup E_2) \bw\le \mu(E_1) +\mu(E_2)$, а если расстояние между множествами положительно, то будет равенство.
Меру на метрическом пространстве, определённую на все его подмножествах, и обладающую такими свойствами, называют
\emph{внешней метрической мерой}. По теореме Каратеодори совокупность $\mu$ измеримых подмножеств
образуют $\si$ алгебру.

На первый взгляд кажется ненужным условие $\ep\ra0$ в определении хаусдорфовой меры. Но на самом деле оно
очень естественно: рассмотрим произвольную кривую $\ga$ бесконечной длины, лежащую в круге радиуса~$1$.
Тогда мы получили бы $\mes^1(\ga)\le 2$, так как она накрывается одним кругом диаметра 2.
В качестве примера такой кривой можно взять $z(t) = \frac{1-t}{1+t}\cdot\exp\hr{\frac{it}{1-t}}$.


Сформулируем некоторые очевидные свойства хаусдорфовых мер.

\begin{points}{-1}
\item Ортогональная\footnote{На самом деле, это тоже не важно: можно брать и косоугольную проекцию: множество <<растянется>>
в конечное число раз.} проекция множества нулевой длины на любую прямую имеет нулевую длину, е
почти каждая прямая, вдоль которой ведётся проекция, не имеет общих точек с проецируемым множеством.
\item
Пусть $E\subs \R^2$. Очевидно, что $\mes^2(E)=0$ тогда и только тогда, когда мера Лебега множества $E$ равна нулю.
\end{points}

\begin{problem}
Доказать, что $\mes^{2+\de}(E)=0$ для любого множества $E\subs\R^2$ и $\de > 0$.
\end{problem}
\begin{hint}
Переформулировка: доказать, что при $\de>0$ имеет место сходимость $x^\de \ra 0$ при $x\ra 0$.
\end{hint}

\begin{problem}
Доказать, что если при некотором $\al \in (0,2]$ выполняется неравенство $\mes^\al(E)>0$, то $\mes^\be(E)=\bes$
при $\be < \al$, а  если $\mes^\al(E)<\bes$, то $\mes^\be(E)=0$ при $\be > \al$.
\end{problem}

\begin{petit}
Данный комментарий относится к последней задаче. Фактически, в ней неявно даётся определение \emph{хаусдорфовой
(фрактальной) размерности} множества. Основная её идея состоит в следующем. Если бы хотим измерить площадь множества,
мы накрываем его достаточно мелкими двумерными множествами и затем суммируем их площадь. Если же мы измеряем объём
множества, то покрываем его параллелепипедами и суммируем объёмы. Но пусть теперь наше множество плохое (например,
канторовское). Видно, что его длина (и тем более площадь) равны нулю, хотя при этом оно непусто и даже континуально.
Мы хотим сопоставить ему некоторое число, которое будет отличать его (по размерности) от прямой (одномерного множества).
Возникает подозрение, что его размерность не является целым числом. Но каким? Заметим, что квадрат разбивается на
4 квадрата вдвое меньшего размера, куб на 8 кубиков с ребром вдвое меньшей длины, четырёхмерный
куб (если читатель может его себе представить) на 16 вдвое меньших по линейным размерам 4-черных кубиков.
То есть, в общем случае размерность множества равна $\log_2 n$, где $n$ число одинаковых частей вдвое
меньшего размера, на которые можно разбить это множество. Канторовское множество подобно самому себе с
коэффициентом $\frac13$ в нём умещается две его копии. Это и наводит нас на ответ $\log_3 2$. Можно доказать, что
это так и есть. А само определение таково:

\begin{df}
\emph{Хаусдорфовой размерностью} множества $E$ называется наименьшее число $d \in \R_+$,
для которого $\mes^d(E)=0$. Точнее говоря,
\eqn{\dim_H(E) = \inf\bc{d\cln \mes^d(E)=0}.}
\end{df}
\end{petit}


\subsubsection{Первая теорема Пенлеве}

Прежде всего заметим, что множество особых точек голоморфной функции всегда  замкнуто.
Это следует из того, что множество точек голоморфности функции открыто (если степенной ряд в некоторой точке
имеет ненулевой радиус сходимости, то есть сходимость и в некоторой окрестности этой точки).

\begin{petit}
Ну удивляйтесь тому, что доказательство следующей теоремы гораздо короче, чем в листочках Долженко.
Просто здесь гораздо меньше лишних слов. Идея то здесь очень простая: накроем множество $E$ особых точек
кругами сколь угодно малого радиуса, тогда для любой точки из дополнения $G\wo E$ верна формула Коши. Но так как
эти самые точки из дополнения могут быть сколь угодно близки к точкам множества $E$ (ибо радиусы кругов сколь угодно малы),
мы можем доопределить функцию и на множестве $E$ (фактически, по непрерывности).
\end{petit}

\begin{theorem}[Первая теорема P.\,Painlev\'e] Пусть ограниченная функция $f$ аналитична в области $G$ за исключением множества
$E$ нулевой длины (по Хаусдорфу). Тогда множество $E$ устранимо, е функция $f$ продолжается до голоморфной функции во
всей области.
\end{theorem}
\begin{proof}
В силу сделанного выше замечания, почти каждая координатная прямая не имеет общих точек с множеством $E$.
Значит, найдётся прямоугольник $Q$, на границе которого функция голоморфна. Его граница лежит в множестве $G\wo E$
с некоторой своей окрестностью, и потому $\rh(\pd Q,E)>0$.

Покажем, что функция голоморфна на множестве $E_0 := E\cap Q$. Рассмотрим покрытие множества $E_0$ кругами
$\si_n$ радиусов $d_n < \ep$, и в силу компактности $E_0$ из них можно выделить конечное подпокрытие.
Так как ${\mes^1(E)=0}$, то и подавно $\mes^1(E_0)=0$. Значит, сумму $\sum d_n$ можно сделать сколь угодно
малой (а значит, и суммарную длину границ $\si_n$, ибо она пропорциональна сумме радиусов).

Пусть $z\in Q\wo E_0$. Тогда верна интегральная формула Коши для составного контура $\Ga := \pd Q \cup \hr{\bigcup \pd \si_n}$.
Так как ориентация границ $\pd Q$ и $\pd \si_n$ разная, имеем
\eqn{f(z) = \frac{1}{2\pi i}\ints{\Ga} \frac{f(\ze)}{\ze-z}\dze = \frac{1}{2\pi i}\ints{\pd Q} \frac{f(\ze)}{\ze-z}\dze - \frac{1}{2\pi i}\cdot \sums{n}\ints{\pd \si_n} \frac{f(\ze)}{\ze-z}\dze.}
Покажем, что второе слагаемое равно нулю. Действительно, $|f(z)|\le M$. Значит,
\eqn{\bbm{\ints{\pd \si_n} \frac{f(\ze)}{\ze-z}\dze} \le \ints{\pd \si_n} \frac{M}{\rh(z,\pd \si_n)}\dze = \frac{M \pi d_n}{\rh(z,\pd \si_n)} \le \frac{M \pi \ep}{\rh(z,\pd \si_n)}.}
Кругов у нас конечное число, $\ep$ произвольно, а расстояние $\rh$ в знаменателе ограничено снизу, так как точка $z$ фиксирована.
Поэтому второе слагаемое сколь угодно мало.

Итак, у нас есть формула
\eqn{f(z) = \frac{1}{2\pi i}\ints{\pd Q} \frac{f(\ze)}{\ze-z}\dze, \quad z  \in Q\wo E_0.}
Но никто не помешает нам доопределить значение функции на множестве $E_0$ по той же формуле.
Тогда она будет голоморфной и на множестве $E_0$ как интеграл типа Коши.
\end{proof}

\begin{lemma}
Модуль непрерывности непостоянной функции обладает следующим свойством: $\om(r)\ge Cr$ для
некоторого $C>0$ и любого $r\in[0,\de]$.
\end{lemma}
\begin{proof}
В самом деле, пусть не существует положительной константы $C$ с таким свойством. Значит, для любой последовательности
$C_n\ra 0$ выполняется обратное неравенство $\om(r) < C_n r$. Значит, $\om(r) =\ol o(r)$. Но из этого следует,
что $f\equiv \const$.
\end{proof}

\begin{petit}
Комментарий от Паши Наливайко: зачем возиться с квадратами, когда можно обойтись кругами?  Ответ: можно и не возиться,
и так всё получится. Просто в этом доказательстве сохранена лекторская идея (на всякий случай, если придётся сдавать ему :).
\end{petit}

\subsubsection{Вторая теорема Пенлеве}

\begin{theorem}[Е.\,П.\,Долженко]
Пусть $f\in \Cb(G) \cap \Oc(G\wo E)$, где $\mes_\ph(E)=0$, $\ph(r) := r\om_f(r)$, а $\om_f(r)$ модуль непрерывности
функции $f$. Тогда множество $E$ устранимо.
\end{theorem}
\begin{proof}
Как и в предыдущей теореме, считаем множество $E$ замкнутым. Покажем, что множество $E$ имеет нулевую лебегову
меру $\mu$. По предыдущей лемме $\frac{\om(r)}{C} \ge r$ для всех $r\in[0,\de]$ и для некоторого $\de>0$.
А по условию у нас $\mes_\ph(E)=0$. Значит, найдётся такое покрытие множества
$E$ кругами $\si_n$, что для всякого $\ep > 0$ выполняется неравенство
\eqn{\mes_\ph(E) \le \sum \ph(d_n) = \sum d_n\om(d_n) <\ep, \quad d_n := \diam \si_n < \de.}
Отсюда получаем оценку сверху на лебегову меру, применив вытекающее из леммы неравенство $d_n^2\le d_n \cdot\frac{\om(d_n)}{C}$:
\eqn{\mu(E) \le \sum \pi\hr{\frac{d_n}{2}}^2 \le \frac{\pi}{4}\cdot \sum \frac{d_n \om(d_n)}{C} < \frac{\pi\ep}{4C}.}
Константа $C$ зависит лишь от самой функции, число $\ep$ произвольно. Значит, $\mu(E)=0$.

Отсюда следует, что $E$ нигде не плотно в $G$ (если бы оно было плотным в некотором круге, то в силу его замкнутости
оно имело бы положительную лебегову меру). Рассмотрим произвольный прямоугольник $Q \subs G$. Покажем, что функция $f$
аналитична внутри $Q$. Пусть $z\in Q\wo E$, и
\eqn{\rh := \min\hc{\dist(z,E);\;\dist(z,\pd Q)} > 0.}
В силу компактности $E$ можно считать покрытие $\hc{\si_n}$ конечным. Накроем круги $\si_n$ квадратами $Q_n$.
Так как множество $E$ нигде не плотно в $Q$, можно считать, что расстояние от точки $z$ до границ $Q_i$ положительно.
Кроме того, можно считать, что $Q_i\subs Q$.
теперь упорядочим квадраты $Q_i$ по убыванию длин сторон и перейдём к дизъюнктному покрытию $\hc{K_i}$: положим
\eqn{\begin{aligned}
K_1 &:= Q_1,\\
K_2 &:= Q_2 \wo \ol Q_1,\\
K_3 &:= Q_3 \wo (\ol{Q_1 \cup Q_2}),\\
\ldots
\end{aligned}\hskip2cm \lower1pc\hbox{\epsfbox{pictures.10}}}
В силу убывания длин сторон имеем $\perim(K_i) \le \perim(Q_i)$. Запишем интегральную формулу Коши для набора контуров $Q \cup \hc{K_i}$.
Имеем
\eqn{f(z) = \frac{1}{2\pi i}\ints{\pd Q} \frac{f(\ze)}{\ze-z}\dze - \frac{1}{2\pi i}\cdot \sums{n}\ints{\pd K_n}\frac{f(\ze)}{\ze-z}\dze.}
Покажем, что второе слагаемое равно нулю. Пусть точки $a_n$ центры кругов $\si_n$. Заметим,
что функция $\frac{f(a_n)}{\ze-z}$ аналитична, если $\ze$ бегает по границам $\pd K_i$, так как знаменатель отделён от нуля.
Значит, в последний интеграл можно безболезненно затащить нулевое слагаемое $\frac{1}{2\pi i}\int\frac{f(a_n)}{\ze-z}\dze$.
Итак, имеем
\eqn{f(z) = \frac{1}{2\pi i}\ints{\pd Q} \frac{f(\ze)}{\ze-z}\dze - \frac{1}{2\pi i}\cdot \sums{n}\ints{\pd K_n}\frac{f(\ze) - f(a_n)}{\ze-z}\dze.}
Если $\ze \in \pd K_n$, то $|\ze-a_n|<d_n$, а потому $|f(\ze)-f(a_n)|\le \om(d_n)$.
Значит,
\eqn{\bbm{\frac{1}{2\pi i}\cdot \sums{n}\ints{K_n} \frac{f(\ze)-f(a_n)}{\ze-z}\dze} \le \frac{1}{2\pi}\cdot \sums{n}\frac{\om(d_n)}{\rh}\cdot \perim(K_n) = \sums{n}\frac{2d_n\om(d_n)}{\pi\rh} \le \frac{2\ep}{\rh}.}
А $\ep$ у нас было произвольным. Дальше рассуждение такое же, как и в первой теореме: функция $f(z)$ представляется интегралом
Коши по границе $\pd Q$. Это интеграл и сама функция $f(z)$ непрерывны, и совпадают друг с другом на всюду плотном множестве $Q\wo E$.
Значит, они совпадают всюду.
\end{proof}

\begin{imp}[Вторая теорема Painlev\'e]
Если непрерывная функция аналитична в области $G$ за исключением множества $E$ конечной хаусдорфовой длины, то она аналитична всюду
в $G$.
\end{imp}
\begin{proof}
Так как длина множества $E$ конечна, для $\fa\ep> 0$ найдётся конечное множество кругов $\si_n$ с диаметрами $d_n<\ep$,
содержащая $E$. Пусть $\ph(r) = r\om_f(r)$. Тогда
\eqn{\mes_\ph(E) \le \sums{n}\ph(d_n) = \sums{n}d_n\om(d_n) \le (\ub{\mes^1(E)+1}_C)\om(\ep) \le C\om(\ep)\ra 0, \quad \ep\ra 0}
в силу непрерывности функции $f$. Значит, можно применить теорему, и $f\in \Oc(G)$.
\end{proof}

\subsection{Принцип симметрии Римана Шварца}

\subsubsection{Продолжение функции через границу}

\begin{df}
Пусть есть две области $G_1$, $G_2$, граничащие по кривой $\ga$, а функция $f$ задана $G_1\cup\ga$.
\emph{Продолжением} функции $f$ через границу $\ga$ на область $G_2$ называется функция $g:G_1\cup G_2\to\Cbb$,
совпадающая с $f$ на $G_1\cup\ga$.
\end{df}

Из только что доказанной теоремы следует, что если функция $g$ аналитична на $G_1 \cup G_2$,
то она аналитична  и на $G_1\cup \ga\cup G_2$.

\begin{petit}
Здесь был пример Помпейю, но к чему он был  неизвестно:
\eqn{F(z) := \iint\frac{d\xi d\eta}{\zeta-z},\quad  \zeta:=\xi+i\eta.}
\end{petit}

Из второй теоремы Пенлеве следует
\begin{theorem}[Об аналитическом продолжении через границу]
Пусть жордановы области  $G_1$ и $G_2$ граничат по дуге $\ga$ конечной длины. Пусть функция $f_1 \in \Oc(G_1)\cap \Cb(G_1 \cup\ga)$,
а $f_2 \in \Oc(G_2)\cap \Cb(G_2 \cup\ga)$, и на дуге $\ga$ эти функции совпадают, то функция $g$,
равная $f_i$ на $G_i$, аналитична на $G_1\cup \ga \cup G_2$.
\end{imp}

\begin{note}
Условие на дугу $\ga$ в этой теореме может быть ослаблено. Пусть $\om_{f_1}(G_1 \cup \ga, \de)=O\br{\la(\de)}$ и
$\om_{f_2}(G_2 \cup \ga, \de)=O\br{\la(\de)}$ при $\de\ra 0$. Здесь $\la(\de)$ неотрицательная непрерывная в нуле
неубывающая функция на некотором отрезке $[0,d]$, что $\mes_\ph(\ga)=0$, где $\ph(r)=r\la(r)$. Тогда справедливо заключение
теоремы.
\end{note}

\begin{ex}
Ели функции  $f_i \in \Lip^\al$, а $\mes^{1+\al}(\ga)=0$, то теорема справедлива.
В частности, если $f_i\in \Lip^1$, и $\mes^2(\ga)=0$, то $g\in \Oc(G_1\cup \ga \cup G_2)$.
\end{ex}


\subsubsection{Принцип симметрии Римана Шварца}

\begin{df}
Если не указано противное, через $G^*$ будем обозначать область, симметричную области $G$ относительно
вещественной оси (а в общем случае относительно некоторой окружности).
\end{df}

\begin{lemma}
Если $f\in\Oc(G)$, то функция $\wt f := \ol{f(\ol{z})}$ голоморфна в симметричной области $G^*$.
\end{lemma}
\begin{proof}
Пусть $a \in G^*$, тогда $\ol a \in G$. Из голоморфности функции $f$ в окрестности $\ol{a}$ следует
разложимость в ряд:
\eqn{f(z)=\sum c_n(z-\ol{a})^n.}
Тогда
\eqn{\wt f = \ol{f(\ol{z})}= \ol{\sum c_n(\ol{z}-\ol{a})^n}= \sum \ol{c}_n(z-a)^n.}
Полученный ряд имеет тот же радиус сходимости, откуда и следует голоморфность.
\end{proof}

\begin{lemma}
Даны две непересекающиеся области, такие что $\pd G_1 \cap \pd G_2=\ga$, где $\ga$ отрезок прямой
или дуга окружности, и  $f_1 \in \Oc(G_1) \cap \Cb(G_1 \cup \ga)$, а $f_2 \in \Oc(G_2) \cap \Cb(G_2 \cup\ga)$.
Пусть $f_1 = f_2$ на $\ga$. Тогда функция
\eqn{f(z)= \case{f_1(z), & z \in G_1 \cup\ga;\\
f_2(z), & z \in G_2 \cup\ga}}
голоморфна в $G_1 \cup \ga \cup G_2$.
\end{lemma}
\begin{proof}
Голоморфность следует из теоремы об аналитическом продолжении через границу.
\end{proof}

\begin{theorem}[Принцип симметрии]
Пусть граница области $G_1$ содержит участок прямой или окружности $\ga_1$, а $G_1^*$
симметричная относительно $\ga_1$ область, и $G \cap G^*=\es$;
Пусть $(G_2, G_2^*,\ga_2)$ набор с теми же свойствами. Пусть функция $f\cln G_1\ra G_2$ конформное
отображение, продолжаемое по непрерывности до взаимно однозначного соответствия границ: $f\cln \ga_1 \ra \ga_2$.
Тогда продолженное по симметрии отображение~$f$ даёт конформное отображение
$G_1\cup\ga_1\cup G_1^* \mapsto G_2\cup\ga_2\cup G_2^*$.
\end{theorem}
\begin{proof}
Если $\ga$ является отрезком прямой, то движением можно перевести этот отрезок на вещественную ось, а далее воспользоваться
леммами. Конформность, очевидно, сохранится (движение ничего не испортит). Если же $\ga$ является дугой окружности, то
сначала дробно линейным преобразованием <<распрямим>> её, передвинем на вещественную ось, сведя тем самым к первому случаю.
\end{proof}
\begin{note}
В данной формулировке есть одна тонкость: если мы делаем симметрию относительно окружностей,
то симметричная точка может оказаться в бесконечности. Именно поэтому у продолжения функции могут появиться полюсы
и голоморфной она НЕ БУДЕТ! Но если ограничиться только симметрией относительно прямых, то всё будет хорошо.
\end{note}

\begin{theorem}[И.\,И.\,Привалов, 1918]
Пусть $\De$ единичный круг, $f,g\cln \De \ra \Cbb$. Если в каждой точке границы
у $f$ и $g$ совпадают пределы по одному и тому же направлению, то $f\equiv g$.
\end{theorem}
Доказывать мы её не будем\dots

\subsection{Аналитическое продолжение по цепи}

\subsubsection{Непосредственное продолжение}

\begin{df}
Пусть $G_0$ и $G_1$ выпуклые области, $G_1 \cap G_0\neq \es$,  и функция $f_0(z)$ аналитична в $G_0$,
а функция $f_1(z)$ аналитична в $G_1$ и $f_0(z)=f_1(z)$ на $G_1 \cap G_0$. Тогда говорят, что $f_0$
\emph{аналитически продолжается} в $G_1$ (и $f_1$ аналитически продолжается в $G_0$) непосредственно.
\end{df}

Рассмотрим  еще одну область $G_2$, пересекающуюся с $G_1$, и пусть $f_1(z)=f_2(z)$ для всех $z \in G_1\cap G_2$,
тогда $f_0$ продолжается непосредственно на $G_2$, и так далее.

\begin{df}
Пара $(G, f)$, где $G$ область, а $f$ функция, аналитическая в $G$, называется \emph{аналитическим элементом}.
\end{df}

Таким образом, при непосредственном продолжении мы получаем цепь аналитических элементов.
$(G_2, f_2)$ аналитическое продолжение $(G_0, f_0)$ (уже не непосредственное).

%Процесс аналитического продолжения некоторого аналитического элемента:

\begin{ex}
Рассмотрим функцию \eqn{f(z)=\ln(1+z)=\suml{n=1}{\infty}\frac{z^n}{n}.}
Этот ряд сходится в круге $|z|<1$. Можно с этого круга продолжать: для этого нужно взять точку внутри круга
и разложить в ряд уже с центром в этой точке. Полученные два ряда будут совпадать на пересечении кругов,
следовательно, функции в соответствующих областях будут аналитическими продолжениями друг друга. Далее можно
продолжать этот процесс. В результате получим функцию $w=\Ln(1+z)$.
\end{ex}

\begin{df}
\emph{Полной аналитической функцией} называется (вообще говоря, многозначная) функция, полученная с помощью всевозможных
непосредственных продолжений.
\end{df}

\subsubsection{Особые точки многозначного характера}

\begin{df} Пусть $(f', G')$, $(f, G)$ аналитические элементы, причем $G' \subs G$. Тогда $(f', G')$
называется \emph{подчинённым} элементом.
\end{df}

Обычно в качестве подчинённых элементов рассматривают так называемые рациональные элементы:
\eqn{g=\hc{z\cln |z-a|<r} \quad r, \Re a, \Im a \in \Q.}

Можно продолжать функцию и через бесконечность: рациональный элемент с центром в $\infty$ выглядит так:
\eqn{g=\hc{z\cln |z|>r}, \quad r\in\Q.}

\begin{ex}
У функции
\eqn{w=\frac{1}{1+\sqrt{z}}}
в точке $z=1$ имеется две ветви. На куске римановой поверхности, где $\sqrt{1}=-1$,
будет полюс первого порядка.
\end{ex}

\begin{df}
Точка называется \emph{изолированной особой точкой многозначного характера}, если ни в какой её окрестности функция не
распадается на однозначные аналитические ветви (е функцию нельзя сделать однозначной аналитической ни в какой
проколотой окрестности точки).

Если после $k$ оборотов получилась исходная ветвь и $k$ минимальное число с таким свойством,
то число $(k-1)$ называется \emph{порядком ветвления}.
\end{df}

\begin{ex}
$z_0=0$ точка ветвления 1-го порядка функции $f(z)=\sqrt{z}$.
\end{ex}

\begin{df}
Если $a$  точка ветвления  $k$-го порядка и существует предел $\liml{z\ra a} f(z) \in \ol \Cbb$,
то такая точка называется \emph{алгебраической точкой ветвления} $k$-го порядка.
\end{df}

\begin{df}
\emph{Рядом Пюизо} называется ряд следующего вида:
\eqn{f(z) = \sums{\Z} c_k \br{\sqrt[n]{z-a}}^k = \sums{\Z} c_k (z-a)^\frac k n.}
\end{df}

\begin{ex}
\eqn{\sin \sqrt[3]{z} = \suml{n=1}{\infty}\frac{z^{\frac{2n-1}{3}}}{(2n-1)!}.}
\end{ex}

\subsection{Аналитическое продолжение вдоль пути}

\begin{df}
Пусть $G$ выпуклая область, тогда аналитический элемент $(G_0,f_0)$ \emph{продолжается
вдоль пути} $\ga(t)$, если существует семейство элементов $(G_t,f_t)$ с центрами $a_t := \ga(t)$ и ненулевыми
радиусами, удовлетворяющих следующему условию: если $\ga\br{U_\de(t_0)} \subs G_{t_0}$, то любой элемент $(G_t,f_t)$ при
$t \in U_\de(t_0)$ является непосредственным продолжением элемента $(G_{t_0},f_{t_0})$.
\end{df}

\begin{df}
\emph{Особой точкой} будем называть такую точку, через которую аналитическое продолжение невозможно.
\end{df}

\begin{theorem}[О монодромии]
Пусть $G$ односвязная область, $G_0\subs G$ односвязная выпуклая область. Если элемент  $(G_0,f_0)$
продолжается вдоль любого непрерывного пути, лежащего в $G$, то полученная в результате продолжения
функция будет однозначной аналитической.
\end{theorem}
\begin{proof}
\begin{points}{-1}
\item Продолжение вдоль кривой можно заменить продолжением вдоль ломаной.
\item Будем доказывать от противного: предположим,  что продолжение вдоль замкнутой конечнозвенной ломаной
даёт разные элементы (если идти вдоль ломаной в разные стороны). Можно считать, что ломаная не имеет
самопересечений. Разбив внутренность на треугольники, получаем, что существует треугольник,  продолжение
вдоль которого даёт разные элементы. Проводим медиану. По какому-то из двух получившихся треугольников получаются
различные элементы, и так далее.
\item В результате получаем последовательность вложенных отрезков на стороне (делим пополам всегда одну и
ту же сторону треугольника),  которая имеет общую точку $\ze$. Продолжим $(G_0,f_0)$ в $\ze$ и обратно
посредством $(G_k,f_k)$. Тогда существует треугольник $\De_k \subs \bigcup G_k$. А это противоречит
теореме единственности, так как по $\De_k$ существуют разные продолжения.
\end{points}
\hfill\end{proof}

\begin{petit}
А вот нормальное доказательство теоремы о монодромии без мухлежа (здесь \emph{росток функции в точке} есть пара $(U,[f])$,
где $U$ окрестность, а $[f]$ класс эквивалентности функций, совпадающих в этой окрестности).

\begin{theorem}[о монодромии]
Пусть $D$ область, и $a,b\in D$. Если две кривые с фиксированными концами $a$ и $b$ гомотопны (в области $D$),
то аналитическое продолжение ростка $[f]_a$ определено однозначно.
\end{theorem}
\begin{proof}
Пусть гомотопия задана как отображение $\ga=\ga(t,\tau) \in \Cb([0,1]^2)$, где $t$ параметр кривой, а $\tau$ параметр
деформации. Пусть $[f_1]_b$ и $[f_2]_b$ два продолжения ростка $[f]_a$. Достаточно доказать, что они совпадают при
малом изменении параметра $\tau$ (так как из того, что функция локально не изменилась, будет следовать, что она не изменилась
при всех $\tau$). Будем действовать аналогично доказательству предыдущего утверждения, но теперь в качестве
параметра будет выступать $\tau$. Из каждой точки кривой $\ga(t)$ осуществим аналитическое продолжение вдоль кривой $\ga(t_0,\tau)$
при каждом фиксированном $t_0$. Так как продолжение вдоль кривой единственно, то найдётся система достаточно малых кружочков
окрестностей каждой точки $t_0\in\ga(t,0)$, таких, что в любой точке, лежащей в их объединении, продолженная функция совпадает
с исходной. Но если параметр гомотопии достаточно мал ($\tau \le \de$, где $\de>0$), то кривая $\ga(t,\tau)$ лежит в объединении
кружочков, так как кривая компактна. Значит, при $\tau\le\de$ аналитическое продолжение совпадает с исходной функцией
на кривой $\ga_\tau(t)= \ga(t,\tau)$. Продолжая делать маленькие шаги по параметру $\tau$, дойдём до $\tau=1$
(снова пользуемся компактностью).
\end{proof}


Выпуклость, конечно, здесь на фиг не нужна. Нужна лишь односвязность. Надо делать так: пусть две кривые с
общими концами гомотопны в области ($\tau$ параметр гомотопии), тогда при малом изменении $\tau$ кривая тоже
меняется слабо (в силу компактности). В малой окрестности работает теорема единственности, и значит, при малом
сдвиге функция останется той же. Конечным числом малых сдвигов (опять в~силу компактности) можно накрыть весь
отрезок изменения параметра $\tau$).
\end{petit}

\begin{ex}
Пусть $a_1\sco a_n$ точки ветвления. В области $G\wo h$, где $h$ ломаная, соединяющая точки ветвления,
функция распадается на однозначные ветви. (внутри полученной односвязной области нет точек ветвления).
Например, для функции $\sqrt{z}$.
\end{ex}

\subsection{Модулярная функция}

Рассмотрим единичный круг $\De$ и вписанный равносторонний треугольник $T_0 := ABC$ с нулевыми углами. Он конформно
эквивалентен полуплоскости (по теореме Римана), причём существует единственное отображение,  переводящее вершины
в точки $0$, $1$ и $\bes$. По принципу симметрии можно продолжить это отображение, отразив треугольник относительно
его сторон. (при этом получится три треугольника $T_1^{(1)}$, $T_1^{(2)}$ и $T_1^{(3)}$.
При этом отраженные вершины попадут на границу круга (стороны треугольника и описанная окружность ортогональны,
а значит, дуга окружности $\pd \De$ переходит в дополнительную дугу). Продолженная функция отображает каждый из
треугольников $T_1^{(i)}$ на нижнюю полуплоскость.

Продолжая процесс отражения относительно сторон $T_1^{(i)}$, получим треугольники $T_2^{(i)}$, и так далее.
В итоге получим функцию $\mu$, голоморфную в круге (она и называется \emph{модулярной функцией}). Легко видеть, что
она не продолжается (даже по непрерывности) на границу круга, так как в сколь угодно малой окрестности границы она
принимает значения, близкие к $0$, $1$ и $\bes$. При этом в самом круге $\De$ она ни разу не принимает этих трех значений.

Теперь построим  функцию, обратную к модулярной. Рассмотрим ее ветвь $z = \mu^{-1}(w)$, голоморфную в верхней полуплоскости.
Она аналитически продолжается в нижнюю полуплоскость до отображения на объединение треугольника $T_0$ и одного из
треугольников $T_1^{(i)}$. Каждую из продолженных ветвей можно снова продолжить по симметрии в верхнюю полуплоскость
(тогда в неё будет переходить один из треугольников $T_2^{(i)}$, и так далее. Функция $\mu^{-1}$ будет бесконечнозначной,
и точки $0,1,\bes$ будут логарифмическими.

\begin{theorem}[Малая теорема Пикара]
Любая целая трансцендентная (е отличная от полинома) функция принимает бесконечное число раз каждое
комплексное значение, за исключением, быть может, одного.
\end{theorem}
\begin{proof}
Пусть целая функция $f(z)$ не принимает двух различных значений $a$ и $b$. Без ограничения общности можно считать, что это $0$ и $1$
(так как всегда можно перейти к функции $\frac{f(z)-a}{b-a}$. В окрестности произвольной точки $z_0$ функция $\ph := \mu^{-1}\circ f$
голоморфна ($\mu^{-1}$ некоторая ветвь функции, обратной к модулярной). В силу того, что функция $f$ целая, она  не принимает
значения $\bes$ (полюсов-то нету). Значит, она не принимает особых значений $0,1,\bes$, и композиция $\ph$ будет
голоморфной и однозначной во всём $\Cbb$ (ввиду односвязности $\Cbb$). Но все значений $\mu^{-1}$ лежат в единичном круге,
а тогда по теореме Лиувилля она постоянна. Значит, $f\equiv\const$.
\end{proof}

\begin{ex}
\begin{nums}{-2}
\item $f(z)=e^z \neq 0$ точка $w=0$ исключительное значение
\item $f(z)=\cos (z)$ не имеет исключительных значений  (обратная функция $z=\arccos w$ всюду определена).
\end{nums}
\end{ex}

\begin{df}
Значение, которое принимается функцией конечное число раз или не принимается вообще,
называется \emph{пикаровским исключительным значением}.
\end{df}

Справедливо и более сильное утверждение, которое мы доказывать не будем.
\begin{theorem}[Большая теорема Пикара]
Пусть $a$ существенно особая точка функции $f(z)$. Тогда для любой точки $w\in \Cbb$,
кроме разве лишь одного значения, найдётся последовательность $z_n \ra a$ такая, что $f(z_n)=w$ для всех~$n$.
\end{theorem}

\begin{ex}
Для функции $f(z)= e^{\frac{1}{z}}$ точка $z=0$ является существенной особой точкой.
\end{ex}

\section{Конформные отображения. Теорема Римана}

\subsection{Компактные семейства аналитических функций}

\subsubsection{Сходимость в топологии $\Oc(G)$}

\begin{df}
Говорят, что последовательность аналитических функций \emph{сходится} в области $G$, если она равномерно сходится
на каждом компакте, лежащем в области $G$.
\end{df}


Далее под сходимостью аналитических функций мы будем понимать только такую сходимость, поэтому
обозначение $f_n\ra f$ означает сходимость в топологии $\Oc(G)$.

Пусть $G\subs \Cbb$. На пространстве $\Oc(G)$ есть счётное семейство полунорм:
\eqn{\hn{f}_n := \bn{f^{(n)}}_{\Cb(G)}.}
Значит, наше пространство метризуемо:
\eqn{\rh(f,g):=\suml{n=1}{\infty} \frac{1}{2^n} \, \frac{\|f-g\|_n}{1+\|f-g\|_n}.}
Сходимость в $\Oc(G)$ эквивалентна сходимости по этой метрике.

\begin{petit}
Контрольный вопрос от злобного экзаменатора (на самом деле от Вани Вегнера): откуда следует эквивалентность?
\end{petit}

\begin{df} Семейство функций $E\subs \Oc(G)$ называется \emph{предкомпактным}, если из
любой последовательности $\hc{f_n}\subs E$ можно выделить сходящуюся подпоследовательность $f_{n_k}$:
\eqn{f_{n_k} \xra{\Oc(G)} f\in \Oc(G).}
Будем называть $E$ \emph{компактным}, если $\lim f_{n_k}\in E$.
\end{df}

\begin{petit}
Тут он долго разглагольствовал о терминологии, и говорил, что слово <<предкомпактный>> ему, видите ли, не нравится.
А нам очень даже нравится, и путаницы не возникает.
\end{petit}


\subsubsection{Критерий компактности}

\begin{df}
Семейство функций $\hc{f_\al}$ на области $G$ называется \emph{локально равномерно ограниченным},
если для любого компакта $K \Subset G$ найдётся константа $M_K$ такая, что $|f_\al(z)| \le M_K$
для $\fa z \in K$ и $\fa \al$.
\end{df}

\begin{df}
Семейство функций $\hc{f_\al}$ называется \emph{локально равностепенно непрерывным}, если для $\fa \ep > 0$ и
любого компакта $K \Subset G$ найдётся $\de$ такое, что при $\fa z_1,z_2\in K$, для которых $|z_1 - z_2| <\de$,
выполняется условие $|f_\al(z_1)-f_\al(z_2)|< \ep$ для любого $\al$.
\end{df}

\begin{df}
Пусть $K \subs \Cbb$. Тогда \emph{$\rh$ раздутием} множества $K$ назовём множество
\eqn{K_\rh := \cups{z \in K}\ol U_\rh(z).}
\end{df}

Смысл определения понятен: множество $K$ раздувается на величину, не превосходящую $\rh$.

\begin{lemma}
Если семейство функций $\hc{f_\al}$, голоморфных в области $G$, равномерно ограничено, то оно и равностепенно
непрерывно в ней.
\end{lemma}
\begin{proof}
Очевидно, расстояние от любого компакта $K$ до границы области больше нуля. Тогда найдётся достаточно малое число $\rh$,
при котором $K_\rh \Subset G$. В силу равномерной ограниченности найдётся число $M$, для которого $|f_\al(z)|\le M$ для $\fa z \in K_\rh$.

Рассмотрим произвольные точки $a,b\in K$ такие, что $|a-b|<\rh$. Тогда имеем $U_\rh(a) \subs K_\rh$ по построению $K_\rh$.
Значит,
\eqn{|f_\al(z)-f_\al(a)| \le |f_\al(z)| + |f_\al(a)|\le 2M}
для любой точки $z \in U_\rh(a)$ и для $\fa \al$. Теперь переведём этот круг в единичный круг $\De$ с центром
в нуле, то есть сделаем линейную замену переменной $\ze = \frac{1}{\rh}(z-a)$. Тогда функция
\eqn{g_\al(\ze) := \frac{1}{2M}\br{f_\al(a + \rh \ze)-f_\al(a)}}
будет удовлетворять условиям леммы Шварца, а значит, $|g_\al(\ze)| \le |\ze|$ для $\fa \ze \in \De$. Возвращаясь
к функции $f_\al$, получаем
\eqn{|f_\al(z)-f_\al(a)|\le \frac{2M}{\rh}|z-a| \text { для } \fa z \in U_\rh(a) \text { и для } \fa \al.}
Значит, выбирая достаточно малое $\de$, можно добиться того, чтобы выполнялось неравенство $|f(a)-f(b)|\le \ep$ при $|a-b|<\de$.
Например, можно взять $\de \le \min \hr{\rh, \frac{\ep\rh}{2M}}$.
\end{proof}

\begin{theorem}[Принцип компактности, теорема Монтеля]
Если семейство голоморфных функций $f_\al$ равномерно ограничено в области $G$, то оно предкомпактно в $G$.
\end{theorem}
\begin{proof}
По предыдущей лемме наше семейство функций будет равностепенно непрерывным.  Рассмотрим счётную всюду плотную
в $G$ последовательность точек $\hc{a_p}$ (скажем, точки с рациональными координатами).
Пусть $\hc{f_n}\subs \hc{f_\al}$ произвольная последовательность функций. Рассмотрим числовую
последовательность $\hc{f_n(a_1)}$. Она ограничена, а потому содержит сходящуюся подпоследовательность.
Пусть $\hc{n^{(1)}}$ подмножество её индексов. Далее, из этой последовательности выберем подпоследовательность,
сходящуюся на точке $a_2$, и так далее. В итоге получатся последовательности функций $f_n^{k}$, сходящиеся на первых $k$
точках из множества $\hc{a_p}$:
\begin{align*}
f_1^1, f_2^1, f_3^1, \ldots\\
f_1^2, f_2^2, f_3^2, \ldots\\
f_1^3, f_2^3, f_3^3, \ldots
\end{align*}
Тогда последовательность $f_n^n$ сойдётся на всех этих точках. Для краткости обозначим её снова через $f_n$.

Остаётся показать, что наша последовательность сходится на самом деле во всех точках компакта. В~силу
равностепенной непрерывности найдётся такое $\de$, что $|f_n(z_1)-f_n(z_2)|<\ep$ при
$|z_1-z_2|<\de$. Покроем компакт $K \subs G$ блинами $\De_i$ радиуса $\frac{\de}{2}$ и выделим конечное
подпокрытие. Рассмотрим произвольную точку $z$ в одном из блинов, тогда в силу равностепенной непрерывности
для $\fa z_1,z_2\in \De_i$ будет выполняться $|f_n(z_1)-f(z_2)|<\ep$. Запишем критерий Коши:
\begin{multline*}
|f_m(z)-f_n(z)| = |f_m(z) - f_m(a_j) + f_m(a_j) - f_n(a_j) + f_n(a_j) - f_n(z)|\le\\\le
 |f_m(z) - f_m(a_j)| + |f_m(a_j) - f_n(a_j)| + |f_n(a_j) - f_n(z)|.
\end{multline*}
Первое и третье слагаемое сколь угодно малы, так как точка $z$ приближается точками $a_j$ ввиду их всюду плотности.
Второе же слагаемое тоже можно устремить к нулю, так как на точках $a_j$ имеется сходимость. Осталось заметить, что
эта сходимость будет равномерной на $K$.
\end{proof}
\begin{note}
Очевидно, что верно и обратное утверждение (хотя оно нам не пригодится).
\end{note}

\subsubsection{Применения принципа компактности}

\begin{theorem}[Витали]
Пусть $G$ область, последовательность $\hc{f_n}\subs \Oc(G)$ предкомпактна, и $f_n$ сходится
в каждой точке подмножества $E\subs G$, имеющего предельную точку в $G$.
Тогда $f_n \ra f\in \Oc(G)$.
\end{theorem}
\begin{proof}
\markr Последовательность $\hc{f_n}$  предкомпактна, поэтому для любого компакта $K \subs G$ найдётся подпоследовательность
$\hc{n_k}$ такая, что $f_{n_k} \convu{K} f\in \Oc(G)$. Докажем, что и вся последовательность сходится
к $f$ в топологии $\Oc(G)$. Допустим, что сходимости нет: существует компакт $K^* \subs G$,
последовательность точек $z_{n'_k}$ и подпоследовательность $\bc{f_{n'_k}}$ такая, что
\eqn{\label{Condition}|f_{n'_k}(z_{n'_k})-f(z_{n'_k})|>d}
для некоторого $d>0$. Из последовательности $\bc{f_{n'_k}}$ в силу её компактности можно выделить
подпоследовательность $\bc{f_{n''_k}}$ такую, что
\eqn{f_{n''_k}\xrightarrow{\Oc(G)} g(z), \quad g(z)\in \Oc(G).}
Поскольку $g(z)=f(z)$ на множестве $E$ по условию, то по теореме единственности
$g \equiv f$ в области $G$. Значит, $f_{n''_k} \stackrel{K}{\rightrightarrows}f$, что противоречит~\eqref{Condition}.
\end{proof}

\begin{theorem}[Гурвица о нулях последовательности аналитических функций]
Пусть $f_n \in\Oc(G)$ и $f_n \ra f \in \Oc(G)$, где $f\not \equiv 0$.
Пусть $\ga$ спрямляемый жорданов контур в $G$, такой, что $\Int \ga \subs G$, и
при этом $f(z) \neq 0$ на $\ga$. Тогда $\exi N$ такое, что при $n > N$ функция $f_n$ имеет внутри $\ga$ столько
же нулей (с учетом кратности), сколько и функция $f$.
\end{theorem}
\begin{proof}
Положим $\de := \min \hc{|f(z)|,\, z \in \ga}$.
Так как $\ga$ компакт, то $f_n(z) \convu{\ga} f(z)$. Значит, найдётся $N$ такое, что при $n > N$
будет выполнено неравенство $|f_n(z)-f(z)|<\de$ для $\fa z \in \ga$.
Представим функцию в виде
\eqn{f(z)=f_n(z)+\br{f(z)-f_n(z)}.}
По теореме Руше функция $f_n(z)$  имеет внутри $\ga$  столько же нулей, сколько и $f(z)$,  что и требовалось.
\end{proof}

\begin{df}
Функция называется \emph{однолистной}, если она осуществляет инъективное отображение.
Функция $f(z)$ \emph{локально однолистна} в точке $z_0$, если она однолистна в $U(z_0)$.
\end{df}

\begin{theorem}[О последовательностях однолистных аналитических функций]
Пусть последовательность однолистных функций $f_n$ в области $G$ сходится к непостоянной функции $f(z)$.
Тогда $f(z)$ также однолистна в области $G$.
\end{theorem}
\begin{proof}
От противного: предположим, что существуют точки $a, b\in G$ такие, что $f(a)=f(b)=A$, но $a\neq b$.
Рассмотрим последовательность $f_n(z) - A$.
\eqn{f_n(z)-A \ra 0, \quad z\ra a,\; z\ra b,\; n \ra \infty.}
Нули функции $f(z)-A$ изолированы. Пусть $U_a$ и $U_b$ столь малые окрестности точек $a,b$, что их
замыкания не пересекаются и целиком лежат в области $G$, причем $f(z)-A\neq 0$ в $\dot U_a$ и в $\dot U_b$.
По теореме Гурвица при $n>N$ функция $f_n(z)-A$ имеет в $U_a$ и в $U_b$ по крайней мере по одному нулю.
Мы получили, что существуют точки $z_a \in U_a$ и $z_b \in U_b$ такие, что $f_n(z_a)-A=f_n(z_b)-A=0$. То есть
$z_a \neq z_b$, но $f_n(z_a)=f_n(z_b)$. Противоречие с однолистностью функции $f_n(z)$.
\end{proof}

\begin{ex}
Покажем, что условие $f(z)\neq\const$ существенно.
Пусть $f_n(z)=\frac{z}{n}$, а $G = \hc{|z| < 1}$. Тогда $f_n \xrightarrow{\Oc(G)} 0$ при $n\ra \infty$.
\end{ex}

\subsection{Отображения посредством аналитических функций}

\subsubsection{Лемма о локальном обращении и её трагические следствия}

\begin{lemma}[О локальном обращении]
Пусть функция $f\in \Oc\br{U(a)}$ непостоянна, причем $f(z)-f(a)$ имеет в точке $a$ нуль $k$ го порядка ($k \ge 1$).
Тогда найдутся $\ep_0, \de_0>0$ такие, что для $\fa \ep \in (0,\ep_0)$
найдётся такое $\de <\de_0$, что при $b = f(a)$ и $w \in U_\de(b)$  уравнение $f(z) = w$  имеет в окрестности $U_{\ep_0}(a)$  ровно $k$ корней с учетом кратности,
причем все корни лежат в окрестности $U_\ep(a)$. При этом, если $w\neq b$, то все эти корни простые
(а при $w=b$ это один $k$-кратный корень).
\end{lemma}
\begin{proof}
Существует $\ep_0$, такое что при $0<|z-a|\le\ep_0$, $f(z)\not=a$ и $f'(z)\not =0$.
(Если $f(z)=f(a)$  в сколь угодно близких к $a$  точках $U_R(a)$, то $f(z)\equiv \const$ по теореме единственности. Если $f'(z) = 0$  в сколь угодно близких к $a$ точках
$U_R(a)$, то из $f'=0$  следует $f\equiv \const$ (снова по теореме единственности).
Положим
\eqn{\de_0 := \min\hc{|f(z)-f(a)|\cln z \in \pd U_{\ep_0}(a)}.}
Рассмотрим точку $w \in U_{\de_0}(b)$. Надо решить уравнение $f(z)-w=0$. Применим теорему Руше:
\eqn{f(z)-w=\br{f(z)-f(a)}+(b-w), \quad |f(z)-f(a)|\ge \de_0 > |w-b|.}
Тогда функция $f(z)-w$  имеет в окрестности $U_{\ep_0}(a)$ столько же нулей,   сколько функция $f(z)-f(a)$,
то есть $k$ штук с учетом кратности (по условию и по выбору $\ep_0$).
Если $w\in U_{\de_0}(b),\, w\neq b$, то все эти нули простые, так как при $z\neq b$ имеем $f'(z) \neq 0$.
Далее все то же самое для $\ep<\ep_0$ и $\de<\de_0$.
\end{proof}

\begin{imp}
Образом области при конформном отображении является областью.
\end{imp}

Решение уравнения $f(z)-w=0$ есть обратная функция в точке $w$: $z=f^{-1}(w)$.
Обратная функция,  несмотря на многозначность,  является непрерывной.

\begin{theorem}[Принцип открытости]
Пусть функция $f\in \Oc(G)$ непостоянна, а $G'\subs G$ произвольное открытое подмножество в $G$. Тогда $f(G')$ также открыто.
\end{theorem}
\begin{theorem}[Принцип области]
Если функция $f \in \Oc(G)$ непостоянна, то $f(G)$ также область.
\end{theorem}
\begin{proof}
То, что множество $f(G)$ открыто, следует из принципа открытости. Докажем\footnote{На самом деле это
общий топологический факт: образ связного множества при непрерывном отображении связен.}, что $f(G)$ связно. Пусть
$b_1, b_2 \in f(G)$, а $a_1,a_2\in G$ какие-либо прообразы точек $b_1, b_2$ соответственно.
Так как $G$ область, то $a_1$ и $a_2$ можно соединить кривой $L = z(t)$, где $t\in [\al,\beta]$,
$z(t)\in G$ для всех $t\in [\al, \be]$, и $a_1=z(\al), a_2=z(\be)$.
Рассмотрим кривую
$w=f\br{z(t)}$, где $t\in [\al,\be]$, и $f(z(t))\in f(G)$ для всех $t \in [\al, \be]$,
и $b_1=f\br{z(\al)}, b_2=f\br{z(\be)}$.
Эта кривая непрерывна и соединяет $b_1$ и $b_2$ в $G$. Значит, $f(G)$ связно, что и требовалось доказать.
\end{proof}

В качестве ещё одного следствия можно доказать одну уже известную нам теорему.
\begin{imp}[Принцип максимума модуля]
Для любой непостоянной функции $f\in\Oc(G)$ и любой точки $a\in G$ найдётся точка $z\in G$ такая, что $|f(z)|>|f(a)|$.
\end{imp}

\subsubsection{Локальное обращение аналитических функций}

\begin{theorem}\label{TheoremaOdin}
Если функция $f\in \Oc\br{U(z_0)}$ и $f'(z_0)\neq 0$, то в некоторой окрестности точки $w_0 := f(z_0)$
определена обратная однолистная функция $f^{-1} \in \Oc\br{U(w_0)}$, причём $\hr{f^{-1}(w_0)}'=\frac{1}{f'(z_0)}$.
\end{theorem}
\begin{proof}
Воспользуемся леммой о локальном обращении. Из нее следует, что существует
непрерывная обратная функция $f^{-1}$ в окрестности $U_{\de_0}(w_0)$:
\begin{gather*}
\fa \ep \le \ep_0 \quad \exi \de>0 \colon
|w-w_0|<\de \Ra |z-z_0|<\ep\\
\exi  \left( f^{-1}(w_0)\right)'=\frac{1}{f'(z_0)}
\end{gather*}
Для точки $w'\in U_\de(w_0)$ применяем лемму и
получаем требуемое. Обратная функция будет однозначна и однолистна.
\end{proof}

\begin{imp}[Критерий локальной однолистности]
Функция $f \in \Oc\br{U(z_0)}$ локально однолистна тогда и только тогда, когда $f'(z_0)\neq 0$.
\end{imp}
\begin{proof}
В одну сторону утверждение уже было доказано. Наоборот: будем доказывать от противного. Пусть
$f'(z_0)=\dots = f^{(k-1)}(z_0)=0$, а $f^{(k)}(z_0)\neq 0$.
По лемме о локальном обращении в любой окрестности точки $w_0$ уравнение
$w=f(z)$ имеет $k$ корней. Поэтому в точке $w_0$ она $k$-листна.
Если $f(z)$ однолистна в области $G$, то $f'(z)\neq 0 \quad \fa z\in G$.
\end{proof}

\begin{theorem}\label{TheoremaDva}
Пусть функция $f(z)$ однозначна и аналитична в окрестности точки $z_0$, причем для некоторого $k\ge 2$
выполнено $f'(z_0)\bw=\dots =f^{(k-1)}(z_0)=0$, а $f^{(k)}(z_0)\neq 0$. Тогда в некоторой окрестности
$U_\de(w_0)$ существует $k$-значная аналитическая функция $f^{-1}(w)$, причем точка
$w_0$ для нее является алгебраическая точка ветвления порядка $k-1$. Функция $z=f^{-1}(w)$в этой
окрестности точки $w_0$ может быть записана в виде
\eqn{f^{-1}(w)=\ta(\sqrt[k]{w-w_0}),}
где $\ta(\ze)$ однозначна, аналитична, однолистна в окрестности точки $\ze=0$.
\end{theorem}

\begin{note}
Теорема \ref{TheoremaOdin} является частным случаем теоремы \ref{TheoremaDva} при $k=1$.
\end{note}

\begin{proof}
Введем промежуточную переменную:
\begin{gather*}
w=f(z)=w_0+c_k(z-z_0)^k+c_{k+1}(z-z_0)^{k+1}+\dots\\
\sqrt[k]{w-w_0}=\ze=(z-z_0)\sqrt[k]{c_k+c_{k+1}(z-z_0)+\dots}
\end{gather*}
Заметим,  что в окрестности точки $z_0$ правая функция распадается на однозначные ветви (всего $k$ штук):
$$c_k=\frac{f^{(k)}(z_0)}{k!}\neq 0.$$
Значит, это равенство между двумя многозначными функциями от $z$. У функции
\eqn{\ze=(z-z_0)\sqrt[k]{c_k+c_{k+1}(z-z_0)+\dots}}
в окрестности выделим какую-нибудь однозначную ветвь. Заметим, что $\ze'(z_0)\neq 0$. По теореме \ref{TheoremaOdin}
у этой функции существует обратная функция $z=\ta(\ze)$ в окрестности $U_{\de_0}(0)$.

Вспомним, что $\ze = \sqrt[k]{w-w_0}$, положим $r = \de_0^k$. Если $w \in U_r(w_0)$, то
$\ze\in U_{\de_0}(0)$, так как $|w-w_0| < r = \de_0^k$. Наша искомая обратная функция:
$f^{-1} = \ta\hr{\sqrt[k]{w-w_0}}$. Из однолистности $\ta$ и того, что для
$\sqrt[k]{w-w_0}$ точка ветвления $(k-1)$-го порядка следует, что $w_0$ точка
ветвления $(k-1)$-го порядка для функции $f^{-1}(w)$.
\end{proof}

\subsubsection{Критерий конформности в точке}

\begin{theorem}[Критерий конформности в точке]
Отображение $f\in \Oc\br{U(z_0)}$ конформно в точке $z_0$ тогда и только тогда, когда $f'(z_0)\neq 0$.
\end{theorem}
\begin{proof}
$\Lar$ Было доказано в 5-м семестре.
\medskip

$\Ra$ Допустим, $f'(z_0) = 0$. Тогда
\eqn{\De w = w-w_0 = c_k (z-z_0)^k + c_{k+1}(z-z_0)^{k+1}+ \dots,}
где $c_k \neq 0$ и  $k \ge 2$. Имеем:
\begin{gather*}
  \De w = c_k(\De z)^k\left(1+O(\De z)\right), \quad \De z = z-z_0;\\
  \Arg \De w=\Arg c_k+k\Arg\De z+O\hr{|\De z|}.
\end{gather*}

Отсюда видно,  что когда вектор $\De w$ вращается вокруг нуля,  то вектор $\De z$
вращается вокруг нуля в $k$ раз быстрее. Значит,  углы увеличиваются в $k$ раз. Поскольку
$k \ge 2$,  то возникает противоречие с конформностью в точке $z_0$.
\end{proof}

\subsection{Конформные отображения круговых областей}

Здесь и далее через $D$ обозначается единичный круг.

\subsubsection{Лемма Шварца}

\begin{lemma}[Шварца]
Пусть функция $f\cln D\ra D$ голоморфна на $D$. Пусть $f(0)=0$ и $|f(z)| \le 1$.
Тогда $|f(z)|\le |z|$, причём если существует точка $z_0\in D$ такая, что $|f(z_0)|=|z_0|$,
то $f(z)=e^{i\ta}z$ при некотором $\ta\in[0,2\pi)$.
\end{lemma}
\begin{proof}
Рассмотрим функцию $\ph(z) := \frac{f(z)}{z}$. Поскольку $f(0)=0$, то нуль будет устранимой точкой для $\ph(z)$.
Значит, $\ph$ голоморфна в круге $D$.

Возьмём замкнутый круг радиуса $\rh < 1$. По принципу максимума функция $\ph$ достигает своего максимума на
границе этого круга. Но так как $|f(z)| \le 1$, то
\eqn{|\ph(z)| = \hm{\frac{f(z)}{z}} \le \frac{1}{\rh}.}
Устремляя $\rh$ к единице, получаем $\bm{\frac{f(z)}{z}} \le 1$, следовательно, $|f(z)|\le|z|$.

Пусть теперь $|f(z_0)|=|z_0|$ в некоторой точке $z_0$. Из доказанного выше следует, что $|\ph| \le 1$.
В точке $z_0$ функция $|\ph|$ достигает значения $1$, а больше единицы быть не может. Значит, по принципу максимума $\ph=\const$
и $|\ph|=1$, то есть $\ph(z) = e^{i\ta}$. Тогда $f(z) = e^{i\ta}z$ поворот на угол $\ta$.
\end{proof}


\subsubsection{Автоморфизмы круговых областей}

\begin{df}
\emph{Круговые области} это круги, полуплоскости и внешности кругов.
\end{df}

\begin{theorem}
 Если аналитическая функция $w=f(z)$  однолистна и конформно отображает
 одну круговую область на другую,  то  $f(z)$ дробно линейная функция.
\end{theorem}

\begin{proof}
Сначала докажем это для автоморфизмов единичного круга. Пусть отображение $f$ однолистно
и конформно отображает круг $D$ на себя.
\begin{points}{-1}
\item Рассмотрим для начала случай, когда $f(0)=0$. Обратная функция аналитична, и
$|f^{-1}(w)|<1,\,f^{-1}(0)=0$. Тогда по лемме Шварца
\eqn{\bm{\hr{f^{-1}}'(0)}\le 1, \quad \br{f^{-1}(w)}'=\frac 1{f'(z)}.}
Значит, $\frac1{|f'(z)|}\le 1$, т.е. $|f'(0)|\ge1$.
Тогда  $|f'(0)|=1$.
А это равенство (также по лемме Шварца) возможно только в том случае,  если
$f(z) = e^{i\al}z$.
\item Теперь рассмотрим общий случай.

Пусть $f\cln G_1\ra G_2$, где $a\in G_1$ и $b = f(a)$. Поскольку $G_1$ и $G_2$ круговые области,
то найдутся дробно линейные отображения $L_i\cln D\ra G_i$ такие, что $L_1(0)=a$ и $L_2(0)=b$.
Положим
\eqn{F := L_2^{-1}\circ f \circ L_1.}
Это конформное однолистное отображение $D$ на себя, такое что $F(0)=0$. По пункту \pt{1} получаем
\markr
\begin{gather*}
\om = F(\ze)= e^{i\al}\ze,  \text{ е}\\
L_2^{-1}(f(L_1(\ze))) = e^{i\al}\ze,\quad z=L_1(\ze);\\
L_2^{-1}(f(z))=e^{i\al}L_1^{-1}(z);\\
f(z)=L_2(e^{i\al}L_1^{-1}(z)).
\end{gather*}
\end{points}
Так как $L_1$ и $L_2$ дробно-линейные функции, то и $f(z)$ дробно линейная функция. Теорема доказана.
\end{proof}

\subsection{Теорема Римана о конформном отображении}

\subsubsection{Доказательство теоремы Римана}

\begin{theorem}[Римана о конформных отображениях]
Пусть $D$ односвязная область. Тогда она конформно эквивалентна одному из следующих множеств:
\begin{items}{-2}
\item $\Card \pd D = 0 \; \Ra \; D \sim \ol{\Cbb}$;
\item $\Card \pd D = 1 \; \Ra \; D \sim \Cbb$;
\item $\Card \pd D > 1 \; \Ra \; D \sim \De := \hc{z\cln |z|<1}$.
\end{items}
\end{theorem}
\begin{proof}
В первом случае область $D$ просто совпадает с $\ol\Cbb$, и даже ничего отображать не надо, во втором достаточно загнать
единственную точку $a$ границы в бесконечность преобразованием $\frac{1}{z-a}$, и мы получим $\Cbb$.
Остался нетривиальный третий случай, когда точек на границе хотя бы две. Тогда загоним их в точки $0$ и $\bes$.
Теперь применим преобразование $\sqrt{z}$. Так как точки $0$ и $\bes$ не лежат в области, то отображение будет конформным
в силу того, что область $D$ односвязна и по теореме о монодромии функция $\sqrt{z}$
допускает выделение двух однозначных ветвей $\ph_1$ и $\ph_2$(отличающихся знаком). Значит, образы $\ph_1(D)$ и $\ph_2(D)$
не пересекаются (предположим противное, тогда $\ph_1(z_1)=\ph_2(z_2)$, а так как это ветви квадратного корня, то
$z_1=z_2$ и $\ph_1(z_1) = -\ph_2(z_1)$, чего быть не может, так как $\ph_i(z)\neq 0$ на области $D$).
Далее, рассмотрим область $\ph_2(D)$, и так как она открыта, то содержит некоторый круг с центром в точке $a$.
Сделаем преобразование $\frac{r^2}{z-a}$ (вывернем круг наизнанку), тогда образ $\ph_1(D)$ попадёт в этот круг.
Таким образом, можно считать, что область $D$ исходно содержалась
в некотором круге. Без ограничения общности можно считать, что это единичный круг с центром в нуле.

Теперь рассмотрим семейство $\Sc$ функций $f_\al$, однолистных в области $D$, причём  таких, что $|f_\al(z)| < 1$
при $\fa z \in D$, и $f(0)=0$. Найдём среди них функцию, у которой достигается максимум производной в нуле и покажем,
что это та самая функция, которая отображает область $D$ на единичный круг.

Пусть $f \in \hc{f_\al}$. По неравенству Коши $|f'(0)| \le \frac{1}{\rh}$, где $\rh$ ненулевой радиус
сходимости ряда для $f$, а значит, множество $\bc{|f_\al(0)|}$ ограничено и имеет верхнюю грань $\al$.
Пусть она достигается на некоторой последовательности $\hc{f_n}$. По теореме Монтеля наше семейство компактно,
а потому можно выделить сходящуюся к некоторой (голоморфной) функции $F$ подпоследовательность. Функция $F$ не постоянна,
так как $f'(0)\neq 0$ (функции то однолистные!). Остаётся показать, что она осуществляет отображение на весь круг.
Заметим сначала, что $F(0)=0$. В самом деле, пусть $F(0)=c\neq 0$. Тогда рассмотрим функцию
\eqn{g(z) := \frac{F(z) -c}{1 - \ol c F(z)}.}
Имеем
\eqn{|g'(0)| = \frac{1}{1- |c|^2} \cdot |F'(0)| \ge |F'(0)|.}
Это противоречит экстремальному свойству $F$, так как $|F(z)|< 1$ и стало быть, функция $g$ также попадёт
в наше семейство.

Пусть нашлась точка $b$ в круге, для которой $F(z)\neq b$. Рассмотрим функцию
\eqn{\psi(z) := \sqrt{\frac{F(z) -b}{1 - \ol b F(z)}}.}
Это композиция конформного автоморфизма, переводящего точку $b$ в нуль, с корнем. Так как <<симметричное>> к $b$
значение $b^* = \frac{1}{\ol b}$ функцией $F$ не принимается (оно вообще вне круга лежит), то у функции $\psi$
выделяется однозначная ветвь. Она опять таки лежит в семействе $\Sc$. Пусть $\psi(0) = d$. Тогда
функция
\eqn{h(z) := \frac{\psi(z) - d}{1 - \ol d \psi(z)}}
будет иметь производную в нуле побольше, чем у $F$:
\eqn{|h'(0)| = \frac{1 + |b|}{2\sqrt b} \cdot |F'(0)| > |F'(0)|,}
ибо $|b|<1$ и $1+|b|>2\sqrt b$. Получилось противоречие. Значит, $F(D) = \De$.
\end{proof}

\subsubsection{Соответствие границ при конформных отображениях}

\begin{theorem}
Пусть функция $f\cln G_1 \ra G_2$ однолистна и конформна. Тогда при
$z_n \ra \pd G_1$  $f(z_n)\ra \pd G_2$ (е$\rh (f(z_n), \pd G_2)\ra 0$).
\end{theorem}

\begin{proof}
Будем доказывать от противного: пусть $f(z_n)\not\ra \pd G_2$, тогда существует
подпоследовательность этой последовательности, т.ч. $f(z_{n_k})\ra w\in G_2$
Так как отображение, обратное к однолистному, однолистно,  то
$$\exi z_0\colon f(z_0)=w_0,  \; w_{n_k}\ra w_0.$$

Тогда и $z_{n_k}=f^{-1}(w_{n_k})\ra z_0$, что противоречит условию $z_n \ra \pd G_1$.
\end{proof}

Следующую теорему еще называют \emph{принципом соответствия границ}. Доказывать мы её не будем.
\begin{theorem}[Каратеодори]
\markr Любое конформное однолистное отображение $f\cln G_1\ra G_2$ жордановых областей в $\ol \Cbb$
продолжается до гомеоморфизма их замыканий $\ol G_1$ и~$\ol G_2$.
\end{theorem}

\begin{note}
При этом конформности на границе мы требовать не можем, там есть только непрерывность.
\end{note}

\begin{theorem}[Кэллог]
 Пусть $\ta (s)$ величина угла в дуговой абсциссе $s$.
Если $\ta^{(k)}(s)\in \Lip \al,  0<\al<1$, то $f(z)\in \Cb^{k+1}(\ol G_1)$, и $f^{(k+1)}\in \Lip \al$.
\end{theorem}

\subsubsection{Достаточные условия однолистности}

\begin{theorem}[Обратный принцип соответствия границ]
Пусть $G_1$ и $G_2$ жордановы области с границами $\ga_1$ и $\ga_2$ соответственно. Пусть
$f\cln G_1\ra G_2$ голоморфная функция, непрерывная на $\ol G_1$ и гомеоморфно отображающая $\ga_1$ на $\ga_2$.
Тогда функция $f$ однолистно и конформно отображает $G_1$ на $G_2$ и сохраняет направление обхода границ.
\end{theorem}
\begin{proof}
Положим $z_0\in G_1,\, z_1\in \ga_1$. Применим принцип аргумента: пусть $z$ бежит по $\ga_1$, тогда $w=f(z)$
бежит по $\ga_2$. Если $w_0=f(z_0)\in G_2$, то $\De_{\ga_1}\Arg(f(z)-f(z_0))=\pm 2 \pi$. Так как внутри контура
нет полюсов, вариант <<$-2\pi$>> невозможен. По принципу аргумента функция $f(z)-w_0=0$ имеет один корень.

Если же $w_0\in \Cbb\wo\ol G_2$, то $\De_{\ga_1}\Arg(f(z)-f(z_0))=0$, значит в $G_1$ нет решений уравнения
$f(z)-w_0=0$ ($w_0\in \Cbb\wo\ol G_2$). Поскольку внутренние точки переходят во внутренние, то $f(z_0)\in G_2$,
если $z_0\in G_1)$. Мы доказали однолистность и конформность. Сохранение направления обхода границ
следует из того, что $\frac1{2\pi}\De_{\ga_1}\Arg(f(z)-f(z_0))=+1$.
\end{proof}

\begin{imp}
При конформных отображениях жордановых областей сохраняется направление обхода их границы.
\end{imp}

\subsubsection{Условие единственности конформного отображения (условие нормировки)}

Пусть отображение $f\cln G_1\ra G_2$ однолистно и конформно. По теореме
Римана существует бесконечно много отображений $f$,  таких что
$f(z_1)=z_2$ (точки $z_1, z_2$ фиксированы).

Условие нормировки Пуанкаре: $\arg f'(z_1)=\al$.
Отображение $f$ с таким условием единственно (следует из теоремы Римана и леммы Шварца).

\textbf{Другие условия нормировки:}

\begin{points}{-1}
\item $a\mapsto b; \quad \arg f'(a)=\al\in[0,2\pi)$. (Пуанкаре)
\item $a\mapsto b; \quad \ze\in\Ga_1, \, \om\in\Ga_2; \quad f(\ze)=\om$.
\item $\ze_1, \ze_2, \ze_3\in\Ga_1;\; \om_1, \om_2, \om_3\in\Ga_2;\;\om_k=f(\ze_k)$,
причем согласованы направления следования в этих тройках.
\end{points}

При этом, условия \pt{2} и \pt{3} годятся только для жордановых областей.

\textbf{Конформно-инвариантные классы областей:}

\begin{points}{-1}
\item Области без границы $G=\ol\Cbb$. На область с одной или несколькими граничными точками нельзя отобразить,
поскольку образ компакта компакт.
\item Области, граница которых состоит из одной точки.
По предыдущему пункту их нельзя отобразить на $\ol\Cbb$. По теореме Лиувилля нельзя
отобразить на область с границей, содержащей более одной точки.
\item Области с границей, состоящей более чем из одной точки, конформно эквивалентны друг другу по теореме Римана.
\end{points}

\section{Гармонические функции}

\subsection{Гармонические функции двух переменных}

\subsubsection{Общее определение гармонической функции}

Пусть в области $G\subs\R^3$ задана функция $u(x_1,x_2,x_3)$. Она называется гармонической в $G$,  если:
\begin{points}{-1}
\item $u$ непрерывна в $G$;
\item
 Существуют $\frac{\pd u}{\pd x_k}$ и непрерывны в $G$ ($k=1,2,3$);
\item
 Существуют $\frac{\pd^2u}{\pd x_k\pd x_j}$ и непрерывны в $G$ ($k,j=1,2,3$);
\item
 $u$ удовлетворяет уравнению Лапласа
 $\De u=\frac{\pd^2 u}{\pd x_1^2}+\frac{\pd^2 u}{\pd x_2^2}+\frac{\pd^2 u}{\pd x_3^2}
 =0$.
\end{points}

Через $\Hc(G)$ будем обозначать линейное пространство функций, гармонических  в области $G$.

\subsubsection{Двумерный случай}

Пусть $f(z)=f(x+iy)=u(x,y)+iv(x,y)=u(z)+iv(z)$.
Если $f(z)$ аналитична в области $G$,  то существует
$f^{(k)}(z),\;\fa k=0,1,2,\dots$.
$$
  f'(z)=\frac{\pd f}{\pd x}=\frac1i \frac{\pd f}{\pd y}=-i\frac{\pd f}{\pd y}.
$$
Из существования производных функции $f(z)$ следует, что частные производные
по $x,y$ существуют и непрерывны.

Условия Коши Римана:
\eqn{\case{\frac{\pd u}{\pd x}=&\frac{\pd v}{\pd y},\\
\frac{\pd u}{\pd y}=&-\frac{\pd v}{\pd x}.}}
Поэтому $\De u=0$. Аналогично получаем $\De v=0$. Тогда  функции $u$ и $v$ гармоничны.

\begin{df}
Гармонические функции $u$, $v$, связанные системой Коши Римана, называются \emph{гармонически сопряженными}.
Точнее говоря, функция $v$ называется гармонически сопряженной к $u$. При $-u$ этом будет сопряжена к$v$.
\end{df}

Очевидно, функция $f(z)=u+iv$ аналитична тогда и только тогда, когда функция $v$ гармонически сопряжена к~$u$.

\begin{stm}
Пусть  $G$ односвязная область в $\Cbb$; $u \in \Hc(G)$.
Тогда существует семейство функций $\hc{v} \subs \Hc(G)$, гармонически сопряженных к~$u$.
Все они записываются в виде
\eqn{v(x,y)=C+\intl{(x_0,y_0)}{(x,y)}-\frac{\pd u}{\pd y}dx +\frac{\pd u}{\pd x}dy,}
где $z_0=(x_0, y_0)\in G, \; z=(x,y)\in G$ (интеграл не зависит от пути). $C\in \R$
\end{stm}
\begin{proof}
Рассмотрим систему
\eqn{\case{\frac{\pd u}{\pd x}=&\phm\frac{\pd v}{\pd y},\\
\frac{\pd u}{\pd y}=&-\frac{\pd v}{\pd x}.}}
Допустим,  что решение этой системы существует. Тогда
\eqn{v(x,y)-v(x_0,y_0)=\ints{L(z_0,z)}dv=\ints{L(z_0,z)}\frac{\pd v}{\pd x}\,dx +\frac{\pd v}{\pd y}\,dy.}
($L$ гладкая или кусочно-гладкая кривая, соединяющая $z$ и $z_0$). Подставляем
$\frac{\pd v}{\pd x}$ и $\frac{\pd v}{\pd y}$ из системы:
\eqn{v(x,y)-v(x_0,y_0)=\ints{L(z_0,z)}-\frac{\pd u}{\pd y}\,dx +\frac{\pd u}{\pd x}\,dy.}
Убедимся, что интеграл не зависит от пути интегрирования. Проверим условие Эйлера: если
$\frac{\pd P}{\pd y}=\frac{\pd Q}{\pd x}$, то интеграл $\int P\,dx+Q\,dy$ не зависит от пути.
В нашем случае
\eqn{P=\frac{\pd F}{\pd x},  \quad Q=\frac{\pd F}{\pd y},}
и
\eqn{\frac{\pd P}{\pd y}=\frac{\pd F}{\pd x \pd y}=\frac{\pd F}{\pd y \pd x}=\frac{\pd Q}{\pd x},}
е условие выполнено. Обозначая $C=v(x_0, y_0)$, немедленно получаем
\eqn{v(x,y)=C+\intl{(x_0,y_0)}{(x,y)}-\frac{\pd u}{\pd y}\,dx +\frac{\pd u}{\pd x}\,dy.}
Легко проверить, что эта функция удовлетворяет системе Коши Римана. Тогда функция $v$ гармонически сопряжена к $u$.
\end{proof}

Таким образом, для любой гармонической функции в односвязной области $G$ существует гармонически
сопряженная к ней функция $v$, причем $f=u+iv$ аналитическая функция.

\begin{df}
Функция $f=u+iv$ называется \emph{комплексным потенциалом}.
\end{df}

\subsubsection{Физическая интерпретация гармонических функций и комплексного потенциала}

Рассмотрим такое скалярное произведение:
\eqn{\begin{aligned}
(\nabla u, \phantom{-i} dz)&=\hr{\pf{u}{x}+i\pf{u}{y}, \; dx+i\dy}= \pf{u}{x}\dx+\pf{u}{y}\dy=du;\\
(\nabla u,          -i  dz)&=\hr{\pf{u}{x}+i\pf{u}{y}, \; dy-i dx}=\pf{u}{x}\dy-\pf{u}{y}\dx.
\end{aligned}}
Очевидно, $dz=\vec \tau\ds$ и $-idz=\vec n \ds$,
Полагаем $\vec \tau=dz/ds,\,ds=|dz|$ единичный вектор, а
$\vec {n}=-idz/ds$ перпендикуляр к нему.

Пусть $u$ гармоническая функция в односвязной области. Тогда функция
\eqn{v(x,y)=C+\intl{a}{z}-\frac{\pd u}{\pd y}\dx +\frac{\pd u}{\pd x}\dy, \quad C \in \R}
будет гармонически сопряженной к $u$.

\eqn{-\pf{u}{y}\,dx + \pf{u}{x}\,dy=(\nabla u, -i dz)= (\nabla u, \vec n\,ds)= (\nabla u,\vec n)\ds.}
Это поток вектора $\nabla u$ через элемент $ds$ (элемент длины).
$$v(z)-v(a)=N(L, \nabla u),$$
где $N(L,\vec v)$ поток вектора $\vec v$.

Итак, гармонически сопряженная к $u$ функция это поток вектора $\nabla u$.
\eqn{u(z)-u(a)=\intl{a}{z}\du=\intl{a}{z}\pf{u}{x}\dx+\pf{u}{y}\dy=
\intl{a}{z}(\nabla u, dz)=\intl{a}{z}(\nabla u,\tau)\ds.}
Если $\nabla u$ силовое поле потенциала $u$, то справа циркуляция векторного
поля $\nabla u$. Таким образом, если силовое поле $\nabla u$ безвихревое,  то
$$A(L,\overrightarrow{\nabla u})=0,$$
для замкнутого контура $L\subs G$ (область $G$ односвязная, функция
$u$ гармоническая, $A(L,\vec v)$ циркуляция вектора $\vec v$).


\subsection{Свойства гармонических функций}

\subsubsection{Инвариантность гармоничности при голоморфной замене переменных}

\begin{theorem}
Пусть $u(z) \in \Hc(G)$, функция $\ph(\ze)$ голоморфна в $G'$, причем $\ph(G')\subs G$.
Тогда функция $U(\ze)=u(\ph(\ze))$ гармонична в $G'$.
\end{theorem}
\begin{proof}
\begin{points}{-1}
\item $G$ односвязна. Строим функцию $f(z)$, такую что $\Re f(z)=u(z)$ в $G$.
Тогда  $U(\ze)=\Re f(\ph(\ze))$ гармоническая в $G'$.
\item $G$ неодносвязна. Возьмем $\ze_0\in G'$ и $\de>0$ столь малым, что
$|\ph(\ze)-\ph(\ze_0)|<\rh\br{\ph(\ze_0),\pd G}$. Пусть $G'$ это $\de$-окрестность точки
$\ze_0$. Тогда мы находимся в случае 1)$\dots$.
\end{points}
\end{proof}

\subsubsection{Принцип экстремума для гармонических функций}

\begin{theorem}
Пусть функция $u(z) \in \Hc(G)$ не постоянна. Тогда ни в какой внутренней точке
области $G$ она не принимает ни максимума, ни минимума (ни абсолютного, ни локального).
\end{theorem}

\begin{proof}
Пусть $a\in G$ фиксировано. Возьмем односвязную подобласть в $G$, содержащую $a$, и функцию $f(z)$,
такую что $\Re f(z)=u(z)$. Если бы $u$ была константой, то и $v$, а тогда по теореме единственности
и $f$ константа. Функция $w=f(z)$ переводит окрестность точки $a$ в область, содержащую $f(a)$.
Значит, есть точки, такие что $\Re f(z)>f(a)$ и $\Re f(z)<f(a)$. Аналогично и для функции $v$.
\end{proof}
\begin{imp}
Если $u(z) \in \Hc(G) \cap \Cb(\ol G)$, то максимума и минимума она достигает только на границе области $G$.
\end{imp}

\subsubsection{Теоремы о среднем}

Пусть $f\in \Oc\br{\ol U_R(a)}$. Запишем теорему Коши для этого круга:
\eqn{f(a)=\frac1{2\pi i}\ints{C_R} \frac{f(\ze)\dze}{\ze-a}= \frac1{2\pi}\intl{0}{2\pi}f(a+Re^{i\ta})\dta
=\frac1{2\pi R}\intl{0}{2\pi}f(a+Re^{i\ta})R\dta=\frac1{2\pi R}\ints{C_R}f(\ze)\dze.}

Это и есть первая теорема о среднем: среднее значение функции $f$ на круге $C_R$ равно значению функции в центре:
\eqn{2\pi r f(a)= r\intl{0}{2\pi}f(a+re^{i\ta})\dta.}
Проинтегрируем это равенство по радиусу:
\eqn{\pi R^2 f(a)=\intl{0}{R}\intl{0}{2\pi}f(a+re^{i\ta})r\dta \dr.}
Получаем вторую теорему о среднем:
\eqn{f(a)=\frac1{\pi R^2}\ints{C_R}f(\ze)\dsi.}

\begin{theorem}
Для того, чтобы непрерывная в области функция $u(z)$ была гармонической, необходимо и достаточно,
чтобы в этой области выполнялась вторая (а значит, и первая) теорема о среднем.
\end{theorem}
Доказательство будет дано позже.

\subsubsection{Аналитичность комплексно сопряженного градиента гармонической функции}

Пусть функция $u(z)$ гармонична в области $G$. Обозначим
$\nabla u :=\frac{\pd u}{\pd x}+i\frac{\pd u}{\pd y}$.
Покажем, что $\ol{\nabla u}=\pf{u}{x}-i \pf{u}{y}$ аналитическая функция.
Проверим условия Коши-Римана для $\ol{\nabla u}$:
\eqn{\case{\frac{\pd^2 u}{\pd x^2}=&-\frac{\pd^2 v}{\pd y^2},\\
\frac{\pd^2 u}{\pd x\pd y}=&-\br{-\frac{\pd^2 v}{\pd y\pd x}}.}}

Поскольку функция $u(z)$ гармонична, то эти условия выполнены. Значит функция $\ol{\nabla u}$
аналитическая и однозначная (голоморфная).

\subsection{Ещё несколько свойств гармонических функций}

\subsubsection{Теоремы единственности}

\begin{theorem}[1]
Пусть $u_1,u_2\in \Hc(G)$, и эти функции совпадают на подобласти $G'\subs G$. Тогда $u_1\equiv u_2$ в $G$.
\end{theorem}
\begin{proof}
Рассмотрим функцию $U(z) :=u_1(z)-u_2(z)$. Тогда $U(z)=0,\,z\in G'$. Тогда и
$\ol{\nabla U(z)}=0$ на $G'$. По теореме единственности для аналитических функций
$\ol{\nabla U(z)}\equiv 0$ на $G$. Получаем, $U\equiv0$ в $G$.
\end{proof}

\begin{theorem}[2]
 Если $u_1$, $u_2$ гармоничны в $G$ и
$\nabla u_1(z_n)=\nabla u_2(z_n), \, z_n\in G,\,z_n\ra a\in G$, то
\eqn{u_1(z)=u_2(z)+C, \quad C\in \R.}
\end{theorem}

\begin{proof}
Рассмотрим аналитическую функцию $f(z)=\ol{\nabla(u_1-u_2)}$. По условию: $f(z_n)=0,\,z_n\in
G,\,z_n\ra a\in G$. По теореме единственности, $f(z)\equiv0$ в $G$. Следовательно $\nabla
u_1=\nabla u_2$ в $G$. Значит $u_1-u_2=C,\,C=\const$.
\end{proof}

Если потребовалось, чтобы $u_1$ и $u_2$ совпали хотя бы в одной точке,
  то можно утверждать, что $u_1\equiv u_2$ в $G$.

\begin{theorem}[3]
Если $u_1,u_2\in\Hc(G)\cap \Cb(\ol G)$, причем $u_1(z)=u_2(z)$ при всех $z\in \pd G$, то $u_1\equiv u_2$ на $\ol G$.
\end{theorem}
\begin{proof}
Используем принцип максимума: $u(z):=u_1(z)-u_2(z)=0$ на $\pd G$ тогда $u\equiv 0$ на
$\ol G$.
\end{proof}

\subsubsection{Теоремы Лиувилля и Гарнака}

\begin{theorem}[Лиувилль]
Если функция $u(z)\in\Hc(\Cbb)$ ограничена хотя бы с одной стороны (е либо $u(z)<C$, либо $u(z)>C$),
то $u(z)\equiv\const$.
\end{theorem}
\begin{proof}
Докажем для случая $u(z)<C$. Рассмотрим функцию $f(z) :=e^{u+iv}$, где $v$ гармонически сопряженная к $u$. Тогда
\eqn{|f(z)|=e^{u}< e^C.}
По теореме Лиувилля $f(z)\equiv \const$, следовательно $e^u\equiv \const$. Значит, $u=\ln C$.

В случае $u(x,y)>C$ нужно рассмотреть функцию $f(z):=e^{-(u+iv)}$.
\end{proof}

\begin{theorem}[Гарнак]
Пусть функция $u(z)\in \Hc\br{\dot U_R(a)}$ ограничена. Тогда точка $a$ устранима,
то есть можно доопределить $u(z)$ в точке $a$ так, что $u(z)$ будет гармонической
во всей окрестности $U_R(a)$.
\end{theorem}
\begin{proof}
Пусть $|u(z)|\le M$, и пусть $z_0\in \dot U_R(a)$. Положим
\eqn{f(z)=u(z)+i\intl{z_0}{z}-\frac{\pd u}{\pd y}\,dx+\frac{\pd u}{\pd x}\,dy.}
Пусть $F(z)=e^{f(z)}$, $f(z)$ не зависит от пути.
Тогда $e^{-M}\le |F(z)|=e^{u(z)}\le e^M$ в $D^0(a,R)$.
Значит, $F(z)$ голоморфная и ограниченная в $D^0(a,R)$. Значит особенность в
точке $a$ устранимая. $F(a)\neq 0$. Тогда положим
$f(z)=(\Ln F(z)=f(z)+2\pi ki)_0, \, z\in D(a,R)$ ветвь логарифма,
соответствующая $k = 0$ в проколотой окрестности $D^0(a,R)$.

Если интеграл зависит от пути, то при каждом повороте вокруг $a$ появляется циклическая постоянная:
\eqn{\intl{z_0}{z}-\frac{\pd u}{\pd y}dx+\frac{\pd u}{\pd x}dy=v_0(z)+n\om, \quad  w\neq 0.}
Тогда $F(z)=\exp\hr{\frac{2\pi(u+iv)}{\om}}$ однозначная функция. Имеем
$|F(z)|=\exp\hr{\frac{2\pi u(z)}{\om}}$, и $\ln |F(z)|=\frac{2\pi}{\om}u(z)$.
Отсюда следует однозначность и гармоничность функции $u(z)$ во всей окрестности.
\end{proof}

\begin{note}(о циклической постоянной) $\ol{\nabla u}$ аналитическая функция.
\eqn{\int \ol{\nabla u}\, dz=   \int \frac{\pd u}{\pd x}dx-\frac{\pd u}{\pd y}dy+i\int -\frac{\pd u}{\pd y}dx+\frac{\pd u}{\pd x}dy.}
\eqn{\int -\frac{\pd u}{\pd y}dx+\frac{\pd u}{\pd x}dy=\Im\int\ol{\nabla u}\, dz= 2\pi i \Im \res \ol{\nabla u}.}
\end{note}

\subsubsection{Гармонические полиномы}

\begin{df}
\emph{Гармоническими полиномами} называются (как это ни парадоксально) полиномы, являющиеся гармоническими функциями.
\end{df}

\begin{ex}

\begin{points}{-1}
\item
Однородный полином:
$$
  \Re z^n = \Re(x+iy)^n=x^n-C_n^2x^{n-2}y^2+\dots
$$
\item
 $\Re c_k z^k$, где $c_k = a_k-ib_k$:
$$
  \Re(c_k z^k)=\Re((a_k-ib_k)r^k(\cos k\ph+i\sin k\ph))=
  r^k(a_k\cos k\ph+b_k\sin k\ph).
$$
\end{points}
\end{ex}
Легко видеть, что любой гармонический полином есть сумма однородных гармонических
полиномов.

\subsection{Задача Дирихле}

\subsubsection{Формула Пуассона}

Далее для краткости положим $U_R  := U_R(0)$, $C_R := C_R(0)$.

\begin{theorem} [формула Пуассона]
Если $u(z)$ гармонична в некоторой окрестности круга $\ol U_R$, то
\eqn{u(z)=u(re^{i\ph})=\frac1{2\pi}\intl{0}{2\pi}u(R e^{i\ph})\frac{R^2-r^2}{R^2+r^2-2rR\cos(\ta-\ph)}\dta.}
\end{theorem}

\begin{proof}
Можно взять круг $U_{R+\ep}$, в котором функция $u(z)$ гармонична, и $f(z)$, такая что $u(z)=\Re f(z)$.
Тогда по теореме Коши имеем
\eqn{f(z)=\frac1{2\pi i}\ints{C_R} \frac{f(\ze)}{\ze-z}\dze, \quad z\in U_R(a).}
Если $z\notin\ol U_R$, то интеграл в правой части равен
нулю по интегральной теореме Коши.
$$
  f(z)=\frac1{2\pi i}\ints{C_R}
  f(\ze)\left(\frac1{\ze-z}-\frac1{\ze-z^*}\right)\dze,
  \quad z^*=\frac{R^2}{\ol z}.
$$
Очевидно, $\ze^*=\ze$. С другой стороны, $\ze^*=\frac{R^2}{\ol\ze}$, значит
$\ze=\frac{R^2}{\ol\ze}$. Поэтому
\begin{multline}
\frac1{\ze-z}-\frac1{\ze-z^*}=\frac{z-z^*}{(\ze-z)(\ze-z^*)}=\frac{z\ol z-R^2}{(\ze-z)(\ze\ol z-R^2)}=
\frac{|z|^2-R^2}{(\ze-z)(\ol z-\ol\ze)}\cdot\frac1{\ze}=\\
=\frac{R^2-r^2}{(\ze-z)(\ol\ze-\ol z)}\cdot\frac1{\ze}=\frac{R^2-r^2}{|Re^{i\ta}-re^{i\ph}|^2}\cdot\frac1{\ze}=
\frac{R^2-r^2}{R^2+r^2-2Rr\cos(\ta-\ph)}\cdot\frac1{\ze}.
\end{multline}
Последнее равенство следует из теоремы косинусов. Поскольку
$\frac{d\ze}{\ze}=\frac{ie^{i\ta}\dta}{e^{i\ta}}=i\dta$, то
\eqn{f(re^{i\ph})=\frac1{2\pi}\intl0{2\pi}f(Re^{i\ph})\frac{R^2-r^2}{R^2+r^2-2Rr\cos(\ta-\ph)}\dta.}
\hfill\end{proof}

\begin{df}
\emph{Ядром Пуассона} называется функция
\eqn{P_R(r,\ta-\ph):=\frac{R^2-r^2}{R^2+r^2-2Rr\cos(\ta-\ph)}.}
\end{df}

Легко проверить, что $P_R(r,\ta-\ph)=\Re\frac{\ze+z}{\ze-z}$, где $\ze=Re^{i\ph}$ и $z=re^{i\ph}$.
Поэтому $P_R(r,\ta-\ph)$ гармонична в $D(0,R)$ для  любого фиксированного значения $\ta$.

\begin{df}
\emph{Ядром Шварца} называется функция
\eqn{\frac{\ze+z}{\ze-z}=\frac{Re^{i\ta}+re^{i\ph}}{Re^{i\ta}-re^{i\ph}}.}
\end{df}
Очевидно, ядро Шварца аналитично. Из формулы Пуассона можно получить весьма похожую \emph{формулу Шварца}:
\eqn{
\markr u(z)=\frac1{2\pi}\intl{0}{2\pi}u(Re^{i\ph})\frac{R^2-r^2}{R^2+r^2-2Rr\cos(\ta-\ph)}\dta=
\Re\frac1{2\pi}\intl{0}{2\pi}u(Re^{i\ta})\frac{Re^{i\ta}+z}{Re^{i\ta}-z}\dta.}

$$f(z)=\frac1{2\pi}\intl{0}{2\pi}u(Re^{i\ta})\frac{Re^{i\ta}+z}{Re^{i\ta}-z}\dta+i c,\quad c\in\R.$$
(по вещественной части восстанавливаем всю функцию). Заметим, что
\eqn{\frac{Re^{i\ta}+z}{Re^{i\ta}-z}=1+2z\frac1{\ze-z}}
и разложим $\frac1{\ze-z}$ в ряд. Проинтегрировав, получим разложение в  степенной ряд функции $u$.
Формулы для коэффициентов $a_k,b_k$ для тригонометрического ряда легко  получаются из коэффициентов
Коши для $c_k$ ($c_k=a_k-ib_k$).

\begin{theorem}[Пуассон]
Если функция $u(z)$ гармонична в $U_R$ и непрерывна на $\ol U_R$, то
\eqn{u(z)=u(re^{i\ph})=\frac1{2\pi}\intl{0}{2\pi}u(Re^{i\ta}) \frac{R^2-r^2}{R^2+r^2-2Rr\cos(\ta-\ph)}\dta.}
\end{theorem}
\begin{proof}
Рассмотрим граничную функцию $g(\ta)=u(Re^{i\ta})$. Она непрерывна и
 $2\pi$-периодична. Значит, существует последовательность
$T_n\stackrel{[o,2\pi)}{\rightrightarrows}g(\ta)$ ($T_n$ тригонометрические  полиномы)
\eqn{T_n(\ta)=\suml{k=0}{n}\hr{a_k^{(n)}\cos k\ta+b_k^{(n)}\sin k\ta}.}
Рассмотрим гармонический полином
\eqn{\tau_n(z):=\suml{k=0}{n}\hr{\frac r R}^k \hr{a_k^{(n)}\cos k\ta+b_k^{(n)}\sin k\ta}.}
Поскольку на границе сходимость равномерна,  то по принципу
максимума сходимость равномерна и во всем круге:
 $|\tau_{n'}-\tau_{n''}|\le\ep$ на $\pd U_R$, значит и на $\ol U_R$.
По формуле Пуассона, так как $\tau_n(Re^{i\ta})=T_n(\ta)$,
\eqn{\tau(re^{i\ph})=\frac1{2\pi}\intl{0}{2\pi}T_n(\ta)\frac{R^2-r^2}{R^2+r^2-2Rr\cos(\ta-\ph)}\dta.}
Переходя к пределу,  получаем
\eqn{u(z)=\frac1{2\pi}\intl{0}{2\pi}u(Re^{i\ta})\frac{R^2-r^2}{R^2+r^2-2Rr\cos(\ta-\ph)}\dta.}
Ядро ограничено снизу и сверху:
\eqn{\frac{R-r}{R+r}\le\frac{R^2-r^2}{R^2+r^2-2Rr\cos(\ta-\ph)}\le \frac{R+r}{R-r}.}
Поэтому переход к пределу законен.
\end{proof}

\subsubsection{Задача Дирихле}

Если $g(\ph)$ $2\pi$ периодическая непрерывная функция,  то функция
\eqn{u(re^{i\ph})=\frac1{2\pi}\intl{0}{2\pi}g(\ta)\frac{R^2-r^2}{R^2+r^2-2Rr\cos(\ta-\ph)}\dta, \quad u(Re^{i\ph})=g(\ph)}
гармонична в круге $U_R$. Она решает задачу Дирихле в круге с произвольными граничными данными.

Если же нам дана произвольная односвязная область, мы переведём её в круг конформным отображением $f$, для круга
напишем формулу Пуассона, а потом обратным отображением $f^{-1}$ перекинем решение на исходную область.
Как было замечено выше, голоморфная замена  переменной сохраняет гармоничность функции.

\subsubsection{Следствие формулы Пуассона}

\begin{theorem}
Пусть функция $u(z) \in \Hc(\ol D)$ непрерывна и $u(z)\in \Lip^\al$ при $\al>1$ на $\ol U_R$. Тогда
сопряжённая функция $v(z)\in \Lip^\al$ на $\ol U_R$.
\end{theorem}

\eqn{f(z)=\frac1{2\pi}\intl{0}{2\pi}u(Re^{i\ta})\frac{Re^{i\ta}+z}{Re^{i\ta}-z}\dta +i c, \quad c\in \R^1.}
Сделаем замену $\ze=Re^{i\ta}$, тогда $d\ze=Rie^{i\ta}\dta$:
\eqn{f(z)=\frac1{2\pi}\ints{C_R}u(\ze)\frac{\ze+z}{\ze-z}\cdot\frac{d\ze}{\ze}.}
Так как $\frac{\ze+z}{\ze-z}=1+2z\frac1{\ze-z}$, то
\eqn{f(z)=\frac1{2\pi i}\ints{C_R}\frac{u(\ze)}{\ze}\dze + 2z\frac1{2\pi i}
\ints{C_R}\frac{u(\ze)}{(\ze-z)\ze}\dze.}
Оба интеграла интегралы типа Коши. Можно разложить в степенной ряд по степеням $\frac z{\ze}$.
\eqn{z^n=r^n(\cos n\ta+i\sin n\ta).}
Отделяя $\Re$ и $\Im$, мы получаем тригонометрические ряды для функций $u$ и $v$.

\subsection{Гармоническое продолжение. Принцип отражения}

Пусть $u_1(z)$ гармонична в $G_1$, $u_2(z)$ гармонична в $G_2$.
Пусть $u_1(z)\equiv u_2(z),\; \fa z\in G_1\cap G_2$.
Тогда $u_1$ и $u_2$ гармонические продолжения одна другой.

Так же,  как для аналитических функций,  можно определить гармоническое продолжение,
риманову поверхность тд

Докажем аналог принципа симметрии Римана Шварца:

\begin{theorem}[Принцип отражения]
Пусть в состав границы жордановой области $G$ входит отрезок $\ga$ (пусть $G$ лежит в верхней
полуплоскости, а отрезок на вещественной прямой), и пусть гармоническая функция $u(z)$ непрерывна на $G\cup \ga$,
причем $u(z)=0$ при $z\in\ga$. Тогда функция $u(z)=u(x,y)$,  доопределённая в области $G^*$
как $u(x,y) := - u(x,-y)$ при $(x,y)\in G^*$, является гармонической в области $G\cup\ga\cup G^*$.
\end{theorem}
\begin{proof}
Проверим,  что $u(x,y)$ гармонична в области $G^{*}$:
\eqn{\De u=-\hr{\frac{\pd^2 u(x,-y)}{\pd x^2}+\frac{\pd^2u(x,-y)}{\pd y^2}}=
-\hr{\frac{\pd^2u(x,y)}{\pd x^2}+\frac{\pd^2u(x,y)}{\pd y^2}}=0.}
Пусть $a\in\ga$. Выберем $R$ так, чтобы $\ol U_R(a)\subs G\cup\ga\cup G^*$. Решим
задачу Дирихле для области $G\cup\ga\cup G^*$ с граничными условиями, задаваемыми функцией (продолженной).
Решение задается интегралом Пуассона:
\eqn{u(z=a+re^{i\ph})=\frac1{2\pi}\intl{\pi}{-\pi}u(a + Re^{i\ta})\frac{R^2-r^2}{R^2+r^2-2Rr\cos(\ta-\ph)}\dta.}
Докажем, что $u(z)=0$ при $z\in (a-R,a+R)$ (на диаметре круга $U_R$):
при этих $z$ имеем $\ph=0$ либо $\ph=\pm\pi$. Подынтегральная функция $u(a+ Re^{i\ta})$ нечетная функция,
а ядро Пуассона четная функция. Значит интеграл равен нулю. Итак, $u(z)=0$  на $\ga\cap U_R$.

Таким образом,  построенная функция совпадает с <<верхней>> функцией на верхней
полуокружности,  с <<нижней>> на нижней и равна $0$ на $\ga\cap U_R$. Поскольку $a$
произвольно,  то всё доказано.
\end{proof}

\begin{ex}
Потенциалы полей.

При отображении $w=L(z)$ гармоничность сохраняется,  поэтому можно восстановить
силовые линии на исходной картинке, а также эквипотенциальные поверхности.
\end{ex}

\section{Операционное исчисление}

\subsection{А на фига оно надо?}

Ещё в глубокой древности было замечено, что некоторые интегральные преобразования позволяют от уравнений одного типа
перейти к уравнениям другого типа. Так, например, преобразование Лапласа, о котором речь пойдёт ниже, позволяет свести
дифференциальное уравнение к алгебраическому. Правда, от этого легче не становится, ибо алгебраические уравнения мы
тоже решать не умеем (и это даже математически строго доказано\footnote{Теорема Абеля о неразрешимости уравнений
выше 4 степени в радикалах доказывается, между прочим, методами комплексного анализа, со ссылкой на неразрешимость
группы $\Sb_5$.}). Но кое-какая польза от этого есть (например, удобнее решать уравнения с разрывной правой частью).

\subsection{Определение преобразования Лапласа и его обращение}

\begin{df}
\emph{Функцией оригиналом} будем называть комплекснозначную функцию $f(t)$, для которой:
\begin{points}{-1}
\item Функция удовлетворяет условию Гёльдера с показателем $\al$ всюду  кроме конечного числа точек разрыва первого рода;
\item $f(t)=0$ при $t < 0$;
\item $|f(t)| \le M e^{s_0 t}$ при некоторых $M,s_0 \ge 0$ (число $s_0$ называют \emph{показателем роста}).
\end{points}
\end{df}

\begin{df}
\emph{Изображением функции (по Лапласу)} будем называть функцию
\eqn{F(p) := \intl{0}{\bes}f(t)e^{-pt}\dt.}
\end{df}

\begin{theorem}
Для всякого оригинала изображение определено в полуплоскости $\Re p \ge s_0$ и аналитично в ней.
\end{theorem}
\begin{proof}
Пусть $s := \Re p > s_0$. Тогда наш интеграл (и все его производные по $p$) мажорируется сходящимися интегралами:
\eqn{\label{eqn:Laplace}\bbm{\frac{d^k}{dp^k}\intl{0}{\bes}f(t)e^{-pt}\dt} = \bbm{\intl{0}{\bes}f(t)t^k e^{-pt}\dt}\le \intl{0}{\bes}Mt^ke^{-(s-s_0)}\dt = \frac{M}{(s-s_0)^{k+1}}.}
Ну и всё. Теорема доказана, однако.
\end{proof}
\begin{imp}Если $s := \Re p\ra \bes$ при $p\ra\bes$, то $\liml{s\ra\bes}F(p) = 0$.
\end{imp}
\begin{proof}
Следует непосредственно из выкладки~\eqref{eqn:Laplace}.
\end{proof}

\begin{theorem}[Формула обращения]
Если функция $f(t)$ является оригиналом, а $F(p)$ её изображение, то в точках непрерывности имеет место формула
\eqn{\label{eqn:reversing}f(t) = \vp\frac{1}{2\pi i} \intl{a-i\bes}{a+i\bes}e^{pt}F(p)\,dp.}
\end{theorem}

\begin{theorem}Если функция аналитична в полуплоскости $\Re p \ge s_0$, стремится к нулю при $|p|\ra\bes$ в $\hc{\Re p \ge a > s_0}$
равномерно относительно $\arg p$, и интеграл
\eqn{\intl{a-i\bes}{a+i\bes}e^{pt}F(p)\,dp}
абсолютно сходится, то применима формула обращения и функция $f(t)$ восстанавливается по формуле \eqref{eqn:reversing}.
\end{theorem}

\subsection{Свойства преобразования Лапласа}

\def\rde{\risingdotseq}

К очевидным свойствам относятся линейность и свойство подобия: $f(\al t) \risingdotseq \frac1\al F\hr{\frac p\al}$.
К чуть менее очевидным относятся такое свойство:
\eqn{f'(t) \rde p F(p) - f(0), \quad f^{(n)} \rde p^{n-1}f(0) - p^{n-2}f'(0) -\dots-f^{(n-1)}(0).}
Разумеется, здесь под значением $f^{(k)}(0)$ понимается правый предел. Доказательство очевидно и сводится к тупому
дифференцированию интеграла.


Аналогично доказывается свойство
\eqn{F^{(n)}(p) \rde (-1)^n t^n f(t).}

\begin{stm}[Теоремы запаздывания и смещения]
Имеют место соотношения
\eqn{f(\tau-t)\rde e^{-p\tau}F(p), \quad e^{p_0 t}f(t)\rde F(p-p_0).}
\end{stm}
\begin{proof}
Громко сказано: <<теоремы>>\dots\ Тупо вычисляем, сделав линейную замену переменной.
\end{proof}

\begin{theorem}[Теорема умножения] Произведение изображений также есть изображение, причём
\eqn{F(p)G(p)\rde \intl{0}{t}f(\tau)g(t-\tau)\,d\tau.}
\end{theorem}
\begin{proof}
Доказывается абсолютно так же, как и про преобразование Фурье, применяя тяжелую артиллерию типа теоремы Фубини.
\end{proof}

\begin{problem}[Бородин]
Существует ли неподвижная точка преобразования Лапласа, то есть функция $f(t)\not\equiv 0$, непрерывная
и ограниченная при $t>0$, такая, что при всех $p>0$ верно равенство
\eqn{f(p) = \intl{0}{\bes} f(t)e^{-pt}\dt?}

\end{problem}

\end{document}

%% Local Variables:
%% eval: (setq compile-command (concat "latex  -halt-on-error -file-line-error  " (buffer-name)))
%% End:
