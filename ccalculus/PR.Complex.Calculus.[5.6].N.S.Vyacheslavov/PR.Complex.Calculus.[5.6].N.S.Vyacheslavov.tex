\documentclass[a4paper]{article}
\usepackage[thmnormal,simple]{dmvn}

\title{Задачи с зачётов по комплексному анализу}
\author{Преподаватель\т Николай Степанович Вячеславов}
\date{5--6 семестр, 2004--2005~г.}

\begin{document}
\maketitle

\section{V семестр}

\subsection{Теорема Руше и принцип аргумента}

\begin{problem}
Доказать, что уравнение $z \sin z = 1$ имеет только вещественные корни.
\end{problem}

\begin{problem}
Доказать, что уравнение $az^3-z+b = e^{-z}(z+2)$ не имеет корней в области $\Rea z > 0$ при ${a>0}$ и~${b > 2}$.
\end{problem}

\begin{problem}
Доказать, что уравнение $az^n+z+1 = 0$ имеет корень в круге $|z|< 2$ при $n \ge 2$ и всяком $a \in \Cbb$.
\end{problem}

\begin{problem}
Найти число корней многочлена $z^4+z^3-4z+1$ в круге $|z| < 2$.
\end{problem}

\begin{problem}
Найти число корней многочлена $z^4+iz^3-7z^2+6z-\frac{27}{8}i$ в IV квадранте ($\Rea z > 0, \Img z < 0$).
\end{problem}

\subsection{Особые точки однозначного характера}

Найти все особые точки функций и указать их тип:

$$\frac{z^2}{(z^2+1)^2}, \qquad \frac{1}{e^z+1} - \frac{1}{\sh z}, \qquad e^{\tg z}, \qquad \sin\frac{1}{\sin\frac1z}$$
$$\ctg z - \frac{1}{\sin z}, \qquad \frac{z^7}{(z^2-4)\cos\frac1{z-2}}, \qquad \frac{z^3\sin\frac1z}{z^2-a^2}, \quad a \in \Cbb,\qquad \frac{z^2}{e^{z^2}-1},$$
$$\frac{\sin \pi z}{z^4-z^2}, \qquad \frac{z^2+4\pi^2}{\sin z(e^z-1)},\qquad \hr{z^2-\frac1{z^2}}\sin\frac{\pi z}{z^2+1},\qquad \frac{1}{\cos (e^z)},$$
$$\frac{(z^2-z-2)^2}{(z^-1)(z-2)^3},\qquad \frac{1}{\sin z - \sin a}, \quad a\in\R,\qquad \frac{\sin \pi z}{(z^2-1)^7},$$
$$\frac{\sh z}{z^2(z^2+\pi^2)},\qquad (z^3-z)\cos\frac{1}{z-2},\qquad \frac{z^2+\pi^2}{\sh^2z}.$$

\subsection{Вычеты}

Найти вычеты функций во всех особых точках:

\eqn{\ctg^2 z,\qquad \frac{z^{35}}{1-z^{16}},\qquad \frac{1}{\sin z (e^z-1)},}
\eqn{z^2e^{\frac{1}{z+1}},\qquad \frac{z^2+1}{e^{\pi z}+1},\qquad \frac{1}{\sin z \sh z},}
\eqn{\frac{z^2-1}{z\sin^2 z},\qquad \frac{1}{\sin z \sh z},\qquad \frac{z}{1-\cos z}.}

\subsection{Интегралы}

\begin{problem}
Вычислить интеграл $\intl{0}{\bes}\frac{\cos(\la x)}{(x^2+1)(x^2+2)}\,dx$, $\la \in \R$.
\end{problem}

\begin{problem}
Вычислить интеграл $\intl{0}{\bes}\frac{\cos(\la x)}{(x^2+1)^2}\,dx$, $\la \in \R$.
\end{problem}


\begin{problem}
Вычислить интеграл $\ints{|z|=2}\frac{z^{47}}{1-z^{16}}\,dz$.
\end{problem}

\begin{problem}
Вычислить интеграл $\intl{0}{\bes}\frac{x^2+1}{x^4+1}\,dx$.
\end{problem}

\begin{problem}
Вычислить интеграл $\ints{|z|=2}z\sin\frac{z+1}{z-1}\,dz$.
\end{problem}

\begin{problem}
Вычислить интеграл $\ints{|z|=4}\frac{z^2}{e^{z^2}-1}\,dz$.
\end{problem}


\subsection{Разложение в ряд}

\begin{problem}
Разложить в ряд c центром в точке $z=2$ функцию $\cos \frac{z^2-4z}{(z-2)^2}$.
\end{problem}

\begin{problem}
Разложить в ряд в кольце $0<|z-1|<\sqrt2$ функцию $\frac{z^2-1}{z^2+1}$.
\end{problem}

\begin{problem}
Разложить в ряд в кольце $1<|z|<2$ функцию $\frac{z^2-z+3}{z^3-3z+2}$.
\end{problem}

\begin{problem}
Разложить в ряд в кольце $2<|z|<\infty$ функцию $\frac{z^5}{z^2-4}$.
\end{problem}

\begin{problem}
Разложить в ряд в кольце $0<|z|<\infty$ функцию $\sin z\sin\frac1z$.
\end{problem}

\begin{problem}
Разложить в ряд в кольце $0<|z-1|<\infty$ функцию $z^2\sin\frac{1}{z-1}$.
\end{problem}

\begin{problem}
Разложить в ряд в кольце $2<|z|<\infty$ функцию $\frac{z^4+1}{z^2-z-2}$.
\end{problem}

\begin{problem}
Разложить в ряд в кольце $0<|z|<\infty$ функцию $\frac1z\sin^2\frac2z$.
\end{problem}

\section{VI семестр}

\subsection{Интегралы}

Вычислить несобственные интегралы

$$\intl{0}{\bes}\frac{x\cos ax\,dx}{(x-1)(x^2+1)^2},\quad a \in \R.$$
$$\intl{0}{\bes}\frac{\ln^3x\,dx}{x^2-1}.$$
$$\ints{\R}\frac{e^{ax} - e^{bx}}{1-e^x}\,dx,\quad a,b \in (0,1).$$
$$\intl{0}{\bes}\frac{\ln x\,dx}{x^p(x-1)}.$$
$$\intl{0}{\bes}\frac{x^p\,dx}{x^2+2x\cos \la + 1}, \quad \la\in (-\pi,\pi), \quad p \in (-1,1).$$
$$\intl{0}{\bes}\frac{\cos x - e^{-x}}{x}\,dx.$$

\subsection{Особые точки многозначного характера}

Найти все особые точки и указать их тип:

\subsubsection{Вариант 1}

$$\frac{1}{\sqrt{z}},\qquad \cos\hr{\frac{i\ln z}{8}}, \qquad \exp\hr{\frac{1}{\sqrt z - 1}},$$
$$\sqrt[18]{z^2(z^2-1)^3(z+1)^6\sin z^2},\qquad \sqrt[3]{\sqrt z - 4},\qquad \sqrt[6]{(z-1)\hr{\sqrt[3]{z} - \sqrt{2-z}}}.$$

\subsubsection{Вариант 2}

$$\frac{1}{\cos\sqrt z},\qquad \sqrt[n]{e^z}, \quad n \in \N \wo \hc{1},\qquad z^i,$$
$$\sqrt{\sqrt[3]{z} + 1},\qquad  \sqrt[3]{\frac{(z^4-16\pi^4)^3}{z^2(1-\cos z)(z^2-4\pi^2)^2}},\qquad  \sqrt[6]{(z-1)\hr{\sqrt[3]{z} - \sqrt{2-z}}}.$$

\medskip
\dmvntrail

\end{document}
