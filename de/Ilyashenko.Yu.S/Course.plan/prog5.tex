\input amstex
\documentstyle{amsppt}

\NoBlackBoxes \TagsOnRight \magnification = \magstep1 \topmatter
\title       Примерный план занятия 10
\endtitle
\endtopmatter


\document
0. 2-4 примера из NN 1194-1210.

1. Вывести уравнение равномерного движения частиц в
бесстолкновительной
%\newline
 среде;
    $u(x,t)$ -- скорость равномерно движущейся частицы, пролетающей через точку
$x$
 в момент времени $t$.

  Ответ:   $u_t+uu_x=0$.

2.  Решить задачу Коши для уравнения $u_t+uu_x=0,$ $
u|_{t=0}=f(x)$,
       и указать максимальный отрезок $[a,b]$, $a\le 0\le b$, для которого
     решения существуют при всех $(t,x),$ $t\in [a,b]$:

$\qquad$a) $f(x)=1$;

$\qquad$б) $f(x)=x$;

$\qquad$в) $f(x)=-x$;

$\qquad$г) $f(x)=\sin x$.


3. При каких $f$ в задаче 2 решение определено на всей плоскости
$(x,t)$?

4.  Написать квазилинейное уравнение, характеристики которого
являются интегральными кривыми уравнения $\dot x =f(x,u),\ \dot
y=g(x,u).$

5. Решить задачу Коши для уравнения $u_t-uu_x=x,$ $
u|_{t=0}=f(x)$,
       и указать максимальный отрезок $[a,b]$, $a\le 0\le b$, для которого
     решения существуют при всех $(t,x),$ $t\in [a,b]$:

$\qquad$a) $f(x)=1$;

$\qquad$б) $f(x)=\alpha x$.

Решить. Исследовать ответ в зависимости от $\alpha$.


6. Решить задачу Коши $u_t-uu_x=\sin x,$ $u|_{t=0}=2\cos \frac
{x}{2}.$

    Ответ: $u(x,t)=2\cos \frac{g^t x}{2}$, где $g^t$ -- фазовый поток уравнения
$\dot x =2\cos \frac{x}{2}$.

\enddocument
