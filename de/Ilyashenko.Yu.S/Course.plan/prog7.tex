\input amstex
\documentstyle{amsppt}

\NoBlackBoxes \TagsOnRight \magnification = \magstep1 \topmatter
\title       Примерный план занятия 13. \  Малые колебания.
\endtitle
\endtopmatter


\document

1.  Оператор $A$  симметричен и имеет базис из собственных
векторов $\xi^j$ с собственными значениями
$\lambda_j=\omega_j^2>0$. Написать общее решение системы
$$
\ddot x=-\text{grad}\, \frac12(Ax,x).  \tag 1
$$

2.  Написать решение системы (1) с начальным условием

$\qquad$a) $x(0)=0,\ \dot x(0)=e_1$\ \ (2)

$\qquad$б) $ x(0)=e_1,\ \dot x(0)=0$\ \ (3)

3.  Нарисовать траектории системы (1) с начальными условиями (2) и
(3) при

$A=\pmatrix 5&3\\ 3& 5\endpmatrix$, $A=\pmatrix 5&4\\ 4&
5\endpmatrix$, $A=\pmatrix 13&5\\ 5& 13\endpmatrix$.

4. Является ли положение равновесия системы (1) устойчивым?
Асимптотически устойчивым?

5. $\ddot x=-\text{grad} U(x), x\in \Bbb R^1.$

a)* Функция $U$ имеет в нуле строгий локальный минимум. Доказать,
что 0 -- устойчивое положение равновесия.

б)** Верно ли обратное утверждение для $U\ne const$, $U\in
C^{\infty}$? $U$ аналитическая?

6.* Найти замыкание орбиты системы $\dot \varphi=\omega,\
\omega=(1,1,\sqrt{2}),\ \varphi\in \Bbb T^3,$ с начальным условием
$\varphi (0)=0$.

7. Найти временное среднее функции $e^{2i\varphi}$ для поворота
$\varphi\mapsto \varphi +2\pi \alpha,\ \varphi \in S^1=\Bbb
R^1/2\pi\Bbb Z,$

$\alpha=\frac 12; \frac 13; \sqrt 2;\sqrt2+\sqrt 3.$

\enddocument
