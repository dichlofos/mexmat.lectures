\documentclass[a4paper]{article}
\usepackage[utf,simple]{dmvn}

\title{Обыкновенные дифференциальные уравнения}
\author{Лекторы: \framebox{В.\,А.\,Кондратьев,} Ю.\,С.\,Ильяшенко}
\date{III--IV семестры, программа экзамена 2003--2004~г, варианты 2001--2009 г.}

\newcommand{\skill}[1]{\textsf{(#1):}}

\begin{document}
\maketitle

\section{Программа экзамена}

\subsection{Первый семестр}

\subsubsection*{Введение. Примеры. Элементарные методы интегрирования}
\begin{enumerate}
\setlength\itemsep{-1.5mm}
\item Уравнения с разделяющимися переменными. Декартовы произведения двух систем.
\item Однородные уравнения. Их группа симметрий.
\item Линейные уравнения первого порядка. Преобразования монодромии и периодические решения
линейных уравнений с периодическими коэффициентами.
\item Уравнения в полных дифференциалах. Гамильтоновы уравнения с одной степенью свободы. Маятник.
\end{enumerate}

\subsubsection{Теорема существования}
\begin{enumerate}
\setlength\itemsep{-1.5mm}
\item Принцип сжимающих отображений.
\item Теорема существования, единственности и непрерывной зависимости решений от начальных условий. Метод Пикара.
\item Сходимость пикаровских приближений к решению (будет использована во втором семестре при доказательстве гладкой
зависимости решения от начального условия и теоремы о выпрямлении).
\item Теорема о продолжении интегральных и фазовых кривых. Её применение к линейным неавтономным системам.
Формула Лиувилля~-- Остроградского.
\end{enumerate}

\subsubsection{Линейные уравнения с постоянными коэффициентами}
\begin{enumerate}
\setlength\itemsep{-1.5mm}
\item Однородные уравнения и уравнения со специальной правой частью.
\item Резонансы. Метод комплексных амплитуд.
\end{enumerate}

\subsection{Второй семестр}

\subsubsection{Линейные системы}
\begin{enumerate}
\setlength\itemsep{-1.5mm}
\item Фазовые потоки. Экспонента линейного оператора.
\item Комплексификация и овеществление. Вычисление экспоненты. Экспонента комплексного числа. Экспонента жордановой клетки.
\end{enumerate}

\subsubsection{Теорема о выпрямлении и её следствия}
\begin{enumerate}
\setlength\itemsep{-1.5mm}
\item Теорема существования и единственности (напоминание). Пикаровские приближения.
\item Производное отображение. Уравнение в вариациях по начальным условиям и параметрам.
Гладкая зависимость решений от начальных условий и параметров.
\item Теорема о выпрямлении и её следствия. Полная система первых интегралов. Задача Коши для
линейных и квазилинейных уравнений. Искажение фазового объёма.
\end{enumerate}

\subsubsection{Устойчивость. Фазовая плоскость}
\begin{enumerate}
\setlength\itemsep{-1.5mm}
\item Устойчивость особых точек дифференциальных уравнений и неподвижных точек отображений.
\item Фазовая плоскость. Топология фазовых кривых. Отображение Пуанкаре. Предельные циклы. Теорема Флоке.
\end{enumerate}

\subsubsection{Детерминизм и хаос}
\begin{enumerate}
\setlength\itemsep{-1.5mm}
\item Малые колебания. Плотные обмотки тора. Равенство пространственных и временных средних для
иррационального поворота окружности.
\item Подкова Смейла. Элементы символической динамики.
\end{enumerate}

\section{Правила игры}

Критерий выставления оценок, как правило, следующий:
\begin{center}
\begin{tabular}{|c|c|c|}
\hline $\ge 9$ & $\ge 14$ & $\ge 20$\\
\hline <<3>>   & <<4>>    & <<5>>\\
\hline
\end{tabular}
\end{center}
В каждом варианте имеется не менее одного теоретического вопроса\footnote{Как показывает
статистика, вопрос берётся из программы 4 семестра.}. Для
получения оценки <<5>> необходимо правильно ответить хотя бы на один теоретический вопрос
или набрать не менее 25 баллов.

В случае, когда задача состоит из двух пунктов \pt{1} и \pt{2}, при проверке засчитывается только
один из них. Если решены оба пункта, то выбирается максимальный балл.

\section{Экзамен 2001 г.}

\subsection{Вариант 1}

\begin{problem}
\skill{4}
Найти и нарисовать фазовые кривые  системы
\eqn{\label{2001.1.1}
\case{\dot x = x^2-y^2,\\ \dot y = 2xy.}}
\end{problem}

\begin{problem}
\skill{1+1+1}
Найти все особые точки системы~(\ref{2001.1.1}). Линеаризовать векторное поле системы~(\ref{2001.1.1}).
Исследовать на устойчивость особые точки линеаризованной системы.
\end{problem}

\begin{problem}
\skill{4}
Исследовать на устойчивость особые точки системы (\ref{2001.1.1}).
\end{problem}

\begin{problem}
\skill{5}
Имеет ли система (\ref{2001.1.1}) непрерывный непостоянный первый интеграл в окрестности
точки~$(0,0)$?
\end{problem}

\begin{problem}
\skill{4} Выпрямить векторное поле системы (\ref{2001.1.1}) в окрестности точки $(1,1)$.
\end{problem}

\begin{problem}
\skill{2+2}
Имеет ли в окрестности точки $(1,1)$ решение задача Коши
\equ{(x^2-y^2)\pf u x + 2xy \pf u y = 0}
с начальным условием: а) $u\rvert_{x=1}=y$, б) $u\rvert_{y=1} = x$?
\end{problem}

\begin{problem}
\skill{5+3} Найти преобразование фазового потока системы (\ref{2001.1.1}) за время $t$ там,
где оно определено. Найти начальные условия для всех решений, определённых на всей оси времени.
\end{problem}

\begin{problem}
\skill{3}
Доказать формулу Лиувилля~-- Остроградского.
\end{problem}

\begin{problem}
\skill{7}
Доказать теорему Флоке.
\end{problem}

\begin{problem}
\skill{8} Вычислить координаты точки по её судьбе для отображения подковы Смейла.
\end{problem}

\subsection{Вариант 2}

\begin{problem}
\skill{4}
Найти и нарисовать фазовые кривые системы
\eqn{\label{2001.2.1}\case{\dot x = y,\\
\dot y = x-x^2.}}
\end{problem}

\begin{problem}
\skill{1+1+3}
Найти все особые точки системы (\ref{2001.2.1}). Линеаризовать векторное поле системы (\ref{2001.2.1}).
Исследовать на устойчивость особые точки линеаризованной системы.
\end{problem}

\begin{problem}
\skill{5}
Исследовать на устойчивость особые точки системы (\ref{2001.2.1}).
\end{problem}

\begin{problem}
\skill{2}
Имеет ли система (\ref{2001.2.1}) непрерывный непостоянный первый интеграл в окрестности
точки~$(0,0)$?
\end{problem}

\begin{problem}
\skill{2+2}
Имеет ли в окрестности точки $(1,1)$ решение задача Коши
\equ{y\pf u x + (x-x^2)\pf u y = 0}
с начальным условием: а) $u\rvert_{x=1}=y$, б) $u\rvert_{y=1} = x$?
\end{problem}

\begin{problem}
\skill{2}
Решить уравнение $x^{(6)} + 64 x = 0$.
\end{problem}

\begin{problem}
\skill{5}
Найти все значения $\om$, при которых уравнение
\equ{x^{(6)} + 64x  = e^{\sqrt3 t} \sin \om t}
имеет хотя бы одно решение, ограниченное на всей оси.
\end{problem}

\begin{problem}
\skill{3}
Доказать теорему о фазовом потоке линейного векторного поля; свойства нормы линейного оператора
можно использовать без доказательства.
\end{problem}

\begin{problem}
\skill{9}
Доказать теорему о равномерном распределении орбиты иррационального поворота окружности.
\end{problem}

\section{Экзамен 29 июня 2001 г.}

\subsection{Вариант 1}

\begin{problem}
\skill{4}
Найти и нарисовать фазовые кривые системы
\eqn{\label{2001R.1}
\case{\dot x = \sin x,\\ \dot y = \sin y.}}
\end{problem}

\begin{problem}
\skill{5}
Найти преобразование фазового потока системы~(\ref{2001R.1}) в квадрате $[0,\pi]\times[0,\pi]$.
\end{problem}

\begin{problem}
\skill{3} Найти все особые точки системы~(\ref{2001R.1}), исследовать их на устойчивость и указать их тип.
\end{problem}

\begin{problem}
\skill{5}
Выпрямить векторное поле системы~(\ref{2001R.1}) в окрестности точки $\hr{\frac\pi2,\frac\pi2}$.
\end{problem}

\begin{problem}
\skill{2}
Решить уравнение $x^{(6)} - 64 x = 0$.
\end{problem}

\begin{problem}
\skill{4}
Найти решение уравнения
\equ{x^{(4)} + 4x = e^{-t}\cos \om t.}
с неопределёнными коэффициентами. Ответ исследовать по $\om$.
Неопределённых коэффициентов не находить.
\end{problem}

\begin{problem}
\skill{3} Сформулировать и доказать принцип сжимающих отображений.
\end{problem}

\subsection{Вариант 2}

\begin{problem}
\skill{4}
Найти и нарисовать фазовые кривые системы
\eqn{\label{2001R.2}
\case{\dot x = 1,\\ \dot y = -y + \sin x.}}
\end{problem}

\begin{problem}
\skill{5}
Найти преобразование фазового потока системы~(\ref{2001R.2}).
\end{problem}

\begin{problem}
\skill{4}
При каком значении параметра $a$ уравнение
\equ{\pf yx = -ay + \sin x}
имеет единственное периодическое решение?
\end{problem}

\begin{problem}
\skill{5}
Найти все точки на прямой $y=1$, в окрестности которых задача Коши
\equ{\pf ux + (-y + \sin x)\pf uy = 0}
имеет единственное решение для любых начальных условий.
\end{problem}

\begin{problem}
\skill{3}
Найти и исследовать на устойчивость особые точки системы
\equ{\case{\dot x = \sin y,\\ \dot y = -y + \sin x.}}
\end{problem}

\begin{problem}
\skill{2}
Найти $e^{At}$, где $A = \rbmat{1&2\\0&1}$.
\end{problem}

\begin{problem}
\skill{3} Сформулировать теорему Ляпунова об устойчивости для уравнений и наметить план её доказательства.
\end{problem}

\section{Экзамен 2002 г.}

\subsection{Вариант 1}

\begin{problem}
\skill{2+3} Решить уравнение
\equ{\frac{dy}{dx} = \frac{2xy}{y^2-x^2}}
и нарисовать его фазовые кривые.
\end{problem}

\begin{problem}
\skill{1+2+3}
Дана система
\eqn{\label{2002.1.2}\case{\dot x = y^2-x^2,\\ \dot y = 2xy.}}
Линеаризовать векторное поле системы (\ref{2002.1.2}) в особой точке $(0,0)$.
Исследовать устойчивость всех особых точек линеаризованной системы (\ref{2002.1.2}).
Исследовать на устойчивость особую точку $(0,0)$ системы (\ref{2002.1.2}).
\end{problem}

\begin{problem}
\skill{2} Для системы (\ref{2002.1.2}) найти все полиномиальные первые интегралы. Разрешается угадать
правильный ответ; то, что найдены ВСЕ полиномиальные первые интегралы, можно не доказывать.
\end{problem}

\begin{problem}
\skill{2+3} Найти и нарисовать фазовые кривые системы
\equ{
\case{\dot x = 3x-5z,\\
\dot y = 5z-3x,\\
\dot z = 5x-3z.}}
\end{problem}

\begin{problem}
\skill{4} Найти фазовый поток системы
\equ{
\case{
\dot x = 3x-5y - 16,\\
\dot y = 5x-3y.}}
\end{problem}

\begin{problem}
\skill{4}
Доказать теорему о гладкой зависимости решений от начальных условий.
\end{problem}

\begin{problem}
\skill{5}
Доказать теорему Флоке.
\end{problem}

\begin{problem}
\skill{9} Найти замыкание траектории $\bc{x(t)\vl t \in \R}$ системы
\equ{\ddot x = -\rbmat{3&1\\1&2}x \quad
\text{с начальным условием}
\quad x(0) = \rbmat{5-\sqrt{5}\\0}, \quad \dot x(0) = \rbmat{0\\0}.}
\end{problem}

\subsection{Вариант 2}

\begin{problem}
\skill{2} Нарисовать фазовые кривые системы
\eqn{\label{2002.2.1}
\case{\dot x = \sin x,\\
\dot y = 1-y^2.}}
\end{problem}

\begin{problem}
\skill{3+2}
Исследовать на устойчивость особые точки системы (\ref{2002.2.1}) и определить их тип.
\end{problem}

\begin{problem}
\skill{6} В окрестности каждой из особых точек системы (\ref{2002.2.1}) найти хотя бы
один непостоянный первый интеграл или доказать, что такового не существует.
\end{problem}

\begin{problem}
\skill{3}
Нарисовать фазовый портрет уравнения Ньютона $\ddot x = x - x^3$.
\end{problem}

\begin{problem}
\skill{6}
Найти все точки на начальной прямой $y = 1$, в окрестности которых задача Коши
\equ{
y \pf ux + (x-x^3)\pf uy = 0, \quad u\rvert_{y=1} = x^2}
имеет хотя бы одно решение.
\end{problem}

\begin{problem}
\skill{4}
Найти с неопределёнными коэффициентами частное решение уравнения
\equ{x^{(4)} + 4x = e^t\sin \om t.}
Ответ исследовать по $\om$. Неопределённых коэффициентов не находить.
\end{problem}

\begin{problem}
\skill{5} Доказать теорему о частном решении линейного уравнения с постоянными коэффициентами
и специальной правой частью в нерезонансном случае.
\end{problem}

\begin{problem}
\skill{9} Определить отображение подковы. Доказать теорему о бесконечности числа периодических орбит.
Теорему о реализации любой последовательности как судьбы точки сформулировать, но не доказывать.
\end{problem}

\subsection{Вариант 3}

\begin{problem}
\skill{3}
Найти периодическое решение уравнения
\eqn{\label{2002.3.1}
\ddot x - \dot x + 25x = \cos \om t}
и доказать, что оно единственно.
\end{problem}

\begin{problem}
\skill{3} Нарисовать график амплитуды периодического решения уравнения (\ref{2002.3.1}) как функцию от~$\om$.
\end{problem}

\begin{problem}
\skill{4}
Дано уравнение
\equ{\dot x = (x-1)^{\al}.}
При каких значениях $\al > 0$ решение, проходящее через точку $(t,x) = (0,1)$, единственно?
\end{problem}

\begin{problem}
\skill{4}
Найти все особые точки системы
\equ{\case{\dot x = x^2 + y^2 - 25;\\
\dot y = xy - 12,}}
исследовать их на устойчивость и указать их тип.
\end{problem}

\begin{problem}
\skill{3} Доказать теорему о фазовом потоке линейного векторного поля.
\end{problem}

\begin{problem}
\skill{4}
Указать все точки на прямой $y=3$, в окрестности которых задача Коши
\equ{(x^2+y^2-25)\pf ux + (xy - 12)\pf uy = 0,\quad u\rvert_{y=3} = f(x)}
имеет решение при любом начальном условии $f(x)$.
\end{problem}

\subsection{Вариант 4}

\begin{problem}
\skill{4}
Найти и нарисовать фазовые кривые уравнения Ньютона
\eqn{\label{2002.4.1}
\case{
\dot x = y,\\
\dot y = \sin x +\frac12.}}
\end{problem}

\begin{problem}
\skill{3} Найти особые точки системы (\ref{2002.4.1}) и исследовать их на устойчивость.
\end{problem}

\begin{problem}
\skill{3} При каких значениях параметра $\om$ система
\eqn{\label{2002.4.2}\case{\dot x = -y + \sin \om t,\\ \dot y = x.}}
имеет хотя бы одно периодическое решение?
\end{problem}

\begin{problem}
\skill{4} Найти общее решение системы (\ref{2002.4.2}) при $\om = 2$.
\end{problem}

\begin{problem}
\skill{3} Доказать теорему о выпрямлении векторного поля.
\end{problem}

\begin{problem}
\skill{4} Найти мультипликатор предельного цикла уравнения, записанного в полярных координатах:
\equ{\case{\dot \ph = 1,\\
\dot r = 1 - r.}}
Нарисовать этот предельный цикл и фазовые кривые в его окрестности.
\end{problem}

\section{Экзамен 30 августа 2002 г.}

\subsection{Вариант 1}

\begin{problem}
Найти преобразование фазового потока системы
$$
\begin{cases}
\dot x = -y,\\
\dot y = x.
\end{cases}
$$
\end{problem}

\begin{problem}
Найти все значения $a$, для которых начало координат является устойчивой по Ляпунову особой точкой
векторного поля
$$
\begin{cases}
\dot x = -a x,\\
\dot y = (a-1)\sin y.
\end{cases}
$$
\end{problem}

\begin{problem}
В окрестности какой из точек а) $(0,1)$  б) $(1,0)$ задача Коши для уравнения
$$
(1+x^2)\pf ux -2xy\pf uy = 0
$$
при начальных условиях, задаваемых на кривой $x^2 + y^2=1$, имеет единственное решение для любого гладкого
начального условия?
\end{problem}

\begin{problem}
Найти решение предыдущей задачи при $u\rvert_{x^2 + y^2 =1} = y\rvert_{x^2 + y^2=1}$ в окрестностях а) и б).
\end{problem}

\begin{problem}
Выпрямить векторное поле
\equ{\case{\dot x = x^2, \\ \dot y = e^y}}
на $\R^2$ в окрестности точки $(1,1)$.
\end{problem}

\begin{problem}
Найти производную по параметру $\mu$ при $\mu = 0$ решения системы
$$
\case{
\dot x = \sin x - y + \mu \sin t,\\
\dot y = x - \ln (1-y)}
\text{с начальным условием } \case{x(0) = \cos \mu-1,\\ y(0) = \sin \mu.}$$
\end{problem}

\begin{problem}
Найти и нарисовать фазовые кривые системы
\eqn{\label{2002.08.1.7}
\case{\dot x = y,\\\dot y = x - x^3.}}
\end{problem}

\begin{problem}
Найти все особые точки системы~(\ref{2002.08.1.7}) и указать их тип.
\end{problem}

\begin{problem}
Сформулировать и доказать теорему о равномерном распределении иррационального поворота окружности.
\end{problem}

\subsection{Вариант 2}

\begin{problem}
Найти преобразование фазового потока системы
$$
\begin{cases}
\dot x = 2y,\\
\dot y = -\frac12 x.
\end{cases}
$$
\end{problem}

\begin{problem}
Найти все значения $a$, для которых начало координат является неустойчивой
по Ляпунову особой точкой векторного поля
$$
\begin{cases}
\dot x = a \tg x,\\
\dot y = -(1+a)y.
\end{cases}
$$
\end{problem}

\begin{problem}
В окрестности какой из точек а) $(0,1)$  б) $(1,0)$ задача Коши
для уравнения
$$
(x^2-y^2)\cdot u_x + 2xy\cdot u_y = 0
$$
при начальных условиях, задаваемых на кривой $x^2 + y^2=1$ имеет единственное
решение для любого гладкого начального условия?
\end{problem}

\begin{problem}
Найти решение предыдущей задачи при $u\rvert_{x^2 + y^2 =1} = y\rvert_{x^2 + y^2=1}$
в окрестностях а) и б).
\end{problem}

\begin{problem}
Выпрямить векторное поле
\equ{\case{\dot x = \sin x, \\ \dot y = y^3}}
на $\R^2$ в окрестности точки $\hr{\frac\pi2,1}$.
\end{problem}

\begin{problem}
Найти производную по параметру $\mu$ при $\mu = 0$ решения системы
$$
\begin{cases}
\dot x = x^2 + y -\mu \sin t,\\
\dot y = \sin x + y^3
\end{cases}
\text{с начальным условием }
\case{x(0) = 1 - \cos \mu, \\ y(0) = \sin \mu.}
$$
\end{problem}

\begin{problem}
Найти и нарисовать фазовые кривые системы
\eqn{\label{2002.08.2.7}
\case{\dot x = y,\\\dot y = x^2 -1.}}
\end{problem}

\begin{problem}
Найти все первые интегралы системы из задачи~(\ref{2002.08.2.7}), указать их тип.
\end{problem}

\begin{problem}
Сформулировать и доказать формулу Лиувилля~-- Остроградского.
\end{problem}


\subsection{Вариант 3}

\begin{problem}
Найти фазовую траекторию векторного поля на плоскости, проходящую при $t=0$ через точку~$(0,1)$:
$$
\begin{cases}
\dot x = y,\\
\dot y = -4x + 4y.\\
\end{cases}
$$
\end{problem}

\begin{problem}
Найти все особые точки векторного поля
$$
\begin{cases}
\dot x = y,\\
\dot y = x^2 - 1.
\end{cases}
$$
\end{problem}

\begin{problem}
В окрестности какой из точек а) $(0,1)$  б) $(1,0)$ задача Коши для уравнения
$$
x^2\frac{\partial u}{\partial x} +(y+1)x\frac{\partial u}{\partial y} = 0
$$
при начальных условиях, задаваемых на кривой $x^2 + y^2=1$ имеет единственное решение для любого гладкого
начального условия?
\end{problem}

\begin{problem}
Найти решение предыдущей задачи при $u\rvert_{x^2 + y^2 =1} = y\rvert_{x^2 + y^2=1}$ в окрестностях а) и
б).
\end{problem}

\begin{problem}
Выпрямить векторное поле
$$\case{\dot x = y, \\ \dot y = y^2}$$
в окрестности точки $(1,1)$.
\end{problem}

\begin{problem}
Найти производную по параметру $\mu$ при $\mu = 0$ решения системы
$$
\begin{cases}
\dot x = \cos x - 1 + y,\\
\dot y = x + e^y - 1 + \mu e^t
\end{cases}
\text{с начальным условием }
\case{x(0) = \cos \mu -1,\\y(0) = \sin \mu.}
$$
\end{problem}

\begin{problem}
Найти и нарисовать фазовые кривые системы
\eqn{\label{2002.08.3.7}\case{\dot x = -x(y + 1), \\ \dot y = x^2 -y.}}
\end{problem}

\begin{problem}
Найти все первые интегралы системы~(\ref{2002.08.3.7}), определённые на всей плоскости.
\end{problem}

\begin{problem}
Экспонента коммутирующих операторов.
\end{problem}

\section{Экзамен 2003 г.}

\subsection{Вариант 1}

\subsubsection{Часть первая}

\begin{problem}
\skill{2}
Нарисовать фазовые кривые системы
\eqn{\label{sys2003.1} \case{\dot x = y,\\\dot y = 1-x^2.}}
\end{problem}

\begin{problem}
\skill{3}
Найти все особые точки системы
\eqn{\label{sys2003.2}
\case{\dot x = y,\\ \dot y = 1-x^2 -y,}}
исследовать их на устойчивость и определить их тип.
\end{problem}

\begin{problem}
\skill{5}
Нарисовать фазовый портрет системы (\ref{sys2003.2}).
\end{problem}

\begin{problem}
\skill{4}
Найти решение системы (\ref{sys2003.1}) с начальным условием $(2,0)$.
\end{problem}

\begin{problem}

\pt{1} \skill{3} Найти все непрерывные первые интегралы системы (\ref{sys2003.2}) в окрестности
точки $(0,1)$.

\pt{2} \skill{6} Найти производную $\pf xa$ при $a=0$ первой компоненты решения системы
(\ref{sys2003.1}) c начальным условием \equ{\case{x(0) = 2,\\ y(0) = a.}}
\end{problem}

\subsubsection{Часть вторая}

\begin{problem}
\skill{3} Найти частное решение с неопределёнными коэффициентами уравнения
\equ{x^{(6)} + 64 x = \sin 2t.}
и общее решение соответствующего однородного уравнения.
\end{problem}

\begin{problem}
\skill{4}
При каких значениях $\om$ уравнение
\equ{x^{(6)} + 64 x = \sin \om t.}
имеет хотя бы одно ненулевое решение, ограниченное на всей оси?
\end{problem}

\begin{problem}
\skill{6} Пусть $f(x) = \sin^2 x+ \cos^{2003} x$. Найти предел
\equ{\liml{n\ra\infty} \frac{1}{n}\suml{k=0}{n-1} f(k).}
\end{problem}

\begin{problem}
\skill{3}
Доказать теорему об искажении фазового объёма. Формулу Лиувилля~-- Остроградского можно считать известной.
\end{problem}

\begin{problem}
\skill{6}
Найти все периодические точки периода $2$ для отображения подковы из лекции 14.
\end{problem}

\subsection{Вариант 2}

\subsubsection{Часть первая}

\begin{problem}
\skill{2}
Найти все $2\pi$-периодические решения уравнения
\eqn{\label{eqn2003.1}
\dot x = 2x + \sin t.}
\end{problem}

\begin{problem}
\skill{3} Найти преобразование монодромии уравнения (\ref{eqn2003.1}) за период.
\end{problem}

\begin{problem}
\skill{2+3} Для любого $m\in \N$ найти $m$-е пикаровское приближение к решению
начальной задачи
\equ{
\dot x= A x,\quad x(0) = x_0,}
где $A\cln \R^n\ra\R^n$~--- линейный оператор, а $x\in\R^n$.
Вывести из полученной формулы теорему о существовании~$e^{At}$ при малых $t$.
\end{problem}

\begin{problem}
\skill{6}
Пусть $\ph(t,x)$~--- решение системы
\eqn{
\case{
\dot{x}_1 = -2\sin x_1 + 3\sin x_3,\\
\dot{x}_2 = -2\sin x_2 + \sin x_3,\\
\dot{x}_3 = -2\sin x_3 + \sin^2 x_2}}
с начальным условием $\ph(0,x) =x$.
Найти
\eqn{X(t) = \pf{\ph(t,x)}{x}\biggl\rvert_{x=0}}
\end{problem}

\begin{problem}
\skill{5+4} Судьба точки $p$ под действием отображения подковы (лекция 14)
имеет вид
\eqn{\om = \ldots\,\om_{-n}\,\ldots\,\om_0\,\ldots\,\om_n\,\ldots,}
причём $\om_n = 1$ при $|n| > 5$. Найти предельные точки орбиты точки $p$ под действием отображения подковы.
Сколько точек удовлетворяют условию задачи?
\end{problem}

\subsubsection{Часть вторая}

\begin{problem}
\skill{3}
Нарисовать фазовый портрет системы
\eqn{\label{eqn2003.3}\case{\dot x = x + x^2,\\\dot y = -y.}}
\end{problem}

\begin{problem}
\skill{6}
Найти хотя бы один непостоянный первый интеграл системы (\ref{eqn2003.3}), определённый на всей плоскости.
\end{problem}

\begin{problem}
\skill{5}
Разрешима ли задача Коши
\equ{(x + x^2)\pf ux - y\pf uy =0,\quad u\rvert_{x+y=1}=x}
в окрестности точки $\hr{\sqrt2 -1, 2-\sqrt2}$.
\end{problem}

\begin{problem}
\skill{3} Доказать принцип сжимающих отображений.
\end{problem}

\begin{problem}
\skill{4}
Останется ли верным принцип сжимающих отображений, если в его формулировке отказаться от условия полноты?
\end{problem}

\section{Основной экзамен 14 июня 2004 г.}

\subsection{Вариант 1}

\subsubsection{Часть первая}

\begin{problem}
\skill{4} Найти общее решение системы
\eqn{\label{ur2} \case{\dot x  = 1, \\ \dot y  = -y + \sin x.}}
\end{problem}

\begin{problem}
\skill{3+1} Найти фазовый поток системы (\ref{ur2}). Сохраняет ли он объём?
\end{problem}

\begin{problem}
\skill{2+1} Выпрямить векторное поле системы (\ref{ur2}) в окрестности точки $(0,0)$. Ответ проверить.
\end{problem}

\begin{problem}
\skill{5} Сколько $2\pi$-периодических решений имеет уравнение
\equ{\dot x = -ax+ \frac{\sin t}{2+\sin t}?}
Ответ исследовать в зависимости от $a$.
\end{problem}

\begin{problem}
\skill{5} Вывести теорему Ляпунова об устойчивости для векторных полей из теоремы Ляпунова об устойчивости
для отображений.
\end{problem}

\subsubsection{Часть вторая}

\begin{problem}\label{p6}
Пусть
\eqn{\label{ur1}A(a) = \rbmat{a+1 & 2 & 1 \\ 2a & 2 & 0 \\ 1 & 1 & 1 }, \quad \dot x = A(a)x.}

\pt{1} \skill{6+1} При каких значениях $a$ уравнение (\ref{ur1})  имеет непостоянные периодические
решения? С каким периодом?

\pt{2} \skill{4} Найти решение уравнения (\ref{ur1}) при $a=4$ с начальным условием $\vec{x}(0)=(4,0,4)$.
\end{problem}

\begin{problem}
\skill{5} Исследовать на устойчивость особую точку $0$ уравнения (\ref{ur1}) при $a=-5$.
\end{problem}

\begin{problem}
\skill{6+2} Рассмотрим арифметическую прогрессию, разность которой  равна $\sqrt[3]a$. Верно ли, что при
$a=\frac19$ в множество
\equ{E= \cups{n \in \Z} \hs{n, n+\frac{1}{2004}}}
попадает бесконечно много членов этой
прогрессии? Верно ли аналогичное утверждение при $a=\frac18$?
\end{problem}

\subsection{Вариант 2}

\subsubsection{Часть первая}
\begin{problem}
\skill{2}
Найти общее решение уравнения
\eqn{\label{2004.2.1}\dot x = 1 + a^2x^2,\quad a \in \R, \quad x \in \R.}
\end{problem}

\begin{problem}
\skill{5} При каких значениях вещественных параметров $a$ и $b$ фазовый поток
уравнения \equ{\dot x = 1 + a^2x^2 + bx^4} определён на всей прямой для всех значений времени?
Найти этот поток для этих значений параметра.
\end{problem}

\begin{problem}
\skill{2+1}
Выпрямить векторное поле уравнения (\ref{2004.2.1}) при $a = 2$ на всей прямой. Ответ проверить.
\end{problem}

\begin{problem}
\skill{5}
Может ли уравнение Ньютона
\equ{\case{\dot x = y,\\ \dot y = f(x)}}
иметь предельные циклы, если $f \in \Cb^1$?
\end{problem}

\begin{problem}
\skill{5}
Доказать основную теорему теории линейных автономных уравнений. Все нужные
для доказательства свойства нормы линейных операторов можно считать известными.
\end{problem}

\subsubsection{Часть вторая}

\begin{problem}

\pt{1} \skill{3} Найти экспоненту линейного оператора
\eqn{\label{op1}A=\rbmat{0&1&\pi\\0&0&2\\0&0&0}.}

\pt{2} \skill{5+1} Найти экспоненту оператора Лапласа $\De$ в пространстве $\Pc_n$ тригонометрических многочленов
степени не выше $n$, то есть в пространстве функций вида
\eqn{p(x) = \sumkzn (a_k \sin kx + b_k \cos kx), \quad a_k,b_k \in \R.}
Какова размерность этого пространства?
\end{problem}

\begin{problem}

\pt{1} \skill{4} Исследовать на устойчивость особую точку $0$ уравнения $\dot x = Ax$, где $A$~---
оператор (\ref{op1}).

\pt{2} \skill{6} Будет ли устойчивой особая точка $0$ уравнения
\equ{\dot p = \De p,\quad p \in \Pc_n?}
\end{problem}

\begin{problem}
\skill{1+7} Сколько периодических точек имеет отображение подковы из лекции 14? Может ли хотя бы одна из этих точек
иметь хотя бы одну иррациональную координату?
\end{problem}

\section{Экзамен 28 июня 2004 г.}

\subsection{Вариант 1}

\begin{problem}
\skill{2+2+2} Найти и нарисовать фазовые кривые системы
\eqn{\case{\dot x = x - x^2,\\ \dot y = y^2.}}
Найти все начальные условия, при которых решение системы определено на всей оси времени.
\end{problem}

\begin{problem}
\skill{2+2+1}
Дана система
\eqn{\case{ \dot x = y, \\ \dot y = x(x-1)(x-2).}}
Нарисовать фазовые кривые системы. Есть ли у системы особые точки, устойчивые
по Ляпунову? Есть ли среди особых точек асимптотически устойчивые?
\end{problem}

\begin{problem}
\skill{4}
При каких $\om$ все решения уравнения $x^{(4)}+ 5 \ddot x + 6 x = \sin \om t$ ограничены
на всей оси времени?
\end{problem}

\begin{problem}
\skill{5}
Сформулировать и доказать теорему Ляпунова об устойчивости для отображений в случае, когда линейная часть
отображения в точке имеет попарно различные собственные значения. Доказывать только
достаточность условий устойчивости.
\end{problem}

\section{Экзамен 4 июня 2009 г.}

\subsection{Вариант 1}

\subsubsection{Часть первая}

\begin{problem}
\skill{3}  Нарисовать фазовый портрет системы \eqn{\label{2009.1.1} \case{\dot x  = y, \\ \dot y  = 5x + x^2}}
\end{problem}

\begin{problem}
\skill{3} Найти общее решение уравнения \eqn{\label{2009.1.2} u_x y + u_y (5x + x^2)=0.}
\end{problem}

\begin{problem}
\skill{4} Существует ли решение задачи Коши для уравнения (\ref{2009.1.2}) с начальными условиями $u\rvert_{y=1} = \sin x$, определённое во всей полуплоскости $y \ge 1$?
\end{problem}

\begin{problem}
\skill{4} В системе (\ref{2009.1.1}) найти периоды малых колебаний и углы наклона сепаратрис.
\end{problem}

\begin{problem}
\skill{6} Будет ли полная энергия системы (\ref{2009.1.1}) убывать вдоль фазовых кривых возмущенной системы
\eqn{\label{2009.1.3} \case{\dot x  = y, \\ \dot y  = 5x + x^2 - 0.1y\quad?}}
Нарисовать фазовый портрет системы (\ref{2009.1.3}).
\end{problem}

\subsubsection{Часть вторая}

\begin{problem}
\skill{5} Теорема об искажении фазового объема.
\end{problem}

\begin{problem}
\skill{5} Найти преобразование фазового потока для системы \eqn{ \case{\dot x  = x + y, \\ \dot y  = y + 2z, \\ \dot z = z}}
\end{problem}

\begin{problem}
\skill{4} Найти производную по параметру $\varepsilon$ при $\varepsilon = 0$ решения системы \eqn{ \case{\dot x  = x^2 + y^2 + \varepsilon t \sin t, \\ \dot y  = 2xy + \varepsilon^2 \frac{\sin t}{t}}} с начальным условием $x(0) = y(0) = 0$.
\end{problem}

\begin{problem}
\skill{6} Найти судьбу точки $(\frac{1}{4}, \frac{3}{4})$ под действием отображения подковы из лекции 6~мая~2009~г.
\end{problem}

\subsection{Вариант 2}

\subsubsection{Часть первая}

\begin{problem}
\skill{3} Найти и~нарисовать фазовые кривые системы \eqn{\label{2009.2.1} \case{\dot x  = 2x + y, \\ \dot y  = -x + 2y}}
\end{problem}

\begin{problem}
\skill{4} Найти особые точки системы \eqn{\label{2009.2.2} \case{\dot x  = \sin(2x + y), \\ \dot y  = \sin(-x + 2y)}}
 и~исследовать их на устойчивость.
\end{problem}

\begin{problem}
\skill{3+2} Найти частное решение уравнения \eqn{\ddot x - 4\dot x + 5x = e^{2t}\sin t.}
а) Неопределённые коэффициенты не находить.\\
б) Найти неопределённые коэффициенты.
\end{problem}

\begin{problem}
\skill{4} Найти производную в~точке~$0$ преобразования фазового потока за время $\pi$ системы (\ref{2009.2.2}).
\end{problem}

\begin{problem}
\skill{4} Выпрямить векторное поле системы (\ref{2009.2.1}) в~окрестности точки $(1,1)$.
\end{problem}


\subsubsection{Часть вторая}

\begin{problem}
\skill{5} Сформулировать теорему о~решении линейной системы с~постоянными коэффициентами и~правой частью в~виде
квазимногочлена. Дать доказательство в~нерезонансном случае.
\end{problem}

\begin{problem}
\skill{4} Нарисовать фазовые кривые системы \eqn{\label{2009.2.3} \case{\dot x  = x^2(1-x)^2, \\ \dot y  = y}}
\end{problem}

\begin{problem}
\skill{5} Верно ли, что все первые интегралы системы (\ref{2009.2.3}), непрерывные на~всей плоскости, постоянны?
\end{problem}

\begin{problem}
\skill{6} Нарисовать замыкание на~плоскости траектории системы \eqn{ \case{\ddot x  = -y, \\ \ddot y  = 2x - 3y}}
с~начальным условием $x(0)=1$, $y(0)=0$, $\dot x(0)=\dot y(0)=0$.
\end{problem}

\medskip\dmvntrail


\end{document}
