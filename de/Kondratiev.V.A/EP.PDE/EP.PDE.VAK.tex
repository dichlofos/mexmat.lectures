\documentclass{article}
\usepackage[utf,simple]{dmvn}

\title{Варианты и программа экзамена по УрЧП}
\author{Лектор В.\,А.\,Кондратьев}
\date{V--VI семестры, 2005--2006 г.}

\begin{document}
\maketitle

\section{Программа экзамена}

\begin{nums}{-3}
\item Постановка задачи Коши для линейных дифференциальных уравнений
с частными производными. Формулировка теоремы Коши. Доказательство
единственности аналитического решения задачи Коши.
\item Определение характеристик линейного уравнения с частными
производными. Обобщённая задача Коши (сведение её к обычной).
\item Определение корректности задачи Коши. Пример Адамара.
\item Приведение линейных дифференциальных уравнений второго порядка
с постоянными коэффициентами к каноническому виду. Классификация
линейных уравнений второго порядка.
\item Единственность решения задачи Коши для волнового уравнения.
\item Формула Кирхгофа. Доказательство существования решения задачи
Коши для волнового уравнения.
\item Доказательство единственности решения смешанной краевой задачи
для волнового уравнения.
\item Задача Штурма~-- Лиувилля. Свойства собственных значений и
собственных функций. Функция Грина.
\item Обоснование метода Фурье для решения смешанной задачи для
волнового уравнения.
\item Принцип максимума для гармонических функций.
\item Лемма о знаке производной по внутреннему направлению для
гармонических функций.
\item Строгий принцип максимума для гармонических функций.
\item Постановка основных краевых задач для гармонических функций.
Необходимое условие разрешимости задачи Неймана.
\item Формула Грина. Функция Грина задачи Дирихле для оператора
Лапласа. Симметрия функции Грина.
\item Формула для решения задачи Дирихле для оператора Лапласа в
круге. Обоснование.
\item Теоремы о среднем.
\item Теорема об устранимой особенности для гармонических функций.
\item Теорема Лиувилля для гармонических функций.
\item Бесконечная дифференцируемость гармонических функций.
\item Оценки производных гармонических функций (неравенство Бернштейна).
\item Обобщённая производная по Соболеву. Пространство $H^1$, полнота.
\item Неравенство Фридрихса, пространство $\os{\circ}{H}{}^1$.
\item Усреднение функций.
\item Обобщённое решение задачи Дирихле для уравнения Пуассона.
Существование и единственность.
\item Гладкость обобщённого решения задачи Дирихле для уравнения
Лапласа.
\item Внешние краевые задачи для уравнения Лапласа.
\item Схема вариационного метода решения задачи Дирихле для
уравнения Лапласа.
\item Принцип максимума для уравнения теплопроводности (в ограниченной и неограниченной области).
\item Существование и единственность решения первой краевой задачи
для уравнения теплопроводности.
\item Существование и единственность решения задачи Коши для
уравнения теплопроводности.
\item Непрерывность интеграла, равномерно сходящегося в точке.
\item Непрерывность потенциала простого слоя.
\item Формулы скачков потенциала двойного слоя.
\item Формулы скачков нормальной производной потенциала простого
слоя.
\item Интегральные уравнения для основных краевых задач для
уравнения Лапласа.
\item Доказательство существования решения задачи Коши методом
мажорант.
\item Доказательство теоремы о стабилизации решений задачи Коши.
\item Аналитичность гармонических функций.
\end{nums}

\section{Досрочный экзамен}

\subsection{Вариант №1}

\begin{nums}{-2}
\item \begin{nums}{-2}
\item Написать формулу решения задачи Коши для волнового уравнения
для трёх пространственных переменных.
\item Доказать единственность решения задачи Коши для волнового
уравнения.
\item $u(x, t)$ -- решение задачи
\begin{gather*}
4u_{tt} = u_{xx}, \quad x > 0, \quad t > 0,\\
u_x\evn{x = 0} = 0, \quad u\evn{t = 0} = 0, \quad u_t\evn{t = 0} = \varphi(x),\\
\varphi \geqslant 0, \quad \varphi = 0 \text{ вне } (2, 4), \quad
\varphi \in C^\infty (\R ).
\end{gather*}
Найти множество точек $(x, t)$, в которых $u_t(x, t) = 0$ для любой
функции $\varphi$.
\end{nums}
\item \begin{nums}{-2}
\item Сформулировать теоремы о среднем для гармонических функций.
\item Доказать вторую теорему о среднем для гармонических функций.
\item Найти решение внешней задачи Дирихле
\begin{gather*}
\Delta u = 0 \text{ в } |x| > 1, \quad x = (x_1, x_2, x_3) \in
\R ^3,\\
u\evn{|x| = 1} = x_1.
\end{gather*}
\end{nums}
\item \begin{nums}{-2}
\item Написать формулу для решения задачи Коши с непрерывными и
ограниченными начальными данными при $t = 0$ для уравнения
теплопроводности
$$
u_t = u_{xx} + 4u_{yy}.
$$
\item Доказать непрерывность решения задачи Коши для уравнения $u_t = u_{xx},\, t >
0$.
\item $u(x, t)$ -- решение задачи
\begin{gather*}
u_t = 4u_{xx}, \quad x > 0, \, t > 0, \text{ такое, что}\\
u(0, t) = 2, \quad u(x, 0) = 2 \cos^2 3\pi x.
\end{gather*}
Найти $\lim\limits_{t \rightarrow \infty} u(\frac{3\pi}{4}, t).$
\end{nums}
\end{nums}

\subsection{Вариант №2}

\begin{nums}{-2}
\item \begin{nums}{-2}
\item Сформулировать теорему Коши~-- Ковалевской.
\item Привести уравнение
$$
u_{xy} - u_{yz} + u_{xz} = 0
$$
к каноническому виду.
\item При каких $\alpha \in \R $ задача Коши
\begin{gather*}
\alpha^2u_{tt} + u_{tx} - u_{xx} + 1 = 0\\
u\evn{t = 0} = \varphi(x), \quad u_t\evn{t = 0} = \psi(x),
\end{gather*}
имеет единственное решение?
\end{nums}
\item \begin{nums}{-2}
\item Сформулировать лемму о знаке производной по внутреннему
направлению для гармонических функций.
\item Бесконечная дифференцируемость гармонических функций.
\item Пусть $u(x)$ гармонична в шаре $|x| < 3$ в $\R ^3$ и
$$
\int\limits_{|x| < 3}|u(x)|\,dx < 2.
$$
Указать $C = const$ такую, что в точке $(0, 0, 1)$ выполнено
$$
\left|\frac{\partial^2 u}{\partial x_1 \partial x_2}\right| < C.
$$
\end{nums}
\item \begin{nums}{-2}
\item Определение пространства $H^1(\Omega)$.
\item Доказать полноту пространства $H^1(\Omega)$.
\item При каких вещественных $\alpha$ и $s > 0$ функция $\ln^\alpha(1 + r^s)$ принадлежит $H^1(B^1_n)$, где $B^1_n$ --
единичный шар в $\R ^n: \left\{r = \sqrt{x_1^2 + \ldots +
x_n^2} < 1\right\}, \, n \geqslant 1$.
\end{nums}
\end{nums}

\section{Основной экзамен}

\subsection{Вариант №3}

\begin{nums}{-2}
\item \begin{nums}{-2}
\item Определение характеристик линейного дифференциального
уравнения порядка $m$. Найти характеристики
$$
2u_{xx} - u_{xy} = 1.
$$
\item Сформулировать теорему Коши~-- Ковалевской. Доказать
единственность решения.
\item Привести пример Адамара. Будет ли корректна задача
\begin{gather*}
u_{tt} + u_{xx} + 2u_x + 4u = 0,\\
u\evn{t = 0} = \varphi(x), \quad u_t\evn{t = 0} = \psi(x)?
\end{gather*}
\end{nums}
\item \begin{nums}{-2}
\item Написать формулу решения задачи Коши для уравнения
\begin{gather*}
u_{tt} = u_{xx} + u_{yy}, \quad t > 0, \,(x, y) \in \R ^2,\\
u\evn{t = 0} = \varphi(x, y), \quad u_t\evn{t = 0} = 0.
\end{gather*}
\item Доказать единственность решения задачи Коши для уравнения
$$
a^2u_{tt} = \Delta u, \quad a = const > 0.
$$
\item Найти решение задачи
\begin{gather*}
u_{tt} = u_{xx} + u_{yy}, \quad t > 0,\, (x, y) \in \R ^2 \\
u\evn{t = 0} = \cos x + \sin y, \quad u_t\evn{t = 0} = \sin x - \cos y.
\end{gather*}
\end{nums}
\item \begin{nums}{-2}
\item Сформулировать и доказать принцип максимума для уравнения
теплопроводности в ограниченной области.
\item Доказать теорему о стабилизации решения задачи Коши для
уравнения теплопроводности.
\item Пусть $u(x, t)$ -- решение задачи Коши
$$
u_t = 4u_{xx}, \quad t > 0, \, x \in \R ,\quad u\evn{t = 0} =
\frac{x^2 + \sin x^2}{1 + x^2}.
$$
Найти $\lim\limits_{t \rightarrow \infty}u_x(x, t).$
\end{nums}
\end{nums}

\subsection{Вариант №4}

\begin{nums}{-2}
\item \begin{nums}{-2}
\item Постановка внутренней задачи Неймана для уравнения Лапласа.
Доказать теорему о единственности решения.
\item Определение функции Грина задачи Дирихле для уравнения
Лапласа в области в $\R ^2$. Построить функцию Грина для круга
радиуса $R$.
\item $\Delta u = 0$, $u \le 0$ в $x^2 + y^2 \le 1$,
$u \in C^2(x^2 + y^2 \le 1)$ и
$$
\iint\limits_{\frac{9}{25} \le x^2 + y^2 \le 1} u\,dx\,dy
= 16.
$$
Указать $C = \const$ такую, что
$$\maxl{x^2 + y^2 = \frac{16}{25}}u(x, y) \le C.$$
\end{nums}
\item \begin{nums}{-2}
\item Дайте определение уравнения эллиптического типа. При каких $\alpha = const$
уравнение
$$
u_{xy} + 2\alpha u_{xx} - 3\alpha^2u_{yy} - \alpha u_y + u_x = 0
$$
является уравнением эллиптического типа?
\item Доказать существование решения задачи Коши
\begin{gather*}
u_{tt} = \Delta u, \quad x \in \R ^3,\\
u\evn{t = 0} = 0, \quad u_t\evn{t = 0} = \psi(x).
\end{gather*}
\item Решить задачу
\begin{gather*}
4u_{tt} = u_{xx} + u_{yy} + u_{zz}, \quad t > 0\\
u\evn{t = 0} = \cos x - e^{-2z}, \quad u_t\evn{t = 0} = \sin y.
\end{gather*}
\end{nums}
\item
\begin{nums}{-2}
\item Дайте определение собственного значения задачи
Штурма~-- Лиувилля. Докажите, что собственные функции, соответствующие
различным собственным значениям, ортогональны.
\item Существование решения смешанной краевой задачи для уравнения
теплопроводности (формулировка и доказательство).
\item Пусть $u(x, t)$ -- решение задачи
$$
u_t = u_{xx} + e^{-t^2}, \quad t > 0, \, x \in \R , \quad
u\evn{t = 0} = e^{-x^2}.
$$
Найти $\lim\limits_{t \rightarrow \infty}u(x, t)$.
\end{nums}
\end{nums}

\medskip\dmvntrail
\end{document}
