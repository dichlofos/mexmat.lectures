\documentclass{article}
\usepackage[utf8]{inputenc}
\usepackage[simple]{dmvn}
\usepackage[russian]{babel}

\title{Варианты экзаменов по ОДУ}
\author{Лектор И.\,Н.\,Сергеев}
\date{III--IV семестры, 2003--2012 г.}

\begin{document}
\maketitle
{\footnotesize Данное издание подготовлено в рамках Программы по Борьбе с Обыкновенными Дифференциальными
Уравнениями. Если у~читателя есть вариант экзамена, которого здесь нет, пришлите
его нам. Только такие методы позволят собрать полную коллекцию
вариантов.

Убедительная просьба собрать ещё немного вариантов 2005, 2007
и~последующих лет, если таковые вообще в~природе имеются!  Хотя общая
тематика и~типы задач уже становится примерно понятными, чёткой
программы сопротивления до~сих пор не выработано, и~дифференциальные
уравнения остаются сложным и~непонятным предметом (главным образом,
с~нашей точки зрения, по~причине отсутствия сколь нибудь вменяемой
литературы по~курсу, написанной современным языком. Две брошюрки,
изданные лектором, конечно, можно считать литературой по~курсу, но
адекватной её назвать всё таки сложно).

Нумерация вариантов и~задач внутри вариантов может не совпадать с~исходной. Мы не можем гарантировать абсолютную
правильность решений и~условий, но стараемся сделать всё для того, чтобы количество ошибок в~данном издании было
строго убывающей функцией.

\textbf{Набор, вёрстка:} DMVN, П.\,В.\,Бибиков.

\textbf{Решения:} DMVN, Д.\,С.\,Котенко, В.\,Осокин, П.\,В.\,Бибиков.

\textbf{Коллекцию пополняли:} Д.\,С.\,Котенко, В.\,Осокин, Baron, К.\,С.\,Коршунов, П.\,В.\,Бибиков,
А.\,Богданова, С.\,Антонов\par}
\medskip\dmvntrail

\section{Экзамен 02.06.2004 (Основной экзамен)}

\subsection*{Правила проведения экзамена}

На решение задач отводилось 1 час 15 минут.
Пусть $x$ количество баллов, набранное на экзамене. Пусть $y$ количество баллов,
полученное за контрольную работу. Имеем $0\le x \le 10$ и $0 \le y \le 5$.
Пусть $z = x+ y$. Если $x \neq 0$, то
\begin{center}
\begin{tabular}{|c|c|}
\hline Баллы    & Оценка \\
\hline $z \ge 3$ & <<3>> \\
\hline $z \ge 6$ & <<4>> \\
\hline $z \ge 9$ & <<5>> \\
\hline
\end{tabular}
\end{center}
В течение первых 20 минут экзамена можно попросить поставить оценку
исходя только из баллов за контрольную работу. В этом случае $z := y$, и смотрим в табличку.
Очевидно, что при этом оценка не превысит трёх баллов.

\subsection{Вариант 1}

\begin{problem}[3 балла]
Известно, что $y_1=9\sin t$ и $y_2 = 3t^2$ решения уравнения
\eqn{\label{02.06.2004.1.eq.diff}y'' + p(t)y' + q(t)y = f(t), \quad p, q, f \in \Cb^1(\R).}
Найти хотя бы одно решение $y_3$ уравнения (\ref{02.06.2004.1.eq.diff}), удовлетворяющее
условию $y_3(\pi) = -\frac12$. Сколько существует таких решений?
\end{problem}
%\begin{solution}
%\end{solution}

\begin{problem}[2 балла]
Может ли уравнение (\ref{02.06.2004.1.eq.diff}) иметь ещё и решение $y_4 = t^2 -1$?
\end{problem}
%\begin{solution}
%\end{solution}

\begin{problem}[2 балла]
Нарисовать характеристики уравнения
\eqn{\label{02.06.2004.1.eq.pd}xu'_x - (2y-3x)u'_y = 0}
на плоскости с координатами $(x,y)$ и определить тип особой точки $(0,0)$.
\end{problem}
%\begin{solution}
%\end{solution}

\begin{problem}[3 балла]
При каких $a \in \R$ для любой функции $\ph \in  \Cb^1(\R)$ в достаточно малой
окрестности точки $(2,1)$ уравнение (\ref{02.06.2004.1.eq.pd}) имеет решение, удовлетворяющее
условию $u(x+2,ax+1) = \ph(x)$?
\end{problem}
%\begin{solution}
%\end{solution}

\setcounter{problem}{0}
\subsection{Вариант 2}

\begin{problem}[3 балла]
Известно, что $y_1=4\cos t$ и $y_2 = 2t^2$ решения уравнения
\eqn{\label{02.06.2004.2.eq.diff}y'' + p(t)y' + q(t)y = f(t), \quad p, q, f \in \Cb^1(\R).}
Найти хотя бы одно решение $y_3$ уравнения (\ref{02.06.2004.2.eq.diff}), удовлетворяющее
условию $y_3\hr{-\frac\pi2} = 1$. Сколько существует таких решений?
\end{problem}
%\begin{solution}
%\end{solution}

\begin{problem}[2 балла]
Может ли уравнение (\ref{02.06.2004.2.eq.diff}) иметь ещё и решение $y_4 = t^2 -1$?
\end{problem}
%\begin{solution}
%\end{solution}

\begin{problem}[2 балла]
Нарисовать характеристики уравнения
\eqn{\label{02.06.2004.2.eq.pd}4xu'_x + (y-3x)u'_y = 0}
на плоскости с координатами $(x,y)$ и определить тип особой точки $(0,0)$.
\end{problem}
%\begin{solution}
%\end{solution}

\begin{problem}[3 балла]
При каких $a \in \R$ для любой функции $\ph \in  \Cb^1(\R)$ в достаточно малой
окрестности точки $(1,2)$ уравнение (\ref{02.06.2004.2.eq.pd}) имеет решение, удовлетворяющее
условию $u(x+1,ax+2) = \ph(x)$?
\end{problem}
%\begin{solution}
%\end{solution}

\setcounter{problem}{0}
\subsection{Вариант 3}

\begin{problem}[3 балла]
Известно, что $y_1=t^2-3$ и $y_2 = 6\cos t$ решения уравнения
\eqn{\label{02.06.2004.3.eq.diff}y'' + p(t)y' + q(t)y = f(t), \quad p, q, f \in \Cb^1(\R).}
Найти хотя бы одно решение $y_3$ уравнения~\eqref{02.06.2004.3.eq.diff}, удовлетворяющее
условию $y_3\hr{-\sqrt3} = 1$. Сколько существует таких решений?
\end{problem}
\begin{solution}
Функция $v := y_2 - y_1$ является решением соответствующего однородного уравнения
\eqn{\label{02.06.2004.eq.hom}y'' + p(t)y' + q(t)y = 0.}
Поскольку решения однородного уравнения образуют линейное пространство, $\la v$ также является
решением~\eqref{02.06.2004.eq.hom} при всяком $\la \in \R$. Отсюда следует, что функция
$y_3 = y_1 + \la v$ будет решением исходного уравнения. Найдём значение $\la$, при котором будет выполнено
условие задачи:
\begin{eqnarray*}
y_1\br{-\sqrt3} + \la \cdot v\br{-\sqrt3} = 1,\\
3-3 + \la \cdot (6\cos\sqrt3 - 3 + 3) = 1,
\end{eqnarray*}
откуда $\la = \frac{1}{6\cos\sqrt3}$.
Итак, искомым решением будет функция
$y_3 = t^2-3 + \frac{1}{6\cos\sqrt3}\hr{6\cos t - t^2 + 3}$.

Теперь покажем, что таких решений существует бесконечно много. Действительно, мы зафиксировали
значение функции в точке $-\sqrt3$, но значение первой производной можно брать любым. Таким образом,
существует бесконечно много разных задач Коши, удовлетворяющих условию, а каждая такая задача Коши
имеет (единственное) решение по теореме существования и единственности.
\end{solution}

\begin{problem}[2 балла]
Может ли уравнение~\eqref{02.06.2004.3.eq.diff} иметь ещё и решение $y_4 = 3t^2$?
\end{problem}
\begin{solution}
Допустим, что $y_4$ является решением этого уравнения. Тогда функции $u := y_4 - y_1$ и $v := y_1 - y_2$ будут
решениями соответствующего однородного уравнения~\eqref{02.06.2004.eq.hom}. Имеем
\eqn{\case{u(t) = 2t^2+3,\\v(t) = t^2-6\cos t -3.}}
Покажем, что~$u$ и~$v$ линейно независимы. Действительно, пусть нашлась их нетривиальная линейная комбинация
$\la u(t) + \mu v(t)$, равная тождественно нулевой функции, е
$t^2(2\la + \mu) + 3(\la-\mu) -6\mu\cos t \equiv 0$. В выражении слева второе и третье слагаемые ограничены,
а первое можно сделать сколь угодно большим за счёт выбора подходящего~$t$. Отсюда следует,
что коэффициент при нём должен быть нулевым: $2\la + \mu = 0$. Выражая~$\mu$, приходим к следующему выражению:
$9\la + 12\la\cos t \equiv 0$. Ясно, что такое равенство возможно только если $\la = 0$, но тогда и $\mu = -2\la = 0$,
что противоречит нетривиальности линейной комбинации.

Поскольку исходное уравнение имеет вторую степень, его пространство решений двумерно. Отсюда следует, что~$u$ и~$v$
образуют базис в пространстве решений однородного уравнения, е составляют его фундаментальную систему решений.
Из теории линейных систем следует, что определитель Вронского $W_{u,v}(t)$ не обращается в нуль на $\R$.
Вычислим его:
\eqn{W_{u,v}(t) = \mbmat{2t^2+3 & t^2-6\cos t -3\\4t & 2t+6\sin t} = 6(3t+2t^2\sin t+3\sin t+4t\cos t).}
Однако при $t=0$ он равен нулю. Противоречие.
\end{solution}

\begin{note}
В других вариантах этой задачи решение может оказаться несколько сложнее. Дело в том,
что там невозможно явно указать значения $t$, при которых $W_{u,v}(t) = 0$. Выход из положения
найти две точки, в которых определитель имеет значения разных знаков, и сослаться на непрерывность.
\end{note}

\begin{note}
Авторское решение этой задачи предполагало использование оценки колеблемости для этого уравнения,
что не очень естественно. Кроме того, додуматься до этого на экзамене можно
только при идеальном знании теории.
\end{note}

\begin{problem}[2 балла]
Нарисовать характеристики уравнения
\eqn{\label{02.06.2004.3.eq.pd}3xu'_x - (x-2y)u'_y = 0}
на плоскости с координатами $(x,y)$ и определить тип особой точки $(0,0)$.
\end{problem}
\begin{solution}
Запишем систему для характеристик этого уравнения:
\eqn{\case{\dot x = 3x;\\\dot y = 2y-x.}}
Матрица этой системы есть $A = \rbmat{\phm 3&0\\-1&2}$. Вычислим характеристический многочлен:
$\det (A-\la E) = (\la -2)(\la-3)$. Корни различны и одного знака, следовательно, особая точка узел.
Найдём собственный базис: при $\la = 2$ имеем $A - 2E = \rbmat{\phm 1&0\\-1&0}$, собственный вектор $h_1 = \rbmat{0\\1}$.
При $\la = 3$ имеем $A - 3E = \rbmat{\phm 0& \phm 0\\-1 & -1}$, собственный вектор $h_2 = \rbmat{\phm 1\\-1}$.
В базисе $\hc{h_1,h_2}$ матрица системы имеет вид $\diag(2,3)$.
Как мы знаем, кривые прижимаются к тому собственному вектору, чьё собственное значение меньше по модулю,
$$
  \epsfbox{pictures.1}
$$
что и подтверждает построенный компьютером рисунок.
\end{solution}



\begin{problem}[3 балла]
  При каких $a \in \R$ для любой функции $\ph \in  \Cb^1(\R)$ в достаточно малой
  окрестности точки $(1,2)$ уравнение (\ref{02.06.2004.3.eq.pd}) имеет решение, удовлетворяющее
  условию $u(x+1,ax+2) = \ph(x)$?
\end{problem}


\setcounter{problem}{0}
\subsection{Вариант 4}

\begin{problem}[3 балла]
Известно, что $y_1=8\sin t$ и $y_2 = 2t^2$ решения уравнения
\eqn{\label{02.06.2004.4.eq.diff}y'' + p(t)y' + q(t)y = f(t), \quad p, q, f \in \Cb^1(\R).}
Найти хотя бы одно решение $y_3$ уравнения (\ref{02.06.2004.4.eq.diff}), удовлетворяющее
условию $y_3\hr{\pi} = -\frac12$. Сколько существует таких решений?
\end{problem}
%\begin{solution}
%\end{solution}

\begin{problem}[2 балла]
Может ли уравнение (\ref{02.06.2004.4.eq.diff}) иметь ещё и решение $y_4 = t^2 -1$?
\end{problem}
%\begin{solution}
%\end{solution}

\begin{problem}[2 балла]
Нарисовать характеристики уравнения
\eqn{\label{02.06.2004.4.eq.pd}2xu'_x - (y-3x)u'_y = 0}
на плоскости с координатами $(x,y)$ и определить тип особой точки $(0,0)$.
\end{problem}
%\begin{solution}
%\end{solution}

\begin{problem}[3 балла]
При каких $a \in \R$ для любой функции $\ph \in  \Cb^1(\R)$ в достаточно малой
окрестности точки $(2,1)$ уравнение (\ref{02.06.2004.4.eq.pd}) имеет решение, удовлетворяющее
условию $u(x+2,ax+1) = \ph(x)$?
\end{problem}
%\begin{solution}
%\end{solution}


\section{Экзамен 28.06.2004 (Пересдача №1)}

\subsection{Вариант 1}

\begin{problem}[2 балла]
  При каждом $\mu \in \R$ найти все решения уравнения \eqn{y'=(x^2+y^5)\cos\ln y,} удовлетворяющие
  условиям
\eqn{y(2)=e^{\frac\pi2}, \quad y'(2)=\mu.}
\end{problem}

\begin{problem}[2]
Выбрав при $\mu=0$ любое решение $y_1$ из найденных,
записать уравнение в вариациях и найти производную решения по начальным условиям вдоль $y_1$.
\end{problem}

\begin{problem}[3 балла]
Исследовать решение $y_1$ из предыдущей задачи на устойчивость.
\end{problem}

\begin{problem}[2 балла]
Существует ли определенное на $\R$ решение $y_2$ уравнения из первой задачи, удовлетворяющее
условию $y_2(2)=30$?
\end{problem}

\setcounter{problem}{0}
\subsection{Вариант 2}

\begin{problem}[2 балла]
  При каждом $\mu \in \R$ найти все решения уравнения \eqn{y'=(x^3+y^4)\sin e^y,} удовлетворяющие
  условиям
\eqn{y(3)=\ln\pi, \quad y'(3)=\mu.}
\end{problem}

\begin{problem}[2]
Выбрав при $\mu=0$ любое решение $y_1$ из найденных, записать уравнение в вариациях и найти
производную решения по начальным условиям вдоль $y_1$.
\end{problem}

\begin{problem}[3 балла]
Исследовать решение $y_1$ из предыдущей задачи на устойчивость.
\end{problem}

\begin{problem}[2 балла]
Существует ли определенное на $\R$ решение $y_2$ уравнения из первой задачи, удовлетворяющее
условию $y_2(3)=5$?
\end{problem}

\section{Экзамен 30.08.2004 (Пересдача №2)}

\subsection{Вариант 1}

Дано уравнение
\eqn{y'''+a(x)y''+b(x)y'+c(x)y=0, \quad a, b, c \in \Cb(\R).}
Известно, что $x+3$ и $\tg 2x - 1$ решения.

\begin{problem}[2 балла]
  Найти какое нибудь решение $y_3$, для которого $y_3(0)=0$, а $y_3'(0) = -7$.
\end{problem}
\begin{solution}
  Используя соображения, аналогичные первой задаче основного варианта этого года, ищем
  подходящие $c_1$ и~$c_2$ для решения $c_1(x+3)+c_2(\tg 2x-1)$. Получаем $-x-3\tg 2x$.
\end{solution}

\begin{problem}[2 балла]
  Сколько существует таких решений?
\end{problem}
\begin{solution}
  По теореме существования и~единственности условия $y_3(0)=0$, $y_3'(0)=-7$, $y_3''(0)=c$ дают единственное решение
  при каждом $c$, но $c$ можно выбирать произвольно, поэтому получаем ответ: бесконечно много решений.
\end{solution}

\begin{problem}[4 балла]
  В какой максимальной окрестности точки $0$ может быть определено данное уравнение с данными
  решениями?
\end{problem}
\begin{solution}
  По теореме продолжаемости для линейных уравнений при $a,b,c \in \Cb(I)$ для любой начальной точки существует
  единственное решение задачи Коши, и оно определено на всем интервале $I$. Отсюда из-за некоторых проблем с
  определённостью у функции $\tg2x-1$  в точках $\pm \frac\pi4$ видим, что оно никак не может быть
  решением линейного уравнения на каком-нибудь интервале более $\hr{-\frac\pi4, \frac\pi4}$. Это и есть ответ.

  \textbf{Дополнение к решению от П.~Бибикова:}

  Уже доказано, что решение не может быть определено на интервале,
  большем, чем $\left(-\frac{\pi}{4}; \frac{\pi}{4}\right)$. Не
  доказано лишь, что \emph{существуют} такие функции $a$, $b$ и $c$,
  что функции $y_1=x+3$ и $y_2=\tg 2x-1$~--- решения. Их необходимо
  предъявить непосредственно, подставив $y_1$ и $y_2$ в уравнение и
  решив полученную систему. Например, можно взять $b(x)=c(x)=0$,
  $a(x)=-4\frac{1+2\sin^2 2x}{\sin 4x}$ (проверьте!).

  \textbf{P.S.} Скорее всего, в условии этой задачи допущена ошибка: функции
  $a$, $b$ и $c$ должны быть непрерывны в~окрестности точки~$0$, а~не на
  всей числовой прямой.
\end{solution}

\setcounter{problem}{0}
\subsection{Вариант 2}

Дано уравнение \eqn{y'''+a(x)y''+b(x)y'+c(x)y=0, \quad a, b, c \in \Cb\br{U(0)}.} Известно, что
$x+2$ и $1 - \tg 3x$ решения.

\begin{problem}[2 балла]
  Найти какое нибудь решение $y_3$, для которого $y_3(0)=7$, а $y_3'(0) = 0$.
\end{problem}

\begin{problem}[2 балла]
  Сколько существует таких решений?
\end{problem}

\begin{problem}[4 балла]
  В какой максимальной окрестности точки~$0$ может быть определено данное уравнение с данными
  решениями?
\end{problem}


\begin{note}
  Чтобы получить <<5>>, надо набрать 9 баллов, откуда следует,
  что оценку <<5>> на этой пересдаче получить нельзя.
\end{note}


\section{Экзамен 2005}

\setcounter{problem}{0}
\subsection{Вариант 1}

\begin{problem}[4 балла]
Какие из функций $$y_1(t)=\cos
8t-1,\quad y_2(t)=\tg 4t,\quad y_3(t)=\cos(8t-5)$$ \emph{не могут}
быть решениями уравнения $$\ddot{y}+p(t)\dot{y}+q(t)y=0,\quad p,g\in
C^1(\mathbb{R}),$$ одним из решений которого служит функция
$y(t)=\cos 4t$?
\end{problem}

\begin{solution}
Согласно следствию из теоремы Штурма, нули линейно
независимых решений линейного однородного уравнения второго порядка
должны перемежаться. Очевидно, что решения $y(t)$ и $y_i(t)$
($i=1,2,3$) линейно независимы. Значит, их корни должны
перемежаться. Однако, как легко проверить, это условие выполняется
только для функции $y_2(t)$. Однако, согласно теореме о
продолжаемости, решение должно быть определено на всей прямой
(поскольку $p,q\in C^1(\mathbb{R})$), что для функции $y_2(t)$
неверно. По этой причине ни одна из предложенных функций не может
быть решением.

\textbf{Ответ:} никакие.
\end{solution}

\begin{problem}[2 балла]
В зависимости от параметра $a\in[0;3)$
определить точный тип и устойчивость особой точки системы
\eqn{\label{eqn:2005.1}
\case{
\dot{x}=3x,\\
\dot{y}=ax+3y.
}}
\end{problem}
\begin{solution}
Для начала найдем собственные вектора матрицы
$A=\hr{\begin{smallmatrix} 3& 0\\ a& 3 \end{smallmatrix}}$:
$$|A-\lambda E|=(\lambda-3)^2=0\Leftrightarrow \lambda_{1,2}=\lambda=3>0.$$
Отсюда сразу следует, что особая точка неустойчива. При $a\neq 0$
$\rk(A-\lambda E)=1$ $\Rightarrow$ это вырожденный узел; при
$a= 0$ $\rk(A-\lambda E)=0$ $\Rightarrow$ это дикритический
узел.

\textbf{Ответ:} при $a\neq 0$ особая точка --- неустойчивый
вырожденный узел;

при $a= 0$ особая точка --- неустойчивый дикритический узел.
\end{solution}

\begin{problem}[1 балл]
Существует ли и единственен ли локально вблизи точки $(1,0)\in G\subset\mathbb{R}^2$
первый интеграл $\varphi$ этой системы, удовлетворяющий условию
$$\varphi(x,y)=1+2y\quad\text{при $x=1$}.$$
\end{problem}

\begin{problem}[3 балла]
Существует ли и единственен ли целиком в области $G=\{(x,y)\vl  x>0,\,y>0\}$
первый интеграл $\varphi$ этой системы, удовлетворяющий условию
$$\varphi(x,y)=1+2y\quad\text{при $x=1$}.$$
\end{problem}

\section{Экзамен 2007}

\setcounter{problem}{0}
\subsection{Вариант 1}

\begin{problem}[1 балл]
Пусть $e^{At}=\hr{\begin{smallmatrix} e^{-2t}\cos 5t&
-e^{-2t}\sin 5t\\ e^{-2t}\sin 5t& e^{-2t}\cos
5t\end{smallmatrix}}$ и~$x_\mu(t)$ решение системы
$\dot{x}=Ax+\hr{\begin{smallmatrix}\mu t^2\\ \sh \mu\end{smallmatrix}}$ с~начальным условием
$x_\mu(0)=\hr{\begin{smallmatrix}\mu-1\\ 1\end{smallmatrix}}$.
Найти нулевой решение $x_0(t)$.
\end{problem}
\begin{solution}
Подставим $\mu=0$ и~решим задачу Коши
$$
\case{
\dot{x}&=Ax\\
x(0)&=\hr{\begin{smallmatrix}-1\\ 1\end{smallmatrix}}.}
$$
Согласно теореме из теории линейных систем, общее решение имеет вид
$x(t)=e^{At}C$. Найдем $C$. При $t=0$ имеем: $x(0)=C=\hr{\begin{smallmatrix}-1\\ 1\end{smallmatrix}}$.
Отсюда получаем

\textbf{Ответ:} $x_0(t)=e^{-2t}\hr{\begin{smallmatrix}-\cos 5t-\sin 5t\\ \cos 5t-\sin 5t\end{smallmatrix}}$.
\end{solution}

\begin{problem}[2 балла]
Исследовать решение $x_0(t)$ задачи~$1$ на устойчивость.
\end{problem}

\begin{solution}
Согласно теореме из теории устойчивости линейных
систем, достаточно исследовать на устойчивость нулевое решение. Для
этого найдем собственные числа матрицы $A$ (см.~задачу~$4$):
$$|A-\lambda E|=\begin{vmatrix}
-2-\lambda& -5\\
5& -2-\lambda
\end{vmatrix}=(\lambda+2)^2+25=0\Leftrightarrow \lambda_{1,2}=-2\pm 5i.$$
Поскольку $\mathrm{Re}\,\lambda_{1,2}=-2<0$, то по критерию устойчивости
нулевое решение асимптотически устойчиво, а значит, и решение
$x_0(t)$ асимптотически (и просто) устойчиво.

\textbf{Ответ:} решение $x_0(t)$ устойчиво и асимптотически устойчиво.
\end{solution}

\begin{problem}[3 балла]
Найти все первые интегралы системы из задачи~$1$,
определенные на всей плоскости.
\end{problem}
\begin{solution}
Пока отсутствует.

\textbf{Ответ:} $\varphi\equiv \mathrm{const}$.
\end{solution}

\begin{problem}[2 балла]
Найти матрицу $A$ из задачи~$1$.
\end{problem}
\begin{solution}
Согласно свойству экспоненты от матрицы,
$\frac{d}{dt}e^{At}=Ae^{At}$, откуда находим
$A=\frac{d}{dt}\big|_{t=0} e^{At}$.

\textbf{Ответ:} $A=\hr{\begin{smallmatrix} -2& -5\\ 5& -2\end{smallmatrix}}$.
\end{solution}

\begin{problem}[3 балла]
Существует ли, и если существует то чему равен
$$\liml{\mu\to0}\supl{t}|\dot{x}_\mu(t)-\dot{x}_0(t)|,\quad\text{
где $|x|=\max\{|x_1|,\,|x_2|\}$}.$$
\end{problem}

\begin{solution}
Пока отсутствует.

\textbf{Ответ:} предел существует и равен~$0$.
\end{solution}

\section{Экзамен 04.06.2011 (Основной экзамен)}
\setcounter{problem}{0}

\subsection{Вариант 5}

\begin{problem}[3 балла]
Про некоторую фиксированную линейную комбинацию
функции $y$ и~её производных $\dot y$, $\ddot y$, $\ldots$
известно, что она обнуляет ровно две из четырёх функций
\eqn{\label{2011.06.04.V.1}
y_1=\cos 2t,\qquad y_2=\cos^2 t, \qquad y_3=t^2\sin t, \qquad y_4=\sin t^2.}
Какие из них она заведомо обнуляет, а~какие заведомо не обнуляет?
\end{problem}

\begin{problem}[2+5 баллов]
Для системы
\eqn{\label{2011.06.04.V.2}
\case{
\dot x_1=\sin ax_1-a\tg x_2,\\
\dot x_2=2\hr{\sqrt{1+bx_2}-1}
}}
а)~при $a=b=4$ изобразите проекции графиков решений $X(t)=\hr{x_1(t), x_2(t)}$ на~фазовую плоскость
вблизи точки $(0,0)$.\\
б) найдите все пары $(a,b)\in \R^2$, при которых для любого $\ep>0$ существует такое $\de>0$,
что если $X(0)<\ep$, то $X(t)<\de$ при всех $t\ge 0$.
\end{problem}

\section{Экзамен 28.06.2011 (Пересдача)}

\subsection{Вариант 3}

Рассматривается уравнение
\eqn{\label{2011.06.28.III.1}y'=y^2-\sqrt[8]{|y|}.}
\begin{problem}[1+2 балла]
Какие точки $(x_0,y_0)\in \R^2$ являются:\\
а)~точками существования,\\
б)~точками единственности?
\end{problem}

\begin{problem}[2~балла]
Исследуйте на устойчивость (в том числе асимптотическую) решение $y(t)=1$, $t\ge 0$.
\end{problem}

\begin{problem}[1+1+2 балла]\\
а)~Сколько существует непродолжаемых решений, для которых $y(7)=0$?\\
б)~Сколько среди них локально различных?\\
в)~Все ли такие решения определены на~$\R$?
\end{problem}



\section{Экзамен 02.06.2012 (Основной экзамен)}

\subsection{Вариант 1}

Функция $y_1(t) = 6t^5 \cdot \ch 3t \cdot \sin 4t$ решение уравнения
$$y^{(n)} + a_1y^{(n-1)}\spl a_ny = 0\quad (a_1\sco a_n \in \R).$$
\begin{problem}[2+2 балла]\\
а)~При каком наименьшем $n \in \N$ это возможно?\\
б)~Обязательно ли это уравнение имеет решение
$$y_2(t) = (t-2)^5\cdot \ch(3t-6)\cdot \sin(4t-8)?$$
\end{problem}

\begin{problem}[2+1+3 балла]
Рассматриваются непродолжаемые решения задачи
$$
\case{
\dot x = 2x - 4y - e^{-t},\\
\dot y = 3x - 6y + 5\sqrt[4]{t^{-2}},
}
\quad
\case{
x(1) = 3,\\
\dot x(1) = -2.
}
$$
а)~Сколько их?\\
б)~На каких интервалах они определены?\\
в)~Устойчивы ли они: по Ляпунову, асимптотически?
\end{problem}

\subsection{Вариант 2}

Функция $y_1(t) = 4t^7 \cdot \sh 2t \cdot \cos 5t$ решение уравнения
$$y^{(n)} + a_1y^{(n-1)}\spl a_ny = 0\quad (a_1\sco a_n \in \R).$$
\begin{problem}[2+2 балла]\\
а)~При каком наименьшем $n \in \N$ это возможно?\\
б)~Обязательно ли это уравнение имеет решение
$$y_2(t) = (t+3)^6\cdot \sh(2t+6)\cdot \cos(5t+15)?$$
\end{problem}

\begin{problem}[2+1+3 балла]
Рассматриваются непродолжаемые решения задачи
$$
\case{
\dot x = 3x - 9y - 3\sqrt[7]{t^{-1}},\\
\dot y = 2x - 6y + t^{-4},
}
\quad
\case{
x(1) = -1,\\
\dot x(1) = 4.
}
$$
а)~Сколько их?\\
б)~На каких интервалах они определены?\\
в)~Устойчивы ли они: по Ляпунову, асимптотически?
\end{problem}


\subsection{Вариант 3}

Функция $y_1(t) = 5t^6 \cdot \ch 2t \cdot \sin 3t$ решение уравнения
$$y^{(n)} + a_1y^{(n-1)}\spl a_ny = 0\quad (a_1\sco a_n \in \R).$$
\begin{problem}[2+2 балла]\\
а)~При каком наименьшем $n \in \N$ это возможно?\\
б)~Обязательно ли это уравнение имеет решение
$$y_2(t) = (t-4)^7\cdot \ch(2t-8)\cdot \sin(3t-12)?$$
\end{problem}

\begin{problem}[2+1+3 балла]
Рассматриваются непродолжаемые решения задачи
$$
\case{
\dot x = 3x - 6y - e^{-t},\\
\dot y = 4x - 8y + 2/\sqrt[3]{t},
}
\quad
\case{
x(1) = 2,\\
\dot x(1) = -5.
}
$$
а)~Сколько их?\\
б)~На каких интервалах они определены?\\
в)~Устойчивы ли они: по Ляпунову, асимптотически?
\end{problem}

\subsection{Вариант 4}

Функция $y_1(t) = 7t^5 \cdot \sh 4t \cdot \cos 3t$ решение уравнения
$$y^{(n)} + a_1y^{(n-1)}\spl a_ny = 0\quad (a_1\sco a_n \in \R).$$
\begin{problem}[2+2 балла]\\
а)~При каком наименьшем $n \in \N$ это возможно?\\
б)~Обязательно ли это уравнение имеет решение
$$y_2(t) = (t+2)^7\cdot \sh(4t+8)\cdot \cos(3t+6)?$$
\end{problem}

\begin{problem}[2+1+3 балла]
Рассматриваются непродолжаемые решения задачи
$$
\case{
\dot x = 2x - 3y - \sqrt{t^{-4}},\\
\dot y = 4x - 6y + 3t^{-5},
}
\quad
\case{
x(1) = -2,\\
\dot x(1) = 3.
}
$$
а)~Сколько их?\\
б)~На каких интервалах они определены?\\
в)~Устойчивы ли они: по Ляпунову, асимптотически?
\end{problem}

\subsection{Вариант 5}

Функция $y_1(t) = 3t^7 \cdot \ch 5t \cdot \sin 2t$ решение уравнения
$$y^{(n)} + a_1y^{(n-1)}\spl a_ny = 0\quad (a_1\sco a_n \in \R).$$
\begin{problem}[2+2 балла]\\
а)~При каком наименьшем $n \in \N$ это возможно?\\
б)~Обязательно ли это уравнение имеет решение
$$y_2(t) = (t-3)^6\cdot \ch(5t-15)\cdot \sin(2t-6)?$$
\end{problem}

\begin{problem}[2+1+3 балла]
Рассматриваются непродолжаемые решения задачи
$$
\case{
\dot x = 3x - 2y - 5t^{-1},\\
\dot y = 9x - 6y + e^{-3t},
}
\quad
\case{
x(1) = 4,\\
\dot x(1) = -1.
}
$$
а)~Сколько их?\\
б)~На каких интервалах они определены?\\
в)~Устойчивы ли они: по Ляпунову, асимптотически?
\end{problem}

\subsection{Вариант 6}

Функция $y_1(t) = 8t^6 \cdot \sh 3t \cdot \cos 2t$ решение уравнения
$$y^{(n)} + a_1y^{(n-1)}\spl a_ny = 0\quad (a_1\sco a_n \in \R).$$
\begin{problem}[2+2 балла]\\
а)~При каком наименьшем $n \in \N$ это возможно?\\
б)~Обязательно ли это уравнение имеет решение
$$y_2(t) = (t+4)^7\cdot \sh(3t+12)\cdot \cos(2t+8)?$$
\end{problem}

\begin{problem}[2+1+3 балла]
Рассматриваются непродолжаемые решения задачи
$$
\case{
\dot x = 3x - 4y - 3/\sqrt[5]{t},\\
\dot y = 6x - 8y + t^{-6},
}
\quad
\case{
x(1) = -5,\\
\dot x(1) = 2.
}
$$
а)~Сколько их?\\
б)~На каких интервалах они определены?\\
в)~Устойчивы ли они: по Ляпунову, асимптотически?
\end{problem}


\medskip\dmvntrail
\end{document}
