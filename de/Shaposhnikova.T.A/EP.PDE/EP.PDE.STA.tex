\documentclass[a4paper]{article}

\usepackage[simple,thmnormal]{dmvn}
\usepackage{dmvnadd}
\usepackage{polyglossia}
\usepackage{unicode-math}
\usepackage{fontspec}
\defaultfontfeatures{Scale=MatchLowercase, Mapping=tex-text}
\setmainfont{CMU Serif}
\setsansfont{CMU Sans Serif}
\setmonofont{CMU Typewriter Text}
\setmathfont{xits-math.otf}



\title{Задачи экзаменов по УрЧП}
\author{Лектор\т Т.\,А.\,Шапошникова}
\date{V--VI семестры, 2004--2005 г.}

\begin{document}
\maketitle

\newcommand{\evub}[3]{% back skip
\newbox\evalbox%
\setbox\evalbox\hbox{{\scriptsize$#1$}}
\mskip2mu\vrule height #3 width .4pt depth #2\mskip2mu\lower#2\hbox{{\scriptsize$#1$}}\hskip-.75\wd\evalbox}
\newcommand{\evb}[2]{\evub{#1}{#2}{.65pc}}
\newcommand{\evnb}[1]{\evb{#1}{10pt}}

\section{Основной экзамен 6 июня 2005 г.}

\subsection{Вариант 2}

\subsubsection{Часть первая (практика)}

\begin{problem}[3 балла]
Найдите плотность распределения зарядов по поверхности сферы $|x|=3, x \in \R^3$, если известно,
что потенциал простого слоя поля, создаваемого этими зарядами равен $\frac4r$ вне шара $|x| \le 3$.
\end{problem}

\begin{problem}[2 балла]
Решите в $\Dc'(\R)$ уравнение: $y''+4y = \de(2-x)$.
\end{problem}
\begin{problem}[4 балла]
Рассмотрим задачу: $ u_{tt} = u_{xx}$, где $x,t \in \R$. Пусть
\eqn{u\evnb{t=x} = \ph(x), \quad u_t\evnb{t=x} = \psi(x), \quad x \in \R.}
При каких $\ph(x)$ и $\psi(x)$ из $\Cb^2(\R)$ существует решение
$u \in \Cb^2(\R^2)$ этой задачи? Ответ обосновать. Найти все такие решения.
\end{problem}
\begin{problem}[3 балла]
Пусть $u(x,t)$\т решение задачи Коши:
\eqn{\case{u_t = u_{xx}, \quad x \in \R, \quad t>0;\\
u\evnb{t=0} = e^{-x-4x^2}.}}
Найдите предел
\eqn{\liml{t\ra\bes}\ints{\R} u(x,t)\,dx.}
\end{problem}
\begin{problem}[3 балла за каждый пункт]
Существует ли в круге $Q := \hc{x^2+y^2 <1}$ решение $u(x,y) \bw\in \Cb^2(Q)\cap \Cb^1(\ol Q)$
уравнения $\De u = xy$, такое, что
\eqn{\ints{x^2+y^2<1} u(x,y)\,dx\,dy=1,}
и при этом: \textbf{а)} $u(0,0)=0$; \textbf{б)} $u(1,0)=0$. Ответ обосновать.
\end{problem}

\medskip

\subsubsection{Часть вторая (теория)}

\medskip
\setcounter{problem}{0}
\begin{problem}[3 балла]
Сформулируйте постановку внешней задачи Неймана. Докажите теорему единственности решения этой задачи.
\end{problem}

\begin{problem}[3 балла]
Сформулируйте и докажите принцип максимума для решения уравнения теплопроводности в ограниченной области.
\end{problem}

\begin{problem}[3 балла]
Сформулируйте и докажите теорему о скачке потенциала двойного слоя.
\end{problem}
\begin{problem}[2 балла]
Дайте понятие корректно поставленной задачи. Приведите пример Адамара.
\end{problem}

\medskip\dmvntrail
\end{document}
