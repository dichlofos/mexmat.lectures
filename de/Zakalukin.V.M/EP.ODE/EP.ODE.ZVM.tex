\documentclass[a4paper]{article}
\usepackage[simple,thmnormal,utf]{dmvn}
\usepackage{dmvnadd}

\title{Программа курса по обыкновенным\\ дифференциальным уравнениям}
\author{Лектор\т В.\,М.\,Закалюкин}
\date{III--IV семестр, 2005--2006 г.}

\begin{document}
\maketitle
\section{Осенний семестр}

\subsection{Программа курса}

\begin{nums}{-3}
\item Дифференциальное уравнение первого порядка в нормальной форме.
Решения. Примеры. Поле направлений на (расширенной) фазовой
плоскости. Изоклины. Формулировки теорем существования и
единственности решения задачи Коши. Пример: уравнение с
разделяющимися переменными.
\item Система дифференциальных уравнений в нормальном виде. Векторная
запись. Норма вектора. Ломанные Эйлера. Конус Пеано. Формулировка
теорем о существовании (для непрерывной правой части) и
единственности (для правой части, удовлетворяющей условию Липшица)
решения задачи Коши для системы.
\item Доказательство теоремы о существовании. Лемма Арцела.
Доказательство теоремы единственности.
\item Решения некоторых уравнений (однородные, линейные). Запись
уравнения в дифференциалах. Решения. Уравнения в полных
дифференциалах. Интегрирующий множитель.
\item Продолжение решения до границы. Примеры.
\item Уравнение, не разрешенное относительно производной.
\item Уравнения высших порядков. Некоторые методы решения. Сведение к
системе.
\item[8--9.]
Линейные системы. Общие свойства: существование, продолжаемость
решений, свойства пространства решений однородной системы,
фундаментальная система решений, определитель Вронского. Формулы
Лиувилля\ч Остроградского. Линейные неоднородные системы. Методы
вариации постоянных. Формула Грина.
\item[10--11.] Линейные уравнения (высших порядков) с постоянными
коэффициентами. Случаи вещественных, комплексных кратных свободных
значений, решения неоднородных уравнений с правой частью в виде
квазимногочлена.
\item[12.]
Линейные уравнения второго порядка. Приведение к нормальному виду.
Теоремы о нулях решений, теоремы сравнения Штурма. Поведение
решений на бесконечности. Преобразование Лиувилля.
\item[13.]
Краевые задачи линейных уравнений второго порядка. Функция Грина.
Примеры.
\item[14.]
Линейные системы с постоянными коэффициентами. Нормы в
пространстве матриц. Матричные ряды. Экспонента матрицы. Свойства.
Способы нахождения.
\item[15.]
Решение линейной системы в случае вещественных простых и кратных
корней.
\end{nums}

\subsection{Практические занятия (по задачнику А.\,Ф.\,Филиппова)}
\medskip
\begin{nums}{-3}
\item[1--2.]
Составление простейших дифференциальных уравнений. Изоклины.
Уравнения с разделяющимися переменными.
\item[3.]
Однородные уравнения. Линейные уравнения первого порядка.
\item[4.]
Уравнения в полных дифференциалах.
\item[5.]
Существование и единственность, продолжение решений.
\item[6.]
Уравнения, не разрешенные относительно производной.
\item[7.]
Уравнения старших порядков.
\item[8.]
Различные типы уравнений первого порядка и сводящиеся к ним.
\item[9.]
Контрольная работа.
\item[10--11.]
Линейные уравнения с постоянными коэффициентами.
\item[12--13.]
Линейные уравнения с переменными коэффициентами. Определитель
Вронского. Теоремы сравнения. Краевые задачи.
\item[14.]
Системы линейных уравнений с постоянными коэффициентами (начало,
тема будет продолжена в следующем семестре).
\item[15.]
Повторение.
\end{nums}

\section{Вопросы к экзамену}

\begin{note}
Из материала осеннего семестра входит только часть вопросов.
\end{note}

\subsection{Осенний семестр}

\begin{nums}{-2}
\item
Решение дифференциального уравнения (системы). Геометрический
смысл. Поле направлений. Векторное поле.
\item
Теорема о существовании решения дифференциального уравнения
(системы) с непрерывной частью.
\item
Ломаные Эйлера. Теорема Арцела.
\item
Теорема о единственности решения с данным начальным условием
системы, удовлетворяющей условию Липшица.
\item
Теорема о продолжении решения до границы компакта.
\item
Сведение уравнения старшего порядка к системе.
\item
Линейные системы, свойства (см. также вопросы 31-33 весеннего
семестра).
\item
Линейные уравнения с постоянными коэффициентами. Общее решение.
Случай простых корней.
\item
Решение неоднородных линейных уравнений.
\item
Линейные уравнения второго порядка (с переменными коэффициентами)
Свойства решений: о чередовании нулей, теорема сравнения Штурма.
\item
Системы линейных уравнений с постоянными коэффициентами.
\item
Нормированная алгебра матриц. Сходимость матричных рядов
\item
Экспонента матрицы. Свойства. Экспонента суммы коммутирующих
матриц.
\end{nums}

\subsection{Весенний семестр}

\begin{nums}{-2}
\item
Экспонента линейного оператора. Экспонента жордановой клетки.
\item
Комплексификация вещественного векторного пространства.
\item
Комплексно\д сопряженные линейные операторы. Сопряженность их
жордановых базисов. Собственные подпространства вещественного
оператора с комплексными собственными значениями.
\item
Экспонента вещественной матрицы, имеющей жордановы клетки с
комплексными собственными значениями
\item
Фазовые портреты линейных систем на плоскости.
\item
Метод вариации постоянных решения неоднородной линейной системы.
\item
Векторные квазимногочлены. Подпространства векторных
квазимногочленов, инвариантные относительно дифференциального
оператора $\frac{d}{dt}-A$
\item
Теорема о виде частного решения линейной системы с неоднородной
частью в виде векторного квазимногочлена.
\item
Если (вектор)\д функция представляется в виде линейной комбинации
квазимногочленов, то это представление единственно (доказать).
\item
Общее решение разностной линейной системы.
\item
Свойства фазовых преобразований системы дифференциальных
уравнений.
\item
Групповое свойство преобразований фазового потока автономной
системы дифференциальных уравнений.
\item
Однопараметрическая группа диффеоморфизмов. Существование
генератора (векторного поля), порождающего произвольную
однопараметрическую группу диффеоморфизмов.
\item
Теорема о непрерывной (дифференцируемой) зависимости решения
системы, удовлетворяющей условию Липшица (дифференцируемой), от
начального условия.
\item
Уравнение в вариациях вдоль некоторого решения системы
дифференциальных уравнений.
\item
Непрерывная (дифференцируемая) зависимость решения от параметров.
Сведение к теореме о зависимости от начальных условий.
\item
Лемма об оценке функции, удовлетворяющей интегральному
неравенству. Доказательство теоремы о не-\break прерывной зависимости от
начальных условий.
\item
Лемма Адамара.
\item
Доказательство теоремы о дифференцируемости решения по начальному
условию.
\item
Теорема о выпрямлении векторного поля в окрестности неособой
точки.
\item
Первый интеграл системы дифференциальных уравнений. Производная
функции вдоль векторного поля.
\item
Теорема о существовании полной системы первых интегралов в
окрестности неособой точи векторного поля.
\item
Общее решение линейного однородного уравнения в частных
производных первого порядка в окрестности неособой точки
характеристического векторного поля.
\item
Существование и единственность решения задачи Коши уравнения в
частных производных в окрестности регулярной
(нехарактеристической) точки.
\item
Решения квазилинейного уравнения в частных производных первого
порядка.
\item
Устойчивость (по Ляпунову) решения системы дифференциальных
уравнений. Асимптотическая устойчивость.
\item
Функция Ляпунова. Теорема Ляпунова об устойчивости положения
равновесия (или решения) при наличии функции Ляпунова.
\item
Устойчивость положения равновесия по линейному приближению. Вид
функции Ляпунова для линейной системы с собственными значениями из
левой комплексной полуплоскости.
\item
Теорема Четаева о неустойчивости положения равновесия.
\item
Свойства решений однородных линейных неавтономных систем систем.
Фундаментальная матрица. Определитель Вронского.
\item
Теорема о продолжаемости решений линейной системы.
\item
Формула Лиувилля\ч Остроградского. Следствие: формула для скорости
изменения объема области при действии фазового потока.
\item
Теорема Дюлака (и ее модификация) об отсутствии периодических
решений на плоскости.
\item
Линейные системы с периодическими коэффициентами. Оператор
монодромии (фазового преобразования за период).
\item
Логарифм невырожденной матрицы.
\item
Теорема Флоке.
\item
Мультипликаторы. Устойчивость периодической системы.
\item
Отображение последования Пуанкаре периодического решения
автономной системы, (доказательство существования, его
производная).
\item
Теорема об устойчивости периодического решения по мультипликаторам
отображения Пуанкаре. Вычисление мультипликатора предельного цикла
на плоскости.
\item
Устойчивость неподвижной точки гладкого отображения при итерациях.
Функция Ляпунова. Теорема об устойчивости (неустойчивости) по
линейному приближению.
\item
Решения систем уравнений малых колебаний.
\item
Равномерное распределение иррациональных обмоток тора.
\item
Число вращения диффеоморфизма окружности, свойства.
\item
Теорема Гробмана\ч Хартмана о топологическом строении фазовых кривых
в окрестности седловой точки (без доказательств).
\item
Понятие о бифуркациях фазовых портретов. Примеры: бифуркация
седло--узел, бифуркация Хопфа.
\item
Индекс особой точки векторного поля.
\item
Теорема Эйлера о сумме индексов особых точек.
\end{nums}

\begin{thebibliography}{Фил}
\bibitem[Фил]{Filippov}Филиппов А.Ф., {\it Сборник задач по обыкновенным дифференциальным уравнениям.}
\bibitem[Ст]{Stepanov}Степанов В.В., {\it Курс дифференциальных уравнений.}
\bibitem[Пе]{Petrovskiy}Петровский И.Г., {\it Лекции по теории обыкновенных дифференциальных уравнений.}
\bibitem[Эл]{Elsgolz}Эльсгольц Л., {\it Дифференциальные уравнения и вариационное исчисление.}
\bibitem[Би]{Bibikov}Бибиков Ю.Н., {\it Курс обыкновенных дифференциальных уравнений.} Высшая школа, 1991.
\bibitem[Ар]{Arnold}Арнольд В.И., {\it Обыкновенные дифференциальные уравнения.}
\bibitem[По]{Pontryagin}Понтрягин Л.С., {\it Обыкновенные дифференциальные уравнения.} 3-е издание.
\end{thebibliography}

\end{document}
