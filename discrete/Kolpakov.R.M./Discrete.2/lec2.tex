\section{Замыкания.}

\subsection{Определения.}

Возьмем множество $F \subseteq P_2$.

\begin{definition}
	Замыкание $[F]$ множества $F$ --- это множество всех булевых функций, получаемых из булевых функций множества $F$ с помощью операций суперпозиции, удаления и добавления фиктивных переменных.
\end{definition}

\begin{definition}
	$F$ --- замкнуто, если $[F] = F$.
\end{definition}

\begin{enumerate}
	\item
	$[\{x \oplus y\}] = \{0,\, x,\, x_1 \oplus \ldots \oplus x_t (t \ge 2)\}$
	\item
	$P_2$ --- замкнуто.	
\end{enumerate}

\begin{definition}
	$P_2(n)$ --- все булевы функции, существенно зависящие от не более, чем $n$ переменных.
\end{definition}

\begin{enumerate}
	\item
	$P_2(1)$ --- замкнуто.
	\item
	$P_2(2)$ --- не замкнуто. $\left( xy \in P_2(2),\, xyz \not\in P_2(2) \right)$
\end{enumerate}
\subsection{Свойства замыкания.}
\begin{enumerate}
	\item $F \subseteq [F].$
	\item $F_1 \subseteq F_2 \Longrightarrow [F_1] \subseteq [F_2]$
	\item $[[F]] = [F]$
	\begin{proof}
		1) $[F] \subseteq [[F]]$ (по 1, 2)

		2)$[[F]] \subseteq [F]$.

		$f(\XXX) \in [[F]] \Rightarrow \exists$ формула $\Phi$, реализующая $f$. Пусть $f_1, \ldots, f_s$ --- все функциональные символы, содержащиеся в $\Phi$. $f_1, \ldots, f_s \in [F] \Rightarrow $ каждая функция $f_i$ реализуется некоторой формулой $\Phi_i$ над $F$ : $\Phi = f_i(F_1, \ldots, F_{n_i})$.

		$\Phi_i(F_1, \ldots, F_{n_i})$ --- формула, полученная из $\Phi$ заменой $x_i \longmapsto F_i$. $\Phi_i(F_1, \ldots, F_n).$

		$\Phi_i(F_1, \ldots, F_n).$

		Так получим: 

		$\Phi'$ --- формулу над $F$, реализующую функцию $F \Rightarrow f \in [F] \Rightarrow [[F]] \subseteq [F]$.
	\end{proof}
	\item  $[F_1] \cap [F_2]$ --- замкнуто.
	\begin{proof}
		Возьмем $f \in [[F_1] \cap [F_2]]$: $f$ реализуется формулой $\Phi$ над $[F_1] \cap [F_2]$. Пусть $f_1, \ldots f_s$ --- все функциональные символы из $\Phi$. $\forall \: i \: f_i$ реализуется и формулой $\Phi_1$ над $F_1$ и формулой $\Phi_2$ над $F_2 \Rightarrow f \in  [F_1] \cap [F_2]$.
	\end{proof}
	\item  $[F_1] \cup [F_2]$ не обязательно замкнуто.
\end{enumerate}

\subsection{СДНФ и СКНФ.}
Пусть $F$ --- замкнутое множество, и $F_1 \subseteq F$.

\begin{definition}
	$F_1$ называется полным в $F$, если $[F_1] = F$.
\end{definition}

\begin{definition}
	$F_1$ называется полным, если $[F_1] = P_2$.
\end{definition}

\begin{example}
	$P_2$ --- полное множество.
\end{example}

\begin{statement}
	$f(\XXX)$ --- булева функция. \\Тогда: $f(\XXX) = (\overline{x_1} \: \& \: f(0, x_2, \ldots, x_n)) \vee (x_1 \: \& \: f(1, x_2, \ldots , x_n))$
\end{statement}
\begin{proof}
	Пусть $\sigma = (\SIG) $ --- набор значений переменных $\XXX$.
\begin{enumerate}
\item $\sigma_1 = 0$. 

$\overline{\sigma_1} \: \& \: f(0, \sigma_2, \ldots, \sigma_n) \vee \sigma_1 \: \& \: f(1, \sigma_2, \ldots, \sigma_n) =\\ = 1 \: \& \: f(0, \sigma_2, \ldots, \sigma_n) \vee 0 \: \& \: f(1, \sigma_2, \ldots, \sigma_n) =\\ = f(0, \sigma_2, \ldots, \sigma_n) = f(\SIG) $

\item $\sigma_1 = 1$.

$0 \: \& \: f(0, \sigma_2, \ldots, \sigma_n) \vee 1 \: \& \: f(1, \sigma_2, \ldots, \sigma_n) =\\ = f(1, \sigma_2, \ldots, \sigma_n) = f(\SIG) $
\end{enumerate}
\end{proof}

$
f(\XXX) = (\overline{x_1} \: \& \: f(0, x_2, \ldots, x_n)) \vee (x_1 \: \& \: f(1, x_2, \ldots , x_n)) = $

$= \overline{x_1} \: \& \: (\overline{x_2} \: \& \: f(0, 0, \ldots, x_n)) \vee (x_2 \: \& \: f(0, 1, \ldots , x_n)) \vee $

$ \vee (x_1 \: \& \: (\overline{x_2} \: \& \: f(1, 0, \ldots, x_n)) \vee (x_1 \: \& \: f(1, 1, \ldots , x_n))) =$

$
= \overline{x_1}\overline{x_2}f(0, 0, \ldots, x_n) \vee \overline{x_1} x_2 f(0, 1, \ldots, x_n) \vee x_1 \overline{x_2}f(1, 0, \ldots, x_n) \vee x_1 x_2 f(1, 1, \ldots, x_n)
$

\begin{definition}
	$x_\sigma = \begin{cases} x \text{, если }\sigma = 1 \\ \overline{x} \text{, если }\sigma = 0 \end{cases}$
\end{definition}


Итак, $f(\XXX)$ можно переписать в виде $\bigvee \limits_{\sigma_1, \sigma_2 \in E} f(\sigma_1, \sigma_2, x_3, \ldots, x_n)$.

Мы также можем аналогично разложить $f$ по $k$ переменным:

$f(\XXX) = \bigvee \limits_{(\sigma_1,\ldots, \sigma_k) \in E^k} f(\sigma_1, \ldots, \sigma_k, \ldots, x_n)$

При $k = n$ получаем:  $f(\XXX) = \bigvee \limits_{(\sigma_1,\ldots, \sigma_n) \in E^n} x_1^{\sigma_1}\ldots x_n^{\sigma_n}f(\SIG) =\\ = \bigvee \limits_{\substack{(\sigma_1,\ldots, \sigma_n) \in E^n \\ f(\SIG) = 1}} x_1^{\sigma_1}\ldots x_n^{\sigma_n} $

\begin{definition} Форма представления функции в виде\\ $f(\XXX)= \bigvee \limits_{\substack{(\sigma_1,\ldots, \sigma_n) \in E^n \\ f(\SIG) = 1}} x_1^{\sigma_1}\ldots x_n^{\sigma_n} $ называется\\ \textit{ Совершенной дизъюнктивной нормальной формой (СДНФ).}
\end{definition}
\begin{definition}
Пусть $f(\XXX) \ne 1$ --- булева функция.\\
$\overline{x} \ne 0 \Rightarrow \overline{f(\XXX)} = \bigvee \limits_{\substack{(\sigma_1,\ldots, \sigma_n) \in E^n \\ \overline{f}(\SIG) = 1}} \overline{x_1^{\sigma_1}\ldots x_n^{\sigma_n}} =\\ = \amper\limits_{\substack{(\sigma_1,\ldots, \sigma_n) \in E^n \\ f(\SIG) = 0}} \overline{x_1^{\sigma_1}} \vee \ldots \vee  \overline{x_n^{\sigma_n}} = \amper\limits_{\substack{(\sigma_1,\ldots, \sigma_n) \in E^n \\ f(\SIG) = 0}} x_1^{\overline{\sigma_1}} \vee \ldots \vee  x_n^{\overline{\sigma_n}}$ --- это так называемая \\ \textit{ Совершенная конъюктивная нормальная форма (СКНФ).}
\end{definition}




\begin{statement}
$	\{x\&y, x \vee y, \bar{x} \}$ --- полное множество.
\end{statement}
\begin{proof}
    Если $f \ne 0$, то СДНФ --- формула над $\{x\&y, x \vee y, \bar{x} \}$
    Если $f = 0$, то $f = \overline{x} \& x \Rightarrow $ любая функция реализуется формулой над $\{x\&y, x \vee y, \bar{x} \}$.
\end{proof}

\begin{lemma}[О сводимости полных множеств]
	$F, F' \subseteq P_2$, $F$ --- полное множество, и любая функция из $F$ может быть реализована формулой над $F' \Rightarrow F'$ --- полное множество.
\end{lemma}
\begin{proof}
	$\forall$ функция из $F$ может быть реализована формулой над $F' \Rightarrow F \subseteq [F'] \Rightarrow [F] \subseteq [[F']] = [F'].$

	$F$ --- полное $\Rightarrow [F] = P_2, [F] \subseteq [F'] \Rightarrow P_2 \subseteq [F'] \Rightarrow F'$ --- полное.
\end{proof}

%\newpage
%зачем??? Может, тут предполагается доделать что-то? Артём, пиши, в общем.
%убрал пока что и вставил заголовок по смыслу. Юра.
\subsection{Ещё примеры полных множеств функций.}

\begin{statement}
$	\{x\&y,\, \bar{x} \}$ --- полное множество. 
\end{statement}
\begin{proof}
			$\{x\vee y,\, x \& y,\, \bar{x} \}$ --- полное множество. Учитывая, что $x \vee y = \overline{\bar{x}\&\bar{y} } $ , то по лемме о сводимости получаем нужное. \end{proof}	
\begin{statement}
	$\{x \vee y,\, \bar{x} \}$ --- полное множество.  
\end{statement}
\begin{proof}
	$	\{x\&y,\, \bar{x} \}$ --- \text{полное множество.}  Учитывая, что: $x \& y =  \overline{\bar{x}\vee\bar{y} }$, то по лемме о сводимости получаем нужное.
\end{proof}
\begin{statement} 
$\{x \oplus y,\, x \& y,\, 1\}$ --- полное множество.
\end{statement}
\begin{proof}
	$\bar{x}=x \oplus 1. $ Получаем нужное по лемме о сводимости и утверждению 2.
\end{proof}	
\begin{statement}
	$\{x | y\}$ --- полное множество. 
\end{statement}
\begin{proof}
	$x|y=\bar{x} \vee \bar{y}=\overline{x \& y};$ \\
	$\bar{x}=x | x; \\
	x \& y=\overline{x|y}=(x|y)|(x|y); \\
	\{x \& y,\, \bar{x}\}$ --- полное по лемме о сводимости, значит $\{x|y\}$ --- полное.   
\end{proof}
\begin{corollary}
	Из любого полного множества можно выделить конечное полное подмножество.
\end{corollary}
\begin{proof}
	$F\subseteq P_2$ --- полное множество. $\Rightarrow$ существует формула Ф над $F$, реализующая ${x|y}$. Пусть $\{f_1, \ldots, f_s\}$ --- множество всех символов функций, содержащихся в Ф.Ф --- формула над $\{f_1, \ldots, f_s\}$ $\Rightarrow  ~ x|y$ содержится в замыкании. $\{x|y\}$ --- полное. $\Rightarrow$ по лемме о сводимости $\{f_1, \ldots, f_s\} \subseteq F$ --- полное.
\end{proof}  