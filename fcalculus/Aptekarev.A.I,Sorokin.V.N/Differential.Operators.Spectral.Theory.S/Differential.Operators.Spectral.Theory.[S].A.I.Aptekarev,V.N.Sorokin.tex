\documentclass[12 pt, a4 paper]{article}
\usepackage[utf8]{inputenc}
\usepackage[russian]{babel}
\usepackage{dmvn}
\usepackage{amssymb, amsmath, amsfonts, theorem}
\theoremstyle{plain}   \newtheorem{Pro}{Задача}
					   \newtheorem{Sta}{Утверждение}
					   \newtheorem{Def}{Определение}
					   \newtheorem{Rem}{Замечание}
			           \newtheorem{The}{Теорема}
					   \newtheorem{Cor}{Следствие}
					   \newtheorem{Lem}{Лемма}
					   \newtheorem{Exe}{Упражнение}
					   \newtheorem{Exa}{Пример}
\begin{document}
%%%%%%%%%%%%%%%%%%%%%%%%%%%%%%%%%%%%%%%%%%%%%%%%%%%%%%%%%%%%%%%%%%%%
%%%%%%%%%   Окончательный вариант   %%%%%%%%%%%%%%%%%%%%%%%%%%%%%%%%
%%%%%%%%%%%%%%%%%%%%%%%%%%%%%%%%%%%%%%%%%%%%%%%%%%%%%%%%%%%%%%%%%%%%
\dmvntitle{Cпектральная теория}
{ разностных операторов}
{лекторы --  А.И.Аптекарев, В.Н.Сорокин}
{}
{Москва 2002 год}
\newpage
$$ \; $$
{\bfseries Аптекарев А.И., Сорокин В.Н.}
$$ \; $$
$$ \; $$
{\bfseries Спектральная теория разностных операторов}
$$ \; $$
Учебное пособие.--Издательство механико- \\
математического факультета МГУ, Москва, \\
2002 г.--168 стр.
$$ \; $$
В настоящем пособии продемонстрирована связь теории непрерывных
дробей и диагональных аппроксимаций Паде с прямой и обратной
спектральной задачей для разностного оператора второго порядка,
заданного матрицей Якоби. Большой раздел уделен классической
проблеме моментов Гамбургера, включая необходимые сведения из
теории граничных свойств аналитических функций. В рамках
исследований Т.Стилтьеса по непрерывным дробям изучаются
задачи о колебаниях дискретной струны и цепочек Ленгмюра. \\
Для студентов, аспирантов и сотрудников математических
факультетов университетов.
\vspace{4 cm}\\
C (2002) А.И.Аптекарев, В.Н.Сорокин.
\newpage
                  %%%%%%%%%%%%%%%%%%%%%%%%%%%%%%%%%%%%%%%%%%%%%%%%%%%%%%%%%
				  %%%%%%%%%%%%%%%%%%%%%%%%%%%%%%%%%%%%%%%%%%%%%%%%%%%%%%%%%
				  %%%%%%%%%%%%%%   Содержание   %%%%%%%%%%%%%%%%%%%%%%%%%%%
				  %%%%%%%%%%%%%%%%%%%%%%%%%%%%%%%%%%%%%%%%%%%%%%%%%%%%%%%%%
				  %%%%%%%%%%%%%%%%%%%%%%%%%%%%%%%%%%%%%%%%%%%%%%%%%%%%%%%%%
{\Large \bfseries Содержание}
$$ \; $$
{\bfseries Введение}
$$ \; $$
{\Large \bfseries 1.}
Алгебраическая теория \\
$ \; $ разностных операторов второго порядка \\
{\bfseries 1.1.}
Аппроксимации Паде \\
{\bfseries 1.2.}
Ортогональные многочлены \\
{\bfseries 1.3.}
Матрицы Якоби \\
{\bfseries 1.4.}
Теорема Чебышева-Фавара \\
{\bfseries 1.5.}
Непрерывные дроби \\
{\bfseries 1.6.}
Резольвента \\
{\bfseries 1.7.}
Пример ограниченного оператора \\
{\bfseries 1.8.}
Пример неограниченного оператора \\
{\bfseries 1.9.}
Спектральные данные, восстанавливающие \\
оператор.
Прямая и обратная задача
$$ \; $$
{\Large \bfseries 2.}
Классическая проблема моментов \\
{\bfseries 2.1.}
Позитивные последовательности \\
{\bfseries 2.2.}
Нули ортогональных многочленов \\
{\bfseries 2.3.}
Квадратурная формула Гаусса-Якоби \\
{\bfseries 2.4.}
Разрешимость проблемы моментов Гамбургера \\
{\bfseries (a)}
Разрешимость проблемы моментов \\
{\bfseries (b)}
Дополнение -- теоремы Хелли \\
{\bfseries 2.5}
Теорема Маркова \\
{\bfseries 2.6}
Формула обращения Стилтьеса-Перрона \\
{\bfseries (a)}
Формула обращения \\
{\bfseries (b)}
Классы аналитических функций \\
{\bfseries 2.7.}
Классы квазианалитичности \\
{\bfseries 2.8.}
Критерий Крейна \\
{\bfseries 2.9.}
Граничные значения аналитических функций \\
{\bfseries (a)}
Постановка задачи \\
{\bfseries (b)}
Гармонические функции \\
{\bfseries (c)}
Ядро Пуассона \\
{\bfseries (d)}
Формула Пуассона \\
{\bfseries (e)}
Теорема Фату \\
{\bfseries (f)}
Ограниченные гармонические функции \\
{\bfseries (g)}
Доказательство теоремы $ \mathcal{C} $ \\
{\bfseries 2.10.}
Теорема об устойчивости \\
ортогональных многочленов \\
{\bfseries 2.11}
Достаточное условие определенности \\
проблемы моментов \\
{\bfseries 2.12.}
Формула Кристоффеля-Дарбу \\
{\bfseries 2.13.}
Экстремальные свойства \\
ортогональных многочленов \\
{\bfseries 2.14.}
Теорема Карлемана
$$ \; $$
{\Large \bfseries 3.}
Теория фон Неймана \\
{\bfseries 3.1.}
Линейные операторы \\
{\bfseries 3.2.}
Лебеговы операторы \\
{\bfseries 3.3.}
Самосопряженные операторы
$$ \; $$
{\Large \bfseries 4.}
Разностные операторы второго порядка \\
и непрерывная дробь Стилтьеса \\
{\bfseries 4.1.}
Дробь Стилтьеса и ее эквивалентные формы \\
{\bfseries 4.2.}
Рекуррентные соотношения и разностные \\
задачи, связанные с дробью Стилтьеса \\
{\bfseries 4.3.}
Решение прямой задачи \\
для спектральных данных Стилтьеса
$$ \; $$
{\Large \bfseries 5.}
Проблема моментов Стилтьеса \\
{\bfseries 5.1.}
Разрешимость проблемы моментов Стилтьеса \\
{\bfseries 5.2.}
Сходимость дробей Стилтьеса \\
{\bfseries 5.3,}
Определенность проблемы моментов Стильеса
$$ \; $$
{\Large \bfseries 6.}
Приложения непрерывных дробей \\
Стилтьеса к линейным и нелинейным \\
механическим системам \\
{\bfseries 6.1.}
Дискретная струна Стилтьеса-Крейна \\
{\bfseries (a)}
Постановка задачи \\
{\bfseries (b)}
Дискретная струна с конечным числом масс \\
{\bfseries (c)}
Дискретная струна с бесконечным числом масс \\
{\bfseries 6.2.}
Цепочка Ленгмюра \\
{\bfseries (a)}
Постановка задачи. Непрерывные аналоги \\
{\bfseries (b)}
Метод обратной спектральной задачи
$$ \; $$
{\bfseries Литература}
\newpage
                        %%%%%%%%%%%%%%%%%%%%%%%%%%%%%%%%%%%%%%%%%%%%%%%%%
						%%%%%%%%%%%%%%%%%%%%%%%%%%%%%%%%%%%%%%%%%%%%%%%%%
						%%%%%%%%%   Предисловие   %%%%%%%%%%%%%%%%%%%%%%%
						%%%%%%%%%%%%%%%%%%%%%%%%%%%%%%%%%%%%%%%%%%%%%%%%%
						%%%%%%%%%%%%%%%%%%%%%%%%%%%%%%%%%%%%%%%%%%%%%%%%%
{\Large \bfseries Предисловие} \\

В настоящем пособии предпринята попытка изложить спектральную
теорию разностных операторов, включая постановку и решение
прямой и обратной спектральной задачи, а также приложения к
некоторым уравнениям математической физики, с единой точки зрения,
а именно, в рамках теории непрерывных дробей.
Такой подход обусловлен историческими причинами, он восходит к
работам П.Л.Чебышева, А.А.Маркова, Т.Стилтьеса и к более поздним
исследованиям М.Крейна. \\

В связи с этим, в первой главе вводятся все основные объекты
теории: аппроксимации Паде и ортогональные многочлены, непрерывные
дроби и матрицы Якоби. Устанавливаются связи между этими понятиями.
Вычисление резольвенты матрицы Якоби (т.е. разностного оператора
второго порядка) позволяет установить связь между спектральными
характеристиками этой матрицы и теорией непрерывных дробей. \\

Вторая глава посвящена классической проблеме моментов, которая
представляет собой естественное продолжение развитой в первой
главе алгебраической теории для матриц Якоби с положительными
(внедиагональными) элементами.
Дано полное и подробное исследование проблемы моментов Гамбургера,
начиная с более простого вопроса о разрешимости проблемы моментов
и заканчивая более сложными вопросами определенности проблемы моментов.
При этом мы существенно опираемся на теорию граничных значений
аналитических функций. В пособии приводятся подробные доказательства
всех необходимых нам фактов этой теории, включая теорему Фату. \\

В третьей главе показано, что проблема моментов Гамбургера,
по-существу, представляет собой спектральную теорему для самосопряженных
операторов с простым спектром. Более того, она позволяет описать
все самосопряженные расширения лебеговых замкнутых симметричных
операторов. В этой сязи доказывается общая теорема Дж. фон Неймана,
дающая описание области определения сопряженного оператора. \\

Опираясь на развитую во второй главе теорию, в четвертой и
пятой главах мы изучаем
важный специальный случай непрерывных дробей, а именно, дроби
Стилтьеса и связанную с ними проблему моментов.
Именно эти дроби применяются в последней главе для решения
некоторых линейных и нелинейных задач математической физики.
Одна из них представляет собой восходящую к М.Крейну трактовку
непрерывной дроби Стилтьеса как задачи о колебании струны,
несущей дискретные массы.
Другая задача заключается в решении бесконечной системы нелинейных
дифференциальных уравнений, которая называется цепочкой Ленгмюра.
Она представляет собой упрощенную одномерную модель колебаний
атомов в кристаллической решетке и служит дискретным аналогом
знаменитого уравнения Кортевега-де-Фриза. \\

Пособие возникло на основе специального годового курса лекций,
прочитанных авторами на кафедре теории функций и функционального
анализа в 2000-2001 учебном году. \\

Изложение материала в данном учебном пособии замкнутое,
т.е. для усоения курса нет необходимости обращатся к какой-либо
дополнительной литературе (от читателя лишь требуется знание
программы первых трех курсов механико-математического факультета).
Тем не менее в конце пособия мы приводим список дополнительной
литературы, в которой заинтересованный читатель может найти дальнейшее
развитие теории. \\

Есть надежда, что этот курс принесет определенную пользу
при обучении студентов и аспирантов математиков.
Работа также расчитана на преподавателей и научных сотрудников,
специализирующихся в области комплексного и функционального
анализа.
\newpage
						 %%%%%%%%%%%%%%%%%%%%%%%%%%%%%%%%%%%%%%%%%%%%%%%%%%
						 %%%%%%%%%%%%%%%%%%%%%%%%%%%%%%%%%%%%%%%%%%%%%%%%%%
						 %%%%%%%%%%   1   Алгебраическая теория   %%%%%%%%%
						 %%%%% разностных операторов второго порядка   %%%%
						 %%%%%%%%%%%%%%%%%%%%%%%%%%%%%%%%%%%%%%%%%%%%%%%%%%
						 %%%%%%%%%%%%%%%%%%%%%%%%%%%%%%%%%%%%%%%%%%%%%%%%%%
\section {Алгебраическая теория\\
разностных операторов второго порядка}
$ \; $
\\
$ \; $
                         %%%%%%%%%%%%%%%%%%%%%%%%%%%%%%%%%%%%%%%%%%%%%%%%%%%
						 %%%%%%%%   1.1 Аппроксимации Паде   %%%%%%%%%%%%%%%
						 %%%%%%%%%%%%%%%%%%%%%%%%%%%%%%%%%%%%%%%%%%%%%%%%%%%
\subsection{Аппроксимации Паде}
$ \; $
\\

Дана последовательность комплексных чисел
$ \mathbf{s}=\{s_n \}_{n=0}^{\infty}. $
Обозначим
$ \mathbb{C}[\lambda ] - $
линейное пространство многочленов с комплексными коэффициентами
от переменной
$ \lambda . $
Определим линейный функционал
$$
  \mathfrak{S} : \mathbb{C} [\lambda ]
  \longrightarrow \mathbb{C}
$$
по формуле
$$
  P(\lambda )=p_0 + p_1 \lambda +...+p_n \lambda ^n
  \mapsto \mathfrak{S} \{ P \} =
  p_0 s_0 + p_1 s_1 +...+p_n s_n .
$$
Это общий вид линейного функционала в пространстве
$ \mathbb{C}[\lambda ]. $
Функционал
$ \mathfrak{S} $
определен своими
{\bfseries степенными моментами}
$$
  s_n = \mathfrak{S} \{ \lambda ^n \} ,
  \quad n=0, \; 1, \; 2, ...
$$
\\

Рассмотрим формальный степенной ряд
$$
  f(z)=\sum _{n=0}^{\infty}
  \frac{s_n }{z^{n+1}}.
$$
Формально имеем
$$
  f(z)=\mathfrak{S}_{\lambda} \biggl \{ \frac{1}{z- \lambda } \biggr \}
$$
в том смысле, что
$$
  \mathfrak{S}_{\lambda} \biggl \{ \frac{1}{z-\lambda} \biggr \}=
  \mathfrak{S}_{\lambda} \biggl \{
  \sum _{n=0}^{\infty}
  \frac{\lambda ^n }{z^{n+1}} \biggr \}=
  \sum _{n=0}^{\infty}
  \frac{\mathfrak{S} \{ \lambda ^n \} }{z^{n+1}}.
$$
Поставим задачу о
{\bfseries диагональных аппроксимациях\\
Паде}.
                               %%%%%%%%%   Задача 1   %%%%%%%%%%%%%%%%%%%%
\begin{Pro}
Для целого неотрицательного числа
$ n \in \mathbb{Z}_+ $
требуется найти многочлен
$ Q_n $
такой, что
\\
1) $ \; \deg Q_n \leq n $\\
2) $ \; Q_n \not \equiv 0 $\\
3) для некоторого многочлена
$ P_n $
формальный степенной ряд
$$
  R_n (z)=Q_n (z) f(z)-P_n (z)
$$
начинается не ниже, чем с
$ (n+1)- $ й степени:
$$
  R_n (z)=\frac{A_n }{z^{n+1}}+...
$$
\end{Pro}
                            %%%%%%%%%%%   Утверждение 1   %%%%%%%%%%%%%%%%%
\begin{Sta}
Решение задачи 1 существует.
\end{Sta}
{\Large Доказательство.}
Ищем многочлен
$ Q_n $
со свойствами 1) и 2) такой, что
$$
  \{ Q_n f \} _m =0,
  \quad m=0,...,n-1,
  \qquad (\ast)
$$
где
$ \{ \cdot \} _m \; - \; $
коэффициент при
$ 1/z^{m+1}. $
Находим
$ P_n $
как полиномиальную часть ряда
$ Q_n f .$
Соотношения
$ (\ast) $
представляют собой систему
$ \; n \; $
линейных однородных уравнений относительно
$ \; n+1 \; $
неизвестных коэффициентов многочлена
$ \; Q_n . $
Такая система имеет нетривиальные решения.
$ \triangle $
                                  %%%%%%%%%%%%   Определение 1   %%%%%%%%%%%
\begin{Def}
Пусть
$ (Q_n , P_n ) - $
любое решение задачи 1. Тогда рациональная функция
$$
  \pi _n =\frac{ P_n }{Q_n }
$$
называется
$ n- $ й
{\bfseries диагональной аппроксимацией Паде}
формального степенного ряда
$ f . $
\end{Def}
Следующее утверждение доказывает корректность этого определения.
                            %%%%%%%%%%%%   Утверждение 2   %%%%%%%%%%%%%%%%
\begin{Sta}
Отношение многочленов
$ P_n /Q_n $
единственно.
\end{Sta}
{\Large Доказательство.}
Пусть есть два решения
\begin{equation*}
  \begin{cases}
    Q_n ^{\prime }f-P_n ^{\prime}=
	\frac{A_n ^{\prime}}{z^{n+1}}+...\\
	Q_n ^{\prime \prime}f-P_n ^{\prime \prime}=
	\frac{A_n ^{\prime \prime}}{z^{n+1}}+...
  \end{cases}
\end{equation*}
Умножим первое равенство на
$ Q_n ^{\prime \prime} ,$
второе - на
$ Q_n ^{\prime} $
и вычтем одно из другого. Получим
$$
  Q_n ^{\prime}P_n ^{\prime \prime}-
  Q_n ^{\prime \prime}P_n ^{\prime}=
  \frac{A}{z}+...
$$
Слева стоит многочлен, следовательно, он тождественно
равен нулю, т.е.
$$
  \frac{P_n ^{\prime}}{Q_n ^{\prime}}=
  \frac{P_n ^{\prime \prime}}
  {Q_n ^{\prime \prime}}.
$$
$ \triangle $
\newpage
                              %%%%%%%%%%%%%%%%%%%%%%%%%%%%%%%%%%%%%%%%%%%%%%
							  %%%%%%%%%   1.2   Ортогональные многочлены %%%
							  %%%%%%%%%%%%%%%%%%%%%%%%%%%%%%%%%%%%%%%%%%%%%%
\subsection{Ортогональные многочлены}
$ \; $
\\

Выше мы заметили, что нахождение знаменателя
$$
  Q_n (z)=q_0 +q_1 z+...+q_n z^n
$$
сводится к решению линейной системы
\begin{equation*}
  \begin{cases}
    s_0 q_0 +s_1 q_1 +...+s_n q_n =0\\
	..........................................\\
	s_{n-1}q_0 +s_n q_1 +...+s_{2n-1}q_n =0.
  \end{cases}
  \qquad (\ast )
\end{equation*}
                        %%%%%%%%%%%%%%   Определение 2   %%%%%%%%%%%%%%%%%%
\begin{Def}
{\bfseries Определителями Ганкеля}
называются
\begin{equation*}
  H_n =
    \begin{vmatrix}
	  s_0 & s_1 & \dots & s_{n-1}\\
	  \hdotsfor{4}\\
	  s_{n-1} & s_n & \dots & s_{2n-2}
	\end{vmatrix},
\end{equation*}
где
$ n=1, \; 2, \; 3,... $
Полагаем
$ H_0 =1. $
\end{Def}
                        %%%%%%%%%%%%%   Определение 3   %%%%%%%%%%%%%%%%%%%
\begin{Def}
Последовательность моментов
$ \mathbf{s} $
(функционал
$ \mathfrak{S}) $
называется
{\bfseries невырожденной}
(или
{\bfseries несингулярной}),
если все ее определители Ганкеля отличны от нуля:
$$
  H_n \not = 0, \quad n=0, \; 1, \; 2,...
$$
\end{Def}
                                    %%%%%%%%%%%%   Утверждение 3   %%%%%%%%
\begin{Sta}
Если последовательность
$ \mathbf{s} $
несингулярна, то для любого
$ n \in \mathbb{Z}_+ $
многочлен
$ Q_n $
определен единственным образом (с точностью до нормировки) и
имеет степень ровно
$ n. $
\end{Sta}
{\Large Доказательство.}
Положим в
$ (\ast ) \; \; q_n =1. $
Получим систему
$ n $
линейных уравнений относительно
$ n $
неизвестных с определителем
$ H_n \not = 0. $
Система имеет единственное решение.
$ \triangle $
\\

Запишем
$ m- $
е уравнение системы
$ ( \ast ) $
следующим образом
$$
  s_m q_0 +s_{m+1} q_1 +...+s_{m+n} q_n =
  \mathfrak{S} \{ q_0 \lambda ^m +q_1 \lambda ^{m+1}+...+
  q_n \lambda ^{m+n} \} =
$$
$$
  =\mathfrak{S} \{ Q_n (\lambda ) \lambda ^m \}=0.
$$
Получим
                             %%%%%%%%%%%%%%%   Замечание 1   %%%%%%%%%%%%%
\begin{Rem}
Задача 1 равносильна следующим
{\bfseries соотношениям ортогональности}
$$
  \mathfrak{S} \{ Q_n (\lambda ) \lambda ^m \} =0,
  \quad m=0,...,n-1,
$$
т.е.
$ Q_n - \; n- $
й
{\bfseries ортогональный многочлен}
относительно функционала
$ \mathfrak{S}. $
\end{Rem}
$ \; $
\\

Начиная с этого момента всегда считаем, что функционал
невырожденный, и тем самым, единственным образом
определена последовательность ортогональных многочленов
$ \{ Q_n (\lambda ) \} _{n=0}^{\infty}. $
Пусть
$  Q_n $
нормирован условием, что старший коэффициент многочлена равен
единице. Напишем детерминантную формулу для таких многочленов.
                          %%%%%%%%%%%%%   Утверждение 4   %%%%%%%%%%%%%%%%%
\begin{Sta}
Справедлива формула
\begin{equation*}
   Q_n (\lambda )=
  \frac{1}{H_n }
    \begin{vmatrix}
	  s_0 & s_1 & \dots & s_n \\
	  \hdotsfor{4}\\
	  s_{n-1} & s_n & \dots & s_{2n-1}\\
	  1 & \lambda & \dots & \lambda ^n
	\end{vmatrix}
\end{equation*}
\end{Sta}
{\Large Доказательство.}
В правой части равенства стоит многочлен степени
$ n $
с единичным старшим коэффициентом. Проверим соотношения ортогональности:
\begin{equation*}
  \mathfrak{S} \{  Q_n (\lambda ) \lambda ^m \} =
  \frac{1}{H_n }
    \begin{vmatrix}
	  \hdotsfor{3}\\
	  s_m & \dots & s_{m+n}\\
	  \hdotsfor{3}\\
	  s_m & \dots & s_{m+n}
	\end{vmatrix}
  =0.
\end{equation*}
Мы получили определитель с двумя одинаковыми строками.
$ \triangle $
                             %%%%%%%%%%%%   Утверждение 5   %%%%%%%%%%%%%%%
\begin{Sta}
Справедливы формулы
$$
   P_n (z) =
  \mathfrak{S}_{\lambda }
  \biggl \{
  \frac{ Q_n (z) -  Q_n (\lambda )}
  {z-\lambda}
  \biggr \} ,
$$
$$
   R_n (z) =
  \mathfrak{S}_{\lambda}
  \biggl \{
  \frac{ Q_n (\lambda )}
  {z-\lambda}
  \biggr \} ,
$$
$$
   A_n =
  \frac{H_{n+1}}{H_n }.
$$
\end{Sta}
{\Large Доказательство.}
Первая формула определяет многочлен, который можно записать
в виде
$$
   P_n (z)=
   Q_n (z)f(z)-  R_n (z).
$$
При этом
$$
   R_n (z)=
  \sum _{m=0}^{\infty}
  \frac{1}{z^{m+1}}
  \mathfrak{S} \{  Q_n (\lambda ) \lambda ^m \}
  =\frac{ A_n }{z^{n+1}}+...,
$$
где
$$
   A_n =
  \mathfrak{S} \{  Q_n (\lambda ) \lambda ^n \}=
  H_{n+1}/H_n .
$$
$ \triangle $
\newpage
                               %%%%%%%%%%%%%%%%%%%%%%%%%%%%%%%%%%%%%%%%%%%%%%
							   %%%%%%%%   1.3     Матрицы Якоби   %%%%%%%%%%%
							   %%%%%%%%%%%%%%%%%%%%%%%%%%%%%%%%%%%%%%%%%%%%%%
\subsection{Матрицы Якоби}
$ \; $
\\

                               %%%%%%%%%%%   Утверждение 6   %%%%%%%%%%%%%%%%
\begin{Sta}
Если
$ \{ Q_n (\lambda ) \} _{n=0}^{\infty} \; - \; $
последовательность ортогональных многочленов с единичным
старшим коэффициентом, то существуют и единственны
комплексные коэффициенты
$$
  a_n \not = 0  \; и \; b_n ,
  \quad n=1, \; 2, \; 3,...,
$$
такие, что последовательность ортогональных многочленов
является единственным решением следующего\\
{\bfseries трехчленного рекуррентного соотношения}:
$$
  Q_{n+1}(\lambda )=
  (\lambda -b_n )Q_n (\lambda )-
  a_n Q_{n-1}(\lambda ),
  \quad n=1, \; 2, \; 3,...,
  \qquad (\ast )
$$
с начальными условиями
$$
  Q_0 (\lambda )=1, \quad
  Q_1 (\lambda )=\lambda - s_1 /s_0 .
$$
\end{Sta}
{\Large Доказательство.}
Начальные условия проверяются непосредственно.\\
Далее, последовательность
$ \{ Q_n \}_0 ^{\infty} - \; $
базис пространства
$ \mathbb{C}[\lambda ]. $
Заметим, что
$$
  \mathfrak{S} \{ Q_n Q_m \} =0, \; если \; n \not = m .
$$
и
$$
  \mathfrak{S} \{ Q_n ^2 \} =
  \mathfrak{S} \{ Q_n \lambda ^n \} =
  A_n \not =0.
$$
Следовательно, коэффициенты разложения любого многочлена по
этому базису
$$
  p=\sum _n c_n Q_n
$$
находятся по
{\bfseries формулам Фурье}
$$
  c_n = \frac{\mathfrak{S} \{ pQ_n \} }
  {A_n }.
$$
В частности,
$$
  \lambda Q_n =Q_{n+1}+b_n Q_n +a_n Q_{n-1}+
  \sum _{k=0}^{n-2} c_k Q_k .
$$
Поскольку
$ \deg \lambda Q_k \leq n-1, \; $
то
$$
  \mathfrak{S} \{ \lambda Q_n Q_k \} =0 \;
  \Rightarrow \; c_k =0.
$$
Далее
$$
  a_n =\frac{\mathfrak{S} \{ \lambda Q_n Q_{n-1} \} }
  {A_{n-1}} =
  \frac{\mathfrak{S} \{ Q_n \lambda ^n \} }{A_{n-1}}=
  \frac{A_n }{A_{n-1}}
$$
и
$$
  b_n =\frac{\mathfrak{S} \{ \lambda Q_n ^2 \} }
  {A_n }.
$$
$ \triangle $
                         %%%%%%%%%%%%   Замечание 2   %%%%%%%%%%%%%%%%%%%%%%
\begin{Rem}
{\bfseries Многочлены второго рода}
$ P_n $
и
{\bfseries функции второго рода}
$ R_n $
удовлетворяют тому же рекуррентному соотношению
$ (\ast ) , $
но с другими начальными условиями:
$$
  P_0 (\lambda )=0, \quad
  P_1 (\lambda )=s_0 ,
$$
$$
  R_0 (\lambda )=f( \lambda ), \quad
  R_1 (\lambda )=Q_1 (\lambda )f(\lambda )-P_1 (\lambda ).
$$
\end{Rem}
$ \; $
\\

Действительно, начальные условия проверяются непосредственно.
Далее,
$$
  P_{n+1}(z)=\mathfrak{S}_{\lambda} \biggl \{
  \frac{Q_{n+1}(z)-Q_{n+1}(\lambda )}{z-\lambda}
  \biggr \}=
$$
$$
  =\mathfrak{S}_{\lambda} \biggl \{
  \frac{1}{z-\lambda} \biggl [ \biggl (
  (z-b_n )Q_n (z)-a_n Q_{n-1}(z) \biggr ) -
  \biggl ( (\lambda -b_n ) Q_n (\lambda ) -
  a_n Q_{n-1} (\lambda ) \biggr ) \biggr ] \biggr \} =
$$
$$
  =\mathfrak{S}_{\lambda} \biggl \{
  \frac{(z-\lambda )Q_n (\lambda ) }{z-\lambda} \biggr \} +
  \mathfrak{S}_{\lambda} \biggl \{
  \frac{z(Q_n (z)-Q_n (\lambda ))}{z-\lambda } \biggr \} -
$$
$$
  -b_n \mathfrak{S}_{\lambda} \biggl \{
  \frac{Q_n (z) - Q_n (\lambda )}{z-\lambda} \biggr \} -
  a_n \mathfrak{S}_{\lambda} \biggl \{
  \frac{Q_{n-1}(z)-Q_{n-1}(\lambda )}{z-\lambda} \biggr \}=
$$
$$
  = 0 + zP_n (z) - b_n P_n (z) -a_n P_{n-1}(z).
$$
Последовательность
$ R_n $
это линейная комбинация последовательностей
$ Q_n \; $
и
$ \; P_n , $
следовательно, она удовлетворяет тому же рекуррентному соотношению.
\\
$ \; $
                           %%%%%%%%%%%%   Замечание 3   %%%%%%%%%%%%%%%%%%%
\begin{Rem}
Если ввести величины
$$
  Q_{-1}=0, \quad P_{-1}=1, \quad R_{-1}=-1
$$
и
$$
  b_0 =s_1 /s_0 , \quad a_0 =-s_0 ,
$$
то в рекуррентных формулах можно брать
$ n=0, \; 1, \; 2,... $
\end{Rem}
$ \; $
\\

Запишем рекуррентные соотношения в матричной форме.
Обозначим
\begin{equation*}
  \mathsf{Q}=
    \begin{bmatrix}
	  Q_0 \\
	  Q_1 \\
	  \dots
	\end{bmatrix}
\end{equation*}
столбец ортогональных многочленов. Рассмотрим матрицу
\begin{equation*}
  \mathsf{A}=
    \begin{pmatrix}
	  b_0 & 1 & \; & \; & \; & \; \\
	  a_1 & b_1 & 1 & \; & \; & \; \\
	  \; & a_2 & b_2 & 1 & \; & \; \\
	  \; & \; & \hdotsfor{3} & \; \\
	  \; & \; & \; & \hdotsfor{3}
	\end{pmatrix}
  \qquad (J)
\end{equation*}
Тогда
$$
  \mathsf{A} \mathsf{Q} = \lambda \mathsf{Q},
$$
т.е.
$ \mathsf{Q} - $
собственный вектор матрицы
$ \mathsf{A}. $
                          %%%%%%%%%%%%%%%   Определение 4   %%%%%%%%%%%%%%%%
\begin{Def}
Матрицы вида
$ (J) , $
где
$ b_0 , \; b_1 , \; b_2 ,... - \; $
произвольные комплексные числа, и
$ a_1 , \; a_2 ,... - \; $
произвольные отличные от нуля комплексные числа,
будем называть\\
{\bfseries матрицами Якоби}.
\end{Def}
\newpage
                    %%%%%%%%%%%%%%%%%%%%%%%%%%%%%%%%%%%%%%%%%%%%%%%%%%%%%%%%
					%%%%%%%%   1.4   Теорема Чебышева-Фавара   %%%%%%%%%%%%%
					%%%%%%%%%%%%%%%%%%%%%%%%%%%%%%%%%%%%%%%%%%%%%%%%%%%%%%%%
\subsection{ Теорема Чебышева-Фавара }
$ \; $
\\

Обозначим
$ \mathbb{S} $
множество всех несингулярных моментных последовательностей
$ \mathbf{s}, $
нормированных условием
$ s_0 =1 , $
и
$ \mathbb{J} - $
множество всех матриц Якоби
$ \mathsf{A} . $
В параграфе 1.3 мы построили отображение
$$
  \phi : \mathbb{S} \longrightarrow \mathbb{J}
$$
                      %%%%%%%%%%%%%%%   Теорема 1   %%%%%%%%%%%%%%%%%%%%%%%
\begin{The}
{\bfseries (Чебышев-Фавар)}
Отображение
$ \phi $
взаимно-однозначное.
\end{The}
{\Large Доказательство.}
Решая рекуррентные соотношения\\
$ \mathsf{A} \mathsf{Q} = \lambda \mathsf{Q} $
с начальным условием
$ Q_0 =1 , $
получим последовательность многочленов
$ Q_n $
степени
$ n $
с единичным старшим коэффициентом.
Это базис пространства
$ \mathbb{C} [ \lambda ] . $
Определим линейный функционал
$ \mathfrak{S} $
на этих базисных векторах следующим образом
\begin{equation*}
  \mathfrak{S} \{ Q_n \} =
    \begin{cases}
	  1, \; если \; n=0, \\
	  0, \; если \; n \geq 1.
	\end{cases}
\end{equation*}
Положим
$$
  s_n = \mathfrak{S} \{ \lambda ^n \} ,
  \quad n \in \mathbb{Z}_+ .
$$
Тогда
$ \mathbf{s} = \{ s_n \} $
и будет искомой последовательностью.
Действительно, проверим соотношения ортогональности для
многочленов
$ Q_n . $
Пусть
$ m<n , $
тогда из рекуррентных соотношений следует, что
$$
  \lambda ^m Q_n (\lambda ) \in
  <Q_{n+m},...,Q_{n-m}> \Rightarrow
  \mathfrak{S} \{ \lambda ^m Q_n (\lambda ) \} =0.
$$
Здесь
$ < \cdot > \; - \; $
линейная оболочка.\\
Докажем по индукции, что последовательность
$ \mathbf{s} $
несингулярна.\\
База индукции:
$ H_0 =1, \; H_1 =1. $
\\
Шаг индукции. Пусть
$ H_0 \not = 0,..., H_n \not = 0. $
Тогда
\begin{equation*}
  Q_n =\frac{1}{H_n }
    \begin{vmatrix}
	  s_0 & \dots & s_n \\
	  \hdotsfor{3}\\
	  s_{n-1} & \dots & s_{2n-1}\\
	  1 & \dots & \lambda ^n
	\end{vmatrix}
  \Rightarrow
  \mathfrak{S} \{ Q_n (\lambda ) \lambda ^n \} =
  \frac{H_{n+1}}{H_n }.
\end{equation*}
Но из рекуррентных формул следует, что
$$
  \lambda ^n Q_n (\lambda ) =
  a_1 ...a_n Q_0 (\lambda ) +...
  \Rightarrow
  \mathfrak{S} \{ \lambda ^n Q_n (\lambda ) \} =
  a_1 ... a_n \not = 0
  \Rightarrow H_{n+1} \not =0.
$$    	      	  					    					
$ \triangle $
\newpage
                         %%%%%%%%%%%%%%%%%%%%%%%%%%%%%%%%%%%%%%%%%%%%%%%%%%%
						 %%%%%%   1.5   Непрерывные дроби   %%%%%%%%%%%%%%%%
						 %%%%%%%%%%%%%%%%%%%%%%%%%%%%%%%%%%%%%%%%%%%%%%%%%%%
\subsection{Непрерывные дроби}
Рассмотрим
{\bfseries непрерывную дробь}
\begin{equation*}
  \Pi _{n+1} =
    \cfrac{A_0 }{B_0 +
	 \cfrac{A_1 }{B_1 + \dotsb +
	  \cfrac{A_n }{B_n
	   }}}
   = \frac{ p_{n+1}}{q_{n+1}},
\end{equation*}
где
$ p_{n+1} \; $
и
$ \; q_{n+1} - \; $
многочлены от параметров дроби.
                  %%%%%%%%%%%%%%%%%   Утверждение   7     %%%%%%%%%%%%%%%%%%
\begin{Sta}
Многочлены
$ p_n $
и
$ q_n $
удовлетворяют трехчленному рекуррентному соотношению
\begin{equation*}
  \begin{cases}
    q_{n+1}=B_n q_n + A_n q_{n-1}\\
	p_{n+1}=B_n p_n + A_n p_{n-1}
  \end{cases}
  \qquad ( n=1, \; 2, \; 3,...)
\end{equation*}
с начальными условиями
$$
  p_0 =0 \quad \quad p_1 =A_0
$$
$$
  q_0 =1 \quad \quad q_1 =B_0
$$
\end{Sta}
{\Large Доказательство.}
Начальные условия выполнены. Далее рассуждаем по индукции.\\
База индукции. Имеем
$$
  \Pi _2 = \frac{A_0}{B_0 +\frac{A_1}{B_1}}=
  \frac{A_0 B_1 }{B_0 B_1 + A_1 }.
$$
По рекуррентным формулам
$$
  q_2 =B_1 q_1 +A_1 q_0 =B_1 B_0 +A_1
$$
$$
  p_2 =B_1 p_1 +A_1 p_0 =B_1 A_0
$$
Что требовалось.
\\
Шаг индукции. Будем считать, что дробь
$ \Pi _{n+1} $
имеет
$ n $
этажей. Тогда, используя два раза предположение индукции, получим
$$
  \frac{p_{n+1}}{q_{n+1}}=
  \frac{p_n}{q_n}(A_0 ,...,A_{n-1}; B_0 ,..., B_{n-2},
  B_{n-1}+ \frac{A_n}{B_n})=
$$
$$
  =\frac{\bigl ( B_{n-1} + \frac{A_n}{B_n} \bigr )
  p_{n-1} + A_{n-1} p_{n-2} }
  {\bigl ( B_{n-1} + \frac{A_n}{B_n} \bigr )
  q_{n-1} + A_{n-1} q_{n-2} } =
$$
$$
  = \frac{B_n ( B_{n-1} p_{n-1}  + A_{n-1} p_{n-2} )
  +A_n p_{n-1} }
  {B_n ( B_{n-1} q_{n-1} + A_{n-1} q_{n-2} )
  +A_n q_{n-1} }=
  \frac{B_n p_n + A_n p_{n-1} }
  {B_n q_n + A_n q_{n-1} }.
$$
$ \triangle $
\\

Сравнивая полученные рекуррентные соотношения с теми,
что были доказаны для многочленов
$ Q_n \; $
и
$ \; P_n ,$
видим, что
$$
  B_n =z-b_n , \quad n=0, \; 1, \; 2,...
$$
$$
  A_n =-a_n , \quad n=1, \; 2, ...
$$
$$
  A_0 =1
$$
Таким образом приходим к следующему утверждению.
                            %%%%%%%%%%   Утверждение 8   %%%%%%%%%%%%%%%%%%%
\begin{Sta}
Аппроксимации Паде имеют следующее разложение в непрерывную дробь
\begin{equation*}
  \pi _{n+1} =
   \cfrac{1}{z-b_0 -
    \cfrac{a_1 }{z-b_1 -
	 \cfrac{a_2}{z-b_2 - \dotsb -
	  \cfrac{a_n}{z-b_n }}}}.
\end{equation*}
\end{Sta}
                             %%%%%%%%%%%%%   Следствие 1   %%%%%%%%%%%%%%%%%
\begin{Cor}
Формальный степенной ряд
$ f(z) $
имеет следующее разложение в непрерывную дробь
\begin{equation*}
  f(z)=
    \cfrac{1}{z-b_0 -
	 \cfrac{a_1 }{z-b_1 - \dotsb }},
\end{equation*}
которая сходится к
$ f(z) $
покоэффициентно (по неархимедовой норме поля формальных
степенных рядов).
\end{Cor}
                        %%%%%%%%%%%%%%%%%   Замечание   4     %%%%%%%%%%%%%
\begin{Rem}
Непрерывная дробь решает
{\bfseries прямую спектральную задачу}
для матрицы Якоби. А именно, по элементам матрицы
восстанавливает последовательность степенных моментов, т.е.
{\bfseries спектральные данные}
задачи.
\end{Rem}
$ \; $
\\

Остановимся еще на одном подходе к изучению непрерывных
дробей, основанном на композициях дробно-линейных отображений.
Рассмотрим ДЛО
$$
  S_{m+1}(Z)=\frac{A_m}{B_m +Z}.
$$
Тогда по определению непрерывной дроби имеем
$$
  \Pi _{n+1}= S_1 \circ S_2 \circ ... \circ S_{n+1} (0).
$$
ДЛО
$ S_{m+1} $
соответствует матрица
\begin{equation*}
  \mathsf{S}_{m+1}=
    \begin{pmatrix}
	  0 & A_m \\
	  1 & B_m
	\end{pmatrix}
\end{equation*}
Композиции ДЛО соответствует произведение матриц
$$
  \mathsf{M}_{n+1}=\mathsf{S}_1 \mathsf{S}_2 ...\mathsf{S}_{n+1}
$$
Следовательно,
\begin{equation*}
  \mathsf{M}_{n+1}=
    \begin{pmatrix}
	  p_n & p_{n+1}\\
	  q_n & q_{n+1}
	\end{pmatrix}
\end{equation*}
Отсюда легко получаются рекуррентные формулы:
\begin{equation*}
  \mathsf{M}_{n+1} =\mathsf{M}_n \mathsf{S}_{n+1}=
    \begin{pmatrix}
	  p_{n-1} & p_n \\
	  q_{n-1} & q_n
	\end{pmatrix}
	\begin{pmatrix}
	  0 & A_n \\
	  1 & B_n
	\end{pmatrix}
	=
	  \begin{pmatrix}
	    p_n & p_{n-1}A_n + p_n B_n \\
		q_n & q_{n-1}A_n + q_n B_n
	  \end{pmatrix}
\end{equation*}
\newpage
				 %%%%%%%%%%%%%%%%%%%%%%%%%%%%%%%%%%%%%%%%%%%%%%%%%%%%%%%%%%
                 %%%%%%%%%%%%%%%%%   1.6   Резольвента   %%%%%%%%%%%%%%%%%%
	             %%%%%%%%%%%%%%%%%%%%%%%%%%%%%%%%%%%%%%%%%%%%%%%%%%%%%%%%%%
\subsection{Резольвента}
$ \; $
\\

Пусть
$ \mathsf{A} - $
матрица Якоби, т.е.
\begin{equation*}
  \mathsf{A}=
    \begin{pmatrix}
	  b_0 & 1 & \; & \; & \; & \; \\
	  a_1 & b_1 & 1 & \; & \; & \; \\
	  \; & a_2 & b_2 & 1 & \; & \; \\
	  \; & \; & \hdotsfor{3}  & \; \\
	  \; & \; & \; & \hdotsfor{3}
	\end{pmatrix}
\end{equation*}
где
$ b_0 , \; b_1 , \; b_2 ,... - $
произвольные комплексные числа, и
$ a_1 , \; a_2 ,... - $
произвольные отличные от нуля комплексные числа.
Вычислим резольвенту этой матрицы.
                             %%%%%%%%%%%%%%   Определение 5   %%%%%%%%%%%%%
\begin{Def}
{\bfseries Резольвентой} матрицы
$ \mathsf{A} $
называется матрица
$$
  \mathsf{R}_z =(z \mathsf{E} - \mathsf{A} ) ^{-1},
$$
зависящая от параметра
$ z . $
Здесь
$ \mathsf{E} - $
бесконечная единичная матрица.
\end{Def}
Другими словами, матрица
$ \mathsf{R}_z $
должна удовлетворять двум матричным уравнениям
$$
  \mathsf{R}_z \cdot (z \mathsf{E} - \mathsf{A} ) =
  (z \mathsf{E} - \mathsf{A} ) \cdot \mathsf{R}_z =
  \mathsf{E} .
$$
Мы докажем, что эти уравнения имеют единственное решение и
найдем его в явном виде.
\\

Прежде всего отметим, что резольвенту легко разложить в
формальный степенной ряд:
$$
  \mathsf{R}_z = \sum _{n=0}^{\infty}
  \frac{ \mathsf{A}^n }{z^{n+1} }.
$$
Действительно,
$$
  (z \mathsf{E} - \mathsf{A} ) \cdot
  \biggl ( \sum _{n=0}^{\infty}
  \frac{\mathsf{A}^n }{z^{n+1}} \biggr ) =
  \mathsf{E}+\sum _{n=0}^{\infty}
  \frac{\mathsf{A}^{n+1}}{z^{n+1}}-
  \sum _{n=0}^{\infty} \frac{ \mathsf{A}^{n+1}}
  {z^{n+1}}= \mathsf{E}.
$$
Тот же ответ получим перемножая матрицы в другом порядке.\\
Таким образом, элементы матрицы
$ \mathsf{R}_z $
суть формальные степенные ряды.
\\

Заметим, что матрица
$ \mathsf{A} $
определяет некоторый линейный оператор. А именно, обозначим
$ \mathtt{D} $
линейное пространство, состоящее из бесконечных финитных
последовательностей (вектор-столбцов) комплексных чисел
\begin{equation*}
  \mathtt{x}=
  \begin{bmatrix}
   x_0 \\
   x_1 \\
   x_2 \\
   \dots
 \end{bmatrix}
 ,
\end{equation*}
(лишь конечное число членов последовательности отлично
от нуля). Определим в
$ \mathtt{D} $
стандартное скалярное произведение по формуле
$$
  (\mathtt{x},\mathtt{y})=\sum _{n=0}^{\infty} x_n \bar y_n .
$$
Тогда
$ \mathtt{D} $
будет предгильбертовым пространством. Оно не полное. Его
пополнением будет стандартное гильбертово пространство
$ \mathtt{H} = l_2 . $
Обозначим
$ \{ \mathtt{e}_n \} _{n=0}^{\infty} $
канонический базис этого пространства
(на $ n- $ м месте стоит $ 1 ,$ остальные элементы - нули).
Это алгебраический базис в линейном пространстве
$ \mathtt{D} $
и полная ортонормированная система в
$ \mathtt{H} . $
\\
Определено произведение матрицы
$ \mathsf{A} $
на столбец
$ \mathtt{x} : $
$$
  \mathtt{y} = \mathsf{A} \mathtt{x} .
$$
При этом
$ \mathtt{y} \in \mathtt{H} $
(фактически
$ \mathtt{y} \in \mathtt{D} ). $
Таким образом, матрицу
$ \mathsf{A} $
можно рассматривать как всюду плотно определенный оператор в
$ \mathtt{H} : $
$$
  \mathsf{A} : \mathtt{D} \longrightarrow \mathtt{H}.
$$
(Если
$ \mathsf{A} $
ограниченный оператор, то он по непрерывности продолжается на все
пространство
$ \mathtt{H} . ) $
\\
Следует различать понятие резольвенты матрица и резольвенты
оператора. В первом случае мы имеем матрицу, элементы которой суть
формальные степенные ряды, а во втором -- матрицу ограниченного
оператора.
\\

Найдем матричный элемент
$$
  (\mathsf{R}_z )_{0,0} =
  (\mathsf{R}_z \mathtt{e}_0 , \mathtt{e}_0 ).
$$
Заметим, что числа
$ s_n $
суть
{\bfseries моменты оператора}
$ \mathsf{A} , $
т.е.
$$
  s_n =(\mathsf{A}^n \mathtt{e}_0 , \mathtt{e}_0 ).
$$
Действительно,
$$
  \mathsf{A} \mathsf{Q}=\lambda \mathsf{Q} \Rightarrow
  \mathsf{A}^n \mathsf{Q} =\lambda ^n \mathsf{Q} \Rightarrow
  \lambda ^n = \sum _{k=0}^{\infty}
  ( \mathsf{A}^n )_{0,k} Q_k \Rightarrow
  s_n =\mathfrak{S} \{ \lambda ^n \} =
  (\mathsf{A} ^n )_{0,0},
$$
поскольку
\begin{equation*}
  \mathfrak{S} \{ Q_k \} =
    \begin{cases}
	  1, \; если \; k=0,\\
	  0, \; если \; k \geq 1.
	\end{cases}
\end{equation*}
Таким образом,
$$
  (\mathsf{R}_z \mathtt{e}_0 , \mathtt{e}_0 )=
  \sum _{n=0}^{\infty} \frac
  {( \mathsf{A}^n \mathtt{e}_0 , \mathtt{e}_0 )}
  {z^{n+1}}=
  \sum _{n=0}^{\infty}
  \frac{s_n}{z^{n+1}}=f(z).
$$
Учитывая эту связь, дадим следующее
                          %%%%%%%%%%%%   Определение  6     %%%%%%%%%%%%%%%%%
\begin{Def}
Формальный степенной ряд
$ f(z) $
называется
{\bfseries резольвентной функцией оператора}
$ \mathsf{A}. $
\end{Def}
$ \; $
\\

Обозначим 1-й столбец матрицы
$ \mathsf{R}_z $
\begin{equation*}
  \mathtt{y}=
    \begin{bmatrix}
	  y_0 \\
	  y_1 \\
	  y_2 \\
	  \dots
	\end{bmatrix}
\end{equation*}
Тогда
$$
  (z \mathsf{E} - \mathsf{A} ) \mathtt{y}= \mathtt{e}_0 .
$$
Другими словами, величины
$ y_n $
удовлетворяют неоднородному рекуррентному соотношению
\begin{equation*}
  \begin{cases}
    b_0 y_0 + y_1 =z y_0 -1 \\
	a_1 y_0 + b_1 y_1 + y_2 =z y_1 \\
	...................................\\
	a_n y_{n-1}+b_n y_n +y_{n+1}=zy_n \\
	...................................
  \end{cases}
\end{equation*}
Общее решение неоднородного уравнения равно сумме
частного решения неоднородного уравнения и
общего решения соответствующего однородного уравнения.
Общее решение однородного уравнения есть
$$
  \{ C(z) Q_n (z) \} _{n=0}^{\infty},
$$
где
$ C - $
произвольная постоянная
$ ( C $
не зависит от
$ n ,$
но зависит от параметра
$ z ). $
Частным решением неоднородного уравнения служит последовательность
$$
 \{ - P_n (z) \} _{n=0}^{\infty}.
$$
Действительно, для нее выполнены все уравнения с
$ n \geq 1. $
Проверим уравнение с номером
$ n=0: $
$$
  (z-b_0 )y_0 -y_1 =(z-b_0 ) \cdot 0 + P_1 =1.
$$
Итак, общее решение неоднородного уравнения имеет вид
$$
  y_n =CQ_n -P_n , \quad n=0, \; 1, \; 2,...
$$
В частности,
$ y_0 =C. $
Но мы имеем начальное условие
$ y_0 =f . $
Следовательно,
$$
  y_n =Q_n f - P_n = R_n ,
  \quad n=0, \; 1, \; 2,...
$$
\\

Обозначим теперь 1-ю строку матрицы
$ \mathsf{R}_z $
$$
  \mathtt{y}^{\prime}=(y_0 , \; y_1 , \; y_2 ,...).
$$
Тогда
$$
  \mathtt{y}^{\prime} \cdot
  (z \mathsf{E}-\mathsf{A})= \mathtt{e}_0 ^{\prime}
$$
или
$$
  (z\mathsf{E}-\mathsf{A}^{\prime})\cdot \mathtt{y} =\mathtt{e}_0 .
$$
Здесь 'штрих' обозначает транспонирование матриц.
Мы пришли к задаче, аналогичной предыдущей, но с транспонированной
матрицей
\begin{equation*}
  \mathsf{A}^{\prime}=
    \begin{pmatrix}
	  b_0 & a_1 & \; & \; & \; \\
	  1 & b_1 & a_2 & \; & \; \\
	  \; & 1 & b_2 & a_3 & \; \\
	  \; & \; & \hdotsfor{3}
	\end{pmatrix}
\end{equation*}
Поэтому как и раньше получим
$$
  y_n = Q_n ^{\ast} f -P_n ^{\ast} =R_n ^{\ast},
$$
где многочлены
$ Q_n ^{\ast} $
удовлетворяют однородному рекуррентному соотношению
$$
  Q_{n-1}^{\ast} +b_n Q_n ^{\ast} + a_{n+1} Q_{n+1}^{\ast}=
  zQ_n ^{\ast} , \quad n=0, \; 1, \; 2,...
$$
с начальными условиями
$$
  Q_{-1}^{\ast}=0, \quad
  Q_0 ^{\ast}=1,
$$
а
$ P_n ^{\ast} - $
соответствующие многочлены второго рода. Заметим, что
$ Q_n ^{\ast} - $
многочлен степени
$ n $
со старшим коэффициентом
$ \frac{1}{a_1 ...a_n } . $
Сделаем перенормировку
$$
  u_n =a_1 ... a_n \; Q_n ^{\ast} .
$$
Получим
$$
  \frac{u_{n-1}}{a_1 ...a_{n-1}}+
  b_n \cdot \frac{u_n}{a_1 ...a_n }+
  a_{n+1} \cdot \frac{u_{n+1}}{a_1 ...a_{n+1}}=
  z \cdot \frac{u_n}{a_1 ...a_n }
$$
или
$$
  a_n u_{n-1}+b_n u_n + u_{n+1} =z u_n ,
  \quad n=0, \; 1, \; 2,...
$$
$$
  u_{-1}=0, \quad u_0 =1.
$$
Следовательно,
$ u_n =Q_n . $
\\

Будем использовать обозначение
$ \{ \cdot \} $
для дробной части произвольного степенного ряда:
$$
  \biggl \{ \sum _{n=-\infty}^{+\infty}
  \frac{ F_n }{z^{n+1}} \biggr \} =
  \sum _{n=0}^{\infty} \frac{F_n }{z^{n+1}}.
$$
Теперь мы готовы сформулировать и доказать следующий
результат.
                %%%%%%%%%%%%%%   Теорема 2   %%%%%%%%%%%%%%%%%%%%%%%%%%%%%%
\begin{The}
Матрица резольвенты имеет вид
$$
  (\mathsf{R}_z )_{j,k}=
  \{ Q_j Q_k ^{\ast} f \} ,
  \quad j,k=0, \; 1, \; 2,...
$$
\end{The}
{\Large Доказательство.}
Заметим, что
\begin{equation*}
  \{ Q_j Q_k ^{\ast} f \} =
    \begin{cases}
	  Q_j R_k ^{\ast}, \; если \; j<k,\\
	  Q_k ^{\ast} R_j , \; если \; j \geq k.
	\end{cases}
\end{equation*}
Проверим, что выполняется равенство
$$
  (z \mathsf{E} - \mathsf{A} ) \cdot \mathsf{R}_z = \mathsf{E}.
$$
Обозначим
$ n- $
й столбец матрицы
$ \mathsf{R}_z $
\begin{equation*}
  \mathtt{y}=
    \begin{bmatrix}
	  Q_0 R_n ^{\ast}\\
	  \dots \\
	  Q_{n-1} R_n ^{\ast} \\
	  Q_n ^{\ast} R_n \\
	  Q_n ^{\ast} R_{n+1} \\
	  \dots
	\end{bmatrix}
\end{equation*}
Мы должны проверить рекуррентные соотношения
$$
  a_m y_{m-1} + b_m y_m +y_{m+1} =
  z y_m - \delta _{n,m},
  \quad m=0, \; 1, \; 2,...
$$
с начальными условиями
$$
  y_{-1}=0, \quad y_0 = R_n ^{\ast}
$$
Пусть
$ m<n, $
тогда приходим к соотношениям
$$
  a_m Q_{m-1} R_n ^{\ast} +
  b_m Q_m R_n ^{\ast} +
  Q_{m+1} R_n ^{\ast} =
  z Q_m R_n ^{\ast}.
$$
Эти соотношения выполняются, поскольку они верны (после деления на
$ R_n ^{\ast}) $
для многочленов
$ Q_m . $
\\
Пусть
$ m>n, $
тогда приходим к соотношениям
$$
  a_m Q_n ^{\ast} R_{m-1}+
  b_m Q_n ^{\ast} R_m +
  Q_n ^{\ast} R_{m+1}=
  z Q_n ^{\ast} R_m ,
  \quad m=n+1, \; n+2,...
$$
Эти соотношения также выполняются, поскольку они верны
(после деления на
$ Q_n ^{\ast} )$
для функций второго рода
$ R_m . $
\\
Рассмотрим
$ n-$
е уравнение
$$
  a_n Q_{n-1} R_n ^{\ast} +
  b_n Q_n ^{\ast} R_n +
  Q_n ^{\ast} R_{n+1} =
  z Q_n ^{\ast} R_n -1
$$
или
$$
  a_n Q_{n-1} R_n ^{\ast} +
  b_n Q_n R_n ^{\ast} +
  Q_{n+1} R_n ^{\ast} +
  Q_n ^{\ast} R_{n+1} -
  Q_{n+1} R_n ^{\ast} =
  z Q_n R_n ^{\ast} -1
$$
Ниже будет доказано тождество
$$
  Q_{n+1} R_n - Q_n R_{n+1} =a_1 ...a_n ,
$$
после чего получаем верное соотношение (оно выполняется для
многочленов
$ Q_n ). $
\\
Аналогичным образом проверяется равенство
$$
  \mathsf{R}_z \cdot
  (z \mathsf{E} - \mathsf{A} )= \mathsf{E} .
$$
$ \triangle $
                  %%%%%%%%%%%%%%%%   Лемма 1   %%%%%%%%%%%%%%%%%%%%%%%%%%%%%
\begin{Lem}
Справедливо
{\bfseries тождество Лагранжа}
$$
  Q_{n+1} R_n -Q_n R_{n+1} =
  a_1 ...a_n , \quad n=0, \; 1, \; 2,...
$$
\end{Lem}
{\Large Доказательство.}
При
$ n=0 $
проверяем базу индукции:
$$
  Q_1 R_0 -Q_0 R_1 = Q_0 P_1 -Q_1 P_0 =1 \cdot 1 -
  Q_1 \cdot 0 =1.
$$
Проведем шаг индукции:
$$
  Q_{n+1} R_n -Q_n R_{n+1} =
  \bigl ( (z-b_n )Q_n -a_n Q_{n-1} \bigr ) R_n -
  Q_n \bigl ( (z-b_n )R_n -a_n R_{n-1} \bigr ) =
$$
$$
  =a_n ( Q_n R_{n-1} - Q_{n-1} R_n ) .
$$
$ \triangle $
\newpage
                    %%%%%%%%%%%%%%%%%%%%%%%%%%%%%%%%%%%%%%%%%%%%%%%%%%%%%%%
					%%%%%%   1.7   Пример ограниченного оператора   %%%%%%%
					%%%%%%%%%%%%%%%%%%%%%%%%%%%%%%%%%%%%%%%%%%%%%%%%%%%%%%%
\subsection{Пример ограниченного оператора}
$ \; $
\\
                     %%%%%%%%%%%%   Пример 1    %%%%%%%%%%%%%%%%%%%%%%%%%%%%
\begin{Exa}
Решим прямую спектральную задачу для оператора с постоянными
коэффициентами
\begin{equation*}
  \mathsf{A}=
    \begin{pmatrix}
	  0 & 1 & \; & \; & \; \\
	  a & 0 & 1 & \; & \; \\
	  \; & a & 0 & 1 & \; \\
	  \; & \; & \hdotsfor{3}
	\end{pmatrix}
\end{equation*}
\end{Exa}
Для резольвентной функции
\begin{equation*}
  f(z)= \sum _{n=0}^{\infty}
  \frac{s_n}{z^{n+1}}
\end{equation*}
имеем следующее разложение в непрерывную дробь
\begin{equation*}
  f(z)=
    \cfrac{1}{z-
	 \cfrac{a}{z-
	  \cfrac{a}{z- \dotsb
	   }}}
\end{equation*}
Отсюда
$$
  f=\frac{1}{z-af}
$$
или
$$
  af^2 -zf+1=0,
$$
и тем самым
$$
  f(z)=\frac{z-\sqrt{z^2 -4a}}{2a}.
$$
В окрестности бесконечности выделяем однозначную ветвь
корня условием
$$
  \sqrt{z^2 -4a} \sim z, \quad z \rightarrow \infty ,
$$
для того, чтобы было
$$
  f(z) \sim \frac{1}{z}, \quad z \rightarrow \infty .
$$
Осталось разложить функцию
$ f(z) $
в ряд Лорана:
\begin{equation*}
  f(z)=\frac{1}{2a} \Biggl ( z-
  z \cdot \sqrt{1- \frac{4a}{z^2}} \Biggr ) =
  -\frac{z}{2a} \sum _{n=1}^{\infty}
  \binom{1/2}{n}\biggl (-\frac{4a}{z^2} \biggr ) ^n =
  \sum _{n=1}^{\infty} \frac{s_{2n-2}}{z^{2n-1}}.
\end{equation*}
Таким образом, моменты с нечетными номерами равны нулю:
$$
  s_1 =s_3 =s_5 =...=0,
$$
а моменты с четными номерами вычисляются по формуле:
$$
  s_{2n}=-\frac{1}{2a} \binom{1/2}{n+1}(-4a)^{n+1}=
  2(-4a)^n \frac{\frac{1}{2}(\frac{1}{2}-1)...(\frac{1}{2}-n)}
  {(n+1)!}=
$$
$$
  = (2a)^n \frac{(2n-1)!!}{(n+1)!}=
  (2a)^n \frac{(2n)!}{(2n)!!(n+1)!}.
$$
Окончательно
$$
  s_{2n}=\frac{1}{n+1} \binom{2n}{n} a^n ,
  \quad n=0, \; 1, \; 2,...
$$
\\

Найдем
{\bfseries спектр}
оператора
$ \mathsf{A} . $
Пусть
$ a - $
вещественное,
$ a>1. $
Выясним, для каких
$ z $
матрица
$ \mathsf{R}_z $
будет матрицей ограниченного оператора,
т.е. найдем
{\bfseries резольвентное множество}.
Поскольку столбцы матрицы суть образы базисных векторов, то
необходимым условием ограниченности будет принадлежность
0-го столбца
\begin{equation*}
  \begin{bmatrix}
    R_0 \\
	R_1 \\
	R_2 \\
	\dots
  \end{bmatrix}
\end{equation*}
пространству
$ l_2 . $
Выясним асимптотическое поведение величин
$ R_n . $
\\

Найдем общее решение рекуррентного соотношения
$$
  a y_{n-1} + y_{n+1}= z y_n ,
  \quad n=0, \; 1, \; 2,...
$$
Это уравнение с постоянными коэффициентами. Составим
характеристическое уравнение
$$
  p^2 -zp +a =0.
$$
Оно имеет два корня
$$
  p_{\pm}(z)=\frac{z \pm \sqrt{z^2 -4a}}{2}
$$
Пусть
$$
  z \in G = \bar{\mathbb{C}} \setminus
  [-2 \sqrt{a}, +2\sqrt{a}]
$$
и
$$
  \sqrt{z^2 -4a} \sim z \quad при \; z \rightarrow \infty .
$$
Тогда
$ p_{\pm} $
это функции, обратные к
{\bfseries функции Жуковского}
(с точностью до линейного преобразования)
$$
  z=p+\frac{a}{p}.
$$
Функция
$ p_- $
конформно отображает область
$ G $
на круг радиуса
$ \sqrt{a} ,$
функция
$ p_+ $
конформно отображает область
$ G $
на внешность этого круга. Функции связаны соотношением
$$
  p_- p_+ =a.
$$
Общее решение рекуррентного соотношения имеет вид
$$
  c_+ p_+ ^n + c_- p_- ^n ,
$$
где
$ c_+ $ и $ c_- - $
произвольные постоянные (они не зависят от
$ n ,$
но зависят от параметра
$ z ). $
Найдем, в частности, величины
$ R_n . $
Они удовлетворяют начальным условиям
$ R_{-1}=1/a , \; R_0 =f. $
Следовательно,
\begin{equation*}
  \begin{cases}
    \frac{c_+}{p_+}+\frac{c_-}{p_-} =\frac{1}{a}\\
	c_+ +c_- =f
  \end{cases}
  \quad \Rightarrow \quad
  \begin{cases}
    c_- =c := \frac{p_- f -1}{p_- -p_+}\\
	c_+ = \frac{p_+ f -1 }{p_+ - p_- } =0.
  \end{cases}
\end{equation*}
Итак,
$$
  R_n (z) = c(z) p_- (z)^n .
$$
Таким образом, необходимым условием ограниченности матрицы
$ \mathsf{R}_z $
будет условие
$ |p_- (z) | < 1 . $
Этому условию удовлетворяют точки, лежащие вне эллипса,
фокусы которого находятся в точках
$ \pm 2 \sqrt{a} , $
а полуоси равны
$ a+1 $ и $ a-1 . $
Точки, лежащие внутри и на границе эллипса, заведомо принадлежат
спектру. Докажем, что это и будет спектр, т.е. условие
$ |p_- |<1 $
будет не только необходимым, но и достаточным, для ограниченности
матрицы
$ \mathsf{R}_z .$
Для того, чтобы это доказать, оценим по модулю все элементы матрицы.
\\

Обозначим
$ p=|p_- (z) | . $
Тогда
$ 0<p<1 . $
Имеем
$ |R_n (z) | \leq cp^n , $
где
$ c - $
некоторая постоянная. Поскольку
$ |p_+ (z)| >1 , $
то
$ |Q_n | \leq cp_+ ^n =c( a/p )^n . $
Тогда
$$
  |Q_k ^{\ast} R_j | \leq c \biggl (\frac{1}{p} \biggr )^k p^j =cp^{j-k},
$$
$$
  |Q_j R_k ^{\ast} | \leq c\biggl (\frac{a}{p} \biggr )^j
   \biggl (\frac{p}{a}\biggr )^k =
  c\biggl (\frac{p}{a}\biggr )^{k-j}.
$$
Обозначим
$ q=p/a . $
Тогда
$ 0<q<p<1. $
Все элементы матрицы
$ \mathsf{R}_z $
по модулю мажорируются (с некоторой постоянной) элементами матрицы
\begin{equation*}
  \mathsf{B}=
    \begin{pmatrix}
	  1 & q & q^2 & \hdotsfor{4}\\
	  p & 1 & q & q^2 & \hdotsfor{3}\\
	  p^2 & p & 1 & q & q^2 & \hdotsfor{2}\\
	  p^3 & p^2 & p & 1 & q & q^2 & \dots \\
	  \hdotsfor{7}
	\end{pmatrix}
\end{equation*}
Докажем ограниченность этого оператора. Обозначим
$ f_0 , \; f_1 , \; f_2 ,... $
столбцы матрицы
$ \mathsf{B}. $
Это векторы из
$ l_2 . $
Пусть
$ \mathtt{x} - $
любой финитный вектор. Тогда
$$
  || \mathsf{B} \mathtt{x} ||^2 =
  || x_0 f_0 +x_1 f_1 +... ||^2 =
  \sum _{j,k=0}^{\infty}
  x_j x_k (f_j , f_k ) .
$$
Ниже мы получим оценки вида
$$
  |(f_j , f_{j+l})| \leq C_l ,
  \quad j,l=0, \; 1, \; 2,...
$$
Оценим сумму
$$
  | \sum _{j=0}^{\infty} x_j x_{j+l} (f_j , f_{j+l})|
  \leq C_l \sum _{j=0}^{\infty}
  \frac{x_j ^2 + x_{j+l}^2}{2}
  \leq C_l || \mathtt{x} ||^2 .
$$
Следовательно,
$$
  || \mathsf{B} \mathtt{x} ||^2 \leq
  || \mathtt{x} ||^2 ( C_0 +2C_1 +2C_2 +...).
$$
Если ряд
$$
  \sum _{l=1}^{\infty} C_l
$$
сходится, то оператор
$ \mathsf{B} $
ограничен.\\
Перемножим $ j- $ й и $ (j+l)- $ й столбцы:
\begin{equation*}
  \begin{bmatrix}
    q^j \\
	\dots \\
	1 \\
	p \\
	\dots \\
	p^l \\
	p^{l+1} \\
	\dots
  \end{bmatrix}
    \cdot
  \begin{bmatrix}
    q^{j+l} \\
	\dots \\
	q^l \\
	q^{l-1} \\
	\dots \\
	1 \\
	p \\
	\dots
  \end{bmatrix}
    \leq
  \begin{cases}
    \sum _{m=0}^j q^m q^{m+l} \leq
	q^l \sum_{m=0}^{\infty} q^{2m} \leq
	c q^l \leq cp^l \\
	+ \\
	 \sum _{m=0}^l q^m p^{l-m} =
	\frac{p^{l+1}-q^{l+1}}{p-q}
	\leq cp^l \\
	+ \\
	 \sum_{m=0}^{\infty} p^m p^{m+l} =
	p^l \sum_{m=0}^{\infty} p^{2m} =
	\frac{p^l}{1-p^2} \leq cp^l
  \end{cases}
    \leq Cp^l .
\end{equation*}
Итак,
$ C_l =Cp^l , $
и ряд
$$
  \sum_{l=1}^{\infty} p^l < +\infty
$$
сходится.
\\
$ \triangle $
               %%%%%%%%%%%%%%%%   Упражнение 1   %%%%%%%%%%%%%%%%%%%%%%%%%%
\begin{Exe}
Найти спектр при любом отличном от нуля комплексном
$ a . $
\end{Exe}
\newpage
                      %%%%%%%%%%%%%%%%%%%%%%%%%%%%%%%%%%%%%%%%%%%%%%%%%%%%%%
					  %%%%  1.8  Пример неограниченного оператора   %%%%%%%%
					  %%%%%%%%%%%%%%%%%%%%%%%%%%%%%%%%%%%%%%%%%%%%%%%%%%%%%%
\subsection{Пример неограниченного оператора}
$ \; $
\\
                       %%%%%%%%%%%%%%%   Пример 2   %%%%%%%%%%%%%%%%%%%%%%%
\begin{Exa}
Решим прямую спектральную задачу для оператора
\begin{equation*}
  \mathsf{A}=
    \begin{pmatrix}
	  0 & 1 & \; & \; & \;& \;  \\
	  1 & 0 & 1 & \; & \; & \; \\
	  \; & 2 & 0 & 1 & \; & \; \\
	  \; & \; & 3 & 0 & 1 & \; \\
	  \; & \; & \; & \hdotsfor{3}
	\end{pmatrix}
\end{equation*}
\end{Exa}
Таким образом ортогональные многочлены удовлетворяют
рекуррентным соотношениям
$$
  nQ_{n-1}+Q_{n+1}=\lambda Q_n ,
  \qquad n=0, \; 1, \; 2,...
$$
с начальными условиями
$ Q_{-1}=0, \; Q_0 =1. $
Будем изучать лишь многочлены с четными номерами. Для них имеем
$$
  \lambda ^2 Q_{2n}=Q_{2n+2}+(4n+1)Q_{2n}+2n(2n-1)Q_{2n-2},
  \qquad n=0, \; 1, \; 2,...
$$
Введем новые многочлены
$ q_n $
по формуле
$$
  Q_{2n}=n!2^n q_n .
$$
Тогда многочлены
$ q_n $
будут удовлетворять следующим рекуррентным соотношениям
$$
  \lambda ^2 q_n =2(n+1)q_{n+1}+(4n+1)q_n +(2n-1)q_{n-1},
  \qquad n=0, \; 1, \; 2,...
$$
и начальным условиям
$$
  q_{-1}=0, \quad q_0 =1.
$$
Вычислим
{\bfseries производящую функцию}
$$
  \Phi (\lambda , w)=\sum _{n=0}^{\infty}
  q_n (\lambda )w^n .
$$
Справедливы формулы
$$
  1) \quad 2\sum _{n=0}^{\infty} (n+1)q_{n+1}w^n =
  2 \frac{\partial}{\partial w} \sum _{n=0}^{\infty}
  q_{n+1}w^{n+1}=2\frac{\partial}{\partial w}
  (\Phi -1)=2\Phi ^{\prime}.
$$
$$
  2) \quad 4\sum _{n=0}^{\infty}(n+\frac{1}{4})q_n w^n =
  4w^{\frac{3}{4}}\frac{\partial}{\partial w}
  w^{\frac{1}{4}}\sum _{n=0}^{\infty}q_n w^n =
  \Phi +4w\Phi ^{\prime}.
$$
$$
  3) \quad \sum _{n=1}^{\infty}(2n-1)q_{n-1}w^n =
  2\sum _{n=0}^{\infty}(n+\frac{1}{2})q_n w^{n+1}=
$$
$$
  =2w^{\frac{3}{2}} \frac{\partial}{\partial w}
  w^{\frac{1}{2}} \sum _{n=0}^{\infty}q_n w^n =
  w\Phi +2w^2 \Phi ^{\prime}.
$$
Умножая рекуррентные соотношения на
$ w^n , $
суммируя полученные равенства по всем
$ n \in \mathbb{Z}_+ $
и учитывая формулы 1), 2) и 3), получим, что функция
$ \Phi $
удовлетворяет следующему линейному однородному
дифференциальному уравнению первого порядка
$$
  (2w^2 +4w +2)\Phi ^{\prime}=(\lambda ^2 -1 -w)\Phi ,
$$
где "штрих" обозначает производную по
$ w. $
\\
Решим уравнение
{\bfseries методом разделения переменных}
$$
  \int \frac{d\Phi}{\Phi}=
  \int \frac{\lambda ^2 -(w+1)}{2(w+1)^2}dw
$$
или
$$
  \ln \Phi =-\frac{\lambda ^2}{2}\frac{1}{w+1}
  -\frac{1}{2}\ln (w+1) + \mathrm{const}.
$$
Учитывая начальное условие
$$
  \Phi (\lambda , w) \biggr | _{w=0}=1,
$$
окончательно получим явный вид производящей функции
$$
  \Phi (\lambda ,w)=\frac{1}{\sqrt{1+w}}
  \exp \Biggl \{ \frac{w}{1+w}\frac{\lambda ^2}{2}
  \Biggr \} ,
$$
где берется главная ветвь корня. Заметим, что функция
$ \Phi $
голоморфна в единичном круге и имеет на его границе единственную
особую точку
$ w=-1. $
\\

Пусть
$ P_n - $
многочлены второго рода и
$$
  P_{2n}=n!2^n p_n .
$$
Вычислим производящую функцию
$$
  \Psi (\lambda , w)=\sum _{n=0}^{\infty}p_n (\lambda )w^n .
$$
Повторяя те же вычисления, что и выше, получим, что
функция
$ \Psi $
удовлетворяет соответсвующему неоднородному дифференциальному
уравнению
$$
  (2w^2 +4w+2)\Psi ^{\prime}=
  (\lambda ^2 -1-w)\Psi +4\lambda .
$$
с начальным условием
$$
  \Psi (\lambda ,w) \biggr | _{w=0}=0.
$$
Решим это уравнение
{\bfseries методом Лагранжа (методом вариации постоянной)},
т.е. будем искать решение в виде
$ \Psi =C\Phi . $
Тогда
$$
  (w+1)^2 C^{\prime}\Phi =2\lambda
$$
и следовательно,
$$
  C(\lambda , w)=\int _0 ^w 2\lambda
  (1+t)^{-3/2}\exp \biggl \{
  -\frac{t}{1+t}\frac{\lambda ^2}{2} \biggr \} dt .
$$
Функция
$ \Psi $
также голоморфна в единичном груге и имеет на его границе
единственную особую точку
$ w=-1. $
Это наблюдение позволяет нам вычислить резольвентную функцию оператора
$$
  f(\lambda )=\lim _{n \rightarrow \infty}
  \frac{P_{2n}(\lambda )}{Q_{2n}(\lambda )},
  \qquad \lambda \in \mathbb{C}_+ .
$$
                             %%%%%%%%%%  Упражнение 2   %%%%%%%%%%%%%%%%%%
\begin{Exe}
Справедлива формула
$$
  f(\lambda )=C(\lambda ,w) \biggr | _{w=-1}.
$$
\end{Exe}
Итак,
$$
  f(\lambda )=- \int _{-1}^0 2\lambda
  \frac{\exp \bigl \{ -\frac{t}{1+t}\frac{\lambda ^2}{2}  \bigr \} }
  {(t+1)^{3/2}}dt=
  \biggl [ t+1=s \biggr ] =
$$
$$
  =-2\lambda e^{-\lambda ^2/2}\int _0 ^1 \exp
   \biggl ( \frac{\lambda ^2}{2s}
  \biggr ) \frac{ds}{s^{3/2}}=
  \biggl [ x=\frac{1}{s} \biggr ] =
  -2\lambda e^{-\lambda ^2 /2}
   \int _1 ^{\infty} e^{\frac{\lambda ^2}{2}x}
   \frac{dx}{\sqrt{x}} .
$$
Положим
$$
  \frac{\lambda ^2}{2}=-y, \qquad y>0 ,
$$
и рассмотрим интеграл
{\bfseries (неполную гамма-функцию или функцию ошибок)}
$$
  \mathrm{erf} (y)=\int _1 ^{\infty} e^{-yx}
  \frac{dx}{\sqrt{x}}.
$$
Осталось разложить этот интеграл в асимптотический ряд.
Используя формулу интегрирования по частям, получим
$$
  \mathrm{erf}(y)=
  \frac{1}{\sqrt{x}}\frac{e^{-yx}}{-y} \biggr | _1 ^{\infty} -
  \int _1 ^{\infty}\frac{e^{-yx}}{-y}\biggl (-\frac{1}{2} \biggr )
  x^{-3/2}dx=
$$
$$
  =\frac{e^{-y}}{y} - \frac{1}{2y} \int _1 ^{\infty}
  e^{-yx}x^{-3/2}dx=...
$$
$$
  ...=e^{-y} \biggl \{
  \frac{1}{y} - \frac{1}{2}\frac{1}{y^2}+
  \frac{1}{2}\frac{3}{2} \frac{1}{y^3} -... \biggr \}.
$$
                      %%%%%%%%%%%   Упражнение  3  %%%%%%%%%%%%%%%%%%%%
\begin{Exe}
Доказать, что действительно получается асимптотический ряд.
\end{Exe}
Возвращаясь к функции
$ f(\lambda ) , $
окончательно получим
$$
  f(\lambda ) \sim \sum _{n=0}^{\infty}
  \frac{(2n-1)!!}{\lambda ^{2n+1}}.
$$
Следовательно,
$$
  \begin{cases}
    s_{2n}=(2n-1)!! \\
	s_{2n+1}=0
  \end{cases}
  \quad (n=0, \; 1, \; 2,...)
$$
Спектральная задача решена.
                            %%%%%%%%%   Замечание 5  %%%%%%%%%%%%%%%%%%
\begin{Rem}
Числа
$ s_n $
суть степенные моменты меры
$$
  e^{-x^2}dx, \qquad x \in \mathbb{R}.
$$
\end{Rem}
Действительно,
$$
  2 \int _0 ^{\infty} x^{2n}e^{-x^2}dx=
  \biggl [ x^2 =t \biggr ] =
  \int _0 ^{\infty} t^{n-1/2}e^{-t}dt=
$$
$$
  =\Gamma \biggl ( n+\frac{1}{2} \biggr )=
  \biggl (n-\frac{1}{2} \biggr )
  \biggl (n-\frac{3}{2} \biggr )...
  \biggl ( \frac{1}{2} \biggr ) =
  (2n-1)!! .
$$
\newpage
              %%%%%%%%%%%%%%%%%%%%%%%%%%%%%%%%%%%%%%%%%%%%%%%%%%%%%%%%%%%%%%%
			  %%%%%%%%%%%   1.9 Спектральные данные, восстанавливающие  %%%%%
			  %%%%%%%%%%%       оператор.        %%%%%%%%%%%%%%%%%%%%%%%%%%%%
			  %%%%%%%%%%%       Прямая и обратная задача   %%%%%%%%%%%%%%%%%%
			  %%%%%%%%%%%%%%%%%%%%%%%%%%%%%%%%%%%%%%%%%%%%%%%%%%%%%%%%%%%%%%%
\subsection{Спектральные данные, восстанавливающие \\
оператор.
Прямая и обратная задача.}
$$ \; $$
Теорема Чебышева-Фавара устанавливает соответствие между
коэффициентами матрицы
\begin{equation*}
  \mathsf{A}=
    \begin{pmatrix}
	  b_0 & 1 & \; & \; & \; \\
	  a_1 & b_1 & 1 & \; & \; \\
	  \; & a_2 & b_2 & 1 & \; \\
	  \; & \; & \dots & \dots & \dots
	\end{pmatrix},
\end{equation*}
задающей некоторый оператор
$ \mathsf{A} \in \mathbb{J} , $
и моментными последовательностями
$ \{ s_n \} , $
задающими некоторый функционал
$ \mathfrak{S} \in \mathbb{S}. $
В свою очередь моментные последовательности
$$
  s_n := ( \mathsf{A}^n \mathtt{e}_0 , \mathtt{e}_0 )
$$
определяют резольвентную функцию оператора
$ \mathsf{A}: $
$$
  f(z)=(\mathsf{R}_z \mathtt{e}_0 , \mathtt{e}_0 )=
    ((z \mathsf{I}-\mathsf{A})^{-1} \mathtt{e}_0 , \mathtt{e}_0 )=
	  \sum _{n=0}^{\infty}\frac
	    {(\mathsf{A}^n \mathtt{e}_0 , \mathtt{e}_0 )}
		  {z^{n+1}}=
		    \sum _{n=0}^{\infty}
			  \frac{s_n}{z^{n+1}}.
$$
Так как, по определению, спектр оператора
$ \mathsf{A} $
есть множество, где оператор резольвенты
$ \mathsf{R}_z $
не существует (т.е. оператор
$ z\mathsf{I}-\mathsf{A} $
не имеет однозначного ограниченного обратного оператора),
поэтому резольвента
$ \mathsf{R}_z , $
резольвентная функция
$ f(z)  $
и ее моменты
$ s_n $
могут быть отнесены к данным характеризующим спектр, т.е. к
{\bfseries спектральным данным.} \\

Задача однозначного восстановления оператора по
какому-то набору его спектральных данных называется
{\bfseries обратной спектральной задачей.}
При этом нахождение спектральных данных по коэффициентам
оператора называют
{\bfseries прямой спектральной задачей.} \\

Формальное разложение степенного ряда
$$
  \sum _{n=0}^{\infty}
    \frac{s_n}{z^{n+1}}
$$
в непрерывную дробь (см. параграф 1.5) дает решение
обратной спектральной задачи:
$$
  f(z) \longrightarrow \mathsf{A}
$$
или
$$
  \{ s_n \} \longrightarrow \mathsf{A}.
$$
С другой стороны, наличие сходимости диагональных аппроксимаций
Паде, или что то же самое, сходимость подходящих дробей
непрерывной дроби (см. параграф 1.5) позволяет решить прямую
спектральную задачу:
$$
  \mathsf{A} \longrightarrow f(z).
$$
\newpage
                        %%%%%%%%%%%%%%%%%%%%%%%%%%%%%%%%%%%%%%%%%%%%%%%%%%%%
						%%%%%%%%%%%%%%%%%%%%%%%%%%%%%%%%%%%%%%%%%%%%%%%%%%%%
						%%%%  2  Классическая  проблема моментов  %%%%%%%%%%
						%%%%%%%%%%%%%%%%%%%%%%%%%%%%%%%%%%%%%%%%%%%%%%%%%%%%
						%%%%%%%%%%%%%%%%%%%%%%%%%%%%%%%%%%%%%%%%%%%%%%%%%%%%
\section{Классическая проблема моментов}
$ \; $ \\
$ \; $ \\

                        %%%%%%%%%%%%%%%%%%%%%%%%%%%%%%%%%%%%%%%%%%%%%%%%%%%%
						%%%%%%%   2.1  Позитивные последовательности   %%%%%
						%%%%%%%%%%%%%%%%%%%%%%%%%%%%%%%%%%%%%%%%%%%%%%%%%%%%
\subsection{Позитивные последовательности}
$ \; $ \\

Пусть
$ \mu - $
конечная положительная борелевская мера на вещественной оси.
Потребуем, чтобы носитель
$ S( \mu ) $
этой меры был бесконечным множеством. Потребуем также, чтобы
сходились интегралы
$$
  \int | \lambda |^n d \mu ( \lambda ) < + \infty ,
  \quad n=0, \; 1, \; 2,...
$$
Множество всех таких мер обозначим
$ \mathcal{M} .$
\\

{\bfseries Классической проблемой моментов Гамбургера}
называется следующая задача
                             %%%%%%%%%%   Задача  2  %%%%%%%%%%%%%%%%%%%%%%%
\begin{Pro}
Дана последовательность действительных чисел
$ \mathtt{s}=\{ s_n \} _{n=0}^{\infty} . $
Требуется найти меру
$ \mu \in \mathcal{M} $
такую, что числа
$ s_n $
суть ее степенные моменты:
$$
  s_n =\int \lambda ^n d \mu ( \lambda ) ,
  \quad n=0, \; 1, \; 2,...
$$
\end{Pro}
Заметим, что в этом случае функционал
$ \mathfrak{S} , $
соответствующий последовательности
$ \mathtt{s} , $
имеет вид
$$
  \mathfrak{S} \{ P(\lambda ) \} =
  \int P(\lambda ) d \mu ( \lambda ).
$$
Установим также связь между формальным степенным рядом
$$
  f(z)= \sum _{n=0}^{\infty}
  \frac{s_n}{z^{n+1}}
$$
и функцией
$$
  \hat \mu (z) = \int \frac{d \mu (\lambda )}{z-\lambda }.
$$
Рассмотрим сперва частный случай, когда носитель меры
$ S( \mu ) - $
компакт. Тогда функция
$ \hat \mu (z) $
голоморфна в области
$ \bar{\mathbb{C}} \setminus S( \mu ) $
и в частности, вне некоторого круга радиуса
$ R , $
содержащего носитель меры. При
$ |z|>R $
имеем
$$
  \hat \mu (z)=\int \sum _{n=0}^{\infty}
  \frac{\lambda ^n}{z^{n+1}}d \mu (\lambda ) =
  \sum _{n=0}^{\infty} \frac{1}{z^{n+1}}
  \int \lambda ^n d \mu ( \lambda )=
  \sum _{n=0}^{\infty} \frac{s_n}{z^{n+1}}.
$$
Почленное интегрирование ряда возможно в силу его равномерной
сходимости. Итак, в случае компактного носителя степенной ряд
$ f $
является
{\bfseries рядом Лорана}
функции
$ \hat \mu $
в окрестности бесконечности. Тогда по
{\bfseries формуле Коши-Адамара}
$$
  \limsup _{n \rightarrow \infty}
  |s_n |^{1/n} < + \infty . \quad ( \ast )
$$
                                %%%%%%%%%%%%%%   Упражнение 4   %%%%%%%%%%%%
\begin{Exe}
Верно и обратное утверждение. Если выполнено условие
$ ( \ast ) , $
то носитель меры -- компакт.
\end{Exe}
Пусть теперь носитель меры неограничен. Тогда интеграл
$ \hat \mu $
определяет, вообще говоря, две аналитические функции: одну --
голоморфную в верхней полуплоскости
$ \mathbb{C}_+ , $
другую -- в нижней. Без ограничения общности всегда
будем рассматривать верхнюю полуплоскость. Степенной ряд
$ f $
теперь расходится, не является рядом Лорана. Но мы можем
утверждать, что это
{\bfseries асимптотический ряд}.
А именно, возьмем произвольное число
$ 0< \varepsilon < \pi /2 . $
Обозначим
$ S_{\varepsilon} $
сектор в верхней полуплоскости
$$
  S_{\varepsilon}=
  \{z \in \mathbb{C} | \varepsilon < Arg z < \pi -
  \varepsilon \} .
$$
Обозначим
$$
  T_N (z)=\sum _{n=0}^{N-1}
  \frac{s_n}{z^{n+1}},
  \quad N=1, \; 2, \; 3,...
$$
частичные суммы ряда
$ f . $
Тогда для любого
$ N \in \mathbb{N} $
имеем
$$
  \hat \mu (z)-T_N (z)=O(1/z^{N+1}),
  \quad z \in S_{\varepsilon}, \; z \rightarrow \infty .
$$
Это соотношение и означает, что
$ f - $
асимптотический ряд. Докажем это соотношение. Имеем
$$
  \hat \mu (z)-T_N (z)=
  \int \frac{d \mu ( \lambda )}{z-\lambda}-
  \int \sum _{n=0}^{N-1} \frac{\lambda ^n}{z^{n+1}}
  d \mu ( \lambda )=
$$
$$
  =\int \frac{d \mu ( \lambda )}{z-\lambda}-
  \int \frac{1-(\lambda /z)^N}{z-\lambda}d \mu (\lambda )=
  \frac{1}{z^N} \int \frac{\lambda ^N}{z-\lambda}d \mu ( \lambda ).
$$
Если
$ z \in S_{\varepsilon} , $
то
$ |z-\lambda | \geq |z| \sin \varepsilon , $
и
$$
  \biggl | \int \frac{\lambda ^N}{z-\lambda}d \mu (\lambda ) \biggr | \leq
  \frac{1}{|z| \sin \varepsilon } \int | \lambda |^N
  d \mu ( \lambda ).
$$
Следовательно,
$$
  | \hat \mu (z)-T_N (z)| \leq
  \frac{C_{\varepsilon , N}}{z^{N+1}},
  \quad z \in S_{\varepsilon}.
$$
$ \triangle $
\\

Вернемся к классической проблеме моментов Гамбургера.
Возникают два вопроса.
\\
{\bfseries (1) Разрешимость проблемы моментов}
\\
{\bfseries (2) Определенность проблемы моментов.}
\\
Первый вопрос относительно простой. Прежде чем
сформулировать критерий разрешимости дадим следующее
                             %%%%%%%%%   Определение 7    %%%%%%%%%%%%%%%%%
\begin{Def}
Последовательность действительных чисел
$ \mathtt{s} $
называется
{\bfseries позитивной},
если все ее определители Ганкеля положительны:
$$
  H_n >0, \quad n=0, \; 1, \; 2,...
$$
\end{Def}
                              %%%%%%%%%%%%%%   Теорема 3 %%%%%%%%%%%%%%%%%%
\begin{The}
Для разрешимости классической проблемы моментов Гамбургера
необходимо и достаточно, чтобы последовательность
$ \mathtt{s} $
была позитивной.
\end{The}							
{\Large Доказательство}
необходимости.\\
Согласно
{\bfseries критерию Сильвестра}
положительность всех
определителей Ганкеля равносильна положительной определенности
квадратичных форм
$$
  \mathsf{Q} ( \mathtt{p} )=
  \sum _{j,k=0}^n s_{j+k}p_j p_k .
$$
Имеем
$$
  \mathsf{Q} (\mathtt{p})=
  \sum _{j,k=0}^{n}p_j p_k \int \lambda ^{j+k} d \mu (\lambda )=
  \int \biggl ( \sum _{j=0}^{n} p_j \lambda ^j \biggr )
  \biggl ( \sum _{k=0}^{n} p_k \lambda ^k \biggr )
  d \mu (\lambda )=
$$
$$
  =\int P( \lambda )^2 d \mu (\lambda ) \geq 0,
$$
где
$ P - $
многочлен, соответствующий вектору
$ \mathtt{p} . $
Более того, если
$ \mathtt{p} \not = 0, $
то
$ P \not \equiv 0 , $
и следовательно,
$$
  \int P(\lambda )^2 d \mu ( \lambda ) >0.
$$
$ \triangle $
\newpage
                          %%%%%%%%%%%%%%%%%%%%%%%%%%%%%%%%%%%%%%%%%%%%%%%%%
						  %%%%%%%   2.2  Нули ортогональных многочленов  %%
						  %%%%%%%%%%%%%%%%%%%%%%%%%%%%%%%%%%%%%%%%%%%%%%%%%
\subsection{Нули ортогональных многочленов}
$ \; $
\\

Нам понадобится следующая
                           %%%%%%%%%%%%   Лемма  2    %%%%%%%%%%%%%%%%%%%%%%
\begin{Lem}
Если
$ P( \lambda ) - $
многочлен, неотрицательный на вещественной оси:
$$
  P( \lambda ) \geq 0, \quad \lambda \in \mathbb{R},
$$
то для некоторых многочленов
$ A $ и $ B $
с вещественными коэффициентами справедливо следующее представление
$$
  P=A^2 +B^2 .
$$
\end{Lem}
{\Large Доказательство.}
Многочлен
$ P $
вещественный. Его корни либо вещественные
(причем четной кратности), либо комплексно сопряженные.
Напишем разложение многочлена
$ P $
на линейные множители (без ограничения общности считаем, что
старший коэффициент многочлена равен единице):
$$
  P(\lambda )=\prod _j (\lambda -x_j )^2
  \prod _k (\lambda -(\alpha _k +i \beta _k ))
  (\lambda -(\alpha _k -i\beta _k ))=
$$
$$
  =\biggl ( \prod _j (\lambda -x_j ) \biggr )^2
  \prod _k ( ( \lambda -\alpha _k )^2 + \beta _k ^2 ).
$$
Применим известное алгебраическое тождество: "произведение
суммы двух квадратов на сумму двух квадратов снова есть
сумма двух квадратов". Получим утверждение леммы.\\
$ \triangle $
\\
Функционал
$ \mathfrak{S} , $
соответсвующий позитивной последовательности, будем называть
позитивным.\\
Из леммы вытекает
                                     %%%%%%%%%%%%   Следствие  2  %%%%%%%%%
\begin{Cor}
Функционал
$ \mathfrak{S} $
{\bfseries позитивный}
, тогда и только тогда,
когда для любого многочлена
$ P( \lambda ) , $
неотрицательного на вещественной оси и не равного тождественно нулю,
имеем
$ \mathfrak{S} \{ P( \lambda ) \} >0. $
\end{Cor}
Действительно, применяя функционал
$ \mathfrak{S} $
к многочлену
$ A^2 , $
получим положительно определенную квадратичную форму.
                        %%%%%%%%%%%   Утверждение 9  %%%%%%%%%%%%%%%%%%%%%%
\begin{Sta}
Пусть
$ Q_n - n- $
й ортогональный многочлен относительно позитивного функционала
$ \mathfrak{S} , $
Тогда все корни многочлена
$ Q_n $
вещественные и простые.
\end{Sta}
{\Large Доказательство.}
Обозначим
$ x_1 ,...,x_m $
все различные вещественные корни многочлена
$ Q_n , $
имеющие нечетную кратность. Пусть
$$
  P(\lambda )=(\lambda -x_1 )...(\lambda -x_m )
$$
многочлен с этими корнями. Тогда
\begin{equation*}
  Q_n P
  \begin{cases}
    \geq 0 \\
	\not \equiv 0
  \end{cases}
\end{equation*}
(вспомним, что старший коэффициент многочлена
$ Q_n $
считаем равным единице). Следовательно,
$ \mathfrak{S} \{ Q_n P \} >0. $
Если предположить, что
$ m \leq n-1 , $
то
$ \mathfrak{S} \{ Q_n P \} =0 $
в силу соотношений ортогональности. Полученное противоречие
доказывает утверждение.\\
$ \triangle $
                         %%%%%%%%%%%%%%%   Следствие  3  %%%%%%%%%%%%%%%%%%%
\begin{Cor}
Пусть
$ \mathfrak{S} - $
позитивный функционал. Тогда аппроксимации Паде имеют
следующее разложение в сумму простейших дробей:
$$
  \pi _n (z) = \sum _{j=1}^n
  \frac{\mu _{n,j}}{z-x_{n,j}},
$$
где
$ x_{n,1}<...<x_{n,n} - $
нули многочлена
$ Q_n , $
и
$$
  \mu _{n,j}=\mathrm{res}_{z=x_{n,j}} \pi _n (z) =
  \frac{P_n (x_{n,j})}{Q_n ^{\prime}(x_{n,j})}.
$$
\end{Cor}
                   %%%%%%%%%%%%%%%   Определение  8   %%%%%%%%%%%%%%%%%%%%%
\begin{Def}
Числа
$ \mu _{n,j} $
называются
{\bfseries коэффициентами Кристоффеля}.
\end{Def}
Аналогичным образом можно доказать следующее
                        %%%%%%%%%%%%   Утверждение 10  %%%%%%%%%%%%%%%%%%%%%
\begin{Sta}
Если
$ \mu - $
конечная положительная борелевская мера с бесконечным носителем
на компактном отрезке
$ [a,b] $
вещественной оси, то все нули $ n- $ го ортогонального
относительно меры
$ \mu $
многочлена лежат на интервале
$ (a,b) . $
\end{Sta}
\newpage
                            %%%%%%%%%%%%%%%%%%%%%%%%%%%%%%%%%%%%%%%%%%%%%%%
							%%%%%%%%%   2.3   Квадратурная формула  %%%%%%%
							%%%%%%%%%   Гаусса-Якоби    %%%%%%%%%%%%%%%%%%%
							%%%%%%%%%%%%%%%%%%%%%%%%%%%%%%%%%%%%%%%%%%%%%%%
\subsection{Квадратурная формула Гаусса-Якоби}
$ \; $
\\

Пусть
$ Q_n - n- $
й ортогональный многочлен относительно позитивного функционала
$ \mathfrak{S} , \; x_{n,j} - $
его нули,
$ \mu _{n,j} - $
соответствующие коэффициенты Кристоффеля.
                       %%%%%%%%%%%%%%%%%   Утверждение  11 %%%%%%%%%%%%%%%%%
\begin{Sta}
Для любого многочлена
$ P $
степени не выше
$ \; 2n-1 \; $
справедлива
{\bfseries квадратурная формула Гаусса-Якоби}
$$
  \mathfrak{S} \{ P(\lambda ) \} =
  \sum _{j=1}^n \mu _{n,j} P(x_{n,j}).
$$
\end{Sta}
{\Large Доказательство.}
Разложим аппроксимации Паде
$ \pi _n (z) $
ряда
$$
  f(z)=\sum _{n=0}^{\infty} \frac{s_n}{z^{n+1}}
$$
в ряд Лорана
$$
  \pi _n (z)= \sum _{j=1}^n \frac{\mu _{n,j}}{z-x_{n,j}}=
  \sum _{k=0}^{\infty}\frac{1}{z^{k+1}}
  \sum _{j=1}^n \mu _{n,j} x_{n,j}^k .
$$
Имеем
$$
  f(z)-\pi _n (z)=\frac{A_n}{z^{2n+1}}+...
$$
Следовательно,
$$
  s_k =\sum _{j=1}^n \mu _{n,j}x_{n,j}^k ,
  \quad k=0,...,2n-1.
$$
$ \triangle $
\\
В дальнейшем всегда считаем, что
$ s_0 =1. $
                       %%%%%%%%%%%%%%%    Следствие  4   %%%%%%%%%%%%%%%%%%
\begin{Cor}
Справедливы следующие свойства коэффициентов Кристоффеля:
\\
$ (1) \; \mu _{n,1}+...+\mu_{n,n}=1, $
\\
$ (2) \; \mu _{n,j}>0, \quad j=1,...,n. $
\end{Cor}
{\Large Доказательство.}
Применим квадратурную формулу к многочлену
$ P(\lambda )\equiv 1. $
Получим (1).\\
Положим
$$
  l(\lambda )=\frac{Q_n (\lambda )}
  {Q_n ^{\prime}(x_{n,j})(\lambda -x_{n,j})}.
$$
Тогда
$$
  \mathfrak{S} \{ l^2 (\lambda ) \} = \mu _{n,j}>0.
$$
$ \triangle $
\newpage      					
               %%%%%%%%%%%%%%%%%%%%%%%%%%%%%%%%%%%%%%%%%%%%%%%%%%%%%%%%%%%%%%
			   %%%%%%%%   2.4  Разрешимость проблемы моментов   %%%%%%%%%%%%%
			   %%%%%%%%%%%%%%%%%%%%   Гамбургера     %%%%%%%%%%%%%%%%%%%%%%%%
			   %%%%%%%%%%%%%%%%%%%%%%%%%%%%%%%%%%%%%%%%%%%%%%%%%%%%%%%%%%%%%%
\subsection{Разрешимость проблемы моментов Гамбургера}
$ \; $
\\
               %%%%%%%  (a) Разрешимость проблемы моментов   %%%%%%%%%%%%%%
{\bfseries (a) Разрешимость проблемы моментов}\\
В этом пункте мы завершим доказательство теоремы 3.\\
{\Large Доказательство достаточности.}
Положим
$$
  \mu _n =\sum _{j=1}^n \mu _{n,j} \delta _{x_{n,j}},
$$
где
$ \delta _a - $
единичная мера, сосредоточенная в точке
$ a . $
Согласно следствию 4
$ \mu _n - $
положительные вероятностные меры. По
{\bfseries теоремам Хелли}
из последовательности
$ \{ \mu _n \} _{n \in \mathbb{N}} $
можно выбрать подпоследовательность
$ \{ \mu _n \} _{n \in \Lambda \subset \mathbb{N}}, $
слабо сходящуюся к некоторой мере
$ \mu . $
Тогда для любого многочлена
$ P(\lambda ) $
имеем
$$
  \int P d \mu _n \rightarrow \int P d \mu
  \quad (n \in \Lambda ).
$$
По квадратурной формуле Гаусса-Якоби для любого
$ m=0, \; 1, \; 2,... $
$$
  \int \lambda ^m d \mu _n =s_m
$$
для всех достаточно больших
$ n $
(а именно, для
$ m \leq 2n-1 ). $
Следовательно,
$$
  \int \lambda ^m d \mu =s_m ,
  \quad m=0, \; 1, \; 2,..
$$
Носитель меры
$ S( \mu ) - $
бесконечное множество. Действительно, в противном случае функция
$$
  \hat \mu (z)=\int \frac{d \mu (\lambda )}{z-\lambda }
$$
будет рациональной, и по
{\bfseries теореме Кронекера}
все ее определители
Ганкеля, начиная с некоторого номера, будут равны нулю.\\
$ \triangle $
                 %%%%%%%%%%%%%%%%   Теорема  4   %%%%%%%%%%%%%%%%%%%%%%%%%%%
\begin{The}{\bfseries (Кронекер)}
Степенной ряд
$$
  f(z)= \sum _{n=0}^{\infty}
  \frac{s_n}{z^{n+1}}
$$
представляет рациональную функцию, тогда и только тогда,
когда все его определители Ганкеля, начиная с некоторого номера,
равны нулю:
$$
  H_n =0, \quad n>N.
$$
\end{The}
{\Large Доказательство.}
Индекс
$ n $
назовем нормальным, если для любого решения задачи Паде
(задачи 1)
$ \deg Q_n =n . $
Очевидно, что индекс
$ n $
нормальный, тогда и только тогда, когда
$ H_n \not =0. $
\\
Пусть
$ f - $
рациональная функция порядка
$ N \; ( N - $
степень знаменателя в несократимом представлении). Тогда
$ f=\pi _N $
и
$ Q_N f -P_N \equiv 0. $
Это значит, что пара
$ (Q_N , P_N ) $
решает задачу Паде и для любого
$ n>N . $
Следовательно, эти индексы не являются нормальными, т.е.
$ H_n =0, \quad n>N. $
\\
Обратно, если
$ H_n =0 , $
т.е. существует решение с
$ q_n =0 , $
то оно будет решением
$ n-1 - $
й задачи Паде, т.е.
$ \pi _n -\pi _{n-1} . $
Поэтому, если
$ H_n =0, \quad n>N, $
то
$ \pi _n =\pi _N , \quad n \geq N. $
Но
$ \pi _n \rightarrow f $
покоэффициентно. Следовательно,
$ f \equiv \pi _N . $
\\
$ \triangle $
\\
				   %%%%%%%%   (b) Дополнение -- теоремы Хелли   %%%%%%%%%%
{\bfseries (b)    Дополнение -- теоремы Хелли.}
\\

Пусть
$ \mu - $
конечная положительная борелевская мера на конечном или бесконечном
промежутке
$ [a,b] $
вещественной оси,
$ F - $
ее
{\bfseries функция распределения},
т.е.
\begin{equation*}
  F(x)=
    \begin{cases}
	  \mu ([0,x)), \quad если \; x>0,\\
	  0, \quad если \; x=0,\\
	  -\mu ([x,0)), \quad если \; x<0.
	\end{cases}
\end{equation*}
Тогда
\\
(1) функция
$ F $
ограничена (следует из конечности меры),\\
(2) $ F(0)=0 $(по определению),\\
(3) $ F(x) $ не убывает (следует из положительности меры),\\
(4) $ F(x) $ непрерывна слева (следует из непрерывности или
$ \sigma- $ аддитивности меры).\\
При этом
$$
  \mu ([\alpha , \beta ))=F(\beta )-F(\alpha ).
  \quad (\ast )
$$
Обратно, если
$ F - $
любая функция распределения, т.е. функция, обладающая свойствами
(1)--(4), то формула
$ (\ast ) $
определяет конечную положительную
$ \sigma - $
аддитивную меру на полукольце
$ \mathbb{P} , $
состоящем из всех промежутков вида
$ [\alpha , \beta ) . $
Эта мера продолжается по Лебегу на некоторую
$ \sigma- $
алгебру измеримых множеств, содержащую, в частности, все
борелевские множества. Меру, определенную таким образом на
борелевской
$ \sigma- $
алгебре, мы обозначим
$ \mu . $
Таким образом, мы имеем взаимно-однозначное соответствие
между конечными положительными борелевскими мерами и их
функциями распределения.\\
Далее, заменим условие (3) на условие
\\
(3*) функция
$ F $
имеет ограниченную вариацию:
$$
  \bigvee _a ^b (F)=\sup _R \sum _{j=1}^n
  |F(x_j )-F(x_{j-1})|< + \infty ,
$$
где supremum берется по всем разбиениям отрезка
$$
  R=\{ a=x_0 <x_1 <...<x_n =b \} .
$$
Тогда функция
$ F $
будет разностью двух неубывающих функций распределения, и
по формуле
$ (\ast ) $
она пораждает заряд (разность двух положительных мер).
Верно и обратное. Мы имеем взаимно-однозначное соответствие между
зарядами и их функциями распределения (ограниченной вариации).
                          %%%%%%%%%%%   Теорема  5   %%%%%%%%%%%%%%%%%%%%%%
\begin{The}
{\bfseries (Первая теорема Хелли)}\\
Если
$ F_n - $
функции распределения ограниченной вариации, и полные вариации
всех этих функций ограничены в совокупности, то из последовательности
$ F_n $
можно выбрать подпоследовательность, сходящуюся в каждой точке отрезка
$ [a,b]. $
\end{The}
                 %%%%%%%%%%%%   Замечание  6   %%%%%%%%%%%%%%%%%%%%%%%%%%%%%
\begin{Rem}
Предельная функция также будет функцией распределения
ограниченной вариации.
\end{Rem}
{\Large Доказательство.}
Можно считать, что функции
$ F_n $
неубывающие. Тогда они ограничены в совокупности.\\
Занумеруем все рациональные числа из отрезка
$ [a,b] , $
а именно:
$ \{ x_1 , \; x_2 ,... \} . $
Из ограниченной последовательности
$$
  \{ F_n (x_1 )| \; n \in \mathbb{N} \}
$$				 						
выберем сходящуюся подпоследовательность
$$
  \{ F_n (x_1 )| \; n \in \Lambda _1 \subset \mathbb{N} \} .
$$
Из ограниченной последовательности
$$
  \{ F_n (x_2 )| \; n \in \Lambda _1 \}
$$
выберем сходящуюся подпоследовательность
$$
  \{ F_n (x_2 ) | \; n \in \Lambda _2 \subset \Lambda _1 \} ,
$$
и т.д. Пусть
$ \{ F_n | \; n \in \Lambda \} - $
подпоследовательность, полученная
{\bfseries диагональным методом Кантора}.
По построению она сходится в каждой рациональной точке:
$$
  F_n (x) \rightarrow F(x),
  \quad (n \in \Lambda )
  \quad x \in \mathbb{Q} \cap [a,b] .
$$
Функция
$ F(x) $
обладает свойствами (1)-(4) на множестве
$ \mathbb{Q} \cap [a,b] . $
Доопределим ее в каждой точке отрезка как функцию непрерывную слева.
Пусть
$ x - $
точка непрерывности этой функции. Докажем, что
$ F_n (x) \rightarrow F(x) \quad (n \in \Lambda ). $
Доказательство:
$$
  \forall \varepsilon >0 \; \exists x^{\prime},x^{\prime \prime}
  \in \mathbb{Q} \cap [a,b] :
  x^{\prime}<x<x^{\prime \prime}
$$
$$
  F(x)-F(x^{\prime})<\varepsilon ,
  F(x^{\prime \prime})-F(x)<\varepsilon
$$
$$
  \exists N \in \mathbb{N} \; \forall n>N, n \in \Lambda
$$
$$
  |F(x^{\prime})-F_n (x^{\prime})|<\varepsilon ,
  |F(x^{\prime \prime})-F_n (x^{\prime \prime})|<\varepsilon
$$
$$
  \Rightarrow F_n (x^{\prime \prime})-F_n (x^{\prime})<4 \varepsilon
  \Rightarrow F_n (x)-F_n (x^{\prime})<4 \varepsilon
$$
$$
  \Rightarrow |F(x)-F_n (x)|<6 \varepsilon
$$
Ч.т.д.\\
Множество точек разрыва монотонной функции не более чем счетное.
Снова применяя диагональный метод Кантора, выделим подпоследовательность,
сходящуюся и в точках разрыва.\\
$ \triangle $
\\

Если
$ F - $
неубывающая функция распределения, то
{\bfseries интеграл Лебега-Стилтьеса}
$ \int _a ^b f dF $
по определению равен интегралу Лебега
$ \int _a ^b f d \mu $
по соответствующей мере
$ \mu . $
{\bfseries Интеграл Римана-Стилтьеса}
по определению равен следующему пределу интегральных сумм
$$
  \int _a ^b fdF = \lim _{\lambda (R) \rightarrow 0}
  \sum _{j=1}^n f(\xi _j )
  (F(x_j )-F(x_{j-1})),
$$
где предел берется при условии, что стремится к нулю параметр
разбиения с отмеченными точками
$$
  \lambda (R) = \max _{j=1,...,n} (x_j -x_{j-1}).
$$
Аналогичным образом определяются интегралы для функции распределения
ограниченной вариации.\\
Для непрерывной функции
$ f $
интегралы Римана-Стилтьеса и Лебега-Стилтьеса совпадают.
                       %%%%%%%%%%%%%%   Теорема  6   %%%%%%%%%%%%%%%%%%%%%%
\begin{The}
{\bfseries (Вторая теорема Хелли)}\\
Если\\
1) вариации функций распределения
$ F_n $
ограничены в совокупности:
$$
  \bigvee _a ^b (F_n )<C,
$$
2) $ F_n \rightarrow F $
в каждой точке отрезка
$ [a,b] , $
\\
то для любой функции
$ f $
непрерывной на отрезке
$ [a,b] $
имеем
$$
  \int _a ^b fdF_n \rightarrow \int _a ^b fdF .
$$
\end{The}
                   %%%%%%%%%%%   Замечание 7   %%%%%%%%%%%%%%%%%%%%%%%%%%%%
\begin{Rem}
В этом случае говорят, что соответствующие меры (заряды)
$ \mu _n $
{\bfseries слабо сходятся}
к мере (заряду)
$ \mu : $
$$
  \mu _n \rightharpoonup \mu .
$$
\end{Rem}
{\Large Доказательство.}
Возьмем произвольное
$ \varepsilon >0 . $
По теореме Кантора найдем столь мелкое разбиение отрезка,
чтобы колебание функции
$ f $
на каждом отрезке разбиения было меньше
$ \varepsilon . $
Тогда
$$
  \biggl | \int _a ^b fdF_n -
  \sum _{j=1}^n f(\xi _j )(F_n (x_j )-F_n (x_{j-1}))
  \biggr | = \biggl | \sum _{j=1}^n \int _{x_{j-1}}^{x_j}
  (f(x)-f(\xi _j ))dF_n (x) \biggr | \leq
$$
$$
  \leq \sum _{j=1}^n \varepsilon \bigvee _{x_{j-1}}^{x_j}
  (F_n ) = \varepsilon \bigvee _a ^b (F_n )
  \leq C \varepsilon .
$$
То же верно и для
$ F . $
В силу поточечной сходимости
$ F_n $
и ограниченности
$ f $
имеем
$$
  \biggl | \sum _{j=1}^n f(\xi _j )(F(x_j )-F(x_{j-1}))-
  \sum _{j=1}^n f(\xi _j )(F_n (x_j )-F_n (x_{j-1})) \biggr |
  < \varepsilon
$$
для всех достаточно больших
$ n . $\\
$ \triangle $
                      %%%%%%%%%%%%   Теорема   7   %%%%%%%%%%%%%%%%%%%%%%%
\begin{The}
{\bfseries (Третья теорема Хелли)}\\
Пусть
$ F_n - $
функции распределения ограниченной вариации на всей вещественной
оси и для некоторого натурального числа
$ m $ \\
(1) $ \; \bigvee _{-\infty}^{+\infty}(F_n ) \leq C,
\quad n \in \mathbb{N}, $\\
(2) $ \; F_n (x) \rightarrow F(x), \quad x \in \mathbb{R}, $ \\
(3) $ \; \int _{-\infty}^{+\infty}|x|^m |dF_n (x)| \leq C,
\quad n \in \mathbb{N}. $\\
Тогда
$$
  \int _{-\infty}^{+\infty} x^m dF_n (x) \rightarrow
  \int _{-\infty}^{+\infty} x^m dF(x), \quad n \rightarrow \infty .
$$
\end{The}
{\Large Доказательство.}
Определим новые функции распределения
$$
  F_n ^{\ast}(x)=\int _0 ^x t^m dF_n (t).
$$
По второй теореме Хелли они сходятся поточечно к функции распределения
$$
  F^{\ast}(x)=\int _0 ^x t^m dF(t).
$$
По условию теоремы вариации этих функций ограничены:
$$
  \bigvee _{-\infty}^{+\infty}(F_n ^{\ast})=
  \int _{-\infty}^{+\infty} |x|^m |dF_n (x)| \leq C.
$$
Снова применим вторую теорему Хелли к функциям
$ F_n ^{\ast} $
и
$ f \equiv 1 . $\\
$ \triangle $
                    %%%%%%%%%%   Замечание   8   %%%%%%%%%%%%%%%%%%%%%%%%%
\begin{Rem}
При доказательстве разрешимости проблемы моментов мы можем
пользоваться следующей оценкой
$$
  \int |\lambda ^m | d \mu _n (\lambda ) \leq
  \int (\lambda ^{2m}+1)d \mu _n (\lambda ) =
  s_{2m}+s_0
$$
для всех достаточно больших
$ n . $
\end{Rem}
\newpage
                       %%%%%%%%%%%%%%%%%%%%%%%%%%%%%%%%%%%%%%%%%%%%%%%%%%%%
					   %%%%%%%   2.5  Теорема Маркова   %%%%%%%%%%%%%%%%%%%
					   %%%%%%%%%%%%%%%%%%%%%%%%%%%%%%%%%%%%%%%%%%%%%%%%%%%%
\subsection{Теорема Маркова}
$ \; $
\\

В качестве еще одного приложения квадратурной формулы
докажем знаменитую
{\bfseries теорему Маркова}.
                        %%%%%%%%%%%%   Теорема  8   %%%%%%%%%%%%%%%%%%%%%%%
\begin{The}
{\bfseries (Марков)}
\\
Пусть
$ \mu - $
конечная положительная борелевская мера на компактном отрезке
$ \Delta $
вещественной оси, и
$$
  f(z)=\int _{\Delta}
  \frac{d \mu (\lambda )}{z-\lambda}.
$$
Тогда диагональные аппроксимации Паде степенного ряда
$ f $
сходятся к функции
$ f $
равномерно внутри области
$ D= \bar{\mathbb{C}} \setminus \Delta . $
\end{The}
{\Large Доказательство.}
Достаточно рассмотреть случай, когда носитель меры --
бесконечное множество.\\
Все нули
$ x_{n,j} $
знаменателя
$ Q_n $
лежат внутри отрезка
$ \Delta . $
Следовательно, аппроксимации Паде
$ \pi _n $
голоморфны в области
$ D . $
\\
Пусть
$ K \subset D - $
произвольный компакт,
$ \rho - $
расстояние между компактами
$ K $
и
$ \Delta . $
Тогда
$ \rho >0 . $
Для любой точки
$ z \in K $
используя свойства коэффициентов Кристоффеля получим оценку
$$
  | \pi _n (z) |=
  \biggl | \sum _{j=1}^n
  \frac{\mu _{n,j}}{z-x_{n,j}} \biggr | \leq
  \sum _{j=1}^n \frac{\mu _{n,j}}{|z-x_{n,j}|} \leq
  \frac{1}{\rho} \sum _{j=1}^n \mu _{n,j} =\frac{1}{\rho}.
$$
Таким образом, последовательность
$ \{ \pi _n \} _{n \in \mathbb{N}} $
равномерно ограничена внутри области
$ D . $
По
{\bfseries теореме Монтеля}
эта последовательность компактна, т.е.
из любой ее подпоследовательности можно выбрать еще
одну подпоследовательность, которая сходится равномерно
внутри области
$ D $
к некоторой голоморфной функции.\\
Пусть
$ \{ \pi _n \} _{ n \in \Lambda \subset \mathbb{N}} - $
произвольная сходящаяся подпоследовательность:
$$
  \pi _n \rightarrow g , \quad n \in \Lambda .
$$
Коэффициенты ряда Лорана суть непрерывные функционалы
(относительно равномерной сходимости), поэтому
$$
  \{ \pi _n \} _{1/z^{k+1}}
  \rightarrow
  \{ g \} _{1/z^{k+1}},
  \quad k=0, \; 1, \; 2,...
$$
Но мы знаем, что
$$
  \{ \pi _n \} _{1/z^{k+1}} =s_k
$$
для всех достаточно больщих
$ n $
(а именно, для
$ k \leq 2n-1 ) . $
Следовательно, функция
$ g(z) $
имеет тоже разложение в ряд Лорана в окрестности бесконечности, что и
функция
$ f(z) : $
$$
  g(z)= \sum _{k=0}^{\infty}
  \frac{s_k}{z^{k+1}} .
$$
Тогда функции
$ f $
и
$ g $
совпадают в некоторой окрестности бесконечности, а именно,
в области сходимости ряда Лорана, т.е. вне наименьшего
круга, содержащего отрезок
$ \Delta . $
По теореме единственности для голоморфных функций
$ f=g $
всюду в области
$ D . $
\\
Мы доказали, что последовательность
$ \{ \pi _n \} _{n \in \mathbb{N}} $
имеет ровно одну предельную точку, а именно,
$ f . $
Если компактная последовательность имеет ровно одну
предельную точку, то она к ней сходится:
$$
  \pi _n \rightrightarrows f , \quad
  n \rightarrow \infty , \quad
  внутри \; D .
$$
$ \triangle $
\\

Приведем еще одно доказательство теоремы Маркова,
которое дает скорость сходимости аппроксимаций.\\
Без ограничения общности считаем, что
$ \Delta =[-1, +1] . $
Обозначим
$ \psi (z) $
функцию, обратную к функции Жуковского:
$$
  \psi (z)=z+\sqrt{z^2 -1}.
$$
Она определена и голоморфна в области
$ D .$
При этом мы берем ту ветвь квадратного корня, для которой функция
$ \psi $
конформно отображает область
$ D $
на внешность единичного круга.\\
Линии уровня этой функции
$$
  L_R =\{ z \in \mathbb{C} | \; | \psi (z) |=R \} ,
  \quad R>1,
$$
суть эллипсы с фокусами в точках
$ z= \pm 1 $
и полуосями
$$
  \frac{1}{2}(R+1/R), \quad \frac{1}{2}(R-1/R).
$$
Пусть
$ K \subset D - $
произвольный компакт. Обозначим
$ R=R_K $
наибольшее число, такое что компакт
$ K $
лежит вне эллипса
$ L_R . $
Это характеристика компакта, определяющая его удаленность от
отрезка
$ \Delta . $
Построим эллипс
$ L_R $
и вспомогательный эллипс
$ L_{\delta } , $
где
$ 1< \delta <R. $
\\
Функция
$ \psi (z ) $
имеет в бесконечно удаленной точке простой полюс.
Функция
$ f(z)-\pi _n (z) $
голоморфна в области
$ D $
и имеет в бесконечно удаленной точке ноль
$ (2n+1)- $
го порядка. Следовательно, функция
$$
  \bigl ( f(z)-\pi _n (z) \bigr )
  \bigl ( \psi (z) \bigr ) ^{2n}
$$
голоморфна в области
$ D . $
Применим к ней
{\bfseries принцип максимума модуля}:
$$
  \bigl | \bigl ( f(z)-\pi _n (z) \bigr )
  \bigl ( \psi (z) \bigr ) ^{2n} \bigr | \leq
  ||(f-\pi _n )\psi ^{2n}||_{L_{\delta}} \leq
  \frac{2}{\rho (\delta )} \delta ^{2n} ,
$$
где
$ \rho (\delta ) - $
расстояние от эллипса
$ L_{\delta} $
до отрезка
$ \Delta , $
а
$ || \cdot ||_E - $
равномерная sup-норма на компакте
$ E . $
Отсюда
$$
  ||f-\pi _n ||_K \leq ||f-\pi _n ||_{L_R}
  \leq \frac{2}{\rho (\delta )}
  \biggl ( \frac{\delta}{R} \biggr ) ^{2n}.
$$
Следовательно,
$$
  q_K := \limsup _{n \rightarrow \infty}
  ||f-\pi _n ||_K ^{1/n} \leq
  \biggl ( \frac{\delta}{R} \biggr ) ^2 .
$$
Поскольку
$ \delta - $
любое число из интервала
$ (1,R) , $
то
$$
  q_K \leq \frac{1}{R_K ^2}.
$$
$ \triangle $
\newpage
                               %%%%%%%%%%%%%%%%%%%%%%%%%%%%%%%%%%%%%%%%%%%%
							   %%%%%%%%%%   2.6   Формула обращения   %%%%%
							   %%%%%%%%%%   Стилтьеса-Перрона   %%%%%%%%%%%
							   %%%%%%%%%%%%%%%%%%%%%%%%%%%%%%%%%%%%%%%%%%%%
\subsection{Формула обращения Стилтьеса-Перрона}
$ \; $
\\
                                %%%%%%%%   (a) Формула обращения   %%%%%%%%
{\bfseries (a) Формула обращения}\\

Большую роль в дальнейших рассмотрениях будет играть следующая
                                %%%%%%%   Теорема   9    %%%%%%%%%%%%%%%%%%%
\begin{The}
Пусть
$ \mu - $
конечная положительная борелевская мера на вещественной оси,
$$
  \hat \mu (z) = \int \frac{d \mu (t)}{z-t},
  \quad z \in \mathbb{C}_+ .
$$
Тогда для любых
$ a<b - $
точек непрерывности меры
$ \mu $
справедлива
{\bfseries формула обращения Стилтьеса-Перрона}:
$$
  \mu ((a,b)) = \lim _{\varepsilon \rightarrow 0+}
  \bigl ( - \frac{1}{\pi} \bigr )
  \int _a ^b \mathrm{Im} \hat \mu (x+i \varepsilon ) dx.
$$
\end{The}
{\Large Доказательство.}
Имеем
$$
  \bigl ( - \frac{1}{\pi} \bigr )
  \mathrm{Im} \hat \mu (x+i \varepsilon ) =
  \bigl ( - \frac{1}{\pi} \bigr )
  \mathrm{Im} \int \frac{1}{(x-t)+i \varepsilon } d \mu (t) =
$$
$$
  =\bigl (-\frac{1}{\pi} \bigr )
  \mathrm{Im} \int \frac{(x-t)-i \varepsilon}{(x-t)^2 + \varepsilon ^2}
  d \mu (t) = \frac{1}{\pi} \int
  \frac{\varepsilon}{(x-t)^2 +\varepsilon ^2}d \mu (t).
$$
Следовательно,
$$
  \bigl ( - \frac{1}{\pi} \bigr ) \int _a ^b
  \mathrm{Im} \hat \mu (x+i \varepsilon )dx=
  \int _a ^b dx \frac{1}{\pi} \int
  \frac{\varepsilon}{(x-t)^2 +\varepsilon ^2} d \mu (t) =
  \int A_{\varepsilon}(t) d \mu (t) ,
$$
где обозначено
$$
  A_{\varepsilon}(t) = \frac{1}{\pi} \int _a ^b
  \frac{\varepsilon}{(x-t)^2 +\varepsilon ^2}dx.
$$
Имеем
$$
  A_{\varepsilon}(t)= \frac{1}{\pi} \int _a ^b
  \frac{\frac{dx}{\varepsilon}}{\bigl (
  \frac{x-t}{\varepsilon} \bigr ) ^2 +1 } =
  \frac{1}{\pi} \biggl ( \arctg \bigl (
  \frac{b-t}{\varepsilon} \bigr ) -
  \arctg \bigl ( \frac{a-t}{\varepsilon} \bigr ) \biggr ).
$$
Семейство функций
$ A_{\varepsilon}(t) $
равномерно ограничено:
$$
  |A_{\varepsilon}(t)| \leq 1.
$$
Далее
\begin{equation*}
  \lim _{\varepsilon \rightarrow 0+} A_{\varepsilon}(t) =
    \begin{cases}
	  0, \quad если \; t<a<b \\
	  1/2, \quad если \; t=a<b \\
	  1, \quad если \; a<t<b \\
	  1/2, \quad если \; a<t=b \\
	  0, \quad если \; a<b<t
	\end{cases}
  = \chi _{(a,b)}(t) \quad \mu - почти \; всюду.
\end{equation*}
Здесь
$ \chi _E - $
характеристическая функция множества
$ E . $
По
{\bfseries теореме Лебега о мажорантной сходимости}
$$
  \lim _{\varepsilon \rightarrow 0+}
  \int A_{\varepsilon}(t) d \mu (t) =
  \int \chi _{(a,b)}(t) d \mu (t) =
  \mu ((a,b)).
$$
$ \triangle $
$$ \; $$
						 %%%%%  (b)  Классы аналитических функций   %%%%%%%
{\bfseries (b) Классы аналитических функций}\\

В дальнейшем нам понадобятся так называемые классы
$ \mathcal{C} $
и
$ \mathcal{R} $
аналитических функций.
                           %%%%%%%%%%%   Определение 9   %%%%%%%%%%%%%%%%%%
\begin{Def}
Функция
$ f $
принадлежит классу
$ \mathcal{C}, $
если\\
$ 1^{\circ}. \; f \; $
голоморфна в единичном круге, \\
$ 2^{\circ}. \mathrm{Re} f \geq 0. $
\end{Def}
                           %%%%%%%%%%%%   Теорема 10   %%%%%%%%%%%%%%%%%%%%%
\begin{The}
{\bfseries (Ф.Рисс и Г.Херглотц)}
\\
Функция
$ f $
принадлежит классу
$ \mathcal{C} , $
тогда и только тогда, когда она допускает следующее
интегральное представление:
$$
  f(z)=i \gamma + \int _0 ^{2 \pi }
  \frac{e^{i \theta}+z}{e^{i \theta}-z} d \tau ( \theta ),
$$
где
$ \gamma - $
вещественная постоянная, а
$ \tau - $
неубывающая функция распределения.
\end{The}
{\Large Доказательство.}
\\
{\Large (I) Достаточность.}
Очевидно, что функция
$ f $
голоморфна в единичном круге.\\
Рассмотрим ДЛО
$$
  z \mapsto w=\frac{\zeta +z}{\zeta -z},
$$
где
$ \zeta - $
фиксированная точка на единичной окружности. Если
$ \zeta = e^{i \theta}, \;z=e^{i \varphi}, \; $
то
$$
  w=\frac{1+e^{i(\varphi - \theta )}}
  {1-e^{i(\varphi- \theta )}}=
  -i \ctg \frac{\varphi - \theta}{2}.
$$
Следовательно,единичная окружность переходит в мнимую ось.
Поскольку
$ 0 \mapsto 1 , $
то единичный круг переходит в правую полуплоскость. Итак,
$$
  \mathrm{Re} \frac{\zeta+z}{\zeta -z} >0, \quad когда \; |z|<1.
$$
Тогда, поскольку
$ d \tau - $
положительная мера, то
$ \mathrm{Re} f \geq 0. $
Отметим также, что
$ \gamma =\mathrm{Im} f(0). $
\\
{\Large (II) Необходимость.}
Функция
$ u=\mathrm{Re}f - $
гармоническая в единичном круге. Предположим вначале, что
$ u $
непрерывна в замкнутом круге. Тогда справедлива
{\bfseries формула Пуассона}
$$
  u(z)=\frac{1}{2\pi } \int _0 ^{2\pi} \mathrm{Re}
  \biggl \{ \frac{e^{i \theta} +z}{e^{i \theta} -z}
  \biggr \} u(e^{i \theta})d \theta .
$$
и
{\bfseries формула Шварца}
$$
  f(z)=i \gamma +\frac{1}{2\pi}\int _0 ^{2\pi}
  \frac{e^{i\theta} +z}{e^{i\theta} -z} u(e^{i\theta})d\theta .
$$
В общем случае рассмотрим семейство функций
$$
  f_r (z)=f(rz), \quad 0<r<1.
$$
Для них имеем
$$
  f_r (z)=i\gamma + \int _0 ^{2\pi}
  \frac{e^{i\theta} +z}{e^{i\theta} -z} d \tau _r (\theta ),
  \quad (\ast)
$$
где
$$
  \tau _r (\theta )=\frac{1}{2\pi} \int _0 ^ {\theta}
  u(re^{i\theta})d\theta
$$
неубывающая функция распределения (поскольку
$ u \geq 0). $
По
{\bfseries теореме о среднем}
$$
  \tau _r (2\pi )=\frac{1}{2\pi} \int _0 ^{2\pi}
  u(re^{i\theta})d\theta =
  u(rz)|_{z=0} = u(0).
$$
Следовательно, семейство функций
$ \tau _r $
ограничено. Тогда по теоремам Хелли можно найти последовательность
$ r_n \rightarrow 1- , $
такую, что функции
$ \tau _{r_n}(\theta ) $
сходятся поточечно к некоторой функции распределения
$ \tau (\theta ) . $
Переходя в равенстве
$ (\ast ) $
к пределу по этой последовательности, получим искомое представление.
\\
$ \triangle $
                           %%%%%%%%%%%%%   Определение 10   %%%%%%%%%%%%%%%
\begin{Def}
Функция
$ F $
принадлежит классу
$ \mathcal{R} , $
если\\
$ 1^{\circ}. \; F $
голоморфна в верхней полуплоскости
$ \mathbb{C}_+ , $
\\
$ 2^{\circ}. \; \mathrm{Im} F \geq 0 . $
\end{Def}
                           %%%%%%%%%%%   Теорема 11   %%%%%%%%%%%%%%%%%%%%%
\begin{The}
{\bfseries (Р.Неванлинна)}
Функция
$ F $
принадлежит классу
$ \mathcal{R} , $
тогда и только тогда,когда она допускает следующее
интегральное представление:
$$
  F(z)=\alpha +\beta z + \int _{-\infty}^{+\infty}
  \biggl( \frac{1}{t-z} - \frac{t}{1+t^2} \biggr )
  d \sigma (\tau ) ,
$$
где
$ \alpha - $
вещественное число,
$ \beta \geq 0, \; \sigma - $
неубывающая функция, такая, что сходится интеграл
$$
  \int _{-\infty}^{+\infty}
  \frac{d\sigma (\tau )}{1+t^2} < + \infty .
$$
\end{The}
{\Large Доказательство.}
Мы получим эту теорему из предыдущей заменой переменных.
Пусть
$$
  z \mapsto \zeta =\frac{z-i}{z+i}
$$
ДЛО, конформно отображающее верхнюю полуплоскость на
единичный круг.
$$
  F \mapsto f=-iF
$$
поворот плоскости, переводящий верхнюю полуплоскость в правую.
Определим функцию
$$
  f(\zeta )=-iF(z).
$$
Имеем
$$
  F \in \mathcal{R} \Longleftrightarrow  f \in \mathcal{C}.
$$
Тем самым
$$
  F(z)=-\gamma +i \int _0 ^{2\pi}
  \frac{e^{i\theta}+\zeta}{e^{i\theta}-\zeta}
  d\tau (\theta ).
$$
Имеем
$$
  \frac{e^{i\theta}+\zeta}{e^{i\theta}-\zeta}=
  \frac{e^{i\theta}+\frac{z-i}{z+i}}
  {e^{i\theta}-\frac{z-i}{z+i}}=
  \frac{z ( e^{i\theta}+1)+i(e^{i\theta}-1)}
  {z(e^{i\theta}-1)+i(e^{i\theta}+1)}=
  \frac{z\frac{e^{i\theta}+1}{e^{i\theta}-1}+i}
  {z+i\frac{e^{i\theta}+1}{e^{i\theta}-1}}.
$$
Далее,
$$
  \frac{e^{i\theta}+1}{e^{i\theta}-1}=
  \frac{e^{i\theta /2}+e^{-i\theta /2}}
  {e^{i\theta /2}-e^{-i\theta /2}}=
  \frac{1}{i}\frac{\cos \frac{\theta}{2}}
  {\sin \frac{\theta}{2}}=
  -i\ctg \frac{\theta}{2}.
$$
Сделаем в интеграле монотонную и непрерывную замену переменной
$ t=-\ctg \frac{\theta}{2} , $
отображающую окружность на вещественную ось. Получим
$$
  F(z)=\alpha +i \int _{[-\infty , +\infty ]}
  \frac{zit+i}{z+i \cdot it}d\tau (\theta (t))=
  \alpha +\int _{[-\infty , +\infty ]}
  \frac{1+tz}{t-z}d \tilde \sigma (t).
$$
Учтем возможный скачок функции
$ \tilde \sigma $
на бесконечности:
$$
  \beta = \lim _{t \rightarrow -\infty}
  \tilde \sigma (t) - \tilde \sigma (-\infty ).
$$
Получим
$$
  F(z)=\alpha +\beta z + \int _{-\infty}^{+\infty}
  \frac{1+tz}{t-z} d \tilde \sigma (t).
$$
Учитывая тождество
$$
  \frac{1+tz}{t-z}=
  \biggl ( \frac{1}{t-z}-\frac{t}{1+t^2} \biggr )
  (1+t^2 )
$$
и обозначая
$$
  d \sigma (t)=(1+t^2 )d \tilde \sigma (t),
$$
получим утверждение теоремы.
\\
$ \triangle $
                          %%%%%%%%%%%%%   Замечание  9   %%%%%%%%%%%%%%%%%%
\begin{Rem}
Формула обращения Стилтьеса-Перрона справедлива и для функций класса
$ \mathcal{R} , $
а именно, в точках непрерывности меры имеем
$$
  \sigma (b)-\sigma (a)=\lim _{\varepsilon \rightarrow 0+}
  \frac{1}{\pi}\int _a ^b \mathrm{Im} F(x+i \varepsilon )dx.
$$
\end{Rem}
                            %%%%%%%%%%   Следствие  5   %%%%%%%%%%%%%%%%%%%
\begin{Cor}
Мера
$ \sigma $
в представлении Неванлинны определена единственным образом.
\end{Cor}
                             %%%%%%%%%%%%   Замечание  10  %%%%%%%%%%%%%%%%%
\begin{Rem}
Коэффициент
$ \beta $
вычмсляется по формуле
$$
  \beta = \lim _{y \rightarrow +\infty}
  \frac{\mathrm{Im} F(iy)}{y}.
$$
\end{Rem}
Действительно,
$$
  \mathrm{Im} F(iy) = \beta y + \int _{-\infty}^{+\infty}
  \frac{y}{t^2 +y^2}d \sigma (t).
$$
Поскольку при
$ y \rightarrow +\infty $
функции
$ \frac{1}{t^2 +y^2} $
стремятся к нулю и мажорируются функцией
$ \frac{1}{t^2 +1}, $
то
$$
  \int _{-\infty}^{+\infty}
  \frac{d\sigma (t)}{t^2 +y^2} \rightarrow 0
  \quad (y \rightarrow +\infty ).
$$
\\

Приведем еще одно мультипликативное представление функций класса
$ \mathcal{R} . $
                        %%%%%%%%%%%%%   Теорема   12   %%%%%%%%%%%%%%%%%%%%%
\begin{The}
Если\\
$ 1) \; F \in \mathcal{R} , $ \\
$ 2) \; F \not \equiv 0 , $ \\
то существует и единственно следующее интегральное представление:
$$
  F(z)=C \exp \Biggl \{ \int _{-\infty}^{+\infty}
  \biggl ( \frac{1}{t-z}-\frac{t}{1+t^2} \biggr )
  f(t)dt \Biggr \} ,
$$
где
$ C - $
положительная постоянная,
$ f - $
локально суммируемая функция, такая, что
$$
  0 \leq f \leq 1 \quad п.в.
$$
и сходится интеграл
$$
  \int _{-\infty}^{+\infty}
  \frac{f(t)dt}{1+t^2} < +\infty .
$$
\end{The}
{\Large Доказательство.}
Из интегрального представления Неванлинны следует, что функция
$ F $
из класса
$ \mathcal{R} $
нигде не обращается в ноль за исключением случая, когда
$ F \equiv 0. $
Поэтому по
{\bfseries теореме о монодромии}
определена голоморфная в верхней полуплоскости функция
$$
  \mathrm{Ln} F(z)= \ln |F(z)|+i \arg F(z),
$$
где
$$
  0 \leq \arg F(z) \leq \pi .
$$
Поскольку
$$
  \mathrm{Im} F(z) \geq 0,
$$
то функция
$ \mathrm{Ln} F $
также принадлежит классу
$ \mathcal{R} . $
По теореме Неванлинны справедливо интегральное представление
$$
  \mathrm{Ln} F(z)= \alpha +\beta z + \int _{-\infty}^{+\infty}
  \biggl ( \frac{1}{t-z}-\frac{t}{1+t^2} \biggr )
  d\tau (t) .
$$
Согласно замечанию 9
$ \beta =0, $
поскольку
$ \mathrm{Im \; Ln} F $
ограничена.
По формуле обращения Стилтьеса-Перрона             						      							
$$
  \tau (t_2 )-\tau (t_1 )=\lim _{\varepsilon \rightarrow 0+}
  \frac{1}{\pi} \int _{t_1}^{t_2}
  \arg F(x+i \varepsilon )dx \leq t_2 -t_1 .
$$
Формула справедлива в точках непрерывности меры
$ \tau , $
но из нее следует, что функция
$ \tau $
непрерывна и удовлетворяет
{\bfseries условию Липшица}
, а значит
{\bfseries абсолютно непрерывна}.
Поэтому почти всюду (по мере Лебега) существует производная
$$
  f(t)=\frac{d\tau (t)}{dt},
$$
которая удовлетворяет всем условиям теоремы. Функция
$ \tau $
восстанавливается по своей производной интегрированием.
\\
$ \triangle $
\newpage
                   %%%%%%%%%%%%%%%%%%%%%%%%%%%%%%%%%%%%%%%%%%%%%%%%%%%%%%%%
				   %%%%%%%%%%  2.7  Классы квазианалитичности   %%%%%%%%%%%
				   %%%%%%%%%%%%%%%%%%%%%%%%%%%%%%%%%%%%%%%%%%%%%%%%%%%%%%%%
\subsection{Классы квазианалитичности}
$ \; $
\\

Пусть имеется разрешимая проблема моментов Гамбургера, т.е.
задана позитивная последовательность
$ \mathbf{s} . $
Начиная с этого параграфа нас будет интересовать вопрос об
определенности проблемы моментов. Мы начнем с самого простого
достаточного условия.
                 %%%%%%%%%%%%   Утверждение 12  %%%%%%%%%%%%%%%%%%%%%%%%%%%
\begin{Sta}
Пусть
$ \mathbf{s} - $
позитивная последовательность, и выполнено
{\bfseries условие аналитичности}
$$
  \limsup _{n \rightarrow \infty }
  |s_n |^{1/n} < + \infty .
$$
Тогда проблема моментов определена.
\end{Sta}
                    %%%%%%%%%%%%%%%%%   Замечание 11   %%%%%%%%%%%%%%%%%%%%
\begin{Rem}
Условие аналитичности можно записать в виде
$$
  |s_n | \leq CR^n
$$
для некоторых положительных постоянных
$ C $
и
$ R . $
\end{Rem}
{\Large Доказательство.}
Пусть
$ \mu $
и
$ \rho - $
два решения проблемы моментов. Их преобразования Коши
$ \hat \mu $
и
$ \hat \rho $
суть функции, голоморфные в верхней полуплоскости.
Они раскладываются в один и тот же степенной ряд
$$
  \sum _{k=0}^{\infty}\frac{s_k}{z^{k+1}},
$$
сходящийся в окрестности бесконечности. Следовательно,
$ \hat \mu = \hat \rho $
в окрестности бесконечности. По теореме
единственности для голоморфных функций
эти функции совпадают во всей верхней полуплоскости.
По формуле обращения Стилтьеса-Перрона
$ \mu = \rho .$
$ \triangle $
\\

Если условие аналитичности не выполнено, то функция
$ \hat \mu $
не будет голоморфна в окрестности бесконечности, и мы не
можем воспользоваться теоремой единственности. Те классы
функций, в которых функция, тем не менее, однозначно
определяются своим асимптотическим разложением, называются
{\bfseries классами квазианалитичности}.
\\
Приведем одно простое, но достаточно широкое условие
квазианалитичности.
                               %%%%%%%%%%%   Теорема  13   %%%%%%%%%%%%%%%%
\begin{The}
Пусть
$ \mathbf{s} - $
позитивная последовательность, и выполнено следующее
{\bfseries условие квазианалитичности}
$$
  |s_n | \leq C R^n n!
$$
для некоторых положительных постоянных
$ C $
и
$ R . $
Тогда проблема моментов определена.
\end{The}
{\Large Доказательство.}
Пусть
$ \mu - $
произвольное решение проблемы моментов. Рассмотрим
{\bfseries интеграл Фурье}
$$
  F_{\mu}(z)=\int _{-\infty}^{+\infty}
  e^{izx}d\mu (x), \quad z \in \mathbb{R}.
$$
Рассмотрим функцию
$$
  \ch(\alpha x), \quad \alpha \in \mathbb{R}.
$$
Тейлоровские многочлены этой функции сходятся к ней
поточечно и монотонно возрастая:
$$
  \sum _{n=0}^N \frac{(\alpha x)^{2n}}{(2n)!}
  \nearrow \ch (\alpha x), \quad N \rightarrow \infty
  \quad x \in \mathbb{R} .
$$
Далее,
$$
  \int _{-\infty}^{+\infty} \sum _{n=0}^N
  \frac{(\alpha x)^{2n}}{(2n)!}d\mu (x)=
  \sum _{n=0}^N\frac{\alpha ^{2n}}{(2n)!}s_{2n} \leq
  \sum _{n=0}^N \frac{\alpha ^{2n}}{(2n)!}CR^{2n}(2n)!=
$$
$$
  =C \sum _{n=0}^N (\alpha R)^{2n} \leq
  C \sum _{n=0}^{\infty} (\alpha R)^{2n}=
  \frac{C}{1-\alpha ^2 R^2} <+\infty ,
$$
если
$ | \alpha |<\frac{1}{R} . $
По
{\bfseries теореме Б.Леви о монотонной сходимости}
функция
$ \ch (\alpha x) $
интегрируема по мере
$ \mu . $
Поскольку
$$
  e^{\alpha x} \leq 2 \ch (\alpha x),
$$
то функция
$ e^{\alpha x} $
также интегрируема. Следовательно, интеграл Фурье сходится
не только для вещественных
$ z , $
но и для комплеасных
$ z , $
лежащих в полосе
$$
  \Pi = \{ z \in \mathbb{C} \; | \; |\mathrm{Im} z |<\frac{1}{R} \} .
$$
По
{\bfseries теореме Вейерштрасса о мажорантной сходимости}
интеграл сходится равномерно на компактах.
По
{\bfseries первой теореме Вейерштрасса}
функция
$ F_{\mu} $
голоморфна в полосе
$ \Pi . $
\\
Далее,
$$
  \frac{d^n}{dz^n}F_{\mu}(z)|_{z=0}=
  \int _{-\infty}^{+\infty}(ix)^n d\mu (x)=
  i^n s_n , \quad n=0, \; 1, \; 2,...
$$
Если
$ \mu $
и
$ \rho - $
два решения проблемы моментов, то ряды Тейлора функций
$ F_{\mu } $
и
$ F_{\rho} $
совпадают, т.е. эти функции совпадают в некоторой окрестности нуля,
а именно, при
$ |z|<1/R . $
По теореме единственности для голоморфных функций
$ F_{\mu}=F_{\rho} $
во всей полосе
$ \Pi , $
и в частности, на вещественной оси.
\\
Рассмотрим
{\bfseries одностороннее преобразование Фурье}
(при
$ z \in \mathbb{C}_+ ) $
$$
  (-i)\int _{-\infty}^0 e^{-iyz}F_{\mu}(y)dy=
  (-i)\int _{-\infty}^0 e^{-iyz}dy
  \int _{-\infty}^{+\infty} e^{iyx}d\mu (x)=
$$
$$
  =\int _{-\infty}^{+\infty} d\mu (x)
  (-i)\int _{-\infty}^0 e^{iy(x-z)}dy=
  \int _{-\infty}^{+\infty} d\mu (x)
  \frac{(-i)e^{iy(x-z)}}{i(x-z)} \Bigg | _{y=-\infty}^{y=0}=
$$
$$
  =\int _{-\infty}^{+\infty} \frac{d\mu (x)}{z-x}=
  \hat \mu (z).
$$
Следовательно,
$ \hat \mu = \hat \rho $
в верхней полуплоскости. Тогда по формуле обращения
Стилтьеса-Перрона
$ \mu = \rho . $
$ \triangle $
                      %%%%%%%%%%%%%   Пример   3   %%%%%%%%%%%%%%%%%%%%%%%%%
\begin{Exa}
Рассмотрим проблему моментов, порожденную
{\bfseries весом Фройда}
$$
  e^{-|x|^{\alpha}}, \quad x \in \mathbb{R} \quad (\alpha >0).
$$
\end{Exa}
Вычислим степенные моменты. Поскольку весовая функция четная, то
$$
  s_1 =s_3 =s_5 =...=0.
$$
Далее,
$$
  s_{2n}=2 \int _0 ^{\infty} x^{2n} e^{-x^{\alpha}}dx=
  \bigl [ \; x^{\alpha}=t \; \bigr ] =
$$
$$
  =2 \int _0 ^{\infty} t^{\frac{2n}{\alpha}}e^{-t}
  \frac{1}{\alpha} t^{\frac{1}{\alpha}-1}dt=
  \frac{2}{\alpha} \Gamma \biggl (
  \frac{2n+1}{\alpha} \biggr ) .
$$
Например, при
$ \alpha =1 $
имеем
$ s_{2n}=2(2n)! , $
и условие квазианалитичности выполнено. При
$ \alpha >1 $
оно тем более выполнено. Таким образом, при
$ \alpha \geq 1 $
проблема моментов определена. Если же
$ 0<\alpha <1 , $
то условие квазианалитичности не выполнено. Вопрос об
определенности или неопределенности проблемы моментов
в этом случае будет решен в следующем параграфе.
\newpage
                        %%%%%%%%%%%%%%%%%%%%%%%%%%%%%%%%%%%%%%%%%%%%%%%%%%%%
						%%%%%%%%%   2.8   Критерий Крейна   %%%%%%%%%%%%%%%%
						%%%%%%%%%%%%%%%%%%%%%%%%%%%%%%%%%%%%%%%%%%%%%%%%%%%%
\subsection{Критерий Крейна}
$ \; $
\\

В этом параграфе мы докажем одно достаточное условие
неопределенности проблемы моментов.
                          %%%%%%%%%%%%%   Теорема  14   %%%%%%%%%%%%%%%%%%
\begin{The}
{\bfseries (М.Крейн)}
\\
Пусть
$ g(x) - $
неотрицательная измеримая функция, такая, что сходится интеграл
$$
  \int _{-\infty}^{+\infty}
  \frac{g(x)}{1+x^2}dx<+\infty .
$$
Положим
$ F(x)=e^{-g(x)} $
и определим абсолютно непрерывную меру
$ d\mu (x)=F(x)dx . $
Потребуем, чтобы эта мера имела конечные степенные моменты
$$
  s_n = \int _{-\infty}^{+\infty} x^n d\mu (x),
  \quad n=0, \; 1, \; 2,...
$$
Тогда проблема моментов Гамбургера с этими моментами неопределена.
\end{The}						
{\Large Доказательство.}
Рассмотрим функцию Неванлинны
$$
  Q(z)=\frac{1}{\pi} \int _{-\infty}^{+\infty}
  \biggl ( \frac{1}{x-z}-\frac{x}{1+x^2} \biggr )
  g(x)dx.
$$
Она принадлежит классу
$ \mathcal{R} . $
Определим функцию
$$
  G(z)=e^{iQ(z)}.
$$
Функция
$ G $
голоморфна в верхней полуплоскости и принимает значения в
замкнутом единичном круге:
$$
  |G(z)| \leq 1, \quad z \in \mathbb{C}_+ .
$$
Из теории граничных значений аналитических функций следует,
что почти всюду существует предел
$$
  Q(x)= \lim _{\varepsilon \rightarrow 0+}
  Q(x+i \varepsilon ),
$$
при этом
$$
  \mathrm {Im} Q(x)=g(x), \quad п.в. \; x \in \mathbb{R} .
$$
Тогда
$$
  G(x)=e^{iQ(x)}=
  \lim _{\varepsilon \rightarrow 0+}
  G(x+i \varepsilon ),
  \quad п.в. \; x \in \mathbb{R} ,
$$
и
$$
  |G(x)|=F(x), \quad п.в. \; x \in \mathbb{R} .
$$
Докажем, что все степенные моменты функции
$ G $
равны нулю:
$$
  \int _{-\infty}^{+\infty}
  x^n G(x)dx=0, \quad n=0, \; 1, \; 2,...
  \quad (\ast )
$$
\\

Зафиксируем числа
$ 0<\delta <R<+\infty $
и рассмотрим в верхней полуплоскости замкнутый контур
$ C_{R, \delta} , $
который является границей кругового сегмента,
отсекаемого от круга
$ |z|<R $
прямой
$ \mathrm {Im}z=\delta . $
Зафиксируем число
$ \varepsilon >0. $
Тогда по
{\bfseries интегральной теореме Коши}
$$
  \int _{C_{R,\delta}}
  \frac{z^n}{(1-i\varepsilon z)^{n+2}}G(z)dz=0.
$$
Перейдем в этом равенстве к пределу при
$ \delta \rightarrow 0+ $
(используя теорему Лебега о мажорантной сходимости). Получим
$$
  \int _{-R}^{+R} \frac{x^n}{(1-i\varepsilon x)^{n+2}}G(x)dx+
  \int _{\gamma _R} \frac{z^n}{(1-i\varepsilon z)^{n+2}}
  G(z)dz=0,
$$
где
$ \gamma _R - $
полуокружность радиуса
$ R , $
лежащая в верхней полуплоскости. Перейдем в этом равенстве к пределу
при
$ R \rightarrow +\infty $
(используя ограниченность функции
$ G . ) $
Получим
$$
  \int _{-\infty}^{+\infty}
  \frac{x^n}{(1-i\varepsilon x)^{n+2}}G(x)dx=0.
$$
Перейдем в этом равенстве к пределу при
$ \varepsilon \rightarrow 0+ $
(используя теорему Лебега и существование степенных моментов).
Получим
$ (\ast ) . $
Таким образом, имеем
$$
  \int _{-\infty}^{+\infty}x^n \mathrm{Re}G(x)dx=0 \; и \;
  \int _{-\infty}^{+\infty}x^n \mathrm{Im}G(x)dx=0
  \quad n=0, \; 1, \; 2,...
$$
По крайней мере одна из функций
$ \mathrm{Re}G(x) $
и
$ \mathrm{Im}G(x) $
не равна нулю на множестве положительной меры. Пусть это будет
для определенности
$ \mathrm{Re}G(x) . $
Тогда
$$
  F(x)dx \; и \; (F(x)+\mathrm{Re}G(x))dx
$$
это две разные меры из класса
$ \mathcal{M} $
и они имеют одинаковые степенные моменты.
\\
$ \triangle $
                          %%%%%%%%%%%%   Пример 4   %%%%%%%%%%%%%%%%%%%%%%%
\begin{Exa}
Пусть
$ F(x)=e^{-|x|^{\alpha}} , $
где
$ 0<\alpha <1. $
\end{Exa}
Тогда
$ g(x)=|x|^{\alpha} $
и
$$
  \int _{-\infty}^{+\infty}
  \frac{|x|^{\alpha}}{1+x^2}dx<+\infty .
$$
Следовательно, проблема моментов с этим весом неопределена.
\newpage
                       %%%%%%%%%%%%%%%%%%%%%%%%%%%%%%%%%%%%%%%%%%%%%%%%%%%%%
					   %%%%%%%%   2.9  Граничные значения   %%%%%%%%%%%%%%%%
					   %%%%%%%%%   аналитических функций    %%%%%%%%%%%%%%%%
					   %%%%%%%%%%%%%%%%%%%%%%%%%%%%%%%%%%%%%%%%%%%%%%%%%%%%%
\subsection{Граничные значения аналитических функций}
						%%%%%%%%   (a) Постановка задачи   %%%%%%%%%%%%%%%%%
{\bfseries (a) Постановка задачи.}  \\

Цель этого параграфа состоит в доказательстве следующей теоремы.
                     %%%%%%%%%%%%   Теорема  15   %%%%%%%%%%%%%%%%%%%%%%%%%%
\begin{The}
{\bfseries ( $ \mathcal{R} $ ) }
\\
Если
$ F \in \mathcal{R} , $
то для почти всех
$ x \in \mathbb{R} $
существует конечный предел
$$
  F(x)=\lim _{y \rightarrow 0+} F(x+iy).
$$
Если при этом
$$
  F(z)= \frac{1}{\pi} \int _{-\infty}^{+\infty}
  \biggl ( \frac{1}{t-z}-\frac{t}{1+t^2} \biggr )
  d\sigma (t)
$$
представление Неванлинны, то
$$
  \mathrm{Im}F(x)=\sigma ^{\prime}(x)
  \quad для \; п.в. \; x \in \mathbb{R}.
$$
\end{The}
Эта теорема равносильна следующей.
                          %%%%%%%%   Теорема 16   %%%%%%%%%%%%%%%%%%%%%%%%%
\begin{The}
{\bfseries ( $ \mathcal{C} $ )}
\\
Если
$ f \in \mathcal{C} , $
то для почти всех
$ \theta \in [-\pi , \pi ] $
существует конечный предел
$$
  h(\theta )=\lim _{r \rightarrow 1-}
  f(re^{i\theta }).
$$
Если при этом
$$
  f(z)=\frac{1}{2\pi}\int _{-\pi}^{\pi}
  \frac{e^{it}+z}{e^{it}-z}d \mu (t)
$$
представление Рисса-Херглотца, то
$$
  \mathrm{Re}h(\theta )= \mu ^{\prime }(\theta )
  \quad для \; п.в. \; \theta \in [-\pi ,\pi ] .
$$
\end{The}
Каждая из этих теорем получается из другой дробно-линейным
преобразованием. При доказательстве теоремы Крейна мы
ссылались на теорему
$ ( \mathcal{R} ) . $
Доказывать будем теорему
$ ( \mathcal{C} ) . $
\\
					 %%%%%%%%%%%%%%%%   (b)   Гармонические функции   %%%%
{\bfseries (b) Гармонические функции.}
                     %%%%%%%%%%%%%%%%%   Определение  11   %%%%%%%%%%%%%%%
\begin{Def}
Пусть
$ u=u(x,y) - $
вещественная функция двух вещественных переменных. Функция
$ u $
называется
{\bfseries гармонической}
в области
$ D , $
если
\\
1) функция
$ u $
дважды непрерывно дифференцируема в области
$ D , $
\\
2) функция
$ u $
удовлетворяет в области
$ D $
{\bfseries уравнению Лапласа}:
$$
  \Delta u := \frac{\partial ^2 u}{\partial x^2}+
  \frac{\partial ^2 u}{\partial y^2} =0.
$$
\end{Def}
  					    %%%%%%%%%%%  Утверждение 13   %%%%%%%%%%%%%%%%%%%%%
\begin{Sta}
Если функция
$ f=u+iv $
голоморфна в области
$ D , $
то функции
$ u $
и
$ v $
гармонические в этой области.
\end{Sta}
{\Large Доказательство.}
Следует из бесконечной дифференцируемости голоморфных
функций и
{\bfseries условий Коши-Римана}.
\\
$ \triangle $
                              %%%%%%%%   Утверждение 14   %%%%%%%%%%%%%%%%
\begin{Sta}
Если функция
$ u $
гармоническая в
{\bfseries односвязной области}
$ D , $
то существует функция
$ f=u+iv $
голоморфная в этой области.
\end{Sta}
{\Large Доказательство.}
Следует из условий Коши-Римана и
{\bfseries формулы Грина}.
\\
$ \triangle $
                                %%%%%%%%%%%   Определение 12   %%%%%%%%%%%%%
\begin{Def}
Гармонические функции
$ u $
и
$ v , $
связанные условиями Коши-Римана, называются
{\bfseries сопряженными гармоническими.}
\end{Def}
                                  %%%%%%%%%%%   Следствие 6   %%%%%%%%%%%%%%
\begin{Cor}
$ \; $\\
1) Гармонические функции бесконечно дифференцируемы. \\
2) Для гармоничнских функций справедлив принцип
максимума и принцип минимума.
\end{Cor}
{\Large Доказательство.}
Следует из принципа максимума модуля для функции
$ e^{\pm (u+iv)}. $
\\
$ \triangle $
\\

{\bfseries Постановка задачи Дирихле}
                      %%%%%%%%%%%%%%   Задача 3   %%%%%%%%%%%%%%%%%%%%%%%%
\begin{Pro}
{\Large Дано:}
$ D - $
ограниченная область с границей
$ \Gamma , $
$ h: \Gamma \longrightarrow \mathbb{R} - $
функция, непрерывная на
$ \Gamma . $
\\
{ \Large Найти:}
функцию
$ u:\bar D \longrightarrow \mathbb{R} , $
такую, что
\\
1) функция
$ u $
гармоническая в области
$ D , $
\\
2) функция
$ u $
непрерывна в замыкании
$ \bar D , $
\\
3) $ u=h $ на $ \Gamma . $
\end{Pro}
                    %%%%%%%%%%%%%%   Утверждение 15   %%%%%%%%%%%%%%%%%%%%
\begin{Sta}
Если задача Дирихле имеет решение, то решение единственно.
\end{Sta}
{\Large Доказательство.}
Следует из принципа максимума и минимума.
\\
$ \triangle $
\\
                     %%%%%%%%%  (c) Ядро Пуассона   %%%%%%%%%%%%%%%%%%%%%%
{\bfseries (c) Ядро Пуассона}
                     %%%%%%%%%%%%%%%%   Определение 13   %%%%%%%%%%%%%%%%%%
\begin{Def}
{\bfseries Ядром Пуассона}
называется следующая функция
$$
  P_r (\theta )=
  \frac{1-r^2}{1-2r\cos \theta +r^2},
  \quad \theta \in \mathbb{R},
  \quad 0<r<1.
$$
\end{Def}
Отметим очевидные свойства этой функции.
Функция
$ P_r (\theta ) - $
четная, непрерывная,
$ 2\pi-$
периодическая на вещественной оси.
Далее,
$$
  P_r (0)=\frac{1+r}{1-r} \rightarrow +\infty ,
  \quad r \rightarrow 1-
$$
и
$$
  P_r (\theta ) \rightarrow 0+ ,
  \quad r \rightarrow 1-  , \quad
  \theta \in (0, \pi ] .
$$
И наконец, функция
$ P_r (\theta ) $
строго монотонно убывает на отрезке
$ [0,\pi ]. $
                           %%%%%%%%%%%%   Замечание 12   %%%%%%%%%%%%%%%%%%
\begin{Rem}
Справедлива формула
$$
  P_r (\theta )= \mathrm{Re}\frac{1+z}{1-z} ,
  \quad z=re^{i\theta}.
$$
\end{Rem}
Действительно,
$$
 \mathrm{Re} \frac{(1+re^{i\theta})(1-re^{-i\theta})}
 {(1-re^{i\theta})(1-re^{-i\theta})}=
 \mathrm{Re}\frac{(1-r^2)+2ri \sin \theta}
 {1-2r \cos \theta +r^2}=P_r (\theta ).
$$
                             %%%%%%%%%%%%   Следствие  7   %%%%%%%%%%%%%%%%
\begin{Cor}
Функция
$$
  z=re^{i\theta} \mapsto P_r (\theta )
$$
гармоническая в единичном круге.
\end{Cor}
                             %%%%%%%%%%%%   Замечание 13   %%%%%%%%%%%%%%%%
\begin{Rem}							
Ядро Пуассона есть сумма следующего тригонометрического ряда
$$
  P_r (\theta )=\sum _{n=-\infty}^{+\infty}
  r^{|n|}e^{in\theta}=1+2\sum_{n=1}^{\infty}
  r^n \cos n\theta .
$$
\end{Rem}
Действительно,
$$
  \mathrm{Re} \Biggl ( 1+2\sum_{n=1}^{\infty}z^n \Biggr ) =
  \mathrm{Re} \Biggl (1+\frac{2z}{1-z} \Biggr )=
  \mathrm{Re}\frac{1+z}{1-z}=P_r (\theta ).
$$
                         %%%%%%%%%%%%   Утверждение 16   %%%%%%%%%%%%%%%%%
\begin{Sta}
Справедливы сдедующие свойства ядра Пуассона
\\
$ 1^{\circ}. \; P_r (\theta )>0. $
\\
$ 2^{\circ}. \; \frac{1}{2\pi}\int_{-\pi}^{+\pi}
P_r (\theta )d\theta =1. $
\\
$ 3^{\circ}. \; \forall \delta \in (0, \pi )
\qquad \frac{1}{2\pi}\int_{-\delta} ^{ +\delta}
P_r(\theta ) d\theta \longrightarrow 1,
\qquad r \rightarrow 1- . $
\end{Sta}
{\Large Доказательство.}
Свойство
$ 1^{\circ} $
следует из определения.
\\
Свойство
$ 2^{\circ} $
получается почленным интегрированием тригонометрического ряда.
\\
Докажем свойство
$ 3^{\circ}: $
$$
  \int_{\delta}^{\pi}P_r (\theta )d\theta
  \leq P_r (\delta ) \pi \rightarrow 0,
  \quad r \rightarrow 1-.
$$
$ \triangle $
                          %%%%%%%%%%%%%   Замечание 14   %%%%%%%%%%%%%%%%%%%
\begin{Rem}
Свойства
$ 1^{\circ}, \; 2^{\circ}, \; 3^{\circ} $
означают, что
$ P_r (\theta ) - $
{\bfseries $ \delta - $ образное семейство функций
(аппроксимативная единица)}
по базе
$ r \rightarrow 1- . $
Другими широко известными примерами аппроксимативных единиц
яляются
{\bfseries фундаментальное решение уравнения теплопроводности}
и
{\bfseries ядро Фейера.}
\end{Rem}
                         %%%%%%%%%   Определение 14   %%%%%%%%%%%%%%%%%%%%
\begin{Def}
Пусть
$ h(\theta )- $
функция, суммируемая на отрезке
$ [-\pi , +\pi ] , $
тогда существует интеграл Лебега
$$
  u(z)=\frac{1}{2\pi}\int _{-\pi}^{+\pi}
  P_r (\theta -t )h(t)dt =
  \frac{1}{2\pi} \int _{-\pi}^{+\pi}
  P_r (t)h(\theta -t ) dt , \quad (1)
$$
где
$ z=re^{i\theta} . $
Этот интеграл называется
{\bfseries интегралом Пуассона функции
$ h $ }
и обозначается как свертка
$$
  u=P_r \ast h .
$$
\end{Def}
                        %%%%%%%%%%%%   Определение 15  %%%%%%%%%%%%%%%%%%%
\begin{Def}
Пусть
$ d\mu - $
заряд на отрезке
$ [-\pi , +\pi ] \; ( \mu -  $
функция ограниченной вариации), тогда существует интеграл
Римана-Стилтьеса
$$
  u(z)=\frac{1}{2\pi}\int _{-\pi}^{+\pi}
  P_r (\theta -t )d\mu (t) . \quad (2)
$$
Он называется
{\bfseries интегралом Пуассона заряда
$ d\mu $ }
и обозначается как свертка
$$
  u=P_r \ast d\mu .
$$
\end{Def}
Интеграл (1) это частный случай интеграла (2).
Действительно, достаточно положить
$$
  \mu (\theta )=\int _0 ^{\theta} h(t)dt.
$$
                       %%%%%%%%%%   Замечание 15   %%%%%%%%%%%%%%%%%%%%%%
\begin{Rem}
Интеграл Пуассона -- функция, гармоническая в единичном круге.
\end{Rem}
Действительно,
$$
  u(z)= \mathrm{Re} \frac{1}{2\pi}\int _{-\pi}^{+\pi}
  \frac{e^{it}+z}{e^{it}-z}d\mu (t)
$$
вещественная часть интеграла типа Коши, т.е. голоморфной функции.
\\
                 %%%%%%%%%%  (d)  Формула Пуассона   %%%%%%%%%%%%%%%%%%%%%
{\bfseries (d) Формула Пуассона}
                 %%%%%%%%%%%   Теорема 17   %%%%%%%%%%%%%%%%%%%%%%%%%%%%%%%%
\begin{The}
{\bfseries (Пуассон)}
\\
Задача Дирихле для круга всегда разрешима. А именно, если
$ h(\theta ) - $
вещественная, непрерывная,
$ 2\pi - $
периодическая фыункция на
$ \mathbb{R} , $
то решение задачи Дирихле с граничными данными
$ u(e^{i\theta})=h(\theta ) $
дает
{\bfseries формула Пуассона}
$$
  u=P_r \ast h.
$$
\end{The}
{\Large Доказательство.}
Необходимо доказать, что для любого
$ \theta \in \mathbb{R} $
существует предел
$$
  \lim _{\substack{z \rightarrow e^{i\theta}\\ z \in \mathbb{D}}}
   u(z)=h(\theta ).
$$
Мы докажем, что этот предел достигается равномерно по
$ \theta . $
\\
По свойству
$ 2^{\circ} $
ядра Пуассона имеем
$$
  h(\theta )=\frac{1}{2\pi}\int _{-\pi}^{+\pi}
  P_r (t)h(\theta )dt.
$$
Следовательно,
$$
  | u(re^{i \varphi})-h(\theta )| \leq \frac{1}{2\pi}
  \int _{-\pi}^{+\pi} |h(\varphi -t)-h(\theta )|P_r (t)dt.
$$
(мы также воспользовались свойством
$ 1^{\circ} ) . $
По теореме Кантора функция
$ h $
равномерно непрерывна на
$ \mathbb{R} , $
т.е.
$$
  \forall \varepsilon >0 \; \; \exists \delta \in (0,\pi ) \; :
  |t^{\prime}-t^{\prime \prime }|<2 \delta \; \Rightarrow \;
  |h(t^{\prime})-h(t^{\prime \prime})|< \varepsilon .
$$
Пусть
$$
  |\varphi - \theta |< \delta . \qquad (a)
$$
Разобьем интеграл на два
$$
  \frac{1}{2\pi}\int _{-\delta}^{+\delta} +
  \frac{1}{2\pi} \int _{[-\pi , +\pi ] \setminus [-\delta ,+\delta ]}
  =(I) + (II).
$$
Оценим первый интеграл
$$
  (I) \leq \frac{1}{2\pi}\int _{-\delta }^{+\delta}
  \varepsilon P_r (t)dt \leq \varepsilon \frac{1}{2\pi}
  \int _{-\pi}^{+\pi}P_r (t)dt =\varepsilon
$$
(мы снова воспользовались свойствами
$ 1^{\circ}, 2^{\circ} $
ядра Пуассона).
\\
По теореме Вейерштрасса функция
$ h $
ограничена на
$ \mathbb{R} : $
$$
  |h(t)| \leq M, \qquad t \in \mathbb{R} .
$$
Оценим второй интеграл
$$
  (II) \leq \frac{2M}{\pi} \int _{\delta}^{\pi}
  P_r (t) dt \rightarrow 0 , \qquad r \rightarrow 1- ,
$$
(по свойству
$ 3^{\circ} ).$
Таким образом,
$ \exists r_{\ast} \in (0,1): $
$$
  \forall r \in (r_{\ast},1) \qquad (b)
$$
выполняется неравенство
$ (II) \leq \varepsilon . $
\\
Итак, для любого
$ \varepsilon >0 $
мы построили окрестность
$ U_{\varepsilon} $
точки
$ e^{i\theta} ,$
которая определяется неравенствами (a) и (b) (причем
$ \delta $ и $ r_{\ast} $
не зависят от
$ \theta ), $
такую, что
$$
  \forall z \in U_{\varepsilon} \quad
  |u(re^{i\varphi})-h(\theta )|<2\varepsilon .
$$
$ \triangle $
\\
                      %%%%%%%%%  (e)  Теорема Фату   %%%%%%%%%%%%%%%%%%
{\bfseries (e) Теорема Фату}
                       %%%%%%%%%%%%%%   Определение 16   %%%%%%%%%%%%%%%%
\begin{Def}
Пусть точка
$ z $
принадлежит единичному кругу
$ \mathbb{D} , $
$$
  e^{i\varphi _0} \in \mathbb{T} =
  \partial \mathbb{D} .
$$
Будем говорить, что точка
$ z $
стремится к
$ e^{i\varphi _0} $
{\bfseries некасательным образом}
и записывать это следующим образом
$$
  z \twoheadrightarrow e^{i\varphi _o},
$$
если
$$
  z=re^{i\theta} \longrightarrow e^{i\varphi _0}
$$
и при этом
$ z $
принадлежит любому, но фиксированному, сектору
$$
  |\arg (e^{i\varphi _0}-z)-\varphi _0 | \leq \varepsilon <
  \frac{\pi}{2}.
$$
Другими словами, с некоторой положительной постоянной
$ c $
выполняется неравенство
$$
  |\theta - \varphi _0 | \leq c(1-r).
$$
\end{Def}
                             %%%%%%%%%%%   Теорема 18   %%%%%%%%%%%%%%%%%%
\begin{The}
{\bfseries (Фату)}
\\
Пусть
$ \mu - $
функция ограниченной вариации, и в точке
$ \varphi _0 $
существует конечная производная
$ \mu ^{\prime}(\varphi _0 ). $
Тогда
$$
  \biggl ( P_r \ast d\mu \biggr )
  (r e^{i\theta}) \longrightarrow
  \mu ^{\prime}(\varphi _0 )
  \qquad при \; \; re^{i\theta}
  \twoheadrightarrow e^{i\varphi _0}.
$$
\end{The}
                            %%%%%%%%   Следствие 8   %%%%%%%%%%%%%%%%%%%%%
\begin{Cor}
Для п.в.
$ \varphi \in [-\pi , +\pi ] $
имеем
$$
  \biggl ( P_r \ast d\mu \biggr )
  (re^{i\varphi}) \longrightarrow
  \mu ^{\prime}(\varphi )
  \qquad при \; \; r \rightarrow 1- .
$$
\end{Cor}
{\Large Доказательство.}
Можно считать, что
$ \varphi _0 =0, \; \mu (0)=0. $
Пусть без ограничения общности
$ \mu ^{\prime}(0)=0. $
По определению производной
$$
  \forall \varepsilon >0 \; \exists \delta >0 \quad
  |t| \leq \delta \Rightarrow
  |\mu (t)| \leq \varepsilon |t|.
$$
Для всех
$ r $
близких к
$ 1 $
имеем
$ 2|\theta |<\delta . $
Пусть для определенности
$ \theta >0. $
Тогда
$$
  \frac{1}{2\pi} \int _{[-\pi , -\delta ] \cup [\delta , \pi]}
  \frac{1-r^2}{1-2r\cos (\theta -t)+r^2}d\mu (t)
  \longrightarrow 0 \qquad при \; r \rightarrow 1-.
$$
Остается оценить интеграл
$$
  \frac{1}{2\pi}\int _{-\delta}^{+\delta}
  \frac{1-r^2}{1-2r\cos (\theta -t)+r^2}d\mu (t).
$$
Проинтегрируем по частям. Получим
$$
  \frac{1}{2\pi}\mu (t)
  \frac{(1-r^2 )2r \sin (\theta -t)}
  {\bigl (1-2r\cos (\theta -t)+r^2 \bigr )^2}
  \Biggr | _{-\delta}^{+\delta}-
  \frac{1}{2\pi}\int _{-\delta}^{+\delta}
  \frac{(1-r^2 )2r \sin (\theta -t)}
  {\bigl (1-2r\cos (\theta -t)+r^2 \bigr )^2}
  \mu (t)dt.
$$
Внеинтегральный член стремится к нулю при
$ r \rightarrow 1- .$
Оставшийся интеграл запишем в виде
$$
  \int _{-\delta}^0 + \int _0 ^{2\theta} +
  \int _{2\theta}^{\delta} = I +II +III.
$$
Оценим второй интеграл
$$
  |II| \leq \frac{1}{2\pi}\int _0 ^{2\theta}
  \frac{2(1-r^2 )r}{(1-r)^4}\theta \varepsilon tdt=
  \frac{1}{2\pi}\frac{4}{(1-r)^3}\theta \varepsilon
  \frac{(2\theta)^2}{2}=
  \frac{4\theta ^3 \varepsilon}{\pi (1-r)^3} \leq
$$
$$
  \leq \frac{4c^3 \varepsilon }{\pi} = \mathrm{const} \varepsilon .
$$
Оценим третий интеграл
$$
  |III| \leq \frac{1}{2\pi} \int _{2\theta}^{\delta}
  \frac{2r(1-r^2 ) \sin (t-\theta )}
  {\bigl ( 1-2r \cos (t-\theta ) +r^2 \bigr ) ^2}
  2\varepsilon (t-\theta )dt=
$$
$$
  =\frac{\varepsilon}{\pi}\int _{\theta}^{\delta -\theta}
  \frac{2r(1-r^2 )\sin t}{\bigl (1-2r\cos t +r^2 \bigr ) ^2}tdt
  \leq \frac{\varepsilon}{\pi}\int _0 ^{\pi}
  \frac{2r(1-r^2 ) \sin t}
  {\bigl ( 1-2r\cos t +r^2 \bigr ) ^2}tdt.
$$
Проинтегрируем по частям. Получим
$$
  -\frac{\varepsilon}{\pi}t
  \frac{1-r^2}{1-2r\cos t +r^2 }
  \Biggr | _0 ^{\pi} + \frac{\varepsilon}{\pi}\int _0 ^{\pi}
  \frac{1-r^2}{1-2r\cos t +r^2 }dt.
$$
Внеинтегральный член стремится к нулю при
$ r \rightarrow 1- , $
а второе слагаемое равно
$ \varepsilon . $
Аналогичным образом оценивается первый интеграл
$$
  |I| \leq \frac{\varepsilon }{2} + o (1)
  \qquad при \; r \rightarrow 1- .
$$
$ \triangle $
\\
               %%%%%%%%%%  (f)  ограниченные гармонические функции   %%%%
{\bfseries (f) Ограниченные гармонические функции}
               %%%%%%%%%%%   Теорема  19   %%%%%%%%%%%%%%%%%%%%%%%%%%%%%%%%
\begin{The}
Если\\
1) функция
$ u $
гармоническая в единичном круге, \\
2) функция
$ u $
ограниченная в единичном круге, \\
то существует функция
$ h \in L^{\infty}(-\pi , +\pi ), $
такая, что
$ u=P_r \ast h . $
\end{The}
                     %%%%%%%%%%%%%%   Следствие 9   %%%%%%%%%%%%%%%%%%%%%%%%
\begin{Cor}
Ограниченная гармоническая в круге функция имеет некасательные
конечные пределы в почти всех точках граничной окружности.
\end{Cor}
{\Large Доказательство.}
Для функции
$$
  u_s (z)=u(sz), \qquad 0 \leq s <1,
$$
напишем формулу Пуассона
$$
  u_s (re^{i\theta})=P_r (\theta ) \ast u_s (\theta ).
  \qquad ( \alpha )
$$
Все функции
$ u_s (e^{i\theta} ) $
принадлежат пространству
$ L^{\infty} $
и по условию
$$
  ||u_s ||_{\infty} \leq M.
$$
Пространство
$ L^{\infty} $
является сопряженным к пространству
$ L^1 . $
По
{\bfseries теореме о компактности шаров в
$ \ast - $
слабой топологии}
можно построить последовательность
$ s_n \nearrow 1 , $
такую, что
$$
  u_{s_n} \rightharpoonup h.
$$
Тогда для любой функции
$ g $
из
$ L^1 $
имеем
$$
  \frac{1}{2\pi}\int _{-\pi}^{+\pi}
  g(t)u_{s_n}(e^{it})dt \longrightarrow
  \frac{1}{2\pi}\int _{-\pi}^{+\pi}
  g(t)h(t)dt.
$$
Возьмем в качестве
$ g $
функцию
$$
  g(t)=P_r (\theta -t).
$$
Переходя в равенстве
$ (\alpha ) $
к пределу по последовательности
$ s_n $
получим
$$
  u(re^{i\theta})=P_r (\theta ) \ast h(\theta ).
$$
$ \triangle $
\\
                %%%%%%%%%%%%  (g)  Доказательство теоремы С   %%%%%%%%%%%%%
{\bfseries (g) Доказательство теоремы
$ \mathcal{C} $ }
\\
Пусть
$ f \in \mathcal{C} , $
т.е. функция
$ f $
голоморфна в единичном круге и принимает значения из
правой полуплоскости. Применим ДЛО
$$
  g=\frac{f}{1+f},
$$
переводящее правую полуплоскость в круг
$ |g-1/2|<1/2. $
Функция
$ g $
ограниченная, поэтому она имеет почти всюду некасательные
граничные значения. Тоже верно и для функции
$$
  f=\frac{g}{1-g}.
$$
$ \triangle $
                        %%%%%%%%%%%%   Упражнение 5  %%%%%%%%%%%%%%%%%%%%%%%
\begin{Exe}
Как быть со значением
$ g=1 ? $
\end{Exe}
\newpage
                        %%%%%%%%%%%%%%%%%%%%%%%%%%%%%%%%%%%%%%%%%%%%%%%%%%%
						%%%%%%%%%  2.10  Теорема об устойчивости  %%%%%%%%%
						%%%%%%%%%%%%   ортогональных многочленов  %%%%%%%%%
						%%%%%%%%%%%%%%%%%%%%%%%%%%%%%%%%%%%%%%%%%%%%%%%%%%%
\subsection{Теорема об устойчивости \\
ортогональных многочленов}
$ \; $
\\
Ранее мы нормировали ортогональный многочлен
$ Q_n $
условием, что старший коэффициент равен единице.
Теперь будем использовать другую нормировку, а именно,
будем считать, что
$ Q_n - $
{\bfseries ортонормированные многочлены},
т.е. старший коэффициент многочлена положителен, и
$$
  \mathfrak{S} \{ Q_n ^2 \} =1.
$$
В этом параграфе мы докажем, что сходимость или расходимость ряда
$$
  \sum _{n=0}^{\infty}|Q_n (z)|^2
  \qquad ( \ast )
$$
не зависит от выбора точки
$ z $
в пределах верхней полуплоскости. Это утверждение и носит
название
{\bfseries теоремы об устойчивости ортогональных многочленов}.
                             %%%%%%%%%%   Определение  17 %%%%%%%%%%%%%%%%%
\begin{Def}
Будем говорить, что позитивная последовательность
$ \mathbf{s} $
принадлежит классу
$ (D) , $
если ряд
$ (\ast ) $
расходится, и
$ \mathbf{s} \in (C) $
в противном случае.
\end{Def}
Прежде чем доказать теорему об устойчивости,
установим несколько вспомогательных утверждений.
\\

Итак, пусть
$ \mathbf{s} - $
позитивная последовательность,
$ \mu - $
произвольное решение соответствующей проблемы моментов,
$$
  R_n (z)=\int \frac{Q_n (\lambda )}{z-\lambda}
  d\mu (\lambda )
$$
функции второго рода,
$$
  P_n (z)=\int \frac{Q_n (z)-Q_n (\lambda)}
  {z-\lambda}d\mu (\lambda )
$$
многочлены второго рода.
                        %%%%%%%%%%   Лемма  3  %%%%%%%%%%%%%%%%%%%%%%%%%%%
\begin{Lem}
Для любой точки
$ z \in \mathbb{C} $
следующий ряд сходится
$$
  \sum _{n=0}^{\infty}|R_n (z)|^2 < + \infty .
$$
\end{Lem}
{\Large Доказательство.}
Функция
$ \lambda \mapsto \frac{1}{z-\lambda} $
непрерывна и ограничена на вещественной оси, следовательно,
она принадлежит пространству
$ L_{\mu}^2 . $
Величины
$ R_n (z) $
суть ее коэффициенты Фурье по ОНС
$ \{ Q_n \} _0 ^{\infty} . $
По
{\bfseries неравенству Бесселя}
$$
  \sum _{n=0}^{\infty} |R_n (z)|^2 \leq
  \int \frac{d\mu (\lambda )}{|z-\lambda |^2}
  < + \infty .
$$
$ \triangle $
                            %%%%%%%%%%%   Следствие   10   %%%%%%%%%%%%%%%
\begin{Cor}
Пусть
$ z \in \mathbb{C}_+ . $
Тогда ряды
$$
  \sum _{n=0}^{\infty} |Q_n (z)|^2 \quad и \quad
  \sum _{n=0}^{\infty} |P_n (z)|^2
$$
либо сходятся одновременно, либо расходятся одновременно.
\end{Cor}
{\Large Доказательство.}
Каждая из трех последовательностей
$ \{ Q_n \} , \; \{ P_n \} \; и \; \{ R_n \} $
является линейной комбинацией двух других:
$$
  R_n = Q_n \hat \mu - P_n .
$$
$ \triangle $
                %%%%%%%%%%%%%   Определение 18   %%%%%%%%%%%%%%%%%%%%%%%%%%%
\begin{Def}
{\bfseries Оператором Вольтерра}
в гильбертовом пространстве
$ l_2 $
будем называть оператор с
{\bfseries нижней треугольной матрицей}
\begin{equation*}
  \mathsf{A}=
    \begin{pmatrix}
	  0 & 0 & 0 & \dots \\
	  a_{1,0} & 0 & 0 & \dots \\
	  a_{2,0} & a_{2,1} & 0 & \dots \\
	  \hdotsfor{4}
	\end{pmatrix}
\end{equation*}
и конечной
{\bfseries нормой Гильберта-Шмидта}
$$
  ||\mathsf{A}||_{\ast}^2 =
  \sum _{\substack{k=0,...,j-1 \\ j=1,2,... }}
  |a_{j,k}|^2 < + \infty .
$$
\end{Def}
                     %%%%%%%%%%%%%   Утверждение  17   %%%%%%%%%%%%%%%%%%%%
\begin{Sta}
Спектр оператора Вольтерра состоит из одной точки ноль.
\end{Sta}
{\Large Доказательство.}
Надо доказать, что
$ \forall \zeta \in \mathbb{C} $
оператор
$ (\mathsf{I}-\zeta \mathsf{A}) $
обратим. Очевидно, что обратный оператор равен сумме ряда
$$
  \sum _{n=0}^{\infty} (\zeta \mathsf{A})^n .
  \quad ( 1)
$$
Необходимо лишь доказать его сходимость, а для этого требуется
оценить
$ ||\mathsf{A}^n ||. $
Заметим, что для любого оператора
$ \mathsf{B}=(b_{i,j}) $
операторная норма не превосходит нормы Гильберта-Шмидта:
$$
  ||\mathsf{B}|| \leq ||\mathsf{B}||_{\ast}
$$
Действительно по неравенсту Коши-Бунякоского имеем
$$
  ||\mathsf{B}\mathtt{x}||^2 =\sum _j |\mathsf{B}\mathtt{x}_j |^2 =
  \sum _j |\sum _k b_{j,k}x_k |^2 \leq
$$
$$
  \leq \sum _j \sum _k |b_{j,k}|^2 \sum _k |x_k |^2 \leq
  ||\mathsf{B}||_{\ast}^2 ||\mathtt{x}||^2 .
$$
Обозначим
$ \mathsf{A}_m $
матрицу, которая получается из матрицы
$ \mathsf{A} $
удалением
$ m $
перых строк и
$ (m-1) $
первых столбцов. Заметим, что у матрицы
$ \mathsf{A}^n $
первые
$ n $
строк равны нулю. Обозначим
$ \tilde{\mathsf{A}}^n $
матрицу, полученную из
$ \mathsf{A}^n $
удалением этих нулевых строк. Тогда
$$
  \tilde{\mathsf{A}}^n = \mathsf{A}_n ...\mathsf{A}_1 .
$$
Следовательно,
$$
  ||\mathsf{A}^n ||^2 =|| \tilde{\mathsf{A}}^n ||^2 \leq
  ||\mathsf{A}_n ||_{\ast}^2 ...||\mathsf{A}_1 ||_{\ast}^2 .
$$
Имеем
$$
  r_n ^2 := ||\mathsf{A}_n ||_{\ast}^2 =
  \sum _{j=n}^{\infty} \sum _{k=n-1}^{j-1}
  |a_{j,k}|^2 \leq
  \sum _{j=n}^{\infty} \sum _{k=0}^{j-1} |a_{j,k}|^2 .
$$
Следовательно,
$ r_n \rightarrow 0 $
при
$ n \rightarrow \infty , $
как остаток сходящегося ряда. Итак,
$$
  ||\mathsf{A}^n || \leq r_1 ...r_n .
$$
Рассмотрим числовой ряд
$$
  \sum _{n=0}^{\infty} |\zeta |^n r_1 ...r_n .
$$
Он сходится по признаку Даламбера. Следовательно, ряд (1)
сходится.
\\
$ \triangle $
                        %%%%%%%%%%%   Теорема 20   %%%%%%%%%%%%%%%%%%%%%%%%
\begin{The}
Если ряд
$ (\ast ) $
сходится при некотором
$ z=z_0 \in \mathbb{C} , $
то он сходится при любом
$ z \in \mathbb{C} . $
\end{The}
{\Large Доказательство.}
Имеем
$$
  \frac{Q_n (z)-Q_n (z_0 )}{z-z_0 }=
  \sum _{k=0}^{n-1}a_{n,k}Q_k (z)
$$
или
$$
  Q_n (z)=Q_n (z_0 ) +(z-z_0 ) \sum _{k=0}^{n-1}
  a_{n,k}Q_k (z).
$$
При этом
$$
  a_{n,k}= \int \frac{Q_n (\lambda )-Q_n (z_0 )}
  {\lambda -z_0 } Q_k (\lambda ) d \mu (\lambda ) =
  Q_k (z_0 ) \int \frac{Q_n (\lambda )-Q_n (z_0 )}
  {\lambda -z_0 } d\mu (\lambda) +
$$
$$
  +\int \biggl ( Q_n (\lambda )-Q_n (z_0 ) \biggr )
  \frac{Q_k (\lambda )-Q_k (z_0 )}{\lambda -z_0 } \mu (\lambda ) =
  Q_k (z_0 )P_n (z_0 ) -Q_n (z_0 )P_k (z_0 ).
$$
Тогда
$$
  \sum _{n=1}^{\infty} \sum _{k=0}^{n-1} |a_{n,k}|^2 \leq
  2 \sum _{n=1}^{\infty}\sum _{k=0}^{n-1}
  (Q_k ^2 P_n ^2 +Q_n ^2 P_k ^2 ) \leq
  4\sum_{n=0}^{\infty} P_n ^2 \sum _{k=0}^{\infty} Q_n ^2
  < +\infty .
$$
Обозначим
$ \mathtt{\eta} - $
столбец
$ \bigl ( Q_n (z) \bigr ) , $
$ \mathtt{c} - $
столбец
$ \bigl ( Q_n (z_0 ) \bigr ) , $
$ \zeta =z-z_0 , $
$ \mathsf{A} = (a_{n,k}). $
Тогда
$ \mathsf{A} - $
оператор Вольтерра и
$ \mathtt{\eta} $
удовлетворяет матричному уравнению
$$
 \mathtt{\eta} =\mathtt{c}+ \zeta \mathsf{A} \mathtt{\eta} .
$$
Откуда
$$
  \mathtt{\eta} = ( \mathsf{I}-\zeta \mathsf{A})^{-1} \mathtt{c} .
$$
Поскольку
$ ( \mathsf{I}-\zeta \mathsf{A})^{-1} - $
ограниченный операор в
$ l_2 $
и
$ \mathtt{c} \in l_2  $
по условию, то и
$ \mathtt{\eta} \in l_2 . $
\\
$ \triangle $
\newpage
                 %%%%%%%%%%%%%%%%%%%%%%%%%%%%%%%%%%%%%%%%%%%%%%%%%%%%%%%%%%
				 %%%%%%%%%%%   2.11   Достаточное условие   %%%%%%%%%%%%%%%
				 %%%%%%%%%%%   определенности проблемы моментов   %%%%%%%%%
				 %%%%%%%%%%%%%%%%%%%%%%%%%%%%%%%%%%%%%%%%%%%%%%%%%%%%%%%%%%
\subsection{Достаточное условие определенности \\
проблемы моментов}
$ \; $
\\
                 %%%%%%%%%%   Теорема 21    %%%%%%%%%%%%%%%%%%%%%%%%%%%%%%%
\begin{The}
Если
$ \mathtt{s} \in (D) , $
то проблема моментов определена.
\end{The}
{\Large Доказательство.}
Предположим противное, т.е. что существуют два различных решения
$ \mu $
и
$ \rho . $
Тогда
$$
  \exists z_0 \in \mathbb{C}_+ :
  \hat \mu (z_0 ) \not = \hat \rho ( z_0 ) .
$$
Действительно, в противном случае, т.е. если
$ \hat \mu = \hat \rho $
в
$ \mathbb{C}_+ , $
по формуле обращения Стилтьеса-Перрона получим
$ \mu = \rho . $
Последовательности
$$
  Q_n (z_0 )\hat \mu (z_0 )-P_n (z_0 ) = \hat \mu _n (z_0 )
$$
и
$$
  Q_n (z_0 ) \hat \rho ( z_0 ) -P_n (z_0 ) = \hat \rho _n (z_0 )
$$
принадлежат
$ l_2 . $
Следовательно,
$$
  Q_n (z_0 ) =\frac{\hat \mu _n (z_0 ) - \hat \rho _n ( z_0 ) }
  {\hat \mu (z_0 ) -\hat \rho (z_0 )} \in l_2 .
$$
Полученное противоречие доказывает теорему.
\\
$ \triangle $
\\
Ниже будет доказана и необходимость условия теоремы.
\newpage
                 %%%%%%%%%%%%%%%%%%%%%%%%%%%%%%%%%%%%%%%%%%%%%%%%%%%%%%%%%%
				 %%%%%%%%%%   2.12 Формула Кристоффеля-Дарбу   %%%%%%%%%%%%
				 %%%%%%%%%%%%%%%%%%%%%%%%%%%%%%%%%%%%%%%%%%%%%%%%%%%%%%%%%%
\subsection{Формула Кристоффеля-Дарбу}
$ \; $
\\
Пусть
$ \mathtt{s}=\{ s_n \} _{n=0}^{\infty} - $
позитивная последовательность,
$ \mathfrak{S} - $
соответствующий ей функционал,
$ \mu - $
любое решение соответсвующей проблемы моментов Гамбургера.
Далее,
$ \{ Q_n \} _{n=0}^{\infty} - $
последовательность ортонормированных многочленов:
$$
  \mathfrak{S} \{ Q_n Q_m \} = \int Q_n Q_m d\mu =
  \delta _{n,m}.
$$
При этом
$$
  Q_n (x) = \alpha _n x^n +...,
  \qquad \alpha _n >0.
$$
И наконец,
$$
  P_n (z)=\mathfrak{S}_x  \Biggl \{
  \frac{Q_n (z)-Q_n (x)}{z-x} \Biggr \} =
  \int \frac{Q_n (z)-Q_n (x) }{z-x}d\mu (x)
$$
многочлены второго рода.
\\

В первой главе нами были
получены рекуррентные формулы для ортогональных многочленов.
Поскольку теперь мы пользуемся другой нормировкой,то
мы передокажем эти формулы заново.
               %%%%%%%%%%%%%%   Утверждение 18  %%%%%%%%%%%%%%%%%%%%%%%%%%%%
\begin{Sta}
Справедливы следующие рекуррентные формулы
$$
  \lambda Q_n =h_n Q_{n+1}+v_n Q_n +h_{n-1}Q_{n-1},
  \qquad ( R)
$$
где
$ n=0, \; 1, \; 2,..., \; \; v_n - $
вещественные,
$ h_n - $
положительные коэффициенты, а именно,
$$
  h_n = \frac{\alpha _n }{\alpha _{n+1}} .
  \qquad ( \ast )
$$
При этом выполняются начальные условия
\begin{equation*}
  \begin{cases}
   Q_{-1}=0 \\
   Q_0 =1.
  \end{cases}
\end{equation*}
\end{Sta}
{\Large Доказательство.}
Разложим многочлен
$ \lambda Q_n $
по ортонормированному базису
$ \{ Q_k \} : $
$$
  \lambda Q_n = \sum _{k=0}^{n+1} a_k Q_k .
$$
Коэффициенты разложения вычисляются по формулам Фурье
$$
  a_k = \mathfrak{S} \{ \lambda Q_n Q_k \} .
$$
Если
$ k \leq n-2 , $
то
$ \deg \lambda Q_k \leq n-1 , $
и поскольку \\
$ Q_n \perp <1,...,\lambda ^{n-1}>, $
то
$ a_k =0 . $
Далее,
\begin{equation*}
  \begin{cases}
    a_{n+1}=\mathfrak{S} \{ \lambda Q_{n+1}Q_n \} \\
	a_{n-1}=\mathfrak{S} \{ \lambda Q_n Q_{n-1} \}
  \end{cases}
  \Rightarrow
  \begin{cases}
    a_{n+1}=h_n \\
	a_{n-1}=h_{n-1}
  \end{cases}
\end{equation*}
При этом, из сравнения старших коэффициентов получим
$ (\ast ) . $
И наконец,
$$
  v_n =\mathfrak{S} \{ \lambda Q_n ^2\} \in \mathbb{R} .
$$
Начальные условия проверяются непосредственно.\\
$ \triangle $
                    %%%%%%%%%%%%%   Замечание 16   %%%%%%%%%%%%%%%%%%%%%%%%%
\begin{Rem}
Рекуррентные соотношения записываются в матричном виде следующим
образом
$$
  \mathsf{A} \mathtt{Q} = \lambda \mathtt{Q} ,
$$
где
$$
  \mathtt{Q}=
  \begin{bmatrix}
    Q_0 \\
	Q_1 \\
	\dots
  \end{bmatrix}
$$
бесконечный столбец ортонормированных многочленов,
$$
  \mathsf{A}=
    \begin{pmatrix}
	  v_0 & h_0 & \; & \; & \;  \\
	  h_0 & v_1 & h_1 & \; & \; \\
	  \; & h_1 & v_2 & h_2 & \;  \\
	  \; & \; & \hdotsfor{3}
	\end{pmatrix}
$$
матрица Якоби, а именно, матрица некоторого симметричного оператора
в гильбертовом пространсте
$ l_2 . $
\end{Rem}
                     %%%%%%%%%%%%   Замечание   17       %%%%%%%%%%%%%%%%
\begin{Rem}
Для многочленов второго рода
$ P_n $
справедливы те же рекуррентные формулы (R), но для
$ n=1, \; 2, \; 3,... $
с начальными условиями
$ P_0 =0, \; P_1 =\alpha _1 =1/h_0 . $
\end{Rem}
                      %%%%%%%%%%%%%%   Утверждение  19      %%%%%%%%%%%%%%
\begin{Sta}
Справедлива
{\bfseries формула Кристоффеля-Дарбу}
$$
  \sum _{k=0}^n Q_k (x)Q_k (y)=h_n
  \frac{Q_{n+1}(x)Q_n (y)-Q_n (x)Q_{n+1}(y)}{x-y}.
  \quad (CD)
$$
\end{Sta}
{\Large Доказательство.}
Имеем
$$
  xQ_k (x)=h_k Q_{k+1}(x)+v_k Q_k (x)+h_{k-1}Q_{k-1}(x)
$$
$$
  yQ_k (y)=h_k Q_{k+1}(y)+v_k Q_k (y)+h_{k-1}Q_{k-1}(y)
$$
Умножим первое равенство на
$ Q_k (y) , $
второе на
$ Q_k (x) $
и вычтем одно из другого. Получим
$$
  (x-y)Q_k (x)Q_k (y)=\xi _k -\xi _{k-1} ,
$$
где
$$
  \xi _k =h_k \biggl ( Q_{k+1}(x)Q_k (y)-Q_{k+1}(y)Q_k (x) \biggr ) .
$$
Складывая полученные равенства по
$ k=0,...,n , $
получим (CD). $ \triangle $
                         %%%%%%%%%%%%   Следствие   11 %%%%%%%%%%%%%%%%%%
\begin{Cor}
$$
  \sum _{k=0}^n \bigl ( Q_k (x) \bigr ) ^2 =
  h_n \biggl ( Q_{n+1}^{\prime}(x)Q_n (x)-
  Q_{n+1}(x)Q_n ^{\prime}(x) \biggr ) .
$$
\end{Cor}
{\Large Доказательство.}
Перейдем в (CD) к пределу при
$ \; y \rightarrow x \; $
предварительно записав правую часть в виде
$$
  \frac{h_n}{x-y} \Biggl ( \bigl ( Q_{n+1}(x)-Q_{n+1}(y) \bigr )
  Q_n (y) - \bigl ( Q_n (x)-Q_n(y) \bigr ) Q_{n+1}(y) \Biggr ) .
$$
$ \triangle $
                       %%%%%%%%%%    Утверждение   20 %%%%%%%%%%%%%%%%%%%%
\begin{Sta}
Справедлива
{\bfseries формула Лагранжа}
$$
  h_n \bigl ( P_{n+1}Q_n -P_n Q_{n+1} \bigr ) =1.
$$
\end{Sta}
{\Large Доказательство.}
Обозначим левую часть формулы
$ \eta _n . $
Имеем
$$
  \eta _n = Q_n \bigl ( xP_n -h_{n-1}P_{n-1} \bigr ) -
  P_n \bigl ( xQ_n -h_{n-1}Q_{n-1} \bigr ) =
$$
$$
  =h_{n-1} \bigl (P_n Q_{n-1}-Q_nP_{n-1} \bigr ) =
  \eta _{n-1} =...=\eta _0 =
  h_0 \bigl ( P_1 \cdot 1-Q_1 \cdot 0 \bigr ) =1.
$$
$ \triangle $
                            %%%%%%%%%%%%   Следствие 12  %%%%%%%%%%%%%%%%%
\begin{Cor}
Справедлива следующая формула для коэффициентов Кристоффеля
$$
  \frac{1}{\mu _{n,j}}=
  \sum _{k=0}^{n-1} \bigl ( Q_k (x_{n,j}) \bigr ) ^2 .
$$
\end{Cor}
{\Large Доказательство.}
Имеем
$$
 \mu _{n,j}=\mathrm{res}_{z=x_{n,j}}
 \frac{P_n (z)}{Q_n (z)}=
 \frac{P_n (x_{n,j})}{Q_n ^{\prime}(x_{n,j})}=
$$
$$
 =\frac{h_{n-1} \bigl ( P_n (x_{n,j})Q_{n-1}(x_{n,j})-
 P_{n-1}(x_{n,j})Q_n (x_{n,j}) \bigr ) }
 {h_{n-1} \bigl ( Q_n ^{\prime}(x_{n,j})Q_{n-1}(x_{n,j})-
 Q_{n-1}^{\prime}(x_{n,j})Q_n (x_{n,j}) \bigr ) } =
 \frac{1}{\sum _{k=0}^{n-1} \bigl ( Q_k (x_{n,j}) \bigr ) ^2 } .
$$
$ \triangle $
\newpage
                            %%%%%%%%%%%%%%%%%%%%%%%%%%%%%%%%%%%%%%%%%%%%%%%%
							%%%%%%%%%   2.13   Экстремальные свойства   %%%%
							%%%%%%%%%   ортогональных многочленов   %%%%%%%%
							%%%%%%%%%%%%%%%%%%%%%%%%%%%%%%%%%%%%%%%%%%%%%%%%
\subsection{Экстремальные свойства \\
ортогональных многочленов}
$ \; $
\\
Пусть
$ \mu \in \mathcal{M}. $
Обозначим
$$
  ||T||_{\mu}^2 =\int _{-\infty}^{+\infty}
  |T(x)|^2 d\mu (x)
$$
квадрат нормы в пространстве
$ L_{\mu}^2 . $
Пусть по прежнему
$$
  Q_n (x)=\alpha _n x^n +... , \quad
  \alpha _n >0,
$$
ортонормированные по мере
$ \mu $
многочлены.
                             %%%%%%%%%%%%%%   Утверждение 21   %%%%%%%%%%%
\begin{Sta}
Пусть
$ T $
пробегает множество всех многочленов степени
$ n $
с единичным старшим коэффициентом, т.е.
$ T(x)=x^n +.... $
Обозначим
$$
  m_n = \inf _T ||T||_{\mu}^2 .
$$
Утверждается, что
\\
1) $ m_n =1/\alpha _n ^2 , $
\\
2) infimum достигается на многочлене
$ \overset{\circ}{Q}_n = \frac{1}{\alpha _n}Q_n , $
\\
3) это единственная точка минимума.
\end{Sta}
              %%%%%%%%%%%%%%%%%%   Замечание  18 %%%%%%%%%%%%%%%%%%%%%%%%
\begin{Rem}
Это экстремальное свойство -- общее для всех ортонормированных
систем, означающее, что перпендикуляр есть кратчайшее расстояние
от точки до плоскости.
\end{Rem}
{\Large Доказательство.}
Разложим многочлен
$ T $
по ортонормированной системе
$ \{ Q_k \} : $
$$
  T=\overset{\circ}{Q}_n+\sum _{k=0}^{n-1}a_k Q_k .
$$
По
{\bfseries равенству Парсеваля}
$$
  ||T||_{\mu}^2 =\frac{1}{\alpha _n ^2 }+\sum _{k=0}^{n-1}|a_k|^2 .
$$
Очевидно, что минимум по
$ a_k $
достигается, тогда и только тогда, когда все
$ a_k =0 , $
т.е.
$ T=\overset {\circ}{Q}_n . \triangle $
                     %%%%%%%%%%%%%   Утверждение 22 %%%%%%%%%%%%%%%%%%%%%%%
\begin{Sta}
Пусть
$ T $
пробегает множество всех многочленов степени не выше
$ n , $
таких, что
$ T(z)=1 , $
где
$ z - $
произвольное, но фиксированное комплексное число.
Обозначим
$$
  \rho _n (z)=\inf _T ||T||_{\mu}^2 .
$$
Утверждается, что
\\
1) имеем
$$
  \rho _n (z)=\frac{1}
  {\sum _{k=0}^n |Q_k (z)|^2},
$$
2) infimum достигается при
$ T=T_z , $
где
$$
  T_z (x)=\rho _n (z) \sum _{k=0}^n Q_k (x) \overline {Q_k (z)},
$$
3) это единственная точка минимума.
\end{Sta}
{\Large Доказательство.}
Разложим многочлен
$ T $
по ортонормированной системе
$ \{ Q_k \} : $
$$
  T(x)=\sum _{k=0}^n b_k Q_k (x) .
$$
По равенству Парсеваля
$$
  ||T||_{\mu}^2 =\sum _{k=0}^n |b_k |^2 .
$$
По условию
$$
  \sum _{k=0}^n b_k Q_k (z) =1 .
$$
Напишем неравенство Коши-Буняковского
$$
  1= \Biggl | \sum _{k=0}^n b_k Q_k (z) \Biggr | ^2
  \leq \sum _{k=0}^n |b_k |^2 \sum _{k=0}^n |Q_k (z)|^2 .
$$
Таким образом
$ ||T||_{\mu}^2 \geq t_n $
(где
$ t_n - $
правая часть равенства 1)). При
$ T=T_z $
достигается равенство. Кроме того известно, что равенство
в неравенстве Коши-Буняковского достигается, тогда и только тогда,
когда  последовательности
$ \{ b_k \} $
и
$ \{ \overline {Q_k (z) } \} $
пропорциональны (с положительным коэффициентом).
Откуда и следуют все утверждения.
$ \triangle $
\\
Обозначим
$ \mathfrak{M}_N $
множество решений усеченной проблемы моментов, т.е. множество
всех конечных положительных борелевских мер
$ \mu $
на вещественной оси, таких, что
$$
  \int _{-\infty}^{+\infty} \lambda ^k d\mu (\lambda )=
  s_k , \qquad k=0,...,N .
$$
                           %%%%%%%%%%%%   Замечание 19 %%%%%%%%%%%%%%%%%%
\begin{Rem}
Последнее утверждение справедливо для всех мер
$ \mu \in \mathfrak{M}_{2n} . $
\end{Rem}
Действительно,
$ ||T||_{\mu}^2 $
зависит только от моментов
$ s_0 ,..., s_{2n} . $
                           %%%%%%%%%%%%%%    Следствие 13 %%%%%%%%%%%%%%%
\begin{Cor}
Справедливо неравенство
$$
  \sup _{\mu \in \mathfrak{M}_{2n}} \mu (\{ x \} )
  \leq \rho _n (x) .
$$
\end{Cor}
{\Large Доказательство.}
Имеем
$$
  \rho _n (x)=\int |T_x (\lambda )|^2d\mu (\lambda )
  \geq \mu ( \{ x \} ).
$$
$ \triangle $    						   						
                     %%%%%%%%%%%%%%%%%     Лемма 4    %%%%%%%%%%%%%%%%%
\begin{Lem}
Справедливо неравенство
$$
  \sup _{\mu \in \mathfrak{M}_{2n-1}}
  \mu ( \{ x \} ) \geq \rho _n (x).
$$
\end{Lem}
{\Large Доказательство.}
Пусть
$ Q_n (x)=0 $
и
$ \mu _n - $
соответствующая этому многочлену дискретная мера
(имеющая массы
$ \mu _{n,j} $
в нулях многочлена
$ x_{n,j} $
). Тогда, с одной строны, по квадратурной формуле Гаусса-Якоби
$ \mu_n \in \mathfrak{M}_{2n-1} . $
С другой стороны, учитывая результаты предыдущего параграфа
имеем
$$
  \mu _n ( \{ x \} ) = \rho _{n-1}(x) =\rho _n (x) .
$$
Пусть теперь
$ Q_n (x) \not = 0. $
Рассмотрим
{\bfseries квазиортогональный многочлен}
$$
  Q_n ^{\ast} = Q_{n+1}-cQ_n ,
$$
где
$ c - $
вещественный параметр. Существует и единственно значение
$ c , $
такое, что
$ Q_n ^{\ast} (x)=0. $
Нетрудно доказать, что свойства нулей и коэффициентов Кристоффеля
у квазиортогональных многочленов такие же, что у ортогональных
многочленов. Обозначим
$ \mu _n ^{\ast} $
соответсвующую дискретную меру. Тогда
$ \mu _n ^{\ast} \in \mathfrak{M}_{2n-1} $
и
$ \mu _n ^{\ast} (\{x \} ) = \rho _n (x) . $
$ \triangle $
                               %%%%%%%%%%%%%   Упражнение  6 %%%%%%%%%%%%%
\begin{Exe}
Доказать все необходимые свойства квазиортогональных многочленов.
\end{Exe}
Обозначим
$$
  m(x)=\sup _{\mu \in \mathfrak{M}} \mu ( \{ x \} ) ,
$$
где
$ \mathfrak{M} - $
множество всех решений проблемы моментов, и
$$
  \rho (x) = \frac{1}{\sum _{n=0}^{\infty}
  |Q_n (x)|^2 } .
$$
                                  %%%%%%%%%%   Следствие  14 %%%%%%%%%%%%%
\begin{Cor}
Справедливо равенство
$$
  m(x)=\rho (x) .
$$
\end{Cor}
{\Large Доказательство.}
Во-первых, для каждого $ n $ имеем
$  m(x) \leq \rho _n (x) , $
следовательно,
$ m(x) \leq \rho (x) . $
\\
Во-вторых, для любого $ n $ найдется мера
$ \mu _n \in \mathfrak{M}_{2n-1} , $
такая, что
$ \mu _n (\{ x \} ) \geq \rho _n (x) . $
Выберем слабо сходящуюся подпоследовательность
$ \mu _n \rightharpoonup \mu , \; n \in \Lambda . $
Тогда
$ \mu \in \mathfrak{M} . $
Возьмем произвольное
$ \varepsilon >0 . $
Тогда
$$
  \mu _n ([x,x+\varepsilon )) \geq \rho _n (x).
$$
Переходя в неравенстве к пределу по подпоследовательности
$ \Lambda $
получим
$$
  \mu ([x,x+\varepsilon)) \geq \rho (x) .
$$
Откуда в силу произвольности
$ \varepsilon $
получим
$ \mu ( \{ x \} ) \geq \rho (x) . \triangle $
\\

Завершим доказательство критерия определенности проблемы моментов.
                             %%%%%%%%   Теорема   22 %%%%%%%%%%%%%%%%%%%%%%
\begin{The}
Если
$ \mathbf{s} \in (C), $
то проблема моментов неопределена.
\end{The}
Если
$ \mathbf{s} \in (C) , $
то
$ \rho (x)>0 $
для всех
$ x \in \mathbb{R} . $
Предположим, что
$ \mu - $
единственное решение проблемы моментов, тогда
$ \mu ( \{ x \} ) >0 $
для всех
$ x \in \mathbb{R} . $
Но число точек разрыва монотонной функции не более чем счетно.
Полученное противоречие доказывает теорему.
$ \triangle $
\newpage
                         %%%%%%%%%%%%%%%%%%%%%%%%%%%%%%%%%%%%%%%%%%%%%%%%%%
						 %%%%%%%%%%%%%   2.14  Теорема Карлемана   %%%%%%%%
						 %%%%%%%%%%%%%%%%%%%%%%%%%%%%%%%%%%%%%%%%%%%%%%%%%%
\subsection{Теорема Карлемана}
$ \; $
\\
Пусть
$ \mathsf{s}=\{ s_n \} _{n=0}^{\infty} - $
позитивная последовательность, и
$$
  \mathsf{A}=
    \begin{pmatrix}
	  v_0 & h_0 & \; & \; & \; \\
	  h_0 & v_1 & h_1 & \; & \; \\
	  \; & h_1 & v_2 & h_2 & \; \\
	  \; & \; &  \hdotsfor{3}
	\end{pmatrix}
$$
соответсвующая ей матрица Якоби. С этими объектами связана
разрешимая проблема моментов. Мы хотим ответить на вопрос
об определенности этой проблемы моментов в терминах моментов
$ s_n $
или элементов матрицы Якоби. Выше было получено полное
решение этой задачи в терминах ортогональных многочленов.
В этом паракрафе в качестве следствия мы
получим простые достаточные условия, принадлежащие Карлеману.
                 %%%%%%%%%%%%%   Теорема  23 %%%%%%%%%%%%%%%%%%%%%%%%%%%%%
\begin{The}
{\bfseries (Карлеман)}
\\
Если расходится ряд
$$
  \sum _{n=0}^{\infty} \frac{1}{h_n }=+\infty ,
  \quad (h)
$$
то проблема моментов определена.
\end{The}
{\Large Доказательство.}
Если проблема моментов неопределена, то при
$ z\in \mathbb{C}_+ $
последовательности
$ \{P_n (z) \} $
и
$ \{ Q_n (z) \} $
принадлежат пространству
$ l_2 . $
Тогда согласно тождеству Лагранжа последовательность
$ \{ \frac{1}{h_n} \} $
принадлежит просранству
$ l_1 , $
что противоречит условию теоремы.
$ \triangle $
\\

Нам понадобится следующее числовое неравенство.
                    %%%%%%%%%%%    Лемма  5 %%%%%%%%%%%%%%%%%%%%%%%%%%%%
\begin{Lem}
{\bfseries (Карлеман)}
\\
Для любых неотрицательных чисел
$ u_n \geq 0, \; n=1,2,3,..., $
справедливо
{\bfseries неравенство Карлемана}
$$
  \sum _{n=1}^{\infty}
  \sqrt[n]{u_1 ...u_n } \leq e
  \sum _{n=1}^{\infty} u_n .
$$
\end{Lem}
{\Large Доказательство.}
Рассмотрим числа
$$
  c_n =\frac{(n+1)^n}{n^{n-1}}, \quad n=1,2,3,...
$$
Известно, что
$ c_n <en. $
Применим к числам
$ c_1 u_1 ,...,c_n u_n $
{\bfseries неравнство о среднем арифметическом и среднем геометрическом}
$$
  \sqrt[n]{c_1 u_1 ...c_n u_n } \leq
  \frac{c_1 u_1 +...+c_n u_n }{n}.
$$
Или, поскольку,
$$
  c_1 ...c_n =(n+1)^n ,
$$
то
$$
 \sqrt[n]{u_1 ...u_n} \leq
 \frac{c_1 u_1 +...+c_n u_n}{n(n+1)}.
$$
Следовательно,
$$
  \sum _{n=1}^{\infty}\sqrt[n]{u_1 ...u_n} \leq
  \sum _{n=1}^{\infty} \frac{1}{n(n+1)}
  \sum _{k=1}^n c_k u_k =
  \sum _{k=1}^{\infty} c_k u_k
  \sum _{n=k}^{\infty} \frac{1}{n(n+1)}.
$$
Далее,
$$
  \sum _{n=k}^{\infty} \frac{1}{n(n+1)}=
  \sum _{n=k}^{\infty} \biggl ( \frac{1}{n}-
  \frac{1}{n+1} \biggr ) =\frac{1}{k} .
$$
Таким образом,
$$
  \sum _{n=1}^{\infty} \sqrt[n]{u_1 ...u_n} \leq
  \sum _{k=1}^{\infty}\frac{c_k u_k}{k} \leq
  e \sum _{k=1}^{\infty}u_k .
$$
Ч.т.д.
$ \triangle $
                   %%%%%%%%%%%%   Теорема 24 %%%%%%%%%%%%%%%%%%%%%%%%%
\begin{The}
{\bfseries (Карлеман)}
\\
Если расходится ряд
$$
  \sum _{n=0}^{\infty} \frac{1}{\sqrt[2n]{s_{2n}}}
  =+\infty , \quad (s)
$$
то проблема моментов определена.
\end{The}
{\Large Доказательство.}
Пусть по прежнему
$ \alpha _n $
обозначает положительный старший коэффициент ортонормированного
многочлена
$ Q_n . $
Тогда
$$
  h_0 h_1 ...h_{n-1}=\frac{1}{\alpha _n}=
  \int _{-\infty}^{+\infty}
  \lambda ^n Q_n (\lambda )d\mu (\lambda ).
$$
Применяя интегральное неравенство Коши-Буняковского, получим
$$
  h_0 h_1 ...h_{n-1} \leq
  \Biggl ( \int _{-\infty}^{+\infty}
  \lambda ^{2n}d\mu (\lambda ) \Biggr ) ^{1/2} \;
  \Biggl ( \int _{-\infty}^{+\infty}
  Q_n ^2 (\lambda )d\mu (\lambda ) \Biggr ) ^{1/2}
  =\sqrt{s_{2n}}.
$$
Применим неравнество Карлемана
$$
  \sum _{n=0}^{\infty} \frac{1}{\sqrt[2n]{s_{2n}}} \leq
  \sum _{n=0}^{\infty} \sqrt[n]{\frac{1}{h_0}...
  \frac{1}{h_{n-1}}} \leq e
  \sum _{n=0}^{\infty} \frac{1}{h_n} .
$$
Если проблема моментов неопределена, то по предыдущей теореме
ряд
$ (h) $
сходится. Тогда и ряд
$ (s) $
сходится, что противоречит услоию теоремы.
$ \triangle $
                              %%%%%%%%%%%%%%%    Замечание  20 %%%%%%%%%%%%
\begin{Rem}
Доказанная теорема усиливает полученное ранее условие
квазианалитичности.
\end{Rem}
\newpage
                             %%%%%%%%%%%%%%%%%%%%%%%%%%%%%%%%%%%%%%%%%%%%%%
							 %%%%%%%%%%%%%%%%%%%%%%%%%%%%%%%%%%%%%%%%%%%%%%
							 %%%%%%%%%   3   Теория фон Неймана   %%%%%%%%%
							 %%%%%%%%%%%%%%%%%%%%%%%%%%%%%%%%%%%%%%%%%%%%%%
							 %%%%%%%%%%%%%%%%%%%%%%%%%%%%%%%%%%%%%%%%%%%%%%
\section{Теория фон Неймана}
$ \; $
\\
$ \; $
\\
                              %%%%%%%%%%%%%%%%%%%%%%%%%%%%%%%%%%%%%%%%%%%%%
							  %%%%%%%%   3.1   Линейные операторы   %%%%%%%
							  %%%%%%%%%%%%%%%%%%%%%%%%%%%%%%%%%%%%%%%%%%%%%
\subsection{Линейные операторы}
$ \; $
\\
В этом параграфе мы приведем основные сведения из теории
линейных неограниченных операторов.
                               %%%%%%%%%%   Определение 19 %%%%%%%%%%%%%%%%
\begin{Def}
Пусть
$ \mathsf{H} - $
гильбертово пространство (бесконечномерное, сепарабельное,
над полем комплексных чисел),
$ \mathsf{D} \subset \mathsf{H} $
линейное подпространство (возможно незамкнутое).
{\bfseries линейным оператором}
в пространстве
$ \mathsf{H} $
называется отображение
$$
  \mathsf{A} : \mathsf{D} \longrightarrow \mathsf{H}
$$
такое, что
$$
  \mathsf{A}(\mathtt{x}+\mathtt{y})=
  \mathsf{A}\mathtt{x}+\mathsf{A}\mathtt{y},
  \quad \mathtt{x},\mathtt{y} \in \mathsf{D} .
$$
$$
  \mathsf{A}(\lambda \mathtt{x})=\lambda \mathsf{A}\mathtt{x},
  \quad \lambda \in \mathbb{C}, \; \mathtt{x} \in \mathsf{D}.
$$
Линейное пространство
$ \mathsf{D}(\mathsf{A})=\mathsf{D} $
называется
{\bfseries областью определения оператора}
$ \mathsf{A} . $
Если
$ \mathsf{D}(\mathsf{A}) $
всюду плотно в
$ \mathsf{H} , $
то говорят, что оператор
$ \mathsf{A} $
{\bfseries плотно определен}.
\end{Def}
                                %%%%%%%%%%%%   Пример  5 %%%%%%%%%%%%%%%
\begin{Exa}
Пусть
$ \mathsf{A} - $
матрица Якоби, т.е. матрица вида
$$
  \mathsf{A}=
    \begin{pmatrix}
	  v_0 & h_0 & \; & \; & \; \\
	  h_0 & v_1 & h_1 & \; & \; \\
	  \; & h_1 & v_2 & h_2 & \; \\
	  \; & \; & \hdotsfor{3}
	\end{pmatrix}
$$
где
$ v_n - $
любые действительные числа,
$ h_n - $
любые положительные числа.
\end{Exa}
Тогда матрица
$ \mathsf{A} $
пораждает плотно определенный оператор в гильбертовом
пространстве
$ \mathsf{H}=l_2 . $
Область определения
$ \mathsf{D}(\mathsf{A}) $
состоит из финитных последовательностей.
Оператор действует по правилу умножения матрицы на столбец.
Будем называть этот оператор оператором Якоби.
                   %%%%%%%%%%   Упражнение 7    %%%%%%%%%%%%%%%%%%%%%%%%%
\begin{Exe}
Доказать,что оператор Якоби ограничен, т.е.
$$
  ||\mathsf{A}||:=\sup _{\substack{\mathtt{x} \in \mathsf{H}\\
  ||\mathtt{x}|| \leq 1 }} ||\mathsf{A}\mathtt{x}|| < +\infty ,
$$
следовательно, непрерывен, и тем самым, по непрерывности
продолжается на все пространство
$ \mathsf{H} , $
тогда и только тогда, когда
$$
  \sup \{ |v_n |, h_n  \} < +\infty .
$$
\end{Exe}
                   %%%%%%%%%%%%%   Определение  20 %%%%%%%%%%%%%%%%%%%%%
\begin{Def}
{\bfseries Графиком}
линейного оператора
$ \mathsf{A} $
называется следующее линейное подпространство в
декартовом произведении
$ \mathsf{H} \times \mathsf{H} : $
$$
  \Gamma (\mathsf{A})=\{ <\varphi , \mathsf{A}\varphi > \; | \;
  \varphi \in \mathsf{D}(\mathsf{A}) \} .
$$
Говорят, что
$ \mathsf{A} \subset \mathsf{B} \; ( \mathsf{B} -  $
{\bfseries расширение}
$ \mathsf{A} ), $
если
$ \Gamma (\mathsf{A} ) \subset \Gamma (\mathsf{B}), $
Другими словами,
$$
  \mathsf{D}(\mathsf{A}) \subset \mathsf{D}(\mathsf{B})
  \quad и \quad \mathsf{B} \biggr | _{\mathsf{D}(\mathsf{A})}=
  \mathsf{A} .
$$
Оператор
$ \mathsf{A} $
называется
{\bfseries замкнутым},
если его график замкнут.
Говорят, что оператор
$ \mathsf{A} $
{\bfseries допускает замыкание},
если он имеет замкнутое расширение.
Наименьшее замкнутое расширение называется
{\bfseries замыканием}
оператора и обозначается
$ \bar{\mathsf{A}}. $
\end{Def}
Здесь и в дальнейшем угловые скобки обозначают упорядоченную пару,
а круглые -- скалярное произведение.
                      %%%%%%%%%%%%%    Упражнение 8    %%%%%%%%%%%%%%%%
\begin{Exe}
Привести пример оператора, не имеющего замыкания.
\end{Exe}
                      %%%%%%%%%%%%%   Замечание  21  %%%%%%%%%%%%%%%%%%%%
\begin{Rem}
Очевидно, что оператор
$ \mathsf{A} $
допускает замыкание, тогда и только тогда, когда замыкание
его графика является графиком некоторого оператора, и при этом
$$
  \overline {\Gamma (\mathsf{A})}=
  \Gamma (\bar{\mathsf{A}}).
$$
\end{Rem}
                             %%%%%%%%%%%%     Определение 21    %%%%%%%%%%
\begin{Def}
{\bfseries (определение сопряженного оператора)}\\
Пусть
$ \mathsf{A} - $
плотно определенный линейный оператор.
Рассмотрим линейный функционал
$$
  \mathtt{x} \mapsto (\mathsf{A}\mathtt{x},\mathtt{y}),
  \quad \mathtt{x} \in \mathsf{D}(\mathsf{A}).
$$
Пусть множество
$ \mathsf{D}(\mathsf{A}^{\ast}) $
состоит из тех и только тех векторов
$ \mathtt{y} , $
для которых этот функционал ограничен. Тогда
$ \mathsf{D}(\mathsf{A}^{\ast}) - $
линейное пространство. В этом случае по
{\bfseries теореме Рисса об общем виде линейного непрерывного
функционала в гильбертовом пространстве}
существует единственный вектор
$ \mathtt{y}^{\ast} , $
такой, что
$$
  (\mathsf{A}\mathtt{x},\mathtt{y})=
  (\mathtt{x},\mathtt{y}^{\ast}),
  \quad \mathtt{x} \in \mathsf{D}(\mathsf{A}) .
$$
Отображение
$$
  \mathsf{A}^{\ast}: \mathtt{y} \mapsto \mathtt{y}^{\ast},
  \quad \mathtt{y} \in \mathsf{D}(\mathsf{A}^{\ast}),
$$
называется
{\bfseries сопряженным оператором}.
Очевидно, это -- линейный
оператор.
\\
Другими словами, сопряженный оператор -- это максимальный оператор
$ \mathsf{A}^{\ast} , $
для которого выполняется равенство
$$
  (\mathsf{A}\mathtt{x},\mathtt{y})=
  (\mathtt{x},\mathsf{A}^{\ast}\mathtt{y}),
  \quad \mathtt{x} \in \mathsf{D}(\mathsf{A}),
  \quad \mathtt{y} \in \mathsf{D}(\mathsf{A}^{\ast}).
$$
\end{Def}
                      %%%%%%%%%%%%%%%   Замечание  22  %%%%%%%%%%%%%%%%%%
\begin{Rem}
Определение сопряженного оператора допускает простую
геометрическую переформулировку, а именно:
$$
  \Gamma (\mathsf{A}^{\ast})=
  \{ <\mathsf{A}\mathtt{x}, -\mathtt{x}> \; | \;
  \mathtt{x} \in \mathsf{D}(\mathsf{A}) \} ^{\bot}.
$$
\end{Rem}
Здесь значок
$ \bot $
обозначает ортогональное дополнение.
                        %%%%%%%%%%%   Следствие 15  %%%%%%%%%%%%%%%%%%%%%
\begin{Cor}
Сопряженный оператор всегда замкнут.
\end{Cor}
                        %%%%%%%%%%%%%   Следствие 16   %%%%%%%%%%%%%%%%%%
\begin{Cor}
$ \mathsf{A}^{\ast}=\bar{\mathsf{A}}^{\ast} . $
\end{Cor}
                         %%%%%%%%%%%%%    Определение 22  %%%%%%%%%%%%%%
\begin{Def}
Плотно определенный оператор
$ \mathsf{A} $
называется
{\bfseries симметричным},
если
$$
  (\mathsf{A}\mathtt{x},\mathtt{y})=
  (\mathtt{x},\mathsf{A}\mathtt{y}),
  \quad \mathtt{x}, \mathtt{y} \in \mathsf{D}(\mathsf{A}).
$$
Другими словами, оператор
$ \mathsf{A} $
симметричный, тогда и только тогда, когда
$ \mathsf{A} \subset \mathsf{A}^{\ast} . $
\end{Def}
                  %%%%%%%%%%%%%   Следствие 17 %%%%%%%%%%%%%%%%%%%%%%%%%%
\begin{Cor}
Симметричный оператор всегда допускает замыкание.
\end{Cor}
Вернемся к примеру оператора Якоби. Этот оператор симметричный
(потому что его матрица симметрична). Следовательно, он допускает
замыкание. Оператор
$ \bar{\mathsf{A}}$
это минимальный замкнутый оператор, который может быть определен
матрицей Якоби. Напротив, сопряженный оператор
$ \mathsf{A}^{\ast} $
это максимальный оператор, который может быть определен
той же матрицей.
  	                    %%%%%%%%%%%%%   Упражнение 23    %%%%%%%%%%%%%%%
\begin{Exe}
Плотно определенный оператор
$ \mathsf{A} $
допускает замыкание, тогда и только тогда, когда сопряженный
оператор
$ \mathsf{A}^{\ast} $
плотно определен. При этом
$ \bar{\mathsf{A}}=\mathsf{A}^{\ast \ast} . $
\end{Exe}
\newpage
                                %%%%%%%%%%%%%%%%%%%%%%%%%%%%%%%%%%%%%%%%%%%%
								%%%%%%%%%    3.2  Лебеговы операторы  %%%%%%
								%%%%%%%%%%%%%%%%%%%%%%%%%%%%%%%%%%%%%%%%%%%%
\subsection{Лебеговы операторы}
$ \; $
\\
                                %%%%%%%%   Определение 23   %%%%%%%%%%%%%%%%%
\begin{Def}
Линейный оператор
$ \mathsf{A} $
называется
{\bfseries лебеговым} (или оператором
{\bfseries с простым спектром}),
если он имеет циклический вектор.
{\bfseries Циклическим вектором} оператора
$ \mathsf{A} $
называется вектор
$ \phi \in \mathsf{H} , $
такой, что векторы
$$
  \{ \mathsf{A}^n \phi \; | \; n \in \mathbb{Z}_+ \}
$$
образуют полную систему в пространстве
$ \mathsf{H} . $
\end{Def}
Вернемся к примеру оператора Якоби.
Этот оператор лебегов. Циклическим вектором служит вектор
$ \mathtt{e}_0 . $
Действительно,
$$
  \mathsf{A}^n \mathtt{e}_0 \in
  <\mathtt{e}_n ,...,\mathtt{e}_0 >,
$$
причем вектор
$ \mathtt{e}_n $
входит в эту линейную комбинацию с ненулевым коэффициентом.
Следовательно, системы векторов
$ \mathsf{A}^n \mathtt{e}_0 $
и
$ \mathtt{e}_n , $
где
$ n \in \mathbb{Z}_+ , $
связаны между собой невырожденным линейным преобразованием.
\\

Верно и обратное утверждение.
                          %%%%%%%%%%   Утверждение  24  %%%%%%%%%%%%%%%%%%%%
\begin{Sta}
Любой симметричный лебегов оператор является расширением
оператора Якоби.
\end{Sta}						
{\Large Доказательство.}
Прежде всего заметим, что векторы
$ \mathsf{A}^n \phi , \\
n=0,1,2,..., $
линейно независимы. Действительно, предположим противное, т.е.,
что существует номер
$ N , $
такой, что вектор
$ \mathsf{A}^N \phi $
линейно выражается через предыдущие. Тогда для всех
$ n \geq N $
имеем
$$
  \mathsf{A}^n \phi \in <\mathsf{A}^0 \phi ,..., \mathsf{A}^{N-1}
  \phi >.
$$
Следовательно, линейная оболочка
$$
  <\mathsf{A}^n \phi \; | \; n \in \mathbb{Z}_+ >
$$
имеет конечную размерность, что противоречит полноте системы.\\
Применим к последовательности векторов
$ \mathsf{A}^n \phi $
{\bfseries процесс ортогонализации Грамма-Шмидта}.
Получим ПОНС
$$
  \mathtt{f}_n =\alpha _n \mathsf{A}^n \phi +...
  \in <\mathsf{A}^n \phi ,..., \mathsf{A}^0 \phi >,
$$
где
$ \alpha _n >0, \; n \in \mathbb{Z}_+ .  $
Напишем матрицу оператора
$ \mathsf{A} $\\
в базисе
$ \{ \mathtt{f}_n \} $
$$
  \mathsf{A}\mathtt{f}_n = \sum _{k=0}^{\infty}
  a_{n,k}\mathtt{f}_k .
$$
Если
$ k \geq n+2, $
то
$$
  a_{n,k}=(\mathsf{A}\mathtt{f}_n ,\mathtt{f}_k )=0,
$$
поскольку
$$
  \mathsf{A}\mathtt{f}_n \in \mathsf{A}
  <\mathsf{A}^n \phi ,..., \mathsf{A}^0 \phi >
  = <\mathsf{A}^{n+1} \phi ,..., \mathsf{A} \phi >
  \subset <\mathtt{f}_{n+1},...,\mathtt{f}_0 > \perp
  \mathtt{f}_k .
$$
Если
$ k \leq n-2 , $
то используя симметричность
$ \mathsf{A} $
аналогичным образом получим
$$
  a_{n,k}=(\mathsf{A}\mathtt{f}_n , \mathtt{f}_k )=
  (\mathtt{f}_n , \mathsf{A}\mathtt{f}_k )=0.
$$
Далее, обозначим
$ v_n =a_{n,n} . $
Поскольку
$$
  v_n =(\mathsf{A}\mathtt{f}_n , \mathtt{f}_n )=
  (\mathtt{f}_n , \mathsf{A}\mathtt{f}_n ) =
  \bar v_n ,
$$
то число
$ v_n - $
вещественное.
\\
И наконец, обозначим
$ h_n =a_{n,n+1} . $
Тогда, с одной стороны, из сравнения старших коэффициентов
получим
$$
  h_n =\frac{\alpha _n}{\alpha _{n+1}}>0.
$$
С другой стороны,
$$
  a_{n,n-1}=(\mathsf{A}\mathtt{f}_n , \mathtt{f}_{n-1})=
  (\mathtt{f}_n , \mathsf{A}\mathtt{f}_{n-1})=
  \bar h_{n-1}=h_{n-1}.
$$
Окончательно,
$$
  \mathsf{A}\mathtt{f}_n =
  h_n \mathtt{f}_{n+1}+v_n \mathtt{f}_n +h_{n-1}\mathtt{f}_{n-1} ,
$$
где
$ v_n - $
вещественные,
$ h_n - $
положительные. Ч.т.д.
$ \triangle $
\newpage
                     %%%%%%%%%%%%%%%%%%%%%%%%%%%%%%%%%%%%%%%%%%%%%%%%%%%%%
					 %%%%%%%%%%%   3.3   Самосопряженные операторы   %%%%%
					 %%%%%%%%%%%%%%%%%%%%%%%%%%%%%%%%%%%%%%%%%%%%%%%%%%%%%
\subsection{Самосопряженные операторы}
$ \; $
\\
                      %%%%%%%%%%%%%%%%%%%   Определение 24 %%%%%%%%%%%%%%%%
\begin{Def}
Оператор
$ \mathsf{A} $
называется
{\bfseries самосопряженным},
если
$ \mathsf{A}=\mathsf{A}^{\ast} . $
Другими словами, оба оператора
$ \mathsf{A} $
и
$ \mathsf{A}^{\ast} $
симметричные.
\end{Def}
В этом параграфе будет доказан критерий самосопряженности.
                           %%%%%%%%%%   Определение 25  %%%%%%%%%%%%%%%%%%
\begin{Def}
{\bfseries Дефектными подпространствами оператора}
$ \mathsf{A} $
называются следующие собственные подпространства
$$
  \mathsf{K}_{\pm}= \mathrm{Ker}
  (\mathsf{A}^{\ast} \mp i).
$$
{\bfseries Индексами дефекта}
называются размерности этих пространств
$$
  d_{\pm}=\dim \mathsf{K}_{\pm}.
$$
\end{Def}
Индексы дефекта могут принимать любые целые неотрицательные
значения и могут равняться бесконечности.
\\

Определим в линейном пространстве
$ \mathsf{D}(\mathsf{A}^{\ast}) $
новую норму, индуцированную графиком
$$
  ||\mathtt{y}||_{\ast}^2 =
  ||\mathtt{y}||^2 +
  ||\mathsf{A}^{\ast}\mathtt{y}||^2 .
$$
Эта норма пораждается соответствующим скалярным произведением.
В дальнейшем значок
$ \oplus _{\ast} $
обозначает ортогональную относительно этого скалярного
произведения сумму.
                    %%%%%%%%%%%%%   Теорема 25  %%%%%%%%%%%%%%%%%%%%%%%%%%%%
\begin{The}
{\bfseries (Дж. фон Нейман)}
\\
Если
$ \mathsf{A} $
замкнутый симметричный оператор, то справедливо следующее разложение
$$
  \mathsf{D}(\mathsf{A}^{\ast})=
  \mathsf{D}(\mathsf{A}) \oplus _{\ast}
  \mathsf{K}_+ \oplus _{\ast} \mathsf{K}_- .
$$
\end{The}
{\Large Доказательство.}
Докажем, что
$ \mathsf{K}_+ \perp _{\ast} \mathsf{K}_- . $
Пусть
$$
  \varphi \in \mathsf{K}_+ , \quad т.е. \quad
  \mathsf{A}^{\ast} \varphi = +i \varphi ,
$$
$$
  \psi \in \mathsf{K}_- , \quad т.е. \quad
  \mathsf{A}^{\ast} \psi =-i \psi .    		
$$
Тогда
$$
  (\varphi , \psi )_{\ast}=(\varphi , \psi )+
  (\mathsf{A}^{\ast} \varphi , \mathsf{A}^{\ast} \psi ) =
  (\varphi , \psi ) + (i\varphi , -i \psi )=
  (\varphi , \psi )-(\varphi , \psi) =0.
$$
Докажем, что
$ \mathsf{D}(\mathsf{A}) \perp _{\ast} \mathsf{K}_+ . $
Пусть
$$
  \varphi \in \mathsf{D}(\mathsf{A}) \quad и \quad
  \psi \in \mathsf{K}_+ , \quad т.е. \quad
  \mathsf{A}^{\ast}\psi =i\psi .
$$
Тогда
$$
  (\varphi , \psi )_{\ast}=(\varphi , \psi )+
  (\mathsf{A}^{\ast} \varphi , \mathsf{A}^{\ast} \psi )=
  (\varphi , \psi )+(\mathsf{A}\varphi , i\psi )=
  (\varphi , \psi ) -i (\mathsf{A}\varphi , \psi )=
$$
$$
  =(\varphi , \psi )-i (\varphi , \mathsf{A}^{\ast}\psi )=
  (\varphi , \psi )-(\varphi , \psi)=0.
$$
Аналогичным образом доказывается, что
$ \mathsf{D}(\mathsf{A}) \perp _{\ast} \mathsf{K}_- . $
\\

Нам понадобится следующая
                      %%%%%%%%%%%%%   Лемма  6  %%%%%%%%%%%%%%%%%%%%%%%%%%
\begin{Lem}
Для любого замкнутого симметричного оператора
$ \mathsf{A} $
справедливо следующее разложение
$$
  \mathsf{H}=\mathrm{Ran}(\mathsf{A}+i)
  \oplus \mathsf{K}_+ .
$$
\end{Lem}
Возьмем произвольный вектор
$ \eta \in \mathsf{D}(\mathsf{A}^{\ast}). $
Воспользуемся леммой. Тогда существуют векторы
$ \varphi \in \mathsf{D}(\mathsf{A}) $
и
$ \psi \in \mathsf{K}_+ , $
такие, что
$$
  (\mathsf{A}^{\ast}+i)\eta =(\mathsf{A}+i)\varphi +2i\psi .
$$
Положим
$$
  \chi =\eta -\varphi -\psi .
$$
Осталось доказать, что
$ \chi \in \mathsf{K}_- . $
Действительно,
$$
 (\mathsf{A}^{\ast}+i)\chi = (\mathsf{A}+i)\varphi +
 2i\psi - (\mathsf{A}^{\ast}+i)\varphi -
 (\mathsf{A}^{\ast}+i)\psi=
$$
$$
 =(\mathsf{A}+i)\varphi +2i\psi -
 (\mathsf{A}+i)\varphi -2i\psi =0.
$$
Теорема доказана.
$ \triangle $
\\
{\Large Доказательство леммы.}
Для любого оператора
$ \mathsf{B} $
по определению сопряженного оператора имеем
$$
  \mathrm{Ker}\mathsf{B}^{\ast}=
  (\mathrm{Ran} \mathsf{B})^{\bot}.
$$
В частности применим эту формулу к оператору\\
$ \mathsf{B}=\mathsf{A}+i . $
Для доказательства леммы осталось показать, что образ
$ \mathrm{Ran}(\mathsf{A}+i) $
замкнутое подпространство.
Снова рассмотрим норму, индуцированную графиком
$$
  ||(\mathsf{A}+i)\varphi ||^2 =
  ||\mathsf{A} \varphi ||^2 +||\varphi ||^2 ,
  \quad \varphi \in \mathsf{D}(\mathsf{A}) ,
  \quad (\ast )
$$
(мы воспользовались симметричностью оператора).
\\
Пусть
$ \{ (\mathsf{A}+i)\varphi _n \} - $
последовательность Коши. Тогда из формулы
$ (\ast ) $
следует, что обе последовательности
$ \{ \varphi _n \} $
и
$ \{ \mathsf{A}\varphi _n \} $
будут последовательносями Коши. Тогда последовательность точек
$ <\varphi _n , \mathsf{A}\varphi _n > , $
принадлежащих графику
$ \Gamma (\mathsf{A}) , $
сходится к точке
$ <\varphi , \mathsf{A}\varphi > , $
также принадлежащей графику (так как по условию оператор
замкнутый). Следовательно,
$$
  (\mathsf{A}+i)\varphi_n \longrightarrow
  (\mathsf{A}+i)\varphi ,
$$
где
$ \varphi \in \mathsf{D}(\mathsf{A}) , $
что и означает замкнутость образа.\\
Лемма доказана.
$ \triangle $
                        %%%%%%%%%%%%%%%%%   Следствие  18  %%%%%%%%%%%%%%%
\begin{Cor}
Замкнутый симметричный оператор является самосопряженным,
тогда и только тогда, когда его индексы дефекта
равны нулю
$ d_{\pm}=0. $
\end{Cor}
Вернемся к примеру оператора Якоби. Вычислим его индексы дефекта.
Решим уравнение
$$
  \mathsf{A}^{\ast}\mathsf{Q}=z\mathsf{Q},
  \quad z \in \mathbb{C}_{\pm} .
$$
Решением (с точностью до числового множителя) будет
столбец ортонормированных многочленов. Возможны два случая.
Либо
$ \mathsf{Q} \in l_2 $
(случай
$ (C)), $
тогда индексы дефекта равны единице
$ d_{\pm}=1. $
Либо
$ \mathsf{Q} \not \in l_2 $
(случай
$ (D)), $
тогда индексы дефекта равны нулю
$ d_{\pm}=0 . $
\\

Таким образом, нами доказана следующая
               %%%%%%%%%%%   Теорема  26  %%%%%%%%%%%%%%%%%%%%%%%%%%%%%%
\begin{The}
Оператор Якоби
$ \bar{\mathsf{A}} $
самосопряженный, тогда и только тогда, когда
соответствующая ему проблема моментов определена.
\end{The}
\newpage
                   %%%%%%%%%%%%%%%%%%%%%%%%%%%%%%%%%%%%%%%%%%%%%%%%%%%%%%%%
				   %%%%%%%%%%%%%%%%%%%%%%%%%%%%%%%%%%%%%%%%%%%%%%%%%%%%%%%%
				   %%%%%%%%%   4   Разностные операторы второго   %%%%%%%%%
				   %%%%%%%%%       порядка и нерерывная дробь     %%%%%%%%%
				   %%%%%%%%%       Стилтьеса                      %%%%%%%%%
				   %%%%%%%%%%%%%%%%%%%%%%%%%%%%%%%%%%%%%%%%%%%%%%%%%%%%%%%%
				   %%%%%%%%%%%%%%%%%%%%%%%%%%%%%%%%%%%%%%%%%%%%%%%%%%%%%%%%
\section{Разностные операторы второго порядка\\
и непрерывная дробь Стилтьеса}
                   %%%%%%%%%%%%%%%%%%%%%%%%%%%%%%%%%%%%%%%%%%%%%%%%%%%%%%%%
				   %%%%%%%%%%   4.1   Дробь Стилтьеса и ее        %%%%%%%%%
				   %%%%%%%%%%          эквивалентные формы        %%%%%%%%%
				   %%%%%%%%%%%%%%%%%%%%%%%%%%%%%%%%%%%%%%%%%%%%%%%%%%%%%%%%
\subsection{Дробь Стилтьеса и ее эквивалентные формы}
{\bfseries Непрерывной дробью Стилтьеса}
называют следующую непрерывную
дробь
\begin{equation*}
  \cfrac{a_0}{z-
    \cfrac{a_1}{1-
	  \cfrac{a_2}{z-
	    \cfrac{a_3}{1-\dotsb
  }}}}
  \qquad (4.1)
\end{equation*}
Иногда ее также записывают в виде
\begin{equation*}
  \cfrac{1}{c_0 z+
    \cfrac{1}{c_1 +
	  \cfrac{1}{c_2 z+\dotsb
  }}},
  \qquad (4.2)
\end{equation*}
где
\begin{equation*}
  a_0 =\frac{1}{c_0}, \quad
  a_k =-\frac{1}{c_k c_{k-1}}, \quad k=1,2,...
\end{equation*}
Принимая во внимание, что часто в приложениях
четные и нечетные коэффициенты дроби (4.2) имеют
различный смысл, дробь (4.2) также записывают в виде
\begin{equation*}
  l_0 +
    \cfrac{1}{m_1 z+
	  \cfrac{1}{l_1 +
	    \cfrac{1}{m_2 z+ \dotsb
	}}}
	\qquad (4.3)
\end{equation*}
В некотором смысле дробь Стилтьеса дает более полную
информацию о степенном ряде, чем рассмотренная ранее
(см. параграф 1.5) непрерывная дробь Чебышева-Якоби.
Поясним это следующим утверждением.
                        %%%%%%%%%%%   Утверждение         %%%%%%%%%%%%%%%%%
\begin{Sta}
Четные подходящие непрерывной дроби Стилтьеса (4.1)
совпадают с подходящими непрерывной дроби Чебышева-Якоби
\begin{equation*}
  \cfrac{A_0}{z-B_0 -
    \cfrac{A_1}{z-B_1 -
	  \cfrac{A_2}{z-B_2 - \dotsb
  }}}
  \qquad (4.4)
\end{equation*}
где
\begin{align*}
  & A_0 =a_0 , \quad & A_n =a_{2n}a_{2n-1}, \quad & n=1,2,...\\
  & B_0 =a_1 , \quad & B_n =a_{2n}+a_{2n+1}, \quad & n=1,2,...
  \qquad (4.5)
\end{align*}
\end{Sta}  	    		    	        						      				
{\Large Доказательство.}
Обозначим
\begin{equation*}
  t_k [w]=\cfrac{a_{2k-1}}{1-
            \cfrac{a_{2k}}{z-w}}
\end{equation*}
Элементарные преобразования дают
\begin{equation*}
  t_k [w]=a_{2k-1}+
  \frac{a_{2k}a_{2k-1}}{z-a_{2k}-w}.
  \qquad (4.6)
\end{equation*}
Четные подходящие непрерывной дроби Стилтьеса (4.1)
можно записать как
\begin{equation*}
  \frac{P_{2n}(z)}{Q_{2n}(z)}=
  \frac{a_0}
  {z-t_1 \circ t_2 \circ ... t_{n-1}
  [\frac{a_{2n-1}}{1}]},
\end{equation*}
что с учетом (4.6) можно переисать как
\begin{equation*}
  \frac{P_{2n}(z)}{Q_{2n}(z)}=
    \cfrac{a_0}{z-a_1 -
	  \cfrac{a_1 a_2}{z-a_2 -a_3 -
	    \cfrac{a_3 a_4}{z-a_4 -a_5 -...-
		  \cfrac{a_{2n-3}a_{2n-2}}{z-a_{2n-2}-
		    \cfrac{a_{2n-1}}{1
	}}}}}.
\end{equation*}
Утверждение доказано. $ \triangle $
                      %%%%%%%%%%%%%   Следствие             %%%%%%%%%%%%%%%
\begin{Cor}
Ввиду Утверждения 8 параграфа 1.5,
четные подходящие непрерывной дроби Стилтьеса
\begin{equation*}
  \cfrac{a_0}{z-
    \cfrac{a_1}{1-
	  \cfrac{a_2}{z-
	    \cfrac{a_3}{1-
  }}}}
  \sim S(z)=\frac{s_0}{z}+\frac{s_1}{z^2}+...
  \qquad (4.7)
\end{equation*}
являются диагональными аппроксимациями Паде
(рассмотренными в параграфе 1.1)
для степенного ряда
$ S(z). $
\end{Cor}
Отметим, что
{\bfseries преобразования Стилтьеса}
(4.5) в литературе по математической физике также называют
{\bfseries преобразованиями Миуры.}
\newpage
				%%%%%%%%%%%%%%%%%%%%%%%%%%%%%%%%%%%%%%%%%%%%%%%%%%%%%%%%%%%
				%%%%%%%%%%%   4.2    Рекуррентные соотношения   %%%%%%%%%%%
				%%%%%%%%%%%          и разностные задачи, связанные   %%%%%
				%%%%%%%%%%%          с дробью Стилтьеса        %%%%%%%%%%%%
				%%%%%%%%%%%%%%%%%%%%%%%%%%%%%%%%%%%%%%%%%%%%%%%%%%%%%%%%%%%
\subsection{Рекуррентные соотношения и разностные\\
задачи, связанные с дробью Стилтьеса}
Общее Утверждение 7 параграфа 1.5 дает трехчленные рекуррентные
соотношения, связывающие полиномы знаменатели подходящих
дробей (4.1) и (4.4). Те же рекуррентные соотношения при других
начальных данных также связывают числители подходящих дробей.
Для дроби (4.1) имеем
\begin{equation*}
  Y_{n+1}(z)=\varepsilon _n Y_n (z)-a_n Y_{n-1}(z),
  \quad n=1,2,..., \qquad (4.8)
\end{equation*}
где
$ \varepsilon _n =z $
для четных
$ n $
и
$ \varepsilon _n =1 $
для нечетных
$ n . $
Тогда для
$ n- $
й подходящей
$ P_n /Q_n $
непрерывной дроби (4.1) справедливо (для
$ n=0,1,...) $
\begin{align*}
  \; & P_n =Y_n , \quad & P_0 =Y_0 =0, \quad & P_1 =Y_1 =a_0 , \\
  \; & Q_n =Y_n , \quad & Q_0 =Y_0 =1, \quad & Q_1 =Y_1 =z .
\end{align*}
При этом
\begin{equation*}
  \deg Q_{2n}=n, \quad \deg Q_{2n-1}=n, \quad
  \deg P_n = \deg Q_n -1.
\end{equation*}
Аналогично, рекуррентные соотношения, задаваемые дробью
(4.4), имеют вид
\begin{equation*}
  \tilde Y_{n+1}(z)=(z-B_n )\tilde Y_n (z)-A_n \tilde Y_{n-1}(z),
  \quad n=1,2,..., \qquad (4.9)
\end{equation*}
где
$ A_n , \; B_n $
задаются соотношениями Стилтьеса (4.5).
Поэтому для
$ n=0, \; 1, \; 2,... $
имеем
\begin{align*}
  \; & P_{2n}=\tilde Y_n , \quad & P_0 =\tilde Y_0 :=0,
  \quad &  P_2 =\tilde Y_1 :=A_0 =a_0,\\
  \; & Q_{2n}=\tilde Y_n , \quad & Q_0 =\tilde Y_0 :=1,
  \quad & Q_2 =\tilde Y_1 :=z-B_0 =z-a_1 .
\end{align*}
Рекуррентные соотношения (4.8) и (4.9) задают разностную задачу
второго порядка. В обозначениях дроби (4.3) приведем эквивалентную
систему разностных уравнений первого порядка.
                    %%%%%%%%%   Утверждение               %%%%%%%%%%%%%%%
\begin{Sta}
{\bfseries (Крейн)}
Пусть
$ \eta (z) $
и
$ \theta (z) $
удовлетворяют системе
\begin{equation*}
  \begin{cases}
    \eta _{n+1}-\eta _n =l_n \theta _n \\
	\theta _{n+1}-\theta _n =m_{n+1}z\eta_{n+1},
  \end{cases}
  \quad n=1,2,3,... \qquad (4.10)
\end{equation*}
Тогда для подходящих
$ \frac{P_n}{Q_n} $
непрерывной дроби Стилтьеса (4.1)
(те же подходящие имеют эквивалентные дроби (4.2) и (4.3)
при соответствующих заменах
$ \{ a_n \} \leftrightarrow \{ c_n \} \leftrightarrow \{ l_n , m_n \} ), $
справедливо
\begin{align*}
  \; & Q_{2n+1}=\eta _n , \quad & n=0,1,...,
  \quad \quad & \eta _0 =1, \\
  \; & Q_{2n}=\theta _n , \quad & n=1,2,...,
  \quad \quad & \theta _1 =m_1 z ,
\end{align*}
соответствующие числители получаются при
$$
  \eta _0 =l_0 , \quad \theta _1 =1+l_0 m_1 z .
$$
\end{Sta}
               %%%%%%%%%%   Упражнение                %%%%%%%%%%%%%%%%%%%
\begin{Exe}
Докажите это утверждение.
\end{Exe}
Рассмотрим другой разностный оператор, связанный с дробью
Стилтьеса (4.1). Пусть оператор
$ \mathsf{A} , $
действующий в
$ l_2 , $
задается в стандартном базисе
$ \{ \mathtt{e}_k \} : $
$$
  \mathtt{e}_k := (0,..,0,1,0,...), \quad k=0,1,2,...
$$
с помощью бесконечной матрицы
\begin{equation*}
  \mathsf{A}:=
    \begin{bmatrix}
	  0 & 1 & \; & \; & \; \\
	  a_1 & 0 & 1 & \; & \; \\
	  \; & a_2 & 0 & 1 \\
	  \; & \; & \dots & \dots & \dots
	\end{bmatrix}
\end{equation*}
Для этого оператора рассмотрим резольвентную функцию
\begin{equation*}
  f(z):=((z\mathsf{I}-\mathsf{A})^{-1}\mathtt{e}_0 ,\mathtt{e}_0 )
  =\sum _{k=0}^{\infty}\frac{(\mathsf{A}^k \mathtt{e}_0 ,
  \mathtt{e}_0 )}{z^{k+1}}=: \sum _{k=0}^{\infty}
  \frac{f_k}{z^{k+1}}.
\end{equation*}
               %%%%%%%%%%%%   Утверждение              %%%%%%%%%%%%%%%%%%
\begin{Sta}
Для спектральных данных
$ f(z), \; \{ f_k \} _{k=0}^{\infty} $
оператора
$ \mathsf{A} $
справедливо
\begin{align*}
  \; & f(z)=zS(z^2 ) \\
  \; & f_{2k}=s_k , \quad f_{2k+1}=0, \quad
  k=0,1,2,... \qquad (4.11)
\end{align*}
где
$ S(z) $
степенной ряд, представляющий дробь Стилтьеса (4.7).
\end{Sta}
{\Large Доказательство.}
Сделаем в дроби Стилтьеса
$ S(\tilde z ) $
замену
$ \tilde z =z^2 $
и рассмотрим
\begin{equation*}
  zS(z^2 )=z
     \cfrac{a_0}{z^2 -
	   \cfrac{a_1}{1-
	     \cfrac{a_2}{z^2 -
		   \cfrac{a_3}{1-\dotsb
	  }}}}
\end{equation*}
Умножение и деление каждого этажа этой непрерывной дроби на
$ z $
дает
\begin{equation*}
  zS(z^2 )=
    \cfrac{a_0}{z-
	  \cfrac{a_1}{z-
	    \cfrac{a_2}{z-\dotsb
	}}}
\end{equation*}
Замечая, что в правой части стоит непрерывная дробь,
соответствующая резольвентной функции оператора
$ \mathsf{A} $
(см. параграф 1.6), получаем справедливость первого
соотношения в (4.11). Справедливость второго соотношения
в (4.11) сразу же вытекает из первого соотношения,
если приравнять коэффициенты степенных рядов,
стоящих в обеих частях равенства.
$ \triangle $ \\
Таким образом, к спектральным данным оператора
$ \mathsf{A} $
мы также можем относить степенной ряд
$ S(z) $
и его коэффициенты
$ \{ s_k \} . $
Мы их будем называть
{\bfseries стилтьесовскими спектральными данными оператора}
$ \mathsf{A}. $
\newpage
               %%%%%%%%%%%%%%%%%%%%%%%%%%%%%%%%%%%%%%%%%%%%%%%%%%%%%%%%%%%%
			   %%%%%%%%%   4.3   Решение прямой задачи для       %%%%%%%%%%
			   %%%%%%%%%         спектральных данных Стилтьеса   %%%%%%%%%%
			   %%%%%%%%%%%%%%%%%%%%%%%%%%%%%%%%%%%%%%%%%%%%%%%%%%%%%%%%%%%%
\subsection{Решение прямой задачи \\
для спектральных данных
Стилтьеса}
Решение прямой задачи
$$
  \{ a_k \} \longrightarrow \{ s_k \}
$$
дается следующей формулой.
                   %%%%%%%%%%%%   Теорема              %%%%%%%%%%%%%%%%%%%
\begin{The}
Если степенной ряд
$$
  S(z)=\sum _{k=0}^{\infty}
    \frac{s_k}{z^{k+1}}, \quad s_0 =1,
$$
разлагается в дробь Стилтьеса
\begin{equation*}
  S(z)=
    \cfrac{a_0}{z-
	  \cfrac{a_1}{1-
	    \cfrac{a_2}{z-\dotsb
	}}}
\end{equation*}
то
\begin{equation*}
  s_k =a_1 \sum _{j_2 =1}^2 a_{j_2}
    \sum _{j_3 =1}^{j_2 +1}a_{j_3}...
	  \sum _{j_k =1}^{j_{k-1}+1}a_{j_k},
	    \quad k=1,2,... \qquad (4.12)
\end{equation*}
\end{The}
Доказательству теоремы предпошлем следующую лемму.
                      %%%%%%%%%%   Лемма             %%%%%%%%%%%%%%%%%%%%%%
\begin{Lem}
Пусть
$ s_k $
определены как (4.12), а
$ s_k ^{(1)} $
есть
\begin{equation*}
  s_k ^{(1)}=a_2 \sum _{j_3 =2}^3 a_{j_3}
    \sum _{j_4 =2}^{j_3 +1}a_{j_4}...
	  \sum _{j_{k+1}=2}^{j_k +1}a_{j_{k+1}},
	    \quad k=1,2,... \qquad (4.13)
\end{equation*}
Тогда
\begin{equation*}
  s_{k+1}=a_1 \sum _{\nu =0}^k s_{\nu}^{(1)}s_{k-\nu}.
  \qquad (4.14)
\end{equation*}
\end{Lem}
{\Large Доказательство леммы.}
Расписывая (4.12) для
$ s_{k+1} $
имеем
\begin{multline*}
  s_{k+1}=a_1 \Biggl (
    s_k +a_2 a_1
	  \sum _{j_3 =1}^2 a_{j_3}...
	    \sum _{j_k =1}^{j_{k-1}+1}a_{j_k}+
		  a_2 (a_2 +a_3 )a_1
		    \sum _{j_4 =1}^2 a_{j_4}...
			  \sum _{j_k =1}^{j_{k-1}+1}a_{j_k}+...\\
   ...+ \biggl ( a_2
     \sum _{i_2 =2}^3 a_{i_2}
	   \sum _{i_3 =2}^{i_2 +1}a_{i_3}...
	     \sum _{i_p =2}^{i_{p-1}+1}a_{i_p} \biggr )
		   \biggl ( a_1
		     \sum _{j_{p+2}=1}^2 a_{j_{p+2}}...
			   \sum _{j_k =1}^{j_{k-1}+1}a_{j_k} \biggr )+
			   ... \Biggr ),
\end{multline*}
что с учетом обозначения (4.13) дает (4.14).\\
Лемма доказана. $ \triangle $ \\
{\Large Доказательство теоремы.}
Рассмотрим степенные ряды:
\begin{equation*}
  S(z)=\sum _{k=0}^{\infty}
    \frac{s_k}{z^{k+1}}
	  \; \; и \; \;
	    S^{(1)}(z)=
		  \sum _{k=0}^{\infty}
		    \frac{s_k ^{(1)}}{z^{k+1}},
			  \qquad s_0 =s_0 ^{(1)} =1.
\end{equation*}
Лемма с учетом правила умножения степенных рядов дает:
$$
  S(z)-\frac{1}{z}=a_1 S(z)S^{(1)}(z).
$$
Откуда
$$
  S(z)=\frac{1}{z-za_1 S^{(1)}(z)}.
  \qquad (4.15)
$$
Обозначим
\begin{equation*}
  s_k ^{(l)}=a_{l+1}
    \sum _{j_{l+2}=l+1}^{l+2}a_{j_{l+2}}
	  \sum _{j_{l+3}=l+1}^{j_{l+2}+1}a_{j_{l+3}}...
	    \sum _{j_{k+l}=l+1}^{j_{k+l-1}+1}a_{j_{k+l}},
		  \quad k=0,1,2,...
\end{equation*}
и
\begin{equation*}
  S^{(l)}(z)=\sum _{k=0}^{\infty}
    \frac{s_k ^{(l)}}{z^{k+1}},
	  \quad s_0 ^{(l)}=1.
\end{equation*}
По лемме имеем
$$
  S^{(l)}(z)=\frac{1}
    {z-za_{l+1}S^{(l+1)}(z)}.
$$
Итеративно подставляя эту формулу в (4.15) получаем для
$ S(z) $
разложение в дробь Стилтьеса с коэффициентами
$ \{ a_k \} : $
$$
  S(z)=
    \frac{1}{z-a_1 z \biggl (
	  \frac{1}{z-za_2 S^{(2)}(z)}
	    \biggr ) } =
$$
$$
   =\cfrac{1}{z-
     \cfrac{a_1}{1-
	   \cfrac{a_2}{z-za_3 S^{(3)}
   }}}=
   \cfrac{1}{z-
     \cfrac{a_1}{1-
	   \cfrac{a_2}{z-
	     \cfrac{a_3}{1-\dotsb
	}}}}.
$$
Теорема доказана. $ \triangle $ \\
Учитывая структуру сумм в правой части (4.12) мы будем
их называть
{\bfseries генетическими суммами}.
\newpage
          %%%%%%%%%%%%%%%%%%%%%%%%%%%%%%%%%%%%%%%%%%%%%%%%%%%%%%%%%%%%%%%%%%
		  %%%%%%%   4.4   Вычисление определителей Ганкеля      %%%%%%%%%%%%
		  %%%%%%%         и решение обратной задачи для спектральных  %%%%%%
		  %%%%%%%         данных Стилтьеса                            %%%%%%
		  %%%%%%%%%%%%%%%%%%%%%%%%%%%%%%%%%%%%%%%%%%%%%%%%%%%%%%%%%%%%%%%%%%
\subsection{Вычисление определителей Ганкеля \\
и решение обратной
задачи \\
для спектральных данных Стилтьеса}
Если определитель Ганкеля состоит из моментов,
задаваемых генетическими суммами (4.12),
то он легко может быть приведен к треугольному виду.
Проиллюстрируем это на примере определителя 3-го порядка:
\begin{equation*}
  H_3 =
    \begin{vmatrix}
	  s_0 & s_1 & s_2 \\
	  s_1 & s_2 & s_3 \\
	  s_2 & s_3 & s_4
	\end{vmatrix}
	=
\end{equation*}
\begin{equation*}
  =
  \begin{vmatrix}
    1 & a_1 & a_1 \sum _{j_2 =1}^{1+1}a_{j_2} \\
	\; & \; & \; \\
	a_1 & a_1 \sum _{j_2 =1}^{1+1}a_{j_2} &
	  a_1 \sum _{j_2 =1}^{1+1}a_{j_2}
	    \sum _{j_3 =1}^{j_2 +1}a_{j_3} \\
		\; & \; & \; \\
	a_1 \sum _{j_2 =1}^{1+1}a_{j_2} &
	   a_1 \sum _{j_2 =1}^{1+1}a_{j_2}
	     \sum _{j_3 =1}^{j_2 +1}a_{j_3} &
		   a_1 \sum _{j_2 =1}^{1+1}a_{j_2}
		     \sum _{j_3 =1}^{j_2 +1}a_{j_3}
			   \sum _{j_4 =1}^{j_3 +1}a_{j_4}
   \end{vmatrix}
\end{equation*}
Вынося из последней строки
$ a_1 , $
вычитая из нее предпоследнюю строку и затем вынося
из последней полученной строки
$ a_2 , $
а из предпоследней строки
$ a_1 , $
получим
\begin{equation*}
  H_3 =a_1 ^2 a_2
    \begin{vmatrix}
	  1 & a_1 & a_1 \sum _{j_2 =1}^{1+1}a_{j_2} \\
	  \; & \; & \; \\
	  1 & \sum _{j_2 =1}^2 a_{j_2} &
	    \sum _{j_2 =1}^2 a_{j_2}
		  \sum _{j_3 =1}^{j_2 +1}a_{j_3} \\
		  \; & \; & \; \\
	  1 & \sum _{j_2 =1}^2 a_{j_2} &
	    \sum _{j_3 =1}^3 a_{j_3}
		  \sum _{j_4 =1}^{j_3+1}a_{j_3}
	\end{vmatrix}
\end{equation*}
Вычитая из 3-й строки вторую, а из 2-й строки первую, продолжим
\begin{equation*}
  H_3 =a_1 ^2 a_2
    \begin{vmatrix}
	  1 & a_1 & a_1 \sum _{j_2 =1}^{1+1}a_{j_2} \\
	  \; & \; & \; \\
	  0 & a_2 & a_2 \sum _{j_3 =1}^{2+1}a_{j_3} \\
	  \; & \; & \; \\
	  0 & a_3 & a_3 \sum _{j_4 =1}^{3+1}a_{j_4}
	\end{vmatrix}
\end{equation*}
Вынося из 3-й строки
$ a_3 , $
из 2-й строки
$ a_2 $
и вычитая из полученной 3-й строки вторую, будем иметь
\begin{equation*}
  H_3 =a_1 ^2 a_2 ^2 a_3
    \begin{vmatrix}
	  1 & a_1 & a_1 \sum _{j_2 =1}^2 a_{j_2} \\
	  \; & \; & \; \\
	  0 & 1 & \sum _{j_3 =1}^3 a_{j_3} \\
	  \; & \; & \; \\
	  0 & 0 & a_4
	\end{vmatrix}
	= a_1 ^2 a_2 ^2 a_3 a_4 .
\end{equation*}
Применяя аналогичные рассуждения к определителю произвольного
порядка получаем:
$$
  H_{n+1}=(a_1 a_2 )^n (a_3 a_4 )^{n-1}...
    (a_{2n-3}a_{2n-2})^2 (a_{2n-1}a_{2n}),
	  \quad n=0,1,...
$$
Таким же образом к треугольному виду приводятся определители
\begin{equation*}
  \tilde H_n =
    \begin{vmatrix}
	  s_1 & s_2 & \dots & s_n \\
	  s_2 & s_3 & \dots & s_{n+1} \\
	  \dots & \dots & \dots & \dots \\
	  s_n & s_{n+1} & \dots & s_{2n-1}
	\end{vmatrix}
\end{equation*}
Имеем
$$
  \tilde H_1 =a_1 , \quad
    \tilde H_2 = a_1 ^2 a_2 a_3 , \quad
	  \tilde H_3 = a_1 ^3 a_2 ^2 a_3 ^2 a_4 a_5 ,\quad ...
$$
$$
  \tilde H_n = a_1 ^n (a_2 a_3 )^{n-1}...(a_{2n-2}a_{2n-1}).
$$
Итак, рекуррентные формулы для определителей
$ H_n $
и
$ \tilde H_n $
$$
  H_{n+1}=a_1 a_2 a_3 ...a_{2n} H_n ,
    \quad H_1 =1
$$
$$
  \tilde H_{n+1}=a_1 a_2 a_3 ...a_{2n+1} \tilde H_n ,
    \quad \tilde H_0 =1
$$
дают решение обратной задачи
$$
  \{ s_n \} \longrightarrow \{ a_k \} .
$$
Справедлива
              %%%%%%%%%%%%%%   Теорема                   %%%%%%%%%%%%%%%%%
\begin{The}
Коэффициенты непрерывной дроби Стилтьеса выражаются через
коэффициенты ее степенного ряда по формулам:
$$
  a_{2n}=\frac{H_{n+1}\tilde H_{n-1}}
    {H_n \tilde H_n},
$$
$$
  a_{2n-1}=\frac{\tilde H_{n-1}H_{n-1}}
    {\tilde H_{n-2}H_n},
	  \quad n=1,2,... \qquad (4.16)
$$
\end{The}
Отметим, что используя исчисление генетических сумм
можно не только вычислить определители Ганкеля, но и
дать формулы для миноров их последнего столбца.
                  %%%%%%%   Упражнение                    %%%%%%%%%%%%%%%%
\begin{Exe}
Пусть
\begin{equation*}
  t[m;n]:=
    \sum _{j_1 =1}^m a_{j_1}
	  \sum _{j_2 =j_1 +1}^{m+1}a_{j_2}...
	    \sum _{j_n =j_{n-1}+1}^{m+n-1}a_{j_n}.
\end{equation*}
Докажите, что
\begin{align*}
  1) & \quad
    s_{2n}=\sum _{k=1}^n (-1)^{k-1}s_{2n-k} \cdot
	  t[2(n-k+1);2k] \\
  2) & \quad
    t[m;n]=t[m-1;n]+a_{m+2(n-1)}
	  \cdot t[m;n-1] \\
  3) & \quad
    s_{2n}= \prod _{j=1}^{2n}a_j +
	  \sum _{k=1}^n (-1)^{k-1}s_{2n-k} \cdot
	    t[2(n-k)+1;2k]
\end{align*}
\end{Exe}
                %%%%%%%%%%%%   Упражнение                        %%%%%%%%%
\begin{Exe}
Пусть
\begin{equation*}
  H_{n,k}:=
    \begin{vmatrix}
	  s_0 & \dots & s_{n-1} \\
	  \dots & \dots & \dots \\
	  s_{n-k-1} & \dots & s_{2n-k-2} \\
	  s_{n-k+1} & \dots & s_{2n-k} \\
	  \dots & \dots & \dots \\
	  s_n & \dots & s_{2n-1}
	\end{vmatrix}
  , \quad k=0,...,n-1.
\end{equation*}
Докажите, что
$$
  H_{n+1}=a_1 a_2 ...a_{2n} H_n
$$
$$
  H_{n,k}=t[2(n-k)+1;2k]H_{n-2}.
$$
\end{Exe}
\newpage
                %%%%%%%%%%%%%%%%%%%%%%%%%%%%%%%%%%%%%%%%%%%%%%%%%%%%%%%%%
				%%%%%%%%%%%%%%%%%%%%%%%%%%%%%%%%%%%%%%%%%%%%%%%%%%%%%%%%%
				%%%%%%%%%   5   Проблема моментов Стилтьеса   %%%%%%%%%%%
				%%%%%%%%%%%%%%%%%%%%%%%%%%%%%%%%%%%%%%%%%%%%%%%%%%%%%%%%%
				%%%%%%%%%%%%%%%%%%%%%%%%%%%%%%%%%%%%%%%%%%%%%%%%%%%%%%%%%
\section{Проблема моментов Стилтьеса}
                %%%%%%%%%%%%%%%%%%%%%%%%%%%%%%%%%%%%%%%%%%%%%%%%%%%%%%%%
				%%%%%%%%%%%   5.1 Разрешимость проблемы моментов   %%%%%
				%%%%%%%%%%%       Стилтьеса                        %%%%%
				%%%%%%%%%%%%%%%%%%%%%%%%%%%%%%%%%%%%%%%%%%%%%%%%%%%%%%%%
\subsection{Разрешимость проблемы моментов Стилтьеса}
{\bfseries Проблемой моментов Стилтьеса}
называется следующая задача.
                %%%%%%%%%%   Задача                   %%%%%%%%%%%%%%%%%%
\begin{Pro}
Дана последовательность действительных чисел
$ \{ s_n \} _{n=0}^{\infty} . $
Требуется найти меру
$ s \in \mathcal{M} $
с носителем на
$ [0,+\infty] $
такую, что числа
$ s_n $
суть ее степенные моменты:
\begin{equation*}
  s_n =\int _0 ^{\infty} \lambda ^n ds( \lambda ).
\end{equation*}
\end{Pro}
Разрешимость проблемы моментов Стилтьеса связана с
существованием спектральной меры для спектральных данных
оператора
$ \mathsf{A} , $
рассмотренного в параграфе 4.2:
\begin{equation*}
  \mathsf{A}=
    \begin{pmatrix}
	  0 & 1 & \; & \; & \; \; \\
	  a_1 & 0 & 1 & \; & \; \; \\
	  \; & a_2 & 0 & 1 & \; \; \\
	  \; & \; & \dots & \dots & \dots
	\end{pmatrix}
\end{equation*}
Напомним, что в качестве спектральных данных оператора
$ \mathsf{A} $
мы рассматривали спектральную функцию
$$
  f(z):=((z \mathsf{I}-\mathsf{A})^{-1}
    \mathtt{e}_0 ,\mathtt{e}_0 )
$$
и коэффициенты ее разложения в точке
$ z=\infty : $
$$
  f_k =(\mathsf{A}^k \mathtt{e}_0 , \mathtt{e}_0 ).
$$
Причем, ввиду специальной структуры оператора
$ \mathsf{A} , $
имеем
$$
  f_{2k+1}=0, \quad f_{2k}=s_k , \quad
     k=0,1,2,..., \qquad (5.1)
$$
где
$ s_k $
коэффициенты степенного ряда
$$
  S(z)=\frac{1}{\sqrt{z}}f(\sqrt{z}),
$$
разложение которого в непрерывную дробь Стилтьеса
имеет коэффициенты
$ \{ a_k \} . $
Если
$ \{ f_k \} $
является последовательностью моментов некоторой меры
$ \mu (y) , $
то можно считать, что эта мера симметрична:
$$
  \mu (y)=-\mu (-y).
$$
Поэтому
$$
  f_{2k}=\int _{-\infty}^{+\infty}y^{2k}d\mu (y)=
    \int _{-\infty}^0 y^{2k}d\mu (y) +
	  \int _0 ^{+\infty} y^{2k}d\mu (y)=
$$
$$
  =\int _0 ^{+\infty}y^{2k}d\mu (-y) +
    \int _0 ^{+\infty} y^{2k}d\mu (y)=
$$
$$
  =2\int _0 ^{+\infty}y^{2k}d\mu (y) =
    \int _0 ^{+\infty}x^k d(2\mu (\sqrt{x}))=
	  \int _0 ^{+\infty}x^k ds(x)=s_k .
$$
Таким образом, существование симметричной спектральной меры
$ d\mu (y) $
для данных
$ f(z), \; \{ f_k \} $
эквивалентно существованию решения (разрешимости)
проблемы моментов Стилтьеса
$ ds $
для
$ S(z), \; \{ s_k \} . $
При этом для оператора
$ \mathsf{A} $
{\bfseries спектральные данные Гамбургера}
$$
  (f, \{ f_k \} , d\mu )
$$
связаны со спектральными данными Стилтьеса  	  		
$$
  (S,\{ s_k \} , ds)
$$
соотношениями:
\begin{equation*}
  \begin{cases}
    f(z)=zS(z^2 ) \\
	f_{2k}=s_k , \quad f_{2k+1}=0 , \quad k=0,1,2,... \\
	d\mu (x)=\frac{1}{2}ds(x^2 )
  \end{cases}
\end{equation*}
Приведем критерий разрешимости проблемы моментов Стилтьеса.
            %%%%%%%%%%   Теорема                   %%%%%%%%%%%%%%%%%%%%%%
\begin{The}{\bfseries (Стилтьес).}
Для разрешимости проблемы моментов Стилтьеса
необходимо и достаточно, чтобы
$$
  H_n [\{ s_k \} ]>0, \quad
  \tilde H_n [ \{ s_k \} ] > 0.
$$
\end{The}
{\Large Доказательство.}
Как мы только что отметили, существование решения
проблемы моментов Стилтьеса для
$ \{ s_k \} $
эквивалентно существованию решения проблемы моментов
Гамбургера для
$ \{ f_k \} : $
$$
  f_{2k}=s_k , \quad f_{2k+1}=0, \quad k=0,1,2,...
$$
Необходимым и достаточным условием разрешимости проблемы
моментов Гамбургера является положительность следующих
определителей:
\begin{equation*}
  H_n [\{ f_k \} ]=
    \begin{vmatrix}
	  f_0 & f_1 & \dots & f_{n-1} \\
	  f_1 & f_2 & \dots & f_n \\
	  \dots & \dots & \dots & \dots \\
	  f_{n-1} & f_n & \dots & f_{2n-2}
	\end{vmatrix}
  >0.
\end{equation*}
Подставляя в последний определитель
$  \{ s_k \} $
и производя соответсвующую перестановку стобцов и строк,
получаем
\begin{equation*}
  H_n [\{ f_k \} ]=
    \begin{vmatrix}
	  1 & 0 & s_1 & \; & \; \\
	  0 & s_1 & 0 & \; & \; \\
	  s_1 & 0 & s_2 & \; & \; \\
	  \; & \; & \; & \dots & \; \\
	  \; & \; & \; & \; & s_{n-1}
	\end{vmatrix}
  =
\end{equation*}
\begin{equation*}
  =
    \begin{vmatrix}
	  1 & s_1 & s_2 & \dots &
	   s_{\bigl [ \frac{n-1}{2} \bigr ] } &
	    \; & \; & \; & \; \\
	  s_1 & s_2 & s_3 \; & \; & \dots &
	   \; & \; & \; & \; \\
	  s_2 & s_3 & s_4 & \; & \dots &
	   \; & \; & \; & \; \\
	   \dots & \; & \; & \dots & \dots &
	   \; & \; & \; & \; \\
	  s_{\bigl [ \frac{n-1}{2} \bigr ] } &
	   \dots & \dots & \dots & s_{2 \bigl [ \frac{n-1}{2} \bigr ] } &
	    \; & \; & \; & \; \\
	       \; & \; & \; & \; & \; &
		           s_1 & s_2 & \dots &
				      s_{\bigl [\frac{n}{2}\bigr ] } \\
		   \; & \; & \; & \; & \; &
		           s_2 & s_3 & \; & \dots \\
		   \; & \; & \; & \; & \; &
		           \dots & \; & \dots & \dots \\
		   \; & \; & \; & \; & \; &
		           s_{ \Bigl [\frac{n}{2} \Bigr ] }
				   & \dots & \dots &
				     s_{2 \bigl [ \frac{n}{2} \bigr ] -1 }
	 \end{vmatrix}
   =
\end{equation*}
\begin{equation*}
  = H_{\bigl [ \frac{n-1}{2} \bigr ] +1} [ \{ s_k \} ] \cdot
    \tilde H_{ \bigl [ \frac{n}{2} \bigr ] } [ \{ s_k \} ] ,
	  \quad n=1,2,...
\end{equation*}
Проиллюстрируем последнее равенство на определителях
третьего и четвертого порядка:
\begin{equation*}
  \begin{vmatrix}
    1 & 0 & s_1 \\
	0 & s_1 & 0 \\
	s_1 & 0 & s_2
  \end{vmatrix}
  =-
  \begin{vmatrix}
    1 & s_1 & 0 \\
	0 & 0 & s_1 \\
	s_1 & s_2 & 0
  \end{vmatrix}
  =
  \begin{vmatrix}
    1 & s_1 & 0 \\
	s_1 & s_2 & 0 \\
	0 & 0 & s_1
   \end{vmatrix}
   ,
\end{equation*}
\begin{equation*}
  \begin{vmatrix}
    1 & 0 & s_1 & 0 \\
	0 & s_1 & 0 & s_2 \\
	s_1 & 0 & s_2 & 0 \\
	0 & s_2 & 0 & s_3
  \end{vmatrix}
  =-
  \begin{vmatrix}
    1 & s_1 & 0 & 0 \\
	0 & 0 & s_1 & s_2 \\
	s_1 & s_2 & 0 & 0 \\
	0 & 0 & s_2 & s_3
  \end{vmatrix}
  =
  \begin{vmatrix}
    1 & s_1 & 0 & 0 \\
    s_1 & s_2 & 0 & 0 \\
    0 & 0 & s_1 & s_2 \\
    0 & 0 & s_2 & s_3
  \end{vmatrix}
  .
\end{equation*}  		   				     	 				    		   		   			  		
Таким образом, учитывая, что решение проблемы моментов
Стилтьеса должно также решать проблему моментов Гамбургера,
получим:
$$
  H_n [ \{ s_k \} ] >0, \quad
  \tilde H_n [ \{ s_k \} ] > 0.
$$
Теорема доказана. $ \triangle $ \\
Из теоремы Стилтьеса и формул решения прямой спектральной
задачи (4.16) вытекает следующая теорема.
               %%%%%%%%  Теорема            %%%%%%%%%%%%%%%%%%%%%%%%%%%
\begin{The}
Для разрешимости проблемы моментов Стилтьеса
необходимо и достаточно, чтобы коэффициенты дроби Стилтьеса
были положительны:
$$
  a_n >0.
$$
\end{The}
\newpage
                 %%%%%%%%%%%%%%%%%%%%%%%%%%%%%%%%%%%%%%%%%%%%%%%%%%%%%%%%%%
				 %%%%%%%%%%   5.2   Сходимость дробей Стилтьеса   %%%%%%%%%
				 %%%%%%%%%%%%%%%%%%%%%%%%%%%%%%%%%%%%%%%%%%%%%%%%%%%%%%%%%%
\subsection{Сходимость дробей Стилтьеса}
Пусть
$ S_n (z) $
есть
$ n- $
я подходящая дробь Стилтьеса (4.2):
\begin{equation*}
  S_n (z) =
    \cfrac{1}{c_0 z+
	  \cfrac{1}{c_1 +
	    \cfrac{1}{c_2 z + \dotsb
		  \cfrac{1}{c_n \varepsilon _n
  }}}},
\end{equation*}
где
$ \varepsilon _n =z, $
если
$ n - $
четное, и
$ \varepsilon _n =1, $
если
$ n - $
нечетное число.\\
Справедлива
                  %%%%%%%%%%%   Теорема               %%%%%%%%%%%%%%%%%
\begin{The}
{\bfseries (Стилтьес).}
Пусть
$ c_n >0 . $
Тогда
\begin{equation*}
  \sum _{n=0}^{\infty} c_n = \infty \Longleftrightarrow
    S_n (z) \rightrightarrows S(z) \quad ( n \rightarrow \infty ) ,
	 \quad z \in K \Subset \mathbb{C}
	 \setminus [-\infty , 0].
\end{equation*}
\end{The}
Отметим сразу, что семейство
$ \{ S_n (z) \} $
компактно в
$ \mathbb{C} \setminus [-\infty , 0]. $
Действительно, условие
$ c_n >0 $
влечет, что проблема моментов Гамбургера
(и Стилтьеса на
$ [-\infty , 0] ) $
разрешима, поэтому повторяя рассуждения,
которые мы проводили при доказательстве теоремы
Маркова, получим:
\begin{equation*}
  S_n (z)=
    \frac{\Delta _{n,1}(z)}{\Delta _{n,0}(z)}=
	  \sum _{j=1}^m
	    \frac{\mu _{j,n}}{z-x_{j,n}},
		  \quad m=m(n), \qquad (5.2)
\end{equation*}
где полиномы
$ \Delta _{n,1} $
и
$ \Delta _{n,0} $
суть числители и знаменатели подходящих дробей
$ S_n , $
нули которых перемежаются и лежат на
$ [-\infty , 0] , $
а коэффициенты Кристоффеля положительны и в сумме дают единицу:
$$
  \mu _{j,n}=\frac{\Delta _{n,1}(x_{j,n})}
    {\Delta _{n,0}^{\prime}(x_{j,n})}>0 ,
	  \quad j=1,...,m,
$$
$$
  \sum _{j=1}^m \mu _{j,n}=1.
$$
Что дает оценку
$$
  |S_n (z)| \leqslant \frac{1}{\eta}
    \sum _{j=1}^m \mu _{j,n} =
	  \frac{1}{\eta}, \quad
	    z \in K \Subset \mathbb{C} \setminus [-\infty , 0],
$$
где
$$
  \eta =\mathrm{dist}(K,[-\infty , 0])=
    \min \{ |z-\lambda |: \; z \in K, \;
	  \lambda \in (-\infty , 0) \} .
$$
Таким образом справедливость теоремы Стилтьеса
будет следовать (по
{\bfseries теореме Витали})
из ее локального варианта, т.е. достаточно доказать
теорему для точек
$ z_0 \in (0,+\infty ) . $
Заметим, что при доказательстве теоремы Маркова выбор
$ z_0 =\infty $
тривиально решал задачу сходимости (вместе со всеми производными)
в одной точке области компактности.
В случае теоремы Стилтьеса этот тривиальный путь закрыт,
так как центр разложения дроби -- точка
$ \infty $
не принадлежит области компактности.\\
Итак, сформулируем локальный вариант теоремы Стилтьеса,
из которого будет следовать справедливость
основной теоремы.
                    %%%%%%%%%%%   Теорема                 %%%%%%%%%%%%%%%
\begin{The}
Пусть
$ c_n >0 , $
тогда
$ \forall z_0 \in (0,+\infty ) $
имеем
$$
  \sum _{n=0}^{\infty}c_n =\infty \Longleftrightarrow
    S_n (z_0 ) \longrightarrow S(z_0 ),
	  \quad n \rightarrow \infty .
$$
\end{The}
{\Large Доказательство.}
Сделаем упрощающие обозначения:
\begin{equation*}
  \mathtt{c}_n:=
    \begin{cases}
	  c_n z_0 , \; \mathrm{для} \; \mathrm{нечетных} \; n \\
	  c_n , \; \mathrm{для} \; \mathrm{четных} \; n ,
	\end{cases}
\end{equation*}
$$
  \Delta _n :=\Delta _{n,0}(z_0 ), \quad
    S_n :=S_n (z_0 ), \quad
	  S:= S(z_0 ).
$$
Наша цель -- доказать, что
$$
  \sum _{n=0}^{\infty}\mathtt{c}_n =\infty
    \Longleftrightarrow S_n \longrightarrow S,
	  \quad n \rightarrow \infty .
$$
Мы имеем рекуррентные соотношения:
$$
  \Delta _n =\mathtt{c}_{n-1} \Delta _{n-1}+
    \Delta _{n-2} , \quad
	  \Delta _0 =1, \; \Delta _{-1}=0.
	    \qquad (5.3)
$$
Введем параметр
$$
  t_n =\mathtt{c}_{n-1}
    \frac{\Delta _{n-1}}{\Delta _n},
	  \quad n \geqslant 1.
$$
Имеем
$$
  t_1 =1 \; \mathrm{и} \; 0<t_n <1, \; \; n>1.
    \qquad (5.4)
$$
Из рекуррентного соотношения получаем
$$
  S_n =t_n S_{n-1}+(1-t_n )S_{n-2},
    \quad S_0 =0, \; S_1 =1 .
	  \qquad (5.5)
$$
Определим последовательности:
$$
  M_n = \max \{ S_n ,S_{n-1} \}  \; \mathrm{и} \;
    m_n = \min \{ S_n , S_{n-1} \} .
	  \qquad (5.6)
$$
Из соотношений выпуклости (5.5) следует, что
$$
  0 \leqslant m_n \leqslant m_{n+1} \leqslant
    M_{n+1} \leqslant M_n \leqslant M_1 ,
	  \quad n \in \mathbb{N} .
	    \qquad (5.7)
$$			  	  		  			  	
Обозначая
$$
  \delta _n =M_n -m_n ,
$$
имеем
$$
  \delta _n \longrightarrow 0
    \Longleftrightarrow
	  S_n \longrightarrow S ,
	    \quad n \rightarrow \infty .
$$
Справедливость следующих двух лемм обеспечит последнюю импликацию,
и тем самым, докажет справедливость теоремы.
          %%%%%%%%%   Лемма (1)                   %%%%%%%%%%%%%%%%%%%%%%
\begin{Lem}
{\bfseries (1)}
$$
  \delta _n =(1-t_n )\delta _{n-1},
   \quad n \geqslant 2, \qquad \delta _1 >0.
$$
\end{Lem}
               %%%%%%%   Лемма (2)              %%%%%%%%%%%%%%%%%%%%%%%%%
\begin{Lem}
{\bfseries (2)}
$$
  \sum _{n=1}^{\infty}t_n =\infty
    \Longleftrightarrow
	  \sum _{n=0}^{\infty} \mathtt{c}_n =\infty .
$$
\end{Lem}
{\Large Доказательство леммы (1).}
Обозначим
\begin{equation*}
  \mathsf{X}_n=
    \begin{pmatrix}
	  S_n \\
	  S_{n-1}
	\end{pmatrix}
 ,
\end{equation*}
тогда
\begin{equation*}
  \mathsf{X}_n =\mathsf{T}_n \mathsf{X}_{n-1},
\end{equation*}
где, в силу (5.5),
\begin{equation*}
  \mathsf{T}_n =
    \begin{pmatrix}
	  \tau _n ^{(1,1)} & \tau _n ^{(1,2)} \\
	  \tau _n ^{(2,1)} & \tau _n ^{(2,2)}
	\end{pmatrix}
  =
    \begin{pmatrix}
	  t_n & 1-t_n \\
	  1 & 0
	\end{pmatrix}
  \qquad (5.8)
\end{equation*}
Замечаем, что для величин
$ M_n $
и
$ m_n , $
определенных в (5.6), выполняются соотношения:
\begin{align*}
  \; & M_n = \alpha _n ^{(1)} M_{n-1}+
    \alpha _n ^{(2)} m_{n-1} \\
  \; & m_n =\beta _n ^{(1)} M_{n-1} +
    \beta _n ^{(2)}m_{n-1},
\end{align*}
где
\begin{equation*}
  \{ \alpha _n ^{(j)} , \beta _n ^{(j)} \} _{j=1,2}
    \subset \{ \tau _n ^{(i,j)} \} _{i,j \in \{ 1,2 \} }
\end{equation*}
и
\begin{equation*}
  \alpha _n ^{(1)} +\alpha _n ^{(2)} =
    \beta _n ^{(1)} + \beta _n ^{(2)} =1 .
\end{equation*}
Таким образом, мы получили, что
\begin{align*}
  \; & M_n =(1-\alpha _n ^{(2)}) M_{n-1}+
     \alpha _n ^{(2)} m_{n-1} \\
  \; & m_n =(1 - \beta _n ^{(2)})M_{n-1}+
    \beta _n ^{(2)} m_{n-1} ,
\end{align*}
и следовательно,
\begin{equation*}
  \delta _n =(1-(\alpha _n ^{(2)} +\beta _n ^{(1)}))
    \delta _{n-1}, \quad n \geqslant 1,
	  \quad \delta _1 >0.
\end{equation*}
Переходя к параметрам матрицы
$ \mathsf{T}  $
имеем
\begin{equation*}
  \delta _n =(1-(\tau _n ^{(i_1 ,j_1 )}+
    \tau _n ^{(i_2 ,j_2 )})) \delta _{n-1},
	  \quad i_1 \not = i_2 ,
	    \quad j_1 \not = j_2 .
\end{equation*}
Тем самым, возможны два случая:
\begin{align*}
  1) & \; (i_1 , j_1 )=(1,1), \;
    (i_2 , j_2 )=(2,2) \Longrightarrow
	  \delta _n =(1-t_n )\delta _{n-1} \\
  2) & \; (i_1 , j_1 )=(1,2) , \;
    (i_2 , j_2 )=(2,1) \Longrightarrow
	  \delta _n =(2-t_n )\delta _{n-1}.
\end{align*}
Но случай 2) не может реализовываться ни для какого
$ n , $
так как в виду (5.4) в этом случае мы бы имели
$$
  \delta _n =(2-t_n )\delta _{n-1} > \delta _{n-1},
$$
что противоречит (5.7). Таким образом для любого
$ n $ мы имеем справедливость случая 1). Лемма доказана.
$ \triangle $ \\
Доказательству леммы 2 мы предпошлем лемму 3.
\begin{Lem}
{\bfseries (3)}
\begin{equation*}
  \sum _{n=0}^{\infty}\mathtt{c}_n =\infty
    \Longleftrightarrow \sum _n \Bigl (
	  1- \frac{\Delta _{n-2}}{\Delta _n}
	    \frac{\Delta _{n-3}}{\Delta _{n-1}}
		  \Bigr ) =\infty .
\end{equation*}
\end{Lem}
{\Large Доказательство леммы (3).}
Сначала мы докажем, что
\begin{equation*}
  \sum _{n=0}^{\infty}\mathtt{c}_n =+\infty
    \Longrightarrow \sum _n \Bigl (
	  1-\frac{\Delta _{n-2}}{\Delta _n}
	    \frac{\Delta _{n-3}}{\Delta _{n-1}}
		  \Bigr ) =+\infty .
		    \qquad (5.9)
\end{equation*}
Из рекуррентных соотношений (5.3) имеем
$$
  \Delta _n >\Delta _{n-2} ,
$$
что влечет
$$
  \Delta _n > \min \{ \Delta _0 , \Delta _1 \} >0
    \Longrightarrow \sum _n \mathtt{c}_{n-1}
	  \Delta _{n-1}^2 =\infty .
$$
Последнее соотношение, в свою очередь, влечет
$$
 \Delta _n \Delta _{n-1} \longrightarrow \infty ,
 \quad n \rightarrow \infty . \qquad (5.10)
$$
Действительно, рекуррентные соотношения (5.3) дают
$$
  \Delta _n \Delta _{n-1}-
    \Delta _{n-1} \Delta _{n-2} =
	  \mathtt{c}_{n-1}\Delta _{n-1}^2
$$
и суммирование последнего равенства дает (5.10).\\
Теперь мы будем считать, что
$$
  \frac{\Delta _{n-2}}{\Delta _n}
    \frac{\Delta _{n-3}}{\Delta _{n-1}}
	  \longrightarrow 1 , \quad
	    n \rightarrow \infty \qquad (5.11)
$$
ибо в противном случае правая часть импликации (5.9)
выполнен атоматически. Тогда
$ \exists N_0 : $			   	    			
$$
 1>\frac{\Delta _{n-2}}{\Delta _n}
   \frac{\Delta _{n-3}}{\Delta _{n-1}}>\frac{1}{2},
     \quad \forall n \geqslant N_0 .
$$
Тем самым получаем неравенства
$$
  1-\frac{\Delta _{n-2}}{\Delta _n }
    \frac{\Delta _{n-3}}{\Delta _{n-1}}
	  >\frac{1}{2} \Bigl (
	    \frac{\Delta _n}{\Delta _{n-2}}
		  \frac{\Delta _{n-1}}{\Delta _{n-3}}-1 \Bigr )
		    >\frac{1}{2} \log
			  \frac{\Delta _n}{\Delta _{n-2}}
			    \frac{\Delta _{n-1}}{\Delta _{n-3}}
$$
Суммируя получаем
$$
  \sum _{n=N_0}^N \Bigl (
    1-\frac{\Delta _{n-2}}{\Delta _n}\frac
	  {\Delta _{n-3}}{\Delta _{n-1}} \Bigr ) >
	    \frac{1}{2} \log \frac
		  {\Delta _N \Delta _{N-1}^2 \Delta _{N-2}}
		    {\Delta _{N_0 -1}\Delta _{N_0 -2}^2 \Delta _{N_0 -3}}.
$$
Теперь (5.10) и и отграниченность
$ \Delta _n $
снизу от нуля для любого
$ n $
дают
$$
  \Delta _n \Delta _{n-1}^2 \Delta _{n-2}
    \longrightarrow \infty ,
	  \quad n \rightarrow \infty ,
$$
что приводит к справедливости импликации (5.9). \\
Осталось доказать импликацию, обратную к (5.9), т.е. что
$$
  \sum _n \Bigl ( 1-
    \frac{\Delta _{n-2}}{\Delta _n}
	  \frac{\Delta _{n-3}}{\Delta _{n-1}} \Bigr )
	    =\infty \Longrightarrow \sum _n \mathtt{c}_n =\infty .
		  \qquad (5.12)
$$
Для справедливости этой импликации достаточно доказать, что
$$
  \Delta _n +\Delta _{n-1} \longrightarrow \infty ,
     n \rightarrow \infty . \qquad (5.13)
$$
Действительно, так как
$$
  \mathtt{c}_{n-1}>\frac{\mathtt{c}_{n-1}\Delta _{n-1}}
    {\Delta _{n-1}+\Delta _{n-2}}=
	  \frac{\Delta _n +\Delta _{n-1}}
	    {\Delta _{n-1} +\Delta _{n-2}}-1
$$
и в виду неравенства
$ \Delta _n >\Delta _{n-2} , $
имеем
$$
  \frac{\Delta _n +\Delta _{n-1}}
    {\Delta _{n-1}+\Delta _{n-2}}>1.
$$
Поэтому из неравенств
$$
  \mathtt{c}_{n-1}>
    \frac{\Delta _n +\Delta _{n-1}}
	  {\Delta _{n-1}+\Delta _{n-2}}-1>
	  \log \frac{\Delta _n +\Delta _{n-1}}
	    {\Delta _{n-1}+\Delta _{n-2}}
$$
получаем
$$
  \sum _{n=0}^{N-1}\mathtt{c}_n >
    \log ( \Delta _N +\Delta _{N-1} ) .
$$
Это доказывает, что (5.13) влечет правую часть (5.12).\\
Докажем (5.13). Имеем 				 			 		  	   								   	  	
$$
  1-\frac{\Delta _{n-1}\Delta _{n-3}}
    {\Delta _n \Delta _{n-1}} <
	  \log \frac{\Delta _n \Delta _{n-1}}
	    {\Delta _{n-2}\Delta _{n-3}},
$$
Следовательно,
$$
  \sum _{n=3}^N \Bigl (
    1-\frac{\Delta _{n-2}\Delta _{n-3}}
	  {\Delta _n \Delta _{n-1}} \Bigr ) <
	    \sum _{n=3}^N \log
		  \frac{\Delta _n \Delta _{n-1}}
		    {\Delta _{n-2}\Delta _{n-3}}=
			  \log
			    \frac{\Delta _N \Delta _{N-1}^2 \Delta _{n-2}}
				  {\Delta _2 \Delta _1 ^2 \Delta _0} ,
$$
Поэтому левая часть (5.12) влечет
$$
  \Delta _N \Delta _{N-1}^2 \Delta _{N-2} \longrightarrow \infty ,
    \quad N \rightarrow \infty ,
$$
откуда, с учетом опять же того, что
$ \Delta _n > \Delta _{n-2} , $
получаем (5.13).
Лемма 3 доказана. $ \triangle $ \\
{\Large Доказательство леммы 2.}
Из рекуррентных соотношений (5.3) имеем
$$
  1-\frac{\Delta _{n-2}\Delta _{n-3}}
    {\Delta _n \Delta _{n-1}}=
	  \frac{\Delta _n (\Delta _{n-1}-\Delta _{n-3})+
	    \Delta _{n-3}(\Delta _n -\Delta _{n-2})}
		  {\Delta _n \Delta _{n-1}}=
$$
$$
  = \mathtt{c}_{n-2}\frac{\Delta _{n-2}}{\Delta _{n-1}}+
    \mathtt{c}_{n-1}\frac{\Delta _{n-3}}{\Delta _n} =
	  t_{n-1}+t_n (1-t_{n-1}).
$$
Оценка (5.4) дает:
$$
  \sum _n t_n =\infty \Longleftrightarrow
    \sum _n (t_{n-1}+t_n (1-t_{n-1}))=\infty .
$$
Следовательно, справедливость леммы 3 доказывает \\
лемму 2.
$ \triangle $ \\
Теорема сходимости Стилтьеса доказана. $ \triangle $
\newpage
             %%%%%%%%%%%%%%%%%%%%%%%%%%%%%%%%%%%%%%%%%%%%%%%%%%%%%%%%%%%%%
			 %%%%%%%%%%%   5.3   Определенность проблемы моментов   %%%%%%
			 %%%%%%%%%%%         Стилтьеса          %%%%%%%%%%%%%%%%%%%%%%
			 %%%%%%%%%%%%%%%%%%%%%%%%%%%%%%%%%%%%%%%%%%%%%%%%%%%%%%%%%%%%%
\subsection{Определенность проблемы моментов Стилтьеса}
Следствием доказанной теоремы Стилтьеса о сходимости
является следующий критерий определенности
проблемы моментов Стилтьеса.
              %%%%%%%%%%%%   Теорема             %%%%%%%%%%%%%%%%%%%%%%%%%
\begin{The}
Пусть
$ \{ c_j \} $
коэффициенты непрерывной дроби (4.2), тогда
$$
  c_j >0, \; j=0,1,2,..., \; \;
    \sum _{j=0}^{\infty}c_j =\infty
	  \Longleftrightarrow
	    \exists ! \; ds(x) : \;
		  s_k = \int _{-\infty}^0 x^k ds(x) .
$$
\end{The}
{\Large Доказательство.(Достаточность).}
Ввиду положительности
$ c_j $
проблема моментов Стилтьеса разрешима, т.е.
$ \exists ds(x): \; \mathrm{supp} \subset \mathbb{R}^- , $
такая что непрерывная дробь (4.2) является формальным
разложением асимптотического ряда, задаваемого функцией
$$
  \int _{-\infty}^0 \frac{ds(x)}{z-x} .
$$
Условие расходимости ряда из коэффициентов дроби
по теореме Стилтьеса влечет сходимость дроби в
$ \mathbb{C} \setminus [-\infty ,0] $
к этой функции. Поэтому если существуют две меры
$ ds(x) $ и $ d \tilde s(x) , $
решающие проблему моментов Стилтьеса, то их
преобразования Коши
$$
  S(z):=\int _{-\infty}^0 \frac{ds(x)}{z-x}, \quad
    \tilde S(z):=\int _{-\infty}^0 \frac
	{d\tilde s(x)}{z-x}
$$
совпадают в
$ \mathbb{C} \setminus [-\infty , 0] , $
и следовательно совпадают предельные граничные значения
двух одинаковых аналитических функций
$ S $ и $ \tilde S , $
и по формуле Стилтьеса-Перрона (см. параграф 2.6)
совпадают меры
$ ds $ и $ d\tilde s .$ \\
{\Large (Необходимость).}
Обозначим числители и знаменатели (см. (5.2)) четных и
нечетных подходящих дробей как	
\begin{align*}
  V_n & = \Delta _{2n+1,1}, \quad
    H_n  = \Delta _{2n,1} , \\
  U_n & = \Delta _{2n+1,0} , \quad
    G_n  = \Delta _{2n,0} ,
\end{align*}
где
$ n=0,1,2,... $
Докажем сначала оценку сверху (при
$ x>0) $
$$
  G_{n+1}(x)U_n (x) < \frac{1}{2}
    (G_{n+1}(x)+U_n (x))^2 <
$$
$$
	 <\frac{1}{2}
	  \exp \Bigl (2 \sum _{k=1}^n l_k \Bigr )
	  \exp \Bigl (2x \sum _{k=1}^{n+1} m_k \Bigr ) ,
	    \qquad (5.14)
$$
где
$ l_k $ и $ m_k $
соответственно обозначают четные  нечетные коэффициенты
$ c_k $
непрерывной дроби (4.3).\\
Действительно, из рекуррентных соотношений (5.3)
следует, что
\begin{align*}
  \; & G_n +U_n \leqslant (1+l_n )
    (G_n +U_{n-1}) \\
  \; & G_{n+1}+U_n \leqslant (1+m_{n+1}x )
    (G_n +U_n ) ,
\end{align*}
окуда суммировнием получаем
$$
  G_{n+1}+U_n \leqslant (1+l_n )...(1+l_1 )
    (1+m_{n+1}x)...(1+m_1 x) <
$$
$$
  < \exp \Bigl ( \sum _{k=1}^n l_k \Bigr )
    \exp \Bigl ( x \sum _{k=1}^{n+1} m_k \Bigr ) ,
$$
что доказывает справедливость (5.14).\\
Также отметим вытекающее из рекуррентного соотношения
тождество
$$
  H_{n+1}U_n -G_{n+1}V_n =1. \qquad (5.15)
$$
Рассмотрим теперь (см. (5.2)) подходящие дроби
$$
  \frac{V_n}{U_n}(z)=\int _{-\infty}^0
    \frac{d\tau _n (x)}{z-x}, \quad
	  \frac{H_{n+1}}{G_{n+1}}(z)=
	    \int _{-\infty}^0
		  \frac{d \omega _n (x)}{z-x},
$$
где
$ \tau _n $
и
$ \omega _n $
дискретные меры, решающие усеченную проблему моментов.
С учетом (5.14) и (5.15) имеем
$$
  \int _{-\infty}^0 \frac{d\omega _n (x)}{z-x}-
    \int _{-\infty}^0 \frac{d\tau _n (x)}{z-x} >
	  2 \exp \Bigl ( -2 \sum _{k=1}^n l_k -
	    2z \sum _{k=1}^{n+1} m_k \Bigr ) .
$$
Поэтому если ряд из коэффициентов дроби сходится, то
по теореме Хелли существуют меры
$ d \omega $
и
$ d \tau , $
решающие проблему моментов Стилтьеса, такие что
$$
  \int _{-\infty}^0 \frac{d\omega (x)}{z-x} -
  \int _{-\infty}^0 \frac{d\tau (x)}{z-x} >0
  \quad \mathrm{при} \quad x>0,
$$
а это противоречит определенности проблемы моментов.\\
Теорема доказана. $ \triangle $\\
Таким образом, мы можем отметить, что проблема
моментов Стилтьеса определена,
тогда и только тогда, когда
дробь Стилтьеса сходится.
При этом, так как подходящие дроби Стилтьеса
с положительными коэффициентами
образуют компактное семейство,
то все предельные точки этого семейства
представляют множество неединственности
решения проблемы моментов.
\newpage
           %%%%%%%%%%%%%%%%%%%%%%%%%%%%%%%%%%%%%%%%%%%%%%%%%%%%%%%%%%%%%%%%%%
		   %%%%%%%%%%%%%%%%%%%%%%%%%%%%%%%%%%%%%%%%%%%%%%%%%%%%%%%%%%%%%%%%%%
		   %%%%%%%%%%   6   Приложения непрерывных дробей   %%%%%%%%%%%%%%%%%
		   %%%%%%%%%%       Стилтьеса к линейным и нелинейным   %%%%%%%%%%%%%
		   %%%%%%%%%%       механическим системам               %%%%%%%%%%%%%
		   %%%%%%%%%%%%%%%%%%%%%%%%%%%%%%%%%%%%%%%%%%%%%%%%%%%%%%%%%%%%%%%%%%
		   %%%%%%%%%%%%%%%%%%%%%%%%%%%%%%%%%%%%%%%%%%%%%%%%%%%%%%%%%%%%%%%%%%
\section{Приложения непрерывных дробей \\
Стилтьеса к линейным
и нелинейным \\
механическим системам}
           %%%%%%%%%%%%%%%%%%%%%%%%%%%%%%%%%%%%%%%%%%%%%%%%%%%%%%%%%%%%%%%%%
		   %%%%%%%%%%%%   6.1   Дискретная струна Стилтьеса- Крейна   %%%%%%
		   %%%%%%%%%%%%%%%%%%%%%%%%%%%%%%%%%%%%%%%%%%%%%%%%%%%%%%%%%%%%%%%%%
\subsection{Дискретная струна Стилтьеса-Крейна}
           %%%%%%   (a)   Постановка задачи   %%%%%%%%%%%%%%%%%%%%%%%%%%%%%%
{\bfseries (a) Постановка задачи.} \\

Пусть точечные массы (бусинки) величины
$$
  m_1 , \; m_2 , \; ...
$$
помещены в точках действительной оси с координатами
$$
  \xi _1 , \; \xi _2 , \; ...
$$
и соединены невесомой упругой нитью, натянутой с единичной силой.
Эту систему будем называть
{\bfseries дискретной струной}.
Рассмотрим движения дискретной струны, при которых каждая точка
$ \xi $
движется параллельно оси ординат по закону
$$
  u(\xi ,t)=\eta (\xi , \omega )\sin \omega t ,
$$
где частота
$ \omega $
одинакова для всех точек, а
$ \eta (\xi ,\omega ) $
-- амплитуда.
(Для большей наглядности можно представлять себе бусинки
нанизанными на вертикальные стержни, по которым они,
соединенные резинкой, перемещаются без трения.
При этом отметим, что присутствие в модели вертикальных
стержней не обязательно, так как мы принебрегаем явлениями
второго порядка малости по сравнению с малыми
гармоническими колебаниями.)\\
Обозначим расстояния между сосредоточенными массами (бусинками)
через
$$
  l_n =\xi _{n+1}-\xi _n ,
    \quad n=1,2,..., \qquad
	  l_0 =\xi _1 =0,
$$
амплитуду угла наклона нити, соединяющей
$ n- $
ю и
$ (n+1)- $
ю бусинки -- через
$$
  \theta _n , \quad n=1,2,..., \qquad \theta _0 =0,
$$
а амплитуду
$ n- $
й бусинки -- через
$$
  \eta _n :=\eta (\xi _n ,\omega ),
    \quad n=1,2,..., \qquad \eta _1 =1.
$$
Начальные условия
$$
  \theta _0 =0, \quad \eta _1 =1,
$$
механически интерпретируются тем, что левый конец струны
свободен (нулевая бусинка помещена в точку
$ -\infty ) $
и совершает малые колебания (нормировка
$ \eta _1 =1 $
характеризует степень малости). \\
Выведем уравнения движения струны.
Согласно закону Гука, сила упругости, действующая
на бусинку в вертикальном направлении, равна
разности ее отклонений от соседних бусинок, т.е.
$$
  F_n (\omega ,t)=\sin \theta _n \sin \omega t-
    \sin \theta _{n-1}\sin \omega t.
$$
Здесь коэффициент упругости считается равным единице.
В то же время, по второму закону Ньютона имеем
$$
  m_n \ddot u(\xi _n ,t)=-m_n \eta _n \omega ^2 \sin \omega t.
$$
Предполагаем относительные угловые отклонения частиц
маленькими:
$$
  \theta _n \sim \sin \theta _n \sim \tg \theta _n ,
    \quad n=1,2,3,...
$$
Условие малости
$ \theta _n $
и закон движения дают нам следующую разностную систему
1-го порядка для определения амплитудных функций
$ \eta _n $
и
$ \theta _n : $
\begin{equation*}
  \begin{cases}
    \eta _{n+1}-\eta _n =l_n \theta _n \\
	\theta _n -\theta _{n-1}=m_n \omega ^2 \eta _n
  \end{cases}
  \quad n=1,2,... \qquad (6.1)
\end{equation*}
с начальными условиями
$$
  \eta _1 =1, \quad \theta _0 =0 .
    \qquad (6.2)
$$
Обратим внимание, что полагая
$ -\omega ^2 =z, $
мы получаем, что амплитудные функции
$ \eta _n (z) $
и
$ \theta _n (z) $
удовлетворяют той же разностной задаче, что и знаменатели
подходящих непрерывной дроби Стилтьеса. \\
Заметим, что амплитудная функция струны, несущей
непрерывное распределение масс
$ \rho (\xi ) , $
удовлетворяет дифференциальному уравнению
$$
  \frac{d}{d\xi}\frac{1}{\rho (\xi )}
    \frac{d}{d\xi} \eta (\xi ,\omega )-
	  \omega ^2 \eta (\xi , \omega )=0.
$$
Обобщая исследования Т.Стилтьеса,
М.Г.Крейн построил теорию струны, которая состоит из
сосредоточенных масс и непрерывно распределенной массы.
$$ \; $$
              %%%%%%   (b) Дискретная струна с конечным числом масс   %%%%
{\bfseries (b) Дискретная струна с конечным числом масс.} \\

Рассмотрим конечную струну
$ \{ m_j ,l_j \} _{j=1}^n , $
закрепленную в точке $ 1: $
$$
  l_0 =0 , \quad
    \sum _{j=1}^n l_j =1.
$$
Величину
\begin{equation*}
  \Gamma (\lambda )=-
    \cfrac{1}{m_1 \lambda -
	  \cfrac{1}{l_1 -\dotsb -
	    \cfrac{1}{m_n \lambda -
		  \cfrac{1}{l_n
	}}}}
	\qquad (6.3)
\end{equation*}
называют
{\bfseries коэффициентом динамической податливости.} \\
Обозначим через
$ Q_n (\lambda ) $
и
$ P_n (\lambda ) $
соответственно знаменатель и числитель этой дроби.
Легко видеть, что
$$
  Q_n (0)=1,
    \quad P_n (0)=l_1 +l_2 +...+l_n =1.
$$
Поэтому
\begin{equation*}
  \Gamma (\lambda )=
    \frac{P_n (\lambda )}{Q_n (\lambda )}=
	  \frac{\prod _{j=1}^n \Bigl ( 1-
	    \frac{\lambda}{\mu _j} \Bigr ) }
	  {\prod _{j=1}^n \Bigl ( 1-
	    \frac{\lambda}{\lambda _j} \Bigr ) }.
   \qquad (6.4)
\end{equation*}
Так как
$ \Gamma (\lambda ) $
обращается в бесконечность, когда
$ \omega = \sqrt{\lambda} $
совпадает с одной из резонансных частот, т.е.
с одной из частот
$ \omega _j , \; j=1,...,n, $
собственных гармонических колебаний, то
$$
  \lambda _j =\omega _j ^2 , \quad j=1,...,n.
$$
Аналогично,
$ \Gamma (\lambda ) $
обращается в нуль, если
$ \omega =\sqrt{\lambda} $
совпадает с одной из антирезонансных частот,
т.е. с одной из резонансных частот
$ \tilde \omega _j $
нашей струны при закреплении ее левого конца.
Таким образом,
$$
  \mu _j =\tilde \omega _j ^2 ,
    \quad j=1,...,n-1.
$$
Отметим, что соотношения (6.3), (6.4) дают решение следующей
обратной задачи.
Пусть заданы
$ 2n-1 $
чисел
$$
  0<\omega _1 <\tilde \omega _1 < \omega _2 <
    \tilde \omega _2 <...<\tilde \omega _{n-1}<\omega _n .
	  \qquad (6.5)
$$
Тогда (6.3) и (6.4) определяют единственную конечную струну
с заданными резонансами
$ \{ \omega _j \} $
и антирезонансами
$ \{ \tilde \omega _j \} . $ \\
Рассмотрим предпоследнюю подходящую дробь непрерывной
дроби (6.3):
\begin{equation*}
  \Gamma _1 (\lambda )=
    \frac{P_{n-1}(\lambda )}{Q_{n-1}(\lambda )}=-
	  \cfrac{1}{m_1 \lambda -
	    \cfrac{1}{l_1 - \dotsb -
		  \cfrac{1}{m_{n-1}\lambda -
		    \cfrac{1}{l_{n-1}-
			  \cfrac{1}{m_n \lambda
	  }}}}}.
\end{equation*}
Легко видеть, что
$$
  Q_{n-1}(\lambda )=-M\lambda +...+a_n ^{(1)} \lambda ^n ,
    \quad M=m_1 +...+m_n .
$$
              %%%%%%%%%%%%%%   Утверждение    %%%%%%%%%%%%%%%%%%%%%%%%%%%
\begin{Sta}
Полная масса струны связана с резонансными частотами
по формуле
$$
  M=-\sum _{j=1}^n \frac{1}
    {\lambda _j ^2 Q_n ^{\prime}(\lambda _j )
	  P_n (\lambda _j )}.
	    \qquad (6.6)
$$
\end{Sta}
{\Large Доказательство.}
Очевидно, что
$$
  \frac{Q_{n-1}(\lambda )}{Q_n (\lambda )}=
    \lambda \sum _{j=1}^n \frac{Q_{n-1}(\lambda _j )}
	  {\lambda _j Q_n ^{\prime}(\lambda _j ) (\lambda -\lambda _j )}.
$$
С другой стороны, полагая
$ \lambda = \lambda _j $
в формуле
$$
  Q_n (\lambda )P_{n-1}(\lambda )-
    Q_{n-1}(\lambda )P_n (\lambda ) =1
$$
находим, что
$$
  -Q_{n-1}(\lambda _j )P_n (\lambda _j )=1,
    \quad j=1,...,n.
$$
Откуда
$$
  \frac{Q_{n-1}(\lambda )}{Q_n (\lambda )}=-\lambda
    \sum _{j=1}^n \frac{1}
	  {\lambda _j Q_n ^{\prime}(\lambda _j )
	    P_n (\lambda _j )(\lambda -\lambda _j )}.
$$
Разделив равенство на
$ \lambda $
и устремляя
$ \lambda $
к нулю, получаем (6.6). \\
Утверждение доказано.
$ \triangle $
               %%%%%%%%%%%   Теорема   %%%%%%%%%%%%%%%%%%%%%%%%%%%%%%%%%%%
\begin{The}
Пусть
$$
  \lambda _1 < \lambda _2 < ...<\lambda _n
$$
произвольная последовательность положительных чисел.
Тогда для всех конечных струн
$ \{ m_j , l_j \} _{j=1}^n , $
имеющих
$$
  \omega _j =\sqrt{\lambda _j},
    \quad j=1,...,n,
$$
своими резонансными частотами, величина
$$
  \tilde M := \Biggl (
    \sum _{j=1}^n \frac{1}
	  {\lambda _j ^{3/2} |Q^{\prime}(\lambda _j ) | }
	    \Biggr ) ^2 ,
$$
где
$$
  Q(\lambda )=\prod _{j=1}^n \biggl ( 1-
    \frac{\lambda}{\lambda _j} \biggr ) ,
$$
минимизирует величину массы струны:
$$
  \tilde M \leqslant \sum _{j=1}^n m_j ,
$$
причем минимум достигается
$$
  \tilde M = \sum _{j=1}^n m_j
$$
для единственной струны, такой что ее динамический
коэффициент податливости равен
$$
  \tilde \Gamma (\lambda )=
    \frac{1}{\sqrt{\tilde M }} \sum _{j=1}^n
	  \frac{1}{\sqrt{\lambda _j}
	    |Q^{\prime}(\lambda _j ) | (\lambda _j -\lambda )}.
$$
\end{The}			   					
{\Large Доказательство.}
Струна, имеющая данный спектр частот
$ \{ \omega _j \} _{j=1}^n =\{ \sqrt{\lambda _j} \} _{j=1}^n , $
вполне определяется заданием своей системы антирезонансных частот
$ \{ \tilde \omega _j \} _{j=1}^{n-1}=\{ \sqrt{\mu _j} \} _{j=1}^{n-1} , $
которые должны удовлетворять неравенствам (6.5).
Согласно (6.6) масса
$ M $
может быть выражена в виде рациональной функции от
$ \{ \mu _j \} _{j=1}^{n-1} : $
$$
  M=M(\mu _1 , \mu _2 ,...,\mu _{n-1}),
$$
При этом, когда
$ \mu _k \rightarrow \lambda _k $
или
$ \mu _k \rightarrow \lambda _{k+1} , $
эта функция будет иметь полюсы. Следовательно, в области
$$
  \lambda _k < \mu _k < \lambda _{k+1} ,
    \quad k=1,...,n-1,
$$
функция
$ M(\mu _1 ,...,\mu _{n-1}) $
имеет по крайней мере один минимум, где выполняется условие:
$$
  -\frac{\partial M}{\partial \mu _k}=
    \frac{1}{\mu _k} \sum _{j=1}^n
	  \frac{1}
	    {\lambda _j Q_n ^{\prime}(\lambda _j )
		  P_n (\lambda _j ) (\mu _k -\lambda _j )}=0,
		    \quad k=1,...,n-1,
$$
Здесь мы воспользовались тем, что
$$
  \frac{\partial}{\partial \mu _k}
    \frac{1}{P_n (\lambda _j )} =
	  \frac{\partial}{\partial \mu _k}
	    \prod _{j=1}^n \Bigl ( 1-
		  \frac{\lambda _j}{\mu _j} \Bigr ) ^{-1} =
		    \frac{\lambda _j}{\mu _k P_n (\lambda _j )
			  (\mu _k -\lambda _j )}
$$
Таким образом, числа
$ \mu _k $
в точке минимума суть нули рациональной функции
$$
  R(\lambda )=\sum _{j=1}^n \frac{1}
    {\lambda _j Q_n ^{\prime}(\lambda _j ) P_n (\lambda _j )
	  (\lambda -\lambda _j )}.
$$
Так как, с другой стороны, эти числа суть нули многочлена
$ P_n (\lambda ) , $
то найдется константа
$ h , $
такая что
$$
  R(\lambda )=h\frac{P_n (\lambda )}{Q_n (\lambda )}=
    h \sum _{j=1}^n \frac{P_n (\lambda _j )}
	  {Q_n ^{\prime}(\lambda _j ) (\lambda -\lambda _j )}.
	    \qquad (6.7)
$$
Сопоставляя две последние формулы, находим, что
$$
  \lambda _j P_n (\lambda _j ) Q_n ^{\prime}(\lambda _j ) =
    \frac{Q_n ^{\prime}(\lambda _j )}
	  {h P_n (\lambda _j )}.
$$
Откуда следует, что
$ h>0 $
и
$$
  P_n (\lambda _j )= \pm \frac{1}
    {\sqrt{h \lambda _j}},
	  \quad j=1,...,n,
	    \qquad (6.8)
$$
Подставляя эти значения в (6.7), получим:
$$
  \frac{P_n (\lambda )}{Q_n (\lambda )}=
    -\frac{1}{\sqrt{h}} \sum _{j=1}^n
	  \frac{1}{\sqrt{\lambda}|Q_n ^{\prime}(\lambda _j )|
	    (\lambda -\lambda _j )}.
$$
Полагая
$ \lambda =0 $
и учитывая, что
$ P_n (0) =Q_n (0) =1 , $
находим
$$
  \sqrt{h}=\sum _{j=1}^n \frac{1}
    {\lambda _j ^{3/2} |Q_n ^{\prime}(\lambda _j )|}.
$$
С другой стороны, подстановка (6.8) в (6.6) дает
$$
  \tilde M = \sqrt{h} \sum _{j=1}^n \frac{1}
    {\lambda _j ^{3/2} |Q_n ^{\prime}(\lambda _j )|}=h.
$$
Последние три равенства полностью доказывают теорему.
$ \triangle $
$$ \; $$
          %%%%%%%%   (c) Дискретная струна с бесконечным   %%%%%%%%%%%%%
		  %%%%%%%%       числом масс                       %%%%%%%%%%%%%
{\bfseries (c) Дискретная струна с бесконечным числом масс.} \\

Если число сосредоточенных масс струны рано бесконечности,
она имеет чисто дискретный спектр, когда соответствующий
разностный оператор является компактным. Справедливо
          %%%%%%%%  Утверждение        %%%%%%%%%%%%%%%%%%%%%%%%%%%%%%%%%
\begin{Sta}
Дискретная струна с бесконечным числом масс имеет только дискретный
спектр (т.е. собственные частоты) тогда и только тогда, когда
$$
  \frac{1}{l_n \sqrt{m_n m_{n+1}}}
    \longrightarrow 0 , \quad n \rightarrow \infty ,
	  \qquad (6.9)
$$
$$
  \frac{1}{m_{n+1}} \Bigl (
    \frac{1}{l_n}+\frac{1}{l_{n+1}} \Bigr )
	  \longrightarrow \Omega ,
	    \quad n \rightarrow \infty . \qquad (6.10)
$$
При этом для собственных частот справедлиа формула:
$$
  \omega _j \longrightarrow \sqrt{\Omega}
    \quad \mathrm{при} \quad j \rightarrow \infty .
$$
\end{Sta}
{\Large Доказательство.}
Разностная система уранений (6.1) может быть записана
в виде одного уравнения
$$
  b_n y_{n+1}+a_n y_n +b_{n-1}y_{n-1}=\lambda y_n ,
$$
где
\begin{align*}
  \; & b_n =\frac{1}
    {l_{n+1}\sqrt{m_{n+1}m_{n+2}}}, \\
  \; & a_n =\frac{1}{m_{n+1}} \Bigl (
    \frac{1}{l_n}+\frac{1}{l_{n+1}} \Bigr ) , \\
  \; & y_n =(-1)^n \sqrt{m_{n+1}}\eta _{n+1} , \\
  \; & \lambda =\omega ^2 .
\end{align*}
Условия (6.9), (6.10) являются условиями компактности оператора,
задаваемого трехдиагональной матрицей Якоби с главной диагональю
$$
  a_n -\Omega , \quad n \in \mathbb{N} ,
$$
и примыкающими диагоналями
$$
  b_n , \quad n \in \mathbb{N}.
$$
При этом точки спектра этого оператора
$ \lambda _n -\Omega $
имеют единственную предельную точку --
точку нуль.
Утверждение доказано.
$ \triangle $
                  %%%%%%%%%%%%   Задача    %%%%%%%%%%%%%%%%%%%%%%%%%%%%%%%
\begin{Exe}
Найдите значения параметров струны
$ m_n , \; l_n , $
удовлетворяющие условиям (6.9) и (6.10), при которых
возможно в явном виде определить значения всех
основных (резонансных) частот.
\end{Exe}
В случае, когда для бесконечной струны условия (6.9)
и (6.10) не выполнены, структура ее спектра становится
сложней. Появляются участки непрерывного спектра,
а резонансы могут исчезнуть. Следующий пример демонстрирует струну,
не принадлежащую классу (6.9), (6.10), и тем не менее обладающую
бесконечным числом резонансов, которые можно вычислить в
явном виде.
                %%%%%%%%   Пример       %%%%%%%%%%%%%%%%%%%%%%%%%%%%%%%%
\begin{Exa}
Струна с параметрами
$$
  m_n =n , \quad l_n =\frac{1}{n+1} ,
    \quad \quad n=1,2,...
$$
имеет следующие резонансные частоты:
$$
  \omega _1 =\frac{3}{\sqrt{2}}, \quad
  \omega _2 =\frac{5}{\sqrt{6}},...,\quad
  \omega _n =\frac{2k+1}{\sqrt{k(k+1)}}, ..., \quad
  \omega _{\infty}=2.
$$
\end{Exa}
              %%%%%%%   Рисунок        %%%%%%%%%%%%%%%%%%%%%%%%%%%%%%%%%
			  %%%%%%%%%%%%%%%%%%%%%%%%%%%%%%%%%%%%%%%%%%%%%%%%%%%%%%%%%%
			  %%%%%%%%%%%   6.2   Цепочка Ленгмюра   %%%%%%%%%%%%%%%%%%%
			  %%%%%%%%%%%%%%%%%%%%%%%%%%%%%%%%%%%%%%%%%%%%%%%%%%%%%%%%%%
\newpage
\subsection{Цепочка Ленгмюра}
               %%%%%%%%%   (a) Постановка задачи. Непрерыные аналоги  %%%
{\bfseries (a) Постановка задачи. Непрерывные аналоги.} \\

{\bfseries Цепочкой Ленгмюра}
называют дискретную динамическую систему
$ \{ a_n (t) \} _{n \in \mathbb{N}}, $
описываемую нелинейной системой обыкновенных
дифференциальных уравнений
\begin{equation*}
  \begin{cases}
    \dot a_n =a_n (a_{n+1}-a_{n-1}),
	  \quad n=1,2,..., \qquad (6.11) \\
	a_0 =0 .
  \end{cases}
\end{equation*}
Для этой системы рассматривается задача Коши,
т.е. ищется эволюция заданных начальных данных
$$
  \{ a_n (0) \} \longrightarrow
    \{ a_n (t) \} .
	  \qquad (6.12)
$$
Кроме самостоятельного интереса цепочка Ленгмюра полезна как
дискретизация некоторых известных уравнений в частных
производных. Во-первых, при соответствующей замене переменной
времени система (6.11) является дискретизацией
{\bfseries уравнения Бюргерса}:
$$
  \frac{\partial a}{\partial t}=-a\frac{\partial a}{\partial x}.
$$
Во-вторых, в качестве непрерывного предела из (6.11)
можно получить
{\bfseries уравнение Кортевега-де Фриза}:
$$
  \frac{\partial v}{\partial \tau}=-
    \frac{\partial ^3 v}{\partial x^3}+
	  6v \frac{\partial v}{\partial x}.
$$
Действительно, записыая (6.11) в виде
$$
  \dot a (x)=a(x) (a(x+\delta )-a(x-\delta ))
$$
и полагая
$$
  a(x)=1-\delta ^2 v(x),
$$
имеем
$$
  \dot v = v(x+\delta )-v(x-\delta )-
    \delta ^2 v(x) (v(x+\delta )-v(x-\delta )).
$$
Разлагая в ряд по
$ \delta $
последнее соотношение, получаем
$$
  \dot v = 2 \delta v_x + \frac{1}{x}\delta ^3 v_{xxx}-
    2\delta ^3 v v_x + O(\delta ^5 ) .
$$
Замена
$ \tilde x =x-2\delta $
приводит к уравнению
$$
  \frac{\partial v}{\partial t}=\frac{1}{3}\delta ^3
    \frac{\partial ^3 v}{\partial \tilde x^3}-
	  2 \delta ^3 v \frac{\partial v}{\partial \tilde x}+
	    O(\delta ^3 ).
$$
И наконец, замена времени
$ \tau =-\frac{\delta ^3}{3}t $
приводит к уравнению
$$
  \frac{\partial v}{\partial \tau}=-
    \frac{\partial ^3 v}{\partial x^3}+6v
	  \frac{\partial v}{\partial x} +
	    O(\delta ^2 ).
$$
$$ \; $$
                       %%%%%%%%%%   (b) Метод обратной спектральной задачи %%
{\bfseries (b) Метод обратной спектральной задачи.} \\

Рассмотрим непрерывную дробь Стилтьеса
\begin{equation*}
  S(z)=
    \cfrac{1}{z-
	  \cfrac{a_1 (t)}{1-
	    \cfrac{a_2 (t)}{z-\dotsb
	}}}.
	  \qquad (6.13)
\end{equation*}
Пусть зависимость от
$ t $
коэффициентов дроби
$ a_n (t) $
описывается системой дифференциальных уравнений (6.11).
Оказывается, что в этом слуае эволюция спектральных
данных Стилтьеса
$$
  ( S, \; \{ s_k \} , \; ds ) ,
$$
где
$$
  S(z)=\sum _{k=0}^{\infty}
    \frac{s_k}{z^{k+1}}=
	  \int _0 ^{\infty}
	    \frac{ds(x)}{z-x},
$$
подчиняется достаточно простым законам.
Следующие три теоремы устанавливают эти законы.
                %%%%%%%%   Теорема       %%%%%%%%%%%%%%%%%%%%%%%%%%%%%%%%%
\begin{The}
Пусть последовательность
$ \{ a_n (t) \} _{n \in \mathbb{N}} $
удовлетворяет системе (6.11). Тогда коэффициенты степенного
разложения (моменты) резольвентной функции
$ S(z) $
удовлетворяют следующей системе:
$$
  \dot s_k (t)=s_{k+1}(t)-s_k (t)s_1 (t) .
    \qquad (6.14)
$$
\end{The}
{\Large Доказательство.}
Продифференцируем генетическую сумму (4.12)
$$
  s_k (t)=a_1 (t) \sum _{i_2 =1}^2 a_{i_2}(t)
    \sum _{i_3 =1}^{i_2 +1}a_{i_3}(t) ...
	  \sum _{i_n =1}^{i_{n-1}+1}a_{i_n}(t),
$$
используя зависимость (6.11) коэффициентов
$ a_i $
от
$ t $
$$
  \dot a_i (t)=a_i (t)(a_{i+1}(t)-a_{i-1}(t)),
    \quad a_0 (t)=0.
$$
Получаем
$$
  \dot s_k =a_1 a_2
    \sum _{i_2 =1}^{1+1}a_{i_2}
	  \sum _{i_3 =1}^{i_2 +1}a_{i_3}...
	    \sum _{i_{k-1}=1}^{i_{k-2}+1}a_{i_{k-1}}
		  \sum _{i_k =1}^{i_{k-1}+1}a_{i_k}+
$$
$$
  +a_1 \sum _{i_2 =1}^{1+1}a_{i_2}(a_{i_2 +1}-a_{i_2 -1})
    \sum _{i_3 =1}^{i_2 +1}a_{i_3}...
	  \sum _{i_k =1}^{i_{k-1}+1}a_{i_k}+...
$$
$$
  ...+a_1 \sum _{i_2 =1}^{1+1}a_{i_2}
    \sum _{i_3 =1}^{i_2 +1}a_{i_3}...
	  \sum _{i_{k-1}=1}^{i_{k-2}+1}a_{i_{k-1}}
	    (a_{i_{k-1}+1}-a_{i_{k-1}-1})
		  \sum _{i_k =1}^{i_{k-1}+1}a_{i_k}+
$$
$$
  +a_1 \sum _{i_2 =1}^{1+1}a_{i_2}...
    \sum _{i_{k-1}=1}^{i_{k-2}+1}a_{i_{k-1}}
	  \sum _{i_k =1}^{i_{k-1}+1}a_{i_k}.
$$
Для преобразования правой части используем тождество
$$
  a_1 \sum _{i_2 =1}^{1+1}a_{i_2}...
    \sum _{i_{p-1}=1}^{i_{p-2}+1}a_{i_{p-1}}
	  \sum _{i_p =1}^{i_{p-1}+1}a_{i_p}(a_{i_p +1}-a_{i_p -1})
	    \sum _{i_{p+1}=1}^{i_p +1}a_{i_{p+1}}...
		  \sum _{i_k =1}^{i_{k-1}+1}a_{i_k}=
$$
$$
  a_1 \sum _{i_2 =1}^{1+1}a_{i_2}...
    \sum _{i_{p-1}=1}^{i_{p-2}+1}a_{i_{p-1}}a_{i_{p-1}+1}a_{i_{p-1}+2}
	  \sum _{i_{p+1}=1}^{i_{p-1}+2}a_{i_{p+1}}
	    \sum _{i_{p+2}=1}^{i_{p+1}+1}a_{i_{p+2}}...
		  \sum _{i_k =1}^{i_{k-1}+1}a_{i_k}-
$$
$$
  -a_1 \sum _{i_2 =1}^{1+1}a_{i_2}...
    \sum _{i_{p-1}=1}^{i_{p-2}+1}a_{i_{p-1}}
	  \sum _{i_p =1}^{i_{p-1}}a_{i_p}a_{i_p +1}a_{i_p +2}
	    \sum _{i_{p+2}=1}^{i_p +3}a_{i_{p+2}}...
		  \sum _{i_k =1}^{i_{k-1}+1}a_{i_k},
$$
где
$ 1<p<k. $ \\
Имеем
$$
  \dot s_k =a_1 a_2
    \sum _{i_2 =1}^{1+1}a_{i_2}
	  \sum _{i_3 =1}^{i_2 +1}a_{i_3}...
	    \sum _{i_k =1}^{i_{k-1}+1}a_{i_k}+
$$
$$
  + \Biggl (
    a_1 a_2 a_3 \sum _{i_3 =1}^{1+2}a_{i_3}
	  \sum _{i_4 =1}^{i_3 +1}a_{i_4}...
	    \sum _{i_k =1}^{i_{k-1}+1}a_{i_k}-
  a_1 a_1 a_2 a_3
    \sum _{i_4 =1}^{1+3}a_{i_4}...
	  \sum _{i_k =1}^{i_{k-1}+1}a_{i_k}
	    \Biggr ) +
$$
$$
  +\Biggl ( a_1
    \sum _{i_2 =1}^{1+1}a_{i_2}a_{i_2 +1}a_{i_2 +2}
	  \sum _{i_4 =1}^{i_2 +2}a_{i_4}...
	    \sum _{i_k =1}^{i_{k-1}+1}a_{i_k}-
$$
$$
  -a_1
    \sum _{i_2 =1}^{1+1}a_{i_2}
	  \sum _{i_3 =1}^{i_2}a_{i_3}a_{i_3 +1}a_{i_3 +2}
	    \sum _{i_5 =1}^{i_3 +3}a_{i_5}...
		  \sum _{i_k =1}^{i_{k-1}+1}a_{i_k}
		    \Biggr ) +...
$$
$$ \; $$
$$
  ..............................................
$$
$$ \; $$
$$
  ...+ \Biggl ( a_1
    \sum _{i_2 =1}^{1+1}a_{i_2}...
	  \sum _{i_{k-3}=1}^{i_{k-4}+1}a_{i_{k-3}}a_{i_{k-3}+1}a_{i_{k-3}+2}
	    \sum _{i_{k-1}=1}^{i_{k-3}+2}a_{i_{k-1}}
		  \sum _{i_k =1}^{i_{k-1}+1}a_{i_k}-
$$
$$
  -a_1
    \sum _{i_2 =1}^{1+1}a_{i_2}...
	  \sum _{i_{k-3}=1}^{i_{k-4}+1}a_{i_{k-3}}
	    \sum _{i_{k-2}=1}^{i_{k-3}}a_{i_{k-2}}a_{i_{k-2}+1}a_{i_{k-2}+2}
		  \sum _{i_k =1}^{i_{k-2}+3} a_{i_k}
		    \Biggr ) +
$$
$$
  + \Biggl ( a_1
    \sum _{i_2 =1}^{1+1}a_{i_2}...
	  \sum _{i_{k-3}=1}^{i_{k-4}}a_{i_{k-3}}
	    \sum _{i_{k-2}=1}^{i_{k-3}+1}a_{i_{k-2}}a_{i_{k-2}+1}a_{i_{k-2}+2}
		  \sum _{i_k =1}^{i_{k-2}+2}a_{i_k}-
$$
$$
  -a_1 \sum _{i_2 =1}^{1+1}a_{i_2}...
    \sum _{i_{k-2}=1}^{i_{k-3}+1}a_{i_{k-2}}
	  \sum _{i_{k-1}=1}^{i_{k-2}}
	    a_{i_{k-1}}a_{i_{k-1}+1}a_{i_{k-1}+2} \Biggr ) +
$$
$$
  +a_1 \sum _{i_2 =1}^{1+1}a_{i_2}...
    \sum _{i_{k-2}=1}^{i_{k-3}+1}a_{i_{k-2}}
	  \sum _{i_{k-1}=1}^{i_{k-2}+1}
	    a_{i_{k-1}}a_{i_{k-1}+1}a_{i_{k-1}+2} .
$$
Складываем генетические суммы, начиная с последней
в обратном порядке. К последней прибавляем предпоследнюю,
получаем
$$
  a_1 \sum _{i_2 =1}^{1+1}a_{i_2}...
    \sum _{i_{k-2}=1}^{i_{k-3}+1}
	  a_{i_{k-2}}a_{i_{k-2}+1}a_{i_{k-2}+2}a_{i_{k-2}+3}.
$$
К результату прибавляем третью с конца, получаем
$$
  a_1 \sum _{i_2 =1}^{1+1}a_{i_2}...
    \sum _{i_{k-3}=1}^{i_{k-4}+1}a_{i_{k-3}}
	  \sum _{i_{k-2}=1}^{i_{k-3}+1}
	    a_{i_{k-2}}a_{i_{k-2}+1}a_{i_{k-2}+2}
		  \sum _{i_k =1}^{i_{k-2}+3} a_{i_k}.
$$
Затем прибавление еще двух предыдущих дает
$$
  a_1 \sum _{i_2 =1}^{1+1}a_{i_2}...
    \sum _{i_{k-4}=1}^{i_{k-5}+1}a_{i_{k-4}}
	  \sum _{i_{k-3}=1}^{i_{k-4}+1}
	    a_{i_{k-3}}a_{i_{k-3}+1}a_{i_{k-3}+2}
		  \sum _{i_{k-1}=1}^{i_{k-3}+3}a_{i_{k-1}}
		    \sum _{i_k =1}^{i_{k-1}+1}a_{i_k},
$$
и т.д. В итоге мы приходим к выражению
$$
  \dot s_k =a_1 a_2
    \sum _{i_2 =1}^{1+1}a_{i_2}
	  \sum _{i_3 =1}^{i_2 +1}a_{i_3}...
	    \sum _{i_k =1}^{i_{k-1}+1}a_{i_k}+
$$
$$
  +a_1 a_2 a_3
    \sum _{i_3 =1}^{1+3}a_{i_3}
	  \sum _{i_4 =1}^{i_3 +1}a_{i_4}...
	    \sum _{i_k =1}^{i_{k-1}+1}a_{i_k}=
$$
$$
  =a_1 a_2 \sum _{i_2 =1}^{3}a_{i_2}
    \sum _{i_3 =1}^{i_2 +1}a_{i_3}...
	  \sum _{i_k =1}^{i_{k-1}+1}a_{i_k}.
$$
И наконец, прибавляя и вычитая
$ s_1 s_k $
к правой части полученного тождества,
будем иметь
$$
  \dot s_k =a_1 a_2
    \sum _{i_2 =1}^{3}a_{i_2}
	  \sum _{i_3 =1}^{i_2 +1}a_{i_3}...
	    \sum _{i_k =1}^{i_{k-1}+1}a_{i_k}+
$$
$$
  +a_1 a_1 \sum _{i_2 =1}^{2} a_{i_2}
    \sum _{i_3 =1}^{i_2 +1}a_{i_3}...
	  \sum _{i_k =1}^{i_{k-1}+1}a_{i_k}
	    -s_1 s_k =
$$
$$
  =s_{k+1}-s_1 s_k .
$$
Теорема доказана.
$ \triangle $
               %%%%%%%%%%%   Теорема   %%%%%%%%%%%%%%%%%%%%%%%%%%%%%%%%%%
\begin{The}
Пусть последовательность
$ \{ a_n (t) \} _{n \in \mathbb{N}} $
удовлетворяет системе (6.11).
Тогда резольвентная функция
$ S(z) $
удовлетворяет уравнению
$$
  \frac{\partial}{\partial t}S(z,t)=
    (z-s_1 (t))S(z,t)-1.
	  \qquad (6.15)
$$
\end{The}
{\Large Доказательство.}
Продифференцируем по
$ t $
степенной ряд
$ S(z,t) $
с помощью формул (6.14). Получим
$$
  \sum _{k=0}^{\infty}
    \frac{\dot s_k}{z^{k+1}}=
	  \sum _{k=0}^{\infty}
	    \frac{zs_{k+1}}{z^{k+2}}-s_1
		  \sum _{k=0}^{\infty}
		    \frac{s_k}{z^{k+1}}
			  +\frac{z}{z}-1.
$$
Теорема доказана.
$ \triangle $
              %%%%%%%%%%%%%   Теорема    %%%%%%%%%%%%%%%%%%%%%%%%%%%%%%%%%
\begin{The}
Пусть последовательность
$ \{ a_n (t) \} _{n \in \mathbb{N}} $
удовлетворяет системе (6.11), и пусть
$$
  a_n (0)>0 .
    \qquad (6.16)
$$
Тогда спектральная мера Стилтьеса
$ ds(x,t) $
сущестует для любого
$ t>0 $
и равна
$$
  ds(x,t)=\frac
    {e^{xt}ds(x,0)}
	  {\int _0 ^{\infty} e^{xt}ds(x,0)}.
	    \qquad (6.17)
$$
\end{The}
{\Large Доказательство.}
Напомним, что условие (6.16) является необходимым и достаточным
для разрешимости проблемы моментов Стилтьеса.
Подставим (6.17) в интегральное представление
для резольвентной функции Стилтьеса и продифференцируем по
$ t : $
$$
  \frac{\partial}{\partial t}S(z,t)=
    \frac{\partial}{\partial t} \Biggl (
	  \frac{\int _0 ^{\infty}
	    \frac{e^{xt}ds(x,0)}{z-x}}
	 {\int _0 ^{\infty}e^{xt}ds(x,0)} \Biggr ) .
$$
Производная по
$ t $
правой части есть
$$
  \frac{\partial}{\partial t}S(z,t)=
    \frac{\int _0 ^{\infty}
	  \frac{xe^{xt}ds(x,0)}{z-x}}
	{\int _0 ^{\infty} e^{xt}ds(x,0)}-
  \frac{\int _0 ^{\infty}xe^{xt}ds(x,0)
    \int _0 ^{\infty} \frac{e^{xt}ds(x,0)}{z-x}}
	{\int _0 ^{\infty}e^{xt}ds(x,0)
	  \int _0 ^{\infty}e^{xt}ds(x,0)}.
$$
Тождественные преобразования интеграла в числителе
первой дроби дают
$$
  \int _0 ^{\infty}\frac{(x-z+z)}{z-x}e^{xt}ds(x,0)=-
    \int _0 ^{\infty}e^{xt}ds(x,0)+z
	  \int _0 ^{\infty}\frac{e^{xt}ds(x,0)}{z-x}.
$$
Таким образом,
$$
  \frac{\partial}{\partial t}S(z,t)=-1+z\frac
    {\int _0 ^{\infty}\frac{e^{xt}ds(x,0)}{z-x}}
	{\int _0 ^{\infty} e^{xt}ds(x,0)}-
	  \frac{\int _0 ^{\infty}x e^{xt}ds(x,0)}
	  {\int _0 ^{\infty} e^{xt}ds(x,0)}
	    \frac{\int _0 ^{\infty}e^{xt}ds(x,0)}
		{\int _0 ^{\infty} e^{xt}ds(x,0)} ,
$$
что ввиду (6.15) является верным тождеством. \\
Теорема доказана.
$ \triangle $ \\
Таким образом, процедура решения задачи Коши
для системы уравнений (6.11)
с помощью спектральной задачи Стилтьеса
выглядит следующим образом.
Сперва по начальным данным решается прямая спектральная
задача:
$$
  \{ a_k (0) \} \longrightarrow
    \{ S(z,0); \; s_k (0); \; ds(x,0) \}
$$
(спектральная мера
$ ds(x,0) $
может быть определена лишь для положительных начальных данных
$ a_k >0 ). $
Затем в соответствии с доказанными теоремами находятся
спектральные данные в момент времени
$ t : $
$$
  \{ S(z,0); \; s_k (0); \; ds(x,0) \}
    \longrightarrow
	  \{ S(z,t); \; s_k (t); \; ds(x,t) \} ,
$$
при этом отметим, что эволюция спектральной меры описывается
явной формулой:
$$
  ds(x,t)=\frac{e^{xt}ds(x,0)}
    {\int _0 ^{\infty} e^{xt} ds(x,0)}.
$$
Наконец, решая обратную спектральную задачу:
$$
  \{ S(z,t); \; s_k (t); \; ds(x,t) \}
    \longrightarrow
	  \{ a_k (t) \} ,
$$
мы получаем решение задачи Коши для системы уравнений (6.11). \\
Метод спектральной задачи служит не только как процедура нахождения
решения задачи Коши для системы уравнений (6.11), но и позволяет
делать заключения общего характера. Справедлива
                  %%%%%%%%%%%%%%   Теорема   %%%%%%%%%%%%%%%%%%%%%%%%%%%%%%
\begin{The}
Для начальных данных
$$
  0<a_k (0) \leqslant C < \infty
    \qquad (6.18)
$$
система уравнений (6.11) имеет глобальное решение
(для любого
$ t>0 ). $
\end{The}
{\Large Доказательство.}
Решая прямую задачу с данными (6.18) мы получаем спектральную меру
с компактным носителем. Так как эволюция спектральной меры
не меняет ее носителя, а вырожение в правой части (6.17)
имеет смысл и представляет меру для любого
$ t>0 , $
то решая обратную задачу мы всегда будем оставаться в классе
(6.18) (напомним, что положительность
$ a_k $
есть не только достаточное, но и необходимое условие
разрешимости проблемы моментов). Поэтому данная процедура
решения задачи Коши может быть продолжена для любого
$ t>0 . $ \\
Теорема доказана.
$ \triangle $
\newpage
\begin{center}
{\bfseries Литература}
\end{center}
$$ \; $$
\\

1. Никишин Е.М., Сорокин В.Н. Рациональные аппроксимации
и ортогональность. М.:Наука, 1988. \\

2. Ахиезер Н.И. Классическая проблема моментов. М.:
Физматгиз, 1961. \\

3. Кусис П. Введение в теорию пространств
$ H^p . $
М.: Мир, 1984. \\

4. Аткинсон Ф. Дискретные и непрерывные граничные задачи.
М.:Мир, 1968.
\newpage
$$ \; $$
Аптекарев Александр Иванович \\
Сорокин Владимир Николаевич
$$ \; $$
{\Large Спектральная теория разностных операторов}
$$ \; $$
Учебное пособие
$$ \; $$
$$ \; $$
Оригинал-макет подготовлен В.Н.Сорокиным с использованием
издательской системы LATEX2$ \varepsilon $
на механико-математическом факультете МГУ.
\end{document}
