\documentclass[a4paper]{article}
\usepackage[russian]{babel}
\usepackage[utf8]{inputenc}
\usepackage[simple]{dmvn}

\title{Программа экзамена по функциональному анализу}
\author{Лектор-- В.\,И.\,Богачёв}
\date{V--VI семестры, 2005--2006 г.}

\begin{document}
\maketitle

\section*{V семестр}

\begin{nums}{-2}
\item Метрические пространства. Непрерывные отображения. Полнота и сепарабельность. Теорема о вложенных шарах. Теорема Бэра.
\item Нормированные пространства. Примеры: пространства непрерывных функций, интегрируемых функций и пространства
      последовательностей. Изометричность метрического пространства $M$ части банахова пространства $B(M)$ и существование пополнения $M$.
\item Топологические пространства. Компактные множества и их свойства.
\item Вполне ограниченные множества. Критерий вполне ограниченности в терминах фундаментальных последовательностей.
\item Равносильность различных определений компакта в метрическом пространстве.
\item Эквивалентность норм на конечномерном пространстве. Некомпактность шара в бесконечномерном нормированном пространстве.
\item Критерий компактности в $B(M)$. Теорема Арцела.
\item Теоремы о неподвижных точках: теорема о сжимающих отображениях и теорема Шаудера.
\item Евклидовы пространства. Существование ортогональной проекции и ортогонального разложения в гильбертовом пространстве.
\item Ортонормированные системы, базисы и ортогонализация. Существование ортонормированного базиса в сепарабельном евклидовом
      пространстве. Примеры базисов.
\item Неравенство Бесселя. Равенство Парсеваля. Теорема Рисса-- Фишера. Изоморфизм сепарабельных гильбертовых пространств.
\item Линейные операторы и линейные функционалы. Норма оператора и непрерывность оператора.
\item Теорема Банаха-- Штейнгауза.
\item Теорема об открытом отображении.
\item Теорема о замкнутом графике. Теорема Банаха об обратном операторе.
\item Теорема Хана-- Банаха и ее следствия. Сопряженное пространство. Отделение выпуклых множеств (без доказательства).
\item Теорема Рисса об общем виде непрерывного линейного функционала на гильбертовом пространстве. Явный вид сопряженных
      к конкретным пространствам (без доказательства).
\item Изометрическое вложение нормированного пространства во второе сопряженное. Ограниченность слабо ограниченных множеств.
\item Топология $\si(E,F)$ и равенство $\br{E, \si(E,F)}^* = F$. Слабая и $*$-- слабая топологии.
\item Теорема о $*$-- слабой компактности шара в сопряженном пространстве (без доказательства).
      Выделение $*$-- слабо сходящейся подпоследовательности из ограниченной последовательности
      функционалов на сепарабельном нормированном пространстве.
      Слабая сходимость и слабая компактность в гильбертовом пространстве.
\item Компактные операторы и их свойства. Примеры компактных и некомпактных операторов.
\end{nums}

\medskip\dmvntrail

\pagebreak

\section*{VI семестр}

\begin{nums}{-2}
\item Спектр оператора. Открытость множества обратимых операторов. Замкнутость спектра.
\item Теорема об отображении спектров для многочленов.
\item Спектр компактного оператора.
\item Альтернатива Фредгольма:
$$\Ker(I-K) = 0 \quad \Longleftrightarrow \quad (I - K)X = X.$$
\item Понятие о локально выпуклом пространстве. Примеры. Пространства $\Dc$ и $\Sc$ и сходимость в них.
      Плотность пространства~$\Dc$ в пространстве $L^2(\R^1)$.
\item Обобщенные функции классов $\Dc'$ и $\Sc'$. Производная обобщенной функции. Носитель и сингулярный носитель.
\item Преобразование Фурье интегрируемых функций и его свойства. Формула обращения. Полнота системы функций Эрмита.
\item Преобразование Фурье в $\Sc$ и его непрерывность. Равенство Парсеваля для интегралов Фурье. Преобразование Фурье в $\Sc'$.
\item Преобразование Фурье в $L^2(\R^1)$ и теорема Планшереля.
\item Свертка интегрируемых функций. Свертка обычной и обобщенной функций. Использование преобразования Фурье
      и свертки для решения дифференциальных уравнений.
\item Самосопряженный оператор и его квадратичная форма. Критерий Вейля и вещественность спектра самосопряженного оператора.
\item Равенства
      $$\hn{A} = \sup \hc{ \hm{(At, х)},\; \hn{x} < 1} = \sup\hc{\hm{\la}\cln \la\text{-- точка спектра }A}$$
      для самосопряженного оператора А.
\item Теорема Гильберта-- Шмидта.
\item Норма и спектр оператора умножения на функцию. Циклические векторы.
\item Унитарные операторы и унитарная эквивалентность операторов. Спектр оператора преобразования Фурье и спектр оператора свертки.
\item Непрерывные функции от самосопряжённых операторов и равенство
$$\hn{f(A)} = \supl{t \in \si(A)} \hm{f(t)}.$$
\item Эквивалентность самосопряженного оператора оператору умножения на функцию
      (случай оператора с циклическим вектором и общий случай).
\item Проекторы и проекторнозначные меры. Представление самосопряженного оператора в виде
      интеграла по проекторнозначной мере. Явное вычисление спектральной меры для оператора умножения на аргумент и для проектора.
\item Пространства С.\,Л.\,Соболева $W^{p,k}$ и их характеризация через пополнение по соболевской
      норме. Описание пространства~$W^{2,k}$ через преобразование Фурье. Теоремы вложения в $L^q$ и $\Cb$ (без доказательства).
\end{nums}

\medskip\dmvntrail
\end{document}


%% Local Variables:
%% eval: (setq compile-command (concat "latex  -halt-on-error -file-line-error " (buffer-name)))
%% End:
