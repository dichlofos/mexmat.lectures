\documentclass[a4paper]{article}
\usepackage{dmvn}

\newcommand{\seq}[3]{\hc{#1}_{{#2}}^{{#3}}}
\newcommand{\n}[1]{^{(#1)}}
\newcommand{\comment}[1]{\par\vskip2pt\hrule\vskip2pt{\footnotesize \textbf{Примечание:} #1\par}\vskip2pt\hrule\vskip2pt}

\newcommand{\eqsymright}[2]
{\par\hbox to \textwidth{\hbox to 0pt{\hbox to \textwidth{\hfil #1}\hss}
\hfil$\displaystyle#2$\hfil}\par}

\newcommand{\hru}{\par\vskip2pt\hrule\vskip2pt\par}
\newcommand{\q}[1]{\par\textbf{[#1]:\,\,}}

\newenvironment{petit}
{\par \smallskip \hrule \smallskip \footnotesize}
{\par \smallskip \hrule \smallskip}

\tocsubsubsectionparam{3.4em}

\begin{document}
\dmvntitle{Курс лекций по}{функциональному анализу}{Лектор\т Анатолий Михайлович Стёпин}
{III курс, 5 семестр, поток математиков}{Москва, 2004 г.} \pagebreak

\pagestyle{plain}
\tableofcontents
\pagebreak

\section*{Введение}

\subsection*{Предисловие}

\hbox to \textwidth{\hfil\parbox{5cm}
{\small \itshape Видишь, в этих строках\\
Где-то спрятан обман\\
А тут\т сто теорем\т\\
Разыщи\д ка его\ldots\\
А когда надоест,\\
Забей на функан,\\
Ботай дифгем,\\
\hbox to .5cm{}Ботай дифгем,\\
\hbox to 1cm{}Ботай дифгем\dots}\par}

\vskip 5pt

Убедительная просьба ко всем читателям: в случае обнаружения ошибок
немедленно сообщайте авторам на \dmvnmail{} или загляните на \dmvnwebsite{} и посмотрите, где можно
достать в настоящее время самих авторов. Все пожелания и предложения по поводу оформления
и содержания документа будут обязательно приняты к сведению.

В этой версии исправлено ещё несколько опечаток, а также устранена неточность в следствии теоремы Рисса\ч Фишера.
Также просим всех читателей обратить внимание на приложение к лекциям. В нём вы найдёте много интересного.

\subsection*{Слова благодарности}

Огромное спасибо Юре Малыхину за обнаружение опечаток и устранение дефектов в доказательствах.
В~настоящее время от его многочисленных пакетов исправлений осталось не так уж много,
а это весьма позитивно.

Почти все поправки от Миши Малинина была успешно внесены в документ. Его решение задачи
про сжимающие отображения выиграло конкурс и было помещено в текст.
Также добавлено решение задачи про вложенные шары, присланное Митей Гусевым.
Исправлена неточность в определении гильбертова пространства, замеченная Колей Масловым.

Отдельная благодарность выносится Юре Притыкину за просвещение в области компактных операторов,
Илье Питерскому за многочисленные замечания и поиск опечаток, а также Мише Берштейну и Мише Левину
за одну очень полезную лемму.

\subsection*{Принятые в тексте соглашения и используемые сокращения}

\begin{points}{-3}
\item Следуя \cite{rokhlin}, топологические понятия обозначаются сокращениями соответствующих
      английских слов. Так, $\Int A$\т множество внутренних точек множества~$A$, $\Cl A$\т
      замыкание множества $A$.
\item Область определения будем обозначать символом $\Dom$ (от английского \emph{domain}).

\item Пространства функций обозначаются жирными буквами: $\VB$\т функции ограниченной вариации,
      $\Cb$\т непрерывные, $\Bb$\т ограниченные.
\item Пространства операторов и линейных функционалов мы иногда будем обозначать буквами
      вида~$\As, \Bs, \Cs$.
\end{points}

Список литературы приведён здесь не случайно. Без этих книжек лекции были бы сборником ошибочно
сформулированных и (не)доказанных теорем.

\begin{thebibliography}{4}
\setlength\itemsep{-.5mm}
\bibitem{rokhlin}
    В.\,А.\,Рохлин, Д.\,Б.\,Фукс. \emph{Начальный курс топологии.}\т М.: Наука, 1977.
\bibitem{kf}
    А.\,Н.\,Колмогоров, С.\,В.\,Фомин. \emph{Элементы теории функций и функционального анализа.}\т М.: Наука, 1981.
\bibitem{ls}
    Л.\,А.\,Люстерник, В.\,И.\,Соболев. \emph{Элементы функционального анализа.}\т М.: Наука, 1965.
\bibitem{vinberg}
    Э.\,Б.\,Винберг. \emph{Курс алгебры.}\т М.: Факториал, 2002.
\bibitem{kg}
    А.\,А.\,Кириллов, А.\,Д.\,Гвишиани. \emph{Теоремы и задачи функционального анализа.}\т М.: Наука, 1988.
\bibitem{glazman}
    И.\,М.\,Глазман. \emph{Прямые методы качественного спектрального анализа сингулярных
    дифференциальных операторов.}\т М.: Физматгиз, 1963.
\bibitem{ag}
    Н.\,И.\,Ахиезер, И.\,М.\,Глазман. \emph{Теория линейных операторов в гильбертовом пространстве.}\т М.: Физматгиз, 1963.
\end{thebibliography}

\dmvntrail


\pagebreak
\pagestyle{headings}

\makeatletter
  \renewcommand{\headheight}{11mm}
  \renewcommand{\headsep}{2mm}
  \renewcommand{\sectionmark}[1]{}
  \renewcommand{\subsectionmark}[1]{}
  \renewcommand{\subsubsectionmark}[1]{\markright{\thesubsubsection. #1}}
  \renewcommand{\@oddhead}{\vbox{\hbox to \textwidth{\scriptsize\thepage\hfil\rightmark\strut}\hrule}}
  \renewcommand{\@oddfoot}{\hfil\thepage\hfil}
\makeatother




\section{Гильбертовы пространства}

\subsection{Операторы в гильбертовых пространствах}

\subsubsection{Основное понятие}

\begin{df}
\emph{Гильбертовым} пространством называется бесконечномерное евклидово пространство, полное относительно нормы,
задаваемой скалярным произведением: $\hn{x} := \sqrt{(x,x)}$. Его мы обычно будем обозначать
буквой~$H$.
\end{df}

\begin{problem}
Проверить, что так заданная норма удовлетворяет всем аксиомам нормы.
\end{problem}

\subsubsection{Сопряжённые операторы}

\begin{df}
Пусть $A$\т ограниченный оператор в~$H$.
Если оператор $B$ таков, что $(Ax,y) = (x,By)$ для всех $x,y\in H$, то $B$ называется \emph{сопряжённым} к $A$
и обозначается $A^*$. Если $A = A^*$, то $A$ называется \emph{самосопряжённым}.
\end{df}

\begin{note}
Существование оператора, сопряжённого к ограниченному, будет доказано позже.
\end{note}

Отношение сопряжённости является симметричным: если $B$ сопряжён к $A$, то $A$ сопряжён к $B$.
Действительно, имеем
$$(Ax,y) = (x,By) \; \Lra \; \ol{(Ax,y)} = \ol{(x,By)} \; \Lra \; (By,x) = (y,Ax),$$
а это и означает, что оператор~$A$ сопряжён к~$B$.

\begin{stm}
Имеет место соотношение $(A^*)^* = A$.
\end{stm}
\begin{proof}
По определению имеем для всех $x,y$
$$\case{(Ax,y) = (x,A^*y),\\ \br{(A^*)^*x,y} = \br{x,A^*y};} \;\Ra \; \br{Ax,y} = \br{(A^*)^*x,y}
\; \Lra \; \br{(A-(A^*)^*)x, y} = 0,$$
но из невырожденности скалярного произведения следует $\br{A-(A^*)^*}x = 0$ для всех $x$,
поэтому $A = (A^*)^*$.
\end{proof}

\begin{stm}
Оператор $A^*A$ является самосопряжённым.
\end{stm}
\begin{proof}
Имеем $(ABx,y) = (Bx,A^*y) = (x,B^*A^*y)$, откуда $(AB)^* = B^*A^*$. Поэтому
$(A^*A)^* = A^*(A^*)^* = A^*A$.
\end{proof}

\begin{lemma}[Фундаментальное равенство]
Имеет место равенство $\hn{A^*A} = \hn{A}^2$.
\end{lemma}
\begin{proof}
Покажем, что $\hn{A^*} = \hn{A}$. Действительно, $\hn{Ax}^2 = (Ax,Ax) = (A^*Ax,x) \le \hn{A^*A}\cdot\hn{x}^2$
по неравенству Коши\ч Буняковского.
Перейдём к верхней грани по $\hn{x} = 1$, получим
$\hn{A}^2 \le \hn{A^*A} \le \hn{A^*} \cdot \hn{A}$, откуда $\hn{A} \le \hn{A^*}$.
Меняя в этих выкладках местами операторы $A$ и $A^*$, получаем обратное неравенство.

Рассмотрим $\hn{Ax}^2 = (Ax,Ax) = (A^*Ax,x) \le \hn{A^*A}\cdot\hn{x}^2$. Снова
переходя к верхней грани по $\hn x = 1$, получим
$\hn{A}^2 \le \hn{A^*A} \le \hn{A^*} \cdot\hn{A} =  \hn{A}^2$. Значит, на самом деле,
тут всюду равенства.
\end{proof}

\subsubsection{Лемма об ортогональной проекции и её следствия}

\begin{lemma}[Об ортогональной проекции]
Пусть $H_0$\т замкнутое подпространство в $H$. Тогда для любого вектора $h \in H \wo H_0$ найдётся единственный
ближайший вектор из $H_0$.
\end{lemma}
\begin{proof}
Имеем $\rho(h, H_0) =: a > 0$ в силу того, что одно из этих множеств замкнуто, а второе компактно.
Выберем последовательность $\hc{h_n} \subs H_0$ так, чтобы $\rho(h_n,h) \ra a$ при $n \ra \infty$.
Покажем, что $\hc{h_n}$ фундаментальна.
Нам понадобится тождество параллелограмма: <<сумма квадратов диагоналей параллелограмма равна
сумме квадратов его сторон>>. В силу этого тождества для достаточно больших $n$ и $m$ получаем
$$\hn{h_n - h_m}^2 = 2\hn{h-h_n}^2 + 2\hn{h-h_m}^2 - 4\hn{h - \frac{h_n+h_m}{2}}^2 \le
2(a^2 + \ep) + 2(a^2 + \ep) - 4a^2 = 4\ep,$$
и тем самым фундаментальность установлена.

Далее, $H_0$\т замкнутое подпространство полного пространства, и потому оно полно.
Следовательно, $\hc{h_n}$ сходится к некоторому элементу $h_0 \in H_0$.
По непрерывности имеем $\rho(h_n,h) \ra \rho(h_0,h)$. С другой стороны, этот предел равен~$a$
в силу выбора~$h_n$. Следовательно, $\rho(h_0,h)=a$.
\end{proof}

\begin{imp}
Пусть $H_0 \subs H$\т замкнутое подпространство.
Всякий вектор $h \in H$ представим в виде $h = h_0 + g$, где $h_0 \in H_0$, а $g \in H_0^\bot$.
\end{imp}
\begin{proof}
Пусть $x \in H_0$. По лемме, функция $d(x) := \hn{h - x}^2$ достигает минимума на
некотором векторе $h_0 \in H_0$. Поэтому функция $\ph(t) := \hn{h - h_0 + tx}^2$
имеет минимум при $t = 0$. Тогда $\ph'(0) = 0$. Распишем скалярный квадрат:
$\ph(t) = (h - h_0 +tx, h-h_0 +tx) = \hn{h-h_0}^2 + 2t\Rea(x,h-h_0) + t^2(x,x)$,
поэтому $\ph'(0) = 2\Rea(x,h -h_0) = 0$. Далее, вместо вектора $x$ рассматривая вектор $i\cdot x$,
получаем $\Img(x,h-h_0) = 0$. Следовательно, $(x,h-h_0) = 0$.
Таким образом, всякий вектор $x \in H_0$ ортогонален вектору $h - h_0$, то есть $h-h_0 \in H_0^\bot$.
Тождество $h = h_0 + (h-h_0)$, очевидно, является искомым разложением.
\end{proof}

\subsubsection{Общий вид линейного функционала в гильбертовом пространстве}

\begin{lemma}[Рисса]
Пусть $f$\т ограниченный функционал. Тогда найдётся вектор $h_0 \in H$, для которого $f(x) = (x,h_0)$.
\end{lemma}
\begin{proof}
Если $f \equiv 0$, то доказывать нечего: берём $h_0 := 0$. Пусть теперь $f \neq 0$.
Очевидно, ядро $K := \Ker f$\т замкнутое подпространство. Покажем, что $\dim K^\bot = 1$.
Рассмотрим ненулевые вектора $h_1, h_2 \in K^\bot$. Рассмотрим вектор
$$v = f(h_1)h_2 - f(h_2)h_1.$$
С одной стороны,
$v \in K^\bot$ как линейная комбинация векторов из $K^\bot$. С другой стороны, он лежит и в~$K$, потому что
$f(v) = f(h_1)f(h_2) - f(h_2)f(h_1) = 0$. Но $K \cap K^\bot = 0$, поэтому $v = 0$, следовательно
вектора $h_1$ и $h_2$ пропорциональны.

Рассмотрим уравнение $f(x) = (x, \mu h_1)$, где $\mu$\т неизвестное.
Определим его, подставив $x = h_1$: получим $F(h_1) = \ol \mu (h_1,h_1)$. Итак, $\mu$ найдено.
Тогда для всякого $x \in K^\bot$ имеем $f(x) = (x, \mu h_1)$. В самом деле, $x = \la h_1$, поэтому
$$f(x) = f(\la h_1) = \la f(h_1) = \la (h_1,\mu h_1) = (\la h_1,\mu h_1) = (x,\mu h_1).$$
Аналогично, если $x \in K$, то равенство тоже верно: и слева, и справа получаем ноль. Но поскольку $H = K\oplus K^\bot$,
по следствию из леммы об ортогональной проекции это верно и на всём пространстве.
\end{proof}

\begin{stm}
Сопряжённый оператор существует.
\end{stm}
\begin{proof}
Пусть $A$\т ограниченный линейный оператор в~$H$.
Зафиксируем $y \in H$ и рассмотрим функционал $f(x) := (Ax,y)$. Линейность его очевидна,
а ограниченность следует из неравенства Коши\ч Буняковского:
$$\hm{(Ax,y)} \le \hn{Ax}\cdot \hn{y} \le \hn{A}\cdot \hn{y} \cdot \hn{x}.$$
По лемме Рисса получаем $f(x) = (x,A^*y)$, где $A^*y$\т обозначение для сопряжённого оператора,
применённого к вектору $y$.

Проверим корректность определения. Пусть мы получили таким способом два
вектора $v_1$ и $v_2$. Для них имеем $(Ax,y) = (x,v_1) = (x,v_2)$,
причём это верно для любого $x$. Таким образом, для всех $x$ имеем $(x, v_1 - v_2) = 0$.
Подставим $x = v_1 - v_2$, получим $(v_1-v_2, v_1-v_2) = 0$, откуда $v_1 = v_2$.

Очевидно, что получаемый таким способом оператор будет линейным. Контрольный вопрос:
а нужно ли доказывать его ограниченность?
\end{proof}

\subsubsection{Ортонормированные системы}

\begin{df}
ОНС называется \emph{полной}, если её линейная оболочка всюду плотна в $H$.
\end{df}

\begin{df}
Пусть $\hc{e_n}$\т ОНС в $H$. \emph{Наилучшим приближением} вектора $x \in H$
по системе $\hc{e_n}$ порядка~$N$ называется число
$$E_N(x) := \infl{\al_k} \Bn{x - \suml{k=1}{N}\al_k e_k}.$$
\end{df}

\begin{theorem}
Пусть $\hc{e_n}$\т ОНС в $H$. Тогда наилучшее приближение порядка~$N$
равно
$$E_N(x) = \Bn{x - \suml{k=1}{N} (x,e_k)e_k}.$$
\end{theorem}
\begin{proof}
Положим $c_k = (x,e_k)$. В силу ортонормированности системы имеем
$$
\Bn{x - \suml{k=1}{N}\al_k e_k  }^2 = \Br{x - \suml{k=1}{N}\al_k e_k, x - \suml{k=1}{N}\al_k e_k } =
\hn{x}^2 - 2\Rea \suml{k=1}{N} \ol\al_k c_k + \suml{k=1}{N}\hm{\al_k}^2  \stackrel{!}{=}
\hn{x}^2 +\suml{k=1}{N}\hm{\al_k - c_k}^2 - \suml{k=1}{N} \hm{c_k}^2.
$$
Проверка равенства, отмеченного знаком <<!>>, предоставляется читателю. Из этой формулы видно,
что выражение достигнет своего минимума, когда станет нулём второе слагаемое в последнем выражении.
А это будет в точности тогда, когда  $\al_k = c_k$.
\end{proof}

\begin{imp}[Неравенство Бесселя]
$$\suml{k=1}{\bes} \hm{(x,e_k)}^2 \le \hn{x}^2.$$
\end{imp}
\begin{proof}
Для конечных сумм это неравенство верно в силу только что доказанной теоремы,
поскольку наилучшее приближение неотрицательно, и
$$E_N^2(x) + \suml{k=1}{N} \hm{(x,e_k)}^2 = \hn{x}^2.$$
Ясно, что при переходе к пределу неравенство не испортится.
\end{proof}

\begin{theorem}[Рисса\ч Фишера]
Пусть $H$\т гильбертово пространство, $\hc{e_k}$\т ОНС в нём, и $(c_k) \in \ell_2$. Тогда
существует $h \in H$, для которого $(h,e_k) = c_k$. Иными словами, существует вектор с предписанными
коэффициентами Фурье из $\ell_2$.
\end{theorem}
\begin{proof}
Поищем $h$ в виде суммы ряда $\sum c_k e_k$ и покажем, что этот ряд сходится. Рассмотрим
$h_n := \sumkun c_k e_k$. Проверим фундаментальность последовательности $\hc{h_n}$.
Пусть $m > n$, тогда
$$
  \hn{h_m - h_n}^2 = \hr{ \suml{k=n+1}{m}c_k e_k, \suml{k=n+1}{m} c_k e_k  } \stackrel{!}{=}
  \suml{k=n+1}{m}|c_k|^2 \ra 0
$$
при $m, n \ra \infty$ как кусок хвоста сходящегося ряда (ведь $(c_k) \in \ell_2$).
Равенство, отмеченное <<!>>, следует из ортонормированности системы $\hc{e_n}$.
В силу полноты пространства, последовательность $h_n$ сходится к некоторому вектору $h \in H$.
То, что вектор $h$ имеет нужные коэффициенты Фурье, очевидно.
\end{proof}

\begin{stm}[Равенство Парсеваля]
Пусть $\hc{e_n}$\т полная ОНС в гильбертовом пространстве $H$, а $c_k := (h,e_k)$. Тогда
\eqn{\hn{h}^2 = \sum |c_n|^2.}
\end{stm}
\begin{proof}
В силу непрерывности скалярного произведения и ортонормированности $\hc{e_n}$ получаем
\eqn{(h,h) = \liml{n} (h_n,h_n) = \liml{n}\sumkun|c_k|^2 = \sum |c_k|^2.}
\end{proof}

\begin{problem}
\label{pr:fourier-coeff-uniqueness}
Доказать, что если $\hc{e_n}$\т полная ортонормированная система, то вектор $h$ в теореме
Рисса\ч Фишера единствен.
\end{problem}
\begin{solution}
Пусть нашлись два вектора с одинаковыми коэффициентами Фурье. Их разность,
очевидно, имеет нулевые коэффициенты Фурье. Но такой вектор может быть
только нулём в силу равенства Парсеваля. Значит, на самом деле векторы равны.
\end{solution}

\begin{theorem}
В сепарабельном евклидовом пространстве $H$ существует полная ортонормированная система.
\end{theorem}
\begin{proof}
Пусть последовательность $\hc{h_i}$ такова, что $\Cl \hc{h_i} = H$. Можно считать,
что все $h_i$ отличны от нуля.
Возьмём $e_1 := \frac{h_1}{\hn{h_1}}$.
Если $\ha{e_1} = H$, то ПОНС найдена. В противном случае
найдётся ещё один вектор из счётного всюду плотного множества (без ограничения
общности это $h_2$) такой, что $h_2 \notin \ha{e_1}$. Если уже
выбрано $(n-1)$ взаимно ортогональных векторов $\hc{e_1\sco e_{n-1}}$ единичной
длины, и $h_n \notin \ha{e_1\sco e_{n-1}}$,
то найдём единичный вектор $e_n \in \ha{e_1\sco e_{n-1}; h_n}$ такой, что $e_n \bot \ha{e_1\sco e_{n-1}}$.
Поищем его в виде
$$e_n = \la_1 e_1 + \la_2 e_2 \spl \la_{n-1}e_{n-1} + h_n.$$
Домножая это равенство скалярно на $e_1\sco e_{n-1}$, получаем систему уравнений на $\la_i$:
$$0 = (e_n, e_i) = \la_i (e_i, e_i) + (h_n, e_i) = \la_i + (h_n, e_i), \quad i = 1\sco n-1.$$
Решая её и нормируя полученный вектор $e_n$, добавляем его в базис.
Если пространство бесконечномерно, этот процесс никогда не оборвётся, и в итоге мы получим
счётную систему взаимно ортогональных векторов $\hc{e_i}$.

Покажем её полноту.
Полнота системы означает, что всякий вектор можно сколь угодно точно приблизить
конечной линейной комбинацией векторов из этой системы. Таким свойством обладало семейство $\hc{h_i}$,
но так как $h_i$ линейно выражаются через $e_i$ (впрочем, и наоборот тоже), то оно переносится и на $\hc{e_i}$.

Покажем, что если $g \bot \hc{e_i}$, то $g = 0$.
В самом деле, приблизим этот вектор линейной комбинацией векторов $\hc{e_i}$, получим
вектор $g_\ep$. Тогда $(g_\ep, g) = 0$, но, переходя
к пределу при $\ep \ra 0$ (увеличивая точность приближения), получаем, что $(g,g) = 0$, откуда $g = 0$.
\end{proof}

\begin{theorem}
Все сепарабельные гильбертовы пространства изоморфны между собой.
\end{theorem}
\begin{proof}
Пусть $\hc{e_n}$\т ПОНС.
Возьмём вектор $h$ и его коэффициенты Фурье по этой системе\т последовательность $\hc{c_n} \in \ell_2$.
Как мы знаем, в силу теоремы Рисса\ч Фишера и задачи \ref{pr:fourier-coeff-uniqueness},
имеется линейная биекция между векторами пространства и наборами коэффициентов Фурье,
то есть изоморфизм произвольного гильбертова пространства на пространство~$\ell_2$.
Он сохраняет расстояние (то есть норму разности) в силу равенства Парсеваля.
Осталось показать, что сохранение нормы влечёт сохранение скалярного произведения.
Для этого достаточно вспомнить факт из линейной алгебры: эрмитова полуторалинейная
функция однозначно восстанавливается по своей квадратичной функции. По этому поводу
см.~\cite[гл. 5, \S 5]{vinberg}.
\end{proof}


\subsection{Спектральная теорема}

\subsubsection{Лемма об отображении спектра}

\begin{lemma}[Об отображении спектра]
Пусть $P$\т многочлен. Тогда $\Sig\br{P(A)} = P\br{\Sig(A)}$.
\end{lemma}
\begin{proof}
Докажем включение <<$\sups$>>. Возьмём $\la \in \Sig(A)$. Рассмотрим $P(z) - P(\la) = (z-\la)Q(z)$, и
подставим $z = A$. Получим $P(A) - P(\la)I = (A - \la I)Q(A)$. Так как сомножители коммутируют, и оператор
$A - \la I$ необратим, поэтому необратим и оператор в левой части. Но это и означает, что $P(\la) \in \Sig\br{P(A)}$.

Докажем обратное включение <<$\subs$>>. Возьмём $c \in \Sig\br{P(A)}$ и покажем, что $\exi \la \in \Sig(A)$,
для которого $c = P(\la)$. Возьмём $P(z) - c = k(z-\la_1)\sd (z-\la_n)$, откуда
$P(A) - c I = k(A - \la_1 I)\sd (A - \la_n I)$. Какой\д то из операторов в этом произведении должен быть
необратим, иначе был бы обратим и оператор $P(A) - cI$, что неверно.
\end{proof}

\begin{note}
Свойство коммутирования тут очень важно. Пример: операторы левого и правого сдвига в $\ell_p$. Имеем $LR=E$
(\те обратимый оператор), а $RL$\т необратим, так как $x_1$ погибает при левом сдвиге, \те имеется ядро $\ha{(1,0,0\etc)}$.
\end{note}

\subsubsection{Спектральный радиус оператора и его оценка сверху}

\begin{df}
\emph{Спектральным радиусом} оператора~$A$ называется число $r(A) := \sup\hc{|\la| \cln \la \in \Sig(A)}$.
Это <<наименьший>> радиус круга, в который умещается спектр оператора.
\end{df}

\begin{df}
Последовательность~$\hc{a_n}$ называется \emph{полуаддитивной}, если $a_{m+n} \le a_m + a_n$ для всех $m,n$.
\end{df}


\begin{lemma}[Фекете]
Для полуаддитивных последовательностей имеет место свойство
$$\liml{n} \frac{a_n}{n} = \infl{n}\frac{a_n}{n}.$$
\end{lemma}
\begin{proof}
Положим $A := \infl{n} \frac{a_n}{n}$. Вообще говоря, может получиться так, что $A = -\infty$.
Но это не повлияет на дальнейшие рассуждения. По определению нижней грани найдётся $n_\ep$,
для которого $\frac{a_{n_\ep}}{n_\ep}- A < \ep$. Рассмотрим произвольное $n$ и поделим с остатком на $n_\ep$:
$n = k\cdot n_\ep + r$. Тогда при $n \ra \infty$  имеем
$$\frac{a_n}{n} \le \frac{k\cdot a_{n_\ep} + a_r}{k\cdot n_\ep + r} = \frac{a_{n_\ep} + \frac{a_r}{k}}{n_\ep + \frac{r}{k}} \ra \frac{a_{n_\ep}}{n_\ep}.$$
Отсюда следует, что $\liml{n}\frac{a_n}{n}$ существует и равен $A$.
\end{proof}

\begin{lemma}[Оценка спектрального радиуса]
Справедливы соотношения:
$$r(A) = \liml{n} \sqrt[n]{\hn{A^n}}.$$
и $r(A) \le \hn{A}$, причём для самосопряжённых операторов неравенство обращается в равенство.
\end{lemma}
\begin{proof}
Рассмотрим $a_n := \ln \hn{A^n}$, тогда $a_{n+m} \le a_n + a_m$. По лемме Фекете
имеем $\frac{a_n}{n} \ra \infl{k} \frac{a_k}{k}$, поэтому существует предел
$\liml{n} \sqrt[n]{\hn{A^n}}$. Положим $s(A) := \liml{n}\sqrt[n]{\hn{A^n}} = \uliml{n}\sqrt[n]{\hn{A^n}}$.

Сначала покажем, что $r(A) \le s(A)$. Действительно, при $\hm{\la} > s(A)$ степенной ряд
для резольвенты
\eqn{\label{eqn:resolventa}(A - \la I)^{-1} = -\frac1\la\suml{k=0}{\infty}\frac{A^k}{\la^k}}
мажорируется по норме в пространстве операторов сходящимся числовым рядом, поэтому
имеет место сходимость.

Теперь докажем обратную оценку. Как мы знаем, резольвента аналитична в дополнении к спектру,
поэтому в кольце $\hm{\la} > r(A)$ она задаётся рядом Лорана~\eqref{eqn:resolventa}.
По формуле Коши\ч Адамара  получаем, что радиус кольца его сходимости равен $s(A)$.
Поэтому верно и обратное неравенство. Таким образом, $r(A) = s(A)$.

Далее, так как $\hn{A^n} \le \hn{A}^n$, то $\sqrt[n]{\hn{A^n}} \le \hn{A}$, поэтому $r(A) \le \hn{A}$.

Покажем, что для самосопряжённых операторов достигается равенство в этом соотношении.
Пусть~$A$\т самосопряжённый оператор. Как мы знаем, $\hn{A^*A} = \hn{A}^2$,
поэтому $\hn{A^2}=\hn{A}^2$. Следовательно, имеет место равенство $\bn{A^{2^k}} = \hn{A}^{2^k}$.
В формуле для $s(A)$ перейдём к пределу по подпоследовательности индексов $n = 2^k$, получим требуемое.
Но так как сама последовательность сходится, предел подпоследовательности совпадает с обычным пределом.
\end{proof}

\begin{problem}
Пусть $M \subs H$\т подмножество гильбертова пространства. Тогда $(M^\bot)^\bot = \Cl\ha{M}$.
\end{problem}

\begin{stm}
Пусть $B$\т оператор в гильбертовом пространстве. Тогда $(\Img B)^\bot = \Ker B^*$.
\end{stm}
\begin{proof}
По определению, $\Img B = \hc{Bx \vl x \in H}$. Если $y \in (\Img B)^\bot$, то
для $\fa x \in H$ имеем $0 = (Bx,y) = (x, B^*y)$. Но это означает, что $B^*y = 0$,
поэтому $y\in \Ker B^*$. Осталось заметить, что рассуждения обратимы.
\end{proof}

\begin{stm}
Спектр самосопряжённого оператора веществен.
\end{stm}
\begin{proof}
Пусть $\la \notin\R$, тогда $\Ker(A-\la I) = 0$, ибо собственные значения
самосопряжённого оператора вещественны. В самом деле, если $Ax = \la x$,
то $\la(x,x) = (\la x, x)=(Ax,x) = (x,Ax) = (x,\la x) = \ol\la(x,x)$, поэтому
$\la \in \R$.

Теперь покажем, что $\Img (A-\la I)$ плотен в~$H$.
Имеем $\Img(A - \la I)^\bot = \Ker(A - \la I)^* = \Ker(A - \ol \la I) = 0$.
Применяя результат задачи к $M = \Img(A - \la I)$, получаем, что $\Cl \Img(A - \la I) = 0^\bot = H$.

Поскольку $\Ker (A-\la I) = 0$, оператор, обратный к $A - \la I$, однозначно определён на образе $\Img (A-\la I)$.
Докажем его ограниченность: пусть $\la = a + bi$, где $b \neq 0$. Тогда
\begin{multline*}
\hn{(A - \la I)x}^2 = \br{ (A - a - bi)x, (A - a - bi)x} =\\=
\br{ (A-a)x - (b i)x, (A - a)x - (b i)x} = \hn{(A-a)x}^2 + b^2\hn{x}^2 \ge b^2 \hn{x}^2,
\end{multline*}
значит, оператор ограничен снизу. Но тогда обратный оператор ограничен сверху.
Поскольку образ всюду плотен, оператор можно продолжить по непрерывности на всё пространство,
значит, он обратим и $\la \notin \Sig(A)$.
\end{proof}


\subsubsection{Общий вид функционала на пространстве непрерывных функций}

\begin{theorem}[Ф. Рисса]
Всякий ограниченный линейный функционал $f\cln \Cb[0,1] \ra \Cbb$ можно представить интегралом Римана\ч Стилтьеса
по функции $g \in \VB[0,1]$, то есть $f(\ph) = \int \ph\,dg$.
\end{theorem}
\begin{proof}
Пусть $\Bb[0,1]$\т пространство ограниченных функций с чебышёвской нормой.
Продолжим наш функционал $f$ на пространство $\Bb$ и обозначим полученное продолжение
через~$F$. Положим $g(t) := F\hr{\chi_{[0,t)}}$, где $\chi$\т индикатор. Для краткости
аргумент индикаторов писать не будем.
Покажем, что $g \in \VB$.

В самом деле, рассмотрим разбиение отрезка $[0,1]$ точками $0=t_0\sco t_n=1$. Рассмотрим
\begin{multline*}
\sumkun \hm{g(t_k)-g(t_{k-1})} = \sumkun \hm{F\hr{\chi_{[0,t_k)}} - F \hr{\chi_{[0,t_{k-1})}}}=
\sumkun \hs{F\hr{\chi_{[0,t_k)}} - F\hr{\chi_{[0,t_{k-1})}}}e^{i\al_k} =\\=
F\hr{\sumkun e^{i\al_k} \chi_{[t_{k-1},t_k)}} \le
\hn{F} \cdot \hn{\sumkun e^{i\al_k} \chi_{[t_{k-1},t_k)}} = \hn{F}\cdot 1 = \hn{F}.
\end{multline*}
В этих выкладках мы <<подкрутили>> слагаемые коэффициентами $e^{i\al_k}$ так, что каждое комплексное число
совпало со своим модулем. Итак, доказано, что $g \in \VB$.

Теперь рассмотрим $\ph \in \Cb[0,1]$ и приблизим её ступенчатыми функциями:
$$\ph_n(t) = \sumkun \ph\hr{\frac kn}\cdot \chi_{\hsr{\frac{k-1}{n}, \frac kn}} =
             \sumkun \ph\hr{\frac kn}\cdot \hr{\chi_{\hsr{0, \frac kn}} -\chi_{\hsr{0, \frac{k-1}{n}}}}.$$
Тогда $\ph_n \rra \ph$ и потому $F(\ph_n) \ra F(\ph)$.
Вспоминая определение интеграла Римана\ч Стилтьеса, получаем
$$F(\ph_n) = \sumkun \ph\hr{\frac kn}\cdot \hs{g\hr{\frac kn} -g\hr{\frac{k-1}{n}}} \ra \intl{0}{1} \ph(t)\,d g(t),$$
что и требовалось доказать.
\end{proof}

\begin{df}
Назовём \emph{$\ep$\д индикатором} отрезка $[\al, \be]$ непрерывную функцию, равную нулю вне отрезка,
равную единице на отрезке $[\al + \ep, \be - \ep]$ и доопределённую линейным образом на интервалах
$(\al, \al + \ep)$ и $(\be - \ep,\be)$.
\end{df}


Следующее утверждение является некоторым дополнением к теореме Рисса об общем виде
функционалов на $\Cb[a,b]$.

\begin{stm}
Пусть $f(\ph) := \int \ph\, dg$\т функционал на $\Cb[a,b]$. Если он вещественный,
то функцию~$g$ можно выбрать вещественной, а если он неотрицателен, то~$g$ можно
взять неубывающей.
\end{stm}
\begin{proof}
Пусть $g$ не является вещественнозначной функцией. Тогда найдётся интервал $(\al,\be)$, на котором
$g(\be) - g(\al) \notin \R$. Возьмём $\ep$\д индикатор отрезка $\al, \be)$,
На такой функции значение нашего функционала, то есть попросту интеграла, не будет вещественным.

Пусть теперь функционал неотрицателен. Допустим, что $g$ убывает на каком\д нибудь отрезке $[\al,\be]$.
Тогда интеграл от $\ep$\д индикатора этого отрезка будет отрицательным.
\end{proof}

\subsubsection{Доказательство спектральной теоремы}

Напомним, что $L_2(\si)$\т пространство $L_2$ интегрируемых в квадрате функций по некоторой мере $\si$.

\begin{df}
Говорят, что оператор $A$ имеет \emph{циклический вектор}, если $\exi h \in H$, для которого линейная оболочка
$\ha{A^n h \vl n \in \Z_+}$ всюду плотна в $H$.
\end{df}

\begin{lemma}
Пусть $A$\т самосопряжённый оператор. Тогда
$\hn{P(A)} \le \hn{P}_C$, где $\hn{\cdot}_C$\т чебышёвская норма на пространстве $\Cb\hs{-\hn{A}, \hn{A}}$.
\end{lemma}
\begin{proof}
В силу фундаментального равенства имеем
$$\hn{P(A)}^2 = \hn{P^*(A)P(A)} = \hn{\ol P P(A)} =
  \sup \bc{|\la|\cln \la \in \Sig\hr{\ol P P(A)}} =
  \sup\bc{\hm{P(\la)}^2\cln \la \in \Sig(A)}.$$
Далее, поскольку $\la \in \R$, можно брать верхнюю грань только по отрезку $\hs{-\hn A, \hn A}$ вещественной оси.
Значит,
$$\hn{P(A)}^2 \le \sup \hc{\hm{P(\la)}^2\cln \la \in \hs{-\hn A, \hn A}},$$
а это и есть определение чебышёвской нормы.
\end{proof}

\begin{theorem}[Спектральная теорема]
Пусть самосопряжённый оператор $A\cln H \ra H$ имеет циклический вектор $h\in H$.
Тогда существует мера $\si$ на отрезке $\hs{-\hn{A}, \hn{A}}$ и изометрическое отображение
$U\cln H \ra L_2(\si)$, для которого
$UAU^{-1}$ есть оператор умножения на независимую переменную: $f(\la) \mapsto \la f(\la)$.
\end{theorem}
\begin{proof}
Сначала докажем, что это верно для многочленов.
Пусть $P$\т многочлен. Рассмотрим функционал
$\al(P):= \br{P(A)h,h}$. Он линейный, неотрицательный и ограниченный.

Покажем, что если $f \ge 0$, то $\al(f) \ge 0$.
Рассмотрим $g = \sqrt{f}$, тогда $\al(f) = \br{g^2(A)h,h} = \br{g(A)h,g(A)h} = \hn{g(A)h}^2 \ge 0$.
Здесь мы воспользовались тем, что $\br{g(A)}^* = g(A)$. Для многочленов это верно, а для функций\т в силу непрерывности.

По теореме Рисса функционал $\al$ имеет представление $\al(P) = \int P\,d\si$, где
$\si$\т мера на $X := \hs{-\hn A, \hn A}$. Построим, наконец, отображение $U$: положим
$U\br{P(A)h} := P$.
Покажем, что это отображение корректно задано, то есть покажем, что если $P(A) = Q(A)$, то $P \eqae Q$ по мере $\si$.
Имеем
\begin{multline*}
0 = \hn{\br{P(A)-Q(A)}h}^2 = \hn{(P-Q)(A)h}^2 = \br{(P-Q)(A)h, (P-Q)(A)h} =\\=
\hr{|P-Q|^2(A)h,h} = \al\br{\hm{P-Q}^2}=\int|P-Q|^2\,d\si = \hn{P-Q}_{L_2(\si)}.
\end{multline*}
Тем самым проверена не только корректность, но и изометричность отображения~$U$,
а также и то, что обратное отображение $U^{-1}$ существует.

Рассмотрим действие на векторах $P(A)h$: имеем
$(UAU^{-1} P)(\la) = \br{UAP(A)h}(\la) = \la P(\la)$ по определению отображения $U$.
В общем случае, приблизим функцию из $L_2(\si)$ многочленами $\hc{P_n}$, тогда получим
$$\hr{UAU^{-1}P_n}(\la) = \la P_n(\la) \convae \la f(\la) = \hr{UAU^{-1}}f(\la).$$
Это и завершает доказательство спектральной теоремы.
\end{proof}

\section{Компактные операторы}

\subsection{Компактные операторы в банаховых пространствах}

\subsubsection{Определение и свойства компактных операторов}

\begin{df}
Оператор называется \emph{компактным}, если образ единичного шара предкомпактен.
\end{df}

\begin{stm}
Сумма компактных операторов есть снова компактный оператор.
\end{stm}
\begin{proof}
Очевидно, если воспользоваться, например, критерием Хаусдорфа.
\end{proof}

\begin{stm}
Произведение компактного и ограниченного операторов есть компактный оператор.
\end{stm}
\begin{proof}
Пусть $A$\т компактный, а $B$\т ограниченный операторы. Сначала покажем,
что оператор $AB$ компактен. Если множество $M$ ограничено, то $B(M)$ тоже ограничено.
Тогда $A\br{B(M)}$ предкомпактно, и всё доказано.

Теперь покажем, что $BA$ тоже компактный оператор. Для этого воспользуемся критерием Хаусдорфа
предкомпактности множества. В силу компактности $A$, для любого $\ep$ в множестве
$A(M)$ существует конечная $\ep$\д сеть. Очевидно, что для множества $B\br{A(M)}$ годится
$\hn{B}\cdot \ep$\д сеть, которая получается из исходной сети после применения оператора $B$.
\end{proof}

\begin{stm}
Ограниченный оператор с конечномерным образом компактен.
\end{stm}
\begin{proof}
Действительно, всякое бесконечное ограниченное множество в конечномерном пространстве
предкомпактно. Следовательно, из образа любой ограниченной последовательности
можно будет выделить фундаментальную.
\end{proof}

\begin{imp}
Компактный оператор в бесконечномерном пространстве необратим.
\end{imp}
\begin{proof}
В самом деле, допустим противное. Поскольку $AA^{-1} = \id$, в силу предыдущего
утверждения получаем, что~$\id$ является компактным оператором. Но это неверно,
поскольку в бесконечномерном пространстве единичный шар не является предкомпактом.
\end{proof}

\begin{theorem}
Пусть $A_n$\т последовательность компактных операторов в банаховом пространстве, и $A_n \ra A$ по норме.
Тогда $A$ компактен.
\end{theorem}
\begin{proof}
Пусть $\hc{x_n}$\т ограниченная последовательность. Нужно доказать, что
из последовательности $\hc{Ax_n}$ можно выбрать фундаментальную.

Так как $A_1$ компактен, то выбираем последовательность $x_n\n1$ такую, что
последовательность $A_1 x_n\n1$ сходится. Из неё выбираем $x_n\n2$ такую, что $A_2x_n\n2$ сходится,
и так далее. Возьмём диагональ $y_i := x_i\n i$ и покажем, что последовательность $Ay_i$ фундаментальна.
По условию $\hn{x_n} \le C$, а $\hn{A_k y_n - A_k y_m} \ra 0$ в силу фундаментальности.
Кроме того, $\hn{A - A_k} \ra 0$. Поэтому
\begin{multline*}
\hn{Ay_n-Ay_m} \le \hn{A y_n-A_k y_n} + \hn{A_k y_n - A_k y_m} + \hn{A_k y_m - A y_m} \le \\ \le
\hn{A-A_k}\cdot\hn{y_n} + \hn{A_k y_n - A_k y_m} + \hn{A-A_k}\cdot\hn{y_m}\le \\ \le
\hn{A-A_k}\cdot C + \hn{A_k y_n - A_k y_m} + \hn{A-A_k}\cdot C \ra 0,
\end{multline*}
а это и значит, что последовательность $\hc{A y_i}$ фундаментальна.
\end{proof}

\begin{lemma}
Собственные векторы с различными собственными значениями линейно независимы.
\end{lemma}
\begin{proof}
Докажем утверждение индукцией по количеству $k$ собственных векторов $e_1\sco e_k$ c собственными
значениями $\la_1\sco \la_k$ соответственно. При $k = 1$ доказывать нечего.
Пусть $k > 1$, и
$$e_1\spl e_{k-1} + e_k = 0,$$
тогда, применяя к этому равенству оператор, получаем
$$\la_1e_1\spl \la_{k-1}e_{k-1} + \la_k e_k = 0.$$
Вычтем отсюда исходное равенство, умноженное на $\la_k$, получим
$$(\la_1-\la_k)e_1\spl (\la_{k-1}-\la_k)e_{k-1} = 0.$$
По предположению индукции такое возможно только если
$e_i = 0$ при $i = 1\sco k-1$. Но тогда и $e_k = 0$.
\end{proof}

\begin{theorem}
Пусть оператор $A\cln X \ra X$\т компактен, пространство $X$\т банахово. Тогда
количество собственных значений вне всякого круга радиуса $r>0$ с центром в
нуле лишь конечное число.
\end{theorem}
\begin{proof}
Пусть $\hc{\la_n}$\т попарно различные ненулевые собственные значения оператора $A$. Покажем, что
$\la_n\ra 0$. Допустим противное, тогда из $\hc{\la_n}$ можно выделить подпоследовательность так, что после перенумерации
последовательность $\bc{\frac1{\hm{\la_n}}}$ ограничена.
Рассмотрим цепочку подпространств $X_n := \ha{e_1\sco e_n}$, где $e_i$\т собственный вектор
с собственным значением $\la_i$. Тогда $e_1\sco e_n$ будут линейно независимыми,
следовательно, $\hc{X_n}$\т строго возрастающая цепочка.
В силу леммы о почти перпендикуляре, найдутся единичные векторы $x_n \in X_n$, для которых
$\rho\hr{x_n,X_{n-1}} > \frac12$. Разложим их по базису подпространств $X_n$:
пусть $x_n = \sumkun c_k e_k$. Как легко видеть, $\frac{A x_n}{\la_n}-x_n \in X_{n-1}$.
По предположению, последовательность $\bc{\frac{x_n}{\la_n}}$ ограничена. Подействуем на неё оператором~$A$
и увидим, что получается ёж. В самом деле, при $n < m$ имеем
$$v := \frac{Ax_n}{\la_n} - \frac{Ax_m}{\la_m} =
\ub{\frac{Ax_n}{\la_n}}_{\in X_{m-1}} - x_m + \ub{x_m - \frac{Ax_m}{\la_m}}_{\in X_{m-1}},$$
значит, $\hn{v}  = \hn{-x_m + \hr{\hbox{вектор из } X_{m-1}}} \ge \frac12$, а это
противоречит компактности оператора~$A$.
\end{proof}

\subsubsection{Слабая сходимость и слабая компактность}

Пусть $X$\т нормированное пространство.

\begin{df}
Говорят, что последовательность $x_n$ \emph{слабо сходится} к $x$, если для любого ограниченного функционала $f$
на $X$ имеем $f(x_n) \ra f(x)$. Обозначение: $x_n \convw x$.
\end{df}

\begin{df}
Говорят, что последовательность функционалов $f_n$ \emph{слабо сходится} к $f$, если для любого вектора $x \in X$
имеем $f_n(x) \ra f(x)$. Обозначение: $f_n \convw f$.
\end{df}


\begin{df}
Говорят, что последовательность $x_n$ \emph{слабо ограничена}, если для любого ограниченного функционала $f$
на $X$ имеем $\hm{f(x_n)} \le C(f)$.
\end{df}


\begin{lemma}
Существует изометричное вложение $X \inj X^{**}$.
\end{lemma}
\begin{proof}
Зададим вложение так: $x \mapsto F_x$, где $F_x \in X^{**}$\т функционал на $X^*$,
действующий на элементах $f \in X^*$ следующим образом:
$$F_x\cln f \mapsto f(x).$$
Это вложение, очевидно, линейно. Докажем, что это изометрия. Обозначим норму в $X^{**}$ через $\hn{\cdot}_2$.
С одной стороны, по определению нормы имеем $\hm{f(x)} \le \hn{f} \cdot \hn{x}$, поэтому
$$\hn{x} \ge \supl{f} \frac{\hm{f(x)}}{\hn{f}} = \hn{x}_2.$$
С другой стороны, в силу одного из следствий теоремы Хана\ч Банаха,
для всякого $x_0 \in X$ найдётся функционал~$f_0$ такой, что
$\hm{f_0(x_0)} = \hn{f_0} \cdot \hn{x_0}$, поэтому
$$\hn{x}_2 = \supl{f}\frac{\hm{f(x)}}{\hn f} \ge \hn{x},$$
следовательно, $\hn{x} = \hn{x}_2$.
\end{proof}

\begin{stm}
Слабо ограниченная последовательность ограничена по норме.
\end{stm}
\begin{proof}
Применим теорему Банаха\ч Штейнгауза к пространствам $X^*$ и $X^{**}$,
то есть вместо последовательности $\hc{x_i}$ рассматривая её образ в $X^{**}$.
В силу этой теоремы семейство образов будет ограниченным, но в силу изометричности вложения
этим свойством будет обладать и исходное семейство векторов.
\end{proof}

\comment{Если это рассуждение непонятно, то можно в лоб доказать аналог ТБШ. Впрочем, в конце
рассуждение о вложении $X \inj X^{**}$ придётся повторить. Доказательство можно найти
в приложении~\ref{sssec:adjoint.banach-shteinhouse}.}

\begin{stm}
Из слабой сходимости следует слабая ограниченность.
\end{stm}
\begin{proof}
Пусть $x_n \convw x$. Это означает, что для каждого $f$ найдется $N$ такое, что при
всех $n>N$ имеем $\hm{f(x_n) - f(x)} \le 1$. Но остальных $n$ лишь конечное число,
поэтому для каждого $f$ последовательность $\hc{f(x_n)}$ ограничена.
\end{proof}

\begin{theorem}[О слабой компактности]
Пусть $X$\т сепарабельное нормированное пространство. Тогда всякое ограниченное
бесконечное подмножество в $X^*$ является слабо предкомпактным.
\end{theorem}
\begin{proof}
Выберем в $X$ счётное всюду плотное множество $D := \hc{x_n}$. Пусть $\hc{f_n}$\т ограниченная
последовательность функционалов. Рассмотрим последовательность чисел $\hc{f_n(x_1)}$. Она ограничена,
а потому содержит сходящуюся. Обозначим её через $f_n\n1(x_1)$. Рассмотрим последовательность чисел
$\bc{f_n\n1(x_2)}$. Она тоже содержит сходящуюся подпоследовательность $f_n\n2(x_2)$.
Продолжая этот процесс и выделяя диагональ $\ph_n := f_n\n n$, получаем последовательность функционалов,
которая сходится на всех векторах $x_i$.

Покажем, что сходимость имеет место для всех векторов $x \in X$. Покажем фундаментальность
последовательности $\hc{\ph_i(x)}$.
Рассмотрим последовательность элементов из $D$, сходящуюся к $x$, тогда, очевидно,
\begin{multline*}
\hm{\ph_m(x) - \ph_n(x)} = \hm{\ph_m(x) - \ph_m(x_k) + \ph_m(x_k) - \ph_n(x_k) + \ph_n(x_k) - \ph_n(x)} \le\\\le
\hm{\ph_m(x) - \ph_m(x_k)} + \hm{\ph_m(x_k) - \ph_n(x_k)} + \hm{\ph_n(x_k) - \ph_n(x)} \ra 0.
\end{multline*}
Теорема доказана.
\end{proof}

\begin{imp}
Пусть $D \subs X$\т счётное всюду плотное множество.
Пусть для каждого $x \in D$ последовательность $f_k(x)$ сходится. Тогда существует
функционал $f$ такой, что $f_k \convw f$.
\end{imp}

\begin{stm}
Слабый предел единствен.
\end{stm}
\begin{proof}
Допустим, что $x_n \convw x$ и $x_n \convw y$, причём $x \neq y$. Тогда, по определению
слабой сходимости, для любого $f$ имеем $f(x_n) \ra f(x)$ и $f(x_n) \ra f(y)$. Следовательно,
для всякого функционала $f$ имеем $f(x) = f(y)$, то есть $f(x-y) = 0$.
Но по лемме о продолжении функционала существует $f$, который равен~$1$ на  векторе $x-y$.
Противоречие.
\end{proof}

\begin{lemma}
Пусть последовательность $\hc{x_n}$ в банаховом пространстве
слабо сходится к $x_0$ и предкомпактна. Тогда $x_n \ra x_0$ по норме пространства.
\end{lemma}
\begin{proof}
В силу предкомпактности из последовательности можно выделить фундаментальную подпоследовательность $x_{n_k}$.
В силу полноты пространства она сходится к некоторому вектору $\wh x$. Из сходимости
по норме очевидно следует слабая сходимость, поэтому $x_{n_k} \convw \wh x$.
Но слабый предел единствен, поэтому $\wh x = x_0$, что и требовалось доказать.
\end{proof}

\begin{imp}
Компактный оператор переводит слабо сходящуюся последовательность в сходящуюся по норме.
\end{imp}
\begin{proof}
Как уже было доказано, слабо сходящаяся последовательность ограничена.
По определению компактного оператора, $\hc{A x_n}$ предкомпактно,
поэтому содержит сходящуюся к некоторой точке $y$ подпоследовательность. Очевидно, что $\hc{Ax_n}$
тоже слабо сходится, а поскольку слабый предел совпадает с сильным (если последний существует),
то и образ всей последовательности сходится к~$y$.
\end{proof}


\subsubsection{Классификация точек спектра}

Пусть $A\cln X \ra X$\т ограниченный оператор в банаховом пространстве. Расклассифицируем точки $\la \in \Cbb$
для оператора $A - \la I$, причём здесь мы будем придерживаться классификации, используемой в книге Глазмана.

\begin{items}{-2}
\item Пусть $\Ker(A-\la I) \neq 0$, но это ядро конечномерно. Тогда $\la$\т собственные значения
конечной кратности. В этом случае, разумеется, $A - \la I$ необратим (даже в алгебраическом смысле).
Множество таких точек обозначим через $\Sig_p(A)$ и назовём \emph{точечным спектром}.
\item Пусть $\Img(A - \la I) \neq X$. Тогда возможно $2$~случая:
  \begin{items}{-2}
  \item[a)] $\Ker(A-\la I) = 0$ и $\Cl \Img (A - \la I) \neq \Img(A - \la I)$;
  \item[b)] $\la$\т собственное значение бесконечной кратности, то есть $\dim \Ker(A - \la I) = \infty$.
  \end{items}
  Такие точки $\la$ называются точками \emph{непрерывного спектра}. Обозначим его через $\Sig_c(A)$.
\item Пусть $\Img(A - \la I) \neq X$ и замкнут. [Ещё какой-то спектр]
\item $\Ker(A - \la I) = 0$ и $\Img(A - \la I) = X$. Тогда в силу теоремы Банаха, существует
  ограниченный обратный оператор $(A - \la I)^{-1}$. В этом случае говорят, что $\la$\т точка
  \emph{резольвентного множества}.
\end{items}

\subsubsection{Сохранение непрерывного спектра при компактном возмущении}

В этом разделе $X$\т сепарабельное гильбертово пространство, а
$A \cln X \ra X$\т ограниченный оператор.

\begin{petit}
\begin{lemma}
Положим $B := A -\la I$. Пусть $\la$\т точка непрерывного спектра, причём $\Ker B = 0$.
Положим $Y := \Img B$. Тогда отображение $C\cln Y \ra X$ является неограниченным оператором.
\end{lemma}
\begin{proof}
По условию $Y$ не является замкнутым подпространством, поэтому оно не может быть полным.
Допустим, что $C$\т ограниченный оператор. Возьмём фундаментальную последовательность $\hc{y_n} \subs Y$.
Положим ${x_n := C y_n}$, тогда, очевидно, $\hc{x_n}$\т тоже фундаментальна. В силу полноты $X$
она сходится к некоторому вектору $x \in X$. Поскольку $B$ ограничен и потому непрерывен,
получаем $Bx_n \ra Bx$. Но $B x_n = y_n$, а $Ax \in Y$. Значит, $y_n \ra y$. Тем самым показано, что $Y$
полное пространство. Противоречие, значит, оператор~$C$ не является ограниченным.
\end{proof}

\begin{lemma}
Положим $B := A - \la I$. Если $\la \in \Sig_c(A)$, то существует непредкомпактная
последовательность $x_n \in X$ такая, что $\hn{x_n} = 1$ и $B x_n \ra 0$.
\end{lemma}
\begin{proof}
Пусть $\la \in \Sig_c(A)$. Точки непрерывного спектра бывают двух видов, и придётся разобрать два случая.

Пусть сначала $\dim \Ker B = \bes$. Так как единичная сфера в бесконечномерном пространстве не
является предкомпактом, можно выбрать непредкомпактную последовательность $\hc{x_n} \subs \Ker B$.
Но в этом случае ${B x_n = 0}$, поэтому всё доказано.

Во второму случае применим только что доказанную лемму, которая говорит, что оператор $C := B^{-1}$,
(определённый, впрочем, только на $\Img B$) является неограниченным.
Тогда, как несложно видеть, существует последовательность $\hc{y_n} \subs \Img B$, для которой $\hn{y_n} = 1$ и
$\hn{C y_n} \ge n$, поэтому, положив
$$x_n := \frac{C y_n}{\hn{ Cy_n}},$$
получаем искомую последовательность, ибо
$\hn{B x_n} \le \frac{1}{n}$.

Осталось показать, что полученная последовательность непредкомпактна. Допустим противное,
тогда из неё можно выделить фундаментальную подпоследовательность, которая в силу полноты пространства
сходится к некоторому вектору $x \in X$. Перенумеруем её, тогда $x_n \ra x$.
В силу непрерывности $B$ имеем $B x_n \ra Bx$, с другой стороны, $Bx_n \ra 0$. Таким образом, $Bx = 0$.
Но поскольку $\Ker B = 0$, то и $x = 0$, с другой стороны, $\hn{x_n} = 1$ и $x_n \ra x$,
поэтому $\hn{x} = 1$. Противоречие.
\end{proof}
\end{petit}

При данном определении непрерывного спектра обратное утверждение леммы неверно, поэтому далее
в этом разделе будет использовано такое определение непрерывного спектра:

\begin{df}
Положим $B := A - \la I$. Точка $\la \in \Sig_c(A)$ тогда и только тогда, когда существует непредкомпактная
последовательность $x_n \in X$ такая, что $\hn{x_n} = 1$ и $B x_n \ra 0$.
\end{df}

\begin{df}
Множество векторов $\hc{x_n}$ называется \emph{$\ep$\д ежом}, если для $\fa n, m$ имеем
$\hn{x_n - x_m} \ge \ep$.
\end{df}

\begin{theorem}
Пусть $A\cln X\ra X$\т ограниченный оператор в сепарабельном гильбертовом пространстве $X$,
а $K\cln X \ra X$\т компактный оператор. Тогда $\Sig_c(A + K) = \Sig_c(A)$.
\end{theorem}
\begin{proof}
Пусть $\la \in \Sig_c(A)$. Положим $B := A - \la I$. По лемме\footnote{То есть по новому определению}
существует непредкомпактная последовательность~$\hc{x_n}$ такая, что $\hn{x_n} = 1$ и $B x_n \ra 0$.
Из доказательства критерия Хаусдорфа следует, что найдётся подпоследовательность $\hc{y_n} \subs \hc{x_n}$,
образующая $\ep$\д ежа. Поскольку $\hc{y_n}$ ограничена, по теореме о слабой компактности\footnote{Мы доказывали
её для сопряжённого пространства. Но в случае гильбертовых пространств мы пользуемся антиизоморфизмом $H$ и $H^*$,
и потому шар в гильбертовом пространстве слабо компактен. Заметим, что именно здесь используется сепарабельность.},
из~$y_n$ можно выделить слабо сходящуюся последовательность $\hc{z_n}$.

Покажем, что последовательность $\hc{z_n - z_{n+1}}$ будет непредкомпактной.
Действительно, допустим, что из неё можно выбрать фундаментальную.
Тогда в силу полноты пространства получаем
$$\De z_k := \br{z_{n_k+1} - z_{n_k}} \ra z \in X.$$
Пусть $f$\т произвольный ограниченный функционал. В силу слабой сходимости имеем
$$f(\De z_k) = f(z_{n_k + 1}) - f(z_{n_k}) \ra 0,$$
поскольку каждое слагаемое сходится к одному и тому же числу.
Таким образом, $\De z_k \convw 0$. Но у этой последовательности существует и сильный
предел $z$, поэтому $z = 0$, то есть $z_{n_k+1} - z_{n_k} \ra 0$.
Но это противоречит тому, что последовательность $\hc{z_n}$ является $\ep$\д ежом.

Осталось показать, что $(B + K)(z_n - z_{n+1}) \ra 0$. В самом деле, имеем
$$(B+K)(z_n - z_{n+1}) = B z_n - B z_{n+1} + K z_n - K z_{n+1}.$$
Первые два слагаемых идут к нулю в силу выбора $\hc{z_n}$, а вторые два
сходятся к одному и тому же вектору, поскольку слабо сходящаяся последовательность $\hc{z_n}$
перерабатывается компактным оператором $K$ в сходящуюся по норме.
Применяя лемму в обратную сторону, получаем, что $\la \in \Sig_c(B+K)$.
\end{proof}

\subsection{Компактные операторы в гильбертовых пространствах}

\subsubsection{Теорема Гильберта\ч Шмидта}

В этом разделе $A$\т компактный самосопряжённый оператор в сепарабельном гильбертовом пространстве~$H$.
Введём обозначение $Q(x) := (Ax,x)$. Заметим, что это число всегда вещественно в силу самосопряжённости
оператора.

\begin{lemma}
Пусть $x_n \convw x$. Тогда $Q(x_n) \ra Q(x)$.
\end{lemma}
\begin{proof}
Имеем
\begin{multline*}
\hm{Q(x_n) - Q(x)} = \hm{(A x_n,x_n) - (Ax,x)} = \hm{(Ax_n, x_n) - (Ax,x_n) + (Ax, x_n) - (Ax,x)} \le\\ \le
\hm{(Ax_n, x_n) - (Ax, x_n)} + \hm{(Ax, x_n) - (Ax, x)} \stackrel{!}{=}
\hm{\br{A(x_n-x), x_n}} + \hm{\br{x, A(x_n-x)}} \le \\ \le \hn{A(x_n-x)}\cdot\hn{x_n} + \hn{x}\cdot\hn{A(x_n-x)} \ra 0.
\end{multline*}
Здесь равенство <<!>> следует из свойств самосопряжённого оператора, а сходимость к нулю вытекает из свойств
компактных операторов и ограниченности $\hn{x_n}$ (а это\т следствие слабой сходимости).
\end{proof}

\begin{lemma}
Если $\hm{Q(x)}$ достигает на единичной сфере своего максимума в точке $x_0$,
то для любого вектора $y$ такого, что $(x_0,y) = 0$, выполнено $(Ax_0, y) = 0$,
то есть $\ha{x_0}^\bot \subs \ha{Ax_0}^\bot$.
\end{lemma}
\begin{proof}
Рассмотрим вектор
$$v := \frac{x_0 + ay}{\hn{x_0 + ay}}, \quad a \in \Cbb.$$
Используя самосопряжённость оператора и теорему Пифагора для векторов $x_0$ и $y$, получаем
$$Q(v) = \frac{1}{1 + \hm{a}^2\cdot\hn{y}^2}\cdot\br{ Q(x_0) + 2\Rea\br{\ol a(Ax_0,y)} + \hm{a}^2Q(y)}.$$
Если $(Ax_0,y) \neq 0$, то выбирая $a$ малым по модулю и подкручивая его аргумент, можно сделать так,
что число $\Rea\br{\ol a(Ax_0,y)}$ будет ненулевым вещественным и будет иметь тот же знак, что и $Q(x_0)$.
Тогда $\hm{Q(v)} > \hm{Q(x_0)}$, а мы предположили, что $x_0$ максимизирует модуль $Q$. Полученное
противоречие показывает, что $(Ax_0,y) = 0$.
\end{proof}

\begin{imp}
Если $\hm{Q(x)}$ достиг максимума на векторе $x_0$, то это собственный вектор оператора~$A$.
\end{imp}
\begin{proof}
По лемме имеем $\ha{x_0}^\bot \subs \ha{Ax_0}^\bot$, поэтому $\br{\ha{x_0}^\bot}^\bot \sups
\br{\ha{Ax_0}^\bot}^\bot$, то есть $\ha{Ax_0} \subs \ha{x_0}$.
\end{proof}

\begin{theorem}[Гильберта\ч Шмидта]
Компактный самосопряжённый оператор $A$ в сепарабельном гильбертовом пространстве $H$
обладает базисом из собственных векторов.
\end{theorem}
\begin{proof}
Будем строить элементы базиса по индукции в порядке убывания модулей собственных значений.

Покажем, что на единичной сфере функция $\hm{Q(x)}$ достигает своего максимума. Пусть $S := \sup \hm{Q(x)}$,
а $x_n$\т последовательность единичных векторов, реализующая~$S$. Поскольку единичный шар
слабо предкомпактен, можно выбрать подпоследовательность $y_n \convw y$. При этом
в силу первой леммы получаем $\hm{Q(y_n)} \ra \hm{Q(y)}$, поэтому $\hm{Q(y)} = S$.

В качестве первого базисного вектора $e_1$ возьмём вектор $y$. Теперь рассмотрим подпространство
$\ha{e_1}^\bot$. Оно в силу самосопряжённости оператора инвариантно относительно~$A$. В нём
повторим эту же процедуру, найдём~$e_2$, и так далее.
Если начиная с какого\д то момента мы получаем $Q(x) \equiv 0$, это означает, что ненулевые собственные
значения кончились, и мы попали в ядро оператора. Во противном случае получаем последовательность
ненулевых собственных значений $\hc{\la_n}$. Они, очевидно, сходятся к нулю. В самом деле, если
бы их модули были ограничены снизу, то образы единичных базисных векторов образовывали бы ежа,
а не предкомпактное множество.

Таким образом, мы представили произвольный вектор $x \in H$ в виде
\eqn{x = \sum c_ie_i + z,}
где $z \in \Ker A$, причём оператор действует диагонально: $Ax = \sum \la_i c_i e_i$.
\end{proof}

\begin{imp}[Об общем виде компактного оператора в гильбертовом пространстве]
Для всякого компактного оператора $A\cln H \ra H$ существуют ОНС
$\hc{\ph_k}$ и $\hc{\psi_k}$, такие, что ряд
$$Ax = \sumkui \la_k (x,\psi_k)\ph_k$$
сходится по норме.
\end{imp}
\begin{proof}
Рассмотрим оператор $A^*A$. Он будет самосопряжённым и компактным, поскольку является произведением
компактного и ограниченного. По предыдущей теореме, существует
ортонормированная система $\hc{\psi_k}$, в которой $A^*A\psi_k = \mu_k\psi_k$.
Легко видеть, что $\mu_k > 0$.
Положим $\la_k := \sqrt{\mu_k}$ и рассмотрим $\ph_k = \frac{1}{\la_k}A\psi_k$.
Можно считать, что $\mu_k$ были упорядочены по убыванию. Осталось проверить, что
для произвольного вектора $x$ имеет место разложение
$$Ax = \suml{k=1}{\infty} \la_k(x,\psi_k)\ph_k.$$
В самом деле, имеем
$$x = \suml{k=1}{\bes}(x,\psi_k)\psi_k + x_0, \quad x_0 \in \Ker A^*A.$$

Покажем, что $Ax_0 = 0$. В самом деле, если $A^*Ax_0 = 0$,
то $(A^*Ax_0,x_0) = 0$, а это эквивалентно $(Ax_0, Ax_0) = 0$, значит, $Ax_0 = 0$.

Следовательно,
$$Ax = \suml{k=1}{\bes}\la_k(x,\psi_k)\frac{A\psi_k}{\la_k} = \suml{k=1}{\infty}\la_k(x,\psi_k)\ph_k,$$
что и требуется. Теперь нужно проверить ортонормированность системы $\hc{\ph_k}$. Имеем
$$(\ph_k, \ph_k) = \hr{\frac{A\psi_k}{\la_k}, \frac{A\psi_k}{\la_k}}
=\frac{1}{\la_k^2}\hr{A^*A\psi_k, \psi_k} = \frac{1}{\la_k^2}\hr{\la_k^2\psi_k, \psi_k} = 1.$$
\hfill\end{proof}

\begin{petit}
Далее приводим доказательство Стёпина. Не факт, что оно правильное, а вот приведённое выше доказательство
является абсолютно строгим.

\begin{df}
\emph{Мера Дирака} $\de_p$\т это мера, сосредоточенная в одной точке $p \in X$.
Она обозначается $\de_p$ и определяется так:
$$\de_p(A) := \case{1,& p \in A;\\ 0, & p \notin A.}$$
\end{df}
Можно рассмотреть новую меру, которая есть линейная комбинация дираковских мер:
$$\si(A)= \sum \la_i \de_{p_i}(A).$$

\begin{note}
Пусть $\si$\т мера на отрезке $[a,b]$. Пространство $L_2(\si)$ конечномерно тогда и только
тогда, когда $\si$\д линейная комбинация дираковских мер.
\end{note}

\begin{theorem}[Гильберта]
Для компактного самосопряжённого оператора в сепарабельном гильбертовом пространстве
существует ортонормированный базис из собственных векторов.
\end{theorem}
\begin{proof}
В силу спектральной теоремы, оператор $A$ может быть реализован как умножение на независимую
переменную: $A\cln f(\la) \mapsto \la f(\la)$. Нам хочется рассмотреть оператор, обратный к $A$,
но этому препятствует деление на $\la$. Откусим от нуля окрестность $U$, тогда на множестве
$K = \hs{-\hn A, \hn A} \wo U(0)$ оператор $A$ обратим: $A^{-1}\cln f(\la) \mapsto \frac{1}{\la}f(\la)$.
Но в силу замечания перед доказательством теоремы, в бесконечномерном пространстве не
бывает обратимых компактных операторов. Значит, $L_2(\si)$ конечномерно на $K$,
поэтому на всём отрезке $\hs{-\hn A, \hn A}$ может быть не более чем счётное число атомов меры $\si$,
причём им разрешено сгущаться только в окрестности $\la = 0$. Ясно [?], что $\de$\д функции будут собственными.
Так как спектр замкнут, то если оператор имеет счётное множество точек в спектре, то и ноль принадлежит спектру.

Если пространство сепарабельно, то можно рассмотреть и общий случай, когда циклического вектора не существует.
Пусть $\hc{f_n}$\т счётное всюду плотное множество в $H$. В этом случае будем рассматривать циклические
подпространства, порождённые векторами $f_n$. Берём первый вектор $f_1$ и рассматриваем $\ha{f_1}$,
для которого проходит наша процедура. Затем берём ортогональное дополнение к $\ha{f_1}$,
в нём выбираем ещё один вектор, и так далее\ldots В итоге получим прямую сумму
$\oplusl{k=1}{\infty}{\ha{f_{n_k}}}$.
\end{proof}

\end{petit}


\subsubsection{Интегральные операторы Гильберта\ч Шмидта}

\begin{df}
Рассмотрим функцию $K(x,y) \in L_2[a,b]^2$ и
оператор $A\cln L_2[a,b] \ra L_2[a,b]$, заданный так:
$$A(f) := \intl{a}{b}K(x,y)f(y)\,dy.$$
Этот оператор называется \emph{интегральным оператором Гильберта\ч Шмидта}.
Функция $K$ называется \emph{ядром} интегрального оператора. Через $A_K$ мы будем обозначать
оператор с ядром $K$.
\end{df}

\begin{problem}
Доказать, что $A_K^* = A_{\ol K}$.
\end{problem}

\begin{df}
\emph{Свёртка} двух ядер $K$ и $L$\т это ядро
$$(K*L)(x,z) := \intl{a}{b}K(x,y)L(y,z)\,dy.$$
\end{df}

\begin{problem}
Докажите, что $A_K A_L = A_{K*L}$.
\end{problem}

Для сокращения записи не будем писать пределы интегрирования по $[a,b]$.
Все нормы для функций понимаются в смысле тех пространств, где эти функции живут.

\begin{stm}
Интегральный оператор $A$ с ядром $K$ ограничен.
\end{stm}
\begin{proof}
По теореме Фубини интеграл $\int K(x,y)f(y)\,dy$ существует. По неравенству Коши\ч Буняковского
$$\hm{\int K(x,y) f(y)\, dy}^2 \le \int \hm{K(x,y)}^2\,dy \cdot \int \hm{f(y)}^2\,dy.$$
Это неравенство можно проинтегрировать по $x$, поскольку интеграл
в правой части существует. Получим
$$\hn{A f}^2 = \int dx \hm{\int K(x,y)f(y)\,dy}^2 \le \int dx \int \hm{K}^2dy \cdot \int\hm{f}^2dy =
\hn{K}^2\cdot \hn{f}^2.$$
\hfill\end{proof}


Следующая теорема на лекциях не формулировалась, однако будет использоваться
при доказательстве компактности операторов Гильберта\ч Шмидта. Доказательство этой теоремы можно
найти в~\cite[гл.VII, \S~3, п.~5]{kf}.

\begin{theorem}
Пусть $\hc{\ph_n}$\т полная ортогональная система в $L_2[a,b]$. Тогда всевозможные произведения
$\hc{\ph_n(x)\cdot\ph_m(y)}$ образуют полную ортогональную систему в $L_2[a,b]^2$.
\end{theorem}


\begin{theorem}
Интегральный оператор Гильберта\ч Шмидта $A$ с ядром $K$ компактен.
\end{theorem}
\begin{proof}
Разложим ядро $K$ нашего оператора по базису пространства $L_2[a,b]^2$:
$$K(x,y) = \suml{m,n=1}{\bes} c_{mn} \ph_m(x)\ph_n(y).$$
Положим
$$K_N(x,y) = \suml{m,n=1}{N} c_{mn} \ph_m(x)\ph_n(y).$$
Покажем, что оператор $A_N$ с ядром $K_N$ имеет конечномерный образ. В самом деле,
$$A_N f = \int K_N(x,y) f(y)\,dy = \suml{m,n=1}{N} c_{mn} \ph_m(x) \int f(y) \ph_n(y) \,dy.$$
то есть образ любой функции $f$ есть конечная линейная комбинация функций $\ph_m(x)$.

Осталось заметить, что $\hn{A_N - A} \ra 0$, так как $\hn{K_N - K} \ra 0$
в силу того, что это частичные суммы ряда Фурье, а $\hn{A} \le \hn{K}$, как следует
из доказанного выше утверждения. Значит, оператор $A$ является пределом конечномерных (а значит,
компактных) операторов и потому сам компактен.
\end{proof}

\section{Метрические и топологические пространства}

\subsection{Топологические пространства. Компактность}

\subsubsection{Понятие топологии. Открытые и замкнутые множества}

\begin{df}
Говорят, что в множестве~$X$ определена \emph{топология}, если в~$X$ отмечен класс подмножеств
$\tau := \hc{U_\al \subs X}$ со
следующими свойствами:
\begin{items}{-3}
\item $\es \in \tau$ и $X \in \tau$.
\item Если $\hc{U_1\sco U_n} \subs \tau$, то и $\bigcap U_i  \in \tau$.
\item Если $\hc{U_\be} \subs \tau$, то и $\bigcup U_\be  \in \tau$.
\end{items}
Пару $(X, \tau)$ называют \emph{топологическим пространством}. Множества из $\tau$ называются \emph{открытыми}.
\end{df}

\begin{df}
\emph{База} топологии $\tau$\т такая подсистема
открытых множеств $\Bc \subs \tau$, что всякое открытое множество представимо в виде объединения
элементов из $\Bc$. Эквивалентная формулировка: совокупность $\Bc$ есть база, если
для всякого открытого множества $U$ и всякой точки $x \in U$ найдётся $V \in \Bc$, для которого
$x \in V \subs U$. Говорят, что топология обладает \emph{счётной базой}, если существует такая база $\Bc \subs \tau$,
что $\Card \Bc \le \aleph_0$.
\end{df}

\begin{df}
Множество называется \emph{замкнутым}, если дополнение к нему открыто.
\end{df}

\begin{df}
\emph{Окрестностью} $U$ точки $x \in X$ называется произвольное открытое множество,
содержащее эту точку.
\end{df}


\begin{df}
Точка $x \in X$ называется \emph{предельной} точкой множества $M \subs X$,
если всякая проколотая окрестность точки $x$ содержит точку множества $M$.
\end{df}

\begin{df}
\emph{Замыканием} $\Cl M$ множества $M \subs X$ называется добавление к нему всех его предельных точек.
\end{df}

\begin{stm}
Множество $\Cl M$ замкнуто.
\end{stm}
\begin{proof}
Возьмём произвольную точку из дополнения к $\Cl M$. Она обладает окрестностью $U$, не пересекающейся
с множеством $\Cl M$. Теперь пробежимся по всем точкам дополнения и объединим все такие
окрестности. Это будет открытое множество, не пересекающееся с $\Cl M$.
\end{proof}

\begin{stm}
Замкнутое множество содержит все свои предельные точки.
\end{stm}
\begin{proof}
Допустим, что замкнутое множество $M$ пространства $X$ не содержит какой\д нибудь
своей предельной точки~$x$. Рассмотрим множество $X \wo M$. Оно содержит точку $x$
и не имеет с $M$ общих точек. Значит, $X\wo M$\т открытая окрестность точки $x$,
не пересекающаяся с $M$. Значит, точка $x$ не может быть предельной для $M$.
\end{proof}

\begin{imp}
Множества, которые не имеют предельных точек, являются замкнутыми.
\end{imp}


\subsubsection{Компактность. Критерии компактности}

\begin{df}
Топологическое пространство называется \emph{компактным}, если из всякого его открытого покрытия
можно выделить конечное подпокрытие.
\end{df}

\begin{df}
Система множеств $X_\al \subs X$ называется \emph{центрированной}, если любое конечное пересечение
множеств из этого семейства непусто.
\end{df}


\begin{theorem}[Критерий компактности в терминах центрированных замкнутых систем]
\label{th:comp.eq.cent.sys}
Пространство $X$ компактно тогда и только тогда, когда любая центрированная система замкнутых множеств имеет
непустое пересечение.
\end{theorem}
\begin{proof}
Пусть $X$ компактно. Рассмотрим произвольную центрированную систему замкнутых множеств $\hc{F_\al}$.
Допустим, что $\bigcap F_\al = \es$. Тогда
$X \wo \bigcap F_\al = X \wo \es = X = \bigcup (X \wo F_\al)$.
Множества $X \wo F_\al$ открыты, поэтому $\hc{X \wo F_\al}$ есть открытое покрытие пространства $X$.
Из него можно выделить конечное подпокрытие $\hc{X \wo F_i}_{i=1}^n$, откуда, снова переходя к дополнениям,
получаем, что $\capiun F_i = \es$. Это противоречит тому, что исходная система центрирована.

Обратно, рассмотрим произвольное открытое покрытие $X = \bigcup G_\al$.
Тогда $\bigcap (X \wo G_\al) = \es$,
поэтому эта система не может быть центрированной. Поэтому найдётся конечная подсистема $\hc{X \wo G_i}_{i=1}^n$,
для которой $\capiun (X \wo G_i) = \es$.
Тогда $\hc{G_1\sco G_n}$ будет искомым конечным подпокрытием.
\end{proof}

\begin{problem}
Замкнутые подмножества компактных топологических пространств компактны.
\end{problem}
\begin{solution}
Пусть $X$\т компактное топологическое пространство, и $F \subs X$\т замкнутое подмножество.
Рассмотрим произвольное открытое покрытие $\hc{G_\al}$ этого множества. Поскольку $X \wo F$ открыто,
система
$$(X \wo F) \cup \hr{\bigcup G_\al}$$
есть открытое покрытие для~$X$. В силу компактности пространства из неё можно выделить конечное подпокрытие
$\hc{G_1\sco G_n}$. Исключив из этого подпокрытия множество $X \wo F$, мы получим
конечное открытое подпокрытие для $F$.
\end{solution}



\begin{df}
Множество называется \emph{счётно\д компактным}, если любое бесконечное подмножество в нём
имеет предельную точку.
\end{df}

\begin{note}
В этом определении слово <<бесконечное>> можно заменить на слово <<счётное>>.
\end{note}

\begin{stm}
Следующие условия эквивалентны:
\begin{items}{-3}
\item Пространство $X$ счётно\д компактно.
\item Всякая счётная центрированная система замкнутых множеств имеет непустое пересечение.
\item Всякое счётное открытое покрытие содержит конечное подпокрытие.
\end{items}
\end{stm}
\begin{proof}
Установим сначала эквивалентность первого и второго утверждения.

Пусть $X$ счётно\д компактно. Рассмотрим центрированную систему замкнутых подмножеств $\hc{F_n}$.
Рассмотрим систему вложенных замкнутых множеств
$$S_1 := F_1,\; S_2 := F_1 \cap F_2,\; S_3 := F_1\cap F_2\cap F_3\etc$$
Все они непусты, поскольку система центрированная. Если эта последовательность стабилизируется,
то всё доказано. Если она убывает, то можно считать, что она убывает строго.
Выберем последовательность $x_i \in S_i$, тогда в силу определения счётной компактности,
она имеет предельную точку $x_0$. Она принадлежит каждому множеству $S_i$, поскольку они замкнуты,
и вся последовательность, за исключением конечного числа точек, содержится в каждом из множеств $S_i$.
Поэтому $x_0$ принадлежит пересечению всех $F_i$, значит, оно непусто.

Обратно, пусть нашлась последовательность $\hc{x_n}$, у которой нет предельных точек.
Следовательно, множества $\hc{x_n}_{n \ge k}$ замкнуты (у них тем более нет предельных
точек). Заметим, что это центрированная система.
Но, очевидно,
$$\capkui \hc{x_n}_{n \ge k} = \es.$$
Значит, эта центрированная система имеет пустое пересечение.

Доказательство эквивалентности второго и третьего утверждений проводится так:
в теореме~\ref{th:comp.eq.cent.sys} заменяем слова <<компактность>> на
<<существование среди всякого счётного покрытия конечного подпокрытия>>,
а произвольные центрированные системы заменяем счётными.
\end{proof}

\begin{stm}
Пусть $X$\т пространство со счётной базой. Тогда все покрытия $X$ можно считать счётными.
\end{stm}
\begin{proof}
Пусть $\hc{U_n}$\т база топологии. Пусть $X = \bigcup G_\al$. Рассмотрим какое\д нибудь
$G_\al$ и произвольную точку~$x$ в~нём. Найдём такое $U_k$, что $x \in U_k \subs G_\al$.
Поступим так со всеми точками $x$ и с каждым $G_\al$, тогда все полученные таким образом
множества $U_k$ будут образовывать искомое счётное покрытие.
\end{proof}

\begin{df}
Пространство $X$ называется \emph{сепарабельным}, если в нём есть счётное всюду плотное множество.
\end{df}


\begin{stm}
Пространство $X$ со счётной базой сепарабельно.
\end{stm}
\begin{proof}
В самом деле, пусть $\hc{U_n}$\т счётная база. Возьмём в каждом множестве $U_n$
по одной точке $x_n$. Покажем, что последовательность $\hc{x_n}$ всюду плотна в $X$.
В самом деле, возьмём произвольную точку $x$ и рассмотрим произвольную её окрестность.
Она является объединением некоторого набора элементов базы, значит, в эту окрестность
попадёт хотя бы одна точка последовательности $\hc{x_n}$.
\end{proof}


\subsection{Метрические пространства}

\subsubsection{Определение метрического пространства}

\begin{df}
\emph{Метрическое пространство}\т это множество $M$ с функцией $\rho\cln M\times M \ra \R$ такой, что:
\begin{items}{-3}
\item $\rho \ge 0$, причём для всех $x, y\in M$ выполнено $\rho(x,y) = 0$ тогда и только тогда, когда $x = y$;
\item $\rho(x,y) = \rho(y,x)$ для всех $x, y \in M$;
\item $\rho(x,y) \le \rho(x,z) + \rho(z,y)$ для всех $x,y,z \in M$ (неравенство треугольника).
\end{items}
\end{df}

\begin{note}
Всякое нормированное пространство является метрическим пространством. Действительно,
легко проверить, что задание метрики по формуле $\rho(x,y) := \hn{x - y}$ удовлетворяет
аксиомам метрического пространства. Обратное, вообще говоря, неверно.
\end{note}

\begin{df}
Пусть $M$\т метрическое пространство.
Говорят, что последовательность $\hc{x_n} \subs M$ \emph{сходится} к $x \in M$, если $\rho(x_n,x) \ra 0$.
Последовательность $\hc{x_n}$ называется \emph{фундаментальной}, если для $\fa \ep > 0$ найдётся $N$
такое, что для $\fa n,m\ge N$ имеем $\rho(x_n,x_m) < \ep$.
Метрическое пространство называется \emph{полным}, если в нём всякая фундаментальная последовательность
сходится к некоторой точке этого пространства.
\end{df}

Очевидно, что замкнутое подмножество полного метрического пространства является полным пространством.

\subsubsection{Принцип сжимающих отображений}

\begin{df}
Пусть $M$\т метрическое пространство. Отображение $f\cln M \ra M$ называется \emph{сжимающим},
если найдётся $\al \in [0,1)$, для которого $\fa x, y \in M$ имеем $\rho\br{f(x),f(y)} \le \al\cdot \rho(x,y)$.
\end{df}

\begin{theorem}[О сжимающих отображениях]
Пусть $M$\т полное метрическое пространство, а~$f$\т сжимающее отображение.
Тогда у него существует единственная неподвижная точка.
\end{theorem}
\begin{proof}
Единственность такой точки сразу следует из определения сжимающего отображения: если бы их было две,
тогда расстояние между ними сохранилось бы, что невозможно. Докажем существование:
рассмотрим произвольную точку $y \in M$ и рассмотрим итерации нашего отображения:
$$y,\; f(y),\;  f\br{f(y)} = f^2(y),\;  f^3(y),\;  \ldots$$
Положим $y_k = f^k(y)$. Последовательность $\hc{y_k}$, очевидно, фундаментальна. В самом деле,
$$\rho(y_k, y_{k+1}) \le \al^k\rho(y, y_1),$$
поэтому
$$
  \rho(y_n, y_m) \le \rho(y_m, y_{m+1}) + \rho(y_{m+1}, y_{m+2})\spl \rho(y_{n-1}, y_n)=
  (\al^m + \al^{m+1} \spl \al^n)\rho(y,y_1),
$$
а последнее выражение можно сделать маленьким как остаток сходящегося ряда для геометрической прогрессии.
В силу полноты пространства, наша последовательность сходится к некоторому пределу $x \in M$.
Покажем, что это и есть искомая неподвижная точка.
Отображение $f$, очевидно, является непрерывным, поскольку близкие точки переходят в близкие.
По свойствам непрерывных отображений имеем $f(y_k) \ra f(x)$, если $y_k \ra x$. Поэтому,
если $f^k(y) \ra x$, то и $f\br{f^k(y)} \ra f(x)$. Но последовательности
$\hc{f^k(y)}$ и $\hc{f\br{f^k(y)}}$ совпадают с точностью до первого члена, поэтому их пределы
одинаковы. Следовательно, $x = f(x)$, что и требовалось доказать.
\end{proof}

\begin{problem}
Пусть $X$\т полное метрическое пространство, а про непрерывное отображение $f\cln X \ra X$ известно,
что некоторая его степень~$f^k$ является сжимающим отображением. Доказать, что оно имеет
единственную неподвижную точку. Можно ли отказаться в этом утверждении от непрерывности~$f$?
\end{problem}
\begin{solution}
По предыдущей теореме, отображение $F := f^k$ имеет единственную неподвижную точку $x_0$.
Очевидно, если $x$\т неподвижная точка отображения $f$, то она тем более является неподвижной
точкой отображения~$F$. Осталось доказать, что~$x_0$ действительно является неподвижной точкой
для~$f$. Допустим противное, то есть $f(x_0) = x_1 \neq x_0$. Тогда
$$F(x_1) = F\br{f(x_0)} = f^k\br{f(x_0)} = f^{k+1}(x_0) = f\br{f^k(x_0)} = f\br{F(x_0)} = f(x_0) = x_1,$$
то есть $x_1$ является ещё одной неподвижной точкой для $F$. Противоречие. А непрерывность не нужна!
\end{solution}

\begin{problem}
Доказать существенность условия полноты в теореме о сжимающих отображениях.
\end{problem}
\begin{solution}
Берём полное метрическое пространство\т прямую $\R$, выкалываем из неё точку $x = 0$,
получаем неполное метрическое пространство.
Рассматриваем отображение $x \mapsto \frac x2$. Оно, очевидно, сжимающее,
но не имеет неподвижных точек, поскольку при $x \neq 0$ равенство $\frac x 2 = x$ невозможно.
\end{solution}

\subsubsection{Теорема о пополнении метрических пространств}

\begin{df}
Метрическое пространство $M$ называется \emph{ограниченным}, если найдётся $x \in M$ и $C > 0$, для которых
при всех $y \in M$ имеем $\rho(x,y) \le C$.
\end{df}

\begin{df}
Полное пространство $(\wt M, \wt \rho)$ называется \emph{пополнением} метрического пространства $(M, \rho)$,
если найдётся инъекция $\ph\cln M \ra \wt M$ такая, что $\Cl \ph(M) = \wt M$ и для $\fa x,y$ имеем
$\rho(x,y) = \wt\rho\br{\ph(x), \ph(y)}$.
\end{df}

\begin{note}
В принципе, можно отказаться от требования $\Cl \ph(M) = \wt M$,
но без него указанное пополнение может оказаться не единственным.
\end{note}

\begin{theorem}
Для всякого метрического пространства $(M, \rho)$ существует
пополнение $(\wt M, \wt \rho)$, причём оно единственно в том смысле,
что если $(\wt M_1, \wt\rho_1)$ и $(\wt M_2, \wt\rho_2)$\т два пополнения одного и того
же пространства, то $(\wt M_1, \wt\rho_1)$ изометрично $(\wt M_2, \wt\rho_2)$.
\end{theorem}

Доказательство этой теоремы можно прочесть в~\cite[гл.~II, \S~3, п.~4]{kf}.

\begin{problem}
Доказать, что полное метрическое пространство из четырёх точек $A, B, C, D$ с расстояниями
$\rho(A,B) = \rho(B, C) = \rho(C,A) = 1$ и $\rho(A, D) = \rho(B, D) = \rho(C, D) = \frac12$
нельзя вложить в гильбертово пространство.
\end{problem}
\begin{solution}
Достаточно показать, что в трёхмерном пространстве нет четырёх точек с такими расстояниями.
Но это очевидно\т шары радиуса $\frac12$ с центрами в вершинах правильного треугольника со стороной~$1$
не имеют общей точки. Комплексный случай сводится к вещественному\т достаточно рассмотреть
наше пространство как вещественное с тем же скалярным произведением.
\end{solution}


\begin{petit}
А вот доказательство теоремы о пополнении, которое было дано на лекциях. Исходно оно было неверным:
ошибка лектора была в том, что нужно было рассматривать непрерывные и \emph{ограниченные} функции.

\begin{proof}
Для начала докажем это для ограниченных метрических пространств.
Рассмотрим пространство $\Cb(M)\cap \Bb(M)$ непрерывных ограниченных функций на пространстве $M$
с чебышёвской нормой
$$\dist(f,g) = \supl{x} \hm{f(x) - g(x)}.$$

Оно, как легко видеть, полное (равномерный предел непрерывных функций непрерывен). Покажем, что
существует изометричное вложение $M \inj \Cb(M)$. Построим отображение $\ph\cln x \mapsto f_x$,
где $f_x(y) = \rho(x, y)$. Понятно, что это корректно,
поскольку для разных точек эти функции будут иметь нули в разных точках:
если $x_1 \neq x_2$, то $f_{x_i}(x_i) = 0$, а $f_{x_1}(x_2) \neq 0 \neq f_{x_2}(x_1)$.

Покажем, что данное вложение является изометрией. Имеем в силу неравенства треугольника
$$
  \dist(f_{x_1}, f_{x_2}) = \supl{x} \hm{\rho(x_1,x) - \rho(x_2,x)} \le \rho(x_1,x_2),
$$
причём при $x = x_1$ получаем как раз значение $\rho(x_1,x_2)$.
Вот оно и построено.
\end{proof}
\end{petit}

\subsubsection{Теорема о вложенных шарах и теорема Бэра о категориях}

В этом параграфе $M$\т метрическое пространство.

\begin{theorem}[О вложенных шарах]
Пусть пространство $M$ полно, и $\hc{B_i(x_i, r_i)}$\т последовательность вложенных
замкнутых шаров, причём $r_i \ra 0$. Тогда их пересечение непусто.
\end{theorem}
\begin{proof}
Поскольку $r_i \ra 0$, а $B_i \sups B_{i+1}$, последовательность $\hc{x_i}$ будет
фундаментальной и потому сходится
к некоторому $x \in M$ в силу полноты пространства. Покажем, что $x$ является искомой точкой.
Действительно, если бы нашёлся шар $B_{i_0}$ такой, что $x \notin B_{i_0}$, тогда бы
точка $x$ не лежала бы ни в одном из шаров, начиная с номера $i_0$. Но поскольку дополнение к $B_{i_0}$
открыто, $x$ можно отделить окрестностью от всех шаров, начиная с номера $i_0$. Это противоречит тому,
что $x$\т предел последовательности центров шаров.
\end{proof}

\begin{note}
Очевидно, что в силу сходимости $r_i \ra 0$ это пересечение будет состоять из одной точки.
Действительно, если бы их было две, то расстояние $d$ между ними было бы
ненулевое. Когда радиусы шаров станут меньше, чем $\frac d3$, эти две точки не поместятся
в шаре такого радиуса одновременно.
\end{note}

\begin{problem}
Показать существенность требования $r_i \ra 0$ в теореме о вложенных шарах.
\end{problem}
\begin{solution}
В качестве примера, подтверждающего необходимость этого условия, рассмотрим пространство~$\N$
с метрикой
\eqn{\rh(n,m) := \case{0, & m=n,\\
1+\frac{1}{m}+\frac{1}{n}, & m \neq n.}}
Рассмотрим замкнутые шары  $B_n$ с центрами в точках $n$ и радиусами $1+\frac{2}{n}$. Тогда они все
вложены друг в друга, но их пересечение пусто. В самом деле, шар $B_n$ состоит из точек $m$ таких, что $m \ge n$,
потому что лишь при таких $m$ имеем $1 + \frac1m+\frac1n \le 1 + \frac2n$. Таким образом, центры шаров
находятся <<с краю>>, и пересечение $\caps{n}[n,+\bes)=\es$.
\end{solution}



\begin{problem}
Показать, что для банаховых пространств требование $r_i \ra 0$ можно убрать.
\end{problem}

\begin{df}
Множество $Y \subs M$ называется \emph{нигде не плотным} в $M$, если всякий шар $B \subs M$ ненулевого радиуса
содержит другой шар $B'$ ненулевого радиуса такой, что $Y \cap B' = \es$.
\end{df}

\begin{stm}
Замыкание нигде не плотного множества $M$ является нигде не плотным.
\end{stm}
\begin{proof}
Допустим, что замыкание плотно в некотором шаре $B$. Это означает,
что всякий шар $B'$ внутри $B$ содержит точку из $\Cl M$, но это значит, что где\д то рядом
есть и точка из множества $M$, причём можно считать, что эта точка принадлежит $B'$.
Но это означает, что $M$ плотно в $B$. Противоречие.
\end{proof}

\begin{df}
Множество $Y \subs M$ называется \emph{всюду плотным} в $M$, если $\Cl Y = M$.
\end{df}

\begin{df}
Множество $Y$ называется \emph{множеством первой категории}, если оно может быть представлено как счётное
объединение нигде не плотных множеств.
\end{df}

\begin{theorem}[Бэра о категориях]
Полное метрическое пространство $M$ не может быть множеством первой категории.
\end{theorem}
\begin{proof}
Допустим, что $M = \bigcup Y_i$, причём $Y_i$ нигде не плотны.
Рассмотрим множество $Y_1$, тогда найдётся замкнутый шар $B_1$, для которого $B_1 \cap Y_1 = \es$.
Рассмотрим множество $Y_2$ и возьмём $B_2 \subs B_1$ так, чтобы $B_2 \cap Y_2 = \es$.
Продолжим этот процесс, получим последовательность замкнутых шаров $\hc{B_i}$. По теореме о вложенных шарах
найдётся $x \in \bigcap B_i$, но это означает, что $x$ не лежит ни в одном из $Y_i$.
\end{proof}

\subsubsection{Компактные метрические пространства}

\begin{df}
Пусть $X$\т метрическое пространство. Множество $M \subs X$ называется \emph{компактным},
если из любой последовательности $\hc{x_i} \subs M$ можно выделить подпоследовательность,
сходящуюся к $x \in M$.
\end{df}

\begin{df}
Множество $M$ называется \emph{предкомпактным}, если
из любой последовательности $\hc{x_i} \subs M$ можно выделить фундаментальную
подпоследовательность.
\end{df}

\begin{df}
Говорят, что множество $N$ образует $\ep$\д \emph{сеть} для множества $M$,
если в $\ep$\д окрестности любой точки $x \in M$ найдётся точка из $N$.
\end{df}

\begin{note}
Иногда требуют, чтобы множество $N$ содержалось в самом множестве $M$,
но, как несложно показать, эти определения эквивалентны.
\end{note}

\begin{df}
Множество $M$ называется \emph{вполне ограниченным}, если для всякого $\ep > 0$
существует конечная $\ep$\д сеть для $M$.
\end{df}

\begin{theorem}[Критерий Хаусдорфа]
Бесконечное подмножество $M$ в метрическом пространстве предкомпактно тогда и только тогда,
когда для $\fa \ep > 0$ существует конечная $\ep$\д сеть для $M$.
\end{theorem}
\begin{proof}
Пусть нашлось такое $\ep_0 > 0$, что для него не существует конечной $\ep_0$\д сети.
Иначе говоря, всякое конечное семейство окрестностей радиуса $\ep_0$ не может покрыть
всё множество $M$. Возьмём $x_1 \in M$ и накроем его $\ep_0$\д окрестностью~$U_1$. Набор
$\hc{U_1}$ не покрывает~$M$, поэтому найдётся $x_2 \in M \wo U_1$.
Накроем его окрестностью~$U_2$, но $\hc{U_1,U_2}$ снова не покроет всё множество~$M$.
Выбирая $x_3 \in M \wo (U_1 \cup U_2)$ и так далее, получим последовательность, у которой
$\rho(x_i, x_j) \ge \ep_0$, поэтому из неё нельзя выделить фундаментальную.
Таким образом, $M$ не предкомпактно.

Обратно, пусть для $\fa \ep > 0$ существует конечная $\ep$\д сеть. Пусть $\hc{x_i} \subs M$\т произвольная
последовательность, выделим из неё фундаментальную.
Возьмём $1$\д сеть, тогда найдётся окрестность, в которой бесконечно много членов последовательности.
Выберем оттуда один элемент $x_1^*$ и в качестве новой последовательности возьмём только то, что
попало в эту окрестность. Далее, существует конечная $\frac12$\д сеть, покрывающая новую
последовательность. Снова выберем ту окрестность сети, в которой бесконечно много элементов,
и в ней возьмём произвольный $x_2^*$.
Продолжим этот процесс, то есть на $n$\д м шаге будем выбирать $\frac1{2^n}$\д сеть.
Ясно, что последовательность $\hc{x_i^*}$ будет фундаментальна.
\end{proof}

\begin{imp}
Любое компактное метрическое пространство сепарабельно.
\end{imp}
\begin{proof}
Пусть $X$\т компактное метрическое пространство.
Покажем, что $X$ полно. В самом деле, если $\hc{x_n}$\т фундаментальная последовательность,
то по определению компактности из неё можно выделить подпоследовательность, сходящуюся к
некоторому элементу $x \in X$. Но из этого, очевидно, следует, что к~$x$ сходится
и вся последовательность. Значит, $X$ полно.

Построим в $X$ счётное всюду плотное множество~$D$.
По критерию Хаусдорфа, для $\fa \ep$ в $X$ существует конечная $\ep$\д сеть
$C_\ep := \bc{x^\ep_1\sco x^\ep_{k_\ep}}$.
Рассмотрим
$$D = \cupnui C_{\frac1n}.$$
Очевидно, что $D$ счётно и всюду плотно в $X$.
\end{proof}

\section{Нормированные и банаховы пространства}

\subsection{Линейные функционалы и операторы}

\subsubsection{Основные понятия}

Пусть $X$\т линейное пространство над полем $\Cbb$.

\begin{df}
\emph{Норма}\т функция $\hn{\cdot}\cln X\ra \R_+$ со свойствами:
\begin{points}{-3}
\item Для $\fa x, y \in X$ имеем $\hn{x+y} \le \hn{x} + \hn{y}$\т неравенство треугольника.
\item Для $\fa x \in X$, $\fa \al \in \Cbb$ имеем $\hn{\al x} = |\al|\cdot\hn{x}$\т однородность.
\item Для $\fa x \in X$ из $\hn{x} = 0$ следует $x = 0$\т точность.
\end{points}
\end{df}

\begin{df}
Линейный оператор $A\cln X \ra X$ называется \emph{ограниченным}, если
$\exi C > 0\cln \hn{Ax} \le C\hn{x}$ для всех $x \in X$.
\end{df}

\begin{df}
\emph{Норма} линейного оператора\т число $\hn{A} := \supl{x\neq 0} \frac{\hn{Ax}}{\hn{x}}$.
\end{df}

\begin{problem}
Доказать, что $\hn{A}$ совпадает с числом $\inf C$, где $C$\т константа из
определения ограниченного оператора. Доказать, что норму можно определять и так:
$\hn{A} = \supl{\hn x =1} \hn{Ax}$.
\end{problem}

\begin{problem}
Пусть $A, B$\т ограниченные операторы. Доказать, что $\hn{AB} \le \hn{A} \cdot \hn{B}$.
\end{problem}
\begin{solution}
Имеем $\hn{AB} = \supl{x \neq 0} \frac{\hn{ABx}}{\hn{x}} \le
\supl{x \neq 0} \frac{\hn{A}\cdot \hn{Bx}}{\hn{x}} = \hn{A}\cdot \hn{B}$.
\end{solution}

\begin{problem}
Доказать, что ограниченность оператора равносильна его непрерывности.
\end{problem}

Буквой $I$ мы будем обозначать тождественный оператор $I\cln X \ra X$.

Очевидно, что вектор $x\neq 0$ является собственным с собственным значением $\la$ тогда и только тогда,
когда $x \in \Ker(A - \la I)$.

\begin{df}
\emph{Обратным} к ограниченному оператору $A$ называется такой
ограниченный оператор $B$, что $AB = BA = I$.
\end{df}

\subsubsection{Спектр оператора}

\begin{df}
Пусть $A$\т ограниченный оператор. Рассмотрим оператор $A - \la I$.
\emph{Спектром} оператора называется множество точек $\la \in \Cbb$, для которых не существует
ограниченного обратного оператора к $A - \la I$. Мы будем обозначать спектр оператора~$A$ через $\Sig(A)$.
\end{df}

\begin{df}
Точки, лежащие в дополнении к спектру, называются \emph{регулярными}.
\end{df}

\begin{df}
Пусть $A\cln X \ra X$\т ограниченный оператор.
\emph{Резольвентой} оператора называется функция
$$\Rc_A \cln \Cbb \wo \Sig(A) \ra \End X, \quad \Rc_A(z) := (A - z I)^{-1}.$$
\end{df}

\subsubsection{Непустота спектра ограниченного оператора}

\begin{lemma}[Тождество Гильберта]
Для резольвенты оператора~$A$ имеет место формула
$$\Rc(z) - \Rc(w) = (z-w)\Rc(z)\Rc(w).$$
\end{lemma}
\begin{proof}
Рассмотрим тождество $(A - w I) - (A - z I) = (z-w)I$. Домножим слева на оператор $\Rc(z)$, а справа на $\Rc(w)$, получим
$$\Rc(z)(A - wI)\Rc(w) - \Rc(z)(A - zI)\Rc(w) = \Rc(z)(z-w)\Rc(w).$$
После сокращения прямых и обратных операторов получим
$\Rc(z) - \Rc(w) = (z-w)\Rc(z)\Rc(w)$.
\end{proof}

\begin{lemma}
Резольвента является дифференцируемой операторнозначной функцией.
\end{lemma}
\begin{proof}
Используя определение производной и тождество Гильберта, получаем
$$\Rc'(z) := \liml{h \ra 0}\frac{\Rc(z+h) - R(z)}{h} = \liml{h\ra 0} \frac{(z+h-z)\Rc(z+h)\Rc(z)}{h} = \Rc^2(z).$$
\hfill\end{proof}

\begin{theorem}
Спектр ограниченного оператора непуст.
\end{theorem}
\begin{proof}
Допустим противное, тогда резольвента определена для любого $z \in \Cbb$.
Заметим, что $\Rc(z) \ra 0$ при $z \ra \infty$. В самом деле,
$$(A - zI)^{-1} = \hr{-z\hr{I - \frac{A}{z}}}^{-1} = -\frac{1}{z}\hr{I - \frac{A}{z}}^{-1} \ra 0,$$
ибо второй множитель ограничен, а первый стремится к $0$.

Рассмотрим какой\д нибудь функционал $\ph \in X^*$. Рассмотрим
функцию $f(z) := \ph\br{\Rc(z)x}$. Покажем, что $f$\т целая функция.
Продифференцируем её:
$$f'(z) = \liml{h\ra0}\frac{f(z+h) - f(z)}{h} = \liml{h\ra0}\frac{\ph\br{\Rc(z+h)x} - \ph\br{\Rc(z)x}}{h} = \liml{h\ra0}\ph\hr{\frac{\Rc(z+h)-\Rc(z)}{h}x} = \ph\br{\Rc^2(z)x}.$$
По доказанному выше, $f(z) \ra 0$ при $z \ra \infty$. Таким образом, $f$ определена всюду
и ограничена. По теореме Лиувилля $f \equiv \const$, но так как
$f(\infty)=0$, то $f \equiv 0$. Следовательно, для всякого $x$ и произвольного функционала~$\ph$
имеем $\ph\br{\Rc^2(z)x}=0$. По следствию из теоремы Хана\ч Банаха, $\Rc^2(z)x = 0$ для всякого~$x$.
Но это означает, что резольвента $\Rc(z)$ является тождественно нулевым оператором при
всех $z$, что невозможно.
\end{proof}


\subsubsection{Теорема Хана\ч Банаха}

\begin{df}
Пусть $(P, \prec)$\т частично упорядоченное множество. \emph{Цепью} называется
произвольное  подмножество в~$P$, в котором любые два элемента сравнимы.
Элемент $p \in P$ называется \emph{максимальным}, если из $p \prec q$ следует, что $p=q$.
Элемент $p\in P$ для цепи $S$ называется \emph{верхней гранью}, если для $\fa q \in S$ имеем $q \prec p$.
\end{df}

\begin{stm}[Лемма Цорна]
Пусть $(P, \prec)$\т частично упорядоченное множество. Если для любой цепи
подмножества $P$ существует верхняя грань, то существует максимальный элемент в $P$.
\end{stm}

\begin{theorem}[Хана\ч Банаха о продолжении функционалов]
Пусть $X$\т нормированное пространство. Пусть $L$\т подпространство в $X$, а $f$\т ограниченный
вещественный функционал на $L$. Тогда существует функционал $\ph\cln X \ra \R$ такой, что
$\hn{\ph} = \hn{f}$ и $\ph\evn{L} = f$.
\end{theorem}
\begin{proof}
Вначале покажем, что $f$ можно продолжить указанным образом на подпространство~${M := L \oplus \ha{x_0}}$,
где $x_0 \notin L$. Всякий вектор $x \in M$ однозначно представляется в виде $x = v + t x_0$, где $v \in L$.

Пусть $x,y\in L$, тогда
$$f(x) - f(y) = f(x-y) \le \hn{f}\cdot\hn{x-y} \le \hn{f}\cdot\hn{x+x_0} + \hn{f}\cdot\hn{y+x_0},$$
поэтому
$$f(x) - \hn{f}\cdot\hn{x+x_0} \le f(y)+\hn{f}\cdot\hn{y+x_0}.$$
Перейдём слева к верхней грани по $x \in L$, а справа к нижней грани по $y \in L$. Получим
$$S := \supl{x\in L}\br{f(x) - \hn{f}\cdot\hn{x+x_0}} \le \infl{y\in L}\br{f(y)+\hn{f}\cdot\hn{y+x_0}} =: I.$$
Возьмём число $c \in [S, I]$. Рассмотрим функционал $\ph$ на $M$, заданный так:
$$\ph(x + t x_0) := f(x) - t c.$$
Он, очевидно, линеен и совпадает с $f$ на $L$. Докажем, что $\hn{\ph} = \hn{f}$.

Пусть $t > 0$. Тогда
$$\hm{\ph(x+t x_0)} = t\hm{f\hr{\frac xt} - c}.$$
Покажем, что $$\hm{f\hr{\frac xt} - c} \le \hn{f}\cdot\hn{\frac xt + x_0}.$$
В самом деле, $c \ge \supl{x\in L}\br{f(x) - \hn{f}\cdot\hn{x+x_0}}$, значит,
в частности, $c \ge f\hr{\frac xt} - \hn{f}\cdot\hn{\frac xt+x_0}$. Аналогично,
$c \le \infl{x\in L}\br{f(x) + \hn{f}\cdot\hn{x+x_0}}$, значит,
в частности, $c \le f\hr{\frac xt} + \hn{f}\cdot\hn{\frac xt+x_0}$. Следовательно,
оценка верна. Поэтому
$$\hm{\ph(x+t x_0)} = t\hm{f\hr{\frac xt} - c} \le t \hn{f} \cdot\hn{\frac xt + x_0} = \hn{f}\cdot\hn{x + t x_0}.$$
Аналогичная оценка получается для $t < 0$. Таким образом, $\hn{\ph} \le \hn{f}$, но при продолжении
норма не может уменьшиться. Итак, $\hn{f} = \hn{\ph}$.

Для сепарабельных пространств дальнейшие рассуждения очевидны. Покажем, как действовать в случае,
когда сепарабельности нет. Рассмотрим всевозможные продолжения $f$ и введём на них частичный порядок:
будем считать, что $f_1 \prec f_2$, если $\Dom  f_1 \subs \Dom f_2$ и $f_1 = f_2$ на $\Dom f_1$.
Пусть $\hc{f_\al}$\т произвольная цепь. Обозначим $L_\al := \Dom f_\al$ и покажем, что её
верхней гранью является функционал~$\wh f$, определённый на $\cups{\al} L_\al$, причём $\wh f(x) = f_\al(x)$,
если $x \in L_\al$. Действительно, очевидно, что $\wh f$ линеен и $\|\wh f\| = \hn{f}$. По лемме Цорна
множество продолжений имеет максимальный элемент. Он определён на всём $X$, в противном случае его
можно было бы продолжить.
\end{proof}



\subsubsection{Лемма Рисса о почти перпендикуляре}

\begin{lemma}[Рисса о почти перпендикуляре]
Пусть $X$\т нормированное пространство, а $Y \subsetneq X$\т замкнутое подпространство.
Тогда для всякого $\ep > 0$ существует <<почти перпендикуляр>> $x \in X$ такой,
что $\hn x = 1$, а $\rho(x,Y) > 1-\ep$.
\end{lemma}
\begin{proof}
Поскольку $Y \neq X$ и замкнуто, найдётся $z \neq 0$, для которого $\rho(z, Y) = a > 0$.
Тогда найдётся последовательность $y_i \in Y$, для которых имеем $\rho(z,y_i) = \hn{z-y_i} \ra \rho(z,Y)$.
Имеем $a = \rho(z,Y) \stackrel{!}{=} \rho(z-y_i, Y)$. В пояснении нуждается только переход,
отмеченный знаком <<!>>, и следует он из того, что всякое линейное пространство инвариантно относительно
сдвигов на \emph{свои} векторы. По определению расстояния, найдётся $i$, для которого $\hn{z - y_i} \le \frac{a}{1-\ep}$.
Тогда
$$\rho\hr{\frac{z-y_i}{\hn{z-y_i}}, Y} = \frac{1}{\hn{z-y_i}}\cdot a \ge \frac{1-\ep}{a}\cdot a = 1-\ep.$$
Таким образом, вектор $x = \frac{z-y_i}{\hn{z-y_i}}$\т искомый.
\end{proof}


\subsubsection{Лемма о продолжении функционала}

\begin{lemma}\label{lem:func.extension}
Пусть $Y \subsetneq  X$\т замкнутое подпространство, и пусть $x \notin Y$. Тогда существует
ограниченный функционал $f$ такой, что $f(x) = 1$ и $f(Y) = 0$.
\end{lemma}
\begin{proof}
В самом деле, на векторах из $\ha{x, Y}$ положим $f(\la x + y) = \la$. Далее этот функционал можно продолжить
на всё пространство с сохранением нормы по теореме Хана\ч Банаха. Осталось понять, почему этот функционал
ограничен на $\ha{x + Y}$. Действительно,
$$\hm{f(\la x + y)} = \hm{\la} = \frac{\hm\la \cdot \hn{\la x + y}}{\hn{\la x + y}} =
\frac{\hn{\la x + y}}{\hn{x + \frac{y}{\la}}} \le \frac{1}{\rho}\hn{\la x + y},$$
где $\rho = \rho(x, Y) > 0$. Таким образом, норма нашего функционала ограничена числом $\frac1\rho$.
\end{proof}

\subsubsection{Критерий конечномерности пространства}

\begin{lemma}\label{lem:bounded.eq.compact.dim.finite}
Нормированное пространство $X$ конечномерно тогда и только тогда,
когда в нём всякое бесконечное ограниченное множество предкомпактно.
\end{lemma}
\begin{proof}
Всякое бесконечное ограниченное множество в конечномерном пространстве
предкомпактно, поскольку в этом случае $X \cong \Cbb^n$ (или $\R^n$),
а для этих пространств предкомпактность эквивалентна ограниченности.

Обратно, пусть всякое ограниченное подмножество в $L$ предкомпактно.
Допустим, что $X$ бесконечномерно, тогда возьмём единичный вектор $e_1 \in X$.
По предположению, $X \neq X_1 := \ha{e_1}$, тогда по лемме Рисса найдётся единичный
вектор $e_2 \notin X_1$, для которого $\rho(e_2, X_1) \ge \frac12$.
Вновь по предположению $X \neq X_2 := \ha{e_1,e_2}$, тогда построим
ещё один вектор $e_3$, для которого $\rho(e_3, X_2) \ge \frac12$, и так далее.
Цепочка подпространств $X_n$ будет строго возрастать, и последовательность $\hc{e_i}$
будет ограниченным и не предкомпактным множеством, так как расстояние между любыми двумя
её элементами не меньше $\frac12$.
\end{proof}


\subsubsection{Теорема Банаха\ч Штейнгауза}

\begin{lemma}
Если замкнутое множество не содержит ни одного шара положительного радиуса, то оно нигде не плотно.
\end{lemma}
\begin{proof}
Если $M$ не является нигде не плотным, то найдётся шар $B$ положительного радиуса
такой, что для всякого шара $B' \subs B$ имеем $M \cap B' \neq \es$. Это означает,
что $M$ всюду плотно в $B$, но тогда $B \subs M$,
ибо $M$ замкнуто (оно содержит все свои предельные точки).
\end{proof}

\begin{theorem}[Принцип равномерной ограниченности Банаха\ч Штейнгауза]
Пусть $X$\т банахово, а $Y$\т нормированное пространство.
Пусть $A_i\cln X \ra Y$\т семейство ограниченных операторов.
Пусть для всякого $x \in X$ существует число $C_x > 0$ такое, что для $\fa i$ имеем
$\hn{A_i x} \le C_x$. Тогда найдётся такое $C > 0$, что $\hn{A_i} \le C$ для всех $i$.
\end{theorem}
\begin{proof}
Рассмотрим семейство множеств
$$X_n := \hc{x \in X \cln \fa i \text{ имеем} \hn{A_i x} \le n}.$$
Очевидно, что $X = \bigcup X_n$. Поскольку $X$ не есть множество первой категории,
найдётся $X_N$ такое, что оно не является нигде не плотным в $X$. Значит, есть шар,
где оно всюду плотно.

Покажем, что все множества $X_n$ замкнуты. Для этого докажем, что дополнения к ним открыты.
Пусть $x \notin X_n$. Значит, $\exi k$, для которого $\hn{A_k x} \ge n+ 2\ep$.
Пусть $v \in X$. Если $\hn{v}\le \frac{\hn{A_kx} - (n+\ep)}{\hn{A_k}}$, то
\eqn{\hn{A_k(x+v)} = \hn{A_k x + A_k v} \ge \hn{A_kx} - \frac{\hn{A_k}\br{\hn{A_k x} - (n+\ep)}}{\hn{A_k}} = n + \ep > n,}
то есть $(x+v)\notin X_n$.

По предыдущей лемме, множество $X_N$ содержит некоторый шар $B$.
Достаточно установить равномерную ограниченность операторов на некотором шаре,
содержащем начало координат. Пусть $\wt B$\т копия шара $B$ с центром в начале координат.
Каждый вектор $v \in \wt B$ можно представить как $w_1 - w_2$, где $w_i \in B$.
По неравенству треугольника и определению множества $X_N$
для всех~$i$ получаем $\hn{A_i v} = \hn{A_i w_1 - A_i w_2} \le N + N = 2N$.
Но это и означает равномерную ограниченность.
\end{proof}

\begin{note}
В этой теореме множество операторов может иметь произвольную мощность.
\end{note}

\subsubsection{Пространство ограниченных операторов}

Пусть $X$ и $Y$\т нормированные пространства. Обозначим через $\Ls(X,Y)$ множество всех линейных отображений
$A\cln X\ra Y$. Это, очевидно, линейное пространство. В нём можно выделить подпространство ограниченных
линейных операторов $\Bs(X,Y)$. Если $X=Y$, то это пространство превращается в алгебру.

\begin{df}
Говорят, что $A_n \convs A$ в $\Bs(X,Y)$ (<<сильно>> сходится), если
для $\fa x \in X$ имеем $A_n x\ra Ax$ по норме пространства~$Y$.
\end{df}

\begin{stm}
Если пространства $X$ и $Y$ банаховы, то пространство $\Bs(X,Y)$ полно
относительно сильной сходимости, то есть сильный предел ограниченных
операторов также является ограниченным оператором.
\end{stm}
\begin{proof}
Пусть для всякого~${x \in X}$ последовательность $\hc{A_nx}$ фундаментальна. Покажем, что
существует ограниченный оператор $A$ такой, что $A_n \convs A$.
В силу фундаментальности, для $\fa x$ последовательность $\hc{A_n x}$ ограничена.
Из теоремы Банаха\ч Штейнгауза
следует, что $\hn{A_n} \le K$, то есть последовательность операторов ограничена по норме.
В силу полноты пространства $Y$, последовательность $A_n x$ сходится к некоторому вектору,
который мы обозначим $Ax$. Отсюда, в частности, следует, что $\hn{A_n x} \ra \hn{Ax}$.
Из равномерной ограниченности следует, что $\hn{A_n x} \le K\hn{x}$. Осталось перейти в
этом неравенстве к пределу при $n\ra\infty$, и мы получим,
что $\hn{Ax} \le K\hn{x}$, то есть норма предельного оператора тоже не превосходит~$K$.
\end{proof}

\begin{stm}
Если пространство $Y$ банахово, то пространство $\Bs(X,Y)$ полно относительно операторной нормы.
\end{stm}
\begin{proof}
Пусть $\hc{A_n} \subs \Bs(X,Y)$\т фундаментальная последовательность. Тогда
$$\hn{A_nx - A_{n+p}x} \le \hn{A_n - A_{n+p}}\cdot \hn{x} \ra 0,$$
поскольку $\hn{A_n - A_{n+p}} \ra 0$.
Отсюда, в силу банаховости~$Y$, последовательность $A_nx$ сходится к некоторому вектору,
который мы обозначим $Ax$. Так как последовательность норм операторов фундаментальна,
она ограничена, то есть $\hn{A_n} \le K$. Отсюда $\hn{A_n x} \le K\hn{x}$, и после
перехода к пределу получаем $\hn{Ax} \le K\hn{x}$.

Покажем, что $\hn{A_n - A} \ra 0$. Для этого достаточно показать, что для $\fa x$ такого, что $\hn{x} \le 1$,
выполняется неравенство $\hn{A_n x - Ax} \le \ep\hn{x}$.
В самом деле, в силу фундаментальности для $\fa \ep > 0$ найдётся $N$ такое, что для
$\fa n \ge N$ и для $\fa p$ выполнено $\hn{A_n x- A_{n+p}x} \le \ep\hn{x}$. Остаётся
перейти к пределу при $p\ra\bes$.
\end{proof}

\begin{stm}[О продолжении оператора по непрерывности]
Пусть $X_0 \subs X$\т всюду плотное подпространство в банаховом пространстве $X$.
Пусть $A_0\cln X_0 \ra X$\т ограниченный линейный оператор.
Тогда существует ограниченное продолжение $A\cln X\ra X$ оператора $A_0$ с сохранением нормы.
\end{stm}
\begin{proof}
Возьмём последовательность $\hc{\xi_n} \subs X_0$, которая сходится к вектору $x$.
Рассмотрим образ этой последовательности под действием оператора $A_0$.
Положим $Ax := \lim A_0 \xi_n$. Этот предел существует, так как
$\hn{A_0 \xi_n - A_0\xi_m} \le \hn{A_0}\cdot\hn{\xi_n-\xi_m} \ra 0$.

Покажем, что такое определение корректно, то есть не зависит от выбора последовательности,
приближающей $x$. Пусть $\xi_n \ra x \ar \eta_n$.
Рассмотрим третью последовательность $\hc{\ze_n} := \xi_1,\eta_1,\xi_2,\eta_2\etc$.
Она тоже сходится к $x$, и $\lim A_0 \ze_n$ тоже существует. Осталось заметить, что
$\lim A_0 \xi_n$ и $\lim A_0 \eta_n$\т это частичные пределы сходящейся последовательности,
значит, они совпадают.

Получилось отображение $A\cln X \ra X$, а так как
$\hn{A_0 \xi_n} \le \hn{A_0} \cdot \hn{\xi_n}$, то, переходя к пределу, получаем, что
$\hn{Ax} \le \hn{A_0} \cdot \hn{x}$. Значит, норма продолженного оператора не увеличилась.
С другой стороны, ясно, что она не могла уменьшиться.
\end{proof}

\subsubsection{Теорема Банаха об обратном операторе}

\begin{lemma}
Пусть $A\cln X \ra Y$\т линейная биекция банаховых пространств.
Положим
$$Y_k := \hc{y \in Y\cln \hn{A^{-1}y} \le k\hn{y}}.$$
Тогда существует такое $Y_N$, что $\Cl Y_N = Y$.
\end{lemma}
\begin{proof}
Поскольку $Y$\т полное пространство, по теореме Бэра существует~$Y_M$, плотное в некотором шаре~$B$.
Обозначим через $P$ пересечение некоторого шарового слоя с центром в точке $y_0 \in Y_M$, целиком лежащего
в шаре $B$, с множеством $Y_M$. Рассмотрим копию $\wt P$ множества~$P$, сдвинутую в начало координат.
Всякий вектор $v \in \wt P$ представляется в виде разности $y - y_0$, где $y \in P$.
Имеем
\begin{multline*}
\hn{A^{-1}v} = \hn{A^{-1}(y-y_0)} \le \hn{A^{-1}y} + \hn{A^{-1}y_0} \le M\br{\hn y + \hn{y_0}} =\\=
M\br{\hn{y - y_0 + y_0} + \hn{y_0} } \le M\br{\hn{y -y_0} + 2\hn{y_0}} = M\hn{y-y_0}\hr{1 + \frac{2\hn{y_0}}{\hn{y-y_0}}}.
\end{multline*}
Заметим, что последний множитель может быть ограничен сверху некоторой константой $C$,
не зависящей ни от чего, поскольку число $\hn{y-y_0}$ отделено от нуля. Беря
в качестве $N := \hs{CM} + 1$, получаем, что $Y_N$ плотно в~$\wt P$. Но поскольку
в силу своего определения множество $Y_N$ инвариантно относительно
гомотетий, оно будет плотно и во всём пространстве.
\end{proof}

\begin{theorem}[Банаха об обратном операторе]
Пусть $A\cln X \ra Y$\т линейная биекция банаховых пространств.
Тогда обратное отображение $A^{-1}\cln Y \ra X$ тоже будет ограниченным оператором.
\end{theorem}
\begin{proof}
Линейность обратного отображения очевидна. Докажем ограниченность.
Рассмотрим ненулевой вектор $y \in Y$. По предыдущей лемме существует всюду плотное в $Y$ множество $Y_N$.
Тогда существует $y_1 \in Y_N$, для которого $\hn{y- y_1} \le \frac{\hn{y}}{2}$, причём $\hn{y_1} \le \hn{y}$.
Далее, существует $y_2 \in Y_N$, для которого $\hn{y-(y_1+y_2)}\le\frac{\hn y}{2^2}$, причём
$\hn{y_2} \le\frac{\hn y}{2}$, и так далее.
На $n$\д м шаге существует $y_n \in Y_N$, для которого
$\hn{y-(y_1+y_2\spl y_n)}\le\frac{\hn y}{2^n}$, причём
$\hn{y_n} \le\frac{\hn y}{2^{n-1}}$.

Рассмотрим $x_n := A^{-1}y_n$. По определению $Y_N$ имеем
$\hn{x_n} \le N\hn{y_n} \le N\frac{\hn{y}}{2^{n-1}}$.
Значит, в силу полноты пространства $X$ и сходимости ряда $\sum \hn{x_n}$ существует предел
$$x := \liml{p\ra\infty}\suml{n=1}{p} x_n.$$
Тогда
$$
  Ax = A\bbr{\liml{p\ra\infty}\suml{n=1}{p}x_n} = \liml{p\ra\infty} \suml{n=1}{p}Ax_n=
  \liml{p\ra\infty}\suml{n=1}{p}y_n=y.
$$
Отсюда $A^{-1}y=x$, поэтому
\begin{multline*}
  \hn{A^{-1}y}=\hn{x} = \Bn{\sumnui x_n} = \liml{p\ra\infty}\Bn{\suml{n=1}{p}x_n} =
  \liml{p\ra\infty}\Bn{\suml{n=1}{p}A^{-1}y_n} \le \\ \le\sumnui\hn{A^{-1}y_n} \le
  \sumnui N\hn{y_n} \le \sumnui N \frac{\hn y}{2^{n-1}} = 2N\hn{y}.
\end{multline*}
Следовательно, оператор $A^{-1}$ ограничен.
\end{proof}

\subsubsection{Устойчивость обратимости оператора при малых возмущениях}

\begin{lemma}
Если $A\cln X \ra X$\т оператор в банаховом пространстве такой, что $\hn{A} < 1$, то оператор $I - A$ обратим.
\end{lemma}
\begin{proof}
Покажем, что оператор
$$P := \sumizi A^i$$
является обратным к оператору $I - A$. Операторный ряд следует понимать как предел частичных сумм.
Покажем, что он сходится, то есть для каждого вектора $x \in X$
последовательность частичных сумм
$$
  S_n x := \sumizn A^ix
$$
фундаментальна.
В самом деле, если $m>n$, то
$$
  \hn{S_m x - S_n x} = \hn{A^{n+1}x\spl A^m x} \le \hn{A^{n+1}x}\spl \hn{A^m x}\le
  \hn{x} \hr{\hn{A}^{n+1}\spl \hn{A}^m},
$$
поэтому, если взять $n$ достаточно большим, эту сумму можно сделать сколь угодно маленькой
как хвост сходящегося ряда $\sum\hn{A}^i$.
В силу полноты пространства, эта последовательность сходится. Очевидно, что
$$\hn{Px} \le \hn{x} \sumizi \hn{A}^i,$$
поэтому оператор $P$ ограничен.
Из определения $P$ выводим, что
$$P(I-A)x = \liml{n \ra \infty} \sumizn A^i(I-A)x = \liml{n\ra\infty}\sumizn\hr{A^ix - A^{i+1}x}
= \liml{n\ra\infty} \hr{x-A^{n+1}x} = x,$$
так как $\hn{A}<1$ и второе слагаемое в пределе даёт нуль. Тем самым доказано, что
$P$ является левым обратным. Покажем, что он и правый обратный.
В самом деле, оператор $I-A$ ограничен и, очевидно, перестановочен с операторами $S_n$.
Поэтому
$$(I - A)P = (I-A)\lim S_n = \lim (I-A)S_n = \lim S_n(I-A) = P(I-A).$$
Таким образом, оператор $I-A$ обратим.
\end{proof}

\begin{theorem}[Устойчивость обратимости при малых возмущениях]
\label{th:resolveness.under.small.pert}
Пусть $A\cln X \ra X$\т ограниченный обратимый оператор в банаховом пространстве. Тогда
для всякого оператора $B$ с нормой $\hn{B} < \frac{1}{\hn{A^{-1}}}$ оператор $A + B$ обратим.
\end{theorem}
\begin{proof}
Ясно, что $A+B$ обратим тогда и только тогда, когда обратим оператор $A^{-1}(A+B) = I + A^{-1}B$.
Поскольку оператор $A$ ограничен, по теореме Банаха оператор $A^{-1}$ тоже ограничен.
Так как $\hn{A^{-1}B} \le \hn{A^{-1}} \cdot \hn{B} < 1$ по условию, то в силу предыдущей леммы
оператор $I + A^{-1}B$ обратим.
\end{proof}

При доказательстве леммы фактически была доказана формула: если $\hn{A} < 1$, то
\eqn{\label{eqn:neyman.series}(I - A)^{-1} = \sumizi A^i.}

\begin{imp}
Резольвента является аналитической функцией в своей области определения.
\end{imp}
\begin{proof}
Сумма степенного ряда голоморфна в круге сходимости.
\end{proof}

Получим из формулы~\eqref{eqn:neyman.series} некоторую оценку для
числа $\hn{(A+B)^{-1}-A^{-1}}$ при условии $\hn{B} < \frac{1}{\hn{A^{-1}}}$.

Имеем $A + B = A(I + A^{-1}B)$, поэтому $(A+B)^{-1} = (I + A^{-1}B)^{-1}A^{-1}$.
По формуле~\eqref{eqn:neyman.series} получаем:
$$(A+B)^{-1} = \sumnzi (-1)^{n}(A^{-1}B)^nA^{-1}.$$
Отсюда
$$\hn{(A+B)^{-1} - A^{-1}} = \Bn{\sumnzi (-1)^n(A^{-1}B)^nA^{-1} + A^{-1}} \le \sumnui \hn{A^{-1}B}^n\hn{A^{-1}} =
\frac{\hn{A^{-1}B}}{1 - \hn{A^{-1}B}}\cdot\hn{A^{-1}}.$$


\subsubsection{Эквивалентность норм в конечномерных пространствах}
\begin{theorem}
В конечномерном пространстве все нормы эквивалентны.
\end{theorem}
\begin{proof}
Пусть $X$\т конечномерное нормированное пространство с нормами $\hn{\cdot}_1$ и $\hn{\cdot}_2$.
Для начала заметим, что оно изоморфно пространству $\Cbb^n$, поэтому можно все рассуждения проводить
для него. Покажем, что все нормы эквивалентны норме $\hn{\cdot}$, заданной как сумма модулей всех координат вектора.
Тогда про норму $\hn\cdot_2$ можно забыть. Пусть $e_1\sco e_n$\т базис в $X$, тогда,
полагая $C := \max \hn{e_i}_1$, для всякого $x \in X$ имеем
$$\hn{x}_1 = \hn{x_1e_1\spl x_n e_n}_1 \le \hm{x_1}\cdot \hn{e_1}_1\spl\hm{x_n}\cdot \hn{e_n}_1 \le C \hn{x}.$$
Тем самым оценка в одну сторону получена. Заметим, что в конечномерном пространстве единичная
сфера компактна. Функция $x \mapsto \hn{x}_1$ непрерывна относительно метрики, задаваемой
нормой $\hn\cdot$ в силу полученной выше оценки. На единичной сфере $\bc{x\cln \hn{x} = 1}$ она достигает своего
минимального значения, которое, очевидно, отлично от нуля. Этот минимум и есть нижняя оценка для отношения норм.
\end{proof}

\begin{problem}
Вывести отсюда, что всякое конечномерное подпространство замкнуто.
\end{problem}

\subsubsection{Отступление про неограниченные операторы}

Приведём пример неограниченного оператора с пустым спектром:
пусть $A\cln D(A) \ra \Cb[a,b]$, где $D(A) \subs \Cb[a,b]$\т область
определения оператора. Именно, рассмотрим оператор дифференцирования $A\cln f \mapsto f'$,
тогда $D(A) = \Cb^1[a,b]$, но мы будем рассматривать только функции, у которых $f(a) = 0$.

Выясним, когда оператор $A - \la$ обратим. Для этого рассмотрим задачу Коши:
$$
\case{f' - \la f = g,\\f(a) = 0.}
$$
Несложно видеть, что её (единственное по теореме существования и
единственности из курса дифференциальных уравнений)
решение выглядит так:
$$f(x) = e^{\la x}\intl{a}{x}e^{-\la y} g(y)\,dy.$$
Таким образом, оператор $A-\la$ обратим при всех $\la$, значит, спектр оператора $A$ пуст.

\subsubsection{О графиках операторов}
\comment{Часто встречающееся здесь обозначение $(a,b)$ для элемента декартового произведения множеств
$A \times B$, $a \in A$, $b \in B$ не следует путать с столь же часто встречающимся обозначением для
скалярного произведения.}

\begin{df}
\emph{Графиком} оператора $A\cln X \ra Y$, где $X, Y$\т нормированные пространства, называется множество
$\Graph A := \hc{(x,y) \vl y = Ax} \subs X \oplus Y$. Легко видеть, что график оператора\т это линейное
подпространство.
\end{df}

Для внешней прямой суммы пространств \emph{норму} вводят естественным образом: $\hn{(x,y)} := \hn{x}_X + \hn{y}_Y$.
Очевидно, что прямая сумма пространств будет банаховым пространством тогда и только тогда,
когда оба слагаемых банаховы.

\begin{df}
Если график оператора замкнут, то оператор называется \emph{замкнутым}.
\end{df}

\begin{stm}
Всюду определённый оператор $A\cln X \ra X$ в банаховом пространстве с замкнутым графиком ограничен.
\end{stm}
\begin{proof}
Поскольку замкнутое подмножество полного пространства полно, график оператора\т это
тоже полное пространство. В нашем случае $\Graph A  = \hc{(x, Ax) \vl x \in X}$.
Рассмотрим проекцию $\pi\cln \Graph A \ra X$ по правилу $\pi(x,Ax) = x$. Этот
оператор ограничен, ибо $\hn{\pi(x)} = \hn{x} \le \hn{x} + \hn{Ax}$,
поэтому $\hn{\pi} \le 1$.
Легко видеть, что $\pi$\т биекция, поэтому обратное отображение $\pi^{-1}\cln x \mapsto (x,Ax)$
ограничено в силу теоремы Банаха. Но это означает, что для некоторого $C> 0$ имеем
$\hn{(x,Ax)} = \hn{x} + \hn{Ax} \le C\cdot \hn{x}$, поэтому $A$ тоже ограничен.
\end{proof}

\subsection{Сопряжённые пространства и операторы}

\subsubsection{Определение сопряжённого оператора}

Пусть $X$ и $Y$\т линейные нормированные пространства.

\begin{df}
\emph{Сопряжённым} к пространству $X$ называется пространство всех
линейных ограниченных функционалов на $X$. Мы будем обозначать его символом $X'$.
\end{df}

Поскольку $X' = \Bs(X, \Cbb)$, автоматически получаем, что сопряжённое пространство всегда
полно.

Пусть теперь $A\cln X \ra Y$\т ограниченный оператор, а $X'$ и $Y'$\т соответствующие сопряжённые пространства.

\begin{df}
\emph{Сопряжённым} к оператору~$A$ называется оператор $A'\cln Y' \ra X'$, который функционалу
$g \in Y'$ ставит в соответствие функционал $f$ по правилу $f(x) := g(Ax)$.
\end{df}

\begin{stm}
$\hn{A'} = \hn{A}$.
\end{stm}
С одной стороны, имеем
$$\hn{A'g} = \supl{\hn{x} \le 1} \hm{g(Ax)} \le \hn{g}\cdot \hn A,$$
таким образом, $\hn{A'} \le \hn{A}$. Чтобы доказать обратное неравенство,
рассмотрим $y = \frac{Ax}{\hn{Ax}}$ и функционал $g \in Y'$ такой, что
$\hn{g} = 1$ и $g(y) = 1$. Тогда
$$\hn{Ax} = g(Ax) = (A'g)(x)  \le \hn{x} \cdot \hn{g} \cdot \hn{A'} = \hn{x} \cdot \hn{A'},$$
значит, $\frac{\hn{Ax}}{\hn x} \le \hn{A'}$,
откуда следует обратное неравенство.

\subsubsection{Компактность оператора, сопряжённого к компактному}

Пусть $X$\т компактное подмножество метрического пространства.

\begin{df}
Семейство $\Phi$ функций $\ph$ на $X$ называется \emph{равномерно ограниченным},
если $\hm{\ph(x)} \le C$ для всех $x \in X$ и всех $\ph \in \Phi$.
\end{df}

\begin{df}
Семейство $\Phi$ функций $\ph$ на $X$ называется \emph{равностепенно непрерывным},
если для всякого $\ep > 0$ найдётся $\de > 0$, для которого
$\hm{\ph(x) - \ph(y)} \le \ep$ для всех $x, y \in X$ таких, что $\rho(x,y) < \de$ и для
всех $\ph \in \Phi$. Проще говоря, это та же равномерная непрерывность,
но число $\de$ универсально для всего семейства функций.
\end{df}

\begin{theorem}[Арцела\ч Асколи]
Множество $M \subs \Cb[a,b]$ предкомпактно тогда и только тогда, когда
оно равномерно ограничено и равностепенно непрерывно.
\end{theorem}
\begin{proof}
Изложено в~\cite[гл.~II, \S~7, п.~4]{kf}.
\end{proof}

\begin{theorem}[Обобщённая теорема Арцела\ч Асколи]
Пусть $X$ и $Y$\т компактные метрические
пространства, и $C := \Cb(X,Y)$\т пространство непрерывных
отображений из $X$ в $Y$ с чебышёвской метрикой. Тогда подмножество
$\Phi \subs C$ предкомпактно тогда и только тогда,
когда $\Phi$ равностепенно непрерывно.
\end{theorem}
\begin{proof}
Мы докажем только достаточность этого утверждения. Доказательство необходимости
ничем не отличается от доказательства необходимости в обычной теореме Арцела,
да и не потребуется нам в дальнейшем.

Пусть $F$\т пространство всех отображений $f\cln X \ra Y$, на котором введена метрика
$$\rho(f,g) := \supl{x \in X} \rho\br{f(x), g(x)}.$$
Равномерный предел непрерывных отображений на компакте непрерывен, поэтому $C$ замкнуто в $F$.
Следовательно, если $\Phi$ предкомпактно в~$F$, то оно предкомпактно и~в~$C$.

Возьмём $\ep > 0$ и по нему выберем $\de$, участвующее в определении
равностепенной непрерывности. Возьмём в $X$ $\frac{\de}{2}$\д сеть $x_1\sco x_n$
и рассмотрим шары $B_i := B\hr{x_i, \frac{\de}{2}}$. Их объединение покрывает $X$.
Получим из этого покрытия дизъюнктное покрытие. Таким будет, например, покрытие
$$E_i := B_i \wo \cups{j < i} B_j.$$
Заметим, что $\diam E_i < \de$.

Рассмотрим в компакте $Y$ некоторую $\ep$\д сеть $y_1\sco y_m$. Рассмотрим набор функций
$\hc{g}$, которые принимают на $E_i$ значения $y_j$. Такой набор, очевидно, конечен.
Покажем, что они образуют $2\ep$\д сеть для $\Phi$ в пространстве $F$. В самом деле,
пусть $f \in \Phi$. Очевидно, для всякой точки $x_i$ найдётся $y_{j(i)}$, для которой
$\rho(f(x_i), y_j) < \ep$. Возьмём в качестве $g$ функцию со значениями $y_{j(i)}$ на множествах $E_i$.
Тогда
$$\rho\br{f(x),g(x)} \le \rho\br{f(x),f(x_i)} + \rho\br{f(x_i),g(x_i)} + \rho\br{g(x_i),g(x)} < 2\ep,$$
что и требовалось доказать.
\end{proof}

\begin{theorem}
Оператор, сопряжённый компактному в банаховом пространстве, компактен.
\end{theorem}
\begin{proof}
Пусть $A\cln X\ra Y$\т компактный оператор. Пусть $B \subs X$\т единичный шар с центром в начале координат.
По определению компактного оператора, множество $A(B)$ предкомпактно. Докажем предкомпактность
множества $A'(B')$, где $B' := \hc{g \in Y'\cln \hn{g} \le 1}$\т единичный шар в $Y'$.

Рассмотрим семейство функционалов из $B'$ на множестве $A(B)$.
Пусть $g \in B'$, а $y = Ax \in A(B)$, тогда
$$\hm{g(y)} \le \hn{g}\cdot \hn{y} = \hn g \cdot \hn{Ax} \le \hn{g} \cdot \hn A \cdot \hn x \le \hn{A}.$$
Следовательно, функционалы из $B'$ равномерно ограничены на $A(B)$.

Пусть теперь $y_1, y_2 \in A(B)$, а $g \in B'$. Тогда
$$\hm{g(y_1) - g(y_2)} = \hm{g(y_1-y_2)} \le \hn{g}\cdot\hn{y_1-y_2} \le \hn{y_1-y_2},$$
а это означает равностепенную непрерывность семейства $B'$ на $A(B)$. В силу обобщения
теоремы Арцела множество $B'$ предкомпактно в смысле равномерной сходимости на $A(B)$.

Теперь рассматриваем произвольную последовательность $\hc{A'g_n} \subs A'(B')$.
Поскольку множество~$B'$ предкомпактно в смысле равномерной сходимости, из последовательности $\hc{g_n}$
можно выделить фундаментальную относительно равномерной сходимости на $A(B)$ подпоследовательность.
Иначе говоря,
$$\supl{y \in A(B)} \hm{g_{n_i}(y) - g_{n_j}(y)} = \supl{x \in B} \hm{g_{n_i}(Ax) - g_{n_j}(Ax)} \ra 0,
\quad n_i, n_j \ra \infty.$$
Но
$$\supl{x \in B} \hm{g_{n_i}(Ax) - g_{n_j}(Ax)}  = \supl{x \in B} \hm{A'(g_{n_i}-g_{n_j})(x)}  =
\hn{A'g_{n_i} - A'g_{n_j}}.$$
А это означает, что последовательность $\hc{A'g_{n_k}}$ фундаментальна в $X'$ и тем самым компактность
доказана.
\end{proof}

\subsection{Теория Фредгольма в банаховых пространствах}

\subsubsection{Вспомогательные леммы}

\begin{lemma}\label{lem:dim.ker.finite}
Пусть $A$\т компактный оператор. Тогда $K := \Ker(A - I)$ конечномерно.
\end{lemma}
\begin{proof}
Пусть $M \subs K$\т произвольное ограниченное множество.
Тогда для $\fa x \in M$ имеем $(A-I)x =0$, то есть $Ax = x$.
Значит, $A(M) = M$, но по определению компактного оператора,
множество $A(M)$ предкомпактно. Таким образом, каждое ограниченное подмножество
в ядре предкомпактно, поэтому по лемме~\ref{lem:bounded.eq.compact.dim.finite}
$K$~конечномерно.
\end{proof}

\begin{note}
В этой лемме вместо оператора $A-I$ можно с тем же успехом рассмотреть оператор $A - \la I$,
если $\la \neq 0$. Тогда мы получим, что компактный оператор может иметь лишь конечное число собственных векторов
с данным ненулевым собственным значением.
\end{note}


\begin{lemma}\label{lem:cl.image}
Пусть $A$\т компактный оператор. Тогда $\Img(A-I)$ замкнут.
\end{lemma}
\begin{proof}
Вначале покажем, что существует $\al > 0$, зависящее только от оператора $A$, такое,
что если $Ax - x = y$ разрешимо, то существует решение~$x$, для которого выполнено неравенство
$\hn x \le \al \hn y$.
Пусть $x_0$\т какое\д нибудь решение этого уравнения. Тогда общее решение имеет
вид $x_0 + z$, где $z \in K := \Ker(A-I)$. Рассмотрим функцию $\ph\cln K \ra \R_+$, заданную по правилу
$$\ph(z) := \hn{x_0 + z}.$$
Положим $d := \infl{z \in K} \ph(z)$, и пусть $\hc{z_n}$\т минимизирующая последовательность,
то есть $\ph(z_n) \ra d$.

Последовательность $\hc{\ph(z_n)}$ сходится, поэтому она ограничена. Следовательно,
ограничена и последовательность $\hc{\hn{z_n}}$, так как
$$\hn{z_n} \le \hn{x_0 + z_n} + \hn{x_0} = \ph(z_n) + \hn{x_0}.$$
В силу леммы~\ref{lem:dim.ker.finite} подпространство $K$ конечномерно, поэтому
из неё можно выделить сходящуюся. Выкинем из неё лишнее и перенумеруем, тогда
можно считать, что $z_n \ra z_0$. В силу замкнутости ядра, $z_0 \in K$.
Тогда в силу непрерывности $\ph(z_n) \ra \ph(z_0)$, и $\ph(z_0) = d$.
Значит, решение $x_0 + z_0$ обладает наименьшей нормой.

Теперь докажем существование $\al$. Через $\wt x(y)$ будем обозначать решение с минимальной нормой,
соответствующее правой части $y$. Рассмотрим отношение $\frac{\hn{\wt x(y)}}{\hn y}$ и допустим,
что оно не ограничено, тогда найдётся последовательность $\hc{y_n}$, для которой
$$\frac{\hn{\wt x(y_n)}}{\hn{y_n}} \ra \infty.$$
Для краткости положим $\wt x_n := \wt x(y_n)$.
В силу линейности, правой части $\mu y_n$ соответствует минимальное по норме решение $\mu \wt x_n$,
поэтому можно считать, что $\hn{\wt x_n} = 1$. Тогда $\hn{y_n} \ra 0$.
В силу компактности $A$ и ограниченности последовательности $\hc{\wt x_n}$,
последовательность $\hc{A\wt x_n}$ содержит сходящуюся подпоследовательность.
Снова выкинем лишнее, тогда можно считать, что $A\wt x_n \ra \wt x_0$,
но так как $A\wt x_n - \wt x_n = y_n$, а правая часть по предположению стремится к нулю,
то $A\wt x_n -\wt x_n \ra 0$, откуда $\wt x_n \ra \wt x_0$. Значит, $A\wt x_n \ra A\wt x_0$,
и $A \wt x_0 = \wt x_0$. Это означает, что  $\wt x_0 \in K$,
но в силу минимальности нормы решения $\wt x_n$ имеем
$\hn{\wt x_n - x_0} \ge \hn{\wt x_n} = 1$, а это противоречит сходимости $\wt x_n \ra \wt x_0$.

Итак, отношение $\frac{\hn{\wt x(y)}}{\hn y}$ ограничено, и осталось положить
$$\al := \sup \frac{\hn{\wt x(y)}}{\hn y}.$$

Теперь докажем замкнутость образа. Пусть $y_n \in \Img(A - I)$ и $y_n \ra y_0$. Без ограничения общности можно
считать, что $\hn{y_{n+1} - y_n} \le\frac1{2^n}$.
Запишем тождество
$$\ub{y_1} - \ub{y_1 + y_2} - \ub{y_2 + y_3} - y_3+\ldots -\ub{y_n+ y_{n+1}} = y_{n+1} \ra y_0.$$
Пусть $x_i$\т прообразы векторов, выделенных скобками, при отображении $(A-I)$, то есть $(A-I)x_1 = y_1$,
$(A-I)x_2 = -y_1 + y_2$, и так далее. Такие, конечно, найдутся, поскольку образ\т линейное подпространство,
а $y_i$ в нём лежат. Но мы специально будем выбирать только такие решения $x_i$, для которых имеет место
свойство
$$\hn{x_{n+1}} \le \al\hn{y_{n+1}-y_n} \le \frac{\al}{2^n}.$$
Значит, ряд $\sum x_i$ сходится. Образ суммы этого ряда накроет $y_0$ в силу непрерывности.
\end{proof}

\subsubsection{Теоремы Фредгольма}

Пусть $X$\т банахово пространство, $A\cln X \ra X$\т компактный оператор. Рассмотрим уравнения
$$\case{Ax - x = y,\\Ax-x = 0;} \quad \case{A'f - f = g,\\A'f - f = 0.}$$

\begin{theorem}[Третья теорема Фредгольма для $A$]
Уравнение $Ax-x = y$ разрешимо тогда и только тогда, когда $f(y) = 0$ для $\fa f \in \Ker(A'-I)$.
\end{theorem}
\begin{proof}
Пусть уравнение разрешимо. Рассмотрим $f \in \Ker(A'-I)$. Тогда
$$f(y) = f(Ax-x) = (A'f)(x) - f(x) = (A'f-f)(x)= \br{(A'-I)f}(x) = 0,$$
что и требовалось.

Докажем теперь обратное утверждение теоремы. Пусть для всякого $f \in \Ker(A' - I)$ имеем $f(y) = 0$.
Нам надо доказать утверждение $A \Ra B$, а мы будем доказывать равносильное ему $\ol B \Ra \ol A$,
то есть если для вектора~$y$ решения нет, то найдётся функционал из ядра, который не обнуляется
на этом векторе.

Пусть уравнение $Ax -x = y$ не решается, то есть $y \notin \Img(A - I)$. Возьмём ограниченный
функционал $f$ такой, что $f(y) = 1$, а $f\br{\Img(A - I)} = 0$.
Это можно сделать в силу леммы~\ref{lem:func.extension}, поскольку $\Img(A-I)$ замкнут
в~силу леммы~\ref{lem:cl.image}.
Далее, для всякого вектора $x$ имеем $0 = f(Ax - x) = (A'f-f)(x)$, значит, $A'f-f = 0$. Таким
образом, $f \in \Ker(A'-I)$, но для него не выполнено условие $f(y) = 0$, и теорема доказана.
\end{proof}

\begin{imp}
Если $\Ker(A'-I) = 0$, то уравнение $Ax - x = y$ разрешимо для всех $y$.
\end{imp}
\begin{proof}
В самом деле, если в ядре $\Ker(A'-I)$ лежит только нулевой функционал, то, очевидно, для
всякого вектора~$y$ выполнено свойство $f(y) = 0$ для $\fa f \in \Ker(A'-I)$, ибо $0(y) \equiv 0$.
Осталось воспользоваться только что доказанной третьей теоремой Фредгольма (справа налево).
\end{proof}

\begin{theorem}[Третья теорема Фредгольма для $A'$]
Уравнение $A'f-f = g$ разрешимо тогда и только тогда, когда $g(x) = 0$ для $\fa x \in \Ker(A-I)$.
\end{theorem}
\begin{proof}
Пусть уравнение $A'f-f = g$ разрешимо. Пусть $x \in \Ker(A - I)$.
Имеем
$$g(x) = (A'f-f)(x) = (A'f)(x) - f(x) = f(Ax-x) = f(0) = 0,$$
что и требовалось доказать.

Обратно, если функционал $f$ есть решение, то $(A'f)(x) - f(x) = g(x)$, то есть $f(Ax-x) = g(x)$.
Поэтому будем конструировать его, исходя из этого равенства.
Зададим функционал $f_0$ на $\Img(A - I)$ так:
$$f_0(Ax-x) := g(x).$$
Проверим, что это корректно, то есть покажем, что значение функционала не зависит от выбора
прообраза $x$ для элемента $Ax - x$. Пусть $Ax_1 - x_1 = Ax_2 - x_2$, нужно
проверить, что $g(x_1) = g(x_2)$. В самом деле,
имеем $(A-I)(x_1-x_2) = 0$, то есть $x_1 - x_2 \in \Ker(A-I)$. По условию теоремы, на
векторах из этого ядра функционал $g$ равен нулю, поэтому $g(x_1 - x_2) = 0$,
то есть $g(x_1) = g(x_2)$.

Теперь покажем, что так определённый функционал $f_0$ ограничен.
Как было установлено в лемме~\ref{lem:cl.image}, по крайней мере для
одного из прообразов $x$ вектора $y$ (то есть $Ax - x = y$) имеет место неравенство $\hn x \le \al \hn y$.
Тогда
$$\hm{f_0(y)} = \hm{f_0(Ax-x)} = \hm{g(x)} \le \hn g \cdot \hn x \le  \al \hn g\cdot\hn y.$$
Тем самым проверена ограниченность $f_0$. Остаётся продолжить его по теореме Хана\ч Банаха
на всё пространство и обозначить это продолжение через $f$. Тогда для всякого $x$ имеем
$$f(Ax -x) = f(y) = f_0(y) = g(x),$$
то есть
$$(A'f-f)(x) = g(x),$$
но  это и означает, что $f$ есть решение исходного уравнения.
\end{proof}

\begin{imp}
Если $\Ker(A-I) = 0$, то уравнение $A'f - f = g$ разрешимо для всех $g$.
\end{imp}

\begin{theorem}[Первая теорема Фредгольма]
Уравнение $Ax-x = y$ разрешимо для любого $y$ тогда и только тогда, когда $\Ker(A-I) = 0$.
\end{theorem}
\begin{proof}
Будем доказывать от противного: пусть ядро $\Ker(A-I) =: X_0$ нетривиально, тогда
найдётся $x_0 \neq 0$, для которого $Ax_0 - x_0 = 0$. Рассмотрим уравнение $Ax -x = x_0$.
По условию у него есть решение $x_1$, то есть $Ax_1 - x_1 = x_0$. Аналогично, у уравнения
$Ax-x = x_1$ есть решение $x_2$, и так далее. Итак, $Ax_i - x_i = x_{i-1}$.
Рассмотрим возрастающую цепочку подпространств $X_n := \Ker(A-I)^{n+1}$.
Покажем, что они строго возрастают. В самом деле, по построению имеем $(A-I)^ix_i = x_0$,
а $(A-I)x_0 = 0$, поэтому $(A-I)^{i+1}x_i = (A-I)x_0 = 0$. Таким образом, вектор $x_i$
аннулируется $(i+1)$\д й степенью оператора $(A-I)$, но не аннулируется его $i$\д й степенью.
Значит, ядра не совпадают. Поскольку подпространства $X_n$ замкнуты (это ядра непрерывных операторов),
то найдётся последовательность единичных векторов $\hc{y_n}$, для которых $\rho(y_n, X_{n-1}) \ge \frac12$.
Покажем, что их образы образуют ежа. Пусть $m > n$, тогда
$$\hn{Ay_m - Ay_n} = \hn{y_m + Ay_m - y_m - Ay_n + y_n - y_n} = \hn{y_m + (A-I)y_m - (A-I)y_n - y_n}.$$
Все слагаемые, кроме первого, погибнут при применении оператора $(A-I)^m$, следовательно, их сумма
представляет собой некоторый вектор из $X_{m-1}$. Поэтому $\hn{Ay_m - Ay_n} \ge \frac12$. Значит,
образы векторов являются ежом, что противоречит компактности оператора.

Симметричное утверждение для сопряжённых операторов доказывается аналогично.

Обратно, пусть $\Ker(A-I)=0$. По следствию из третьей теоремы Фредгольма
для сопряжённого оператора, в этом случае уравнение $A'f-f = g$ всегда разрешимо.
Пользуясь уже доказанным  для сопряжённого оператора, получаем $\Ker(A'-I) = 0$.
Далее, применяя следствие из третьей теоремы Фредгольма для обычных операторов,
получаем, что уравнение $Ax -x = y$ разрешимо для любого $y$.
\end{proof}

\begin{df}
Пусть $V$\т (конечномерное) векторное пространство с базисом $e_1\sco e_n$,
а $V^*$\т сопряжённое ему пространство. Базис $\ep^1\sco \ep^n$ пространства $V^*$
называется \emph{сопряжённым} к $e_1\sco e_n$, если $\ep^j(e_i) = \de_i^j$.
\end{df}

\begin{theorem}[Вторая теорема Фредгольма]
$\dim \Ker(A-I) = \dim \Ker(A'-I)$.
\end{theorem}
\begin{proof}
Пусть $x_1\sco x_n$\т базис в $\Ker(A - I)$, а $f_1\sco f_m$\т базис в $\Ker(A'-I)$.
Возьмём сопряжённый базис $\ph_1\sco\ph_n$ к $x_1\sco x_n$, и $\xi_1\sco \xi_m$\т сопряжённый базис
к $f_1\sco f_m$. Допустим, что $n \neq m$ и рассмотрим два случая.

\pt{1} Пусть $n < m$. Рассмотрим оператор
$$Ux = Ax + \sumiun \ph_i(x)\xi_i.$$
Этот оператор компактен, поскольку это сумма компактного и конечномерного операторов.

Покажем, что $\Ker(U - I) = 0$. В самом деле, пусть $Ux - x = 0$.
Распишем выражение для оператора $U$:
$$Ax + \sumiun \ph_i(x)\xi_i = x,$$
$$Ax - x + \sumiun \ph_i(x)\xi_i = 0.$$
Подействуем на это выражение функционалами $f_j$, получим
$$f_j(Ax-x) + \sumiun \ph_i(x)f_j(\xi_i) = 0, \quad j =1\sco m,$$
или, что то же самое,
$$(A'f_j-f_j)(x) + \sumiun \ph_i(x)f_j(\xi_i) = 0, \quad j =1\sco m.$$
Но поскольку $f_j$ лежат в ядре, то первое слагаемое равно нулю. При каждом значении $j$ в сумме
выживает только слагаемое с номером $j$ в силу сопряжённости базисов.
Следовательно, $\ph_i(x) = 0$ при всех $i$, поэтому $Ux = Ax$, откуда $Ax - x = 0$, следовательно,
$x \in \Ker(A-I)$. Но так как $\ph_i$ образуют базис пространства, сопряжённого этому ядру, и все
обнуляются на векторе $x$, то $x$ может быть только нулём, что и требовалось доказать.

Применим следствие третьей теоремы Фредгольма: если $\Ker(U - I) = 0$, то уравнение
$Ux - x = y$ разрешимо при любой правой части, в частности, при $y = \xi_{n+1}$.
Пусть $x$\т решение этого уравнения. Тогда
\begin{multline*}
1 = f_{n+1}(\xi_{n+1}) = f_{n+1}\hr{U x - x} = f_{n+1}\Br{Ax - x + \sumiun \ph_i(x)\xi_i} =\\=
f_{n+1}(Ax-x) + \sumiun\ph_i(x)f_{n+1}(\xi_i) \stackrel{!}{=}
(A'f_{n+1} - f_{n+1})(x) + 0 = \br{(A'-I)f_{n+1}}(x) \stackrel{!!}{=} 0(x) = 0.
\end{multline*}
Здесь переход <<!>> следует из того, что $i < n+1$ (а потому всё убивает сопряжённость базисов),
а переход <<!!>>\т из того, что $f_{n+1}$ лежит в ядре $\Ker(A'-I)$.
Но так как $1 \neq 0$, мы получаем противоречие, значит, случай $n < m$ невозможен.

\pt{2} Теперь допустим, что $m < n$. Рассмотрим оператор
$$U'f = A'f + \sumium f(\xi_i)\ph_i.$$

Покажем, что $\Ker(U' - I) = 0$. В самом деле, пусть $U'f - f = 0$.
Распишем выражение для оператора $U'$:
$$A'f + \sumium f(\xi_i)\ph_i = f,$$
$$A'f - f + \sumium f(\xi_i)\ph_i = 0.$$
Подействуем этим выражением на векторы $x_j$, получим
$$(A'f-f)(x_j) + \sumium f(\xi_i)\ph_i(x_j) = 0, \quad j =1\sco n,$$
или, что то же самое,
$$f(Ax_j-x_j) + \sumium f(\xi_i)\ph_i(x_j) = 0, \quad j =1\sco n.$$

Но поскольку $x_j$ лежат в ядре, то первое слагаемое равно нулю. При каждом значении $j$ в сумме
выживает только слагаемое с номером $j$ в силу сопряжённости базисов.
Следовательно, $f(\xi_i) = 0$ при всех $i$, поэтому $U'f = A'f$, откуда $A'f - f = 0$, следовательно,
$f \in \Ker(A'-I)$. Но так как $\xi_i$ образуют базис пространства, сопряжённого этому ядру
(здесь мы неявно пользуемся тем, что для конечномерных пространств имеет место
канонический изоморфизм $V^{**} \cong V$),
и ковектор $f$ на них равен нулю, то $f$ может быть только нулём, что и требовалось доказать.

Применим следствие третьей теоремы Фредгольма: если $\Ker(U' - I) = 0$, то уравнение
$U'f - f = g$ разрешимо при любой правой части, в частности, при $g = \ph_{m+1}$.
Пусть $f$\т решение этого уравнения. Тогда
\begin{multline*}
1 = \ph_{m+1}(x_{m+1}) = \hr{U'f - f}(x_{m+1}) = \Br{A'f - f + \sumium f(\xi_i)\ph_i}(x_{m+1}) =\\=
(A'f-f)(x_{m+1}) + \sumium f(\xi_i)\ph_i(x_{m+1}) \stackrel{!}{=}
f(Ax_{m+1} - x_{m+1}) + 0 = f\br{(A-I)x} \stackrel{!!}{=} f(0) = 0.
\end{multline*}
Здесь переход <<!>> следует из того, что $i < n+1$ (а потому всё убивает сопряжённость базисов),
а переход <<!!>>\т из того, что $x_{m+1}$ лежит в ядре $\Ker(A-I)$.
Но так как $1 \neq 0$, мы получаем противоречие, значит, случай $m < n$ тоже невозможен.

Итак, остаётся единственная возможность $m=n$, но это и требовалось доказать.
\end{proof}


\subsubsection{Альтернатива Фредгольма}

\begin{theorem}[Альтернатива Фредгольма]
Рассмотрим уравнение $(A-I)x = y$. Тогда либо ядро
$\Ker(A-I)$ ненулевое, либо оператор $A - I$ обратим.
\end{theorem}
\begin{proof}
Пусть ядро нулевое, тогда в силу первой теоремы Фредгольма,
уравнение $(A-I)x = y$ разрешимо для любого $y$, то есть $\Img(A-I) = X$.
Кроме того, оператор $A - I$ инъективен. Следовательно, это биекция $X \lra X$.
По теореме Банаха оператор $A - I$ будет иметь ограниченный обратный.
\end{proof}

\begin{imp}
Ненулевые точки спектра компактного оператора суть его собственные значения
конечной кратности.
\end{imp}
\begin{proof}
В самом деле, при $\la \neq 0$ можно применить альтернативу Фредгольма:
если $\Ker(A - \la I) = 0$, то оператор $A - \la I$ обратим,
поэтому такие $\la$ не принадлежат спектру. Если же ядро ненулевое,
то его размерность конечна, как было доказано выше.
\end{proof}

\begin{note}
Первая теорема Фредгольма допускает слегка парадоксальную переформулировку:
<<Если решение единственно, то оно существует>>.
\end{note}

\subsubsection{Частный случай: гильбертовы пространства}

Строго говоря, сопряжённое пространство $H^*$ к гильбертову пространству $H$ не изоморфно
исходному пространству. Между $H$ и $H^*$ имеется так называемый \emph{антиизоморфизм}.
Задаётся он очевидным образом: Пусть $h \in H$. Сопоставим этому вектору некоторый функционал:
$$h \corr{\ph} f_h(x) := (x,h).$$
Это  корректно, так как все функционалы имеют такой вид, то есть отображение сюръективно,
а инъективность сомнений не вызывает. Но скалярное произведение в пространстве над
полем $\Cbb$ антилинейно по второму аргументу, поэтому и $\ph$ будет антилинейным:
$$\ph(\la h) = f_{\la h} = (x,\la h) = \ol \la (x,h) = \ol \la \ph(h).$$
Такие отображения и называют антиизоморфизмами.

А вот если пространство над полем $\R$, то всё совсем хорошо и $H \cong H^*$.

Тогда теоремы Фредгольма переформулируются так:


\eqn{\Ker(A-I)=0 \Lra \Img(A-I)=H;}
\eqn{\dim \Ker(A-I)=\dim \Ker(A^*-I);}
\eqn{\Ker(A^*-I)=\Img(A-I)^\perp.}


\section{Приложение}

\subsection{Service Pack 1 (Миша Берштейн, Миша Левин)}

Здесь под сохраняющимся спектром понимаются либо собственные значения бесконечномерной кратности,
либо те значения $\la$, при которых будет незамкнутым образ (при этом вполне может быть конечномерное ядро).

\begin{lemma}
Следующие два утверждения эквивалентны:
\begin{nums}{-2}
\item Выполняется хотя бы одно из двух условий:
$\dim\Ker(A-\la I)=\infty$ или $\Img (A-\la I)$ незамкнут;
\item $\exi$ непредкомпактная последовательность $\{x_n\}$, такая, что $\hn{x_n}=1$ и $(A-\la I)x_n\ra 0$.
Предполагается, что исходное пространство банахово.
\end{nums}
\end{lemma}
\begin{proof}
В процессе доказательства леммы о замкнутости образа оператора $A-I$, если $A$ компактен был установлен
следующий факт: если ядро оператора $A-I$ конечномерно (не обязательно компактного), то
$\fa y \in \Img (A-I) \exi \wt{x}$, являющееся наименьшем по норме решением
уравнения $(A-I)x=y$. Понятно, что это верно и для оператора $A-\la I, \fa \la$.

Пусть есть оператор $B,\dim\Ker B<\infty$, тогда профакторизуем все пространство по $\Ker B$ и
введем в этом новом пространстве норму: $\|x\|_1=\|\wt{x}\|$, где $\wt{x}$\т
наименьшее решение уравнения $Bz=y$, где $y=Bx$.

Докажем корректность определения. Если $x_1$ и $x_2$ лежат в одном классе эквивалентности, то
$Bx_1=Bx_2$ и, соответственно, $\wt{x}$ для них одинаково.

$\hn{x}_1=0 \Lra x \in \Ker B$, значит, его образ при факторизации равен $0$ --- первое свойство нормы.

$\hn{\la x}_1=\la \hn{x}_1$\т очевидно.

$\hn{x+y}_1=\hn{\wt{x+y}} \le \hn{\wt{x}+\wt{y}} \le \lcomm$ так как $B(\wt{x+y})=B\wt x +B\wt y $, а
$\wt{x+y}$\т наименьшее решение $\rcomm \le \hn{\wt x}+\hn{\wt y} \bw= \hn{x}_1+\hn{y}_1$.

Значит, это действительно будет нормой.

Будем доказывать, что $1\Ra2$.

1. $\dim \Ker(A-\la I)=\infty$ --- этот случай был разобран ранее.

2. $\dim \Ker(A-\la I)<\infty$ и $\Img(A-\la I)$ незамкнут.

Профакторизуем по $\Ker(A-\la I)$ и введем норму, как это было указано выше.  Мы получим оператор
$B-\la I$, индуцированный оператором $A-\la I$ на факторизованном пространстве, причем $\Ker
(B-\la I)=0$, а $\Img(B-\la I)$ не замкнут. Такой случай также разбирался и, значит, для оператора
$B-\la I$ $\exi$ искомая $\{x_n\}$. Тогда в качестве последовательности для $A-\la I$ возьмем
$\{\wt{x_n}\}$. Докажем, что она подходит. Имеем $\hn{\wt{x_n}}=\hn{x_n}_1=1$ и $(A-\la
I)\wt{x_n}=(B-\la I)x_n \ra 0$.

Осталось доказать непредкомпактность $\{\wt{x_n}\}$.
Предположим противное. Тогда найдётся фундаментальная подпоследовательность
$\{\wt{x_{n_k}}\}$. Тогда для $\fa \ep>0$ найдётся $N$ такое, что для $\fa k,l>N$ выполнено $\hn{\wt{x_{n_k}}-\wt{x_{n_l}}}<\ep$.
Но для $\fa k,l$ имеем
$$\|\wt{x_{n_k}}-\wt{x_{n_l}}\|\ge \|\wt{x_{n_k}-x_{n_l}}\|=\|x_{n_k}-x_{n_l}\|.$$
Значит $\{x_{n_k}\}$ также фундаментальна. Противоречие.

Теперь докажем, что $2\Ra1$.

Если $\Img(A-\la I)$ незамкнут, то все доказано. Пусть тогда  он замкнут. Если $\dim \Ker(A-\la I)=\infty$,
то все доказано. Предположим противное. Отфакторизуем по $\Ker(A-\la I)$. Тогда пусть $t_n$\т образ $x_n$
при факторизации. Докажем, что $\exi c>0: \fa n$ $\|t_n\|_1>c$. Предположим противное. Тогда найдётся
подпоследовательность $\hc{t_{n_k}}$ такая, что $\hn{t_{n_k}}_1\ra 0$. Тогда
$\|\wt{x_{n_k}}\|\ra 0$. Обозначим $z_k=x_{n_k}-\wt{x_{n_k}}$. Тогда
$\|z_k\|=\|x_{n_k}-\wt{x_{n_k}}\|\le\|x_{n_k}\|+\|\wt{x_{n_k}}\|\le 2\|x_{n_k}\|=2$. Очевидно,
$z_k \in \Ker(A-\la I)$. Тогда мы получаем ограниченную последовательность в конечномерном пространстве.
Тогда она предкомпактна. После перенумерации можно считать, что $\{z_k\}$ сходится. Но тогда $\{x_{n_k}\}$
сходится как сумма сходящихся последовательностей $\{z_k\}$ и $\{\wt{x_{n_k}}\}$ --- противоречие с
тем, что $\{x_n\}$ непредкомпактна. Обозначим $X$ - отфакторизованное пространство. Обозначим
$Y=\Img(A-\la I)$. Так как $Y$ --- замкнутое подпространство банахова пространства, то $Y$ --- банахово.

\begin{lemma}
Пространство $X$ банахово.
\end{lemma}
\begin{proof}
Пусть $\{x_n\}$ --- фундаментальная  последовательность в $X$. Пусть $\fa i,\,j>N_1$
$\|x_i-x_j\|_1<\frac1{2}$. Пусть $y_1=x_{N_1}$. Пусть $\fa i,\,j>N_2>N_1$. $\fa i,\,j>N_2$
$\|x_i-x_j\|_1<\frac1{4}$ и $y_2=x_{N_2}$ и т.д. Тогда $\fa n \fa i,\,j\ge n$
$\|y_i-y_j\|_1<\frac1{2^n}$. Возьмем $z_1=\wt{y_1},\, z_n=z_{n-1}+(\wt{y_n-y_{n-1}})$. Тогда
$z_n$ является прообразом для $y_n$ при отображении факторизации.
$\|z_{n+1}-z_n\|=\|\wt{y_{n+1}-y_n}\|=\|y_{n+1}-y_n\|_1$. Тогда $\fa n \fa i>j\ge n$ имеем
$$\|z_i-z_j\|\le\|z_i-z_{i-1}\|+\dots+\|z_{j+1}-z_j\|<\frac1{2^{i-1}}+\dots+\frac1{2^j}<\frac1{2^{n-1}},$$
т.е. $\{z_n\}$ --- фундаментальна. Тогда $z_n\ra z$. Пусть $y$ --- образ $z$ при факторизации. Тогда
$y_n\ra y$. Тогда $\{x_n\}$ --- фундаментальная последовательность, подпоследовательность которой
сходится к $y$. Значит, $x_n\ra y$.
\end{proof}

Обозначим $B$ --- оператор, индуцированный $A-\la I$ на $X$.  Тогда $B:X\ra Y,\, \Ker B=0,\, X$ и
$Y$ --- банаховы. $B$ ограничен, так как $\|Bx\|=\|(A-\la I)\wt{x}\|\le \|A-\la I\|\|\wt{x}\|=\|A-\la I\|\|x\|_1$.
Тогда по теореме Банаха существует ограниченный $C=B^{-1}$. Обозначим $w_n=Bt_n$. Тогда
$t_n=Cw_n$. Но $w_n\ra 0$, а $\exi c: \fa n$ $\|t_n\|_1>c$. Противоречие с ограниченностью $C$.
\end{proof}

\subsection{Service Pack 2 (Юра Малыхин)}

\subsubsection{Теорема Хана\ч Банаха}

Первый случай, когда пространство сепарабельно, т.е. $\exi (x_n)\subset H$ плотное.
Продолжаем $F$ на $\ha{X_0,x_1,\ldots,x_n}$ последовательно. Получим в итоге $F$ на $\ha{X,x_i}$\т
это плотное подмножество. Продолжим $F$ по непрерывности: надо проверить корректность и что норма
не испортится. Корректность: пусть $a_i,b_i\to c$. Мы хотим положить $F(c):=\lim F(a_i)$. Т.к $a_i$
фундаментальна, то и $F(a_i)$ тоже, поэтому предел существует; т.к
$\hn{a_i-b_i}\ra 0$, то $\hm{F(a_i)-F(b_i)} \le \hn{F_0}\cdot \hn{a_i-b_i} \ra 0$, т.е.
пределы равны. Далее,
$$\|a_i\|\to \|c\|, \quad |F(a_i)|\le \hn{F_0}\cdot \hn{a_i}, \quad |F(a_i)|\ra|F(c)|.$$
Отсюда $|F(c)|\le \|F_0\|\cdot \|c\|$.


\subsubsection{Спектральная теорема}

Более подробно про построение $U$: сначала определяем $U$ на всюду плотном множестве $\ha{A^n h}$ так: если
$v=P(A)h$, то $Uv=P$.  Корректность проверена (по определению меры $\sigma$). Изометрия проверена\т это ровно
то же тождество, только без равенства нулю. Значит, все продолжается на $C[-\|A\|,\|A\|]$, причем остается
свойство изометричности. Отсюда следует инъективность $U$. Проверим сюръективность: берем $f\in L_2(\sigma)$
и строим многочлены $P_n\to f$ (сходимость по норме $L_2$ - существование такой последовательности
многочленов см. ниже - лемма). Берем $x_n=P_n(A)h$ - прообразы. В силу изометрии последовательность $x_n$
фундаментальна и имеет предел $x$. Тогда $Ux=f$. Теперь нам надо доказать следующее равенство функций:
$(UAU^{-1}f)(\la)=\la f(\la)$. Если теперь $P_n\to f$, то:
$$\la f(\la)\stackrel{1}{\gets} \la P_n(\la)\stackrel{2}=
UAU^{-1}P_n\stackrel{3}{\to}UAU^{-1}f.$$
1 означает сходимость по норме $L_2$, эта сходимость следует из того, что
$\|\la P_n(\la)-\la f(\la)\|_2 \le C\|P_n-f\|_2$, где $|\la|\le C$.\\
2 следует из определения $U$\\
3 означает сходимость по норме $L_2$, она следует из непрерывности операторов $U, U^{-1}, A$.

Поскольку предел в $L_2$ единствен, левая часть равна правой.

\begin{lemma}
Пусть $\sigma$\т конечная борелевская мера на $[a,b]$ (\те она задана на борелевских подмножествах отрезка).
Тогда многочлены плотны в $L_2(\sigma)$.
\end{lemma}
\begin{proof}
Нам надо приблизить многочленами все функции из $L_2$. Поскольку
все непрерывные приближаются многочленами равномерно, то приближаются и по норме
$L_2$. Поскольку простые функции плотны в $L_2$, остается лишь приблизить непрерывными функциями
индикаторы измеримых (=борелевских) множеств. Теперь рассмотрим те $E$, индикаторы которых
приближаются непрерывными функциями, т.е. $F=\{E:\chi_E\in\overline{C[a,b]}\}$. Покажем, что $F$ есть
$\sigma$-алгебра. Действительно, если $f_n$ приближают $\chi_E$, то $1-f_n$ приближают
$\chi_{[a,b]\setminus E}$, если $f_n\to E_1$, $g_n\to E_2$, то $f_ng_n\to \chi_{E_1\cap E_2}$,
если есть счетное число непересекающихся $E_i$, то берем $f_1:\|f_1-\chi_{E_1}\|<\frac12$,
$f_2:\|f_2-\chi{E_2}\|<\frac14$, и так далее. Тогда $\|\sum f_i-\chi_{E_1\sqcup E_2\sqcup\ldots}\|<1$.
Покажем, что в $F$ входят полуинтервалы вида $[a,t)$. Такой интервал приближается функцией $f_n(x)$,
которая есть 0 при $x\ge t$, 1 при $x\le t-\frac{1}{n}$, концы соединяем по линейности. Действительно,
тогда ошибка есть $\int_{[t-1/n,t)} |1-f_n|^2 \le \sigma([t-1/n,t)) \to 0$.
Значит F содержит все борелевские множества, чего нам и надо было доказать.
\end{proof}

\subsubsection{Теорема Ф. Рисса}

\begin{stm}
Пусть для любой вещественнозначной $f$ имеем $F(f)\in\R$. Покажем, что построенную
функцию $g$ можно выбрать окажется вещественной.

Пусть для любой вещественнозначной неотрицательной $f$ имеем $F(f)\ge 0$. Покажем, что $g$
можно выбрать вещественной неубывающей.
\end{stm}
\begin{proof}
Заметим, что мы используем теорему Рисса в несколько другой форме, а именно, пользуемся не самой
функцией $g$, а мерой $dg$. Поэтому после того, как построили функцию конечной вариации $g$,
изменим ее в счетном числе точек, чтобы представить в виде разности $g'=g'_1-g'_2$ функций
распределения некоторых мер $\mu_2,\mu_1$ на $[a,b]$. Это делаем так: известно, что $g$
есть разность $g=g_1-g_2$ двух монотонных функций. Далее переопределим, если надо, $g_1$ и $g_2$
так, чтобы они стали непрерывны слева на $(a,b)$. С точкой $b$ происходит \textbf{жопа}.
Тогда $dg'_1$ и $dg'_2$\т некоторые меры, поэтому интеграл по $dg$ есть разность интегралов
Лебега по мерам $dg'_1$ и $dg'_2$, а с интегралом Лебега работать намного приятнее!
Техническое утверждение \cite[гл.~6, \S~6, п.~4]{kf} состоит в том, что для всякой непрерывной
функции $f$ значение интеграла не изменится, т.е. $\int fdg=\int fdg'=\int fdg'_2-\int fdg'_1$,
где первые два интеграла понимаются в смысле Римана\ч Стилтьеса, а последние два\т в
смысле Лебега\ч Стилтьеса. Далее везде считаю функцию $g$ подправленной. (Замечу, что рассуждения
в этом параграфе необходимо проводить, чтобы получить именно меру в спектральной теореме!)

1) итак, пусть $F$ вещественный. Сделаем $g(c)\in\mathbb R$ для какой-то фиксированной
$c\in(a,b)$. Возьмем произвольное $d>c$ и докажем $g(d)\in\mathbb R$. Для этого построим последовательность
<<почти индикаторов>> $f_n$, сходящихся почти всюду к $\psi=\chi_{[c,d)}$. Тогда
$\int f_ndg_2\to\int\psi dg_2$ (из свойств интеграла Лебега), $\int f_ndg_1\to\int\psi dg_2$,
откуда $\int f_ndg\to\int\psi dg=g(d)-g(c)$. Поскольку слева стоят вещественные числа, то и
предел получится вещественным. Аналогично доказывается $g(d)\in\mathbb R$ для $d<c$ и
для $g(b)$. (Тут я сжульничал! На самом деле проблема имеется в точке $b$, по хорошему $g_1$
и $g_2$ должны быть непрерывны в точке $b$ слева, но тогда нужно знать еще меру точки $b$, \те
$g(b+0)-g(b)$. Подробности слишком техничны и неинтересны)

2) Пусть $F$ неотрицателен. Совершенно аналогично доказывается, что $g$ монотонна. Выбирая $g$
вещественной в какой-нибудь точке, получим вещественную монотонную функцию, которая будет функцией
распределения некоторой меры (опять возникает небольшая проблема с мерой $b$). Эту меру и надо
будет использовать в спектральной теореме.
\end{proof}

\subsubsection{Теорема о компактном возмущении}

Контрпример, показывающий неравносильность определений непрерывного спектра: рассмотрим $A\cln \ell_2\ra \ell_2$,
$(x_1,x_2,\ldots)\mapsto(x_1,x_2/2,x_3/3,\ldots)$. Он компактен, причем $0 \in \Sig_c(A)$.
Рассмотрим компактное возмущение $K=-A$: $A+K=A-A=0$, но у нуля нет
непрерывного спектра. По лемме (см. соотв. главу) из того, что $\la\in\Sig_с(A)$ следует
существование некомпактная последовательности $\hn{x_n} = 1$ и $(A-\la)x_n\ra 0$.
Обратно же неверно. Пример: $A=0$, $\la=0$, $h_n=e_n$\т ОНС.

\subsection{Полезные утверждения, примеры, факты}

\subsubsection{К теореме Банаха\ч Штейнгауза}

Покажем, что в принципе равномерной ограниченности нельзя убрать требование банаховости пространства~$X$.

\begin{ex}
Пусть $X$\т пространство финитных последовательностей, а $Y = \ell_1$.
Определим семейство операторов так:
$$A_n(x_1\sco x_n\etc) := (0\sco nx_n,0\etc).$$
Для каждой финитной последовательности $x = (x_1\sco x_N, 0,0\etc)$ найдётся нужная константа $C_x$:
достаточно взять $C_x = N$.
С~другой стороны, $\hn{A_n} = n \ra \infty$ при $n \ra \infty$.
\end{ex}


\subsubsection{К теореме Банаха об обратном операторе}

Покажем, что полнота пространств в теореме Банаха существенна.

\begin{ex}
Пусть $X = \Cb[0,1]$ с чебышёвской нормой, а $Y = \Cb[0,1]$ с интегральной нормой.
Тогда $X$ полно, а $Y$, очевидно, нет. Рассмотрим тождественный оператор $\id\cln X \ra Y$.
Очевидно, $\hn{\id} \le 1$, но оператор $\id^{-1}$ не является ограниченным.
В самом деле, рассмотрим $f_n := n \chi_{\hs{0,\frac1n}}$ и чуть\д чуть их сгладим,
чтобы они стали непрерывными. Тогда их интегральная норма будет близка к~$1$, а чебышёвская\т
неограниченно возрастать. Значит, $\id^{-1}$ не является ограниченным оператором.
\end{ex}


\begin{df}
Базис Гамеля\т такая система векторов $\Bc \subs L$,
что всякий вектор $x \in L$ единственным образом представляется в виде конечной линейной комбинации
векторов из $\Bc$.
\end{df}

\begin{theorem}
Всякое линейное пространство $L$ обладает базисом Гамеля.
\end{theorem}
\begin{proof}
Существенно использует аксиому выбора, поэтому в справедливость этой теоремы
можно либо верить, либо не верить.
\end{proof}

В предположении справедливости этой теоремы, можно легко построить пример
неограниченного оператора в любом бесконечномерном пространстве. Пусть $\Ga$\т
базис Гамеля пространства $L$. Выберем счётное подмножество среди базисных элементов
и занумеруем их, получим набор $\hc{\ga_i}$. Считаем, что базисные вектора имеют единичную
длину. Зададим действие оператора на базисных векторах: положим $A\ga_i = i\ga_i$,
а для всех остальных базисных векторов $e \in \Ga \wo \hc{\ga_i}$ положим $Ae = 0$. Тем самым
мы задали действие оператора на всех векторах пространства $L$, ибо всякий вектор
единственным образом разлагается по нашему базису. Поэтому, если $x = \sumiun a_i e_i$,
то $Ax = \sumiun a_i Ae_i$ и тем самым оно однозначно определено. Вместе с этим ясно, что оператор~$A$
неограничен, поскольку для всякого $M > 0$ найдётся вектор, который растягивается этим оператором
больше, чем в $M$ раз.

\begin{ex}
Из теоремы о базисе Гамеля следует, что на всяком бесконечномерном пространстве
существует неограниченный линейный оператор.
Аналогично строится и неограниченный линейный функционал,
именно, возьмём $\ph(\ga_i) = i\hn{\ga_i}$, а на всех остальных векторах базиса положим его
равным нулю. Тогда возьмём какое\д нибудь полное пространство $X$ и оснастим его другой
нормой $\hn{\cdot} := \hn{\cdot}_L + \hn{\cdot}_\ph$, то есть
положим $\hn x := \hn{x}_L + \hm{\ph(x)}$. Обозначим новое пространство через $L_\ph$
и рассмотрим неограниченный оператор $\id\cln L \ra L_\ph$.
Рассмотрим оператор $\id\cln L_\ph \ra L$. Его норма, очевидно, не превосходит~$1$,
однако обратный оператор неограничен, поскольку норма $i$\д ого базисного вектора
увеличивается в $(1+i)$ раз.
\end{ex}

Покажем что в теореме Банаха существенно требование существования алгебраического
обратного отображения.

\begin{ex}
Возьмём оператор правого сдвига в пространстве $\ell_p$. Он действует так:
$$Rx = R(x_1,x_2\etc) = (0,x_1,x_2\etc).$$
Понятно, что $\hn{R} = 1$. Очевидно, что левый обратный оператор существует\т это левый сдвиг $L$,
причём он тоже ограничен. Но правого обратного не существует, поскольку $\Img R \neq \ell_p$, значит,
не для каждого вектора будет выполнено равенство $R L x=x$.
\end{ex}


\subsubsection{Сопряжённый аналог ТБШ}
\label{sssec:adjoint.banach-shteinhouse}

\begin{theorem}[Аналог принципа равномерной ограниченности для сопряжённых пространств]
Слабо ограниченная последовательность ограничена по норме.
\end{theorem}
\begin{proof}
Пусть $x_i$\т слабо ограниченная последовательность. Это значит, что
для всякого $f \in X^*$ существует число $C_f$ такое, что для $\fa i$ имеем
$\hm{f(x_i)} \le C_f$. Докажем, что найдётся такое $C$, что $\hn{x_i} \le C$ для всех~$i$.

Рассмотрим семейство множеств
$$F_n := \hc{f \in X^* \cln \fa i \text{ имеем} \hm{f(x_i)} \le n}.$$
Очевидно, что $X^* = \bigcup F_n$. Поскольку $X^*$ полно и потому не есть множество первой категории,
найдётся $F_N$ такое, что оно не является нигде не плотным в $X^*$. Значит, есть шар,
где оно всюду плотно.

Покажем, что все множества $F_n$ замкнуты. Для этого докажем, что дополнения к ним открыты.
Возьмём какой\д нибудь элемент $f \notin F_n$. Значит, $\exi j$, для которого имеем $\hn{f(x_j)} > n$.
Покажем, что найдётся окрестность элемента $f$, не пересекающаяся с $F_n$. Пусть $g \in X^*$, тогда
$$\hm{(f+g)(x_j)} \ge \hm{f(x_j)} - \hm{g(x_j)}.$$
Поскольку набор множеств $F_n$ покрывает всё пространство $X^*$,
найдётся $M$, для которого имеем $g \in F_M$. Поэтому для всех $i$ имеем $\hm{g(x_i)} \le M$.
Рассмотрим функционал $h = \la g$. Поскольку $M$ уже зафиксировано, число $\la$ можно выбрать таким,
чтобы для всех $i$ число $\hm{h(x_i)}$ было сколь угодно маленьким. Если это число выбрать правильно,
то можно считать, что сохраняется неравенство
$$\hm{(f+h)(x_j)} \ge \hm{f(x_j)} - \hm{h(x_j)} > n.$$
Действительно, $\hm{f(x_j)} >n$, значит, можно отнять от него настолько маленькое число так, чтобы результат всё ещё
был больше $n$. Это и означает, что искомая окрестность элемента $f$ найдена.

По лемме, множество $F_N$ содержит некоторый шар $B$.
Достаточно установить равномерную ограниченность для функционалов в некотором шаре,
содержащем начало координат. Пусть $\wt B$\т копия шара $B$ с центром в начале координат.
Каждый функционал $g \in \wt B$ можно представить как $f_1 - f_2$, где $f_i \in B$.
В силу неравенства треугольника и определения множества $F_N$
для всех~$i$ получаем $\hm{g(x_i)} = \hm{f_1(x_i)-f_2(x_i)} \le N + N = 2N$.

Рассмотрим образ исходной последовательности в пространстве $X^{**}$.
Докажем, что он ограничен. Но это только что было установлено, поскольку
мы получили, что некоторое множество коковекторов равномерно ограничено
на шаре с центром в начале координат.
\end{proof}

\subsection{Service Pack 3 (Юра Притыкин)}

Доказательство леммы о слабой компактности шара в гильбертовом пространстве $H$ без свойства сепарабельности:

Надо доказать, что из последовательности $\hc{x_n}$ можно выбрать слабо сходящуюся подпоследовательность. Рассмотрим подпространство
$H_0$, порожденное всеми элементами последовательности $\hc{x_n}$, и замкнём его. Получим некоторое сепарабельное
гильбертово подпространство в $H$. Оно не более чем счётномерно. Выберем в нем счетный базис и применим к нему доказанную лемму.
Поэтому если мы можем найти такую последовательность в сепарабельном гильбертовом, то и в любом гильбертовом.

\end{document}

%=================================================================
\subsection{Фрагменты лекций, признанные ошибочными}

\subsubsection{Какое-то следствие из теоремы Фредгольма}

Здесь видимо используется общепринятое определение точечного спектра
оператора: $\Ker (A - \la I) \neq 0$. Во-вторых, непонятно, откуда следует то,
что $B^{-1}$ ограничен (в условии этого НЕ предполагалось).

\begin{imp}
Пусть $A$ и $B$\т ограниченные операторы, причём $A-B$ компактен. Тогда из
$\la \in \Sig_A \wo \Sig_A^P$ следует, что $\la \in \Sig_B$.
\end{imp}
\begin{proof}
Фактически, надо доказать следующее: если $A - \la I$ необратим, но $\Ker(A-\la I) = 0$, то и $B - \la$ необратим.
Напишем отрицание этого утверждения: если $B-\la I$ обратим, то либо $A -\la$ обратим, либо $\Ker(A-\la I) \neq 0$.
Но можно от операторов $A -\la$ и $B - \la$ перейти соответственно к операторам $A$ и $B$, то есть
сделать замену. Тогда утверждение переформулируется так: если $B$ обратим, то либо $A$ обратим,
либо $\Ker A$ нетривиально. Рассмотрим
$$A = A - B + B = B\bs{B^{-1}\hr{A-B} + 1}.$$
Но $A - B$ компактен по условию, а $B^{-1}$ ограничен, поэтому оператор $B^{-1}(A-B)$ тоже компактен.
Значит, для оператора в квадратных скобках имеет место альтернатива Фредгольма. Но это как раз то, что нам надо.
\end{proof}

\subsubsection{Теория Фредгольма}

\textbf{Постановка задачи и аналогия с конечномерными пространствами}

Пусть $H$\т гильбертово пространство, а $A\cln H\ra H$\т компактный оператор.
Рассмотрим уравнения
\begin{gather}
\mat
{(I - A) h = f,\\
 (I - A) h = 0.} \quad
\mat
{(I - A^*) \wt h = \wt f,\\
 (I - A^*) \wt h = 0.}
\end{gather}
Здесь $f$ и $\wt f$\т фиксированные векторы, а $h$ и $\wt h$\т неизвестные.

Рассмотрим также конечномерную ситуацию:
\begin{gather}
\mat
{B h = f,\\
 B h = 0.} \quad
\mat
{B^* \wt h = \wt f,\\
 B^* \wt h = 0.}
\end{gather}


\begin{theorem}[Теоремы Фредгольма]
\begin{nums}{-2}
\item $\Img B = H$ тогда и только тогда, когда $\Ker B =0$.
\item $\dim \Ker B = \dim \Ker B^*$.
\item $\Ker B^* = (\Img B)^\bot$.
\end{nums}
\end{theorem}
\begin{proof}
Докажем эти три теоремы для конечномерного случая.

Докажем Ф1. Пусть $f \bot \Img B$, тогда для всякого $h \in H$ имеем $0 = (f,Bh) = (B^*f,h)$,
откуда следует, что $B^*f = 0$. Но это и значит, что $f \in \Ker B^*$.
Обратно, пусть $x \in \Ker B^*$, то есть $B^* x = 0$. Проверим, что $x \bot Bf$ для всякого $f \in H$.
В самом деле, имеем $0 = (0,f) = (B^*x,f) = (x,Bf)$, что и требовалось доказать.

\comment{
Этого не было доказано на лекциях, но, думается мне, теоремы Ф2 и Ф3 проще всего вывести из формулы
\eqn{\label{eqn:dimension.formula}\dim \Img \ph + \dim \Ker \ph = \dim V,}
доказанной в курсе линейной алгебры. В этой формуле $\ph$ есть линейное отображение
$\ph\cln V \ra W$, а $V$ и $W$\т конечномерные векторные пространства. В частности, если
$W = V$, то $\ph$\т это просто линейный оператор.

Докажем~Ф2 и~Ф3.
В ортонормированном базисе матрица оператора~$B^*$ записывается транспонированной
матрицей оператора~$B$. А $\rk B = \rk B^\top$, поэтому $\dim \Img B = \dim \Img B^*$,
а отсюда получаем $\dim \Ker B = \dim \Ker B^*$. Теорема~Ф3\т прямое следствие формулы~\eqref{eqn:dimension.formula}.}
\end{proof}

\textbf{Доказательство для операторов с конечномерным образом}

Так называемая альтернатива Фредгольма заключается в следующем: либо оператор $I - A$ обратим,
либо $\Ker(I - A)$ нетривиально. Очевидно, что в том случае, когда пространства конечномерны, так оно и будет.
Идея доказательства теорем Фредгольма в бесконечномерном случае будет состоять в приближении
компактного оператора конечномерными.

Рассмотрим сначала случай, когда $\dim \Img A < \infty$. При этом мы уже не будем предполагать,
что оператор действует в конечномерном пространстве.

\begin{note}
Точкой мы будем обозначать опущенный аргумент отображения. Иными словами,
если у нас есть отображение $F(x) = f(x,y)$, то это может быть записано так: $F = f(\cdot,y)$.
\end{note}

\begin{prop}
Если $A = \suml{n=1}{N}(\cdot, e_n^*)e_n$, то
$A^* = \suml{n=1}{N}(\cdot, e_n)e_n^*$.
\end{prop}
\comment{На лекции доказано не было (упражнение). Чтобы его доказать, нужно сначала понять,
что такое $e_n^*$.}

Из этого предложения следует [?], что
\begin{gather*}
\dim \Img A^* \le \dim \Img A,\\
\dim \Img A^{**} \le \dim \Img A^*.
\end{gather*}
Но $A^{**} = A$, поэтому на самом деле
имеем $\dim \Img A = \dim \Img A^*$.

Рассмотрим уравнение $(I - A)x = 0$. Как уже было замечено,
$Ax = \sums{k}(x,e_k^*)e_k$, поэтому это можно подставить в однородное уравнение:
$$x - \sums{k}(x,e_k^*)e_k = 0.$$
Отсюда следует, что $x \in \ha{\hc{e_k}}$, поэтому $x = \sum \xi_k e_k$. Подставим это выражение
в исходное уравнение, получим
$$\sums{k}\xi_ke_k - \sums{k}\hr{\sums{l}\xi_l e_l,e_k^*}e_k = 0.$$
Раскрываем скобочки, приводим подобные\ldots
$$\sums{k}\xi_ke_k - \sums{k,l}\xi_l(e_l,e_k^*)e_k = 0.$$
Приравнивая коэффициенты при $e_k$, получим систему уравнений:
$$\xi_k - \sums{l}\xi_l(e_l,e_k^*) = 0.$$
Возьмём исходное неоднородное уравнение, умножим его скалярно на $e_l^*$, получим
$$\br{(I-A)x, e_l^*} = (f, e_l^*).$$
$$(x,e_l^*) - \sums{k}(x,e_k^*)(e_k,e_l^*) = (f,e_l^*).$$
Это уже система относительно неизвестных $(x, e_k^*)$.

Совершенно аналогично можно всё написать для $A^*$, поскольку в силу
предложения имеем
$$A^*y = \sums{k}(y,e_k)e_k^*.$$

\comment{Да, это всё хорошо, но что из этого? Тут чего-то опять не хватает.}

\textbf{Общий случай}

Пусть компактный оператор $B$ представлен как $B = A + C$, где $A$\т оператор
с конечномерным образом, а $C$ таков, что $\hn{C} < 1$. Как мы знаем, для компактных операторов такое
разложение имеет место.

Заметим, что оператор $I - C$ обратим.
Исходное уравнение можно переписать как $(I - A - C)h = f$, откуда
$(I-C)h - Ah = f$, и, следовательно, $h - (I-C)^{-1}Ah = (I-C)^{-1}f$.
Понятно, что оператор $(I-C)^{-1}A$ тоже имеет конечномерный образ, поскольку таким
свойством обладает уже оператор $A$.

Рассмотрим $(I - B^*)y = g$, тогда $(I - A^* - C^*)y = g$, поэтому $(I - C^*)y - A^*y = g$.
Домножая на $(I - C^*)^{-1}$, получаем
\eqn{
\label{eqn:fredgolm.adjoint-eq}
y - (I - C^*)^{-1}A^*y = (I-C^*)^{-1}g.}
Но вот беда: оператор $(I - C^*)^{-1}A^*$ не является
сопряжённым к $(I-C)^{-1}A$. Пойдём другим путём.

Сделаем замену переменой: $(I - C)h = x$, тогда исходное уравнение перепишется в виде
$x - A(I-C)^{-1}x = f$. А вот это уже то, что нам надо, поскольку оператор в левой части является в точности
сопряжённым к оператору в правой части уравнения~\eqref{eqn:fredgolm.adjoint-eq}. Действительно,
$\br{A(I - C)^{-1}}^* = \br{(I-C)^{-1}}^*A^* = (I - C^*)^{-1}A^*$.


\subsubsection{Компактность операторов Гильберта\ч Шмидта}

\begin{theorem}
Интегральный оператор Гильберта\ч Шмидта $A$ с ядром $K$ компактен.
\end{theorem}
Будем приближать по норме ядро $K$ ступенчатыми функциями $K_\ep$ так, чтобы $\hn{K - K_\ep} < \ep$.
Покажем, что оператор с ядром $K_\ep$ будет конечномерным.

\comment{Почему верно всё то, что написано ниже? Тот факт, что интегрируемую функцию
можно приблизить в $L_2$ ступенчатыми, вопросов не вызывает, но почему прообразы
ступенек являются произведениями проекций на оси $X$ и $Y$? Ведь функция может быть очень поганая...
Итак, я думаю, что то, что написано ниже\т полный бред.}

Пусть $\hc{K_n}$\т последовательность ступенчатых ядер таких, что $K_n \ra K$ в $L_2$.
Представим $K_n$ в виде
$$K_n = \sum c_{ij} \chi_{\De_{ij}},$$
то есть в виде линейной комбинации ступенек\д индикаторов, причём $\De_{ij} = X_i \times Y_j$.

Имеем
$$\hn{K - K_n} \ge \hn{A_{K-K_n}} = \hn{A_K - A_{K_n}}.$$

Далее,
$$\int \sum \chi_{\De_{ij}} f(y)\,dy = \sum c_{ij} \chi_{X_i}\ints{Y_j} f(y)\,dy.$$

Заметим, что функция в правой части этого равенства является ступенчатой.
Но ясно, что пространство ступенчатых функций с фиксированным разбиением конечномерно.
Поэтому оператор $K_n$ имеет конечномерный образ.


\subsubsection{Теория Фредгольма: случай $\dim \Img < \infty$ и сведение общего случая к конечномерному}

Мы рассматривали уравнение
$$
  (I - A) u = f,
$$
где $A$\т компактный оператор. Представим $A$ в виде $A = A_1 + A_2$,
где $\hn{A_1} < 1$, а $\dim \Img A_2 < \infty$.

Перепишем наше уравнение в новых терминах.
Имеем
$$
  (I - A_1)u - A_2 u = f.
$$
Обозначим $x := (I - A_1)u$, то есть сделаем замену переменной.
Тогда $u = (I - A_1)^{-1}x$ (то, что оператор $I - A_1$ обратим, следует из того, что $\hn{A_1} < 1$).
В новой переменной уравнение перепишется следующим образом:
$$
x - A_2(I - A_1)^{-1}x = f,
$$
и, обозначая $B := A_2(I - A_1)^{-1}$, получаем следующее
уравнение
$$(I - B) x = f.$$

Поскольку наши рассуждения обратимы, получаем, что $f \in \Img(I - A)$ тогда и только тогда,
когда $f \in \Img(I - B)$.

Рассмотрим сопряжённое однородное уравнение $(I - A^*)v = 0$.
Рассуждая аналогично, получим уравнение $(I - B^*)y = 0$.
Действительно, распишем уравнение
$$(I - A^*)v = 0.$$
Имеем
$$(I - A_1^*)v  - A_2^*v = 0,$$
откуда
$$v - (I - A_1^*)^{-1}A_2^*v = 0.$$
Но, как легко видеть, оператор, стоящий во втором слагаемом\т это и есть $B^*$.
\comment{Это в силу правила $(AB)^* = B^*A^*$, которое ещё никто не отменял...} Окончательно,
$(I - B^*)v = 0$, откуда мы делаем вывод, что
$$\Ker(I - A^*) = \Ker(I - B^*).$$

Далее,
$$
  \dim \Ker(I - B) = \dim \Ker(I - B^*).
$$
\comment{Насколько я понимаю, далее следует объяснение, почему это верно.}
$$\Img(I - B)\oplus \Ker(I - B^*) = H = \Img(I - B^*) \oplus \Ker(I - B).$$
Ф2 говорит, что уравнение $(I - B)x = f$ имеет решение тогда и только тогда,
когда $f \bot \Ker(I - B^*)$.
\comment{Наверное, это одно и тоже, но формулировали мы её НЕ ТАК.}

Ф1 говорит, что уравнение $(I - B)x = f$ всегда имеет решение тогда и только тогда,
когда $\Ker (I - B) = 0$. Выведем её из Ф2 и Ф3 для гильбертовых пространств.
Если решение всегда есть, то образ есть всё пространство. Поэтому $\Ker(I - B^*) = 0$.
Но тогда и $\Ker(I - B) = 0$, поскольку их размерности совпадают.
\comment{Рассуждения, похоже, обратимы, поэтому в обратную сторону это тоже доказано.}

Докажем Ф2 и Ф3. Пусть $\hc{e_i}$\т базис в $\Img B$. Тогда
$$Bx = \sum(x,e_i^*)e_i =\sum\xi_i e_i,$$
то есть мы будем искать вектор в таком виде.

Из уравнения имеем
$$x = Bx + f = \sum \xi_i e_i + f = \sums{i}\hr{\sums{k} \xi_k e_k + f, e_i^*}e_i,$$
поскольку всякий вектор разлагается по базису. Приравнивая коэффициенты, получаем
$$\sums{k}(e_k, e_i^*)\xi_k + (f,e_i^*) = \xi_i \stackrel{!}{=} \sums{k}\de_{ki}\xi_k.$$
Вынесем отовсюду $\xi_k$, получим
\eqn{\label{eqn:fredgolm-patch-xi-system}
\sums{k}\hs{\de_{ki} - (e_k, e_i^*)}\xi_k = (f,e_i^*) =: \ph_i.}
Получилась линейная система.

Пусть
$$
  y = \sums{k} \eta_i e_i^*.
$$

Проделаем то же самое для $I - B^*$, получим
$$\sums{k}\hs{\de_{ki} - \ol{(e_i,e_k^*)}}\eta_k = 0,$$
откуда
\eqn{\label{eqn:fredgolm-patch-eta-system}
\sums{k}\hs{\de_{ki} - (e_k^*,e_i)}\eta_k = 0.}
Это ещё одна линейная однородная система, причём она сопряжена системе~\eqref{eqn:fredgolm-patch-xi-system}.
А из линейной алгебры известно, что размерности ядер обычной и сопряжённой системы совпадают.

\begin{note}
Чтобы это понять, достаточно заметить, что в ортонормированном базисе матрица сопряжённой системы
совпадает с звезданутой\footnote{Это синоним для <<транспонированной комплексно сопряжённой>>.}
матрицей исходной системы. Но никто не мешает нам выбрать в качестве $\hc{e_i}$ именно ортонормированный базис.
\end{note}

Покажем, что $f \bot \Ker(I - B^*)$. В самом деле,
$$(f,y) = \hr{f, \sums{i}\eta_i e_i^*} = \sums{i} \ol{\eta}_i(f,e_i^*) = \sums{i} \ph_i \ol{\eta}_i = 0,$$
\comment{Ничего не понимаю. Откуда следует последнее равенство?}
что и требовалось доказать.

\end{document}
