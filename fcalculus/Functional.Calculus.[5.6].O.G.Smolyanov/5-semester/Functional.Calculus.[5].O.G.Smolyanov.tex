\documentclass[10pt]{article}
\usepackage[russian]{babel}
\usepackage[utf8]{inputenc}
\usepackage{dmvn}
\usepackage[all]{xy}
\newcommand*{\p}[1]{#1\nobreak\discretionary{}{\hbox{$\mathsurround=0pt #1$}}{}}
\newcommand*{\pr}[1]{\mathop{\mathrm{pr_{#1}}}}
%\usepackage{style}
%\CompileMatrices

\begin{document}
\dmvntitle{Курс лекций}
{ по функциональному анализу}
{Лектор -- Олег Георгиевич Смолянов}
{III курс, 5 семестр, поток математиков}
{Москва, 2007г}


%% \vspace{40pt}}%
%% III курс, 5 семестр, 1 поток (2007 -- 2008 гг.)
%% \end{center}
%% \end{titlepage}
%---------------------------Lecture 1-----------------------------------------------%
\section*{Автор}
Летописец --- Павел Витальевич Бибиков {\normalfont \verb"<tsdtp4u@proc.ru>"}
(группа 303), телефон: 137-45-97.

\section{Метрические пространства}

\begin{df}
Пусть $E$ --- произвольное множество. \emph{Метрикой}
(\emph{расстоянием}) \emph{на $E$} называется функция $\rho\colon
E\times E\to \mathbb{R}^+$, обладающая следующим свойствам:

1) $\rho(x,z)\geqslant 0$, $\rho(x,z)=0$ $\Leftrightarrow$ $x=z$;

2) $\rho(x,z)=\rho(z,x)$;

3) $\rho(x,y)\leqslant\rho(x,z)+\rho(y,z)$.

\emph{Метрическое пространство} --- это пара $(E,\rho)$, где $\rho$
--- метрика на $E$.
\end{df}

\begin{ex}
   $E=\mathbb{R}^n$,
  $\rho(x,y)=\sqrt{\sum\limits_{j=1}^n(x_j-y_j)^2}$.
\end{ex}
\begin{ex}
   Пусть $\Omega$ --- произвольное множество. Положим
  $E=\mathcal{B}(\Omega)$ --- множество всех ограниченных функций на
  $\Omega$, а также
  $\rho(f,g)=\sup\limits_{\omega\in\Omega}|f(\omega)-g(\omega)|$.
\end{ex}
\begin{ex}
  $E=C[a;b]$, $\rho(f,g)=\max\limits_{x\in[a;b]}|f(x)-g(x)|$.
\end{ex}
\begin{ex}
   $E=\mathbb{Q}$. Определим \emph{$p$-адическую норму} следующим
  образом. Пусть $p$ --- фиксированное простое число. Рациональное
  число $0\neq r\in\mathbb{Q}$ представим в виде $r=p^\gamma\frac m
  n$, где $\gamma\in\mathbb{Z}$ и $(m;n)=1$. \emph{$p$-адической
    нормой} назовем величину $|r|_p=\frac{1}{p^\gamma}$, тогда
  $p$-\emph{адическая метрика} вводится следующим образом:
  $\rho_p(r_1,r_2)=|r_1-r_2|_p$. Заметим, что в этом случае аксиома 3
  выполнена в усиленной форме, а именно,
  $\rho(x,z)\leqslant\max(\rho(x,y);\rho(y,z))$. Метрические
  пространства с такими метриками называются
  \emph{ультраметрическими}.
\end{ex}
\begin{df}
Пусть $(E_1,\rho_1)$ и $(E_2,\rho_2)$ --- метрические пространства.
Их \emph{прямым произведением} называется метрическое пространство
$(E_1\times E_2,\rho)$, где метрика $\rho$ вводится так, чтобы она
индуцировала метрики $\rho_1$ и $\rho_2$ на пространствах $E_1$ и
$E_2$ соответственно.
\end{df}

\begin{note}
Для прямого произведения нет канонической метрики, т.е. метрику
$\rho$ можно задавать разными способами, например, следующими:
\begin{align*}
\rho((x_1;x_2);(y_1;y_2))&=\rho_1(x_1;y_1)+\rho_2(x_2;y_2);\\
\rho((x_1;x_2);(y_1;y_2))&=\max\{\rho_1(x_1;y_1);\rho_2(x_2;y_2)\};\\
\rho((x_1;x_2);(y_1;y_2))&=\sqrt{\rho_1^2(x_1;y_1)+\rho_2^2(x_2;y_2)}.
\end{align*}
\end{note}

\begin{df}
Пусть $(E,\rho)$ --- метрическое пространство и $G\subset E$. Тогда
пара $(G,\rho\mid_G)$ называется \emph{подпространством метрического
пространства $(E,\rho)$}.
\end{df}

\begin{df}
\emph{Открытым шаром с центром в точке $x\in E$ и радиусом $r>0$}
называется множество $S(x,r)=\{z\in E\mid\rho(x;z)<r\}$.

\emph{Замкнутым шаром с центром в точке $x\in E$ и радиусом
$r\geqslant0$} называется множество $F(x,r)=\{z\in
E\mid\rho(x;z)\leqslant r\}$.
\end{df}

\begin{df}\label{def.open}
Множество $G\subset E$ называется \emph{открытым}, если $G$ ---
объединение семейства открытых шаров, или, что все равно, $\forall\,
x\p\in G$ $\exists\,r>0: S(x,r)\subset G$. В частности, открытый шар
--- это открытое множество.
\end{df}

Пусть $\tau_\rho$ --- множество открытых подмножеств пространства
$(E,\rho)$. Это множество обладает следующими свойствами:

1) $\varnothing\in \tau_\rho$;

2) $E\in\tau_\rho$;

3) $\{V_\alpha\}\in\tau_\rho$ $\Rightarrow$ $\bigcup\limits_\alpha
V_\alpha\in\tau_\rho$;

4) $V_1$,\ldots,$V_n\in\tau_\rho$ $\Rightarrow$
$\bigcap\limits_{j=1}^nV_j\in\tau_\rho$.

\begin{df}
Множество $T$ называется \emph{топологическим простра\-нством}, если
в нем выделена совокупность $\tau$ подмножеств, обладающая
свойствами 1)--4). В этом случае $\tau$ называется \emph{топологией
на $T$}, а ее элементы --- \emph{открытыми множествами}.
\end{df}

\begin{note}
Любая метрика порождает топологию согласно
определению~\ref{def.open}, но не наоборот. В дальнейшем, если не
оговорено противное, мы будем считать, что в метрическом
пространстве введена именно такая топология.
\end{note}

\begin{problem}
Проверьте условия 1) -- 4) для системы открытых подмножеств
метрического пространства.
\end{problem}

\begin{problem}
Привести пример метрического пространства, в котором есть открытый
шар, являющийся замкнутым множеством, но не замкнутым шаром, и
пример метрического пространства, в котором есть замкнутый шар,
являющийся открытым множеством, но не открытым шаром.
\end{problem}

\begin{df}
Множество $F\subset E$ называется \emph{замкнутым}, если множество
$E\setminus F$ открыто. В частности, замкнутый шар --- замкнутое
множество.

\emph{Окрестностью точки} $x$ топологического пространства
называется всякое множество, содержащее открытое подмножество,
которому принадлежит точка $x$.
\end{df}

\begin{df}
\emph{Базой} (или \emph{фундаментальной системой}) окрестностей
точки $x$ называется такое множество $\mathcal{V}$ окрестностей
точки $x$, что $\forall\,G\ni x$ $\exists\,W\in\mathcal{V}:W\subset
G$, где $G$ --- открытое множество.
\end{df}

Рассмотрим примеры фундаментальных систем окрестностей точки $x$
метрического пространства.

\begin{ex}


  1. $\mathcal{V}=\{S(x,r)\mid r\in\mathbb{R}^+\}$.

  2. $\mathcal{V}=\{S(x,r)\mid r\in\mathbb{Q}^+\}$.

  3. $\mathcal{V}=\{S(x,1/n)\mid r\in\mathbb{N}\}$.

  4. $\mathcal{V}=\{F(x,1/n)\mid r\in\mathbb{N}\}$.
\end{ex}

\begin{df}
Пусть $A\subset E$ --- произвольное множество. Точка $x\p\in E$
называется \emph{точкой прикосновения множества $A$}, если для
каждой окрестности $V(x)$ точки $x$ имеем: $V(x)\cap
A\neq\varnothing$. \emph{Замыканием множества $A$} называется
множество всех его точек прикосновения. Обозначение: $\bar{A}$.

Точка $x\p\in E$ называется \emph{предельной точкой множества $A$},
если для каждой окрестности $V(x)$ точки $x$ пересечение $V(x)\cap
A$ бесконечно.
\end{df}

\begin{prop}
Множество $A$ замкнуто $\Leftrightarrow$ $A=\bar{A}$.
\end{prop}

\begin{proof}
1. Пусть $A$ замкнуто, тогда $E\setminus A$ открыто. Значит,
$E\setminus A$
--- окрестность точки $x$ для любой точки $x\in E\setminus A$,
кроме того, $(E\setminus A)\cap A=\varnothing$, а значит, $x$ не
точка прикосновения.

2. Обратно, пусть $A=\bar{A}$. Тогда каждая точка $x\in E\setminus
A$ --- не точка прикосновения. Поэтому существует окрестность $V(x)$
точки $x$, такая, что $V(x)\cap A=\varnothing$. Т.к. $x\in V(x)$, то
$\bigcup\limits_{x\in E\setminus A}V(x)=E\setminus A$. Значит,
множество $E\setminus A$ открыто, а множество $A$ замкнуто.
\end{proof}

\begin{df}
Последовательность $\{x_n\}$ \emph{сходится к $x$} (обозначение:
$x_n\to x$), если $\forall\, V(x)$ $\exists\,n:\forall\,k>n$ $x_k\in
V(x)$ (где $V(x)$ --- окрестность точки $x$). В метрическом
пространстве это эквивалентно следующему условию: $\rho(x_n;x)\to
0$.
\end{df}

\begin{df}
Пусть $E$ --- метрическое пространство. Последовательность
$\{x_n\}\subset E$ называется \emph{фундаментальной}, если
$\forall\,\varepsilon>0$ $\exists\,n:\forall\,k_1,k_2>n$
$\rho(x_{k_1};x_{k_2})<\varepsilon$. В произвольном топологическом
пространстве понятие фундаментальной последовательности не имеет
смысла.
\end{df}

\begin{note}
Любая сходящаяся последовательность фундаментальна. Однако обратное
неверно: например, это неверно в метрическом пространстве
$(\mathbb{Q},\rho_p)$.
\end{note}

\begin{df}
Метрическое пространство называется \emph{полным}, если любая его
фундаментальная последовательность сходится.

Например, пространство $\mathcal{B}(E)$ полно.
\end{df}

\begin{df}
\emph{Пополнением метрического пространства $(E,\rho)$} называется
полное метрическое пространство $(\bar{E},\bar{\rho})$, содержащее
$(E,\rho)$ в качестве подпространства и всюду плотного подмножества
(т.е. замыкание $E$ в $\bar{E}$ совпадает с $\bar{E}$).
\end{df}

\begin{df}
Пространства $(E,\rho_E)$ и $(G,\rho_G)$ называются
\emph{изоморфными}, если $\exists\,f\colon E\to G$, где $f$ ---
биекция, сохраняющая расстояния, т.е.
$\rho_G(f(x),f(z))=\rho_E(x,z)$.
\end{df}

%---------------------------Lecture 2-----------------------------------------------%
%\clearpage


\begin{theorem}
Всякое метрическое пространство $E$ обладает пополнением,
единственным с точностью до изоморфизма, тождественного на $E$.
\end{theorem}

\begin{proof}
Докажем сначала, что если $E$ полно и $A\subset E$ замкнуто, то
пространство $(A,\rho\mid_A)$ тоже полно\footnote{Если
$(A,\rho\mid_A)$ полно, то $A$ замкнуто в $E$, даже если $E$ не
является полным.}.

Пусть $\{x_n\}\subset A$ --- фундаментальная последовательность.
Тогда эта последовательность фундаментальна и в $E$, поэтому
$\exists\, x\in E:x_n\to x$. Т.к. $A=\bar{A}$, то $x\in A$, поэтому
$A$ полно.

Возьмем теперь пространство $\mathcal{B}(E)$ всех ограниченных
функций на $E$ с метрикой $\rho(f;g)=\sup\limits_{x\in
E}|f(x)-g(x)|$. Докажем, что оно полно. В самом деле, пусть
$\{f_n\}$ --- фундаментальная последовательность в $\mathcal{B}(E)$,
тогда при всех $z\in E$ последовательность $\{f_n(z)\}$ тоже
фундаментальна, т.к. для всех $z$ имеем: $\forall\,\varepsilon>0$\;
$\exists\,n:\forall\,k,r>n$
$$\varepsilon>\rho(f_k;f_r)=\sup\limits_{x\in
E}|f_k(x)-f_r(x)|\geqslant|f_k(z)-f_r(z)|.$$ Отсюда следует, что
$\forall\,z\in E$\;\;$\exists\,f(z)=\lim\limits_{n\to\infty}f_n(z)$,
поэтому, переходя в предыдущем неравенстве к пределу при
$r\to\infty$, при всех $z$ имеем:
$|f_k(z)\p-f(z)|\leqslant\varepsilon$. Значит,
$\rho(f_k;f)\p=\sup\limits_{z\in
E}|f_k(z)-f(z)|\leqslant\varepsilon$, и пространство
$\mathcal{B}(E)$ полно.

Вложим $E$ в $\mathcal{B}(E)$. А именно, пусть $x_0\in E$, тогда
$E\ni x_1\mapsto f_{x_1}\p\in\mathcal{B}(E)$, где
$f_{x_1}(x)=\rho(x;x_1)-\rho(x;x_0)$. Понятно, что
$f_{x_1}\in\mathcal{B}(E)$, т.к. по неравенству треугольника
$|f_{x_1}(x)|\leqslant\rho(x_0;x_1)$.

Проверим, что
$\rho_E(x_1;x_2)=\rho_{\mathcal{B}(E)}(f_{x_1};f_{x_2})$. В самом
деле, $$\rho_{\mathcal{B}(E)}(f_{x_1};f_{x_2})=\sup\limits_{x\in
E}|\rho(x;x_1)-\rho(x;x_2)|\leqslant\rho(x_1;x_2).$$ Кроме того,
равенство достигается (например, при $x=x_1$).

Таким образом, можно считать, что $(E,\rho_E)$ --- это
подпространство в $(\mathcal{B}(E), \rho_{\mathcal{B}(E)})$.
Рассмотрим $\bar{E}$, тогда $E\subset \bar{E}$ и $\bar{E}$ полно.
Понятно, что $\bar{E}$
--- это искомое пополнение. Докажем, что оно единственно (с точностью
до изоморфизма, тождественного на $E$).

Пусть $G$ --- другое пополнение $E$. Докажем, что $\exists\, F\colon
\bar{E}\to G$, причем $F(x)=x$ при всех $x\in E$. Для каждой точки
$z\in\bar{E}$ найдется такая последовательность $\{x_n\}\subset E$,
что $x_n\to z$. Тогда $\{x_n\}$ фундаментальна в $\bar{E}$ и в $E$,
а значит, и в $G$. Поэтому
$\exists\,\lim\limits_{n\to\infty}F(x_n)=F(z)$.

\begin{problem}
Докажите корректность определения функции $F$, т.е. тот факт, что
она не зависит от выбора последовательности, сходящейся к $z$.
\end{problem}

Если $z_1,z_2\in\bar{E}$, то
$\rho_G(F(z_1);F(z_2))=\lim\limits_{n\to\infty}\rho_E(x_n^1;x_n^2)=\rho_E(z_1;z_2)$
(поскольку расстояние непрерывно по совокупности аргументов в силу
неравенства треугольника). Таким образом, $F$ --- изометрия и
отображение <<на>> (т.к. $G$ тоже полно и $E$ плотно в $G$).
\end{proof}

\begin{df}
\emph{Диаметром множества $F$} называется величина $\diam
F=\sup\limits_{x,y\in F}\rho(x;y)$.
\end{df}

\begin{theorem}[Теорема о вложенных шарах]
Пусть $E$ полно и $\{F_j\}$ --- такая последовательность замкнутых
множеств, что $E\supset F_1\p\supset F_2\supset\ldots\supset
F_j\supset\ldots$ и $\diam F_j\to 0$. Тогда $\bigcap\limits_j
F_j\neq\varnothing$.\footnote{На самом деле, сформулированное
предложение является критерием полноты метрического пространства.}
\end{theorem}

\begin{proof}
Пусть $x_j\in F_j$, тогда последовательность $\{x_j\}$
фундаментальна, а значит, $\exists\,\lim\limits_{j\to\infty}x_j=x$.
Т.к. $F_j$ замкнуты, то $\forall\, j\;x\in F_j$, а значит,
$x\in\bigcap\limits_j F_j$.
\end{proof}

\begin{theorem}[Бэр]\label{th.Ber}
Пусть $E$ полно и $E=\bigcup\limits_{n=1}^\infty F_n$, где множества
$F_n$ замкнуты. Тогда $\exists\,n:F_n\supset S(x,r)$.
\end{theorem}

\begin{proof}
Предположим противное. Возьмем шар $S(x_0,1)$. Тогда множество
$S(x_0,1)\setminus F_1$ открыто и непусто, а значит, $\exists\,
S(x_1,r_1)\subset S(x_0,1)\p\setminus F_1$ (где $r_1<1/2$), а
значит, $S(x_0,r_1/2)\subset S(x_0,1)\p\setminus F_1$. Тогда
$S(x_1,r_1/2)\p\setminus F_2\neq\varnothing$, поэтому
$\exists\,S(x_2,r_2)\subset S(x_1,r_1/2)\setminus F_2$ (где
$r_2<1/4$). Продолжая, получаем последовательность вложенных шаров:
$$S(x_0,1)\supset S(x_1,r_1)\supset S(x_1,r_1/2)\supset
S(x_2,r_2)\supset\ldots,$$ причем $S(x_n,r_n)\cap F_n=\varnothing$,
поэтому $\bar{S}(x_n,r_n/2)\cap F_n=\varnothing$. Таким образом, мы
получаем последовательность замкнутых вложенных шаров
$\bar{S}(x_1,r_1/2)\supset \bar{S}(x_2,r_2/2)\supset \ldots$, причем
$\diam \bar{S}(x_n,r_n/2)\to0$ (т.к. $r_n\p<2^{-n}$). По предыдущему
следствию $\bigcap\limits_{n=1}^\infty
\bar{S}(x_n,r_n/2)\neq\varnothing$ и не содержится в $F_i$ при всех
$i$ --- противоречие.
\end{proof}

Топологическое пространство, в котором справедлива
теорема~\ref{th.Ber}, называется \emph{бэровским}. Т.о., всякое
полное метрическое пространство --- бэровское.

\begin{problem}
Привести пример неполного метрического пространства, являющегося тем
не менее бэровским.
\end{problem}

%\clearpage

\section{Компактность}

\begin{df}
Подмножество $K$ топологического пространства $E$ называется
\emph{компактным}, если из любого его покрытия открытыми множествами
можно извлечь конечное подпокрытие.
\end{df}
\begin{df}
  Подмножество $K$ топологического пространства $E$ называется
  \emph{относительно компактным}, если множество $\bar{K}$ компактно.
\end{df}

\begin{df}
  Множество $K$ топологического пространства $E$ называется
  \emph{счетно-компактным}, если всякое бесконечное подмножество $K$
  имеет предельную точку из $K$.
\end{df}

\begin{df}
  Множество $K$ топологического пространства $E$ называется
  \emph{сек\-вен\-ци\-аль\-но-компактным}, если из любой
  последовательности из $K$ можно выделить подпоследовательность,
  сходящуюся к элементу из $K$.
\end{df}

\begin{df}
  Множество $K$ метрического пространства $E$ называется
  \emph{предкомпак\-тным} (или \emph{вполне ограниченным}), если
  $\forall\,\varepsilon>0$
  $\exists\,S(x_j,\varepsilon):\bigcup\limits_{j=1}^n
  S(x_j,\varepsilon)\p\supset K$. В этом случае последовательность
  $\{x_j\}$ называется \emph{$\varepsilon$-сетью}.
\end{df}


\begin{problem}
Очевидно, что из вполне ограниченности следует ограниченность.
Покажите, что обратное неверно.
\end{problem}
%---------------------------Lecture 3-----------------------------------------------%


\begin{theorem}
1. В топологическом пространстве счетная компактность следует из
секвенциальной компактности, а также из компактности.

2. В метрическом пространстве компактность равносильна
секвенциальной компактности, а также счетной компактности, а также
предкомпактности и полноте одновременно.
\end{theorem}

\begin{proof}
Сначала докажем п.1.

Докажем, что из секвенциальной компактности следует счетная
компактность. Пусть $K$ --- секвенциально компактное множество в
топологическом пространстве, и $A$ --- его бесконечное подмножество.
Возьмем бесконечную последовательность $\{x_n\}\subset A$ и выделим
из нее сходящуюся подпоследовательность: $\exists\, x_{n_k}\to x\in
K$. Значит, $x$ --- предельная точка множества $A$.

Теперь докажем, что из компактности следует секвенциальная
компактность. Предположим противное: пусть $A\subset K$ ---
бесконечное множество, у которого в $K$ нет предельных точек.
Значит, $\forall\,x\in K$\;\;$\exists\,V(x):|V(x)\cap
A|<\infty$.\footnote{Если $M$ --- множество, то $|M|$ --- число его
элементов.} Т.к. $K\subset \bigcup\limits_x V(x)$, то
$\exists\,x_1,\ldots,x_n:\bigcup\limits_{k=1}^nV(x_k)\supset
K\supset A$ --- противоречие, т.к. $A$ бесконечно.

Теперь докажем п.2.

 Докажем, что из счетной компактности следует
секвенциальная компактность. Пусть $\{x_n\}\subset K\subset E$.
Возможны два случая.

1) Множество различных элементов последовательности $\{x_n\}$
конечно. Тогда найдется точка $x\in\{x_n\}$, встречающаяся
бесконечное количество раз, а значит, подпоследовательность
${x_{n_j}}$, где $x_{n_j}=x$, будет сходиться к точке $x$.

2) Множество различных элементов последовательности $\{x_n\}$
бесконечно. Тогда найдется предельная точка $z\in K$ этой
последовательности. А значит,
$\forall\,k\in\mathbb{N}$\;\;$\exists\,x_{n_k}\in S(z,1/k)$. Можно
считать, что $n_1<n_2<\ldots$, поэтому $x_{n_k}\to z$ (т.к.
$\rho(x_{n_k};z)\to 0$).

Докажем теперь, что из секвенциальной компактности следуют полнота и
предкомпактность.

1) Пусть $K$ секвенциально компактно и $K\supset\{x_n\}$ ---
фундаментальная последовательность. Тогда $\exists\,x_{n_j}\to x\in
K$. Докажем, что в этом случае $x_n\to x$. В самом деле, для любого
$\varepsilon>0$ имеем:
$$\exists\,k_0:\forall\,m,r>k_0\;\;\rho(x_m;x_r)<\varepsilon/2,$$
а также
$$\exists\,j_0:\forall\,j>j_0\;\;\rho(x_{n_j};x)<\varepsilon/2.$$
Среди таких $j$ найдется $j_1$, для которого $n_{j_1}>k_0$. Тогда
$$\forall\,m>k_0\;\;\rho(x_m;x)\leqslant
\rho(x_m;x_{n_{j_1}})+\rho(x_{n_{j_1}};x)<\varepsilon.$$

2) Пусть теперь $K$ секвенциально компактно и не предкомпактно, т.е.
$\exists\,\varepsilon>0:\forall\,x_1,\ldots,x_n\in
K$\;\;$\bigcup\limits_{j=1}^n S(x_j,\varepsilon)\not\supset K$. Но
тогда найдется бесконечное множество элементов $\{z_j\}\subset K$,
таких, что
$\forall\,j_1,j_2$\;\;$\rho(z_{j_1};z_{j_2})\geqslant\varepsilon$, а
значит, из последовательности $\{z_j\}$ нельзя выделить
фундаментальную подпоследовательность --- противоречие.

Наконец, докажем, что из полноты и предкомпактности следует
компактность. Предположим противное: Пусть $K$ не компактно, тогда
$\exists\,V_\alpha: \bigcup\limits_\alpha V_\alpha\supset K$ и
нельзя выделить конечное подпокрытие. Т.к. $K$ --- предкомпактно, то
$\forall\,k>0$\;\;$\exists\,x_1^k,\ldots,x_{n(k)}^k:\bigcup\limits_
{j=1}^{n(k)}(F(x_j^k,1/k)\cap K)\supset K$ (как обычно, $F(x,r)$
--- это замкнутый шар радиуса $r$ с центром в точке $x$), а значит,
$\exists\,F(x_{j_1}^1,1):F(x_{j_1}^1,1)\cap K$ не покрывается
конечным числом $\{V_\alpha\}$. Аналогично, при $k=2$ имеем:
$$\bigcup\limits_{j=1}^{n(2)} (F(x_j^2,1/2)\cap F(x_{j_1}^1,1)\cap K)\supset F(x_{j_1}^1,1)
\cap K,$$ а значит, $\exists\,F(x_{j_2}^2,1/2):F(x_{j_2}^2,1/2)\cap
F(x_{j_1}^1,1)\cap K$ не покрывается конечным числом $\{V_\alpha\}$.
Проводя аналогичные рассуждения, получаем последовательность
замкнутых вложенных множеств $\bigcap\limits_{n=1}^r
F(x^n_{j_n},1/n)\cap K$, диаметры которых стремятся к 0. Т.к. $K$
полно, то $\exists\,x\in\bigcap\limits_{n=1}^\infty
F(x_{j_n}^n,1/n)\cap K$. Но тогда $\exists\,V_{\alpha(x)}\ni x$ и
$\exists\,\varepsilon(x): S(x,\varepsilon(x))\subset V_{\alpha(x)}$,
поэтому $\exists\,n: 1/n\p<\varepsilon(x)/2$ и
$F(x_{j_n}^n,1/n)\subset S(x,\varepsilon(x))\subset V_{\alpha(x)}$
--- противоречие.
\end{proof}

\begin{df}
Пусть $(E,\rho)$ --- метрическое пространство. Отображение $f\colon
E\to E$ называется \emph{сжимающим}, если
$\exists\,\alpha\in(0;1):\forall\,x_1,x_2\in E$ \;\;
$\rho(f(x_1);f(x_2))\leqslant\alpha\rho(x_1;x_2)$.
\end{df}

Всякое сжимающее отображение непрерывно (проверьте это).

\begin{theorem}[Пикар]
Всякое сжимающее отображение $f$ полного ме\-трического пространства
$(E,\rho)$ в себя обладает ровно одной неподвижной точкой.
\end{theorem}

\begin{proof}
Пусть $x_0\in E$. Рассмотрим последовательность $\{x_n\}$, где
$x_{n+1}=f(x_n)$. Легко видеть, что она фундаментальна:
$$\rho(x_{n};x_{n+1})=\rho(f(x_{n-1});f(x_n))\leqslant\alpha\rho(x_{n-1};n_n)
\leqslant\ldots\leqslant\alpha^n\rho(x_0;x_1),$$ поэтому
$$\rho(x_{n+k};x_n)\leqslant\rho(x_n;x_{n+1})+\ldots+\rho(x_{n+k-1};x_{n+k})
\leqslant\frac{\alpha^{n+k}}{1-\alpha}\rho(x_0;x_1)\to 0.$$ Значит,
$\exists\,z\in E:z=\lim\limits_{n\to\infty} x_n$. Докажем, что точка
$z$ неподвижна. В самом деле, $$\rho(z;f(z))=\rho(\lim x_n;\lim
f(x_n))=\rho(\lim x_n;\lim x_{n+1})=\lim\rho(x_n;x_{n+1})=0,$$ так
что $z=f(z)$.

Докажем, что неподвижная точка единственна. Пусть $z_1$ и $z_2$ ---
неподвижные точки. Тогда
$$\rho(z_1;z_2)=\rho(f(z_1);f(z_2))\leqslant\alpha\rho(z_1;z_2),$$
откуда $(1-\alpha)\rho(z_1;z_2)=0$, а значит, $\rho(z_1;z_2)=0$ и
$z_1=z_2$.
\end{proof}

\begin{problem}
Привести пример неполного метрического пространства, в котором
теорема Пикара неверна.
\end{problem}

\begin{problem}
Привести пример неполного метрического пространства, в котором
теорема Пикара верна.
\end{problem}

\begin{problem}
Доказать, что отображение $f$ метрического пространства $E$, для
которого $\rho(f(z_1);f(z_2))<\rho(z_1;z_2)$, может не иметь
неподвижной точки, даже если пространство $E$ полно.
\end{problem}

\begin{problem}
Если $E$ компактно, то отображение $f$, для которого
$\rho(f(z_1);f(z_2))<\rho(z_1;z_2)$, обладает ровно одной
неподвижной точкой.
\end{problem}
%---------------------------Lecture 4-----------------------------------------------%
\clearpage



\vspace{-20pt}

\section{Непрерывность}

\begin{df}
Отображение $f\colon E\to G$ называется \emph{непрерывным в точке
$x$}, если для каждой окрестности $V(f(x))$ точки $f(x)$ существует
такая окрестность $W(x)$ точки $x$, что $f(W(x))\p\subset V(f(x))$.
\end{df}

\begin{df}
Отображение $f\colon E\to G$ называется \emph{непрерывным на
множестве $E$}, если $f$ непрерывно во всех точках множества $E$.
\end{df}

\begin{prop}
Отображение $f\colon E\to G$ непрерывно на $E$ $\Leftrightarrow$
прообраз любого открытого множества в $G$ открыт в $E$.
\end{prop}

\begin{proof}
Пусть множество $V\subset G$ --- открытое, и $x\in f^{-1}(V)$. Тогда
по определению непрерывности отображения $f$ в точке $x$ имеем:
$\forall\,V(f(x))$\;\;$\exists\,W(x):f(W(x))\subset V(f(x))$.
Значит, $W(x)\subset f^{-1}(V)$ и $f^{-1}(V)$ представляется в виде
объединения открытых множеств, а именно,
$f^{-1}(V)=\bigcup\limits_{x\in f^{-1}(V)}W(x)$.

Докажем обратное утверждение. Пусть $x\in E$ и $V(f(x))$ ---
произвольная открытая окрестность точки $f(x)$ в $G$. Тогда
множество $W\p=f^{-1}(V(f(x)))$ открыто и $x\in W$. Значит,
$f(W)=V(f(x))$, что и доказывает непрерывность отображения $f$ в
точке $x$.
\end{proof}

\begin{prop}\label{predl.zamk->zamk}
Отображение $f$ непрерывно $\Leftrightarrow$ прообраз любого
замкнутого множества в $G$ замкнут в $E$.
\end{prop}

\begin{proof}
Это утверждение следует из предыдущего предложения и следующей
выкладки: $$f^{-1}(F)=f^{-1}(G\setminus (G\setminus F))=(f^{-1}(G))
\setminus(f^{-1}(G\setminus F))$$ (т.к. множество $f^{-1}(G\setminus
F)$ открыто).
\end{proof}

\begin{df}
Пусть $E$ --- топологическое пространство. В точке $x\in E$
выполняется \emph{первая аксиома счетности}, если существует не
более чем счетная фундаментальная система окрестностей точки $x$.

Пространство, в каждой точке которого выполнена первая аксиома
счетности, называется \emph{пространством с первой аксиомой
счетности}.
\end{df}

\begin{theorem}
Отображение $f$ топологического пространства с первой аксиомой
счетности в топологическое пространство непрерывно в точке $x\in E$
$\Leftrightarrow$ $\forall\,\{x_n\}\p\subset E:x_n\to x$
$\Rightarrow$ $f(x_n)\to f(x)$.
\end{theorem}

\begin{proof}
Пусть отображение $f$ непрерывно и $x_n\to x$. Тогда
$\forall\,W(f(x))$\;\;$\exists\, V(x):f(V(x))\subset W(f(x))$.
Поскольку $\exists\,n_0:\forall\,n>n_0$\;\;$x_n\p\in V(x)$, то
$f(x_n)\in W(f(x))$, так что $f(x_n)\to f(x)$. (В этой части первая
аксиома счетности не используется.)

Докажем утверждение в другую сторону. Предположим противное: пусть
$\exists\, W(f(x)):\forall\,V(x)$\;\;$\exists\,z\in f(V(x)):z\not\in
W(f(x))$. Т.к. $\forall\,n$\;\;$\exists\,x_n:x_n\p\in V(x)$ и
$f(x_n)=z_n\not\in W(f(x))$, то $x_n\to x$, но $z_n\nrightarrow
f(x)$ --- противоречие.
\end{proof}

\begin{prop}
Пусть $E$ и $G$ --- топологические пространства, и отображение
$f\colon E\to G$ непрерывно и $K\subset E$ --- компакт. Тогда $f(K)$
--- компакт
\end{prop}

\begin{proof}
В самом деле, если $\bigcup\limits_\alpha W_\alpha\supset f(K)$, то
$\bigcup\limits_{\alpha}f^{-1}(W_\alpha)\p\supset
f^{-1}(f(K))\supset K$. Т.к. $K$ --- компакт, то
$\exists\,\{W_{\alpha_j}\}:\bigcup\limits_{j=1}^n
f^{-1}(W_{\alpha_j})\supset K$. Но в таком случае
$\bigcup\limits_{j=1}^n W_{\alpha_j}\supset f(K)$, что и
требовалось.
\end{proof}

\begin{df}
Топологическое пространство называется \emph{хаусдорфовым}, если у
любых двух его точек есть непересекающиеся окрестности.
\end{df}

\begin{lemma}\label{lemm.hau.komp->zamk}
Пусть $G$ --- хаусдорфово пространство и $K\subset G$ --- компакт.
Тогда $K$ замкнуто.
\end{lemma}

\begin{proof}
$\forall\,z\in K$\;\;$\exists\,V(z):V(z)\cap V_z(x)=\varnothing$,
где $x\not\in K$. Поскольку $\bigcup\limits_{z\in K}V(z)\supset K$,
то $\exists\,\{V(z_j)\}:\bigcup\limits_{j=1}^n V(z_j)\supset K$.
Поскольку множество $\bigcap\limits_{j=1}^n V_{z_j}(x)=W(x)$ открыто
и $W(x)\cap K=\varnothing$, то $K$ замкнуто.
\end{proof}

\begin{prop}
Пусть $f\colon E\to G$ --- непрерывная биекция, $E$ компактно, а $G$
хаусдорфово. Тогда $G$ тоже компактно и отображение $f^{-1}$ тоже
непрерывно.
\end{prop}

\begin{proof}
Утверждение следует из предложения \ref{predl.zamk->zamk} и леммы
\ref{lemm.hau.komp->zamk}, поскольку $(f^{-1})^{-1}(F)=f(F)$ ---
замкнутое в $G$ множество (т.к. если $F\subset E$ замкнуто в
компакте, то и само $F$ компакт).
\end{proof}

\begin{df}
Пусть $(E,\rho_E)$ и $(G,\rho_G)$ --- метрические пространства.
Отображение $f\colon E\to G$ \emph{равномерно непрерывно}, если
$\forall\,\varepsilon>0$\;\;$\exists\,\delta>0:\forall\, x_1,x_2\in
E$\;\;$\rho(x_1;x_2)<\delta\Rightarrow\rho(f(x_1);f(x_2))<\varepsilon$.
\end{df}

\begin{prop}
Непрерывное отображение компактного метрического пространства в
произвольное метрическое пространство равномерно непрерывно.
\end{prop}

\begin{proof}
Если отображение $f$ не является равномерно непрерывным, то
$\exists\,\varepsilon>0:\forall\,n\in\mathbb{N}\;\exists\,x_n, z_n:
\rho(x_n;z_n)<1/n$, но $\rho(f(x_n);f(z_n))\p>\varepsilon$. Пусть
$\{x_{n_k}\}$ --- сходящаяся подпоследовательность
последовательности $\{x_n\}$, т.е. $x_{n_k}\to x$. Тогда $z_{n_k}\to
x$, т.к. $\rho(x_{n_k};z_{n_k})\to 0$. Поэтому последовательность
$x_{n_1}$, $z_{n_1}$, $x_{n_2}$, $z_{n_2}$,\ldots тоже сходится к
$x$. Но последовательность $f(x_{n_1})$, $f(z_{n_1})$, $f(x_{n_2})$,
$f(z_{n_2})$,\ldots даже не является фундаментальной и потому
сходиться не может.
\end{proof}

%\clearpage

\section{Нормированные пространства}

\begin{df}
Пусть $E$ --- векторное пространство (над $\mathbb{R}^1$ или
$\mathbb{C}^1$). Функция $p\colon E\to\mathbb{R}^1$ называется
\emph{полунормой на $E$}, если выполнены следующие аксиомы:

1) $p(x)\geqslant0$;

2) $p(\alpha x)=|\alpha|p(x)$;

3) $p(x_1+x_2)\leqslant p(x_1)+p(x_2)$.

Если аксиому 1) усилить, а именно, потребовать к тому же, чтобы
$p(x)=0$ тогда и только тогда, когда $x=0$, то функция $p$ будет
называться \emph{нормой на $E$}.
\end{df}

Во всяком нормированном пространстве вводится расстояние с помощью
равенства $\rho(x,z)=p(x-z)$.

\begin{df}
\emph{Локально выпуклое пространство} --- это пара
$(E,\mathcal{P})$, где $\mathcal{P}$ --- семейство полунорм на $E$.

\emph{Нормированное пространство} --- это пара $(E,p)$, где $p$ ---
норма на $E$. Нормированное пространство наделяется
\emph{канонической метрикой}: $\rho(x_1;x_2)=p(x_1-x_2)$. Если
полученное метрическое пространство будет полным, то нормированное
пространство $E$ называется \emph{банаховым}.
\end{df}

\begin{ex}
  1. $E=\mathbb{R}^1$, $\|x\|=|x|$.

  2. $E=C[a;b]$, $\|f\|=\max\limits_{t\in[a;b]}|f(t)|$.

  3. $E=C_2[a;b]$,
  $\|f\|=\Big(\int\limits_a^b\!|f(t)|^2\,dt\Big)^{1/2}$.

  4. $E=c_0$ --- пространство всех последовательностей, сходящихся к
  0, $\|\{x_n\}\|=\max\limits_n|x_n|$.

  5. $E=l_\infty$ --- пространство всех ограниченных
  последовательностей, $\|\{x_n\}\|=\sup\limits_n|x_n|$.
\end{ex}
\begin{problem}
Докажите, что нормированные пространства в примерах 1, 2, 4, 5
банаховы, а в примере 3 нет.
\end{problem}

\begin{df}
Пусть $f\colon E\to G$ --- линейное непрерывное отображение.
\emph{Нормой $f$} называется величина
$\|f\|=\sup\limits_{\|x\|_E\leqslant1}\|f(x)\|_G$. В случае, когда
$G=\mathbb{R}^1$, отображение $f$ называется \emph{линейным
функционалом}. Множество всех непрерывных функционалов на
пространстве $E$ образуют линейное пространство (нормированное),
которое называется \emph{сопряженным к $E$}. Обозначение --- $E^*$.
\end{df}

%---------------------------Lecture 5-----------------------------------------------%


\begin{df}
Пусть $E$ --- нормированное пространство. Линейное отображение
$A\colon E\to G$ называется \emph{ограниченным}, если образ любого
ограниченного множества ограничен.

Множество называется \emph{ограниченным}, если оно содержится в
некотором шаре.

\emph{Нормой отображения $A$} называется величина
$\|A\|=\sup\limits_{\|x\|\leqslant 1}\|Ax\|$.
\end{df}

\begin{prop}
$\|A\|=\sup\limits_{\|x\|=1}\|Ax\|=\sup\limits_{x\neq
0}\frac{\|Ax\|}{\|x\|}$.
\end{prop}

\begin{problem}
Докажите это предложение.
\end{problem}

Рассмотрим пространство $\mathcal{L}(E,G)$ всех непрерывных линейных
отображений из $E$ в $G$. Введенная выше функция $\|\cdot\|$
действительно является нормой. Проверим, например, неравенство
треугольника. Имеем:
\begin{align*}
\|A_1+A_2\|&=\sup\limits_{\|x\|_E\leqslant 1}\|(A_1+A_2)x\|_G
\leqslant \sup\limits_{\|x\|_E\leqslant
1}(\|A_1x\|_G+\|A_2x\|_G)\leqslant
\\&\leqslant\sup\limits_{\|x\|_E\leqslant
1}\|A_1x\|_G+\sup\limits_{\|x\|_E\leqslant
1}\|A_2x\|_G=\|A_1\|+\|A_2\|.
\end{align*}

\begin{prop}
Если отображение $A$ линейно, от его ограниченность равносильна
непрерывности.
\end{prop}

\begin{proof}
Пусть $A$ ограничено, тогда $A(S(0,1))\subset S(0,r)$, поэтому
$\forall\,\varepsilon$\;\;$A(S(0,\varepsilon/r))\subset
A(0,\varepsilon)$.

Обратно, пусть $A$ непрерывно, тогда
$\forall\,r>0$\;\;$\exists\,\varepsilon
>0:A(S(0,\varepsilon))\p\subset S(0,r)$, а значит, $A(S(0,1))\subset
S(0,r/\varepsilon)$ и $\|A\|\leqslant r/\varepsilon$.
\end{proof}

\begin{prop}
Пусть $A\in\mathcal{L}(E,G)$. Тогда $\|Ax\|\leqslant \|A\|\|x\|$ и
$\|A\|\p=\inf\{M>0\mid \forall\,x\;\;\|Ax\|\leqslant M\|x\|\}$.
\end{prop}

\begin{proof}
Т.к. $\|A\|=\sup\limits_{\|x\|\leqslant 1}\|Ax\|$, то
$\forall\,x\neq 0$\;\;$\Big\|\frac{Ax}{\|x\|}\Big\|\leqslant\|A\|$,
откуда $\|Ax\|\leqslant\|A\|\|x\|$. Поэтому, если $M_0=\inf\{M>0\mid
\forall\,x\;\;\|Ax\|\leqslant M\|x\|\}$, то $M_0\leqslant\|A\|$. Но
если $M_0<\|A\|$, то
$\exists\,\varepsilon>0:M_1=M_0+\varepsilon<\|A\|$. Тогда
$\forall\,x\neq 0$\;\;$\frac{\|Ax\|}{\|x\|}\leqslant M_1$, а значит,
$\|A\|=\sup\limits_{x\neq0}\frac{\|Ax\|}{\|x\|}\leqslant M_1<\|A\|$
--- противоречие. Т.о., $\|A\|=M_0$.
\end{proof}

\begin{theorem}
Если $G$ --- банахово пространство, а $E$ --- нормированное
пространство, то пространство $\mathcal{L}(E,G)$ банахово.
\end{theorem}

\begin{proof}
Пусть $\{A_n\}\subset\mathcal{L}(E,G)$ --- фундаментальная по норме
последовательность. Тогда $\forall\,x\in
E$\;\;$\|A_nx-A_kx\|\leqslant \|A_n-A_k\|\|x\|$, поэтому при всех
$x\in E$ последовательность $\{A_nx\subset G\}$ фундаментальна, а
значит, $\forall\,x\in E$\;\;$\exists\,\lim\limits_{n\to\infty}
A_nx=Ax$. Докажем, что $A\in\mathcal{L}(E,G)$. В самом деле,
понятно, что $A$ линейно в силу линейности предела и отображений
$A_n$, поэтому необходимо доказать только непрерывность.

$\forall\,\varepsilon>0$\;\;$\exists\,n_0:\forall\,n,k>n_0$\;\;
$\|A_n-A_k\|<\varepsilon$, поэтому
$\forall\,x$\;\;$\|A_nx-A_kx\|\p\leqslant\varepsilon\|x\|$ и
$\|A_n-A\|\leqslant\varepsilon$. Отсюда следует, что функционал
$A_n-A$ непрерывен. Но функционал $A_n$ также непрерывен, поэтому
$A=A_n-(A-A_n)$ тоже будет непрерывным. Кроме того, понятно, что
функционал $A$ --- предел последовательности $\{A_n\}$, т.к.
$\|A_n-A\|\to 0$.
\end{proof}

В частности, при $G=\mathbb{R}^1$ получаем, что пространство $E^*$
всегда банахово (в силу полноты пространства $\mathbb{R}^1$).

\begin{df}
Множество $E$ называется \emph{выпуклым}, если
$\forall\,x_1,x_2\p\in
E,\;\;\forall\,\tau_1,\tau_2\geqslant0:\tau_1+\tau_2=1$\;\;$\tau_1x_1+\tau_2x_2\in
E$.
\end{df}

\begin{theorem}[Банах -- Штейнхаус]
Пусть $E$ полно, $G$ нормировано и
$\{A_\alpha\}\subset\mathcal{L}(E,G)$ и
$$\forall\,x\in E \quad \sup\limits_\alpha\|A_\alpha x\|_G<\infty$$
Тогда $\sup\limits_\alpha\|A_\alpha\|<\infty$.
\end{theorem}

\begin{proof}
Для каждого натурального $n$ рассмотрим множество $M_n=\{x\in E\mid
\forall\,\alpha\;\;\|A_\alpha x\|\leqslant n\}$. Тогда
$\bigcup\limits_{n=1}^\infty M_n=E$. Представим множества $M_n$ в
следующем виде: $M_n=\bigcap\limits_\alpha \{x\in E\mid \|A_\alpha
x\|\leqslant n\}\p=\bigcap\limits_\alpha A^{-1}_\alpha(F(0,n))$.
Т.к. $A_\alpha$ непрерывны, то множества $M_n$ замкнуты, и по
теореме Бэра $\exists\,n:M_n\supset S(z,r)$.

Множество $M_n$ выпукло, содержит шар $S(z,r)$ и симметрично
относительно точки 0. Т.к. $M_n$ симметрично, то $M_n\supset
S(-z,r)$, а т.к. $M_n$ выпукло, то $M_n\supset \frac 1 2
S(-z,r)+\frac 1 2 S(z,r)=S(0,r)$. Т.о., $M_n$ содержит шар $S(0,r)$
радиуса $r$ с центром в 0. Отсюда следует, что
$\forall\,\alpha$\;\;$\forall\,x:\|x\|\leqslant r$ $\Rightarrow$
$\|A_\alpha x\|\leqslant n$, поэтому $\sup\limits_{\|x\|\leqslant
r}\|A_\alpha x\|\leqslant n$, т.е.
$\forall\,\alpha$\;\;$\|A_\alpha\|\leqslant \frac n r$.
\end{proof}
%---------------------------Lecture 6-----------------------------------------------%


\begin{theorem}[Хан--Банах]
Пусть $E$ --- произвольное линейное пространство, и $p\colon
E\to\mathbb{R}^1$
--- такая функция на нем, что выполняются следующие свойства:

1) $p(\alpha x)=\alpha p(x)$;

2) $p(x_1+x_2)\leqslant p(x_1)+p(x_2)$.

Пусть также $E_1\subset E$ --- подпространство и $f\colon
E_1\to\mathbb{R}^1$ --- линейный функционал на нем, причем
$\forall\,x\in E_1$\;\;$f(x)\leqslant p(x)$. Тогда
$\exists\,\bar{f}\colon E\to \mathbb{R}^1$ --- такое линейное
отображение, что $\forall\,x\in E$\;\;$\bar{f}(x)\leqslant p(x)$ и
$\forall\,x\p\in E_1$\;\;$\bar{f}(x)=f(x)$.
\end{theorem}

\begin{proof} Оно состоит из двух частей --- аналитической и
теоре\-ти\-ко-множественной. Первая часть --- аналитическая.

Пусть $z\in E\setminus E_1$ и $E^z=\mathrm{conv}(E_1,z)$
--- линейная оболочка. Докажем, что существует искомое продолжение
функционала $f$ на пространство $E^z$. $\forall\,v\in
E^z$\;\;$v=tz+x$, где $z\in E_1$, а $t\in\mathbb{R}^1$. Понятно, что
$\bar{f}(tz+x)\p=t\bar{f}(z)+f(x)\leqslant p(tz+x)$. Найдем величину
$C=\bar{f}(z)$. Возможны два случая.

1) $t>0$. Тогда $tC+f(x)\leqslant p(tz+x)$, а значит, $C\leqslant
p(z+x/t)-f(x/t)$ для всех $x$.

2) $t<0$. Тогда $tC+f(x)\leqslant p(tz+x)$ для всех $x$. разделив
обе части неравенства на $-t>0$, получим: $-C-f(x/t)\leqslant
-\frac{1}{t}p(tz+x)=p(-z-x/t),$ т.е. $C\geqslant -f(x/t)-p(-z-x/t)$.

Но $\forall\,x_1,x_2$\;\;$-p(-x_2-z)-f(x_2)\leqslant
-f(x_1)+p(z+x_1)$. В самом деле,
\begin{multline*}
f(x_1)-f(x_2)=f(x_1-x_2)\leqslant p(x_1-x_2)=p((x_1+z)-(x_2+z))\leqslant\\
\leqslant p(x_1+z)+p(-x_2-z).
\end{multline*}
Поэтому можно выбрать произвольное $C$, удовлетворяющее двойному
неравенству
$$-p(-z-x_1/t)-f(x_1/t)\leqslant C\leqslant -f(x_2/t)+p(z+x_2/t)\qquad
(\forall\,x_1,x_2).$$

Для завершения доказательства нам потребуется лемма
Куратовско\-го--Цорна.

\begin{df}
Множество $\Omega$ называется \emph{упорядоченным} (или
\emph{частично упорядоченным}), если на нем введено \emph{отношение
порядка} <<$\leqslant$>>, удовлетворяющее следующим аксиомам:

1) $x\leqslant x$ (\emph{рефлексивность});

2) $x\leqslant y$, $y\leqslant z$ $\Rightarrow$ $x\leqslant z$
(\emph{транзитивность});

3) $x\leqslant y$, $y\leqslant x$ $\Rightarrow$ $x=y$
(\emph{антисимметричность}).

Множество $\Omega$ называется \emph{линейно упорядоченным}, если
каждые два его элемента сравнимы (т.е. если $\forall\,x,z\in\Omega$
или $x\leqslant z$ или $z\leqslant x$).

Пусть $\Omega_1\subset\Omega$. Тогда элемент $\omega\in\Omega$
называется \emph{мажорантой $\Omega_1$}, если
$\forall\,x\in\Omega_1$\;\;$x\leqslant\omega$.

Элемент $a\in\Omega$ называется \emph{максимальным элементом
$\Omega$}, если $\forall\,x\in\Omega$ $x\geqslant a$ $\Rightarrow$
$x=a$.
\end{df}

\begin{lemma}[Куратовский--Цорн]
Если для каждого линейно упорядоченного подмножества
$\Omega_1\subset \Omega$ существует мажоранта $\omega\in\Omega$, то
в $\Omega$ есть максимальные элементы.\footnote{Ее доказательство
можно найти, например, в книге Н.\,Бурбаки <<Теория множеств>>.}
\end{lemma}

Теперь мы готовы завершить доказательство. Пусть $\Omega=(G, f_G)$,
где $E_1\subset G\subset E$ и $f_{G_1}$ --- продолжение $f$ на $G_1$
для которого $\forall\,x\in G\;\;f_G(x)\leqslant p(x)$. Введем на
$\Omega$ следующее отношение порядка:
$(G_1,f_{G_1})\leqslant(G_2,f_{G_2})$, если $G_1\subset G_2$ и
$f_{G_2}$ --- продолжение $f_{G_1}$ на $G_2$. Пусть $\Omega_1\subset
\Omega$ --- линейно упорядоченное подмножество, тогда найдется
мажоранта $\omega=(G_{\Omega_1},f_{\Omega_1})$, где
$G_{\Omega_1}=\bigcup\limits_{G_\alpha\in\Omega_1}G_\alpha$ и
$f_{\Omega_1}$ --- продолжение $f$ на $G_{\Omega_1}$, определенное
следующим образом: если $x\in G_\alpha$, то
$f_{\Omega_1}(x)=f_{G_\alpha}(x)$ (из линейной упорядоченности
$\Omega_1$ вытекает корректность этого определения). По лемме Цорна
в $\Omega$ есть максимальный элемент $(G_{\max},f_{\max})$.

В силу первой части доказательства $G_{\max}=E$. Действительно, если
$G_{\max}\neq E$, то $\exists\,z\in E\setminus G_{\max}$, и согласно
первой части, $f_{\max}$ можно продолжить на подпространство
$\mathrm{conv}(G_{\max},z)$ в противоречие с максимальностью
$(G_{\max},f_{\max})$.
\end{proof}

\begin{imp}
Пусть $E$ --- нормированное пространство и $f\colon
E_1\to\mathbb{R}^1$ --- непрерывный линейный функционал на
пространстве $E_1\subset E$, причем $\|f\|=C>0$. Тогда
$\exists\,\bar{f}\colon E\to \mathbb{R}^1:\bar{f}\mid_{E_1}=f$ и
$\|\bar{f}\|=\|f\|$.
\end{imp}

\begin{proof}
Пусть $p(x)=C\|x\|$, тогда $\forall\,x\in E_1\;\;|f(x)|\p\leqslant
C\|x\|\p=p(x)$, а значит, по теореме Хана-Банаха
$\exists\,\bar{f}\colon E\to\mathbb{R}^1: \bar{f}\mid_{E_1}=f$,
причем $\bar{f}(x)\p\leqslant C\|x\|$. Но неравенство
$\bar{f}(x)\leqslant C\|x\|$ влечет
$-\bar{f}(x)=\bar{f}(-x)\p\leqslant C\|x\|$, а из этих двух
неравенств вытекает, что $|\bar{f}(x)|\leqslant C\|x\|$, т.е. что
$\|\bar{f}\|\leqslant C$. Значит, $\|\bar{f}\|=C$ (поскольку
$\|\bar{f}\|\geqslant\|f\|$ ввиду того, что $\bar{f}\mid_{E_1}=f$).
\end{proof}

\begin{prop}
$\forall\,x\in E$\;\;$\exists\,f^x\in E^*:\|f^x\|=1$ и
$f^x(x)=\|x\|$.
\end{prop}

\begin{proof}
Положим $E_1=\{\lambda x\mid \lambda\in\mathbb{R}^1\}$ и $f_0\colon
E_1\to\mathbb{R}^1$, $f_0(\lambda x)\p=\lambda\|x\|$. Тогда
$\|f_0\|=1$. Тогда продолжение этого функционала без увеличения
нормы будет искомым.
\end{proof}

Рассмотрим пространство $E^{**}$. Можно считать, что $E\subset
E^{**}$; а именно, рассмотрим отображение $x\mapsto F_x\in E^{**}$,
где $F_x(g)=g(x)$. Это отображение --- вложение: если $x\neq 0$, то
$F_x\neq 0$ по предыдущему предложению. Поскольку
$|F_x(g)|=|g(x)|\leqslant \|g\|\|x\|$, то $\|F_x\|\leqslant \|x\|$,
причем равенство достигается при $g=f^x$. Значит, $\|F_x\|=\|x\|$.
Т.о., вложение $E\hookrightarrow E^{**}$, $x\mapsto F_x$ является
изометрическим на образ $f(E)$.

\begin{df}
Пространство $E$ называется \emph{рефлексивным}, если образ $E$ при
этом вложении совпадает с $E^{**}$.
\end{df}

\begin{df}
Нормированные пространства $E_1$ и $E_2$ называются
\emph{изоморфными}, если существует линейная биекция между этими
пространствами, сохраняющая норму.
\end{df}

\begin{df}
\emph{Пополнением нормированного пространства $E$} называется такое
нормированное пространство $\bar{E}\supset E$, что $E$ всюду плотно
в $\bar{E}$.
\end{df}

\begin{theorem}
Для любого нормированного пространства $E$ существует его пополнение
$\bar{E}$, однозначное с точностью до изоморфизма, тождественного на
$E$.
\end{theorem}

\begin{proof}
Вложим $E$ в банахово пространство $E^{**}$ и рассмотрим его
замыкание $\bar{E}$ в $E^{**}$. Оно и будет искомым. Доказательство
единственности аналогично доказательству единственности в теореме о
пополнении метрического пространства.
\end{proof}

\begin{df}
\emph{Графиком отображения $f\colon E\to G$} называется множество
$\Gamma_f=\{(x,f(x))\mid x\in E,\;f(x)\in G\}\subset E\times G$.
\end{df}

Норма в произведении $E\times G$ вводится так, чтобы ее сужения на
подпространства $E\times\{0\}$ и $\{0\}\times G$, изоморфные (как
линейные пространства) соответственно, пространствам $E$ и $G$,
совпадали с нормами, порожденными нормами пространств $E$ и $G$.

\begin{ex}
  1. $\|(x,z)\|=\|x\|+\|z\|$;

  2. $\|(x,z)\|=\max\{\|x\|,\|z\|\}$;

  3. $\|(x,z)\|=\sqrt{\|x\|^2+\|z\|^2}$.
\end{ex}
\begin{prop}
Если отображение $f$ непрерывно, то его график замкнут.
\end{prop}

\begin{problem}
Докажите это предложение.
\end{problem}

%---------------------------Lecture 7-----------------------------------------------%


\begin{theorem}[Банах]\label{th.obr.otobr.}
Если $f$ --- линейное непрерывное биективное ото\-бражение, то
отображение $f^{-1}$ непрерывно.
\end{theorem}

Теорема Банаха равносильна следующему утверждению.

\begin{theorem}\label{th.graf}
Пусть $E$ и $G$ --- банаховы пространства и $f\colon E\to G$
--- линейное отображение, график $\Gamma_f$
которого замкнут. Тогда отображение $f$ непрерывно.
\end{theorem}

\begin{proof}[Доказательство равносильности теорем~\ref{th.obr.otobr.} и \ref{th.graf}]
Докажем, сначала, что теорема \ref{th.obr.otobr.} влечет теорему
\ref{th.graf}. Т.к. график отображения $f$ является замкнутым
линейным пространством в $E\times G$, то он является банаховым
пространством. Рассмотрим отображение $F\colon (x,f(x))\p\mapsto x$.
Оно линейно, биективно и непрерывно, поэтому по теореме Банаха об
обратном отображении получаем, что и $F^{-1}$ непрерывно. Значит,
непрерывно отображение $f$ как композиция непрерывных отображений
$x\mapsto (x,f(x))\mapsto f(x)$ (первое из них --- это $F^{-1}$, а
второе --- проекция $E\times G$ на $G$).

Теперь докажем, что из теоремы \ref{th.graf} следует теорема
\ref{th.obr.otobr.}. Пусть отображение $f\colon E\to G$ линейно и
непрерывно, тогда $\Gamma_f\subset E\times G$ замкнут. Пусть
$\varphi=f^{-1}$, тогда
$\Gamma_\varphi=\{(z,\varphi(z))\}\p=\{(f(x),x)\}\p\subset G\times
E$. Отображение $E\times G\to G\times E$, $(x,z)\mapsto (z,x)$
биективно и непрерывно, причем $\Gamma_f$ отображается на
$\Gamma_\varphi$. Значит, $\Gamma_\varphi$ замкнут вместе с
$\Gamma_f$ и $\varphi$ непрерывно по теореме \ref{th.graf}.
\end{proof}

\begin{theorem}
Пусть $f\colon E\to G$ --- линейное непрерывное сюръективное
отображение банаховых пространств. Тогда образ всякого открытого
подмножества из $E$ открыт в $G$.
\end{theorem}

\begin{proof}
Пусть $V\subset E$ --- открытое подмножество. Сначала докажем
теорему для случая, когда $V=S(0,r)$ --- открытый шар.

Докажем, что $\overline{f(S(0,\varepsilon))}\supset S(0,\eta)$. В
самом деле,

\begin{equation*}
  \bigcup\limits_{n=1}^\infty
  n\cdot\overline{f(S(0,\varepsilon))}\supset\bigcup\limits_{n=1}^\infty
  n\cdot f(S(0,\varepsilon))= f\Big(\bigcup\limits_{n=1}^\infty
  n\cdot S(0,\varepsilon)\Big)=f\Big(\bigcup\limits_{n=1}^\infty
  S(0,n\varepsilon)\Big)=f(E)=G.
\end{equation*}


По теореме Бэра $\exists\,n:\overline{n\cdot
f(S(0,\varepsilon))}\supset S(x,r)$. Т.к. слева стоит выпуклое
симметричное множество, то $\overline{f(S(0,n\varepsilon))}\supset
S(0,r)$ и $\overline{f(S(0,\varepsilon))}\p\supset S(0,r/n)$.

Докажем, что $f(S(0,2\varepsilon))\supset S(0,\eta)$. Возьмем
последовательность $\{\varepsilon_j\}$, такую, что
$\sum\limits_{j=1}^\infty \varepsilon_j<\varepsilon$, и произвольное
$z\in S(0,\eta)$. Найдем такое $x\p\in S(0,2\varepsilon)$, что
$z=f(x)$. По доказанному ранее
$\forall\,j$\;\;$\exists\,\eta_j:\overline{f(S(0,\varepsilon_j))}\p\supset
S(0,\eta_j)$, причем $\eta_j\to 0$. Поэтому $\exists\,x_0\in
S(0,\varepsilon): \|z-f(x_0)\|<\eta_1$, т.е. $z-f(x_0)\in
S(0,\eta_1)$. Аналогично, $\exists\,x_1\in
S(0,\varepsilon_1):\|z-f(x_0)-f(x_1)\|\p<\eta_2$, т.е.
$z-f(x_0)-f(x_1)\in S(0,\eta_2)$, и т.д. Таким образом, мы получаем
последовательность $\{x_n\}$, где $x_0\in S(0,\varepsilon)$,
$x_j\p\in S(0,\varepsilon_j)$ и $z-\sum\limits_{j=0}^nf(x_j)\p\in
S(0,\eta_{n+1})$. Последовательность $\Big\{\sum\limits_{j=0}^n
x_j\Big\}$ фундаментальна в $E$, т.к.
$$\Big\|\sum\limits_{j=0}^{n+k}x_j-\sum\limits_{j=0}^n x_j\Big\|\leqslant
\Big\|\sum\limits_{j=n+1}^{n+k}x_j\Big\|\leqslant\sum\limits_{j=n+1}^{n+k}\|x_j\|
<\sum\limits_{j=n+1}^{n+k}\varepsilon_j<\sum\limits_{j=n+1}^\infty
\varepsilon_j\to 0.$$ Поэтому $\exists\,E\ni
x_\infty=\lim\limits_{n\to\infty}\sum\limits_{j=0}^n x_j$. Кроме
того, $\sum\limits_{j=0}^nf(x_j)\to z$, и в силу непрерывности $f$
получаем: $f\Big(\sum\limits_{j=0}^n
x_j\Big)=\sum\limits_{j=0}^nf(x_j)\to f(x_\infty)$, откуда
$z\p=f(x_\infty)$.

Таким образом,
$\forall\,\delta>0$\;\;$\exists\,r(\delta):f(S(0,\delta))\supset
S(0,r(\delta))$, откуда получаем, что $f(S(x,\delta))\supset
S(f(x),r(\delta))$.

Теперь докажем теорему для произвольного открытого подмножества $V$.
Пусть $z\in f(V)$, тогда $z=f(x)$, где $x\in V$. Т.к.
$\exists\,\delta>0:S(x,\delta)\subset V$, то
$S(f(x),r(\delta))\subset f(S(x,\delta))\subset V$.
\end{proof}

\section{Локально выпуклые пространства}

\begin{df}
\emph{Локально выпуклое пространство} --- это пара
$(E,\mathcal{P})$, где $\mathcal{P}$ --- семейство полунорм на $E$.
\end{df}

\begin{df}
На локально выпуклом пространстве $(E,\mathcal{P})$ можно
\emph{задать топологию}: множество $V\subset E$ назовем открытым,
если $$\forall\,x\in E\;\exists\,n\in\mathbb{N},\;
p_1,\ldots,p_n\in\mathcal{P},\;\varepsilon_1,\ldots,\varepsilon_n>0:\bigcap\limits_{
j=1}^n\{z\mid p_j(x-z)<\varepsilon_j\}\subset V.$$
\end{df}

Если $E$ и $G\subset E^*$ --- линейные пространства, то
$p\in\mathcal{P}_G$ $\Leftrightarrow$ $\exists\,f\p\in
G:\forall\,x\in E$\;\;$p_f(x)\equiv p(x)=|f(x)|$. Тогда пространство
$(E,\mathcal{P}_G)$ будет локально выпуклым.

\begin{df}
Топология на пространстве $(E,\mathcal{P}_G)$ называется
\emph{слабой топологией на $E$, порожденной $G$}, и обозначается
через $\sigma(E,G)$.

Если $E$ нормировано и $G=E^*$, то топология $\sigma(E,E^*)$
называется \emph{слабой топологией нормированного пространства $E$}.

Для пространства $E^*$ возьмем $G=\{F_x\mid x\in E\}$, тогда
топология $\sigma(E^*, E^{**})$ называется \emph{*слабой на $E^*$}.
\end{df}
%---------------------------Lecture 8-----------------------------------------------%


\begin{lemma}\label{lemm.algebra}
Пусть $f_{12}$ и $f_{13}$ --- линейные отображения, причем $\ker
f_{13}\p\supset \ker f_{12}$. Тогда существует такое линейное
отображение $f_{23}$, что следующая диаграмма коммутативна:
$$\xymatrix{
K_1 \ar@{->}[dr]_{f_{13}} \ar@{->}[rr]^{f_{12}}&& K_2\ar@{->}[dl]^{f_{23}}\\
&K_3}$$
\end{lemma}

\begin{proof}
Пусть $K_2=f_{12}(K_1)\oplus K$. Тогда положим
$$f_{23}(x)=
\begin{cases}
f_{13}(f_{12}^{-1}(x)),&\text{если $x\in f_{12}(K_1)$;}\\
0,&\text{если $x\in K.$}
\end{cases}$$
Это определение корректно ввиду того, что $\ker f_{13}\supset
\ker f_{12}$.
\end{proof}

\begin{theorem}
Пусть $E$ --- линейное пространство и $f$ --- линейный функционал на
$E$. Тогда он непрерывен в слабой топологии $(E,\sigma(E,G))$
$\Leftrightarrow$ $f\in G$\footnote{По другому утверждение теоремы
можно записать так: $(E,\sigma(E,G))^*=G$.}.
\end{theorem}

\begin{proof}
Если $x\in E$ таково, что $p_g(x)<\varepsilon$, где $g\in G$, то
$|g(x)|\p=p_g(x)<\varepsilon$, а значит, $g$ непрерывен в 0.

Обратно, пусть $g\in (E,\sigma(E,G))^*$. Тогда $g$ непрерывен в 0,
поэтому $\forall\,\varepsilon$\;\;$\exists\,V(0)\subset
V:\forall\,x\in V$\;\;$|g(x)|<\varepsilon$. Отсюда следует, что
$$\exists\, g_k:\{x\in E\mid p_{g_i}(x)<1\}=\{x\in E\mid
|g_i(x)|<1\}\subset V.$$ Поэтому, если $|g_k(x)|<1$, то
$|g(x)|<\varepsilon$, а значит, т.к. $\ker g\supset
\bigcap\limits_{k=1}^n\ker g_k$, то $\exists\,
\lambda_k:g=\sum\lambda_kg_k$, откуда $g\in G$.

Существование таких $\lambda_k$ следует из леммы~\ref{lemm.algebra}.
В самом деле, возьмем $K_1=E$, $K_2=\mathbb{R}^n$,
$K_3=\mathbb{R}^1$, $f_{12}(x)=(g_1(x),\ldots,g_n(x))$ и $f_{13}=g$.
Тогда $\exists\,f_{23}:f_{23}((x_1,\ldots,x_n))=\sum\lambda_kx_k$,
что и требовалось.
\end{proof}

\begin{theorem}
Пусть $E$ --- нормированное пространство и $B\subset E$. Тогда $B$
ограничено в топологии $\sigma(E, E^*)$ $\Leftrightarrow$ $B$
ограничено по норме.
\end{theorem}

\begin{proof}
Пусть $B$ ограничено по норме и $g\p\in E^*$. Тогда $\forall\,x\p\in
B$\;\;$|g(x)|\leqslant \|g\|\cdot\|x\|$, откуда $\|x\|<\infty$ и
$\sup\limits_{x\in B}|g(x)|<\infty$.

Докажем обратное утверждение. Вложим $E$ в $E^{**}$ и применим ко
множеству $B\subset E\subset E^{**}$ теорему Банаха--Штейнхауса.
Тогда $B$ ограничено в топологии $\sigma(E,E^*)$ $\Leftrightarrow$
$\forall\,f\in E^*$\;\;$\sup\limits_{x\in B}|f(x)|<\infty$, т.е.
$\sup\limits_{x\in B}|F_x(f)|<\infty$, т.к. $B$ поточечно ограничено
на банаховом пространстве $E^*$. Значит, $B$ ограничено по норме в
$E^{**}$. Но вложение $E\hookrightarrow E^{**}$ является изометрией,
поэтому $B$ ограничено и в пространстве $E$.
\end{proof}

\begin{theorem}
Пусть $E$ --- нормированное пространство  и $V\subset E$ ---
выпуклое подмножество в нем. Тогда $V$ замкнуто по норме
$\Leftrightarrow$ $V$ замкнуто в топологии $\sigma(E, E^*)$.
\end{theorem}

\begin{proof}
В одну сторону утверждение очевидно, т.к. топология по норме сильнее
слабой.

Докажем обратное утверждение. Для этого рассмотрим следующее
понятие.

\begin{df}
\emph{Функционалом Минковского множества $W$} называется функционал
$p_W(x)=\inf\{\lambda>0\mid x/\lambda\in W\}$.
\end{df}

Функционал Минковского обладает следующими свойствами.

1) $p_W(0)=0$  (обратное неверно!);

2) $p_W(\alpha x)=\alpha p_W(x)$, где $\alpha>0$;

3) $p_W(x_1+x_2)\leqslant p_W(x_1)+p_W(x_2)$.

Докажем свойство 3) (остальные очевидны). Нам будет достаточно
доказать его в случае, когда $\forall\,x\in E$\;\;$W\cap\{\lambda
x\}$ открыто в $\{\lambda x\}$ и $0\in W$. Пусть $x_1,x_2\in E$,
тогда
$\forall\,\varepsilon>0$\;\;$\frac{x_j}{p_W(x_j)+\varepsilon}\in W$
($j=1,2$). Т.к. $W$ выпукло, то при
$\tau_j=\frac{p_W(x_j)+\varepsilon}{p_W(x_1)+p_W(x_2)+2\varepsilon}$
и $z_j=\frac{x_j}{p_W(x_j)+\varepsilon}$ имеем: $\tau_1
x_1+\tau_2x_2\in V$. Но тогда $$p_W(\tau_1 x_1+\tau_2x_2)=p_W\left(
\frac{x_1+x_2}{p_W(x_1)+p_W(x_2)+2\varepsilon}\right)<1,$$ откуда
$p_W(x_1+x_2)<p_W(x_1)+p_W(x_2)+2\varepsilon$, что и требовалось.

Теперь докажем обратное утверждение теоремы. Можно считать, что
$0\in V$. Тогда $\exists\,S(z,\varepsilon):S(z,\varepsilon)\p\cap
V=\varnothing$, поэтому
$(V+S(0,\varepsilon/2))\p\cap(S(z,\varepsilon/2))=\varnothing$.
Положим $W=V+S(0,\varepsilon/2)$. Тогда $W$ --- это выпуклое
открытое множество, поскольку $W=\bigcup\limits_{v\in
V}(S(0,\varepsilon/2)+v)$. Пусть $p_W$ --- функционал Минковского
множества $W$, тогда $\exists\,\delta>0:p_W(z)\geqslant 1+\delta$.

На одномерном пространстве $\{\lambda z\}$ определим функционал
$f(\lambda z)\p=\lambda p_W(z)$. Тогда $\forall\,x\in\{\lambda
z\}$\;\;$f(x)\leqslant p_W(x)$. Значит, по теореме
Банаха--Штейнхауса функционал $f$ можно продлить до функционала
$\bar{f}$ на $E$, такого, что $\bar{f}(x)\leqslant p_W(x)$. Этот
функционал непрерывен: пусть $S(0,\varepsilon/2)\p\subset W$, тогда
$\forall\,x\in S(0,\varepsilon/2)$\;\;$p_W(x)<1$, поэтому
$\bar{f}(x)\leqslant p_W(x)<1$ и $\bar{f}(-x)<1$, откуда
$|\bar{f}(x)|<1$.

Рассмотрим множество $U=\{x\in E\mid \bar{f}(x)>1+\delta/2\}$. Тогда
$z\in U$ и $U\cap W=\varnothing$ (т.к. $\bar{f}(x)<p_W(x)$, откуда
$U\subset \{x\in E\mid p_W(x)>1\p+\delta/2\}$, а последнее множество
не пересекается с $W$). Но отсюда следует, что $U\cap
V=\varnothing$, т.е. $V$ содержит вместе с каждой точкой некоторую
ее окрестность в слабой топологии топологии, что и означает
открытость $V$.
\end{proof}
%---------------------------Lecture 9-----------------------------------------------%


\vspace{-25pt}

\section{Гильбертовы пространства}

\begin{df}
Пусть $E$ --- линейное пространство над $\mathbb{R}^1$ или
$\mathbb{C}^1$. \emph{Скалярным произведением на $E$} называется
функция $b\colon E\times E\to\mathbb{C}^1$, удовлетворяющая
следующим аксиомам:

1) $b(\lambda x,z)=\lambda b(x,z)$ и $b(x,\lambda z)=\bar{\lambda}
b(x,z)$;

2) $b(x_1+x_2,z)=b(x_1,z)+b(x_2,z)$ и
$b(x,z_1+z_2)=b(x,z_1)+b(x,z_2)$;

3) $b(x,z)=\overline{b(z,x)}$;

4) $b(x,x)\geqslant 0$, причем $b(x,x)=0$ $\Leftrightarrow$ $x=0$.

Если пространство $E$ вещественное, аксиомы немного другие, а
именно, в аксиоме 1) $b(x,\lambda z)=\lambda b(x,z)$, и в аксиоме 3)
$b(x,z)=b(z,x)$.
\end{df}

Если $(E,b)$ --- евклидово пространство, то на нем можно ввести
норму, а именно, $\|x\|^2=b(x,x)$.

В дальнейшем скалярное произведение будем обозначать через $(x,z)$.

\begin{prop}[Неравенство Коши--Буняковского--Шварца]
$$|(x,z)|\leqslant \|x\|\|z\|.$$
\end{prop}

\begin{proof}
При $z=0$ утверждение очевидно. Пусть теперь $z\neq 0$. Поскольку
неравенство
$$0\leqslant(x-\lambda z,x-\lambda z)=\|x\|^2-2(x,z)\lambda
+\lambda^2\|z\|^2$$ верно при всех $\lambda$, то дискриминант
квадратного трехчлена, стоящего в правой части, должен быть
отрицательным. А он как раз равен $(x,z)^2\p-\|x\|^2\|z\|^2$.
\end{proof}

\begin{note}
В этом доказательстве предполагалось, что пространство $E$
вещественно. Доказательство для комплексного случая будет дано в
теореме \ref{th.kompl.Koshi}.
\end{note}

\begin{df}
Полное евклидово пространство называется \emph{гильбертовым}. В
дальнейшем мы будем обозначать его через $H$.
\end{df}

\begin{ex}


  1. Пространство $\mathcal{L}_2(\Omega,\mathfrak{B},\nu)$ со
  скалярным произведением
  $(f,g)\p=\int\limits_\Omega\!f(x)g(x)\,\nu(dx)$ является
  гильбертовым.

  2. Пространство $l_2$ суммируемых последовательностей со скалярным
  произведением $(\{x_n\},\{z_n\})=\sum x_nz_n$ является гильбертовым
  (на самом деле, это частный случай пространства
  $\mathcal{L}_2(\Omega,\mathfrak{B},\nu)$, когда $\Omega=\mathbb{N}$,
  а $\nu$ --- считающая мера).
\end{ex}
\begin{df}
Вектора $a,b\in H$ называются \emph{ортогональными}, если $(a,b)=0$.

Вектор $a$ называется \emph{нормированным}, если $\|a\|=1$.
\end{df}

\begin{prop}
Если $\{x_j\}\subset E$ --- линейно независимая система векторов, то
$\exists\,\{e_j\}\subset E:{e_j}$ --- ортонормированная система
векторов и $\forall\,k$\;\;$\langle e_1,\ldots,e_k\rangle=\langle
x_1,\ldots,x_k\rangle$.
\end{prop}

\begin{proof}
Для доказательства воспользуемся \emph{процессом ортогонализации
Грама--Шмидта}: положим $e_1=\frac{x_1}{\|x_1\|}$ и
$$e_n=\frac{x_n-\sum\limits_{j=1}^{n-1}
(x_n,e_j)e_j}{\Big\|x_n-\sum\limits_{j=1}^{n-1}
(x_n,e_j)e_j\Big\|}.$$ Легко видеть, что система векторов $\{e_j\}$
искомая.
\end{proof}

\begin{df}
Ортонормированная система векторов $\{e_i\}$ простра\-нства $E$
называется \emph{тотальной}, если $\overline{\langle e_i\rangle}=E$.
\end{df}

\begin{df}
  Ортонормированная система векторов $\{e_i\}$ пространства $E$
  называется \emph{замкнутой}, если $\forall\,x\in
  E$\;\;$\|x\|^2\p=\sum\limits_{i=1}^\infty (x,e_i)^2$.
\end{df}

\begin{df}
  Ортонормированная система векторов $\{e_i\}$ пространства $E$
  называется \emph{полной}, если $\forall\,x\in E: (x,e_n)=0$
  $\p\Rightarrow$ $x=0$.
\end{df}
\begin{df}
Ортонормированная система векторов $\{e_j\}$ называется
\emph{базисом} пространства $E$, если $\forall\,x\p\in
E$\;\;$x=\sum\limits_{n=1}^\infty (x,e_n)e_n$.
\end{df}
%---------------------------Lecture 10-----------------------------------------------%



\begin{prop}[Неравенство Бесселя]
$\forall\,x\in E$\;\;$\|x\|^2\geqslant\sum\limits_n(x,e_n)^2$.
\end{prop}

\begin{proof}
В самом деле, $$0\leqslant
\Big\|x-\sum\limits_{n=1}^k(x,e_n)e_n\Big\|^2=
\|x\|^2-\sum\limits_{n=1}^k(x,e_n)^2,$$ откуда следует, что при всех
$k$ выполнено неравенство
$\sum\limits_{n=1}^k(x,e_n)^2\leqslant\|x\|^2$, а значит,
$\sum\limits_{n=1}^\infty(x,e_n)^2\leqslant\|x\|^2$.
\end{proof}

\begin{prop}\label{predl.inf.}
$\inf\limits_{\{\alpha_n\}}\Big\|x-\sum\alpha_ne_n\Big\|=\Big\|x-\sum(x,e_n)e_n\Big\|$.
\end{prop}

\begin{proof}
Несложно убедиться, что $$\Big\|x-\sum\alpha_ne_n\Big\|^2=
\Big\|x-\sum(x,e_n)e_n\Big\|+\Big\|\sum(x,e_n)e_n-\sum\alpha_ne_n\Big\|,$$
откуда следует искомое неравенство.
\end{proof}

\begin{theorem}
Имеет место следующая диаграмма:
$$
\xymatrix{
\text{(1) тотальность}\ar@{<=>}[r] \ar@{<=>}[d]& \text{(2) замкнутость}\ar@{=>}[d] \\
\text{(4) базисность}\ar@{=>}[r]& \text{(3) полнота} }
$$
\end{theorem}

\begin{proof}
Сначала докажем, что $(2)\Leftrightarrow(4)$. Пусть система векторов
$\{e_n\}$ замкнута. Тогда
$\Big\|x-\sum\limits_{n=1}^k(x,e_n)e_n\Big\|^2=\|x\|^2-\sum\limits_{n=1}^k(x,e_n)^2\to
0$ при $k\to\infty$, поэтому
$x=\lim\limits_{k\to\infty}\sum\limits_{n=1}^k(x,e_n)e_n=\sum(x,e_n)e_n$.

Обратно, пусть система векторов $\{e_n\}$ является базисом, тогда
получаем, что
$x=\lim\limits_{k\to\infty}\sum\limits_{n=1}^k(x,e_n)e_n$, откуда
$\Big\|x-\sum\limits_{n=1}^k(x,e_n)e_n\Big\|^2=\|x\|^2\p-\sum\limits_{n=1}^k(x,e_n)^2\to
0$, а значит, $\|x\|^2=\sum(x,e_n)^2$.

Теперь докажем, что $(1)\Leftrightarrow(4)$. Пусть система векторов
$\{e_n\}$ тотальна. Тогда $\forall\,x\in E$,
$\forall\,\varepsilon>0$\;\;$\exists\,k\in\mathbb{N},
\{\alpha_n\}_{n-1}^k:\Big\|x-\sum\limits_{n=1}^k\alpha_ne_n\Big\|<\varepsilon$.
В силу предложения~\ref{predl.inf.}, отсюда следует, что
$\Big\|x-\sum\limits_{n=1}^k(x,e_n)e_n\Big\|<\varepsilon$, а значит,
$x=\sum(x,e_n)e_n$.

Обратная импликация очевидна.

Наконец, докажем, что $(4)\Rightarrow(3)$. Пусть система векторов
$\{e_n\}$ является базисом и $\forall\,x,n$\;\;$(x,e_n)=0$. Тогда
$x=\lim\limits_{k\to\infty}\sum\limits_{n=1}^k(x,e_n)e_n=0$.
\end{proof}

\begin{theorem}[Рисс--Фишер]
Если пространство $E$ гильбертово, то $(3)\Rightarrow(4)$.
\end{theorem}

\begin{proof}
Поскольку $\sum(x,e_n)^2<\infty$, то для всякого $x\in E$ имеем:
$\Big\|\sum\limits_{n=k_1}^{k_2}(x,e_n)e_n\Big\|\p=\sum\limits_{n=k_1}^{k_2}
(x,e_n)^2\to 0$ при $k_1,k_2\to\infty$. Т.к. $E$ гильбертово,
$\exists\,z\in E:z\p=\sum(x,e_n)e_n$. Остается доказать, что $z=x$.
Это следует из следующей цепочки равенств и полноты:
$$(z-x,e_l)=(z,e_l)-(x,e_l)=\lim\limits_{k\to\infty}\Big(\sum\limits_{n=1}^k
(x,e_n)e_n, e_l\Big)-(x,e_l)=0,$$ что и требовалось.
\end{proof}

\begin{df}
Пространство называется \emph{сепарабельным}, если оно обладает
счетным всюду плотным множеством.
\end{df}

\begin{theorem}
Любые два бесконечномерных сепарабельных гильбертовых пространства
изоморфны
\end{theorem}

\begin{proof}
Докажем, что в сепарабельном гильбертовом простра\-нстве бесконечной
размерности есть ортонормированный базис. Поскольку пространство
сепарабельно, то в нем есть счетное всюду плотное множество
$\{x_n\}$. Пусть $z_1=x_{n_1}$ --- первый ненулевой элемент этой
системы. Далее, $z_2=x_{n_2}$ --- первый среди последующих элементов
этой системы, независимый с $z_1$. Продолжая этот процесс, мы в
конце концов получим систему независимых векторов $\{z_n\}$, причем
мы всегда сумеем выбрать следующий элемент $z_k$ в силу
бесконечномерности пространства. Применяя процесс ортогонализации
Грама--Шмидта, мы получаем тотальную ортонормированную систему
векторов $\{e_n\}$. Т.к. пространство гильбертово, то по доказанному
ранее эта система является базисом.

Докажем теперь утверждение теоремы. Пусть $E_1$ и $E_2$ --- два
пространства. Согласно доказанному выше, выберем в них
ортонормированные базисы $\{e_n^1\}$ и $\{e_n^2\}$. Тогда
$\forall\,x\in E_1$\;\;$x=\sum(x,e_n^1)e_n^1$. Положим $F\colon
E_1\to E_2$, $x\mapsto F(x)=\sum(x,e_n^1)e_n^2$. В силу неравенства
Бесселя указанный ряд сходится, поэтому отображение определено
корректно. Докажем, что оно является автоморфизмом. В самом деле,
\begin{multline*}
(x_1,x_2)=\lim\limits_{k\to\infty}\Big(\sum\limits_{n=1}^k(x_1,e_n^1)e_n^1,
\sum\limits_{n=1}^k(x_2,e_n^1)e_n^1\Big)=\\
=\lim\limits_{k\to\infty} \sum\limits_{n=1}^k(x_1,e_n^1)(x_2,e_n^1)=
\sum\limits(x_1,e_n^1)(x_2,e_n^1)=(F(x_1),F(x_2))\footnotemark.
\end{multline*}
\footnotetext{Это равенство называется \emph{равенством Парсеваля.}}
\end{proof}

\begin{lemma}[Равенство параллелограмма]
$\forall\,x,z\in E$ имеет место следующее равенство: $\|\frac 1
2(x-z)\|^2+\|\frac 1 2(x+z)\|^2= \frac 1 2\|x\|^2+\frac 1 2\|z\|^2$.
\hfill{$\square$}
\end{lemma}

\begin{lemma}\label{lemm.koltso}
Пусть $d>\delta\geqslant 0$, $S=\{x\in E\mid
d\leqslant\|x\|^2\leqslant d+\delta\}$ и $A\subset S$ --- выпуклое
множество. Тогда $\forall\,x_1,x_2\in A$\;\;$\|x_1-x_2\|\leqslant
\sqrt{12d\delta}$.
\end{lemma}

\enlargethispage{\baselineskip}

\begin{proof}
Т.к. $\frac 1 2(x_1+x_2)\in A$, то $\|\frac 1 2(x_1+x_2)\|\geqslant
d$. Кроме того, согласно правилу параллелограмма,
\begin{equation*}
\Big\|\frac 1 2(x_1-x_2)\Big\|^2=\frac 1 2\|x_1\|^2+\frac 1
2\|x_2\|^2-\Big\|\frac 1
2(x_1+x_2)\Big\|^2\leqslant(d+\delta)^2-d^2\leqslant3d\delta,
\end{equation*}
откуда $\|x_1-x_2\|\leqslant\sqrt{12d\delta}$.
\end{proof}

%---------------------------Lecture 11-----------------------------------------------%


\vspace{-27pt}

\section{Теорема Рисса}

\begin{note}
В дальнейшем, если не оговорено противное, мы будем считать, что
основным полем является \emph{либо $\mathbb{R}$, либо $\mathbb{C}$}.
\end{note}

\begin{prop}
Пусть $V$ --- выпуклое замкнутое множество гильбертова пространства
и $h\not\in V$. Тогда $\exists!\,x_h\in
V:c=\|h-x_h\|\p=\inf\limits_{z\in V}\|h-z\|$.
\end{prop}

\begin{df}
Элемент $x_h$ называется \emph{проекцией элемента $h$ на $V$} и
обозначается через $\pr{V}h$.
\end{df}

\begin{proof}
Пусть последовательность $\{z_n\}\subset V$ такова, что
$\|h\p-z_n\|\to c$, т.е.
$\forall\,\varepsilon>0$\;\;$\exists\,n_0:\forall\,n\geqslant
n_o$\;\;$z_n\in\{x:c\leqslant\|x-h\|\leqslant c+\varepsilon\}\cap
V$. По лемме~\ref{lemm.koltso} получаем, что
$\|z_n-z_k\|\leqslant\sqrt{12c\varepsilon}$ при $n,k\geqslant n_0$,
а значит, последовательность $\{z_n\}$ фундаментальна. Поэтому
$\exists\,x_h=\lim\limits_{n\to\infty}z_n$. Легко видеть, что
элемент $x_h$ искомый, т.е. $\|h-x_h\|=c$.

Докажем единственность. Пусть есть два элемента $x_h$ и $\bar{x}_h$,
удовлетворяющие условию. Тогда по лемме~\ref{lemm.koltso} имеем:
$\|\bar{x}_h-x_h\|\leqslant\sqrt{12c\varepsilon}$ при всех
$\varepsilon>0$. Отсюда следует, что $\bar{x}_h=x_h$.
\end{proof}

\begin{prop}
$\Re(h-x_h,z-x_h)\leqslant 0$ при всех $z\in V$.
\end{prop}

\begin{proof}
Т.к. множество $V$ выпукло, то при всех $\lambda\in[0;1)$ имеем:
$x_h+\lambda(z-x_h)\in V$. В таком случае
$\|h-(x_h+\lambda(z-x_h))\|^2\geqslant\|h-x_h\|^2$, что равносильно
следующему неравенству:
$$\|h-x_h\|^2+\lambda^2\|z-x_h\|^2-2\lambda\Re(h-x_h,z-x_h)\geqslant \|h-x_h\|^2.$$
Отсюда следует, что $\Re(h-x_h,z-x_h)\leqslant\frac\lambda 2
\|z-x_h\|^2$. Устремляя $\lambda $ к 0, получаем требуемое.
\end{proof}

\begin{theorem}
Пусть $G\subset H$ --- замкнутое подпространство гильбертова
пространства. Тогда $\forall\,h\in H$\;\;$\exists!\,x_h=\pr{G}h\in
G:h-\pr{G}h\bot G$.
\end{theorem}

\begin{proof}
Пусть $x_h=\pr{G}h$. При $z=0$ по предыдущему предложению получаем,
что $\Re (h-x_h,-x_h)\leqslant 0$. Отсюда следует, что при всех $z$
выполнено неравенство $\Re(h-x_h,z)\leqslant 0$. Если теперь
$(h-x_h,z)=re^{i\theta}$, где $r\geqslant0$, то
$\Re(h-x_h,e^{i\theta}z)=r\leqslant0$. Поэтому $r=0$, что и
требовалось.
\end{proof}

\begin{note}
На самом деле, условие теоремы является и достаточным, т.е. по
условию теоремы вектор $x_h$ определяется однозначно.
\end{note}

\begin{theorem}[Рисс]
Пусть $E$ --- гильбертово пространство и $f\in E^*$. Тогда
$\exists!\,h_f\in E:\forall\,x\in E$\;\;$f(x)=(x,h_f)$, причем
$\|f\|=\|h_f\|$.
\end{theorem}

\begin{proof}
Если $f\equiv0$, то утверждение очевидно.

Пусть $f\not\equiv 0$. Рассмотрим подпространство $G=\ker f$. Тогда
$\exists\,z\p\in H\setminus G:z-\pr{G}z=h\bot G$, причем $h\neq 0$.
Рассмотрим функционал $F\colon x\mapsto (x,h)$. Тогда $\ker F\supset
\ker f=G$, поэтому по лемме~\ref{lemm.algebra}
$\exists\,\alpha\in\mathbb{C}: f=\alpha F$ и $\alpha\neq 0$. В таком
случае положим $h_f=\bar{\alpha}h$. Легко проверить, что
$f(x)=(x,h_f)$.

Докажем, что $\|f\|=\|h_f\|$. По неравенству
Коши--Буняковского--Шва\-рца
$|f(x)|=|(x,h_f)|\leqslant\|x\|\|h_f\|$, поэтому
$\|f\|\leqslant\|h_f\|$. Кроме того, при $x=\frac{h_f}{\|h_f\|}$
неравенство обращается в равенство, поэтому $\|f\|=\|h_f\|$.

Докажем единственность. Пусть есть два элемента $h_f$ и $\bar{h}_f$,
удовлетворяющие условию. Тогда $\forall\,x\in H$\
$(x,h_f)=(x,\bar{h}_f)$. Отсюда $\forall\,x\in H$\
$(x,h_f-\bar{h}_f)=0$. Подставив $x=h_f-\bar{h}_f$, получим
$(h_f-\bar{h}_f,h_f-\bar{h}_f)=0\p\Rightarrow
h_f-\bar{h}_f=0\Rightarrow h_f=\bar{h}_f.$
\end{proof}
%---------------------------Lecture 12-----------------------------------------------%



\begin{theorem}
Пусть $G\subset H$ --- замкнутое подпространство гильбертова
пространства, $h\in H$, $\exists\, \pr{G}h\in G $ и $\exists\,z\in
G$, $(h-z)\bot G$. Тогда $\pr{G}h=z$.
\end{theorem}

\begin{proof}
$(h-\pr{G}h)\bot G$, $(h-z)\bot G$ $\p\Rightarrow$
$(h-\pr{G}h)-(h-z)=(z\p-\pr{G}h)\bot G$. Но $(z-\pr{G}h)\in G$.
Следовательно, $(z-\pr{G}h,z-\pr{G}h)=0\;$$ \p\Rightarrow
z-\pr{G}h=0\;$$\Rightarrow z=\pr{G}h$.
\end{proof}

\begin{df}
Пусть $E_1$ и $E_2$ - линейные пространства над полем $\mathbb{C}$.
Отображение $f:E_1\rightarrow E_2$ называется \emph{полулинейным},
если $\forall g_1, g_2\in E_1$ и $\lambda \in \mathbb{C}$ выполнено
$f(g_1+g_2)=f(g_1)+(g_2)$ и $f(\lambda g_1)=\bar{\lambda} f(g_1)$.
\end{df}

\begin{note}
Пусть $E$ --- гильбертово пространство и $f\in E^*$. По теореме
Рисса  $\exists!\,h_f\in E:\forall\,x\in E$\;\;$f(x)=(x,h_f)$ Тогда
отображение $F\colon E^*\rightarrow E$, $F(f)=h_f$ полулинейно.
\end{note}

\begin{theorem}
Пусть $E$ --- это произвольное линейное пространство, а $p\colon
E\to\mathbb{R}^+$ --- такая функция на нем, что выполняются
следующие свойства:

1\textup{)} $p(\alpha x)=|\alpha| p(x)$;

2\textup{)} $p(x_1+x_2)\leqslant p(x_1)+p(x_2)$.

Пусть также $G\subset E$ --- подпространство и $f\colon
G\to\mathbb{C}$ --- линейный функционал на нем, причем
$\forall\,x\in G$\;\;$|f(x)|\leqslant p(x)$. Тогда
$\exists\,\tilde{f}\colon E\to \mathbb{C}$ --- такое линейное
отображение, что $\forall\,x\in E$\;\;$|\tilde{f}(x)|\leqslant p(x)$
и $\forall\,x\p\in G$\;\;$\tilde{f}(x)=f(x)$.
\end{theorem}

\begin{proof}
Положим $\forall\, x\in G\ f_1(x)=\Re f(x)$. Тогда $f_1\colon
G_{\mathbb{R}}\rightarrow \mathbb{R}$ линеен и $\forall\, x \in
G\;\; f(x)=f_1(x)-if_1(ix)$. По условию $\forall\,x\p\in
G$\;\;$|f(x)|\p\leqslant p(x)$, значит, $\forall\,x\in
G$\;\;$|f_1(x)|\leqslant p(x)$. По теореме Хана-Банаха найдем такой
линейный функционал $\tilde{f_1}\colon E_{\mathbb{R}}\rightarrow
\mathbb{R}$, что $\forall\, x\in E_{\mathbb{R}}
\;\;|\tilde{f_1}(x)|\leqslant p(x)$. Положим $\forall x \in E\;\;
\tilde{f}(x)=\tilde{f_1}(x)-i\tilde{f_1}(ix)$. Тогда $\tilde{f}$ и
будет искомым функционалом. Надо лишь проверить, что $\forall\,x\in
E$\;\;$|\tilde{f}(x)|\leqslant p(x)$.

Допустим, что это не так, т.е. $\exists\, x\in E:
|\tilde{f}(x)|>p(x)$. Пусть $\tilde{f}(x)\p=\rho e^{i\theta}$. Тогда
$\tilde{f}(e^{-i\theta}x)=e^{-i\theta}\tilde{f}(x)=e^{-i\theta}\rho
e^{i\theta}=\rho>0$. Поэтому
$|\tilde{f}(e^{-i\theta}x)|=\;\;=\tilde{f}(e^{-i\theta}x)=\tilde{f_1}(e^{-i\theta}x)\Rightarrow
\tilde{f_1}(e^{-i\theta}x)>p(x)=p(e^{-i\theta}x)$. Противоречие.
\end{proof}

\begin{df}
Пусть E и G --- банаховы пространства и $A\colon E\rightarrow G$
--- линейный оператор. Тогда \emph{банахов сопряженный оператор} $A^*\colon
G^*\rightarrow E^*$ определяется следующим образом: $\forall g\in
G^* \;\;\forall x\in E \;\;g(Ax)=(A^*g)(x)$.
\end{df}

\begin{prop}
$A^*$ линеен и непрерывен.
\end{prop}

\begin{proof}
Линейность очевидно вытекает из определения. Для проверки
непрерывности докажем более сильное утверждение: $\|A^*\|\p=\|A\|$.

\begin{multline*}
\|A\|=\sup\limits_{\substack{x\in F\\ \|x\|\leqslant
1}}\|Ax\|=\sup\limits_{\|x\|\leqslant 1}\sup\limits_{\substack{g\in
G^*\\ \|g\|\leqslant 1}}|g(Ax)|=\sup\limits_{\|g\|\leqslant
1}\sup\limits_{\|x\|\leqslant 1}|g(Ax)|=\\
=\sup\limits_{\|g\|\leqslant 1}\sup\limits_{\|x\|\leqslant
1}|(A^*g)(x)|=\sup\limits_{\|g\|\leqslant 1}\|A^*g\|=\|A^*\|.
\end{multline*}
\end{proof}

\begin{df}
Пусть H --- гильбертово пространство и $A\colon H\rightarrow H$
--- линейный оператор. Тогда \emph{гильбертов сопряженный оператор}
$A^*\colon H\p\rightarrow H$ определяется следующим образом:
$\forall x\in H \;\;\forall z\in H \;\;(A^*x,z)\p=(x,Az)$.
\end{df}
%---------------------------Lecture 13-----------------------------------------------%

\clearpage



\begin{theorem}[Неравенство Коши-Буняковского в комплексном
случае]\label{th.kompl.Koshi} Пусть $H$ --- гильбертово пространство
над $\mathbb{C}$ и $f,g\in H$. Тогда верно неравенство
$|(f,g)|\leqslant \|f\|\|g\|$.
\end{theorem}

\begin{proof}
Если $\|g\|=0$, то $g=0$ и неравенство Коши-Буняковс\-кого
выполнено.

Если же $(f,g)=0$, то неравенство Коши-Буняковского тоже выполнено.

Ну а если $\|g\|(f,g)\neq0$, то пусть $\lambda$ --- произвольное
действительное число. Тогда
$(f-\lambda(f,g)g,f-\lambda(f,g)g)\geqslant 0.$ Значит,
$(f,f)-2\lambda|(f,g)|^2\p+\lambda^2|(f,g)|^2\|g\|^2\geqslant0.$ Это
квадратный трехчлен, неотрицательный при всех $\lambda$, поэтому его
дискриминант неотрицателен. Таким образом, получаем, что
$|(f,g)|^4\leqslant \|f\|\|g\||(f,g)|^2$, откуда следует
$|(f,g)|\leqslant \|f\|\|g\|.$
\end{proof}

\begin{theorem}
Пусть $A$ --- произвольный линейный непрерывный оператор в
гильбертовом пространстве $H$. Тогда $\ker A=(\Im A^*)^{\perp}$, где
$A^*$ --- оператор, сопряженный к оператору $A$ и $\Im C=\{Cx\mid
x\in H\}$.
\end{theorem}

\begin{proof}
Проверим включение в одну сторону. $x\in \ker A$ $\p\Leftrightarrow$
$\forall\, z\in H$ \;\;$(Ax,z)=(x,A^*z)$. Итак, любой элемент ядра
оператора перпендикулярен любому элементу образа.

Обратно: $x\perp \Im A^*$ $\Leftrightarrow$ $\forall\, z\in H$\;\;
$(x,A^*z)=0$ $\Leftrightarrow$ $\forall\, z\in H$ \;\;$(Ax,z)=0$
$\p\Leftrightarrow$ $Ax=0$ $\Leftrightarrow$ $x\in \ker A$.
\end{proof}

\begin{imp}
1. $(\ker A)^{\perp}=(\Im A^*)^{\perp\perp}=\overline{\Im A^*}$.

2. $\ker A^*=(\Im A)^{\perp}$.

3. $(\ker A^*)^{\perp}=\overline{\Im A}$.
\end{imp}

\begin{problem}
Докажите это следствие.
\end{problem}

Пусть теперь $A\colon E\to G$, где $E$ и $G$ --- банаховы
пространства. Тогда для оператора $A^*\colon G^*\to E^*$ верны те же
свойства (здесь $(\Im A^*)^{\perp}=\{x\p\in E\mid\forall\, g\p\in
\Im A^*,\; g(x)=0\}$).

Доказательства аналогичны предыдущим.

\section{Обобщенные функции}

Определим три пространства так называемых пробных функций:

1. $D=D(\mathbb{R}^n)$ --- пространство всех бесконечно
дифференцируемых функций (действительнозначных или
комплекснозначных) с компактным носителем.

2. $S=S(\mathbb{R}^n)$ --- пространство всех бесконечно
дифференцируемых быстро убывающих функций (действительнозначных или
комплекснозначных).

3. $\mathcal{E}=\mathcal{E}(\mathbb{R}^n)$ --- пространство всех
бесконечно дифференцируемых функций (действительнозначных или
комплекснозначных).

\begin{df}
\emph{Носителем функции} $\varphi$ называется следующее множество:
$\supp\varphi=\overline{\{x\mid \varphi(x)\neq0\}}$.
\end{df}

\begin{df}
Функция $\varphi$ называется \emph{быстро убывающей}, если
$\forall\,
n,k$\;\;$p_{n,k}=\sup_x(1+\|x\|^n)\|\varphi^{(k)}(x)\|<\infty$.
\end{df}

\begin{df}
Здесь $\|\varphi^{(k)}(x)\|=\sum\limits_{\sum
r_j=k}\Big|\cfrac{\partial^k\varphi(x)}{\partial x^{r_1}_{1 } \ldots
\partial  x^{r_n}_{n }}\Big|$.
\end{df}

Каждое из пространств $D$, $S$, $\mathcal{E}$ является линейным. В
этих пространствах задается топология следующим образом.

Начнем с пространства $S$. В нем топологию определяет система
полунорм $P=\{p_{n,k}: n,k=0,1,2,\ldots\}$.

Теперь рассмотрим пространство $D$.
$D=\bigcup\limits_{r=1}^{\infty}D_r$, где $D_{r}=\{\varphi\in D\mid
\supp \varphi\subset S(0,r)\}$. Но $D_r \subset D\subset S$.
Получаем в $D_r$ индуцированную топологию из $S$.

Определим фундаментальную систему окрестностей нуля:

$W\in V$ $\Leftrightarrow$ $W$ выпукло и $\forall\, r\in
\mathbb{N}\;\; W\cap D_r$ --- открытая окрестность нуля в $D_r$.

\begin{df}
$V\subset D$ \emph{открыто} тогда и только тогда, когда оно
представляет собой объединение (возможно, сдвинутых) окрестностей
нуля.
\end{df}

Теперь определим топологию $P_E$ в $\mathcal{E}$. Для этого снабдим
пространство $\mathcal{E}$ семейством полунорм: $p\in P_\mathcal{E}$
$\Leftrightarrow$ $\exists$ компакт $K\subset \mathbb{R}^n$ и
$r,k\p\in \{0,1,2,\ldots\}: \forall\, \varphi \in E\;\;
p_{r,k}=\sup\limits_{x\in K}\sum\limits_{\sum
r_j=r}\Big|\cfrac{\partial^k\varphi(x)}{\partial x^{r_1}_{1 } \ldots
\partial  x^{r_n}_{n }}\Big|$.

Это семейство полунорм и задает топологию.

\begin{theorem}
Пусть $E$ хаусдорфовое и локально выпуклое пространство и топология
может быть задана не более чем счетным семейством полунорм. Тогда
$E$ метризуемо.
\end{theorem}

\begin{proof}
Если $\{p_n\}$ --- полунормы из условия, то метрика такова:
$\rho(\varphi,\phi)=\sum_{n=1}^{\infty}\frac{p_n(\varphi-\phi)}{2^n(1+p_n(\varphi-\phi))}$.
\end{proof}
%---------------------------Lecture 14-----------------------------------------------%



\begin{prop}
$F_g=0$ $\Leftrightarrow$ $g(x)=0$ почти всюду.
\end{prop}

\begin{proof}
Рассмотрим семейство гладких функций $$ \psi_{a,b,n}(x)=
\begin{cases}
1,&\text{$x\in(a+1/n; b-1/n)$};\\
0,&\text{$x\not\in[a;b]$}.
\end{cases}$$
Понятно, что $\forall\,x\;\; 0\leqslant\psi_{a,b,n}(x)\leqslant1$.
Пусть теперь
$\int\limits_{-\infty}^{+\infty}\!g(x)\varphi(x)\,dx=0$. Докажем,
что $\forall\,a<b\in\mathbb{R}^1\;\;\int\limits_{a}^b\!g(x)\,dx=0$.
В самом деле, поскольку $\psi_{a,b,n}\to\gamma_{(a;b)}$ при
$n\to\infty$, то, подставляя $\varphi=\psi_{a,b,n}$ и переходя к
пределу под знаком интеграла (это возможно по теореме Лебега),
получаем:
$$0=\int\limits_{-\infty}^{+\infty}\!g(x)\psi_{a,b,n}(x)\,dx\to\int\limits_a^b\!
g(x)\,dx.$$

Докажем теперь, что для любого ограниченного измеримого подмножества
$A$ из $\mathbb{R}^1$ верно равенство $$\int\limits_A\!g(x)\,dx=0$$
Действительно,
$\forall\,\varepsilon\p>0\;\;\exists\,A_\varepsilon:\nu(A_\varepsilon\triangle
A)<\varepsilon$ и
$\forall\,\nu>0\;\;\exists\,\delta>0:\forall\,B\subset
\mathbb{R}^1$, $\nu(B)<\delta$, $B\subset[a-1;b+1]$ $\Rightarrow$
$\int\limits_{B}\!|g(x)|\,dx<\nu$. Значит,
$\int\limits_{A_\varepsilon}\!g(x)\,dx=0$ и $\int\limits_{A\triangle
A_\varepsilon}\!g(x)\,dx<\nu$, откуда $\int\limits_A\!g(x)\,dx=0$.
\end{proof}

Пусть $\nu\colon\mathfrak{B}(\mathbb{R}^1)\to\mathbb{R}^1$ --- мера.
Построим следующее отображение этой меры в пространство обобщенных
функций: $\nu\mapsto F_\nu\in D^*$,
$F_\nu(\varphi)\p=\int\limits_{\mathbb{R}^1}\!\varphi(x)\,\nu(dx)$.
Рассуждая так же, как и при доказательстве предыдущей теоремы,
получаем, что $\forall\,a<b\in\mathbb{R}^1\;\;\nu((a;b))=0$. Если
$A\in\mathfrak{B}(\mathbb{R}^1)$, то по теореме Хана
$\nu=\nu^+-\nu^-$, поэтому
$\forall\,\varepsilon>0\;\;\exists\,A^{\pm}_\varepsilon:\nu^\pm(A\triangle
A^\pm_\varepsilon)<\varepsilon$. Отсюда следует, что
$\exists\,A_\varepsilon:\nu^\pm(A\triangle
A_\varepsilon)<\varepsilon$ и $\nu(A_\varepsilon)=0$. Отсюда
получаем, что $|\nu^+(A)-\nu^-(A)|<2\varepsilon$, а значит,
$\nu(A)=0$.

Введем теперь некоторые операции в пространстве обобщенных
фун\-кций. Для этого прежде всего заметим, что $D^*$ --- это модуль
над $\mathcal{E}$. Поэтому, например, $\varphi F_g=F_{\varphi d}$.

\begin{df}
\emph{Производной обобщенной функции $F$} называется такая
обобщенная функция $F'$, что $(F',\varphi)=-(F,\varphi')$
\end{df}

%Еще одна важная операция --- это преобразование Фурье.

\section{Преобразование Фурье}

Сначала определим преобразование Фурье для функций из класса
$\mathcal{L}_1(\mathbb{R}^n)$.\footnote{В дальнейшем мы для простоты
часто будем считать, что $n=1$.}

\begin{df}
Пусть $\varphi\in\mathcal{L}_1(\mathbb{R}^n)$. Ее
\emph{преобразованием Фурье} называется функция
$$\hat{\varphi}(x)=c_1\int\limits_{\mathbb{R}^n}\!e^{-i(x,z)}\varphi(z)\,dz$$
где $c_1$ --- некоторая ненулевая константа.
\end{df}

\begin{note}
В дальнейшем мы докажем т.н. \emph{формулу обращения}:
$\varphi(x)\p=c_2\int\limits_{\mathbb{R}^n}\!e^{i(x,z)}\hat{\varphi}(z)\,dz$.
При этом константы $c_1$ и $c_2$ выбираются таким образом, чтобы
$c_1c_2=(2\pi)^n$. В дальнейшем мы будем считать, что $c_1=1$ и
$c_2\p=\frac{1}{(2\pi)^n}$.
\end{note}

Рассмотрим некоторые свойства преобразования Фурье.

\begin{theorem}
Преобразование Фурье $\mathcal{L}_1(\mathbb{R}^n)\to
C^0(\mathbb{R}^n)$ непрерывно.
\end{theorem}

\begin{proof}
В самом деле, функция $\hat{\varphi}$ непрерывна по теореме Лебега и
$\|\hat{\varphi}\|_{C^0}=\max\limits_x|\hat{\varphi}(x)|\leqslant\int\limits_
{\mathbb{R}^n}\!|\varphi(z)|\,dz=\|\varphi\|_{\mathcal{L}_1}$. Кроме
того, $|\hat{\varphi}(x)|\to 0$ при $x\to\infty$: это верно для
функций $\varphi(x)=\gamma_{[a;b]}(x)$, которые плотны в
$\mathcal{L}_1$, а значит, ими можно приблизить любую другую функцию
и применить теорему Лебега.
\end{proof}

\begin{theorem}
Пусть $g(x)\in C^1(\mathbb{R}^1)\cap\mathcal{L}_1(\mathbb{R}^1)$,
тогда $\widehat{g'}(x)=ix\hat{g}(x)$.
\end{theorem}

\begin{proof}
Действительно,
\begin{multline*}
\widehat{g'}(x)=\int\limits_{\mathbb{R}^1}\!e^{-ixz}g'(z)\,dz=\lim\limits_{n\to\infty}
\int\limits_{-n}^n\!e^{-ixz}g'(z)\,dz=\\=\lim\limits_{n\to\infty}\left(e^{-ixz}g(z)\mid_{-n}^n
+ix\int\limits_{-n}^n\!e^{-ixz}g(z)\,dz\right)=ix\hat{g}(x),
\end{multline*}
т.к. $g(z)=g(0)+\int\limits_{0}^z\!g'(z)\,dz$, откуда
$\exists\,\lim\limits_{z\to\infty}g(z)=c_1$ и
$\exists\,\lim\limits_{z\to-\infty}g(z)=c_2$. Оба этих предела равны
0, т.к. $g'\in\mathcal{L}_1(\mathbb{R}^1)$, откуда и следует искомое
равенство.
\end{proof}
%---------------------------Lecture 15-----------------------------------------------%



%$\hat f(z)=\int_{\mathbb{R}}e^{-ixz}f(x)dx$.

\begin{theorem}
Если $f\in \mathcal{L}_1(\mathbb{R})$ и $[x\mapsto xf(x)]\in
\mathcal{L}_1(\mathbb{R})$, то $(\hat f)'(z)\p=\widehat{-ixf(x)}$.
\end{theorem}

\begin{proof}
$\forall\, \alpha, \beta\;\; |e^{i\alpha}-e^{i\beta}|\leqslant
|\alpha-\beta|$. Отсюда $|\frac{e^{-ix(z+\Delta z)}-e^{-ixz}}{\Delta
z}|\leqslant |x|$. Поэтому $\hat f'(z)=\lim\limits_{\Delta z\to
0}\int\limits_{\mathbb{R}} \Big|\frac{e^{-ix(z+\Delta
z)}-e^{-ixz}}{\Delta
z}\Big|\,dx=\int\limits_{\mathbb{R}}-ixe^{-ixz}f(x)dx$.
%Итак, $(\hat f)'(z)=\widehat{-ixf(x)}(z)$.
\end{proof}

%$\widehat {f'}(z)=iz\hat f(z)$

\begin{theorem}
$\widehat{f(\frac{x}{a})}(z)=a\hat f(az)$.
\end{theorem}

\begin{proof}
$\widehat{f(\frac{x}{a})}(z)=\int\limits_{\mathbb{R}^n}f(\frac{x}{a})e^{-ixz}dx=
\int\limits_{\mathbb{R}^n}af(y)e^{-iayz}dx=a\hat f(az)$, где была
сделана замена $x=ay$.
\end{proof}

\begin{theorem}
$\hat f(x+a)(z)=e^{iaz}f(z)$.
\end{theorem}

\begin{proof}
$\hat f(x+a)(z)=\int\limits_{\mathbb{R}^n}f(x+a)e^{-ixz}dx\p=
\int\limits_{\mathbb{R}^n}f(y)e^{-i(y-a)z}dx\p=e^{iaz}\hat f(z)$,
где была сделана замена $x=y+a$.
\end{proof}

\begin{theorem}[Равенство Парсеваля]
Если функции $f,g\in \mathcal{L}_1(\mathbb{R})$, то
$\int\limits_\mathbb{R}\hat
f(x)g(x)dx=\int\limits_\mathbb{R}f(x)\hat g(x)dx$.
\end{theorem}

\begin{proof}
Согласно теореме Фубини, имеем:
\begin{multline*}
\int\limits_\mathbb{R}\hat f(x)g(x)dx=
\int\limits_\mathbb{R}\Big(\int\limits_\mathbb{R}f(z)e^{-ixz}dz\Big)g(x)dx=\\=
\int\limits_\mathbb{R}\Big(\int\limits_\mathbb{R}g(x)e^{-ixz}dx\Big)f(z)dz=
\int\limits_\mathbb{R}\hat g(z)f(z)dz.
\end{multline*}
Применение теоремы Фубини возможно, поскольку
$\iint\limits_{\mathbb{R}^2}|f(x)||g(x)|dzdx\p=\int
|f(z)|dz\cdot\int |g(x)|dx <\infty$.
\end{proof}

\begin{ex}
  Пусть $f(x)=\frac{1}{\sqrt{2\pi}}e^{-\frac{x^2}{2}}$. Тогда $(\hat
  f)'(z)=\widehat {-ixf(x)}(z)=\widehat {if'}=i\cdot iz\hat
  f(z)\p=-z\hat f(z)$. Получаем дифференциальное уравнение: $(\hat
  f)'(z)=-z\hat f(z)$. Общее решение $\hat f(z)=Ce^{-\frac{x^2}{2}}$.
  Константа $C$ определяется из условия $\hat f(0)=C$. Но $\hat
  f(0)=\int\limits_{\mathbb{R}}f(x)dx=1$. Поэтому $\hat
  f(z)=e^{-\frac{z^2}{2}}$.
\end{ex}

\begin{note}
$S\subset \mathcal{L}_1$.
\end{note}

\begin{theorem}
$\forall\,\varphi\in S\;\; \hat \varphi\in S$ и отображение $\varphi
\mapsto \hat \varphi$ непрерывно.
\end{theorem}

\begin{proof}
Проверим, что $\forall\, n,k\;\; \sup\limits_{x\in\mathbb{R}}
(1\p+|x|^{2k})\cdot |\hat {\varphi}^{(n)}(x)|<\infty$.
%тут нечетко
$(1+x^{2k})\widehat {(-iz)^n\varphi(z)}(x)+\widehat
{((-iz)^n\varphi)^{2k}}(x)\cdot (-i)^{2k}$. Оценивая по модулю это
выражение, получаем требуемое.%тут надо дополнить
\end{proof}

\begin{theorem}
$\varphi\in S$ $\Rightarrow$ $\forall\, P,\; \forall\,
m=0,1,2,\ldots\;\; P(x)\varphi^m(x)\in S$, где $P$ --- многочлен.
\end{theorem}

\begin{theorem}
Если $\varphi_n \xrightarrow[n\rightarrow \infty]{S} 0$, то
$P\varphi_{n}^{(m)} \xrightarrow[n\rightarrow \infty]{S} 0$.
\end{theorem}

\begin{theorem}
$\varphi_n \xrightarrow[n\rightarrow \infty]{S} 0 \Rightarrow
\varphi_n \xrightarrow[n\rightarrow \infty]{\mathcal{L}_1} 0$.
\end{theorem}

\begin{proof}
$\varphi_n(x)=\frac{1}{1+x^2}\cdot (1+x^2)\varphi_n(x)$.

$|\varphi_n(x)|=\Big|\frac{1}{1+x^2}\cdot
(1+x^2)\varphi_n(x)\Big|\leqslant \|\varphi\|\cdot \frac{1}{1+x^2}$.
$\|\varphi_n\|_{\mathcal{L}_1}=\int\limits_\mathbb{R}|\varphi_n(x)|dx\p=\|\varphi_n\|_{2,0}\cdot
\Big(\int\limits_\mathbb{R}\frac{dx}{1+x^2}\Big)\leqslant
C\|\varphi_n\|_{2,0}$.
\end{proof}

\begin{note}
Из этой теоремы следует, что отображение $\varphi \mapsto \hat
\varphi$ непрерывно.
\end{note}

\begin{theorem} Пусть $\Lambda \colon S \rightarrow S$ --- преобразование Фурье.
Тогда $$\varphi
(x)\p=\frac{1}{(2\pi)^n}\int\limits_{\mathbb{R}^n}e^{i(x,z)}\hat
\varphi(z)dz.$$
\end{theorem}

\begin{proof}
Действительно,
$$\int\limits_\mathbb{R}\varphi(\frac{x}{a})\hat\psi(x)dx=
a\int\limits_\mathbb{R}\varphi(y)\hat\psi(ay)dy=
\int\limits_\mathbb{R}\varphi(y)\hat\psi(\frac{x}{a})(y)dy=
\int\limits_\mathbb{R}\hat\varphi(x)\psi(\frac{x}{a})dx.$$ При
$a\rightarrow \infty$ получаем
$\varphi(0)\int\limits_\mathbb{R}\hat\psi(x)dx$=
$\psi(0)\int\limits_\mathbb{R}\hat\varphi(x)dx$.

Если $\psi (x)=\frac{1}{\sqrt{2\pi}}e^{-\frac{x^2}{2}}$, то
$\frac{1}{\sqrt{2\pi}}\int\limits_{\mathbb{R}}\hat\varphi (x)dx=$
$\varphi (0)\int\limits_{\mathbb{R}}e^{-\frac{x^2}{2}}dx=$ $\varphi
(0)\sqrt{2\pi}$. Отсюда $\varphi
(0)=\frac{1}{2\pi}\int\limits_{\mathbb{R}}\hat\varphi (x)dx$.
Положим $\varphi_1(x)=\varphi (x+z)$, тогда $\varphi_1\in S$.
$\varphi
(z)=\varphi_1(0)=\frac{1}{2\pi}\int\limits_{\mathbb{R}}\hat\varphi_1
(x)dx=$$\frac{1}{2\pi}\int\limits_{\mathbb{R}}\widehat{\varphi
(x_1+z)}(x)dx=$
$\frac{1}{2\pi}\int\limits_{\mathbb{R}}e^{izx}\hat\varphi (x)dx$.
\end{proof}

Обозначим $\check \varphi(z)=
\frac{1}{2\pi}\int\limits_{\mathbb{R}}e^{izx}\varphi(x)dx$.

\begin{prop}
Если $\varphi\in S$, то $\check\varphi\in S$.
\end{prop}

\begin{prop}
$\Lambda$ --- сюръекция.
\end{prop}

\begin{proof}
$\varphi\in S$ $\Rightarrow$ $\check\varphi\in S$.
$\check{\hat\varphi}=\varphi$ $\Rightarrow$ $\varphi\in \Im
\Lambda$.
\end{proof}

\section{Преобразование Фурье обобщенных функций}

\begin{df}
\emph{Преобразованием Фурье для $F\in S^*$} называется обобщенная
функция $\hat F : (\hat F,\varphi)=(F,\hat\varphi)$.
\end{df}

\begin{prop}
$\hat F$ непрерывно на $S$.
\end{prop}

\begin{proof}
Пусть $\varphi_n\rightarrow\varphi$, проверим, что $(\hat
F,\varphi_n)\rightarrow (\hat F,\varphi)$.

$(\hat F,\varphi_n)=(F,\widehat{\varphi_n})\rightarrow
(F,\hat\varphi)=(\hat F,\varphi)$.
\end{proof}
%---------------------------Lecture 16-----------------------------------------------%



\begin{df}
\emph{Преобразованием Фурье функции $F\in D^*$} называется
обобщенная функция $\hat{F}$, такая, что
$(\hat{F},\varphi)=(F,\hat{\varphi})$. Здесь $\varphi\in
\mathcal{Z}=\hat{D}$ и $\hat{\varphi}\in D$. Т.е.,
$\hat{\mathcal{Z}}=\check{\mathcal{Z}}=D$ и
$\hat{F}\in\mathcal{Z}^*$.
\end{df}

В дальнейшем мы будем использовать следующие обозначения:
$$(F,\varphi)=F(\varphi)=\int\limits_{\mathbb{R}^1}\!F(x)\varphi(x)\,dx.$$
Если же $g\in\mathcal{L}^{\mathrm{loc}}_1(\mathbb{R}^1)$, то положим
$$(F_g,\varphi)=F_g(\varphi)=(g,\varphi)=\int\limits_{\mathbb{R}^1}\!g(x)\varphi(x)\,dx.$$

\begin{df}
\emph{Регуляризацией функции $g$} (или обобщенной функции $F_g$)
называется продолжение $F_g$ на все пространство $D$ с сохранением
непрерывности.
\end{df}

\begin{note}
Продолжение вовсе не обязано быть единственным!
\end{note}

\begin{problem}
Докажите, то регуляризации функции $g(x)=x^{-n}$ образуют
подпространство размерности $n$ в $D^*$.
\end{problem}

\begin{problem}
Обозначим через $D_0^k$ множество функций из $D$, которые равны 0
вместе со всеми своими производными порядка не больше $k$ на
некотором интервале $(-\varepsilon;\varepsilon)$. Пусть также
$g(x)=x^{-n}$. Докажите, что

1. Обобщенная функция $F_g$ непрерывна на $D_0=D_0^0$.

2. Обобщенная функция $F_g$ однозначно продолжается на $D_0^{n-1}$.

3. $\exists\,K:K\oplus D_0^{n-1}=D$ и $\dim K=n$.
\end{problem}

\begin{theorem}
Пусть $F\in D^*$. Тогда выполняются следующие свойства:

1. $\hat{F}'=\widehat{-iz F(z)}$.

2. $\widehat{F'}=ix\hat{F}$.
\end{theorem}

\begin{proof}
В самом деле,
\begin{multline*}
(\hat{F}',\varphi)=-(\hat{F},\varphi')=-(F,\widehat{\varphi'})=-\int\limits_{\mathbb{R}^1}\!
F(z)(iz\hat{\varphi}(z))\,dz=\\=
-\int\limits_{\mathbb{R}^1}\!izF(z)\hat{\varphi}(z)\,dz=-\int\limits_{\mathbb{R}^1}\!
\widehat{izF(z)}(x)\varphi(x)\,dx=(\widehat{-izF(z)},\varphi),
\end{multline*}
а также

\begin{multline*}
(\widehat{F'},\varphi)=(F',\hat{\varphi})=-(F,\hat{\varphi}')=-(F,\widehat{-ix\varphi(x)})=
-\int\limits_{\mathbb{R}^1}\!\hat{F}(x)(-ix\varphi(x))\,dx=\\=
\int\limits_{\mathbb{R}^1}\!ix\hat{F}(x)\varphi(x)\,dx=(ix\hat{F}(x),\varphi).
\end{multline*}
\end{proof}

\begin{prop}
Если $F\in D^*$ и $F$ обладает компактным носителем, то
$\hat{F}(z)=(F,[x\mapsto e^{-ixz}])$.
\end{prop}

Строгое доказательство мы дадим позже, а пока что приведем
правдоподобное рассуждение, позволяющее обосновать это предложение.
А именно, $\forall\,\varphi\in
D$\;\;$(\hat{F},\varphi)=\int\limits_{\mathbb{R}^1}\!\hat{F}(z)\varphi(z)\,dz=
\int\limits_{\mathbb{R}^1}\!\Big(\int\limits_{\mathbb{R}^1}\!F(x)e^{-ixz}\,dx\Big)\varphi(z)\,dz$;
с другой стороны,
$(\hat{F},\varphi)=(F,\hat{\varphi})=\int\limits_{\mathbb{R}^1}\!F(x)\Big(\int\limits_{\mathbb{R}^1}\!
e^{-ixz}\varphi(z)\,dz\Big)\,dx$. Если бы речь шла об обычных
функциях, то правые части этих равенств были бы равны в силу теоремы
Фубини. Однако для обобщенных функций, формально говоря, применять
эту теорему нельзя. Поэтому это рассуждение не может считаться
строгим доказательством.
\section{Отдельная лекция}
\begin{petit}
  В комплекте к этим лекциям идет отдельная лекция, датированная
  27.11.2007. Сомнительно, что это важно, но содержимое лекции неплохо
  было бы влить в общий поток сознания.
\end{petit}

\begin{theorem}[Неравенство Коши-Буняковского (в комплексном случае)]
Пусть $H$ --- гильбертово пространство над $\mathbb{C}$ и $f,g\in
H$. Тогда верно неравенство $|(f,g)|\leqslant \|f\|\|g\|.$
\end{theorem}
\begin{proof}
  \begin{petit}
    Кажется, это дубль.
  \end{petit}
  Если $\|g\|=0$, то $g=0$ и неравенство Коши-Буняковского
  выполнено.

  Если же $(f,g)=0$, то неравенство Коши-Буняковского тоже
  выполнено.

  Ну а если $\|g\|(f,g)\neq0$, то пусть $\lambda$ --- произвольное
  действительное число. Тогда
  $(f-\lambda(f,g)g,f-\lambda(f,g)g)\geqslant 0.$ Значит,
  $(f,f)-2\lambda|(f,g)|^2+\lambda^2|(f,g)|^2\|g\|^2\geqslant0.$ Это
  квадратный трехлен, неотрицательный при всех $\lambda$, поэтому его
  дискриминант неотрицателен. Получаем, что $|(f,g)|^4\leqslant
  \|f\|\|g\||(f,g)|^2$, откуда следует $|(f,g)|\leqslant \|f\|\|g\|.$
\end{proof}

\begin{theorem}
  Пусть $A$ --- произвольный линейный непрерывный оператор
  в гильбертовом пространстве $H$. Тогда $\ker A=(Im A^*)^{\perp}$,
  где $A^*$ --- оператор, сопряженный к оператору $A$ и $Im C=\{Cx:
  x\in H\}$.
\end{theorem}

\begin{proof}
  Проверим включение в одну сторону. $x\in \ker A
  \Longrightarrow \forall z\in H (Ax,z)=(x.A^*z)$. Итак, любой
  элемент ядра оператора перпендикулярен любому элементу образа.

  Обратно: $x\perp ImA^*\Longrightarrow \forall z\in H (x,A^*z)=0
  \Longrightarrow \forall z\in H (Ax,z)=0 \Longrightarrow Ax=0
  \Longrightarrow x\in \ker A$.
\end{proof}

\begin{imp}
  $(\ker A)^{\perp}=(ImA^*)^{\perp\perp}=\overline{ImA^*}$.
\end{imp}

\begin{imp}
  $\ker A^*=(ImA)^{\perp}$.
\end{imp}

\begin{imp}
  $(\ker A^*)^{\perp}=\overline{ImA}$.
\end{imp}

\begin{theorem}
  Пусть теперь $A\colon E\longrightarrow G$, где $E$ и $G$---
  банаховы пространства. Тогда для оператора $A^*\colon
  G^*\longrightarrow E^*$ верны те же свойства (там $(Im
  A^*)^{\perp}=\{x\in E:\forall g\in Im A^*, g(x)=0\}$).
\end{theorem}

\begin{proof}
  Доказательства аналогичны предыдущим.
\end{proof}

\subsection{Обобщенные функции}


Определим три пространства так называемых пробных функций:
\begin{enumerate}

\item $D=D(\mathbb{R}^n)$ --- пространство всех бесконечно
  дифференцируемых функций (действительнозначных или комплекснозначных)
  с компактным носителем.

\item $S=S(\mathbb{R}^n)$ --- пространство всех бесконечно
  дифференцируемых быстро убывающих функций (действительнозначных
  или комплекснозначных).

\item $E=E(\mathbb{R}^n)$ --- пространство всех бесконечно
  дифференцируемых функций (действительнозначных или
  комплекснозначных).
\end{enumerate}
\begin{df}
  Носителем функции $\varphi$ называется множество
  $supp\varphi=\overline{\{x: \varphi(x)\neq0\}}$.
\end{df}

\begin{df}
  $\varphi$ --- быстро убывающая функция, если $\forall
  n,k  p_{n,k}=\sup_x(1+\|x\|^n)\|\varphi^k(x)\|<\infty$.
\end{df}

\begin{df}
  Здесь $\|\varphi^k(x)\|=\sum_{\sum
    r_j=k}|\frac{\partial^k\varphi(x)}{\partial x^{r_1}_{1 } \ldots
    \partial  x^{r_n}_{n }}|$.
\end{df}

Каждое из пространств $D, S, E$ является линейным. В этих
пространствах задается топология следующим образом.

Начнем с пространства $S$. В нем топологию определяет система
полунорм $P=\{p_{n,k}: n,k=0,1,2,\ldots\}$.

Теперь рассмотрим пространство $D$. $D=\bigcup_{r=1}^{\infty}D_r$,
где $D_{r}=\{\varphi\in D: supp \varphi\subset S(0,r)\}$. Но $D_r
\subset D\subset S$. Получаем в $D_r$ индуцированную топологию из
$S$.

Определим фундаментальную систему окрестностей нуля:

$W\in V \Longleftrightarrow W $ выпукло и $\forall r\in \mathbb{N}
W\bigcap D_r$ --- открытая окрестность нуля в $D_r$.

Определение. $V\subset D$ открыто тогда и только тогда, когда оно
представляет собой объединение (возможно) сдвинутых окрестностей
нуля.

Теперь определим топологию $P_E$ в $E$. Для этого снабдим
пространство $E$ семейством полунорм:

$p\in P_E \Longleftrightarrow \exists$ компакт $K\subset
\mathbb{R}^n$ и $r,k\in \{0,1,2,\ldots\} \forall \varphi \in
E=\sup_{x\in K}\sum_{\sum
r_j=r}|\frac{\partial^k\varphi(x)}{\partial x^{r_1}_{1 } \ldots
\partial  x^{r_n}_{n }}|$.

Это семейство полунорм и задает топологию.

\begin{theorem}
  Пусть $E$ хаусдорфовое и локально выпуклое пространство и
  топология может быть задана не более чем счетным семейством
  полунорм. Тогда $E$ метризуемо.
\end{theorem}

\begin{proof}
  Если $\{p_n\}$ --- полунормы из условия, то
  метрика такова:
  $\rho(\varphi,\phi)=\sum_{n=1}^{\infty}\frac{p_n(\varphi-\phi)}{2^n(1+p_n(\varphi-\phi))}$.
\end{proof}
\clearpage

\begin{center}
\textbf{\textsc{Приложение. \\ Экзаменационные вопросы.}}
\end{center}

\vspace{7pt}

\noindent 1. Равносильность счетной компактности и секвенциальной
компактности для подмножеств метрических пространств.

\noindent 2. Доказательство того, что всякое секвенциально
компактное подмножество  метрического пространства полно и
предкомпактно.

\noindent 3. Доказательство того, что всякое компактное подмножество
метрического пространства секвенциально компактно.

\noindent 4. Доказательство того, что всякое полное предкомпактное
подмножество метрического пространства компактно.

\noindent 5. Теорема о вложенных шарах.

\noindent 6. Теорема Бэра.

\noindent 7. Теорема о пополнении метрического пространства.

\noindent 8. Равносильность непрерывности и ограниченности для
отображений метрических пространств.

\noindent 9. Доказательство того, что непрерывный образ компактного
множества является компактным множеством.

\noindent 10. Доказательство того, что непрерывное отображение
компактного ме\-трического пространства в метрическое пространство
равномерно непрерывно.

\noindent 11. Теорема Банаха--Штейнхауза.

\noindent 12. Теорема Хана--Банаха для линейных функционалов на
линейных пространствах.

\noindent 13. Теорема Хана--Банаха для линейных функционалов на
линейных нормированных пространствах (над полем вещественных чисел).

\noindent 14. Сохранение нормы при каноническом вложении
нормированного линейного пространства в его второе сопряженное.

\noindent 15. Теорема о пополнении нормированного линейного
пространства.

\noindent 16. Полнота нормированного линейного пространства линейных
непрерывных отображений нормированного линейного пространства в
банахово пространство.

\noindent 17. Теорема Банаха о гомоморфизме.

\noindent 18. Равносильность теорем Банаха об обратном отображении и
о замкнутом графике.

\noindent 19. Теорема Банаха о замкнутом графике.

\noindent 20. Теорема Рисса--Фишера.

\noindent 21. Теорема Рисса об общем виде линейного непрерывного
функционала на гильбертовом пространстве.

\noindent 22. Лемма о трех гомоморфизмах.

\noindent 23. Всякий линейный функционал, непрерывный в слабой
топологии, является элементом пространства, задающего слабую
топологию.

\noindent 24. Ограниченность слабо сходящейся последовательности в
нормированном пространстве.

\noindent 25. Для выпуклых множеств в нормированном пространстве
замкнутость в слабой топологии и в топологии, определяемой нормой,
равносильны.

\noindent 26. Для подмножеств нормированного линейного пространства
ограниченность по норме и слабая ограниченность равносильны.

\noindent 27. Неравенство Бесселя.

\noindent 28. Существование ортонормированного базиса в
сепарабельном евклидовом пространстве.

\noindent 29. Если $A$ --- линейный непрерывный оператор в
гильбертовом пространстве, то $\|A\|=\|A^*\|$.

\noindent 30. Для счетной ортонормированной системы векторов в
евклидовом пространстве тотальность, замкнутость и свойство быть
базисом равносильны и каждое из этих свойств влечет полноту
ортонормированной системы.

\noindent 31. Пространства $\mathcal{D}$, $\mathcal{S}$,
$\mathcal{E}$. Плотность образов при вложениях $\mathcal{D}\subset
\mathcal{S}$, $\mathcal{S}\subset\mathcal{E}$.

\noindent 32. непрерывность преобразования Фурье в пространстве
$\mathcal{S}$. Формула обращения для преобразования Фурье в
пространстве $\mathcal{S}$.

\noindent 33. Вложение локально интегрируемых функций и локально
конечных мер в пространство $\mathcal{D}^*$.

\noindent 34. Операции над обобщенными функциями. Связь
дифференцирования и преобразования Фурье.

\noindent 35. Прямые и обратные образы обобщенных функций при
отображениях пространств.

\noindent 36. Преобразование Фурье интегрируемых функций.

\noindent 37. Теорема Хана--Банаха для отображения комплексных
пространств.

\noindent 38. Теорема о существовании ортогональной проекции.
\end{document}


%% Local Variables:
%% eval: (setq compile-command (concat "latex  -halt-on-error -file-line-error " (buffer-name)))
%% End:
