\documentclass[12pt,titlepage, a4paper]{article}
\usepackage[utf8]{inputenc}
\usepackage[russian]{babel}
\usepackage{amsmath,amssymb}
\begin{document}
Докажем единственность. Пусть есть два элемента $h_f$ и
$\bar{h}_f$, удовлетворяющие условию. Тогда $\forall\,x\in H$\
$(x,h_f)=(x,\bar{h}_f)$. Отсюда $\forall\,x\in H$\
$(x,h_f-\bar{h}_f)=0$. Подставив $x=h_f-\bar{h}_f$, получим
$(h_f-\bar{h}_f,h_f-\bar{h}_f)=0\Rightarrow
h_f-\bar{h}_f=0\Rightarrow \ \Rightarrow h_f=\bar{h}_f.$

\; Теорема. Пусть $G\subset H$ --- замкнутое подпространство
гильбертова пространства, $h\in H, \exists\ pr_{G}h\in G $ и
$\exists\ z\in G, (h-z)\bot G$. Тогда $pr_{G}h=z$.\;

Доказательство. $(h-pr_{G}h)\bot G, (h-z)\bot G \Rightarrow
(h-pr_{G}h)-(h-z)=$$(z--pr_{G}h)\bot G$. Но $(z-pr_{G}h)\in G$.
Следовательно, $(z-pr_{G}h,z-pr_{G}h)=0\;$$ \Rightarrow
z-pr_{G}h=0\;$$\Rightarrow z=pr_{G}h$.

Определение. Пусть $E_1$ и $E_2$ - линейные пространства над полем
$\mathbb{C}$. Отображение $f:E_1\rightarrow E_2$ называется
полулинейным, если $\forall g_1, g_2\in E_1$ и $\lambda \in
\mathbb{C}$ выполнено $f(g_1+g_2)=f(g_1)+(g_2)$ и $f(\lambda
g_1)=\bar{\lambda} f(g_1)$.

Замечание. Пусть $E$ --- гильбертово пространство и $f\in E^*$. По
теореме Рисса  $\exists!\,h_f\in E:\forall\,x\in
E$\;\;$f(x)=(x,h_f)$ Тогда отображение $F\colon E^*\rightarrow E$,
$F(f)=h_f$ полулинейно.

Теорема.  Пусть $E$ --- произвольное линейное пространство, и
$p\colon E\to\mathbb{R}^+$
--- такая функция на нем, что выполняются следующие свойства:

1) $p(\alpha x)=|\alpha| p(x)$;

2) $p(x_1+x_2)\leqslant p(x_1)+p(x_2)$.

Пусть также $G\subset E$ --- подпространство и $f\colon
G\to\mathbb{C}$ --- линейный функционал на нем, причем
$\forall\,x\in G$\;\;$|f(x)|\leqslant p(x)$. Тогда
$\exists\,\tilde{f}\colon E\to \mathbb{C}$ --- такое линейное
отображение, что $\forall\,x\in E$\;\;$|\tilde{f}(x)|\leqslant
p(x)$ и $\forall\,x\in G$\;\;$\tilde{f}(x)=f(x)$.

Доказательство. Положим $\forall x\in G\ f_1(x)=Re f(x)$. Тогда
$f_1\colon G_{\mathbb{R}}\rightarrow \mathbb{R}$ линеен и $\forall
x \in G\;\; f(x)=f_1(x)-if_1(ix)$. По условию $\forall\,x\in
G$\;\;$|f(x)|\leqslant p(x)$, значит, $\forall\,x\in
G$\;\;$|f_1(x)|\leqslant p(x)$. По теореме Хана-Банаха найдем
такой линейный функционал $\tilde{f_1}\colon
E_{\mathbb{R}}\rightarrow \mathbb{R}$, что $\forall x\in
E_{\mathbb{R}} \;\;|\tilde{f_1}(x)|\leqslant p(x)$. Положим
$\forall x \in E\;\;
\tilde{f}(x)=\tilde{f_1}(x)-i\tilde{f_1}(ix)$. Тогда $\tilde{f}$ и
будет искомым функционалом. Надо лишь проверить, что
$\forall\,x\in E$\;\;$|\tilde{f}(x)|\leqslant p(x)$.

Допустим, что это не так, т.е. $\exists x\in E:
|\tilde{f}(x)|>p(x)$. Пусть $\tilde{f}(x)=\rho e^{i\theta}$. Тогда
$\tilde{f}(e^{-i\theta}x)=e^{-i\theta}\tilde{f}(x)=e^{-i\theta}\rho
e^{i\theta}=\rho>0$. Поэтому
$|\tilde{f}(e^{-i\theta}x)|=\;\;=\tilde{f}(e^{-i\theta}x)=\tilde{f_1}(e^{-i\theta}x)\Rightarrow
\tilde{f_1}(e^{-i\theta}x)>p(x)=p(e^{-i\theta}x)$. Противоречие.

Определение. Пусть E и G --- банаховы пространства и $A\colon
E\rightarrow G$ --- линейный оператор. Тогда банахов сопряженный
оператор $A^*\colon G^*\rightarrow E^*$ определяется следующим
образом: $\forall g\in G^* \;\;\forall x\in E
\;\;g(Ax)=(A^*g)(x)$.

Предложение. $A^*$ линеен и непрерывен.

Доказательство. Линейность очевидно вытекает из определения. Для
проверки непрерывности докажем более сильное утверждение:
$\|A^*\|=\|A\|$.

$\|A\|=\sup\limits_{\substack{x\in F\\ \|x\|\leqslant
1}}\|Ax\|=\sup\limits_{\|x\|\leqslant
1}\sup\limits_{\substack{g\in G^*\\ \|g\|\leqslant
1}}|g(Ax)|=\sup\limits_{\|g\|\leqslant
1}\sup\limits_{\|x\|\leqslant
1}|g(Ax)|=\sup\limits_{\|g\|\leqslant
1}\sup\limits_{\|x\|\leqslant
1}|(A^*g)(x)|=\;=\sup\limits_{\|g\|\leqslant 1}\|A^*g\|=\|A^*\|$.

Определение. Пусть H --- гильбертово пространство и $A\colon
H\rightarrow H$ --- линейный оператор. Тогда гильбертов
сопряженный оператор $A^*\colon H\rightarrow H$ определяется
следующим образом: $\forall x\in H \;\;\forall z\in H
\;\;(A^*x,z)=(x,Az)$.
\end{document}
