\documentclass[12pt,titlepage, a4paper]{article}
\usepackage[utf8]{inputenc}
\usepackage[russian]{babel}
\usepackage{amsmath,amssymb}
\begin{document}

$\hat f(z)=\int_{\mathbb{R}}e^{-ixz}f(x)dx$.

Теорема. Если $f\in L_1(\mathbb{R})$ и $[x\mapsto xf(x)]\in
L_1(\mathbb{R})$, то $(\hat f)'(z)=\widehat{-ixf(x)}$.

Доказательство. $\forall \alpha ,\beta
|e^{i\alpha}-e^{i\beta}|\leqslant |\alpha-\beta|$. Отсюда
$|\frac{e^{-ix(z+\Delta z)}-e^{-ixz}}{\Delta z}|\leqslant |x|$.
Поэтому $\hat f'(z)=\lim_{\Delta z\longrightarrow
0}\int_{\mathbb{R}} |\frac{e^{-ix(z+\Delta z)}-e^{-ixz}}{\Delta
z}|=\int_{\mathbb{R}}-ixe^{-ixz}f(x)dx$.

Итак, $(\hat f)'(z)=\widehat{-ixf(x)}(z)$

$\widehat {f'}(z)=iz\hat f(z)$

Теорема.$\widehat{f(\frac{x}{a})}(z)=a\hat f(az)$.

Доказательство.
$\widehat{f(\frac{x}{a})}(z)=\int_{\mathbb{R}^n}f(\frac{x}{a})e^{-ixz}dx=
\int_{\mathbb{R}^n}af(y)e^{-iayz}dx=a\hat f(az)$, где была сделана
замена $x=ay$.

Теорема.$\hat f(x+a)(z)=e^{iaz}f(z)$.

Доказательство. $\hat
f(x+a)(z)=\int_{\mathbb{R}^n}f(x+a)e^{-ixz}dx=
\int_{\mathbb{R}^n}f(y)e^{-i(y-a)z}dx=e^{iaz}\hat f(z)$, где была
сделана замена $x=y+a$.

Теорема. Равенство Парсеваля.

Если $f,g\in L_1(\mathbb{R})$, то $\int_\mathbb{R}\hat
f(x)g(x)dx=\int_\mathbb{R}f(x)\hat g(x)dx$.

Доказательство. $\int_\mathbb{R}\hat f(x)g(x)dx=
\int_\mathbb{R}(\int_\mathbb{R}f(z)e^{-ixz}dz)g(x)dx=
\int_\mathbb{R}(\int_\mathbb{R}g(x)e^{-ixz}dx)f(z)dz=
\int_\mathbb{R}\hat g(z)f(z)dz$.

Здесь была использована теорема Фубини. Это возможно, поскольку
$\iint_{\mathbb{R}^2}|f(x)||g(x)|dzdx=\int |f(z)|dz\cdot\int
|g(x)|dx <\infty$.

Пример. Пусть $f(x)=\frac{1}{\sqrt{2\pi}}e^{-\frac{x^2}{2}}$.
Тогда $(\hat f)'(z)=\widehat {-ixf(x)}(z)=\widehat {if'}=i\cdot
iz\hat f(z)=-z\hat f(z)$. Получаем обыкновенное дифференциальное
уравнение $(\hat f)'(z)=-z\hat f(z)$. Общее решение $\hat
f(z)=Ce^{-\frac{x^2}{2}}$. Константа $C$ определяется из условия
$\hat f(0)=C$. Но $\hat f(0)=\int_{\mathbb{R}}f(x)dx=1$. Поэтому
$\hat f(z)=e^{-\frac{z^2}{2}}$.

Замечание. $S\subset L_1$.

Теорема. $\forall\varphi\in S \hat \varphi\in S$ и отображение
$\varphi \mapsto \hat \varphi$ непрерывно.

Доказательство. Проверим, что $\forall n,k \sup_{x\in\mathbb{R}}
(1+|x|^{2k})\cdot |\hat {\varphi}^{(n)}(x)|<\infty$.
%тут нечетко
$(1+x^{2k})\widehat {(-iz)^n\varphi(z)}(x)+\widehat
{((-iz)^n\varphi)^{2k}}(x)\cdot (-i)^{2k}$. Оценивая по модулю это
выражение, получаем требуемое.%тут надо дополнить

Теорема. $\varphi\in S \Rightarrow \forall P \forall
m=0,1,2,\ldots P(x)\varphi^m(x)\in S$, где $P$ --- многочлен.

Теорема. Если $\varphi_n \xrightarrow[n\rightarrow \infty]{S} 0$,
то $P\varphi_{n}^{(m)} \xrightarrow[n\rightarrow \infty]{S} 0$.

Теорема. $\varphi_n \xrightarrow[n\rightarrow \infty]{S} 0
\Rightarrow \varphi_n \xrightarrow[n\rightarrow \infty]{L_1} 0$.

Доказательство. $\varphi_n(x)=\frac{1}{1+x^2}\cdot
(1+x^2)\varphi_n(x)$.

$|\varphi_n(x)|=|\frac{1}{1+x^2}\cdot
(1+x^2)\varphi_n(x)|\leqslant \|\varphi\|\cdot \frac{1}{1+x^2}$.
$\|\varphi_n\|_{L_1}=\int_\mathbb{R}|\varphi_n(x)|dx=\|\varphi_n\|_{2,0}\cdot
(\int_\mathbb{R}\frac{dx}{1+x^2})\leqslant C\|\varphi_n\|_{2,0}$.

Замечание. Из этой теоремы следует, что отображение $\varphi
\mapsto \hat \varphi$ непрерывно.

Теорема. Пусть $\Lambda \colon S \rightarrow S, \Lambda
(\varphi)=\hat \varphi$. Тогда $\varphi
(x)=\frac{1}{(2\pi)^n}\int_{\mathbb{R}^n}e^{i(x,z)}\hat
\varphi(z)dz$.

Доказательство (в одномерном случае). Пусть $\varphi, \phi\in S$.
Тогда $\int_\mathbb{R}\varphi(\frac{x}{a})\hat\psi(x)dx$=
$a\int_\mathbb{R}\varphi(y)\hat\psi(ay)dy$=
$\int_\mathbb{R}\varphi(y)\hat\psi(\frac{x}{a})(y)dy$=
$\int_\mathbb{R}\hat\varphi(x)\psi(\frac{x}{a})dx$. При
$a\rightarrow \infty$ получаем
$\varphi(0)\int_\mathbb{R}\hat\psi(x)dx$=
$\psi(0)\int_\mathbb{R}\hat\varphi(x)dx$.

Пусть $\psi (x)=\frac{1}{\sqrt{2\pi}}e^{-\frac{x^2}{2}}$. Тогда
$\frac{1}{\sqrt{2\pi}}\int_{\mathbb{R}}\hat\varphi (x)dx=$
$\varphi (0)\int_{\mathbb{R}}e^{-\frac{x^2}{2}}dx=$ $\varphi
(0)\sqrt{2\pi}$. Отсюда $\varphi
(0)=\frac{1}{2\pi}\int_{\mathbb{R}}\hat\varphi (x)dx$. Положим
$\varphi_1(x)=\varphi (x+z)$,Тогда $\varphi_1\in S$. $\varphi
(z)=\varphi_1(0)=\frac{1}{2\pi}\int_{\mathbb{R}}\hat\varphi_1
(x)dx=$$\frac{1}{2\pi}\int_{\mathbb{R}}\widehat{\varphi
(x_1+z)}(x)dx=$ $\frac{1}{2\pi}\int_{\mathbb{R}}e^{izx}\hat\varphi
(x)dx$.

Итак, $\varphi(z)=
\frac{1}{2\pi}\int_{\mathbb{R}}e^{izx}\hat\varphi(x)dx$.

Обозначим $\check \varphi(z)=
\frac{1}{2\pi}\int_{\mathbb{R}}e^{izx}\varphi(x)dx$.

Предложение. Если $\varphi\in S$, то $\check\varphi\in S$.

Предложение. $\Lambda$ --- сюръекция.

Доказательство. $\varphi\in S\Rightarrow \check\varphi\in S$.
$\check{\hat\varphi}=\varphi\Rightarrow \varphi\in Im \Lambda$.

\begin{center}
Преобразование Фурье обобщенных функций
\end{center}

Определение. Преобразованием Фурье для $F\in S^*$ называется
обобщенная функция $\hat F : (\hat F,\varphi)=(F,\hat\varphi)$.

Предложение. $\hat F$ непрерывно на $S$.

Доказательство. Пусть $\varphi_n\rightarrow\varphi$, проверим, что
$(\hat F,\varphi_n)\rightarrow (\hat F,\varphi)$.

$(\hat F,\varphi_n)=(F,\widehat{\varphi_n})\rightarrow
(F,\hat\varphi)=(\hat F,\varphi)$.
\end{document}
