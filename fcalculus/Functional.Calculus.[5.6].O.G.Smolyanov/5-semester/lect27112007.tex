\documentclass[12pt,titlepage, a4paper]{article}
\usepackage[cp1251]{inputenc}
\usepackage[russian]{babel}
\usepackage{amsmath,amssymb}
\begin{document}
Теорема. Неравенство Коши-Буняковского (в комплексном случае).
Пусть $H$ --- гильбертово пространство над $\mathbb{C}$ и $f,g\in
H$. Тогда верно неравенство $|(f,g)|\leqslant \|f\|\|g\|.$

Доказательство.

Если $\|g\|=0$, то $g=0$ и неравенство Коши-Буняковского
выполнено.

Если же $(f,g)=0$, то неравенство Коши-Буняковского тоже
выполнено.

Ну а если $\|g\|(f,g)\neq0$, то пусть $\lambda$ --- произвольное
действительное число. Тогда
$(f-\lambda(f,g)g,f-\lambda(f,g)g)\geqslant 0.$ Значит,
$(f,f)-2\lambda|(f,g)|^2+\lambda^2|(f,g)|^2\|g\|^2\geqslant0.$ Это
квадратный трехлен, неотрицательный при всех $\lambda$, поэтому
его дискриминант неотрицателен. Получаем, что $|(f,g)|^4\leqslant
\|f\|\|g\||(f,g)|^2$, откуда следует $|(f,g)|\leqslant
\|f\|\|g\|.$

Теорема. Пусть $A$ --- произвольный линейный непрерывный оператор
в гильбертовом пространстве $H$. Тогда $\ker A=(Im A^*)^{\perp}$,
где $A^*$ --- оператор, сопряженный к оператору $A$ и $Im C=\{Cx:
x\in H\}$.

Доказательство. Проверим включение в одну сторону. $x\in \ker A
\Longrightarrow \forall z\in H (Ax,z)=(x.A^*z)$. Итак, любой
элемент ядра оператора перпендикулярен любому элементу образа.

Обратно: $x\perp ImA^*\Longrightarrow \forall z\in H (x,A^*z)=0
\Longrightarrow \forall z\in H (Ax,z)=0 \Longrightarrow Ax=0
\Longrightarrow x\in \ker A$.

Следствия.

1. $(\ker A)^{\perp}=(ImA^*)^{\perp\perp}=\overline{ImA^*}$.

2. $\ker A^*=(ImA)^{\perp}$.

3. $(\ker A^*)^{\perp}=\overline{ImA}$.

Пусть теперь $A\colon E\longrightarrow G$, где $E$ и $G$---
банаховы пространства. Тогда для оператора $A^*\colon
G^*\longrightarrow E^*$ верны те же свойства (там $(Im
A^*)^{\perp}=\{x\in E:\forall g\in Im A^*, g(x)=0\}$).

Доказательства аналогичны предыдущим.

\begin{center}
Обобщенные функции
\end{center}

Определим три пространства так называемых пробных функций:

1. $D=D(\mathbb{R}^n)$ --- пространство всех бесконечно
дифференцируемых функций (действительнозначных или
комплекснозначных) с компактным носителем.

2. $S=S(\mathbb{R}^n)$ --- пространство всех бесконечно
дифференцируемых быстро убывающих функций (действительнозначных
или комплекснозначных).

3. $E=E(\mathbb{R}^n)$ --- пространство всех бесконечно
дифференцируемых функций (действительнозначных или
комплекснозначных).

Определение. Носителем функции $\varphi$ называется множество
$supp\varphi=\overline{\{x: \varphi(x)\neq0\}}$.

Определение. $\varphi$ --- быстро убывающая функция, если $\forall
n,k  p_{n,k}=\sup_x(1+\|x\|^n)\|\varphi^k(x)\|<\infty$.

Определение. Здесь $\|\varphi^k(x)\|=\sum_{\sum
r_j=k}|\frac{\partial^k\varphi(x)}{\partial x^{r_1}_{1 } \ldots
\partial  x^{r_n}_{n }}|$.

Каждое из пространств $D, S, E$ является линейным. В этих
пространствах задается топология следующим образом.

Начнем с пространства $S$. В нем топологию определяет система
полунорм $P=\{p_{n,k}: n,k=0,1,2,\ldots\}$.

Теперь рассмотрим пространство $D$. $D=\bigcup_{r=1}^{\infty}D_r$,
где $D_{r}=\{\varphi\in D: supp \varphi\subset S(0,r)\}$. Но $D_r
\subset D\subset S$. Получаем в $D_r$ индуцированную топологию из
$S$.

Определим фундаментальную систему окрестностей нуля:

$W\in V \Longleftrightarrow W $ выпукло и $\forall r\in \mathbb{N}
W\bigcap D_r$ --- открытая окрестность нуля в $D_r$.

Определение. $V\subset D$ открыто тогда и только тогда, когда оно
представляет собой объединение (возможно) сдвинутых окрестностей
нуля.

Теперь определим топологию $P_E$ в $E$. Для этого снабдим
пространство $E$ семейством полунорм:

$p\in P_E \Longleftrightarrow \exists$ компакт $K\subset
\mathbb{R}^n$ и $r,k\in \{0,1,2,\ldots\} \forall \varphi \in
E=\sup_{x\in K}\sum_{\sum
r_j=r}|\frac{\partial^k\varphi(x)}{\partial x^{r_1}_{1 } \ldots
\partial  x^{r_n}_{n }}|$.

Это семейство полунорм и задает топологию.

Теорема. Пусть $E$ хаусдорфовое и локально выпуклое пространство и
топология может быть задана не более чем счетным семейством
полунорм. Тогда $E$ метризуемо.

Доказательство. Если $\{p_n\}$ --- полунормы из условия, то
метрика такова:
$\rho(\varphi,\phi)=\sum_{n=1}^{\infty}\frac{p_n(\varphi-\phi)}{2^n(1+p_n(\varphi-\phi))}$.
\end{document}
