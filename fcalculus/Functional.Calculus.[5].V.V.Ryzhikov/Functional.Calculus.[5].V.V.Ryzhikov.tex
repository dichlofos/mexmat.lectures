\documentclass[a4paper]{article}
\usepackage{dmvn}

\newcommand{\n}[1]{^{(#1)}}

\DeclareMathOperator{\Ess}{Ess}

\newcommand{\seq}[3]{\hc{#1}_{{#2}}^{{#3}}}
\newcommand{\comment}[1]{\par\vskip2pt\hrule\vskip2pt{\footnotesize \textbf{Примечание:} #1\par}\vskip2pt\hrule\vskip2pt}

\makeindex

\begin{document}
\dmvntitle{Курс лекций по}{функциональному анализу}{Лектор\т Валерий Валентинович Рыжиков}
{III курс, 5 семестр, поток математиков}{Москва, 2004 г.} \pagebreak

\pagestyle{plain}
\tableofcontents
\pagebreak

\section*{Введение}

\subsection*{Предисловие}

Убедительная просьба ко всем читателям: в случае обнаружения ошибок
немедленно сообщайте автору на \dmvnmail{} или загляните на \dmvnwebsite{} и посмотрите, где
можно достать в настоящее время самого автора. Все пожелания и предложения по поводу оформления
и содержания документа будут обязательно приняты к сведению.

Комментарии к тексту набраны мелким шрифтом с пометкой вида
\comment{Замечания по поводу...}
Как правило, такие примечания указывают на дырки и непонятности
в рассуждениях лектора. Следовательно, к тексту в окрестности таких
пометок нужно относиться очень осторожно и внимательно его читать.

В этой редакции не хватает одного вопроса: теорем Фредгольма. Читайте по КФ.

Последнее обновление: \сегодня~года.

\subsection*{Слова благодарности}

За некоторые замеченные опечатки благодарность выносится Виктору Осокину, Анастасии Абрашитовой и Дарье Ярцевой.

\subsection*{Принятые в тексте соглашения и используемые сокращения}

\begin{points}{-3}
\item Следуя \cite{rokhlin}, топологические понятия обозначаются сокращениями соответствующих английских слов.
      Так, $\Int A$\т множество внутренних точек множества $A$, $\Cl A$\т замыкание множества $A$.
\item Следуя \cite{bogachev}, пространства интегрируемых в $p$\д й степени функций обозначаются $L^p$.
\item Индикатор множества $A$ обозначается через $\Ibb_A$.

\end{points}

\begin{thebibliography}{4}
\setlength\itemsep{-.5mm}
\bibitem{bogachev}
В.\,И.\,Богачёв. \emph{Основы теории меры}.\т Москва\т Ижевск: RCD, 2003.
\bibitem{kf}
    А.\,Н.\,Колмогоров, С.\,В.\,Фомин. \emph{Элементы теории функций и функционального анализа}.\т
    М.: Наука, 1981.
\bibitem{rokhlin}
В.\,А.\,Рохлин, Д.\,Б.\,Фукс. \emph{Начальный курс топологии}.\т М.: Наука, 1977.
\end{thebibliography}

\pagebreak
\pagestyle{headings}

\makeatletter
  \renewcommand{\headheight}{11mm}
  \renewcommand{\headsep}{2mm}
  \renewcommand{\sectionmark}[1]{}
  \renewcommand{\subsectionmark}[1]{}
  \renewcommand{\subsubsectionmark}[1]{\markright{\thesubsubsection. #1}}
  \renewcommand{\@oddhead}{\vbox{\hbox to \textwidth{\scriptsize\thepage\hfil\rightmark\strut}\hrule}}
  \renewcommand{\@oddfoot}{\hfil\thepage\hfil}
\makeatother

\section{Элементарное введение}

\subsection{Основные понятия}

\subsubsection{Нормированные и метрические пространства}

\index{Нормированное пространство}
\index{Пространство!нормированное}
\index{Норма}
\begin{df}
\emph{Нормированным пространством} называется линейное (векторное) пространство $X$ над полем $K$,
в котором задана функция $\hn{\cdot}\cln X \ra \R^+$, называемая нормой
и удовлетворяющая следующим свойствам:
\begin{points}{-3}
\item Для $\fa x, y \in X$ имеем $\hn{x+y} \le \hn{x} + \hn{y}$\т неравенство треугольника.
\item Для $\fa x \in X$, $\fa \al \in K$ имеем $\hn{\al x} = |\al|\cdot\hn{x}$\т однородность.
\item Для $\fa x \in X$ из $\hn{x} = 0$ следует $x = 0$.
\end{points}
\end{df}

Как следует из определения, поле $K$ должно быть снабжено своей нормой. Мы будем рассматривать
случаи $K = \R$ и $K = \Cbb$.

\index{Метрическое пространство}
\index{Пространство!метрическое}
\index{Метрика}
\begin{df}
\emph{Метрическим пространством} называется множество $M$, в котором задана функция $\rho\cln M \times M \ra \R^+$,
удовлетворяющая свойствам:
\begin{points}{-3}
\item Для $\fa x, y, z \in M$ имеем $\rho(x,z) \le \rho(x,y) + \rho(y,z)$\т неравенство треугольника.
\item Для $\fa x, y \in M$ имеем $\rho(x,y) = \rho(y,x)$\т симметричность.
\item Для $\fa x, y \in M$ из $\rho(x,y) = 0$ следует $x = y$.
\end{points}
\end{df}

Если $(X, \hn{\cdot})$\т нормированное пространство, то норма в нём задаёт некоторую метрику:
положим $\rho(x,y) := \hn{x-y}$. Выполнение всех свойств метрики очевидным образом следует
из аксиом нормы. Отметим, что такая метрика является трансляционно\д инвариантной, то
есть $\fa x, y, z \in X$ имеем $\rho(x + z, y + z) = \rho(x,y)$, или, проще говоря, расстояние
между точками сохраняется при параллельных переносах.

\index{Фундаментальная последовательность}
\index{Последовательность!фундаментальная}
\begin{df}
Последовательность $\hc{x_n} \subs X$ называется \emph{фундаментальной}, если $\fa \ep > 0 \exi N\cln
\fa n, m \ge N$ имеем $\hn{x_n - x_m} < \ep$.
\end{df}

\index{Сходимость последовательности}
\begin{df}
Говорят, что последовательность $\hc{x_n} \subs X$ \emph{сходится} к $x \in X$,
если $\hn{x_n - x} \ra 0$ при $n \ra \infty$.
\end{df}

\index{Полное пространство}
\index{Пространство!полное}
\begin{df}
Пространство называется \emph{полным}, если в нём любая фундаментальная
последовательность имеет предел.
\end{df}

\index{Банахово пространство}
\index{Пространство!банахово}
\begin{df}
\emph{Банаховым пространством} называется полное нормированное пространство.
\end{df}

\begin{ex}
Пространство $L_p(X, \mu) = \hc{f\cln \int|f|^p d\mu < \infty}$, $p \in [1, +\infty)$
является банаховым пространством с нормой
\eqn{\hn{f} := \bbr{\ints{X}|f|^pd\mu}^{1/p}.}
Здесь мы, конечно, должны сделать небольшую поправку. Чтобы все свойства нормы были выполнены,
нужно рассматривать не само множество функций с интегрируемой $p$\д ой степенью, а
факторпространство $\fact{X}{\sim}$, в котором $f \sim g$ тогда и только тогда, когда $f \eqae g$.
В противном случае не будет выполнено свойство \pt{3} нормы.
\end{ex}

\begin{ex}
Пространство $\ell_p = L_p(\N, \#)$, где $\#$\т мера, задаваемая как мощность множества.
Иначе говоря, это пространство последовательностей $x = \hc{x_n}$ с нормой
\eqn{\hn{x} := \hr{\suml{n=1}{\infty}|x_n|^p}^{1/p}.}
\end{ex}

\begin{theorem}
Пространство $\ell_1$ является полным пространством.
\end{theorem}
\begin{proof}
Мы будем писать индексы элементов пространства сверху.
Фиксируем фундаментальную последовательность $\hc{x^n} \subs \ell_1$. Фиксируем $i \in \N$
и рассмотрим последовательность $\seq{x_i^n}{n=1}{\infty}$. Она, очевидно, является фундаментальной,
поэтому в силу критерия Коши для числовых последовательностей она сходится к некоторому
числу $x_i$. Положим $x = (x_1,x_2\etc)$. Запишем теперь условие Коши для исходной
последовательности: $\fa \ep > 0 \exi N\cln \fa n,m \ge N$ имеем $\hn{x^n - x^m} < \ep$.
Перейдём в этом неравенстве к пределу при $n \ra \infty$ и фиксированном $m$, получим
$\hn{x - x^m} \le \ep$, но это и означает, что $x^m \ra x$.
\end{proof}
\comment{Мы не обосновали, почему, собственно говоря, $x \in \ell_1$, но это несложно доказывается.
Достаточно выбрать достаточно близкие к $x_i$ элементы исходных последовательностей.}

\index{Равномерная норма}
\index{Норма!равномерная}
\index{Чебышёвская норма}
\index{Норма!чебышёвская}
\begin{df}
Норма в пространстве $\Cb[a,b]$ непрерывных на отрезке $[a,b]$ функций
обычно задаётся так:
\eqn{\hn{f} := \maxl{x \in [a,b]} |f(x)|.}
Эта норма называется \emph{равномерной}, или \emph{чебышёвской}.
\end{df}

\begin{theorem}
Пространство $\Cb[a,b]$ полно по равномерной норме.
\end{theorem}
\begin{proof}
Пусть $\hc{f_n}$\т фундаментальная последовательность функций.
Тогда $\fa \ep > 0$ $\exi N\cln \fa n, m > N$ имеем $\hn{f_n - f_m} < \ep$.
Отсюда сразу получаем, что $\fa x \in [a,b]$ имеем $|f_n(x) - f_m(x)| < \ep$.
Следовательно, при каждом фиксированном $x \in [a,b]$ существует
предел $\liml{n} f_n(x) =: f(x)$. Покажем, что $f_n \rra f$. Для этого достаточно перейти
к пределу в неравенстве при $n \ra \infty$, тогда получим
$\fa \ep > 0$ $\exi N\cln \fa m > N$ имеем $\hn{f - f_m} \le \ep$.
Это в точности означает равномерную сходимость. Но равномерный предел непрерывных функций непрерывен,
поэтому $f \in \Cb[a,b]$.
\end{proof}

\subsubsection{Полные метрические пространства}

\index{Открытый шар}
\index{Шар!открытый}
\index{Замкнутый шар}
\index{Шар!замкнутый}
\begin{df}
Пусть $(M, \rho)$\т метрическое пространство. \emph{Шаром} (открытым) называется множество
\eqn{B(x,r) := \hc{y \in M\cln \rho(x,y) < r}.}
Если неравенство нестрогое, то шар называется замкнутым.
\end{df}

\comment{Определения топологии на лекции не было. Не было сказано и то, что открытый шар
является открытым множеством в топологическом смысле, и аналогично про замкнутый.
Доказательства этих фактов несложные, но всё это надо помнить.}

\index{Теорема!Бэра о вложенных шарах}
\begin{theorem}[Бэра о вложенных шарах]
Пусть $(M, \rho)$\т полное метрическое пространство.
Пусть $\bc{\wt B_i(x_i, r_i)}$\т последовательность вложенных замкнутых шаров, причём $r_i \ra 0$.
Тогда $\exu x \in \bigcap \wt B_i$.
\end{theorem}
\begin{proof}
Поскольку $r_i \ra 0$, а $\wt B_i \sups \wt B_{i+1}$, последовательность $\hc{x_i}$ будет
фундаментальной и потому сходится
к некоторому $x \in M$ в силу полноты пространства. Покажем, что $x$ является искомой точкой.
Действительно, если бы нашёлся шар $\wt B_{i_0}$ такой, что $x \notin \wt B_{i_0}$, тогда бы
точка $x$ не лежала бы ни в одном из шаров, начиная с номера $i_0$. Но поскольку дополнение к $\wt B_{i_0}$
открыто, $x$ можно отделить окрестностью от всех шаров, начиная с номера $i_0$. Это противоречит тому,
что $x$\т предел последовательности центров шаров.

Покажем единственность такой точки. В самом деле, если бы их было две, то расстояние между ними было бы
ненулевое. Но это противоречит условию $r_i \ra 0$.
\end{proof}

Пусть $(M, \rho)$\т метрическое пространство.

\index{Множество!нигде не плотное}
\index{Нигде не плотное множество}
\begin{df}
Множество $Y \subs M$ называется \emph{нигде не плотным в $M$}, если всякий шар $B \subs M$ ненулевого радиуса
содержит другой шар $B'$ ненулевого радиуса такой, что $Y \cap B' = \es$.
\end{df}

\index{Множество!всюду плотное}
\index{Всюду плотное множество}
\begin{df}
Множество $Y \subs M$ называется \emph{всюду плотным в $M$}, если $\Cl Y = M$.
\end{df}

\index{Множество!первой категории}
\begin{df}
Множество $Y$ называется \emph{множеством первой категории}, если оно может быть представлено как счётное
объединение нигде не плотных множеств.
\end{df}

\index{Теорема!Бэра о категориях}
\begin{theorem}[Бэра о категориях]
Полное метрическое пространство не может быть множеством первой категории.
\end{theorem}
\begin{proof}
Пусть $M$\т наше метрическое пространство. Допустим, что $M = \bigcup Y_i$, причём $Y_i$ нигде не плотны.
Рассмотрим множество $Y_1$, тогда найдётся замкнутый шар $\wt B_1$, для которого $\wt B_1 \cap Y_1 = \es$.
Рассмотрим множество $Y_2$ и возьмём $\wt B_2 \subs \wt B_1$ так, чтобы $\wt B_2 \cap Y_2 = \es$.
Продолжим этот процесс, получим последовательность замкнутых шаров $\hc{\wt B_i}$. По теореме о вложенных шарах
найдётся $x \in \bigcap \wt B_i$, но это означает, что $x$ не лежит ни в одном из~$Y_i$.
\end{proof}


\begin{theorem}
Множество непрерывных функций на отрезке $[0,1]$, имеющих конечную производную хоть в одной точке,
образуют множество первой категории в $\Cb[0,1]$.
\end{theorem}
\begin{proof}
Назовём функцию хорошей, если она дифференцируема хотя бы в одной точке.
Мы будем рассматривать полуинтервал $[0,1)$, а не отрезок $[0,1]$, то есть исключим
пока из рассмотрения точку $x=1$. Ясно, что множество хороших функций, дифференцируемых в
единице, ещё более дырявое, чем множество хороших функций,
дифференцируемых где\д то на $[0,1)$, поэтому его можно не рассматривать.
Рассмотрим функцию $f$, и пусть существует $f'(x)$ для некоторого $x \in [0,1)$. Это означает, что
$\exi n \in \N\cln |f(x+t)-f(x)| \le nt$ для всякого $t \in \hs{0,\frac1n}$.
В самом деле, из определения производной во всяком случае следует, что найдётся $\de$, такое, что при $|t|<\de$
будет выполняться неравенство
\eqn{\hm{\frac{f(x+t)-f(x)}{t}} \le 2|f'(x)| + 1.}
Теперь рассмотрим множества
\eqn{Y_n := \hc{f \in \Cb[0,1] \bvl \exi x \in \hs{0,1-\frac1n} \cln \fa t \in \hs{0,\frac1n} \text{ имеем } |f(x+t)-f(x)| \le nt}.}
Итак, все функции, которые дифференцируемы в точках из $[0,1)$, попадут в множество $\bigcup Y_n$.

\begin{lemma}
Множество $Y_n$ замкнуто в $\Cb[0,1]$.
\end{lemma}
\begin{proof}
В самом деле, покажем, что если $f_k \rra f$, и $\hc{f_k} \subs Y_n$, то $f \in Y_n$.
Для каждой функции $f_k$ имеем $\exi x_k \in \hs{0,1-\frac1n} \cln \fa t \in \hs{0,\frac1n}$
имеем $|f(x_k +t) -f(x_k)| \le nt$. Покажем, что $\exi x \in [0,1)$, для которого
$\fa t \in \hs{0,\frac1n}$ имеем $\hm{f(x+t)-f(x)} \le nt$.
Рассмотрим в качестве этого $x$ произвольную предельную точку последовательности $\hc{x_k}$
и оставим от всей последовательности только ту подпоследовательность, которая сходится
к точке $x$.

Через $a \mp b$ будем обозначать выражение $a - b + b$ (вычли и тут же прибавили). Имеем
$$\hm{f(x+t)-f(x)} = \hm{f(x+t) \mp f(x_k+t) \mp f_k(x_k+t) \mp f_k(x_k) \mp f(x_k) -f(x)} \le \ep + \ep + nt + \ep + \ep.$$
Разберёмся, откуда вылезло столько $\ep$: первый\т из равномерной непрерывности $f$ на отрезке $[0,1]$,
второй\т из равномерной сходимости; третье слагаемое\т из определения $x_k$, предпоследнее\т из равномерной сходимости;
последнее\т из непрерывности $f$. Таким образом, лемма доказана.
\end{proof}

\begin{lemma}\label{lemma:cl.wo.balls.hence.nondense}
Пусть замкнутое множество не содержит ни одного шара положительного радиуса.
Тогда оно нигде не плотно.
\end{lemma}
\begin{proof}
Если $M$ не является нигде не плотным, то найдётся шар $B$ положительного радиуса
такой, что для всякого шара $B' \subs B$ имеем $M \cap B' \neq \es$. Это означает,
что $M$ всюду плотно в $B$, но тогда $B \subs M$,
ибо $M$ замкнуто (оно содержит все свои предельные точки).
\end{proof}

\begin{lemma}
Всякое множество $Y_n$ не содержит ни одного шара положительного радиуса.
\end{lemma}
\begin{proof}
Покажем, что для $\fa f \in Y_n$ найдётся $g \in \Cb[0,1]$, для которой
$\hn{f-g} < \ep$, но $g \notin Y_n$. Действительно, приблизим нашу непрерывную функцию
кусочно\д линейной, выберем максимум среди всех значений её производной, а затем возьмём
<<пилу>> с зубьями малой высоты, у которой наклон зубьев в 100 раз больше,
чем этот максимум, и в 10 раз больше, чем $n$, да прибавим к функции $f$.
Это и будет наша функция $g$, поскольку её производная там, где она есть, всегда будет
принимать значения, большие, чем $n$. Значит, она не лежит в $Y_n$.
\end{proof}

Применим последовательно все эти леммы и получим доказательство теоремы.
\end{proof}

\begin{imp}
В пространстве $\Cb[0,1]$ существует нигде не дифференцируемая функция.
\end{imp}
\begin{proof}
Это пространство полно, поэтому оно не является множеством первой категории.
А мы только что доказали, что совокупность хороших функций есть множество первой категории.
Значит, в $\Cb[0,1]$ есть ещё какие\д то функции\т они и будут нигде не дифференцируемыми.
\end{proof}


\index{Сжимающее отображение}
\index{Отображение!сжимающее}
\begin{df}
Пусть $(M, \rho)$\т метрическое пространство.
Отображение $\Ph\cln M\ra M$ называется \emph{сжимающим} с коэффициентом $\al \in [0,1)$, если $\fa x, y \in M$
имеем $\rho(\Ph x, \Ph y) \le \al \rho(x,y)$.
\end{df}

\index{Теорема!о сжимающих отображениях}
\begin{theorem}[О сжимающих отображениях]
Пусть $(M, \rho)$\т полное метрическое пространство. Пусть $\Ph$\т сжимающее отображение.
Тогда у него существует единственная неподвижная точка.
\end{theorem}
\begin{proof}
Единственность такой точки сразу следует из определения сжимающего отображения: если бы их было две,
тогда расстояние между их образами сохранилось бы, что невозможно. Докажем существование:
рассмотрим произвольную точку $y \in M$ и рассмотрим итерации нашего отображения:
\eqn{y,\; \Ph y,\;  \Ph(\Ph y) = \Ph^2 y,\;  \Ph^3 y,\;  \ldots}
Положим $y_k = \Ph^k y$. Последовательность $\hc{y_k}$, очевидно, фундаментальна. В самом деле,
\eqn{\rho(y_k, y_{k+1}) \le \al^k\rho(y, y_1),}
поэтому
\eqn{\rho(y_n, y_m) \le \rho(y_m, y_{m+1}) + \rho(y_{m+1}, y_{m+2})\spl \rho(y_{n-1}, y_n)=
  (\al^m + \al^{m+1} \spl \al^n)\rho(y,y_1),}
а последнее выражение можно сделать маленьким как остаток сходящегося ряда для геометрической прогрессии.
В силу полноты пространства, она сходится к некоторому пределу $x \in M$.
Покажем, что это и есть та самая неподвижная точка.
Отображение $\Ph$, очевидно, является непрерывным, поскольку близкие точки переходят в близкие.
По свойствам непрерывных отображений имеем $\Ph y_k \ra \Ph x$, если $y_k \ra x$. Поэтому,
если $\Ph^k y \ra x$, то и $\Ph\br{\Ph^k y} \ra \Ph(x)$. Но последовательности
$\hc{\Ph^k y}$ и $\hc{\Ph\br{\Ph^k y}}$ совпадают с точностью до первого члена, поэтому их пределы
одинаковы. Следовательно, $x = \Ph x$, что и требовалось доказать.
\end{proof}

\begin{ex}
Рассмотрим отображения
\eqn{\mat{\Ph_0\cln \R \ra \R,\\
\Ph_0\cln x \mapsto \frac{1}{3} x,} \qquad
\mat{\Ph_1\cln \R \ra \R,\\
\Ph_1\cln x \mapsto \frac{1}{3} (x-1) + 1.\\}}
Пусть $M$\т все компакты в $\R$. На $M$ можно задать естественную метрику $\wt\rho$,
по которой оно будет полным метрическим пространством.
Рассмотрим отображение $\Psi\cln K \mapsto \Phi_1 K \cup \Phi_2 K$. Оно будет иметь единственную неподвижную
точку, поскольку, очевидно, является сжимающим.
\end{ex}

\index{Теорема!о пополнении}
\begin{theorem}[О пополнении]
Каждое метрическое пространство изометрично вкладывается в полное. Более точно, если $(M,\rho)$\т метрическое
пространство, то существует полное метрическое пространство $(\ol M, \ol \rho)$ и инъективное отображение
$\ph\cln M \ra \ol M$ такое, что $\fa x,y \in M$ имеем $\rho(x,y) = \ol \rho\br{\ph(x), \ph(y)}$.
\end{theorem}
\begin{proof}
Мы приведём здесь только основную идею доказательства.
Пусть $N$\т множество всех фундаментальных последовательностей элементов из пространства $M$.
Введём на $N$ отношение эквивалентности: $x \sim y$ тогда и только тогда, когда
$\lim \rho(x_i,y_i) = 0$. Положим $\ol M = \fact{N}{\sim}$, а метрику зададим так:
$\ol\rho(x,y) := \lim \rho(x_i,y_i)$. Совершенно очевидно, что вложение $M \inj \ol M$ строится так:
отображаем элемент $x\in M$ в класс эквивалентности, в котором есть
стационарная последовательность $\hc{x_i} \in \ol M$, где $x_i = x$.
Остаётся доказать то, что $\ol \rho$ является метрикой (что, впрочем, совсем несложно проверяется),
и, что противнее, полноту этого пространства.
\end{proof}

\section{Нормированные пространства}

\subsection{Операторы в нормированных пространствах}

\subsubsection{Определение и свойства операторов}

\index{Оператор}
\index{Линейный функционал}
\index{Функционал!линейный}
\begin{df}
Пусть $X$ и $Y$\т нормированные пространства. \emph{Оператором} мы будем называть
произвольное линейное отображение $A\cln X \ra Y$, а \emph{линейным функционалом}\т
линейное отображение $\ph\cln X \ra K$ (частный случай оператора).
\end{df}

\index{Норма!оператора}
\index{Ограниченный оператор}
\index{Оператор!ограниченный}
\begin{df}
\emph{Нормой} оператора $A$ называется число $\hn{A} := \supl{\hn{x} = 1}\hn{Ax}$.
Оператор $A$ называется \emph{ограниченным}, если $\hn A < \infty$.
\end{df}

\index{Непрерывный оператор}
\index{Оператор!непрерывный}
\begin{df}
Оператор $A$ называется \emph{непрерывным}, если из $\hn{x_i - x} \ra 0$ следует,
что $\hn{Ax_i - Ax} \ra 0$. Иначе говоря, если $x_i \ra x$ в $X$, то $Ax_i \ra Ax$ в $Y$.
\end{df}

\index{Оператор!тождественный}
\index{Тождественный оператор}
Тождественный оператор мы будем обозначать либо буквой $E$, либо символом $\id$.

\begin{lemma}[Свойства нормы оператора]
Имеют место следующие свойства:
\begin{gather*}
  \hn{A} = \supl{0 < \hn x \le 1} \frac{\hn{A x}}{\hn x} =
  \supl{\hn x \neq 0}\frac{\hn{A x}}{\hn x}, \quad
  \hn{A x} \le \hn A \cdot \hn x,\\
  \hn{\la A} = \hm{\la}\cdot\hn{A},\quad
  \hn{A+B} \le \hn{A} + \hn{B}, \quad
  \hn{A B} \le \hn{A} \cdot \hn{B}.
\end{gather*}
\end{lemma}
\begin{proof}
Первые два равенства напрямую следуют из линейности норм пространств $X$ и $Y$,
а последние четыре свойства вытекают напрямую из определения нормы оператора.
\end{proof}

\begin{imp}
Множество ограниченных операторов образуют подалгебру в алгебре всех операторов.
Эта алгебра является нормированным пространством.
\end{imp}

\begin{theorem}
Непрерывность оператора равносильна его ограниченности.
\end{theorem}
\begin{proof}
Пусть оператор $A$ ограничен. В силу линейности имеем
$Ax_i - Ax = A(x_i -x)$. Отсюда получаем $\hn{Ax_i - Ax} \le \hn{A} \cdot \hn{x_i - x}$,
поэтому из ограниченности оператор следует его липшицевость (и тем более непрерывность).

Вместо того, чтобы выводить ограниченность из непрерывности, докажем, что из неограниченности
следует разрывность. Это согласуется с логическим правилом $(A \ra B) \Lra (\ol B \ra \ol A)$.
Пусть $\hn{A} = \infty$, тогда, используя определение нормы оператора, можно выбрать
последовательность $\hc{x_i}$, для которой $\hn{x_i} = 1$, но $\hn{Ax_i} \ra \infty$. Рассмотрим
другую последовательность $y_i := \frac{x_i}{\hn{Ax_i}}$. Тогда $\hn{y_i} \ra 0$, но $\hn{Ay_i} = 1
\nra 0$, поэтому $A$ не может быть непрерывным.
\end{proof}

\index{Оператор!неограниченный}
\index{Неограниченный оператор}
При рассмотрении операторов возникает естественный вопрос: а существуют ли неограниченные операторы?
Для неполных пространств примеры таких операторов строятся совсем легко. Возьмём,
например, в качестве $X$ пространство $\Cb[0,1]$ с нормой $\hn{\cdot}_{L_1}$,
а в качестве $Y$\т то же пространство с равномерной нормой $\hn{\cdot}_\Cb$.
Рассмотрим тождественный оператор на функциях $f_n$ вида
$$\epsfbox{pictures.10}$$
Понятно, что равномерная норма каждой функции $f_n$ равна $1$, а нормы этих функций в смысле $L_1$ стремятся к
нулю с ростом~$n$. Поэтому оператор не будет непрерывным.

\subsubsection{Базис Гамеля}

\index{Базис Гамеля}
\begin{df}
\emph{Базис Гамеля}\т такая система векторов $\Bc \subs X$,
что всякий вектор $x \in X$ единственным образом представляется в виде конечной линейной комбинации
векторов из $\Bc$.
\end{df}

\begin{theorem}
Всякое линейное пространство~$X$ обладает базисом Гамеля.
\end{theorem}
\comment{При доказательстве мы будем использовать лемму Цорна. О том, что это такое,
см. ниже в том разделе, где доказывается теорема Хана\ч Банаха.
Доказательство придумал я сам, поэтому возможна лажа.}
\begin{proof}
В основе доказательства будет лежать следующая идея:
возьмём какой\д нибудь ненулевой вектор $x_1 \in X$ и рассмотрим $X_1 := \ha{x_1}$.
Если $X \neq X_1$, то найдётся ещё вектор $x_2 \notin X_1$. Положим $X_2 := \ha{X_1,x_2}$,
тогда $X_1 \subs X_2$. Если и на этом шаге нам не повезло, и вновь $X_2 \neq X$,
то найдём третий вектор, и так далее. Так мы описали процесс расширения конечномерного подпространства.

Формализуем это построение с использованием леммы Цорна.
Система векторов называется линейно независимой, если любая конечная подсистема в ней линейно независима.
Пусть $(L, \hc{e_i})$\т подпространство с базисом $\hc{e_i}$, а $(M, \hc{f_j})$\т подпространство
с базисом $\hc{f_j}$. Будем говорить, что $L \prec M$, если $\hc{e_i} \subs \hc{f_j}$.
Рассмотрим множество всех расширений, получается, что мы ввели на нём
частичное упорядочение. Пусть $(M_\al, \hc{e_i}_\al)$\т цепь расширений.
Очевидно, что расширение
$$\hr{\bigcup M_\al, \bigcup \hc{e_i}_\al}$$ является верхней гранью этой цепи.
По лемме Цорна в нашем множестве найдётся максимальный элемент $(S,\hc{e_i})$.
Ясно, что $S = X$, поскольку если бы это было не так, то можно было бы расширить~$S$ на
вектор из $X$, которого ещё нет в линейной оболочке базиса $S$.
Но тогда $\hc{e_i}$\т базис $X$, что и требовалось доказать.
\end{proof}

Если верить в лемму Цорна, то с помощью этой теоремы  можно легко построить пример неограниченного оператора
в любом бесконечномерном пространстве. Пусть $\Ga$\т базис Гамеля пространства $X$.
Выберем счётное подмножество среди базисных элементов и занумеруем их, получим
набор $\hc{\ga_i}$. Зададим действие оператора на базисных векторах: положим $A\ga_i = i\ga_i$,
а для всех остальных базисных векторов $e \in \Ga \wo \hc{\ga_i}$ положим $Ae = 0$. Тем самым
мы задали действие оператора на всех векторах пространства $X$, ибо всякий вектор
единственным образом разлагается по нашему базису. Поэтому, если $x = \sumiun a_i e_i$,
то $Ax = \sumiun a_i Ae_i$ и тем самым оно однозначно определено. Вместе с этим ясно, что оператор $A$
неограничен, поскольку для всякого $M > 0$ найдётся вектор, который растягивается этим оператором
больше, чем в $M$ раз.


\subsection{Важнейшие теоремы функционального анализа}

\subsubsection{Принцип равномерной ограниченности Банаха\ч Штейнгауза}

\index{Теорема!Принцип равномерной ограниченности Банаха\ч Штейнгауза}
\begin{theorem}[Принцип равномерной ограниченности Банаха\ч Штейнгауза]
Пусть $X$\т банахово, а $Y$\т нормированное пространство.
Пусть $A_i\cln X \ra Y$\т семейство ограниченных операторов.
Пусть для всякого $x \in X$ существует число $C_x > 0$ такое, что для $\fa i$ имеем
$\hn{A_i x} \le C_x$. Тогда найдётся такое $C > 0$, что $\hn{A_i} \le C$ для всех $i$.
\end{theorem}
\begin{proof}
Рассмотрим семейство множеств
$$X_n := \hc{x \in X \cln \fa i \text{ имеем} \hn{A_i x} \le n}.$$
Очевидно, что $X = \bigcup X_n$. Поскольку $X$ не есть множество первой категории,
найдётся $X_N$ такое, что оно не является нигде не плотным в $X$. Значит, есть шар,
где оно всюду плотно.

Покажем, что все множества $X_n$ замкнуты. Для этого докажем, что дополнения к ним открыты.
Пусть $x \notin X_n$. Значит, $\exi k$, для которого $\hn{A_k x} \ge n+ 2\ep$.
Пусть $v \in X$. Если $\hn{v}\le \frac{\hn{A_kx} - (n+\ep)}{\hn{A_k}}$, то
\eqn{\hn{A_k(x+v)} = \hn{A_k x + A_k v} \ge \hn{A_kx} - \frac{\hn{A_k}\br{\hn{A_k x} - (n+\ep)}}{\hn{A_k}} = n + \ep > n,}
то есть $(x+v)\notin X_n$.

По лемме~\ref{lemma:cl.wo.balls.hence.nondense}, множество $X_N$ содержит некоторый шар $B$.
Достаточно установить равномерную ограниченность операторов на некотором шаре,
содержащем начало координат. Пусть $\wt B$\т копия шара $B$ с центром в начале координат.
Каждый вектор $v \in \wt B$ можно представить как $w_1 - w_2$, где $w_i \in B$.
По неравенству треугольника и определению множества $X_N$
для всех~$i$ получаем $\hn{A_i v} = \hn{A_i w_1 - A_i w_2} \le N + N = 2N$.
Но это и означает равномерную ограниченность.
\end{proof}

\begin{note}
В этой теореме множество операторов может иметь какую угодно мощность.
Иначе говоря, все индексы операторов\т это не натуральные числа, а элементы произвольного
индексного множества.
\end{note}

\begin{note}
Покажем, что нельзя опустить требование банаховости пространств.
Пусть $X$\т пространство финитных последовательностей, а $Y = \ell_p$.
Определим семейство операторов так:
\eqn{A_i(x_1,x_2\sco x_{i-1},x_i,x_{i+1}\etc) := (0,0\sco 0,ix_i,0\etc).}
Ясно, что для каждой финитной последовательности $x$ найдётся нужная константа $C_x$, ибо
$A_i x = 0$ при достаточно больших $i$ (если последовательность финитна,
то она имеет вид $(x_1\sco x_N, 0,0\etc)$, поэтому достаточно будет взять $i > N$),
но, с другой стороны, $\hn{A_i} = i \ra \infty$ при $i \ra \infty$.
\end{note}


\subsubsection{Теорема Банаха об обратном операторе}

Заметим, что если оператор $A\cln X \ra Y$ обладает тем свойством, что $\Img A = Y$,
а $\Ker A = 0$, то $A$ биективен и потому существует обратное отображение $A^{-1}\cln Y \ra X$.

\begin{lemma}
Пусть $A\cln X \ra Y$\т линейная биекция банаховых пространств.
Положим
\eqn{Y_k := \hc{y \in Y\cln \hn{A^{-1}y} \le k\hn{y}}.}
Тогда существует такое $Y_N$, что $\Cl Y_N = Y$.
\end{lemma}
\begin{proof}
Поскольку $Y$\т полное пространство, по теореме Бэра существует~$Y_M$, плотное в некотором шаре~$B$.
Обозначим через $P$ пересечение некоторого шарового слоя с центром в точке $y_0 \in Y_M$, целиком лежащего
в шаре $B$, с множеством $Y_M$. Рассмотрим копию $\wt P$ множества~$P$, сдвинутую в начало координат.
Всякий вектор $v \in \wt P$ представляется в виде разности $y - y_0$, где $y \in P$.
Имеем
\begin{multline*}
\hn{A^{-1}v} = \hn{A^{-1}(y-y_0)} \le \hn{A^{-1}y} + \hn{A^{-1}y_0} \le M\br{\hn y + \hn{y_0}} =\\=
M\br{\hn{y - y_0 + y_0} + \hn{y_0} } \le M\br{\hn{y -y_0} + 2\hn{y_0}} = M\hn{y-y_0}\hr{1 + \frac{2\hn{y_0}}{\hn{y-y_0}}}.
\end{multline*}
Заметим, что последний множитель может быть ограничен сверху некоторой константой $C$,
не зависящей ни от чего, поскольку число $\hn{y-y_0}$ отделено от нуля. Беря
в качестве $N := \hs{CM} + 1$, получаем, что $Y_N$ плотно в~$\wt P$. Но поскольку
в силу своего определения множество $Y_N$ инвариантно относительно
гомотетий, оно будет плотно и во всём пространстве.
\end{proof}

\index{Теорема!Бахаха об обратном операторе}
\begin{theorem}[Банаха об обратном операторе]
Пусть $A\cln X \ra Y$\т линейная биекция банаховых пространств.
Тогда обратное отображение $A^{-1}\cln Y \ra X$ тоже будет ограниченным оператором.
\end{theorem}
\begin{proof}
Линейность обратного отображения очевидна. Докажем ограниченность.
Рассмотрим ненулевой вектор $y \in Y$. По предыдущей лемме существует всюду плотное в $Y$ множество $Y_N$.
Тогда существует $y_1 \in Y_N$, для которого $\hn{y- y_1} \le \frac{\hn{y}}{2}$, причём $\hn{y_1} \le \hn{y}$.
Далее, существует $y_2 \in Y_N$, для которого $\hn{y-(y_1+y_2)}\le\frac{\hn y}{2^2}$, причём
$\hn{y_2} \le\frac{\hn y}{2}$, и так далее.
На $n$\д м шаге существует $y_n \in Y_N$, для которого
$\hn{y-(y_1+y_2\spl y_n)}\le\frac{\hn y}{2^n}$, причём
$\hn{y_n} \le\frac{\hn y}{2^{n-1}}$.

Рассмотрим $x_n := A^{-1}y_n$. По определению $Y_N$ имеем
$\hn{x_n} \le N\hn{y_n} \le N\frac{\hn{y}}{2^{n-1}}$.
Значит, в силу полноты пространства $X$ и сходимости ряда $\sum \hn{x_n}$ существует предел
\eqn{x := \liml{p\ra\infty}\suml{n=1}{p} x_n.}
Тогда
\eqn{Ax = A\bbr{\liml{p\ra\infty}\suml{n=1}{p}x_n} = \liml{p\ra\infty} \suml{n=1}{p}Ax_n=
\liml{p\ra\infty}\suml{n=1}{p}y_n=y.}
Отсюда $A^{-1}y=x$, поэтому
\begin{multline*}
  \hn{A^{-1}y}=\hn{x} = \Bn{\sumnui x_n} = \liml{p\ra\infty}\Bn{\suml{n=1}{p}x_n} =
  \liml{p\ra\infty}\Bn{\suml{n=1}{p}A^{-1}y_n} \le \\ \le\sumnui\hn{A^{-1}y_n} \le
  \sumnui N\hn{y_n} \le \sumnui N \frac{\hn y}{2^{n-1}} = 2N\hn{y}.
\end{multline*}
Следовательно, оператор $A^{-1}$ ограничен.
\end{proof}

Покажем теперь, что для ограниченности полнота $X$ и $Y$ существенна. В качестве примера
возьмем $X = \hr{\ell_1, \hn{\cdot}_{\ell_1}}$,
а $Y = \hr{\ell_1, \hn{\cdot}_{\ell_2}}$. Оператор возьмём, как водится, тождественным. Очевидно, что
\eqn{\sums{i} \hm{x_i}^2 \le \hr{\sums{i}\hm{x_i}}^2,}
поэтому
\eqn{\sqrt{\sums{i} \hm{x_i}^2} \le \sums{i}\hm{x_i}.}
Отсюда $\hn{x}_2 \le \hn{x}_1$. Рассмотрим последовательность
\eqn{x^{(N)} = \Br{\ub{\frac1N\sco \frac1N}_N,0\etc}.}
Очевидно, $\hn{x^{(N)}}_1 = 1$, а $\hn{x^{(N)}}_2 = \frac{1}{\sqrt N}$, поэтому
оператор $\id\cln Y \ra X$ не является ограниченным.

Как мы знаем, из теоремы о базисе Гамеля следует, что на всяком бесконечномерном пространстве
существует неограниченный линейный оператор. Аналогично строится и неограниченный линейный функционал.
Рассмотрим счётное подмножество $\hc{\ga_i}$ базиса Гамеля. Положим $\ph(\ga_i) = i\hn{\ga_i}$, а на
всех остальных векторах базиса положим его
равным нулю. Тогда возьмём какое\д нибудь полное пространство $X$ и оснастим его другой
нормой $\hn{\cdot} := \hn{\cdot}_X + \hn{\cdot}_\ph$, то есть
положим $\hn x := \hn{x}_X + \hm{\ph(x)}$. Обозначим новое пространство через $X_\ph$.
Рассмотрим оператор $\id\cln X_\ph \ra X$. Его норма, очевидно, не превосходит~$1$,
но обратный оператор не будет ограниченным, поскольку он увеличивает норму $i$\д го вектора
в $(1+i)$ раз.

\index{Теорема!Устойчивость обратимости оператора при малых возмущениях}
\begin{theorem}[Устойчивость обратимости оператора при малых возмущениях]
Пусть $A\cln X \ra X$\т ограниченный обратимый оператор в банаховом пространстве. Тогда
существует $C> 0$ такое, что для всякого оператора $B$ с нормой $\hn{B} \le C$ оператор $A + B$ будет обратим.
\end{theorem}
\begin{proof}
Для начала докажем вспомогательное утверждение, полезное само по себе.
\begin{lemma}
Если $A\cln X \ra X$\т оператор в банаховом пространстве такой, что $\hn{A} < 1$, то оператор $E - A$ обратим.
\end{lemma}
\begin{proof}
Покажем, что оператор
\eqn{(E-A)^{-1} := \sumizi A^i}
является искомым. Для начала объясним, как следует понимать операторный ряд и почему он сходится.
Определим значение оператора
\eqn{P = \sumizi A^i}
на векторе $x$ следующим образом:
\eqn{Px := \liml{n\ra \infty} S_n(x), \text{ где } S_n x := \sumizn A^ix.}
Покажем, что последовательность частичных сумм этого ряда является фундаментальной последовательностью.
В самом деле,
\eqn{\hn{S_{n+l}x - S_n x} = \hn{A^{n+1}x\spl A^{n+l}x} \le \hn{A^{n+1}x}\spl \hn{A^{n+l}x}\le
  \hn{x} \hr{\hn{A}^{n+1}\spl \hn{A}^{n+l}},}
поэтому, если взять $n$ достаточно большим, эту сумму можно сделать сколь угодно маленькой.
В силу полноты пространства, эта последовательность сходится. Очевидно, что
\eqn{\hn{Px} \le \hn{x} \sumizi \hn{A}^i,}
поэтому оператор $P$ ограничен.
Из определения $P$ выводим, что
\eqn{P(E-A)x = \liml{n \ra \infty} \sumizn A^i(E-A)x = \liml{n\ra\infty}\sumizn\hr{A^ix - A^{i+1}x}
= \liml{n\ra\infty} \hr{x-A^{n+1}x} = x.}

Далее, поскольку оператор $E-A$ ограничен, знак предела можно перенести, и значит, ${(E-A)Px = x}$.
Таким образом, оператор $E-A$ обратим.
\end{proof}

Покажем теперь, что существенно требование обратимости.
Возьмём оператор правого сдвига в пространстве~$\ell_p$. Он действует так:
\eqn{Rx = R(x_1,x_2\etc) = (0,x_1,x_2\etc).}
Понятно, что $\hn{R} = 1$. Очевидно, что левый обратный оператор существует\т это левый сдвиг $L$,
причём он тоже ограничен. Но правого обратного не существует, поскольку $\Img R \neq \ell_p$, значит,
не для каждого вектора будет выполнено равенство $R L x=x$. Таким образом, из существования левого обратного
не вытекает существование правого обратного. Меняя местами операторы $R$ и $L$, получаем пример, когда
есть правый обратный, но нет левого.

Ясно, что $A+B$ обратим тогда и только тогда, когда оператор $A^{-1}(A+B) = E + A^{-1}B$ обратим.
Поскольку $A$ ограничен, по теореме Банаха обо обратном операторе $A^{-1}$ тоже ограничен.
Поскольку $\hn{A^{-1}B} \le \hn{A^{-1}} \cdot \hn{B}$, в силу леммы достаточно взять
$C  = \frac{1}{\hn{A^{-1}}}$.
\end{proof}

\begin{imp}
Пусть $A$\т обратимый ограниченный оператор в банаховом пространстве.
Тогда $\exi C>0$, для которого $\frac{\hn x}{\hn{Ax}} < C$, то есть ограниченный обратимый оператор
не может как угодно сильно сжимать вектора.
\end{imp}

\index{Теорема!Достаточное условие необратимости оператора}
\begin{imp}[Достаточное условие необратимости оператора]
\label{imp:unrevertibility}
Если найдётся последовательность векторов $\seq{x_i}{i=1}{\infty}$, для которой имеем $\hn{x_i} = 1$,
а $\hn{Ax_i} \ra 0$, то оператор $A$ не может быть обратим.
\end{imp}


\subsection{Спектр оператора и его свойства}

\subsubsection{Определение спектра и его простейшие свойства}

Говоря о спектре операторов, мы всегда будем рассматривать только комплексные пространства.

\index{Спектр оператора}
\index{Оператор!спектр}
\begin{df}
Пусть $A$\т оператор. \emph{Спектром оператора} называется множество
\eqn{\Spec A := \hc{\la \in \Cbb\cln A - \la E \text{ необратим}}.}
\end{df}

Основные свойства спектра ограниченного оператора можно сформулировать так:
\emph{спектр является непустым компактным подмножеством в $\Cbb$} (всё это мы аккуратно докажем).

Строго говоря, это утверждение не всегда верно. Пусть наше пространство состоит
из одной точки $\hc{0}$. В таком пространстве всякий оператор является нулевым
и обратимым, поэтому имеет пустой спектр. Но это слишком вырожденный случай,
чтобы его было интересно рассматривать.

\begin{theorem}
Пусть $A$\т ограниченный оператор в банаховом пространстве.
Тогда $\Spec A$ является компактным множеством, и справедливо включение $\Spec A \subs \hc{\la \cln \hm{\la} \le \hn{A}}$.
\end{theorem}
\begin{proof}
Рассмотрим оператор $A - \la E$. Если он обратим при $\la \neq 0$, то доказывать нечего\т спектр состоит
не более чем из одной точки, и включение, очевидно, выполнено.
Пусть теперь $\la \neq 0$. Очевидно, что оператор $A-\la E$ обратим тогда и только тогда,
когда обратим $E - \frac{A}{\la}$. А это, в свою очередь, имеет место, как только $\hn{\frac{A}{\la}}<1$.
Следовательно, при $\hm{\la} > \hn{A}$ имеем $\la \notin \Spec A$.

Осталось показать, что спектр замкнут. Это следует в точности из теоремы об устойчивости обратимости оператора
при малых возмущениях: если оператор $A-\la_0 E$ обратим, то при добавлении к нему любого оператора
$\la E$ достаточно малой нормы это свойство сохранится. Но это означает, что множество
$\Cbb \wo \Spec A$ открыто, поэтому $\Spec A$ замкнуто, что и требовалось доказать.
\end{proof}

Теперь выясним, откуда берутся необратимые операторы.
Посмотрим, когда оператор $A - \la E$ может быть необратим.
Во\д первых, может случиться так, что $\Ker (A - \la E) \neq 0$, и тогда
причина необратимости в том, наш оператор не инъективен.
Кроме того, $\Img (A - \la E)$ может не совпадать с исходным пространством,
тогда обратное отображение $A^{-1}\cln X\ra X$ не может быть корректно задано.

\subsubsection{Пример нахождения спектра}

\begin{ex}
Вычислим спектр оператора правого сдвига в $\ell_1$.
Рассмотрим
\eqn{R(x_1,x_2,x_3\etc) = (0,x_1,x_2,x_3\etc).}
Поищем собственные векторы: с собственным значением $\la = 0$ их, очевидно, нет, кроме нулевого.
Пусть $\la \neq 0$, тогда имеем цепочку равенств
\eqn{\case{0 =\la x_1,\\x_1 =\la x_2,\\x_2 =\la x_3,\\ \ldots}}
Поскольку $\la \neq 0$, решая первое уравнение, сразу получаем $x_1 = 0$.
Отсюда $x_2 = 0$, и так далее. Значит, собственных векторов у этого оператора нет.
Очевидно, $\hn{R} = 1$, поскольку $\hn{Rx} = \hn{x}$. Следовательно,
весь спектр содержится в круге радиуса $1$. Покажем, что при $\hm{\la} \le 1$
оператор $R - \la E$ необратим, а именно, $e_1 \notin \Img (R - \la E)$.
В самом деле, решим уравнение
$$(R - \la E)y = e_1 = (1,0,0\etc).$$
Имеем
$$
\case{
1 =0-\la y_1,\\
0 =y_1-\la y_2,\\
0 =y_2-\la y_3,\\
  \ldots
} \text{ откуда }
\case{
y_1 = -\frac{1}{\la},\\
y_2 = -\frac{1}{\la^2},\\
y_3 = -\frac{1}{\la^3},\\
  \ldots
}
$$
В силу ограничения на $\la$, решение этого уравнения не лежит в пространстве $\ell_1$.
Значит, оператор необратим, поэтому спектр состоит из круга $\hc{\hm{\la} \le 1}$.
\end{ex}

\begin{ex}
Пусть $L$\т левый сдвиг:
$$
  L(x_1,x_2,x_3\etc) = (x_2,x_3\etc).
$$
Очевидно, что $\hn{L} = 1$, поскольку вектора с первыми нулевыми координатами сохраняют норму.
Поэтому $\Spec L \subs \hc{\hm{\la} \le 1}$.
Из соотношения
$$(x_2,x_3,x_4\etc) = (\la x_1, \la x_2, \la x_3\etc)$$
видно, что собственные векторы этого оператора\т в точности геометрические прогрессии. Более точно,
при $\hm{\la} < 1$ решением является геометрическая прогрессия
$(x_1,\la x_1, \la^2 x_1\etc)$. Следовательно, круг $\hc{\hm{\la} < 1}$ содержится в спектре.
Но так как спектр замкнут, он совпадает с замкнутым кругом $\hc{\hm{\la} \le 1}$.
\end{ex}

\subsubsection{Ещё две теоремы о спектрах}

\begin{stm}
Для любого непустого компакта $K \subs \Cbb$ найдётся оператор $A\cln \ell_p \ra \ell_p$ такой,
что $\Spec A = K$.
\end{stm}
\begin{proof}
Пусть $\seq{a_i}{i=1}{\bes}$\т последовательность, плотная в $K$.
\comment{Ясно, что эту последовательность надо строить с помощью метода
половинного деления. Как конкретно\т додумайте сами. Если кому\д то покажется,
что это сложно, напишите на \dmvnmail{}, и в следующей версии я напишу подробнее.}
Положим
$$A(x_1,x_2,x_3\etc) = (a_1x_1,\,a_2x_2,\,a_3x_3\etc).$$
В силу ограниченности последовательности $\hc{a_i}$ этот оператор ограничен.
Очевидно, что $\hc{a_i} \subs \Spec A$, ибо $A e_i = a_i e_i$, но в силу замкнутости спектра
имеем $\Cl \hc{a_i} = K \subs \Spec A$. Пусть $\mu \notin K$, тогда найдётся окрестность
$U_\de(\mu)$, не пересекающаяся с $K$ и тем более, с элементами последовательности.
Рассмотрим
$$(A - \mu E)x =\br{(a_1-\mu)x_1, (a_2-\mu)x_2\etc}.$$
Очевидно, к этому оператору имеется обратный оператор
$$(A - \mu E)^{-1}x =\hr{\frac{1}{a_1-\mu}x_1, \frac{1}{a_2-\mu}x_2\etc}.$$
Он будет ограниченным, поскольку числа $a_i-\mu$ отделены от нуля числом $\de$.
\end{proof}

\begin{theorem}
Пусть $\mu$\т мера Лебега на $(0,1)$.
Пусть $A\cln L_1(\mu) \ra L_1(\mu)$\т оператор, заданный по правилу $A\cln f\mapsto \ph f$,
где $\ph$\т ограничена $\mu$\д почти всюду.
Покажем, что
$$\Spec A = \Ess \ph := \hc{\la \in \Cbb\cln \fa \ep > 0 \text{ имеем } \mu\Br{\ph^{-1}\br{B(\la, \ep)}} > 0},$$
то есть спектр состоит из всех существенных значений функции $\ph$.
\end{theorem}
\begin{proof}
Докажем теорему в случае, когда $\Img \ph \subs \R$. Рассмотрим случай
$\la = 0$. В этом случае для $\fa \ep > 0$ имеем $\mu\br{\ph^{-1}(-\ep,\ep)} > 0$. Рассмотрим
$Y_n = \ph^{-1}\hr{-\frac1n,\frac1n}$ и $f_n := \Ibb_{Y_n}$. Сравним теперь нормы
$\hn{f_n}$ и $\hn{A f_n}$. Имеем
$\hn{f_n} = \mu(Y_n) > 0$, а
$$
  \hn{A f_n} = \int \hm{\ph f_n}d\mu =
  \ints{Y_n}\hm{\ph}d\mu \le \frac1n\mu(Y_n) = \frac1n\hn{f_n}.
$$
В силу следствия \ref{imp:unrevertibility} из теоремы Банаха, оператор $A$ необратим.
Общий случай сводится к рассмотрению функции $\wt\ph = \ph - \la$. В самом деле, если $\la$\т существенное
значение для $\ph$, то $0$ будет существенным значением для $\wt\ph$. Таким образом,
$\Ess \ph \subs \Spec A$. Пусть $\nu \notin \Ess\ph$, тогда $\exi \ep > 0$,
для которого имеем $\mu\br{\ph^{-1}(\nu-\ep,\nu+\ep)} = 0$. Следовательно, для $\mu$\д почти всех $x$
имеем $\hm{\ph(x) - \nu} > \frac{\ep}{2}$. Оператор $A - \nu E$ будет обратим, если можно делить на функцию
$\ph - \nu$. Но это как раз можно делать, поскольку функция $\frac{1}{\ph(x) - \nu}$ ограничена
для $\mu$\д почти всех $x$. Итак, $\nu \notin \Spec A$, поэтому $\Ess \ph = \Spec A$.
\end{proof}


\subsubsection{Резольвента. Непустота спектра ограниченного оператора}

\index{Резольвента оператора}
\index{Оператор!резольвента}
\begin{df}
Пусть $A\cln X \ra X$\т оператор. \emph{Резольвентой} оператора называется функция
$$\Rc_A \cln \Cbb \wo \Spec A \ra \End X,$$
определённая по правилу
$$\Rc_A(z) := (A - z E)^{-1}.$$
\end{df}

\begin{note}
В тех случаях, когда понятно, о каком операторе идёт речь, мы будем опускать индекс у резольвенты.
\end{note}

\index{Теорема!Тождество Гильберта}
\begin{lemma}[Тождество Гильберта]
\label{lemma:hilbert.equality}
Для резольвенты справедливо следующее соотношение:
$$\Rc(z) - \Rc(w) = (z-w)\Rc(z)\Rc(w).$$
\end{lemma}
\begin{proof}
Рассмотрим тождество $(A - wE) - (A - z E) = (z-w)E$. Домножим его слева на оператор $\Rc(z)$, а справа на $\Rc(w)$,
получим
$$\Rc(z)(A - wE)\Rc(w) - \Rc(z)(A - zE)\Rc(w) = \Rc(z)(z-w)\Rc(w).$$
После сокращения прямых и обратных операторов и перенесения коэффициента $z-w$ на первое место получим
$$\Rc(z) - \Rc(w) = (z-w)\Rc(z)\Rc(w),$$
а ровно это и требовалось доказать.
\end{proof}

Покажем, что резольвенту можно дифференцировать как оператор.
Используя обычное определение производной и тождество Гильберта, получаем
$$\Rc'(z) := \liml{h \ra 0}\frac{\Rc(z+h) - R(z)}{h} = \liml{h\ra 0} \frac{(z+h-z)\Rc(z+h)\Rc(z)}{h} = \Rc^2(z).$$

\comment{Здесь всё правильно. В литературе встречается другое определение резольвенты,
а именно $\Rc(z) = (zE-A)^{-1}$. В этом случае в тождестве Гильберта появится перемена знаков.}

\begin{theorem}
Спектр ограниченного оператора непуст.
\end{theorem}
\begin{proof}
Допустим, что спектр оператора пуст. Заметим, что $\Rc(z) \ra 0$ при $z \ra \infty$. В самом деле,
$$(A - zE)^{-1} = \hr{-z\hr{E - \frac{A}{z}}}^{-1} = -\frac{1}{z}\hr{E - \frac{A}{z}}^{-1}=
-\frac{1}{z}\hr{E + \frac{A}{z} + \frac{A^2}{z^2}+\ldots} \ra 0,$$
ибо второй множитель ограничен, а первый стремится к $0$.

Рассмотрим какой\д нибудь линейный непрерывный функционал $\ph \in X^*$. Рассмотрим обычную
комплекснозначную функцию
$f(z) = \ph\br{\Rc(z)x}$. Покажем, что $f$\т целая функция в $\Cbb$.
Продифференцируем её:
$$\frac{f(z+h) - f(z)}{h} = \frac{\ph\br{\Rc(z+h)x} - \ph\br{\Rc(z)x}}{h} = \ph\hr{\frac{\Rc(z+h)-\Rc(z)}{h}x},$$
и, переходя к пределу при $h \ra 0$, получаем $\ph\br{\Rc^2(z)x}$.
Далее, в силу доказанного выше, имеем $f(z) \ra 0$, откуда следует её ограниченность.
По теореме Лиувилля $f \equiv \const$, но поскольку
$f \ra 0$, получаем $f \equiv 0$. Следовательно, для всякого $x$ и произвольного функционала $\ph$
имеем $\ph\br{R(z)x}=0$.

В силу следствия \ref{imp:about.zero.vector} из теоремы Хана\ч Банаха, если все функционалы
из $X^*$ обнуляются на векторе,
то этот вектор нулевой. Следовательно, для $\fa z$ и любого $x \in X$ имеем $\Rc(z)x = 0$,
но это означает, что резольвента $\Rc(z)$ является тождественно нулевым оператором при
всех $z$. Но это противоречит нашему допущению.
\end{proof}

\subsection{Теорема Хана\ч Банаха о продолжении линейных функционалов}

\subsubsection{Вещественный вариант ТХБ}

В своей слабой формулировке теорема Хана\ч Банаха утверждает следующее:

\begin{theorem}
Пусть $X$\т нормированное пространство, а $M \subs X$\т подпространство.
Пусть $\ph \in M^*$\т ограниченный функционал, тогда
существует $\psi \in X^*$ такой, что $\psi\evn{M} = \ph$ и $\hn{\psi} = \hn{\ph}$.
\end{theorem}

Мы докажем теорему Хана\ч Банаха для случая $K = \R$ в следующей формулировке:

\index{Теорема!Хана\ч Банаха о продолжении ограниченного функционала}
\begin{theorem}[Хана\ч Банаха о продолжении ограниченного функционала]
Пусть $X$\т вещественное нормированное пространство. Пусть $M \subs X$\т линейное подпространство.
Пусть на $M$ задан функционал $\ph$ такой, что существует калибровочная функция
$f\cln X \ra \R$, удовлетворяющая следующим свойствам:
\begin{gather*}
\fa x \in M \text{ имеем }\ph(x) \le f(x),\\
\fa x,y \in X \text{ имеем }f(x+y) \le f(x) + f(y),\\
\fa x \in X, \fa t \ge 0\text{ имеем }f(tx) = t f(x).
\end{gather*}
Тогда существует продолжение $\psi$ функционала $\ph$ на пространство $X$ с сохранением требования $\psi \le f$.
\end{theorem}
\begin{proof}
Первый шаг доказательства будет состоять в том, что можно продолжить функционал $\ph$ на пространство
$\ha{M, x_1}$, где $x_1 \notin M$, то есть на пространство <<на единицу большей размерности>>.

Запишем условие калибровки
$\ph(x + \la x_1) \le f(x + \la x_1)$. Мы считаем, что $\la \neq 0$.
В силу линейности имеем $\ph(x)  + \la\ph(x_1) \le f(x + \la x_1)$.
Пусть сначала $\la > 0$. Тогда
$$\ph(x_1) \le \frac1\la f(x + \la x_1) - \frac1\la\ph(x).$$
Положительные коэффициенты можно вносить под знак калибровочной функции. Поэтому
$$\ph(x_1) \le f\hr{\frac{x}{\la} + x_1} - \ph\hr{\frac{x}{\la}}.$$
Заметим, что когда $x$ бегает по $M$, вектор $\frac{x}{\la}$ тоже бегает по всему $M$.
Переобозначив $y_1 = \frac{x}{\la}$, получим
\eqn{\label{eqn:banach-han-th.upper}\ph(x_1) \le f(y_1 + x_1) - \ph(y_1).}

Теперь посмотрим, что получится, если $\la < 0$. Имеем
$$\ph(x_1) \ge \frac{1}{\la}f(x+\la x_1) -\frac1\la\ph(x).$$
Мы будем вносить под знак $f$ положительное число $-\frac{1}{\la}$, поэтому запишем это в виде
$$\ph(x_1) \ge -\hr{-\frac{1}{\la}}f(x+\la x_1)+ \hr{-\frac1\la}\ph(x).$$
Отсюда
$$\ph(x_1) \ge -f\hr{-\frac{x}{\la}-x_1} + \ph\hr{-\frac{x}{\la}}.$$
Обозначая $y_2 = -\frac{x}{\la}$, получаем второе условие
\eqn{\label{eqn:banach-han-th.lower}\ph(x_1) \ge -f(y_2 - x_1) + \ph(y_2).}

Итак, чтобы были выполнены условия теоремы, нужно, чтобы для всех $y_1$ и $y_2$ из пространства $M$
были выполнены условия \eqref{eqn:banach-han-th.upper} и \eqref{eqn:banach-han-th.lower}.
Для этого достаточно показать, что
$$f(y_1+x_1) + f(y_2-x_1) \ge \ph(y_1) + \ph(y_2).$$
В самом деле, это условие можно переписать в виде
$$f(y_1+x_1) + f(y_2-x_1) \ge \ph(y_1 + y_2),$$
а в силу свойств функции $f$ имеем
$$f(y_1+x_1) + f(y_2-x_1) \ge f(y_1+x_1+y_2-x_1) = f(y_1+y_2) \ge \ph(y_1 + y_2).$$
Но это ровно то, что нам нужно.
Итак, мы доказали, что можно продолжить функционал на пространство $\ha{M, x_1}$.

Чтобы теперь показать, что можно продлить функционал на всё пространство $X$,
нужно либо предполагать сепарабельность пространства $X$ и доказывать это утверждение
методом математической индукции, либо воспользоваться трансфинитными средствами,
а именно так называемой леммой Цорна. Мы пойдём по второму пути.
Для начала напомним необходимые понятия
из курса математической логики.

\index{Бинарное отношение}
\index{Отношение!бинарное}
\index{Отношение!рефлексивное}
\index{Отношение!симметричное}
\index{Отношение!транзитивное}
\index{Отношение!антисимметричное}
\begin{df}
\emph{Бинарным отношением} $\Rs$ на множестве $P$ называется подмножество $\Rs \subs P\times P$. Говорят,
что $x \Rs y$, если $(x,y) \in \Rs$. Отношение называется \emph{рефлексивным}, если имеет место $x\Rs x$;
\emph{симметричным}, если из $x \Rs y$ следует $y \Rs x$; \emph{антисимметричным}, если из $x \Rs y$ и $y \Rs x$ следует $x = y$;
\emph{транзитивным}, если из $x\Rs y$ и $y \Rs z$ следует $x \Rs z$.
\end{df}

\index{Частичный порядок}
\index{Частичное упорядочение}
\index{Упорядочение!частичное}
\index{Отношение!частичного порядка}
\begin{df}
Бинарное отношение называется \emph{отношением частичного порядка},
если оно антисимметрично и транзитивно.
\end{df}

\index{Линейное упорядочение}
\index{Упорядочение!линейное}
\index{Линейный порядок}
\begin{df}
Частичное упорядочение называется \emph{линейным упорядочением}, если всегда выполнено $x \Rs y$ или $y \Rs x$.
Иными словами, любые два элемента сравнимы.
\end{df}

\index{Максимальный элемент}
\index{Элемент!максимальный}
\index{Верхняя грань}
\begin{df}
Пусть $(P, \prec)$\т частично упорядоченное множество.
Элемент $p \in P$ называется \emph{максимальным}, если из $p \prec q$ следует, что $p=q$.
Элемент $p\in P$ для цепи $S$ называется \emph{верхней гранью}, если для $\fa q \in S$ имеем $q \prec p$.
\end{df}

\index{Лемма!Цорна}
\begin{stm}[Лемма Цорна]
Пусть $(P, \prec)$\т частично упорядоченное множество. Если для любой цепи, то есть линейно упорядоченного
подмножества $P$ существует верхняя грань, то существует максимальный элемент в $P$.
\end{stm}

Покажем теперь, как с использованием леммы Цорна доказать нашу теорему. Пусть
$(M_\al, \ph_\al)$\т цепь, полученная последовательным расширением пространств
с помощью присоединения одного вектора. Рассмотрим $\bigcup M_\al$
и определим на нём функционал $\ph$ так: $\ph(x) := \ph_\al(x)$, как только $x \in M_\al$.
Корректность данного определения очевидна. В силу леммы Цорна в нашем множестве существует максимальный элемент
$\wt M$. Если он совпадает с $X$, то всё доказано. Если же $\wt M \neq X$, то найдётся вектор
из $X$, не лежащий в $\wt M$, но это противоречит тому, что продолжать дальше некуда.
\end{proof}

\begin{imp}
Пусть $M$\т замкнутое подпространство, и $x\notin M$. Тогда есть такой функционал, что $\ph(x)=1$, а $\ph(M)=0$.
\end{imp}
\begin{proof}
Пусть $z = y + \al x$, где $y\in M$. Положим $\ph(x) := 1$ и $\ph(y)=0$ для $\fa y \in M$.
Продолжая его на всё пространство, получим искомый функционал. Остается проверить его ограниченность.
Имеем
\eqn{\hn{\ph} := \supl{y+\al x} \frac{|\ph(y+\al x)|}{\hn{y+\al x}} \le \frac{1}{\rh(M,x)}.}
 Но так как $x\notin M$ и
подпространство $M$ замкнуто, то $\rh(M,x)> 0$, а значит, $\ph$ ограничен.
\end{proof}

\begin{imp}\label{imp:about.zero.vector}
Пусть про вектор $x$ известно, что $\fa \ph \in X^*$ имеем $\ph(x) = 0$. Тогда $x = 0$.
\end{imp}
\begin{proof}
Рассмотрим линейную оболочку $M = \ha{x}$.  Определим на ней линейный функционал:
положим $\ph(x) = 1$ и распространим его по линейности на все одномерное пространство $M$.
По теореме Хана\ч Банаха его можно продлить до функционала $\psi$ на всё пространство $X$ так, что
на пространстве $M$ имеем $\psi = \ph$. Но тогда $\psi(x) = 1$, а это противоречит условию.
\end{proof}


\bigskip
\comment{Начиная с этого места текст набирался крайне быстро, поэтому вероятность ошибок очень велика.}
\bigskip

\subsubsection{Обобщение ТХБ на комплексный случай}

\begin{df}
\emph{Полунормой} называется функция $p(x)\cln X\ra \R$ со свойствами:
\begin{items}{-2}
\item $p(x+y)\le p(x)+p(y)$;
\item $p(\al x) = |\al|p(x)$;
\item $p(x)\ge 0$;
\end{items}
\end{df}

Как видно из определения, полунорма отличается от нормы отсутствием свойств точности ($\hn{x}=0\Lra x=0$).

Пусть на пространстве $X$ задана полунорма $p(x)$, и $M\subs X$\т подпространство, на котором задан
ограниченный функционал $\ph$, для которого $|\ph(x)|\le p(x)$. Тогда существует продолжение $\psi$ функционала $\ph$,
ибо норма играет роль калибровочной функции.

Пусть $\ph$\т комплексный функционал. Тогда $\ph(x) = \Rea \ph(x)+i\Img\ph(x)$. Обозначим вещественную и
мнимую части через $R(x)$ и $I(x)$ соответственно.
Ясно, что не всякие вещественные функционалы могут быть в роли $R$ и $I$. Действительно, имеем

$$\ph(ix) = i\ph(x) = R(ix)+iI(ix) = iR(x)+(-1)I(x).$$
Значит, $I(x) = -R(ix)$.

\begin{problem}
Доказать, что функция $\ph(x)=R(x)-iR(ix)$, где $R$\т вещественный функционал, является (комплексным) линейным
функционалом.
\end{problem}

Функционал $R$ можно продолжить до $\wt R$ с сохранением условия $|\wt R(x)|\le p(x)$.
Для всех $\al$ таких, что $|\al|=1$, имеем
$$|\ph(x)|=|\ph(\al x)| = |\ub{\al_0\ph(x)}_{\in \R}| = p(\al_0x) = p(x).$$
Далее,
$$|\psi(x)| = |\ub{\psi(\al_0 x)}_{\in \R}| = |\wt R(\al_0 x)| \le p(x).$$

Значит, можно продолжить $\psi$ полностью: $\wt \psi(x) := \wt R(x)-i\wt R(ix)$.

\begin{note}
Аналога ТХБ для операторов не существует. Например, пространство $c_0$ не является ретрактом, то есть
оператор $E\cln c_0\ra c_0$ нельзя продолжить на $\ell_\bes$ так, чтобы $A\cln\ell_\bes\ra \ell_\bes$ совпадал с $E$
на подпространстве $c_0$ и $\hn{A}=1$.
\end{note}

\begin{problem}
Доказать, что всякий непрерывный функционал $\ph$ на $c_0$ можно единственным образом
продолжить на $\ell_\bes$ с сохранением нормы.
\end{problem}

\subsection{Компактность. Слабая сходимость и слабая компактность}

\subsubsection{Компактные и предкомпактные множества}

\begin{df}
Пусть $X$\т метрическое пространство. Множество $M \subs X$ называется \emph{компактным},
если из любой последовательности $\hc{x_i} \subs M$ можно выделить подпоследовательность,
сходящуюся к $x \in M$.
\end{df}

\begin{df}
Множество $M$ называется \emph{предкомпактным}, если
из любой последовательности $\hc{x_i} \subs M$ можно выделить фундаментальную
подпоследовательность.
\end{df}

\begin{df}
Говорят, что множество $N$ образует $\ep$\д \emph{сеть} для множества $M$,
если в $\ep$\д окрестности любой точки $x \in M$ найдётся точка из $N$.
\end{df}

\begin{theorem}[Критерий Хаусдорфа]
Бесконечное подмножество $M$ полного метрического пространства предкомпактно тогда и только тогда,
когда для $\fa \ep > 0$ существует конечная $\ep$\д сеть для $M$.
\end{theorem}
\begin{proof}
Пусть нашлось такое $\ep_0 > 0$, что для него не существует конечной $\ep_0$\д сети.
Иначе говоря, всякое конечное семейство окрестностей радиуса $\ep_0$ не может покрыть
всё множество $M$. Возьмём $x_1 \in M$ и накроем его $\ep_0$\д окрестностью~$U_1$. Набор
$\hc{U_1}$ не покрывает~$M$, поэтому найдётся $x_2 \in M \wo U_1$.
Накроем его окрестностью~$U_2$, но $\hc{U_1,U_2}$ снова не покроет всё множество~$M$.
Выбирая $x_3 \in M \wo (U_1 \cup U_2)$ и так далее, получим последовательность, у которой
$\rho(x_i, x_j) \ge \ep_0$, поэтому из неё нельзя выделить фундаментальную.
Таким образом, $M$ не предкомпактно.

Обратно, пусть для $\fa \ep > 0$ существует конечная $\ep$\д сеть. Пусть $\hc{x_i} \subs M$\т произвольная
последовательность, выделим из неё фундаментальную.
Возьмём $1$\д сеть, тогда найдётся окрестность, в которой бесконечно много членов последовательности.
Выберем оттуда один элемент $x_1^*$ и в качестве новой последовательности возьмём только то, что
попало в эту окрестность. Далее, существует конечная $\frac12$\д сеть, покрывающая новую
последовательность. Снова выберем ту окрестность сети, в которой бесконечно много элементов,
и в ней возьмём произвольный $x_2^*$.
Продолжим этот процесс, то есть на $n$\д м шаге будем выбирать $\frac1{2^n}$\д сеть.
Ясно, что последовательность $\hc{x_i^*}$ будет фундаментальна.
\end{proof}

\begin{lemma}[Критерий конечномерности пространства]
Нормированное пространство $X$ конечномерно тогда и только тогда,
когда в нём всякое бесконечное ограниченное множество предкомпактно.
\end{lemma}
\begin{proof}
Всякое бесконечное ограниченное множество в конечномерном пространстве
предкомпактно, поскольку в этом случае $X \cong \Cbb^n$ (или $\R^n$),
а для этих пространств предкомпактность эквивалентна ограниченности.

Обратно, пусть всякое ограниченное подмножество в $L$ предкомпактно.
Допустим, что $X$ бесконечномерно, тогда возьмём единичный вектор $e_1 \in X$.
По предположению, $X \neq X_1 := \ha{e_1}$, тогда по лемме Рисса найдётся единичный
вектор $e_2 \notin X_1$, для которого $\rho(e_2, X_1) \ge \frac12$.
Вновь по предположению $X \neq X_2 := \ha{e_1,e_2}$, тогда построим
ещё один вектор $e_3$, для которого $\rho(e_3, X_2) \ge \frac12$, и так далее.
Цепочка подпространств $X_n$ будет строго возрастать, и последовательность $\hc{e_i}$
будет ограниченным и не предкомпактным множеством, так как расстояние между любыми двумя
её элементами не меньше $\frac12$.
\end{proof}

\subsubsection{Слабая сходимость и слабая компактность}

\begin{df}
Говорят, что последовательность $x_n$ \emph{слабо сходится} к $x$, если для любого ограниченного функционала $f$
на $X$ имеем $f(x_n) \ra f(x)$. Обозначение: $x_n \convw x$.
\end{df}

\begin{df}
Говорят, что последовательность функционалов $f_n$ \emph{$*$-слабо сходится} к $f$, если для любого вектора $x \in X$
имеем $f_n(x) \ra f(x)$. Обозначение: $f_n \convws f$.
\end{df}

\comment{Не знаю, зачем лепить звездочки куда ни попадя. И так понятно, что слабая сходимость векторов\т на функционалах,
а слабая сходимость функционалов\т на векторах. Поэтому будем писать $f_n \convw f$.}

\begin{df}
Говорят, что множество \emph{слабо компактно}, если из любой его последовательности элементов можно выделить
слабо сходящуюся.
\end{df}

\begin{theorem}[О слабой компактности единичной сферы]
Пусть $X$\т сепарабельное нормированное пространство. Тогда единичный шар в $X^*$ слабо компактен.
\end{theorem}
\begin{proof}
Выберем в $X$ счётное всюду плотное множество $D := \hc{x_n}$. Пусть $\hc{f_n}$\т ограниченная
последовательность функционалов. Рассмотрим последовательность чисел $\hc{f_n(x_1)}$. Она ограничена
(мы сидим в единичном шаре $\hn{f} \le 1$), а потому содержит сходящуюся. Обозначим её через $f_n\n1(x_1)$.
Рассмотрим последовательность чисел
$\bc{f_n\n1(x_2)}$. Она тоже содержит сходящуюся подпоследовательность $f_n\n2(x_2)$.
Продолжая этот процесс и выделяя диагональ $\ph_n := f_n\n n$, получим последовательность функционалов,
сходящуюся на всех векторах~$x_i$.

Покажем, что сходимость имеет место для всех векторов $x \in X$. Покажем фундаментальность
последовательности $\hc{\ph_i(x)}$.
Рассмотрим последовательность элементов из $D$, сходящуюся к $x$, тогда, очевидно,
\begin{multline*}
\hm{\ph_m(x) - \ph_n(x)} = \hm{\ph_m(x) - \ph_m(x_k) + \ph_m(x_k) - \ph_n(x_k) + \ph_n(x_k) - \ph_n(x)} \le\\\le
\hm{\ph_m(x) - \ph_m(x_k)} + \hm{\ph_m(x_k) - \ph_n(x_k)} + \hm{\ph_n(x_k) - \ph_n(x)} \ra 0.
\end{multline*}
Но это и значит, что $\ph_n$ слабо сходится. Теорема доказана.
\end{proof}

\begin{problem}
Доказать, что на втором шаге доказательства всё корректно: результат не зависит от выбора
последовательности, сходящейся к $x$.
\end{problem}

\begin{lemma}
Существует изометричное вложение $X \inj X^{**}$.
\end{lemma}
\begin{proof}
Зададим вложение так: $x \mapsto F_x$, где $F_x \in X^{**}$\т функционал на $X^*$,
действующий на элементах $f \in X^*$ следующим образом:
$$F_x\cln f \mapsto f(x).$$
Это вложение, очевидно, линейно. Докажем, что это изометрия. Обозначим норму в $X^{**}$ через $\hn{\cdot}_2$.
С одной стороны, по определению нормы имеем $\hm{f(x)} \le \hn{f} \cdot \hn{x}$, поэтому
$$\hn{x} \ge \supl{f} \frac{\hm{f(x)}}{\hn{f}} = \hn{x}_2.$$
С другой стороны, в силу одного из следствий теоремы Хана\ч Банаха,
для всякого $x_0 \in X$ найдётся функционал~$f_0$ такой, что
$\hm{f_0(x_0)} = \hn{f_0} \cdot \hn{x_0}$, поэтому
$$\hn{x}_2 = \supl{f}\frac{\hm{f(x)}}{\hn f} \ge \hn{x},$$
следовательно, $\hn{x} = \hn{x}_2$.
\end{proof}

\begin{stm}
Слабо ограниченная последовательность ограничена по норме.
\end{stm}
\begin{proof}
Применим теорему Банаха\ч Штейнгауза к пространствам $X^*$ и $X^{**}$,
то есть вместо последовательности $\hc{x_i}$ рассматривая её образ в $X^{**}$.
В силу этой теоремы семейство образов будет ограниченным, но в силу изометричности вложения
этим свойством будет обладать и исходное семейство векторов.
\end{proof}

\comment{На лекциях следующего утверждения не было, но наверняка было на семинарах.
А кому\д нибудь, возможно, попадётся на экзамене.}

\begin{stm}
Слабый предел единствен.
\end{stm}
\begin{proof}
Допустим, что $x_n \convw x$ и $x_n \convw y$, причём $x \neq y$. Тогда, по определению
слабой сходимости, для любого $f$ имеем $f(x_n) \ra f(x)$ и $f(x_n) \ra f(y)$. Следовательно,
для всякого функционала $f$ имеем $f(x) = f(y)$, то есть $f(x-y) = 0$.
Но по лемме о продолжении функционала существует $f$, который равен~$1$ на  векторе $x-y$.
Противоречие.
\end{proof}

\subsection{Ещё две теоремы о сопряжённых пространствах}

\subsubsection{Общий вид функционалов в $L_1[0,1]$. Несепарабельность $L_\bes^*$}

\begin{theorem}
Рассмотрим пространство $X := L_1([0,1],\mu)$ и любой ограниченный функционал $\ph\in X^*$. Тогда найдётся
функция $g \in L_\bes$ такая, что
\eqn{\ph(f) = \int f g \,d\mu.}
\end{theorem}
\begin{proof}
Докажем сначала для индикаторов. Рассмотрим функцию $\nu(A) = \ph(\Ibb_A)$, где $A$\т измеримое множество.
Легко видеть, что эта функция абсолютно непрерывна относительно меры $\mu$. Если мы покажем, что $\nu$ является зарядом, то
по теореме Радона\ч Никодима найдётся функция $g$ такая, что $\nu(A) = \int g \cdot \Ibb_A \,d\mu$.
Имеем
\eqn{\label{SupremumBoundness}\supl{A\cln \mu(A) > 0} \frac{|\ph(\Ibb_A)|}{\mu(A)} < M,}
так как $\mu(A) = \hn{\Ibb_A}$, а функционал $\ph$ ограничен.

Покажем, что $g \in L_\bes$. В самом деле, положим $A_c := \hc{x\cln g(x) > c}$. Допустим, что $g \notin L_\bes$. Тогда
для $\fa c$ имеем $\mu(A_c) > 0$. Значит,
\eqn{\frac{\ph(\Ibb_{A_c})}{\hn{\Ibb_{A_c}}} \ge \frac{c \cdot \mu(A_c)}{\mu (A_c)} = с.}
Но в силу \eqref{SupremumBoundness} это выражение не превосходит $M$, а мы предположили, что $c$ произвольно.
Противоречие.

Таким образом, для индикаторов утверждение теоремы проверено. В общем случае, как обычно, приближаем
произвольную функцию ступенчатыми: $f_n\ra f$. Тогда
$$\mat{
\ph(f_n) & = & \int f_n g\,d\mu\\
\downarrow &  & \downarrow \\
\ph(f) & = & \int f g\,d\mu}
$$
и всё доказано.

Покажем, что $\nu$ является зарядом, то есть проверим её счётную аддитивность. Имеем
\eqn{\nu\br{\cupsql{i=1}{\bes} A_i} = \nu\br{\cupsql{i=1}{N-1}A_i} + \nu\br{\cupsql{i=N}{\bes}A_i}.}
Конечная аддитивность очевидна, а второе слагаемое стремится к нулю в силу абсолютной непрерывности, так как
$\cupsql{i=N}{\bes}A_i \searrow \es$ при $N\ra \bes$.
\end{proof}

\begin{lemma}
Пусть пространство $X$ несепарабельно. Тогда $X^*$ также несепарабельно.
\end{lemma}
\begin{proof}
Допустим, что $X^*$ сепарабельно. Пусть $\hc{\ph_i}$\т счётное всюду плотное семейство функционалов. Найдем семейство
единичных векторов $x_i$ таких, что $\ph_i(x_i) \ge \frac12\hn{\ph_i}$.
Пусть $M := \Cl \ha{x_i}$\т замкнутое подпространство. Покажем, что $M \neq X$. В самом деле, если бы
$M$ совпало с~$X$, то линейные комбинации векторов $x_i$ с рациональными координатами были бы плотны в $X$,
что противоречит несепарабельности $X$.

Возьмём вектор $y$ такой, что $\rh(y, M) > 0$ и $\hn{y}=1$.
По следствию теоремы Хана\ч Банаха найдётся функционал~$\psi$ такой, что $\psi(y)=1$ и $\psi(M)=0$.
Тогда, очевидно, $\hn{\psi - \ph_i} \ge |\psi(y)-\ph_i(y)|$, а кроме того,
\eqn{\hn{\psi - \ph_i} \ge |\psi(x_i)-\ph_i(x_i)|\ge \frac12\hn{\ph_i}.}
Но по предположению о сепарабельности пространства найдётся последовательность функционалов $\ph_{i_k}$, для которой ${\hn{\psi - \ph_{i_k}}\ra 0}$
при $k\ra\bes$. Значит, в силу написанных неравенств $\hn{\psi_{i_k}}\ra 0$. Но отсюда следует, что
$\hn{\psi}=0$. Это противоречит тому, что $\psi(y)\neq 0$.
\end{proof}

\begin{theorem}
Пространство $L_\bes^*[0,1]$ несепарабельно.
\end{theorem}
\begin{proof}
В самом деле, пространство $L_\bes[0,1]$ несепарабельно, так как можно взять семейство континуальное семейство
функций вида $I_x:= \Ibb_{[0,x]}$, и если $x\neq y$, то $\hn{I_x-I_y} = 1$, \те в пространстве существует
несчётномерный <<ёж>>. А тогда по лемме и пространство $L^*_\bes[0,1]$ будет несепарабельным.
\end{proof}
\begin{imp}
Существуют непрерывные функционалы на $L_\bes$, не задающиеся формулой $\ph(f) = \int fg\,d\mu$, где $g\in L_1$, а $f\in L_\bes$.
\end{imp}

\section{Гильбертовы пространства}

\subsection{Операторы в гильбертовых пространствах}

\subsubsection{Понятие гильбертова пространства}

\begin{df}
\emph{Гильбертовым} пространством называется евклидово пространство, полное относительно нормы,
задаваемой скалярным произведением: $\hn{x} := \sqrt{(x,x)}$. Его мы всегда будем обозначать
буквой~$H$.
\end{df}

Напомним, что скалярное произведение предполагается невырожденным! Именно поэтому норма задана корректно.

\subsubsection{Сопряжённые операторы}

\begin{df}
Пусть $A$\т ограниченный оператор в~$H$.
Если оператор $B$ таков, что $(Ax,y) = (x,By)$ для всех $x,y\in H$, то $B$ называется \emph{сопряжённым} к $A$
и обозначается $A^*$. Если $A = A^*$, то $A$ называется \emph{самосопряжённым}.
\end{df}

\begin{note}
Существование сопряжённого оператора для всякого ограниченного оператора будет доказано несколько позже.
\end{note}

Отношение сопряжённости является симметричным: если $B$ сопряжён к $A$, то $A$ сопряжён к $B$.
Действительно, имеем
$$(Ax,y) = (x,By) \; \Lra \; \ol{(Ax,y)} = \ol{(x,By)} \; \Lra \; (By,x) = (y,Ax),$$
а это и означает, что оператор~$A$ сопряжён к~$B$.

\begin{stm}
Имеет место соотношение $(A^*)^* = A$.
\end{stm}
\begin{proof}
По определению имеем для всех $x,y$
$$\case{(Ax,y) = (x,A^*y),\\ \br{(A^*)^*x,y} = \br{x,A^*y};} \;\Ra \; \br{Ax,y} = \br{(A^*)^*x,y}
\; \Lra \; \br{(A-(A^*)^*)x, y} = 0,$$
но из невырожденности скалярного произведения следует $\br{A-(A^*)^*}x = 0$ для всех $x$,
поэтому $A = (A^*)^*$.
\end{proof}

\subsubsection{Лемма об ортогональной проекции и её следствия}

\comment{Следующие леммы были неоправданно свалены лектором в одну кучу. Они полезны сами по себе.}

\begin{lemma}[Об ортогональной проекции]
Пусть $H_0$\т замкнутое подпространство в $H$. Тогда для любого вектора $h \in H \wo H_0$ найдётся единственный
ближайший вектор из $H_0$.
\end{lemma}
\begin{proof}
Имеем $\rho(h, H_0) =: a > 0$ в силу того, что одно из этих множеств замкнуто, а второе компактно.
Выберем последовательность $\hc{h_n} \subs H_0$ так, чтобы $\rho(h_n,h) \ra a$ при $n \ra \infty$.
Покажем, что $\hc{h_n}$ фундаментальна.
Нам понадобится тождество параллелограмма: <<сумма квадратов диагоналей параллелограмма равна
сумме квадратов его сторон>>. В силу этого тождества для достаточно больших $n$ и $m$ получаем
$$\hn{h_n - h_m}^2 = 2\hn{h-h_n}^2 + 2\hn{h-h_m}^2 - 4\hn{h - \frac{h_n+h_m}{2}}^2 \le
2(a^2 + \ep) + 2(a^2 + \ep) - 4a^2 = 4\ep,$$
и тем самым фундаментальность установлена.

Далее, $H_0$\т замкнутое подпространство полного пространства, и потому оно полно.
Следовательно, $\hc{h_n}$ сходится к некоторому элементу $h_0 \in H_0$.
По непрерывности имеем $\rho(h_n,h) \ra \rho(h_0,h)$. С другой стороны, этот предел равен~$a$
в силу выбора~$h_n$. Следовательно, $\rho(h_0,h)=a$.
\end{proof}

\begin{imp}
Пусть $H_0 \subs H$\т замкнутое подпространство.
Всякий вектор $h \in H$ представим в виде $h = h_0 + g$, где $h_0 \in H_0$, а $g \in H_0^\bot$.
\end{imp}
\begin{proof}
Пусть $x \in H_0$. По лемме, функция $d(x) := \hn{h - x}^2$ достигает минимума на
некотором векторе $h_0 \in H_0$. Поэтому функция $\ph(t) := \hn{h - h_0 + tx}^2$
имеет минимум при $t = 0$. Тогда $\ph'(0) = 0$. Распишем скалярный квадрат:
$\ph(t) = (h - h_0 +tx, h-h_0 +tx) = \hn{h-h_0}^2 + 2t\Rea(x,h-h_0) + t^2(x,x)$,
поэтому $\ph'(0) = 2\Rea(x,h -h_0) = 0$. Далее, вместо вектора $x$ рассматривая вектор $i\cdot x$,
получаем $\Img(x,h-h_0) = 0$. Следовательно, $(x,h-h_0) = 0$.
Таким образом, всякий вектор $x \in H_0$ ортогонален вектору $h - h_0$, то есть $h-h_0 \in H_0^\bot$.
Тождество $h = h_0 + (h-h_0)$, очевидно, является искомым разложением.
\end{proof}

\subsubsection{Общий вид линейного функционала в гильбертовом пространстве}

Гильбертовы пространства хороши тем, что всякий функционал в них устроен очень просто: это скалярное умножение
не некоторый (фиксированный) вектор.

\begin{lemma}[Рисса]
Пусть $f$\т ограниченный функционал. Тогда найдётся вектор $h_0 \in H$, для которого $f(x) = (x,h_0)$.
\end{lemma}
\begin{proof}
Если $f \equiv 0$, то доказывать нечего: берём $h_0 := 0$. Пусть теперь $f \neq 0$.
Очевидно, ядро $K := \Ker f$\т замкнутое подпространство. Покажем, что $\dim K^\bot = 1$.
Рассмотрим ненулевые вектора $h_1, h_2 \in K^\bot$. Рассмотрим вектор
$$v = f(h_1)h_2 - f(h_2)h_1.$$
С одной стороны,
$v \in K^\bot$ как линейная комбинация векторов из $K^\bot$. С другой стороны, он лежит и в~$K$, потому что
$f(v) = f(h_1)f(h_2) - f(h_2)f(h_1) = 0$. Но $K \cap K^\bot = 0$, поэтому $v = 0$, следовательно
вектора $h_1$ и $h_2$ пропорциональны.

Рассмотрим уравнение $f(x) = (x, \mu h_1)$, где $\mu$\т неизвестное.
Определим его, подставив $x = h_1$: получим $F(h_1) = \ol \mu (h_1,h_1)$. Итак, $\mu$ найдено.
Тогда для всякого $x \in K^\bot$ имеем $f(x) = (x, \mu h_1)$. В самом деле, $x = \la h_1$, поэтому
$$f(x) = f(\la h_1) = \la f(h_1) = \la (h_1,\mu h_1) = (\la h_1,\mu h_1) = (x,\mu h_1).$$
Аналогично, если $x \in K$, то равенство тоже верно: и слева, и справа получаем ноль. Но поскольку $H = K\oplus K^\bot$,
по следствию из леммы об ортогональной проекции это верно и на всём пространстве.
\end{proof}

\begin{stm}
Сопряжённый оператор существует.
\end{stm}
\begin{proof}
Пусть $A$\т ограниченный линейный оператор в~$H$.
Зафиксируем $y \in H$ и рассмотрим функционал $f(x) := (Ax,y)$. Линейность его очевидна,
а ограниченность следует из неравенства Коши\ч Буняковского:
$$\hm{(Ax,y)} \le \hn{Ax}\cdot \hn{y} \le \hn{A}\cdot \hn{y} \cdot \hn{x}.$$
По лемме Рисса получаем $f(x) = (x,A^*y)$, где $A^*y$\т обозначение для сопряжённого оператора,
применённого к вектору $y$.

Проверим корректность определения. Пусть мы получили таким способом два
вектора $v_1$ и $v_2$. Для них имеем $(Ax,y) = (x,v_1) = (x,v_2)$,
причём это верно для любого $x$. Таким образом, для всех $x$ имеем $(x, v_1 - v_2) = 0$.
Подставим $x = v_1 - v_2$, получим $(v_1-v_2, v_1-v_2) = 0$, откуда $v_1 = v_2$.

Очевидно, что получаемый таким способом оператор будет линейным.
\end{proof}

\subsubsection{Свойства сопряжённых операторов}

\begin{lemma}Для любого ограниченного оператора $A$ имеет место равенство $\hn{A^*} = \hn{A}$.
\end{lemma}
\begin{proof}
Действительно, $\hn{Ax}^2 = (Ax,Ax) = (A^*Ax,x) \le \hn{A^*A}\cdot\hn{x}^2$ по неравенству Коши\ч Буняковского.
Перейдём к верхней грани по $\hn{x} = 1$, получим $\hn{A}^2 \le \hn{A^*A} \le \hn{A^*} \cdot \hn{A}$,
откуда $\hn{A} \le \hn{A^*}$. Меняя в этих выкладках местами операторы $A$ и $A^*$, получаем обратное неравенство.
\end{proof}

\begin{theorem}
Пусть про операторы $A$ и $B$ известно, что $(Ax,y)=(x,By)$. Тогда $\hn{A}\le\bes$.
\end{theorem}
\begin{proof}
Покажем, что ограниченности $A$ эквивалентна тому, что
\eqn{\supl{\substack{\hn x =1\\\hn y=1}} |(Ax,y)| < \bes.}
В самом деле, имеем
\eqn{\supl{\substack{\hn x =1\\\hn {Ax}\neq 0}} \hr{Ax, \frac{Ax}{\hn{Ax}}} = \hn A.}

Рассмотрим функционал $\ph_x(y) := (Ax,y)$. Пусть $\hn x = 1$, тогда мы имеем семейство
непрерывных функционалов $\hc{\ph_x}$. Тогда
\eqn{|\supl{\hn x=1}\ph_x(y)| = \supl{\hn x =1} |(x,By)| \le \hn{By} \le C(y),}
то есть мы имеем семейство поточечно ограниченных функционалов. По теореме Банаха\ч Штейнгауза $\hn{\ph_x}\le C$.
Отсюда $\supl{\substack{\hn x =1\\\hn y=1}} |\ph_x(y)|\le \const$, то есть $\sup |(Ax,y)|\le \const$.
А это и значит, что оператор $A$ ограничен.
\end{proof}


\subsection{Компактные (вполне непрерывные) операторы}

\subsubsection{Определение и свойства компактных операторов}

\begin{df}
Оператор называется \emph{компактным}, если образ единичного шара предкомпактен.
\end{df}

\begin{stm}
Сумма компактных операторов есть снова компактный оператор.
\end{stm}
\begin{proof}
Очевидно, если воспользоваться, например, критерием Хаусдорфа.
\end{proof}

\begin{stm}\label{CompactIdeal}
Произведение компактного и ограниченного операторов есть компактный оператор.
\end{stm}
\begin{proof}
Пусть $A$\т компактный, а $B$\т ограниченный операторы. Сначала покажем,
что оператор $AB$ компактен. Если множество $M$ ограничено, то $B(M)$ тоже ограничено.
Тогда $A\br{B(M)}$ предкомпактно, и всё доказано.

Теперь покажем, что $BA$ тоже компактный оператор. Для этого воспользуемся критерием Хаусдорфа
предкомпактности множества. В силу компактности $A$, для любого $\ep$ в множестве
$A(M)$ существует конечная $\ep$\д сеть. Очевидно, что для множества $B\br{A(M)}$ годится
$\hn{B}\cdot \ep$\д сеть, которая получается из исходной сети после применения оператора $B$.
\end{proof}

\begin{imp}
Компактные операторы образуют двусторонний идеал в алгебре операторов.
\end{imp}

\begin{imp}
Компактный оператор в бесконечномерном пространстве необратим.
\end{imp}
\begin{proof}
В самом деле, допустим противное. Поскольку $AA^{-1} = \id$, в силу предыдущего
утверждения получаем, что~$\id$ является компактным оператором. Но это неверно,
поскольку в бесконечномерном пространстве единичный шар не является предкомпактом.
\end{proof}

\begin{theorem}Следующие утверждения эквивалентны:
\begin{nums}{-2}
\item Оператор $A$ компактен;
\item Оператор $A^*$ компактен;
\item Оператор $A^*A$ компактен.
\end{nums}
\end{theorem}
\begin{proof}
Мы уже знаем, что от умножения на ограниченный с любой стороны компактный оператор не теряет своих чудесных свойств,
поэтому <<$1 \Ra 3$>> и <<$2\Ra 3$>> доказаны.

Докажем, что $3\Ra 1$. Рассмотрим последовательность $\hc{x_n}$ такую, что $\hn{x_n}=1$. Так как оператор
$A^*A$ компактен, то $\hc{A^*Ax_n}$ содержит сходящуюся подпоследовательность. Тогда перенумеруем её, и будем
считать, что $\hc{x_n}$\т это она и есть. Покажем, что $\hc{Ax_n}$ тоже содержит сходящуюся.
Действительно, имеем
\eqn{(Ax_n-Ax_m,Ax_n-Ax_m) = \br{(A^*A)(x_n-x_m),\ub{x_n-x_m}_{\hn{\cdot}\le 2}}\ra 0,}
так как $(A^*A)(x_n-x_m)\ra 0$.
\end{proof}

\begin{theorem}
Пусть $A_n$\т последовательность компактных операторов в банаховом пространстве, и $A_n \ra A$ по норме.
Тогда $A$ компактен.
\end{theorem}
\begin{proof}
Пусть $\hc{x_n}$\т ограниченная последовательность. Нужно доказать, что
из последовательности $\hc{Ax_n}$ можно выбрать фундаментальную.

Так как $A_1$ компактен, то выбираем последовательность $x_n\n1$ такую, что
последовательность $A_1 x_n\n1$ сходится. Из неё выбираем $x_n\n2$ такую, что $A_2x_n\n2$ сходится,
и так далее. Возьмём диагональ $y_i := x_i\n i$ и покажем, что последовательность $Ay_i$ фундаментальна.
По условию $\hn{x_n} \le C$, а $\hn{A_k y_n - A_k y_m} \ra 0$ в силу фундаментальности.
Кроме того, $\hn{A - A_k} \ra 0$. Поэтому
\begin{multline*}
\hn{Ay_n-Ay_m} \le \hn{A y_n-A_k y_n} + \hn{A_k y_n - A_k y_m} + \hn{A_k y_m - A y_m} \le \\ \le
\hn{A-A_k}\cdot\hn{y_n} + \hn{A_k y_n - A_k y_m} + \hn{A-A_k}\cdot\hn{y_m}\le \\ \le
\hn{A-A_k}\cdot C + \hn{A_k y_n - A_k y_m} + \hn{A-A_k}\cdot C \ra 0,
\end{multline*}
а это и значит, что последовательность $\hc{A y_i}$ фундаментальна.
\end{proof}

\begin{lemma}
Пусть последовательность $\hc{x_n}$ в банаховом пространстве
слабо сходится к $x_0$ и предкомпактна. Тогда $x_n \ra x_0$ по норме пространства.
\end{lemma}
\begin{proof}
В силу предкомпактности из последовательности можно выделить фундаментальную подпоследовательность $x_{n_k}$.
В силу полноты пространства она сходится к некоторому вектору $\wh x$. Из сходимости
по норме следует слабая сходимость, поэтому $x_{n_k} \convw \wh x$.
Но слабый предел единствен, поэтому $\wh x = x_0$, что и требовалось.
\end{proof}

\begin{imp}
Компактный оператор переводит слабо сходящуюся последовательность в сходящуюся по норме.
\end{imp}
\begin{proof}
Как уже было доказано, слабо сходящаяся последовательность ограничена.
По определению компактного оператора, $\hc{A x_n}$ предкомпактно,
поэтому содержит сходящуюся к некоторой точке $y$ подпоследовательность. Очевидно, что $\hc{Ax_n}$
тоже слабо сходится, а поскольку слабый предел совпадает с сильным (если последний существует),
то и образ всей последовательности сходится к~$y$.
\end{proof}

\subsubsection{Свойства спектра компактных операторов}

\begin{lemma}
Собственные векторы с различными собственными значениями линейно независимы.
\end{lemma}
\begin{proof}
Докажем утверждение индукцией по количеству $k$ собственных векторов $e_1\sco e_k$ c собственными
значениями $\la_1\sco \la_k$ соответственно. При $k = 1$ доказывать нечего.
Пусть $k > 1$, и
$$e_1\spl e_{k-1} + e_k = 0,$$
тогда, применяя к этому равенству оператор, получаем
$$\la_1e_1\spl \la_{k-1}e_{k-1} + \la_k e_k = 0.$$
Вычтем отсюда исходное равенство, умноженное на $\la_k$, получим
$$(\la_1-\la_k)e_1\spl (\la_{k-1}-\la_k)e_{k-1} = 0.$$
По предположению индукции такое возможно только если
$e_i = 0$ при $i = 1\sco k-1$. Но тогда и $e_k = 0$.
\end{proof}

\begin{note}
Отрицание линейной независимости было сделано для линейной комбинации с единичными
коэффициентами. Но ясно, что это не существенно, так как вектора $y_i := a_i e_i$ также являются
собственными, и можно предполагать, что зависимы именно они (но уже с единичными коэффициентами).
\end{note}

\begin{theorem}
Пусть оператор $A\cln X \ra X$\т компактен, пространство $X$\т банахово. Тогда
количество собственных значений вне всякого круга радиуса $r>0$ с центром в
нуле лишь конечное число.
\end{theorem}
\begin{proof}
Пусть $\hc{\la_n}$\т попарно различные ненулевые собственные значения оператора $A$. Покажем, что
$\la_n\ra 0$. Допустим противное, тогда из $\hc{\la_n}$ можно выделить подпоследовательность так, что после перенумерации
последовательность $\bc{\frac1{\hm{\la_n}}}$ ограничена.
Рассмотрим цепочку подпространств $X_n := \ha{e_1\sco e_n}$, где $e_i$\т собственный вектор
с собственным значением $\la_i$. Тогда $e_1\sco e_n$ будут линейно независимыми,
следовательно, $\hc{X_n}$\т строго возрастающая цепочка.
В силу леммы о почти перпендикуляре, найдутся единичные векторы $x_n \in X_n$, для которых
$\rho\hr{x_n,X_{n-1}} > \frac12$. Разложим их по базису подпространств $X_n$:
пусть $x_n = \sumkun c_k e_k$. Как легко видеть, $\frac{A x_n}{\la_n}-x_n \in X_{n-1}$.
По предположению, последовательность $\bc{\frac{x_n}{\la_n}}$ ограничена. Подействуем на неё оператором~$A$
и увидим, что получается ёж. В самом деле, при $n < m$ имеем
$$v := \frac{Ax_n}{\la_n} - \frac{Ax_m}{\la_m} =
\ub{\frac{Ax_n}{\la_n}}_{\in X_{m-1}} - x_m + \ub{x_m - \frac{Ax_m}{\la_m}}_{\in X_{m-1}},$$
значит, $\hn{v}  = \hn{-x_m + \hr{\hbox{вектор из } X_{m-1}}} \ge \frac12$, а это
противоречит компактности оператора~$A$.
\end{proof}

\subsubsection{Теорема Гильберта\ч Шмидта}

В этом разделе $A$\т компактный самосопряжённый оператор в сепарабельном гильбертовом пространстве~$H$.
Введём обозначение $Q(x) := (Ax,x)$. Заметим, что это число всегда вещественно в силу самосопряжённости
оператора.

\begin{lemma}
Пусть $x_n \convw x$. Тогда $Q(x_n) \ra Q(x)$.
\end{lemma}
\begin{proof}
Имеем
\begin{multline*}
\hm{Q(x_n) - Q(x)} = \hm{(A x_n,x_n) - (Ax,x)} = \hm{(Ax_n, x_n) - (Ax,x_n) + (Ax, x_n) - (Ax,x)} \le\\ \le
\hm{(Ax_n, x_n) - (Ax, x_n)} + \hm{(Ax, x_n) - (Ax, x)} \stackrel{!}{=}
\hm{\br{A(x_n-x), x_n}} + \hm{\br{x, A(x_n-x)}} \le \\ \le \hn{A(x_n-x)}\cdot\hn{x_n} + \hn{x}\cdot\hn{A(x_n-x)} \ra 0.
\end{multline*}
Здесь равенство <<!>> следует из свойств самосопряжённого оператора, а сходимость к нулю вытекает из свойств
компактных операторов и ограниченности $\hn{x_n}$ (а это\т следствие слабой сходимости).
\end{proof}

\begin{lemma}
Если $\hm{Q(x)}$ достигает на единичной сфере своего максимума в точке $x_0$,
то для любого вектора $y$ такого, что $(x_0,y) = 0$, выполнено $(Ax_0, y) = 0$,
то есть $\ha{x_0}^\bot \subs \ha{Ax_0}^\bot$.
\end{lemma}
\begin{proof}
Рассмотрим вектор
$$v := \frac{x_0 + ay}{\hn{x_0 + ay}}, \quad a \in \Cbb.$$
Используя самосопряжённость оператора и теорему Пифагора для векторов $x_0$ и $y$, получаем
$$Q(v) = \frac{1}{1 + \hm{a}^2\cdot\hn{y}^2}\cdot\br{ Q(x_0) + 2\Rea\br{\ol a(Ax_0,y)} + \hm{a}^2Q(y)}.$$
Если $(Ax_0,y) \neq 0$, то выбирая $a$ малым по модулю и подкручивая его аргумент, можно сделать так,
что число $\Rea\br{\ol a(Ax_0,y)}$ будет ненулевым вещественным и будет иметь тот же знак, что и $Q(x_0)$.
Тогда $\hm{Q(v)} > \hm{Q(x_0)}$, а мы предположили, что $x_0$ максимизирует модуль $Q$. Полученное
противоречие показывает, что $(Ax_0,y) = 0$.
\end{proof}

\begin{imp}
Если $\hm{Q(x)}$ достиг максимума на векторе $x_0$, то это собственный вектор оператора~$A$.
\end{imp}
\begin{proof}
По лемме имеем $\ha{x_0}^\bot \subs \ha{Ax_0}^\bot$, поэтому $\br{\ha{x_0}^\bot}^\bot \sups
\br{\ha{Ax_0}^\bot}^\bot$, то есть $\ha{Ax_0} \subs \ha{x_0}$.
\end{proof}

\begin{theorem}[Гильберта\ч Шмидта]
Компактный самосопряжённый оператор $A$ в сепарабельном гильбертовом пространстве $H$
обладает базисом из собственных векторов.
\end{theorem}
\begin{proof}
Будем строить элементы базиса по индукции в порядке убывания модулей собственных значений.

Покажем, что на единичной сфере функция $\hm{Q(x)}$ достигает своего максимума. Пусть $S := \sup \hm{Q(x)}$,
а $x_n$\т последовательность единичных векторов, реализующая~$S$. Поскольку единичный шар
слабо предкомпактен, можно выбрать подпоследовательность $y_n \convw y$. При этом
в силу первой леммы получаем $\hm{Q(y_n)} \ra \hm{Q(y)}$, поэтому $\hm{Q(y)} = S$.

В качестве первого базисного вектора $e_1$ возьмём вектор $y$. Теперь рассмотрим $\ha{e_1}^\bot$.
Оно в силу самосопряжённости оператора инвариантно относительно~$A$. В нём
повторим эту же процедуру, найдём $e_2$, и так далее.
Если начиная с какого\д то момента мы получаем $Q(x) \equiv 0$, это означает, что ненулевые собственные
значения кончились, и мы попали в ядро оператора. Во противном случае получаем последовательность
ненулевых собственных значений $\hc{\la_n}$. Они, очевидно, сходятся к нулю. В самом деле, если
бы их модули были ограничены снизу, то образы единичных базисных векторов образовывали бы <<ежа>>,
а не предкомпактное множество.
\end{proof}

\subsubsection{Интегральные операторы Гильберта\ч Шмидта}

\begin{df}
Рассмотрим функцию $K(x,y) \in \Cb[a,b]^2$ и
оператор $A\cln L_1[a,b] \ra L_1[a,b]$, заданный так:
$$A(f) := \intl{a}{b}K(x,y)f(y)\,dy.$$
Этот оператор называется \emph{интегральным оператором Гильберта\ч Шмидта}.
Функция $K$ называется \emph{ядром} интегрального оператора. Через $A_K$ мы будем обозначать
оператор с ядром $K$.
\end{df}

\begin{problem}
Доказать, что $A_K^* = A_{\ol K}$.
\end{problem}

Для сокращения записи не будем писать пределы интегрирования по $[a,b]$.
Все нормы для функций понимаются в смысле тех пространств, где эти функции живут.

\begin{stm}
Интегральный оператор $A$ с ядром $K$ ограничен.
\end{stm}
\begin{proof}
Очевидным образом следует из ограниченности ядра $K(x,y)$.
\end{proof}

\begin{theorem}
Интегральный оператор Гильберта\ч Шмидта $A$ с ядром $K$ компактен.
\end{theorem}
\begin{proof}
Идея доказательства состоит в том, чтобы приблизить оператор по норме конечномерными операторами: как мы знаем,
конечномерный оператор компактен и предел компактных компактен. Разобьём квадрат сеткой $n\times n$, и вместо
функции $K(x,y)$ возьмём ступенчатую функцию $K_n(x,y)$, которая на каждом квадратике сетки равна значению функции $K$
в центре квадратика. Ясно, что при увеличении $n$ имеем $\hn{K_n-K}\ra 0$ и потому $\hn{A_{K_n}-A_K}\ra 0$.
Конечномерность образов операторов $K_n$ очевидна. Значит, оператор $A$ является пределом компактных
операторов и потому сам компактен.
\end{proof}

\subsection{Теория Фредгольма}

Три теоремы Фредгольма очень хорошо изложены в книге Колмогорова и  Фомина. Не вижу смысла
переписывать их сюда. Читайте гл. IX, \S 2, п. 4.

\begin{note}
В пятом издании этой книги на странице 469 (доказательство теорем Фредгольма, перед первой леммой) имеется опечатка.
Там написано <<... где $A$\т комплексный оператор...>> имелся в виду, конечно, компактный оператор.
\end{note}

В стёпинских лекциях (см. \dmvnwebsite) можно прочесть доказательство этих теорем для банаховых пространств.

\newpage
\scriptsize
\input{"Functional Calculus [5] - V.V. Ryzhikov.ind"}

\end{document}
