\documentclass[a4paper]{article}
\usepackage[utf8]{inputenc}
\usepackage[russian]{babel}
\usepackage[simple]{dmvn}

\def\apoint{\par\textbf{Указание.} }
\def\answer{\par\textbf{Ответ.} }

\title{Программа экзамена по функциональному анализу}
\author{Лектор Александр Яковлевич Хелемский}
\date{V семестр, 2006 г. + Архив 2001--2002 г.}

\begin{document}
\maketitle

\section*{2006 г.}

\subsection*{Программа экзамена}
\begin{nums}{-3}
\item Преднормированное и нормированное пространство. Примеры. Сопряжённо-билинейный функционал и полярное тождество.
\item Скалярное произведение. Почти-гильбертово пространство. Примеры. Неравенство Коши-Буняковского.
Гильбертова норма и непрерывность по ней скалярного произведения. Равенство параллелограмма.
Теорема фон Нойманна~--- Йордана (без док.).
\item Ортогональные векторы. Ортогональные и ортонормированные системы. Их свойства и их примеры.
Процесс ортогонализации. Определение функций и многочленов Эрмита.
\item Ряд Фурье в почти-гильбертовом пространстве. Предложение о ближайшем векторе в конечномерном подпространстве.
Неравенство Бесселя. Предложение о тотальных ортонормированных системах. Теорема о разложении в ряд Фурье.
Понятие о базисе Шаудера.
\item Ограниченный оператор между преднормированными пространствами. Пространство $\mathcal{B}(E,F)$ и операторная преднорма;
достаточное условие того, что это норма. Случай почти-гильбертовых пространств. Сопряжённое пространство.
Мультипликативное неравенство для операторной преднормы.
\item Некоторые классы операторов: сжимающие, изометрические, коизометрические. Топологические, изометрические,
унитарные изоморфизмы. Примеры операторов. Интегральный оператор в $L_2[a,b]$ и оценка его нормы.
\item Характеризация ограниченных операторов как непрерывных. Возможность продолжения операторов в линейной алгебре (без док.).
Пример Филлипса оператора, непродолжаемого с сохранением ограниченности (без док.).
\item Теорема Хана~--- Банаха; случай действительного поля скаляров и сепарабельного пространства.
\item Комплексная версия теоремы Хана~--- Банаха. Достаточность семейства ограниченных функционалов на нормированном
пространстве. Теорема Рисса об описании ограниченных функционалов на $\Cb[a,b]$ (без док.).
\item Банахово и гильбертово пространство. (Достаточное) условие того, когда $\mathcal{B}(E,F)$ банахово.
Суммируемые векторные ряды и <<признак Вейерштрасса>>. Принцип продолжения по непрерывности.
Случай изометрического изоморфизма.
\item Теорема Рисса~--- Фишера. Теорема Гауэрса (без док.). Предложение о ближайшем векторе в замкнутом подпространстве.
\item Ортогональное дополнение к подмножеству и его свойства. Теорема об ортогональном дополнении.
Условие тотальности системы в гильбертовом пространстве.
\item (Орто)проектор. Теорема Рисса об описании ограниченных функционалов на гильбертовом пространстве. (без док.).
Банахов сопряжённый оператор.
\item Гильбертов сопряжённый оператор. Соотношения сопряжённости. Свойства операции <<гильбертова звёздочка>>.
Связь между ядром оператора и образом сопряжённого. Алгебраическая характеризация проектора.
\item Теорема Банаха~--- Штейнхауса. Её следствие о раздельно и совместно непрерывных (билинейных) операторах.
Теорема Банаха об обратном операторе (без док.).
\item Пополнение нормированного пространства; два подхода к этому понятию и их эквивалентность. Примеры.
Теорема единственности пополнения.
\item Топология и топологическое пространство. Первые примеры. База и предбаза топологии.
Топология метрического пространства и метризуемость. Хаусдорфово пространство. Непрерывное отображение.
Тихоновское произведение топологических пространств.
\item Компактное топологическое пространство. <<Теорема Вейерштрасса>>. Теорема Александрова. Теорема Тихонова (без док.)
\item Сверхограниченное (вполне ограниченное) метрическое пространство. Эквивалентные условия сверхограниченности.
Эквивалентные условия компактности метрического пространства.
\item Сверхограниченность ограниченных множеств в $\Cbb^n_1$. Свойства конечномерных нормированных пространств
 (топологическая изоморфность нормированных пространств одной размерности, полнота и др.).
\item Лемма о почти перпендикуляре. Теорема Рисса о сверхограниченных единичных шарах. Теорема Арцела (без док.).
\item Компактный оператор. Пространство $\mathcal{K}(E,F)$ и его замкнутость в $\mathcal{B}(E,F)$.
Компактность интегрального оператора.
\item Свойство аппроксимации и его наличие у гильбертовых пространств. Понятие о примере Энфло (без док.).
\item Теорема Шмидта о строении компактных операторов между гильбертовыми пространствами.
\item Слабо унитарно эквивалентные операторы. Гильбертова сумма гильбертовых пространств.
Предложение о слабо унитарно эквивалентной модели компактного оператора.
\item Ядерный оператор в гильбертовом пространстве и его след.
\item Компактность оператора, сопряжённого к компактному. Коядро оператора и его связь с ортогональным дополнением к образу.
Случай конечномерного коядра. Фредгольмов оператор и его индекс. Мультипликативное свойство индекса (без док.).
\item Теорема о фредгольмовости компактного возмущения тождественного оператора.
\item Альтернатива Фредгольма.
\item Теорема об индексе компактного возмущения тождественного оператора. Тройная теорема Фредгольма.
\end{nums}

\subsection*{Задачи}

\begin{nums}{-1}
\item Любое метрическое пространство изометрически вкладывается в нормированное пространство,
причём если оно сепарабельно, то в качестве нормированного можно выбрать $\ell_\infty$.
\item Восстанавливается ли сопряжённо-билинейный функционал по своей квадратичной форме
в случае действительного поля скаляров?
\item Если норма гильбертова, то из $\hn{x+y} = \hn x + \hn y$ следует, что $x$ и $y$
пропорциональны.
\item Норма в $\Cb[a,b]$ и $L_1[a,b]$ не гильбертова.
\item Ближайших точек до вектора в подпространстве нормированного пространства может быть много.
\item Система Радемахера не тотальна.
\item Найти норму диагонального оператора в $\ell_p;\,p = 1,2,\infty$, оператора умножения
на функцию в $L_p[a,b];\,p = 1,2,\infty$ и оператора неопределённого интегрирования в
$\Cb[a,b]$ и $L_1[a,b]$.
\item Для диагонального оператора, операторов правого и левого сдвига в $\ell_p;\,p=1,2,\infty$,
оператора сдвига в $\ell_p(\mathbb{Z});\,p=1,2,\infty$ и оператора умножения на функцию узнать,
когда они являются изометрическими? коизометрическими? изометрическими изоморфизмами?
топологическими изоморфизмами?
\item Описать мономорфизмы и эпиморфизмы в категории нормированных пространств.
\item Найти общий вид ограниченных функционалов на $c_0,\ell_1,\ell_2$.
\item Когда заданный функционал на прямой в $\mathbb{R}^2_1$, $\mathbb{R}^2_2$ и $\mathbb{R}^2_\infty$
обладает единственным сохраняющим норму продолжением?
\item В метрическом пространстве измеримых функций на отрезке не существует ненулевых непрерывных
функционалов.
\item Пусть $E$~--- нормированное пространство, $E_0$~--- его замкнутое подпространство,
$x \in E \wo E_0$. Тогда существует ограниченный функционал на $E$, равный нулю на $E_0$ и
отличный от нуля на $x$.
\item Ближайших точек до вектора в подпространстве почти-гильбертова пространства может и не существовать.
\item Найти банаховы сопряжённые операторы к диагональному оператору, оператору левого сдвига и оператору
правого сдвига в $c_0$.
\item То же для $\ell_1$.
\item То же для $\ell_2$.
\item Банахов сопряжённый и гильбертов сопряжённый операторы к оператору $\la \mathbf{1}$ в $\ell_2$
не являются топологически эквивалентными. (?)
\item Найти гильбертов сопряжённый для диагонального оператора в $\ell_2$, операторов правого и левого сдвига в $\ell_2$,
оператора сдвига в $\ell_2(\mathbb{Z})$, оператора умножения на функцию в $L_2[a,b]$ и оператора неопределённого интегрирования
в $L_2[0,1]$.
\item Верно ли, что ортогональное дополнение к ядру оператора всегда есть образ его гильбертова сопряжённого?
\item Охарактеризовать в алгебраических терминах унитарный оператор.
\item Оператор в гильбертовом пространстве равен своему сопряжённому и своему обратному. Как он действует?
\item Как действует оператор $V$ такой, что $V^*V = 1$? $VV^* = 1$? $V^*VV^*V = V^*V$?
\item Верна ли теорема Банаха~--- Штейнхауса без предположения о полноте заданного пространства?
\item Привести пример раздельно, но не совместно ограниченного билинейного оператора между нормированными пространствами.
\item Верна ли теорема Банаха об обратном операторе без предположения о полноте обоих заданных пространств?
\item Если норма гильбертова, то такова же норма в пополнении.
\item Не существует метрики, задающей поточечную сходимость в $\Cb[a,b]$.
\item Указать топологию, задающую поточечную сходимость в $\Cb[a,b]$
\item Когда топология предметрического пространства хаусдорфова?
\item Если два непрерывных отображения в хаусдорфово пространство совпадают на плотном подмножестве, то они равны.
\item Сходимость последовательности в тихоновском произведении~--- это покоординатная сходимость.
\item Показать, не используя теорему Рисса, что единичный шар в $\ell_p;\,p=1,2,\infty$, $\Cb[a,b]$, $L_2[a,b]$ и
$\mathcal{B}(\ell_2)$ не компактен.
\item Если $f \in E^*$ таков, что верхняя грань в определении его нормы не достигается, то для любого $x \in E \wo \Ker(f)$
выполнено $\hn x \ge d(x,\Ker(f))$.
\item Привести пример равномерно ограниченного, но не равностепенно непрерывного семейства в $\Cb[a,b]$.
\item Охарактеризовать сверхограниченные множества в $\ell_2$ в терминах норм <<хвостов>>.
\item В классе сепарабельных гильбертовых пространств из предложения о модели следует теорема Шмидта.
\item (повышенной трудности) Интегральный оператор в $L_2[a,b]$ является оператором Шмидта.
\item (повышенной трудности) Каждый оператор Шмидта, действующий в сепарабельном гильбертовом пространстве, топологически
эквивалентен интегральному оператору в $L_2[a,b]$.
\item Привести пример фредгольмова оператора с заданным целым индексом.
\item Когда диагональный оператор, оператор умножения на функцию фредгольмовы?
\item Компактный оператор между гильбертовыми пространствами, одно из которых бесконечномерно, не фредгольмов.
\end{nums}

\medskip\dmvntrail

\newpage

\section*{Архив: 2001--2002 год}

\subsection*{V семестр}
\begin{nums}{-3}
\item Топологические пространства. База и предбаза. Метризуемые, хаусдорфовы и сепарабельные пространства.
Топологическое произведение.
\item (Пред)нормированные пространства. Примеры. (Пред)метрика, порожденная (пред)нормой.
\item Скалярные произведения и почти гильбертовы пространства. Неравенство Коши Буняковского. Закон параллелограмма.
Полярное тождество и характеризация гильбертовых норм (без доказательства).
\item Ортогональные векторы и ортонормированные системы. Примеры. Равенство Пифагора. Процесс ортогонализации.
\item Ряд Фурье по ортонормированной системе. Вектор в конечномерном подпространстве, ближайший к данному.
Неравенство Бесселя.
\item Тотальные ортонормированные системы и теорема о разложении в ряд Фурье по таким системам.
Базис Шаудера нормированного пространства и ортонормированный базис сепарабельного почти гильбертова пространства.
\item Ограниченные операторы и операторная (пред)норма. Мультипликативное неравенство. Интегральный оператор.
\item Эквивалентность непрерывности и ограниченности оператора. Топологический и изометрический изоморфизм.
Случай конечномерных пространств (без доказательства). Способы отождествления произвольных операторов и эндоморфизмов.
\item Вопрос о продолжении операторов и пример Филлипса (без доказательства). Теорема Хана~--- Банаха и её
доказательство для сепарабельного действительного пространства.
\item Теорема Хана~--- Банаха для комплексных пространств. Следствие о достаточности функционалов. Теорема
Рисса о функционалах на $\Cb[a,b]$.
\item Банаховы и гильбертовы пространства. Полнота пространства операторов. Признак сходимости рядов.
Принцип продолжения по непрерывности.
\item Теорема Рисса Фишера. Общая теорема о классификации гильбертовых пространств (без доказательства).
Плохие свойства общих банаховых пространств. Теорема Гауэрса и теорема Энфло Рида (без доказательства).
\item Вектор, ближайший к данному, в подпространствах гильбертовых пространств. Теорема об ортогональном дополнении.
\item Ортогональный проектор. Теорема Рисса о функционалах в гильбертовом пространстве.
\item Пример биективного ограниченного оператора без ограниченного обратного. Теорема Банаха об обратном
операторе (без доказательства). Теорема Банаха~--- Штейнхауса и существенность условия полноты.
\item Банахов сопряженный оператор и его простейшие свойства. Эквивалентные определения пополнения
нормированного пространства. Теорема единственности пополнения.
\item Теорема существования пополнения нормированного пространства. Пополнение почти гильбертова пространства.
\item Компактные топологические пространства. <<Теорема Вейерштрасса>> и теорема Александрова. Теорема Тихонова (без доказательства).
\item Сверхограниченные метрические пространства. Случай конечномерных нормированных пространств. Сво-\break йства
метрических пространств, эквивалентные сверхограниченности.
\item Свойства метрических пространств, эквивалентные компактности. Случай конечномерных нормированных пространств.
\item Лемма о почти перпендикуляре и теорема Рисса. Теорема Арцела (без доказательства).
\item Компактные операторы. Операции, сохраняющие компактность. Компактность интегрального оператора.
\end{nums}

\subsection*{VI семестр}

\begin{nums}{-3}
\item Теорема Шмидта о строении компактного оператора между гильбертовыми пространствами.
\item Теорема Аллахвердиева об $s$ числах (без доказательства). Операторы Шмидта и
ядерные операторы. След ядерного оператора.
\item Фредгольмовы операторы и их индексы. Первая теорема Фредгольма.
\item Альтернатива Фредгольма и третья теорема Фредгольма. Мультипликативное свойство
индекса и его устойчивость (без доказательства).
\item Полинормированные пространства: определение, примеры, топология, сходящиеся последовательности, условие хаусдорфовости.
Примеры. Топология полинормированного пространства. Выражение сходимости преобразований и хаусдорфовости на языке преднорм.
\item Полинормированные пространства: условие непрерывности оператора, пространство операторов, сопряженное пространство,
условие достаточности функционалов.  Выражение непрерывности оператора между полинормированными пространствами на языке преднорм.
Пространство операторов и сопряженное пространство в контексте полинормированных пространств. Связь между хаусдорфовостью
полинормированного пространства и условием достаточности функционалов.
\item Полинормированные пространства: слабая и $*$ слабая топологии, условия их хаусдорфовости, условие $*$ слабой
плотности подпространства сопряженного пространства.
\item Сопряженные операторы в контексте полинормированных пространств и их $*$-слабая непрерывность.
Сравнение трех топологий в пространстве, сопряженном к банахову. Теорема о банаховых сопряженных
операторах и теорема Банаха-Алаоглу (обе без доказательства).
\item  Три пространства пробных функций: состав элементов, преднормы, сходящиеся последовательности.
\item Плотность $\Dc$ и $\Sc$ в $\Ec$. Сравнение топологий этих трех пространств.
Непрерывность оператора дифференцирования в $\Dc$.
\item  Определение трех видов обобщенных функций. Критерий принадлежности
функционала к обобщенным функциям. Регулярные обобщенные функции.
Дельта-функция и ее сингулярность.
\item  Топология в пространствах обобщенных функций. Отождествление одних пространств пробных
и обобщенных функций с частью других. Плотность $\Dc$ в $\Dc^*$ и $\Sc$ в $\Sc^*$.
\item Оператор дифференцирования обобщенных функций; его существование и
единственность. Эквивалентность двух подходов к обобщенным функциям с
компактным носителем. Выражение последних через регулярные функции (без доказательства).
\item Спектр оператора. Точечный, непрерывный и остаточный спектр. Спектр компактного оператора.
\item Абстрактные алгебры и спектры их элементов. Многочлен от элемента
алгебры и его спектр.
\item Банаховы алгебры. Топологические свойства группы их обратимых элементов.
Следствие о спектрах. Спектр изометрического изоморфизма.
\item Резольвентная функция. Теорема о непустоте спектра. Формула спектрального радиуса (без доказательства).
\item Экспонента элемента банаховой алгебры и обратный к ней элемент. Связь
между спектром экспоненты и <<экспонентой спектра>>.
\item Гильбертов сопряженный оператор. Соотношения сопряженности. Свойства операции
<<$*$>>. Оператор, сопряженный к интегральному.
\item Взаимосвязь между ядрами и образами исходного и сопряженного оператора.
Сохранение операцией <<$*$>> конечномерности и компактности.
Алгебраическая характеризация унитарного оператора. $C^*$ тождество.
\item Самосопряженные операторы и их простейшие свойства. Алгебраическая
характеризация проектора. Местоположение спектра самосопряженного оператора.
\item Теорема Гильберта Шмидта. Реализация произвольного самосопряженного оператора
как оператора ум-\break ножения на функцию (без доказательства).
\item Классическое преобразование Фурье и его общие свойства. Теорема единственности (без доказательства).
Взаимозамена, посредством преобразования Фурье, свойств гладкости и быстроты убывания.
\item Свертка интегрируемых функций и действие на нее преобразования Фурье.
\item Преобразование Фурье в пространстве $\Sc$. Лемма об операторах,
перестановочных с умножением на независимую переменную.
\item Лемма об операторах, перестановочных с дифференцированием. Теорема обращения преобразования Фурье
в $\Sc$. Оператор Фурье в $\Sc^*$: существование и единственность. Преобразование Фурье дельта-функции.
\item Теорема Планшереля. Гильбертов оператор Фурье и его спектр.
\end{nums}

\subsection*{Задачи}

\subsubsection*{V семестр}

\begin{nums}{-1}
\item Поточечная сходимость в $\Cb[a,b]$ не может быть задана никакой метрикой.
\def\trap[#1,#2][#3,#4]{\mathbf{F}_{[#1,#2]}^{[#3,#4]}}

\begin{solution}
$$\epsfbox{pictures.1}$$

Предположим, что существует метрика $\rho$, задающая поточечную сходимость в $\Cb[0,1]$. Пусть
функция $\trap[a,b][c,d]\in \Cb[0,1]$~--- трапецевидная функция с нижним основанием $[a,b]$ и
верхним основанием $[c,d]$ (см. рисунок). Докажем следующее утверждение:

\begin{lemma}
Для любого $\ep>0$ и любого отрезка $[a,a']\subset[0,1]$ существует трапеция $f=\trap[a,b][c,d]$,
$a<b<a'$, такая, что $\rho(f,0)<\ep$.
\end{lemma}
\begin{proof}
Рассмотрим монотонно убывающую последовательность чисел $x_n$,  $a<x_n<a'$, так чтобы
$x_n\rightarrow a$. На каждом отрезке $[a,x_n]$  построим трапецию $\mathbf{F}_n$ с нижним
основанием $[a,x_n]$ (и верхним основанием $[c_n,d_n]$, которое можно взять любым таким,  чтобы
$a<c_n<d_n<x_n$). Последовательность $\mathbf{F}_n$ стремится поточечно к нулю. (В самом деле, если
$x\in[a,a']$, то существует такой   номер $N$, что $\forall n>N\> a<x_n<x$ и потому
$\mathbf{F}_n(x)=0\>\forall n>N$). Значит, последовательность $F_n$ сходится по метрике (по нашему
предположению) и $\exists N\>\rho(\mathbf{F}_N,0)<\ep$. Обозначив $b=x_N$, $c=c_N$, $d=d_N$,
получим, что для $f=\trap[a,b][c,d]=\mathbf{F}$ выполнено $\rho(f,0)<\ep$, причем $a<b<a'$.
\end{proof}

Докажем теперь утверждение задачи. Пусть $f_1=\trap[a_1,b_1][c_1,d_1]$~--- произвольная
трапецевидная функция. Тогда возьмем трапецию с основанием $[a_2,b_2']$, такую, чтобы
$c_1<a_2<b_2'<d_1$. По лемме существует функция $f_2=\trap[a_2,b_2][c_2,d_2]$, такая, что
$a_2<b_2<b_2'$ и $\rho(f_2,0)<\frac{\textstyle\rho(f_1,0)}{\textstyle2}$. Пусть построена функция
$f_{n-1}=\trap[a_{n-1},b_{n-1}][c_{n-1},d_{n-1}]$. Возьмем трапецию с основанием $[a_n,b_n']$,
такую, чтобы $c_{n-1}<a_n<b_n'<d_{n-1}$. По лемме существует функция $f_n=\trap[a_n,b_n][c_n,d_n]$,
такая, что $a_n<b_n<b_n'$ и $\rho(f_n,0)<\frac{\textstyle\rho(f_{n-1},0)}{\textstyle2}$.

Последовательность $f_n$ сходится к нулю по метрике, значит, она должна сходиться к нулю поточечно.
Но $[c_n,d_n]$~--- последовательность вложенных отрезков, которые имеют имеют общую точку $x$,
причем $\forall n\>f_n(x)=1$, поэтому $\liml{n\rightarrow\infty} f_n(x)=1$, а не $0$~---
противоречие.
\end{solution}

\item Открытые подмножества метрического пространства образуют топологию.
\item Любая совокупность подмножеств есть предбаза некой топологии.
\item Если непрерывные отображения в хаусдорфово пространство совпадают на плотном множестве, то они равны.
\item Сходимость последовательности в тихоновской топологии есть поточечная сходимость.
\item Топологическое произведение счетного числа двоеточий гомеоморфно  канторову множеству.
\item Сходимость почти всюду на множестве измеримых функций на отрезке не задается никакой топологией.
\item Норма~--- непрерывная функция на нормированном пространстве.
\apoint Доказать, что если $x_n\rightarrow x$, то $\hn{x_n}\rightarrow \hn x$.
\item Пространства $\Cb[0,1]$ и $L_2[0,1]$~--- сепарабельны, а $L_\infty[0,1]$~--- нет.
\item Непрерывное и равномерно непрерывное отображение метрических пространств~--- это разные вещи.
\apoint $f\cln\R\ra\R_+$, $f(x)=x^2$, непрерывно, но не равномерно непрерывно. Или
$\tg\cln(-\frac\pi2,\frac\pi2)\ra \R$.
\item Скалярное произведение~--- непрерывная функция на топологическом квадрате почти гильбертова пространства.
\apoint Доказать, что если $x_n\rightarrow x$, $y_n\rightarrow y$, то $\langle
x_n,y_n\rangle\rightarrow\langle x,y\rangle$.
\item Если норма гильбертова, то $\hn{x+y}=\hn x +\hn y$ влечет $x = \la y$.
\item Векторов в подпространствах нормированных пространств, ближайших к данному, может быть много.
\apoint В $\ell_\infty$ к последовательности $(1,0,0,0,\dots)$ в подпространстве
$\{(0,a_2,a_3,a_3,\dots)\in \ell_\infty\}$ любой вектор с $|a_i|\leqslant1$, $i=2,3,4,\dots$,
является ближайшим.
\item Норма диагонального оператора в $\ell_2$.
\answer Норма оператора $T=\mathop{\mathrm{diag}}(\lambda_1,\lambda_2,\dots)$ равна
$\hn{T}=\supl{i} |\lambda_i|^2$.
\item Норма оператора неопределенного интегрирования в $\Cb[0,1]$ и $L_1[0,1]$.
\answer $\hn{T_{\int}}_{C[0,1]}=\hn{T_{\int}}_{L_1[0,1]}=1$.
\item Слабая топологическая эквивалентность и топологическая эквивалентность операторов~--- это разные вещи.
\item Общий вид ограниченных функционалов в $c_0$, $\ell_1$ и $\ell_2$.
\item Пример, когда продолжений функционала, сохраняющих норму, много, и пример, когда такое продолжение всего одно.
\item Если $E_0$~--- замкнутое подпространство в нормированном пространстве $E$, и $x\in E\wo E_0$, то существует
$f\in E^*$, такой, что $f=0$ на $E_0$ и $f(x)\ne0$.
\item Не существует ни одного ненулевого непрерывного линейного
функционала на пространстве измеримых функций на [0,1] с метрикой
$$d(f,g)\mathrel{:=}\intl{0}{1}\frac{|f(t)-g(t)|}{1+|f(t)-g(t)|}\,dt.$$
\item Пространства $\Cb[a,b]$, $L_2[a,b]$ и $L_\infty[a,b]$~--- банаховы, а пространство непрерывных
функций на отрезке с интегральной нормой~--- не банахово.
\item Нормированное пространство со счетным алгебраическим базисом никогда не банахово.
\item Для нормированных пространств $E$ и $F$ пространство $\mathfrak{B}(E,F)$ банахово $\iff$ $F$ банахово.
\item Принцип продолжения по непрерывности для равномерно непрерывных отображений метрических пространств.
\item В замкнутых подпространствах почти гильбертовых пространств может и не быть вектора, ближайшего к данному.
\item В замкнутых подпространствах банаховых пространств может и не быть вектора, ближайшего к данному.
\item Пример изометрического изоморфизма между $L_2(X,\mu)^*$ и $L_2(X,\mu)$.
\item В теореме Банаха об обратном операторе нельзя отбросить требование полноты ни первого, ни второго
из заданных пространств. \apoint Рассмотреть тождественный оператор
$T\cln(\ell_1,\hn{\cdot}_1)\ra(\ell_1,\hn{\cdot}_\infty)$. Он ограничен, а обратный к нему~--- нет.
Чтобы обосновать необходимость полноты первого пространства, использовать симметричность теоремы.
\item Если линейное пространство сделано банаховым относительно двух норм,
и первая мажорирует вторую, то и вторая мажорирует первую. \apoint Рассмотреть тождественный
оператор из $(E,\hn{\cdot})$ в $(E,\hn{\cdot}')$, доказать его ограниченность и применить теорему
Банаха об обратном операторе.
\item В теореме Банаха Штейнгауза требование полноты первого из заданных пространств нельзя отбросить.
\apoint Рассмотреть семейство операторов $f_n\cln(c_{00},\hn{\cdot}_\infty)\ra\Cbb$,
$f_n\br{(a_1,a_2,\dots,a_n,\dots)}=na_n$.
\item Норма сопряженного оператора равна норме исходного.
\item Оператор, сопряженный к изометрическому, коизометрический.
\item Сопряженные операторы к диагональному оператору и операторам сдвига в пространствах, сопряженных
к $c_0$, $\ell_1$ и $\ell_2$.
\item Свойство универсальности пополнения относительно произвольных ограниченных операторов.
\item Эквивалентность двух определений компактного подмножества.
\item Эквивалентные определения компактности в терминах центрированных множеств.
\item Компактность~--- топологическое свойство.
\item В теореме Александрова нельзя отбросить ни требование компактности
первого, ни требование хаусдорфовости второго из заданных пространств.
\item Эквивалентность двух определений сверхограниченного множества.
\item Замыкание сверхограниченного множества само сверхограниченно.
\apoint $\ep$-сеть множества есть $2\ep$-сеть его замыкания в силу неравенства треугольника.
\item Алгебраическая сумма и растяжение сверхограниченных множеств сверхограниченны.
\item Единичный шар во всех предлагавшихся примерах бесконечномерных нормированных пространств
не сверх\-ограничен.
\item Единичный шар в $C_1[a,b]$ как подмножество в $\Cb[a,b]$ сверхограничен, но не компактен.
\item Подмножество в единичном шаре $C[a,b]$, состоящее из таких $f$, что
$|f(s)-f(t)|\leqslant|s-t|$, $s,t\in[a,b]$, компактно.
\item Критерий компактности подмножества в $\ell_2$.
\item Критерий компактности диагонального оператора в $\ell_2$.
\item Оператор умножения на функцию в $L_2(X,\mu)$, не равную нулю почти всюду, не компактен.
\item Оператор неопределенного интегрирования компактен.
\item Оператор дифференцирования из $\Cb^1[a,b]$ в $\Cb[a,b]$ не компактен.
\end{nums}

\subsubsection*{VI семестр}

\begin{nums}{-1}
\item Оператор, слабо топологически эквивалентный компактному, сам компактен.
\item Интегральный оператор является оператором Шмидта.

\item Оператор, слабо топологически эквивалентный фредгольмову, сам фредгольмов.
\item Компактный оператор между банаховыми пространствами с бесконечномерной областью
значений никогда не фредгольмов.

\item Когда диагональный оператор в $\ell_2$ фредгольмов?
\item Когда оператор умножения на непрерывную функцию в $L_2[a,b]$ фредгольмов?

\item Указать фредгольмов оператор заданного индекса.
\item Задано интегральное уравнение второго рода (1) в $L_2[a,b]$
и соответствующее однородное уравнение (2). Тогда:

\begin{items}{-1}
\item существует набор $x_1\sco x_m$ линейно независимых решений
уравнения (2), такой, что всякое решение этого уравнения есть линейная комбинация указанных
решений;
\item существует набор $z_1\sco z_n$ линейно независимых функций в $L_2[a,b]$,
такой, что правые части уравнения (1), для которых это уравнение разрешимо -- это в точности те
$y$, для которых
$$\intl{a}{b} y(t)\ol{z_k(t)}\,dt=0, \quad k=1\sco n.$$
\item $m=n$.
\end{items}
\item Размерность ядра и коразмерность образа фредгольмова оператора не устойчивы
при малых возмущениях.

\item Классическая сходимость в $\Cb^\infty[a,b]$ не может быть задана
преднормой.
\item Покоординатная сходимость в $c_\infty$ не может быть задана преднормой.

\item Эквивалентные системы преднорм задают одну и ту же топологию.
\item Как по семейству преднорм судить о том, что пространство нормируемо?

\item  Пространство, сопряженное к бесконечномерному хаусдорфову полинормированному
пространству, бесконечномерно.
\item Любая слабая$^*$ окрестность нуля в пространстве, сопряженном к
бесконечномерному хаусдорфову полинормированному пространству, содержит бесконечномерное
подпространство.

\item Из ограниченной по норме последовательности функционалов на сепарабельном нормированном
пространстве можно выделить слабо$^*$ сходящуюся подпоследовательность.
\item Для любых возрастающих $a,b,c,d\in \R$ существует <<шляпа>> $\ph \in\Dc$:
$0\le\ph\le 1$, и $\ph=1$ внутри $[b,c]$ и $\ph=0$ вне $[a,d]$.

\item Оператор дифференцирования в $\Sc$ и в $\Ec$ непрерывен.
\item Оператор умножения на гладкую функцию в $\Dc$ и в $\Ec$ и оператор умножения на
умеренно растущую гладкую функцию в $\Sc$ непрерывен.
\item Если функционал на $\Ec$ непрерывен, то он непрерывен по некоторой стандартной норме.
\item $\Dc$ плотно в $\Sc$.
\item Функционал $\ph\mapsto\ints{\R}\frac{\ph(t)}{t}dt$ это сингулярная обобщенная функция порядка $1$.
\item Функционал $\ph\mapsto\suml{k=1}{\infty}\ph^{(k)}(k)$ это сингулярная обобщенная
функция бесконечного порядка.
\item Какова обобщенная функция с нулевой производной?
\item Каждая обобщенная функция обладает первообразной.
\item Существование и единственность оператора дифференцирования в $\Sc^*$.
\item Существование и единственность оператора умножения на гладкую функцию в $\Dc^*$.
\item Существование и единственность оператора умножения на умеренно растущую гладкую функцию в $\Sc^*$.
\item Спектры, а также точечные, непрерывные и остаточные спектры топологически эквивалентных операторов совпадают.
\item Если $\la$ точка непрерывного спектра оператора $T:E\to E$, то существует
последовательность $x_n\in E;\|x_n\|=1$, такая, что $Tx_n-\la x_n\to0; n\to\infty$.
\item Каков спектр (и его выделенные части) проектора?
\item Каков спектр (и его выделенные части) диагонального оператора в $\ell_2$?
\item Каков спектр (и его выделенные части) оператора левого сдвига в $\ell_2$?
\item Каков спектр (и его выделенные части) оператора правого сдвига в $\ell_2$?
\item Каков спектр (и его выделенные части) оператора сдвига в $\ell_2(\Z)$?
\item Каков спектр (и его выделенные части) оператора умножения на непрерывную
функцию в $L_2[a,b]$?
\item Любое непустое подмножество в $\Cbb$ может служить спектром элемента (чистой) алгебры.

\item Указать элемент (чистой) алгебры с пустым спектром.
\item Спектр сопряженного оператора <<комплексно-сопряжен>> к спектру исходного.

\item Если $\la$ точка остаточного спектра оператора, то $\overline{\la}$ точка точечного
спектра его сопряженного.
\item Если $\la$т точка точечного спектра оператора, то $\overline{\la}$ точка точечного или
остаточного спектра его сопряженного.
\item Если $\la$ точка непрерывного спектра оператора, то $\overline{\la}$ точка
непрерывного спектра его сопряженного.
\item Каков оператор, сопряженный к диагональному оператору в $\ell_2$?
\item Каков оператор, сопряженный к оператору левого сдвига в $\ell_2$?
\item Каков оператор, сопряженный к оператору правого сдвига в $\ell_2$?

\item Каков оператор, сопряженный к оператору умножения на непрерывную функцию в $L_2[a,b]$?
\item Каков оператор, сопряженный к оператору неопределенного интегрирования в $L_2[0,1]$?
\item Задано интегральное уравнение второго рода (1) в $L_2[a,b]$, его сопряженное
уравнение (1$^*$) и соответствующие однородные уравнения (2) и (2$^*$). Пусть, далее, $x_1\sco x_m$
и $z_1\sco z_n$ функции, о которых говорится в задаче 8. Тогда:
\begin{items}{-1}
\item $z_1\sco z_n$ образуют максимальную линейно независимую систему  решений уравнения (2$^*$)
\item правые части уравнения (1$^*$), для которых это уравнение разрешимо это в точности те $y$, для которых
$$\intl{a}{b} y(t)\ol{x_k(t)}\,dt=0,\quad k=1\sco m.$$
\end{items}
\item Спектр самосопряженного оператора может быть любым подмножеством $\R$.
\item Спектр унитарного оператора может быть любым подмножеством $S^1$ (единичная окружность).
\item Спектр нормального оператора может быть любым подмножеством ${\Cbb}$.
\item У самосопряженного оператора нет остаточного спектра.
\item Теорема Гильберта Шмидта для сепарабельного бесконечномерного пространства эквивалентна следующему утверждению:
всякий компактный самосопряженный оператор унитарно эквивалентен оператору умножения на сходящуюся
к нулю последовательность действительных чисел в $\ell_2$.
\item Квадраты $s$ чисел компактного оператора $T$ суть собственные числа оператора $T^*T$.
\item Преобразование Фурье оставляет на месте функцию $e^{-\frac{t^2}{2}}$.
\item $T_a(\ph*\psi)=(T_a\ph)*\psi=\ph*(T_a\psi)$, где  $T_a$ сдвиг на $a$.
\item Объясните фразу: <<свертка горба со ступенькой это шляпа>>.
\item Пусть функции $г_n$ таковы, что $г_n=0$ вне $\hs{-\frac{1}{n},\frac{1}{n}}$ и $\ints{\R}г_n(t)\,dt=0$.
Тогда $г_n*\ph$ сходится в $L_1(\R)$ к $\ph$ для любой $\ph$.
\item Каково преобразование Фурье заданной линейной комбинации производных дельта-функции?
\item Каково преобразование Фурье заданного многочлена?
\item Каково преобразование Фурье обобщенной функции $\ph\mapsto\ph(a);a\in\R$ (<<сдвига дельта-функции>>)?
\end{nums}


{\it Следующие задачи могут предлагаться студентам, претендующим на оценку <<5>>:}

\begin{nums}{-1}
\item Если образ оператора между банаховыми пространствами имеет конечную коразмерность, то он замкнут.

\item Вейерштрассова сходимость в пространстве голоморфных функций в круге не может быть задана преднормой.
\item Как устроены сходящиеся последовательности в сильнейшем полинормированном пространстве?

\item Как по семейству преднорм судить о том, что пространство метризуемо?
\item Слабо сходящаяся последовательность в $\ell_1$ сходится и по норме.

\item Если кратная производная обобщенной функции $f$ равна нулю на $(a,b)$,
то $f$ равна на $(a,b)$ некоторому многочлену.
\item Каков спектр (и его выделенные части) оператора сдвига в $L_2(\R)$?

\item Каков спектр (и его выделенные части) оператора неопределенного
интегрирования в $L_2[0,1]$?
\item В банаховой алгебре нет $a,b$, таких, что $ab-ba=i\hbar$ {\bf 1}.

\item Какова норма оператора неопределенного интегрирования в $L_2[0,1]$?
\item Пусть $\ph$ и $b>0$ таковы, что $\ph(t)e^{b|t|}\in L_1(\R)$. Тогда
$F(\ph)$ может быть продолжена до функции, аналитической в полосе $\{z: -b<\Img z<b\}$.

\item Функции Эрмита образуют ортонормированный базис в $L_1(\R)$.
\item Гильбертов оператор Фурье унитарно эквивалентен оператору умножения
на последовательность $\la_n\bw{:=}(-i)^n$ в $\ell_2$.
\end{nums}


\end{document}
