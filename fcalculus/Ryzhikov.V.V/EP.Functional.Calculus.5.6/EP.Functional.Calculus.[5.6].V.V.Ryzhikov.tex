\documentclass[a4paper]{article}
\usepackage[russian]{babel}
\usepackage[utf8]{inputenc}
\usepackage[simple]{dmvn}

\title{Программа экзамена по функциональному анализу}
\author{Лектор-- В.\,В.\,Рыжиков}
\date{V--VI семестр, 2004--2005 г.}

\begin{document}
\maketitle

\section*{V семестр}

\begin{nums}{-2}
\item Полные метрические пространства. Теорема Бэра о вложенных шарах. Теорема о категориях.
\item Существование непрерывной функции на отрезке $[0,1]$, не имеющей конечной производной
      ни в одной точке отрезка.
\item Теорема о пополнении метрического пространства (без доказательства).
\item Теорема о неподвижной точке сжимающего отображения.
\item Примеры банаховых пространств. Полнота пространств $\Cb[0,1]$ и $\ell_1$.
\item Эквивалентность непрерывности и ограниченности операторов (функционалов) в
нормированных пространствах. Понятие базиса Гамеля и доказательство его
существования. Доказательство существования неограниченных операторов в
бесконечномерном банаховом пространстве.
\item Теорема Банаха-- Штейнгауза о равномерной ограниченности (принцип равномерной
ограниченности).
\item Геометрический смысл линейного функционала. Общий вид линейного функционала
в гильбертовом пространстве. Полнота сопряжённого пространства.
\item Общий вид непрерывного линейного функционала в пространстве $L_1[0,1]$.
Несепарабельность пространства $L_{\infty}^*[0,1]$.
\item Теорема Хана-- Банаха о продолжении линейного функционала (вещественный и
комплексный вариант) и следствия из неё.
\item Ограниченность слабо ($*$-- слабо) сходящейся последовательности в банаховом
пространстве. Изометрическое вложение $L$ в $L^{**}$.
\item Слабая-- $*$ компактность единичного шара в пространстве,
сопряжённом к сепарабельному пространству.
\item Теорема Банаха об ограниченности обратного оператора. Достаточное условие
необратимости оператора: $\hn{x_i} = 1$, $\hn{Ax_i} \ra 0$.
\item Устойчивость обратимости ограниченного оператора.
\item Свойства спектра ограниченного оператора в банаховом пространстве (ограниченность,
замкнутость, непустота).
\item Спектр оператора умножения на ограниченную измеримую функцию (в $L_p[0,1]$).
\item Свойства компактных операторов.
\item Компактность интегрального оператора в $L_1[0,1]$ с непрерывным ядром.
\item Некомпактные операторы. Лемма о почти перпендикуляре.
\item Эквивалентность компактности операторов $А$, $А^*$, $А^*А$.
\item Теорема Фредгольма.
\item Ограниченность сопряжённого оператора. Существование оператора, сопряжённого
ограниченному, равенство их норм.
\item Теорема Гильберта-- Шмидта.
\end{nums}

\pagebreak

\section*{VI семестр}

\begin{nums}{-2}
\item Спектральный радиус. Формула вычисления. Теорема об отображении спектра для полиномов.
\item Следствие из теоремы Фредгольма о спектре компактного оператора. Теорема Ломоносова
об инвариантном подпространстве для компактного оператора в бесконечномерном пространстве.
\item Унитарные и самосопряжённые операторы. Их спектр.
\item Разложение сепарабельного гильбертова пространства в сумму ортогональных циклических пространств
для унитарного оператора.
\item Спектральная теорема для унитарного оператора с циклическим вектором.
Представление унитарного оператора в виде оператора умножения
$V f(z) = z f(z)$ в~$L_2(\T,\si)$ и $V f(z,n) = z f(z,n)$  в~$L_2(\T\times\N, \si)$.
\item Преобразование Кэли. Эквивалентность самосопряжённого оператора некоторому оператору
умножения на вещественную функцию.
\item Теорема об отображении спектра для аналитической функции (в односвязной области).
\item Классическое преобразование Фурье, его свойства.
\item Инъективность преобразования Фурье.
\item Преобразование Фурье в пространстве Шварца $\Sc$. Непрерывность преобразования Фурье в $\Sc$.
\item Преобразование Фурье в $L_2(\R)$. Теорема Планшереля.
\item Полнота системы Эрмита в $L_2(\R)$.
\item Теорема Пэли-- Винера.
\item Формула суммирования Пуассона.
\item Пространства $\Dc_N$, $\Dc(\R)$, $\Sc$ и непрерывные функционалы над ними.
\item Общий вид непрерывного функционала на $\Dc_N$.
\item Решение уравнений $\La'=0$ и $\La'=F$ в $\Dc'(\R)$.
\item Общий вид непрерывного функционала, носитель которого есть точка.
\item Общий вид непрерывного функционала на $\Sc$.
\item Общий вид непрерывного функционала из $\Dc'(\R)$ с компактным носителем.
\end{nums}

\begin{note}
Вопросы 8 и 10 не очень полно освещались в лекциях. В случае недостатка
материала лекций предлагается использовать книгу Колмогорова -- Фомина.
\end{note}

\medskip\dmvntrail
\end{document}


%% Local Variables:
%% eval: (setq compile-command (concat "latex  -halt-on-error -file-line-error " (buffer-name)))
%% End:
