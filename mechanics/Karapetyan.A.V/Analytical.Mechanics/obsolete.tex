%%%%%%%%%%%%%%%%%%%%%%%%%%%%%%%%%%%%%%%%%%%%%%%%%%%%%%%%%%%%%
%% Малые колебания и все с ними связанное. Почему-то нет в программе.
%%%%%%%%%%%%%%%%%%%%%%%%%%%%%%%%%%%%%%%%%%%%%%%%%%%%%%%%%%%%%
\subsubsection{Принцип виртуальных перемещений}

В этом разделе мы будем рассматривать голономные
системы, то есть системы со связями вида
\equ{f_{\al}(\vb{r}_1,\ldots,\vb{r}_n)=0,\ \al=1\sco a,} причем
связи предполагаются идеальными. Мы уже выяснили, что движения таких
систем происходят в некотором конфигурационном пространстве, на
котором можно ввести обобщенные координаты $q^1\sco q^m,\ m=3n-a.$
Силы, действующие на систему, будут предполагаться зависящими только
от координат и потенциальными (таким образом, рассматриваемые системы будут ещё и консервативными): \equ{\vb{F}_i=\vb{F}_i(\vb{r}_1\sco\vb{r}_n).}
\begin{df}
Положение системы $\vb{r}_i=\vb{r}^0_i$ называется \tdf{положением
равновесия}, если при $\vb{r}_i(t_0)=\vb{r}_i^0,\ \vd{r}_i(t_0)=0$
имеет место равенство $\vb{r}_i(t)=\vb{r}_i^0$ при любом $t$.
\end{df}
Вспоминая принцип д'Аламбера--Лагранжа \eqref{DL}, запишем
\eql{\suml{i=1}{n}\spr{\vb{F}_i(\vb{r}^0)}{\de\vb{r}_i}=0.}{pvm} Это
и есть \tdf{принцип виртуальных перемещений: работа сил в положении
равновесия равна нулю на виртуальных перемещениях.} Выпишем теперь
некоторые свойства, характерные для описанной только что системы.

Вспоминая представление виртуальных перемещений через обобщенные
координаты \eqref{virt_gen} и определение обобщенных сил, можно
записать \equ{Q_j(q_0)\de q^j=0\Ra Q_j(q_0)=0.} Если обобщенные силы
потенциальны, то есть \equ{Q_j(q)=-\pf{V(q)}{q^j},} получим,
во-первых, \equ{\de V(q_0)=Q_j(q_0)\de q^j=0,} а также
\equ{\pf{V}{q^j}\evu{q=q_0}{12pt}{1.4pc}=-Q_j(q_0)=0.}
%\begin{nbb}
%Далее всюду в этом разделе мы, по-видимому, предполагаем
%обобщенные силы потенциальными и пользуемся тем, что уравнения
%Лагранжа \eqref{L2} имеют в этом случае вид
%\equ{\frac{d}{dt}\pf{T}{\dot{q}^j}-\pf{T}{q^j}=-\pf{V}{q^j}.}
%\end{nbb}
Так как система предполагается консервативной,
справедливо следующее представление для кинетической энергии:
\equ{T=T_2+T_1+T_0=T_2+0+0=\frac{1}{2}a_{jk}(q)\dot{q}^j\dot{q}^k.}
И последнее: при рассмотрении системы, имеющей в $q^0$ положение
равновесия, целесообразно сделать замену $q^j=q^j_0+x^j,\ j=1\sco
m.$

\begin{nbb}
Все результаты, полученные в этом параграфе, будут
использоваться в последующих параграфах (до конца текущего раздела)
без прямого на то указания, дабы не загромождать текст обилием
ссылок на тривиальные факты.
\end{nbb}

\subsubsection{Малые колебания}

Будем исследовать характер поведения системы вблизи положения
равновесия. В терминах описанной выше замены это означает, что мы
будем предполагать величины $x$ и $\dot{x}$ малыми. Ввиду этого
факта, выражения для кинетической и потенциальной энергии можно
переписать в следующем виде:
\equ{T=\frac{1}{2}a_{jk}(q_0+x)\dot{x}^j\dot{x}^k=\frac{1}{2}a_{jk}(q_0)\dot{x}^j\dot{x}^k+\o(x^2+\dot{x}^2).}
\equ{V=V(q_0)+\underbrace{\hr{\pf{V}{q^j}}\evu{q=q_0}{12pt}{1.4pc}}_{=0}\cdot
x^j+\frac{1}{2}\hr{\frac{\pd^2 V}{\pd q^j \pd
q^k}}\evu{q=q_0}{12pt}{1.4pc}\cdot x^j x^k+\o(x^2).}
\begin{note}
Чтобы получить представление для потенциальной энергии, мы просто
воспользовались разложением $V(x)$ в ряд Тейлора в точке $x=0$. А
вот запись для кинетической энергии содержит небольшую неточность:
размерности величин $x$ и $\dot{x}$, вообще говоря, различны,
поэтому не вполне понятно, что такое $\o(x^2+\dot{x}^2)$. Дабы
упростить наши рассуждения, остановимся на том, чтобы все-таки
считать эти величины приведенными к одной размерности.
\end{note}
Теперь перепишем полученные выражения в матричной форме. Для этого
введем матрицы \equ{A=(a_{jk}^0)=(a_{jk}(q_0)),\
C=(c_{jk}^0)=\frac{1}{2}\hr{\frac{\pd^2 V}{\pd q^j \pd
q^k}}\evu{q=q_0}{12pt}{1.4pc}.} Сразу отметим, что обе они
симметричны, а матрица $A$ еще и положительно определена
(\tdf{утверждение \ref{kin_en_rep}}). Получим
\equ{T\approx\frac{1}{2}a_{jk}^0\dot{x}^j\dot{x}^k,\
V\approx\frac{1}{2}c_{jk}^0x^jx^k.}

\begin{nbb}
Здесь мы как-то мимоходом потеряли константное слагаемое
$V(q_0)$ в выражении для $V$. Впрочем, потенциальная энергия все
равно определена с точностью до константы, так что не жалко.
\end{nbb}
Подставляя полученные выражения в уравнения Лагранжа \eqref{L2},
получим их <<линеаризацию>> вблизи положения равновесия
\equ{a_{jk}^0\ddot{x}^k+c_{jk}^0x^k=0,} или, в совсем уж матричной
форме \eql{A\ddot{x}+Cx=0.}{L2lin}

В силу симметричности матриц $A$ и $C$ найдется невырожденной
ортогональное преобразование координат $x=B\xi$, такое что в новых
координатах уравнения примут вид \equ{\ddot{\xi}+K\xi=0,} где
$K=\diag(\kappa_1,\sco\kappa_m),$ а величины $\kappa_j$ являются
корнями уравнения $\det(A\kappa-C)=0.$ При этом координаты $\xi$
называют \tdf{нормальными}.
\begin{note}
Если все $\kappa_j$ положительны, то уравнения \eqref{L2lin}
действительно описывают <<малые колебания>>. В противном случае
некоторые решения будут иметь экспоненциальный рост и ни о каких
колебаниях речи уже быть не может. Это, впрочем, не мешает
уравнениям \eqref{L2lin} и в этом случае тоже называться <<уравнениями
малых колебаний>>.
\end{note}
Теперь рассмотрим пример, демонстрирующий только что описанную
технику.

\subsubsection{Пример: двойной математический маятник}


Потенциал изображенной на~\emph{рис.~5}\xtrpicturep{5}{5}
системы имеет вид
\equ{V=-m_1gl_1\cos{\ph_1}-m_2g(l_1\cos{\ph_1}+l_2\cos{\ph_2}).} Пользуясь тем, что
\equ{\pf{V}{\ph_1}\evu{\ph_1=\ph_2=0}{12pt}{1.4pc}=\pf{V}{\ph_2}\evu{\ph_1=\ph_2=0}{12pt}{1.4pc}=0,}
запишем приближенное значение потенциала вблизи положения равновесия
\equ{V\approx\frac{1}{2}\hr{(m_1+m_2)gl_1\ph_1^2+m_2gl_2\ph_2^2}.} Теперь посчитаем кинетическую
энергию. Ясно, что $T=\frac{1}{2}\hr{m_1v_1^2+m_2v_2^2}$, а значит, необходимо найти скорости точек
$m_1$ и $m_2$. Чтобы упростить себе жизнь, скорость второй точки сразу будем искать приближенно.
Опуская тривиальные выкладки, запишем окончательный результат \equ{v_1^2=l_1^2\dot{\ph}_1^2,\
v_2^2\approx\hr{l_1\dot{\ph}_1+l_2\dot{\ph}_2}^2,} откуда
\equ{T\approx\frac{1}{2}m_1l_1^2\dot{\ph}_1^2+\frac{1}{2} m_2\hr{l_1\dot{\ph}_1+l_2\dot{\ph}_2}^2.}

Для простоты будем полагать единицы измерения и свойства
рассматриваемой системы такими, чтобы выполнялись равенства
$m_1=m_2=1, l_1=l_2=1, g=1$. В этом случае выражения для $T$ и $V$
станут короче:
\equ{T\approx\frac{1}{2}\hr{2\dot{\ph}_1^2+2\dot{\ph}_1\dot{\ph}_2+\dot{\ph}_2^2},\
V\approx\frac{1}{2}\hr{2\ph_1^2+\ph^2_2}.} Теперь уже ясно, что
матрицы $A$ и $C$, введенные в предыдущем параграфе, в данном случае
имеют вид \equ{A=\rbmat{2&1\\1&1},\ C=\rbmat{2&0\\0&1}.} Решая
систему \equ{\det(A\kappa-C)=\mbmat{2\kappa-2&\kappa\\ \kappa&\kappa-1}=0,}
находим $\kappa_{1,2}=2\pm\sqrt{2}$ --- квадраты частот малых
колебаний.

%%%%%%%%%%%%%%%%%%%%%%%%%%%%%%%%%%%%%%%%%%%%%%%%%%%%%%%%%%%%%
%% Пример для уравнений Раусса. В программе ныне отсутствует.
%%%%%%%%%%%%%%%%%%%%%%%%%%%%%%%%%%%%%%%%%%%%%%%%%%%%%%%%%%%%%
\subsubsection{Пример: волчок Лагранжа в углах Крылова}

\xtrpicture{6}{6}
Вначале определим \tdf{углы Крылова} (\emph{рис. 6}). Обозначать их
будем просто $\al,\ \be,\ \ga$. Они задаются следующим образом:
поворот системы $Oxyz$ вокруг оси $x$ на угол $\al$ должен
переводить ее в $Ox_1y_1z_1$, такую, что линия пересечения
плоскостей $O\xi\eta$ и $Ox_1y_1$ совпадает с $Oy_1$; далее, поворот
$Ox_1y_1z_1$ вокруг $Oy_1$ на угол $\be$ переведет ее в
$Ox_2y_2z_2$, в которой $Oz_2=O\zeta$; и, наконец, поворот вокруг
$Oz_2=\zeta$ на угол $\ga$ окончательно совместит системы. В этих
углах скорость подвижного (с закрепленным началом) репера
$Oe_{\xi}e_{\eta}e_{\zeta}$ имеет вид
\equ{\om=\dot{\al}\vb{e}_x+\dot{\be}\vb{e}_{y_1}+\dot{\ga}\vb{e}_{\zeta}=p\vb{e}_{\xi}+q\vb{e}_{\eta}+r\vb{e}_{\zeta}.}
Выразим коэффициенты $p,\ q,\ r$ через углы Крылова:
\equ{\begin{cases}
        p=\dot{\al}\cos\be\cos\ga+\dot{\be}\sin\ga,\\
        q=-\dot{\al}\cos\be\sin\ga+\dot{\be}\cos\ga,\\
        r=\dot{\al}\sin\be+\dot{\ga}.
     \end{cases}
}

В этих обозначениях кинетическая и потенциальная энергия волчка
Лагранжа будут иметь следующий вид:
\equ{T=\frac{1}{2}A\hr{\dot{\al}^2\cos^2\be+\dot{\be}^2}+\frac{1}{2}\hr{\dot{\al}\sin\be+\dot{\ga}}^2,\
V=mg\cos\al\cos\be.} Теперь уже легко видеть, что $\ga$
циклическая переменная, а циклический интеграл имеет вид
\equ{\pf{L}{\ga}=0\Ra\pf{L}{\dot{\ga}}=\pf{T}{\dot{\ga}}=C\hr{\dot{\al}\sin\be+\dot{\ga}}=C\om=\const.}
Отсюда, пользуясь тем, что $\dot{\ga}=\dot{\om}-\dot{\al}\sin\be$,
получим
\equ{R=T-V-C\om\dot{\ga}=R(\al,\be,\dot{\al},\dot{\be},\om).}
Слагаемые стандартного разложения $R=R_2+R_1+R_0$ будут выглядеть
так: \equ{R_2=\frac{1}{2}(\dot{\al}^2\cos^2\be+\dot{\be}^2),\
R_1=C\om\dot{\al}\sin\be,\ R_0=-mg\cos\al\cos\be.} Положение
равновесия приведенной системы, таким образом, будет при
\equ{\pf{R_0}{\al}=\pf{R_0}{\be}=0,\ \al=\be=0.} Стационарные
движения при этом будут такими: \equ{\al=\be=\dot{\al}=\dot{\be}=0,\
\dot{\ga}=\om.} Отбрасывая члены старше второго порядка, вблизи
положения равновесия приведенной системы получим:
\equ{R_2\approx\frac{1}{2}A(\dot{\al}^2+\dot{\be}^2),\ R_1\approx
C\om\dot{\al}\be,\ R_0=\frac{1}{2}mg_S(\al^2+\be^2).} Можно,
наконец, написать уравнения малых колебаний: \equ{\begin{cases}
         A\ddot{\al}+C\om\be-mg_S\al=0, \\
         A\ddot{\be}-C\om\dot{\al}-mg_S\be=0.
       \end{cases}
}
Лучше переписать эту систему в комплексной форме, вводя угол $\ph=\al+i\be$:
\equ{A\ddot{\ph}-C\om i\dot{\ph}-mg_S\ph=0.} Характеристическое уравнение этого диффура будет
таким: \equ{A\la^2-C\om i\la -mg_S=0,} а его решения будут выглядеть так:
\equ{\la_{1,2}=\frac{C\om i\pm \sqrt{-C^2\om^2+4Amg_S}}{2A}.}
\begin{imp}[условие Майевского]
Если $C^2\om^2-4Amg_S<0$, то уравнения малых колебаний имеют
экспоненциально растущие решения.
\end{imp}
%%%%%%%%%%%%%%%%%%%%%%%%%%%%%%%%%%%%%%%%%%%%%%%%%%%%%%%%%%%%%
%% Ещё один пример, которого нет в программе.
%%%%%%%%%%%%%%%%%%%%%%%%%%%%%%%%%%%%%%%%%%%%%%%%%%%%%%%%%%%%%
\subsection{Пример: волчок Эйлера}

\begin{nbb}
Напомню, что волчок Эйлера это твердое тело с неподвижным центром масс в однородном поле
тяжести. Через $A,\ B,\ C$ мы обозначаем главные моменты инерции (предполагая при этом, что все они
попарно не равны друг другу), а через $p,\ q,\ r$ компоненты угловой скорости.
\end{nbb}
Уравнения движения волчка Эйлера в этих обозначениях записываются так: \equ{ \begin{cases}
    A\dot{p}+(C-B)qr=0,\\
    B\dot{q}+(A-C)rp=0,\\
    C\dot{r}+(B-A)pq=0.
  \end{cases} } А первые интегралы будут такими: \equ{
\begin{array}{l}
2H=Ap^2+Bq^2+Cr^2=2h,\\
K^2=A^2p^2+B^2q^2+C^2r^2=k^2.
\end{array}
} Положение равновесие у этой системы такое: $p=q=0,\ r=\om\neq0$; мы будем рассматривать близкие к
нему решения \equ{p=x,\ q=y,\ r=\om+z.} Перепишем для этого случая уравнения и первые интегралы:
\equ{ \begin{cases}
    A\dot{x}+(C-B)y(\om+z)=0,\\
    B\dot{y}+(A-C)x(\om+z)=0,\\
    C\dot{z}+(B-A)xy=0,\\
2H=Ax^2+By^2+2C\om z+Cz^2=2h,\\
K^2=A^2x^2+B^2y^2++2C^2\om z+C^2z^2=k^2.
\end{cases}}
Введем теперь функцию Ляпунова (в двух вариантах):
\equ{V_{\pm}=\frac{1}{\om^2}(2H)^2\pm(K^2-2CH)=4C^2z^2+\ldots\pm(A(A-C)x^2+B(B-C)y^2)+\ldots}
члены старше второго порядка здесь сознательно опущены ввиду их незначимости вблизи положения
равновесия. Условия теоремы Ляпунова будут выполнены, если $A>C$ и $B>C$ (в качестве функции
Ляпунова берем $V_{+}>a(\hn{p;q;r})$), либо если $A<C$ и $B<C$ (берем $V_{-}$), следовательно, в
этих случаях положение равновесия волчка Эйлера будет устойчивым.

Осталось рассмотреть случай, когда $(A-C)(B-C)<0$. Чтобы показать, что в данном случае применима
теорема Четаева (а значит, имеет место неустойчивость), достаточно взять в качестве функции Четаева
функцию $V=xy$.

%%%%%%%%%%%%%%%%%%%%%%%%%%%%%%%%%%%%%%%%%%%%%%%%%%%%%%%%%%%%%
%% Еще один ненужный пример.
%%%%%%%%%%%%%%%%%%%%%%%%%%%%%%%%%%%%%%%%%%%%%%%%%%%%%%%%%%%%%


\subsection{Пример: система полудисков}

Пусть два полудиска массы и радиуса, соответственно, $m_1,\ r_1$ и
$m_2,\ r_2$, расположены так, как показано на \tpic{рис.~9}.
\xtrpicturep{9}{9}
Положению равновесия этой системы, очевидно, соответствует равенство
нулю углов $\ph_1$ и $\ph_2$. Выясним, когда это положение
равновесия будет устойчивым.

Положения центров масс $S_i$ обоих дисков находятся из формул
$\frac{O_iS_i}{r_i}=k=1-\frac{2}{\pi}.$ C учетом этого факта, несложно выписать потенциал этой
системы:
\ml{V=m_1gr_1(1-k\cos\ph_1)+m_2g(r_1+r_2((1-k\cos\ph_2)\cos\ph_1+(k\sin\ph_2-\ph_2)\sin\ph_1))=\\
=m_2gr_2(p(1-\cos\ph_1)+(1-k\cos\ph_2)\cos\ph_1+(k\sin\ph_2-\ph_2)\sin\ph_2)+m_1gr_1,} где
$p=\frac{m_1r_1}{m_2r_2}$, а последнее слагаемое можно не учитывать, так как оно является
константой; кроме того, для упрощения расчетов, будем предполагать, что единицы измерения выбраны
так, что $m_2gr_2=1.$ Чтобы воспользоваться теоремой 2.8 нам необходимо вычислить матрицу вторых
производных в точке равновесия. Найдем вначале первые:
\equ{V_1'=\pf{V}{\ph_1}=pk\sin\ph_1-(1-k\cos\ph_2)\sin\ph_1+(k\sin\ph_2-\ph_2)\cos\ph_1,}
\equ{V_2'=\pf{V}{\ph_2}=k\sin\ph_2\cos\ph_1+(k\cos\ph_2-1)\sin\ph_1.} Заметим, кстати, что
$\ph_1=\ph_2=0$ действительно является решением уравнения $V_1'=V_2'=0$. Вторые производные будем
сразу выписывать в этой точке равновесия: \equ{V_{11}''\evu{0}{12pt}{1pc}=pk+(k-1);\
V_{12}''\evu{0}{12pt}{1pc}=k-1;\ V_{22}''\evu{0}{12pt}{1pc}=k.} Нетрудно убедиться, что
определитель матрицы вторых производных в этой точке будет равен $pk^2+k-1$, откуда, в силу теоремы
\ref{th_lyap_3}, немедленно получаем следующее условие на устойчивость:
\equ{p>p_0=\frac{1-k}{k^2}=\frac{2\pi}{(\pi-2)^2}.} Соответственно, при $p<p_0$ имеет место
неустойчивость.
