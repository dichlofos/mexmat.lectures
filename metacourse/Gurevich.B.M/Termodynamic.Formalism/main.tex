\documentclass[10pt]{article}
\usepackage{dmvn}
\usepackage{polyglossia}
\usepackage{unicode-math}
\usepackage{fontspec}
\usepackage{makeidx}
\usepackage{amsmath}
\usepackage{wrapfig}
\usepackage{stackrel}
\defaultfontfeatures{Scale=MatchLowercase, Mapping=tex-text}
\setmainfont{CMU Serif}
\setsansfont{CMU Sans Serif}
\setmonofont{CMU Typewriter Text}
\setmathfont{Asana-Math.otf} %% I had to make decision. I dit it.
\newcommand{\type}[2]{\ensuremath{\colon\!#1→#2}}
\newcommand{\inline}[1]{{\setmainfont{URW Chancery L} (#1)}}
\newcommand{\inv}[1]{{\frac{1}{#1}}}


\begin{document}
\dmvntitle{Курс естествнно-научного содержания}
{«Термодинамический формализм»}
{Лектор -- Гуревич Борис Маркович}
{IV(V) курс, 7(9) семестр, поток математиков, ЕНС}
{Москва, 2012г}
\section*{Dead letter mail}

%% Структуризация оставляет желать лучшего. Пока реализовано на глазок
%% наборщика. И, ещё. Я даю себе отчет, что Asana-Math выглядит
%% непривычно, но этот шрифт поддерживает юникод значительно лучше.
%% Надеюсь, вам не придется переверстывать это под другой матшрифт,
%% ибо местами я вставлял пробелы руками из-за специфики шрифта.

Для удобства изучения материал разбит в соотвествии с пунктами программы.


\section*{Программа экзамена}
\subsection*{\inline{Осень 2012}}
\begin{enumerate}
\item Распределение Максвелла-Больцмана-Гиббса на конечном множестве и
  обосновывающий его вариационный принцип.
\item Двумерная модель Изинга. Определение и свойства контуров. Неравенство Пайерлса.
\item Теорема о равномерной по $V$ сходимости к нулю при $β→∞$
  вероятности $P_{V,+}^{(β)}(ω(0,0) = -1)$ того, что в плоском
  ферромагнетике Изинга с обратной температурой $β$ спин в начале
  координат $(0,0)$ равен $-1$ при граничном условии $+1$ вне квадрата
  $V$ с центром в $(0,0)$. Вывод отсюда существования при низкой
  температуре по крайней мере двух предельных распределений Гиббса.
\item Основные состояния. Утверждение о стремлении меры
  $P^{(β)}_{V,\bar ω}$ к равномерному распределению на множестве
  основных состояний. Примеры основных состояний для ферромагнетика
  Изинга при различных граничных условиях. Модель Изинга с внешним полем.
\item Одномерная решеточная модель с конечнозначным спином и
  взаимодействием ближайших соседей. \par Существования предела
  $\liml{n→∞}λ^{-n}B^n$, где $B$ -- квадратная матрица с
  положительными элементами и $λ$ -- её максимальное собственное
  значение (вывод из эргодической теоремы для конечных цепей
  Маркова). Существование единственного предельного распределения
  Гиббса, не зависящего от граничных условий.
\item Общая модель с конечнозначным спином на решентке
  $\Z^d$. Банаховы пространства трансляционно инвариантных потенциалов
  с сильной и слабой нормами. Взаимодействие конечного радиуса: оценки
  статсуммы с пустыми граничными условиями, отвечающей объединию двух
  конечных множеств, через статсуммы, отвечающие каждому из множеств и
  характеристике потенциала.
\item Существование давления как предела нормированного логарифма
  статсуммы по последовательности параллелепипедов с растущими
  сторонами (случай взаимодействия конечного радиуса).
\item Свойства давления как функции потенциала конечного радиуса:
  оценка константы Липшица.  Продолжение давления по непрерывности на
  множество потенциалов с конечной сильной нормой. Давление как предел
  по последовательности параллелепипедов в классе потенциалов с
  конечной нормой.
\item Статсумма при наличнии внешних условий: существование
  отвечающего ей давления и его независимость от этих условий.
\item Давление как выпуклая функция на пространстве потенциалов.
\item Свойства энтропии верояностной меры на пространстве конфигураций
  $Ω_V$, $V⊂Z^d$, $\hm V < ∞$. Существование для всякой инвариантной
  относительно группы сдвигов вероятностной меры на $Ω$ предела при
  $n→∞$ нормированной энтропии её проекции на $Ω_{V_n}$, где
  $\hc{V_n}$ -- последовательность Фельнера.
\item Свойства функции $f_Φ(ω)$ (вклада начала координат в энергию
  конфигурации $ω∈Ω$). Вариационны принцип для давления.
\item Лемма о прореженной последовательности Фельмера. Лемма
  Линденштраусса о случайном выборе элементов последовательности
  Фельмера.
\item Теорема о существовании для всякой последовательности Фельмера
  предела, определяющего давление P (см. вопрос 7).
\end{enumerate}

\begin{note}
  Материал, касающйся вопросов 6 -- 12, можно найти в книге Д. Рюэль
  «Термодинамический формализм», 2002. Однако, обозначения в этой
  книге в большинстве случаев отличаются от использованных в
  лекциях.
\end{note}

%% Local Variables:
%% compile-command: "xelatex -halt-on-error -file-line-error main.tex"
%% End:

\section{Распределение Гиббса на конечном множестве.}
\newcommand{\sumx}{\ensuremath{\suml{x∈X}{}}}
Рассмотрим систему с конечным множеством состояний $X=\hc{x_i}$. Для
произвольного состояния $x∈X$ полагаем определенными следующие функции
\begin{nums}{-2}
\item $E(x)$ -- энергия системы в состоянии $x$. $E\type X\R$
\item $P(x)$  -- вероятность состояния, $\sumx P(x) = 1$
\end{nums}
\begin{df}
  Постулатом Гиббса называется следующее соотношение
  \begin{equation}
    \label{dist:gibs}
    \begin{array}{cc}
      P(x) = ce^{-βE(x)} & c = \frac 1{\sumx e^{-βE(x)}}
    \end{array}
  \end{equation}
  \indent  Константа $c$ называется \emph{статистической суммой}. В англоязычной
  литературе -- `partition function'.
\end{df}
\newcommand{\intX}{\intl{X}{}}
Приведем некое обоснование постулата Гиббса. Рассмотрим следующую функцию от распределения
\begin{denote}
$    φ(\vec P) ≝ -\sumx P(x)\ln P\,(x) - \sumx P(x)E(x) = -\intX \ln P\, dP -\intX EdP $
\end{denote}
\begin{denotes}
  Выражение $(-\intX \ln P dP)$ называется энтропией, $(\intX EdP\, )$ -- средней энергией.
\end{denotes}
\begin{lemma}
  Распределение Гиббса~(\ref{dist:gibs}) является стационарной точкой функции $φ(\vec P)$.
\end{lemma}
\begin{proof}
 Введем обозначения $P(x_i) = p_i$, $E(x_i) = u_i$. Множество всех
 допустимых значений $p_i$ образует симплекс
 \begin{equation*}
   \hc{(p_1,…,p_n) : p_i≥0, \sumiun p_i = 1}
 \end{equation*}
Найдем стационарные точки функции

$φ(p_1,…,p_n) = -\sumiun p_i\ln p_i - \sumiun p_iu_i$,

где $u_i$ полагаем фиксированными. В соответствии с теорией
вариационного исчисления, вводим функцию
\begin{equation*}
  \begin{array}{lr}
    Φ = φ + λs &\text{где } s(p_1,…,p_n) = \sumiun p_i
  \end{array}
\end{equation*}
Тогда $\pf Φ{p_i} = -\ln p_i -1 -u_i +λ$, откуда $p_i = ce^{-u_i}$,
\begin{equation*}
  c = \frac 1{\sumiun e^{u_i}} = \frac 1{\sumiun e^{-βu_i^0}}
\end{equation*}
где параметр $β$ вводится заменой $u_i = βu_i^0$.
\end{proof}

Таким образом, мы показали, что распределение Гиббса является
стационарной точкой функции $φ(\bar p)$. Для того, что бы доказать,
что это будет максимум, покажем, что $φ$ -- выпуклая вниз(вогнутая).

\begin{wrapfigure}{l}{0.15\textwidth}
    \includegraphics[width=0.15\textwidth]{image1.eps}
\end{wrapfigure}
Действительно, рассмотрим функцию $f(y) =-y\ln y$, $y∈[0,1]$,
доопределенную в нуле по непрерывности (график слева).

\begin{df}
  Линейная комбинация $ap+bp'$, где $a,b≥0$, $a+b = 1$ называется
  выпуклой комбинацией.
\end{df}

\begin{df}
  Функция $g$ -- выпуклая, если $ g(ap+bp') ≤ag(p)+bg(p') $
\end{df}

Отсюда получаем, что распределение Гиббса действительно доставляет
максимум функции $φ$, выпуклой как линейной комбинации выпуклых
функций и с линейным ограничением на $\bar p$.

%% Local Variables:
%% compile-command: "xelatex -halt-on-error -file-line-error main.tex"
%% End:

\section{Двумерная модель Изинга}

Рассмотрим множество $T=\Z²$ -- целочисленную решетку. Также
рассмотрим подмножество $V=[-N, N]²∩T$, называемое \emph{сосудом}.

Функция $ω\cln V→S$ называется \emph{конфигурацией}, а $S$ --
\emph{спиновым пространством}. В дальнейшем будем полагать, что
$S=\hc{-1, 1}$.

Аналогично определим конфигурацию на всей решетке
\begin{equation*}
  \tilde ω\cln T→S
\end{equation*}
\begin{denote}
  $\tilde ω_{V} = \tilde ω\evn{V}$
\end{denote}

\begin{denote}
  $Ω_T=Ω$ -- множество всех конфигураций на $T$.
\end{denote}
\begin{denote}
  $Ω_V$ -- множество всех конфигураций на $V$.
\end{denote}

\begin{stm}
  Пусть $V_1⊂V_2$. Тогда, если $ω∈Ω_{V_2}$, то $ω\evn{V_1} ∈Ω_{V_1}$.
\end{stm}
\begin{denote}
  Определим функцию $U\cln S×S→S$, $U(s,s') = ss'$.
\end{denote}

\begin{df}
  Определим функцию энергии
  \begin{equation*}
    E(ω) =  -J \suml{\scriptstyle\mat{t,t'∈V\\ \hm{t-t'}=1}}{}U(ω(t),ω(t'))
  \end{equation*}
  \indent и вероятности
  $$ P(ω) = \frac{e^{-E(ω)}}{Z_V} $$
  \indent Величина $Z_V$ называется \emph{статистической суммой}.
  \end{df}
\subsection{Внешние условия}

Рассмотрим конфигурацию $\tilde ω$ на всей целочисленной сетке:
\begin{equation}
        \tilde ω(t) = \bcase{ω(t) \quad&t∈V\\ \bar ω(t)\quad &t∈T\wo V}
\end{equation}
\begin{denote}
        Функция $\bar ω$ называется внешними(граничными) условиями.
\end{denote}

 Для комбинации конфигурации в сосуде и внешних условий определим энергию
 \begin{equation*}
        E(ω,\bar ω) = -J\suml{\mat{t∈V\\t'∈T\\\hm{t-t'} = 1}}{}U(\tilde ω(t),\tilde ω(t'))
 \end{equation*}
и вероятность
\begin{equation*}
        P^β_{V,\bar ω}(ω) = \frac{e^{-βE(ω,\bar ω)}}{Z_{V,\bar ω}(β)}
\end{equation*}

\begin{denote}
        Если параметр $J=1$,  то конструкция называется ферромагнетиком Изинга,
        если $J= - 1$, то антиферромагнетиком.
\end{denote}

\newcommand{\Pp}{P^{(β)}_{V,+}}
\newcommand{\Pm}{P^{(β)}_{V,-}}
\begin{denote}
  Распределение, соответствующее $\bar ω ≡ +1$, обозначим $\Pp$.
\end{denote}
\begin{denote}
    $ \hat T = T + (½,½) $
\end{denote}

  Для каждой точки $t$ конфигурации $ω$, в которой $ω(t) = -1$, построим
  $K_t∈\hat T$ -- квадрат единичной стороны с центром в точке $t$.

  \begin{df}
    $V^-(ω) ≝\bigcup\limits_{t\cln ω(t) = -1} K_t$
  \end{df}
  \begin{df}
    $∂ω ≝ ∂V^-(ω)$
  \end{df}

  \begin{df}
    Контур($γ$) -- связная компонента границы $∂ω$
  \end{df}
  \begin{petit}
    Здесь должен быть комментарий про то, что угловые точки не считаются
    связностью, или, что-то там про скругленные углы. Ну нет картинок, увы…
  \end{petit}

  \begin{lemma}
    \begin{petit}
      Все леммы очевидно показываются на картинках. К сожалению, у меня
    нет времени на это.
    \end{petit}
    \begin{enumerate}
    \item Ребро принадлежит границе $∂ω$ тогда и только тогда, когда по
      разные стороны лежат точки $t_1,t_2∈ T$  с разными знаками.
    \item Каждый контур $γ$ -- несамопересекающаяся замкнутая кривая(ломаная).
    \item Из каждой точки, принадлежащей контуру, выходит ровно два ребра контура.
    \item Пусть $ω((0,0)) = -1$. Тогда существует такой контур  $γ$, что $(0,0) ∈ \Int(γ)$
    \item Граница $ ∂ω$ определяет конфигурацию $ω$.
    \end{enumerate}
  \end{lemma}

  \begin{denote}
   $Ω^+(γ) = \hc{ω∈Ω_V\vert \text{ $γ$ -- контур в $∂ω$ }} $
  \end{denote}
  \begin{lemma}[Неравенство Пайерлса.]
    Пусть $γ$  -- замкнутая простая ломаная, принадлежащая $\hat T$.
    Тогда $\Pp(Ω^+(γ)) ≤e^{-2β\hm{γ}}$.
  \end{lemma}
  \begin{proof}
    \begin{df}
      Преобразование стирания контура $φ_γ\cln Ω_V→Ω_V$ -- меняет знаки внутри $γ$.
    \end{df}
    \begin{note}
      Очевидно, $φ_γ² = Id$.
    \end{note}
    На картинке можно показать, что $E(ω,\bar ω) - E(φ_φω,\bar ω) =
    -2\hm γ$.  Отсюда, огрубляя,  $\Pp(ω)/\Pp(φ_γω) ≤ e^{-2β\hm γ}$, из чего
    следует утверждение леммы.
  \end{proof}
\begin{theorem}[Пайерлс, Рудольф(1907 -- 1995)]
  Равномерно по сосуду (по $N$) выполнено
  \begin{equation*}
    \liml{β→∞} \Pp(ω(0,0) = -1) = 0
  \end{equation*}
\end{theorem}
\begin{proof}
\newcommand{\Goo}{Γ_0}
\begin{denote}
  Если замкнутый контур $γ$ охватывает точку $(0,0)$, то обозначаем
  $γ∈\Goo$
\end{denote}
\newcommand{\sumgoo}{\suml{γ∈\Goo}{}}
\begin{equation*}
  P(ω(0,0) = -1 ) ≤ \sumgoo P(∂ω⊂γ) ≤
  \sumnui\suml{\tiny\mat{γ∈\Goo\\\hm{γ} = n}}{} P(∂ω⊃γ) ≤
  \suml{n=4}{\infty}\hc{γ\vert \hm{γ} = n, γ∈\Goo} e^{-2βn} ≤
  \suml{n=4}{\infty}4n^24^ne^{-2βn}
\end{equation*}
Последний переход обосновывается тем, что контур задается начальной
точкой ($4n^2$) и последовательностью из $n$ перемещений в одном
из четырех направлений.

Начав с нижней точки пересечения с $OY$ можно получить оценку $n3^{n-1}$.
Таким образом, если $\ln 3 - 2β < 0$, то ряд сходится. При
этом, при достаточно больших $β$ сумма ряда стремится к $0$.
\end{proof}
\newcommand{\convarg}[1]{\stackrel{#1}{\longrightarrow}}
\begin{note}
  Эта теорема дает «двухфазное регулирование», т.е
  \begin{equation*}
    \Pp(ω(0,0) = -1) \convarg{β→+∞} 0\qquad
    \Pm(ω(0,0) = 1) \convarg{β→-∞} 0
  \end{equation*}
  Получаем, что значение в $(0,0)$ зависит от граничных условий,
  которые находятся сколь угодно далеко.
\end{note}



%% Local Variables:
%% compile-command: "xelatex -halt-on-error -file-line-error main.tex"
%% End:

\section{Основные состояния}
Посмотрим, что происходит при различных $β$.
\begin{df}
 Спонтанная намагниченность
 \begin{equation*}
   μ(β) ≝\uliml{N→+∞} P^{(β)}_{V_N,+}(ω(0,0) = +1)
 \end{equation*}
\end{df}

\begin{problem}
  Найти наименьшее $β_0$, что для всех $β>β_0$ выполнено $μ(β) > ½$.
\end{problem}
\begin{answer}
  Нобелевский лауреат Ларс Анзайгер нашел решение. Доказательство
  довольно сложное с использованием алгебры. Есть альтернативное
  доказательство Ландау-Фавыченко (возможно с ошибкой). В курсе
  решение задачи не рассматривается.
\end{answer}

\newcommand{\Po}{P^{(β)}_{V,\bar ω}}

Зафиксируем сосуд $V$ и
граничные условия $\bar ω$ и устремим $β→+∞$.  Тогда вероятностная
мера $\Po$ будет сходиться к равномерному распределению на
конфигурациях, имеющих минимум энергии (обозначаются $Ω^{\min}$).

\begin{df}
  $Ω^{\min}$ -- точки основного состояния (`ground state').
\end{df}
\begin{petit}
  Возможно, это и не так. Проверяйте.
\end{petit}
\begin{note}
  Точки основного состояния для $\bar ω ≡ ±1$ -- конфигурации $ω ≡ ±1$.
\end{note}
\begin{problem}
  Найти $Ω^{\min}$ для квадрата, у которого противоположные стороны
  имеют граничные условия -- константы разного знака.
\end{problem}

\begin{petit}
  Далее идет `best efford`.
\end{petit}

Можно рассмотреть альтернативную модель, в которой есть так называемое
«внешнее поле» и энергия определяется следующим образом
\begin{equation*}
  E(ω,\bar ω, h) = E(ω,\bar ω) - h\suml{t∈V}{}ω(t)
\end{equation*}
Оказывается, что при $h≠0$ вся картинка существенно меняется.
\begin{problem}
  Найти основные состояния для этой модели.
\end{problem}
\begin{answer}
  При любых граничных условиях основное состояние $ω ≡ \sgn(h)$ при достаточно
  большом объеме сосуда.
\end{answer}
\begin{problem}
  $\displaystyle P^{(β, h)}_{V, \bar ω(V)}(ω(0,0) = 1) \stackrel[N→∞]{β→∞}{\longrightarrow} Φ(h)$
\end{problem}
%% Local Variables:
%% compile-command: "xelatex -halt-on-error -file-line-error main.tex"
%% End:

\section{Одномерная модель Изинга.}

Пусть теперь $S$ -- спиновое пространство, $\hm{S} < ∞$, $T = \Z$,
$V⊆T$, $ω\type{V}{S}$. Пусть определена функция $U\type{S×S}{R}$ и рассмотрим
формальный ряд
\begin{equation*}
  H(ω) = \suml{\mat{t, t'∈T\\ \hm{t - t'} = 1}}{} U(ω(t), ω(t'))
\end{equation*}

Задача та же, что и в предыдущем разделе. Сосуд $V_N = [-N, N]$, граничные условия
$s^- = ω(-N - 1)$, $s^+ = ω(N+1)$. Интересно, что происходит при $N→∞$.
\begin{df}
  Матрица Больцмана $B(s,s') ≝ e^{-U(s,s')}$
\end{df}
\begin{lemma}
  Статсумма $Z = B^{2N+2}(s^-,s^+)$
\end{lemma}
\begin{proof}
  Доказывается по индукции.
\end{proof}
\begin{problem}
  Пусть $A$ -- матрица, все элементы которой положительны, и пусть
  $λ = λ(A)$ -- максимальное собственное значение. Тогда существует в смысле
  поэлементной сходимости предел $ \liml{n→+∞}\frac{A^n}{λ^n}$.
\end{problem}
\begin{df}
  Будем называть матрицу положительной, если все её элементы положительны.
\end{df}
\begin{theorem}[Перрон, 1909]
  Пусть $A$ -- положительная матрица. Тогда её максимальное по модулю
  собственное значение вещественно и кратности $1$. Причем выполнено
  \begin{equation*}
    Ax = λx \quad yA = λy\quad x_i,y_i >0
  \end{equation*}
\end{theorem}
\begin{theorem}[Фробениус]
  Пусть существует $n_0$, что $A^{n_0}$ -- положительная матрица.
  Если при этом матрица $A$ -- неотрицательна, то для неё верно
  утверждение теоремы Перрона.
\end{theorem}
\begin{hint}
  Утверждение вытекает из эргодической теоремы для цепей Маркова.
\end{hint}
\begin{petit}
  Здесь пропущено примерно пол-страницы которые я не понимаю.
\end{petit}



%% Local Variables:
%% compile-command: "xelatex -halt-on-error -file-line-error main.tex"
%% End:

\section{Ещё одна лекция}
Перейдем другому подходу -- пусть теперь энергия взаимодействия бывает
не только у соседей. Теперь полагаем, что $T = \Z^d$, $\hm{S} < ∞$,
$V⊂T$ -- сосуд, $t∈T$, $V+t$ -- тоже сосуд. Отображение
$τ_t\type{Ω_{V+t}}{Ω_V}$ задается естественным образом: $(τ_tω)(t') =
ω(t+t')$.
\begin{df}
  Потенциал  $\displaystyle Φ=\hc{Φ_V\type{Ω_V}{\R}, \hm{V} < ∞}$
\end{df}
\begin{denote}
  $\displaystyle \Fc = \hc{V⊂T\cln \hm{V} < ∞}$
\end{denote}

В дальнейшем нас будут интересовать только трансляционно-инвариантные
потенциалы, т.е $Φ_{V+t}(ω) = Φ_V(τ_t(ω))$.

\begin{df}
  Взаимодействие конечного радиуса («финитное»), если
  \begin{equation*}
    ∃V_0∈\Fc\cln ∀V(∃ω∈Ω_V\cln Φ_V(ω) → ∃t\cln V+t ⊆ V_0)
  \end{equation*}
\end{df}
\newcommand{\nw}[1]{\hn{#1}_w}
\newcommand{\ns}[1]{\hn{#1}_s}
\begin{df}
  Определим нормы потенциалов
  \begin{align*}
    \nw Φ =\suml{0∈V∈\Fc}{}\frac{1}{\hm V}\max_{ω∈Ω_V}\hm{Φ_V(ω)}\\
    \ns Φ = \suml{0∈V∈\Fc}{}\max_{ω∈Ω_V}\hm{Φ_V(ω)}
  \end{align*}
\end{df}

\begin{problem}
  С любой из этих норм пространство $Φ$ является Банаховым со
  стандартным сложением и умножением.
\end{problem}

\begin{df}
  Давлением называется величина, определяемая как $P≝\liml{\hm
    V→+∞}\frac{\ln Z_V}{\hm V}$, где
  \begin{equation*}
    E_V(ω) = \suml{V'⊆V}{}Φ_{V'}(ω) \qquad
    Z_V = \suml{ω∈Ω_V}{} e^{-E_V(ω)}
  \end{equation*}
\end{df}
\begin{petit}
  Шесть строчек, на которых парсер захлебнулся.
\end{petit}

\begin{df}
  Последовательность $\hc{a_n}_{n=1}^∞$ называется субаддитивной, если
  $a_{n+m} ≤ a_n + a_m$. Тогда существует
  $$\liml{n→+∞}\frac{a_n}{n} = \inf\limits_n \frac{a_n}{n}$$
\end{df}

\begin{lemma}
  $\displaystyle Z_{V_1\sqcup V_2} ≤ Z_{V_1}·Z_{V_2}·e^{N(V_2)\ns Φ}$,
  где $N(V_2)$, грубо говоря, описывает размер приграничной полосы в
  терминах $Λ$.
\end{lemma}
\begin{note}
  Далее, полагаем, что $\hn ·  ≡ \ns ·$
\end{note}
\begin{proof}
  \newcommand{\vsc}{{V_1\sqcup V_2}}
  Для произвольного $ω∈Ω_{\vsc}$ распишем
  \begin{equation*}
    E_\vsc(ω) = E_{V_1}(ω\evn {V_1}) + E_{V_2}(ω\evn {V_2}) +E_{V_1|V_2}(ω)
  \end{equation*}
  \begin{petit}
    Почему бы не использовать динамические переменные в математическом
    тексте?
  \end{petit}
  \begin{denote}
    Множество $V$ обладает свойством $⊛$, обозначаем $⊛V$, если
    $V⊂\vsc, V∩V_1 ≠ ∅, V∩V_2 ≠ ∅$.
  \end{denote}
  \indent где $E_{V_1|V_2}(ω) = \suml{⊛V}{} Φ(ω\evn V)$


  Для того, что бы оценить $E_\vsc(ω)$, возьмем любую точку $t$ из
  $V_2$ и рассмотрим все $V$, содержащие эту точку. Оцениваем их
  количество как количество множеств $V$, для которых $V∩V_2 ≠ ∅$.
  \begin{petit}
    Парсер сдох. Попробуйте запустить его снова. Пропущено примерно
    половина страницы.
  \end{petit}
\end{proof}
\begin{denote}
  Пусть $\bar a = (a_1\sco a_d) ∈ ℵ^d$. Тогда будем рассматривать
  $V(a) = [0, a_1 -1] × [0, a_2 -1] ×…× [0, a_d-1]$. Будем писать $a→∞$, подразумевая, что
  $\min\limits_{i}a_i → ∞$.
\end{denote}

\newcommand{\zvfr}{\frac{\ln{Z_{V(a)}}}{\hm{V(a)}}}
\newcommand{\zvfra}{\frac{\ln{Z_{V(a^*)}}}{\hm{V(a^*)}}}
\newcommand{\limai}{\liml{a→+∞}{}}
\begin{theorem}
  Существует предел $\limai\zvfr$
\end{theorem}
\begin{proof}
  Определим $P ≝ \inf\limits_a \zvfr$. Пусть $P > -∞$. Тогда $∀ε>0$ $ ∃\,a^*$, что
  $\zvfra < P+ε$. Пусть $a = n*a^*$ (т.е $a_j = n*a_j^*$).
\end{proof}
%% Local Variables:
%% compile-command: "xelatex -halt-on-error -file-line-error main.tex"
%% End:

Рассмотрим более общий случай (последовательность Фельмера).
\begin{df}
  Пусть $G$ -- счетная группа, $A_n ⊂ G$, $\hn{A_n} < ∞$.
  Последовательность $\hc{A_n}_{n=1}^∞$ -- последовательность Фельмера,
  если для любого $g∈G$
  \begin{equation*}
    \liml{n→∞}\frac{\hm{gA_n\triangle A_n}}{\hm{A_n}} = 0
  \end{equation*}
  Если такая последовательность существует, то группа называется аменабельной.
\end{df}
\begin{petit}
 И к чему это? Дальше идет совершенно другое. Все это надо вычитывать,
 желательно с пониманием происходящего.
\end{petit}
\begin{lemma}
   $P_V^Φ$ как функция от $Φ$ является Липшицевой с $\const = 1 $.
\end{lemma}
\newcommand{\dl}{\frac{d}{dλ}}
\newcommand{\flp}{{Φ+λΨ}}
\begin{proof}
  \begin{equation*}
    \dl P_V^{\flp} = \frac 1{\hm V}\frac{1}{Z_\flp}\suml{ω∈Ω_V}{} e^{-E_V^\flp}·(-E_V^Ψ(ω))
    = \suml{ω∈Ω_V}{}\inv{\hm V}\frac{e^{-E_V^\flp}}{Z_V^\flp}·(-E_V^Ψ(ω))
  \end{equation*}
  Заметим, что $\frac{e^{-E_V^\flp}}{Z_V^\flp} = P(ω)$, а $\suml{ω∈Ω_V}{} P(ω) = 1$.
  Выполним оценку
  \begin{equation*}
    \hm{E_V^Ψ(ω)} ≤ \suml{t∈V}{}\suml{\mat{V'⊂V\\t∈V'}}{}\inv{\hm{V'}}\hm{Ψ_{V'}(ω')}
    ≤ \suml{t∈V}{} \nw Ψ = \hm V\nw Ψ
  \end{equation*}
  Таким образом, $\hm{\dl P_V^\flp} ≤ \nw Ψ$.  Введя обозначение $f(λ) =
  P_V^{λΦ_1 + (1-λ)Φ_2}$, имеем
  \begin{equation*}
    \hm{P^{Φ_1}-P^{Φ_2}} = \hm{f(0) - f(1)} ≤ \maxl{α∈(0,1)}\hm{f'(α)} ≤ \nw{Φ_1 - Φ_2}
  \end{equation*}
\end{proof}
\begin{lemma}
  Функция $P_V^\flp$ выпуклая по $λ$.
\end{lemma}
\newcommand{\ddl}{\frac{d²}{dl²}}
\newcommand{\sumo}{\suml{ω∈Ω_V}{}}
\begin{proof}
  \begin{equation*}
    \dl P_V^\flp  = \suml{ω∈Ω_V}{} -\frac{E_V^Ψ(ω)}{\hm V} p(ω)
  \end{equation*}

  \begin{equation*}
    \ddl P_V^\flp =
    \sumo\hc{ \frac{-E_V^Ψ(ω)}{\hm V}· \inv{(Z_V^\flp)²}·\hs{Z_V^\flp·e^{-E^\flp_V(ω)}·(-E_V^Ψ(ω))}
      - (\dl Z_V^\flp)·e^{-E_V^\flp(ω)}}
  \end{equation*}
  \begin{equation*}
    \ddl P_V^\flp(0) = K^2 \suml{ω,ω'}{}e^{-E_V^Φ(ω) - E_V^Φ(ω')}·\hs{(E^Ψ_V(ω))² - E_V^Ψ(ω')E_V^Ψ(ω)}
    = K^2\suml{ω,ω'}{}e^{-E_V^Φ(ω) - E_V^Φ(ω')} · \frac 12\hs{E_V^Ψ(ω) - E_V^Ψ(ω')}²
  \end{equation*}
  \begin{note}
    Последнее равенство имеет место в силу соотношения
    \begin{equation*}
      \suml{a,b}{}g(a)f(b)\hs{f²(a) - f(a)f(b)} = \suml{a,b}{} g(a)g(b)\hs{f(a) - f(b)}²
    \end{equation*}
  \end{note}
  Получаем, что $\ddl P_V^\flp\evn{λ=0} ≥ 0$. Ввиду произвольности $Φ$,
  $Ψ$ получаем, что вторая производная положительна для любых $λ$.
  Таким образом, функция $P_V^\flp$ выпуклая по $λ$.
\end{proof}

%% Local Variables:
%% compile-command: "xelatex -halt-on-error -file-line-error main.tex"
%% End:

%% -*- mode: latex -*-
Функцию $P_V^Φ$ можно рассматривать как функцию потенциала $Φ$. При
этом, это отображение непрерывно на $B_ω = \hc{Φ\cln\nw Φ < ∞}$.  При
любом фиксированном $V$ функция $P_V\type{B_ω}{\R}$ выпуклая, т.к
свойство выпуклости выполнено вдоль любой прямой по доказанному.

%% Ранее замечали, что $\hm{P_V^{Φ_1} - P_V^{Φ_2}} ≤ \nw{Φ_1 - Φ_2}$. Но
%% это верно даже равномерно по $V$. Воспользуемся теоремой (\ref{limit-pressure}).
%% \inline{Как?}.
\begin{lemma}
Пусть $Φ∈B_ω$. Тогда существует $\hc{Φ^{(n)}}_{n=1}^∞$, что
\begin{itemize}
\item  $Φ^{(n)}$ -- конечного радиуса.
\item  $\nw{Φ - Φ^{(n)}} \stackrel{n→∞}{\longrightarrow} 0$
\end{itemize}
\end{lemma}
\begin{proof}
        Вспомним определение нормы $\displaystyle \nw Φ =
        \suml{0∈V\in\Fc}{}\maxl{ω∈Ω_V}\frac{\hm{Φ_V(ω)}}{\hm{V}}$. Возьмем
        семейство кубов $K_n = V(2n)$. Объявим потенциал $Φ^{(n)}$
        равным нулю, если область не содержится в $K_n$ (после сдвига)
        и совпадает с $Φ$ иначе. Очевидно, что $Φ^{(n)}$ -- конечного
        радиуса и при этом $\nw{Φ-Φ^{(n)}}
        \stackrel{n→∞}{\longrightarrow} 0$ из сходимости ряда в
        определении $\nw ⋅$.
\end{proof}
\begin{theorem}
  Для любого $Φ∈B_ω$, не обязательно конечного ранга, существует предел
  $P^Φ = \liml{a→∞} P^Φ_{V(a)}$.
\end{theorem}
\begin{proof}
  Формально,
  \begin{equation*}
    \hm{P_{V(a)}^Φ - P^Φ} ≤ \nw{Φ- Φ^{(n)}}  + \hm{P^{Φ^{(n)}} - P^Φ}
    + \hm{P_{V(a)}^{Φ^{(n)}} - P^{Φ^{(n)}}}
  \end{equation*}
  Сначала выбираем $ n = n_ε$, при котором сумма первых двух слагаемых
  меньше $ε/2$, потом находим $a_ε = a(n_ε)$, что при любом $a > a_ε$
  третье слагаемое также меньше $ε/2$.
\end{proof}
Функция $P^Φ$  -- выпуклая как предел выпуклых функций.

%% Local Variables:
%% compile-command: "xelatex -halt-on-error -file-line-error main.tex"
%% End:

\section{Cтатсумма с граничными условиями.}
\newcommand{\oarg}{(ω,\bar ω)}
\begin{denotes}
  $\bar ω ∈ Ω_{T\backslash V} = Ω_{\bar V}$, $(ω, \bar ω) ∈ Ω$,
  $(ω,\bar ω)\evn{V} = ω$, $(ω, \bar ω)\evn{Τ\backslash V} = \bar ω$
\end{denotes}
\begin{denote}
  $\Fg(V,\bar V) = \hc{V' ∈ \Fc\,| V'∩V≠∅, V'∩\bar V ≠∅}$
\end{denote}
\begin{df}
  \begin{align}
    Z_V^Φ(\bar ω) ≝ \sumo e^{-E_V(ω,\bar ω)}\\
    E_V\oarg ≝ E_V(ω) + E_V(ω|\bar ω) \\
    E_V(ω|\bar ω) ≝ \suml{V'∈\Fg(V,V')}{} Φ_V'(\oarg\evn{V'})
  \end{align}
\end{df}
\begin{theorem}
  Если $\ns Φ < ∞$, то равномерно по $\bar ω$ $\liml{a→∞}\frac{\ln
    Z_{V(a)}(\bar ω)}{\hm{V(a)}} = P^Φ$.
\end{theorem}
\begin{proof}
  Оценим $\ln Z_V(\bar ω) - \ln Z_V$. Покажем, что если $V$ -- элемент
  последовательности Фельмера, то эта разность растет медленнее объема.

  \begin{denote}
    $\Fc_r = \hc{V'|∃t∈T\cln V'+t∈B(0,r)}$
  \end{denote}
  Для фиксированного $r$ запишем, что
  \begin{equation*}
    \newcommand{\farg}{Φ_{V'}(\oarg\evn{V'})}
    E_V(ω|\bar ω) = \suml{V'∈\Fg(V,V')}{}\farg
    =\suml{V'∈\Fg{V,\bar V}∩\Fc_r}{}\farg + \suml{V'∈\Fg(V,\bar V)\backslash\Fc_r}{}
    ≝ Ε^{ ≤r} + Ε^{>r}
  \end{equation*}

  Оценим суммы отдельно. Зафиксируем $t∈V$. Если есть $V'∋t$,
  то ($t+B(0,2r) ∩\bar V ≠∅$).
  \begin{equation*}
    Ε^{≤r} ≤ \suml{t∈V, B(t,2r) ∩\bar V ≠∅}{}\suml{V'∋t}{}…
    ≤ \suml{t∈V, B(t,2r) ∩\bar V ≠∅}{} \ns Φ =\hm{\hc{t∈V, B(t,2r) ∩\bar V ≠∅}} \ns Φ
  \end{equation*}
  \begin{equation*}
    \hm{E^{>r}} ≤ \suml{t∈V}{} \ns{Φ}^{>r} = \hm V \ns{Φ}^{>r}
  \end{equation*}
  где $\ns{Φ}^{>r} = \suml{0∈V'∉\Fc_r}{}\maxl{ω∈Ω_{V'}}…$
  Получаем в итоге оценку
  \begin{denote}
    $\hm{V}_{2r} = \hm{\hc{t∈V, B(t,2r) ∩\bar V ≠∅}}$
  \end{denote}
  \begin{equation*}
    \hm{E_V\oarg} ≤ \hm V\ns Φ^{>r} + \hm{V}_{2r}\ns Φ
  \end{equation*}
  Потенцируя, получаем, что
  \begin{denote}
    $C(V,r) = \hm V\ns Φ^{>r} + \hm{V}_{2r}\ns Φ$
  \end{denote}
  \begin{equation*}
    \frac{e^{-E_V\oarg}}{e^{-E_V(ω)}} = e^{-E_V(ω|\bar ω)}
    ≤ e^{C(V,r)}
  \end{equation*}
  Аналогично, с другой стороны,
  \begin{equation*}
    \frac{e^{-E_V\oarg}}{e^{-E_V(ω)}} = e^{-E_V(ω|\bar ω)}
    ≥ e^{-C(V,r)}
  \end{equation*}
  Просуммировав неравенства получаем
  \begin{align*}
    e^{-E_V(ω)}e^{-C(V,r)} ≤  e^{-E_v\oarg} ≤ e^{-E_V(ω)}e^{C(V,r)}\\
    e^{-C(V,r)} ≤ \frac{Z_V(\bar ω)}{Z_V} ≤ e^{C(V,r)}\\
    e^{-C(V,r)}Z_V≤Z_V(\bar ω) ≤ Z_V e^{C(V,r)}\\
    \ln Z_V - C(V,r) ≤ \ln Z_V(\bar ω) ≤ \ln Z_V + C(V,r)
  \end{align*}
  Разделим последнее равенство на $\hm V$ и положим, что $V = V_n$ --
  последовательность Фельмера. Выбрав большое $r$ можно сделать $\ns
  Φ^{≥r} < ε/2$.  А за счет Фельмеровости при фиксированном $r$ начиная
  с некоторого $N$ получить $\frac{\hm{V}_{2r}}{\hm{V}} < \frac{ε}{2\ns Φ}$.
  Отсюда, $\ln Z_V(\bar ω) - \ln Z_V = \bar{\bar o}(\hm{V})$.
\end{proof}

%% Local Variables:
%% compile-command: "xelatex -halt-on-error -file-line-error main.tex"
%% End:

\begin{petit}
  Счастье то какое! Мне не придется это доказывать самому!
\end{petit}
Вспомним, что нам оставался случай $\lliml{a→∞} P_V(a)^Φ = ±∞$.
Поймем, что этого быть не может. Число слагаемых в $E_V^Φ$ меньше, чем
$C·\hm V$. Каждое слагаемое тоже ограничено, откуда $E_V^Φ(ω) < K·\hm V$.
\begin{align*}
  \hm S^{\hm V} e^{-K\hm{V}} ≤ Z_V^Φ ≤ \hm S^{\hm V} e^{K\hm{V}}\\
  (\ln S -K) \hm V ≤ \ln Z_V^Φ ≤ \hm V(\ln S + K)\\
  (\ln S -K)  ≤ P_V^Φ ≤ (\ln S + K)
\end{align*}
И все.

\begin{df}
  Пусть $\nw Φ <∞$. Тогда $f^Φ(ω) = \suml{0∈V∈\Fc}{}\inv{\hm V}Φ_V(ω_V)$.
  Определим $\hn{f^Φ} ≝ \supl{ω}\hm{f^Φ(ω)}$.
\end{df}
\begin{denote}
  $K(ω, ω') = \min\hc{n∈ℵ| ω_{V_n} ≠ ω'_{V_n}}$, где $V_n = [-n, n] ^d$.
\end{denote}
\begin{df}
  $d(ω,ω') = 2^{-K(ω,ω')}$
\end{df}
\begin{note}
  Метрика это есть. Любое цилиндрическое множество (в фиксированных
  местах есть фиксированные спины) является открытым в этой топологии.
\end{note}
Отображение $Φ→f^Φ$ линейно для любого $ω$. В этой топологии $Ω$
компактно.


Вспомним про преобразование $τ_t\cln τ_t(ω)(x) = ω(t+x)$. При каждом
$t$ отображение непрерывно в нашей топологии на $Ω$. Более того, это
преобразования образую группу. Рассмотрим вероятностные меры на $Ω$
--- $M$ На ней $σ$ - алгебра -- это Борелевская $σ$ - алгебра (она же
цилиндрическая).  Пусть $μ∈M$, $B∈\Bc(Ω)$ и $μ(τ_t^{-1}B) =μ(B)$,
тогда $μ$ называется $τ_t$--инвариантной мерой. Если она инвариантна
для любого $τ_t$, то она $\hc{τ_t}$--инвариантна.
\begin{df}
  Энтропия -- неотрицательное число, характеризующее динамическую
  систем $h_μ = h_μ(\hc{τ_t})$.
\end{df}
\begin{denote}
  $τ = \hc{τ_t}$
\end{denote}
\begin{df}
  $f(τ)$ -- множество $τ$ -- инвариантных борелевских мер на $Ω$.
\end{df}
\begin{theorem}[Вариационный принцип]
  Если $\ns{P^Φ} < ∞$, то $\supl{μ}(h_μ - \intl{Ω}{}f^Φdμ) = P^Φ$
\end{theorem}
\begin{proof}
  \begin{petit}
    Ереси много здесь есть. Мастера Йоды мудрость требуется нам.
  \end{petit}
  Мы будем доказывать, что этот супремум достигается. Равенство
  доказывать не будет.  Пусть $(X, τ)$ -- динамическая система, $X$ --
  компактное метрическое пространство. Пусть $f\type{X}{\R}$ --
  непрервына, $M(τ)$ -- $τ$-инвариантные вероятностные борелевские меры.

  Если $τ$ -- аминабельная группа(пусть даже не коммутативная), то
  можно рассмотреть функционал $μ→h_μ(τ) - μ(τ)$. Т.е можно
  рассмотреть $\supl{μ} μ→h_μ(τ) - μ(τ) ≝ P$.
\end{proof}

%% Local Variables:
%% compile-command: "xelatex -halt-on-error -file-line-error main.tex"
%% End:





%% Local Variables:
%% compile-command: "xelatex -halt-on-error -file-line-error main.tex"
%% End:

\tableofcontents
\end{document}

%% Local Variables:
%% compile-command: "xelatex -halt-on-error -file-line-error main.tex"
%% End:
