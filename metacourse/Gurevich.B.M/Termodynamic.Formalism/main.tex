\documentclass[10pt]{article}
\usepackage{dmvn}
\usepackage{polyglossia}
\usepackage{unicode-math}
\usepackage{fontspec}
\usepackage{makeidx}
\defaultfontfeatures{Scale=MatchLowercase, Mapping=tex-text}
\setmainfont{CMU Serif}
\setsansfont{CMU Sans Serif}
\setmonofont{CMU Typewriter Text}
\setmathfont{xits-math.otf}
\newcommand{\type}[2]{\ensuremath{\colon\!#1→#2}}
\begin{document}
\dmvntitle{Курс естествнно-научного содержания}
{ Термодинамический формализм}
{Лектор -- Гуревич Борис Маркович}
{IV(V) курс, 7(9) семестр, поток математиков, ЕНС}
{Москва, 2012г}

\section{Dead man letter}
Пока лектор не предоставляет  вразуметельной структуризации,
каждая лекция будет отдельной секцией.
\section{Лекция}
\newcommand{\sumx}{\ensuremath{\suml{x∈X}{}}}
Рассмотрим систему с конечным множеством состояний $X=\hc{x_i}$. Для
произвольного состояния $x∈X$ полагаем определенными следующие функции
\begin{enumerate}
\item $E(x)$ -- энергия системы в состоянии $x$. $E\type X\R$
\item $P(x)$  -- вероятность состояния, $\sumx P(x) = 1$
\end{enumerate}
\begin{df}
  Постулатом Гибса называется следующее соотношение
  \begin{equation}
    \label{dist:gibs}
    \begin{array}{cc}
      P(x) = ce^{-βE(x)} & c = \frac 1{\sumx e^{-βE(x)}}
    \end{array}
  \end{equation}
  Константа $c$ называется статистической суммой. В англоязычной литературе -- `partition function'.g
\end{df}
\newcommand{\intX}{\intl{X}{}}
Приведем некое обоснование постулата Гибса. Рассмотрим следующую функцию от распределения
\begin{equation*}
  φ(\vec P) = -\sumx P(x)\ln P(x) - \sumx P(x)E(x) = -\intX \ln P dP -\intX EdP
\end{equation*}
\begin{df}
  Выражение $-\intX \ln P dP $ называется энтропией, $\intX EdP$ -- средней энергией.
\end{df}
\begin{lemma}
  Распределение Гибса~(\ref{dist:gibs}) является стационарной точкой функции $φ(\vec P)$.
\end{lemma}
\begin{proof}
 Введем обозначения $P(x_i) = p_i$, $E(x_i) = u_i$. Множество всех
 допустимых значений $p_i$ образует симплекс
 \begin{equation*}
   \hc{(p_1,…,p_n) : p_i≥0, \sumiun p_i = 1}
 \end{equation*}
Найдем стационарные точки функции $φ(p_1,…,p_n) = -\sumiun p_i\ln p_i - \sumiun p_iu_i$, где
$u_i$ полагаем фиксированными. В соотвествии с теорией вариационного исчисления, вводим функцию
\begin{equation*}
  \begin{array}{lr}
    Φ = φ + λs & s(p_1,…,p_n) = \sumiun p_i
  \end{array}
\end{equation*}
Тогда $\pf Φ{p_i} = -\ln p_i -1 -u_i +λ$, откуда $p_i = ce^{-u_i}$,
\begin{equation*}
  c = \frac 1{\sumiun e^{u_i}} = \frac 1{\sumiun e^{-βu_i^0}}
\end{equation*}
где параметр $β$ вводится заменой $u_i = βu_i^0$.
\end{proof}

Таким образом, мы показали, что распределение Гиббса является
стационарной точкой функции $φ(\bar p)$. Для того, что бы доказать,
что это будет максимум, покажем, что $φ$ -- выпуклая вниз(вогнутая).

Действительно, рассмотрим функцию $f(y) =-y\ln y$, $y∈[0,1]$,
доопределенную в нуле по непрерывности. \{Здесь будет график\}

Линейная комбинация $ap+bp'$, где $a,b≥0$, $a+b = 1$ называется выпуклой комбинацией.

\begin{df}
  Функция $g$ -- выпуклая, если
  \begin{equation*}
    g(ap+bp') ≤ag(p)+bg(p')
  \end{equation*}
\end{df}

Отсюда получаем, что распределение Гиббса действительно доставляет
максимум функции $φ$, выпуклой как линейной комбинации выпуклых
функций и с линейным ограничением на $\bar p$.

\section{Модель Изенга}

Рассмотрим множество $T=\Z²$ -- целочисленную решетку. Также
рассмотрим подмножество $V=[-N, N]²∩T$, называемое \emph{сосудом}.

Функция $ω\cln V→S$ называется \emph{конфигурацией}, а $S$ --
\emph{спиновым пространством}. В дальнейшем будем полагать, что
$S=\hc{-1, 1}$.

Аналогично определим конфигурацию на всей решетке
\begin{equation*}
  \tilde ω\cln T→S
\end{equation*}
\begin{denote}
  $\tilde ω_{V} = \tilde ω\evn{V}$
\end{denote}

\begin{denote}
  $Ω_T=Ω$ -- множество всех конфигураций на $T$.
\end{denote}
\begin{denote}
  $Ω_V$ -- множество всех конфигураций на $V$.
\end{denote}

\begin{stm}
  Пусть $V_1⊂V_2$. Тогда, если $ω∈Ω_{V_2}$, то $ω\evn{V_1} ∈Ω_{V_1}$.
\end{stm}
\begin{denote}
  Определим функцию $U\cln S×S→S$, $U(s,s') = ss'$.
\end{denote}

\begin{df}
  Определим функцию энергии
  \begin{equation*}
    E(ω) =  -J \suml{\scriptstyle\mat{t,t'∈V\\ \hm{t-t'}=1}}{}U(ω(t),ω(t'))
  \end{equation*}
  и вероятности.
  $$ P(ω) = \frac{e^{-E(ω)}}{Z_V} $$
    Величина $Z_V$ называется \emph{статистической суммой}.
  \end{df}
\subsection{Внешние условия}

Рассмотрим конфигурацию $\tilde ω$ на всей целочисленной сетке:
\begin{equation}
        \tilde ω(t) = \bcase{ω(t) \quad&t∈V\\ \bar ω(t)\quad &t∈T\wo V}
\end{equation}
\begin{denote}
        Функция $\bar ω$ называется внешними(граничными) условиями.
\end{denote}

 Для комбинации конфигурации в сосуде и внешних условий определим энергию
 \begin{equation*}
        E(ω,\bar ω) = -J\suml{\mat{t∈V\\t'∈T\\\hm{t-t'} = 1}}{}U(\tilde ω(t),\tilde ω(t'))
 \end{equation*}
и вероятность
\begin{equation*}
        P^β_{V,\bar ω}(ω) = \frac{e^{-βE(ω,\bar ω)}}{Z_{V,\bar ω}(β)}
\end{equation*}

\begin{denote}
        Если параметр $J=1$,  то конструкция называется ферромагнетиком Изенга,
        если $J= - 1$ -- антиферромагнетиком.
\end{denote}

\end{document}

%% Local Variables:
%% compile-command: "xelatex -halt-on-error -file-line-error main.tex"
%% End:
