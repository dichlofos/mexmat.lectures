\section{Лекция}
\newcommand{\sumx}{\ensuremath{\suml{x∈X}{}}}
Рассмотрим систему с конечным множеством состояний $X=\hc{x_i}$. Для
произвольного состояния $x∈X$ полагаем определенными следующие функции
\begin{enumerate}
\item $E(x)$ -- энергия системы в состоянии $x$. $E\type X\R$
\item $P(x)$  -- вероятность состояния, $\sumx P(x) = 1$
\end{enumerate}
\begin{df}
  Постулатом Гибса называется следующее соотношение
  \begin{equation}
    \label{dist:gibs}
    \begin{array}{cc}
      P(x) = ce^{-βE(x)} & c = \frac 1{\sumx e^{-βE(x)}}
    \end{array}
  \end{equation}
  \indent  Константа $c$ называется статистической суммой. В англоязычной
  литературе -- `partition function'.
\end{df}
\newcommand{\intX}{\intl{X}{}}
Приведем некое обоснование постулата Гибса. Рассмотрим следующую функцию от распределения
\begin{equation*}
  φ(\vec P) = -\sumx P(x)\ln P\,(x) - \sumx P(x)E(x) = -\intX \ln P\, dP -\intX EdP
\end{equation*}
\begin{df}
  Выражение $(-\intX \ln P dP)$ называется энтропией, $(\intX EdP\, )$ -- средней энергией.
\end{df}
\begin{lemma}
  Распределение Гибса~(\ref{dist:gibs}) является стационарной точкой функции $φ(\vec P)$.
\end{lemma}
\begin{proof}
 Введем обозначения $P(x_i) = p_i$, $E(x_i) = u_i$. Множество всех
 допустимых значений $p_i$ образует симплекс
 \begin{equation*}
   \hc{(p_1,…,p_n) : p_i≥0, \sumiun p_i = 1}
 \end{equation*}
Найдем стационарные точки функции

$φ(p_1,…,p_n) = -\sumiun p_i\ln p_i - \sumiun p_iu_i$,

где $u_i$ полагаем фиксированными. В соотвествии с теорией
вариационного исчисления, вводим функцию
\begin{equation*}
  \begin{array}{lr}
    Φ = φ + λs &\text{где } s(p_1,…,p_n) = \sumiun p_i
  \end{array}
\end{equation*}
Тогда $\pf Φ{p_i} = -\ln p_i -1 -u_i +λ$, откуда $p_i = ce^{-u_i}$,
\begin{equation*}
  c = \frac 1{\sumiun e^{u_i}} = \frac 1{\sumiun e^{-βu_i^0}}
\end{equation*}
где параметр $β$ вводится заменой $u_i = βu_i^0$.
\end{proof}

Таким образом, мы показали, что распределение Гиббса является
стационарной точкой функции $φ(\bar p)$. Для того, что бы доказать,
что это будет максимум, покажем, что $φ$ -- выпуклая вниз(вогнутая).

\begin{wrapfigure}{l}{0.15\textwidth}
    \includegraphics[width=0.15\textwidth]{image1.eps}
\end{wrapfigure}
Действительно, рассмотрим функцию $f(y) =-y\ln y$, $y∈[0,1]$,
доопределенную в нуле по непрерывности (график слева).

\begin{df}
  Линейная комбинация $ap+bp'$, где $a,b≥0$, $a+b = 1$ называется
  выпуклой комбинацией.
\end{df}

\begin{df}
  Функция $g$ -- выпуклая, если
  \begin{equation*}
    g(ap+bp') ≤ag(p)+bg(p')
  \end{equation*}
\end{df}

Отсюда получаем, что распределение Гиббса действительно доставляет
максимум функции $φ$, выпуклой как линейной комбинации выпуклых
функций и с линейным ограничением на $\bar p$.
