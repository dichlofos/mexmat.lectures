\section{Модель Изенга}

Рассмотрим множество $T=\Z²$ -- целочисленную решетку. Также
рассмотрим подмножество $V=[-N, N]²∩T$, называемое \emph{сосудом}.

Функция $ω\cln V→S$ называется \emph{конфигурацией}, а $S$ --
\emph{спиновым пространством}. В дальнейшем будем полагать, что
$S=\hc{-1, 1}$.

Аналогично определим конфигурацию на всей решетке
\begin{equation*}
  \tilde ω\cln T→S
\end{equation*}
\begin{denote}
  $\tilde ω_{V} = \tilde ω\evn{V}$
\end{denote}

\begin{denote}
  $Ω_T=Ω$ -- множество всех конфигураций на $T$.
\end{denote}
\begin{denote}
  $Ω_V$ -- множество всех конфигураций на $V$.
\end{denote}

\begin{stm}
  Пусть $V_1⊂V_2$. Тогда, если $ω∈Ω_{V_2}$, то $ω\evn{V_1} ∈Ω_{V_1}$.
\end{stm}
\begin{denote}
  Определим функцию $U\cln S×S→S$, $U(s,s') = ss'$.
\end{denote}

\begin{df}
  Определим функцию энергии
  \begin{equation*}
    E(ω) =  -J \suml{\scriptstyle\mat{t,t'∈V\\ \hm{t-t'}=1}}{}U(ω(t),ω(t'))
  \end{equation*}
  \indent и вероятности
  $$ P(ω) = \frac{e^{-E(ω)}}{Z_V} $$
  \indent Величина $Z_V$ называется \emph{статистической суммой}.
  \end{df}
\subsection{Внешние условия}

Рассмотрим конфигурацию $\tilde ω$ на всей целочисленной сетке:
\begin{equation}
        \tilde ω(t) = \bcase{ω(t) \quad&t∈V\\ \bar ω(t)\quad &t∈T\wo V}
\end{equation}
\begin{denote}
        Функция $\bar ω$ называется внешними(граничными) условиями.
\end{denote}

 Для комбинации конфигурации в сосуде и внешних условий определим энергию
 \begin{equation*}
        E(ω,\bar ω) = -J\suml{\mat{t∈V\\t'∈T\\\hm{t-t'} = 1}}{}U(\tilde ω(t),\tilde ω(t'))
 \end{equation*}
и вероятность
\begin{equation*}
        P^β_{V,\bar ω}(ω) = \frac{e^{-βE(ω,\bar ω)}}{Z_{V,\bar ω}(β)}
\end{equation*}

\begin{denote}
        Если параметр $J=1$,  то конструкция называется ферромагнетиком Изенга,
        если $J= - 1$ -- антиферромагнетиком.
\end{denote}

\newcommand{\Pp}{P^{(β)}_{V,+}}
\newcommand{\Pm}{P^{(β)}_{V,-}}
\begin{denote}
  Распределение, соответствующее $\bar ω ≡ +1$, обозначим $\Pp$.
\end{denote}

\begin{theorem}[Пайерлс, Рудольф(1907 -- 1995)]
  Равномерно по сосуду (по $N$) выполнено
  \begin{equation*}
    \liml{β→∞} \Pp = 0
  \end{equation*}
\end{theorem}
\begin{proof}
  \begin{denote}
    $ \hat T = T + (½,½) $
  \end{denote}

  Для каждой точки $t$ конфигурации $ω$, в которой $ω(t) = -1$, построим
  $K_t∈\hat T$ -- квадрат единичной стороны с центром в точке $t$.

  \begin{df}
    $V^-(ω) ≝\bigcup\limits_{t\cln ω(t) = -1} K_t$
  \end{df}
  \begin{df}
    $∂ω ≝ ∂V^-(ω)$
  \end{df}

  \begin{df}
    Контур($γ$) -- связная компонента границы $∂ω$
  \end{df}
  \begin{petit}
    Здесь должен быть комментарий про то, что угловые точки не считаются
    связностью, или, что-то там про скругленные углы. Ну нет картинок, увы…
  \end{petit}

  \begin{lemma}
    \begin{petit}
      Все леммы очевидно показываются на картинках. К сожалению, у меня
    нет времени на это.
    \end{petit}
    \begin{enumerate}
    \item Ребро принадлежит границе $∂ω$ тогда и только тогда, когда по
      разные стороны лежат точки $t_1,t_2∈ T$  с разными знаками.
    \item Каждый контур $γ$ -- несамопересекающаяся замкнутая кривая(ломаная).
    \item Из каждой точки, принадлежащей контуру, выходит ровно два ребра контура.
    \item Пусть $ω((0,0)) = -1$. Тогда существует такой контур  $γ$, что $(0,0) ∈ \Int(γ)$
    \item Граница $ ∂ω$ определяет конфигурацию $ω$.
    \end{enumerate}
  \end{lemma}

  \begin{df}
    $Ω^+ = \hc{ω∈Ω_V\vert \text{ $γ$ -- контур в $∂ω$ }} $
  \end{df}

  \begin{lemma}
    Пусть $γ$  -- замкнутая простая ломаная, принадлежащая $\hat T$.
    Тогда $\Pp(Ω^+) ≤e^{-2β\hm{γ}}$.
  \end{lemma}
  \begin{proof}
    \begin{df}
      Преобразование стирания контура $φ_γ\cln Ω_V→Ω_V$ -- меняет знаки внутри $γ$.
    \end{df}
    \begin{note}
      Очевидно, $φ_γ² = Id$.
    \end{note}
    На картинке можно показать, что $E(ω,\bar ω) - E(φ_φω,\bar ω) =
    -2\hm γ$.  Отсюда, огрубляя,  $\Pp(ω)/\Pp(φ_γω) ≤ e^{-2β\hm γ}$, из чего
    следует утверждение леммы.
  \end{proof}
Теперь будем доказывать теорему:
\newcommand{\Goo}{Γ_0}
\begin{denote}
  Если замкнутый контур $γ$ охватывает точку $(0,0)$, то обозначаем
  $γ∈\Goo$
\end{denote}
\newcommand{\sumgoo}{\suml{γ∈\Goo}{}}
\begin{equation*}
  P(ω(0,0) = -1 ) ≤ \sumgoo P(∂ω⊂γ) ≤
  \sumnui\suml{\tiny\mat{γ∈\Goo\\\hm{γ} = n}}{} P(∂ω⊃γ) ≤
  \suml{n=4}{\infty}\hc{γ\vert \hm{γ} = n, γ∈\Goo} e^{-2βn} ≤
  \suml{n=4}{\infty}4n^24^ne^{-2βn}
\end{equation*}
Последний переход обосновывается тем, что контур задается начальной
точкой ($4n^2$) и последовательностью из $n$ перемещений в одном
из четырех направлений.

Начав с нижней точки пересечения с $OY$ можно получить оценку $n3^{n-1}$.
Таким образом, если $\ln 3 - 2β < 0$, то ряд сходится. При
этом, при достаточно больших $β$ сумма ряда стремится к $0$.
\end{proof}
\newcommand{\convarg}[1]{\stackrel{#1}{\longrightarrow}}
\begin{note}
  Эта теорема дает «двухфазовое регулирование», т.е
  \begin{equation*}
    \Pp(ω(0,0) = -1) \convarg{β→+∞} 0\qquad
    \Pm(ω(0,0) = 1) \convarg{β→-∞} 0
  \end{equation*}
  Получаем, что значение в $(0,0)$ зависит от граничных условий,
  которые находятся сколь угодно далеко.
\end{note}

Посмотрим, что происходит при различный $β$.
\begin{df}
 Cпонтанная намагниченность
 \begin{equation*}
   μ(β) ≝\uliml{N→+∞} P^{(β)}_{V_N,+}(ω(0,0) = +1)
 \end{equation*}
\end{df}

\begin{problem}
  Найти наименьшее $β_0$, что для всех $β>β_0$ выполнено $μ(β) > ½$.
\end{problem}
\begin{answer}
  Нобелевский лауреат Ларс Анзайгер нашел решение. Доказательство
  довольно сложное с использованием алгебры. Есть альтернативное
  доказательство Ландау-Фавыченко (возможно с ошибкой). В курсе
  решение задачи не рассматривается.
\end{answer}


%% Local Variables:
%% compile-command: "xelatex -halt-on-error -file-line-error main.tex"
%% End:
