\section{Ещё одна лекция}
Перейдем другому подходу -- пусть теперь энергия взаимодействия бывает
не только у соседей. Теперь полагаем, что $T = \Z^d$, $\hm{S} < ∞$,
$V⊂T$ -- сосуд, $t∈T$, $V+t$ -- тоже сосуд. Отображение
$τ_t\type{Ω_{V+t}}{Ω_V}$ задается естественным образом: $(τ_tω)(t') =
ω(t+t')$.
\begin{df}
  Потенциал  $\displaystyle Φ=\hc{Φ_V\type{Ω_V}{\R}, \hm{V} < ∞}$
\end{df}
\begin{denote}
  $\displaystyle \Fc = \hc{V⊂T\cln \hm{V} < ∞}$
\end{denote}

В дальнейшем нас будут интересовать только трансляционно-инвариантные
потенциалы, т.е $Φ_{V+t}(ω) = Φ_V(τ_t(ω))$.

\begin{df}
  Взаимодействие конечного радиуса («финитное»), если
  \begin{equation*}
    ∃V_0∈\Fc\cln ∀V(∃ω∈Ω_V\cln Φ_V(ω) → ∃t\cln V+t ⊆ V_0)
  \end{equation*}
\end{df}
\newcommand{\nw}[1]{\hn{#1}_w}
\newcommand{\ns}[1]{\hn{#1}_s}
\begin{df}
  Определим нормы потенциалов
  \begin{align*}
    \nw Φ =\suml{0∈V∈\Fc}{}\frac{1}{\hm V}\max_{ω∈Ω_V}\hm{Φ_V(ω)}\\
    \ns Φ = \suml{0∈V∈\Fc}{}\max_{ω∈Ω_V}\hm{Φ_V(ω)}
  \end{align*}
\end{df}

\begin{problem}
  С любой из этих норм пространство $Φ$ является Банаховым со
  стандартным сложением и умножением.
\end{problem}

\begin{df}
  Давлением называется величина, определяемая как $P≝\liml{\hm
    V→+∞}\frac{\ln Z_V}{\hm V}$, где
  \begin{equation*}
    E_V(ω) = \suml{V'⊆V}{}Φ_{V'}(ω) \qquad
    Z_V = \suml{ω∈Ω_V}{} e^{-E_V(ω)}
  \end{equation*}
\end{df}
\begin{petit}
  Шесть строчек, на которых парсер захлебнулся.
\end{petit}

\begin{df}
  Последовательность $\hc{a_n}_{n=1}^∞$ называется субаддитивной, если
  $a_{n+m} ≤ a_n + a_m$. Тогда существует
  $$\liml{n→+∞}\frac{a_n}{n} = \inf\limits_n \frac{a_n}{n}$$
\end{df}

\begin{lemma}
  $\displaystyle Z_{V_1\sqcup V_2} ≤ Z_{V_1}·Z_{V_2}·e^{N(V_2)\ns Φ}$,
  где $N(V_2)$, грубо говоря, описывает размер приграничной полосы в
  терминах $Λ$.
\end{lemma}
\begin{note}
  Далее, полагаем, что $\hn ·  ≡ \ns ·$
\end{note}
\begin{proof}
  \newcommand{\vsc}{{V_1\sqcup V_2}}
  Для произвольного $ω∈Ω_{\vsc}$ распишем
  \begin{equation*}
    E_\vsc(ω) = E_{V_1}(ω\evn {V_1}) + E_{V_2}(ω\evn {V_2}) +E_{V_1|V_2}(ω)
  \end{equation*}
  \begin{petit}
    Почему бы не использовать динамические переменные в математическом
    тексте?
  \end{petit}
  \begin{denote}
    Множество $V$ обладает свойством $⊛$, обозначаем $⊛V$, если
    $V⊂\vsc, V∩V_1 ≠ ∅, V∩V_2 ≠ ∅$.
  \end{denote}
  \indent где $E_{V_1|V_2}(ω) = \suml{⊛V}{} Φ(ω\evn V)$


  Для того, что бы оценить $E_\vsc(ω)$, возьмем любую точку $t$ из
  $V_2$ и рассмотрим все $V$, содержащие эту точку. Оцениваем их
  количество как количество множеств $V$, для которых $V∩V_2 ≠ ∅$.
  \begin{petit}
    Парсер сдох. Попробуйте запустить его снова. Пропущено примерно
    половина страницы.
  \end{petit}
\end{proof}
\begin{denote}
  Пусть $\bar a = (a_1\sco a_d) ∈ ℵ^d$. Тогда будем рассматривать
  $V(a) = [0, a_1 -1] × [0, a_2 -1] ×…× [0, a_d-1]$. Будем писать $a→∞$, подразумевая, что
  $\min\limits_{i}a_i → ∞$.
\end{denote}

\newcommand{\zvfr}{\frac{\ln{Z_{V(a)}}}{\hm{V(a)}}}
\newcommand{\zvfra}{\frac{\ln{Z_{V(a^*)}}}{\hm{V(a^*)}}}
\newcommand{\limai}{\liml{a→+∞}{}}
\begin{theorem}
  Существует предел $\limai\zvfr$
\end{theorem}
\begin{proof}
  Определим $P ≝ \inf\limits_a \zvfr$. Пусть $P > -∞$. Тогда $∀ε>0$ $ ∃\,a^*$, что
  $\zvfra < P+ε$. Пусть $a = n*a^*$ (т.е $a_j = n*a_j^*$).
\end{proof}
%% Local Variables:
%% compile-command: "xelatex -halt-on-error -file-line-error main.tex"
%% End:
