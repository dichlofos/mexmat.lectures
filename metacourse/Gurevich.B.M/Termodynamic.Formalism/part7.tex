%% -*- mode: latex -*-
Функцию $P_V^Φ$ можно рассматривать как функцию потенциала $Φ$. При
этом, это отображение непрерывно на $B_ω = \hc{Φ\cln\nw Φ < ∞}$.  При
любом фиксированном $V$ функция $P_V\type{B_ω}{\R}$ выпуклая, т.к
свойство выпуклости выполнено вдоль любой прямой по доказанному.

%% Ранее замечали, что $\hm{P_V^{Φ_1} - P_V^{Φ_2}} ≤ \nw{Φ_1 - Φ_2}$. Но
%% это верно даже равномерно по $V$. Воспользуемся теоремой (\ref{limit-pressure}).
%% \inline{Как?}.
\begin{lemma}
Пусть $Φ∈B_ω$. Тогда существует $\hc{Φ^{(n)}}_{n=1}^∞$, что
\begin{itemize}
\item  $Φ^{(n)}$ -- конечного радиуса.
\item  $\nw{Φ - Φ^{(n)}} \stackrel{n→∞}{\longrightarrow} 0$
\end{itemize}
\end{lemma}
\begin{proof}
        Вспомним определение нормы $\displaystyle \nw Φ =
        \suml{0∈V\in\Fc}{}\maxl{ω∈Ω_V}\frac{\hm{Φ_V(ω)}}{\hm{V}}$. Возьмем
        семейство кубов $K_n = V(2n)$. Объявим потенциал $Φ^{(n)}$
        равным нулю, если область не содержится в $K_n$ (после сдвига)
        и совпадает с $Φ$ иначе. Очевидно, что $Φ^{(n)}$ -- конечного
        радиуса и при этом $\nw{Φ-Φ^{(n)}}
        \stackrel{n→∞}{\longrightarrow} 0$ из сходимости ряда в
        определении $\nw ⋅$.
\end{proof}
\begin{theorem}
  Для любого $Φ∈B_ω$, не обязательно конечного ранга, существует предел
  $P^Φ = \liml{a→∞} P^Φ_{V(a)}$.
\end{theorem}
\begin{proof}
  Формально,
  \begin{equation*}
    \hm{P_{V(a)}^Φ - P^Φ} ≤ \nw{Φ- Φ^{(n)}}  + \hm{P^{Φ^{(n)}} - P^Φ}
    + \hm{P_{V(a)}^{Φ^{(n)}} - P^{Φ^{(n)}}}
  \end{equation*}
  Сначала выбираем $ n = n_ε$, при котором сумма первых двух слагаемых
  меньше $ε/2$, потом находим $a_ε = a(n_ε)$, что при любом $a > a_ε$
  третье слагаемое также меньше $ε/2$.
\end{proof}
Функция $P^Φ$  -- выпуклая как предел выпуклых функций.

%% Local Variables:
%% compile-command: "xelatex -halt-on-error -file-line-error main.tex"
%% End:
