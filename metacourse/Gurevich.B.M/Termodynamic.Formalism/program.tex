\section*{Программа экзамена}
\subsection*{\inline{Осень 2012}}
\begin{enumerate}
\item Распределение Максвелла-Больцмана-Гиббса на конечном множестве и
  обосновывающий его вариационный принцип.
\item Двумерная модель Изинга. Определение и свойства контуров. Неравенство Пайерлса.
\item Теорема о равномерной по $V$ сходимости к нулю при $β→∞$
  вероятности $P_{V,+}^{(β)}(ω(0,0) = -1)$ того, что в плоском
  ферромагнетике Изинга с обратной температурой $β$ спин в начале
  координат $(0,0)$ равен $-1$ при граничном условии $+1$ вне квадрата
  $V$ с центром в $(0,0)$. Вывод отсюда существования при низкой
  температуре по крайней мере двух предельных распределений Гиббса.
\item Основные состояния. Утверждение о стремлении меры
  $P^{(β)}_{V,\bar ω}$ к равномерному распределению на множестве
  основных состояний. Примеры основных состояний для ферромагнетика
  Изинга при различных граничных условиях. Модель Изинга с внешним полем.
\item Одномерная решеточная модель с конечнозначным спином и
  взаимодействием ближайших соседей. \par Существования предела
  $\liml{n→∞}λ^{-n}B^n$, где $B$ -- квадратная матрица с
  положительными элементами и $λ$ -- её максимальное собственное
  значение (вывод из эргодической теоремы для конечных цепей
  Маркова). Существование единственного предельного распределения
  Гиббса, не зависящего от граничных условий.
\item Общая модель с конечнозначным спином на решентке
  $\Z^d$. Банаховы пространства трансляционно инвариантных потенциалов
  с сильной и слабой нормами. Взаимодействие конечного радиуса: оценки
  статсуммы с пустыми граничными условиями, отвечающей объединию двух
  конечных множеств, через статсуммы, отвечающие каждому из множеств и
  характеристике потенциала.
\item Существование давления как предела нормированного логарифма
  статсуммы по последовательности параллелепипедов с растущими
  сторонами (случай взаимодействия конечного радиуса).
\item Свойства давления как функции потенциала конечного радиуса:
  оценка константы Липшица.  Продолжение давления по непрерывности на
  множество потенциалов с конечной сильной нормой. Давление как предел
  по последовательности параллелепипедов в классе потенциалов с
  конечной нормой.
\item Статсумма при наличнии внешних условий: существование
  отвечающего ей давления и его независимость от этих условий.
\item Давление как выпуклая функция на пространстве потенциалов.
\item Свойства энтропии верояностной меры на пространстве конфигураций
  $Ω_V$, $V⊂Z^d$, $\hm V < ∞$. Существование для всякой инвариантной
  относительно группы сдвигов вероятностной меры на $Ω$ предела при
  $n→∞$ нормированной энтропии её проекции на $Ω_{V_n}$, где
  $\hc{V_n}$ -- последовательность Фельнера.
\item Свойства функции $f_Φ(ω)$ (вклада начала координат в энергию
  конфигурации $ω∈Ω$). Вариационны принцип для давления.
\item Лемма о прореженной последовательности Фельмера. Лемма
  Линденштраусса о случайном выборе элементов последовательности
  Фельмера.
\item Теорема о существовании для всякой последовательности Фельмера
  предела, определяющего давление P (см. вопрос 7).
\end{enumerate}

\begin{note}
  Материал, касающйся вопросов 6 -- 12, можно найти в книге Д. Рюэль
  «Термодинамический формализм», 2002. Однако, обозначения в этой
  книге в большинстве случаев отличаются от использованных в
  лекциях.
\end{note}

%% Local Variables:
%% compile-command: "xelatex -halt-on-error -file-line-error main.tex"
%% End:
