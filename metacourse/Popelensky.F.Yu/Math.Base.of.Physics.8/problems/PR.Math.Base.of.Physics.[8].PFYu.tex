
\subsection{Решения некоторых задач}

\def\psp{\psi^+}
\def\psm{\psi^-}

\def\pspm{\psi^\pm}

\def\apsp{\overset{*}{\psi}{}^+}
\def\apsm{\overset{*}{\psi}{}^-}
\def\apspm{\overset{*}{\psi}{}^\pm}

\def\avpm{\overset{*}{v}{}^\pm}

\def\aphp{\overset{*}{\ph}{}^+}

\def\aapm{\overset{*}{a}{}^\pm}
\def\aamp{\overset{*}{a}{}^\mp}

\def\avp{\overset{*}{v}{}^+}
\def\avm{\overset{*}{v}{}^-}

Ниже следует более общее доказательство написанной в соответствующей главе теоремы. (С) А.\,Юхименко
\begin{problem}[Задача 23]
Для вектора энергии-импульса вещественного скалярного поля получить выражение через
$a^{\pm}(\k)$.
\end{problem}
\begin{solution}
Для начала получим выражение для энергии, \те $P^0$. Как известно,
\mln{
P^0=\int \dxt  T^{00}=\frac12\int \dxt
\hr{\sum_{\mu=0}^3\pd_\mu \ph\pd_\mu \ph+m^2\ph^2}=\\=\frac12\int \dxt
\hr{\sum_{\mu=0}^3\pd_\mu (\php +\ph^-)\pd_\mu (\php +\ph^-)+m^2(\php +\ph^-)^2}=\\=
\frac12\int \dxt
\hr{\sum_{\mu=0}^3\pd_\mu \php \pd_\mu \php +m^2(\php )^2}+I^{+-}+I^{-+}+I^{--}.
}
Посчитаем лишь $I^{++}$ и $I^{+-}$, тогда станет ясно чему равны остальные. Итак,
\mln{
I^{++}=\frac12\frac1{(2\pi)^3}\int \dxt \dkt \dlt  \hr{\sum_{n=0}^3ik^nil^n+m^2}\frac{e^{i(k^0+l^0)x^0}e^{i(\k+\l)\mathbf{x}}}{\sqrt{2k^0}\sqrt{2l^0}}a^+(\k)a^+(\l)=\\=
\hc{(2\pi)^{-3}\int \dxt e^{i\k\mathbf{x}}=\de(\k)}= \frac12\int \dkt \dlt \hr{m^2-k^0l^0-\k\l}\frac{e^{i(k^0+l^0)x_0}}{\sqrt{2k^0}\sqrt{2l^0}}\,a^+(\k)a^+(\l)\de(\k+\l)=\\=\hc{(k^0)^2=m^2+\k^2}=0.
}

\mln{
I^{+-}=\frac12\frac1{(2\pi)^3}\int \dxt \dkt \dlt \hr{\sum_{n=0}^3ik_n(-il_n)+m^2}\frac{e^{i(k^0-l^0)x^0}e^{i(\k-\l)\mathbf{x}}}{\sqrt{2k^0}\sqrt{2l^0}}a^+(\k)a^-(\l)=\\=
\hc{(2\pi)^{-3}\int \dxt e^{i\k\mathbf{x}}=\de(\k)}= \frac12\int \dkt \int
\dlt \hr{m^2+k^0l^0+\k\l}\frac{e^{i(k^0-l^0)x^0}}{\sqrt{2k^0}\sqrt{2l^0}}\,a^+(\k)a^+(\l)\de(\k-\l)=\\=
\frac12\int \dkt
(m^2+(k^0)^2+\k^2)\frac1{\sqrt{2k^0}\sqrt{2k^0}}\,a^+(\k)a^-(\k)= \frac12\int
\dkt  k^0a^+(\k)a^-(\k).
}
Итак,
\eqn{P^0=\frac12\int \dkt  k^0\hr{a^+(\k)a^-(\k)+a^-(\k)a^+(\k)},}
а общий ответ такой:
\eqn{P^{\mu}=I^{-+}+I^{+-}=\frac12\int \dkt k^{\mu}\hr{\ap (\k)a^-(\k)+a^-(\k)\ap(\k)}.}
\end{solution}

\begin{problem}[Задача 43]
Вывести из уравнения Дирака уравнения на $\chi^{\pm}(\k)$ и $(\chi^*)^{\pm}(\k)$.
\end{problem}
\begin{solution}
$\psi(t)$ должно удовлетворять уравнению $(i\ga^{\mu}\pd_\mu -m)\psi(x)=0$, распишем чему равно
$\psi(x)$:
$$
\psi(x)=\psp (x)+\psi^-(x)=\int \dkt  e^{ikx}\chi^+(\k)+\int
\dkt  e^{-ikx}\chi^-(\k),
$$
поэтому
$$
0=\int \dkt  e^{ikx}(ii\ga^{\mu}k_\mu-m)\chi^+(\k)+\int \dkt e^{-ikx}(-ii\ga^{\mu}k_\mu-m)\chi^-(\k).
$$
Поэтому
$$
(\ga^{\mu}k_\mu\pm m)\chi^{\pm}(\k)=0.
$$
Уравнение на $(\chi^*)^{\pm}(\k)$ получается из этого простым сопряжением:
$$
(\chi^*)^{\pm}(\k)(\ga^{\mu})^* k_\mu\pm m=0.
$$
\end{solution}

\begin{problem}[Задача 45]
Получить условие нормировки решений, сопряженных по Дираку:
$$
\ol v^{\pm}_s(\k)v^{\mp}_r(\k)=\pm\frac{m}{k_0}\,\de^s_r.
$$
\end{problem}
\begin{solution}
Доказательство нагло скатаем с книжки Боголюбова\ч Ширкова. Умножим равенство
$$
(\ga^{\mu}k_\mu+m)v^+_r(\k)=0
$$
слева на $\avm_s(\k)$, получим (учитывая, что $\avpm_r\ga^0=\ol v^{\pm}_r$)
$$
k_0\ol v^-_sv^+_r-k_\mu \avm_s\ga^{\mu}v^+_r+m \avm_s v^+_r=0.
$$
Применим к полученному эрмитово сопряжение (и воспользуемся тем, что $\avpm_r=(v^{\mp}_r)^*$ и
$\ga^{\mu *}=-\ga^{\mu}$):
$$
k_0\ol v^-_rv^+_s+k_\mu \avm_r\ga^{\mu}v^+_s+m\avm_rv_s^+=0.
$$
Сравнивая эти два выражения, получаем требуемое.
\end{solution}
