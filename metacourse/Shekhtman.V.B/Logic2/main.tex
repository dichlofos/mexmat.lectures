\documentclass{article}
\usepackage{dmvn}
\usepackage{polyglossia}
\usepackage{unicode-math}
\usepackage{makeidx}
\setmainfont[Mapping=tex-text]{DejaVu Serif}
\setsansfont[Mapping=tex-text]{DejaVu Sans}
\setmonofont[Mapping=tex-text]{DejaVu Sans Mono}
\defaultfontfeatures{Scale=MatchLowercase, Mapping=tex-text}
\setmathfont{Asana-Math.ttf}

\DeclareMathOperator{\Cnst}{Cnst}
\DeclareMathOperator{\Fn}{Fn}
\DeclareMathOperator{\Tm}{Tm_Ω}
\DeclareMathOperator{\CTm}{CTm_Ω}
\DeclareMathOperator{\CTmu}{CTm_{Ω∪M}}
\DeclareMathOperator{\CFmu}{CFm_{Ω∪M}}
\DeclareMathOperator{\CFm}{CFm_Ω}
\DeclareMathOperator{\Fm}{Fm_Ω}
\DeclareMathOperator{\FV}{FV}
\DeclareMathOperator{\Th}{Th}

\newcommand{\baggr}[1]{\left\[\begin{aligned}#1\end{aligned}\right.}
\newcommand{\Ml}{\ul{M}}
\newcommand{\sst}[2]{\hs{#2/#1}}
\newcommand{\need}{{\bf Необходимость({⇐}).}\par}
\newcommand{\suff}{{\bf Достаточность(⇒).}\par}
\makeindex
\begin{document}
\dmvntitle{Математическая логика}
{ и теория алгоритмов}
{Лектор -- Валентин Борисович Шехтман}
{IV курс, 8 семестр, поток математиков, ЕНС}
{Москва, 2012г}
\tableofcontents
\newpage
\section{The beginning}
\begin{df}
  Рассмотрим тройку множеств $Ω=(\Cnst,\Fn,\Pr)$, где $\Cnst$ --
  константы, $\Fn$ -- функциональные символы и $\Pr$ -- предикатные
  символовы. Будем полагать, что $\Pr ≠ ∅$. Также пусть определена
  функция валентности \mbox{$γ\colon \Fn ∪\Pr → ℵ₀$} и множество переменных
  $\Var = \hc{v_1,v_2\etc}$.  $Ω$ называется сигнатурой.
  \begin{denote}
    Пусть $f∈\Fn$ и $γ(f) = k$, тогда будем дополнительно указывать
    валентность индексом сверху: $f^k$.
  \end{denote}
\end{df}

\begin{df}
  Реккурентно определим множество термов $\Tm$:
  \begin{enumerate}
  \item $ c∈\Cnst ⇒ c∈\Tm$
  \item $ x∈\Var ⇒ x∈ \Tm$
  \item $ f^k∈\Fn,\; t_1,t_2,…,t_k∈\Tm ⇒f^k(t_1,t_2,…,t_k) ∈ \Tm$
  \end{enumerate}
\end{df}

\begin{petit}
Вот здесь происходит что-то странное. Во-первых, атомарная формула
определеяется через понятие формулы, а формула через атормарную
формулу. Нестрого, но все довольно прозрачно. А во-вторых от общей
формулировки мы переходим к конкретном алфавиту и в его терминах
определяем понятие формулы. Ну, лектору виднее.
\end{petit}

\begin{df}
  Формулы вида $p^k(t_1,t_2,…,t_k)$, где $p^k∈\Pr$, $t_1,…,t_k ∈\Tm$
  называются атомарными.
\end{df}

\begin{note}
  Далее будем рассматривать следующий алфавит: $\hc{¬,∧,∨,→,(,),∀,∃}$.
\end{note}

\begin{petit}
  Квантеры, похоже, есть предикатные символы, хотя не совсем понятно,
  как их загнать в общее определение.
\end{petit}

\begin{df}
  Вхождение переменной $x$ в слово $α$:
  $(x,α,n)$ -- вхождение, если $α_n = x$.
\end{df}
\begin{note}
  Здесь формула рассматривается как строка, а не как дерево.
\end{note}

\begin{df}
  Реккурентно определим множество формул $\Fm$:
  \begin{enumerate}
  \item $φ$ -- атормарная формула ⇒ $φ ∈ \Fm$
  \item $φ,ψ ∈\Fm$ ⇒ $(φ∧ψ), (φ∨ψ), (φ→ψ) ∈\Fm$
  \item $ φ∈ \Fm $ ⇒ $¬φ ∈ \Fm$
  \item $φ ∈\Fm$, $x ∈\Var$  ⇒ $(∀x φ), (∃x φ) ∈\Fm$
  \end{enumerate}
\end{df}

\begin{df}
  Определим свободные и связанные вхождения переменных в формулы:
  \begin{enumerate}
  \item $φ$ -- атомарная ⇒ все вхождения свободные.
  \item $φ = φ_1∘φ_2$, $∘ ∈\hc{∨,∧,→}$ ⇒
    $\bcase{
      (x,φ,k+1)  \text{ -- свободно, если свободно } (x,φ_1, k) \\
      (x,φ, n+k+2) \text{ -- свободно, если свободно } (x,φ_2,k)
    }$
  \item $φ = ¬ψ$
  \item $φ = Δyψ$, $Δ∈\hc{∀,∃}$ ⇒ $(x,ψ,k)$ -- свободное и $x ≠ y$ → $(x,φ,k+2)$ -- свободно
  \end{enumerate}
  Все остальные вхождения называются связанными.
\end{df}

\begin{df}
  Множество переменных со свободными вхождениями в формулу называется
  параметрами формулы. Обозначение: $\FV(φ)$.
\end{df}

\begin{df}
  Формула $φ$ называется замкнутой, если $\FV(φ) = ∅$.
\end{df}
\begin{denotes}
  Множество замкнутых термов обозначается $\CTm$, множество замкнутых формул --- $\CFm$.
\end{denotes}

\section{Интерпретации}
\begin{df}
  Интерпретацией для $Ω$ называется пара $M=(\Ml, \Jc)$, где $\Ml ≠ ∅$ -- носитель интерпретации $M$,
  а $\Jc$ -- интерпретирующая функция, $\Dom(\Jc) = \Cnst ∪\Pr∪\Fn$, действующая следующим образом:
  %% Функции высшего порядка. Да я любить логику должен…
  \begin{itemize}
  \item $ c ∈ \Cnst$  ⇒ $\Jc(c) ∈\Ml$
  \item $ f^k ∈\Fn$ ⇒ $\Jc(f^k)\colon \Ml^k → \Ml$
  \item $ p^k ∈ \Pr$ ⇒$ \Jc(p^k)\colon \Ml^k → \hc{0,1}$
  \end{itemize}
\end{df}

\begin{denotes}
  $c^M ≔ \Jc(c)$, $f^M ≔ \Jc(f)$, $p^M ≔ \Jc(p)$.
\end{denotes}

\begin{denote}
  \begin{displaymath}
    Ω∪M ≔ (\Cnst∪\Ml, \Pr, \Fn, γ)
  \end{displaymath}
\end{denote}

\begin{df}
  $\CFmu$ называется множестом оцененных формул, $\CTmu$ --- множестом
  оцененных термов.
\end{df}

\begin{df}
  Подробно распишем операцию оценики:
  \begin{enumerate}
  \item $c ∈\Cnst$ ⇒ $ c^M≔ \Jc(c) $
  \item $ c∈ \Ml$ ⇒ $ c^M ≔ c $
  \item $ (f(t_1,t_2,…,t_k))^M ≔ f^M(t_1^M,t_2^M,…,t_k^M)$
  \item $ (p(t_1,t_2,…,t_k))^M ≔ p^M(t_1^M,t_2^M,…,t_k^M)$
  \item $ (¬φ)^M ≔ 1 - φ^M $
  \item $ (∀x φ)^M = 1 ≔ \hr{\sst{x}aφ}^M\!= 1\; ∀ a∈\Ml $
  \item $ (∃x φ)^M = 1 ≔  ∃\; a∈ \Ml\;\; \hr{\sst{x}aφ}^M\!= 1 $
  \end{enumerate}
\end{df}
\begin{denote}
  \begin{displaymath}
    M ⊧φ ≔ φ^M = 1
  \end{displaymath}
\end{denote}

\begin{df}
  $φ$ --- выполнима, если $∃ M$ $M⊧φ$.
\end{df}
\begin{df}
  $φ$ --- общезначима, если $∀M$ $M⊧φ$.
\end{df}
\begin{df}
  Пусть $φ∈\CFm$. Говорим, что $M$ -- модель $φ$, если $M⊧φ$.
\end{df}
\begin{df}
  Теорией первого порядка называется произвольное $T⊂\CFm$
\end{df}
\begin{df}
  $M$ -- модель теории, если $M⊧T$.
\end{df}
\subsection{Примеры теорий}
\subsubsection{Теория графов}
$\Pr=\hc{P^2,=^2}$
\begin{enumerate}
\item $∀x¬Px$
\item $∀x∀y(P(x,y) →P(y,x))$
\end{enumerate}
\begin{df}
  Модель называется нормальной, если
  \begin{displaymath}
    =^M(a,b)=\bcase{1,\quad &a≡b\\ 0,\quad &otherwise}
  \end{displaymath}
\end{df}
\subsubsection{Теория групп}
$\Pr=\hc{=^2},\quad\Fn=\hc{⋅^2}$
\begin{enumerate}
\item $∀x∀y∀z((x⋅y)⋅z = x⋅(y⋅z))$ --- ассоциативность
\item $∃x∀y(x⋅y=y⋅x∧y⋅x=y)$ --- существование единицы
\item $∀x∃y∀e(e\text{ -- единица} →(x⋅y = e\; \&\; y⋅x =e))$ --- существование обратного элемента.
\end{enumerate}
\subsubsection{Теория натуральных чисел}
$Pr=\hc{=^2,P^2}$
\begin{enumerate}
\item[$φ_1$:] $∀x¬Px$
\item[$φ_2$:] $∀x∀y∀z\;\hr{P(x,y)\,\&\,P(y,z) →P(x,z)}$
\item[$φ_3$:] $∀x∃y P(x,y)$
\end{enumerate}
У этой теории есть бесконечная модель $(ℵ, <)$. Конечных моделей нет.
\begin{note}
Заметим, что для $φ=¬(φ_1∧φ_2∧φ_3)$ верно интересное свойство --- она
выводима в любой конечной модели, но не общезначима.
\end{note}
\section{Гомоморфизмы}
Пусть $M$, $M'$ --- интерпретации одной сигнатуры. Рассмотрим отображение $π\colon
M→M'$. Оно называется гомоморфизмом, если
\begin{enumerate}
\item $π\colon \Ml→\Ml'$
\item $π(c^M) = c^{M'}$
\item $π(f^M(a_1,…,a_k)) = f^{M'}(π(a_1),…,π(a_k))$
\item $π(p^M(a_1,…,a_k)) = p^{M'}(π(a_1),…,π(a_k))$
\end{enumerate}

\begin{df}
  Биективный гомоморфизм называется изоморфизмом.
\end{df}

\begin{lemma}
  Пусть $π\colon M→M'$ --- гомоморфизм, $t(x_1,…,x_n) ∈\Tm$,
  $\FV(t(x_1,…,x_n))⊆\hc{x_1,…,x_n}$.  Тогда
  \begin{displaymath}
    π(t(a_1,…,a_m)^M) = t\hr{π(a_1),…,π(a_m)}^{M'}
  \end{displaymath}
\end{lemma}
\begin{proof}
Индукция по построению функции.
\end{proof}
\begin{petit}
  Почему так, непонятно. Но, видимо, очевидно.
\end{petit}
\begin{imp}
  \begin{displaymath}
    M ⊧ a=b ⇔ M' ⊧ π(a) = π(b)
  \end{displaymath}
\end{imp}
\begin{petit}
  Исходники в этом месте были не очень понятны, так что, возможно,
  лемма и неверна. По крайней мере, я её не понимаю. Похоже, все
  модели подразумеваются нормальными.
\end{petit}
\begin{lemma}
  \index{unknown}
  Пусть $π\colon M→M'$ --- сюрьективно, $φ(x_1,…,x_n)∈\Fm$, $FV(φ)⊆\hc{x_1,…,x_n}$. Тогда для
  любых $a_1,…,a_n∈M$ выполнено
  \begin{displaymath}
    \hm{φ(a_1,…,a_n)}_M = \hm{φ(π(a_1),…,π(a_n))}_{M'}
  \end{displaymath}
\end{lemma}
\begin{proof}
  Проведем доказательство по потроению формулы.
  \begin{itemize}
  \item Пусть $φ$ --- атомарная формула. Тогда она имеет вид $φ(\bar
    a) = P^k(a_{i_1},…,a_{i_k})$. По определению
    \begin{displaymath}
      \hm{P^k(a_{i_1},…,a_{i_k})}_M =\hm{P^k(π(a_{i_1}),…,π(a_{i_k}))}_{M'}
    \end{displaymath}
  \item
    $M\;⊧ a_{i_1}=a_{i_2}$ ⇔ $a_{i_1}$ равно $a_{i_2}$. \par
    $M'⊧ π(a_{i_1})=π(a_{i_2})$ ⇔ $π(a_{i_1})$ равно $π(a_{i_2})$.
  \item Пусть $φ=∃yψ(y,x_1,…,x_n)$. Из сюрьективности,
    $∀c∈M'\;∃b∈M\colon π(b)=c$.
    \par $M⊧φ(\bar a)$ ⇔ $∃b∈M\colon M⊧ψ(b,\bar a)$
    \par $M'⊧φ(π(\bar a))$ ⇔ $∃c∈M'\colon M'⊧ψ(c,π(\bar a))$
  \end{itemize}
\end{proof}

\begin{lemma}
  \begin{enumerate}
  \item Композиция гомоморфизмов --- гомоморфизм.
  \item Композиция изоморфизмов --- изоморфизм.
  \item Обратный к изоморфизму --- изоморфизм.
  \end{enumerate}
\end{lemma}
\begin{denote}
  \begin{math}
    M ≅ M' ≔ \;∃ π\colon M→M', π \text{ --- изоморфизм}
  \end{math}
\end{denote}
\begin{lemma}
  $≅$ -- отношение эквивалентности.
\end{lemma}
\begin{df}
  Пусть $M$ --- модель сигнатуры $Ω$, $Φ\colon \Ml^n→\hc{0,1}$ ---
  предикат. $Φ$ выражается (определяется) формулой $φ(x_1,x_2,…,x_n)$,
  если для всех $a_1,a_2,…,a_n∈M$
  \begin{displaymath}
    Φ(a_1,a_2,…,a_n)=\hm{φ(a_1,a_2,…,a_n)}_M
  \end{displaymath}
\end{df}
\begin{petit}
  На самом деле, пример не очень аккуратен -- мы смешиваем в кучу
  сигнатуру, теорию и модель. Впрочем, все довольно естественно.
\end{petit}
\begin{ex}
  Выразим предикат «x -- простое число» в $(ℵ, 1, ⋅ , =)$:
  \begin{displaymath}
    ∀a∀b(x=a⋅b →a=1∨b=1) ∧x≠1
  \end{displaymath}
\end{ex}

\begin{lemma}
  Пусть $Φ(x_1,…,x_n)$ выразима в $M$. Тогда $∀a_1,…,a_n$ и
  $∀π\colon M≅M$ выполнено
\begin{displaymath}
  Φ(a_1,…,a_n) = Φ(π(a_1),…,π(a_n))
\end{displaymath}
Такое свойство называется инвариантностью относительно автоморфизмов.
\end{lemma}
\begin{proof}
  Очевидно из предыдущих лемм.
\end{proof}
\begin{ex}
  Предикат $(y=x^2)$ невыразим в $(ℝ,<,+,=)$.
\end{ex}
\begin{proof}
  Рассмотрим $π: x→2x$. В предположении, что предикат можно выразить,
  по свойству инвариантности относительно автоморфизмов получаем, что
  $ Φ(a,b) =Φ(2a,2b) $. Т.е
  \begin{displaymath}
    a=b^2 ⇔ 2a = (2b)^2
  \end{displaymath}
  Противоречие.
\end{proof}
\section{Теории}
\newcommand{\pargs}[2]{#1_{i_1},…,#1_{i_{#2}}}
\newcommand{\enum}[2]{#1_1,#1_2,…,#1_{#2}}
\begin{df}
  Пусть $M$ -- интерпретация сигнатуры $Ω$. Элементарной теорией
  называется
  \begin{displaymath}
    \Th(M) = \hc{φ∈\CFm \,|\; M⊧φ}
  \end{displaymath}
\end{df}
\begin{df}
  Модели $M$, $M'$  элементарно эквивалентны, если
  $\Th(M) = \Th(M')$.
  \begin{displaymath}
    M≡M'\,≔\, \Th(M) = \Th(M')
  \end{displaymath}
\end{df}
\begin{df}
  Теория $T$ называется сильно категоричной, если все её модели с
  равенством изоморфны.
\end{df}

\begin{theorem}
  Пусть $M$ --- конечная интерпретация конечной сигнатуры с
  равенством. Тогда $\Th(M)$ --- сильно категорична.
\end{theorem}
\begin{petit}
  Здесь должно быть {\it адекватное} доказательство.
\end{petit}
\begin{proof}
  Т.к $M$ --- конечна, то $\Ml = \hc{a_1,…,a_n}$. Рассмотрим формулу
  \begin{displaymath}
    φ = ∃v_1…∃v_nψ_M
  \end{displaymath}
  где $ψ_M$ рассказывает все о $M$.
  %% Вот тут код непортабелен. Ибо уж слишком сильно от шрифта зависит.
  %% Уверен, что пробелы можно расставлять грамотней. diff -U4 господа.
  \begin{displaymath}
    \begin{gathered}
      ψ_M≔\bigwedge_{i≠j}v_i≠v_j ∧ ∀y\bigvee_i( y=v_i) ∧ \bigwedge\hc{c=v_i\,|\, c∈\Cnst,\, M⊧c=a_i} ∧\\
      ∧\bigwedge\hc{P^k(\pargs vk)\,|\,P^k∈\Pr,\, M⊧P^k(\pargs ak)}
      ∧\bigwedge\hc{¬P^k(\pargs vk)\,|\,P^k∈\Pr,\, M⊧¬P^k(\pargs ak)}∧\\
      ∧\bigwedge\hc{f^k(\pargs vk) = v_j\,|\,f^k ∈\Fn,\, M⊧f^k(\pargs ak) = a_j}
    \end{gathered}
  \end{displaymath}
 Формула инвариантна относительно автоморфизмов, поэтому
  сохраняется  множесто выводимых формул.
\end{proof}

\begin{lemma}
  $ N ⊧φ_M ⇔ N≅M$
\end{lemma}
\begin{proof}
  \need
  \begin{petit}
    Из этого все должно следовать, но мне это непонятно.
  \end{petit}
  $$M⊧ψ_M(\enum an)$$
  \suff
  \begin{displaymath}
    N⊧φ_M⇒N⊧ψ_M(\enum bn)
  \end{displaymath}
  Покажем, что $π\colon a_i→b_i$ --- изоморфизм.
  \begin{enumerate}
  \item $π$  -- инективен
  \item $π$  -- сюрьективен
  \item $M⊧ c = a_j ⇒ N⊧ c = b_j$
    \begin{petit}
      Далее в исходниках каша, поэтому то, что дальше -- не более, чем
      приблизительное графическое равенство.
    \end{petit}
    Получаем, что $π(c^M) = c^N$, т.к $c^M$ равно $a$ и $c^N$ равно
    $b$.
    \begin{displaymath}
      \begin{array}{c}
        M⊧P^k(\pargs ak) ⇔ N⊧P^k(\pargs bk) \\
        M ⊧f^k(\pargs ak) = a_j ⇔ N⊧f^k(\pargs bk) = b_j\\
        f^M(\pargs ak) \text{ равно } a_j ⇔ f^N(\pargs bk) \text{ равно } b_j
      \end{array}
    \end{displaymath}
    \begin{displaymath}
      N⊧\Th(M),\, φ_M\in\Th(M) ⇒N⊧φ_M ⇒N≅M.
    \end{displaymath}
  \end{enumerate}
\end{proof}

\begin{df}
  $ T_M ≔ \hc{φ_{M,Ω'} \,|\, Ω' \text{ -- конечна, } Ω' ⊆ Ω }$
\end{df}

\begin{lemma}
  $N ⊧T_M ⇔ N≅M$
\end{lemma}
\begin{note}
  Заметим, что это есть обобщение предыдущей леммы, однако мы не
  требуем конечности модели или сигнатуры.
\end{note}
\begin{proof}
  \need Заметим, что $M⊧T_M$ $⇔$ $∀ Ω'⊆Ω\;M⊧φ_{M,Ω'}$. Применяем
  доказанную лемму для конечного случая и это наблюдение, получаем
  \begin{displaymath}
    N≅M ⇒N⊧φ_{M,Ω'} ⇒N⊧T_M
  \end{displaymath}
\end{proof}
\end{document}
