\documentclass[a4paper]{article}
\usepackage[utf,simple]{dmvn}

\setcounter{section}{1}
\begin{document}

\section*{Задачи по экономике}

\subsection*{II семестр, 2003 г. Преподаватель~--- Д.\,Медведко}

\subsection*{29.03.2003}

\begin{problem}
Известно, что средняя урожайность зерновых в Англии
20 млн центнеров в год, средняя цена 40 шиллингов за четверть
(12.7 кг). В неурожайный год количество зерна
снижается на 30 проц, цена возрастает на 16 проц. Найти функцию
цены от количества зерна, если считать, что она линейна.
\end{problem}
\begin{solution}
Выберем удобные единицы измерения: цена --- шиллинги,
количество зерна --- кг. Выразим среднее количество зерна:
$20 \cdot 10^6 \cdot 100 = 2 \cdot 10^9$ кг. Выразим цену:
$\frac{40}{12.7} \approx 3.1496$ шиллингов за кг. В неурожайный год
цена составит $1.16 \cdot 40=46.4$ шиллинга, а количество упадёт
до $0.7 \cdot 2 \cdot 10^9 = 14 \cdot 10^8$ кг. Отсюда угловой коэффициент
составит $$K=-\frac{46.4-40}{2\cdot 10^9-14 \cdot 10^8} = -\frac{16}{15}\cdot 10^{-8}.$$
Смещение определим так:
$$B=40 - K\cdot 2 \cdot 10^9=40+\frac{16}{15}\cdot 20 = 61.(3).$$
Тогда искомая функция будет $P(Q)=K\cdot Q+B$.
\end{solution}

\begin{problem}
Во время работы консультантом на фанерной фабрике Леонид Канторович
установил, что зависимость издержек, приходящихся на единицу выпускаемой
продукции, от объёма этой продукции выражается функцией $C(Q) = 2Q^2 - 10Q + 20$,
где $C$ --- средние издержки, а $Q$ --- количество выпускаемой продукции.
Определить, при каком уровне выпуска средние издержки фабрики будут минимальными.
\end{problem}
\begin{solution}
Очевидно, минимум квадратичной функции будет в точке $Q=2.5$, но т.к. $Q$ выражает
количество произведённой продукции, оно должно быть натуральным числом. Тогда, выбирая
наиболее близкое целое решение, получаем $Q \in \hc{2;\;3}$.
\end{solution}

\begin{problem}
Иван отказался от работы столяром с зарплатой 25 тыс руб в год и отказался от работы
референтом за 45 тыс руб в год. Вместо этого он поступил в колледж с годовой платой за
обучение в размере 30 тыс руб. Половину платы за обучение ему компенсируют родители.
Определите, какова упущенная выгода за 1 год обучения, и какова упущенная выгода за весь
курс обучения (4 года), если в последний год обучения Иван имел возможность параллельно
работать референтом с зарплатой 60 тыс руб.
\end{problem}
\begin{solution}
Очевидно, за первый год Иван потерял $45+30-15=60$ тыс руб, а за 4 года
потери составят $60\cdot 3-60+60=180$ тыс руб. Примечание: из условия не вполне ясно,
действительно ли он работал референтом на 4-й год учёбы, или нет. Если нет, то он потерял
не 180, а $180+60=240$ тыс руб.
\end{solution}

\begin{problem}
На путешествие самолётом из Москвы в Париж менеджер тратит сутки с учётом
сопутствующих затрат времени. Поездка в поезде занимает двое суток. Авиабилет стоит 300 долл,
а железнодорожный билет - 180 долл. Определите, какой способ передвижения дешевле для
менеджера, зарабатывающего 150 долл в день, если путешествие совершается в рабочий день.
Определите, при каком дневном заработке менеджера ему будет безразлично, каким видом
транспорта добираться из Москвы в Париж.
\end{problem}
\begin{solution}
Разъезжая на поезде, менеджер теряет $180 + 2\cdot 150=480$ долл за поездку,
а на перелёте теряет $300+1\cdot 150=450$ долл. Итак, самолёт выгоднее на 30 долл. II.
Пусть менеджер получает $x$ долл в день. Тогда $180+2x=300+x$, откуда $x=120$ долл. Итак,
при 120-долларовом дневном заработке ему всё равно. Но летать-то приятнее...
\end{solution}

\begin{problem}
После окончания 11-го класса была возможность пойти работать в магазин продавцом
с оплатой 8000 руб в месяц или поступить в техникум, где платят стипендию --- 500 руб
в месяц. Третья возможность --- пойти учиться в ВУЗ с годовой оплатой 30000 руб. Найдите
альтернативную стоимость каждого решения в расчёте на 1 год.
\end{problem}
\begin{solution}
Посчитаем выгоду от каждого решения. Магазин: $+8000 \cdot 12 = +96000$ руб, техникум:
$+500 \cdot 12=+6000$ руб, ВУЗ: $-30000$ руб. Альтернативные стоимости соответственно будут
равны 0, 90000 и 126000 руб.
\end{solution}


\subsection*{12.04.2003}

\begin{problem}
Даны функции спроса и предложения:
$$Q_d=3000-1.5P,~Q_s=3.5P-600.$$ Определить равновесную цену.
\end{problem}
\begin{solution}
Равновесная цена --- цена в точке пересечения графиков спроса и
предложения. Решим уравнение
$$3000-1.5P=3.5P-600,$$ $$3600=5P,$$
откуда $P=720$.
\end{solution}

\begin{problem}
Даны функции $Q_d=7-P$ и $Q_s=-5+2P$. Определить равновесную цену и
равновесный объём продаж. Что произойдёт на рынке, если
правительство установит цену $P=3$?
\end{problem}
\begin{solution}
Имеем $7-P_E=-5+2P_E$, откуда $P_E=4$. Если правительство установит
$P=3$, то спрос составит $Q_d=7-3=4$, а предложение составит $Q_s=-5+2\cdot 3=1$,
таким образом, возникнет дефицит товара в размере 3 единиц.
\end{solution}

\begin{problem}
Даны значения спроса по цене:
$$Q_{d_0}=10,~P_0=300,~~Q_{d_1}=30,~P_1=200.$$ Определить
эластичность спроса по цене, степень эластичности, изменение
выручки продавца вследствие снижения цены. Посчитать то же при
$$Q_{d_0}=10,~P_0=300,~~Q_{d_1}=12,~P_1=200.$$
\end{problem}
\begin{solution}
Процентное изменение спроса есть
$$\Delta Q_{proc}=\frac{\Delta Q}{Q_{avg}}=\frac{30-10}{\frac{30+10}{2}}\cdot 100=100,$$
процентное изменение цены
$$\Delta P_{proc}=\frac{\Delta P}{P_{avg}}=\frac{300-200}{\frac{300+200}{2}}\cdot 100=40.$$
Формулы из [Economics, 15-th Edition, Paul A. Samuelson, William D. Nordhaus].
Отсюда эластичность равна $$E=\frac{\Delta Q_{proc}}{\Delta P_{proc}}=\frac{100}{40}=2.5>1.$$
Следовательно, спрос эластичен. Изменение выручки $$\Delta S = 30 \cdot 200 - 10 \cdot 300 = 3000.$$
Для второй ситуации:
$$\Delta Q_{proc}=\frac{12-10}{\frac{12+10}{2}}\cdot 100= 18.(18),$$
$$\Delta P_{proc}=\frac{300-200}{\frac{300+200}{2}}\cdot 100=40,$$
$$E=\frac{18.(18)}{40}=0.(45)<1.$$
Следовательно, спрос неэластичен. Изменение выручки $$\Delta S = 12 \cdot 200 - 10 \cdot 300 = -600.$$
Мораль: при неэластичном спросе снижение цен при прочих равных условиях приводит к убыткам
для предприятия, а при эластичном~--- к повышению прибыли.
\end{solution}


\subsection*{26.04.2003}

\begin{problem}
При цене 5 руб за килограмм величина спроса на картошку за день на базаре составляет
200 кг. Эластичность спроса на данный продукт равна $-2$. Найдите величину спроса при цене 7 руб за
килограмм.
\end{problem}

\begin{solution}
Пусть искомая величина спроса есть $x$ килограммов. Очевидно, при большей цене
спрос не может возрасти, поэтому $x < 200$. Тогда составим уравнение
$$\frac{2(x-200)}{x+200}: \frac{2(7-5)}{7+5}=-2.$$
Отсюда $x=100$.
\end{solution}

\begin{problem}
Функция спроса задана уравнением $Q_D(P)=2400-6P$. Выведите формулу точечной
эластичности этого спроса. При какой цене эластичность спроса по цене составит $-0.5$?
\end{problem}

\begin{solution}
По определению точечной эластичности, эластичность в точке $P_0$ равна
$$E(P_0)=\lim_{P \rightarrow P_0} \left(\frac{Q(P)-Q(P_0)}{Q(P)+Q(P_0)}:\frac{P-P_0}{P+P_0}\right).$$
Теперь наша задача тривиальна:
$$E(P_0)=\lim_{P \rightarrow P_0} \left(\frac{2400-6P-2400+6P_0}{2400-6P+2400-6P_0}:\frac{P-P_0}{P+P_0} \right)=
\lim_{P \rightarrow P_0} \left(-6\frac{P+P_0}{4800-6P-6P_0}\right)=\frac{P_0}{P_0-400}.$$
Теперь найдём $P$ такое, что $E(P)=-0.5$:
$$\frac{P}{P-400}=-\frac{1}{2} \Rightarrow P = 133.(3).$$
\end{solution}

\begin{problem}
Кривая спроса на продукцию монополиста описывается уравнением $Q=600-P$.
Монополист установил такую цену на товар, при которой точечная эластичность спроса на
него равна $-2$. Найти выручку монополиста.
\end{problem}
\begin{solution}
Запишем формулу точечной эластичности, используя накопленный опыт от задачи 9:
$$E(P)=\frac{P}{P-600}.$$
Отсюда эластичность $E(P)=-2$ будет наблюдаться при цене $P=400$. Тогда выручка монополиста
составит $S=400 \cdot (600-400)= 80000$ единиц.
\end{solution}

\begin{problem}
Даны следующие индивидуальные предложения субъектов рынка:
$$Q_1(P)=\case{
0,     & P < 2; \\
16+4P, & 2 \le P \le 6;\\
40,    & P > 6;} \quad\quad
Q_2(P)=\case{
0,     & P < 3;\\
10+4P, & 3 \le P \le 5;\\
40,    & P > 5.}$$
Определить функцию суммарного рыночного предложения, считая, что на данном рынке
присутствуют только эти два игрока. Найдите [точечную] эластичность предложения в
точке $P=3.5$.
\end{problem}
\begin{solution}
Надо полагать, что эти два игрока имеют независимые источники факторов производства,
в противном случае мы ничего не можем сказать о их совместном поведении. Итак, очевидно,
что суммарное предложение есть просто сумма наших кусочно заданных функций:
$$Q(P)=\case{0, & P < 2 \\ 16+4P, & 2 \leqslant P < 3 \\
26+8P, & 3 \le P \le 5 \\ 56+4P, & 5 < P \leqslant 6 \\ 80, & P > 6}$$
Найдём точечную эластичность предложения в точке $P=3.5$, учитывая, что в этой
точке действует закон изменения предложения $Q(P)=26+8P$:
$$E(P)=\frac{4P}{13+4P}=\frac{4\cdot 3.5}{13+4\cdot 3.5}=0.(518).$$
\end{solution}

\begin{problem}
До введения налога функция предложения на рынке прохладительных напитков
описывалась уравнением $Q_s=10P-100$, где $P$ -- средняя цена в рублях, а
$Q_s$ выражается в миллионах литров. Вывести уравнение функции предложения, если:
1. Правительство вводит акцизный налог, равный 1 р, взимаемый с каждого проданного
литра прохладительных напитков. 2. Правительство вводит налог с продаж в размере
7 процентов к цене проданных товаров (в базу расчёта налога с продаж включаются
все остальные налоги). 3. Введены оба налога одновременно. 4. Помимо первых двух
налогов введён НДС в размере 10 процентов (в базу расчёта НДС другие налоги не
включаются).
\end{problem}
\begin{solution}
Используем следующие соображения. Пусть была цена $P$. На неё накрутили
налогов: $P_T=F(P)$. Тогда надо подставить в $Q_s$ вместо $P$ его выражение
через $P_T$, т.е. $P=F^{-1}(P_T)$. Действительно, если на рынке установится
цена, численно равная накрученной, то, когда налоги сдерут, величина предложения
останется такая же, как и была. Вообще говоря, вполне логично.

1. Ясно, что в целях компенсации убытков от акциза надо поднять цену на
1 р. Тогда $$Q_s=-100+10(P-1)=-110+10P.$$

2. Очевидно, надо завысить на 7 процентов цену
для компенсации: $$Q_s=-100+\frac{10P}{1.07}=-100+9.346P.$$

3. Сначала прибавим акциз, а на него навесим налог:
$$Q_s=-100+10\hr{\frac{P}{1.07}-1}=-110+9.346P.$$

4. Аналогично (3):
$$Q_s=-100+\frac{10\hr{\frac{P}{1.07}-1}}{1.1}=-109+8.496P.$$
\end{solution}

\end{document}
