\documentclass[a4paper,12pt]{article}
\usepackage[cp1251]{inputenc}
\usepackage[russian]{babel}
\usepackage{amssymb,amsmath,mathrsfs}
\usepackage{../dmvn}

\begin{document}
\begin{center} {\LARGE Economy and finance: mathematical methods in economics} \end{center}

I would like to speak of some mathematical methods that are involved in economics.
I've chosen this aspect because it's closer to me than others. Now almost all branches of science are connected with mathematics to some extent,
and economics is not an exception.

There are a lot of problems in economics that are related to the theory of probability, differential equations and the theory of optimization and
approximation.
The first one is used to predict changes of the costs of currencies and stocks. Differential calculus helps to formulate and prove
economic some laws (for example, relation between demand and supply, etc.) And the third mentioned branch of mathematics is connected with very old
problem which was not solved by economists of the past, is the classical transporting problem. It's usually formulated in following way:
we have some mining factories situated in different points of the country; this factories produce some kind of resources. The question of the problem is:
where should we build a manufacturing plant that utilizes all types of our resources, to reduce transporting costs to minimums?

If we try to formalize all this conditions, we'll get a system of equations and inequalities. Of course, it depends on quantity of factories and other
parameters that we are taking into account. For example, we can add some conditions that are dictated by quality of the transport routes between plant and
factories. Or, we can take into consideration costs of transporting of each type of resource. Anyway, the result will be a some set of equations.

Our main goal is not only to translate our task into mathematical language. Actually the real task is to solve it. But there is no method in mathematics
which can be used to solve any system of equations, and in most cases it's not necessary. Instead of this we can approximate our system using equations
of simpler type. These ideas were used by russian mathematician Leonid Cantorovich, who was first to apply linear algebra methods in economics.
By the way, he got a Nobel prize in the field of economics for his investigations. He proved that the most of the tasks described above could be approximated
by linear equations, and the difference between approximate and exact solutions won't be very large. This makes a way to solution: as we know,
a system of linear inequalities defines a polyhedron in affine space, and we have a special method to find a minimum of a function, if it's domain is a
polyhedron. This is called \textit{linear programming}.

The main theorem in this theory states that the minimal value of our function can be found only in vertices of the polyhedron, not on the edges.
So the algorithm of searching for minimal vertex will be following: To begin with, we take any vertex on the polyhedron. Then shift it along that edge,
which corresponds to decreasing value of the function. Minimal vertex is found if and only if we come to vertex $A$, where values on neighbor edges are
greater than in $A$.

So, I've observed one of the application of maths in economics. Of course, it's not the only one, but it seemed to me as most understandable and easiest
to describe without any complicated formulas, terms, etc.  Now we can make one but very important conclusion: there is a bit of mathematics in all of the
sciences.






\end{document}
