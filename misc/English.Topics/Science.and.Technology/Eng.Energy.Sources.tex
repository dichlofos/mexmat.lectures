\documentclass[a4paper]{article}
\usepackage[simple]{dmvn}

\begin{document}
\cent{{\Large <<Science and Technology>>\\Energy Sources\par}}

Solar energy is one the most resourceful sources of energy for the future. One of the reasons
for this is that the total energy we recieve each year from the sun is around 35,000 times
the total energy used by man. However, about 1/3 of this energy is either absorbed by the
outer atmosphere or reflected back into space (a proccess called albedo)1.
Solar energy is presently being used on a smaller scale in furnaces for homes and to heat up
swimming pools. On a larger scale use, solar energy could be used to run cars, power plants,
and space ships (like the picture you see above).

Geothermal energy is an alternative energy source, although it is not resourceful enough to
replace more than a minor amount of the future's energy needs. Geothermal energy is obtained
from the internal heat of the planet and can be used to generate steam to run a steam turbine.
This in turn generates electricity, which is a very useful form of energy.

The radius of the Earth is about 4000 miles, with an internal core temperature of about 4000
degrees celsius at the center. The mantle surrounds the outercore and is only about 45 miles
below the surface, depending on location. The temperature at the mantle-surface crust boundary is
about 375 degrees, celsius. (This is too deep to get to\dots as of today)

So, how does this help us? It turns out that if we drill down only three miles we can reach
temperatures of 100 degrees, celsius, which is enough to boil water to run a steam-powered electric
powerplant. Drilling three miles through the earth is possible, but not easy, so luckily there are
easier routes to access this power source, known as geothermal hotspots.

Geothermal hotspots are volcanic features which are found all around the world. Basically a hotspot
is an area of reduced thickness in the mantle which transmits excess internal heat from the interior
of the earth to the outer crust. These hotspots are well known for their unique effects on the surface,
such as the volcanic islands of Hawaii, the mineral deposits and gyesers in Yellowstone National Park,
or the hotsprings in Iceland. These geothermal hotspots can easily be used to generate electricity.

Some systems pump hot-water into permeable sedimentary hospots found underground and then use the steam
to generate electricity. Then the used steam is condensed and sent back down to the permeable
sedimentary stream. Another system utilizes volcanic magma which is still partly molten at around
650 degrees, celsius, to boil water which would generate electricity. Also there is a system which
uses hot dry rock, which is just hardened magma, but still is extremely hot. To recover this heat
from these rocks, a system is used which circulates water through the rock and transfers the heat
up to a steam generator. The first system listed here is not as useful as other methods because of
the acidic nature of the fluids found under the ground. These acidities require a lot of maintenance
and upkeep on the equipment, and this cost reduces the economic effectiveness of the system.
Therefore, geothermal energy systems are more inefficient than other alternative energy sources because
of the costs required in upkeep and the shortage of potential sites.

Even the power of the tides can be harnessed to produce electricity.

Similar to the more conventional hydroelectric dams, the tidal process utilizes the natural motion
of the tides to fill reservoirs, which are then slowly discharged through electricity-producing turbines.
The former USSR produced 300 MW in its Lumkara plant using this method.

Hydroelectricity comes from the damming of rivers and utilizing the potential energy stored in the water.
As the water stored behind a dam is released at high pressure, its kinetic energy is transferred onto
turbine blades and used to generate electricity. This system has enormous costs up front, but has
relatively low maintenance costs and provides power quite cheaply. In the United States approximately
180,000 MW of hydroelectric power potential is available, and about a third of that is currently
being harnessed.
\end{document}