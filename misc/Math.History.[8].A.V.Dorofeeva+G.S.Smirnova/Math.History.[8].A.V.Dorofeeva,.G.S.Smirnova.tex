\documentclass[a4paper,oneside,fleqn,10pt]{article}
% ----------------------------------
\usepackage[utf8]{inputenc} \usepackage[russian]{babel}
\usepackage[simple]{dmvn} \usepackage{color} \usepackage{epigraph}
\tocsubsectionparam{3em} \tocsubsubsectionparam{3.4em}

\newcommand{\isp}{\Psi} \newcommand{\vr}{\vrule height 11pt depth 6pt
  width 0pt} \newcommand{\phz}{\phantom{0}}
\newcommand{\pe}[2]{${#1}\ldots{#2}$}
\newcommand{\cpic}[1]{$$\epsfbox{pictures.#1}$$}
\newcommand{\hwbox}[2]{\hbox to #1{
    \hfil\hbox{\vtop{\hsize=#1\advance\hsize
        by-10pt\relax\noindent#2\vskip8pt}}\hfil}}
\newcommand{\dx}{\,dx} \newcommand{\dy}{\,dy}

\begin{document}
\dmvntitle{Курс лекций} {по истории математики} {лекторы --- Алла
  Владимировна Дорофеева} {IV курс, 8-й семестр} {Москва, 2012г}
\section*{Предисловие}
\epigraph{Дела давно минувших дней, \par Преданья старины
  глубокой}{А.С.Пушкин}

Данные лекции записываются мной в реальном режиме прямо на лекциях, и
потому текст далеко не всегда страдает связностью и понятностью.
После лекции в нём исправляются опечатки и несуразности, но почти
никакой стилистической правки не вносится. Это конспекты, а не книжные
тексты.

Проблема с картинками очевидна (я не умею рисовать в MetaPost'e с
такой скоростью), но по возможности мы будем её локализовывать после
лекций при наличии времени).  Посему к большинству картинок я пишу
доступные текстовые пояснения.  Если кто-то считает, что он хорошо
рисует картинки, дайте мне конспект, я его отсканирую и вставлю их в
eps файлах, хотя это не очень красиво.

Также, некоторые вопросы были пополнены, используюя данные википедии.
Вики ни в коем случае не является авторитетным источником, вас
предупреждали.
\subsection*{Обозначения}

Фрагменты, в которых лектор ошибался, говорил что то неразборчивое,
выделяются квадратными скобками, иногда ставится знак вопроса.

О нумерации дат. Я очень часто буду указывать отрицательные годы и
века, игнорируя стандартную римскую нумерацию, естественно,
подразумевая под этим просто соответствующие века и годы \emph{до
  нашей эры}).  Для пущей ясности, перед числом явно ставится унарный
плюс, если нужно подчеркнуть, что это именно наша эра.

Кроме того, знак тильды перед годом обозначает неточности в её
определении.  Знак ${+}{+}$ после года обозначает <<число, большее
указанного>>.

Об используемом лексиконе. Не считайте меня некультурным человеком, но
я иногда буду сохранять просторечные (но вполне цензурные) выражения в
этом тексте, чтобы не так скучно было его читать. Ничего не значащие
выражения типа <<ну-с, поехали дельше>>~ отчасти мои, отчасти
лекторские комментарии, набираются скорее по инерции, чтобы не думать,
поэтому часть таковых будет оставаться для уменьшения сухости текста.

\subsection*{Благодарности}

Я глубоко признателен Юре Малыхину за то, что он взял на себя труд
читать это и даже исправил грубые математические ляпы.

\subsection*{Пожелания от автора}

Замечания и предложения, а также всяческая помощь, в том числе
техническая, приветствуется.  Я буду рад, если кто-то возьмётся
дополнять тексты упущенными мною фрагментами (а равно как исправлять
ошибки в набранном тексте).

Хотелось бы внести правки в список литературы, а то как-то некрасиво
смотрится!  Народы, у кого сохранились лекции, напишите правильный
список литературы, без глюков, пожалуйста. Это важно.

\subsection*{Ботва с датировками}

В общем, не знаю, что уж там на Фоменко валят, но вот наши лекторы уж
точно его переплёвывют.  Я проанализировал датировки, и получилось,
что, например, тов. Платон жил примерно в течение 900 лет (нехило,
правда?).  Датам верить нельзя, но и выверять их все тоже
немыслимо. Думаю, не добавить ли годы жизни к самим именам, чтобы лажа
всплывала в индексе?  Часть косяков я исправил, но их ещё, наверное,
очень много.

\subsection*{О литературе}

Понятно, что никакой литературы для подготовки этого текста не
использовалось.  Это просто почти дословная стенограмма потока
сознания, исходящего от лектора.  Однако, на стенде кафедры истории
математики и механики имеется некоторый список позиций на~20, не
меньше. Наверное, именно эта литература используется лекторами при
подготовке к лекциям. В общем, если кому интересно, посмотрите, а
переписывать его сюда не полностью не считаю нужным.

\rightline{\emph{Миша}}
\medskip
\dmvntrail


\tableofcontents

\marklec{01}{07.02.06}

\section{Об истории математики вообще}

\subsection{Математики о математике}

Для начала приведём несколько цитат известных математиков, которые
что-то говорили о математике.

\auth{Хинчин}: Математика не естественная наука. Обычно есть свой
объект исследования, биология и физика. У математики другая структура
исследования, никаких внешних требований, широкий круг применения,
неэкспериментальная теория доказательств.

\auth{Аристотель} (о метафизике). Математика исследует, отсекая всё
чувственное, оставляя только количественное и непрерывное.

\auth{Декарт}: математика это порядок и мера.

\sauth{Леонард Эйлер}{Эйлер}: Математика: наука о величинах и их
измерении, величин много.  Вообще Эйлер получил очень много
результатов, но он их не доказывал.

\sauth{Бертран Рассел}{Рассел}: математика не знает о том, что она
говорит и о чём она говорит.

\sauth{Анри Пуанкаре}{Пуанкаре}: Математика это искусство называть
разные вещи одинаковыми именами и одинаковые вещи разными именами.

\auth{Вейль}: Математика наука о бесконечности. Конечный человек
достигает с помощью конечных символов бесконечности.

Начало XX века, Франция. Группа \sauth{Бурбаки}{Бурбаки, Н.}
(псевдоним <<Н.\,Бурбаки>>) \auth{Дьедонне}, \auth{Гротендик} и
др. Это была группа, которая взялась изложить современную математику.

\auth{Энгельс}: математика \authorcomment{что-то там делает} с
пространственными формами и количественными соотношениями.

\sauth{Исаак Ньютон}{Ньютон}, \sauth{Готфрид Вильгельм
  Лейбниц}{Лейбниц, Готфрид Вильгельм} изобрели классическое
дифференциальное исчисление.

\auth{Лейбниц}: полезно знать истинное происхождение открытия; оно
важно тем, что познание метода ведёт к совершенствованию метода
открытия.

\auth{Ньютон}: если я и увидел что-то больше других, то только потому
что стоял на плечах гигантов.

Идея от \sauth{Колмогорова}{Колмогоров}: в историческом исследовании
нужно отступать от настоящего времени лет пятьдесят, потому что более
современное нельзя осознать и правильно оценить и осмыслить.

\subsection{Математика и её развитие в общем и целом}

\subsubsection{Преданья старины глубокой}

Вавилон: первые источники это глиняные таблички (\pe{-30}{-20}~века),
времена \auth{Хамураппи}.

Египет: древний Египет: папирус \sauth{Ахмеса}{Ахмес}.

Китай: первые источники порядка $+1$~века. У них там случилась
<<культурная революция>>, и все более древние источники были
уничтожены.

Индия: ведические знания.

\subsubsection{Периоды развития математики}

Первый период: \pe{-\bes}{-600}~годы. Это период накопления
математического материала, при это известно достаточно малое
количество источников.

Второй период: \pe{-600}{+600}~годы.  Это период развития античной
науки (в основном в Греции).  Характерно то, что в этот момент
появились первые доказательства, до этого момента существовали только
готовые решения, <<рецепты>>.

Потом всё это благополучно заглохло, и следующий всплеск произошёл
только в эпоху Возрождения.  Это был $+13$~век это, например, времена
\auth{Фибоначчи} (он же \auth{Леонардо Пизанский}).

Кроме того, существовала ещё и арабская математика.  Когда это всё
происходило, чёрт его знает, но ясно, что не раньше $+654$~года.  Там
был такой товарищ \auth{Аль Хорезми} (кстати, мы обязаны именно ему за
слова <<Алгоритм>> и <<Алгебра>>).

\subsubsection{Системы счисления}

\textbf{Египетская система} выглядела так \authorcomment{когда-то у
  нас было время, теперь у нас есть дела\ldots\ так что картинок с
  лягушками и лотосом пока нет\ldots}

\ctab{|c|c||c|c|}{ \hline $1$ & \epsfbox{pic-egypt-0.eps} & $10^4$ &
  \epsfbox{pic-egypt-4.eps}\\ \hline $10$ & \epsfbox{pic-egypt-1.eps}
  & $10^5$ & \epsfbox{pic-egypt-5.eps}\\ \hline $10^2$ &
  \epsfbox{pic-egypt-2.eps} & $10^6$ &
  \epsfbox{pic-egypt-6.eps}\\ \hline $10^3$ &
  \epsfbox{pic-egypt-3.eps} & $10^7$ &
  \epsfbox{pic-egypt-7.eps}\\ \hline}


\textbf{Римская система}:

\ctab{|c|c||c|c|} { \hline 1 & I & 100 & C\\ \hline 5 & V & 500 &
  D\\ \hline 10 & X & 1000 & M\\ \hline 50 & L & &\\ \hline }

\textbf{Вавилонская система} счисления: это была клинопись.
горизонтальный клин $1\cdot 60^n$, вертикальный клин $10\cdot 60^n$.

\textbf{Греческая ионическая система}:

%\centerline{
\begin{center}
\tab{|c|c|} { \hline $\alpha$ & 1\\ \hline $\beta$ & 2\\ \hline
  $\gamma$ & 3\\ \hline $\delta$ & 4\\ \hline $\varepsilon$ &
  5\\ \hline $?$ & 6\\ \hline $\zeta$ & 7\\ \hline $\nu$ & 8\\ \hline
  $\theta$ & 9\\ \hline }\hskip1cm \tab{|c|c|} { \hline $\iota$ &
  10\\ \hline $\ka$ & 20\\ \hline $\la$ & 30\\ \hline $\mu$ &
  40\\ \hline $\nu$ & 50\\ \hline $\xi$ & 60\\ \hline $o$ &
  70\\ \hline $\pi$ & 80\\ \hline $?$ & 90\\ \hline }\hskip1cm
\tab{|c|c|} { \hline $\rho$ & 100\\ \hline $\sigma$ & 200\\ \hline
  $\tau$ & 300 \\ \hline $\upsilon$ & 400 \\ \hline $\phi$ & 500
  \\ \hline $\chi$ & 600 \\ \hline $\psi$ & 700 \\ \hline $\om$ & 800
  \\ \hline $?$ & 900 \\ \hline }
\end{center}
%}
\authorcomment{Таблица одолжена с википедии. Там, где вопросительные
  знаки, это технические трудности поддержания обратной совместимости
  с \LaTeX}

\marklec{02}{09.02.06}

\section{Древнейшая математика}

Мы начнём с периода зарождения математики. Датировка примерно такая:
\pe{-\bes}{-6}~век. Конец этого периода соответствует появлению
математики в Древней Греции. Принципиальная разница была в том, что у
греков появились доказательства, а до этого момента были только
готовые рецепты решения задач.

В этот период математика развивалась в таких цивилизациях, как
Вавилон, Египет, Китай и Индия. В последних двух была проблема с
первоисточниками, которые по некоторым причинам плохо сохранились.  От
вавилонян до нас дошли глиняные таблички, а от египтян папирусы.

Предмет древней математики это в основном счёт. Измерение длин,
площадей (потому что нужно было строить и измерять земельные
площади). \auth{Геродот} пишет, что в Египте житель получал изначально
прямоугольный участок земли, но затем в силу разных причин он мог
изменить свою форму, и тогда нужно было пересчитывать его площадь,
чтобы платить правильную сумму налога.

Доказательств не было. В них просто не было потребности, потому что
математика решала весьма прикладные задачи, там не было проблем с
обоснованиями.

Развитие шло медленно, математика носила прикладной характер.
Например, нужно было проводить астрономические расчёты (календарь и
прочая фигня), вести торговый счёт; была необходимость математики в
строительном и военном деле.

Носители знаний жрецы (писцы) и чиновники (они вели статистику).


\subsection{Математика древнего Египта}

Это период примерно $-2000$ года. Самые первые папирусы, которые дошли
до нас это папирусы \sauth{Ринда}{Ринд} (размеры $5.25 \times 0.33$
м$^2$) и Московский папирус ($5.44 \times 0.08$ м$^2$).

У египтян была десятичная непозиционная система счисления.

Дроби в Египте были, и оперировали с ними, раскладывая всякую дробь в
сумму дробей с числителем~$1$.  (Иногда встречались дроби $\frac23$ и
$\frac34$, но это была крайняя редкость.)  Для оперирования с дробями
вида $\frac2n$ использовались таблицы разложения этих дробей в сумму
дробей вида $\frac1n$ с различными знаменателями. Эти таблицы были
доведены до $n = 101$. Например, $\frac23 = \frac12 + \frac16$;
$\frac25 = \frac13 + \frac1{15}$; $\frac27 = \frac14 + \frac1{28}$.

Деление натуральных чисел выполняли так. Возьмём, например, дробь
$\frac{37}{17}$.  Выделяем целую часть, получаем $2$. Затем оставшиеся
$\frac{3}{17}$ разлагаем в сумму: $\frac{3}{17} = \frac2{17} +
\frac1{17} = \frac1{12} + \frac1{51} + \frac1{68}$.  Ну и записываем
ответ в виде суммы $2$ и только что полученного разложения.

Умножение: перемножим, например, $12 \cdot 12$.  \ctab{|c|c|c|}{
  \hline & $1$ & $12$ \\ \hline & $2$ & $24$ \\ \hline $/$ & $4$ &
  $48$ \\ \hline $/$ & $8$ & $96$ \\ \hline $=$ & $12$ & $144$
  \\ \hline }

То есть, иначе говоря, сначала выписываем один множитель, умноженный
на все возможные степени двойки, а потом выделяем из них те, которые
составляют второй сомножитель (отмечаем их чертой слева), и числа при
них складываем.

Деление \authorcomment{на степени двойки, судя по всему} делали
так. Разделим, например, $19$ на $8$.

\ctab{|c|c|c|}{ \hline / & \vr$2$ & $16$\\ \hline & \vr$1$ &
  $8$\\ \hline & \vr$\frac12$ & $4$\\ \hline / & \vr$\frac14$ &
  $2$\\ \hline / & \vr$\frac18$ & $1$\\ \hline $\sum$ & \vr$2 +\frac14
  + \frac18$ & $19$\\ \hline}

Кроме дробей, египтяне возились с объёмами и площадями.  Вычисляли
площади прямоугольников, треугольников, объёмы цилиндров.  Так, для
площади круга использовалась формула: $S_\text{кр} =
\hr{\frac89d}^2$. То есть значение $\pi$ принимается приближённо
равным $4\cdot \hr{\frac89}^2 \approx 3.16$.

Усечённая пирамида называлась \emph{корзиной}.  Для вычисления объёма
корзины используется формула $V = (a^2 + ab + b^2) \frac h3$. В
папирусах имеется пример для $a =4$, $b=2$ и $h = 6$, получен ответ $V
= 56$.


Для приближённого вычисления площади произвольного четырёхугольника со
сторонами $a, b,c,d$ применялась формула $S = \frac{a+c}{2} \cdot
\frac{b+d}{2}$.

Неизвестные величины как правило назывались \emph{кучами} (<<аха>>).

\begin{ex} \textbf{(Пример 26 из папируса Ринда).}
Количество и его $\frac14$ часть дают вместе $15$.
\end{ex}

\begin{solution}
Возьмём $4$. Далее, $\frac14 \cdot 4 = 1$, вместе $5$.  Стало быть,
нужно взять побольше: $15:5 = 3$. Получаем $4 \cdot 3 = 12$.  Это и
есть ответ.
\end{solution}

\begin{ex} \textbf{(Пример из Московского папируса).}
$\frac34$ длины равны ширине. Найти стороны, если площадь равна $12$.
\end{ex}
\begin{solution}
Рассматриваем $12:\frac34 = 16$, потом извлекаем корень, получаем
$4$. Значит, одна из сторон равна $4$, а вторая, соответственно, $3$.
\end{solution}

\begin{ex}
Разделить $7$ хлебов на $8$ человек.
\end{ex}
\begin{solution}
Для этого использовалось разложение дроби $\frac78 =
\frac12+\frac14+\frac18$. Известно, что с практической точки зрения
такие расщепления достаточно экономны и требуют мало разрезов
(конечно, с точки зрения теоретической математики, можно было бы
каждый хлеб разделить на $8$ частей и раздать каждому по $7$ кусочков,
но это непрактично, а в то время людей интересовала не теория, а
практика).
\end{solution}

В Египте рассматривали также геометрические прогрессии.
\begin{problem}
Есть 7 домов, в каждом доме по 7 кошек, каждая кошка может поймать 7
мышей, каждая мышь может съесть 7 колосьев, каждый колос даёт по 7 мер
зерна; сколько всего?
\end{problem}
Получаем такую сумму: $7 + 7^2 + 7^3 + 7^4 + 7^5$.

В папирусе Ринда встречается и задача про арифметическую прогрессию:
\begin{problem}
Раздели 10 мер хлеба на 10 человек так, чтобы разность между хлебом у
каждого человека и следующего равна $\frac18$ меры.
\end{problem}

На нашем языке это означает, что $a_1 = x$, $a_{10} = x+\frac98$.
Ответ записывался в виде разложения дроби по степеням двойки.
Получаем уравнение $\frac{2x + \frac98}{2}\cdot 10 = 10$, откуда $x =
\frac{7}{16}$, а $a_{10} = \frac{25}{16} = 1 + \frac12 + \frac1{16}$.

Значительным достижением были пифагоровы тройки тройки целых чисел
$(x,y,z)$, удовлетворяющие уравнению $x^2 + y^2 = z^2$. В записях
встречаются $(3,4,5)$, $(60,45,75)$, и, более того, встречаются тройки
с четырёхзначными числами.

\subsection{Вавилон}

Датировка порядка $-3000$ года.  Вавилон -- город, существовавший в
Междуречье (сегодня Ирак, 90 км к югу от Багдада).

У вавилонян были глиняные таблички, на которых они писали клинышками.
Вавилоняне использовали 60 ричную систему счисления. Вертикальный клин
обозначает $1\cdot 60^n$, горизонтальный клин $10\cdot 60^n$.  При
этом всякий раз в записи числа минимальное $n$ приходилось определять
из контекста.

Поскольку умножение было весьма трудоёмкой операцией, были составлены
большие таблицы умножения. Существовали таблицы для $\frac1n$, $n^2$,
$n^3$, $n^2+ n^3$, $\sqrt n$.  При этом для числа $7$ в ранних
таблицах говорилось, что обратного элемента просто нет, а в поздних
таблицах появились приближённые значения.

В Вавилоне появились первые квадратные уравнения (полные).  Я вычел из
площади моего квадрата длину его стороны и получил $870$ (на
современном языке $x^2-x=870$). Рецепт решения: берём $1$, делим
пополам ($-\frac{p}{2}$), возведём в квадрат (получим
$\frac{p^2}{4}$). Прибавим $870$, извлекаем корень, сложим с
$\frac{1}{2}$, то есть фактически словесно описывается формула для
решения квадратного уравнения.

Первый успех теоретической мысли системы, приводящие к квадратным
уравнениям.  Именно, рассматривается система вида
$$\case{xy = S,\\x+y=p.}$$

Однако вообще уравнений рассматривалось крайне мало (всего $7$ штук).
Вот некоторые задачи, приводящие к (нелинейным) уравнениям. Все они
так или иначе возникали из геометрии и имели геометрическую
интерпретацию.

Дана площадь, дана гипотенуза. Найти катеты. Приходим к системе
$$\case{x^2 + y^2 = a,\\ xy = b.}$$

Кроме того, рассматривались такие системы:
$$\case{xy + (x-y)(x+y) = a,\\ x+y = b.}$$

$$\case{xyz + xy = 1\frac16,\\ y = \frac23 x,\\ z = 12x.}$$

\marklec{03}{14.02.06}

\section{Математика Древней Греции}

\subsection{О математике и математиках}

\textbf{Период:} \pe{-6}{+6}~века.

Это было то время, когда появлялись полисы Спарта, Афины, Милет.  В
полисах была демократия, а чтобы оказывать влияние на других, нужно
было уметь аргументировать, доказывать свою правоту логическими
рассуждениями. Только тогда можно было завоевать признание.

Традиционно в развитии греческой науки выделяют период древней Греции
(\pe{-5}{-3}~века).  К этому периоду относятся такие мыслители, как
\auth{Гиппократ Хиосский}, \auth{Зенон Элейский}, \auth{Менехм},
\auth{Фалес}, \auth{Пифагор}, \auth{Архит}, \auth{Феодор},
\auth{Евдокс} (придумал некоторое подобие теории действительных чисел,
которая очень напоминает систему дедекиндовых сечений).

Второй эпохой развития греческой науки является так называемая
\emph{Александрийская} эпоха, начавшаяся в $-3$ веке и закончившаяся
примерно в $+4$ м веке.  Тут можно назвать таких товарищей, как
\auth{Птолемей}, \auth{Герон}, \auth{Диофант}, \auth{Прокл},
\auth{Папп}, \auth{Гепатия Александрийская} и \auth{Теон
  Александрийский} (эти граждане жили в \pe{+1}{+4} веках).

\auth{Платон}, \auth{Аристотель} и \auth{Евдем Родосский}
(\pe{-4}{-3}~века).  \authorcomment{Касательно датировок исправлено
  много ошибок. Никакого $+5$ века не было!  Платон если и был
  комментатором, то совсем не позднейшим!}

Известнейший труд <<Начала>>, приписываемый \sauth{Евклиду}{Евклид},
скорее всего не был его монографией. Это было скорее совместное
творчество многих учёных того времени.

\textbf{Роль доказательства в математике.}

\begin{items}{-2}
\item Для установления истинности;
\item Необходимость отталкиваться только от аксиом;
\item Для установления связи между объектами;
\item Для открытия новых утверждений.
\end{items}

\auth{Фалес Милетский} принадлежал к группе греческих математиков,
которая называлась \emph{Ионической школой}.  Он известен, в
частности, тем, что предсказал солнечное затмение ($-585$ год). Фалес
прожил довольно долгую жизнь (\pe{-640}{-550}).

Ещё были такие граждане, как \auth{Анаксимандр} и \auth{Анаксимент}.

\textbf{Некоторые факты геометрии, известные древним грекам.}

\begin{items}{-2}
\item Окружность делится диаметром пополам.
\item Вертикальные углы равны.
\item Признак равенства треугольников по стороне и углу.
\item Угол, опирающийся на диаметр прямой.
\end{items}

\subsubsection{Пифагор и его школа}

\auth{Пифагор Самосский} (\pe{-570}{-477}). Родился на острове Самос.
Насчёт даты смерти есть сомнения.

Они с \sauth{Фалесом}{Фалес} слегка пересеклись, поэтому у него был
шанс немного поучиться у великого геометра.  Мать Пифагора из знатного
рода (Портемия, от Анкая, вождя племени элегов, а тот сын Зевса).
Отца его звали Мнесарх, по происхождению он был чужестранец, по
профессии камнерез.  Но он был очень хорошим человеком, и потому его
тоже уважали.

Его имя происходит от слова <<пифия>> (жрица дельфийского
оракула). Опять-таки, в раннем возрасте Пифагор свалил из родной
страны, в 22 года укатил в Египет, потом провёл в Вавилоне 24 года в
общей сложности. Потом вернулся в Грецию, в связи с долгим отсутствием
многие его после приезда не узнали, но вскоре весть о приезде учёного
распространилась в Греции. Вообще он был весьма уважаемым в обществе
человеком.

Про него говорили, что он пророк и чудотворец. Например, говорят, что
однажды его укусила змея, а Пифагор укусил её, и змея сдохла.  Ещё
одно чудо по легенде, с ним поздоровалась река, сказав ему <<Привет,
Пифагор>>.  Про него говорили: <<высокий, статный, длинноволосый
эллин>>.

По возвращении из странствий народ устраивал множество встреч с ним,
слушали его выступления, и все его понимали, потому что говорил он
доходчиво.  Как рассказывали про \sauth{Колмогорова}{Колмогоров}, его
лекции для школьников понимали только студенты, студенческие лекции
понимали аспиранты, а на научных семинарах его не понимал никто.  А
вот Пифагор говорил понятно, так, что его понимали все, кто его
слушал.

У него появилась куча учеников. Однако в~$-517$~г. победил кто-то из
плохих правителей, и Пифагор свалил в г.~Кротон. Там на собраниях
(собиралось по 600 человек), Пифагор организовал Кротонское братство
(обсуждались там в основном наука, религия, политика). Потом, правда,
их разогнали, потому что политические взгляды пифагорейцев не одобряло
правительство.

\textbf{О происхождении слова <<математика>>.}

<<Знание>> по-гречески пишется мудрёными буковками примерно так:
$\mu\al\ta\eta\mu\al\tau\al$.  В Греции знание делилось на
4~части. Главное \emph{арифметика}, но это не наука о счёте.  Учёные
изучали законы, а не какие-то чиселки. Второе \emph{геометрия}.
Площадь измеряли рабы, а вот учёные люди интересовались
доказательствами фактов.  Третье \emph{гармония} (то есть музыка и её
теория).  Последняя часть науки <<математика>> это \emph{астрономия}.

\textbf{Теория чисел}

1) Учение о делимости. Про знания пифагорейцев писал
\auth{Аристотель}: <<числа элементы всего мира. И они их изучали>>.
Числа делили на чётные и нечётные, и вообще изучалась делимость чисел.
Греки знали, что если $xy$ четно, тогда либо $x$ чётно, либо $y$
чётно.

2) Учение о совершенных числах (напомним, что число называется
\emph{совершенным}, если сумма всех собственных делителей равна самому
числу).  Примеры: $6 \bw= 1 \bw+ 2 \bw+ 3$, $28 \bw= 1 \bw+ 2 \bw+ 4
\bw+ 7 \bw+ 14$.  Если простое число $p$ равно сумма степеней двойки,
то $2^k \cdot p$ будет совершенным. \authorcomment{Это неправда, как
  говорил Юра. На самом деле чётное число совершенно тогда и только
  тогда, когда оно равно $2^{p-1}\cdot (2^p-1)$, причём второй
  множитель прост. Я сам не проверял, но мы поверим математикам.}
Эйлер доказал, что других чётных совершенных чётных чисел не
существует, а про нечётные совершенные числа ничего не известно до сих
пор.

3) Фигурные числа. Была Книга о многоугольных числах (треугольные,
квадратные, тд).  Число называется \emph{треугольным}, если это число
точек в треугольнике в целочисленной решётке.  Например, такое число:
\begin{verbatim}
   * * * * * * * * * *
\end{verbatim}
(суммы первых членов арифметической прогрессии).  Далее,
\emph{квадратные} числа это просто квадраты натуральных чисел.

4) Отношение к единице: <<неделимое целое>>. Нельзя говорить о частях
единицы.  Каждая дробь это отношение целых, и только так.  Учение о
подобии (и в частности, методы деления отрезка в заданном отношении)
не выводило их за пределы целых (или в крайнем случае рациональных
чисел).

Греки всегда находили общую меру (приводили к кратному, и всё тут).
Короче, $\Q = \Z^2/\mskip-5mu\sim$.  И потому для них $\sqrt2$
оказался страшным психологическим препятствием, приведшим к тому, что
пифагорейцы бросили изучение чисел как таковых.

\textbf{Геометрия.}

1) Планиметрия изучение плоских фигур.  Это наиболее классическая
геометрия 2 я книга <<Начал>> \sauth{Евклида}{Евклид}.

2) Знали 5~правильных тел, в частности, пентагон и пентаграмму (как мы
знаем, это приводит к золотому сечению из решения уравнения $\frac ax
= \frac x{a-x}$).  Пифагорейцы настолько пропёрлись, что сделали это
символом своего братства.  Один раз так получилось, что одного из
пифагорейцев выручил какой-то человек, но отблагодарить было
нечем. Тогда пифагореец нарисовал на воротах дома пентаграмму, и когда
её впоследствии увидели другие пифагорейцы, то отблагодарили того
человека по справедливости.

\textbf{Гармония, музыка.} Звук. Они знали, что длина струны связана с
тем звуком, который она издаёт.  Круто, когда длины струн
согласованы. У вавилонян не было модели Мира, а у пифагорейцев была.
Космос нечто живое. Земля сама по себе, все планеты крутятся вокруг
её, а Земля стоит на месте, ибо зачем ей крутиться, она и так
главная. Говорили, что есть небесные сферы (9~штук). Последняя сфера
сфера неподвижных звёзд.  Говорили, что есть музыка небесных сфер,
которую могут слышать только избранные.  Это были первые, кто занялись
наукой, и они сказали, что весь мир является гармонией и числом.  Это
так называемая \emph{арифметизация} Вселенной.

\textbf{Сказка о страшном Корне из Двойки.}  Арифметика греков была
главной. Вскоре произошло событие, которое всё порушило, и чуть было
её не развалило. Когда они доказали, что корень из двух несоизмерим с
единицей, всё стало совсем плохо.  Диагональ квадрата вполне реальный
объект оказался несоизмеримым со стороной.

Приведём их доказательство, про которое пишет \auth{Аристотель}.

\begin{proof}
Пусть $\sqrt2 = \frac{p}{q}$ несократимая дробь. Возведём в квадрат.
Получим $\frac{p^2}{q^2} = \frac{AB^2}{AГ^2} = \frac{AB^2}{2AB^2}$,
отсюда получаем $2p^2 = q^2$. В предположении несократимости дроби,
получаем, что $q^2$ чётно, стало быть и $q$ чётно. А мы то думали, что
дробь нельзя сократить.
\end{proof}

На самом деле не все доказательства, приписываемые Пифагору, были его
собственными.  Это были совместные разработки пифагорейцев, и вообще
они всё своё творчество приписывали Пифагору за то, что он был сильно
крут (из уважения к нему) И, например, когда про платоновы тела вышел
базар с \sauth{Архитом}{Архит} на тему того, кто это придумал, Архита
прогнали из школы.

Ещё про иррациональные числа. Число $\sqrt2$ это первое, что они
заметили.  А потом \auth{Феодор} нашёл ещё кучу гадости. Улитка
Феодора это геометрический объект, порождающий последовательность
$\sqrt3, \sqrt5\sco\sqrt{17}\ldots$.  \auth{Теэтег} показал, что если
натуральное число не есть точный квадрат, то корень из него не будет
рациональным.  Рассматривает рациональности вида $\sqrt a \pm \sqrt
b$, и показывает, что с числами всё плохо.

\textbf{Геометрическая алгебра.}  И что же делать? Ну ладно, тогда
давайте плясать от отрезков. То есть не будем работать с числами, а
определим арифметику отрезков.  Сложение, вычитание (из большего
меньший), умножение отрезков (результат это прямоугольник, то есть
декартово произведение). Проблемы ясны: ограничение на размерность
(трёхмерные объекты не рассматривали) и ещё получалась полная ерунда,
если пытаться складывать объекты разных размерностей. Например, $x^2 +
x + 1$ это не то, что могли рассматривать греки.

Чтобы складывать прямоугольники, нужно их приводить к одной общей
стороне.  (так называемое \emph{квадрирование}, решение уравнений вида
$ab = x^2$). Но были существенные ограничения.  Именно, делить отрезок
на отрезок было нельзя, произведение трёх отрезков тоже брать нельзя.
Но был один плюс: при сложении и умножении не надо ломать голову над
тем, какими получаются эти длины (рациональными или иррациональными).
Всё это тоже было описано во второй книге <<Начал>>
\sauth{Евклида}{Евклид}.

Приложение к <<Началам>>: учение об аддитивности площадей
(дистрибутивность сложения относительно умножения).  Алгебраические
тождества: $(a+b)^2 = a^2 + b^2 + 2ab$. Куб суммы сумма двух кубов $+$
тройные произведения $+\ldots$

Были ещё пифагоровы тройки: $x^2 + y^2 = z^2$.  Но, конечно, полного
исследования решения этого уравнения в целых числах получено не было.
Это было сделано чуть позже.

Зато была геометрия отрезков. Это послужило толчком для развития
аппарата для геометрических построений.  Например, был хорошо известен
метод построения отрезка $\sqrt{ab}$. Откладываем отрезок~$a$, к нему
приклеиваем отрезок~$b$, на полученный отрезок натягиваем окружность,
как на диаметр, потом восстанавливаем перпендикуляр в точке
склейки. Длина полученного перпендикуляра до пересечения с окружностью
и есть отрезок длины~$\sqrt{ab}$.

А вот ещё одно тождество, хотя и очевидное, но имеющее геометрический
смысл: $ab \bw= \hr{\frac{a+b}{2}}^2 - \hr{\frac{a-b}{2}}^2$.
Квадратное уравнение: $x(a+x) = S$.  Интерпретация: нужно приложить
такой отрезок, чтобы площадь была заданной.  Они рассматривали избыток
и недостаток площадей (вторая книга <<Начал>>).  Естественно,
уравнения третьей степени не рассматривались.

Но они возникали. И вот, всё шло к тому, что греки наткнулись на три
знаменитые задачи древности: удвоение куба, трисекция угла и
квадратура круга. К ним мы и переходим, но это уже другая история.

\marklec{04}{16.02.06}

\subsection{Три классические задачи древности}

В Древней Греции была геометрическая алгебра. По сути это работа с
циркулем и линейкой. Греки откладывали отрезки и проводили
окружности. Если перенести на язык уравнений, то это линейные и
квадратные уравнения, а также системы из них.

Значит, если задача такова, что она не сводится к квадратному
уравнению, то она неразрешима в принципе.  Среди других выделились три
знаменитые, которые все пытались решать, но тщетно.  $-6$й век. Они
были поставлены в Греции, а потом их решали арабы, европейцы\etc

\begin{items}{-3}
\item Удвоение куба;
\item Трисекция угла;
\item Квадратура круга.
\end{items}

Эти задачи нельзя решить с помощью циркуля и линейки, но этого тогда
ещё не умели доказывать.  Поэтому создавались новые методы. Обогащался
математический аппарат.  Сейчас мы про это поподробнее поговорим.

Это был золотой век Афин. Это была демократия все собирались на
площади, и что то обсуждали. Женщины сидели дома, а рабы делали чёрную
работу.  А свободный народ вёл дискуссии и споры на тему политики и
т.п.  Поэтому нужно было уметь доказывать, а не как звери силой.
Поэтому было стремление к образованию, и был спрос на учителей
(софистов, мудрых людей).

Софисты учили риторике, философии, математике, астрономии. Долгие годы
потом это была основа образования риторика, философия. Светское
образование давалось именно софистами, а не просто учителями
арифметики.

Так, ну поймём, что дала каждая из этих задач в математике.

\subsubsection{Удвоение куба (делосская задача)}

Дело было в городе Делос. Там была чума. Ну вот, жертвенник в виде
куба, и его нужно удвоить. Они поставили на куб ещё один куб, но это
не покатило. Нужно было получить куб вдвое большего объёма, при этом
можно пользоваться только циркулем и линейкой. Сторона старого куба:
$a$, нового куба $x$, поэтому $x^3 = 2a^3$, откуда $x = a\sqrt[3]{2}$.

На Востоке решали задачу приближённо, а вот грекам нужно было именно
точное решение.  Менталитет был разным, и древним было пофигу, как оно
там, по таблицам или как, а вот грекам это уже было не всё равно.

Первые попытки что-то сделать с этой задачей делал, по-видимому,
\auth{Гиппократ Хиосский} ($-5$ век). Он сказал: $x^3 = a^2b$, $b =
2a$. Он делал \emph{вставки}: $\frac{a}{x} = \frac{x}{y} =
\frac{y}{b}$. Перемножаем: Так получаем $ay = x^2$ и $ xy = ab$. Это
соответственно, парабола и гипербола, но греки этого пока не
осознавали. Это было потом конические сечения
(см.~с.~\pageref{sec:conicae}).  После перемножения получаем $x^2
\cdot xy = a^2 by$, поэтому $x^3 = a^2b$. \authorcomment{Бред
  какой-то}

\subsubsection{Трисекция угла}

Разделить угол на три равные части с помощью циркуля и линейки.
Древние греки умели делить угол пополам. Арабы всё свели к кубическому
уравнению.  Тригонометрию они знали хорошо. По существу нам нужно было
вот что: $\sin \al = 3 \sin\frac\al3 - 4\sin^3\frac\al3$.  Полагая
$\sin\al = a$, а $\sin\frac\al3 = x$, получаем кубическое уравнение
$3x-4x^3 = a$.

Греки понимали, что как то по простому не получается, и изобретали
разные механические методы. Потом они уже поняли, что решается не
всё. Механика третий сорт, с помощью конических сечений это второй
сорт, а первый сорт это циркуль и линейка. Это точно, красиво,
аккуратно\ldots

Однако вот некоторый механический способ решения этой задачи.
Рассматриваем прямоугольник $ABCO$, при этом двигаем $AB$ параллельно
самой себе вниз до $OC$, а при этом $AO$ вращается равномерно до $OC$
по часовой стрелке.  Их точка пересечения прочерчивает одномерное
многообразие в прямоугольнике.

 \cpic{1}


Пусть они пересекаются в точке $M$.  Пусть отрезок образует с
вертикалью угол $\ph$.  Поскольку равномерно, то $\frac{\ph}{\ph_1} =
\frac{y}{y_1}$.  Кривая называется \emph{квадратрисой} (так её называл
\auth{Лейбниц}).

\auth{Леонардо Пизанский} ($+13$ век). Первая попытка доказать, что
уравнение $x^3 \bw+ 2x^2 \bw+ 10x \bw- 20 \bw= 0$ неразрешимо с
помощью циркуля и линейки \authorcomment{Как правильно заметил Юра,
  тут у лектора, была опечатка неправильный знак.}

\auth{Декарт} ($+17$ век) критерий отыскания кубических уравнений,
которые можно решить с помощью циркуля и линейки.

Строгое доказательство того, что угол нельзя разделить на три части, и
нельзя удвоить куб, было дано \sauth{Пьером Вантцелем}{Вантцель, Пьер}
в $+19$ веке (1837~г).  Всё это было уже тогда, когда появились
группы, кольца и поля. Вантцель (\pe{1814}{1848}) был меньшего
масштаба, по сравнению с \sauth{Гауссом}{Гаусс}.  Учился в Ecole
Polytechnique в Париже, был репетитором.

\subsubsection{Квадратура круга}

Построить квадрат, площадь которого равна площади круга радиуса~$1$.
Получалось уравнение $x^2 = \pi r^2$. Отсюда $x = \sqrt\pi \cdot R$.

А теперь посмотрим, что такое $\pi$?  Как его приближённо найти?
25~столетий, чтобы понять, что это трансцендентное число, и что его
нельзя построить отрезок такой длины с помощью циркуля и линейки.  Про
трудные задачи говорили: <<да что же это за задача, квадратура круга
какая то\ldots>>.


\textbf{Луночки Гиппократа.}  Эта задача была популярна в народе, все
про неё знали, и даже сейчас ещё маньяки находятся.  А почему думали,
что они её найдут? А вот почему: потому что \auth{Гиппократ Хиосский}
в $-5$м веке изобрёл \emph{луночки}.  Так вот, с ними всё было гораздо
лучше, но они никак не помогают решить задачу о квадратуре.

Что такое луночка Гиппократа? Рассмотрим окружность, от неё откусим
четвертушку.  Пусть $B$ центр круга, $AB= BC$, $A$ и $C$ точки на
окружности.  Имеем $AB = BC = r$. Тогда по теореме
\sauth{Пифагора}{Пифагор} $AC = r\sqrt2$.  На $AC$ как на диаметре
построим новую окружность. У луночки площадь $S_1$.  Луночка это то,
что вылазит из второй окружности. $S_2$ это круговой сегмент в большом
круге. А $S_3$ это площадь треугольника. Тогда имеем
$$S_2 + S_3 = \frac{\pi r^2}{4}.$$ \cpic{2}
$$S_1 + S_2 = \frac12 \pi \hr{\frac{r}{\sqrt2}}^2 = \frac{\pi
  r^2}{4}.$$ Тогда имеем $S_1 + S_2 = S_2 + S_3$. После сокращения
имеем $S_1 = S_3$.

Итак, Гиппократ нашёл три луночки, а потом, в 1840~году немецкий
математик \auth{Клаузен} нашёл ещё две луночки.  Окончательно вопрос о
луночках (1949~г.) был решён советскими математиками доклады АН
(\auth{Чеботарёв}, \auth{Дороднов}).  Доказано, что всего существует
5~луночек, и других нет.  Доказано средствами теории \auth{Галуа}.

Кроме того, сама задача о квадратуре решена с отрицательным
результатом даже раньше.  Это произошло после того как была построена
строгая теория алгебраических чисел. Тем самым доказано, что нельзя
построить квадрат той же площади.

\textbf{Приближения числа $\pi$.}  Естественно, кроме самой задачи о
квадратуре круга, пытались приближённо вычислять это число. Прогресс
имел место в Китае в $+5$~веке. А именно, астроном, инженер,
математик, \auth{Цзу Чун Чжи} (\pe{430}{501}) нашёл удивительное
приближение для $3.1415926 < \pi < 3.1415927$.

\sauth{Архимеду}{Архимед} принадлежит такое приближение:
$3\frac{10}{71} < \pi < 3 \frac{1}{7}$.  Это даёт приближение до
второго знака после запятой.

Немецкий математик \auth{Отто} в $+16$ м веке придумал, что $\pi$
можно приблизить дробью $\frac{355}{113}$.

А как приблизить $\pi$? Ну ясно, нужно вписывать в окружность
многоугольники, и считать их длину.

1427~год. \auth{Джемшид Аль Каши} написал знаменитый трактат <<Ключ к
арифметике>>. Он нашёл $17$~верных знаков числа $\pi$.  Родился в
Иране, работал в Самарканде в обсерватории Улугбека. Там была работа
таблицы, астрономия, астрология, и прочая наука. В Европе не было
известно об успехах арабов, и там начинали с меньшего числа знаков.

Ну а что же в Европе? \sauth{Франсуа Виет}{Виет}, замечательный
математик, изобрёл буквенное исчисление, перевёл астрономические
таблицы в десятичные дроби. В 1579~году взял правильный многоугольник
для $n = 393216 = 3\cdot 2^{17}$.  С помощью него вычислил
9~правильных знаков.

Хуже того, в 1596~году голландский математик \sauth{Лудольф ван
  Цейлен}{Цейлен, Лудольф ван} нашёл 20~точных знаков числа~$\pi$.
Его многоугольник имел $\sim32512\cdot 10^6$ сторон. В его рукописях
нашли ещё 15~знаков числа~$\pi$.  С помощью ЭВМ в 1961 году нашли
100\,625 знаков числа $\pi$.

\textbf{Природа числа $\pi$.}  В 1766~году \sauth{Иоганн
  Ламберт}{Ламберт} (немецкий математик) доказал, что $\pi$
иррационально.

1882~год. \sauth{Карл Линдеман}{Линдеман} (немецкий математик) и
\sauth{Шарль Эрмит}{Эрмит} (французский математик) доказали, что
$\pi$~трансцендентно, и таким образом решили проблему квадратуры
круга.

Как мы знаем из теории чисел, сходная проблема (о выразимости в
квадратурах) это построение правильных многоугольников с помощью
циркуля и линейки.  Приближённо их строили без проблем.

Сдвиг в решении проблемы произошёл, когда пришёл \sauth{Карл Фридрих
  Гаусс}{Гаусс} (\pe{1777}{1855}).  В~1801~году он написал книгу
<<Арифметические исследования>>.  Гаусс доказал, что нечётноугольники
строятся тогда и только тогда, когда $n = 2^k \cdot p_1 \sd p_k$, где
$p_i$ различные простые числа вида $2^{2^\al} +1$.  Случаи $\al = 0$ и
$\al = 1$ были известны грекам, а Гаусс разработал алгоритм для $\al =
2$.  Сначала Гаусс не получал признания, хотел уехать в Россию.

Он работал в обсерватории, и ему приходилось много считать. Он
задавался вопросом, как измерить Землю?  В частности, ему нужны были
расчёты для геодезии. Преподавать не любил, но практик был великий.
Много двигал теоретические вопросы. \auth{Абель} и \auth{Галуа}
использовали многие его идеи.

Гаусс завещал, чтобы на его могиле был изображён 17 угольник.  Так оно
и получилось. Постамент памятника Гауссу \authorcomment{кажется, этот
  памятник расположен в Гёттингене} имеет форму правильного 17
угольника.  Эта задача шла из Греции. Какая польза от математики?  Не
так важна роль того или иного результата. Главное его красота.

\marklec{05}{21.02.06}

\subsection{Эпоха эллинизма}

\auth{Филипп Македонский} завоевал довольно много к~$(-4)$ му веку. А
потом его сынишка, \auth{Александр Македонский}, пошёл ещё дальше,
завоевал почти всё до Индии.

В $-331$ году основана Александрия Александром Македонским.  Это и
есть начало эпохи \emph{эллинизма}. Государственный язык во всей
империи Александра греческий, во всём подражали грекам, и даже после
смерти Александра Македонского, когда почти всё рухнуло, и стали
править его военачальники, некоторое время оставался греческий язык.
Научная элита говорила на греческом, они то понимали друг друга, но
коренное население не понимало греческого, и ничего похожего на
деятельность софистов не происходило. Такое различие в языке привело к
смерти театра.  Зато филология развилась. Хоть что-то радует.

Итак, после смерти Александра правит династия Птолемеев, и первый же
\auth{Птолемей} основывает в Александрии \emph{Мусейон} (храм
муз). Это фактически был научный центр, самый крупный научный центр
того времени. Там были почти все крупные математики, за исключением,
разве что Архимеда, и ещё двух трёх других.

Поскольку это был новый город, на него и влияния было оказано немало.
Купцы, торговля и всё такое. В школах детей учили счёту по египетски,
гречески, а всё остальное -- по вавилонски, в 60 ричной системе
счисления (то, что касалось астрономии). Своих то традиций не было,
вот и заимствовали то, что было известно.

В нём же образовалась библиотека. Известное событие пожар в
Александрийской библиотеке в $-30$ е годы, когда туда пришли злые
римляне и всё сожгли, сволочи.  Там было более 700000 свитков, основа
знаменитая библиотека \sauth{Аристотеля}{Аристотель}.

Что ещё было в этом Мусейоне? Раньше были школы.  Были научные
организации (академия \sauth{Платона}{Платон}, Ликей
\sauth{Аристотеля}{Аристотель}).  Мусейон был организован правителем
государства и поддерживался правительством.  <<Обитатели>> Мусейона
преподаватели, профессора получали жалование.  И это было новым
элементом, потому что до этого момента содержание всех научных
сообществ было частным делом.  Во главе Мусейона стоял главный жрец,
назначаемый правителем самый крупный учёный того времени.  Важно, что
это были профессиональные преподаватели. В это время начали учить по
книгам, раньше этого не было. Кроме того, особенностью являлось и то,
что это научное сообщество фактически было заизолировано от основного
населения, которое вообще не разговаривало по гречески. Вот вкратце и
все отличия от предыдущих школ.

\textbf{Астрономия.} В древней Греции (например, в школе пифагорейцев)
была геоцентрическая система мира (9~небесных сфер, Земля в центре
всего).  В других теориях в центре всего было ещё что-нибудь (огонь,
например).

\auth{Евдокс} попытался построить модель, которая содержит 27~сфер,
при этом все они движутся.  Но и это было не верхом совершенства. В
древней Греции были и \emph{гелиоцентрические} системы мира.  Так что
\auth{Коперник} был не первым, кто рискнул предположить, что Земля
крутится вокруг Солнца, а не наоборот. Большая астрономическая работа
закончилась созданием звёздного каталога <<Альмагеста>>
\sauth{Птолемея}{Птолемей}.

\auth{Архимед} изучал статику и гидростатику.  \auth{Евклид},
\auth{Архимед} геометрическая оптика.  Было довольно развито учение о
каноне (то есть о музыке).  \auth{Эратосфен} с помощью довольно
неточных приборов ($колодец \bw+ верблюд$) измерил Землю (длину
градуса меридиана).

Ну да ладно\ldots\ Это всё ещё так, цветочки. А важно то, что такие
граждане, как \auth{Евклид}, \auth{Архимед} и \auth{Аполлоний}, были
тремя гениями эпохи эллинизма.

\subsubsection{Евклид}

\auth{Евклид} помер около $-270$ года. Больше ничего не известно.
Возможно это был Николя Бурбаки своего времени, но всё таки, видимо,
это был реальный man (комментаторы писали про него, например
\auth{Папп} пишет, что он был очень хорошим человеком).

<<К математике нет царской дороги>> сказал Евклид одному из
правителей.  Один ученик спросил своего учителя Евклида: <<А если я
выучу всё, что мне тогда будет?>> <<Получишь три копейки>>, ответил
Евклид \authorcomment{естественно, в греческой валюте, но по нашим
  меркам это примерно столько, то есть почти ничего}.

Ну да ладно, давайте про его науку.  Это автор <<Начал>>. На самом
деле мы изучаем уже только то, что до нас дошло, так что нет никаких
возможностей проверить, кто на самом деле автор.  \auth{Гиппократ
  Хиосский}, говорят, тоже что-то в них писал. Оригинал не дошёл до
нас (арабские копии <<Начал>> это \pe{+6}{+9}~века), латинские копии
($+11$~век).  Но копий много, и различий не так много, так что о
содержимом известно многое.

\textbf{Научные работы Евклида}

\begin{items}{-2}
\item <<Ложные заключения>> -- логические основы математики.
\item <<Данные>> -- что нужно задать, чтобы можно было решить задачу.
  (учение о степенях свободы или о системе параметров).
\item <<Конические сечения>> -- попытка решить задачу об удвоении
  куба.
\item Phenomena.
\item Оптика учение о перспективе.
\item Учение о каноне учение о музыке.
\end{items}

Полная теория конических сечений разработана \sauth{Аполлонием
  Пергским}{Аполлоний Пергский} (родом из Пергама).

Говорят, что <<Начала>> это энциклопедия. Но на самом деле она сильно
не полна. Это не всё то, что знали. Там нет знаменитых задач
древности, конических сечений и многого другого.

Целью \sauth{Евклида}{Евклид} было изложить базу, то, из чего строится
здание современной математики, что-то вроде книг группы
\auth{Бурбаки}.  Кстати, даже труд Бурбаки был похож по названию на
<<Начала>> те же <<Элементы математики>>.

Евклид излагает то, что было известно до него. Кроме того, в
<<Началах>> \auth{Прокл} излагает труды \sauth{Евдокса}{Евдокс}, при
этом даёт доказательства более строгие, подправляет предков.

<<Начала>> это 13~книг + ещё 2~книги, написанные позже (не совсем
ясно, кем, правда, написанные, но есть некоторая общепринятая точка
зрения). В каждой книге даны определения.  Перед первой книгой даны
определения, аксиомы, постулаты, чтобы было понятно, на чём базируется
рассуждение.

\textbf{Определения.}  Точка то, что не имеет частей.  Прямая то, что
длина без ширины, её концы точки (на самом деле, это определение
отрезка прямой).  Прямая (настоящая прямая) характерна тем, что ровно
расположена по отношению ко всем точкам.  Поверхность имеет только
длину и ширину.

\textbf{Постулаты.}

\begin{nums}{-2}
\item От всякой точки до всякой можно провести прямую.
\item Отрезок можно продолжить до всей прямой непрерывно.
\item Всяким раствором циркуля из любой точки можно описать круг.
\item Все прямые углы равны между собой, то есть продолжение прямой
  единственно.
\item Постулат о параллельных прямых: если сумма внутренних
  односторонних меньше с одной стороны, то там эти прямые и
  пересекаются.
\end{nums}

Неархимедова величина типа угла между окружностью и её касательной.
Это так называемые роговидные углы. Это то, что не удовлетворяет
принципу \sauth{Архимеда}{Архимед}: умножив его на достаточно большое
число, можно было получить число, большее~$1$.

Интересно то, что греки знали про то, что в сферическом треугольнике
сумма углов строго больше $\pi$.

\textbf{Аксиомы.}
\begin{nums}{-2}
\item Транзитивность равенства объектов.
\item ${a=b}$, ${c=d}$, тогда $a+c\bw=b+d$.
\item То же для вычитания.
\item Совмещающиеся равны между собой (намёк на то, что движение
  сохраняет расстояния и все метрические свойства).
\item Целое больше части.
\end{nums}

В отличие от постулатов, тут про геометрию ни слова. Это абстрактная
часть.  Каждый постулат считается \textbf{решённой} задачей.

Аксиома~4 слова о том, как доказывается равенство фигур конгруэнтность
при движении для равных фигур.

Понятно, что это не все аксиомы, чтобы из них вырастить всю
математику.  Как известно, \sauth{Давид Гильберт}{Гильберт} вновь
поднял базар об выборе аксиом, сказав, что геометрия это теория
инвариантов какой нибудь фиксированной группы преобразований
пространства.  Но, заметим, что у Евклида тоже получилось не так
плохо.

\subsubsection{<<Начала>>}

Для начала заметим, что <<Начала>> это не только геометрия.

Книги \textbf{1--4}: Планиметрия.

\textbf{1}: (треугольники, прямоугольники, трапеции\ldots)

\textbf{2}: Основы геометрической алгебры.

\textbf{3}: Круг, хорды, касательные.

\textbf{4}: Построения и правильные $n$ угольники, в том числе.  Греки
умели строить правильные 3-, 4-, $5$ угольники, но вот 7 угольник
\auth{Архимед} умел строить только с помощью механических
приспособлений.  Далее через уполовинивание сторон можно ещё много
чего настроить.  $15$ угольник \auth{Евклид} сделал сам, чем очень
гордился.

Знаменитый алгебраист \sauth{ван дер Варден}{Варден, ван дер} говорил,
что книги \textbf{1--4} это <<Начала>> \sauth{Гиппократа
  Хиосского}{Гиппократ Хиосский}.

В~\textbf{1--4} нет никакого учения о подобии. Про теорему о хордах
говорили в терминах площадей, и доказывали тоже так, через площади.
(Теорема о хордах говорит, что если у нас есть две хорды $AB$ и $CD$,
пересекающиеся в точке~$X$, то $AX \cdot BX = CX \cdot DX$.

\textbf{5} я книга это теория отношений \sauth{Евдокса
  Книдского}{Евдокс} для абстрактных величин.

После этого получается учение о подобии, которое описано в \textbf{6}
й книге.  Там же говорится о решении квадратных уравнений вида $x
\cdot (a + x) = S$.

\textbf{7--9} приложение теории Евдокса к числам. Арифметика, теория
делимости.  Алгоритм Евклида, теорема, доказанная Евклидом, о
бесконечности множества простых чисел.  Это доказательство было одним
из первых доказательств от противного.

В \textbf{10} й книге описываются классы квадратичных
иррациональностей.  $a \pm \sqrt b$, $\sqrt a \pm \sqrt b$. Евклид
строит пифагорово поле, которое получается из $\Q$ расширением с
помощью квадратичной иррациональностью.

Метод исчерпания \sauth{Евдокса}{Евдокс} \textbf{12} я книга (первая
теория бесконечно малых).

\textbf{11--13} Стереометрия. Это результаты \sauth{Архита
  Торрентского}{Архит Торрентский}.  Что тут было:

Про отношение площадей двух окружностей $S_1$ и $S_2$ с радиусами $D$
и $d$ известно, что $\frac{S_1}{S_2} \bw= \frac{D^2}{d^2}$.
Аналогично, про отношение объёмов двух шаров было известно, что
$\frac{V_1}{V_2} \bw= \frac{D^3}{d^3}$.

Объём тетраэдра равен $\frac13$ объёма призмы (с тем же основанием и
высотой).

\emph{Платоновы тела} \textbf{13} я книга.

Про ещё две книги известно, что она вроде принадлежит
\sauth{Гипсиклу}{Гипсикл} ($+2$ век), \textbf{15} я \auth{Исидор
  Милетский} ($+4$ век).  Они продолжили теорию многогранников.

1729~год первый перевод <<Начал>> на русский язык, последний 1950~год.
Школьная традиция записи доказательств пошла оттуда, из <<Начал>>.

\marklec{06}{28.02.06}

\subsection{Бесконечность в математике}

\subsubsection{Как в математику пришла бесконечность, и как с ней боролись}

\emph{Бесконечность} появилась вместе с несоизмеримыми отрезками.
Греки считали, что всё есть число, и чтобы найти отношение отрезков,
нужно найти общую меру. А числа то были рациональные, и вот тут была
вся ботва.

Когда поняли, что ничего не получается, перешли к геометрической
алгебре.  Изобрели алгоритм \sauth{Евклида}{Евклид} (попеременное
вычитание). Вот краткое изложение этого алгоритма:

Пусть $a > b$. Представим в виде $a = nb + b_1$. Потом начинаем
отрезок $b$ мерить отрезком $b_1$, и так далее, $b = n_1b_1 + b_2$ и
так далее.  Иначе говоря, имеем такое разложение
$$\frac ab = n + \frac{b_1}{b} = n + \cfrac{1}{n_1 +
  \cfrac{b_2}{b_1}},$$ и получается цепная дробь. Для соизмеримых
отрезков это конечное выражение, а для иррациональных получаем
бесконечную цепочку.  Так появилась бесконечность.

Кроме того, были философские вопросы: а из чего состоит отрезок, можно
ли его бесконечно делить?  В итоге сложилось две школы (философские)
по взглядам на бесконечность.

1. \emph{Континуалистская} концепция (те, кто считал, что деление
возможно до бесконечности).

2. \emph{Атомистическая} концепция (существует неделимый элемент).

Естественно, они боролись между собой, но были и нейтрально
настроенные ученые.  Так вот, один из таких нейтралов был
\auth{Зенон}.

\subsubsection{Апории Зенона}

\emph{Апория} по гречески значит <<трудность>>. \auth{Зенон Элейский}
($-5$ век).  Зенон возражал и той, и другой школе.  Это любимый ученик
\sauth{Парменида}{Парменид}, а тот был отцом логики.  Зенона называли
двуязыким.  Как пишет \auth{Платон}, Парменид писал все свои
рассуждения в стихах.

Известна апория Зенона об отрезке как о множестве точек.  С одной
стороны, точки имеют нулевой размер, так что если просуммировать, то и
отрезок будет нулевого размера. А если большой отрезок это много
маленьких отрезков, то получается, что мера всего должна быть
бесконечной, что тоже неверно.

Была ещё апория о \emph{дихотомии} (о делении пополам), она же апория
движения.  Чтобы пройти отрезок $[0,1]$, нужно пройти середины
отрезков, то есть точки $0.5$, $0.75$, $0.875$, и так далее.  Ну, а на
самом деле движения вообще нет, потому что чтобы дойти до первой
середины, нужно сначала дойти до $\frac14$, но чтобы до неё дойти,
нужно дойти до $\frac18$, и так далее.

Зенон говорит, что рассмотрение ряда (геометрическая прогрессия) не
даёт того, чего надо. Нам нужно пересчитать весь натуральный
ряд. Здесь \auth{Аристотель} пишет, что Зенон путает актуальную и
потенциальную бесконечность.

Кстати, этот парадокс сохранялся довольно долго, до тех пор, пока не
пришёл \sauth{Георг Кантор}{Кантор} и не доказал, что для бесконечных
множеств <<часть>> может равняться (то есть быть равномощной)
<<целому>>.

Следующая апория Ахиллес и черепаха: Ахиллес никогда не догонит
черепаху.  А чем она, собственно, отличается от первой? Пусть $V =
kv$. Тогда Ахиллес проходит $a$, черепаха $\frac{a}{k}$.  Когда он
добежит, черепаха отползёт ещё на $\frac{a}{k^2}$, и так далее.  На
практике, конечно, всё не так. У Ахиллеса было $\infty + 1$ шагов, у
черепахи $\infty$ шагов. Итак, часть равна целому.

Насчёт того, что часть может быть равна целому, задумывались не только
греки. У \sauth{Галилея}{Галилей} был такой парадокс: число квадратов
было равно числу натуральных чисел.

Эти парадоксы были направлены в основном против континуалистов. Однако
Зенон, как мы сейчас увидим, не оставлял права на существование и
атомистической концепции.

Если считать, что существуют неделимые элементы, то тоже ничего не
получится. Если считать, что бесконечное деление невозможно, то вот
вам пожалуйста такое рассуждение парадокс о стреле: пусть время
дискретно. Тогда в каждый момент времени стрела покоится, стало быть,
вообще не движется.

\subsubsection{Попытки решить проблему бесконечности}

\sauth{Гильберт и компания}{Гильберт} об основании математики:
Современная физика: когда мы изучаем микромир, там другие законы
движения, и мы не должны применять эту же науку <<в малом>>.

Ну и что же греки придумали?  Наиболее радикальным оказался
\auth{Протагор}, который сказал, что нафиг всё, типа, долой
математическую абстракцию, уж если она приводит нас к таким нелепым
выводам.  \auth{Аристотель} возразил, что так нельзя, и что наука есть
знание общее, и нельзя полностью отказываться.

\auth{Демокрит}: конечная математика: можно дробить, но не сильно. Это
подтолкнуло \sauth{Архимеда}{Архимед} к его инфинитезимальным
изысканиям.

\auth{Анаксагор} ($-5$ век). Он говорил, что делить то можно сколько
угодно, но нужно быть аккуратным, когда потом начинаешь всё это
суммировать.

\subsubsection{Евдокс Книдский}

Мы поговорим о теории исчерпаний \sauth{Евдокса Книдского}{Евдокс
  Книдский} (\pe{-406}{-355}).  Остров Книд это юг Малой Азии.  Евдокс
был гениальным астрономом, пошёл в академию \sauth{Платона}{Платон},
там отгрохал свои 27~небесных сфер, а вообще он ваял свои дела под
руководством \sauth{Архита Торрентского}{Архит Торрентский}.  Потом
уехал на остров и там организовал обсерваторию, и там, кстати, впервые
наблюдения стали систематическими.

Исчерпания излагаются в <<Началах>> \sauth{Евклида}{Евклид}.  При этом
ключевым фактом его теории является следующая лемма.

\begin{lemma}
Для любого отрезка $a > b$, вычитая всякий раз из него больше его
половины, а из его остатка больше его половины, и так далее, через
конечное число шагов мы можем получить остаток, который меньше~$b$.
\end{lemma}
\begin{proof}
Пусть $\al_1 = a-a_1 < \frac{a}{2}$, $\al_2 = \al_1 - a_2 <
\frac{\al_1}{2} < \frac{a}{2^2}$, ну и так далее. Итого получаем
$\al_n = \al_{n-1} - a_n < \frac{a}{2^n}$, а с другой стороны,
поскольку числа архимедовы, то есть есть такое~$n$, что $nb > a$, то
получается, что $nb > a > 2^n \cdot \al_n > n \cdot \al_n$.  Отсюда
следует, что $\al_n < b$.
\end{proof}

Эта важная лемма лежит в основе принципа интегрирования,
установленного \sauth{Архимедом}{Архимед}.  Вписав монотонную
последовательность, нужно ещё просуммировать ряд, который у нас
получается.

\emph{Метод исчерпания} состоит из трёх шагов:

\begin{items}{-2}
\item Выписывается последовательность, исчерпывающая фигуру.
\item Вычисляется арифметический предел (на нашем языке).
\item потом говорится, что измеряемая площадь равна найденному
  пределу, потом заново в каждой задаче доказывается, что он равен
  тому, что надо.
\end{items}

Главное, что последовательность была именно такая, как в лемме.

\textbf{Квадратура параболы}, выполненная \sauth{Архимедом}{Архимед}.

Находим точку $B$, в которой касательная параллельна хорде $AC$.
Рассматриваем сопряжённый диаметр $O_1B$, где $O_1$ середина
$AC$. Потом проводим ему параллельные через точки $A$ и $C$. Полагаем
$A_1 := S_{ABC}$. Это будет по площади ровно половина
параллелограмма. Значит, то, что осталось от сегмента, имеет площадь
меньше половины. Потом рассматриваем ещё два сегмента параболы, у них
точки касания будут, соответственно, $D$~и~$E$. Архимед показывает,
что $S_{ADB} + S_{BEC} = \frac14 S_{ABC}$.  Затем полагаем $A_2 :=
S_{ABC} \bw+ S_{ADB} \bw+ S_{BEC}$, ну и так далее.

 \cpic{4}

Итого получаем ряд $\sum\frac{1}{4^n}$. В соответствии с методом
\sauth{Евдокса}{Евдокс}, видим, что такая последовательность
исчерпывает нашу фигуру. Для того, чтобы посчитать предел,
соответственно, пишем формулу для $A_n$, и видим, что хвост стремится
к нулю, и всё получается: $\frac43S_{ABC}$.

А потом для каждого конкретного примера доказываем, что именно число и
должно получиться.

Эти инфинитезимальные методы имени Архимеда, в котором встречается
суммирование рядов и построение касательных к кривым.

Конечно, всё это было очень хрупким в том смысле, что стояло на
непрочных ногах.  Как мы знаем, строгость появилась более чем через
2000~лет, когда появились \auth{Вейерштрасс}, \auth{Коши} и прочие
граждане, которые не любили махать руками.

\subsubsection{Архимед}

\auth{Архимед} (\pe{-287}{-212}).

Родился на Сицилии, учился в Александрии, но потом прожил всю жизнь в
Сиракузах.  Мы знаем про его результаты из писем (\auth{Досифей},
\auth{Эратосфен}). Про него говорят, что иногда он нарочно посылал
неточные или неверные результаты, чтобы убедиться, что его работы
читают внимательно.

Корона царя Гиерона (<<электрон>> = сплав серебра и золота). Поскольку
удельный вес золота и серебра был известен, осталось найти только
объём короны. А это, как хорошо известно, история с <<эврикой>>.

История с кораблём и системой рычагов, с помощью которых спустили
корабль на воду.  <<Дайте мне точку опоры, и я переверну Землю!>>
сказал Архимед.  Архимедов винт тоже его изобретение.

Могила Архимеда говорят, была с цилиндром и вписанным шаром и конусом.
Впрочем, до наших дней она не дошла.

Архимед круто продвинул инженерное искусство, и римляне не смогли
взять город Сиракузы приступом.  Камнемёты помогли сдержать
осаду. Тогда римляне осадили город.  Архимеду в это время было
75~лет. Говорят, он даже не заметил, как его убили.

Ну, механика это всё хорошо, но вот математика это главное.  Изучает
архимедовы тела (полуправильные многогранники).  Изучает кубическое
уравнение $x^2 (a\pm x) = b$, то есть вопрос о том, сколько и когда
решений оно имеет.

Говорят, что он в одном из писем изучал теорию игр. Считается, что это
первое в истории математики изучение этой теории.

<<Псаммит>> исчисление песчинок (наука о том, как именовать большие
числа).

Вычислял приближённое значение числа~$\pi$, установил, что
$3\frac{10}{71} < \pi < 3\frac{10}{70}$, дойдя до 96 угольника
дроблением пополам.

Занимаясь вопросами с кораблём, нашёл положения устойчивости для
параболоида вращения.

Некоторые сочинения Архимеда дошли до нас в арабском переводе
\sauth{Сабита ибн Корра}{Сабит ибн Корра} ($+9$~век).  Ещё была вроде
работа о параллельных прямых, но она до нас не дошла.

\textbf{Направления исследований:}

\begin{items}{-2}
\item Измерение площадей криволинейных фигур и объёмов тел.
\item Построение касательных к кривым.
\item Задача об экстремумах.
\end{items}

Всё это инфинитезимальные методы Архимеда. До нас дошли 6~работ по
инфинитезимальным методам (всё это отрывки из писем, потому что
Архимед никогда не писал монографий, а всего лишь излагал свои
рассуждения в письмах). До нас дошли такие фрагменты:

\begin{nums}{-2}
\item О квадратуре параболы.
\item О коноидах и сфероидах.
\item О спиралях (дифференциальные методы для спирали $r(\ph) = a\ph$)
  + построение касательных.
\item О сфере и цилиндре (сам Архимед её ценил).
\item Измерение круга (приближение числа $\pi$).
\item Послание \sauth{Эратосфену}{Эратосфен} о механических методах.
\end{nums}

В дальнейшем Архимед начинает модифицировать метод
\sauth{Евдокса}{Евдокс}.  Он рассматривает не только вписанные фигуры,
но и описанные фигуры, и стремит их к общему пределу.

Формулы: \eqn{\suml{\nu=1}{n} \nu h = \frac{n(n+1)}{2}\cdot h.}
\eqn{\suml{\nu=1}{n} (\nu h)^2 = \frac{n(n+1)(2n+1)}{6}\cdot h^2.}
\eqn{\suml{\nu=1}{n} \sin\frac{\nu \pi}{n} = \ctg \frac{\pi}{2n}.}
Как считался объём параболоида вращения?  Мы устраиваем разбиение
параболоида на блины, и аппроксимируем.  Очевидно, что
\eqn{\frac{n^3}{3} \cdot h^3< \suml{\nu=1}{n} (\nu h)^2\cdot h <
  \frac{(n+1)^3}{3}\cdot h^3.}  Ну и дальше простенькая оценка
пределов. Так и получается одна треть.

\marklec{07}{02.03.06}

\subsection{Теория конических сечений}

\label{sec:conicae}

Это сочинение принадлежит \sauth{Аполлонию Пергскому}{Аполлоний
  Пергский} (\pe{-260}{-170}~годы).

Это математик и астроном. Учился в Александрии, ученик
\sauth{Евклида}{Евклид}. Сам родился на побережье малой Азии, потом
уехал в Александрию.

Как вошли в математику конические сечения? \auth{Гиппократ Хиосский}
($-5$ век) решал задачу об удвоении куба, и придумал, что нужно бы
рассмотреть такие величины: $\frac ax = \frac xy = \frac yb$. Но нужно
было подтвердить существование этих вставок.  А были то только прямая,
окружность, и то, что получается скольжением конус, цилиндр.

\subsubsection{Принцип непрерывности.}

\auth{Архит} из школы \sauth{Пифагора}{Пифагор} показал, как построить
$x$ с помощью пересечения конуса, цилиндра и тора (построение очень
сложное, и это нетривиально).  Там получаются именно те соотношения,
которые нам нужны.  Движение обеспечивало непрерывность.

\auth{Аристотель} говорил: <<Непрерывность это то, что в движении не
обнаруживает разрывов.  Это абсолютная связь последующего с
предыдущим. Непрерывность присуща пространству, времени и движению.>>

Долгое время непрерывные кривые так и понимали: линия непрерывна, если
она начерчена одним движением руки, говорил \auth{Эйлер} в 17 м веке.
Как использовалась эта идея?

\auth{Евклид} в <<Началах>> пишет: на отрезке~$CD$ построить
равносторонний треугольник.  Ставим ножку циркуля в~$C$,
раствором~$CD$ проводим окружность, потом аналогично из
точки~$D$. Точка~$A$ пересечения существует, так как окружность
непрерывна.

Есть ещё такая задача: пусть точка $A$ внутри окружности, а $B$ вне
её. Тогда отрезок $AB$ пересекает окружность в точке $C$. Она
существует, потому что прямая и окружность непрерывны.

\textbf{Конические сечения.}

\auth{Менехм} ($-4$-й век) сказал, что нужно делать сечения конуса,
потому что будет от этого польза.  Получаются линии $x^2 = ay$, $xy =
ab$ как сечения конусов вращения \authorcomment{эллиптическое сечение
  было забыто лектором}.  Рассматривались 3~вида конусов:
прямоугольный, остроугольный, тупоугольный, и, соответственно, три
линии парабола, эллипс и гипербола. Названия этих кривых дал уже
\auth{Аполлоний}. Крупных учёных интересовал мир, и ради этого,
кстати, и создавались разные теории.

\textbf{О связи времён (дерево школ).}

\auth{Архит} (самая древняя школа пифагорейцы). У него учился
\auth{Евдокс}.  У Евдокса учился \auth{Менехм}. Известно, что Евдокс
учился всюду, у \sauth{Архита}{Архит}, \sauth{Платона}{Платон}, в
Италии изучал медицину. \auth{Платон} сказал: построй модель
мира. Этого не было в восточной математике, там было много вычислений,
данных, но не было реальной модели всего этого, общей картины мира.  А
чтобы это сделать, нужно было всё-таки довольно много поработать.

\subsubsection{Модель мира.}

\auth{Евдокс} сказал, что мир устроен так:

\begin{items}{-2}
\item Каждое тело вращается по своей сфере, внешняя используется для
  неподвижных звёзд.
\item Всего имеется 27~сфер, все, кроме внешней, равномерно
  крутятся. Это важно, что равномерное, другого тогда помыслить не
  могли.
\item 3~сферы описывают движение Солнца, 3 движение Луны, а ещё по
  4~сферы нужно на каждую из 5~планет.  Всего тогда знали Меркурий,
  Марс, Венеру, Юпитер, Марс. Это построение использовалось много
  столетий. Кстати, вавилонские астрономические данные изучал в
  Вавилоне Аполлоний.
\end{items}

\sauth{Клавдий Птолемей}{Птолемей, Клавдий} ($+2$ й век).  Его
сочинение <<Математическое построение>> обычно называется
<<Альмагеста>> \emph{величайшее} по арабски. Птолемей соединил
греческие модели и астрономические таблицы. Потом его этим каталогом
пользовались до тех пор, пока \auth{Кеплер} не пришел и не сказал, что
всё неправда.  Надо сказать, что поправок было очень много, и все их
терпеливо учитывали.  И вот когда \auth{Коперник} пришёл и сказал, что
на самом деле есть гелиоцентрическая система, это и вызвало большой
базар.

\subsubsection{Книга <<Конические сечения>> Аполлония Пергского}

Итак, переходим к классическому сочинению
\sauth{Аполлония}{Аполлоний}. Теория конических сечений.  Оно состоит
из 8~книг. Книги \textbf{1--4} на греческом языке, \textbf{5--7} в
арабском переводе, \textbf{8} я книга утеряна.  Однако \auth{Папп}
($+3$ й век) привёл кое какие сведения из этой книги.

В $+18$ м веке у \sauth{Э.~Галлея}{Галлей} встречается в сочинениях
реконструкция \textbf{8} й книги.

Рассмотрим прямой круговой конус. Тогда $\angle AOB = 90^\circ$. $PK
\bot OB$. $PL = p$.  Из треугольника $KPB$ имеем $BK^2 = PK^2 + PB^2$.
$\angle OAB = 45^\circ$. $BK^2 = 2PK^2$. $BK = PK\sqrt2$.  $KM^2 =
AK\cdot KB$.

\cpic{3}

$PP' = AK$, $PP' = LP\sqrt2$, $AK = KL\sqrt2$.

$KM^2 = 2KP\cdot LP$. В современных обозначениях $LP = p$, $KP = x$,
$KM = y$, $y^2 = 2px$.

Симптом параболы \authorcomment{тут ещё была картинка}.  Квадрат,
построенный на полухорде $KM$, равен прямоугольнику, построенному на
отрезке $PK$ и постоянном отрезке $PR$.  Здесь $PR = 2p$.

\textbf{Терминология Аполлония.}  $y^2 = 2px$ \emph{парабола}.  $y^2 =
2px - \frac pa x^2$ \emph{эллипс} (недостаток по гречески), $y^2 = 2px
+ \frac pa x^2$ \emph{гипербола} (избыток по гречески).

Потом до него дошло, что на самом деле конус можно брать не только
стандартный, но и любой. Исследовался переход от одной системы к
другой.

Кроме всего этого, была ещё книга Аполлония про \emph{касательные} и
\emph{нормали}.  Но там всё было очень сложно, поэтому сейчас нет
времени про это рассказывать.

\subsubsection{Астрономия}

\auth{Коперник} (\pe{+1473}{+1543}). Он был первый, кто сказал, что
это гелиоцентрическая система, но опубликовал свою работу уже прямо
перед смертью.  При этом он обманывал церковников, говоря, что просто
так удобно считать, а само-то оно крутится, конечно, вокруг Земли, так
что всё хорошо.

А вот \auth{Кеплер} уже понял, что всё не так, как говорил
\auth{Птолемей}.  К тому времени уже про них всё было известно, и
выяснилось, что закон равномерности для планет имеет место (то есть
заметание телом равных секторов за равные промежутки времени).

Он написал <<Стереометрию винных бочек>>. Он увидел, что бочки сделаны
хитро так, что можно измерять объём жидкости, который находится в
бочке, довольно просто, опуская в бочку мерную рейку.  В этой книге он
написал про измерение площадей и объёмов.

\auth{Кавальери} исследовал интегралы, получил $\int x^n\,dx =
\frac{1}{n+1} x^{n+1} + C$, и $ \int \frac1x\,dx = \ln x + C$ (при $x
> 0$).

Но важно со всеми этими эллипсами то, что их просто заранее изучили.
<<Я преклоняюсь перед гением \sauth{Кеплера}{Кеплер}, но ещё больше
перед природой, которая позволяет всё описать такими красивыми
уравнениями>> (неточная цитата \sauth{Эйнштейна}{Эйнштейн}).

17 й век. Это уже математика нового времени. \auth{Кеплер} использовал
эллипсы. \auth{Галилей} сказал, что брошенный камень летит по
параболе. \auth{Ньютон} в своей работе <<Математические начала
натуральной философии>> ссылается на учение
\sauth{Аполлония}{Аполлоний}.

После греков были арабы, но у них не было революций, не было веры в
перспективу.

Создание аналитической геометрии. Это были \auth{Ферма} и
\auth{Декарт}.  \auth{Мерсенн} был координатором переписки с другими
учёными (в том числе Ферма и Декарта).  Тогда был большой спор по
поводу первенства в открытиях, но вот по поводу аналитической
геометрии они почему то не спорили, они говорили, что это уже и раньше
было, что мы тут ничего не придумали.

Это был перевод Аполлония на другой язык, на алгебраический.

\sauth{Пьер Ферма}{Ферма, Пьер} (\pe{+1601}{+1655}) по образованию
юрист, жил в Тулузе.  В 1679 году опубликованы работы Ферма (но
главное было уже сделано: 1636 год <<Введение в теорию плоских и
пространственных мест>>.

Нужно сказать, что он уважал греков, и что нужно всё писать их языком.
В частности, он сохранил принцип однородности.

Геометрия \sauth{Декарта}{Декарт} 1637 год (публикация <<Геометрии>>).
У него всё было уже числами, там не было ничего геометрического.  Всё
число, и $ab$, и $\frac ab$, и $a+b$. То есть нужно забыть про
площади.  Декарт был философ, он много чего хотел. Его труды были
заумные.  А вот Ферма писал по простецки. Древняя Греция: прямая,
окружность (плоские тела).  Конические сечения пространственные
тела. Ферма сказал, что нет, это всё фигня. Надо писать уравнениями, и
главное это степени уравнения.  первая степень прямая, а вторая
степень это как раз конические сечения окружности, эллипсы и тому
подобное кривые второго порядка.

Потом были ещё преобразования координат, приведение к каноническому
виду и так далее.  Но про него ещё потом скажем.

Важно, что они эти два математика заговорили на языке алгебры при
изучении геометрии.  Ещё дальше пошёл \auth{Ньютон}, он начал изучать
кривые третьего порядка.

\marklec{08}{07.03.2006}

\subsection{Архимед}
\authorcomment{Одолжено с Википедии.}
Сведения о жизни Архимеда оставили нам Полибий, Тит Ливий, Цицерон,
Плутарх, Витрувий и другие. Они жили на много лет позже описываемых
событий, и достоверность этих сведений оценить трудно.

Архимед родился в Сиракузах, греческой колонии на острове
Сицилия. Отцом Архимеда был математик и астроном Фидий, состоявший,
как утверждает Плутарх, в близком родстве с Гиероном II, тираном
Сиракуз. Отец привил сыну с детства любовь к математике, механике и
астрономии. Для обучения Архимед отправился в Александрию Египетскую —
научный и культурный центр того времени.

Архимед написал своё сочинение <<О спиралях>>. Задача поставлена
\sauth{Конэном}{Конэн} \sauth{Архимеду}{Архимед}.  Спираль имеет
уравнение $\rho = a\ph$ в полярных координатах. На самом деле можно
считать, что $a = 1$, это ни на что не влияет. Архимед вычислил
площадь витка спирали, более точно, витка, параметризованного $\ph \in
[0,2\pi]$.  Мы делим плоскость на~$n$~частей прямыми, выходящими из
нуля. и складывать площади круговых секторов вписанного и описанного.

$S_{sect} = \frac{\pi}{n}\rho^2$. Впишем сектора вписанный и
описанный.  Итого получаем формулу
$$\frac\pi n \hr{\rho^2 + (2\rho)^2 + \dots + ((n-1)\rho)^2}.$$ Это
интегральные методы.

Если нужно построить касательную. Соединяем точку $P$ и центр спирали
$O$.  При этом $OT \bot OP$ и нам нужно теперь найти точку $T$.  Пусть
$Q$ близка к $P$ на спирали. Тогда $OQ$ тоже радиус вектор. Она
пересекается с касательной в точке $F$. Провели окружность с центром в
точке $O$ радиусом $OP$.  Она пересекает $OF$ в точке $R$. Тогда,
говорит Архимед, дуга $PR$ ортогональна $OF$.  В криволинейном
треугольнике угол $QPR$ равен углу $OTP$ (на самом деле это всё не
очень строго, потому что углы то все кривые\ldots) Далее, угол $\ph$
(угол $FOP$) примерно равен $\angle FPR$. Рассматриваем $FPR$ и $PTO$.
Это два прямоугольника прямоугольные, поэтому треугольники почти
подобны.  Посему $\frac{FR}{RP} = \frac{PO}{OT} = \tg \al$.  Далее мы
говорим, что на самом деле, поскольку приращение бесконечно малое,
можно считать, что $\frac{QR}{RP} = \frac{FR}{RP}$. С другой стороны,
$QR$ приращение радиуса нашей кривой $\De\rho$. А $RP$ это примерно
$\rho \De\rho$.  Стало быть, получаем формулу $OT =
\frac{\rho^2\De\ph}{\De\rho} = \rho^2\cdot \frac{\De\ph}{\De\rho} =
\frac{\rho^2}{a} = \rho\ph$. Итого, $\frac{PO}{OT} = \tg \al =
\liml{\De\ph \ra 0} \frac{\De\rho}{\rho \De\ph}$.  Такой подход
называется дифференциальным методом Архимеда.

Ещё один подход был развит Архимедом для нахождения экстремумов.

Задача. Разбить шар на две части с заданным отношением объёмов.

Задача о делении отрезка: $\frac{S}{x^2} = \frac{a-x}{c}$.  На самом
деле Архимед приводит это уравнение к виду $\frac{x^2}{p}
=\frac{bc}{x}$, где $S = pb$.  Тогда это задача о пересечении
гиперболы и параболы. Он показывает, в каком случае у уравнения будет
один корень (на геометрическом языке). То есть мы решение уравнения
$f(x) \cdot g (x) = M$ сводим к решению $f = \frac{M}{g(x)}$.  Важно,
что Архимед спокойно оперировал с интегральными суммами, пределами, но
ничего не формулировал явно.

\subsection{Римская империя}

Римская империя времени упадка. У римлян к науке было отношение прямо
скажем, наплевательское.  $-30$ е годы падение Александрии. После
того, как все основные войны закончились, наука потихоньку
оправляется. Но центр науки всё равно остался в Александрии.
\auth{Блок}: \auth{Цицерон} собрал жалкие остатки того мёда, который
остался от греков.

\auth{Вергилий} тоже говорил, что математика это уж во всяком случае
не то, что должны знать римляне.  Но потом как то так получилось, что
потом возникло греческое возрождение.  К его описанию мы сейчас и
переходим.

\subsection{Поздняя греческая наука}

По времени это \pe{+1}{+2}~века.  \auth{Герон Александрийский} ($1$ й
век), \auth{Менелай Александрийский} (1 й век), \sauth{Клавдий
  Птолемей}{Птолемей, Клавдий} (тоже из Александрии) ($2$ й~век).

\auth{Герон} преподавал в Мусейоне, комментировал
\sauth{Евклида}{Евклид}.  От Герона осталось много метрических методов
<<Метрика>> Герона.

\auth{Диофант Александрийский} (Герон был его предшественником, тоже
изучал неопределённые уравнения).  $u^2 + v^2 \bw= 2w^2$ и $u^2 + v^2
\bw= w^2$ решения самых двух древних неопределённых уравнения.  Ещё у
Герона были приближения для квадратного корня. Что нужно? Берём
подходящий квадрат, потом множим всё на~$10$, опять подбираем, и так
далее. Кстати, беда в том, что система счисления у греков была
извратная (алфавитная, непозиционная).

\auth{Менелай} был создателем первой сферической
геометрии. <<Сферика>> дошло только в арабском переводе.

\auth{Птолемей}. Самое важное, что у него было это геоцентрическая
картина мира. Был построен полный звёздный каталог <<Альмагеста>>. Там
использовалась полностью вавилонская система счисления.  То, что мы
час делим на 60 минут, мы обязаны ему. Кроме того, у него были таблицы
синусов.  Там была развита техника вычислений.

В это время математики начинают отказываться от принципов
геометрической алгебры. Происходит вторичная арифметизация науки.

\subsubsection{Диофант}

Ну вот, значит, теперь поговорим про \sauth{Диофанта
  Александрийского}{Диофант Александрийский}.

Его работы это теория чисел, алгебра, и диофантов анализ. Это примерно
середина третьего века нашей эры.  Но точных датировок фактически
нет. Его рукописи нашли в Ватикане (\sauth{Рафаэль Бомбелли}{Бомбелли,
  Рафаэль}).  Зато очень хорошо известно, сколько прожил Диофант,
потому что сохранилась стихотворная эпитафия Диофанта, в которой его
возраст можно узнать, решив несложную математическую задачу.

У \sauth{Диофанта}{Диофант} были книги <<Книга о многоугольных
числах>>, <<Арифметика>>. Их переводил Бомбелли вместе с одним
товарищем, когда им нечего было делать. Так вот, 6~книг они перевели,
а осталось ещё сколько то. Они говорили, что остатки переведём потом,
но так потом видимо, ничего не было сделано. Более того, книги были
утеряны.

Сравнительно недавно нашли ещё 4~книги, но оказалось, что на самом
деле это комментированные 4~книги из первых~6.  Есть мнение, что
Диофанта комментировала только \auth{Гепатия Александрийская}, так что
есть шанс, что это именно её комментарии.

Итак, как мы уже говорили, геометрическая алгебра начинает умирать. У
Диофанта впервые появились обозначения для неизвестных и их
степеней. Его книги это фактически сборники задач с решениями.

В начале книг имеется введение, в котором всё совсем круто и
алгебраично.  Для пифагорейцев число это множество единиц. Разделить
нельзя. \auth{Герон}, например, забил на это, и у него уже была
нормальная числовая область $\Q_+$.

Слово $\al\rh\iota\ta\mu o\varsigma$ по гречески означало <<число>>.
Но Диофант, что важно, уже говорил, что числа бывают и в смысле
недостатков, или, как он их называл по гречески, $\la\ep\iota\pi\om$.
А оперировать с ними можно так (по естественным правилам перемножать и
складывать).  Знак недостатка у него выглядел так: $\isp$ . Таким
образом, Диофант расширил область до $\Q^*$.

Примерно так же \sauth{Рафаэль Бомбелли}{Бомбелли, Рафаэль} ввёл
мнимые числа (можно использовать только в промежуточных вычислениях).

С неизвестными, правда, была такая проблема. Буквы были забиты под
числа.  Так вот, использовали $\varsigma$. Это была единственная
буква, которая была не забита под цифирное обозначение.

А что же делать со степенями? Они выглядели так: $\De^\upsilon$,
$K^\upsilon$, $\De^\upsilon\De$, $\De^\upsilon K$, $K^\upsilon K$.
Для отрицательных степеней использовали опять $\isp$, которую рисовали
сверху.

По гречески $\iota\si o\varsigma$ равный. Поэтому символы $\iota\si$
использовали для обозначения равенства.

Уравнение $x^3 = 2-x$ в этой нотации записывалось так: $\al\ol
K^\upsilon \iota \si \ol\be\isp\varsigma$.

Правила оперирования. Умели прибавлять равные величины. \textbf{1} я
книга определённые уравнения (1, 2 степень). Но там это всё не нужно.
Появились неопределённые уравнения. Вот там уже без этой алгебры было
совсем плохо.

Они умели строить рациональные параметризации для плоских кривых.
\textbf{2}$_8$: там было написано уравнение $x^2 + y^2 = \square$.
Этим значком обозначалось то, что уравнение не имеет решений.

Как избавиться от 2~неизвестных, если они были? Фактически он
переходил к параметрам, считая их уже известными, но не
фиксированными. Кроме того, использовался подход рационализации
плоских кривых. Это были своего рода начала алгебраической геометрии.

А ещё, помимо Диофанта были такие граждане: \auth{Теон},
\auth{Гепатия}, \auth{Прокл}, \auth{Симпликий}, \auth{Евтокий}.  Но
про них ничего сказать мы уже не успеваем.

\marklec{09}{09.03.2006}

\section{Китай}

\epigraph{Старик Декарт матриц не знал}{\emph{И.\,В.\,Аржанцев}}

Китай. 2000 лет до нашей эры. Первые поселения возникли на реке
Хуанхэ.  Но на самом деле источники дошли до нас только со 2~века до
нашей эры, а всё то, что было до этого, было по всей видимости,
тщательно уничтожено самими китайцами во времена культурных революций.
Первый трактат <<Об измерительном шесте>> появился в то время, когда в
Китае изобрели бумагу.

Второй источник <<Математика>> в 9~книгах. В период \pe{+3}{+8}~веков
появилось ещё 8~трактатов.  А потом было сделано следующее. Из всего
этого склеили 10~классических трактатов.  Так же, как и на Востоке,
был догматический характер изложения.  Веками ничего не менялась. Тоже
рецепты решений, ничего прогрессивного.

Все чиновники сдавали экзамены по математике по этим классическим
трактатам.  Во всяком случае, тех кто не сдал, не пускали. Так что у
них политики были образованные.

В Китае раньше всех научились работать с дробями, с рациональными
числами.  Они делали так же, как и мы это делаем сейчас, но не думали,
что дроби можно сокращать.  То есть они просто перемножали
знаменатели.

Итак, первая китайская математика появилась уже в $-2$~веке.
Десятичная система, десятичные дроби ($+3$~век).  В Европе это только
лишь $16$ й век.

Метод <<\emph{фан чен}>> выстраивание чисел по клеткам.  Это был метод
решения систем линейных уравнений методом исключения неизвестных. В
Европе это было сделано значительно позднее ($19$~век, \auth{Гаусс}).
В Европе матрицы появились только в 19~веке.

Что получалось, мы сейчас посмотрим на примере. Возьмём что нибудь
простенькое.
$$ \case{ x + \frac12 y = 48,\\ \frac23 x + y = 48.  }
$$ Обычно они приводили систему к целым числам, домножая уравнения на
константы.
$$ \case{ 2x + y = 96,\\ 2x + 3y = 144.  }
$$ В Китае это всё записывалось именно в виде матриц:
$$ \rbmat{ \phz\phz2&\phz2\\ \phz\phz3&\phz1\\ 144&96 }
$$ Там говорили: вычитай (по сути, это метод Гаусса). После вычитания
$$ \rbmat{ \phz0&\phz2\\ \phz2&\phz1\\ 48&96 }
$$ отсюда $y = \frac{48}{2}$, а $x = 36$.

Кроме того, в Китае были отрицательные числа (\emph{фу} долг).
Положительные числа назывались \emph{чжэн} имущество.  Однако просто
непонятно было, как их умножать.  Но у китайцев отрицательные числа
появлялись при решении систем, и потому приходилось что-то с этим
делать.

Ну вот ещё один пример системы.
$$ \case{ 5x - y = 7,\\ 10x - 5y = 5.  }
$$ Снова пишем китайскую табличку:
$$ \rbmat{ \phm10&\phm5\\ -\phz5&-1\\ \phm\phz5&\phm7 }
$$ После вычитания получаем таблицу
$$ \rbmat{ \phm0&\phm5\\ -3& -1\\ -9 &\phm7 },
$$ откуда $y = \frac{-9}{3} = -3$, потом $x = \frac{10}{5} =2$.  На
счётной доске они числа изображали камешками (отрицательные и
положительные разного цвета).

\emph{Тянь нюань} (небесный элемент) решение уравнений степени~$n$ до
нужного десятичного знака. Это то, что в Европе появилось только в
19~веке и называлось там метод \auth{Руфини} \sauth{Горнера}{Горнер}.


\section{Индия}

Первые поселения возникли на реке Ганг. Отметим, что первые источники
это первое тысячелетие до нашей эры священные книги <<Веды>> (в
переводе <<знание>>). Первые письменные источники, где была
математика, это уже \pe{5}{7} века.

<<Шулва-Сутра>> это было учение о том, как нужно строить алтари~тп

\subsection{Система счисления}

Самое главное, что было у индусов, и что пошло в ход далее это
десятичная позиционная система счисления. Это было важно, что именно
позиционная, потому что очень удобно считать.

Есть 9~знаков. Писали их не так, как сейчас, но похожим образом.  И
ещё был~0 <<шунья>>. По арабски это <<сифр>>, при переводе на латынь
уже получилось <<ciffra>>.

Каждая цифра имела своё символическое название.  Например,
<<Луна>>$=1$, <<Дыра>>$=0$, <<Ноги>>$=2$, <<Страны света>>$=4$ и так
далее.  Так, например, <<Луна дыра ноги дыра>>$= 1020$.

\subsection{А что ещё?}

В Индии умели рассматривать много неизвестных.  Для них были тоже свои
названия.  <<Калака>> чёрный (ка), <<Нилака>> голубой (ни), <<Питака>>
жёлтый (пи), <<Лохита>> красный (ло).

Решали уравнение $ax + b = cy$ в целых числах.

Вообще у индийцев всё было очень красиво. Они писали в стихах, чтобы
легче запоминалось.

$12$~век \auth{Бхаскара} 
    <<Венец учения>>. Это были 4 книги:

\textbf{1}: <<Лилавати>> (прекрасное) арифметика.

\textbf{2}: <<Биджаланита>> алгебра, то есть искусство вычисления с
элементами.

\textbf{3--4}: Астрономия.

Итак, самое главное, что было сделано, это была \emph{десятичная
  позиционная система счисления}.  А что ещё?

\subsection{Терминология}

\emph{Корень.} Пример. Вот есть у нас квадрат. Нужно найти его сторону
(мула основание квадрата).  То же слово (мула) означало <<корень
растения>>. Арабы перетащили это себе, стали говорить <<джизр>>. Потом
в латынь оно перешло как <<radix>> корень.

\emph{Синус} \authorcomment{картинка с этакой дугой окружности и
  полухордой} Архаждива~ полухорда (тетива лука).  Потом про <<арха>>
забыли, стали говорить <<джива>>.  Потом это было перенесено в
арабский как <<джайб>> выпуклый. А в латыни <<выпуклый>> пишется
<<sinus>>.

\section{Математика арабского Востока}

\epigraph{ Восток дело тонкое\ldots\hskip1cm} {<<\emph{Белое солнце
    пустыни}>>}

Язык науки арабский. Но на самом деле, там было не так много чистых
арабов. Ещё это можно называть математикой стран Ислама.

Как мы знаем, религия Ислама возникла в $7$ м веке (точнее,
$+654$~г.). \emph{Ислам} означает <<покорность>>.  Вначале
уничтожалась культура, насаждённая греками, но потом поняли, что это
неправильно, и решили, что надо бы отгрохать несколько научных
центров.

Первым таким центром стал город Багдад (\pe{9}{10}~века). Турки
сельджуки сделали центром Исфахан в Иране (11~век), в 13 м веке Марага
(южный Азербайджан), основан внуком \sauth{Чингис Хана}{Чингис Хан}.
Наконец, 15 й век это то, что когда-то относилось к СССР Ташкент,
Бухара, Улугбек (основано внуком \sauth{Тимура}{Тимур}). А что было
дальше? А дальше все учёные свалили в Средиземноморье.  Потому что там
тоже была потребность в науке.  Это была уже <<западная>> наука там
были созданы специальные центры по переводу с арабского на латынь. В
этом смысле Европа не пренебрегала арабской математикой, и
преемственность сохранилась.

\textbf{Багдадский халифат}. Там был построен <<Дом мудрости>>. Там
была библиотека, обсерватория, и там жили учёные, которых туда
приводили из разных стран.  В те времена правил \auth{Аль Мамун} (сын
\sauth{Гаруна ар Рашида}{Гарун ар Рашид}).  Он правил
\pe{813}{833}~годы. \emph{Зидж} аль Мамуна это собрание научных знаний
(астрономия, география, математика) того времени.

\subsection{Аль Хорезми}

\auth{Аль Хорезми} (\pe{783}{850}). Сочинения Аль Хорезми переведены
на много языков. Это был крупнейший учёный того времени. Кстати, в
Европе было довольно массовое образование, а в Азии всё было
распределено между маленькой кучкой людей).

\textbf{Сочинения аль Хорезми.}

$\bullet$ Сочинение <<Об индийском счёте>>. Сам Хорезми удивлялся
тому, что с помощью 10~знаков можно на самом деле записать любое
число.  Это были <<правила вычисления>> алгоритм.  Сам он писал
<<хисаб ал-хинд>>, то есть <<индийский счёт>>. Это была его первая
заслуга.

$\bullet$ Алгебра Аль Хорезми <<Аль Джебр и аль мукабала>>.  Аль джебр
перенос слагаемых из одной части уравнения в другую (с одного берега
на другой).  А мукабала приведение подобных слагаемых.  Сам Хорезми
рекомендовал покупать свою книгу, ибо она была полезна при расчётах.

Однако всё писали словами. Формул не было, но неизвестное, конечно
приходилось обозначать какой-нибудь буквой. Неизвестное обозначали
словом <<\emph{шай}>>.  Квадрат <<\emph{мал}>> (в Европе
\emph{census}).

В качестве примера мы приведём вот что. Чтобы решить довольно простое
уравнение $\frac{10+x}{3} = x$, требовалось исписать целую страницу,
чтобы получить ответ $x =5$.

\textbf{Классификация квадратных трёхчленов.}

\begin{items}{-2}
\item Квадраты равны корням $x^2 = 5x$ (на самом деле, конечно,
  коэффициенты любые).
\item Квадраты равны числу $x^2 = 80$.
\item Корень равен числу $4x = 20$.
\item Квадрат и корень равен числу $x^2 + 10x = 39$
\item Квадрат и число равно корням $x^2 + 21 = 10x$.
\item Корни и числа равны квадратам $3x + 4 = x^2$.
\end{items}

Ну, конечно, ничего принципиально нового не было сделано.

\subsection{Омар Хайям и другие}

Однако было не так всё плохо. Жил был такой поэт \sauth{Омар
  Хайям}{Хайям} (\pe{1048}{1131}), жил в Самарканде, бедствовал, но
потом в~1074~году его пригласили в Исфахан (там правил визирь
\auth{Низам-ал Мумк}). Это была столица турков сельджуков.  Там Хайям
и написал свой зидж Малик-Шаха (реформа календаря).  Важно то, что
Омар Хайям классифицировал уравнения третьей степени.  Хайям писал:
хорошо бы, чтобы люди нашли общую формулу, а то я не могу её найти.
Уравнения: $x^3 + bx = a$, $x^2 + y^2 = \frac{a}{b} x$ окружность, $y
\sqrt b = x^2$ парабола

Хайям вообще был разносторонним человеком, писал стихи.  И всё было
хорошо до тех пор, пока правителя не убили, и вот тогда Хайяму
пришлось плохо, и он убежал в Мекку.

Новый центр науки появился в Мараге. Там жил \auth{ат-Туси}
(\pe{1201}{1274}).  Там он делал астрономические таблицы.  Но был и
прогресс. У него впервые тригонометрия оторвалась от астрономии.
1260~г. трактат о полном четырёхстороннике.  Эти идеи использовались
впоследствии \sauth{Региомонтаном}{Региомонтан} и
\sauth{Коперником}{Коперник}.

Последним корифеем арабской науки был \auth{Джемшид Аль Каши}
(\pe{?}{1430}).  Его главное сочинение <<Ключ к арифметике>>, где он
рассказывал о десятичных дробях.  Второе сочинение трактат <<Об
окружности>>. Он вычислил $\pi$ до 17 знаков, вписав правильный
многоугольник с числом сторон, равным $3\cdot 2^{28}$, то есть порядка
миллиарда.  Ну, а в Европе у \sauth{Виета}{Виет} (значительно позже!)
был всего лишь $3\cdot 2^{17}$ угольник и всего 9~знаков.

Была классификация уравнений первой и второй степени.

\subsection{Замечание о терминологии}

$9$ й век все тригонометрические функции были известны. Как получилось
название для функции косинус?  <<Джайб тамам>> значит дополнительный
синус. На латыни это, естественно, complementi sinus, то есть уже
почти cosinus.

Были ещё тригонометрические тождества: $\sin^2 \al \bw+ \cos^2\al =
1$, потом $\sin \al = \cos(90-\al)$, $\sin({\al \pm\be}) = \ldots$.
Через некоторое время это всё было перенесено в Европу.

\marklec{10}{14.03.2006}

\section{Математика эпохи Возрождения}

Мы опять перемещаемся в Европу. При этом арабские знания просачиваются
в Европу.

Из Багдада через Сицилию и юг Италии знание проникало в Европу.  12
век время великих переводов на латынь и другие европейские языки.

Самый знаменитый переводчик \auth{Герардо Кремонский}
(\pe{1114}{1187}).  Перевёл много рукописей на латынь, в основном с
арабского языка.  Знание об индийском счёте проникло в Европу из
сочинений \auth{Аль Хорезми}.

Однако главный герой математик, о котором сегодня пойдёт речь, всё
таки иногда использовал 60 ричную систему счисления.

\subsection{Фибоначчи}

\auth{Леонардо Пизанский}, он же filius Bonacci (сын доброго), он же
Fibonacci, он же \auth{Фибоначчи}. Жил, как удалось выяснить, примерно
вот когда: (\pe{\sim 1180}{1250{+}{+}}).  Его отец был нотариусом в
Пизе, крупном торговом городе.  Там его прозвали добрым. Его потом
отправили примерно в Алжир на службу.

Ну а сам товарищ Фибоначчи, чтобы научиться уму разуму, поехал на
Сицилию, в Египет, Сирию (то есть Вавилон).  И вот в 1202 году он
написал своё самое известное произведение Liber Abaci (книга об
абаке).  Это была полная коллекция того, что ему удалось собрать.

В ней, в частности, была решена задача о кроликах, то есть исследована
последовательность Фибоначчи.  Что там ещё было:

\begin{items}{-2}
\item Индийский счёт, пропаганда десятичной системы счисления.
\item Арифметические и геометрические прогрессии.
\item Последовательность (ряд) чисел Фибоначчи, первый пример
  рекуррентной последовательности.
\item Квадратные уравнения в духе \auth{Аль Хорезми}.
\item Начала теории неопределённых уравнений.
\end{items}

Как мы знаем, на арабском Востоке не писали уравнения буковками. Это
делал Диофант, но это не прижилось в арабской науке. Впрочем, всякое
слово для степеней переменных в арабском языке было очень кратким,
поэтому там это видимо было не нужно.

Очень важно было то, что Фибоначчи выработал терминологию.  Наконец то
неизвестные стали обозначать так: co [cosa], ce [census], cu [cubus].
Но почему-то они любили перемножать степени, а не складывать их,
поэтому они 5~называли первой невыразимой, 7 второй невыразимой. И
такие обозначения будут встречаться вплоть до 28 степени (у
\sauth{Луки Пачолли}{Пачолли, Лука}).

Корень (радикал) radici.

1223 год второе издание Liber Abaci. Оно посвящено Микелли Скотту,
который был при первом просвященном монархе (который был, впрочем,
очень жесток, Фридрих II Гугенштаут).

Потом Фибоначчи ещё написал книгу <<Flo\ss>> и Liber quadratorum.  В
них, в частности, появилась и такая система:
$$\case{x^2 -5 = u^2,\\ x^2 + 5 = v^2.}$$

Он исследовал разрешимость уравнения $x^3 + 2x^2 + 10x = 20$.
Сократим его на $10$, получим $\frac{x^3}{10} + \frac{x^2}{5} + x =
2$.  Тогда отсюда следует, что $1 < x < 2$
\authorcomment{Почему?}. Ботва, значит, целых решений нет.  Более
того, рациональных нет. Допустим, что там есть $\frac{p}{q}$ и получим
противоречие.

Более того, $x \ne \sqrt n$. Потому что мы напишем так: $x(x^2 + 10) =
2(10-x^2)$.  Ясно, что $x = 2 \frac{10-x^2}{10+x^2}$, то есть не
судьба.  Это, кстати, уже приём, являющийся классическим приёмом
алгебраической геометрии.  Более того, он показывает, что у этого
уравнения нет корней $x = \sqrt m \pm \sqrt n$.  А ответ у этого
уравнения (вычисленный с изумительной точностью) он почему-то даёт в
$60$ ричной системе счисления.
$$x = 1;22'7''47'''39''''4'''''40''''''.$$ Здесь именно важен скорее
подход, чем сами задачи.

Во второй книге было вот что. Там разбирались задачи от
\sauth{Диофанта}{Диофант} (как мы знаем, Диофант писал всё в виде
задач с решениями).  Разложение $a^2 = x^2 + y^2$ (\textbf{2}$_8$ и
\textbf{2}$_9$).  Здесь же разбиралась задача о том, что $x^2 + y^2 =
a^2 + b^2 \ne \square$.

Решение Леонардо близко к тому, что предлагали арабские математики.
Использовалась некоторая простая геометрическая конструкция с подобием
прямоугольных треугольников с рациональными сторонами.

Для решения другой задачи возникает переформулировка такого вида:
площадь прямоугольного треугольника не может быть квадратом. $z^4 =
x^4 + y^4$ это была фактически формулировка теоремы Ферма для $n=4$,
и, кстати, Ферма формулировал своё утверждение очень похожим образом.

\subsection{Позднее Возрождение}

Потом примерно 2.5 века вообще ничего не было слышно. А вот потом, в
самом конце 15 века, пошёл век алгебры. И вот тогда, в Италии
появляются Болонский университет (там учились \sauth{Альбрехт
  Дюрер}{Дюрер}, \auth{Коперник}).  Это было \emph{позднее
  Возрождение}.

Что же было в то время, и что возрождалось? Возрождалась греческая
наука.  Это было время гуманизма, но впрочем, инквизиция не дремала, и
церковь вообще довольно сильно мешала развитию науки.

\auth{Кардано}. Имеется автобиография Джироламо Кардано <<О моей
жизни>>. Там он пишет о великих открытиях его времени.

Итак, значит, Болонский университет. Почему он стал таким крупным
научным центром?  Вспомним древние школы греков, Александрию, дом
мудрости в Багдаде. Вплоть до падения Византии в 15 веке там
хранились греческие рукописи. А потом они возвратились в Европу,
причём через Сицилию и Италию. А это означало, что в это время Италия
и Сицилия оказались после падения Византии научными центрами.

Но всё-таки не было академий. Были популярны турниры, диспуты. При
этом некоторые открытия приходилось скрывать (так появились
анаграммы). Кроме того, в 1434 году было изобретено книгопечатание.

1478 год коммерческая арифметика \auth{Тревизо}.

1482 год вышли из печати <<Начала>> (в переводе).

1494 год <<Сумма знаний по арифметике\ldots>>, написанная \sauth{Лукой
  Пачолли}{Пачолли, Лука} (работал тоже в Болонском университете). Про
уравнения третьей степени написал, что <<искусство алгебры ещё не
дошло до того, чтобы решать кубические уравнения, подобно тому как и
не научилось квадрировать круг.>>

\sauth{Лука Пачолли}{Пачолли, Лука} (\pe{{\sim}1445}{{\sim}1515})
Кроме уже упомянутых, имеется ещё книга <<О божественной пропорции>>,
которую он написал по просьбе \sauth{Леонардо да Винчи}{да Винчи,
  Леонардо}.

\subsubsection{Борьба с уравнениями ну очень высоких степеней}

Как уже было сказано, это был век алгебры. В $16$ м веке были получены
формулы для решения кубического уравнения в радикалах. Вообще говоря,
квадратные уравнения появились ещё в древнем Вавилоне. И их там умели
решать. В древней Греции их пытались решать, но именно циркулем и
линейкой. Кроме того, побочным продуктом этих попыток стали коники.

Кроме того, \sauth{Омар Хайям}{Хайям} провёл классификацию. Но ничего
не было решено. Много авторов писали о том, что вот, дескать, решили,
но всё это было неправдой. И греки не могли.  И когда итальянцы решили
то, что не умели решать греки, вот тут они подумали,что превзошли
своих богов (греков).

1515 год \auth{Сципион дель Ферро} научился решать приведённые
уравнения третьей степени.  Правда, рассматривали три с виду различных
(но на самом деле ничем не отличающихся) типа уравнений (в зависимости
от знаков коэффициентов, наверное).

Итак, Сципион нашёл формулу для уравнения $x^3 + px = q$. Однако он не
раскрывал формулу для народа.  Но пользовался, а потом уже когда
собрался помирать, сообщил сию формулу ученику \auth{дель Фьеро}.  Но
ученик был недалёким, и вот так другие типы уравнений не разрюхал.

А был ещё такой немеряно крутой чувак \sauth{Никкола Фонтана}{Фонтана}
(\pe{1500}{1557}). Это был самоучка.  Но его прозвали \auth{Тарталья}
(он стал заикой после того, как ему повредили гортань).  Дель Фьеро
вызвал Тарталью на математический диспут, думая, что тот не решит
кубических уравнений.  Но по легенде, в ночь перед диспутом Тарталья
сам вывел все формулы и всё решил.  Это было в~1535~году.

Тарталья готовил книгу, в которой собирался открыть секрет. Но не
успел.

И тут, как назло, появился \sauth{Джироламо Кардано}{Кардано,
  Джироламо} (\pe{1501}{1576}). Он был очень разносторонним человеком,
написал 16~книг, по математике была всего одна. Погиб по своему
гороскопу.  Вылечил епископа от ожирения.

Его самое знаменитое произведение ($1545$ год) \emph{Ars Magna}
(великое искусство), или наука об алгебраических правилах. Но были ещё
его книги про арифметику, геометрию\ldots

Он выманил Тарталью в свой город, и уговорил его выдать его секрет. И
поклялся, что не издаст, пока тот сам не напечатает. И не сразу всё
было понятно.  Кроме того, его ученик \sauth{Лодовико
  Феррари}{Феррари, Лодовико} свёл 4 ю степень к 3 ей. Он таки нашёл
рукописи Сципиона, и таки опубликовал своё произведение. Так он и стал
первооткрывателем.

Но и тут была одна загвоздка. Уравнение для $x^3 = 15x + 4$ даёт
комплексные корни в радикалах.  И вот, уже тогда увидели, что иногда
бывает плохо. Это называлось <<неприводимый>> случай.

Разбирая случай квадратного уравнения с отрицательным дискриминантом,
Кардано заметил, что вообще говоря можно попробовать ввести какие то
числа, квадраты которых будут отрицательными.  Но у него это далеко не
пошло.

А вот \sauth{Рафаэль Бомбелли}{Бомбелли} (\pe{1526}{1579}) в 1550 году
пишет свою <<Алгебру>> (5~книг).  И вот в 3-ей книге, он, изучая труды
\sauth{Диофанта}{Диофант}, переписал 143 его задачи и ввёл мнимые
числа аксиоматическим образом. Он вводит $\pm\sqrt{-1}$ и определяет
операции в полученном расширении поля. И вот, с этими комплексными
числами он и напускается на формулу Кардано. После этого он говорит,
что ну вот, видим, чему равны корни, всё сокращается и всё
получается. В~1572~году он изложил полную теорию~$\Cbb$.

Ну и он, кстати, уже ввёл нормальные обозначения для степеней число с
дужкой внизу.  И таким способом он легко записывал длинные
алгебраические выражения.

\subsubsection{Франсуа Виет}

Так, ну теперь немного про \sauth{Франсуа Виета}{Виет, Франсуа}
(\pe{1540}{1603}). Самое главное это буквенное исчисление.  Родился в
семье юриста, в Тулузе. Это было время Дюма (гугеноты католики и всё
такое).  Исследовал выражения $\sin nx$, $\cos nx$. Его взяли на
службу к королю Генриху $n$ му.  В частности, Виет дешифровывал письма
от испанцев, чем доставил большой геморрой всем, что на него
жаловались даже Папе Римскому.

\sauth{Адриан ван Роомен}{Роомен, Адриан ван} поставил задачу о
решении уравнения
$$x^{45} \bw- 45x^{43} \bw+ 945x^{41} \bw+ 9534x^5 -3795x^3 + 45x =
A,$$ где
$$A = \sqrt{1\frac34-\sqrt{\frac{5}{16}} - \sqrt{1\frac78 -
    \sqrt{\frac{45}{64}}}}.$$ Виет нашёл корни, увидев, что $A$
сторона правильного $15$ угольника.

И вообще, в этот момент символьное исчисление победило.  Именно
поэтому символика Виета считается одним из самых важных его
изобретений.  До Виета написать хотя бы что то вроде $ax^2 + bx + c =
0$ никто не умел. Работа Виета <<Введение в аналитическое искусство>>
была написана в~1591~году. Виет делил алгебру на две части. Простая
часть числовая алгебра (logistica numerosa), вторая часть более важная
(logistica speciosa).

Виет предлагал разделить величины на ступени (по нашему степени).  Для
величин одной ступени мы определяем их сложение, вычитание.  Далее,
при умножении двух первоступенных величин получаем величину второй
ступени.

Как и у греков, складывать величины разной ступени было нельзя
(фактически, это конструкция градуированной алгебры). Он использовал
буквы $A, E, I, O$ для неизвестных, $B, C, D$ для известных.

\begin{ex}
Уравнение $x^3 + 3bx = 2d$ записывалось примерно так:
Пишем $$A~cubus\bw+B~planoin~3A~aequatur~D~solido~2.$$ Здесь
<<\emph{planoin}>> плоский, <<\emph{aequatur}>> равенство,
<<\emph{solido}>> телесный.
\end{ex}

\begin{note}
Знаки <<плюс>> и <<минус>> были введены уже в 15~веке каким-то Яном.
\end{note}

\begin{ex}
Уравнение $x^3 \bw+ 30 x^2 \bw+ 44 x \bw= 1560$, как ни странно, Виет
писал примерно так: <<$1\;C \bw+ 30\;Q \bw+ 44\;N\;aequatur\;1560$>>.
Это уравнение записано для чисел, поэтому тут нам на размерность
наплевать.  Здесь C <<\emph{cubus}>> (куб), Q <<\emph{quadratus}>>
(квадрат), N <<\emph{numerus}>> (число).
\end{ex}

Сумму степеней (кубов, 5 х степеней, их разности) Виет тоже знал.  Но
общая формула для $n$ й степени уже принадлежит
\sauth{Ньютону}{Ньютон}.

Немного особняком стоящая работа \sauth{Виета}{Виет} <<Genesis
Triangulorum>> как показали исследования 20~века работа, в которой
нашли первые попытки ввести комплексные числа.  Но там всё было
довольно непонятно, поэтому нельзя говорить о том, что Виет хорошо
рубил в комплексных числах.

\marklec{11}{16.03.2006}

\section{Европа XVII века}

Когда мы говорили про арабский Восток, мы уже отмечали, что это была
замкнутая группа учёных, там практически не было <<утечки
информации>>.  А вот в Европе всё уже стало по другому. Обмен был, и
очень значительный.

17 век это век переменных величин.  Это период, когда математика
начинает изучать движение.  Появилась система координат
(\auth{Декарт}). Самое важное создание математического анализа
\sauth{Ньютоном}{Ньютон} и \sauth{Лейбницем}{Лейбниц}.  Но не только
эти двое были создателями математического анализа: практически все
занимались кто касательными, кто интегрированием, кто ещё чем-нибудь,
связанным с дифференциальным и интегральным исчислением.

Первой страной, в которой произошёл большой всплеск науки, это была
\textbf{Италия.}  Это, например, такие люди: \auth{Галилей},
\auth{Кавальери}, \auth{Торричелли}.  Но тут была инквизиция, Галилея
преследовали.  (Это, кстати, вызвало резонанс не только в Италии, но и
в других странах).

\textbf{Германия.} Это была раздробленная в княжества страна, что
 не способствовало прогрессу и развитию науки. Подъём произошёл
позже, во времена \sauth{Эйлера}{Эйлер}. Но это уже следующий
век. Однако \auth{Кеплер} был одним из тех немцев, кого можно было
назвать крутым.  Кроме того, был ещё \auth{Лейбниц} (но это уже самый
конец 17~века).  Он официально был библиотекарем, но вообще был
довольно разносторонним человеком. Известно, что он встречался с
\sauth{Петром Первым}{Пётр Первый} и помогал организовывать академию.

\textbf{Англия} и \textbf{Шотландия.}

\sauth{Джон Непер}{Непер, Джон}. Шотландский барон, создатель
логарифмов.  Были ещё \auth{Валлис}, \auth{Барроу} (учитель
\sauth{Ньютона}{Ньютон}) Кроме того, многие приезжали в Англию, потому
что по религиозным взглядам многие не выдерживали Европы.

\sauth{Н.~Меркатор}{Меркатор, Н.} и \sauth{Г.~Меркатор}{Меркатор, Г.}
продвинули логарифмы, и, что самое важное, создали географические
карты (проекция Меркатора).  Один из них проектировал фонтаны в
Версале, второй зарабатывал созданием карт.

\textbf{Франция.} Ключевые фигуры там это \auth{Ферма} и
\auth{Декарт}.  Ферма был юристом. Декарт был из небогатых дворян, но
хватало на то, чтобы прожить.

\auth{Мерсенн} (священник) был координатором переписки многих учёных.
Ему писали многие учёные, он пересылал их письма другим учёным.

\subsection{Европейские Академии наук}

\begin{items}{-2}
\item 1603~г. Италия. Академия <<Accademia dei Lincei>>
  (<<\emph{lincei}>> рысь).
\item 1662~г. Лондонское Королевское Общество.
\item 1666~г. Парижская Академия наук.
\item 1700~г. Берлинская академия наук.
\item 1725~г. Петербургская академия наук.  Но это уже был 18~век, и
  тут в поддержании учёных уже играло роль и государство тоже.  Потому
  что они много делали именно для государства. Госзаказ дело
  оплачиваемое.
\end{items}


\subsection{Логарифмы}

Логарифмы появились в математике почти как гром среди ясного неба.  У
них нет предистории. Зачем они были нужны? Идея: нужно было составлять
таблицы.  Мы хотим умножение заменить сложением, потому что умножать
сложно, а складывать легко.

Надо сказать, что некоторые тригонометрические формулы тоже направлены
на преобразование умножения в сложение. Например, $\sin a \cdot \sin b
= \frac12 \hs{\cos(a-b) - \cos(a+b)}$. Вместо того, чтобы перемножать
два синуса, проще было по таблицам вычислить два косинуса и вычесть
их.

15~век. \sauth{Никола Шюке}{Шюке, Никола}. Он заметил, что степени
складываются при умножении.  Иначе говоря, он открыл закон $a^m \cdot
a^n = a^{n+m}$.

16~век. Идею продвинули. \auth{Штифель} написал степени двойки
(положительные и отрицательные) и сами показатели. Но дальше почему то
не пошёл: <<Я мог бы написать целую книгу про свойства этих чисел, но
я должен пройти этого мимо с закрытыми глазами.>> Он, анализируя
цифирь Священного Писания, пообещал конец света в 1533 году, но когда
этого не случилось, ему пришлось сваливать куда подальше, чтобы не
убили.

Голландец \sauth{Симон Стевин}{Стевин, Симон}
(\pe{1548}{1620}). <<Десятая>> так он называл десятичные дроби. Я не
изобрёл, я открыл их, говорил он. Кроме того, ввёл сложные
проценты. Именно, рассматривал $(1+r)^n$, $r = 0.04$, $r = 0.05$.  Но
это не удивительно, он был бухгалтером, поэтому ему это было нужно.

\auth{Бюрги} (1620~г.) работал у \sauth{Кеплера}{Кеплер}, открыл
логарифмы. Но он был скромным человеком, и никому не рассказал о том,
что он открыл. Поэтому \auth{Непер} ругал его за эту скромность.  Так
получилось, что он опубликовал свои результаты уже после того, как это
сделал Непер, хотя само открытие сделал первым.  Непер опубликовал
свои таблицы уже в 1614 году и они сразу расползлись по всему миру.

Бюрги написал таблицы для $r = \frac{1}{10^4}$. Расписал $a_n\cdot
10^8$, и тогда получится так:

$$\mat{10^8 & 0\\ 10^8\cdot\hr{1+\frac{1}{10^4}} &
  10\\ 10^8\cdot\hr{1+\frac{1}{10^4}}^2 & 20\\ \hdotsfor{2}}$$ Степени
писал красной краской, числа чёрной.  Итак, у него получилось
соотношение:
$$R = \log_{\sqrt[10]{1+\frac{1}{10^4}}} \hr{\frac{B}{10^8}}$$

\auth{Непер} (\pe{1550}{1617}).  Учился в Эдинбурге, в 21~год закончил
университет, потом жил в своём поместье.  Очень активный человек, был
<<предводителем дворянства>>, и конкретно протестовал против
католиков. Например, доказывал <<теоремы>>, утверждающие, что Папа
Римский $=$ Антихрист.

Его антицерковные книги пользовались довольно большой популярностью во
Франции, Англии, Голландии, Германии.  Его таблицы были известны
гораздо меньше в народе.  Он предсказал конец света в \pe{1689}{1700}
году, но этого, естественно, не случилось.  После этого ему, правда,
перестали верить.

1614~г. вышла из печати книга Непера <<Описание удивительной таблицы
логарифмов>> (<<\emph{логос}>> знание, \emph{арифмос}>> число).  Но
вот беда, графиков не было. То есть нарисовать функцию, обратную к
экспоненте, никто тогда ещё не умел.

Делали так: брали два единичных отрезка, $AB$ и $A_1B_1$.  Пусть по
ним движутся точки $M$ и $m$ (дискретно, шагом в $\frac{1}{10^7}$),
причём $m$ движется равномерно, а скорость $M$ равна в точности длине
того пути, который ей остался до конца отрезка.

В наших обозначениях $\frac{d(1-y)}{y} = -dx$.  Итого получаем решение
$\ln y = -x$.  Таблица логарифмов была такого вида: $x_n \bw= n \cdot
\frac1{10^7}$, $y_n \bw= \hr{1-\frac1{10^7}}^n$.

\sauth{Генри Бригс}{Бригс, Генри} таблицы десятичных логарифмов
1617~год.


\subsection{Аналитическая геометрия}

Ферма и Декарт. 1679~год выход трудов Ферма в свет. Вообще известно,
что \auth{Ферма} писал в духе Древней Греции, простым и понятным
языком.

\auth{Декарт} 1637~год. К тому времени, когда Ферма написал своё
произведение, <<Геометрию>> Декарта уже перевели на разные языки, так
что в основном, конечно, аналитическая геометрия принадлежит Декарту.

Название его работы было такое: <<Рассуждения о методе и его
приложения.  Диоптрика. Метеоры. Геометрия>>.

Однако, ни у \sauth{Декарта}{Декарт}, ни у \auth{Ферма} не было таких
картинок, которые мы рисуем сейчас (то есть система координат была
какая то кривая).

Так вот, Декарт сказал, в отличие от греков, что всё есть число, и
объём, и площадь, и нам неважно, что с чем складывать.

В качестве примера он приводил пример с теоремой о пропорциональных
отрезках: $\frac{a}{1} = \frac{b}{x}$, откуда $x = \frac{b}{a}$, и всё
тут.  То есть он позволил себе забить на размерность.

\textbf{1} я книга: <<О задачах, которые можно строить можно было
только кругами и прямыми линиями>>.  Тут он чётко сказал: это
уравнения второй степени.

\textbf{2} я книга: <<О природе кривых линий>>. Он сказал: нет у нас
никаких механических методов, это всё чепуха. Нужно исследовать только
алгебраические кривые. А на другие у нас нет метода.

\textbf{3} я книга: <<Изучение полиномов>>.

Кстати, \auth{Лейбниц} раскритиковал Декарта, потому что применял
степенные ряды там, где методы Декарта не годились (Декарту было
неведомо, что можно было делать с трансцендентными функциями, которые
представляются именно рядами.  Его подход был чисто алгебраический,
все его построения не выводили его за класс алгебраических чисел.)

\marklec{12}{21.03.2006}

\section{Математический анализ XVII века}

Итак, наша тема сегодня дифференциальное и интегральное исчисление.

Это вторая треть 17~века. Раз это 17 й век, то вспомним, что тогда
было в математике.  Так вот, 16~век это век алгебры, а 17~век век
математического анализа.

Мы вспомним таких граждан, как \sauth{Кеплера}{Кеплер} и
\sauth{Галилея}{Галилей}.  В их работах возрождаются инфинитезимальные
методы античности, созданные \sauth{Архимедом}{Архимед}. Тут нужно
вспомнить даже \sauth{Пьера Ферма}{Ферма, Пьер}. Опричь его знаний про
теорию чисел, он много продвинул теории для математического анализа.
После работ Декарта его принцип однородности стал не нужен, и работы
Ферма быстро забыли.  Но важно то, что он сделал кое что и в
математическом анализе.

\textbf{Интегральное исчисление.} 1-я треть 17 века. Кроме
<<Стереометрии винных бочек>>, которую продвигал \auth{Кеплер},
интегралы считал \auth{Ферма}.  Он посчитал площадь под параболой. Он
брал отрезок от 0 до $x$ и считал площадь сегмента: делил отрезок на
кучу частей и рассматривал их сумму.

Он проделывал это только для фиксированной параболы. Там, правда,
нигде не используется то, что это именно парабола. То есть он посчитал
предел интегральных сумм, но сделал это общим способом. Ещё бы чуть
чуть, и он первым создал бы теорию интегрирования.

Кроме Ферма, примерно этим же занимались ещё трое товарищей.  Именно,
был вычислен интеграл $\intl{0}{x}x^n\,dx \bw= \frac{x^{n+1}}{n+1}$,
$n \in \Q$.  Это сделали \sauth{Бонавентура Кавальери}{Кавальери,
  Бонавентура}, \sauth{Эванджелиста Торричелли}{Торричелли,
  Эванжелиста}, \sauth{Жюль Роберваль}{Роберваль}.

Кроме того, эти математики забили на описанные фигуры, то есть решили,
что достаточно переходить к пределу только по вписанным фигурам. Ещё
они упразднили необходимость доказательства от противного: напомним,
что \auth{Архимед} всякий раз, вычисляя тот или иной предел, потом ещё
доказывал от противного, что он единствен.

Самое интересное, что опять возник базар на тему неделимости.
Некоторые считали, что бесконечно малые всё таки конечны.  И вот с
ними опять начинают бороться.

\textbf{Задача о касательных.} Первым был, естественно, \sauth{Пьер
  Ферма}{Ферма, Пьер}.  Опять-таки, он работал только с параболой. Для
того, чтобы её провести, нам потребуется знать длину отрезка проекции
положительной части касательной на $OX$.  Кроме того, Ферма сказал,
что касательная это \emph{предельное положение секущей}.  Это
1629~год.  \cpic{5} Пусть $M$ точка касания, пусть $T$ точка
пересечения касательной с $OX$, далее, $P$ проекция на $OX$ точки
$M$. Дадим точке $P$ приращение (бесконечно малое).  Нам нужен отрезок
$NQ \sim f(x +h)$, $h = PQ$, $N$ точка на касательной.  (За неточность
формулировок его, кстати, часто ругал \auth{Декарт}.)  Из подобия
треугольников имеем $\frac{f(x)}{S_t = PT} = \frac{f(x+h)}{S_t + h}$.
Отсюда выражается число $f(x+h) = \hr{1 + \frac{h}{S_t}}f(x)$.  Ферма
работал с многочленами, так что ему легко было всё разлагать в ряды по
степеням $h$.  Он получал разложения $f(x+h) \bw= f(x) \bw+ A(x) h
\bw+ B(x)\cdot h^2 +\ldots$, и у Ферма ряд всегда обрывался (а хитрые
англичане пошли ещё дальше, и сказали, что то же самое можно делать и
не для многочленов).

\auth{Ферма} (а потом и \sauth{Лейбница}{Лейбниц}) любили пинать за
то, что он неосторожно обращается с бесконечно малыми.  Он говорил
примерно так: <<А потом сокращаем на $h$, и кладём $h = 0$.>> Многим,
не привыкшим в повседневной жизни делить на нуль, это казалось
дикостью: как же так, сначала делим, потом кладём равным нулю?\ldots

\auth{Ферма} пытался идти по пути арифметизации математики.

А в это время в Англии \sauth{Исаак Барроу}{Барроу, Исаак} был
заведующим кафедрой математики в Кембридже. Он всё это видел, и ему
это не нравилось, потому что не казалось строгим.  Барроу был первым,
кто заметил, что дифференцирование обратно интегрированию.  Он создал
некоторую теорию интегрирования и дифференцирования, но всё это было
тоже ещё не то, что можно было назвать единой теорией.  К середине 17
века аналитические методы интегрирования и дифференцирования кое-как
оформились.  Но придать полную форму всему этому безобразию смогли
только \sauth{Исаак Ньютон}{Ньютон, Исаак} (Англия) и \sauth{Готфрид
  Вильгельм Лейбниц}{Лейбниц, Готфрид Вильгельм} (Германия). Это была
вторая половина 17~века.  Заметим, что термин <<производная>> ввёл
\auth{Лагранж}, а <<интеграл>> \auth{Лейбниц}.

\subsection{Исаак Ньютон}

\sauth{Исаак Ньютон}{Ньютон, Исаак} (\pe{1643}{1727}). Родился в 75
километрах от Кембриджа, в поместье фермера. Мать его противилась
тому, чтобы он учился в колледже Тринити (1661).  Кстати, ему не
нравилась другая наука, кроме математики, поэтому его средний балл в
колледже был невысок.  Спасло его то, что в \pe{1665}{1666} году была
чума, Ньютон уехал подальше, и вот тогда ему никто не мог помешать
заниматься математикой.

Неизвестно, в силу каких обстоятельств, возможно, в силу характера
своей матушки, Ньютон по жизни не любил женщин, и так и остался на всю
жизнь холостяком.

В 1669 году он занял кафедру математики после смерти \auth{Барроу}.
Кстати говоря, в своих конспектах он не упоминал Барроу, так что на
самом деле считал себя самоучкой, хотя, несомненно, Барроу был его
учителем.

В это время Ньютон сбацал свой знаменитый бином, изучил оптику.


\textbf{Работы Ньютона.}

1669 (опубл.~1711) Анализ при помощи уравнений с бесконечным числом
членов.

1671 (опубл.~1736) Метод флюксий и бесконечные ряды.

1676 (опубл.~1704) Рассуждение о квадратуре кривых.

Он не любил, когда его критиковали. Поэтому он не публиковал работы
сразу.  Когда он построил телескоп и притащил его в Лондонское
Королевское Общество (ЛКО), его избрали в него.  В 1672~году написал
работу по теории света и излучения, за что его стразу возненавидели
\auth{Гук} и \auth{Гюйгенс}.  Ньютон очень сильно обиделся. Тем не
менее, он был избран в 1703~году президентом ЛКО. И вот тут он начал
публиковать свои работы, понимая, что теперь в случае гнилого базара
будет что возразить.

В 1696~году его пригласили в Лондонский монетный двор, там работа была
проще.  Там нужно было бороться с фальшивомонетчиками, и надо сказать,
что он справился с задачей чеканкой новой монеты (тут проявились его
организаторские способности).  Но на университет к тому времени забил.

Знаменитый спор между Ньютоном и Лейбницем разразился в конце 17~века.
И вот тут надо сказать, что Ньютон был неприступен. Он не хотел
признавать, что кто то ещё претендует на первенство в открытии
дифференциального исчисления.  Однако сам он иногда впадал в
скромность (его слова про берег океана).

Итак, про математику. Бесконечные ряды. Ему не нужно было никакой их
сходимости, он работал с ними формально. Однако, когда он говорит, что
$\frac{1}{1+x^2} = 1 - x^2 + x^4-\dots$, то пишет, что $x \ra 0$, а
при $x \ra \bes$ пишет разложение по обратным степеням совершенно
спокойно, то есть $\frac{1}{1+x^2} = \frac{1}{x^2} - \frac{1}{x^4} +
\dots$.

Но основные его задачи это были задачи механики. И вот для обоснования
того, что он делал в механике, он стал вводить свои флюксии и флюэнты.
Он говорит, что флюксия это производная, а флюэнта собственно,
функция.

Когда \auth{Лейбниц} написал Ньютону: толкни мне свою теорию, чувак!
тот в ответ написал анаграммой.  Это были две зашифрованные фразы:
<<По данному уравнению найти флюксии из флюэнтов, и наоборот>>.  И
второе <<умение решать дифференциальные уравнения методом подстановки
степенных рядов>>.

Он пытался обойти проблему деления на нуль при оперировании с
бесконечно малыми так: их нужно делить одно на другое, тогда что-то
будет получаться. Кроме того, Ньютону тоже стало понятно, что
интегрирование противоположно дифференцированию (ведь это знал ещё
\auth{Барроу}).

\subsection{Готфрид Вильгельм Лейбниц}

Итак, выясним, что же мы знаем про товарища по имени \sauth{Готфрид
  Вильгельм Лейбниц}{Лейбниц, Готфрид Вильгельм} (\pe{1646}{1716})?
Философ и юрист. Но когда он закончил университет (ему было тогда 20
лет), и накатал диссертацию, то её даже не приняли. Но в 1666 году он
наконец защитил диссертацию по юриспруденции.

Он пишет, что хотел путешествовать и ботать математику. Кроме
математики, он занимался много чем. Чтобы получить работу, ему нужно
было служить всяким местными правителям, и, кстати, умер в забвении,
был довольно беден в итоге.  Он ездил в Париж со своим арифмометром,
но кроме того, он вёл большую политическую работу. Кроме того, в
1673~году поехал в Лондон, везде забивал всем стрелки, показал в
Лондоне арифмометр, просил у Ньютона рукопись, но Ньютон его послал.
И вообще в ЛКО его сочли молодым дилетантом. Но вот, подружившись с
\sauth{Гюйгенсом}{Гюйгенс}, он сам начал ботать математику (это было
уже достаточно поздно).

Некоторое время живёт в Париже, потому что ему поначалу не досталось
места.  Потом в 1676 году уезжает в Ганновер, там работает
библиотекарем.  Но там он начинает всякую разную работу, помогает
князю, и этот князь сильно продвинулся в политике.

В~1700 году в Берлине создана Академия наук, по инициативе и
рекомендации Лейбница. Кроме того, по его советам была создана
Российская академия (а всё то началось с устройства брака русского
царевича и какой то там принцессы).  Но тем не менее он ничего почти
не собрал на жизнь.

\textbf{Работы Лейбница.}

1684 Новый метод для максимумов и минимумов, а также касательных, для
которого не служат препятствием ни дробные, и иррациональные величины
и особый для этого род исчисления.

1686 О глубокой геометрии и анализе неделимых и бесконечных.

Первый научный журнал \emph{Actae eruditorium}. Там собственно,
говоря, он и написал статейку всего на 7~страниц с ошибками и
опечатками.  У него просто не было времени на более фундаментальную
публикацию.

Характеристический треугольник функции $dx,dy$. На самом деле, он
говорил, что касательная есть, когда существует конечный предел
$\frac{dy}{dx}$.  Это было очень похоже на то, о чём говорил
\auth{Ферма}. Далее он говорит, что для них имеют место формулы для
суммы, произведения, тп

Впрочем, из его работ всё равно не совсем понятно, что такое были его
эти бесконечно малые.  То есть строгости всё равно не было. Он говорит
о том, что бывают и дифференциалы высших порядков.  Более того,
притягивает сюда за уши философию.

Плюсы: обозначения (они так и остались). Считал как то раз Лейбниц
площадь под графиком.  Увидел он сумму $\hbox{Summa }y_i dx_i$, ну а
потом сама <<Summa>> просто вытянулась в интеграл.

С его наукой познакомились братья \auth{Бернулли}. Так вот, они читали
курс, а потом лекции остались \sauth{Лопиталю}{Лопиталь}, и вот он,
гад, издал этот курс.  И потому появилось правило Лопиталя.

Говорят, что \auth{Ньютон} тратил слишком много на спор с Лейбницем.

\subsection{Дерево Бернулли}

\auth{Бернулли}. Вот их фамильное дерево:

 \cpic{6}

Один из трёх сыновей Николай был художником.  Николай~II и Даниил
стали первыми российскими академиками.  У Иоганна~I учился
\auth{Эйлер}, который тоже был российским академиком, потому что его
тоже пригласили в Россию.

\marklec{13}{23.03.2006}

\section{Математика XVIII века}

\subsection{Математика в Европе}

Как известно, 18~век век просвещения. С одной стороны, это капитализм,
который укрепляется.  Англия захватывает новые колонии, вытесняя
Францию и Португалию. Кроме того, все ценят образование, учёных,
создаются академии. Просвещённые монархи приглашают учёных, в общем,
процесс идёт.

1789 буржуазная революция во Франции. Но её подготавливали
\auth{Вольтер}, \auth{Руссо} просветители.  Кроме того, была
подготовлена энциклопедия (а называлось это <<Толковый словарь наук,
искусств и ремёсел>>).  Это было 28 томов, изданных в
\pe{1754}{1772}. Главным издателем там был \auth{Дидро}, а
\auth{Даламбер} писал в ней статьи по математике.

С одной стороны все её приветствовали, а с другой стороны она
подрывала основы монархии, и в итоге Дидро прогнали. \sauth{Екатерина
  Великая}{Екатерина} тоже потом поняла, что это плохо для её
монархии, и забила на просвещение.

Была создана механистическая картина мира. То есть мир огромный
механизм, который работает по законам дифференциальных
уравнений. Поэтому матан и дифуры главные предметы изучения.  И вот
тут потребовался многомерный матан функции многих переменных, кратные
интегралы.

Тут появились и степенные ряды, и ряды \auth{Фурье}. Аналитическая
геометрия вышла в трёхмерное пространство.  \auth{Лаплас} говорил, что
если мы имеем дифференциальное уравнение Мира, и начальные условия, то
мы знаем всё.  Тогда мы всё можем предсказать, а равно заглянуть в
прошлое.  \authorcomment{К счастью, он не был знаком с теорией хаоса в
  динамических системах.}

\auth{Ньютон} сказал, что Бог просто дал нам импульс для изучения
математики.  \auth{Лаплас} сдал свои труды
\sauth{Наполеону}{Наполеон}, и тот спросил а почему в его работах нет
ни слова про Бога?  А Лаплас ответил: <<Я не нуждался в этой
гипотезе.>>

Важно, что в основном всё работало на механику.

\auth{Ньютон} Математические начала натуральной философии. Это
1687~год.  Но важно, что тогда ещё никто, кроме него не знал его
метода флюксий.  И вот тогда он решил, что это всё надо написать так,
чтобы все поняли, то есть в понятных терминах.

А вот \sauth{Эйлеру}{Эйлер} было проще. Он написал свою <<Механику>> в
1736~году уже с использованием нормального математического
анализа. Кроме того, в 1788 году \auth{Лагранж} написал
<<Аналитическую механику>>. Идея такая: есть только матан, а механики
как таковой нет, есть просто обычные дифференциальные уравнения, и
всё. Это вогнало его в колоссальный депресняк.

\pe{1799}{1825} \auth{Лаплас} написал 5 томов своей <<Небесной
механики>>. И в конце 18~века уже появлялись крамольные мысли: а что
дальше то делать? Типа, всё известно.  И потому 19 век это век
абстрактной науки.

\textbf{Франция:} тут были \auth{Лагранж}, \auth{Лаплас}, \auth{Клеро}
(он продвинул трёхмерную аналитическую геометрию), \auth{Монж}
(начертательная геометрия), \auth{Даламбер}, \auth{Мопертюи}
(измерение меридиана, принцип наименьшего действия).

В Париже Была создана политехническая школа (Ecole Polytechnique,
1793) и Нормальная школа (\`Ecole Normale) для учителей.  EP это была
самая крутая школа, многие не выдерживали, и ходили жаловаться.  Идея
была в том, чтобы там преподавали те, кто, собственно, делает науку.

\textbf{Англия:} тут всё было беднее, а именно: \auth{Муавр},
\auth{Стирлинг}, \auth{Тейлор}, \auth{Маклорен}, \auth{Варинг},
\auth{Байес}.  В Англии не поддерживали науку на государственном
уровне. Королевское общество никак не спонсировало бедных, это как раз
было общество состоятельных людей.  Они консервативно не признавали
то, что делалось в Европе. Поэтому и отстали, разрыв был сокращён
только в~19~веке, но это были совсем другие люди.

Вообще, 18~век демократизация состава учёных.  Это было очень важно,
потому что до этого математикой занималось фактически дворянство.

\auth{Лагранж} (\pe{1736}{1813}) учился на адвоката в Турине. Его отец
был французом, но сам жил уже в Италии. Он выиграл несколько
конкурсов, получил кучу денег, и\ldots Итак, в~1755 году стал
профессором артиллерийской школы, с 1766 года профессор Берлинской
академии наук. С 1788 года живёт во Франции. С 1995 года профессор
Политехнической школы.

Самые важные работы Лагранжа: 1797 теория аналитических функций, и
1801 лекции об исчислении функций.

Вообще говоря, когда заваривалась война 1812 года, всех иностранцев
погнали из Франции.  Но вот его оставили, хотя он был итальянцем,
потому что немеряно много сделал для дела Франции.

\auth{Даламбер} (\pe{1717}{1783}). Его главная работа Математический
анализ.  Это дифференциальные уравнения, математическая физика, и
обоснования анализа (статьи в Энциклопедии).  Он был
незаконнорожденным, поэтому его подкинули в семью ремесленников.  Но
потом, когда он стал крут, то, когда его вновь нашли, он отказался от
состояния и титула.

\auth{Лаплас} (\pe{1749}{1827}) Крестьянская семья, учился в
монастырской школе.  В 1766 году приехал в Париж. Он кормил Даламбера
рекомендациями, но крутой Даламбер ему не верил. Потом он как-то раз
дорешал ему задачу, и тот пропёрся, и взял его на работу.  Более
точно, стал преподавать в Артиллерийской школе. В 1799 году стал
министром.

\auth{Фурье} \pe{1768}{1830} сын портного, учился в школе при церкви,
потом стал профессором EP.  Большое его открытие его ряды.

Это была его идея читать лекции стоя, и во вторых, что нужна история
математики.  Главное, не надо лекции превращать в диспут. Типа,
читайте как читается\ldots

\subsection{Математика в России}

1701 школа математических и навигацких наук.  Пригласили англичан, но
они (вот беда!), не знали русского. А русские не знали английского. К
счастью, тут Магницкий поспособствовал. Он переводил тексты на
русский, и обратно.

1711 Инженерная школа. 1712 Артиллерийская школа. Всё это было в
Москве.  Потом появилась Морская Академия в Петербурге в 1715
году. Первые школы принимали всех некрепостных. А вот Морская академия
была для дворян.

1725 год создание Академии Наук. Они сразу стали издавать журнал
записки академии наук (они назывались commentarii). Кроме того, в 1755
году появился МГУ. Тут, надо сказать, был солидный прогресс: студентов
академии (первых) набрали вообще из Европы, а учить их никто не хотел
-- все хотели двигать науку.  Другое дело МГУ, тут сразу дело пошло как
надо.

\auth{Лейбниц} был знаком с Петром~I. Кроме того, у него было двое
знакомых математиков \sauth{Иоганн Бернулли}{Бернулли, Иоганн}
(\pe{1667}{1748}) и \sauth{Якоб Бернулли}{Бернулли, Якоб}.  У Иоганна
были сыновья \sauth{Николай}{Бернулли, Николай} (\pe{1695}{1726}) и
\sauth{Даниил}{Бернулли, Даниил} (\pe{1700}{1782}). И вот их то и
послали в Россию.  Потом они вытащили сюда \sauth{Эйлера}{Эйлер},
правда, пришлось сначала найти ему место на медицинском факультете,
потому что других вакансий не было. Но Эйлер, будучи человеком
продвинутым, заботал медицину и приехал в Россию. Ну а потом,
естественно, получил и нормальное место.

\auth{Эйлер} (\pe{1707}{1783}). Когда он умер, то про него сказал кто
то из великих, что он перестал вычислять и жить.  Жил в Петербурге
(\pe{1725}{1741}), Берлине (\pe{1741}{1766}) и снова в Петербурге
(\pe{1766}{1783}).  Написал более 850 работ, 40 монографий. Он был
большим практиком.

Когда появилась \auth{Анна Иоанновна}, тут платить перестали, и он
свалил в Германию.  Но регулярно получал работы русских математиков,
писал отзывы. Но вот его правитель в Германии, Фридрих, не был большим
математической деятелем, и сказал, что ему непонятно, чем тот
занимается.  Эйлер обиделся и вновь уехал в Россию.

\subsection{Борьба с функциями}

Сам термин <<функция>> был введён \sauth{Лейбницем}{Лейбниц}. Кто то
из многочисленных \auth{Бернулли} сказал, что функция это
аналитическое выражение.  \auth{Эйлер} сказал, что функции бывают
алгебраические и трансцендентные.  Алгебраические делятся на
иррациональные и рациональные, а рациональные на целые и дробные.  Что
до трансцендентных, то тут он выделял экспоненту с логарифмом и
тригонометрические функции.

\auth{Эйлер}: функция непрерывна, если она задана одним аналитическим
выражением.  Обратные функции тоже непрерывные.

А непрерывность в классическом понимании он называл связностью. Это
то, что можно написать одним росчерком пера. И вот тогда стали думать,
какие функции допустимы в математике.

Речь шла о задаче колебания струны. \auth{Даламбер} вывел УрЧП для
волнового уравнения $\pf{^2 u}{t^2} = a^2\pf{^2u}{x^2}$.  В
1747~Даламбер сказал, что общее решение есть $u = f(x+at) +
g(x-at)$. Эйлер через год сказал, что нужно ещё задать начальное
условие, чтобы можно было искать решение.

Самое интересное, что говорили: $f$ и $g$ любые функции. Но что такое
\emph{любые}?  Сам \auth{Даламбер} сказал, что должна быть дважды
дифференцируема и непрерывна по \sauth{Эйлеру}{Эйлер}.  А иначе у нас
нет и самого аппарата. А вот сам Эйлер сказал так: что неважно, есть
ли средства, или их нет: нужно исходить из того, что говорит физика в
данной задаче.

\auth{Эйлер} сказал, что нельзя ограничиваться <<непрерывными>>
функциями. Нужна связность, а не непрерывность. Их надо изучать, хотя
не знаем никакой формулы.  Конечно, неправы были оба.

Основным аппаратом были степенные ряды. \sauth{Даниил
  Бернулли}{Бернулли, Даниил} сказал, что решение надо искать в таком
виде:
$$y = \al \sin \frac{\pi x}{l} + \be \sin \frac{2\pi x}{l} +
\ga\frac{\sin 3\pi x}{l}+\ldots.$$ При этом коэффициенты $\al, \be,
\ga\etc$ зависят от времени. Фактически это уже были коэффициенты
Фурье, но они применяли всё это к конкретным задачам. Всё было сложно.

В 1807 году (1822 второе издание, стереотипное) Фурье выписал ряд по
синусам и косинусам:
$$f(x) \sim \frac{a_0}{2} + \sumnui (a_n \cos nx + b_n \sin nx).$$
\auth{Лагранж} сказал (он был тогда уже немолодой): <<Да, я тоже всё
это писал. Но вот вы сходимость то изучите!>> Но Фурье не послушал
совета. Он ничего не сумел обосновать, и опубликовал копию своей
книги.

Но кое что сдвинулось, правда, не сразу. В 19~веке поменялось понятие
функции. Стали говорить, что функция это соответствие. То есть
элементу области определения ставится в соответствие элемент области
значений.

Ну и вот тут пришёл \auth{Дирихле} со своей
функцией. \auth{Лобачевский} сказал так: Функция есть закон, по
которому $x$ ставят в соответствие $y$. При этом неважно, а знаем ли
мы само правило.


\marklec{14}{28.03.2006}


\section{Математика XVIII века}

Мы будем говорить про 18~век. Тут всего будет много. Век алгебры
16~век, в~17~веке был матан. Тем не менее грань между алгеброй и
анализом ещё не стёрлась. Называлось эта смесь как-то хитро, типа,
<<алгебраический анализ>>.  Так вот, значит, мы поговорим о двух
теоремах: именно, об основной теореме алгебры, и о решении уравнений в
радикалах.


\subsection{Основная теорема алгебры и комплексные числа}

Те, кто хорошо учился на первом курсе, знают, что это теорема о том,
что уравнение $n$ й степени имеет над полем $\Cbb$ ровно $n$~корней с
учётом кратности.

Как уже было замечено нами ранее, комплексные числа появились у
\auth{Бомбелли} в~1572 году.  Он называл их \emph{piu di meno} (корень
из $-1$).

В~1629 году \sauth{Альбер Жирар}{Жирар, Альбер} сказал, что любое
уравнение имеет ровно $n$~корней, с учётом невозможных и
кратностей. \auth{Декарт} (1637~год) говорил, что они
<<воображаемые>>.  Как истинные, так и ложные корни, могут быть
действительными, так и мнимыми.  (<<Ложные>> корни это отрицательные
корни). Мнимые они же комплексные (\emph{imaginaire} воображаемый,
мнимый)

\auth{Эйлер} отбросил кусок слова \emph{i}maginaire и получил
обозначение~$i$.  Однако, что касается самой ОТА, то были только
формулировки. Заметим, что доказательство является не чисто
алгебраическим, там используется матан, топология.

При интегрировании рациональных дробей возникает разложение на
множители.  (\auth{Бернулли}). Так вот, он говорил, что всегда можно
разложить если не на линейные, то хотя бы на квадратичные с
отрицательным дискриминантом.

1742 \auth{Эйлер}: всякий многочлен разлагается в линейные и
квадратичные множители с $D < 0$.

1746 \auth{Даламбер} тоже доказал теорему (опубликовал в 1748~году)
чисто аналитическое доказательство. Даже на том уровне, что было
тогда, оно было нестрогим до крайности.  Работа называлась
<<исследование по интегральному исчислению>>.

Вот его доказательство:

Теорема~1. Пусть есть уравнение, задающее кривую: $P(z,y) = 0$.  Пусть
при $z = 0$ имеем $y = 0$ или $\bes$.  Тогда для любого сколь угодно
малого $z$ соответствующее $y$ может быть либо действительным, либо
комплексным.

Без всяких обоснований он пишет, что $z = a_1 y^{q_1} + a_2 y^{q_2}+
\ldots$, при этом $q_i \in \Q$ монотонно возрастают. Здесь $a_i \in
\R$.  Хвост ни на что не повлияет, потому что он мал. Ну вот и
получается, что оно может быть либо действительным, либо комплексным.

Далее, мы говорим, что если $z_0 \in \R$, все $y_0$ достаточно близкие
будут комплексными.

Далее, то же верно для конкретного $z \in \R$ то же верно, потому что
можно всякое представить как сумму бесконечно малых.

Следствие: то же, что теорема 1, но для всякого $z \in \R$.

Теорема~2. Если многочлен никогда не равен нулю, то всяко существует
мнимый корень, который его таки зануляет.  Чтобы это доказать,
рассмотрим всё, кроме свободного члена, как кривую в теореме 1; тогда
берём $z = -a_m$ и получаем противоречие с п.~3 теоремы.

Теорема~3 (следствие) всякий многочлен разложим, потому что
сопряжённый корень тоже корень.

Как было сказано, это доказательство не оказало влияния на
алгебру\footnote{Одно из самых кратких доказательств теоремы Гаусса,
  которое умещается на страничку и не использует ничего, кроме
  элементарных соображений из математического анализа и свойств
  комплексных чисел, основано на лемме Даламбера. Оно приведено в
  книге Э.\,Б.\,Винберга <<Курс алгебры>>. Кроме того, рассуждения, по
  сути являющиеся доведёнными до строгости словами Даламбера,
  используются при доказательстве некоторых важных теорем комплексного
  анализа. \emph{Прим. наб.}}. Его не время забыли.

1751~год доказательство ОТА \sauth{Эйлером}{Эйлер}. Вообще говоря,
чисто алгебраического доказательства этой теоремы пока не существует
(и вряд ли появится). И вот Эйлер использует несколько утверждений из
математического анализа. Именно:

1. Любой многочлен нечётной степени имеет хотя бы один действительный
корень.  Если корней несколько, то их число нечётно (теорема Коши о
промежуточном значении функции).

2. Многочлен чётной степени с отрицательным свободным членом имеет по
крайней мере один положительный и один отрицательный корень.

Всё остальное это уже алгебра. Итак, мы говорим: пусть нам задано
уравнение, и нехай коэффициенты $x^n + Ax^{n-1} + B x^{n-2}+ \dots + N
= 0$.  Пусть $\al\sco \nu$ корни (символы). Кстати, за использование
объектов неизвестной природы его сильно ругал \auth{Гаусс}.

Рассмотрим $(x + \al)(x + \be)\sd(x+\nu) = P_n(x)$.  Раскрываем
скобки, получаем теорему Виета:
\begin{gather*}
\al \spl \nu = A\\ \al\be \spl \mu\nu = B\\ \ldots\\ \al\be\sd\nu = N.
\end{gather*}
В соответствии с предыдущим говорим так: если степень нечётна, то
неинтересно корень уже есть.  А если чётно: из любого уравнения
степени $n = 2m$ делаем уравнение, дописывая к нему ещё множителей
\authorcomment{Совершенно непонятно, почему это законно: с тем же
  успехом можно сказать и так: пусть мы умеем доказывать, что
  уравнение какой-то степени всегда имеет корень. Тогда возьмём другое
  уравнение меньшей степени, домножим на многочлен нужной степени и
  скажем, что полученное уравнение уже имеет корень. Но это ничего не
  доказывает относительно исходного уравнения. Так что тут у Эйлера
  ещё одна дыра.}. Тогда новое уравнение уже переписывается как
(полагая $\la := 2^{k-1}$)
$$ P_{2^k} = Q_\la \cdot R_\la= (x^\la + u x^{\la-1} + \mu x^{\la-2} +
\dots)(x^\la - u x^{\la-1} + \mu x^{\la-2}+\dots )
$$ На самом деле Эйлер не умел ничего рассматривать в общем случае. Он
говорил так:

Рассмотрим уравнение для $n=4$, получаем уравнение
$$ x^4 + ax^2 + bx + c = (x^2 + ux + d) (x^2 - ux + f).
$$ Имеем $a = d + f - u^2$, $b = u(f-d)$, $c = df$.  Получили
систему. Для $u$ получаем уравнение 6 й степени, но важно, что его
коэффициенты являются во первых многочленами, а во вторых, у него
отрицательный свободный член.  После этого говорим, что у него есть
два корня, положительный и отрицательный.

\epigraph{\par--- Ничего не понимаю!  \par --- Аналогично!}
         {\emph{Следствие ведут Колобки}}

Далее, пусть $\al, \dots$ корни уравнения 4 й степени. В силу теоремы
Виета их сумма равна нулю, потому что члена при кубе нет. Значит,
какие могут быть значения для~$u$?  Возможны варианты, когда
$$ u = \al + \be =:p,\quad u = \al + \ga=:q,\quad u = \al + \de=:r,
$$
$$ u = \ga + \de,\quad u = \be + \de,\quad u = \be + \ga.$$ Тогда наше
уравнение имеет вид
$$ (u^2-p^2)(u^2-q^2)(u^2-r^2) = 0.
$$ Из общих соображений его свободный член равен $-(pqr)^2$, и важно,
что $pqr$ есть функция от корней уравнения, симметричная при всех их
перестановках.

Вот тут у Эйлера была дырка. Тут бы ему самое время сослаться на факт,
доказанный (впоследствии) Лагранжем: если некоторая рациональная
функция от корней инвариантна относительно всех перестановок корней,
то она рационально выражается через коэффициенты заданного уравнения.
Кроме того, далее Эйлеру нужна была теорема о базисных инвариантах при
действии группы~$\Sb_n$ на алгебре многочленов.

Эйлер использовал без доказательства такое утверждение.  Если функция
при всех подстановках принимает $k$~различных значений, то
$\ph$~является корнем уравнения степени~$k$, коэффициенты которого
рационально выражаются через $A, B, \dots$.

Рассматривая примеры, Эйлер видел, что всё получается. Ну и в общем
случае он говорил, что всё аналогично.

Это доказательство было за\emph{patch}ено \sauth{Лагранжем}{Лагранж}
в~1772~году.  Важно, впрочем, что, несмотря на большое количество
дырок, это доказательство, возможно, повлияло на дальнейшее развитие
теории алгебраических расширениях поля рациональных чисел и создании
теории \auth{Галуа}.

\subsection{Решение уравнений в радикалах}

Тут прославились такие граждане, как \auth{Эйлер}, \auth{Чирнгауз},
\auth{Безу}, \auth{Лагранж}, ну и ещё \auth{Вандермонд}.

\auth{Чирнгауз} хотел сделать такую подстановку, чтобы убить
максимальное количество коэффициентов.  Так вот, для степени~6 ничего
не получалось, поэтому много было написано по приближённым вычислениям
корней (\auth{Ньютон}, \auth{Стирлинг}, \auth{Бернулли}, тп).

\auth{Лагранж} примерно в \pe{1771}{1772} хотел сказать, почему все
методы, которые были ранее, не канают в общем случае. В общем, важно,
что он рассматривал всякие разные функции, инвариантные относительно
перестановок. Иначе говоря, он работал с подгруппами подстановок
корней.  Так вот, значит, он рассматривает группу корней $n$ й
степени, шаманит с ними, но слов <<группа>>, <<обратный элемент>>, не
произносит.  Вообще его справедливо считают отцом теории групп.

А потом ещё был \auth{Руффини} (1799) неполное доказательство
разрешимости.  Что касается полного доказательства, то точку в вопросе
о неразрешимости общих уравнений степени выше 4 поставил молодой
Н.\,Х.\,Абель, который сначала <<решил>> уравнение 5 й степени, потом
нашёл у себя ошибку, и в~1824~году накатал полное доказательство.


\marklec{15}{30.03.2006}

\section{XIX век}

Что важно думать об этих периодах? Это начало периода современной
математики.  Более точно, сие происходит с 70 х годов 19 века (хотя
всё условно).  Важно, что многие идеи были осознаны к этому времени.

Приложения механика, астрономия, изучение теплоты, упругости,
гидромеханика и аэромеханика, электричество, магнетизм.

Особенность: очень много математиков, и при этом им было где
работать. Появилось до фига университетов, технических и военных
институтов. Сейчас, кстати, в Питере много юбилеев всяких институтов,
и вообще центр науки тогда был скорее там, нежели в Москве.
\auth{Чебышёв}, например, жил в Питере, хотя учился в МГУ.

Вообще это время, когда происходил обмен идеями, появились
математические журналы, командировки, стажировки. Это был единый
научный мир. Национальные языки (немецкий, французский, английский,
ранее латынь). Вот например, \auth{Лобачевский}, сначала всё написал
на русском, потом на немецком.

\textbf{Франция.} \auth{Коши}, \auth{Пуассон}, \auth{Фурье} (эти
граждане в основном ботали математическую физику).  Был ещё
\auth{Галуа}, он был молодой, всегда ходил на заседания, но его не
понимали.  Перед дуэлью он написал, чтобы его друг послал
\sauth{Гауссу}{Гаусс} все его работы.  Галуа погиб в возрасте 20 лет.

\auth{Пуанкаре} топология, небесная механика, теория
относительности. Очень крупный учёный.

\textbf{Германия.} \auth{Гаусс}, \auth{Якоби}, \auth{Риман},
\auth{Вейерштрасс}, \auth{Дедекинд}, \auth{Кантор}, \auth{Грассман},
\auth{Минковский}, \auth{Гильберт} (вершина немецкой
математики). Гильберт был основательным человеком, а вот
\auth{Пуанкаре} больше мыслил идеями, недоделками.

А что в \textbf{России}? Тут жили \auth{Остроградский},
\auth{Буняковский} (математическая физика), \auth{Лобачевский}
(геометрия). Ещё был \auth{Чебышёв}, и его школа: \auth{Лобачевский},
\auth{Ляпунов}, \auth{Марков}, \auth{Стеклов}. Ещё была
\auth{Ковалевская}.

\textbf{Англия.} \auth{Буль}, \sauth{де Морган}{Морган, де},
\auth{Пикок}, \auth{Грин} (у него тоже были все формулы типа формул
\sauth{Остроградского}{Остроградский}).  Его работа 1828 года <<Опыт
применения математической физики к теории электричества и
магнетизма>>.  Опричь того был \auth{Гамильтон} (механика, астрономия)
Сам Гамильтон был директором обсерватории.  Кроме того, он придумал
кватернионы, и очень много про них написал.  Но тут народ подумал: а
на фига нам эти кватернионы? Нам векторы нужны!

1873 \auth{Максвелл} <<Трактат об электричестве и магнетизме>>.

Тут уже появляются физические лаборатории, появились телефон,
телеграф, электричество.

\auth{Фарадей} был до \sauth{Максвелла}{Максвелл}, он не знал
математику. Так вот, Максвелл построил математическую теорию того, что
придумал дядя Фарадей.

\textbf{Преобразования математики}

\halign to \textwidth{\tabskip=0pt plus 1 fil%
  \hwbox{.18\textwidth}{#}& \hwbox{.25\textwidth}{#}&
  \hwbox{.57\textwidth}{#}\cr & \textbf{До XIX века}& \textbf{XIX
    век}\cr Алгебра & Решение уравнений & Изучение алгебраических
  структур\cr Геометрия & Евклидова геометрия & Неевклидова геометрия,
  Много геометрий\cr Аналитическая\hfil\break геометрия & Геометрия
  плоскости\hfil\break и пространства & $n$ мерная геометрия.\cr Матан
  & Что такое бесконечно малая величина? Анализ не был обоснован &
  Коши, теория пределов.  Потом полную строгость навёл Вейерштрасс,
  когда построил теорию $\R$. Теория бесконечных множеств\cr}

Кстати, письмо \auth{Галуа} про его теорию прочли только в 1870 е
годы, то есть через 40~лет \authorcomment{Хорошо ещё, что не дней!}
после его написания.

\sauth{Лобачевского}{Лобачевский} признали примерно в 1870~г, хотя
написал он работу значительно раньше.

\auth{Грассман} сказал, что вообще $e_1 e_2 e_3$ это просто наш мир. А
вот математика должна изучать общую ситуацию, то есть всё в $n$ мерном
случае.

\section{Реформа математического анализа}

\subsection{Парадоксы}

Как мы знаем, математический анализ был создан в~17~веке. Но проблемы
были, были парадоксы, которые не умели объяснять.

1. Что такое $\frac{1}{1+x} = 1 -x + x^2 -x^3 +\dots$.  Возьмём $x =
1$, а справа <<грамотно>> расставим скобочки.  Справа получим сумму
$\sum (+1 -1) = 0,$ то есть $\frac{1}{2} = 0$.

2. Или вот ещё такое: $\frac{1}{1-x} = 1 + x + x^2 + x^3 + \dots$.  $x
= 2$, справа получаем сумму, равную бесконечности.

\auth{Абель} очень негативно относился к тому, что так пишут. И вот он
сказал, что нужно вводить сходимость ряда. Он учился и во Франции, и в
Германии, но, однако, довольно рано заболел и умер.

\auth{Лежандр}: если $f(x) > 0$, то $\intl{a}{b} f(x) dx > 0$.  А
когда Лагранж это увидел, он сказал: нехорошо. $f$ должна быть
ограниченной, а то как-то нехорошо.

Пример \sauth{Лагранжа}{Лагранж}: $\intl{-1}{1} \frac{1}{x^2}dx =
\bes$.  Но тогда не было таких слов: неопределённый интеграл как
предел сумм.  А с другой стороны, если банально проинтегрировать, то
получается (по формуле Ньютона Лейбница).  $-1-2 = -2$. Это явно
противоречит геометрии.

Ну, тем не менее, был посажен пробел в теории вариационного исчисления
(\auth{Якоби} придумал теорию слабого экстремума, которая устраняла
этот дефект).

\subsection{История математического анализа}

$\bullet$ 17 век, \auth{Лейбниц}. Исчисление работает, основы не
ясны. Лейбниц говорил, что $\al$ бесконечно малая (песчинка по
сравнению с горой).

\auth{Ферма}. Приведём его метод вычисления производной от функции
$x^2$.
$$\frac{(x+h)^2 - x^2}{h} = \frac{2xh + h^2}{h} = 2x + h.$$ Потом
кладём $h = 0$. Возражения: если $h = 0$, то нельзя делить на
$h$. Если не нуль, то нельзя отбрасывать.

$\bullet$ Потом наступил 18~век. Это рациональный период. Епископ
\auth{Беркли} в книге <<Аналист, или рассуждения, обращённые к
неверующему математику>>.  критикует \sauth{Ньютона}{Ньютон}. И вот на
его критику начинает отвечать.

\auth{Даламбер} в Энциклопедии написал статью <<Предел>>. Он говорил,
что речь идёт не о бесконечно малых, но только о пределах конечных
величин.

$\bullet$ 19 век. Теория пределов. \auth{Коши} и \auth{Вейерштрасс}.

Сначала скажем о том, что сделал \auth{Коши}.  Он накатал две книжки,
даже учебника. Они сразу были переведены на много языков. Итак, Коши
\pe{1789}{1857}. Учебники 1821 <<Курс анализа>>, 1823 <<Резюме лекций
по исчислению бесконечно малых>>.

Все его понятия основаны на базе теории пределов.  Непрерывность: если
$\de x \ra 0$, то $\de y \ra 0$.

Производная то же, что у нас.

Интеграл: он первый сказал, что интеграл это определённый интеграл,
предел интегральных сумм.

Кроме того, предел частичных сумм ряда это и есть его сумма. То есть
его определения уже не отличались от наших. Но у Коши всё было
словами, значков не было.  И вот он в своих текстах местами врал. В
общем, неточностей была масса, и многие его потом шпыняли.

И вот потом пришёл дядя \auth{Вейерштрасс} \pe{1815}{1897} и ввёл
$\ep$-$\de$ аппарат.  кроме того, он разработал условия
дифференцируемости и интегрируемости степенных рядов, исследовал
равномерную сходимость.

Он ввёл верхнюю и нижнюю грань. И вот когда стали смотреть, что вообще
такое число, стало ясно, что нужна строгая теория. Так вот, она была у
Вейерштрасса. В его работах было то, что сейчас называют
<<Вейерштрассовой строгостью>>.

Он учился на юриста, но это ему не нравилось, и он забил.  Его отец
сказал, что второго высшего образования ему не дадут, и ему пришлось
идти в школу учителем.

Он начал изучать функции комплексного переменного, многие учёные его
заметили, и с 1865 года он стал профессором Берлинского
университета. Он не хотел печатать свои работы, потому что не любил
критику, которой ранее подвергался.

1875 год пример нигде не дифференцируемой, но непрерывной функции
(\emph{гора Вейерштрасса}).  Он сам рассказал это на заседании в
университете, но опубликовал этот пример не он сам, а
\auth{Дюбуа-Реймон}.  Вообще, раньше думали, что непрерывная функция
это путь, так что куда же касательная то денется\ldots\ Ан
нет. \auth{Пуанкаре} писал: Как интуиция до такой степени могла
обмануть нас?!

Доверие к интуиции было подорвано, появилась необходимость в
строгости.  В 1870 е годы много разных учёных создавали строгие теории
действительных чисел.  Одним из таких был \auth{Дедекинд} (у него была
книжка <<Непрерывность и иррациональные числа>>).  \auth{Пуанкаре}
сказал, что матан арифметизирован: в нём есть только числа,
неравенства, тп

Теория бесконечных множеств. Больше всего продвинулся \sauth{Георг
  Кантор}{Кантор, Георг}, хотя иногда упоминают ещё и
\sauth{Дедекинда}{Дедекинд}. Он в первую очередь исследовал
равномощные множества и само понятие равномощности.

До него в 1851 году \auth{Больцано} (в книге <<Парадоксы
бесконечного>>) тоже говорил про равномощность.  Кантор сильно
продвинул теорию. Он определил счётные множества, построил свой
диагональный процесс, потом доказал, что множество действительных
чисел не может быть счётным, сказал, что оно имеет мощность континуум.

Вопрос: а бывают ли множества промежуточной мощности? Это была
знаменитая континнум гипотеза.  Он хотел доказать, что континуум
гипотеза верна. Даже \auth{Гильберт} сказал, что это неплохо бы
доказать (первая проблема Гильберта).

Вообще Кантора не все понимали. Потому что не понимали, зачем вообще
была нужна теория множеств.

\auth{Кантор} основатель первое германское математическое общество
(сам Кантор был его президентом в период \pe{1890}{1893}).  Он же был
инициатором созыва международного конгресса (Цюрих, 1897~г).  К тому
времени теорию множеств таки признали. И тогда появились первые
парадоксы теории множеств. Гильберт сказал, что никто не изгонит нас
из рая, который для нас создал Кантор, что мы никогда не перестанем в
неё верить.

И вот тогда разработали аксиоматику (ZF) \auth{Цермело},
\auth{Френкель}.  \auth{Гёдель} в 1940 году сказал: возьмём
аксиоматику Цермело Френкеля.  Добавим C (аксиому о
континууме). Получим аксиоматику ZFC.  А потом докажем, что она
непротиворечива.  А в~1963~году \auth{Коэн} сказал, что всё ещё лучше:
он доказал непротиворечивость ZF$\ol{\rm C}$.  То есть можно развивать
обе теории.

\marklec{16}{04.04.2006}

\section{Появление групп в математике}

Надо сказать, что в 19 веке было много интересного в алгебре.  Однако,
создание теории определителей и прочая радость остаётся для нас за
бортом.  Мы поговорим о другом не менее важном алгебраическом объекте.
В начале 19 века в математику входит понятие группы. И вот мы сегодня
будем про них говорить.

Говоря про алгебру 18~века, мы поняли, что это было два вопроса: ОТА
(мы разобрали доказательства \sauth{Эйлера}{Эйлер} и
\sauth{Даламбера}{Даламбер}) и вопрос о разрешимости уравнений в
радикалах.

Трактат \sauth{Лагранжа}{Лагранж} \pe{1770}{1771} <<Размышление об
решении алгебраического уравнения>>.  Он думал думал, и придумал, что
для исследовании нужно рассматривать группы подстановок корней. И вот
тут и возникает множество с операцией.

У Лагранжа всё изложено именно для группы подстановок. Самого термина
<<группа>> не было.  Чтобы доказать теорему о неразрешимости уравнений
пятой степени, ему не хватало двух фактов: того, что $\Sb_5$ не имеет
подгрупп индекса~3 и~4, а также того факта, что всякий промежуточный
радикал, который возникает при решении уравнений, представляет собой
рациональную функцию корней уравнения.

В самом конце 18~века \sauth{Паоло Руффини}{Руффини, Паоло}
опубликовал статью о неразрешимости уравнений 5 й степени. Он показал,
что если функция от 5~переменных принимает меньше 4~значений, то их на
самом деле~2. Но всё таки в его рассуждениях таки была дырка и даже
неверное утверждение касательно выразимости радикалов.

На прошлой лекции мы не сказали ничего про \auth{Коши}, великого
преобразователя математического анализа.  Коши самый плодовитый
математик (около 800 опубликованных и ещё большая куча
неопубликованных работ).  Так вот, он тоже занимался алгеброй, и в
1815 году обобщил результат \auth{Руффини} на произвольное $n$.  Его
обозначения для подстановок сохранились и по сей день. Но он тоже не
дал полного доказательства неразрешимости.

После всего этого возник вопрос (до \sauth{Абеля}{Абель}, конечно): ну
хорошо, а какие уравнения вообще разрешимы в радикалах? Вообще говоря,
именно под таким соусом и появилась теория \auth{Галуа}, но это было
позже.

Первым продвиженцем в этой области был, разумеется, король математиков
19~века \sauth{Карл Фридрих Гаусс}{Гаусс, Карл Фридрих}
(\pe{1777}{1855}).  Важные результаты касательно разрешимости были
получены в работе 1801 года <<Арифметические исследования>>.  Сам
Гаусс был из небогатой семьи. Отец из рода фермеров. Его не принимали
в различные сословия, жить было тяжело, устроиться на работу тоже.
Мать Гаусса из сословия каменщиков.

Формула Гаусса для арифметической прогрессии известна всем. Его
арифметические способности проявились в три года, когда маленький
Гаусс нашёл ошибку в расчётах отца.  Его родители не оценивали его
способностей. Его учителя в начальной школе помогали ему, и в
\pe{1795}{1798} годах Гаусс учился в Гёттингене в университете (там же
он был потом профессором математики до конца дней своих).  Надо
сказать, что при Гауссе Гёттинген стал научным центром.

Гаусс не закончил университет, найдя хорошую работу, но, однако, в
1799~году защитил докторскую.  Всё время думал, чем же заняться,
математикой или филологией. Однако, когда он построил 17 угольник, он
забил на филологию, и стал заниматься математикой, потому что уверовал
в её красоту.

Вообще, фактически, он нашёл первый класс уравнений, которые разрешимы
в радикалах: уравнения деления круга.  Надо сказать, что он не
опубликовал ни одной работы, учась в Гёттингене.  Однако это было
сделано в 1801 году, в той самой работе <<Арифметические
исследования>>.  В той работе было 7~разделов, и уравнениям был
посвящён самый последний раздел.  Остальные разделы были посвящены
теории чисел (там был, в частности, доказан квадратичный закон
взаимности, развита теория сравнений по модулю и вообще было много
результатов по теории чисел).

Итак, \emph{уравнения деления круга}: это уравнения такого вида:
$$\frac{x^n-1}{x-1} = x^{n-1} + a_1 x^{n-2} + \dots.$$ Он доказал, что
если $n$ простое, то уравнение неприводимо над~$\Q$.  Гаусс: мы
покажем, что если $n-1 = \al\be\ga$, где $\al\be\ga$ простые, то
уравнение раскладывается на $\al$ множителей степени
$\frac{n-1}{\al}$, потом каждый множитель разваливаем дальше тд При
этом Гаусс отслеживает, как расширяется поле при добавлении корней.

Мы далее говорим, что если $n=2^k+1$, то мы получаем цепочку
квадратных уравнений, но $n$ должно быть простым, а $k = 2^s$.  И при
$n=2$ получаем как раз $17$. Итак, мы понимаем, что можно построить
$17$ угольник циркулем и линейкой, потому что все иррациональности
квадратичны.

\auth{Ферма} думал, что все числа Ферма (числа вида $2^{2^s} + 1$)
простые. Но \auth{Эйлер} нашёл контрпример для $s=5$.

Надо сказать, что этими уравнениями занимались и до Гаусса.  Так,
\auth{Эйлер} решил круговое уравнение для $n=6$, в 1771 году
\auth{Вандермонд} решил его для $n=11$.  Эйлер воспользовался
подстановкой, которая помогает решать симметрические уравнения 6 й
степени.

Далее, Гаусс начинает исследовать выбор первообразных корней, то есть
таких корней, что их степени порождают всё множество корней.  Он
говорит, что поля разложения многочлена и группа подстановок корней
связаны.

\subsection{Нильс Хенрик Абель}

\sauth{Нильс Хенрик Абель}{Абель, Нильс Хенрик} (норвежский математик,
\pe{1802}{1829}).  Прожил, как видно, немного, умер от туберкулёза, но
дел за это время успел наворочать немало.

Он начал интересоваться разрешимостью уравнений довольно рано.  У него
были исследования по делению лемнискаты на~$n$~равных частей.
Возможно, это навело его на свою теорему.

В 1823 году он <<доказал>> разрешимость уравнения 5 й степени. Он
получил стипендию, и на следующий год доказал неразрешимость
уравнения. Тут он вовсю пользовался работой \auth{Коши}. Полное
доказательство невозможности было опубликовано в 1826~году.  Оно было
опубликовано в одном из первых периодических изданий журнале
\sauth{Крейля}{Крейль}.

А когда он понял, что не для всех уравнений всё так плохо, он написал
работу про класс уравнений, которые таки разрешимы.

\auth{Крейль} послал работу Абеля \sauth{Гауссу}{Гаусс} и
\auth{Якоби}.  Получил, надо сказать, положительные отзывы.

Итак, Абель ввёл область рациональности уравнения, то есть множества
всех величин, рационально выражающихся через коэффициенты и корни
уравнения (в народе эту штуку называют алгебраическим расширением,
полученным присоединением коэффициентов и корней уравнения).  Так,
$\Q(\sqrt2) \supset \Q$ поле расширения уравнения $x^2 -2=0$.  Кроме
того, он вводит понятие нормального уравнения (то есть когда все корни
выражаются рационально через один из них). После этого Абель выводит
теорему: уравнение разрешимо, если оно нормально и группа монодромии
абелева (отсюда и название для коммутативных групп, которые, конечно
же, разрешимы).  Далее, он показал, что абелевы группы разлагаются в
произведение циклических групп.  Но он не успел завершить своего дела.

\subsection{Эварист Галуа}

Эварист Галуа (\pe{1811}{1832}). В 15 лет начал ботать математику, но,
когда тому было 18 лет, отец покончил с жизнью самоубийством. Правда,
не всё в этом тёмном деле объяснялось политикой, но вот для сына это
оказалось важным, и он стал революционером.  Он не поступил один раз в
Ecole Polytechnique, потому что подумал, что слишком лёгкие вопросы
задают. Когда он всё таки туда поступил, его однако, вскоре выгнали за
политическую деятельность.  То ли из за его политической активности,
то ли из вредности, его работы не публиковались и терялись в бумагах
Коши, Фурье и прочих великих математиков того времени.  Жизнь Галуа
была далеко не безоблачной. Например, его посадили за то, что он
<<пожелал здоровья>> Луи Филиппу с ножом в руке. Его быстро выпустили,
но скоро опять посадили.  Ну, в общем, дело кончилось не так
красиво. В тюрьме здоровье Галуа было подорвано, и его перевели в
лечебницу. Потом он вклеился в какую то тёмную любовную историю, и в
конце концов его замочили на дуэли. Ходили слухи, что это было
специально подстроено.

Галуа, впрочем, успел отправить все результаты
\sauth{О.\,Шевалье}{Шевалье}, но торопился и старался всё изложить
кратко. Шевалье выполнил волю Галуа, передав его рукописи другим
математикам, но только в 1846 году \sauth{Жозеф Лиувилль}{Лиувилль,
  Жозеф} смог это сделать. Этот Лиувилль сам был довольно
крут. Например, его считают отцом дифференциальной геометрии во
Франции. У него был даже собственный журнал, и часто была большая
ботва с xref-ами, потому что, как мы знаем, был ещё журнал
\sauth{Крейна}{Крейн}.

Однако полное признание работ Галуа случилось только тогда, когда
\sauth{Камило Жордан}{Жордан, Камило} изложил его теорию с
просветлением тёмных мест в трактате <<О подстановках>>.  Надо
сказать, что у Галуа уже было понятие группы (но тоже только для
подстановок).  Вообще говоря, он рассматривал нормальные подгруппы,
нормальные ряды, разрешимость и прочую требуху.

\subsection{История развития понятия группы}

В Англии в это время алгебраисты развлекаются СЛУ, определителями,
матрицами, (а после того, как были построены комплексные числа, начали
строить алгебры кватернионов тп) Но тем не менее, там тоже возникли
группы. Итак, Артур Кэли в 1854 году сформировал первое определение
группы.  Группа множество различных символов $1, \alpha,
\beta,\ldots$, таких, что произведение двух даёт тоже некоторый
элемент группы, называется группой.  У него не было аксиом
ассоциативности. Он вообще почему то исследовал только циклические
группы. Таблица Кэли таблица умножения в группе.

1870 \auth{Жордан}, применяя теорию \auth{Галуа} для дифференциальных
уравнений, вводит понятие группы монодромии для дифференциальных
уравнений (эта тематика была добита \sauth{Пикаром}{Пикар}).

Далее, \auth{Клейн} запихнул группы в геометрию (он сказал, что
геометрия это теория инвариантов действия некоторой группы на нашем
пространстве).

\auth{Пуанкаре} ввёл группы в топологию (он ввёл понятие $\pi_n$, то
есть фундаментальной группы многообразия).

\sauth{Софус Ли}{Ли, Софус} ввёл непрерывные группы, потом был ещё
\auth{Александров}.

Жордан представлял группы матрицами (теория представлений).  В конце
19 века был построен гармонический анализ.  Тут были
\sauth{Вейль}{Вейль, Герман}, \auth{Понтрягин}.

Кристаллограф \auth{Фёдоров} классифицировал группы, появляющиеся в
кристаллографии.

\auth{Гамильтон} долго строил свои кватернионы, и ему не удалось
победить коммутативность.

Ну вот и всё на этот раз!


\marklec{17}{06.04.2006}

\section{Дифференциальные уравнения и вариационное исчисление}

\subsection{Дифференциальные уравнения}

\subsubsection{17 и 18 века}

Ещё \auth{Ньютон} в своём методе флюксий говорил, что нужно по сути
решать две задачи: прямая задача нахождение производной по функции, и
обратная задача по уравнению на производную найти функцию.

Ньютон чувствовал, что задача весьма сложна, потому и не спешил с
публикациями.  Он использовал следующий метод: искал решение в виде
степенного ряда.  Он говорил, что решений много, поэтому задавал
условие $y(0) = 0$.

\auth{Лейбниц} совсем иначе подходил к решению этой задаче. Вообще он
хотел оформить весь матан, так, чтобы последователям уже ничего не
досталось. У Лейбница было много корешей математиков, например,
\sauth{Иоганн Бернулли}{Бернулли, Иоганн} (1691 лекции по исчислению
дифференциалов), годом позже лекции о методе интегралов.

1696 год \auth{Лопиталь} написал первый учебник по матану (<<Анализ
бесконечно малых>>).  Важно то, что очень много сделал
\auth{Бернулли}, а Лопиталь хотел сделать книжку.  Но вот правило
Лопиталя так и осталось за ним, хотя Бернулли против этого возражал
потом.

Ну неважно. Первые искатели, естественно, начинали с конкретных задач.
Потому что задачи возникали вполне конкретные, из практики.

\auth{Бернулли} ввёл обозначения, используемые и по сей день.  Он ввёл
степень уравнения <<gradus>>, метод разделения переменных,
интегрирующие множители.

Он решал уравнение
$$ax\dy - y\dx = 0.$$ Умножим на $\frac{y^{a-1}}{x^2}$. Получим
\begin{gather*}
\frac{y^{a-1}\cdot a}{x}dy = \frac{y^a}{x^2} dx =
0,\\ \frac{y^{a-1}\cdot a \cdot x\dy - y^a\dx}{x^2} = 0.
\end{gather*}
отсюда $\frac{y^a}{x^2} = C$, и получаем ответ.

\auth{Лейбниц}: $\frac{dy}{dx} = f\hr{\frac yx}$, тогда $y = xt$.

\auth{Бернулли}: $\frac{dy}{dx} = p(x) y + q(x) y^n$, тогда $z =
y^{1-n}$ (подстановка).

\auth{Эйлер}: $Pdx + Qdy = dF$, тогда получаем условие $P_y = Q_x$. Он
написал труд <<Интегральное исчисление>> в трёх томах.

1723 год уравнения \auth{Риккати} $\frac{dy}{dx} = ay^2 + bx^\al$.

\auth{Эйлер}, 1746 год $\frac{dy}{dx} = P(x)y^2 + Q(x)y + R(x)$.
Здесь $P, Q, R$ непрерывные функции.

\auth{Эйлер}, 1743 год: Общее решение уравнения порядка $n$:
$$Ay + B y' \spl N y^{(n)} = 0,$$ тогда в общем решении будет $N$
произвольных постоянных.  Как его найти? Пишем характеристическое
уравнение:
$$A + Bp \spl N \cdot p^n =0$$ Случай первый. Пусть все корни различны
и действительные $p_1\sco p_n$.  Тогда решение сумма экспонент: $y =
\sum \al_i e^{p_i x}$.  Однако \auth{Эйлер} думал, что общее решение
это всё, что мы можем получить.  Он не думал о том, что бывают
огибающие семейства интегральных кривых.

\auth{Лагранж} это заметил, и написал такое уравнение:
$$x dx + y dy = dy \sqrt{x^2 + y^2 - b^2}.$$ Пусть под корнем не нуль,
тогда поделим:
$$\frac{x dx + y dy}{\sqrt{x^2 + y^2 - b^2}} = dy$$ тогда получаем
$$d(\sqrt{x^2 + y^2 - b^2}) = dy$$ Отсюда
$$\sqrt{x^2 + y^2 - b^2} = y + \rho,$$ получаем семейство парабол $x^2
- 2cy - b^2 -c^2 = 0$.  А если под корнем ноль, то получаем
окружность.

\subsubsection{19 век}

Во-первых, были развиты приближённые методы в работах по физике,
астрономии.  Именно, были решения в виде рядов (степенных, Фурье и
других).

Самое первое достижение \auth{Коши} доказал теорему существования и
единственности для уравнения $n$ й степени, разрешённого относительно
старшей производной.  Если задано нужное количество начальных условий,
то получаем единственное решение при некоторых условиях.

Обобщение на УрЧП провела \sauth{С.\,В.\,Ковалевская}{Ковалевская,
  Софья Васильевна}.

Но качественные методы исследования это уже конец 19 века.  Пуанкаре
говорил о том, что нам нужно исследовать не только локальные вопросы,
но и вопросы о предельном поведении решений, вопросы устойчивости.
Особенно это нужно в задачах небесной механики.

\auth{Пуанкаре} (\pe{1854}{1912}). Он начал заниматься этими вопросами
в \pe{1881}{1886}, написал 4 мемуара <<О кривых, определяемых
дифференциальными уравнениями>>

\sauth{А.\,М.\,Ляпунов}{Ляпунов, А. М.} \pe{1857}{1918} исследовал
устойчивость решений уравнений.  В 1892 году защитил докторскую
диссертацию по теории устойчивости.

Кроме того, иногда выгодно было переходить к интегральным уравнениям.
Кстати, из этого потом вырос функан. С 1896 года \sauth{Вито
  Вольтерра}{Вольтерра, Вито} занялся своими интегральными
уравнениями. Это была аналогия с решением линейных систем
алгебраических уравнений степени $n$. Сам он занимался математической
физикой, и там очень многое сумел разрюхать.

\subsubsection{20 век}

Потом пришёл \auth{Фредгольм} (1900) и начал решать свои уравнения.

\pe{1903}{1910} \auth{Гильберт}. Он сказал, что вообще есть аналогия с
квадратичными формами.  Сначала для $n$ переменных, а потом и для
бесконечномерных пространств.

Решающий переход сделал \auth{Шмидт}, когда в 1907 году ввёл
бесконечномерные векторные пространства.  Он сказал, что будем
рассматривать пространство $\ell_2$, где квадрат нормы вектора
конечен.

А потом пришёл \auth{Рисс}, и сказал, что вся ботва в линейности и
связи с алгеброй.  1922 год диссертация \sauth{Банаха}{Банах, Стефан}
<<Операции на абстрактных множествах и теория интегральных
уравнений>>.

\subsection{Вариационное исчисление}

1696 год. По какой кривой должна катиться точка, чтобы скатиться из
точки $A$ в точку $B$ за минимальное время. Задача была поставлена
\sauth{Иоганном Бернулли}{Бернулли, Иоганн}.  \auth{Лейбниц} объявил
конкурс на год, и через год был получен ответ.  \auth{Лопиталь}, двое
\auth{Бернулли} получили сходные пути решения, потому что использовали
принцип \auth{Ферма} о скорейшем распространении света.  Далее,
\auth{Ньютон} прислал решение, не подписанное. Но все догадались, что
это был Ньютон (<<Я узнаю льва по его когтям>>, сказал
\auth{Лейбниц}).

\auth{Лейбниц} сказал, что нужно действовать так.  Проведём
горизонталь. Найдём на ней такую точку $D$, чтобы оба отрезка пути
пройти за минимальное время. Тем самым проблема была сведена к
дифференциальному исчислению.

\auth{Эйлер}, 1744. <<Метод нахождения кривых линий, обладающих
свойствами максимальности либо минимальности, или решении
изопериметрической задачей в самом широком смысле>> Именно,
рассматривалась задача
$$J = \intl{a}{b} f(x, y, y')\dx \ra \extr.$$ И для решения нужно было
решать уравнение $f_y = \frac{d}{dx} f_{y'}$.  Изопериметрическая
задача: пусть, кроме того, задано условие $\intl{a}{b} \ph(x,y,y')\dx
= K$. Тогда нужно писать $F = \al J + \be K \ra \extr$.  Ну и
аналогично делаем для того случая, когда у нас функционал зависит от
многих производных (задачи высших порядков).

Но беда в том, что мы не умеем работать с многомерными функциями.  То
есть методы Эйлера не работали.

\auth{Лагранж} написал \sauth{Эйлеру}{Эйлер} письмо\footnote{Подробнее
  об истории экстремальных задач, в частности, о задаче Эйлера
  Лагранжа, можно посмотреть, например, в книге В.\,Ю.\,Протасова
  <<\emph{Максимумы и минимумы в геометрии}>> М.: МЦНМО,
  2005. \emph{Прим. наб.}}  в 1755 году про свой метод, а в 1761 году
была первая публикация. Мы скажем, что необходимое условие это то, что
первая вариация равна нулю. После дифференцирования получаем
$\intl{a}{b} f_y \de y + f_{y'} \de y'\dx = 0$.  По частям:
$\intl{a}{b} f_{y'} \de y'dx = f_y\de y - \intl{a}{b} \frac{d}{dx}
f_{y'}\de y \dx$.  Поскольку на концах у нас нули, первого слагаемого
нет, и потому получаем интегральное уравнение $\intl{a}{b} \hr{f_y -
  \frac{d}{dx} f_{y'} }\de y \dx = 0$.  После снятия интеграла
получаем искомое уравнение Эйлера Лагранжа.

\auth{Эйлер} ввёл понятия вариации и вариационного исчисления.

1761 год \auth{Лагранж} победил пространственную задачу, то есть у
него ядро функционала могло зависеть уже от переменных $x,y,z$ и
производных $y', z'$.

Потом, в 1788 году он сказал, что можно вообще многомерную задачу
решать.  Именно, $y$ теперь считаем векторным. Ограничения $\ph_\be =
0$ можно рассматривать в количестве не более $n-1$ штуки.

Потом появилось и правило множителей \sauth{Лагранжа}{Лагранж} правило
перехода от условного экстремума к безусловному экстремуму $F = f +
\la_1 \ph_1\spl \la_m \ph_m \ra \extr$.

Достаточные условия \auth{Лежандр}, 1788~г. Мы знаем, что $y' = 0$
необходимое условие.  Но нужно $y''> 0$ достаточное условие минимума.
Аналогичное правило было сформулировано и для функционалов: $\de^2 J >
0$, но тут у него был глюк в рассуждениях. Именно, ему было нужно,
чтобы вспомогательная функция $V$ была такова, чтобы было выполнено
уравнение
$$f_{y'y'} \hr{f_{yy} + \frac{dV}{dx}} = \hr{f_{yy'} + V}^2.$$ Ну а
как решать такое уравнение, непонятно даже сейчас.

1837 год достаточные условия слабого минимума. В общем, важно, чтобы
было выполнено условие малости $\de y$ и $\de y'$.

Полная формулировка такова:

1) Условие \sauth{Эйлера}{Эйлер}.

2) Условие $f_{y'y'} > 0$ условие \sauth{Лежандра}{Лежандр}.

3) Условие \auth{Якоби} об отсутствии сопряжённых точек на отрезке.

\auth{Вейерштрасс} (Лекции \pe{1865}{1889}) в 1879 году получил
достаточные условия для сильного экстремума (когда мы разрешаем
производной вариации быть большой).  Надо сказать, что вейерштрассова
строгость проявилась и здесь.

Начало 20 века дифференциал \auth{Фреше}. Он стал спрашивать, а что
такое сходится, что такое предел, и так далее. \auth{Вольтерра} ввёл
понятие вариационной производной.

\auth{Адамар}, 1910 год учебник. Там была глава Вариационное
исчисление, первая глава Функционального исчисления.

Надо сказать, что \auth{Эйлер} не говорил слов <<функционал>>.

\marklec{18}{11.04.2006}

\section{Теория функций комплексного переменного}

\subsection{Путь к $\R^2 \cong \Cbb$}

Как случилось так, что комплексные числа получили права гражданства,
нужно спрашивать у \sauth{Гаусса}{Гаусс}. Во всяком случае, видимо,
именно он вручил им паспорт в виде изоморфизма $\Cbb \cong \R^2$.

\auth{Кардано} (1545) говорил, что бывают числа, квадраты которых
отрицательны.  \auth{Бомбелли} (1572), говоря про неприводимый случай
кубического уравнения, пытался ввести их аксиоматически. Надо сказать,
что у Бомбелли было нечто похожее на то, что потом придумал
\auth{Декарт}. Но надо сказать, что уже Бомбелли не говорил, что нам
нужно <<явное>> геометрическое решение, что не обязательно всё делать
в терминах греческой геометрической алгебры.

В последних работах \auth{Кардано} (надо сказать, что работы были
недоделанные и неопубликованные) пытался мнимые числа изображать
отрезками (работа <<Рассуждения о плюсе и минусе>>).  Он писал на
латыни, но её не знал (потому что знал только итальянский), поэтому
его рассуждения весьма непонятны.

Но очевидной геометрической интерпретации не было. К ним относились
как к символам.  Дескать, нефиг им смысла придавать. \auth{Эйлер}
говорил, что обоснованиями займутся потомки, а нам лишь бы
посчитать. Так и в комплексных числах поначалу не пытались искать
какой либо смысл.

Однако \auth{Валлис} в 1685 году сказал, что $\sqrt{-bc}$ можно
интерпретировать как отрезок, перпендикулярный отрезкам $b$ и $c$. Но
дальше этого дело не пошло.

\auth{Виет} (1599) <<Порождение треугольника>>. То, что он там творил
с треугольниками, хорошо переносится на комплексную арифметику.

\auth{Лейбниц} (1702) говорил, что $\Cbb$ амфибия бытия с небытиём.

Но всякий раз, когда они возникали в конкретной задаче (например,
$\log(-1) = ?$).  Так, \sauth{Иоганн I Бернулли}{Бернулли, Иоганн I}
считал, что нулю, потому что $\log(-x) = \log(x)$, ибо $$d\log(-x) =
\frac{d(-x)}{-x} = \frac{dx}{x} = d\log x.$$ Ну, тут конечно, всё
неправда, но на том уровне это было ещё ничего себе.

Но хитрый \auth{Лейбниц}, который всё любил раскладывать в ряды,
спорил с ним, говоря, что
$$\ln(1+x) = x - \frac{x^2}{2} + \frac{x^3}{3} - \frac{x^4}{4} +
\dots,$$ и при $x = -2$ получаем расходящийся ряд, поэтому это не есть
действительное число.

Спор был разрешён \sauth{Эйлером}{Эйлер}, который сказал, что бывают
многозначные функции.  В работе <<Введение в анализ бесконечно
малых>>.  Он писал
$$e^z := \hr{1 + \frac{z}{i}}^i, \quad i = \bes.$$ (Это то же самое,
что предел при $i \ra \bes$.)  Он говорил далее, что $y = \ln x$, то
есть $x = e^y = \lim\hr{1 + \frac{y}{n}}^n$, поэтому $x \approx \hr{1
  + \frac{y}{n}}^n$.  Отсюда $x^{1/n} = 1 + \frac{y}{n}$, а у корня
значений много, поэтому тут их вообще бесконечно много должно быть. Ну
и поэтому полагаем
$$\ln z = \lim n \hr{z^{1/n} - 1}.$$ Ну, заодно Эйлер получил формулу
\sauth{Муавра}{Муавр} и тригонометрические функции комплексного
переменного.

\auth{Даламбер} доказал, что все мнимости имеют вид $M + N \sqrt{-1}$.

То, что $\Cbb \cong \R^2$, придумал \sauth{Курт Вессель}{Вессель,
  Курт} \authorcomment{не путать с Бесселем!}  в 1799 году. Но нашли и
поняли её только в 19 веке, потому что работа Весселя была написана на
датском языке.

Первые работы, которые публиковались в \authorcomment{каком-то там
  журнале или другом периодическом издании}, были работами мало
известных людей.  Например, \auth{Арган}, \auth{Бюэ} (он выдвинул в
\pe{1814}{1815} годах то, что придумал 15 лет назад \auth{Вессель},
заодно придумал и модуль для комплексного числа).

\subsubsection{Гаусс и его целые числа}

1799 год, ОТА, \auth{Гаусс}. Надо сказать, что он использовал в своей
работе указанное соответствие плоскости и множества комплексных чисел.
В письме к \sauth{Бесселю}{Бессель} (1811) он изложил не только то,
что имеет место изоморфизм $\R^2 \cong \Cbb$, но и даже ближайшую
программу комплексного анализа на ближайшее время. Но оно было
опубликовано заметно позднее (1880).

В \pe{1828}{1832} году он в <<Теории биквадратичных вычетов>> написал,
что вычеты можно изготовить и для комплексных чисел тоже.

Он сказал, что числа вида $a+ bi$ назовём целым. Построил кольцо
$\Z[i]$, ввёл ассоциированные числа (то, что представимо в виде $a =
be$, где $e = \pm1$ или $\pm i$.  Чтобы доказать, что оно факториально
с точностью до ассоциированности, он вводит норму.  Не всякое простое
в $\Z$ неразложимо в $\Z[i]$, потому что $2 = (1-i)(1+i)$.

Теорема: если норма является простым числом, то тогда оно простое в
$\Z[i]$, и обратно.  \authorcomment{С другой стороны было сказано, что
  простые это единицы, числа вида $4n+3$ и те, у которых норма $4m+1$,
  это явно противоречит сказанному только что. Где то лажа!}  После
этого он построил НОД в $\Z[i]$ и теорию вычетов. В конце концов
получил биквадратичный закон взаимности (новый результат теории
чисел).  Тем самым он оправдал законность целых гауссовых чисел.

Что важно, так это то, что \auth{Коши} такого издевательства над
комплексными числами поначалу не признавал.

\subsubsection{Кошизм в комплексном анализе}

\sauth{Огюстен Луи Коши}{Коши, Огюстен Луи} (\pe{1760}{1857}).  Про
него хорошо написал \sauth{Bruno Belhoste}{Bruno, Belhoste}
(французский биограф).  Мало того, что Коши реконструировал матан, он
сделал очень много для комплексного анализа. Но вот к комплексным
числам относился (поначалу) как к символам, то есть на уровне 18~века.
Это, типа, просто необходимый инструмент вычисления, который только
упрощает выкладки (1821 <<Алгебраический анализ>>).

В 1847 году Коши пишет <<Теорию алгебраических эквивалентностей>>, в
которой продолжает идею Куммера о сравнении многочленов по модулю.
Так вот, если рассмотреть многочлен $x^2+1$, по нему отфакторизовать,
то мы получим то, что нужно, а именно расширение поля $\R(i) \cong
\Cbb$.  Примерно также соображали Кронекер, Куммер, и вообще немецкие
математики 19 века. Так появилась теория расширений полей и всё такое.

1849 \sauth{Баре де Сен Венан}{Сен-Венан, Баре, де} (работы по теории
упругости).

К тому времени \auth{Коши} уже нужно было то, что $\R^2 \cong \Cbb$,
потому что иначе очень трудно было интегрировать по путям (там нужна
непрерывность).  И вот тут он признал точку зрения
\sauth{Гаусса}{Гаусс}, то есть геометрическую интерпретацию.

\subsection{Голоморфные функции и теоремы Коши}

В задачах о гидродинамики, в 1752 году \auth{Даламбер} сказал, что
если $P(x,y)$ и $Q(x,y)$ составляющие поля скоростей частиц, то для
наличия полного дифференциала $f(x + iy) = p + i q$ нужно $P_x = Q_y$
и $P_y = -Q_x$ (условия Коши Римана).  Он сказал, что это достаточное
условие. \auth{Эйлер} доказал, что это необходимое условие (1755
Эйлер, 1752 Даламбер).  Более, в одной работе \auth{Эйлер} построил
что-то типа конформного отображения.

Далее, 1811 год. \auth{Гаусс} \sauth{Бесселю}{Бессель}: рассмотрим
интеграл по пути.  Заметим, что он не зависит от пути, если при
гомотопии мы не пройдём сквозь плохие точки.

Итак, \auth{Гаусс} сделал вот что:

1) Геометрическая интерпретация $\Cbb$.

2) Определения интеграла по пути.

3) Интегральная теорема Коши.

4) Разложимость аналитической функции в ряд.

5) Посчитал интеграл $\ints{C\cdot k} \frac{dx}{x} = 2\pi i \cdot k$.

Но этого не узнал никто, кроме \sauth{Бесселя}{Бессель}. Написал всё
это уже \auth{Коши}.

\auth{Коши}:

1) Интегрирование по замкнутому контуру.

2) Разложение функции в ряд в области без особых точек, радиусы
сходимости и так далее.

Он создал свою теорию к середине 19 века.

1861 год лекции по ТФКП дяди \sauth{Вейерштрасса}{Вейерштрасс}. Его
идея идея продолжения аналитических функций по цепи.  Именно, если
функцию можно разложить в ряд, то в точке, которая близка к границе,
можно написать другое разложение в другой точке. Ну и иногда есть
шанс, что мы сумеем продолжить функцию подальше.

Но при продолжении мы можем потерять однозначность. И тут пришёл
\auth{Риман}, и он сказал, что можно строить римановы поверхности.

1814 год первая работа \auth{Коши} о вычислении интегралов по
прямоугольникам (он показал, что неважно, по каким сторонам ходить:
$A{-}B{-}C$ и $A{-}D{-}C$.  В этот момент \auth{Эйлер} как раз доказал
теорему \auth{Фубини} для прямоугольников, потом \auth{Коши} вывел
свои условия и применил теорему Фубини в этом случае. Если у нас есть
$\pf{f}{\ol z} = 0$, то всё хорошо, если функция не уходит на
бесконечность.

Далее (1825) \auth{Коши} обобщает результат на произвольный
контур. Далее он говорит, что если есть особая точка, то разность двух
интегралов мы будем называть вычетом. Далее (1826), Коши сказал, что
если функция имеет особую точку $z$, то разложение в этой точке даст
ряд Лорана с отрицательными степенями, а вычет это и есть $c_{-1}$.

Интегральный вычет сумма вычетов по всем полюсам. А в 1840 году
\auth{Коши} всё это изложил в работе <<Основная теорема о вычетах>>.

\subsection{Комплексные степенные ряды}

Строгость, как обычно, началась только в 19 веке.  1813 год работа
\sauth{Гаусса}{Гаусс} <<О гипергеометрическом ряде>>.

Но \auth{Коши} это тоже интересовало.  1831 год <<Туринские мемуары>>
Коши. Там он вывел формулу \auth{Коши} \sauth{Адамара}{Адамар} для
радиуса степенного ряда.

\auth{Абель}, 1826 год. Он исследовал комплексные ряды $(1 +x )^m = 1
+ mx + \frac{m(m-1)}{2} x^2 + \dots$.  где $m,x \in \Cbb$. Тут он и
вывел две свои теоремы о сходимости рядов.

При доказательстве этой теоремы он заметил, что Коши был неправ, когда
говорил, что сумма ряда непрерывных функций непрерывна. Тут ему
потребовалось исследование равномерной сходимости.

Кроме того, появилась интегральная теорема \auth{Коши} о представлении
функции.  Отсюда можно, интегрируя ряд почленно, считать вычеты.

\marklec{19}{13.04.2006}

\section{Неевклидова геометрия}

\auth{Евклид}, <<Начала>>. Пятый постулат говорил, что если сумма
внутренних односторонних углов меньше $\pi$, то прямые пересекутся с
той стороны, где меньше.  Это утверждение казалось странным. Было
известно 26 теорем, которые можно было доказать и без пятого постулата
(то есть факты абсолютной геометрии).  Поэтому греческие геометры не
верили, что его нельзя доказать.

\auth{Прокл} в 5 м веке выработал более привычную формулировку пятого
постулата: через данную точку вне прямой можно провести только одну
прямую, параллельную данной.

\subsection{Первый этап: <<доказательства>>}

Конечно, эти доказательства были ошибочными в том смысле, что неявно
делалось допущение, эквивалентное пятому постулату (аксиоме
параллельных).

Например, тот факт, что сумма углов треугольника равна $\pi$,
естественно, нельзя считать очевидным без пятого постулата, но его
иногда таки считали.  В основном такими вещами занимались арабские
математики.

\auth{Ибн аль-Хасам} (\pe{965}{1039}) книга <<О разрешении сомнений в
книге Евклида>>.  Он использовал движение, и \sauth{Омар Хайям}{Хайям,
  Омар} его критиковал.  Другой арабский математик \auth{ат-Туси}
предполагал, что сумма углов треугольника равна $\pi$.  Часто
использовался принцип <<две сближающиеся прямые не могут с некоторого
момента начать расходиться>>. Это, конечно, неверно в геометрии
Лобачевского.

Рассматривался четырёхугольник с тремя прямыми углами, и четвёртым
углом, равным $\al$ (\auth{Саккери}, \auth{Ламберт}).  Скажем, что
если $\al$ прямой, то это евклидова геометрия.  Если он тупой, то
противоречие. Если меньше, то говорили, что <<третья гипотеза верна>>.
Тогда (далее следовала куча каких нибудь рассуждений) получаем
противоречие (естественно, при этом врали по дороге).  Им казалось,
что всё это бред полный, но на самом деле они выводили факты новой
геометрии.

\auth{Ламберт}, 1761 год. Сумма углов треугольника меньше $\pi$, тогда
определим дефект треугольника $\de = \pi - A - B - C$. Кроме того,
площадь $S = \rho^2(\pi - A - B - C)$.  Надо сказать, что в тот момент
уже была известна сферическая геометрия. На сфере $S = r^2(A + B + C -
\pi)$.  <<Я почти вынужден прийти к заключению, что третья гипотеза
находит применение на мнимой сфере>>, сказал Ламберт. Он оказался
прав.

\subsection{Создание неевклидовой геометрии}

\sauth{Карл Фридрих Гаусс}{Гаусс, Карл Фридрих} (\pe{1777}{1855}). Он
делал свои открытия, но не публиковал этого, боясь критики.

\auth{Швейкарт}. 1819 юрист. В \pe{1812}{1816} преподавал право в
Харькове.  Он в харьковской библиотеке читал работы
\sauth{Гурьева}{Гурьев} по неевклидовой геометрии.  Потом, когда он
вернулся к себе, он написал заметку, ссылаясь на \auth{Саккери} и
\sauth{Ламберта}{Ламберт}, что есть две геометрии: геометрия
\sauth{Евклида}{Евклид} и геометрия астральная.  В последней сумма
углов треугольника меньше $\pi$. Он написал \sauth{Гауссу}{Гаусс}, но
тот ему ответил, что всё знает, и всё может посчитать, если только
известна константа~$c$.

\auth{Тауринус} (племянник \sauth{Швейкарта}{Швейкарт}): возьмём сферу
радиуса $r = i\rho$. Тогда получим новую геометрию, которую назовём
логарифмо сферической. Но \auth{Гаусс} раскритиковал его работы, и тот
сжёг свои труды после публикации.

\sauth{Янош Бойяи (Janos Bolyai)}{Бойяи, Янош} (\pe{1802}{1860})
венгерский математик.  Он вывел положения новой геометрии. Его отец
\sauth{Фаркош Бойяи (Farkos Bolyai)}{Бойяи, Фаркош} написал работу
<<Опыт введения учащегося юношества в начала чистой математики>>.
1832 Янош Бойяи приложение к книге отца (<<Appendix>>). Но
\auth{Гаусс} не оценил его работ.  Более того, когда Гаусс сказал ему,
что Лобачевский, который <<живёт в далёкой, покрытой льдами стране
(\copyright~Гаусс)>>, тоже что-то такое придумал.  Это настолько
огорчило Яноша, что тот сдвинулся на геометрии, и его отец отговаривал
его от занятий геометрией. Кроме того, он говорил ему, что, вообще
говоря, идеи могут возникать одновременно, и это вполне естественно,
что несколько людей изобрели одну и ту же геометрию.

\sauth{Николай Иванович Лобачевский}{Лобачевский, Николай Иванович}
(\pe{1793}{1856}) учился и потом работал в Казанском университете
\pe{1811}{1856}.

\auth{Бартельс} \pe{1769}{1836} учился чуть раньше Гаусса, но потом
вместе с ним изучал неевклидову геометрию. Так, \pe{1808}{1820}
профессор Казанского университета, \pe{1820}{1836} университет в
Дерпте (Тарту).

Итак, кратенько о том, что написал Лобачевский.

1829 первая публикация <<О началах геометрии>>.

1840 работа в Берлине на немецком языке <<Геометрические исследования
по теории параллельных линий>>.

1855 <<Пангеометрия>>.

Но Лобачевский не получил признания в России.  Лучшие умы России того
времени (\auth{Остроградский}, например) раскритиковали его.

\subsection{Непротиворечивость новой геометрии}

\auth{Лобачевский} рассуждал примерно так.  Ну пусть через точку $M$
проходят две прямые, параллельные данной прямой $AB$. Пусть расстояние
от $M$ до $AB$ равно $x$.  Пусть угол между перпендикуляром и прямой
равен $\om$.  Тогда
$$\ctg \frac{\om}{2} = \exp\hr{\frac{x}{\rho}}.$$ Здесь $\rho$ радиус
кривизны пространства.

\auth{Гаусс} высоко оценил работу \sauth{Лобачевского}{Лобачевский},
но не похвалил его самого.  В России всё равно говорили, что всё это
бред полный.  Он предлагал сделать Лобачевского ЧК Гёттингенского
университета.

Как объяснить непротиворечивость? Лобачевский ссылался на астрономию,
и говорил, что треугольник <<звезда Земля Солнце>> на самом деле не
евклидов. Но тут была проблема: измерения неточны, так что доказать
что либо было очень сложно.

\auth{Гаусс} (1822) работа по геодезии <<Общие исследования о кривых
поверхностях>>.  Он ввёл квадратичную форму $ds^2 = E\,du^2 +
2F\,du\,dv + G\, dv^2$.  Он говорил, что её помощью можно посчитать
длины дуг, углы, площади, кривизны.

Ввёл понятие своей (гауссовой) кривизны. Он сказал, что кривизна сферы
равна $\frac{1}{R^2}$, а для псевдосферы $-\frac{1}{R^2}$.

\auth{Миндинг} (\pe{1806}{1885}). Ученик \sauth{Гаусса}{Гаусс}.  С
1843 профессор в Дерпте. Развил тригонометрию треугольников на
поверхностях отрицательной кривизны. Он построил воронку
\auth{Бельтрами}, которую получил вращением трактрисы.  Получил
тригонометрические формулы, которые полностью совпали с формулами
гиперболической геометрии \sauth{Лобачевского}{Лобачевский}.

Но этого никто не заметил. Это сделал уже \auth{Бельтрами}.  В 1865
году были опубликованы письма \sauth{Гаусса}{Гаусс} и его дневники.
Это породило колоссальный интерес к неевклидовой геометрии.

Первая интерпретация (локальная модель плоскости Лобачевского) 1865 г
Бельтрами доказал, что внутренняя геометрия псевдосферы совпадает с
геометрией Лобачевского.  Но этого было мало. Модель пространства
отрицательной кривизны не давала модели всей плоскости Лобачевского.

\auth{Риман} (прочёл лекцию в 1854 году), в 1868 году её наконец
опубликовали.  Он сказал, что на самом деле всё равно, что
рассматривать, потому что можно взять любое риманово многообразие (то
есть хорошее многообразие с заданной на ём римановой метрикой $ds^2 =
g_{ij}dx^idx^j$ при $i = 1\sco 3$).  Многомерная геометрия (и
многомерная риманова метрика) появилась в конце 19 века.

\auth{Колмогоров} сказал, что риманова метрика определяет гауссову
кривизну, значит, это инвариант многообразия.

\auth{Гаусс} прослушал лекцию Римана молча, не задавал вопросов и
молча ушёл.  Никто не знает, что он подумал потом.

Надо сказать, что всё это нашло применение в общей теории
относительности.

1906 \auth{Пуанкаре} <<О динамике электрона>>.

1909 \auth{Минковский} <<Zeit und Raum>> (<<Время и Пространство>>).
Он сказал, что можно рассматривать пространство $\R^{3,1}$.

1915 ОТО. \auth{Эйнштейн} положил в основу ОТО риманову геометрию.  Он
не хотел заниматься математикой, чтобы она не затмила физику, которой
он хотел заниматься.

\auth{Гильберт}, правда, подвёл математический аппарат под теорию
Эйнштейна.  Иногда даже его считают первооткрывателем ОТО.


\marklec{20}{18.04.2006}

\section{Математика в России}

\authorcomment{Эта лекция началась с довольно активной рекламы книжек
  по истории математики.  Именно, упоминались книга \auth{Гнеденко} по
  истории математики издательства УРСС, книжки
  \sauth{И.\,З.\,Штокало}{Штокало, И. З.} <<История отечественной
  науки>>, аглицкие издания (\auth{Вусинич}, а также 6-й том издания
  \auth{AMS}, в котором довольно неплохо написано про московскую
  математику 30-х годов).}

\medskip
\hbox to \columnwidth{\hfil\hbox to
  .44\columnwidth{\vbox{\footnotesize\itshape\hsize=.44\columnwidth

3000 год нашей эры\ldots

На территории древней Америки при раскопках нашли куски меди. Пришли к
выводу, что древние американцы использовали выделенные линии.

На территории древней Германии нашли куски стекла. Пришли к выводу,
что древние германцы пользовались в основном оптоволоконными линиями.

На территории древней Руси копали копали, копали копали, но так ничего
и не нашли. Пришли к выводу, что древние славяне пользовались
спутниковой связью.\par}}}
\medskip

Первые летописи 11 век. Поэтому ничего раньше этого не случилось.

А что было до этого? Крещение Руси (988 год). В этот момент
развивается арабская математика. По свидетельствам арабских историков,
славяне доплывали и до Багдада, и до стран на побережье Каспийского
моря тп Значит, что-то они знали. А именно, целые числа (в общем,
арифметику).  Это примерно 10 век. Рубль часть гривны, стало быть, они
знали дроби.

1134 год вычисляется число дней от сотворения мира (по другим данным,
1136 год, или ошибка в тексте). <<Кирика диакона Новгородского
Антониева монастыря \ldots учение о числе лет от сотворения
мира>>. Там в тексте ставится задача: вычислить день Пасхи (упражнение
для читателя).

1492 год в России не осталось людей, которые умели вычислять день
Пасхи.  \sauth{Геннадий Гонзов}{Гонзов, Геннадий} привёз пасхалий на
70 лет из Рима, но он сказал, что ими можно пользоваться лет 20, не
больше.  В общем, полная деградация (возможно, в ожидании конца
Света).

В религиозной литературе из византийской культуры: <<Шестиднев>>. Это
учение \sauth{Аристотеля}{Аристотель} в несколько искажённом виде.  В
нём \sauth{Иоанн Экзарх}{Экзарх, Иоанн} излагает, в частности, расчёты
размеров Земли, и там используется $\pi \approx \frac{25}{8}$, хотя у
\sauth{Архимеда}{Архимед} было $\frac{22}{7}$, что гораздо лучше. Ещё
был <<Шестислов>>.

<<Толковая Палея>>.

\auth{Ярослав Мудрый} (\pe{980}{1054}) <<Русская Правда>>. Это был
новый свод законов, и там были некоторые правила по поводу того, как
нужно давать деньги в долг (до трёх гривен без свидетелей), и много
всего другого.

Договор между \sauth{Олегом}{Олег (князь)} и греками о выдаче пленных
(там за каждого пленника полагалась ещё сумма денег, так что там тоже
нужна была какая-то арифметика).

В основу системы счисления была положена греческая система, но там над
буквами рисовались тильды.  Для <<буки>> числового значения тоже, как
и у греков, не существовало.

Ну, конечно, после татаро монгольского ига всё стало плохо.  Стоглавый
собор 1551 года: если не просвещать безграмотных, то в России придётся
умирать без пения, потому что образованных людей, которые умели петь в
хоре (и при этом понимать, что они поют), стало совсем мало.  От этого
в Европе даже шли всякие странные крамольные мысли: 1620 \sauth{Иоанн
  Ботвид}{Ботвид, Иоанн} <<А христиане ли московиты?>>.

Важно, что в \pe{16}{18} веках церковники глушили всяческое
проникновение литературы с Запада.  Они были противниками изучения
геометрии, астрономии и учения греков. Они предписывали, что нужно
учить только закон Божий, а вся эта наука только совращает умы.

У \sauth{Татищева}{Татищев} были математические рукописи 17 века по
поводу измерения земли.  Они пропали.

Коллекция профессора \auth{Браузе} разных математических книг сгорела
в пожаре 1812 года.

<<Устав Ратных дел>> (1607, 1621). Там всё было настолько непонятно,
что становилось понятно, что автору самому было не всё
понятно. Никаких доказательств, только примеры и рецепты.

1629 <<Книга сошного письма>> (наука об измерении площадей).  Там тоже
не было никаких доказательств, а геометрические формулы большей частью
неправильные.

Очень много уделялось внимания арифметическим задачам.  Правила
действий с целыми числами, с дробями (вообще было много торговых
задач).

У греков был квадривиум: Арифметика, Астрономия, Гармония, Геометрия.
Римляне к этому добавили ещё Грамматику, Диалектику и Риторику.  Самое
интересное, что потом арифметику всё таки признали полезной, но
<<бездуховной>>.

Конец 17 века оживление в науке. $\sim1670$ Киевская духовная
академия.  Для подготовки пастырей программы берутся иезуитские, что
интересно.  Вообще это первый ВУЗ в России.

1687 год Славяно Греко Латинская академия в Москве (например, там
учился \auth{Ломоносов}). Также очень был известен
\rauth{Л}{Ф}{Магницкий}, который был автором первой <<Арифметики>>.

В 1701 году по указу \sauth{Петра Первого}{Пётр Первый} открылась
<<Навигацкая школа>>.  Пётр хотел вывести Россию на первый уровень, в
том числе и в науке.

Людей часто посылали за границу. Кроме того, печатались книги за
границей.  Но всё это было дорого, и в общем, далеко не пошло.  В
Навигацкой школе были аглицкие преподаватели.  В 1715 году эту школу
перевели в Санкт Петербург, назвали Морской академией.  Её выпускники
должны были отработать в цифирных школах.  Но детей было мало, всё шло
медленно.

Надо сказать, что обучение в цифирных школах было сильно
дифференцированным (один умел только делить, второй только умножать).
Всё это жило так до 1744 года.

Самый мощный учебник того времени 1703 Арифметика
\sauth{Магницкого}{Магницкий}.  Ноль <<цифра>>. Но по крайней мере это
была уже десятичная система счисления.  Там была арифметика, дроби,
задачи на прогрессии. Кроме того, были правила для решения квадратных
уравнений $Ax^2 + Bx + C=0$ (много вариантов, потому что не было
отрицательных чисел). Кроме того, там есть плоская и сферическая
тригонометрия, и, надо сказать, неправильных формул там не было.

По инициативе \sauth{Лейбница}{Лейбниц} была создана Берлинская
Академия наук.  В 1725 году по его же инициативе создали академию наук
в России.  Первые академики иностранцы \sauth{Николай}{Бернулли,
  Николай} и \sauth{Даниил Бернулли}{Бернулли, Даниил}.  Было ещё
двое: \sauth{Якоб Герман}{Герман, Якоб} (ученик \auth{Бернулли}),
\sauth{Христиан Гольдбах}{Гольдбах, Христиан} (был знаком с Иоганном
Бернулли).

В 1726 году Николая Бернулли не стало. В 1727 году приехал
\auth{Эйлер}.

Надо сказать, что в 1725 году был создан Питерский университет (8
учеников, 17 преподов) при Академии Наук. Но в 1787 году его
благополучно закрыли, потому что там было всего два студента.

1755 МГУ им. \sauth{Ломоносова}{Ломоносов}. Княгиня \auth{Дашкова},
князь \auth{Шувалов} внесли неоценимый вклад в создание университета.

Ценно было то, что хотя бы поначалу была поддержка со стороны
государства.  Однако, ВФР пошатнула планы \sauth{Екатерины}{Екатерина}
в плане просвещения, и программа просвещения была свёрнута
<<сверху>>. Но начало было положено.

Вообще, первая половина 19 века время основания множества
университетов.  1802 Дерптский университет (там
\sauth{Петерсоном}{Петерсон} был основан российский дифгем, который
потом пошёл по российским геодезическим через снега и болота в
Москву).

1803 Вильнюс, 1804\ Казань 1805 Харьков, 1819 СПб, 1834 Киев.

В 1804 году сбацали новый университетский устав.  Церковь и настоящее
образование в России были разделены (церковно приходские школы не в
счёт).

В целом, всё образование развивалось достаточно вяло.  Бывало так, что
на факультете был только один студент.  До середины 19 века
образование в МГУ было довольно слабым.

\auth{Чебышёв} (\pe{1821}{1894}) в 1851 году окончил МГУ (лекции
\sauth{Зернова}{Зернов}, \sauth{Перевощикова}{Перевощиков}).
\auth{Брашман} был тем, кто не возражал против трудов
\sauth{Лобачевского}{Лобачевский}, в отличие от
\sauth{Остроградского}{Остроградский}. Он говорил, что если он сам не
может понять, что придумал Лобачевский, то это не означает, что это
неправильно.  С его именем связано создание ММО в 1864 году.

Что развивалось в России в то время? Да почти ничего. Во всяком случае
до первой половины 19 века.  \sauth{Виктор Яковлевич
  Буняковский}{Буняковский, Виктор Яковлевич} (\pe{1804}{1889}) (отец
российской демографии), \sauth{Михаил Васильевич
  Остроградский}{Остроградский, Михаил Васильевич}
(\pe{1801}{1861}). Оба они учились в Париже.  Для этого времени
характерны УрЧПы, продолжение исследований \sauth{Эйлера}{Эйлер}
(алгебра, теория чисел).

\auth{Чебышёв} увлёкся теорией чисел, заботал теорию вероятностей
(ЗБЧ, ЦПТ), и вообще считается основателем московской математической
школы.

\marklec{21}{20.04.2006}

\section{Математика в МГУ в 19-м и первой трети 20-го века}

Итак, мы будем говорить о том, что происходило с математикой в Москве
вообще и в МГУ в частности.

Всё началось с того, что в 1804 году был создан физико математический
факультет.  Вот \textbf{список кафедр}:
\begin{nums}{-2}
\item физики
\item чистой математики
\item прикладной математики
\item астрономии
\item химии
\item ботаники
\item минералогии и сельского хозяйства (чтобы привлечь помещиков к
  науке)
\item науки о торговле
\end{nums}

Мы будем говорить о своих кафедрах, то есть о кафедрах чистой
математики и прикладной математики. Первые годы студентов было совсем
мало.

Поначалу (\pe{1825}{1836}) выпускалось в среднем по 11 человек в год.
\pe{1837}{1854} примерно 25 человек в год. Тут были уже учёные с
мировыми именами. Например, \auth{Чебышёв}, \auth{Сомов} (академики),
\auth{Цингер}, \auth{Давидов} (профессора).

\textbf{Список дисциплин:}

\begin{nums}{-2}
\item высшая алгебра (решение уравнений)
\item аналитическая геометрия
\item математический анализ
\item дифференциальные уравнения
\item вариационное исчисление
\end{nums}

Учились 4 года. Система лекционная. И только в 1870 е годы появились
семинарские занятия, кружок \sauth{Жуковского}{Жуковский} (он был
прикладником и считал, что семинары необходимы).

ММО. Его организовал \rauth{Н}{Д}{Брашман} (\pe{1796}{1866}).  Он был
в МГУ с 1834 года.  В архивах было найдено дело Брашмана (он
заканчивал Политехнический институт, ему предложили ехать в
Россию. Сначала он поехал в Питер, потом в Казань, потом оказался в
МГУ.

15.09.1864 год первое заседание ММО. Это были уже последние годы жизни
\sauth{Брашмана}{Брашман}.  Заседания проходили дома у Брашмана,
потому что он тогда был уже стар.  Было 13
учёных. \rauth{А}{И}{Давидов} (дифференциальные уравнения с частными
производными), А.\,В.\,Лети\authorcomment{\ldots НАРОД! Как пишется
  его фамилия, чёрт побери?}ов ОДУ. \rauth{К}{М}{Петерсон} геометрия,
\rauth{Н}{В}{Бугаев} теория чисел.

В 1901 году в ММО уже было 100 человек.

Важно, что в 1866 году они решили начать издавать математический
сборник, который издаётся и по сей день.

Самый первый номер опубликованы письма \sauth{Гаусса}{Гаусс}
\sauth{Шумахеру}{Шумахер}, из которых стало ясно, что Гаусс много знал
про геометрию \sauth{Лобачевского}{Лобачевский}. Кроме того,
опубликовали работы Лобачевского по неевклидовой геометрии.

125 лет ММО (\pe{1864}{1989}) 60 человек, председатель
\rauth{С}{П}{Новиков}.

Вообще, можно выделить три основных направления:

\textbf{1)} \rauth{Н}{Е}{Жуковский} (\pe{1847}{1921}) отец русской
авиации.  Закончил МГУ в 1868 году. Организовал \hbox{ЦАГИ}
(Центральный аэрогидродинамический институт), построил
аэродинамическую трубу.  У него было немало учеников (механики, физики
и математики): \rauth{С}{A}{Чаплыгин}, \rauth{М}{В}{Келдыш},
\rauth{М}{А}{Лаврентьев}.

Всякий математик обычно тяготеет либо к теории, либо к практике.  Так,
\auth{Келдыш} ушёл из чистой математики в прикладную науку, хотя
\auth{Лузин} (его руководитель) его пытался остановить.

\textbf{2)} Классическая дифференциальная
геометрия. \rauth{К}{М}{Петерсон} (\pe{1818}{1881}).  Диссертация <<Об
изгибании поверхностей>>.  \auth{Гаусс} $\ra$ \auth{Миндинг} $\ra$
\auth{Петерсон} $\ra$ \auth{Егоров} $\ra$ \auth{Финников}.  Кроме
того, работы итальянских математиков часто использовали результаты
этих граждан.

\textbf{3)} Московская школа теории функций.  \sauth{Николай
  Николаевич Лузин}{Лузин, Николай Николаевич} \pe{1883}{1952} хороший
лектор.  Думал, что девушки не способны понимать математику.
Диссертация \sauth{Лузина}{Лузин} называлась так: <<Интеграл и
тригонометрический ряд>>, 1915.

\sauth{Дмитрий Фёдорович Егоров}{Егоров, Дмитрий Фёдорович}
(\pe{1869}{1931}).

Все они примыкали к французской школе, и \auth{Лузин} направлял своих
учеников по этим направлениям.

1907 \auth{Жегалкин} (диссертация <<Трансфинитные числа>>). Тут мы
выходим на тему, которая связана с разладом Московской и Петербургской
школы.

\textbf{Некоторые президенты ММО:}

\begin{items}{-2}
\item \pe{1866}{1886} \rauth{А}{И}{Давидов} (механика)
\item \pe{1886}{1891} \rauth{Б}{Я}{Цингер} (механика)
\item \pe{1891}{1903} \rauth{Н}{В}{Бугаев} (теория функций, теория
  чисел)
\item \pe{1903}{1905} \rauth{П}{А}{Некрасов} (теория вероятностей)
  ретроград, консерватор
\item \pe{1905}{1921} \rauth{Н}{Е}{Жуковский} (ТФКП)
\item \pe{1921}{1923} \rauth{Б}{К}{Млодзиевский}
\item \pe{1923}{1931} \rauth{Д}{Ф}{Егоров} (теория функций)

Неприятный момент: культ личности задел крупных математиков того
времени, например, \sauth{Лузина}{Лузин} и \sauth{Егорова}{Егоров}.

\item С 1972 года \sauth{Павел Сергеевич Александров}{Александров
  Павел Сергеевич} \authorcomment{Кажется, опять ботва с датировками}

\item \sauth{Андрей Николаевич Колмогоров}{Колмогоров, Андрей
  Николаевич} (\pe{1973}{1985})
\end{items}

Надо сказать, что в начале века разруха гражданской войны не помешала
математике развиться.

\rauth{Н}{В}{Бугаев} закончил МГУ в 1863 году. Защитил диссертацию по
поводу сходимости рядов.  Командировки во Францию,
Германию. Докторская диссертация была по теории чисел.  Был деканом
МехМата. Кроме того, он увлекался не только математикой, он основал
психологическое общество.  У него была куча друзей, например,
композиторы \auth{Танеев}, \auth{Рубинштейн}, историк \auth{Соловьёв},
юрист \auth{Плевако}, медик \auth{Склифософский}, писатели
\rauth{Л}{Н}{Толстой}, \auth{Тургенев}, \auth{Писемский}.

\sauth{Андрей Белый}{Белый, Андрей} сын Бугаева.  <<В глазах его
застыла математическая сушь\ldots>> сказал он про своего отца.

\textbf{Древо Лузина.}  Первые ученики \sauth{Лузина}{Лузин}
занимались тем, что было во Франции.

Из его учеников очень известны такие топологи, как
\rauth{П}{С}{Александров} (\pe{1896}{1982}), \rauth{П}{С}{Урысон}
(\pe{1898}{1924}).  Урысон погиб очень рано (утонул).

\rauth{Д}{Е}{Меньшов} (\pe{1892}{1988}) тригонометрические ряды.

\rauth{А}{Н}{Колмогоров} (\pe{1903}{1987}) теория вероятностей, и ещё
много чего.  Его ученики хотели удариться в прикладную науку.

\rauth{Л}{А}{Люстерник} (\pe{1899}{1981}) топология, вариационное
исчисление.

\rauth{А}{О}{Гельфонд} (\pe{1906}{1968}) теория чисел (решена одна из
проблем Гильберта).

1933 год создание Мехмата МГУ. 1934 год академия наук переехала из
Ленинграда в Москву.  Был ещё упомянут \rauth{В}{А}{Стеклов}
(\pe{1864}{1926}), но к чему, так и осталось неизвестным.

\textbf{Математика и \emph{то время}}

Дело \sauth{Лузина}{Лузин}: его объявили врагом народа. Всё было
подстроено.  Его пригласили в одну школу, она ему понравилась, а
директор, как потом оказалось, был врагом народа, и вот его за
покрывательство врага народа и подшили.  Начали на него катить баллон,
но, к счастью, \emph{то время} только начиналось, и потом про него
забыли (\auth{Сталин} сказал, что вродэ нэлогично вэсти дэло
матэматика). Но сам Лузин ушёл из университета, преподавал в другом
институте.

На \sauth{Егорова}{Егоров} накатили за религиозные воззрения. Как
рассказывают очевидцы, один раз его замели \emph{дяди в кепках} вместе
со студентами. Потом студентов выпустили, а Егорова выслали в Казань.

\auth{Шнирельман} погиб в 30 лет. Он покончил жизнь самоубийством
после того, как его \emph{вызвали}.  Так Россия лишилась великого
теоретико числовика.


\marklec{22}{25.04.2006}

\section{Математика XX века}

Когда мы говорим о математике 20 века, так сразу вспоминаются проблемы
Гильберта.

В конце 19 века в математике важное изменение. 17, 18, 19 век это
математика Европы, арабская математика заглохла, Индия вообще
молчит. Всё развитие только в Европе.  Так вот в конце 19 века
математика потихоньку переползает на Дикий Запад, то есть в Америку.
Например, Сильвестр (Англия$\ra$университет Хопкинса), то есть
поначалу математики в Америке были, но импортные. Создаются общества.

\begin{items}{-2}
\item 1864 год ММО.
\item 1865 году Лондонское МО (аналог Академии Наук).
\item 1872 год французское МО.
\item 1888 год Создано NY MS (Нью-Йоркское математическое общество).
\item 1891 год немецкое МО.
\item 1894 год переименовано AMS.
\end{items}

Это время кооперации математиков через журналы тд В воздухе витает
идея создания международного математического союза.

\subsection{Первый математический конгресс}

1897 Цюрих, первый математический Конгресс. Это была не очень удачная
попытка.  13 человек в нём были американцы.

Инициатор объединения \sauth{Феликс Клейн}{Клейн, Феликс}, который
пригласил \sauth{Давида Гильберта}{Гильберт, Давид} в Гёттинген.

Итак, первый международный конгресс в Цюрихе.

\begin{items}{-2}
\item Единодушно были признана теория множеств Кантора.
\item Гурвиц 1859--1919 и Вольтерра 1860--1940. Теория аналитических
  функций и функциональный анализ.
\item Шрёдер (\pe{1841}{1902}) и Пеано (\pe{1851}{1932}) символика.
\item Пуанкаре не приехал, но прислал доклад <<Об отношениях между
  чистым анализом и математической физикой>>.
\end{items}

Отличие математики 19 и 20 века применения математики на практике.
Заключительный доклад Клейна (\pe{1849}{1925}) был посвящён
реорганизации математического образования.

\subsection{Второй математический конгресс}

Более важный конгресс, который оказал большее влияние на развитие
математики Париж, 1900.  4 важных секции (арифметика и алгебра,
анализ, геометрия, механика и математическая физика, история и
библиография, преподавания и методологии (5 и 6 секция), на которых и
состоялся знаменитый доклад \sauth{Гильберта}{Гильберт}.

\textbf{Пленарные заседания}:

\begin{items}{-2}
\item \sauth{Мориц Кантор}{Кантор, Мориц}. Об историографии математики
  со времён \eauth{Ж}{Монтюкла} и \eauth{Г}{Либри}
\item \eauth{В}{Вольтерра} Обзор творчества \eauth{Э}{Бетти},
  \eauth{Ф}{Бриоски}, etc
\item \auth{Пуанкаре}
\end{items}

Последний конгресс 200? год (19 секций, 18 и 19 история математики).

\subsection{Проблемы Гильберта}

8 августа 1900 года доклад Гильберта.  Это было объединённое заседание
5 и 6 секции Конгресса.

Вообще он сформулировал 10 проблем, но в письменном докладе было 23
штуки.

На рубеже 20 века всем математикам казалось, что всё будет
хорошо. Когда 20 век прошёл, то в России как раз всё стало
плохо. Более того, оглядываясь назад, мы видим, что будущее оказалось
совсем не таким светлым. Наш (20) век были чудовищным и великим.

\medskip

\textbf{1.} Проблема континуума, сформулировать арифметически понятие
континуума, существует ли кардинальное число между числом,
соответствующим счётному множеству, и числом, соответствующим
континууму. Можно ли рассматривать континуум как вполне упорядоченное
множество?

Не может быть решена методами математической логики и одной
общепринятой аксиоматической теорией множеств. 1936, \auth{Гёдель}: не
может быть опровергнута.  1963, \auth{Коэн}: не может быть доказана,
так как представляет собой утверждение, независимое от системы аксиом
\auth{Цермело} \sauth{Френкеля}{Френкель} (ZF).

\medskip

\textbf{2.} Непротиворечивость арифметики (1931 \auth{Гёдель}) не
может быть доказана финитными (как настаивал \auth{Гильберт})
средствами. С привлечением более сильных средств доказали в 1936 году
\eauth{Г}{Генцен} и в 1943 году \rauth{П}{С}{Новиков} (ученик
\sauth{Лузина}{Лузин}).

\medskip

\textbf{3.} Проблема существования неравносоставленных тетраэдров с
равными основаниями и высотами (1901, \auth{Ден}).  Невозможность
построения стереометрии без инфинитезимальных методов.

\medskip

\textbf{4.} Определение все проективных метрик (1903, \auth{Гамель}).

\medskip

\textbf{5.} Всякая связная локально евклидова топологическая группа
топологически изоморфна некоторой группе Ли. \sauth{фон
  Нейман}{Нейман, фон} (1933), \auth{Понтрягин} (1936), \auth{Шевалле}
(1941), \auth{Мальцев} (1946), \auth{Глисон}, \auth{Монтгомери}
(1952).

\medskip

\textbf{6.} Аксиоматизация теории вероятностей (\auth{Колмогоров},
1933).

\medskip

\textbf{7.} Трансцендентность $\al^\be$ ($\al \in \ol{\Q}$, $\al \ne
0$, $\al \ne 1$; $\be \in \R\wo\Q$).  \auth{Гельфонд}, \auth{Шнейдер}.

\medskip

\textbf{8.} Гипотеза \sauth{Римана}{Риман} о нулях дзета функции,
проблема \sauth{Гольбаха}{Гольбах}, проблема \sauth{Эйлера}{Эйлер}.
Успехи: 1914 \auth{Харди}, 1930 \auth{Шнирельман}, 1937
\auth{Виноградов}, 1941 \auth{Вейль}.

\medskip

\textbf{9.} Общий закон взаимности (в 1848 году решил
\rauth{И}{Р}{Шафаревич}, до него \auth{Артин}, \auth{Хассе} и~др.)

\medskip

Отступление о математических школах. Питер классический матан, теория
чисел, алгебра.  Москва дифгем, топология, теория функций
действительного переменного.  Этого питерцы не приветствовали.

Когда в 190? году освободилось место в Академии Наук, \auth{Жуковский}
отказался переезжать в Питер.  \auth{Стеклов} пишет письмо
\sauth{Ляпунову}{Ляпунов} в Харьков: <<Жуковский отказался, ну и
ладно: одним безграмотным москвичом меньше>>.

\medskip

\textbf{10.} Алгоритмическая разрешимость диофантовых уравнений
высоких степеней (1970, \rauth{Ю}{В}{Матиясевич}, отрицательный
результат).

\medskip

\textbf{11.} Построение теории квадратичных форм с любым числом
переменных и коэффициентами из произвольного поля алгебраических
чисел. Решена в 1924 году \auth{Хассе}.

\medskip

\textbf{12.} Обобщение теоремы \sauth{Кронекера}{Кронекер}
\sauth{Веба}{Веб} на произвольные поля алгебраических чисел.
Окончательное решение 1961 (\eauth{Г}{Шимура} и \eauth{Т}{Танияма}).

\medskip

\textbf{13.} Гипотеза о невозможности решения алгебраического
уравнения седьмой степени в общем случае посредством суперпозиции
непрерывных функций только двух переменных.  Опровергнута в 1957 году
в работах \sauth{Колмогорова}{Колмогоров} и \sauth{Арнольда}{Арнольд}.

\medskip

\textbf{14.} Доказательство теоремы конечности в теории инвариантов
для произвольных алгебраических групп. Результаты в этом направлении
для различных групп получены: \eauth{Г}{Вейль} (1939),
\eauth{М}{Нагата} (1964), \eauth{В}{Хабуш} (1975), В общем случае
гипотеза неверна: контрпример построил \eauth{М}{Нагата} в 1959 году.

\medskip

\textbf{15.} Обоснование исчислительной геометрии Г.Шуберта. В 1930
году ВанДерВарден предложил для исчисления над полем комплексных
чисел.

\medskip

\textbf{16.} Совокупность двух задач. Первая о топологии
алгебраических кривых и поверхностей. Гильберт предположил гипотезу о
взаимном расположении ветвей плоской алгебраической кривой 6 го
порядка.  Вопрос о топологии кривых изучил \rauth{И}{Г}{Петровский}
(1933, 1938).  В 1969 году \rauth{Д}{А}{Гудков} опроверг гипотезу
(построил контрпример).  Вопрос о расположении алгебраических
поверхностей 4 й степени в трёхмерном пространстве был изучен
\sauth{Варламовым}{Варламов} (1976, 1978, 1984) и
\sauth{В.\,Никулиным}{Никулин, В.}.

Вторая --- о числе предельных циклов ОДУ $\frac{dx}{dy} =
\frac{P(x,y)}{Q(x,y)}$.  где $P,Q \in K[x,y]_n$.

\medskip

\textbf{17.} О представимости рациональной функции от $n$ переменных.

\medskip

\textbf{18.} Три задачи по теории дискретных групп (о числе
кристаллографических групп: \auth{Либербах} в 1910 году сказал, что их
конченое число).  Вторая: Могут ли полиэдры разбиения $R^n$ быть
фундаментальной областью группы движений?  В 1928 \eauth{К}{Рейнгардт}
решил её с отрицательным результатом.  Третья о плотной упаковке шаров
не решена.

\medskip

\textbf{19.} Задача об уравнениях в частных производных Лагранжа про
существование только аналитических интегралов, даже если граничные
значения только непрерывны.

\medskip

\textbf{20.} Всякая ли регулярная вариационная задача имеет решение
при заданных граничных условиях?

\medskip

\textbf{21.} О существовании системы ОДУ с заданной группой монодромии
(\auth{Биркгоф}, 1913 доказал, что вроде бы существует, но в
доказательстве была ошибка), 1990 \rauth{А}{А}{Болибрух} построил
контрпример.

\medskip

\textbf{22.} Проблема унифоримизации аналитических отношений
посредством автоморфных функций. Для одномерных $\Cbb$ многообразий
решена в 1907 году \auth{Пуанкаре}.

\medskip

\textbf{23.} Развитие методов вариационного исчисления.

\subsection{Что дальше?}

Прикладной характер математики 20 века. Математика и физика.  В 19
веке казалось, что физика уже вся изучена.  Опыт
\sauth{Майкельсона}{Майкельсон} и теория излучения не рассеялись, а
превратились в СТО в 1905 году, а потом и ОТО.

Квантовая механика, парадоксы СТО. Всё грустно: мы столько лет думали,
что живём в трёхмерном пространстве с независимой временной
координатой. Чёрта с два! \auth{Лоренц}, \auth{Эйнштейн},
\auth{Пуанкаре}.  Связь СТО и геометрии
\sauth{Лобачевского}{Лобачевский}. Пространство
\sauth{Минковского}{Минковский} это примерно то, где мы живём.  В 1916
году \auth{Эйнштейн} добавил гравитацию, и тем самым показал, что
пространство Минковского это всего лишь локальное приближение к тому
кривому многообразию, в котором на самом деле живёт наш мир.

Так появлялась квантовая механика.

Броуновское движение. Оказалось, что функция
\sauth{Вейерштрасса}{Вейерштрасс} вполне естественный объект.  Более
того, выяснилось, что хаос это то, что очень часто возникает при
исследовании хороших, гладких отображений.

Надо сказать, что к концу 20 века проблемы \sauth{Гильберта}{Гильберт}
потихоньку исчерпались.  Но возникли другие проблемы.

Работы по математической логике, в частности, показали, что некоторые
теории неразрешимы. Значит, вообще говоря, решение наобум взятой
задачи невозможно алгоритмически.

\sauth{Алан Матисон Тьюринг}{Тьюринг, Алан Матисон} (40 е
годы). \eauth{Н}{Винер} отец кибернетики.

В 1970 е годы заканчивается запас задач, но появляется новая
гадость. Это теория \emph{солитонов}.  Это воскрешает дифференциальные
уравнения в конечных интегралах, хотя до этого момента казалось, что с
ними всё понятно.

Итак, видно, что математика не закончится даже с решением последней
проблемы Гильберта. История продолжается\ldots

\tbk

\newpage
\scriptsize


\end{document}
