\documentclass[a4paper]{article}
\usepackage{dmvn}
\newcounter{EUnits}
\stepcounter{EUnits}
\newcommand{\EUnit}{\par\medskip{\huge \textbf{Unit \arabic{EUnits}}}\par\stepcounter{EUnits}}
\newcommand{\ESect}[1]{\medskip\par{\large \textbf{#1}}\par}
\newcommand{\ETask}[2]{\medskip\par\textbf{#1.} \textit{#2}\par}

\newcommand{\Eqn}[1]{\begin{equation}#1\end{equation}}

\newcommand{\ETypeWr}[1]{\par\begin{ttfamily}#1\end{ttfamily}\par}

\hyphenation{expe-ri-men-tally}

\title{English for Students of Mathematics and Mechanics. Part II}
\author{\LaTeX{} reconstruction by DMVN Corporation}
\date{Last update: \today}

\begin{document}
\maketitle

This document was reconstructed from book to electronic version in \LaTeX{} by DMVN Corporation.
Look for new versions of this document on \dmvnwebsite{}, comments and reports about bugs
please send to \dmvnmail{}.

\EUnit
\ESect{Reading}
\ETask{I}{Pre-reading questions:}
1. Why do they say that mathematics is a language?

2. Have you any idea of the distributive axiom of algebra?

\ETask{II}{Read the text and see whether your points of view coincide with those expressed in the text. Make a list of mathematical terms. Consult your dictionary if necessary.}
\ESect{Text}
We begin our mathematical discussion with a review of some of the basic notions of algebra. However, because we want to be
very careful about the ideas involved, we shall not be able to do very much with algebra in this section. We shall instead
become concerned, but in not too neurotic a fashion, with the language of mathematics. It has been said that mathematics is
a language; this contention is a little difficult to support if we accept any of the ordinary descriptions of language. However,
it is true that there is a standard sort of terminology in mathematics that is much more concise and much briefer than the
garden variety of English. All of mathematics can be done without using this shorthand notation, but its incredible usefulness
makes it, practically speaking, a necessity.

The notions of number, addition, and multiplication are undefined. One of the axioms of algebra, called the distributive axiom,
is usually stated: \Eqn {x(y+z)=xy+xz.}
Let us be very certain that we understand just what is meant by this statement; in fact, let us discuss briefly what some refer
to as the `$x$-cessive $x$-cresence of $x$'s' involved in algebra. The proposition which is stated in (1) certainly does not
require this $x-y-z$ sort of language; the proposition can be stated: for any three numbers, the product of the first with
the sum of the second and third is equal to the sum of the products of the first with the second and the first with the third.
Of course, since we have all studied some algebra, the statement (1) seems considerably simpler than the translation into
vernacular which we have just given. And this is one of the points we want to emphasize: the mathematical language is not only
shorter, it is easier to comprehend.

We shall not usually abbreviate our statements to quite the extent that (1) is abbreviated. We shall usually include the
qualification that is supposed to be understood in (1), and we shall write

For all numbers $x$, $y$, and $z$
\Eqn{x(y+z)=xy+xz.}
Instead of `for all' we may frequently use `for every', or we may write `for each number $x$, each $y$, and each $z$'.
These several different expressions are supposed to mean the same thing. What we are really asserting is that if, in the expression,
we replace `$x$', `$y$', and `$z$' by numerals, then the resulting statement is always correct.

\ETask{III}{Comprehension tasks.}
1. State the proposition given in the text without using the signs and symbols of mathematics.

2. Comment on this statement: The mathematical language is not only shorter, it is easier to comprehend.

\ETask{IV}{What do the words in italics refer to? Check against the text.}
1. ...but \textit{its} incredible usefulness makes \textit{it} a necessity.

2. And \textit{this} is one of the points we want to emphasize.

3. ...\textit{it} is easier to comprehend.

4. \textit{These} several different expressions are supposed to mean \textit{the same thing}.

\ESect{Vocabulary}
\ETask{V}{Give the Russian equivalents of the following expressions:}
a review of some of the basic notions of algebra; to be careful about; the ideas involved; to do very much with algebra;
we shall instead become concerned with...; it is true; the garden variety of English; practically speaking; let us be very
certain that...; what some refer to as...; involved in algebra; to quite the extent; instead of.

\ETask{VI}{Join these notes with prepositions to make sentences. Then check against the text.}
1. Because we want to be very careful --- the ideas involved, we shall not be able to do very much --- algebra in this section.

2. It is true that there is a standard sort --- terminology --- mathematics.

3. One --- the axioms --- algebra is usually stated:...

4. Let us be very certain that we understand what is meant --- this statement.

5. We begin our mathematical discussion --- a review --- some --- the basic notions --- algebra.

6. We shall instead become concerned --- the language --- mathematics.

7. All --- mathematics can be done --- using this shorthand notation.

8. Let us discuss briefly what some refer --- as the `$x$-cessive $x$-cresence of $x$'s' involved --- algebra.

\ETask{VII}{Suggest meanings for \textbf{some} and \textbf{any} in these statements.}
1. We begin our mathematical discussion with a review of some of the basic notions of algebra.

2. We accept any of the ordinary descriptions of language.

3. Let us discuss briefly what some refer to as the `$x$-cessive $x$-cresence of $x$'s' involved in algebra.

4. For any three numbers the product of the first with the sum of the second and third is equal to the sum
of the products of the first with the second and the first with the third.

5. We have all studied some algebra.

\ETask{VIII}{In the following sentences pay attention to the verbs \textbf{make} and \textbf{do}.}
1. We shall not be able to do very much with algebra in this section.

2. All the mathematics can be done without using this shorthand notation.

3. Its incredible usefulness makes it a necessity.

4. The definition of `gizmo' is made in terms with which we are already familiar.

5. It is a little difficult to regard axioms as `self-evident truths', as has sometimes been done historically.

{\it Note: These two verbs have similar meanings, and sometimes it can be difficult to know which one to use.

\textbf{Do} is usually used when we are talking about work and it often means `be engaged in an activity'.

\textbf{Make} often expresses the idea of creation or construction.

But there are exceptions to these rules. We often use \textbf{do} and \textbf{make} in fixed phrases, where they go with particular nouns.

Try to remember some of the \textbf{make/do + noun} combinations. Then write sentences using these phrases:

\textbf{do +:} (me) a favour, harm, the housework, a lesson, the shopping, one's best, homework.

\textbf{make +:} an agreement, a demand, a mess, a mistake, a promise, a proposal, an attempt, progress, an impression, an appointment.
}

\ESect{Grammar}
\ETask{IX}{Rewrite these sentences in the Passive.}
1. The language of mathematics will concern us.

2. We can do mathematics without using this shorthand notation.

3. We call this axiom of algebra a distributive axiom.

4. We understand what we mean by this statement.

5. We can state this proposition without this $x-y-z$ sort of language.

\ETask{X}{Supply comparative or superlative forms.}
1. There is a standard sort of terminology in mathematics that is much (concise) and much (brief) than the garden variety of English.

2. ...the statement seems considerably (simple) than the translation...

3. ...the mathematical language is not only (short), it is (easy) to comprehend.

4. It is worth while to examine the notion of definition a little (closely).

\ETask{XI}{Rewrite these sentences using the Complex Subject construction.}

1. It has been said that mathematics is a language.

2. It is certain that we understand...

3. It seems that the statement is considerably simpler.

4. We suppose that this qualification is understood.

5. It is supposed that these several different expressions mean the same thing.

\ETask{XII}{Combine modals and their equivalents (should, can, may, must, have to, be able to) with the verbs in brackets.}

1. We shall not --- (to do) very much with algebra in this section.

2. All of mathematics --- (to do) without using this shorthand notation.

3. The proposition --- (to state):...

4. Instead of `for all' we --- frequently (to use) `for every', or we --- (to write) `for each number $x$, each $y$ and each $z$'.

5. There is another important fact about this mathematical language which --- (to notice).

6. Our mathematical language has the curious property that the letters which occur --- (to vary) almost at random.

7. One object --- (to have) many names, and we --- (to use) the names interchangeably.

8. Anything that --- (to say) about 4 --- (to say) with the same amount of truth about IV.

9. In each mathematical system there --- (to be) undefined terms.

10. We were going to describe a mathematical theory in a very careful way, so we --- (to define) every single term.

\ETask{XIII}{Use the Present Perfect (Active or Passive) of the verbs in brackets.}
1. It (to say) that mathematics is a language.

2. Since we (to study) some algebra, the statement seems considerably simpler than the translation into
vernacular which we just (to give).

3. Many students of geometry (to relieve) to discover that it is never necessary either to understand or to
use this cryptic definition.

4. We (to run) across the word `gizmo' and we (to look) in a dictionary to find its meaning.

5. This term (not to define) yet.

6. This field of knowledge (to advance) a great deal since the beginning of the 20th century.

\ETask{XIV}{Complete the sentences with proper forms of the words in brackets.}

1. Let (we, to be) very certain that we understand just what is meant by this statement.

2. The professor made (I, to redo) my report because he wasn't satisfied with it.

3. The teacher usually let (we, to consult) the dictionary while translating a text.

4. Don't let (he, to know) that we have finished the experiment.

5. The doctor made (she, to stay) in bed.

6. Let (we, to imagine) the situation.

7. Let (we, to suppose) that we are going to describe a mathematical theory in a very careful way.

8. Let (we, to consider) the very first definition.

\ETask{XV}{Write in the Past Tense forms and the Past Participles of the following irregular verbs.}
to begin, to write, to do, to become, to have, to be, to make, to speak, to understand, to mean, to let, to give.

\ETask{XVI}{Complete the sentences with ing-forms or Past Participles of the verbs in parentheses.}
1. We want to be very careful about the ideas (to involve).

2. We shall become (to concern) with the language of mathematics.

3. All of mathematics can be done without (to use) this shorthand notation.

4. One of the axioms of algebra, (to call) the distributive axiom, is usually stated:...

5. The proposition (to state) here doesn't require this sort of language.

6. What we are really (to assert) is that if in this expression we replace $x$, $y$ and $z$ by numerals, then the
(to result) statement is always correct.

7. There is a number which when (to add) to 2 yields 5.

8. There is one more question of meaning which we would like to discuss before (to end) this linguistic introspection.

9. The axioms tell us the nature of the undefined objects by (to state) relations between them.

\ETask{XVII}{Make the verbs in parentheses either active or passive.}

1. This contention is a little difficult to support if we (to accept) any of the ordinary descriptions of language.

2. The proposition can (to state) in the following way.

3. We shall not (to abbreviate) our statements to quite the extent that this one (to abbreviate).

4. We usually (to include) the qualification that (to suppose) (to understand).

5. These several different expressions (to suppose) (to mean) the same thing.

6. There is another important fact about this mathematical language which should (to notice).

7. The letters that (to use) in a statement of this sort are inconsequential.

8. Our mathematical language has the curious property that the letters which (to occur) can (to vary) almost at random.

\ESect{Writing}
\ETask{XVIII}{1. Make a list of three to five word-combinations which you think best characterize the language of mathematics.
2. Compare your list with those of your classmates. Do you agree with what is on their lists? Why or why not? 3. Write a
paragraph describing the language of mathematics. Use your own words as far as possible.}

\ESect{Supplementary Texts}
\ESect{Text 1}
It is worth while to examine the notion of definition a little more closely. Let us imagine this situation. We have run across
the word `gizmo' and we look in a dictionary to find its meaning. In the dictionary we find a list of synonyms --- say, `frazimer',
`whatsis', and `sicklebob' --- and it turns out that every one of these is unfamiliar. We then look up the word `frazimer', and
find that `whatsis', `sicklebob', and `gizmo' are listed as synonyms. Continuing, we look for `whatsis' and then `sicklebob'. But
the same list of words reappears. Quite clearly, there is no way for us to discover the meaning of the word `gizmo' unless the
definition of `gizmo' is made in terms with which we are already familiar.

Now let us suppose that we are going to describe a mathematical theory in a very careful way, and, in particular, suppose
that we want to define every single term. (This is precisely, what Euclid attempted in his treatment of geometry.) Let us
consider the very first definition; perhaps it reads `A gizmo is a...'. Then we may ask: In terms of what is the
gizmo to be defined? If this is the very first definition, with what sort of thing can we fill in the blank in the
definitional statement: `A gizmo is...'? It is clearly impossible to manage a definition without using some term, and if
this is the first definition, then that term has not been defined!

\ESect{Text 2}
In each mathematical system there must be undefined terms. This fact need not cause us excessive anguish, no more than our
inability to comprehend the notion of `that which is without breadth' causes difficulty in plane geometry. The only things
we needed to know about lines were asserted for us in the axioms of geometry.

It is a little difficult to regard axioms as `self-evident truths', as has sometimes been done historically, because they
are statements about objects which are themselves undefined. Intuitively, the axioms tell us the nature of the undefined
objects by stating relations between them, and we use the axioms to prove, by means of reasoning, mathematical theorems.
It is perfectly clear that we must have axioms, for if we start with undefined terms and have no axioms we have absolutely
no way to begin to prove theorems.

\ESect{Text 3}
There is another important fact about the mathematical language which should be noticed. For all numbers $a$, $b$, and $c$
$$a(b+c)=ab+ac$$
and for all numbers $a$, $r$, and $x$
$$a(r+x)=ar+ax$$
state precisely the same fact that is stated by (2) much earlier in Task II. That is, the particular letters that are used in
a statement of this sort, are inconsequential; so our mathematical language has the curious property that the letters which
occur can be varied almost at random!

There is another sort of statement which will occur frequently in our work. Consider the following:

There is a number $x$ such that $x+2=5$.

For some number $a$, $a+2=5$.

There exists a number $r$ such that $r+2=5$.

Clearly, all of these statements assert the same fact: namely, that there is a number which when added to 2 yields 5.
It is sometimes said that statements of the form `$x+2=5$' are conditional equations, and that statements of the form
`$x+y=y+x$' are identities. We shall not use this technical sort of jargon.

There is one more question of meaning which we would like to discuss before ending this linguistic introspection. In
just what sense is equality used? If, in a discussion of arable and roman numerals, we assert that $4=$~IV, what is to be
inferred from this statement? We shall always use equality in the sense of logical identity, and the assertion: `$4=$~IV'
is simply to mean that `4' and `IV' are both names for the same object. One object may have many names, and we may use
the names interchangeably. Anything which can be said about 4 can be said with the same amount of truth about IV.

\ESect{Text 4}
There are several statements about equality which are sometimes taken as axioms: for example, `each thing is equal to
itself', `things equal to the same thing are equal to each other', and `if, in an equation, equals are substituted for
equals, the results are equal'. Because we use equality only in the sense of identity, we can accept such statements
(and many more precise statements of this kind) as part of our natural conception of the notion of identity. Of course,
$4=4$ since each object is identical with itself. We may infer that $2+2=4$ if we know that $2+2=3+1$ and $3+1=4$. These
last two equalities tell us that `$2+2$' and `$3+1$' are names for the same object, and that `$3+1$' and `4' are names
for the same object; we simply have three different names for the same number, and quite evidently $2+2=4$. A statement
of equality is always to be considered intuitively as an assertion that the symbols on the left of the equality sign name
the same thing that is named by the symbols on the right.

We shall use letters `$x$', `$y$', etc., as if they were names. Strictly speaking, they are not names, although one
frequently finds in mathematics books such statements as `Let $x$ denote a fixed, but arbitrary number...'. Statements
such as these are part of the technical jargon which is psychologically useful in communication between mathematicians,
but one must not try to take such statements literally. In mathematics a name always refers to a single object, and pot,
in promiscuous fashion, to any one of a collection of objects. We use letters in much the same way that pronouns are used
and just as pronouns are used in sentence structure like nouns, so letters are used in mathematical structure like names.
Similar rules of `grammar' are to be used for letters and for names. Thus, if $x$ is a number and $x+5=7$, we take the
view that `$x+5$' names the same number as is named by `7', and hence infer without ado that $(x+5)+(-5)=7+(-5)$. In more
detail, we might phrase the reasoning as follows. It is true that $7+(-5)=7+(-5)$ because each thing is identical with
itself. If $x+5=7$, then `$x+5$' and `7' are names for the same thing, and we may replace `7' in `$7+(-5)$', using the
other name `$x+5$', and so find that $(x+5)+(-5)=7+(-5)$.

We shall not need to use arguments like the preceding one; our only objective in presenting such an argument here is to
obtain a clear intuitive understanding of the meaning of equality. The student should be able to see that each of the
following statements is true simply because equality means identity.

If $A$ is a triangle and $A=B$, then $B$ is a triangle.

If $x$, $y$, $u$ and $v$ are numbers and if $x=y$ and $u=v$, then $x+u=y+v$, $x+u=x+v$, and $x-u=y-u$.

If $x$ and $y$ are numbers and $x=y+2$, then $17x=17(y+2)$ and $x+2(y+2)+x=(y+2)+2x+x$.

On the other hand, the following statements are true, but their truth depends on additional algebraic facts and not just
on the notion of equality.

If $x$ and $y$ are numbers, then $x+y=y+x$.

If $x$ is a number, then $x+2x+3=3(x+1)$.

\EUnit
\ESect{Reading}
\ETask{I}{Pre-reading questions:}
1. What different sorts of numbers do you know?

2. What can you say on historical development of the number system?

\ETask{II}{Read the text and try to get the main points.}
\ESect{Text} We began our discussion of algebra with axioms that
apply to all numbers. We shall see that there are several
different sorts of numbers: the natural numbers, the integers, the
rational numbers, and the irrational numbers. We shall define
these various sorts of numbers in this section. There is a sort of
chronology among the several sorts, in the following sense. An
intuitive conception of number certainly preceded the formal
description that we are giving, and the sets of numbers that we
define in this section were intuitively understood before any
clear understanding of the complete number system, as we now know
it, was accomplished. Historically, these various kinds of numbers
were not discovered, or if you prefer, invented, simultaneously.
The natural numbers, 1, 2, 3,..., were certainly used first. The
number 0 was first employed only a few hundred years ago, and
still bears the stigma of being unnatural. Negative numbers
appeared very late in history. Finally, those mysterious objects,
the irrational numbers, achieved respectability and a secure
position only in the 19th century in spite of an abortive effort
to enter mathematics during the Hellenic age.

The axioms we have used for the numbers are not the only ones possible. It is quite possible to begin with axioms for
the natural numbers, and then to construct all other numbers. It is also possible to begin with set theory and construct
the natural numbers and then the rest of the numbers. We have chosen the set of axioms which we use just because we learn
more about numbers in less time than with either of the other possible approaches.

There is one difficulty with our approach. We do not know, so far, which numbers should be called natural, or integral,
or rational. This section is devoted to the definitions of these special kinds of numbers. The future development of this
book does not require the ideas expounded in this section, and we shall not go too deeply into the subject. But it seems
appropriate to try to connect our axiom system with the sorts of notions of number that you have studied before.

\ETask{III}{Comprehension questions:}
In the text they say:

1. `Historically these various kinds of numbers were not discovered simultaneously'. What is meant by this?

2. `There is one difficulty with our approach.' What is this difficulty?

\ETask{IV}{What do the words in italics refer to? Check against the text.}

1. We shall define {\it these} various sorts of numbers in this section.

2. ...the sets of numbers that we define in this section were intuitively understood before any clear understanding of
the complete number system, as we know {\it it}, was accomplished.

3. The axioms we have used for the numbers are not the only {\it ones} possible.

4. It is quite possible to begin with axioms for the natural numbers and then to construct {\it all other} numbers.

5. It is also possible to begin with set theory and construct the natural numbers and then {\it the rest of the numbers}.

6. There is one difficulty with {\it our} approach.

7. {\it It} seems appropriate to try to connect {\it our} axiom system with the sorts of notions that you have studied before.

\ESect{Vocabulary}

\ETask{V}{Give the Russian equivalents of the following expressions:}
there is a sort of chronology; in the following sense; or if you prefer; in spite of; the only possible; the rest of the
numbers; so far; go too deeply into the subject; it seems appropriate.

\ETask{VI}{Find words in the text that mean:}
make use of something (3); some but not many; meaning; statement that defines; diverse; go before; full, entire; attain (2);
invent, make known; happening or done at the same time; to come into, to join; to build; suited to.

\ETask{VII}{Give nouns corresponding to these verbs:}
to begin, to discuss, to describe, to define, to understand, to discover, to invent, to require, to identify,
to achieve, to construct, to choose, to approach, to apply, to develop, to connect.

\ETask{VIII}{Make these adjectives negative using \textbf{un, in, il, ir, im}. Consult your dictionary if necessary.}
natural, rational, possible, regular, important, capable, complete, legal, mobile

\ETask{IX}{Supply the necessary prepositions to make the sentences. Check against the text.}

1. We begin our discussion --- algebra --- axioms that apply --- all numbers.

2. There is a sort --- chronology --- the several sorts.

3. The irrational numbers achieved respectability and a secure position only --- the 19th century --- spite --- an abortive
effort to enter mathematics --- the Hellenic age.

4. It is also possible to begin --- set theory and construct the natural numbers and then the rest --- the numbers.

5. This section is devoted --- the definitions --- these special kinds --- numbers.

6. We shall not go too deeply --- the subject.

\ETask{X}{Explain the use of one in these sentences:}

1. The axioms we have used for the numbers are not the only ones possible.

2. There is one difficulty with our approach.

{\it Think of some other examples with the word one in different meanings.}

\ESect{Grammar}
\ETask{XI}{Make these sentences interrogative or negative.}
1. There is a sort of chronology among the several sorts.

2. There is one difficulty with our approach.

3. There is a very nice exposition of this construction in Landau's `Foundations of Analysis'.

4. Unfortunately, there is no written treatment of this construction which is even semi-elementary.

\ETask{XII}{Make the comparative and superlative forms of the given words:}
much, many, little, good, bad

{\it Give some examples with comparative or superlative degrees of these words.}

\ETask{XIII}{Supply the appropriate forms (Active or Passive) of the verbs given in brackets. Then check against the text.}

1. We began our discussion with axioms that (to apply) to all numbers.

2. An intuitive conception of number certainly (to precede) the formal description that we (to give), and the sets of
numbers that we (to define) in this section (to understand) intuitively before any clear understanding of the complete
number system, as we now (to know) it, (to accomplish).

3. Historically, these various kinds of numbers (not to discover), or if you (to prefer), (to invent), simultaneously.

4. The natural numbers, 1, 2, 3,..., (to use) certainly first.

5. The number 0 (to employ) first only a few hundred years ago, and still (to bear) the stigma of being unnatural.

6. We (not to know) which numbers should (to call) natural, or integral, or rational.

7. This section (to devote) to the definitions of these special kinds of numbers.

8. The future development of this book (not to require) the ideas expounded in this section.

\ETask{XIV}{Supply the proper forms (Past Indefinite or Present Perfect) of the verbs given in brackets.}

1. The sets of numbers that we define in this section (to understand) intuitively before any clear understanding of the
complete number system, as we now know it, was accomplished.

2. The axioms we (to use) for the numbers are not the only ones possible.

3. We (to choose) the set of axioms which we use just because we learn more about numbers in less time than with either
of the other possible approaches.

4. The number 0 (to employ) first only a few hundred years ago.

5. But it seems appropriate to try to connect our axiom system with the sorts of notions that you (to study) before.

6. Negative numbers (to appear) very late in history.

7. The irrational numbers (to achieve) respectability and a secure position only in the 19th century.

8. We (to list) so far what might be called the purely algebraic axioms about the numbers, and we (to examine) the
consequences of these axioms in some detail.

\ETask{XV}{Supply articles (a, the, ---). Then check against the text.}

1. There is --- sort of --- chronology among --- several sorts, in --- following sense.

2. --- intuitive conception of number certainly preceded --- formal description that we are giving.

3. --- number 0 was first employed only --- few hundred years ago.

4. We shall not identify --- set of negative numbers until --- next section.

5. --- axioms we have used for --- numbers are not --- only ones possible.

6. It is also possible to begin with --- set theory and construct --- natural numbers and then --- rest of --- numbers.

7. We shall not go too deeply into --- subject.

\ETask{XVI}{Each of the following sentences has one mistake. Find it and give the correct variant.}

1. We shall define this various sorts of numbers.

2. There is a sort of chronology between the several sorts.

3. These various kinds of numbers did not discovered simultaneously.

4. The number 0 still bear the stigma of being unnatural.

5. The irrational numbers achieved respectability only in the 19th century in spite an abortive effort to enter
mathematics during the Hellenic age.

6. It is quiet possible to begin with axioms for the natural numbers.

7. We begin with the set of axioms because we learn more about numbers in less time then with either of the other possible approaches.

8. We do not know which numbers shall be called natural, or integral, or rational.

9. An intuitive conception of number preceded the formal description that we giving.

10. It seem appropriate to try to connect our axiom system with the sorts of notions of number that you have studied before.

\ESect{Discussion}

\ETask{XVII}{What can you say concerning different sorts of numbers: the natural numbers, the integers, the rational numbers and
the irrational numbers? Try to define them.}

\ESect{Writing}
\ETask{XVIII}{There are many ways of showing sequential relationships. The text under consideration gives one of them.
1. Look through it again and pick out all the linking words used to describe chronological development of numbers.
2. Put away the original, use only the list of linking words, write a paragraph presenting the complete number system.
Remember to use your own words.}

\ESect{Supplementary Texts}
\ESect{Text 1}
In our discussion of the elementary algebra of the number system there have been already three undefined notions --- number,
addition, and multiplication. In this section we want to discuss briefly an undefined notion which is even more fundamental than
that of number. We digress for a moment to explain its importance.

There are a large number of abstract concepts in mathematics: number, addition, multiplication, line, plane, vector, and so on.
We may very well ask if it is necessary, as we go more and more deeply into mathematical theory, to keep listing more and more
undefined terms. That is, as the theory grows, must the list of undefined objects also grow? It turns out that this is not the
case. It is only necessary to have a single, short list of such undefined objects, and the length of this list is really
surprising. It is one of the achievements of twentieth century mathematics that a single undefined notion, that of set membership,
is adequate for all of mathematics! Numbers, addition, and all the other mathematical concepts can be defined in terms of this
single notion. Unfortunately, actually beginning with this single notion and developing mathematics is rather too complicated
a task for this course, but perhaps knowing that it is possible will explain why the notions of set and of set membership are
so pervasive in modern mathematics.

The terms `set', `collection', and `class' will all be used interchangeably. Intuitively, a set is just a bunch of objects,
and the objects are called members of the set. A bunch of grapes, a covey of quail, and a pride of lions can be considered to be
sets with members being, respectively, grapes, quail, and lions. If an object $x$ is a member of a set $A$, we write $x \in A$.
Thus, if $\mathbb{R}$ is the set of numbers, then $0 \in \mathbb{R}$ and $1 \in \mathbb{R}$. We assume that if we are given an
object $x$ and a set $A$, then it either is or is not the case that the object is a member of the set. In the first case we
write `$x \in A$' and in the latter we write `$x \notin A$'.

We frequently describe a set by listing its members in the following fashion. $\{0,2,1\}$ is the set whose members are 0, 2
and 1, and $\{0, 1, -1\}$ has the members 0, 1, and -1. The order of listing is unimportant; the set $\{0, 1, 2\}$ is identical
with the set $\{2, 1, 0\}$. Moreover, we do not `count a member more than once', and $\{0,2,2\}$ is identical with $\{0,2\}$.
The critical fact about sets is:

Axiom of Extent. The sets $A$ and $B$ are identical if they have the same members; that is, if every member of $A$ is a member
of $B$ and every member of $B$ is a member of $A$, then $A = B$.

A set is completely described if we know its elements, and we shall frequently define sets by giving a condition which enables
us to decide whether or not an object belongs to the set. The following notation is usually used. `$\{x:~~(some~condition~
about~x)\}$' is read as `the set of all $x$ such that the condition about $x$ is the case'.

\ESect{Text 2}
Let $\varnothing$ be the set $\{x: x = 0$ and $x = 1\}$. This is a very curious set because, of course, there is no object which
is equal to 0 and also equal to 1. It is called the empty set or the void set; it has no members. That such a set `exists' may
surprise you a little, but the set $\varnothing$ is not `nothing'; an empty box is very different from no box at all.

It should be noticed, that there are other ways of describing the empty set $\varnothing$. For example, $\{x: x \neq x \}$ is
the empty set because it is always the case that $x = x$. Any condition such that no object satisfies the condition could
be used to define $\varnothing$. In general, a condition permits us to define a set, but many different conditions may give
the same set.

\ESect{Text 3}
Some rather interesting intellectual calisthenics can be based on the notion of set and the rather surprising fact that many
declaratory sentences can be interpreted as statements about sets. We will give two examples.

EXAMPLE We are given the statements:

(1) Socrates is a man.

(2) All men are mortal.

(3) Therefore Socrates is mortal.
\medskip

The problem is to find and state a mathematical theorem of which this is a special case. We begin by noticing that the first
statement can be considered as the assertion that Socrates is a member of a certain set. Let $A$ be the set of all men; then
the first statement can be paraphrased as `Socrates $ \in A$'. The second statement is a statement of set inclusion;
if $M$ denotes the set of all mortal beings, then this statement says that each member of $A$ is a member of $M$, that is,
$A \subset M$. The conclusion is supposed to be: Socrates $\in M$. Thus, from `Socrates $\in A$' and `$A \subset M$' we are
supposed to deduce that Socrates $\in M$. Using different terms, it is proposed that: If $x \in B$ and $B \subset C$, then
$x \in C$. This is surely a theorem of set theory.

Before leaving this example, let us consider how certain other statements could be translated into set theoretic language.
Suppose it is asserted that no man is mortal; this simply asserts that there is no object which belongs to both $A$ and to $M$
--- that is, $A \cap M$ is the empty set. Again, if it is asserted that some men are mortal, this amounts to saying that
there are objects belonging to both $A$ and $M$, that is, $A \cap M$ is not the empty set.

EXAMPLE (From Lewis Carroll, with salutations to Schroeder.) From the following three assertions we are to make whatever
deductions are possible.

(1) Nobody who really appreciates Beethoven fails to keep silence, while the Moonlight Sonata is being played.

(2) Guinea-pigs are hopelessly ignorant of music.

(3) No one who is hopelessly ignorant of music ever keeps silence while the Moonlight Sonata is being played.

These can be interpreted as statements about various sets. Let $G=$~the set of guinea-pigs, $H=$~the set of creatures that
are hopelessly ignorant of music, $K=$~the set of creatures who keep silence while the Moonlight Sonata is being played,
and $R=$~the set of creatures that really appreciate Beethoven. The three statements now have the translations:

(1)' $R \subset K$

(2)' $G \subset H$,

(3)' $H \cap K$ is the empty set $\varnothing$.

Figure 1 shows the relationship of the four sets $R$, $K$, $G$, and $H$. It is clear that there are no objects belonging
to both $R$ and $G$ (the mathematical theorem is: if $R \subset K$, $G \subset H$, and $H \cap K = \varnothing$, then
$R \cap G = \varnothing$). Translating this back from the mathematical language, we conclude that guinea-pigs do not
really appreciate Beethoven.

Before leaving this whimsical problem I should like to point out that this is precisely the way in which mathematics is
applied. We always begin with some sort of physical assumptions (in this case the statements (1), (2), and (3)) then by
analogy or guess work translate these into mathematical hypotheses (the statements (1)', (2)' and (3)'), establish a
mathematical theorem or theorems ($R \subset K$, $G \subset H$, and $H \cap K = \varnothing$, then
$R \cap G = \varnothing$) and finally, retranslate the mathematical theorem to infer something about the physical problem
(guinea-pigs do not really appreciate Beethoven).

\ESect{Text 4}
Our first task is this: how do we define the natural numbers? We want 1 to be a natural number, and if $n$ is a natural
number then $n + 1$ is to be a natural number. We begin by looking at some sets that are so large that every natural number
is a member.

DEFINITION A set $A$ of numbers is inductive if and only if 1 is a member of $A$ and $x + 1$ is a member of $A$ whenever
$x$ is a member of $A$.

Let us give several examples of inductive sets. For the sake of the examples, we'll assume that we understand the notion
of `greater than or equal to' although we shall not study inequalities until the next section. The set of all numbers is
certainly inductive, because of the closure axiom; the set of positive numbers is inductive; the set $\{x : x = 1$ or $x=2$
or $x \geqslant 3 \}$ is also inductive; and we shall certainly want the set of natural numbers to be inductive. No finite
set of numbers is inductive, and the set of all even integers fails to be inductive. The number 1 belongs to every
inductive set, and so does 2 (why?), and 3. In fact it seems pretty clear that every natural number ought to belong to
every inductive set, and this is the key to our definition.

DEFINITION A number $x$ is a natural number if and only if $x$ belongs to every inductive set. There is a perfectly obvious
consequence of this definition: if $A$ is an inductive set then every natural number belongs to $A$. This important theorem
has a name.

THEOREM (Principle of mathematical induction) If $A$ is an inductive set, then every natural number belongs to $A$.
We will use this theorem, and the definitions which precede it, to establish a few simple properties of the natural numbers.
First we observe that 1 is a natural number because 1 belongs to every inductive set. Next, we show that:

THEOREM If $x$ is a natural number, then so is $x + 1$. Proof: If $x$ is a natural number, then $x$ belongs to every inductive
set by definition. But if $x$ belongs to an inductive set, so does $x + 1$ because of the definition of inductive set.
Consequently $x + 1$ belongs to every inductive set, and hence, by the definition of natural number, $x + 1$ is a natural number.
The preceding theorem, together with the fact that 1 is a natural number, shows that the set of natural numbers is an inductive set.

\ESect{Text 5}
As we have remarked, it is possible to show that if we are given two systems that satisfy all of the axioms that we list,
then one system is simply a carbon copy of the other. In brief, the axiom of this section really completes the list of
axioms about numbers.

We begin with the notion of the smallest element of a set of numbers. If $A$ is a set of numbers it may happen that there is
a member $a$ of $A$ which has the property that it is smaller than every other member of $A$. Formally, we make the definition:

DEFINITION A number $a$ is the smallest or least member of a set $A$ of numbers if and only if $a \subset A$ and $a < b$ for
every other member $b$ of $A$.

The smallest member of a set is just the member that is furthest to the left in our geometrical interpretation. The number 1 is
the smallest member of the set $\mathbb{N}$ of natural numbers; the number 0 is the smallest member of the set of non-negative numbers,
and the number -2 is the smallest member of $\{1, -2, 4\}$. However, there are many sets that have no smallest member. For example,
there is no smallest member of the set of all numbers (if $x$ is a number, then $x - 1$ is another number which is smaller).
There is no smallest member of the set of positive numbers, for if $x$ is any positive number, then $x/2$ is a positive number
which is smaller. It is also easy to see that $\{x : 2 < x$ and $x < 3\}$ has no smallest member. And of course the empty set
has no smallest member, since it has no member.

We could also define the largest member of a set and much the same sort of situation would occur. However, we want to consider
numbers that are in a slightly different relationship to a set $A$. We will say that a number $b$ is an upper bound for a set $A$
if and only if $b$ is at least as large as every member of $A$. The number $b$ may or may not belong to $A$, this is irrelevant;
but $b$ is supposed to be no smaller than any member of $A$. Formally:

DEFINITION A number $b$ is an upper bound for a set $A$ if and only if $b > x$ or $b = x$ for every member $x$ of $A$.
Thus 1 is not an upper bound for the set $\{0, -4, 2\}$ because a member of the set, namely 2, is larger than 1; 2 is an upper bound
for $\{0, -4, 2\}$, and so is 5, and so is 367. In general, if a set has an upper bound $b$ it has many other upper bounds ---
for example, $b + 1$ is also an upper bound. In fact, if $b$ is an upper bound for a set $A$ then every number larger than $b$ is also
an upper bound. The set of all numbers has no upper bound. The set of natural numbers also fails to have an upper bound (this is
a very important fact which, curiously enough, cannot be proved without assuming the axiom of continuity), It is also true that
the set $P$ of positive numbers has no upper bound - that is, there is no number which is greater than or equal to every positive
number (proof: if $b$ is an upper bound for $P$, then $b > 1$, and since $1 > 0$, it follows that $b > 0$; thus $b$ is positive,
and clearly $b + 1$ is a larger positive number, which is a contradiction). Every number is an upper bound for the empty set.

\ESect{Text 6}
The purpose of this section is very restricted: it is to introduce the terms `finite', `countable', and `infinite'. It provides
a basis for the study of cardinal numbers, but it does not pursue this study. Although the theories of cardinal and ordinal numbers
are fascinating in their own right, it turns out that very little exposure to these topics is really essential for the material
in this text. A reader wishing to learn about these topics would do well to read the books of P.R. Halmos and W. Sierpinski.
We shall assume familiarity with the set of natural numbers. We shall denote this set by the symbol $\mathbb{N}$; the elements
of $\mathbb{N}$ are denoted by the familiar symbols

$$1,2,3,...$$

The set $\mathbb{N}$ has the property of being ordered in a very well-known way: we all have an intuitive idea of what is meant
by saying that a natural number $n$ is less than or equal to a natural number $m$. We now borrow this notion, realizing that
complete precision requires more analysis than we have given. We assume that, relative to this ordering, every non-empty subset
of $\mathbb{N}$ has a smallest element. This is an important property of $\mathbb{N}$; we sometimes say that $\mathbb{N}$ is
well-ordered, meaning that $\mathbb{N}$ has this property. This Well-Ordering Property is equivalent to mathematical induction.
We shall feel free to make use of arguments based on mathematical induction, which we suppose to be familiar to the reader.
By an initial segment of $\mathbb{N}$ is meant a set of natural numbers which precede or equal some fixed element of $\mathbb{N}$.
Thus an initial segment $S$ of $\mathbb{N}$ determines and is determined by an element $n$ of $\mathbb{N}$ as follows:
An element $x$ of $\mathbb{N}$ belongs to $S$ if and only if $x \leqslant n$. For example, the subset $\{1,2\}$ is the initial
segment of $\mathbb{N}$ determined by the natural number 2; the subset $\{1, 2, 3, 4\}$ is the initial segment of $\mathbb{N}$
determined by the natural number 4; but the subset $\{1,3,5\}$ of $\mathbb{N}$ is not an initial segment of $\mathbb{N}$, since
it contains 3 but not 2, and 5 but not 4.

DEFINITION. A set $B$ is finite if it is empty or if there is a one-one function with domain $B$ and range in an initial segment
of $\mathbb{N}$. If there is no such function, the set is infinite. If there is a one-one function with domain $B$ and range equal
to all of $\mathbb{N}$, then the set $B$ is denumerable (or enumerable). If a set is either finite or denumerable, it is said to
be countable.

When there is a one-one function with domain $B$ and range $C$, we sometimes say that $B$ can be put into one-one correspondence
with $C$. By using this terminology, we rephrase the definition and say that a set $B$ is finite if it is empty or can be put
into one-one correspondence with a subset of an initial segment of $\mathbb{N}$. We say that $B$ is denumerable if it can be put
into one-one correspondence with all of $\mathbb{N}$.

It will be noted that, by definition, a set $B$ is either finite or infinite. However, it may be that, owing to the description
of the set, it may not be a trivial matter to decide whether the given set $B$ is finite or infinite. In other words, it may not
be easy to define a one-one function on $B$ to a subset of an initial segment of $\mathbb{N}$, for it often requires some
familiarity with $B$ and considerable ingenuity in order to define such a function.

The subsets of $\mathbb{N}$ denoted by $\{1,3,5\}$, $\{2,4,6,8, 10\}$, $\{2, 3,..., 100\}$, are finite since, although they are
not initial segments of $\mathbb{N}$, they are contained in initial segments of $\mathbb{N}$ and hence can be put into one-one
correspondence with subsets of initial segments of $\mathbb{N}$. The set $E$ of even natural numbers $$E = \{2,4,6,8,...\}$$
and the set $O$ of odd natural numbers $$O = \{1,3,5,7,...\}$$ are not initial segments of $\mathbb{N}$, and they cannot be put
into one-one correspondence with subsets of initial segments of $\mathbb{N}$. (Why?) Hence both of the sets $E$ and $O$ are infinite,
but since they can be put into one-one correspondence with all of $\mathbb{N}$ (how?), they are both denumerable. Even though,
the set $\mathbb{Z}$ of all integers $$\mathbb{Z} = \{...,-2,-1,0,1,2,...\},$$ contains the set $\mathbb{N}$, it may be seen that
$\mathbb{Z}$ is a denumerable set.

\EUnit

\ESect{Reading}

\ETask{I}{Pre-reading questions:}
1. What do you know about `ordered pairs'?

2. Try to give a definition of the notion `ordered pairs'.

\ETask{II}{Read the text. Think of the questions to the main points of it. Then try to answer the questions given by other students.}

\ESect{Text}
Our first task is to define the notion of ordered pair. That is, for every two objects a and b we want to have an object $(a, b)$
which has the property that if $(a, b) = (c, d)$, then $a = c$ and $b = d$. It is quite reasonable to take the notion of ordered
pair as undefined, and to assume the property just listed as an axiom. However, this seems a little wasteful since it is not at
all difficult to define an ordered pair in terms of sets. Consequently, in order to give the reader a little more practice in set
theory, we shall make this definition and prove the desired property as a theorem. If the reader doesn't want any more practice
in set theory he can assume, without damaging his understanding of this course, that ordered pair is undefined and that our first
theorem is an axiom.

The idea underlying the definition we give is quite simple. Given two objects $a$ and $b$, we want to construct a set involving
$a$ and $b$ and having some special structure so that we can `recover' $a$ and $b$ from the set; this special set will then be
called the ordered pair $(a, b)$. (The term `ordered' is syntactical, not mathematical; it derives from the fact that $(a, b)$
is not necessarily equal to $(b, a)$.) Our first guess might be to set $(a, b) = (a, b)$, but this doesn't work because
$(a, b) = (b, a)$ and the theorem we want would be false. However, a slightly more complicated definition does work. The
definition we give suggests strongly the terminology `first coordinate' and `second coordinate' which will presently be used.

DEFINITION $(a, b) = \{\{a, 1\}, \{b, 2\}\}$. Thus the members of $(a, b)$ are $\{a, 1\}$ and $\{b,2\}$.
It is convenient, to prove a lemma before establishing the single theorem on ordered pairs.

LEMMA If $\{x, y\} = \{x, z\}$, then $y = z$. Proof If $\{x, y\} = \{x, z\}$, then $y \in \{x, z\}$ and therefore either $y = z$,
in which case the lemma is established, or $y = x$. In the latter case $\{x\} = \{x, y\} = \{x, z\}$ and hence $z = x$. Thus in this
case $y = x$ and $z = x$ and therefore $y = z$.

THEOREM (on ordered pairs) If $(a, b) = (c, d)$, then $a = b$ and $c = d$.
Proof By hypothesis $\{\{a, 1\}, \{b, 2\}\} = \{\{c, 1\}, \{d, 2\}\}$ and therefore $\{a, 1\} = \{c, 1\}$ or $\{a, 1\} = \{d, 2\}$.
In the first case $a = c$ by reason of the preceding lemma, and in the second case it is easy to see that $a = 2$ and $d = 1$.
Similarly $\{c, 1\} = \{a, 1\}$ or $\{c, 1\} = \{b, 2\}$ and we infer that $a = c$ or $c = 2$ and $b = 1$. Thus either $a = c$ or
all of the following hold: $a = 2$, $d = 1$, $c = 2$, and $b = 1$. Thus in any case $a = c$. We may therefore apply the preceding
lemma to the case $\{\{a, 1\}, \{b, 2\}\} = \{\{c, 1\}, \{d, 2\}\}$ and deduce that $\{b, 2\} = \{d, 2\}$. If we apply the lemma
again, we see that $b = d$.

If a set $A$ is an ordered pair, then $A = (a, b)$ for unique objects $a$ and $b$, because of the preceding theorem. Hence we can,
without ambiguity, define:

DEFINITION The first coordinate of $(a, b)$ is $a$ and the second coordinate is $b$.

\ETask{III}{In the text they say:}
1. It is quite reasonable to take the notion of ordered pair as undefined, and to assume the property as an axiom.

2. It is not at all difficult to define an ordered pair in terms of sets.

\textit{Comment on these statements.}

\ESect{Vocabulary}

\ETask{IV}{Give the Russian equivalents of the following:}
it is quite reasonable; this seems a little wasteful; in terms of; in order to give; in the latter case; not necessarily equal;
by hypothesis; by reason of the preceding lemma; all of the following hold; in any case.

\ETask{V}{Find words in the text that mean:}
concept; special quality; to believe before there is proof; nevertheless; as ; accordingly; to form the basis of something;
incorrect; to put forward for consideration or as a possibility; soon, at present time; thing well done or successfully completed;
greater or more important; inside; so; to set up; for that reason(2); the second of two things already mentioned; without ambiguity.

\ETask{VI}{All these verbs have appeared in the text. Give nouns for these verbs. Consult your dictionary for their meaning,
spelling and pronunciation:}
to define, to derive, to assume, to suggest, to list, to use, to prove, to establish, to damage, to apply, to recover, to deduce.

\ETask{VII}{In the text there are such pairs of nouns as set theory and number system. Here the first noun behaves like an adjective,
and it is an attribute to the second one. Translate into Russian the pairs given below and think of your own examples:}
town library; iron bridge; paper bag; life insurance; oil difficulties; flight delay; price index; steel production; steel
demand; market position; car key; computer keyboard.

\ETask{VIII}{Put in the correct prepositions.}
1. It is quite reasonable to take the notion --- ordered pair as undefined.

2. If the reader doesn't want any more practice --- set theory he can assume, --- damaging his understanding --- this course,
that ordered pair is undefined and that our first theorem is an axiom.

3. The term `ordered' is syntactical, not mathematical; it derives --- the fact that $(a, b)$ does not necessarily equal $(b,a)$.

4. This accomplishment and C. S. Pierce's brilliant idea --- defining relations as sets --- ordered pairs were major steps --- the
construction --- all mathematics --- set theory.

5. It is convenient to prove a lemma before establishing the single theorem --- ordered pairs.

\ESect{Grammar}

\ETask{IX}{Rewrite these sentences using the ing-form instead of the italicized verbs.}

1. We want to have an object which \textit{has} the property...

2. The idea which \textit{underlies} the definition we give is quite simple.

3. ...we want to construct a set which \textit{involves} $a$ and $b$ and \textit{has} some special structure...

4. ...a definition which \textit{involves} only set theory can be given.

\ETask{X}{Supply with proper forms of the verbs in brackets.}

1. Our first task is (to define) the notion of (to order) pair.

2 He can (to assume) without (to damage) his (to understand) of this course that (to order) pair is undefined.

3. (to give) two objects a and b, we want (to construct) a set (to involve) a and b and (to have) some special structure.

4. This accomplishment and C. S. Pierce's brilliant idea of (to define) relations as sets of (to order)
pairs (to be) major steps in the construction of all mathematics within set theory.

5. It is convenient (to prove) a lemma before (to establish) the single theorem of (to order) pairs.

6. We shall (to make) this definition and (to prove) the (to desire) property as a theorem.

\ETask{XI}{You know that the verb to do as an auxiliary verb is used to make questions and negatives in the simple present and simple
past tenses, and also in place of a verb in short answers and question tags.}

E.g.

\textit{Do} you agree with me? --- Yes, I do.

You \textit{don't} know this rule.

She speaks English, \textit{doesn't} she?

\textit{Sometimes the verb \textbf{to do} is used for emphasis.}

E.g.

I \textit{did} answer your question.

\textit{Do} sit down.

\textit{Here \textbf{do, does, did} always have stress, and emphasize positive meaning. We use the simple form of the verb after them.
Comment on the use of the verb \textbf{to do} in the sentences from the text.}

1. ...but this doesn't work...

2. However, a slightly more complicated definition does work.

\textit{Now make the statements below more emphatic.}

1. I \textit{wrote} that letter. I am positive of it.

2. She \textit{took} the book. She told me so.

3. You are mistaken. I \textit{want} to study English.

4. This student doesn't study hard, but he \textit{attends} class regularly.

5. Columbus didn't reach the Indies, but he \textit{reached} a new continent.

\ETask{XII}{In the text there is a sentence with \textbf{in order to}.}
`Consequently, in order to give the reader a little more practice in set theory, we shall make this definition and prove the
desired property as a theorem.'

\textit{The table below shows how to use \textbf{in order to} and \textbf{for}.}

(a) He came here in order to study English. \textbf{In order to} is used to express purpose. It answers the question `Why?'

(b) He came here to study English. \textbf{In order} is often omitted, as in (b).

I went to the store for some bread. \textbf{For} is sometimes used to express purpose, but it is a preposition and is
followed by a noun object.

I went to the store (in order) to buy some bread. Use \textbf{in order to} not for with a verb.

\textit{Insert \textbf{in order} to or \textbf{for}:}

1. She borrowed my dictionary --- look up the spelling of `occurrence'.

2. I went to the library --- study last night.

3. My friend went to Chicago --- a business conference.

4. I came to this school --- learn English.

5. Mary went to the market --- some vegetables.

6. I need a part-time job --- earn some money --- my school expenses.

\ETask{XIII}{In the text there are sentences with Subjunctive Mood.}

1. Our first guess might be to set $(a, b) = \{a, b\}$.

2. ...and the theorem we want would be false.

\textit{Complete the sentences below with would, could, might and the verb in brackets. Translate them into Russian.}

1. I --- (to read) the book, but I can't find it anywhere.

2. He --- (to visit) you, but he doesn't know your address.

3. I --- (not to finish) the work without your help.

4. My friend --- (to solve) the problem, but unfortunately he is out.

5. I --- (to answer) the phone, but I didn't hear it ring.

6. He --- (to finish) his education, but he had to quit school and find a job in order to support his family.

\ETask{XIV}{Put the words in the right order to form a sentence. Then check against the text.}

1. underlying, the, we, idea, the, quite, is, simple, give, definition.

2. undefined, ordered, is, and, first, is, an, theorem, axiom, our, pair.

3. hold, all, the, of, following.

4. slightly, however, a, more, definition, complicated, work, does.

5. two, given, want, we, objects, set, a, to construct.

\ESect{Writing}
\ETask{XV}{Give an illustration to clarify the following idea.}

The method of labeling points in the geometric plane establishes a one-to-one correspondence between geometric points and ordered pairs.
\textit{Use: for example, for instance, an example of this, as an example, that is, etc.}

\ESect{Supplementary Texts}
\ESect{Text 1}
These sections begin the study of the geometry and algebra of 2- and 3-dimensional vector spaces. We define ordered pairs
and triples, and then describe geometric points as ordered pairs (or triples) of numbers; we also call these vectors. We are
anxious to utilize the geometric intuition which the student may have obtained in plane geometry, and we consequently use
geometric illustrations systematically. Later we begin to define geometric objects precisely, in terms of the undefined terms
and axioms of our system. Finally, we apply vector algebra to the study of straight lines; it is interesting to notice that
all of the theorems about lines which occur here are true in either two or three dimensions and, in fact, in any number
of dimensions.

Lastly, just as our study of the number system led us to consider the notion of set, so the notion of function arises here.
We define `function' in terms of sets. It is noteworthy that the notion of function is, with the possible exception of the
notion of set, the most important concept in mathematics.

\ESect{Text 2}

DEFINITIONS An ordered pair of numbers is called a 2-dimensional vector, or a point in the coordinate plane, or simply a point in
the plane. The coordinate plane, or simply the plane, or 2-dimensional vector space, is the set of all ordered pairs of numbers;
it is $\{(x, y) : x$ and $y$ are numbers $\}$.

There is a very good geometric reason for calling the set of pairs
of numbers the coordinate plane. Let us consider the geometric
interpretation shown in Figure 1 [Omitted in this Edition].

A horizontal line is called the $x$-axis, a vertical line is called the $y$-axis, and the point where they intersect is called
the origin. The $x$-direction is to the right, and the $y$-direction is vertically upward; if you think of the figure as a map,
then the $y$-direction is north and the $x$-direction is east. To each pair of numbers --- for example, $(5, 2)$ --- we assign
a point in the plane by means of the prescription: go from the origin in the $x$-direction a number of units equal to the first
coordinate of the pair and then proceed in the $y$-direction a number of units equal to the second coordinate of the pair.
Thus $(5, 2)$ is 5 units in the $x$-direction and 2 units in the $y$-direction. The first coordinate of a pair is frequently
called the $x$-coordinate (or the abscissa), and the second the $y$-coordinate (or ordinate). If the $x$-coordinate is negative,
we understand that we are to proceed in the direction opposite to the $x$-direction the stated number of units, and similarly
for the $y$-coordinate. Thus $(-3,1)$ is the point as shown in the upper left-hand corner; $(4, -3)$ is as shown in the lower
right-hand corner. We notice that the $x$-coordinate of a point is the distance from the $y$-axis to the point, with the sign
correctly chosen, and the $y$-coordinate is the distance from the $x$-axis to the point.

This method of labeling points in the geometric plane establishes a one-to-one correspondence between geometric points and ordered
pairs; that is, each point is labeled by just one ordered pair of numbers, and each ordered pair of numbers labels just one point.
This is an important accomplishment. It makes it possible for us to study properties of the geometric plane by means of the algebraic
properties of the set of pairs of numbers.

There is no great difficulty in defining ordered triple of objects and in finding a geometric interpretation of the set of ordered
triples of numbers. In much the same fashion as with pairs, we seek a definition of $(a, b, c)$ such that if $(a, b, c) = (p, q, r)$,
then $a = p$, $b = q$, and $c = r$. The following definition will do.

DEFINITION $(a, b, c)= ((a, b), c)$.

That is, the ordered triple $(a, b, c)$ is the ordered pair whose first coordinate is $(a,b)$ and whose second coordinate is $c$.

THEOREM If $(a, b, c) = (p, q, r)$, then $a = p$, $b = q$, and $c = r$. Proof: If $(a, b, c) = (p, q, r)$, then by the definition
of ordered triple we see that ((a, b), c) = ((p, q), r). Applying the theorem on ordered pairs we infer that $(a, b) = (p, q)$
and $c = r$, and applying the same theorem again, we see that $a = p$ and $b = q$.

Again, if the student prefers to take the notion of ordered triple as undefined and assume this Theorem as an axiom, no harm is done.

\ESect{Text 3}
Before beginning the study of vector geometry we want to define one of the most important notions of mathematics, that of
function. Intuitively, a function is supposed to be a correspondence which assigns to each object in a certain class, called the
domain of the function, some corresponding object. For example, we may consider the correspondence that assigns to each person
his mother; this correspondence is a function, the essential feature being that to each member of the domain, which in this case
is the set of all people, there is assigned precisely one member of another set (in this case, the set of all mothers). Of course,
several different people (brothers and sisters) may correspond to the same mother; in mathematical terms, this correspondence isn't
one to one. But to each person, there corresponds just exactly one mother. Let us denote the correspondence by $M$, we emphasize
that $M$ is not the set of all persons, nor the set of all mothers, but is the correspondence itself. If $C$ is a member of the
domain of $M$, that is, $C$ is a person, then the mother of $C$ is denoted by $M(C)$. Thus, $M$(George VII) = Victoria and
$M$(Elizabeth) = Ann Boleyn. In general, if $f$ is a function and $x$ is a member of the domain of $f$, then $f(x)$ is that
object that corresponds to $x$. It is called the value of $f$ at $x$.

A more mathematical example may be helpful. Consider the correspondence $f$ which is pictured schematically in Figure 1. The
domain of $f$ is supposed to consist of points $a$, $b$, $c$, $d$ and the correspondence $f$ sends $a$ into a point $p$, $b$
into a point $q$, $c$ into $r$, and $d$ into $p$; that is, $f(a) = p$, $f(b) = q$, $f(c) = r$, and $f(d) = p$.

Before giving further examples of functions let us consider the major mathematical problem of the section. What are we going
to define a function to be? The key to the definition is the fact that a correspondence is completely described if we know
what object corresponds to each member of the domain. This suggests that if we know the set of all ordered pairs of the form
(member of the domain, corresponding object), then the function should be completely described. In other words, the set of all
pairs $(x,f(x))$ completely describes it. Thus the function $f$ defined in the preceding paragraph is completely described by
the set $\{(a, p), (b, q), \{c, r), (d, p)\}$. If this seems confusing, don't worry. The intuitive notion of a function is
just that of a correspondence, and the only facts about functions which we need are given in the single theorem of this section.

DEFINITION $f$ is a function if and only if $f$ is a set of ordered pairs, no two of which have the same first coordinate.
More formally, $f$ is a function if and only if $f$ is a set, each member of $f$ is an ordered pair, and if $(a, y)$ and
$(x, z)$ belong to $f$, then $y = z$.

For example, $\{(0, 1), (1, 1)\}$ is a function; but $\{(1,0), (1, 1)\}$ is not a function.

We will use the terms `function', `correspondence', and `map' or `mapping', interchangeably.

A member of a function $f$ is then a pair with first coordinate a member of the domain and second coordinate the object which
corresponds to this member. The requirement that no two pairs belonging to $f$ have the same first coordinate is just a way of
ensuring that there is just one object corresponding to each member of the domain. We can define very easily the domain of a
function, and the value of the function at a member of its domain.

DEFINITIONS The domain of a function $f$ is the set of first coordinates of members of $f$. The value of $f$ at a member $x$
of its domain, denoted $f(x)$, is the second coordinate of that member of whose first coordinate is $x$. The set of all second
coordinates of members of $f$ is the range of $f$.

\EUnit

\ESect{Reading}
\ETask{I}{Pre-reading questions:}

1. What is a quadratic equation?

2. What is necessary to know for solving quadratic equations?

\ETask{II}{Read the text. Make a list of unknown words. Consult your dictionary for their meaning and pronunciation.}

\ESect{Text}
It is natural after considering the solution of linear equations to attempt to solve quadratic equations. That is, we attempt to
find for just what numbers $x$ it is true that $ax^2+bx+c=0$, where $a$, $b$, and $c$ are numbers. If $a$ is 0, then the equation
to be solved is actually linear and we have already discussed this problem in detail; so we will always assume that the
coefficient of $x^2$ is not 0.

We will prove a single theorem and make a single application of the theorem. But it should be said that the main purpose of the
section is not just to know and understand the theorem, but also to understand and be able to apply the process by which the
theorem is proved. The student, in brief, should acquire some skill in solving quadratic equations, both by means of the theorem
of this section, and by means of the procedure which underlies the theorem. It is to be regretted that acquiring skill is
sometimes a little uninteresting; nevertheless, anyone interested in mathematics must learn to solve quadratic equations just
as every child has to learn to tie shoelaces; otherwise one stumbles.

The single idea that underlies the solution of quadratic equations is this: it is easy to find the numbers $x$ such that
$x^2+2px+p^2=q$, if $p$ and $q$ are any numbers. The reason it is easy is that $x^2+2px+p^2=(x+p)^2$, and the numbers $x$ such
that $(x+p)^2=q$ can be classified as follows: if $q$ is positive, then $x + p$ must be either $\sqrt{q}$ or $-\sqrt{q}$ and
$x$ must be either $-p+\sqrt{q}$ or $-p-\sqrt{q}$, if $q$ is negative there is no number $x$ such that $(x+p)^2=q$, and
if $q = 0$ the only number such that $(x+p)^2=0$ is $-p$. The method of solving other quadratic equations is roughly this:
we think fondly of the equation
$$x^2+2px+p^2=(x+p)^2=q$$
and attempt, without employing violence, to arrange the equation to be solved so that it resembles the preceding. Somewhat
more precisely: we attempt to arrange the equation so there is a `perfect square' on the left, and a number on the right.

\ETask{III}{In the text they say:}
1. Anyone interested in mathematics must learn to solve quadratic equations.

2. We attempt to arrange the equation so there is a `perfect square' on the left and a number on the right.

\textit{Comment on these statements.}

\ETask{IV}{What do the italicized words refer to?}
1. ...we have already discussed \textit{this} problem in detail.

2. ...otherwise \textit{one} stumbles.

3. The method of solving \textit{other} quadratic equation is roughly this...

4. ...so that it resembles the preceding.

\ESect{Vocabulary}

\ETask{V}{Give the Russian equivalents of the following:}
it is natural; that is; in detail; in brief; by means of; underlies the theorem; it is to be regretted; as follows;
anyone interested in; the only number; somewhat more precisely; on the left; on the right.

\ETask{VI}{Find words in the text that mean:}
to think about; to find the answer to...; to try; to exchange ideas on; to believe before there is proof; to make
practical use of; putting to a special or practical use; most important; intention; to gain by skill, ability or by one's
own efforts; ability to do something well; to be sorry for; however; if not; single; to put into order; to be like.

\ETask{VII}{Supply these sentences with nouns instead of verbs in brackets.}
1. It is natural after considering (to solve) of linear equation to attempt to solve quadratic equations.

2. We will make a single (to apply) of the theorem.

3. The student should acquire some skill in solving quadratic equations by (to mean) of this theorem and by (to mean)
of (to proceed) which underlies the theorem.

\textit{Give nouns for these verbs:}
to attempt, to acquire, to discuss, to classify, to assume, to employ, to know, to arrange, to prove, to solve

\ETask{VIII}{What is the opposite of these adjectives? if necessary consult your dictionary:}
interesting, possible, expected, honest, agreeable, thinkable, legal, moral

\ETask{IX}{In the text you have come across some paired conjunctions. This table shows how to use \textbf{both ... and; no only ...
but also; either ... or; neither ... nor}.}
(a) Both by brother and my sister are here.

(b) Not only my mother but also my sister is here.

(c) Not only my sister but also my parents are here.

(d) Neither my mother nor my sister is there.

(e) Neither my sister nor my patents are there.

\textit{Two subjects connected by both... and take a plural verb.
When two subjects are connected by not only... but also, either... or, neither... nor the subject that is closer to the verb
determines whether the verb is singular or plural.}

(f) The research project will take both time and money.

(g) Yesterday is not only rained but also snowed.

(h) I'll take either chemistry or physics next quarter.

(i) That book is neither interesting nor accurate.

Notice the parallel structure in the examples. The same grammatical form should follow each word of the pair.

\textit{Supply \textbf{is} or \textbf{are} in the following.}

1. Both the teacher and the student ... here.

2. Neither the teacher nor the student ... here.

3. Not only the teacher but also the student ... here.

4. Not only the teacher but also the students ... here.

5. Either the students or the teacher ... planning to come.

6. Either the teacher or the students ... planning to come.

\textit{Complete these sentences with \textbf{both ... and; either ... or; neither ... nor}.}

1. The student should acquire some skill in solving quadratic equations --- by means of this theorem --- by means of the
procedure which underlies the theorem.

2. If $q$ is positive, then $x + p$ must be --- $\sqrt{q}$ --- $-\sqrt{q}$ and $x$ must be --- $-p+\sqrt{q}$ --- $-p-\sqrt{q}$.

3. This problem is --- difficult --- interesting.

4. We can --- prove this theorem --- apply it.

5. The city suffers from --- air --- water pollution.

6. I'm studying --- math --- chemistry.

7. Our country has --- good schools --- good universities.

8. The result was --- good --- bad.

9. --- the library --- the bookstore has the book I need.

10. --- coal --- oil are irreplaceable natural resources.

\ETask{X}{Suggest the meaning of the italicized words:}
1. The student should acquire \textit{some} skill in solving quadratic equations.

2. Nevertheless, \textit{anyone} interested in mathematics must learn to solve quadratic equations just as \textit{every}
child has to learn to tie shoelaces.

3. It is easy to find the numbers $x$ such that $x^2+2px+p^2=q$, if $p$ and $q$ are \textit{any} numbers.

4. If $q$ is negative there is \textit{no} number $x$ such that $(x+p)^2=q$.

\ESect{Grammar}

\ETask{XI}{Put questions to these sentences, either general or to the italicized words.}

1. We have already discussed this problem \textit{in detail}. (General)

2. We will always assume that \textit{the coefficient of} $x^2$ \textit{is not} 0.

3. The student should acquire some skill \textit{in solving quadratic equations}.

4. Anyone interested in mathematics must learn \textit{to solve quadratic equations}.

5. This idea underlies \textit{the solution of quadratic equations}.

6. We attempt to \textit{arrange the equation}. (General)

7. There is no number $x$ such that $(x+p)^2=q$. (General)

\ETask{XII}{Note the use of the Infinitive in the following sentence:}
If $a$ is 0, then the equation \textit{to be solved} is actually linear.
\textit{The to-infinitive is often used after a noun to convey advice, purpose, etc. This construction is like a relative clause.}

E.g. The person \textit{to ask} is Mike ($=$the person whom you should ask).

I've got an essay to write ($=$an essay which I must write).

\textit{Sometimes active and passive infinitives are interchangeable.}

E.g. After the discussion, there was some work \textit{to do / to be done}.

\textit{When the subject of the sentence is the person who is to do the action described by the infinitive, we do not normally
use the passive.}
E.g. I have a report to prepare (Not `to be prepared').

\textit{Now translate the sentences into Russian.}

1. All the data to be presented here refer also to the above problems.

2. The explanation will probably be considerably modified in the years to come.

3. The method to be followed is based upon some peculiar properties of these rays.

4. Here are some more figures to be referred to later.

5. This is the important question to be answered.

6. He was the first to note this phenomenon.

7. This theory will be adequate for practical applications through centuries to come.

\ETask{XIII}{Supply these sentences with modal verbs. Then refer
to the text. Alternatives are possible. In each case give a reason
for your choice.}

1. It --- be said that the main purpose of the section is not just to know and understand the theorem, but also to understand
and --- to apply the process by which the theorem is proved.

2. The student --- acquire some skill in solving quadratic equations.

3. It --- to be regretted that acquiring skill is sometimes a little uninteresting.

4. Anyone interested in mathematics --- learn to solve quadratic equations just as every child --- to learn to tie shoelaces.

5. The numbers $x$ such that $(x + p)^2 = q$ --- be classified as follows.

6. If $q$ is positive, then $x + p$ --- either $\sqrt{q}$ or $-\sqrt{q}$ and $x$ --- either $-p+\sqrt{q}$ or $-p-\sqrt{q}$.

\ETask{XIV}{In the text we have sentences with \textbf{so that}, used to express purpose. It expresses the same meaning as
\textbf{in order to}. The word \textbf{that} is often omitted, especially in speaking.}
We attempt to arrange the equation to be solved so that it resembles the preceding.

We attempt to arrange the equation so there is a `perfect square' on the left, and a number on the right.

\textit{Look at the examples with \textbf{so that}. Mind the tense of the verb in the adverb clause after it.}

Please turn down the radio so (that) I can get to steep.

My wife turned down the radio so (that) I could get to sleep.

Put the milk in the refrigerator so (that) it won't (doesn't) spoil.

I put the milk in the refrigerator so (that) it wouldn't spoil.

\textit{Combine the ideas by using so (that) according to the patterns given above.}

1. Please be quiet. I want to be able to hear what the teacher is saying.

2. I asked the students to be quiet. I wanted to be able to hear what the teacher was saying.

3. I am going to cash a check. I want to make sure that I will have (or have) enough money to go to the market.

4. I cashed a check yesterday. I wanted to make sure that I had enough money to go to the market.

5. I'm going to leave the party early. I want to be able to get a good night's sleep tonight.

6. It's a good idea for you to learn how to type. You'll be able to type your own papers when you go to the university.

7. I turned on the TV. I wanted to listen to the news while I was making dinner.

8. I unplugged the phone. I didn't want to be interrupted while I was working.

\ETask{XV}{Put the words in the right order to form sentences. Then check against the text.}

1. to be, the, linear, is, actually, solved, equation

2. discussed, we, already, in, problem, this, have, detail

3. prove, theorem, a, of, will, single, make, application, theorem, we, a, and, single, the

4. should, skill, quadratic, student, some, solving, the, acquire, in, equations

5. in, learn, equations, interested, must, quadratic, anyone, mathematics, to solve

\ESect{Writing}

\ETask{XVI}{In the text below, the paragraphs are not in the correct order. Rearrange the paragraphs into what you consider to
be a suitably coherent order.}

1. In order to get the variable alone on one side of the equation, we must perform inverse operations to `undo' the operations
on that side. Remember, addition undoes subtraction (and vice-versa), and multiplication undoes division (and vice-versa).

2. In equations which contain only one variable, the highest power of the variable is called the equation's degree. For
example, $5x-2=3x+8$ is called a first degree, or linear equation; $x^2+2x=10-x$ is called a second degree, or quadratic equation;
and $x^3+5=x^2+6$ is called a third degree, or cubic equation.

3. Since we want the new, transformed equation to have the same roots as the given equation, we must follow the equivalence
principle stated below.

4. The particular method used to solve an equation depends upon the equation's degree. For first degree (linear) equations,
we use the method of inverse operations. The idea behind this method is to transform the given equation into an equivalent
equation (an equation having the same roots) in which the variable appears alone on one side of the equation, and a number
appears alone on the other. This number will be the root of the given equation.

5. Whenever a number is added to, subtracted from, multiplied by, or divided into one side of an equation, the same thing must
be done on the other side of the equation.
\textit{What is in your opinion the central idea of the text? Write it out in one sentence. Use: The main idea of the text under review
is... The text is devoted to (deals with, is concerned with) ... etc.}

\ESect{Supplementary Texts}
\ESect{Text 1}
We have come to the point where we shall introduce a formal description of the real number system $\mathbb{R}$. Since we are more
concerned in this text with the study of real functions than the development of the number system, we choose to introduce
$\mathbb{R}$ as an Archimedean field which has one additional property.

The reader will recall that if $F$ is an ordered field and if $a$, $b$ belong to $F$ and $a < b$, then the closed interval
determined by $a$, $b$, which we shall denote by $[a, b]$, consists of all elements $x$ in $F$ satisfying $a \leqslant x \leqslant b$.
It will also be recalled that if $x$ is any element of an Archimedean field $F$, then there is a nested sequence $\{I_n\}$ of
nonempty closed intervals whose only common point is $x$. However, it was seen that a nested sequence of closed intervals does
not always have a common point in certain Archimedean fields (such as $Q$). It is this property that we now use to characterize
the real number system among general Archimedean fields.

DEFINITION. An Archimedean field $\mathbb{R}$ is said to be complete if each sequence of nonempty closed intervals $I_n = [a_n, b_n]$,
$n$ belong to $\mathbb{N}$, of $\mathbb{R}$ which is nested in the sense that
$$I_1 \supseteq I_2 \supseteq \ldots \supseteq I_n \supseteq \ldots,$$
has an element which belongs to all of the intervals $I_n$.

ASSUMPTION. We shall assume that there exists a complete ordered field which we shall call the real number system and shall
denote by $\mathbb{R}$. An element of $\mathbb{R}$ will be called a real number.

We have introduced $\mathbb{R}$ axiomatically, in that we assume that it is a set which satisfies a certain list of properties.
This approach raises the question as to whether such a set exists and to what extent it is uniquely determined. Since we shall
not settle these questions, we have frankly identified as an assumption that there is a complete ordered field. However a few
words supporting the reasonableness of this assumption are in order.

The existence of a set which is a complete ordered field can be demonstrated by actual construction. If one feels sufficiently
familiar with the rational field $\mathbb{Q}$, one can define real numbers to be special subsets of $\mathbb{Q}$ and define
addition, multiplication, and order relations between these subsets in such a way as to obtain a complete ordered field.
There are two standard procedures that are used in doing this: one is Dedekind's method of `cuts' which is discussed in the
books of W. Rudin and E. Landau. The second way is Cantor's method of `Cauchy sequences' which is discussed in the book of
N. T. Hamilton and J. Landin.

In the last paragraph we have asserted that it is possible to construct a model of $\mathbb{R}$ from $\mathbb{Q}$ (in at least
two different ways). It is also possible to construct a model of $\mathbb{R}$ from the set $\mathbb{N}$ of natural numbers and
this is often taken as the starting point by those who, like Kronecker, regard the natural numbers as given by God. However,
since even the set of natural numbers has its subtleties (such as the Well-ordering Property), we feel that the most
satisfactory procedure is to go through the process of first constructing the set $\mathbb{N}$ from primitive set theoretic
concepts, then developing the set $\mathbb{Z}$ of integers, next constructing the field $\mathbb{Q}$ of rationals, and finally
the set $\mathbb{R}$. This procedure is not particularly difficult to follow and it is edifying; however, it is rather lengthy.
Since it is presented in detail in the book of N.T. Hamilton and J. Landin, it will not be given here.

From the remarks already made, it is clear that complete ordered fields can be constructed in different ways. Thus we cannot
say that there is a unique complete ordered field. However, it is true that all of the methods of construction suggested above
lead to complete ordered fields that are `isomorphic'. This means that if $\mathbb{R}_1$ and $\mathbb{R}_2$ are complete
ordered fields obtained by these constructions, then there exists a one --- one mapping $\varphi$ of $\mathbb{R}_1$ onto
$\mathbb{R}_2$ such that (I) $\varphi$ sends a rational element of $\mathbb{R}_1$ into the corresponding rational element of
$\mathbb{R}_2$, (II) $\varphi$ sends $a + b$ into $\varphi(a)+\varphi(b)$, (III) $\varphi$ sends $ab$ into
$\varphi(a)\varphi(b)$, and (IV) $\varphi$ sends a positive element of $\mathbb{R}_1$ into a positive element of
$\mathbb{R}_2$. Within naive set theory, we can provide an argument showing that any two complete ordered fields are isomorphic
in the sense described. Whether this argument can be formalized within a given system of logic depends on the rules of
inference employed in the system. Thus the question of the extent to which the real number system can be regarded as being
uniquely determined is a rather delicate logical and philosophical issue. However, for our purposes this uniqueness
(or lack of it) is not important, for we can choose any particular complete ordered field as our mode) for the real number system.

\ESect{Text 2}
This section begins the study of solution of equations. It will turn out that the linear equations whose solutions we discuss
are closely connected with certain functions which we will also call linear. We discuss the geometry of linear functions briefly.
The same sort of approach is used in the next sections in the discussion of quadratic equations and functions.

The basic problem is the following: given numbers $m$ and $b$, we are required to find all numbers $x$ such that $mx + b = 0$.
Such numbers are called solutions of the equation: $mx + b = 0$; and the set of all solutions is called the solution set. Of
course, this problem is entirely trivial, and the results may be summarized: if $m \neq 0$, then $-b/m$ is the only solution,
if $m = 0$ and $b \neq 0$, then there is no solution; and if $m = 0$ and $b = 0$, then every number is a solution. In other words,
in these three cases the solution set is, respectively, $\{- b/m\}$, the empty set, and the set of all numbers.

It is instructive, in considering the equation $mx+b=0$, to consider simultaneously the function $f$ whose value at $x$ is
$mx + b$. That is, the domain of $f$ is the set of all numbers, and $f(x)=mx+b$ for each number $x$, alternatively, the function
$f$ (or if you prefer, the graph of $f$) is $\bc{(x,mx+b)\cln x \text{ is a number }}$. The solutions of the equation $mx + b = 0$
are then just the numbers $x$; such that $f(x)=0$, and the solution set is simply $\{x: f(x) = 0\}$. We digress to comment
on the algebraic descriptions of sets. There are three very common ways of describing sets. Many sets can be described very
naturally as the set of numbers, or pairs of numbers, satisfying certain inequalities; such sets are sometimes called solution
sets of inequalities. Again, it is frequently convenient to describe a set as the range of a function, the most noteworthy
example being a line. The line through $A$ with direction number $B$ is just the range of the function $X$, where $X(t) = A + tB$
for all numbers $t$. Finally, as we have noticed, a set may sometimes be described as the set of points where a function $f$ is
equal to 0, or as the solution set of the equation $f(x) = 0$. The later work will display many more examples of this sort
of description.

\EUnit

\ESect{Reading}
\ETask{I}{Pre-reading questions:}
1. What is a prime number?

2. Is 1 a prime number? If not, give your reasons.

3. Is there a point in the list of primes after which there are no more?

\ETask{II}{Now read the text. Check your answers to the questions above. How many of them did you answer correctly?}

\ESect{Text}
Prime numbers are such integers as have only one and themselves for divisors, i.e., $$2, 3, 5, 7, 11, 13, 17, 19, 23, 29,...$$
Most of these are odd numbers; in fact only one of them is even. However, this is not a very deep property, for actually only
one prime is divisible by three, etc.

The number 1 is not considered as a prime number for it gives no additional information concerning the nature of a number when
it is decomposed into prime factors, e.g., $12=2\times 2 \times 3$. In such multiplicative building up of numbers we construct
(or decompose) a number from (into) prime factors. Of course our numbers can be built up in an additive manner e.g.,
$$6=2+4=1+1+1+3=1+1+1+1+1+1.$$ But reducing all numbers to a sum of units is of little interest.

At the very beginning of the list, the primes are quite dense; however, they become less dense as we proceed to higher numbers.
This seems reasonable as we expect a high number to have a greater chance of being divisible by a prime number than a low one.
Now, do the primes stop altogether? Is there a point in the list of primes after which there are no more? This question is posed
and answered in the `Elements' of Euclid (which contains much of number theory) in the following way. Euclid asserts that there
can be no last prime number.

\ETask{III}{Comprehension check:}
1. `However, this is not a very deep property'. What is meant by this?

2. At the very beginning of the list, the primes are quite dense. What happens to their density as we proceed to higher numbers?

\ETask{IV}{What do the words in italics refer to? Check against the text and note them down.}

1. Prime numbers are such integers as have only one and \textit{themselves} for divisors.

2. The number 1 is not considered as a prime number, for \textit{it} gives no additional information concerning the nature
of a number when \textit{it} is decomposed into prime factors.

3. \textit{This} seems reasonable as we expect a high number to have a greater chance of being divisible by a prime number
than a low one.

4. \textit{This} question is posed and answered in the Elements of Euclid (\textit{which} contains much of number theory)
in the following way.

\ETask{V}{Join these notes to make sentences. Then check against the text.}

1. Most --- are --- numbers --- fact --- one --- even.

2. Of course --- numbers --- built --- manner.

3. Now --- the primes --- altogether?

4. But reducing --- a sum of units --- interest.

\ESect{Vocabulary}

\ETask{VI}{Find words in the text that mean:}
1. entirely, completely

2. and the reason is that; because

3. the number by which another number is divided

4. about, with regard to; in connection with

5. final

\ETask{VII}{Give the Russian equivalents of the following:}
in fact; is divisible by; information concerning the nature of a number: in such multiplicative building up of numbers;
a number is decomposed into prime factors; is of little interest; at the very beginning; this seems reasonable; the question
is posed and answered; in the following way.

\ETask{VIII}{Match these abbreviations with their meanings:}

\ETypeWr{
i.e.~~~~~and the rest and so on etc.

etc.~~~~~for example

e.g.~~~~~that is}

\ETask{IX}{The following words are in the text. Use dictionary to find the
other parts of speech. Check the pronunciation.}
\ETypeWr{
Noun~~~~~~~~~~~~~~~Adjective~~~~~~~~~~~Verb

================================================

divisor~~~~~~~~~~~~divisible~~~~~~~~~~~-

information~~~~~~~~-~~~~~~~~~~~~~~~~~~~-

-~~~~~~~~~~~~~~~~~~multiplicative~~~~~~-

-~~~~~~~~~~~~~~~~~~reasonable~~~~~~~~~~-

-~~~~~~~~~~~~~~~~~~-~~~~~~~~~~~~~~~~~~~construct

-~~~~~~~~~~~~~~~~~~additive~~~~~~~~~~~~-

-~~~~~~~~~~~~~~~~~~-~~~~~~~~~~~~~~~~~~~consider}

\ETask{X}{Pay attention to the meaning of \textbf{for} in the following sentences. Think of your own examples with \textbf{for}
in different meanings.}

1. Prime numbers are such integers as have only one and themselves for divisors.

2. This is not a very deep property, for actually only one prime is divisible by three.

3. The number 1 is not considered as a prime number, for it gives no additional information concerning the nature of a number
when it is decomposed into prime factors.

\ETask{XI}{Supply the suitable prepositions. Then check against the text.}

1. Most --- these are odd numbers; --- fact only one --- them is even.

2. Only one prime is divisible --- three.

3. ...when it is decomposed --- prime factors.

4. --- such multiplicative building up --- numbers we construct (or decompose) a number --- (---) prime factors.

5. Of course our numbers can be built up --- an additive manner.

6. Reducing all numbers --- a sum --- units is --- little interest.

7. --- the very beginning --- the list, the primes are quite dense; however they become less dense as we proceed --- higher numbers.

8. Is there a point --- the list --- primes --- which there are no more?

9. This question is posed and answered --- the Elements of Euclid (which contains much --- number theory) --- the following way.

\ETask{XII}{What does one mean in these sentences?}

1. Prime numbers are such integers as have only one and themselves for divisors.

2. Most of these are odd numbers; in fact only one of them is even.

3. However, this is not a very deep property, for actually only one prime is divisible by three.

4. This seems reasonable as we expect a high number to have a greater chance of being divisible by a prime number than a low one.

\ESect{Grammar}

\ETask{XIII}{Themselves is a reflexive pronoun. What are the other forms of reflexive pronouns?}

singular: 1...2...3a)...b)...c)...

plural: 1...2...3...

\textit{Which reflexive pronoun form goes into the space?}

1. I hurt ...

2. They fooled ...

3. We saw ... on television.

4. You ... said so.

5. The little girl wrote the letter all by ...

6. The government made ... unpopular.

7. It is important to make a clear distinction between the function ... and the values of the function.

8. The students ... may discover shortcuts for solving the problem.

\ETask{XIV}{Give the proper form of the verb (Active or Passive).}

1. The number 1 (not to consider) as a prime number, for it (to give) no additional information concerning the nature
of a number when it (to decompose) into prime factors.

2. We (to construct) or (to decompose) a number from (into) prime factors.

3. Our numbers (can, to build up) in an additive manner.

4. The primes (to become) less dense as we (to proceed) to higher numbers.

5. This (to seem) reasonable as we (to expect) a high number to have a greater chance of being divisible by a prime
number than a low one.

6. This question (to pose) and (to answer) in the Elements of Euclid.

7. It (to contain) much of number theory.

8. Euclid (to assert) that there (can, to be) no last prime number.

\ETask{XV}{Supply the comparative or superlative forms of the words in brackets.}

1. (Many) of these are odd numbers.

2. They become (dense) as we proceed to (high) numbers.

3. We expect a high number to have a (great) chance of being divisible by a prime number than a low one.

4. Is there a point in the list of primes after which there are no (many)?

5. Since Euler's identity no (long) holds, we must seek a new relation.

6. Let $P$ be the (large) prime (little) than $2^N$.

7. The next (large) class of numbers which we consider is the set of integers.

\ETask{XVI}{Put questions to the following sentences, either general or to the italicized words.}

1. \textit{Most of these} are odd numbers. (General)

2. \textit{Only} one prime is divisible by three.

3. The number 1 is not considered as a prime number. (General)

4. It gives no additional information. (General)

5. \textit{We} construct a \textit{number} from prime factors.

6. Our numbers can be built up \textit{in an additive manner}. (General)

7. \textit{They} become less dense \textit{as we proceed to higher numbers}. (General)

8. This seems reasonable. (General)

9. There are no more. (General)

10. There can be no last prime number. (General)

\ESect{Writing}

\ETask{XVII}{1. The text under consideration deals with prime numbers. By analogy with it write a definition of a composite
number and explain why all even numbers greater than two are referred to as composite numbers. 2. Write each of the following
composite numbers as a product of prime number factors: 60, 315, 176, 825.}

\ESect{Discussion}
\ETask{XVIII}{The text says that reducing all numbers to a sum of units is of little interest. Why? Give your reasons.}

\ESect{Supplementary Texts}

\ESect{Text 1}
From the uniqueness of the factorization of a number into prime factors, it follows that any term on the left will be formed once
and only once when this product is multiplied out. Thus, this is a formal identity. Now if $s \rightarrow 1$, we know that
$$\sum_{n=1}^\infty \frac{1}{n^s}$$ is divergent. But the identity must hold for all $s$ --- so there must be infinitely many factors
on the right, if the identity is to hold.
$$\sum_{n=1}^\infty \frac{1}{n^s}$$ depends only on $s$ and when $s$ is any complex number
$$\zeta(s)=\sum_{n=1}^\infty \frac{1}{n^s},$$
a meromorphic function with $s = 1$ as a pole, is known as the Riemann $\zeta$-function. It is of considerable importance in number
theory and is so named after Riemann (1826-1866), who discussed it in a famous paper published in 1860.

Euler's proof tacitly assumes that any number can be decomposed into prime factors and that this decomposition can be accomplished
in just one way. This must be proved, but this we shall defer until the next section. A more serious objection is that we have used
the notions of infinite series and products and the notion of a limit, none of which are elementary. Now let us seek a proof
modeled on Euler's --- one that is truly elementary.

To eliminate these notions, let us put $s = 1$ and replace the infinite series and product of Euler's identity by a finite
sum and product. Since Euler's identity no longer holds, we must seek a new relation:
$$ \sum_{n=1}^{\nu}\frac{1}{n}= \prod_{p=2}^P\frac{1}{1-\frac{1}{p}},$$
where $\nu$ and $P$ are numbers to be chosen.

First let us choose a number $N$, preferably a large one, and form the sum
$$1+\frac{1}{2}+\frac{1}{3}+\frac{1}{4}+\ldots+\frac{1}{2^N-1}+\frac{1}{2^N},$$
i.e., choose $\nu=2^N$. And then let $P$ be the largest prime less than $2^N$ which is surely composite, so that we consider
the sequence of primes $2,3,5, \ldots, P < 2^N$.

If we make $N$ larger, and do so continuously, then we expect that the number of primes less than $2^N$ will indeed increase,
for if the set of all prime numbers were finite, then we could not increase the number of primes less than $2^N$ by taking $N$
larger and larger. This gives us the motivation for a proof: can we show that some strictly increasing function of $N$ is always
less than some function of the primes less than $2^N$?

\ESect{Text 2}
It is true that the product of two natural numbers is a natural number. Thus the set of natural numbers is closed under addition
and multiplication. There is a natural number which is a multiplicative identity (namely 1), but no natural number is an additive
identity (unless we remove the bar sinister from 0). The natural numbers are not closed with respect to subtraction or division,
in the sense that the difference or quotient of two natural numbers may fail to be a natural number. For example, neither
$1 - 2$ nor $\frac{1}{2}$ is a natural number.

The next larger class of numbers which we consider is the set of integers.

DEFINITION A number $x$ is an integer if and only if $x = m - n$ for some natural numbers $m$ and $n$.

In other words, the integers are the differences of natural numbers. The integers satisfy all the axioms of addition (that
is, the axioms with `number' replaced by `integer'), but not all integers have integral inverses with respect to multiplication.
The next larger class of numbers which we consider is called the set of rational numbers.

DEFINITION A number $x$ is a rational number if and only if for some integer $a$ and some natural number $b$ it is true that
$x = \frac{a}{b}$.

Alternatively, the rational numbers may be described as the quotients, with non-zero denominators, of integers.
The rational numbers satisfy all of the axioms for numbers which have so far been listed. Only the continuity axiom for the
numbers fails to be satisfied by the set of rational numbers. (Again, we mean the axioms, with `number' replaced by `rational number'.)

At this stage we cannot assert that there are numbers which are not rational numbers (that is, are irrational numbers). But,
eventually, we shall attain that objective.

\EUnit
\ESect{Reading}
\ETask{I}{Pre-reading questions:}

1. What are twin primes?

2. Is it true that every even number can be written as the sum of two primes?

\ETask{II}{Read the text. Try to guess the meaning of the words you don't know. Then consult your dictionary to check their
meaning and pronunciation.}

\ESect{Text}
Dirichlet proved that each simple arithmetic progression whose first term and difference are coprime contains an infinite
number of primes. Now does the arithmetic progression of second order
$$2^2+1,4^2+1,6^2+1,8^2+1,10^2+1, \ldots, (2x)^2+1, \ldots,$$
contain an infinite number of primes? No one knows. The answer seems to be beyond our present strength.

Another famous unsolved problem is: Are there an infinite number of twin primes, i.e., primes that differ by 2, e.g., 5 and 7,
or 11 and 13? No one knows. Such twin primes are scarcer than prime numbers. The best result in this direction was obtained by
a Norwegian mathematician, Viggo Brun. Curiously the fact that he was isolated from other mathematicians seems to have favored
his work --- for they, no doubt, would have convinced him that it would be useless to employ the Sieve of Eratosthenes to prove
propositions on prime numbers. Using this outmoded device, he was able to show that the sum of the reciprocals of the twin
primes converges, i.e.,
$$\frac{1}{3}+\frac{1}{5}+\frac{1}{7}+\frac{1}{11}+\frac{1}{13}+\frac{1}{17}+\frac{1}{19}+\ldots$$
actually converges. This is surprising, for it can be shown that the sum of the reciprocals of all primes diverges. Thus Viggo
Brun's result is an advance. If the number of twin primes is finite, then the sum of their reciprocals certainly converges.
While if their number is infinite, it shows that we lose very many primes in extracting this convergent series from the divergent
sum of the reciprocals of all primes, so that the twin primes are indeed scarce.

In addition Brun considered numbers that were not prime but consisted of at most 9 prime factors and succeeded in showing that
among such numbers there are infinitely many twins, i.e., numbers differing by 2.

Since we know that the primes become less dense as we go to higher numbers and suspect that they appear again and again as twins,
we expect that the prime numbers are very irregularly distributed. Are there arbitrarily large gaps in the sequence of primes?
Yes. For example, consider $M = 1000!$: The numbers $M+2, M+3, M+4,..., M+1000$ are not prime, for $M+2$ is divisible by 2,
$M+3$ is divisible by 3,...,$M+1000$ is divisible by 1000. Thus we construct 999 consecutive numbers no one of which is prime.
The same result could be obtained from the product, $N$, of all the primes less than 1000. Adding $2, 3, 4, 5,..., 1000$ successively
to $N$, we would have again 999 consecutive non-prime numbers, e.g., $N+6$ is divisible both by 2 and 3.

In a letter to Euler, a Russian named Goldbach asked if he could prove that every even number can be written as the sum of
two primes. Mathematicians of the 18th century communicated their discoveries one to the other by letter, as there were
few journals, so that their collected works consist in large part of correspondence. As a result of this fact, this unproved
proposition is known as the Goldbach conjecture, though he is known for nothing else. However, the proposition is reasonable,
for $4=2+2$, $12=5+7$, $6=3+3$, $14=7+7=3+11$, $8=5+3$, $16=5+11=3+13$, $10=5+5=3+7$, $18=7+11=5+13$, and no one has ever found
an even number contradicting the Goldbach conjecture. However, it is unproved. A major difficulty in proving this is the nature
of prime numbers --- they are made for multiplication, while the proposition is of an additive nature.

If the Goldbach conjecture were true, then adding 3 to every number we should have: Every odd number can be expressed as the sum
of three primes. However, this weaker proposition does not imply the Goldbach conjecture --- even if it were valid, still some
even numbers might not be expressible as the sum of two primes. This has not been completely proved, but Vinogradov (1937),
using ideas developed by Hardy and Littlewood in the early 1920's, succeeded in proving that from a certain number, $M$, onward,
all odd numbers are the sum of three primes. Unfortunately his proof is an existence proof; it does not yield a method of
estimating $M$. Nevertheless, the importance of this result is not to be underestimated.

Here we have a statement about all odd numbers greater than $M$ - one we could never verify experimentally. A function of
mathematics is to prove these things which are beyond experimental verification, and in large part, the importance and interest
of mathematics lie in its `infinite tail', in those propositions which are not experimentally verifiable. For example,
$5 \cdot 3 = 3 \cdot 5$ is mathematically dull, for we can check it; while $ab = ba$ is fascinating, for it is a statement
about all numbers.

We have noticed that the primes seem to be distributed irregularly. However, Bertrand observed that between $a$ and $2a$
there always is a prime number. This is known as Bertrand's Postulate and has been proved by rather elementary means. Comparing
the size of the interval, $a < p < 2a$, with the first number $a$, we have the relative length of the interval
$\frac{2a - a}{a} = 1$. Thus the primes are regularly distributed in the sense that there is at least one in each interval of
relative length 1.

Now it would seem reasonable to ask if a smaller interval can be assigned in which we can always find at least one prime.
For example, is the inequality $a^2<p<(a+1)^2$ valid for all a? This yields $1 < 2.3 < 4.4 < 5.7 < 9$, for $a$ equal to 1 and 2.
Here the relative length of the interval is
$$\frac{(a+1)^2-a^2}{a^2}=\frac{2a+1}{a^2}=\frac{2}{a}+\frac{1}{a^2}\sim \frac{2}{a}$$
so that for large numbers the inequality, if valid, would assign a relative interval in which we could find a prime much
smaller than that of Bertrand's Postulate. Unfortunately this is unproved.

\ETask{III}{Answer the questions:}

1. What did Dirichlet prove concerning simple arithmetic progressions?

2. What result was obtained by V. Brun in this field?

3. Why do we expect that the prime numbers are very irregularly distributed?

4. What did Goldbach ask in his letter to Euler? What is he known for?

5. What was proved by Vinogradov?

6. What is an existence proof?

\ETask{IV}{What do the italicized words refer to? Check against the text.}

1. \textit{Such twin} pairs are scarcer than prime numbers.

2. The best result in \textit{this direction} was obtained by a Norwegian mathematician V. Brun.

3. \textit{This} is surprising.

4. If \textit{their} number is infinite, \textit{it} shows that we lose very many primes in extracting \textit{this} convergent series
from the divergent sum of the reciprocals of all primes.

5. \textit{The same result} could be obtained from the product, $N$, of all the primes less than 1000.

6. As a result of \textit{this fact, this} unproved proposition is known as the Goldbach conjecture.

7. \textit{This} has not been completely proved.

8. \textit{It} does not yield a method of estimating $M$.

9. Here we have a statement about all odd numbers greater than $M$ - one we could never verify experimentally.

10. \textit{This} is known as Bertrand's Postulate.

\ESect{Vocabulary}

\ETask{V}{Give the Russian equivalents of the following word combinations:}

to be beyond; no doubt; outmoded device; in addition; at most; the same result; both ... and; one to the other; in large part;
as a result; he is known for nothing else; in the early 1920's; from a certain number onward; in the sense that...; at least;
it would seem reasonable; if valid.

\ETask{VI}{Find in the text the adverbs that end in \textbf{-ly} and adverbs that don't end in \textbf{-ly}. Think of some
examples of such adverbs.}

\ETask{VII}{Give antonyms of the following words:}
finite, useless, to converge, regularly, reasonable, successful, known, divisible, addition, multiplication, to discover, to appear.

\ETask{VIII}{The following suffixes are used to form different parts of speech.}
\ETypeWr{
Nouns~~~~~~~~~-ment~~~~-ness~~~~-sion~~~~-tion~~~~-ty~~~~-al

Adjectives~~~~-ful~~~~~-ic~~~~~~-able~~~~-ous~~~~~-y~~~~~-ive~~~~~-al

Verbs~~~~~~~~~-ize~~~~~-ise}

\textit{The words below have all appeared in the text. Use your dictionary to find the other parts of speech, their translation and
pronunciation. The above suffixes are used (but not always):}

\sloppy
difference, to converge, to surprise, addition, to succeed, to construct, product, divisible, to prove, to communicate,
to discover, correspondence, reasonable, to contradict, difficult, multiplication, to express, existence, statement, variable,
important, equal.

\fussy
\ETask{IX}{Use either among or between in these sentences: Note that we commonly use between to show a division between two
people, things or times, e. g. `Divide this between you both.' We use among to refer to a mass of people, things, etc., e.g.
`Were you among the people present?'}

1. Brun succeeded in showing that --- such numbers there are infinitely many twins.

2. Bertrand observed that --- $a$ and $2a$ there always is a prime number.

3. The lines were drawn --- two corresponding dots to indicate the cancellation.

4. --- all the ordinary fractions there are many that are equal.

5. On closer examination of these two sequences we see that there is indeed a relation --- them.

\ESect{Grammar}

\ETask{X}{Rewrite these questions with the words provided.}

1. Does the arithmetic progression of second order contain an infinite number of primes? \textbf{We want to know...}

2. Are there an infinite number of twin primes? \textbf{It is interesting to know...}

3. Are there arbitrarily large gaps in the sequence of primes? \textbf{Now we should answer the question...}

4. Can he prove that every even number can be written as the sum of two primes? \textbf{In a letter to Euler, a Russian named
Goldbach asked...}

5. Can a smaller interval be assigned in which we can always find at least one prime? \textbf{It would seem reasonable to ask...}

6. Is the inequality $a^2<p<(a+1)^2$ valid for all $a$? \textbf{We would like to know...}

\ETask{XI}{Supply comparative or superlative forms. Then check against the text.}

1. Such twin primes are (scarce) than prime numbers.

2. The (good) result in this direction was obtained by V. Brun.

3. We know that the primes become (dense) as we go to (high) numbers.

4. The same result could be obtained from the product of all primes (little) than 1000.

5. This (weak) proposition does not imply the Goldbach conjecture.

6. Here we have a statement about all odd numbers (great) than $M$.

7. Now it would seem reasonable to ask if a (small) interval can be assigned.

\ETask{XII}{Rewrite the sentences using Subjunctive Mood.}

1. They convinced him that it was useless to employ the Sieve of Eratosthenes to prove propositions on prime numbers.

2. The same result can be obtained from the product of all the primes less than 1000.

3. If we add $2,3,4,5,...,1000$ successively to $N$, we will have again 999 consecutive non-prime numbers.

4. If the Goldbach conjecture is true, then adding 3 to every even number we have: Every odd number can be expressed as the
sum of three primes.

5. Even if it is valid, still some even numbers may not be expressible as the sum of two primes.

6. It seems reasonable to ask if a smaller interval can be assigned in which we can always find at least one prime.

7. If for large numbers the inequality is valid it will assign a relative interval in which we can find a prime much smaller
than that of Bertrand's Postulate.

\ETask{XIII}{Rewrite these sentences with suitable forms of seem.}

1. The answer is beyond our present strength.

2. No one knows it.

3. The fact that he was isolated from other mathematicians favored his work.

4. This was not completely proved.

5. These things are beyond experimental verification.

6. We cannot check it.

7. The primes are distributed irregularly.

8. The essential idea was developed in Alexandria in the Third Century B.C.

\ETask{XIV}{Supply Present Perfect of the verbs given in brackets.}

1. No one (ever to find) an even number contradicting the Goldbach conjecture.

2. This (not to prove) completely by anyone yet.

3. We (to notice) that the primes seem to be distributed irregularly.

4. This is known as Bertrand's Postulate and (to prove) by rather elementary means.

5. A more serious objection is that we (to use) the notions of infinite series and products and the notion of a limit,
none of which are elementary.

6. So far we (to speak) only of finite sets.

7. We (already to set forth) a first axiom of the theory of probability, namely $m(u) = p(u) = 1$.

8. Thus we (to transform) the problem into the much simpler one.

\ESect{Writing}

\ETask{XV}{There are many different ways of expressing cause and effect, the text under consideration gives some of them.
1. Look through it again and pick out all the linking words used for these purposes. 2. Make a list of them and add some more
examples. 3. Choose one of the theorems you know and prove it in writing, using linkers of the above type.}

\ESect{Supplementary Texts}

\ESect{Text 1}
Euler's proof of the infinity of primes provides the motivation for Dirichlet's proof of the theorem that in every arithmetic
progression there is an infinity of primes. This last theorem is much too difficult to consider here.

Up to now everything has been proved before our eyes. A list of primes less than 10000000 has been computed by D. N. Lehmer (1956).
How are such lists prepared? The essential idea seems to have been developed in Alexandria in the Third Century B.C., for it
is attributed to Eratosthenes (ca. 250 B.C.) and does not appear in Euclid. Let us write down the list of integers:
$$1,2,3,4,5,6,7,8,9,10,11,12,13,14,15,16,17,...$$
First we cross out 2 and all its multiples. The first number, always excluding 1, which is not hit is a prime, for it has no
lower divisor. This is 3. So we strike out 3 and its multiples. The next uncancelled number, 5, is prime, having no smaller
factor. Eliminating the multiples of 5, we find that 7 is a prime, etc. This purely mechanical method is known as the Sieve
of Eratosthenes. It depends not upon the numbers themselves, but rather on their position in the sequence of the integers.

We may think of the row of equally spaced dots in Plate 1 [Omitted in this Edition] as being continued indefinitely to the
right. These we can number, calling the first one 0, the next, to the right, 1, the next 2, etc. The method sketched above
is applicable to these dots.

Let us cross out every other dot, starting with 0 --- a mechanical process. (To avoid confusion this has been done in a second
row and lines have been drawn between corresponding dots to indicate the cancellation.) Next we strike out the dots whose
distance from the origin is a multiple of the distance of the first dot, excluding always, 0 and 1, not previously cancelled.
This yields line 3, and repeating the process gives line 5. Stepwise we eliminate all dots that are at multiples of the distance
or some previous dot from the origin and find at the end of each step that the first uncancelled dot corresponds to a prime number.
Thus it is clear that the Sieve of Eratosthenes depends upon the position of the integers in sequence rather than on the
properties of the numbers themselves. This method was the basis on which Lehmer's list of primes was computed.

\EUnit

\ESect{Reading}

\ETask{I}{Pre-reading questions:}
1. What rules of operations performed on numbers do you know?

2. What are fractions?

\ETask{II}{Read the text. Try to guess the meaning of the words which you probably don't know. Then consult your dictionary if
necessary.}

\ESect{Text}

In order to introduce common fractions into our number system, we take for granted that we know all the integers, positive and
negative, the operations, addition and multiplication, that we can perform on them, and the rules by which we may combine
these operations. As we become more sophisticated in mathematics, we find a need to enlarge our number system --- from the
integers to the rational numbers, to the irrational numbers, and to the complex numbers. But this is a heuristic approach;
whereas a more formal approach is desirable in order to show that the operations performed with fractions fit the same
pattern as the operations with integers.

What are these rules of operation that we take for granted? They seem to have been first enumerated by the Irish mathematician
William Hamilton.

We have a commutative law: for addition: $a+b=b+a$, for multiplication: $ab = ba$, an associative law: for addition:
$a+(b+c)=(a+b)+c$, for multiplication: $a(bc)=(ab)c$; (the associative law for addition is merely codification of the
fact that we can only add two numbers together, e.g., the method we use in adding long columns of numbers);
and a distributive law: $a(b+c)=ab+ac$, which connects addition and multiplication. This last law is clearly of a different
nature from the first two in which we pass from the law for addition to the law for multiplication by replacing the plus sign
by a dot. If we apply this replacement to the distributive law, we have nonsense: $a+(bc) \neq (a+b)(a+c)$. The usual rule
of multiplication is a clear application of the distributive law.

Finally we have a law of cancellation: for addition: if $a+b=a+c$, then $b=c$, for multiplication: if $ab = ac$, and if
$a \neq 0$, then $b=c$.
In addition to these laws we have two invariants, one for each operation: for addition: zero, i.e., $a+0=a$, for multiplication:
one, i.e., $a\cdot 1=a$.

The foregoing are postulates and from them we can derive the other familiar rules of arithmetic. For example we have as a
theorem: $a\cdot 0=0$. Proof. $ab = a(b + 0)$ (the invariant for addition), but $a(b + 0) = ab + a\cdot 0$ or $ab=ab+a\cdot 0$
(the distributive law) whence $a \cdot 0=0$ (the law of cancellation).

Knowing the integers and having these rules of operation in mind, we may now introduce fractions as an ordered pair of
integers $(a|b)$, where $b \neq 0$. In this ordered pair $a$ plays the conventional role of the numerator, while $b$ plays the
role of the denominator. We introduce purposely a new symbol $(a|b)$ in order to be sure that we do not inadvertently make use
of the rules of fractions which we are now going to introduce.

Definitions:

Multiplication of Fractions: $(a|b)(c|d)=(ac|bd)$, Addition of Fractions: $(a|b)+(c|d)=(ad+bc|bd)$. In addition to these two
operations we need an equivalence relation --- when are two pairs of numbers equal --- for among all the ordinary fractions
there are many that are equal, e.g.

Definition:

Equivalence of Fractions: $(a|b)=(c|d)$, if and only if $ad=bc$.

This definition of equivalence is based on the known operations of the integer realm and reduces the decision of equivalence
to a question therein. An equivalence relation must have three properties:

1. Reflexivity: every pair must be equal to itself, $(a|b)=(a|b)$ implies $ab=ba$ which is true in the integer realm.

2. Symmetry: if a pair is equal to a second, the second pair is equal to the first. $(a|b)=(c|d)$ implies $ad=bc$ which
can be written as $cb=da$. But the latter is merely the integer formulation of $(c|d)=(a|b)$.

3. Transitivity: if a pair is equal to a second, and the second is equal to a third, then the first pair is equal to the
third. If $(a|b)=(c|d)$ and $(c|d)=(e|f)$, then $(a|b)=(e|f)$. These imply
$$ad=bc, \; cf=de,$$
$$adf=bcf, \; bcf=bde,$$
whence
$$adf=bde,$$
and by cancellation (since $d \neq 0$) $af=be$, so that $(a|b)=(e|f)$. And we have shown that our proposed equivalence relation
indeed does have the properties required of it.

In so defining equivalence between fractions, we have divided them into classes such that $\frac{3}{4}$ and $\frac{12}{16}$
belong to the same class. Here the essential difficulty with fractions is exhibited, namely that we have no unique notation
for the classes.

Suppose that we have two fractions of the same class, i.e., $(a|b)=(a'|b')$; then it would seem reasonable to assert that
$(a|b)(c|d)=(a'|b')(c|d)$, or multiplication by a member of a class is equivalent to multiplication by any other member of the
same class. $(a|b)=(a'|b')$ implies that $ab'=a'b$, and we must show that $(a|b)(c|d)=(a'|b')(c|d)$, or
$(ac|bd)=(a'c|b'd)$, which implies that $acb'd=a'cbd$. But this last relation holds in the realm of integers.

Likewise, we could show that the addition of a member of a class is equivalent to the addition of any other member of the
same class. And similarly, the commutative, associative, and distributive laws also hold for fractions as we have defined them.
These are left as exercises.

Like 0 among the integers there is a class of fractions invariant with respect to addition, namely $(0|a)=(0|b)=(0|1)$,
provided that $a$ and $b$ are not zero. To show this we note that $(a|b)+(0|d)=(ad+b\cdot 0|bd)=(ad|bd)=(a|b)$, for the last
equality implies $adb = bda$. Similarly we find a class of fractions invariant with respect to multiplication, namely
$(a|a)=(b|b)=(1|1)$, provided $a$ and $b$ are not zero, for $(a|a)(b|c)=(ab|ac)=(b|c)$.

Now we can show that there is a fraction $(x|y)$ such that $(a|b)(x|y)=(c|d)$, provided that neither $a$, $b$ or $d$ are zero.
$(a|x)(x|y)=(ax|by)=(c|d)$ can be fulfilled by $ax=kc$ and $by=kd$. If we put $k=ab$, so that $ax=abc$ and $by=abd$, then
by the law of cancellation we have $x=be$ and $y=ad$. Thus the fractions always permit division, provided that of all the terms
only the first term of the dividend may be zero.

\ETask{III}{Give your ideas.}

1. What is a number system?

2. Enumerate the main laws of operations on integers.

3. Give the definition of equivalence.

4. What are the three properties of an equivalence relation?

5. Comment on the statement: `The fractions always permit division.'

\ETask{IV}{What do the words in italics refer to? Check against the text.}

1. ...we can perform on \textit{them}...

2. \textit{This last law} is clearly of a different nature from the \textit{first two}.

3. If we apply \textit{this replacement} to the distributive law, we have nonsense.

4. Knowing the integers and having \textit{these rules} of operation in mind, we may now introduce fractions as
an ordered pair of integers.

5. In \textit{so} defining equivalence between fractions, we have divided them into classes.

\ESect{Vocabulary}

\ETask{V}{Give the Russian equivalents of the following expressions:}
in order to; to take for granted; to become more sophisticated in; the same pattern; to be of a different nature from; the
foregoing are postulates; to have something in mind; to make use of; in addition; if and only if; a question therein; it
would seem reasonable; these laws hold for...; with respect to; provided that.

\ETask{VI}{Supply with between or among:}
1. --- all the ordinary fractions there are many that are equal.

2. In so defining equivalence --- fractions, we have divided them into classes.

3. Like 0 --- the integers there is a class of fractions invariant with respect to addition.

4. --- these fractions there is a subset of classes.

5. Now we can show the difference --- these two numbers.

\ETask{VII}{Supply prepositions. Then check against the text.}

1. We know all the integers and the operations that we can perform --- them, and the rules --- which we may combine them.

2. We find a need to enlarge our number system --- the integers --- the rational numbers.

3. The associative law --- addition is merely codification --- the fact that we can only add two numbers, e.g., the method we
use --- adding long columns --- numbers.

4. This last law is clearly --- a different nature --- the first two --- which we pass --- the law --- addition --- to law
--- multiplication --- replacing the plus sign --- a dot.

5. This definition --- equivalence is based --- the known operations --- the integer realm and reduces the decision ---
equivalence --- a question therein.

6. Every pair must be equal --- itself.

7. We have divided them --- classes.

8. Multiplication --- a member --- a class is equivalent --- multiplication --- any other member --- the same class.

\ETask{VIII}{Translate into Russian the following pairs of nouns:}
number system; equivalence relation; integer realm; integer formulation; continuity axiom; right-hand corner; vector
geometry; vector addition; zero vector; circle tangent.

\ETask{IX}{Pay attention to the use of \textbf{another, other/others, the other}.}
\textit{The meaning of another: one more in addition to the one(s) already mentioned. The meaning of other/others (without the):
several more in addition to the one(s) already mentioned.}

The students in the class come from many countries. One of them is from Great Britain. Another student is from France.
Another is from Germany. Other students are from Brazil. Others are from Japan.

\textit{The meaning of the other(s): all that remains from a given number, the rest of a specific group.}

1. I have three books. Two are mine. The other book is yours. The other is yours. 2. I have three books. One is mine.
The other books are yours. The others are yours.

\textit{Another is used with expressions of time, money and distance, even if these expressions contain plural nouns.}

1. I will be there for another three years. 2. I need another five dollars. 3. We drove another ten miles.

\textit{Every other gives the idea of `alternate'.}

1. Please write on every other line. 2. I see her every day other week.

\textit{The other day can mean a few days ago.}

I met my friend the other day (= a few days ago).

\medskip
\textit{Supply the following sentences with another, other, others, the other.}
1. The foregoing are postulates and from them we can derive all --- familiar rules of arithmetic.

2. Multiplication by a member of a class is equivalent to multiplication by any --- member of the same class.

3. Now let us consider --- Euclid's postulate.

4. $a$ and $b$ have no common divisor --- than $\pm 1$.

5. Thinking over --- examples, we see that this is impossible.

6. If there are two systems which satisfy all the axioms which we list, then one system is simply a copy of ---.

7. Numbers, addition and all --- mathematical concepts can be defined in terms of this single notion.

8. Now we shall prove --- theorems about natural numbers.

9. If $x$ is a number, then $x-1$ is --- number which is smaller.

10. Some problems are quite difficult, --- are rather easy.

\ESect{Grammar}

\ETask{X}{Explain the use of the verb to do in these sentences.}

1. We have shown that our proposed equivalence relation indeed does have the properties required of it.

2. Our language does have remnants of other systems of notation.

3. We do now know that this is true.

\ETask{XI}{What is the meaning of provided in these sentences?}

1. ...there is a class of fractions invariant with respect to addition namely $(0|a)=(0|b)=(0|1)$, provided that $a$ and $b$
are not zero.

2. ...there is a fraction $(x|y)$ such that $(a|b)(x|y)=(c|d)$, provided that neither $a$, $b$, or $d$ are zero.

3. The fractions always permit division, provided that of all the terms only the first term of the dividend may be zero.

\ETask{XII}{Combine modals with the verbs in brackets.}

1. ...we know all the integers, positive and negative, the operations, addition and multiplication, that we --- (to perform)
on them, and the rules by which we --- (to combine) these operations.

2. The associative law for addition is merely codification of the fact that we --- (to add) two numbers together.

3. The foregoing are postulates and for them we --- (to derive) the other familiar rules of arithmetic.

4. Knowing the integers and having these rules of operation in mind, we --- (to introduce) fractions as an ordered pair of integers.

5. An equivalence relation --- (to have) three properties.

6. Every pair --- (to be) equal to itself.

7. $(a|b)=(c|d)$ implies $ad=bc$ which --- (to be) written as $cb = da$.

8. Now we --- (to show) that...

9. The fractions always permit division, provided that of all the terms only the first term of the dividend --- (to be) zero.

\ETask{XIII}{Supply the right finite and non-finite forms of the verbs in brackets.}

1. They (to seem, to enumerate) first by the Irish mathematician W. Hamilton.

2. (to know) the integers and (to have) these rules of operation in mind, we may now introduce fractions as an (to order)
pair of integers.

3. This definition of equivalence (to base) on the (to know) operations of the integer realm and (to reduce) the decision
of equivalence to a question therein.

4. We (to show) that our (to propose) equivalence relation indeed does (to have) the properties (to require) of it.

5. In so (to define) equivalence between fractions, we (to divide) them into classes.

6. Here the essential difficulty with fractions (to exhibit).

7. The commutative, associative and distributive laws also (to hold) for fractions as we (to define) them. These
(to leave) as exercises.

8. It can (to fulfill) by $ax = kc$.

9. We find after (to complete) the operation, that the correspondence (to preserve).

\ETask{XIV}{Use Subjunctive Mood in the following sentences.}

1. It (to seem) reasonable to assert that $(a|b)(c|d)=(a'|b')(c|d)$.

2. If there (to be) a last prime $P$, then the right-hand side (can, to have) only a finite number of factors and the value
(to be) definitely finite. But this is impossible.

3. We (may, to go) deeper into the theory of probability, but we content ourselves with the exposition of these four postulates.

4. What (to happen) if we (to cut) these circles by a horizontal line?

5. If we (can, to show) that this last infinite sequence of points lay in the finite part of the planes, then clearly
the Bolzano-Weierstrass theorem (to be) applicable and we (to be sure) that they had at least one limit point.

6. In more detail, we (may, to phrase) the reasoning as follows.

7. We shall use letters `$x$', `$y$', etc. as if they (to be) names.

8. If we shift the whole plane 3 units to the right and then shift it 4 units up, we have it in a new position which we
(can, to put) it in with one shift. What single shift (to do) it?

\ESect{Writing}
\ETask{XV}{There are many reasons to show how ideas are related in scientific discourse. The following linking words are used
to express the comparison of like items: similarly, likewise, in the same way, moreover, also, furthermore, besides. Linking
words that show contrast include although, but, however, in contrast, on the other hand, even though, nevertheless, on the
contrary, yet, in spite of, despite. 1. Look through the text again to find the sentences, if any, that contain linking words
of the above type. 2. Write a paragraph of comparison or contrast on one of the points of the text.}

\ESect{Supplementary Texts}

\ESect{Text 1}

In mathematics, we frequently classify things and then single out from each class an element, called the reduced element, to
stand for the whole class. This process proves so useful that when it is impossible to exhibit a method for finding a reduced
element, to avoid the difficulty we postulate the existence of such an element as in the axiom of choice in the theory of sets.

Our classes consist of equivalent fractions ($(a|b)=(c|d)$, if and only if $ad = bc$) and we choose as the reduced element
that pair of the class which has the smallest positive second element (denominator). Of all the fractions of a class, only
one can be the reduced element. (Suppose there were two, then $(a|b)=(c|b)$, as they must have the same second element, which
implies $ab=cb$, so that $a=c$.) We do not need to make an involved proof of the existence of the reduced pair. Suppose we have
our class of fractions in a bag. Pulling one from the bag we note its denominator. If it is negative, changing the signs of
both members of the pair will give us a fraction with positive denominator which also belongs to our class. Now we have only
to make a finite number of comparisons to determine if a smaller positive number can serve as a denominator of a fraction of
our class. Since there can be only a finite number of smaller denominators, we can certainly determine the smallest by inspection.
When only a finite number of objects are to be compared, inspection is indeed a legitimate and feasible method of mathematical
proof. Thus we can always find one fraction, $(a|b)$, the reduced element of the class, such that $b$ is the smallest positive
denominator of the equivalent fractions.

In what sense does this element represent the class? We shall show that if ($(a|b)$ is the reduced fraction of a class and
$(c|d)$ belongs to the same class, then $(c|d) = (ka|kb)$.

Consider $(a|b)=(c|d)$, where $b > 0$, $d > 0$ and $d > b$, where $b$ is the smallest positive denominator of all equivalent
fractions, $(c|d)$ is a fraction different from $(a|b)$, only if $d > b$; for if $d=b$, then $a=c$, as $ad=be=dc$. So let us
suppose definitely that $d > b$. Performing division we have $d = qb + r$, where $0 < r \leqslant b$. (Note the curious
specification of the remainder $r$ --- zero is excluded, while $b$ is included as a possible value.) Set $c=qa+s$, where $q$
is defined by the previous division. We know that $(a|b)=(c|d)=(qa + s|qb + r)$, implying that $a(qb+r)=b(qa + s)$, or $ar=bs$:
whence $(a|b)=(s|r)$. But as $b$ was the smallest positive denominator contained in the class, it is impossible that $r$ be
different from $b$, i.e., $r=b$. This implies that $d=qb+b=(q+1)b$. Now we must show that $s=a$; but this follows from
$(a|b)=(s|r)=(s|b)$, so that $c=(q+l)a$. And thus $(c|d)=\bigl((q+l)a|(q+1)b\bigl)=(ka|kb)$.

\ESect{Text 2}
We have defined the reduced fraction as the member of the class that has the smallest positive denominator, and we have shown
that any other member of the class is merely an amplification of the reduced fraction in which the numerator and denominator
are both multiplied by the same factor. This leads naturally to another definition of the reduced fraction, one that is more
convenient: $(a|b)$ is reduced if and only if $a$ and $b$ have no common divisor other than $\pm 1$. We must show this to be
equivalent to the first definition. First suppose that $a$ and $b$ have a common divisor, i.e., $a = \delta \alpha$,
$b = \delta \beta$, where $\delta \neq \pm 1$. Then $(a|b)=(\delta \alpha|\delta \beta)=(\alpha|\beta)$, $b > \beta$, so that
$(a|b)$ is not a reduced fraction in the old sense. Conversely, if $c$ and $d$ have no common divisor, then $(c|d)$, $d > 0$,
must be a reduced fraction. Indeed, if it were not reduced, then there would be another specimen in the same class with a
smaller positive denominator, say $(a|b)$. But we just showed that $(c|d)=(ka|kb)$ and by hypothesis $k=\pm 1$, so that
$(c|d)$ is in fact the reduced fraction.

The notion that $a$ and $b$ have no common divisor is perhaps more familiarly expressed by saying that the fraction is in
lowest terms. For example, consider
$$\frac{31759}{112753}=\frac{\ldots}{\ldots}.$$
which may or may not be in lowest terms. Why should it be impossible that two fractions, both in lowest terms, should be
equal? Thinking over the example, what it implies in the realm of integers, we see that this impossibility is equivalent to
the theorem on the uniqueness of prime factorization of a number. Thus our next goal will be to establish this in full.
To this end we shall use the previous result to prove Euclid's lemma.

Suppose $a$ and $b$ are coprime, i.e., $(a, b) = 1$ and suppose $a$ divides the product $bc$, then $a$ divides $c$. Let us
write out what $a$ divides be means. That is $bc = ad$, or $ad = bc$, which can be written in fractional notation as
$(a|b)=(c|d)$, where $(a|b)$ is a reduced fraction. Hence $c=ak$, or $c$ is divisible by $a$.

By a specialization of Euclid's lemma we have: If $p$ is a prime, and $p$ divides $bc$, then $p$ divides either $b$ or $c$.
1. Suppose $p$ divides $b$, then the theorem is true. 2. Suppose $p$ does not divide $b$, then $(p, b) = 1$, for 1 is a
divisor of both and $p$ has only two divisors, 1 and $p$, so that $(p, b)$, the greatest common divisor, is in fact equal
to 1. Hence by Euclid's lemma $p$ divides $c$.

But this is tantamount to the theorem on the uniqueness of prime factorization of integers, being the essential theorem
used in the proof of that important proposition.

\EUnit
\ESect{Reading}
\ETask{I}{Pre-reading questions:}
1. What are two-dimensional vectors?

2. What do the terms real and complex numbers mean?

\ETask{II}{Read the text. Make a list of unknown words. Consult your dictionary for their meaning and pronunciation.}

\ESect{Text}

We turn to the problem of defining a `good' multiplication for vectors. We have defined the scalar product of a scalar and a
vector, which yields a vector, and we have defined the inner product of two vectors, which yields a scalar, but we have not
defined a multiplication of two vectors which yields a vector. In this section we define such a multiplication for 2-dimensional
vectors, and we shall later see that this multiplication has interesting applications in the problem of solving equations.

There is, unfortunately, a plethora of notation in mathematics. Two-dimensional vectors are usually called complex numbers in
circumstances where one is likely to multiply them. Formally:

DEFINITION. A complex number is a 2-dimensional vector; that is, a complex number is an ordered pair of numbers. A real number
is a number. That is, what we have simply called a number up to now, will be called a real number to distinguish it from a complex
number.

Thus $(1, 2)$ is a complex number and 2 is a real number. The terminology is, of course, historical; in problems concerned with
scalar multiplication a number is called a scalar, and in problems concerned with what we shall call complex multiplication a
number is called a real number. The adjective `real' is not supposed to imply that complex numbers are hallucinatory. We now
compound the semantic confusion.

DEFINITION. The real part of a complex number is its first coordinate; the imaginary part of a complex number is its second
coordinate.

Thus 2 is the real part of $(2,3)$, and 3 is the imaginary part of $(2,3)$. (Don't ask me what is imaginary about 3.) In line
with this interesting terminology, the $x$-axis is frequently referred to as the real axis, and the $y$-axis as the imaginary axis.

The standard way of writing complex numbers is as the sum of scalar multiples of two particular complex numbers. We define:

DEFINITION. $I=(1,0)$ and $i=(0,1)$.

In other words, $I$ is the unit vector in the $x$-direction and $i$ is the unit vector in the $y$-direction. If $(a,b)$ is
an arbitrary complex number, it is evident that $(a, b) = a(1,0)+b(0,1)=aI+bi$, where the addition is the usual vector addition.
Conversely, if $c$ and $d$ are real numbers, then $cI+di=(c,d)$. Thus the real part of $cI+di$ is $c$ and the imaginary part is
$d$, for example, the real part of $2I+7i$ is 2 and the imaginary part is 7. On the other hand, the real and imaginary parts
determine the number; if the real part of $A$ is 4 and the imaginary part is 3, then $A=4I+3i=(4,3)$.

There is one other notational matter which must be explained. It is customary, in discussing complex numbers, to suppress the
symbol $I$ entirely and, for example, to write $(2,3)$ as $2+3i$. We make the convention:

CONVENTION. If $a$ and $b$ are real numbers, then $a+bi=aI+bi=(a,b)$.

We should not gloss over the fact that this convention may cause confusion. It is usual to write $(4,0)$ as $4I+0i$, or $4+0i$,
or simply as 4. But this is inconsistent; if we write `4', are we to mean the real number 4 or the complex number $(4,0)$? We
can only say, rather lamely, that the context should make clear which usage is meant; if we say `the complex number 4' we certainly
mean $(4,0)$, and the real number 4 is just 4.

All of these notational devices may tend to obscure some of our hard-won knowledge. If two complex numbers are equal, then their real
parts (first coordinates) are equal and their imaginary parts (second coordinates) are equal, and, conversely, if the real and
imaginary parts of two complex numbers are equal, then the numbers are equal. Thus, if $x$ and $y$ are real numbers such that
$2x+yi=4-3i$, then we infer that $2x=4$ and $y=-3$.

Finally, we come to the definition of complex multiplication. First, we want the product of the complex numbers $a + 0i$ and
$c+di$ to be $ac+adi$. In other words, multiplication by $(a,0)$ is to be scalar multiplication by $a$. Second, we want all of
the algebraic axioms for real numbers to be satisfied by complex numbers. Stated more precisely, if `number' is replaced by
`complex number' in each of the axioms A1---A5, M1 --- M5, D and AM we want the resulting statements to be theorems.

\ETask{III}{Comment on these statements.}

1. There is, unfortunately, a plethora of notations in mathematics.

2. The context should make clear which usage is meant.

\ESect{Vocabulary}

\ETask{IV}{Give the Russian equivalents of the following expressions:}
to be likely to; that is; up to now; problems concerned with...; in line with...; is referred to as...; in other words;
on the other hand; it is customary; it is usual to...; this is inconsistent; rather lamely; to make clear; hard-won knowledge;
and conversely; stated more precisely.

\ETask{V}{Join these notes to make sentences. Then check against the text.}

1. We turn --- the problem --- defining a `good' multiplication --- vectors.

2. --- this section --- define such --- multiplication --- 2-dimensional vectors.

3. The standard way --- writing complex numbers is --- the sum --- scalar multiples --- two particular complex numbers.

4. All --- these notational devices may tend --- obscure some --- our hard-won knowledge.

5. Finally we come --- the definition --- complex multiplication.

6. --- other words, multiplication --- $(a,0)$ is --- be scalar multiplication --- $a$.

7. We want --- of --- algebraic axioms --- real numbers --- be satisfied --- complex numbers.

\ETask{VI}{Suggest meanings for \textbf{one} in these sentences.}
1. Two-dimensional vectors are usually called complex numbers in circumstances where one is likely to multiply them.

2. There is one other notational matter which must be explained.

\ETask{VII}{Supply the following sentences with what or that.}

1. We shall later see --- this multiplication has interesting applications in the problem of solving equations.

2. --- we have simply called a number up to now, will be called a real number to distinguish it from a complex number.

3. In problems concerned with --- we shall call complex multiplication a number is called a real number.

4. The adjective `real' is not supposed to imply --- complex numbers are hallucinatory.

5. Don't ask me --- is imaginary about 3.

6. We can only say --- the context should make clear which usage is meant.

7. Then we infer --- $2x=4$ and $y=-3$.

8. If $x$ and $y$ are real numbers such --- $2x+yi=4-3i$.

\ESect{Grammar}

\ETask{VIII}{Supply the right forms and tenses of the verbs in brackets, then refer to the text. In each case give your reason
for the form you have chosen.}

1. We (define) the scalar product of a scalar and a vector, which (yield) a vector, and we (define) the inner product of two
vectors, which (yield) a scalar, but we (not, define) a multiplication of two vectors which (yield) a vector.

2. We (see) later that this multiplication (have) interesting applications in the problem of (solve) equations.

3. Two-dimensional vectors usually (call) complex numbers.

4. What we (call) simply a number up to now, (call) a real number (distinguish) it from a complex number.

5. In problems (concern) with scalar multiplication a number (call) a scalar, and in problems (concern) with what we (call)
complex multiplication a number (call) a real number.

6. The $x$-axis frequently (refer) to as the real axis.

7. `Number' (replace) by `complex number' in each of the axioms.

\ETask{IX}{Combine modals and their equivalents must, may, should, can, to be to with the verbs in brackets. Check against the text.}

1. There is one other notational matter which --- (to explain).

2. We --- (not to gloss) over the fact that this convention --- (to cause) confusion.

3. If we write `4', --- we (to mean) the real number 4 or the complex number $(4,0)$?

4. We --- only (to say) that the context --- (to make) clear which usage is meant.

5. All of these notational devices --- (to tend) to obscure some of our hard-won knowledge.

6. Multiplication by $(a,0)$ --- (to be) scalar multiplication by $a$.

\ETask{X}{In which sentence can you insert the words \textbf{in order} before the italicized Infinitive to express the purpose?}

1. It is customary, in discussing complex numbers, to \textit{suppress} the symbol `$I$' entirely and, for example, to \textit{write}
$(2,3)$ as $2+3i$.

2. What we have simply called a number up to now, will be called a real number \textit{to distinguish} it from a complex number.

3. All of these notational devices may tend to obscure some of our hard-won knowledge.

4. To \textit{satisfy} curiosity we state here Fermat's Last Theorem.

5. To introduce common fractions into our number system we take for granted that we know all the integers, the operations
that we can perform on them and the rules to combine these operations.

6. We introduce purposely a new symbol $(a|b)$ to be sure that we do not inadvertently make use of the rules of fractions which
we are now going to introduce.

7. It would seem reasonable to assert that multiplication by a member of a class is equivalent to multiplication by any
other member of the same class.

8. A more formal approach is desirable to show that the operations performed with fractions fit the same pattern as the
operations with integers.

\ETask{XI}{Rewrite these sentences beginning with the words we want using Complex Object. Then check against the text.}

1. The product of the complex numbers $a + 0i$ and $c+di$ is $ac + adi$.

2. All of the algebraic axioms for real numbers are satisfied by complex numbers.

3. The resulting statements are theorems.

\ETask{XII}{Pay attention to the use of Complex Subject in these examples. Think of other sentences with this construction.}
1. Two-dimensional vectors are usually called complex numbers in circumstances where one is likely to multiply them.

2. The adjective `real' is not supposed to imply that complex numbers are hallucinatory.

3. The essential idea seems to have been developed in Alexandria in the Third Century B.C.

\ETask{XIII}{Make these sentences interrogative.}

1. There is a plethora of notation in mathematics.

2. There is one other notational matter which must be explained.

3. There is a very nice exposition of this construction in Landau's work.

4. In every arithmetic progression there is an infinity of primes.

5. There can be only a finite number of smaller denominators.

6. If the fraction were not reduced, then there would be another specimen in the same class with a smaller positive denominator.

\ETask{XIV}{Give the proper forms (ing-form or Past Participle) of the verbs in parentheses.}

1. We turn to the problem of (to define) a `good' multiplication for vectors.

2. In problems (to concern) with scalar multiplication a number is called a scalar, and in problems (to concern) with what
we shall call complex multiplication a number is called a real number.

3. The standard way of (to write) complex numbers is as the sum of scalar multiples of two particular complex numbers.

4. It is customary in (to discuss) complex numbers to suppress the symbol `$I$' entirely and, for example, to write $(2,3)$ as $2+3i$.

5. (to state) more precisely, it means that if `number' is replaced by `complex number' in each of the axioms, we want the
(to result) statements to be theorems.

6. (to use) ruler and compass, construct the tangent to $P$ at $(3, \frac{9}{4})$.

7. Derive a theorem analogous to this one for the parabola (to describe) in the problem above.

8. Prove Heron's Theorem (to use) elementary (non-vector) geometry.

9. Assume here that the tangent to an ellipse at a (to give) point is the unique line (to intersect) the ellipse only at that point.

\ESect{Writing}

\ETask{XV}{In scientific writing you should add details if you think the people who read your paper won't understand the
main idea without more explanation. Some expressions used for this purpose are: \textbf{in other words, to put it differently,
that is to say, by . . . we mean, by . . . is meant}. 1. Now look through the text again to see if there are sentences containing
linkers of the above type. 2. Write a paragraph on one of the points under consideration in the text. Begin with a topic
sentence and then add some details that would make your idea clear.}

\ESect{Supplementary Texts}

\ESect{Text 1}

This section is devoted to a study of those properties of 2-dimensional vector space which do not extend to 3-space,
the primary purpose being to define and exploit a multiplication of 2-dimensional vectors. We begin the study by considering
the solutions of equations. Linear equations are easily disposed of, but the situation is a little more complicated for
quadratic equations. If $a$, $b$, and $c$ are numbers and $a$ is not zero, then it may or may not happen that there is a
number $x$ such that $ax^2+bx+c=0$. For example, it is easy to see that there is no number $x$ such that $x^2+1=0$, although
there certainly is a number $x$ such that $x^2-1=0$. We construct a multiplication for 2-dimensional vectors such that there is
a 2-dimensional vector $A$ with the property that the vector sum of $A^2$ and the multiplicative identity is the zero vector,
and in this sense, we obtain a solution of the equation $x^2+1=0$. We shall call 2-dimensional vectors `complex numbers' and,
to emphasize the difference between numbers and complex numbers, we call the ordinary numbers `real'. Thus it will appear that,
whereas there are no real numbers $x$ such that $x^2+1=0$, there are complex numbers with this property. (We have just given a
more precise statement of this fact.)

We consider briefly two topics other than the algebraic and geometric properties of complex numbers. Parabolas are closely
connected with quadratic functions, and I've succumbed cheerfully to the temptation to exhibit some of the beautiful geometry
of the parabola.

It may be wise to mention, in concluding, something of the usefulness of the algebra of 2-dimensional vectors. There is a
large and beautiful mathematical theory, called the theory of complex functions, which is based on complex numbers. This
theory was constructed just as an abstract mathematical creation more than a half century before alternating electric
current was first generated, and much before the first airplane was flown. Yet it has turned out that the theory of complex
functions is precisely the mathematical tool needed to study the transmission of alternating current or to design an airfoil
for subsonic flight. There are other equally startling applications.

\ESect{Text 2}
Any number which can be found as a point on the number scale extending from $-\infty$ to $+\infty$ (an indefinitely long
straight line with some fixed point 0 used as the zero mark, and a convenient scale to represent one unit) is known as a
real number, and these are subdivided into rational and irrational numbers. A rational number is any number that can be
expressed in the form $\frac{p}{q}$, where $p$ and $q$ are integers, and an irrational number is any real number that cannot
be expressed in this form.

The square root of $-1$ is denoted by the symbol $i$ $(i = \sqrt{-1})$. The value of this quantity cannot be determined as a
real number, and therefore the product of any real number and $i$, which also cannot be placed on the number scale, is known
as an imaginary number. From this it can be seen that no real number, except zero, can be equal to an imaginary number. (Zero
is a neutral number and can be taken to be real or imaginary.)

A complex number is the sum of a real number and an imaginary number. Thus, if $a$ and $b$ be real numbers, a complex number
will be represented by $a+ib$, which is the standard form of representation. It will have been found that complex numbers have
arisen as the roots of certain quadratic equations and, therefore, all the processes of algebra are applicable to complex numbers.

It is to be noted that $i^2=-1$, $i^3=-i$, $i^4=+1$, $i^5=i$, and these values occur in cycles for higher powers of $i$.

\ESect{Text 3}
Once the real number system is at hand. it is a simple matter to create the complex number system. We shall indicate in this
section how the complex field can be constructed.

As seen before, the real number system is a field which satisfies certain additional properties. We constructed the Cartesian
space $\mathbb{R}^p$ and introduced some algebraic operations in the $p$-fold Cartesian product of $\mathbb{R}$. However, we
did not make $\mathbb{R}^p$ into a field. It may come as a surprise that it is not possible to define a multiplication which
makes $\mathbb{R}^p$, $p \geqslant 3$, into a field. Nevertheless, it is possible to define a multiplication operation in
$\mathbb{R} \times \mathbb{R}$ which makes this set into a field. We now introduce the desired operations.

DEFINITION. The complex number system $\mathbb{C}$ consists of all ordered pairs $(x,y)$ of real numbers with the operation
of addition defined by $$(x,y)+(x',y')=(x+x',y+y'),$$ and the operation of multiplication defined by
$$(x,y) \cdot (x',y')=(xx'-yy',xy'+x'y).$$

Thus the complex number system $\mathbb{C}$ has the same elements as the two-dimensional space $\mathbb{R}^2$. It has the same
addition operation, but it possesses a multiplication as $\mathbb{R}$ does not. Therefore, considered merely as sets,
$\mathbb{C}$ and $\mathbb{R}$ are equal since they have the same elements; however, from the standpoint of algebra, they are
not the same since they possess different operations.

An element of $\mathbb{C}$ is called a complex number and is often denoted by a single letter such as $z$. If $z=(x,y)$, then
we refer to the real number $x$ as the real part of $z$ and to $y$ as the imaginary part of $z$, in symbols, $x = \Re z, y = \Im z$.
The complex number $z=(x,-y)$ is called the conjugate of $z=(x,y)$.

\EUnit
\ESect{Reading}

\ETask{I}{Pre-reading questions:}
1. What do you know of P. de Fermat?

2. State Fermat's Last Theorem.

\ETask{II}{Read the text. Make a list of mathematical terms. If necessary consult a special dictionary for their meaning and
pronunciation.}

\ESect{Text}
This is the famous Fermat-Euler theorem, If $p$ is a prime, then for all natural $n$ $n^{p-1}-1$ can be divided by $p$, for
example $10^{p-1} - 1$ is divisible by $p$, provided $p \neq 2,5$. This theorem was known to Fermat, one of the very greatest
mathematicians of the 17th century, if not of all time --- a jurist whose contributions to mathematics, his hobby, made him
immortal, whereas his jurisprudence is forgotten. Euler put the theorem in its most general form, which we shall shortly produce,
but the essential idea is due to Fermat. Let us make a few examples. Taking $p$ as small as possible, i.e., 3, we have $10^2-1=99$
is divisible by 3; or for $p=7$, we have $10^6-1=999999$ is divisible by 7.

Here we have the special number 10 in our formula, as a result of the fact that we used the decimal system to write our fractions.
However, the whole argument could be reproduced using a $g$-adic number system, in which $\alpha \beta \gamma$ means
$\alpha g^2 + \beta g + \gamma$ and, $0.\alpha\beta\gamma$ means $\alpha g^{-1}+ \beta g^{-2} + \gamma g^{-3}$ defining periodic
decimal fractions in the same way. (To the mathematician the discussion of these different number systems isn't very interesting,
for they are merely notations that have nothing to do with the fundamental nature of numbers. The one that we use is merely a
linguistic heritage --- probably a biological accident in that we have 10 fingers --- although our language does have remnants
of other systems of notation, notably dozen, score, and gross.)

Thus we may replace 10 by any number coprime to $b$, and obtain: $g^{\varphi(b)}-1$ is divisible by $b$, provided that $g$ and $b$
are coprime, the Fermat-Euler theorem in its most general formulation. This theorem, we see, provides the background for the
systematic study of decimal fractions.

To satisfy curiosity we state here Fermat's Last Theorem. We could show that there are infinitely many integers, the so-called
Pythagorean numbers, that satisfy the equation $a^2+b^2=c^2$, e.g., $3^2+4^2=5^2$, $5^2+12^2=13^2$, etc. Format claimed there
were no integers satisfying $a^n+b^2=c^2$ for $n>2$. We do now know that this is true for many $n$, but it is still not proved
in full generality. \textsc{Corrector Hint: The Great Fermat Theorem was successfully proved in 1995-1996, hence author of
this book hadn't heard about this great event.} In part the interest of the theorem lies in the provocative way in which it
was first stated. Format wrote the assertion on his copy of Diophantus together with the remark, `...I have discovered a truly
marvellous demonstration which this margin is too narrow to contain.' However, the importance of the theorem lies not in its
content, but in the mathematics developed in the attempts to prove it --- the efforts to do so in the 19th century yielded the
new field of algebraic number theory and the notion of ideal numbers developed first by Kummer.

In recent years some remarkable applications of the Fermat-Euler theorem have come to light. It was a surprise to many people to
learn that a procedure was developed whereby a secret message could be encoded and the person encoding the message would not be
able to reverse the process and decode the message.

\ETask{III}{Answer the questions:}
1. Why is the famous Fermat-Euler theorem called so?

2. What is a number system we use?

3. Why is the Fermat Theorem so interesting and important?

4. What applications of the Fermat-Euler theorem do you know?

5. `Fermat's hobby made him immortal, whereas his jurisprudence is forgotten'. Comment on this statement.

\ETask{IV}{What do the italicized words refer to?}

1. This theorem was known to Fermat.

2. The one that we use is merely a linguistic heritage.

3. This theorem provides the background for the systematic study of decimal fractions.

4. We do now know that this is true, but it is still not proved in full generality.

5. The efforts to do so in the 19th century yielded the new field of algebraic number theory.

\ESect{Vocabulary}
\ETask{V}{Give the Russian equivalents of the following words and word combinations:}
the formula reads; is divisible by; provided; the very greatest; if not of all time; we shall shortly produce; is due to;
let us; in the same way; have nothing to do with; in part; too narrow; come to light; as follows; let $N$ be; at least; it
takes a computer hundreds of years to factor $N$; as shown above.

\ETask{VI}{What is the function of one in these sentences?}
1. This theorem was known to Fermat, one of the very greatest mathematicians of the 17th century.

2. The one that we use is merely a linguistic heritage.

3. To decode the message, one simply takes $R^p$ and computes the remainder after dividing by $N$.

\ETask{VII}{Compare the meanings of italicized words in the following sentences.}

1. This theorem we see \textit{provides} the background for the systematic study of decimal fractions.

2. If $p$ is a prime, $\varphi(p)=p-1$, the formula reads $10^{p-1}-1$ is divisible by $p$, \textit{provided} $p \neq 2,5$.

3. Thus we may replace 10 by any number coprime to $b$, and obtain: $g^{\varphi(p)}-1$ is divisible by $b$, \textit{provided} that
$g$ and $b$ are coprime.

\ETask{VIII}{What is the meaning of for in the sentences below? Give examples of other meanings offer that you know.}
1. This theorem we see provides the background for the systematic study of decimal fractions.

2. Assume for simplicity that $N = p_1p_2$, the product of two large primes.

3. To the mathematician the discussion of these different number systems isn't very interesting, for they are merely
notations that have nothing to do with the fundamental nature of numbers.

\ETask{IX}{The words in the chart below have all appeared in the text. Use your dictionary to find the other parts of speech,
their translation and pronunciation.}
\ETypeWr{
Noun~~~~~~~~~Adjective~~~~~Verb

=======================================

-~~~~~~~~~~~~divisible~~~~~-

-~~~~~~~~~~~~-~~~~~~~~~~~~~to use

-~~~~~~~~~~~~-~~~~~~~~~~~~~to reproduce

-~~~~~~~~~~~~-~~~~~~~~~~~~~to mean

discussion~~~-~~~~~~~~~~~~~-

-~~~~~~~~~~~~systematic~~~~-

-~~~~~~~~~~~~general~~~~~~~-

-~~~~~~~~~~~~-~~~~~~~~~~~~~to define

idea~~~~~~~~~-~~~~~~~~~~~~~-

-~~~~~~~~~~~~remarkable~~~~-

application~~-~~~~~~~~~~~~~-

surprise~~~~~-~~~~~~~~~~~~~-

-~~~~~~~~~~~~-~~~~~~~~~~~~~to assume

product~~~~~~-~~~~~~~~~~~~~-

-~~~~~~~~~~~~-~~~~~~~~~~~~~to satisfy

-~~~~~~~~~~~~-~~~~~~~~~~~~~to prove
}

\ETask{X}{Make adverbs of the following adjectives. Pay attention to their spelling. Check against the text.}
Short, mere, notable, infinite, true, real, exact, simple.

\ESect{Grammar}
\ETask{XI}{Join the notes with right forms of verbals. Then refer to the text.}

1. (to take) $p$ as small as possible, i.e. 3, we have...

2. The whole argument could be reproduced (to use) a $g$-adic number system.

3. Fermat claimed there were no integers (to satisfy) $a^n+b^n=c^n$ for $n>2$.

4. The importance of the theorem lies in the mathematics (to develop) in the attempts to prove it.

5. The person (to encode) the message would not be able to reverse the process and decode the message.

6. A person (to know) only $N$ could never really determine $\varphi(N)$.

7. Let $E$ and $D$ be two integers (to satisfy) $ED=\varphi(N)+1$.

8. As (to show) above, he cannot know $D$.

9. Let us assume he has a message $M$, (to give) in the form of a large number.

10. Then $R$ will be the (to encode) message.

\ETask{XII}{Supply the right active and passive forms and tenses of the verbs in brackets.}

1. This theorem (to know) to Fermat.

2. ...his hobby (to make) him immortal, whereas his jurisprudence (to forget).

3. The whole argument could (to reproduce) using a $g$-adic number system.

4. Thus we may (to replace) 10 by any number coprime to $b$.

5. ...it (not, to prove) in full generality.

6. In part the interest of the theorem (to lie) in the provocative way in which it (to state).

7. I (to discover) a truly marvellous demonstration which this margin (to be) too narrow to contain.

8. In recent years some remarkable applications of the Fermat-Euler theorem (to come) to light.

9. It was a surprise to many people to learn that a procedure (to develop) whereby a secret message (can, to encode) and the
person encoding the message (to be able) to reverse the process and decode the message.

10. The person encoding the message (to give) the numbers $N$ and $E$.

11. This (to base) on the fact that...

\ETask{XIII}{Use Present Perfect in the following sentences.}
1. Format wrote, `...I (to discover) a truly marvellous demonstration which this margin is too narrow to contain'.

2. In recent years some remarkable applications of the Fermat-Euler theorem (to come) to light.

3. The rational numbers satisfy all of the axioms for numbers which (to list) so far.

4. As we (to remark) already, it is possible to show that one system is simply a copy of the other.

5. We (to finish) our review and restatement of elementary algebra and are ready to study the topics which form the principal
subject matter of this course.

6. We (not to prove) from the axioms that every non-negative number has a square root.

\ETask{XIV}{What is the function of the verb to do in these sentences?}
1. ...although our language does have remnants of other systems of notation, notably dozen, score, and gross.

2. We do now know that this is true for many n, but it is still not proved in full generality.

\ETask{XV}{Use Subjunctive Mood in the following sentences.}
1. The whole argument (can, to reproduce) using a $g$-adic number system.

2. If $p_1$ and $p_2$ are very large, then it (can, to take) a computer hundreds of years to factor $N$, and hence a person knowing
only $N$ (can never, to determine) $\varphi(N)$.

3. A procedure was developed whereby a secret message (can, to encode) and the person encoding the message (not to be able) to
reverse the process and decode the message.

4. We (can, to show) that there are infinitely many integers.

\ESect{Writing}

\ETask{XVI}{1. Give a brief oral summary of what you consider to be the most important information in the text under discussion.
2. Take notes on the important information. 3. Use your notes to write a summary of the text. After you have written your summary:
1. In groups, share what you have written. 2. Decide which is the best summary in your group. 3. Discuss the characteristics of
a good summary.}

\EUnit
\setcounter{equation}{0}
\ESect{Reading}

\ETask{I}{Pre-reading questions:}

1. What is dynamics?

2. What do you know about Newton's laws of motion?

\ETask{II}{Read the text and try to guess the meaning of unknown words. Use dictionaries if necessary.}

\ESect{Text}
Dynamics is the branch of mechanics which deals with the physical laws governing the actual motion of material bodies. One
of the fundamental tasks of dynamics is to predict, out of all possible ways a material system can move, which particular
motion will occur in any given situation.

\textbf{Newton's Laws of Motion}

You are already familiar with Newton's laws of motion. They are as follows:

I. Every body continues in its state of rest or of uniform motion in a straight line, unless it is compelled by a force to
change that state.

II. Change of motion is proportional to the applied force and takes place in the direction of the force.

III. To every action there is always an equal and opposite reaction, or, the mutual actions of two bodies are always equal
and oppositely directed.

Let us now examine these laws in some detail.

\textbf{Newton's First Law. Inertial Reference Systems.}

The first law describes a common property shared by all matter, namely inertia. The law states that a moving body travels
in a straight line with constant speed unless some influence called force prevents the body from doing so. Whether or not
a body moves in a straight line with constant speed depends not only upon external influences (forces) but also upon the
particular reference system that is used to describe the motion. The first law actually amounts to a definition of a
particular kind of reference system called a Newtonian or inertial reference system. Such a system is one in which Newton's
first law holds.

The question naturally arises as to how it is possible to determine whether or not a given coordinate system constitutes
an inertial system. The answer is not simple. In order to eliminate all forces on a body it would be necessary to isolate
the body completely. This is impossible, of course, since there are always at least some gravitational forces acting unless
the body was removed to an infinite distance from all other matter.

For many practical purposes not requiring high precision, a coordinate system fixed to the earth is approximately inertial.
Thus, for example, a billiard ball seems to move in a straight line with constant speed as long as it does not collide with
other balls or hit the cushion. If the motion of a billiard ball were measured with very high precision, however, it would be
discovered that the path is slightly curved. This is due to the fact that the earth is rotating and so a coordinate system
fixed to the earth is not actually an inertial system. A better system would be one using the center of the earth, the center
of the sun, and a distant star as reference points. But even this system would not be strictly inertial because of the
earth's orbital motion around the sun. The next best approximation would be to take the center of the sun and two distant
stars as reference points, for example. It is generally agreed that the ultimate inertial system, in the sense of Newtonian
mechanics would be one based on the average background of all the matter in the universe.

\textbf{Mass and Force. Newton's Second and Third Laws.}

We are all familiar with the fact that a big stone is not only hard to lift, but that such an object is more difficult to
set in motion (or to stop) than, say, a small piece of wood. We say that the stone has more inertia than the wood. The
quantitative measure of inertia is called mass. Suppose we have two bodies $A$ and $B$. How do we determine the measure of
inertia of one relative to the other? There are many experiments that can be devised to answer this question. If the two
bodies can be made to interact with one another, say by a spring connecting them, then it is found, by careful experiments,
that the accelerations of the two bodies are always opposite in direction and have a constant ratio. (It is assumed that the
accelerations are given in an inertial reference system and that only the mutual influence of the two bodies $A$ and $B$
is under consideration.) We can express this very important and fundamental fact by the equation
\Eqn{\frac{dv_A}{dt}=-\frac{dv_B}{dt}\mu_{BA}.}
The constant $\mu_{BA}$ is, in fact, the measure of the relative inertia of $B$ with respect to $A$. From Equation (1) it
follows that $$\mu_{BA}=\frac{1}{\mu_{AB}}.$$
Thus we might express $\mu_{BA}$ as a ratio
$$\mu_{BA}=\frac{m_B}{m_A}$$
and use some standard body as a unit of inertia. Now the ratio $\frac{m_B}{m_A}$ ought to be independent of the choice of the
unit. This will be the case if, for any third body $C$,
$$\frac{\mu_{BC}}{\mu_{AC}}=\mu_{BA}.$$
This is indeed found to be true. We call the quantity $m$ the mass.

Strictly speaking, $m$ should be called the inertial mass, for its definition is based on the properties of inertia. In
actual practice mass ratios are usually determined by weighing. The weight or gravitational force is proportional to what
may be called the gravitational mass of a body. All experience thus far, however, indicates that inertial mass and gravitational
mass are strictly proportional to one another. Hence for our purpose we need not distinguish between the two kinds of mass.

The fundamental fact expressed by Equation (1) can now be written in the form
\Eqn{m_A\frac{dv_A}{dt}=-m_B\frac{dv_B}{dt}.}
The product of mass and acceleration in the above equation is the `change of motion' of Newton's second law and, according to
that law, is proportional to the force. In other words, we can write the second law as
\Eqn{F=km\frac{dv}{dt}}
where $F$ is the force and $k$ is a constant of proportionality. It is customary to take $k = 1$ and write
\Eqn{F=m\frac{dv}{dt}}
The above equation is equivalent to
\Eqn{F=\frac{d(mv)}{dt}}
if the mass is constant. As we shall see later, the theory of relativity predicts that the mass of a moving body is not constant
but is a function of the speed of the body, so that Equations (4) and (5) are not strictly equivalent. However, for speeds that
are small compared to the speed of light, $3 \times 10^8$ m/sec, the change of mass is negligible.

According to Equation (4) we can now interpret the fundamental fact expressed by Equation (2) as a statement that two interacting
bodies exert equal and opposite forces on one another:
$$F_A = -F_B$$
This is embodied in the statement of the third law. Forces are mutual influences and occur in equal amounts on any two bodies
that affect each other's motion.

One great advantage of the force concept is that it enables us to restrict our attention to a single body. The physical
significance of the idea offeree is that, in a given situation, there can usually be found some relatively simple function
of the coordinates, called the force function, which when set equal to the product of mass and acceleration correctly
describes the motion of a body.

\ETask{III}{Think of the questions to the main points of the text. Then try to answer the questions given by other students.}

\ESect{Vocabulary}

\ETask{IV}{Give the Russian equivalents of the following words and word combinations:}
as follows; to take place; let us do it; in some detail; namely; reference system; the law holds; as to how; in order to;
at least; as long as; due to; reference points; because of; it is generally agreed; in the sense of; to be under consideration;
with respect to; from this it follows that...; this will be the case if...; strictly speaking; in actual practice; thus far;
the above equation; according to that law; in other words; it is customary; compared to.

\ETask{V}{The following combinations have all appeared in the text. Give their Russian equivalents and pay your attention to
the use of words another and other. If necessary see the table in Unit 7.}
from all other matter; it does not collide with other balls; one relative to the other; to interact with one another;
proportional to one another; in other words; to exert equal and opposite forces on one another; to affect each other's motion

\ETask{VI}{Supply prepositions where necessary. Then check against the text.}

1. Change of motion is proportional --- the applied force and takes place --- the direction --- the force.

2. --- every action there is always an equal and opposite reaction.

3. Let us now examine these laws --- some detail.

4. The law states that a moving body travels --- a straight line --- constant speed unless some influence called force
prevents the body --- doing so.

5. This depends not only --- external influences (forces) but also --- the particular reference system.

6. The question naturally arises as --- how it is possible to determine this.

7. This is impossible since there are always --- least some gravitational forces acting unless the body was removed --- an
infinite distance --- all other matter.

8. Suppose we have two bodies $A$ and $B$. How do we determine the measure --- inertia --- one relative --- the other?

9. Only the mutual influence --- the bodies $A$ and $B$ is --- consideration.

10. The constant $\mu_{BA}$ is, --- fact, the measure --- relative inertia --- $B$ --- respect --- $A$.

11. --- our purpose we need not distinguish --- the two kinds --- mass.

12. --- other words, we can write the second law as...

13. --- speeds that are small compared --- the speed --- light the change --- mass is negligible.

14. Forces are mutual influences and occur --- equal amounts --- any two bodies that affect --- each other's motion.

15. One great advantage --- the force concept is that it enables --- us to restrict our attention --- a single body.

\ETask{VII}{What is the meaning of one in the following sentences?}

1. One of the fundamental tasks of dynamics is to predict which particular motion will occur in any given situation.

2. Such a system is one in which Newton's first law holds.

3. A better system would be one using the center of the earth, the center of the sun, and a distant star as reference points.

4. The ultimate inertial system, in the sense of Newton's mechanics, would be one based on the average background of all the
matter in the universe.

5. Suppose we have two bodies $A$ and $B$. How do we determine the measure of inertia of one relative to the other?

6. All experience thus far indicates that inertial mass and gravitational mass are strictly proportional to one another.

7. One great advantage of the force concept is that it enables us to restrict our attention to a single body.

\ETask{VIII}{Make adverbs of the following words. Mind their spelling. Give their meanings:}
actual; natural; particular; simple; mutual; ultimate; fast; hard; opposite; strict; first; above; name; late.

\ETask{IX}{Fill in the chart with forms of the given verbs. Mind their spelling.}
\ETypeWr{
Infinitive~~~Past Simple~~~Past Participle~~~Ing-form~~~Primary Meaning

to deal

to occur

to compel

to apply

to hold

to (a)rise

to raise

to hit

to set

to find

to found
}

\ETask{X}{Supply some, any, all, every, each. Then check against the text.}
1. One of the fundamental tasks of dynamics is to predict which particular motion will occur in --- given situation.

2. To --- action there is always an equal and opposite reaction.

3. Let us now examine these laws in --- detail.

4. The first law describes a common property shared by --- matter, namely inertia.

5. The law states that a moving body travels in a straight line with constant speed unless --- influence called force
prevents the body from doing so.

6. In order to eliminate --- forces on a body it would be necessary to isolate the body completely.

7. There are always at least --- gravitational forces acting unless the body was removed to an infinite distance from
--- other matter.

8. We are --- familiar with the fact.

9. We might use - standard body as a unit of inertia.

10. This will be the case if, for --- third body $C$, $\frac{\mu_{BC}}{\mu_{AC}}=\mu_{BA}$.

11. Forces are mutual influences and occur in equal amounts on --- two bodies that affect --- other's motion.

12. There can usually be found --- relatively simple function of the coordinates, called the force function.

\ETask{XI}{Connect each pair of the sentences with if or unless.}

1. The two bodies can be made to interact with one another, say by a spring connecting them. It is found, by careful experiments,
that the accelerations of the two bodies are always opposite in direction and have a constant ratio.

2. Every body continues in its state of rest or of uniform motion in a straight line. It is not compelled by a force to
change that state.

3. A moving body travels in a straight line with constant speed. No influence prevents the body from doing so.

4. The form of the function $f$ is known. We know the motion of the fluid.

5. There are always at least some gravitational forces acting. The body was not removed to an infinite distance from all
other matter.

6. We draw the streamline through each point of a closed curve. We obtain a stream tube.

\ESect{Grammar}

\ETask{XII}{Put in proper forms (Past Participle or ing-forms) of the verbs given in brackets.}

1. Dynamics is the branch of mechanics (to deal) with the physical laws (to govern) the actual motion of material bodies.

2. The first law describes a common property (to share) by all matter, namely inertia.

3. The law states that a (to move) body travels in a straight line with constant speed unless some influence (to call) force
prevents the body from (to do) so.

4. There are always at least some gravitational forces (to act) unless the body was removed to an infinite distance from
all other matter.

5. For many practical purposes (not to require) high precision, a coordinate system (to fix) to the earth is approximately inertial.

6. A better system would be one (to use) the center of the earth, the center of the sun, and a distant star as reference points.

7. The ultimate inertial system would be one (to base) on the average background of all the matter in the universe.

8. The two bodies can be made to interact with one another, say by a spring (to connect) them.

9. In actual practice mass ratios are usually determined by (to weigh).

10. For speeds that are small (to compare) to the speed of light the change of mass is negligible.

11. We can now interpret the fundamental fact (to express) by this equation as a statement that two (to interact) bodies exert
equal and opposite forces on one another.

12. There can usually be found some relatively simple function of the coordinates, (to call) the force function, which when
(to set) equal to the product of mass and acceleration correctly describes the motion of a body.

\ETask{XIII}{Put in modals (can, may, might, need, should, ought to) and appropriate forms of the verbs (Passive or Active)
given in brackets. Then check against the text.}

1. One of the fundamental tasks of dynamics is to predict, out of all possible ways a material system --- (to move), which
particular motion will occur in any given situation.

2. There are many experiments that --- (to devise) to answer this question.

3. We --- (to express) this very important and fundamental fact by the equation.

4. Thus we --- (to express) $\mu_{BA}$ as a ratio.

5. Now the ratio $\frac{\mu_B}{\mu_A}$ --- (to be) independent of the choice of the unit.

6. Strictly speaking, $m$ --- (to call) the inertial mass, for its definition is based on the properties of inertia.

7. The weight or gravitational force is proportional to what --- (to call) the gravitational mass of a body.

8. Hence for our purpose we --- (not to distinguish) between the two kinds of mass.

9. The fundamental fact expressed by Equation (1) --- now (to write)
in the form...

10. In other words, we --- (to write) the second law as...

11. According to Equation (4) we --- now (to interpret) the fundamental fact expressed by Equation (2) as a statement that
two interacting bodies exert equal and opposite forces on one another.

12. There --- usually (to find) some relatively simple function of the coordinates, called the force function.

\ETask{XIV}{Connect the following pairs of sentences with whether.}

1. A body moves or doesn't move in a straight line with constant speed. It depends not only upon external influences
(forces) but also upon the particular reference system that is used to describe the motion.

2. It is possible to determine. A given coordinate system constitutes or doesn't constitute an inertial system.

3. We need to know... The work done usually depends on the particular route the particle takes in going from one point to another.

4. We are not sure. This knowledge can be of use in predicting the motion of the particle.

5. We'd like to know... It is the most fundamental type of force that occurs in nature.

6. The question naturally arises. The corresponding statement is or is not true in our case.

7. It is easily verified. The definitions stated above have the following interpretations.

8. We want to know... The proofs of these statements follow directly from the definition.

9. Now we can find out... The remaining equations are easily proved in a similar manner.

\ETask{XV}{Supply the proper forms of the verbs in brackets. Mind the use of infinitives after the verbs \textbf{to make}
and \textbf{to let}.}

1. Let us now (to examine) these laws in some detail.

2. Our task is to make theory and experiment (to agree) as closely as possible.

3. The two bodies can be made (to interact) with one another.

4. Let us (to examine) the geometric significance of the velocity vector.

5. Attractive forces may make molecules (to collide).

6. Let us (to consider) a rigid body in a uniform gravitational field, say at the surface of the earth.

7. The body was made (to move) along the given path with constant speed.

8. Let us (to choose) the $z$-axis of an appropriate coordinate system as the axis of rotation.

\ETask{XVI}{Put in the forms of the Subjunctive mood of the verbs in brackets.}

1. In order to eliminate all forces on a body it (to be) necessary to isolate the body completely.

2. If the motion of a billiard ball (to measure) with very high precision, it (to discover) that the path is slightly curved.

3. A better system (to be) one using the center of the earth, the center of the sun, and a distant star as reference points.

4. But even this system (not to be) strictly inertial because of the earth's orbital motion around the sun.

5. The next best approximation (to be) to take the center of the sun and two distant stars as reference points.

6. It is generally agreed that the ultimate inertial system, in the sense of Newtonian mechanics, (to be) one based on the
average background of all the matter in the universe.

7. We (may, to express) this as a ratio.

8. If two of the three principal moments of inertia (to be) equal, then the ellipsoid of inertia (to be) one of revolution.

9. The earth attracts as if all of its mass (to concentrate) at a single point.

10. A uniform spherical body attracts an external particle as if the entire mass of the sphere (to locate) at the center.

11. The center of mass of the shrapnel from an artillery shell that has burst in mid-air will follow the same parabolic path
that the shell (to take) if it (not to burst).

\ETask{XVII}{Translate the sentences with Complex Subject.}

1. A billiard ball seems to move in a straight line with constant speed as long as it does not collide with other balls or
hit the cushion.

2. The accelerations of the two bodies are found to be always opposite in direction and have a constant ratio.

3. The accelerations are assumed to be given in an inertial reference system.

4. This is indeed found to be true.

5. The force in this case is said to be conservative.

6. This inverse-square relation is also found to be the law of force for the electric fields of elementary particles.

7. The same function turns out to give the correct force for the three-dimensional case.

8. When the force F is a function of the positional coordinates only, it is said to define a static force field.

9. The stone is said to have more inertia than the wood.

\ETask{XVIII}{Put the words in the correct order to make sentences.}
1. Inertia, the, measure, of, is, mass, called, quantitative.

2. Practice, ratios, in, usually, by, determined, weighing, are, mass, actual.

3. Bodies, two, equal, forces, another, interacting, exert, opposite, one, on, and.

4. Our, need, between, kinds, for, we, distinguish, two, mass, purpose, not, the, of.

5. To, forces, body, be, isolate, completely, order, all, a, would, to, body, in, eliminate, on, it, necessary, the.

\ESect{Writing}

\ETask{XIX}{1. Give a brief oral summary of what you consider to be the most important information in the text you have read.
2. Take notes on the important information. 3. Use your notes to write a summary of the text. \textbf{NB!} When writing your
summary, put aside the original text and work from your notes, putting information into complete sentences in your own words.}

\ESect{Supplementary Texts}

\ESect{Text 1}
To describe the motion of mechanical systems one uses a variety of mathematical models which are based on different `principles'
--- laws of motion. The simplest and most important model of motion of real bodies is Newtonian mechanics, which describes the
motion of a free system of interacting point masses in three-dimensional Euclidean space.

\textbf{Newtonian Mechanics}

\textbf{Space, Time, Motion.} Space, in which motion takes place, is three-dimensional and Euclidean, with a fixed orientation.
We shall denote it by $\mathbb{E}^3$. Fix a point $0 \in \mathbb{E}^3$ an `origin' or `reference point'. Then the position of
each point $s$ in $\mathbb{E}^3$ is uniquely specified by its position (radius) vector $\overrightarrow{os}= r$ (with its tail
and tip at 0 and $s$, respectively). The set of all position vectors is the three-dimensional linear space $\mathbb{R}^3$. This
space is equipped with the scalar product $<,>$.

Time is one-dimensional; we denote it uniformly by $t$. The set $R = \{ t \}$ is called the time axis.

A motion (or path) of the point $s$ is a smooth mapping $\Delta \rightarrow \mathbb{E}^3$, where $\Delta$ is a time interval.
We say that the motion is defined on the interval $\Delta$. To each motion there corresponds a unique smooth vector-function
$r: \Delta \rightarrow \mathbb{R}^3$.

The velocity $v$ of the point $s$ at time $t \in \Delta$ is the derivative $\frac{dr}{dt} = r'(t) \in \mathbb{R}^3$. Velocity is
clearly independent of the choice of the reference point.

The acceleration of the point $s$ is the vector $a=\dot v=\ddot r \in \mathbb{R}^3$. It is customary to represent the velocity
and acceleration as vectors with tail at the point $s$.

The set $\mathbb{E}^3$ is also known as the position (or configuration) space of the point $s$. The pair $(s, v)$ is called a
state of $s$, and the space $\mathbb{E}^3 \times \mathbb{R}^3\{v\}$ is the state space (or the velocity phase space).

\ESect{Text 2}
\textbf{The Newton-Laplace Principle of Determinacy.} This principle asserts that the state of a mechanical system at any
fixed moment of time uniquely determines all of its (future and past) motion.

Suppose we know the state $(r_0,v_0)$ of the system at the moment of time $t_0$. Then by the principle of determinacy we also
know the motion $r(t)$, with $r(t_0)=r_0$ and $\dot r(t_0) = \dot r_0 = v_0$, for all $t \in \Delta \subset \mathbb{R}$. In
particular, we can calculate the acceleration $\ddot r$ at $t = t_0$. The result is $\ddot r(t_0)=f(t_0,r_0,\dot r_0)$, where
$f$ is a function whose existence follows from the Newton-Laplace principle. Since we may choose an arbitrary value for
$t_0$, we conclude that the equation
$$\ddot r = f(t, r, \dot r)$$
holds for all $t$. This differential equation is known as the equation of motion or as Newton's equation. The existence of
Newton's equation with a smooth vector-function
$$f: \mathbb{R}\{t\}\times\mathbb{R}^{3n}\{r\}\times\mathbb{R}^{3n}\{r\} \rightarrow \mathbb{R}^{3n}$$
and the determinacy principle are actually equivalent. This is a consequence of the theorem of existence and uniqueness
of solutions from the theory of ordinary differential equations. The function $f$ in Newton's equation is usually determined
experimentally. Its specification is part of the definition of the mechanical system under consideration.

We will now give examples of Newton's equation.

a) The equation describing the fall of a point (small body) in vacuum near the Earth's surface (obtained experimentally by
Galileo) has the form: $\ddot r = -ge_z$, where $g \approx 9.8 m/s^2$ (the free fall or gravitational acceleration), and $e_z$
is the vertical unit vector.

b) R. Hooke showed that the equation governing the small oscillations of a body attached to the extremity of an elastic spring
has the form: $\ddot x = -\alpha x, \alpha > 0$. The constant coefficient $\alpha$ depends on the body and on the spring in question. This
mechanical system is known as the harmonic oscillator.

As it turns out, in experiments, rather than determining the acceleration $f$ appearing in the right-hand side of Newton's
equation, it is more convenient to determine the product $mf=F$, where $m$ is a positive number called the mass of the point
(unraveling the physical meaning of the mass concept is not among the tasks of dynamics). Thus, in Hooke's experiments the
constant $m \alpha=c$ depends on the properties of the elastic spring, but not on the choice of the attached body; $c$ is called
the elasticity constant (or coefficient).

The pair $(s,m)$ (or $(r, m)$, where $r$ is the position vector of the point $s$ is called a material point (or point mass,
or particle) of mass $m$. Hereafter we shall often use the letter $m$ to denote both the point $s$ and its mass $m$. If a
system consists of $n$ material points with masses $m_1, \ldots m_n$, then Newton's equations
$$\ddot r_i = f_i(t,r_1, \ldots, r_n, \dot r_1, \ldots, \dot r_n), \; 1 \leqslant i \leqslant n,$$
may be rewritten as
$$m_i \ddot r_i = F_i(t,r,\dot r), \; 1 \leqslant i \leqslant n.$$
The vector $F_i = m_i f_i$ is called the force acting on the point $m_i$. The word force does not appear in the basic laws
of dynamics that we just indicated. As a matter of fact, we may also manage without it. The last equations will be also referred
to as Newton's equations.

c) Newton established that if one considers $n$ point masses $(r_1,m_1),...,(r_n,m_n)$ in space, then the force acting on
the $i$-th point is
$$F_i = \sum_{i \neq j} F_{ij}$$
where
$$F_{ij}=\frac{\gamma m_i m_j}{|r_{ij}|^3} r_{ij}, \; r_{ij}=r_i-r_j, \; \gamma = const > 0.$$
This is the law of universal gravitation (attraction).

d) The resistance force acting on a body moving rapidly in air is proportional to the square of its velocity (Stoke's law).
Accordingly, the equation describing the motion of a body falling in air is: $$m \ddot z = mg - c \dot z^2, \; c > 0.$$
One can show that the limit $\lim\limits_{t \rightarrow \infty} v(t)$ always exists and equals $\sqrt{\frac{mg}{c}}$, regardless
of the initial state of the body.

The determinacy principle holds also in relativistic mechanics. Newton's classical mechanics is distinguished from relativistic
mechanics by Galileo's principle of relativity.

\ESect{Text 3}
\textbf{The Principle of Relativity.} The direct product $\mathbb{E}^3 \times \mathbb{R}\{t\}$ (space-time) carries a natural
structure of affine space. The Galilean group is the group of all affine transformations of $\mathbb{E}^3 \times \mathbb{R}$ which
preserve time intervals and which for every fixed $t \in \mathbb{R}$ are isometries of $E$. Thus, if $g: (s,t) \mapsto (s',t')$
is a Galilean transformation, then

1) $t_\alpha - t_\beta = t_\alpha' - t_\beta'$,

2) if $t_\alpha=t_\beta$ then $|s_\alpha - s_\beta| = |s_\alpha' - s_\beta'|$.

Obviously, the Galilean group acts on $\mathbb{R}^3\{t\} \times \mathbb{R}\{t\}$. We mention three examples of Galilean
transformations of this space. First, uniform motion with velocity $v$:
$$g_1(r,t)=(r+tv,t).$$
Next, translation of the reference point (origin) in space-time:
$$g_2(r,t)=(r+x,t+a).$$
Finally, rotation of the coordinate axes:
$$g_3(r,t)=(Gr,t),$$
where $G: \mathbb{R}^3 \rightarrow \mathbb{R}^3$ is an orthogonal transformation.

Proposition 1. Every Galilean transformation $g: \mathbb{R}^3 \times \mathbb{R} \rightarrow \mathbb{R}^3 \times \mathbb{R}$ can
be uniquely represented as a composition (product) $g_1 \circ g_2 \circ g_3$ of transformations of the type indicated above.

Let us introduce in $\mathbb{E}^3$ a `fixed' coordinate system (reference frame): we fix a point $0 \in \mathbb{E}^3$ and pick
three mutually orthogonal axes through 0. Every Galilean transformation takes this coordinate system into a new coordinate system
which is in uniform rectilinear motion with respect to the original system. Such coordinate systems are called inertial.

The action of the Galilean group on $\mathbb{E}^3 \times \mathbb{R}$ extends to an action on
$\mathbb{E}^3 \times \ldots \times \mathbb{E}^3 \times \mathbb{R}$ by the rule: if $g: (s,t) \rightarrow (s',t')$, then
$g: (s_1,\ldots,s_n,t) \rightarrow (s_1',\ldots,s_n', t')$.

The principle of relativity states that Newton's equations, written in inertial systems, are invariant with respect to the
Galilean transformation group.

This principle imposes a series of conditions on the right-hand side of Newton's equation, written in an inertial coordinate
system. Thus, since among the Galilean transformations there are the time translations, the forces do not depend on the time $t$:
$$m_i\ddot r_i = F_i(r, \dot r), \; 1 \leqslant i \leqslant n.$$

Forces that do depend on $t$ may arise in Newtonian mechanics only in simplified models of motion.

Translations in three-dimensional space $\mathbb{E}^3$ are also Galilean transformations. From the homogeneity of $\mathbb{E}^3$
it follows that in inertial coordinate systems forces can depend only on the relative coordinates $r_k-r_l$. Also, from the
invariance of Newton's equations with respect to the subgroup of uniform motions $g_1$, it follows that forces can depend
only on the relative velocities of the points:
$$m_i \ddot r_i = F_i(r_k-r_l, \dot r_k - \dot r_l), \; i,k,l=1,\ldots,n.$$

Finally, from the isotropy of $\mathbb{E}^3$ (that is, the invariance under the subgroup of rotations $g_3$) it follows that
$$F(Gr,G \dot r)= GF(r,\dot r).$$

If a mechanical system consists of only one point, then its motion in any inertial coordinate system is uniform and rectilinear.
In fact, in this case the force $F$ does not depend on $t, r, \dot r$, and is invariant under rotations. Consequently, $f \equiv 0$.

If the given system consists of two points, then the forces $F_1$ and $F_2$ acting on these points are directed along the
straight line connecting them. Moreover, according to the principle asserting the equality of action and reaction, $F_1 = -F_2$.
This principle, which is independent of the principle of relativity, leads to the general notions of forces of interaction and
closed mechanical system. Thus, a system of $n$ material points $(r_i, m_i), i = 1,...,n$, on which the forces $F$, act, is
said to be closed if
$$F_i = \sum_{
\begin{smallmatrix}
1 \leqslant j \leqslant n \\
i \neq j \end{smallmatrix}}
F_{ij}, \; F_{kl} = -F_{lk}.$$

The vector $F_{ij}$ is called the force with which the $j$-th point acts on the $i$-th. An important example of interaction is
universal gravitation. We note that if a system consists of three material points, then from the principle of relativity it
follows that the forces acting on the points lie in the plane which contains them.

Among the examples of laws of motion given above, only universal gravitation is Galilean-invariant. If, however, in a system of
material points interacting gravitationally, one of the masses is infinitesimally small (say, a speck of dust in the Solar
system), then its influence on the motion of the other points can be neglected. This leads to a `restricted' problem (with
many important applications in astronomy) for which Galileo's principle of relativity is no longer valid. All laws of motions,
encountered in Newtonian mechanics, which are not Galilean-invariant, are obtained from invariant laws by making similar
simplifying assumptions.

\ESect{Text 4}
In 1905 Albert Einstein proposed that the whole concept of the ether and `absolute' motion through it was sheer nonsense.
With amazing insight he discarded the ether as superfluous and, instead, offered a radical new approach based on two fundamental
postulates:

I. Physical laws are equally valid in all inertial reference systems.

II. The speed of light is the same for all observers regardless of any relative motion of the source and observer.

These postulates form the basis of the special theory of relativity.

The first postulate is an extension of earlier discussions about inertial reference systems to include all physical laws,
not just Newton's laws of motion. Einstein had in mind particularly the laws of electrodynamics. In his own words
`--- the unsuccessful attempts to discover any motion of the earth relative to the light medium', suggest that `the
phenomena of electrodynamics as well as of mechanics possess no properties corresponding to the idea of absolute rest'.
Einstein went on in the same paragraph of his famous work to assert the second, and more remarkable of the two postulates
`--- and also introduce another postulate, which is only apparently irreconcilable with the former, namely, that light
is always propagated in empty space with a definite velocity $c$ which is independent of the state of motion of the emitting body.'

\EUnit
\ESect{Reading}

\ETask{I}{Pre-reading questions:}
1. What fundamental concepts of mechanics do you know?

2. Try to explain the difference between scalar and vector quantities. Give some examples.

\ETask{II}{Read the text. Make a list of mathematical and mechanical terms. If necessary use your dictionary to check
their meaning and pronunciation.}

\ESect{Text}
In any scientific theory, and in mechanics in particular, it is necessary to begin with certain primitive concepts. It is
also necessary to make a certain number of reasonable assumptions. Two of the most basic concepts are space and time. In our
initial study of the science of motion, mechanics, we shall assume that the physical space of ordinary experience is adequately
described by the three-dimensional mathematical space of Euclidean geometry. And with regard to the concept of time, we shall
assume that an ordered sequence of events can be measured on a uniform absolute time scale. We shall further assume that space
and time are distinct and independent entities. Later, when we study the theory of relativity, we shall reexamine the concepts
of space and time and we shall find that they are not absolute and independent. However, this is a matter to which we shall
return after we study the classical foundations of mechanics.

In order to define the position of a body in space, it is necessary to have a reference system. In mechanics we use a coordinate
system. The basic type of coordinate system for our purpose is the Cartesian or rectangular coordinate system, a set of three
mutually perpendicular straight lines or axes. The position of a point in such a coordinate system is specified by three numbers
or coordinates, $x$, $y$, and $z$. The coordinates of a moving point change with time; that is, they are functions of the
quantity $t$ as measured on our time scale.

A very useful concept in mechanics is the particle, or mass point, an entity that has mass but does not have spatial extension.
Strictly speaking the particle is an idealization that does not exit --- even an electron has a finite size --- but the idea
is useful as an approximation of a small body, or rather, one whose size is relatively unimportant in particular discussion.
The earth, for example, might be treated as a particle in celestial mechanics.

\textbf{Physical Quantities and Units}

The observational data of physics are expressed in terms of certain fundamental entities called physical quantities --- for
example, length, time, force, and so forth. A physical quantity is something that can be measured quantitatively in relation
to some chosen unit. When we say that the length of a certain object is, say 7 in., we mean that the quantitative measure
7 is the relation (ratio) of the length of that object to the length of the unit (1 in.). It has been found that it is
possible to define all of the unit physical quantities of mechanics in terms of just three basic ones, namely length, mass and
time.

\textbf{The Unit of Length}

The standard unit of length is the meter. The meter was formerly the distance between two scratches on a platinum bar kept
at the International Bureau of Metric Standards, Sevres, France. The meter is now defined as the distance occupied by
exactly 1,650,763.73 wavelengths of light of the orange spectrum line of the isotope krypton 86.

\textbf{The Unit of Mass}

The standard unit of mass is the kilogram. It is the mass of a cylinder of platinum iridium also kept at the International
Bureau.

\textbf{The Unit of Time}

The basic unit for measurement of time, the second, was formerly defined in terms of the earth's rotation. But, like the
meter, the second is now defined in terms of a specific atomic standard. The second is, by definition, the amount of time
required for exactly 9,192,631,770 oscillations of a particular atomic transition of the cesium isotope of mass number 133.

The above system of units is called the \textit{mks} system. The modern atomic standards of length and time in this system are not
only more precise than the former standards, but they are also universally reproducible and indestructible. Unfortunately,
it is not at present technically feasible to employ an atomic standard of mass.

Actually, there is nothing particularly sacred about the physical quantities length, mass, and time as a basic set to
define units. Other sets of physical quantities may be used. The so-called gravitational systems use length, force, and time.

In addition to the \textit{mks} system, there are other systems in common use, namely, the \textit{cgs}, or centimeter-gram-second, system,
and the \textit{fps}, or foot-pound-second, system. These latter two systems may be regarded as secondary to the \textit{mks} system because
their units are specifically defined fractions of the \textit{mks} units:

\begin{center}
1 cm = $10^{-2}$ m

1 g = $10^{-3}$ kg

1 ft = $0.3048$ m

1 lb = $0.4536$ kg
\end{center}

\textbf{Scalar and Vector Quantities}
A physical quantity that is completely specified by a single magnitude is called a scalar. Familiar examples of scalars are
density, volume, and temperature. Mathematically, scalars are treated as ordinary real numbers. They obey all the regular
rules of algebraic addition, subtraction, multiplication, division, and so on.

There are certain physical quantities that possess a directional characteristic, such as a displacement from one point in
space to another. Such quantities require a direction and a magnitude for their complete specification. These quantities are
called vectors if they combine with each other according to the parallelogram rule of addition as discussed below.\footnote{An
example of the directed quantity that does not obey the rule for addition is a finite rotation of an object about a given axis.
The reader can readily verify that two successive rotations about different axes do not produce the same effect as a single
rotation determined by the parallelogram rule. For the present we shall not be concerned with such non-vector directed quantities,
however.} Besides displacement in space, other familiar examples of vectors are velocity, acceleration, and force. The vector
concept and the development of a whole mathematics of vector quantities have proved indispensable to the development of the
science of mechanics.

\ETask{III}{Comprehension questions:}
In the text they say:

1. `A very useful concept in mechanics is the particle or mass point'. What does this concept mean and why is it so useful
in mechanics?

2. `The observational data of physics are expressed in terms of certain fundamental entities called physical quantities.'
What is a physical quantity?

\ETask{IV}{What do the words in italics refer to? Check against the text.}

1. However, this is a matter to which we shall return after we study the classical foundations of mechanics.

2. The basic type of coordinate system for our purpose is the Cartesian or rectangular coordinate system.

3. ...but the idea is useful as an approximation of a small body, or rather, one whose size is relatively unimportant in
a particular discussion.

4. The above system of units is called the \textit{mks} system.

5. The modern atomic standards of length and time in this system are not only more precise than the former, but they are
also universally reproducible and indestructible.

6. These latter two systems may be regarded as secondary to the \textit{mks} system.

7. These quantities are called vectors.

\ESect{Vocabulary}

\ETask{V}{Give the Russian equivalents of the following words and expressions:}
in any scientific theory; in particular; with regard to...; in order to...; strictly speaking; and so forth; in relation to...;
namely; by definition; the above system; the former standards; at present; in addition to...; these latter two systems; and
so on; with each other; according to...; as discussed below; the same effect; for the present; we shall not be concerned with...

\ETask{VI}{These words and expressions have the same meaning. Match a line in $A$ with a line in $B$.}
\ETypeWr{
A~~~~~~~~~~~~~~~~~~~~~~~~B

certain~~~~~~~~~~~~~~~~~~besides

and so on~~~~~~~~~~~~~~~~to treat smth

in addition to~~~~~~~~~~~as

necessary~~~~~~~~~~~~~~~~by means of

because~~~~~~~~~~~~~~~~~~in relation to

reference system~~~~~~~~~some

to be concerned with~~~~~to define

in terms of~~~~~~~~~~~~~~to consider

with regard to~~~~~~~~~~~notion

to require~~~~~~~~~~~~~~~indispensable

to specify~~~~~~~~~~~~~~~to demand

concept~~~~~~~~~~~~~~~~~~coordinate system

to regard~~~~~~~~~~~~~~~~and so forth
}

\ETask{VII}{Prefixes \textit{un}, \textit{in}, \textit{il}, \textit{ir}, \textit{im} change the meaning of adjectives. Give antonyms of the following words. Translate them
into Russian. Consult your dictionary if necessary:}
necessary, certain, reasonable, adequate, dependent, finite, important, possible, destructible, regular, complete, dispensable.

\ETask{VIII}{Translate the following combinations of words:}
time scale; coordinate system; mass point; unit physical quantities; standard unit; platinum bar; spectrum line; cesium
isotope; \textit{mks} unit; parallelogram rule; vector concept; vector quantity.

\ETask{IX}{Make these nouns plural. Use your dictionary if necessary:}
datum, axis, equilibrium, spectrum, maximum, minimum, phenomenon, polyhedron, analysis, basis, crisis, vertex, hypothesis,
index, phasis, calculus, focus, genius, locus, nucleus, radius, rhombus, means, apparatus, news, series.

\ETask{X}{Supply prepositions where necessary. Check against the text.}
1. It is necessary to begin --- certain primitive concepts.

2. And --- regard --- the concept --- time, we shall assume that an ordered sequence --- events can be measured --- a uniform
absolute time scale.

3. We mean that the quantitative measure 7 is the relation (ratio) --- the length --- that object --- the length --- the unit (1 in.).

4. --- addition --- the \textit{mks} system, there are other systems --- common use.

5. These quantities are called vectors if they combine --- each other according --- the parallelogram rule --- addition.

6. An example --- a directed quantity that does not obey --- the rule --- addition is a finite rotation --- an object --- a given axis.

7. --- the present we shall not be concerned --- such non-vector directed quantities.

\ESect{Grammar}

\ETask{XI}{Make these sentences interrogative. Begin with the words in brackets.}
1. An ordered sequence of events can be measured on a uniform absolute time scale. (How)

2. It is necessary to have a reference system. (What)

3. In mechanics we use a coordinate system. (What)

4. The coordinates of a moving point change with time. (What)

5. These latter two systems may be regarded as secondary to the \textit{mks} system because their units are specifically defined
fractions of the \textit{mks} units. (Why)

6. They obey all the regular rules of algebraic addition, subtraction, multiplication, division, and so on. (What)

7. Such quantities require a direction and a magnitude for their complete specification. (What)

8. Other sets of physical quantities may be used. (What)

\ETask{XII}{Make these sentences interrogative and negative if possible.}

1. There is nothing particularly sacred about the physical quantities length, mass and time as a basic set to define units.

2. There are other systems in common use.

3. There are certain physical quantities that possess a directional characteristic.

4. There can be no confusion when the null vector is denoted by a `zero'.

5. There are many experiments that can be devised to answer this questions.

6. There are some external forces, acting on the respective particles.

7. There may be internal forces of interaction between any two particles of the system.

\ETask{XIII}{Put the words given in brackets into Present Perfect or Past Indefinite. Check against the text.}
1. The meter (to be) formally the distance between two scratches on a platinum bar kept at the International Bureau of
metric Standards.

2. It (to find) that it is possible to define all of the unit physical quantities of mechanics in terms of just three basic
ones, namely length, mass and time.

3. The basic unit for measurement of time, second, (to define) formerly in terms of the earth's rotation.

4. The vector concept and the development of a whole mathematics of vector quantities (to prove) indispensable to the
development of the science of mechanics.

\ETask{XIV}{Supply appropriate forms of the verbs given in brackets (Present and Future tenses).}

1. Later, when we (to study) the theory of relativity, we (to reexamine) the concepts of space and time and we (to find)
that they are not absolute and independent.

2. This is a matter to which we (to return) after we (to study) the classical foundations of mechanics.

3. If our discussion (to limit) to vectors in a plane, only two components (to be necessary).

4. There will be no confusion when the null vector ( to denote) by a `zero'.

5. The vector sum will be defined in such a way even if the vectors (not to have) a common point.

6. If the system (to be) a rigid body and if the sum of all the external forces (to vanish), then the center of mass, if
initially at rest, will remain at rest.

7. The light wave will travel outward in all directions if speed (to equal) $c$.

\ETask{XV}{Supply the comparative or superlative forms of the adjectives given in brackets.}
1. Two of the (basic) concepts are space and time.

2. The modern atomic standards of length and time are (precise) than the former standards.

3. The (simple) type of rigid-body motion is that in which the body is constrained to rotate about a fixed axis.

4. In a (late) chapter we shall study alternative ways of expressing the laws of motion in (advanced) equations of
Lagrange and Hamilton.

5. A (good) system would be one using the center of the earth, the center of the sun, and a distant star as reference points.

\ETask{XVI}{Complete the sentences with proper forms of verbals.}

1. We shall assume that an (to order) sequence of events can be measured on a uniform absolute time scale.

2. The coordinates of a (to move) point change with time.

3. They are functions of the quantity t as (to measure) on our time scale.

4. Strictly (to speak), the particle is an idealization that does not exist.

5. A physical quantity is something that can be measured quantitatively in relation to some (to choose) unit.

6. Kilogram is the mass of a cylinder of platinum iridium (to keep) at the International Bureau.

7. There are certain physical quantities (to possess) a directional characteristic.

8. An example of a (to direct) quantity (not to obey) the rule for addition is a finite rotation of an object about a (to give) axis.

\ETask{XVII}{Supply the appropriate forms (Active or Passive) of the verbs given in brackets.}

1. We (to assume) that an ordered sequence of events (can, to measure) on a uniform absolute time scale.

2. The coordinates of a moving point (to change) with time; that is, they are functions of the quantity $t$ which (to measure)
on our time scale.

3. The earth (may, to treat) as a particle in celestial mechanics.

4. The observational data of physics (to express) in terms of certain fundamental entities that (to call) physical quantities.

5. Other sets of physical quantities (may, to use).

6. These two systems (may, to regard) as secondary to the \textit{mks} system.

7. A physical quantity that (to specify) completely by a single magnitude (to call) a scalar.

8. Mathematically, scalars (to treat) as ordinary real numbers.

9. These quantities (to call) vectors if they (to combine) with each other according to the parallelogram rule of addition as
it (to discuss) below.

10. The vector concept and the development of a whole mathematics of vector quantities (to prove) indispensable to the development
of the science of mechanics.

11. The reader (can, to verify) readily that two successive rotations about different axes (not to produce) the same effect as a
single rotation that (to determine) by the parallelogram rule.

12. We (not to concern) with non-vector directed quantities.

\ESect{Writing}

\ETask{XVIII}{Look through the text again to produce a set of notes on: The Cartesian coordinate system. Then reconstitute
your notes in the form of a short paragraph. Remember to use your own words. Do not refer to the original text but only your
notes when writing the paragraph.}

\ESect{Supplementary Texts}

\ESect{Text 1}
\textbf{Scalars and vectors}. Pure numbers and physical quantities which do not require direction in space for their complete
specification are called \textit{scalar quantities}, or simply \textit{scalars}. Volume, density, mass and energy are familiar examples. Fluid
pressure is also a scalar. The thrust on an infinitesimal plane area due to fluid pressure is, however, not a scalar, for to
describe this thrust completely, the direction in which it acts must also be known.

A \textit{vector quantity}, or simply a \textit{vector}, is a quantity which needs for its complete specification both magnitude
and direction, and which obeys the parallelogram law of composition (addition), and certain laws of multiplication which will be
formulated later. Examples of vectors are readily furnished by velocity, linear momentum and force. Angular velocity and angular
momentum are also vectors, as is proved in books on Mechanics.

A vector can be represented completely by a straight line drawn in the direction of the vector and of appropriate magnitude to
some chosen scale. The sense of the vector in this straight line can be indicated by an arrow.

In some cases a vector must be considered as \textit{localized} in a line. For instance, in calculating the moment of a force,
it is clear that the position of the line of action of the force is relevant.

In many cases, however, we shall be concerned with \textit{free} vectors, that is to say, vectors which are completely specified
by their direction and magnitude, and which may therefore be drawn in any convenient positions. Thus if we wish to find only the
magnitude and direction of the resultant of several given forces, we can use the polygon of forces irrespectively of the actual
positions in space of the lines of action of the given forces.

We shall represent a vector by a single letter in clarendon (heavy) type and its magnitude by the corresponding letter in italic
type. Thus if \textbf{q} is the velocity vector, its magnitude is $q$, the speed. Similarly the angular velocity $\omega$ has
the magnitude $\omega$.

A \textit{unit vector} is a vector whose magnitude is unity. Any vector can be represented by a numerical (scalar) multiple of
a unit vector parallel to it. Thus if $i_a$ is a unit vector parallel to the vector \textbf{a}, we have $\textbf{a}=ai_a$.

\ESect{Text 2}
\textbf{The scalar product of two vectors}. Let \textbf{a}, \textbf{b} be two vectors of magnitudes $a$, $b$, represented by
the lines $OA$, $OB$ issuing from the point $O$.

Let $\theta$ be the angle between the vectors, i.e. the angle $AOB$ measured positively in the sense of minimum rotation
from \textbf{a} to \textbf{b}.

The \textit{scalar product} of the vectors is then \textbf{ab} and is defined by the relation
$$\textbf{ab}=ab \cos \theta.$$

The scalar product is a scalar and is measured by the product $OA \cdot OM$, where $M$ is the projection of $B$ on $OA$, so
that $OA = a$, $OM = b \cos \theta$. It is clear from the definition that
$$\textbf{ba} = ba \cos(-\theta)=ab \cos \theta = ab,$$
so that the order of the two factors is irrelevant.

When the vectors are perpendicular, $\cos \theta = 0$, so that $\textbf{ab} = 0$. Conversely this relation implies either
that $a$, $b$ are perpendicular, or that $a = 0$, or that $b = 0$.

If $ab = 0$, where \textbf{b} is an arbitrary vector, then $a = 0$, for \textbf{a} cannot be perpendicular to every vector
\textbf{b}.

If $\theta$ is an obtuse angle, the scalar product is negative.

If $\textbf{i}_a$ is a unit vector, then $\textbf{i}_a\textbf{b} = b \cos \theta$, which is the resolved part of the vector \textbf{b}
along the direction of any vector which is parallel to $\textbf{i}_a$.

If $\textbf{i}_a$, $\textbf{i}_b$ are both unit vectors, then $\textbf{i}_a\textbf{i}_b = \cos \theta$, which is the cosine of
the angle between any two vectors parallel to $\textbf{i}_a$ and $\textbf{i}_b$.

If the point of application of a force $F$ moves with velocity $v$, the rate at which the force is doing work is the scalar
product $Fv$.

\EUnit
\setcounter{equation}{0}
\ESect{Reading}
\ETask{I}{Pre-reading questions:}
1. What is Pressure?

2. What does Hydrodynamics deal with?

\ETask{II}{Read the text. Make a list of terms used in mechanics. If necessary use your dictionary to check their meaning and
pronunciation.}

\ESect{Text}
\textbf{Pressure}. Consider a small plane of infinitesimal area $d\sigma$, whose centroid is $P$, drawn in the fluid, and draw
the normal $PN$ on one side of the area which we shall call the positive side. The other side will be called the negative side.

We shall make the hypothesis that the mutual action of the fluid particles on the two sides of the plane can, at a given
instant, be represented by two equal but opposite forces $p d \sigma$ applied at $P$, each force being a push not a pull, that
is to say, the fluid on the positive side pushes the fluid on the negative side with a force $p d \sigma$.

Experiment shows that in a fluid at rest these forces act along the normal. In a real fluid in motion these forces make an
angle $\varepsilon$ with the normal (analogous to the angle of friction). When the viscosity is small, as in the case of air
and water, $\varepsilon$ is very small. In an inviscid fluid which can exert no tangential stress $\varepsilon = 0$, and in
this case $p$ is called the pressure at the point $P$.

Pressure is a scalar quantity, i.e. independent of direction. The dimensions of pressure in terms of measure ratios $M$, $L$, $T$
of mass length and time are indicated by $ML^{-1}T^{-2}$.

The thrust on an area $d \sigma$ due to pressure is a force, that is a vector quantity, whose complete specification requires
direction as well as magnitude. Pressure in a fluid in motion is a function of the position of the point at which it is measured
and of the time. When the motion is steady the pressure may vary from point to point, but at a given point it is independent of
the time.

It should be noted that $p$ is essentially positive.

\textbf{Bernoulli's theorem} (special form).

In the steady motion of a liquid the quantity
$$\frac{p}{\rho}+\frac{1}{2}q^2+gh$$
has the same value at every point of the same streamline where $p$, $\rho$, $q$ are the pressure, density, and speed, $g$ is
the acceleration due to gravity, and $h$ is the height of the point considered above a fixed horizontal plane.

\textbf{Hydrodynamic pressure.} In the steady motion of a liquid Bernoulli's theorem enables us to elucidate the nature of
pressure still further. In a liquid at rest there exists at each point a hydrostatic pressure $p_H$, and the principle of
Archimedes states that a body immersed in the fluid is buoyed up by a force equal to the weight of the liquid which it displaces.
The particles of the liquid are themselves subject to this principle and are therefore in equilibrium under the hydrostatic
pressure $p_H$ and the force of gravity. It follows at once that $\frac{p_H}{\rho}+gh$ is constant throughout the liquid.
When the liquid is in motion the buoyancy principle still operates, so that if we write
$$p=p_D+p_H,$$
Bernoulli's theorem gives
$$\frac{p_D}{\rho}+\frac{1}{2}q^2+\frac{p_H}{\rho}+gh=C,$$
and therefore
\Eqn{\frac{p_D}{\rho}+\frac{1}{2}q^2=C'}
where $C'=C-(\frac{p_H}{\rho}+gh)$ is a new constant.

Now (1) is the form which Bernoulli's theorem would assume if the force of gravity were non-existent.

The quantity $p_D$ may be called the hydrodynamic pressure, or the pressure due to motion. This pressure $p_D$ measures the force
with which two fluid particles are pressed together (for both are subject to the same force of buoyancy). It will be seen that
the knowledge of the hydrodynamic pressure will enable us to calculate the total effect of the fluid pressure on an immersed body,
for we have merely to work out the effect due to $p_D$ and then add the effect due to $p_H$, which is known from the principles
of hydrostatics. This is a very important result, for it enables us to neglect the external force of gravity in investigating
many problems, due allowance being made for this force afterwards.

It is often felt that hydrodynamical problems in which external forces are neglected or ignored are of an artificial and
unpractical nature. This is by no means the case. The omission of external forces is merely a device for avoiding unnecessary
complications in our analysis.

It should therefore be borne in mind that when we neglect external forces we calculate in effect the hydrodynamic pressure.

We also see from (1) that the hydrodynamic pressure is greatest where the speed is least, and also that the greatest
hydrodynamic pressure occurs at points of zero velocity.

It should be observed, however, that the device of introducing hydrodynamic pressure can be justified only when the boundaries
of the fluid are fixed, for only in these conditions is the hydrostatic pressure constant at a given point. When the liquid has
free surfaces which undulate, the hydrostatic pressure at a fixed point will vary, and we must consider the total pressure.

In the case of compressible fluids the pressure due to motion is usually called aerodynamic pressure.

\ETask{III}{Comprehension questions:}
In the text they say:

1. `...in this case $p$ is called the pressure at the point $P$.' What case do they mean?

2. It should be observed, however, that the device of introducing hydrodynamic pressure can be justified only when the
boundaries of the fluid are fixed.

Explain this statement.

\ETask{IV}{What do the words in italics refer to? Check against the text.}

1. Experiment shows that in a fluid at rest \textit{these} forces act along the normal.

2. The particles of the liquid are themselves subject to \textit{this principle}.

3. \textit{This pressure} $p_D$ measures the force with which two fluid particles are pressed together (for both are subject to the
same force of buoyancy).

4. \textit{This} is a very important result.

5. \textit{This} is by no means the case.

\ESect{Vocabulary}

\ETask{V}{Give the Russian equivalents of the following expressions:}
the two sides; that is to say; in terms of; due to; as well as; it should be noted; at rest; it follows that; at once; both are
subject to the same force; the forces are of artificial nature; this is not the case; by no means; it should be borne in mind;
in effect; it should be observed.

\ETask{VI}{The words in the charts below have all appeared in the text. Give the other parts of speech, their translation and
pronunciation. Use your dictionary if necessary.}
\ETypeWr{
================================================

verb~~~~~~~~~~~~~~noun

================================================

to represent~~~~~~-

-~~~~~~~~~~~~~~~~~push

-~~~~~~~~~~~~~~~~~pull

-~~~~~~~~~~~~~~~~~force

to exist~~~~~~~~~~-

to state~~~~~~~~~~-

to occur~~~~~~~~~~-

-~~~~~~~~~~~~~~~~~omission

-~~~~~~~~~~~~~~~~~weight

to assume~~~~~~~~~-

================================================

noun~~~~~~~~~~~~~~adjective

================================================

quantity~~~~~~~~~~-

-~~~~~~~~~~~~~~~~~analogous

-~~~~~~~~~~~~~~~~~independent

-~~~~~~~~~~~~~~~~~long

nature~~~~~~~~~~~~-

condition~~~~~~~~~-

================================================

verb~~~~~~~~~~~~~~noun~~~~~~~~~~~~~~adjective

================================================

to consider~~~~~~~-~~~~~~~~~~~~~~~~~-

-~~~~~~~~~~~~~~~~~-~~~~~~~~~~~~~~~~~direct

-~~~~~~~~~~~~~~~~~-~~~~~~~~~~~~~~~~~equal

to buoy~~~~~~~~~~~-~~~~~~~~~~~~~~~~~-

to vary~~~~~~~~~~~-~~~~~~~~~~~~~~~~~-

-~~~~~~~~~~~~~~~~~specification~~~~~-

-~~~~~~~~~~~~~~~~~-~~~~~~~~~~~~~~~~~compressible
}

\ETask{VII}{Supply prepositions. Then check against the text.}
1. Experiment shows that --- a fluid --- rest these forces act --- the normal.

2. --- a real fluid --- motion these forces make an angle $\varepsilon$ --- the normal (analogous --- the angle --- friction).

3. When the motion is steady the pressure may vary --- point --- point, but --- a given point it is independent --- the time.

4. The principle --- Archimedes states that a body immersed --- the fluid is buoyed up --- a force equal --- the weight --- the
liquid which it displaces.

5. The particles --- the liquid are themselves subject --- this principle and are therefore --- equilibrium --- the
hydrostatic pressure $p_H$ and the force --- gravity.

6. This is --- no means the case.

7. It should be borne --- mind that when we neglect external forces we calculate --- effect the hydrodynamic pressure.

8. --- the case --- compressible fluids the pressure due --- motion is usually called aerodynamic pressure.

\ETask{VIII}{Pay attention to the meaning of \textbf{for} in the following sentences. Think of some other examples.}
1. This is a very important result, \textit{for} it enables us to neglect the external force of gravity in investigating many problems.

2. The omission of external forces is merely a device \textit{for} avoiding unnecessary complications in our analysis.

3. There is no need \textit{for} us to prove this theorem.

4. For many years mathematicians have looked \textit{for} the solution of the problem.

\ETask{IX}{Translate the following combinations of nouns.}
fluid particle; vector quantity; buoyancy principle; fluid pressure; zero velocity.

\ESect{Grammar}
\ETask{X}{Put in the proper forms (Present Participle or Past Participle) of the verbs given in brackets.}

1. Consider a small plane of infinitesimal area $d \sigma$, whose centroid is $P$ (to draw) in the fluid.

2. The mutual action of the fluid particles on the two sides of the plane can, at a (to give) instant, be represented by
two equal but opposite forces $p d \sigma$ (to apply) at $P$.

3. A body (to immerse) in the fluid is buoyed up by a force equal to the weight of the liquid which it displaces.

4. It enables us to neglect the external force of gravity (to investigate) many problems.

5. When the liquid has free surfaces which undulate, the hydrostatic pressure at a (to fix) point will vary.

6. In an inviscid fluid (to exert) no tangential stress $\varepsilon = 0$.

7. In the above discussion there is nothing (to show) that the pressure $p$ is independent of the orientation of the element
$d \sigma$ (to use) in defining $p$.

\ETask{XI}{Put in the modals can, may, must, have to, should and proper forms of the verbs in brackets. Alternatives are possible.}
1. When the motion is steady the pressure --- (to vary) from point to point.

2. It --- (to note) that $p$ is essentially positive.

3. We -- (to be) content with a definition of density given above.

4. The mutual action of the fluid particles on the two sides of the plane --- (to represent) by two equal but opposite forces
$p d \sigma$ applied at $P$.

5. The quantity $p_D$ --- (to call) the hydrodynamic pressure.

6. We --- (to work out) the effect due to $p_D$.

7. It --- (to bear) in mind that when we neglect external forces we calculate in effect the hydrodynamic pressure.

8. It --- (to observe) that the device of introducing hydrodynamic pressure --- (to justify) only when the boundaries of the
fluid are fixed.

9. When the liquid has free surfaces which undulate, the hydrostatic pressure at a fixed point will vary, and we --- (to
consider) the total pressure.

\ETask{XII}{Supply articles \textbf{a/an, the, ---}.}

1. Draw --- normal $PN$ on one side of --- area which we shall call --- positive side --- other side will be called ---
negative side.

2. We shall make --- hypothesis that --- mutual action of --- fluid particles on --- two sides of --- plane can, at --- given
instant, be represented by --- two equal but opposite forces.

3. In --- fluid at --- rest these forces act along --- normal.

4. Pressure in --- fluid in --- motion is --- function of --- position of --- point at which it is measured and of --- time.

5. Both are subject to --- same force of --- buoyancy.

6. It will be seen that --- knowledge of --- hydrodynamic pressure will enable us to calculate --- total effect of --- fluid
pressure on --- immersed body.

7. This is --- very important result, for it enables us to neglect --- external force of --- gravity in --- investigating many problems.

8. This is by no means --- case.

\ETask{XII}{Put in the proper comparative or superlative forms.}

1. In the steady motion of a liquid Bernoulli's theorem enables us to elucidate the nature of pressure still (far).

2. We also see that the hydrodynamic pressure is (great) where the speed is (little).

3. The depth and breadth increase together if and only if, $\omega^2<gh$, i.e., if $u$ is (little) than the speed of
propagation of long waves in the channel.

4. The gravitational field is clearly (important) of conservative fields of force.

5. Then Bernoulli's theorem will take a (general) form.

6. A (precise) definition of viscosity will be given (late).

\ETask{XIV}{Supply the correct forms of the verbs. Use the Subjunctive mood. Translate the sentences into Russian.}
1. This is the form which Bernoulli's theorem (to assume) if the force of gravity (to be) non-existent.

2. If the breadth of the channel (to vary) slightly, there (to be) a small consequent change in $u$.

3. If we (to denote) by $\Omega$ the potential energy per unit mass in a conservative field, Bernoulli's theorem (to take)
the more general form, and the same method of proof (can, to use).

4. If the motion (to be) steady, the path lines (to coincide) with the streamlines.

5. If we (to draw) the streamlines through each point of a closed curve we (to obtain) a stream tube.

6. If we (to place) an obstacle $A$ in the middle of the tube, the flow in the immediate neighbourhood of $A$ (to derange),
but at a great distance either upstream or downstream the flow (to be) undisturbed.

7. If we (to impose) on the whole system a uniform velocity $V$ in the direction opposite to that of the current, the liquid
at a great distance (to reduce) to rest and $A$ (to move) with uniform velocity $V$.

8. (to be) the vortex alone in the otherwise undisturbed fluid, the velocity at the point (to have) the value in question.

\ETask{XV}{Pay attention to the Absolute Participle construction. Translate the sentences into Russian.}
1. We shall make the hypothesis that the mutual action of the fluid particles on the two sides of the plane can, at a given
instant, be represented by two equal but opposite forces applied at $P$, each force being a push not a pull.

2. This is a very important result, for it enables us to neglect the external force of gravity in investigating many
problems, due allowance being made for this force afterwards.

3. The vortex lines being straight and parallel, all vortex tubes are cylindrical.

4. The motion being dependent on the time, the configuration of the stream tubes and filaments changes from instant to instant.

5. The point of application of a force $F$ moving with velocity $V$, the rate at which the force is doing work is the
scalar product $Fv$.

6. Another important example of this principle occurs when the surface $S$ separates not two different fluids, but two regions
of the same fluid, there being a discontinuity of tangential velocity at the surface $S$.

\ETask{XVI}{Put questions to the following statements. Translate them into Russian.}
1. In a liquid at rest there exists at each point a hydrostatic pressure.

2. In the above discussion there is nothing to show that the pressure $p$ is independent of the orientation of the element
$d \sigma$ used in the defining $p$.

3. In air at ordinary temperatures there are about $3 \times 10^{19}$ molecules per $cm^3$.

4. There is no flow into the region across $S$.

5. To each point of space there corresponds a scalar.

6. There cannot be two different forms of acyclic irrotational motion of a confined mass of liquid in which the boundaries
have prescribed velocities.

7. There are many experiments that can be devised to answer this question.

8. To every action there is always an equal and opposite reaction.

9. There can usually be found some relatively simple function of the coordinates, called the force function, which describes
the motion of a body.

\ESect{Writing}
\ETask{XVII}{Look through the text again to produce a set of notes on: \textbf{Hydrodynamic pressure}. Then reconstitute your notes
in the form of a paragraph. Remember to use your own words. Do not refer to the original text but only your notes when writing
the paragraph.}

\setcounter{equation}{0}
\ESect{Supplementary Texts}
\ESect{Text 1}
\textbf{Density.} If $M$ is the mass of the fluid within a closed volume $V$, we can write
\Eqn{M=V \rho_1}
and $\rho_1$ is then the average density of the fluid within the volume at that instant. In a hypothetical medium continuously
distributed we can define the density $\rho$ as the limit of $\rho_1$ when $V \rightarrow 0$.

In an actual fluid which consists of a large number of individual molecules we cannot let $V \rightarrow 0$, for at some stage
there might be no molecules within the volume $V$. We must therefore be content with a definition of density given by (1) on
the understanding that the dimensions of $V$ are to be made very small, but not so small that $V$ does not still contain a
large number of molecules. In air at ordinary temperatures there are about $3 \times 10^{19}$ molecules per $cm^3$. A sphere
of radius 0.001 cm will then contain about $10^{11}$ molecules, and although small in the hydrodynamical sense will be
reasonably large for the purposes of measuring average density.

\ESect{Text 2}
\textbf{Remarks on Bernoulli's Theorem.} The form in which the theorem has been stated is called special for two reasons.
Firstly, we have assumed the external forces to be due to gravity alone. The field of gravitational force is a conservative
field, meaning by this that the work done by the weight when a body moves from a point $P$ to another point $Q$ is independent
of the path taken from $P$ to $Q$ and depends solely on the vertical height of $Q$ above $P$. A conservative field of force
gives rise to potential energy, which is measured by the work done in taking the body from one standard position to any
other position. In order that potential energy of a unit mass at a point may have a definite meaning, it is obviously
necessary that the work done by the forces of the field should be independent of the path by which that point was reached.
The gravitational field is clearly the most important of conservative fields of force, but it is by no means the only
conceivable field of this nature; for example, an electrostatic field has the conservative property. If more generally we
denote by $Q$ the potential energy per unit mass in a conservative field, Bernoulli's theorem would take the more general form that
$$\frac{p}{\rho}+\frac{1}{2}q^2+\Omega$$
is constant along a streamline, and the same method of proof could be used.

Secondly, we have assumed the fluid to be incompressible, and of constant density. More generally, when the pressure is a
function of the density, the theorem assumes the form that
$$\int \frac{d p}{\rho}+\frac{1}{2}q^2+\Omega$$
is constant along a streamline.

\ESect{Text 3}
\textbf{Streamlines and Paths of the Particles.} A line drawn in the fluid so that its tangent at each point is in the direction
of the fluid velocity at that point is called a streamline.

When the fluid velocity at a given point depends not only on the position of the point but also on the time, the streamlines
will alter from instant to instant. Thus photographs taken at different instants will reveal a different system of streamlines.
The aggregate of all the streamlines at a given instant constitutes the flow pattern at that instant.

When the velocity at each point is independent of the time, the flow pattern will be the same at each instant and the motion is
described as steady. In this connection it is useful to describe the type of motion which is relatively steady. Such a motion
arises when the motion can be regarded as steady by imagining superposed on the whole system, including the observer, a
constant velocity. Thus when a ship steams on a straight course with constant speed on an otherwise undisturbed sea, to an
observer in the ship the flow pattern which accompanies him appears to be steady and could in fact be made so by superposing
the reversed velocity of the ship on the whole system consisting of the ship and sea.

If we fix our attention on a particular particle of the fluid, the curve which this particle describes during its motion is
called a path line. The direction of motion of the particle must necessarily be tangential to the path line, so that the path
line touches the streamline which passes through the instantaneous position of the particle as it describes its path.

Thus the streamlines show how each particle is moving at a given instant.

The path lines show how a given particle is moving at each instant. When the motion is steady, the path lines coinside with
the streamlines.

\setcounter{equation}{0}
\ESect{Text 4}
\textbf{Acyclic and Cyclic Irrotational Motion.} When the region occupied by fluid moving irrotationally is simply connected,
the velocity potential is one-valued, for the velocity potential at $P$ is defined by
\Eqn{\varphi_P=-\int_{(OAP)} q\, dr}
and this integral is the same for all paths from $O$ to $P$, for all such paths are reconcilable. Motion in which the velocity
potential is one-valued is called acyclic. Thus in a simply connected region the only possible irrotational motion is acyclic.
This result depends essentially on the possibility of joining any two paths from $O$ to $P$ by a surface lying entirely within
the fluid and then applying Stokes' theorem.

When the region is not simply connected, two paths from $O$ to $P$ can only be joined by a surface lying entirely within the
fluid when these paths are reconcilable. When the paths are not reconcilable, the inference from Stokes' theorem cannot be
made, and the velocity potential may then have more than one value at $P$, according to the path taken from $O$ to $P$.

When the velocity potential is not one-valued the motion is said to be cyclic.

In the continuous motion of a fluid the velocity at any point must be perfectly definite. Thus, even when $\varphi$ has more
than one value at a given point, $\nabla\varphi$ must be one-valued. It follows that although two paths from $O$ to $P$ may
lead to different values of $\varphi_P$, these values can only differ by a scalar $k$, such that $\nabla k = 0$, and $k$ is
therefore independent of the coordinates of $P$. This scalar $k$ may be identified with the circulation in any one of a
family of reconcilable irreducible circuits, for, if $C$ be any circuit, (1) shows that
\Eqn{circ~C=decrease~in~\varphi~on~describing~the~circuit~once.}

We shall have occasion later to consider particular types of cyclic motion. For the present we shall consider only acyclic
irrotational motion, and the general theorems which follow must be considered as applying to that type of motion only. In
that sense the regions concerned may always be considered as simply connected, but it should be remembered that acyclic motion
is also possible in multiply connected regions.

\setcounter{equation}{0}
\ESect{Text 5}
\textbf{Uniqueness Theorems.} We shall now prove some related theorems concerning acyclic irrotational motion of a liquid.
The proofs are all based on the following equivalence of the expressions for the kinetic energy,
\Eqn{\frac{1}{2}\rho\int q^2 d\,\tau=-\frac{1}{2}\rho\int\varphi \frac{\partial \varphi}{\partial n} dS,}
where the volume integral is taken throughout the fluid and the surface integral is taken over the boundary.

(I) Acyclic irrotational motion is impossible in a liquid bounded entirely by fixed rigid walls.

For $\frac{\partial \varphi}{\partial n} =0$ at every point of the boundary, and therefore $\int q^2 \, d\tau = 0$. Since $q^2$
cannot be negative, $q = 0$ everywhere and the liquid is at rest.

(II) The acyclic irrotational motion of a liquid bounded by rigid walls will instantly cease if the boundaries are brought
to rest. This is an immediate corollary to (I).

(III) There cannot be two different forms of acyclic irrotational motion of a confined mass of liquid in which the boundaries
have prescribed velocities.

For, if possible, let $\varphi_1, \varphi_2$ be the velocity potentials of two different motions subject to the condition
$\frac{\partial \varphi_1}{\partial n} = \frac{\partial \varphi_2}{\partial n}$ at each point of the boundary.

Then $\varphi = \varphi_1-\varphi_2$ is a solution of Laplace's equation and therefore represents a possible irrotational
motion in which
$$\frac{\partial \varphi}{\partial n}=\frac{\partial \varphi_1}{\partial n} - \frac{\partial \varphi_2}{\partial n} = 0.$$

Therefore, as in (I), $q = 0$ at every point; and therefore $\varphi_1 - varphi_2 = constant$, so that the motions are
essentially the same. This theorem shows that acyclic motion is uniquely determined when the boundary velocities are given.

(IV) If given impulsive pressures are applied to the boundaries of a confined mass of liquid at rest, the resulting motion,
if acyclic and irrotational, is uniquely determinate.

If possible, let $\varphi_1$ and $\varphi_2$ be velocity potentials of two different motions. The impulsive pressure which
would start the first motion is $\rho\varphi_1$, that which would start the second is $\rho\varphi_2$ , and since the
pressures are given at the boundaries
$$\rho\varphi_1=\rho\varphi_2$$
at each point of the boundary.

Therefore $\varphi=\varphi_1-\varphi_2$ is the velocity potential of a possible irrotational motion such that $\varphi = 0$ at
each point of the boundary. Therefore, from (1), $q = 0$ at each point of the liquid. It follows that $\varphi_1-\varphi_2$ is
constant and the motions are essentially the same.

(V) Acyclic irrotational motion is impossible in a liquid which is at rest at infinity and is bounded internally by fixed
rigid walls.

Since the liquid is at rest at infinity and there is no flow over the internal boundaries, the kinetic energy is still
given by (1) and the proof is therefore the same as in (I).

(VI) The acyclic irrotational motion of a liquid at rest at infinity and bounded internally by rigid walls will instantly
cease if the boundaries are brought to rest. This is an immediate corollary to (V).

(VII) The acyclic irrotational motion of a liquid, at rest at infinity, due to the prescribed motion of an immersed solid,
is uniquely determined by the motion of the solid.

If possible, let $\varphi_1, \varphi_2$ be the velocity potentials of two different motions. The boundary conditions are
$\frac{\partial \varphi_1}{\partial n} = \frac{\partial \varphi_2}{\partial n}$ at the surface of the solid, $q_1=q_2 = 0$
at infinity.

Thus $\varphi=\varphi_1-\varphi_2$ is the velocity potential of a possible motion, such that $\frac{\partial \varphi}{\partial n} = 0$
 at the surface of the solid, $q = 0$ at infinity. It then follows from (1) that $q = 0$ everywhere, so that
 $\varphi_1-\varphi_2 = constant$, and the motions are essentially the same.

(VIII) If the liquid is in motion at infinity with uniform velocity, the acyclic irrotational motion, due to the prescribed
motion of an immersed solid, is uniquely determined by the motion of the solid.

For the relative kinematical conditions are unaltered if we superpose on the whole system of solid and liquid a velocity
equal in magnitude and opposite in direction to the velocity at infinity. This brings the liquid to rest at infinity.
The resulting motion is then determinate by (VII) and we return to the given motion by reimposing the velocity at infinity.

\ESect{Text 6}
\textbf{Motion in Two Dimensions.} Two-dimensional motion is characterized by the fact that the lines of motion are all
parallel to a fixed plane and that the velocity at corresponding points of all planes, parallel to the fixed plane has the
same magnitude and direction. To explain this more fully, suppose that the fixed plane is the plane of $xy$ and that $P$ is
any point in that plane. Draw $PQ$ perpendicular to the plane $xy$ (or parallel to $Oz$). Then points on the line $PQ$ are
said to correspond to $P$. Take any plane (in the fluid) parallel to $xy$ and meeting $PQ$ in $R$. Then, if the velocity
at $P$ is $q$ in the $xy$ plane in a direction making an angle $\theta$ with $Oy$, the velocity at $R$ is equal in magnitude
and parallel in direction to the velocity at $P$. The velocity at corresponding points is then a function of $x$, $y$ and the
time $t$, but not of $z$. It is therefore sufficient to consider the motion of fluid particles in a representative plane, say
the $xy$ plane, and we may properly speak of the velocity at the point $P$, which represents the other points on the line $PQ$
at which the velocity is the same. [Picture Omitted].

In order to keep in touch with physical reality it is often useful to suppose the fluid in two-dimensional motion to be
confined between two planes parallel to the plane of motion and at unit distance apart, the fluid being supposed to glide
freely over these planes without encountering any resistance of a frictional nature. Thus in considering the problem of the
flow of liquid past a cylinder in a two-dimensional motion in planes perpendicular to the axis of the cylinder, instead of
considering a cylinder of infinite length, a more vivid picture is obtained by restricting attention to a unit length of
cylinder confined between the said planes.

In considering the motion of a cylinder in a direction perpendicular to its axis, we can profitably suppose the cylinder to
be of unit thickness\footnote{The term `thickness' will be used to denote dimensions perpendicular to the plane of the motion.}
and to encounter no resistance from the barrier planes. This method of envisaging the phenomena in no way restricts the
generality and does not affect the mathematical treatment.

To complete the picture we shall adopt as our representative plane of the motion the plane which is parallel to our hypothetical
fixed planes and midway between them.

Thus in the case of a circular cylinder moving in two dimensions the diagram will show the circle $C$ which represents the
cross-section of the cylinder by the aforesaid reference plane, and the center $A$ of this circle will be the point where the
axis of the cylinder crosses the reference plane. This point may with propriety be called the center of the cylinder. More
generally any closed curve drawn in the reference plane represents a cross-section of a cylindrical surface bounded by the
fixed planes.

Two-dimensional motion, presents opportunities for special mathematical treatment and enables us to investigate the nature
of many phenomena which in their full three-dimensional form have to the present proved intractable.

\ESect{Text 7}
\textbf{Simple Closed Polygons.} The elementary idea of a polygon exemplified by, say, a rectangle or a regular hexagon
is familiar. For hydrodynamical applications it will be necessary to extend this concept to rectilinear configurations which
do not at first sight appear to resemble the polygons of elementary geometry. Let us consider two properties of the rectangle
(or of the regular hexagon).

(a) It is possible to go from any assigned point of the boundary to any other assigned point of the boundary by following a
path which never leaves the boundary. The boundary is connected.

(b) The boundary divides the points of the plane into two regions the points of which may be called interior points and
exterior points respectively. The interior points are such that any two of them can be joined by a path which never intersects
the boundary. The same holds of the exterior points. On the other hand, it is impossible to go from an interior point to an
exterior point without crossing the boundary somewhere.

Any configuration of straight lines in a plane which has the properties (a) and (b) will be called a simple closed polygon.
The adjective `simple' refers to the property that every point of the plane is either an interior point, a point of the
boundary, or an exterior point, the points of each class forming a connected system.

In many problems of hydrodynamical interest the boundaries of the polygon extend to infinity.
\end{document}
