
Таким образом, $\pi(x)=\frac{x}{\ln x}+R(x)$, $R(x)$ есть 
$o\left(\frac{x}{\ln x}\right)$. Если постараться, можно показать, что 
$R(x)\bw=O\left(\frac{x}{\ln^2x}\right)$, но не более того.	

\subsubsection{Следствия асимптотического закона}

Отметим важное следствие.
\begin{imp}
  $p_n \sim n \ln n$
\end{imp}
\begin{proof}
  Имеем $\pi(p_n) = n = \dfrac{p_n}{\ln p_n} (1 + o(1))$. 

	Прологарифмируем, получим: $\ln n = \ln p_n - \ln\ln p_n + \ln (1 + o(1))$. 

	Теперь перемножим	эти равенства: 
	$$
		n \ln n = \frac{p_n}{\ln p_n} (1 + o(1))(\ln p_n - \ln\ln p_n
	 + \ln(1 + o(1))) \sim p_n.
	$$
\end{proof}

Очевидно, $\ln[1,2,\dots,n] = \psi(n) \sim n$, откуда получаем еще одно
\begin{imp}
	$[1,2,\dots,n] = e^{n(1+o(1))}$.
\end{imp}

\subsubsection{Интегральный логарифм и нерешенные задачи}

Оказывается, $\pi(x)$ лучше приближает \notion{интегральный логарифм}: 
$$\li x = \int\limits_2^x\frac{dt}{\ln t},$$ и для функции $R(x)$ в этом 
случае можно добиться гораздо лучших оценок.

Очевидно, по правилу Лопиталя
$\lim\limits_{x\to+\infty} \frac{\li x}{x}\ln x = 1$. Рассмотрим 
$\pi(x)=\li x+R(x)$. Используя то, что $\ze(s)$ не обращается в $0$ в
$\{\si>1-\frac{c}{\ln t}\}$, можно доказать (что и было сделано Вале-Пуссеном),
что $$R(x)<xe^{-c\sqrt{\ln x}}<\frac{x}{(\ln x)^c}\,\text{ для любого } c>1.$$

Лучшая известная на данный момент оценка (Виноградов, Коробов, 1959 год) для
функции $R(x)$: $$R(x)<xe^{-c(\ln x)^{3/5-\ep}}.$$

При этом, если верна гипотеза Римана о том, что все нетривиальные нули 
$\ze$-функции лежат на прямой $\si=\frac12$, то для $R(x)$ выполняется куда 
более сильная оценка: $$R(x)<\sqrt{x}\ln x.$$

В 30-е годы было доказано, что $\pi(x) - \li x $ бесконечное число раз меняет 
знак, хотя во всех известных таблицах $\pi(x) < \li x$.

С простыми числами связано множество нерешенных задач. Среди них~— проблема 
Эйлера-Гольдбаха: правда ли, что любое нечетное натуральное число $n>8$
представимо в виде суммы трех простых. Виноградов доказал, что это верно для 
достаточно больших $n$ ($n>N$, где $N\sim10^{20}$). 

Известно также предположение Эйлера о том, что всякое четное число представимо
в виде суммы двух простых. Пока что доказано лишь, что 
$$\frac{\Card\{\text{все четные, непредставимые в таком виде}< n\}}
{\Card\{\text{все натуральные числа}< n\}}\xra{n\to\infty}0.$$

\section{Теорема Дирихле о бесконечном количестве простых чисел в 
				арифметических прогрессиях}

	Цель этого раздела~— доказательство открытой Дирихле в 1839 году теоремы, 
	гласящей, что в любой арифметической прогрессии,
	первый член и разность которой~— взаимно простые натуральные числа, 
	содержится бесконечно много простых чисел.

Стоит отметить, что, хотя задача для арифметических прогрессий была
решена более полутора веков назад, до сих пор неизвестно, бесконечно ли много
простых чисел в последовательности чисел вида, например, $n^2+1$.
	
% Среди $mx + a, (a,m) = 1$ содержится бесконечно много простых чисел~— доказано Дирихле в 1839 году. Нужно доказать бесконечность множества простых чисел, удовлетворяющих сравнению $ p = a \pmod m$.

\subsection{Сравнения и их свойства}

\subsubsection{Определение сравнений, элементарные свойства}

\begin{df}
Говорят, что числа \notion{$a$ и~$b$ сравнимы по модулю~$m$}, если они дают 
одинаковые остатки при делении на $m$.
\end{df}

Здесь $a$ и $b$~— целые числа, $m$~— натуральное, не меньшее $2$.
\begin{denote}
$a \eq b \pmod m$.
\end{denote}

Справедливо следующее очевидное утверждение.
\begin{stm}
Числа $a$ и~$b$ сравнимы по модулю~$m$ тогда и только тогда, когда $m \divs a-b$.
\end{stm}

Имеют место свойства:
\begin{points}{0}
  \item $a \eq b \pmod m \llra b \eq a \pmod m$\par
  \item $a \eq b \pmod m \llra a+c \eq b+c \pmod m$\par
  \item $a \eq b \pmod m \lRa ac \eq bc \pmod m$\par
  \item $a \eq b \pmod m,\:c \eq d \pmod m \lRa a + c \eq b + d \pmod m,\:
		ac \eq bd \pmod m$ 
\begin{proof}
	$ac \eq bc \eq bd \pmod m$
\end{proof}
  \item $ac \eq bc \pmod m,\:(c,m) = 1 \lRa a \eq b \pmod m$
\begin{proof}
  $m \divs c(a - b),\,(c,m) = 1 \lRa m \divs a - b$
\end{proof}

	\item $a \eq b \pmod m \llra ac \eq bc \pmod{mc},\,c \neq 0$
\end{points}

\subsubsection{Уравнения в сравнениях, линейные сравнения}

Можно решать уравнения относительно сравнений: 
$a_nx^n + \dots + a_0 \eq 0 \pmod m$. Очевидно, если $x_0$~— решение, то 
весь класс $x \eq x_0 \pmod m$~— решение. Таким образом, количество классов 
вычетов по модулю $m$ и есть количество решений этого сравнения.

Уравнение $a_nx^n + \dots + a_0 \eq 0 \pmod p $ по модулю
простого числа имеет не более $n$ классов решений. Действительно, в этом случае
мы фактически ищем корни многочлена $a_nx^n + \dots + a_0 = 0$ в факторгруппе 
$\Z_p=\fact{\Z}{p\Z}$, которая является полем при простых $p$; в поле же число 
корней многочлена не превышает его степень.

Если $m$ не является простым, то это уже не верно: например, уравнение
$x^2 \eq 1 \pmod 8$ имеет корни $x \eq 1, 3, 5, 7 \pmod 8$.

Рассмотрим линейные сравнения.
\begin{stm}
  Сравнение $ax \eq b \pmod m,\,(a,m)=1$ всегда имеет и ровно одно решение.
\end{stm}
\begin{note}
Условие $(a,m)=1$, понятное дело, никак не ограничивает класс рассматриваемых 
сравнений: если $(a,m)\neq1$, то мы всегда можем сократить на него и
свести к нужному.
\end{note}
\begin{proof}
	Докажем существование. Найдутся целые $u,\,v$: $au+mv=1$, значит, 
	$au\eq1\pmod m$, откуда $abu\eq b\pmod m$.
\begin{note}
Это сразу же дает нам и алгоритм решения~— алгоритм Евклида.
\end{note}
	Покажем единственность. Действительно, пусть найдутся два решения 
	$ax_1\eq b\pmod m$, $ax_2\eq b\pmod m$. Тогда $a(x_2-x_1)\eq0\pmod m$. Но
	$a$ взаимнопросто с $m$, значит (\pt 5), $x_2-x_1\eq0\pmod m$.
\end{proof}

\subsubsection{Группы $\Z_m$ и $\Z_m^*$}

Пусть $m \geqslant 2$~— натуральное число. Тогда определена факторгруппа 
$\Z_m = \fact{\Z}{m\Z}=\{\overline{a}: a + mt, t \in \Z \}$ с операциями на
классах $\overline{a+b} = \overline{a} + \overline{b},\,
\overline{a\cdot b} = \overline{a} \cdot \overline{b}$

Очевидно, $\Z_m$~— группа по сложению при любом $m$, $\Z_m\setminus\{0\}$
~— группа по умножению тогда и только тогда, когда $m$~— простое: 
в противном случае $m = m_1m_2$, а значит, $\overline{m_1}\cdot\overline{m_2} = 
\overline{0}$.

\begin{df}
$\Z_m^*$ есть набор тех классов из $\Z_m$, которые порождены элементами, 
взаимнопростыми с $m$, то есть 
$\Z_m^* =\{\overline{a}: a + mt,\,t \in \Z,\,(a,m) = 1 \}$.
\end{df}
\begin{stm}
  $\Z_m^*$~— абелева группа по умножению.
\end{stm}
\begin{proof}
  Во-первых, проверим корректность умножения:
  \begin{points}{0}
    \item Если $(a,m) = 1,\, b \eq a \pmod m$, то $(b,m) = 1$. \par
		Действительно, $b = a+mt$, поэтому, если $p \divs b,\,p \divs m$, то 
		$p \divs a$~—противоречие.
    \item Если $(a,m) = 1,\,(b,m) = 1$, то $(ab,m) = 1$. \par
		Предположим, $p \divs ab,\,p \divs m$, тогда $p \divs a$ или $p \divs b$
		~— противоречие.
  \end{points}

  Во-вторых, групповые свойства:
  \begin{points}{0}
    \item Коммутативность и ассоциативность очевидны.
    \item Единица группы есть $\overline{1}$: $\overline{a}\cdot\overline{1}=\overline{a}$.
    \item Для любого элемента группы найдется обратный.\par
		Действительно, доказали, что $ax \eq 1 \pmod m$~— разрешимо. Если $x_0$~— 
		решение, то $\overline{x_0} = \overline{a}^{-1}$.
  \end{points}
\end{proof}

\subsubsection{Функция Эйлера, теорема Лагранжа и следствия}

Нам потребуется один результат из алгебры.

\begin{theorem}[Лагранж]
Пусть группа $G$ конечна, и $H$ — её подгруппа. Тогда порядок $G$ равен порядку 
$H$, умноженному на количество её левых или правых классов смежности.
\end{theorem}

\begin{imp}
Порядок любого элемента конечной группы $G$ делит порядок $G$.
\end{imp}

\begin{proof}
Следует из того, что порядок элемента равен порядку циклической подгруппы, им
порожденной.
\end{proof}

\begin{df}
\notion{Функцией Эйлера} называется 
$\ph(m) = \Card\{x\in\N: 1 \leqslant x \leqslant m,\, (x,m)=1\}$.
\end{df}

Ясно, что $|\Z_m^*| = \varphi(m)$. 
По только что доказанному следствию для любого
$\overline{a}$ из $\Z_m^*$ верно равенство $\overline{a}^{\varphi(m)} 
= \overline{1}$.

Отсюда получаем два важных результата.
\begin{imp}[Теорема Эйлера]
  Для произвольного целого $a$ такого, что $(a,m) =1$, верно $a^{\varphi(m)} 
	\eq 1 \pmod m$.
\end{imp}

\begin{imp}[Малая теорема Ферма]
  Для каждого простого $p$, не делящего $a$, $a^{p-1} \eq 1~\pmod p$.
\end{imp}

\begin{proof}
  В силу очевидного $\ph(p)=p-1$.
\end{proof}

\begin{imp}
В условиях предыдущего следствия $a^p\eq a\pmod p$.
\end{imp}

\subsubsection{Бесконечность количества простых чисел в некоторых частных 
							арифметических прогрессиях}

\begin{lemma}
	\label{p divs a^2+b^2}
  Пусть нечетное простое $p$ делит $a^2 + b^2$, и $(a,b) = 1$. Тогда $p$ имеет 
	вид $4n + 1$.
\end{lemma}

\begin{proof}
  Понятно, $p \ndivs a$ и $p \ndivs b$: иначе $(a,b) \not = 1$. Тогда 
	$a^2 \eq -b^2 \pmod p$. А значит, $a^{p-1} \eq (-1)^{\frac{p-1}{2}} b^{p-1} 
	\pmod p$. С другой стороны, в силу малой теоремы Ферма 
	$a^{p-1}\eq b^{p-1}\eq1\pmod p$, откуда по свойству \pt 5 сравнений 
	$1 \eq (-1)^{\frac{p-1}{2}} \pmod p$. Но $p\geqslant 3$, 
	так что это означает просто $(-1)^{\frac{p-1}{2}} = 1$, а следовательно, 
	$\frac{p-1}{2} = 2n$, то есть $p = 4n + 1$.
\end{proof}

\begin{stm}
  В прогрессии вида $4n - 1,\,n = 1,2,\dots$ содержится бесконечное количество 
	простых чисел.
\end{stm}

\begin{proof}
  Пусть конечное: $p_1,\ldots,p_m$. Рассмотрим $4p_1\dots p_m - 1 = q_1\dots q_s$. 
	Среди $q_j$ есть число вида $4k - 1$, потому как произведение чисел вида 
	$4k + 1$ есть число того же вида. Значит, для некоторого $k$ $q_j=p_k$, тогда
	$p_k \divs 4p_1\dots p_m-1$, откуда $p_k\divs 1$, что невозможно.
\end{proof}

\begin{note}
  Для прогрессии вида $6n - 1$ все аналогично.
\end{note}

\begin{stm}
 В прогрессии вида $4n + 1,\,n = 1,2,\dots$ содержится бесконечное количество 
	простых чисел.
\end{stm}

\begin{proof}
  Пусть конечное: $p_1,\dots,p_m$. Рассмотрим 
	$(2p_1\dots p_m)^2 + 1 = q_1\ldots q_s$. По лемме $(\ref{p divs a^2+b^2})$
	все $q_i=4n_i+1$, поэтому для	некоторого $k$ $q_l=p_k$, значит, 
	$p_k\divs(2p_1\dots p_m)^2 + 1$, откуда	$p_k\divs 1$, чего быть не может.
\end{proof}
