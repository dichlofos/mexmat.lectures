\section{Алгебраические и трансцендентные числа}

\begin{df}
  Число $\alpha \in \Cbb$ называется \emph{алгебраическим}, если
  существует ненулевой многочлен с рациональными коэффициентами,
  который зануляется на $\alpha$.
\end{df}
\begin{denote}
  Поле алгебраических чисел обозначается~$\A$.
\end{denote}
\begin{df}
  Число называется \emph{трансцендентным}, если оно не является
  алгебраическим.
\end{df}

Замечание: можно считать, что коэффициенты у многочлена целые.

Можно заметить, что многочленов счетно много, отсюда следует, что
алгебраических чисел тоже счетно много, а всех комплексных чисел
континуально много. Значит, трансцендентные числа существуют (Кантор,
1872).

В 1844 году Лиувилль явно построил пример трансцендентного числа. В
1873 году Эрмит доказал, что $e$ трансцендентно. В 1882 году Линдеман
доказал трансцендентность~$\pi$.

\begin{df}
  Многочлен $x^n + a_{n-1}x^{n-1} + \ldots + a_0 \in \Q[x]$, который
  зануляет $\alpha$ такой, что $n$ минимально, называется
  \emph{минимальным многочленом $\alpha$}.
\end{df}

\begin{stm}
  Минимальный многочлен единственный.
\end{stm}
\begin{proof}
  Если он не единственный, то вычтем один минимальный многочлен из
  другого. Получится многочлен на единицу меньшей степени, который
  зануляет $\alpha$. Противоречие.
\end{proof}

\begin{df}
  \emph{Степенью} алгебраического числа $\alpha$ называется степень
  минимального многочлена $\alpha$.
\end{df}

Разложим минимальный многочлен на множители: $(x - \alpha_1) \ldots (x
- \alpha_n)$.

\begin{df}
  $\alpha_1, \ldots, \alpha_n$ называется
  \emph{сопряженными}~$\alpha$.
\end{df}

\begin{stm}
  Допустим, что $A$~--- минимальный многочлен $\alpha$, $B(\alpha) =
  0$. Тогда $A \divs B$.
\end{stm}
\begin{proof}
  Разделим $B$ на $A$ с остатком: $B = AQ + R$. Подставим сюда
  $\alpha$, получим, что $R(\alpha) = 0$, но степень $R$ меньше
  степени~$A$. Значит, $R$ тождественный ноль.
\end{proof}

\begin{stm}
  Пусть $0 \ne A \in \Q[x]$, $A(\alpha) = 0$, коэффициент при $x^n$
  равен единице. Утверждается, что $A$ минимальный многочлен $\alpha$
  тогда и только тогда, когда $A$ неприводим над $\Q$.
\end{stm}
\begin{proof}
  Необходимость. Если $A$ приводим, то разложим его на множители и
  $\alpha$ попадает \emph{куда-то}.

  Достаточность. Пусть $A_0$~--- минимальный многочлен $\alpha$. Тогда
  $A$ делится на $A_0$, но $A$ неприводим, значит, $A = A_0$.
\end{proof}

\begin{stm}
  Если $\alpha_1$, \ldots, $\alpha_n$~--- сопряженные для $\alpha$, то
  этот набор является сопряженным для всех $\alpha_i$.
\end{stm}

\begin{stm}
  Минимальный многочлен не имеет кратных корней.
\end{stm}
\begin{proof}
  Если корень кратный, то он корень производной, у которой степень
  меньше.
\end{proof}

Наряду с минимальным многочленом рассматривают другое близкое
определение.

\begin{df}
  $A' = a_n x^n + \ldots + a_0$, $A'(\alpha) = 0$, $a_k \in \Z$, $a_n
  > 0$, $a_k$ взаимно просты. Такой многочлен называют
  \emph{каноническим} для $\alpha$.
\end{df}

\begin{df}
  Число $\alpha$ называется \emph{целым алгебраическим}, если
  минимальный многочлен $\alpha$ имеет целые коэффициенты.
\end{df}
\begin{denote}
  Поле целых алгебраических чисел обозначается~$\Z_A$.
\end{denote}

\begin{df}
  Пусть $A \in \Z[x]$. $A$ \emph{примитивный}, если его коэффициенты
  взаимно просты.
\end{df}

\begin{lemma}[Гаусса]
  Произведение примитивных многочленов~— примитивный многочлен.
\end{lemma}
\begin{proof}
  Пусть $A$ и $B$~— примитивные. $C = AB = c_{m+n}x^{m+n} + \ldots +
  c_1 x + c_0$. Хотим доказать, что $C$ примитивный. Что значит, что
  многочлен примитивный? Очевидно, что это эквивалентно тому, что для
  любого простого найдется коэффициент, который на него не
  делится. Возьмем простое $p$. Пусть $s$~— минимальное такое, что $p
  \ndivs a_s$, аналогично, $t$~— минимальное такое, что $p \ndivs
  b_t$. Докажем, что $p \ndivs c_{s+t}$. Действительно, $c_{s+t} =
  \sum_{k+l = s+t} a_k b_l = a_s b_t + \sum_{k+l=s+t, (k, l) \ne
    (s,t)} a_k b_l$. Первое слагаемое не делится на $p$, сумма
  делится. Что и требовалось доказать.
\end{proof}

\begin{theorem}
  Если $\alpha$ является корнем \emph{какого-то} многочлена $C$ с
  целыми коэффициентами, и $c_n = 1$, то $\alpha$ является целым
  алгебраическим ($\alpha \in \Z_A$).
\end{theorem}
\begin{proof}
  Пусть $A \in \Q[x]$~— минимальный многочлен $\alpha$. $C = AB$, $B
  \in \Q[x]$. Приведем коэффициенты $A$ и $B$ к общему знаменателю:
  $$ C = \frac{u}{v} A' B',
  $$ где $A'$ и $B'$~— примитивные, $u, v \in \N$. По лемме Гаусса
  получаем, что $u = v = 1$, Отсюда получаем, что старший коэффициент
  $A = A'$ равен единице. Отсюда следует, что $\alpha \in \Z_A$.
\end{proof}

\begin{stm}
  Если $\alpha \in \A$, то его можно представить как $\alpha = \beta /
  \gamma$, где $\beta, \gamma \in \Z_A$.
\end{stm}
\begin{proof}
  Пусть $A = a_n x^n + \ldots$~— канонический многочлен $\alpha$. $0 =
  a_n^{n-1} A(\alpha) = B(a_n \alpha)$. Значит, $\alpha a_n \in \Z_A$.
\end{proof}

\begin{theorem}
  $\A$~— поле, $\Z_A$~— кольцо.
\end{theorem}
\begin{proof}
  Докажем для начала, что если $\A \ni \alpha \ne 0$, то $1 / \alpha
  \in \A$. Действительно, возьмем минимальный многочлен для $\alpha$ и
  переставим его коэффициенты в обратном порядке.
  
  Пусть $\alpha$~— корень $A \in \Q[x]$, $\beta$~— корень $B \in
  \Q[x]$, $\deg A = n$, $\deg B = m$.  Докажем, что $\alpha \beta$
  является корнем какого-то многочлена из $\Q[x]$. Рассмотрим линейное
  пространство над $\Q$, порожденное числами вида $\alpha^k \beta^l$,
  где $k, l \geqslant 0, k, l \in \Z$. Размерность пространства не
  превосходит $mn$, так как $\alpha^n$ и $\beta^m$ выражаются через
  младшие степени. Рассмотрим числа $1$, $\alpha \beta$, \ldots,
  $(\alpha \beta)^{mn}$. Их $mn + 1$, значит, что есть линейная
  зависимость: $c_{mn}(\alpha \beta)^{mn} + \ldots + c_0 = 0$. Значит,
  $\alpha\beta$ алгебраическое.
  
  Аналогично можно доказать, что $\alpha + \beta$ алгебраическое.
  
  Неясно, как это доказательство перенести на целые алгебраические
  числа. Поэтому приведем другое доказательство факта, что $\A$~—
  поле.
  
  Вспомним некоторые сведения из алгебры. Пусть $K$~— коммутативное
  кольцо. Рассмотрим симметрический $P(x_1, \ldots, x_n) \in K[x_1,
    \ldots, x_n]$. Можно представить $P(x_1, \ldots, x_n) =
  S(\sigma_1, \ldots, \sigma_n)$, где $\sigma_i$~— элементарные
  симметрические многочлены, $S \in K[\ldots]$.
  
  Можно обобщить эту теорему на многочлены, которые симметрические по
  нескольким системам переменных. Их можно представить как многочлены
  от элементарных многочленов: для каждой системы свой набор.
  
  \begin{lemma}
    Пусть $\alpha_1$, \ldots, $\alpha_m$~— алгебраические числа
    степеней $n_1, \ldots, n_m$. $\overline{\alpha_j}$~— набор
    сопряженных для $\alpha_j$.
    
    \begin{enumerate}
    \item
    Рассмотрим $P(x,\overline{\alpha_1}, \ldots, \overline{\alpha_m})
    \in \Q[\ldots]$, который является симметрическим по каждому из
    этих наборов. Тогда $P$ можно представить как $R(x) \in \Q[x]$.
    \item Если $\alpha_i \in \Z_A$, $P \in \Z[\ldots]$, тогда $R \in
      \Z[x]$.
    \end{enumerate}
  \end{lemma}
  \begin{proof}
    Пусть $B \in \Q[x]$~— минимальный многочлен с корнями
    $\beta_i$. Тогда по теорема Виетта $\sigma_j(\beta_1, \ldots,
    \beta_n)$ рациональные, а если $B \in \Z[x]$, то
    $\sigma_j(\beta_1, \ldots, \beta_n) \in \Z$.
    
    Докажем пункт 1. Индукция по~$m$. Если $m = 1$, то все просто~—
    представляем $P$ как многочлен от элементарных симметрических
    многочленов и пользуемся предыдущим абзацем.
    
    Переход. $P(x, \overline{\alpha_1}, \ldots, \overline{\alpha_m}) =
    S(\overline{\alpha_m})$, где $S$~— многочлен над кольцом
    многочленов над $x, \ldots,
    \overline{\alpha_{m-1}}$. $S(\overline{\alpha_m}) = T(\sigma_{m1},
    \ldots, \sigma_{mn_m}) = V(x, \overline{\alpha_1}, \ldots,
    \overline{\alpha_{m-1}}) \in \Q[x]$.
    
    Пункт 2 доказывается точно так же.
  \end{proof}
  Давайте докажем, что если $\alpha, \beta \in \A$, то $\alpha +
  \beta, \alpha \beta \in \A$. Пусть $\alpha_1, \ldots, \alpha_n$~—
  сопряженные $\alpha$, $\beta_1, \ldots, \beta_m$~— сопряженные
  $\beta$. Рассмотрим многочлен $\prod_{i=1}^n \prod_{j=1}^m (x -
  (\alpha_i + \beta_j))$. $\alpha + \beta$ является его корнем, в то
  же время он является симметрическим по $\alpha_i$ и по
  $\beta_j$. Значит, что он с рациональными
  коэффициентами. Произведение доказывается так же.
  
  Про целые алгебраические доказывается точно так же.
\end{proof}

Рассмотрим многочлен $A = x^n + a_{n-1}x^{n-1} + \ldots + a_0$, где
$a_i \in \A$. Пусть $A(\alpha) = 0$. Оказывается, что $\alpha \in \A$.

\begin{theorem}
  Поле $\A$ алгебраически замкнуто.
\end{theorem}
\begin{proof}
  Пусть $a_j^{(1)}, \ldots, a_j^{(m_j)}$~— алгебраически
  сопряженные~$a_j$. Рассмотрим многочлен
  $$ B(x) = \prod_{j_0 = 1}^{m_0} \ldots \prod_{j_{n-1} = 1}^{m_{n-1}}
  \left( x^n + a_{n-1}^{(j_{n-1})} x^{n-1} + \ldots + a_0^{(j_0)}
  \right)
  $$ $B(\alpha) = 0$ и он является симметрическим по всем системам
  сопряженных чисел. Значит, $\alpha$ алгебраическое.
\end{proof}

Видно, что $\A$ является алгебраическим замыканием~$\Q$.



