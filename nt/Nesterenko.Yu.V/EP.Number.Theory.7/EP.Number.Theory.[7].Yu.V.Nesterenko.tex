\documentclass[a4paper]{article}
\usepackage[simple,utf]{dmvn}

\title{Программа экзамена по теории чисел}
\author{Лектор Ю.В.Нестеренко}
\date{VII семестр, 2005 г.}

\begin{document}
\maketitle

\begin{nums}{-3.2}

\item  Теорема о единственности разложения целых чисел в произведение простых.

\item  Оценки Чебышева для функции $\pi(x)$.

\item  Определение функции $\ze(s)$
и ее простейшие свойства в области $\Re s > 1$ (аналитичность, представление функций
$\ze'(s)$ и $\frac{\ze '(s)}{\ze(s)}$ в виде ряда Дирихле,
отсутствие нулей).

\item  Тождество Эйлера для $\ze(s)$.

\item  Аналитическое продолжение $\ze(s)$ в область $\Re s > 0$.

\item  Отсутствие нулей у $\ze(s)$ на прямой $\Re s = 1$.

\item  Оценка сверху $\hm{\ze(s)}$ и $\hm{\ze'(s)}$
в области $1\le \sigma \le 2$, $\hm{t} \ge 3$.

\item  Оценка сверху $\hm{\frac{\ze '(s)}{\ze (s)}}$ в области $1 \le \sigma \le 2$,
$\hm{t} \ge 3$.

\item  Асимптотический закон распределения простых чисел. Функция Чебышева~$\psi(x)$.
Доказательство равенства $\psi(x) - \pi(x)\ln x = o(x)$ при $x \ra\infty$.
Сведение доказательства асимптотического закона к равенству
$\om(x) = x + o(x)$ для функции $\om (x) = \intl{1}{x}{\frac{\psi(y)}{y}dy}$.

\item  Доказательство тождества
$
\om (x) = \frac{1}{2\pi i}\intl{2 - i\infty }{2 + i\infty }{\hr{-\frac{\ze '(s)}{\ze (s)}}} \cdot \frac{x^s}{s^2}ds.
$

\item Выделение предполагаемого главного члена функции $\om(x)$.

\item Оценка остаточного члена $\om(x)$ и доказательство равенства $\om (x) = x + o(x)$ при $x \ra \infty$.

\item  Теорема Эйлера и малая теорема Ферма. Бесконечность множеств простых чисел в прогрессиях $4n\pm 1$.

\item Построение характеров Дирихле.

\item Свойства характеров (вычисление сумм $\suml{n = 1}{m}\chi(n)$ и
$\sums{\chi}\chi (n)$, доказательство неравенства
$\Bm{\suml{n = 1}{x} \chi (n)} \le m$ для неглавного характера).

\item  $L$ функции Дирихле и их простейшие свойства в области $\Re s > 1$
(аналитичность, представление $L'(s,\chi)$ и $\frac{L'(s,\chi)}{L(s,\chi)}$
в виде рядов Дирихле, отсутствие нулей).

\item  Тождество Эйлера для $L$ функций, аналитическое продолжение $L$ функций в
область $\Re s > 0$.

\item  Доказательство утверждения $L(1,\chi) \ne 0$ для неглавных
действительных характеров $\chi$.

\item  Доказательство утверждения $L(1,\chi) \ne 0$ для неглавных комплексных
характеров $\chi$.

\item  Доказательство теоремы Дирихле о простых числах в арифметической
прогрессии.

\item  Множество $\A$ алгебраических чисел. Замкнутость $\A$ относительно арифметических операций.

\item  Целые алгебраические числа. Замкнутость множества целых алгебраических
чисел относительно сложения, вычитания и умножения.

\item  Теорема о примитивном элементе. Степень конечного расширения.

\item  Алгебраическая замкнутость множества алгебраических чисел.

\item  Вложения конечного расширения в $\Cbb$. Нормальные расширения.
Группа Галуа.

\item  Множество образов элемента при различных вложениях
конечного расширения. Норма в алгебраическом расширении.

\item  Теорема Дирихле о приближении действительных чисел рациональными.
Иррациональность числа~$e$.

\item  Теорема Лиувилля о приближении рациональными числами алгебраических
чисел. Иррациональность и~трансцендентность числа $\suml{n =0}{\infty} \frac{1}{2^{n!}}$.

\item  Трансцендентность числа $e$.

\item  Иррациональность числа $\pi$.

\item  Теорема Линдемана Вейерштрасса. Следствия из неё. Сведение
доказательства к предложению об экспоненциальной линейной форме.

\item  Доказательство предложения об экспоненциальной линейной форме.
\end{nums}

\medskip
\dmvntrail
\end{document}
