\documentclass[a4paper]{article}
\usepackage[xe]{dmvn}
%\usepackage{dmvnadd}

%\input marginmarker

\renewcommand{\mod}{\mathop{\mathrm{mod}}}

\def\nequiv{\not\equiv}
\def\dx{\,dx}
\def\dt{\,dt}
\def\ds{\,ds}

\newenvironment{petit}
{\par\smallskip\hrule\smallskip\footnotesize}{\par\smallskip\hrule\smallskip}

\begin{document}
\dmvntitle{Курс лекций по}{теории чисел}{Лектор\т Юрий Валентинович Нестеренко}
{IV курс, 7 семестр, поток математиков}{Москва, 2006~г.}

\section*{Предисловие}

Текст был набран Александром Юхименко, затем отредактирован, исправлен и дополнен DMVN Corporation.
Версия 2005 года. Документ был также отредактирован самим лектором и грубой лажи содержать не должен.

\medskip

\dmvntrail

\subsection*{Благодарности}

Благодарность за поиск лажи выражается Паше Наливайко, Володе Филатову, Юре Дружинину,
Саше Юхименко, Альмире Червовой, Сергею Гладких, Коле Рудому, Мише Левину,
Мише Берштейну, Юре Притыкину и даже Сене Акопяну, который вообще с другого потока.

\subsection*{Используемые обозначения}

Обычно простое число мы будем обозначать буквой $p$ (если $p$ занята, то далее по алфавиту $q$ или $r$).
Если встречается запись $\sums{p<x}$, то это означает, что суммирование ведется по всем простым числам,
меньшим $x$. Иногда, разумеется, мы будем использовать букву $p$ для обозначения произвольного натурального числа,
не обязательно простого.

\begin{items}{-2}
\item $p\divs x$\т означает, что число $p$ делит число $x$.
\item $x\dv p$\т означает, что число $x$ делится на $p$.
\item $(a,b)$\т наибольший общий делитель (НОД) чисел $a$ и $b$. Аналогичное обозначение используется
      и для НОД нескольких чисел: $(a_1\sco a_n)$.
\item $[a,b]$\т наименьшее общее кратное (НОК) чисел $a$ и $b$. Аналогичное обозначение используется
      и для НОК нескольких чисел: $[a_1\sco a_n]$.
\item $f(x)\sim g(x)\Lra \frac{f(x)}{g(x)}\ra 1$, $x\ra \bes$.
\item $\# A$\т количество элементов в множестве $A$.
\item $\Z_+$\т множество целых неотрицательных чисел.
\end{items}

\begin{thebibliography}{1}
\setlength{\itemsep}{-2pt}
\bibitem{gnsh} \emph{Галочкин\,А.\,И., Нестеренко\,Ю.\,В., Шидловский\,А.\,Б.} Введение в теорию чисел.\т М.: Изд\д во Моск. ун-та, 1984.
\bibitem{hass} \emph{Хасс.} Лекции по теории чисел.
\end{thebibliography}

\newpage

\tableofcontents

\newpage

%\makeatletter
%  \renewcommand{\headheight}{11mm}
%  \renewcommand{\headsep}{2mm}
%  \renewcommand{\sectionmark}[1]{}
%  \renewcommand{\subsectionmark}[1]{}
%  \renewcommand{\subsubsectionmark}[1]{\markright{\thesubsubsection. #1}}
%  \renewcommand{\@oddhead}{\vbox{\hbox to \textwidth{\scriptsize\hbox to .5cm{\thepage\hfil}\textsc{[\сегодня~года, \now]}\hfil\rightmark\strut}\hrule}}
%  \renewcommand{\@oddfoot}{\hfil\thepage\hfil}
%\makeatother

\section{Введение}

\subsection{Простые числа}

\begin{df}
Натуральное число $n$ называется \emph{составным}, если может быть представлено в виде ${n=uv}$, где ${u,v>1}$.
Натуральное число, не являющееся составным, называется \emph{простым}. Число $1$ не является ни простым,
ни составным по определению.
\end{df}

Вот пара интересных примеров простых чисел: $\underbrace{11\dots11}_{32}4\underbrace{11\dots11}_{32}$ и
$2^{13466917}-1$.

\begin{stm}
Пусть $M\in\N$ и $p>1$\т наименьший делитель $M$. Тогда $p$\т простое.
\end{stm}
\begin{proof}
Предположим, что $p=uv$, где $u,v>1$. Тогда $u$ и $v$\т делители $M$, меньшие $p$.
\end{proof}

\begin{theorem}[Евклид] Множество простых чисел бесконечно.
\end{theorem}
\begin{proof}
Допустим, существует лишь конечное множество простых чисел $\hc{p_1\sco p_n}$. Рассмотрим число $M\bw{:=}p_1\cdot p_2\sd p_n+1$.
Пусть $p$\т его наименьший делитель. Очевидно, $M$ не делится ни на одно из чисел $p_i$. Согласно предыдущему
утверждению, либо число $p$ является простым, либо $p=1$ (то есть само число $M$ простое).
В обоих случаях мы получили ещё одно простое число.
\end{proof}

С древних времен известен способ нахождения простых чисел (решето Эратосфена). К настоящему моменту имеются
некоторые его модификации, которые лишь незначительно ускоряют процесс поиска.

\begin{theorem}[Решето Эратосфена]
Выписываем числа $1,2,3\sco n$ и вычёркиваем единицу. Далее, первое незачёркнутое число $p$ обводим рамкой
и вычёркиваем все числа, кратные ему, начиная c $p^2$. Затем берём первое невычеркнутое и необведённое число,
с ним делаем то  же самое, и так далее. Невычеркнутые числа суть все простые числа в диапазоне от $1$ до $n$.
\end{theorem}
\begin{proof}
Заметим, что вычеркиваются лишь составные числа, поэтому простые останутся невычеркнутыми.
Предположим, что число $a$ не вычеркнуто, и $a=pv$, где $p,v>1$. Пусть $p$\т минимальный делитель $a$.
Тогда он прост. Но $p^2 \le pv =a$, значит, число $a$ в свое время нужно было вычеркнуть.
\end{proof}

Составных чисел бесконечно много, но их в некотором смысле гораздо больше, чем простых. Вот пример отрезка
натурального ряда длины $n-1$, состоящего сплошь из составных чисел: $n!+2,n!+3\sco n!+n$. Таким образом,
между простыми числами встречаются сколь угодно большие пробелы.

Тем не менее, в натуральном ряду встречаются и отрезки, на которых простых чисел сравнительно много.
Так, на отрезке $\hs{10^{15},10^{15}+150000}$ расстояние между соседними простыми числами не превосходит~$276$.

До сих пор не доказан факт бесконечности пар простых чисел вида $(n-1,n+1)$ (гипотеза <<близнецов>>).
Пример относительно большой такой пары\т это $(291\cdot2^{1553}\pm1)$.

Еще одна до сих пор нерешённая задача (проблема Гольдбаха): каждое чётное число представимо в виде суммы двух простых. В 1937 году
И.\,М.\,Виноградов доказал, что каждое достаточно большое нечётное число представимо в виде суммы трёх простых.

А вот пример уравнения в целых числах (уравнение Пелля), которое очень непросто решить прямым перебором:
$x^2-109y^2=1$. Его минимальное по модулю решение $(x_{\min},y_{\min}) = (158070671986249, 15140424455100)$.

\subsection{Основная теорема арифметики}

\begin{theorem}[Основная теорема арифметики]
    Для любого натурального $m$ существует и единственно (с точностью до порядка множителей) его представление в виде
    $m=p_1^{\al_1}\sd p_t^{\al_t}$, где $p_i$\т простые числа.
\end{theorem}

Существование такого разложения легко доказывается по индукции. Мы не будем проводить его здесь.
Единственность легко доказывается с помощью двух полезных лемм, приведённых ниже.

\begin{lemma}[О линейном представлении НОД]
Если $(a,b)=d$, то существуют числа $x,y\in\Z$, такие что $d=ax+by$.
\end{lemma}

Мы приведём два доказательства этой леммы. Одно\т так называемое неэффективное (поскольку не даёт явных
значений $x$ и $y$), а второе\т эффективное (то есть даёт явный алгоритм построения чисел $x$ и $y$).

\begin{proof}
\pt{1} \emph{Неэффективное:} рассмотрим множество
\eqn{M=\hc{m=ax+by\vl m>0,\; x, y\in\Z}.}
Пусть $d = (a,b)$. Тогда, очевидно, для любого $m\in M$ имеем $d\divs m$.
Пусть $z$\т минимальное число в~$M$. Число $d$ делит~$z$ как и всякое прочее число из~$M$.
Пусть $m$\т произвольное число из $M$. Докажем, что $z \divs m$. Предположим, что это не так. Разделим $m$ на $z$
с остатком: $m=qz+m'$, $0<m'<z$. Тогда $m' = m -qz$\т это линейная комбинация чисел из $M$, поэтому $m'$ тоже является числом из $M$.
Это противоречит минимальности $z$ в $M$. Поэтому $z\divs m$. А поскольку $m$\т произвольное число из $M$, то, в частности,
$z\divs a$ и $z \divs b$, поскольку $a,b\in M$. Следовательно, $z \divs (a,b) = d$. С другой стороны, $d \divs z$, поскольку $z \in M$.
Значит, $z = d$. А так как $z$ имеет вид $ax+by$, числа $x$ и $y$ найдены.

\pt{2} \emph{Алгоритмическое:} Проводим алгоритм Евклида, который выглядит так:
\eqn{\begin{aligned}
a&=q_1b+r_1,\\
b&=q_2r_1+r_2,\\
&\ldots\\
r_n&=q_{n+2}r_{n+1}+r_{n+2},\\
r_{n+1}&=q_{n+3}r_{n+2}.\\
\end{aligned}}
Легко видеть, что $r_{n+2}=(a,b)$. Теперь, чтобы найти $x$ и $y$, нужно воспользоваться обратным ходом алгоритма Евклида.
Именно, перепишем равенства в следующем виде:
\eqn{\begin{aligned}
(a,b) = r_{n+2} &= r_n - q_{n+2}r_{n+1},\\
        r_{n+1} &= r_{n-1} - q_{n+1}r_n,\\
        &\dots
\end{aligned}}
Затем последовательно выражаем остатки с большими номерами через остатки с меньшими номерами.
В итоге получим представление вида $(a,b) = r_{n+2} = ax + by$.
\end{proof}

\begin{lemma}\label{lem:div}
Если $a\divs bc$ и $(a,b)=1$, то $a\divs c$.
\end{lemma}
\begin{proof}
    Имеем $(a,b)=1$, значит, по предыдущей лемме найдутся $x,y \in \Z$, для которых $ax+by=1$. Умножим это равенство
    на $c$, получим $acx+bcy=c$. По условию $bc\dv a$, значит, левая часть равенства делится на $a$. Стало быть, $c\dv a$.
\end{proof}

Вывод основной теоремы арифметики из этих лемм предоставляется читателю.

\section{Асимптотический закон распределения простых чисел}

Мы будем обозначать через $\pi(x)$ количество простых натуральных чисел, не превосходящих~$x$.
История определения асимптотики функции $\pi(x)$ такова:

\begin{items}{-2}
\item Евклид: $\pi(x) \ra \bes$ при $x\ra \bes$.
\item Эйлер: $\frac{\pi(x)}x \ra 0$ при $x \ra \bes$.
\item Чебышев (1848~г.): Если предел $\frac{\pi(x)\ln(x)}{x}$ существует, то он равен $1$.
\item Адамар и Валле\д Пуссен (1896~г.): $\pi(x)\sim \frac{x}{\ln x}$.
\end{items}

\subsection{Оценки Чебышева для функции $\pi(x)$}

Мы докажем для любого $x\ge 6$ неравенства
\eqn{a\frac{x}{\ln x}\le\pi(x)\le b\frac{x}{\ln x}.}
Константы, которые получатся у нас, будут такими: $a=\frac{\ln2}2\approx0.3465$, а $b=5\ln2\approx3.4657$.
У Чебышева константы были более точные: $a\approx0.92129$, $b\approx1.10555$, но при достаточно больших значениях~$x$.

\begin{lemma}[Нижняя оценка для НОК]
    $K:=[1,2,3,\dots,2n+1]>4^n$.
\end{lemma}
\begin{proof}
Рассмотрим функцию $f_n := \br{x(1-x)}^n$.
Поскольку $x(1-x) < \frac14$ всюду на отрезке $[0,1]$, за исключением одной точки $x=\frac12$, получаем
\eqn{I:=\intl01 \br{x(1-x)}^n\,dx<\frac1{4^n}.}
Раскроем скобки, получим некоторый многочлен с целыми коэффициентами:
\eqn{f_n = x^n(1-x)^n=a_nx^n\spl a_{2n}x^{2n},\quad a_j\in\Z.}
Проинтегрируем его:
\eqn{I=\frac{a_n}{n+1}\spl\frac{a_{2n}}{2n+1}>0,}
поскольку $f_n > 0$ на $(0,1)$.
Заметим, что число $KI$ целое (все знаменатели убьёт множитель $K$). Оно положительное,
поэтому по крайней мере $KI \ge 1$. Пользуясь оценкой для $I$, получаем, что $K > 4^n$.
\end{proof}

\begin{theorem}\label{thm:lowerBoundary}
При $x \ge 6$ выполнена оценка $a\frac{x}{\ln(x)}\le\pi(x)$ для некоторой константы $a = \frac12\ln 2$.
\end{theorem}

\begin{proof}
По всякому числу $x$ можно однозначно определить натуральное $n$, такое что $2n+1\le x< 2n+3$.
Рассмотрим $K:=[1,2,3,\dots,2n+1]$. Рассмотрим разложение этого числа на простые множители:
\eqn{K=p_1^{k_1}\sd p_r^{k_r}.}
Заметим, что каждое простое число в диапазоне от $1$ до $2n+1$ входит в разложение $K$. Значит, $r = \pi(2n+1)$.
Далее, $p_i^{k_i} \le 2n+1$ при всех~$i$. Следовательно,
$K\le(2n+1)^{\pi(2n+1)}$. С другой стороны, по предыдущей лемме имеем $4^n < K$.
Следовательно,
\eqn{4^n < (2n+1)^{\pi(2n+1)}.}
Логарифмируя это неравенство, получаем
\eqn{\pi(2n+1)\log_2(2n+1)>n\log_24 = 2n\quad\Ra\quad\pi(2n+1)>\frac{2n}{\log_2(2n+1)}
>\frac{x-3}{\log_2(2n+1)}\stackrel{!}{\ge}\frac{\frac x2}{\log_2(2n+1)} \ge a\frac{x}{\ln x}.}
Переход, отмеченный <<!>>, обусловлен неравенством $x-3\ge \frac{x}2$, справедливым при $x \ge 6$.
\end{proof}

\begin{lemma}
$\prods{p\le x}p<4^x$.
\end{lemma}
\begin{proof}
В силу монотонного возрастания функции $4^x$ достаточно доказать это неравенство для натуральных $x$.
Будем вести индукцию по $x$. При $x =2$ и $x=3$ это верно.
Пусть теперь это неравенство верно для всех чисел, меньших чем $x$. Докажем, что оно верно и для $x$.

Если $x = 2m$ ($m \ge 2$), то всё доказано, поскольку $\prods{p \le 2m}p = \prods{p \le 2m-1}p < 4^{2m-1} < 4^x$.

Пусть теперь $x = 2m-1$. Имеем
\eqn{\prods{p \le x}p = \Br{\prods{p\le m}p} \cdot \Br{\prods{m < p \le 2m-1} p}.}
Рассмотрим число $\Cb_{2m-1}^m = \frac{(2m-1)!}{m!(m-1)!}$. Заметим, что это (целое) число делится на любое простое
число $p$, для которого $m < p \le 2m-1$, потому что в знаменателе таких больших простых делителей не встречается.
Значит,
\eqn{\hr{\Cb_{2m-1}^m}\dv \Br{\prods{m<p\le 2m-1}p},}
откуда следует, что
\eqn{\Cb_{2m-1}^m\ge\prods{m<p\le 2m-1}p.}
Пользуясь предположением индукции и полученной оценкой для второго множителя, получаем
\eqn{\label{eqn:prodIneq}\prod_{p\le x}p <4^m \Cb_{2m-1}^m.}
Далее, поскольку $\Cb_{2m-1}^{m-1}=\Cb_{2m-1}^m$ и
$\suml{k=0}{2m-1}\Cb_{2m-1}^k=2^{2m-1}$,
имеет место оценка $\Cb_{2m-1}^m\le 2^{2m-2} = 4^{m-1}$. Пользуясь ей и неравенством~\eqref{eqn:prodIneq},
получаем, что $\prods{p\le x}p< 4^m \cdot 4^{m-1} = 4^{2m-1}=4^x$, и шаг индукции полностью  доказан.
\end{proof}

\begin{theorem}\label{thm:upperBoundary}
$\pi(x)\le b\frac{x}{\ln x}$ для  константы $b = 5\ln2$.
\end{theorem}
\begin{proof}
Обозначим $k := \pi(x)$.  Выпишем все простые числа, не превосходящие $x$: $p_1<p_2<\ldots<p_k\le x$.
Перемножая очевидные неравенства $i < p_i$ по $i = 1\sco k$ и используя предыдущую лемму, получаем
\eqn{\label{eqn:factUpperIneq}k!<p_1\cdot p_2\sd p_k=\prod_{p\le x}p<4^x.}
Легко видеть, что
\eqn{\label{eqn:factLowerIneq}
(k!)^2=\br{1\cdot k}\cdot\br{2\cdot(k-1)}\sd\br{(k-1)\cdot 2}\cdot\br{k\cdot1}\ge k^k\text{ (так как каждая скобка $\ge k$)}.}
Таким образом, из \eqref{eqn:factUpperIneq} и \eqref{eqn:factLowerIneq} следует, что
\eqn{\label{eqn:factUL}k^{k/2}\le k!<4^x.}

Докажем, что $k\le5\frac{x}{\log_2 x}$. Предположим, что это не так, и $k> 5\frac{x}{\log_2 x}$. Покажем, что
\eqn{\label{eqn:logIneq}5\frac{x}{\log_2 x}\ge x^{4/5}.}
Действительно,
\eqn{5\frac{x}{\log_2x}\ge x^{4/5}\Lra x^{1/5}\ge\frac15\log_2 x=\log_2(x^{1/5})\Lra t\ge\log_2 t,}
а последнее неравенство всем хорошо известно.
Пользуясь \eqref{eqn:logIneq}, получаем:
\eqn{k^{k/2}>  \hr{5\frac{x}{\log_2 x}}^{\frac52\frac{x}{\log_2 x}} \ge (x^{4/5})^{\frac52\frac{x}{\log_2 x}}=4^x,}
а это противоречит \eqref{eqn:factUL}.
Итак,
\eqn{\pi(x) = k \le 5\frac{x}{\log_2 x},}
откуда следует утверждение теоремы.
\end{proof}

\begin{imp}
Пусть $p_1<p_2<p_3<\ldots$\т последовательность всех простых чисел. Тогда найдутся константы $\al, \be > 0$, такие что
$\al n\ln n<p_n<\be n\ln n$.
\end{imp}
\begin{proof} Пользуясь теоремами \ref{thm:lowerBoundary} и \ref{thm:upperBoundary}, получаем:
\eqn{\label{eqn:asymp}a\frac{p_n}{\ln p_n}\le\pi(p_n)=n\le b\frac{p_n}{\ln p_n}.}
Возьмём $\ln$ от этого неравенства:
\eqn{\label{eqn:asympLn}\ln a+\ln p_n-\ln\ln p_n\le\ln n\le \ln b+\ln p_n-\ln\ln p_n.}
Теперь перемножим \eqref{eqn:asymp} и \eqref{eqn:asympLn} и получим:
\eqn{ap_n\ga_n\le n\ln n\le bp_n\de_n\quad\text{где }\ga_n,\de_n \ra 1\text{ при }n\ra\bes.}
Поэтому
\eqn{0<\al\le\frac{p_n}{n\ln n}\le\be,}
что и требовалось доказать.
\end{proof}

\begin{imp}
    $\sum\frac1p$ расходится.
\end{imp}
\begin{proof}
    Имеем $p_n\le\be n\ln n$, а ряд $\sum \frac1{n\ln n}$ расходится.
\end{proof}

\subsection{Функция Чебышева и ее связь с $\pi(x)$}

\begin{df}
\emph{Функцией Чебышева} называется функция
\eqn{\psi(x)=\sums{p\le x}\hs{\frac{\ln x}{\ln p}}\ln p.}
\end{df}

Заметим, что
\eqn{\label{eqn:psiIneq}
\psi(x) = \sums{p\le x}\hs{\frac{\ln x}{\ln p}}\ln p \le
\sums{p\le x}\frac{\ln x}{\ln p}\cdot \ln p = \ln x\cdot \sums{p \le x} 1 = \ln x\cdot \pi(x).}

\begin{df}
\emph{Функцией Мангольдта} называется функция:
\eqn{\La(m)=\case{\ln p, &m=p^{\al},\\0 &\text{иначе.}}}
\end{df}

Заметим, что $\hs{\frac{\ln x}{\ln p}}= \#\hc{a \in \N\cln p^a\le x}$. Поэтому
\eqn{\label{eqn:mangoldtEq}\sums{m\le x}\La(m) = \sums{p^a \le x} \La(p^a) = \sums{p^a \le x} \ln p=
\sums{p \le x} \hs{\frac{\ln x}{\ln p}}\ln p = \psi(x).}


\begin{petit}
Судя по всему, следующее утверждение имеет более просто доказательство, чем то, которые изложено ниже.
\end{petit}

\begin{stm}
$\pi(x)\sim\frac{x}{\ln x}\Lra\psi(x)\sim x$.
\end{stm}
\begin{proof} Сначала докажем прямое утверждение.

\framebox{$\Ra$} Пусть $\frac12<\be<1$, тогда
\eqn{\br{\pi(x)-x^\be}\be\ln x\le \br{\pi(x)-\pi\hr{x^\be}}\ln x^\be \le\sums{x^\be <p\le x}\ln p\stackrel{!}{=}
\sums{x^\be < p\le x}\hs{\frac{\ln x}{\ln p}}\ln p\le\psi(x)\stackrel{!!}{\le}\pi(x)\ln x.}
Переход <<!>> обусловлен тем, что $p>x^\be >x^{1/2}$, значит, $\frac{\ln x}{\ln p}<2$, поэтому $\hs{\frac{\ln x}{\ln p}}=1$.
Что касается перехода <<!!>>, то он следует из неравенства~\eqref{eqn:psiIneq}.

Итак,
\eqn{\be\ln x\br{\pi(x)-x^\be}\le\psi(x)\le\ln x\cdot \pi(x).}
Разделим это неравенство на $x$:
\eqn{-\frac{\be\ln x}{x^{1-\be}}+\frac{\be\pi(x)}{x/\ln x}\le\frac{\psi(x)}x\le\frac{\pi(x)}{x/\ln x}.}
Зададим $\ep>0$. Пусть $\be=1-\ep$. Тогда найдется $x_0(\ep)$, для которого если
$x\ge x_0(\ep)$, то $1-\ep<\frac{\pi(x)}{x/\ln x}<1+\ep$ (по~предположению теоремы)
и $\frac{\ln x}{x^{\ep}}<\ep$. Поэтому
\eqn{(1-\ep)(-\ep+1-\ep)\le\frac{\psi(x)}x\le 1+\ep.}
А это и означает, что $\psi(x)\sim x$.

\framebox{$\Lar$} Из неравенств предыдущего пункта не сложно составить следующее соотношение:
\eqn{\frac{\psi(x)}x\le\frac{\pi(x)}{x/\ln x}\le\frac1\be \frac{\psi(x)}x+\frac{\ln x}{x^{1-\be}}.}
Опять рассмотрим $\ep>0$. Положим $\be=1-\ep$. Тогда найдется $x_0(\ep)$ такое, что если $x\ge x_0(\ep)$,
то $1-\ep<\frac{\psi(x)}x<1+\ep$ (по предположению теоремы) и $\frac{\ln x}{x^{\ep}}<\ep$. Поэтому
\eqn{1-\ep\le\frac{\pi(x)}{x/\ln x}\le\frac1{1-\ep}(1+\ep)+\ep.}
А это и означает, что $\pi(x)\sim \frac{x}{\ln x}$.
\end{proof}

Введем новую функцию
\eqn{\om(x)=\intl{1}{x}\frac{\psi(t)}t\dt.}

\begin{stm}
Если $\om(x)\sim x$, то и $\psi(x)\sim x$.
\end{stm}
\begin{proof}
Очевидно, $\psi(x)$\т монотонная функция, поэтому
\eqn{\om((1+\ep)x)-\om(x)=\intl{x}{(1+\ep)x}\frac{\psi(t)}tdt\ge\psi(x)\intl{x}{(1+\ep)x}\frac{dt}t=\psi(x)\ln(1+\ep).}
Поделим это неравенство на $x$:
\eqn{(1+\ep)\frac{\om((1+\ep)x)}{(1+\ep)x}-\frac{\om(x)}x\ge\frac{\psi(x)}x\ln(1+\ep).}
Левая часть неравенства при возрастании $x$ стремится к $\ep$, поэтому можно записать, что
\eqn{\ep\ge\ln(1+\ep)\ulim\frac{\psi(x)}x\Ra\ulim\frac{\psi(x)}x\le1.}
Абсолютно аналогично
\eqn{\om(x)-\om\br{(1-\ep)x}=\intl{(1-\ep)x}{x}\frac{\psi(t)}t\dt\le-\psi(x)\ln(1-\ep).}
Поделим это неравенство на $x$:
\eqn{\frac{\om(x)}x-(1-\ep)\frac{\om((1-\ep)x)}{(1-\ep)x}\le\frac{-\psi(x)}x\ln(1-\ep).}
Левая часть неравенства при возрастании $x$ стремится к $\ep$, поэтому можно записать, что
\eqn{\ep\le-\ln(1-\ep)\llim\frac{\psi(x)}x\Ra\llim\frac{\psi(x)}x\ge1.}
Окончательно получаем:
\eqn{1\le\llim\frac{\psi(x)}x\le\ulim\frac{\psi(x)}x\le1,}
значит, на самом деле существует обычный предел, и $\lim\frac{\psi(x)}x=1$.
\end{proof}

\subsection{Дзета-функция Римана}

\subsubsection{Определение и простейшие свойства}

\begin{df}
\emph{Дзета-функцией Римана} называется функция
\eqn{\label{eqn:dzetaFunction}\ze(s):=\suml{n=1}{\bes}\frac1{n^s}, \quad s \in \Cbb.}
\end{df}

Сформулируем некоторые очевидные свойства $\ze$\д функции.

\begin{items}{-2}
\item Ряд \eqref{eqn:dzetaFunction} сходится абсолютно в области $\Re s>1$.
\item Ряд \eqref{eqn:dzetaFunction} сходится неравномерно в области $\Re s>1$ (иначе сходился бы и гармонический ряд).
\item Ряд~\eqref{eqn:dzetaFunction} сходится равномерно по признаку Вейерштрасса в области $\Om_\de=\hc{s\cln\Re s>1+\de}$.
      Следовательно, по теореме Вейерштрасса функция $\ze(s)$ аналитична в $\Re s>1$.
\item $\ze'(s)=-\suml{n=2}{\bes}\frac{\ln n}{n^s}$ в области $\Re s > 1$ (потому что равномерно сходящийся ряд
      можно дифференцировать почленно).
\end{items}


\begin{df} \emph{Функцией Мёбиуса} называется функция
\eqn{
\mu(n):=\case{1,&n=1;\\
0,& n\dv p^2\text{ для некоторого простого }p;\\
(-1)^r,&n=p_1\sd p_r\;(p_i\text{\т различные простые числа}).}}
\end{df}

\begin{stm}
Функция $\ze(s)$ не имеет нулей в области $\Re s>1$, и
\eqn{\frac1{\ze(s)}=\suml{1}{\bes}\frac{\mu(n)}{n^s}.}
\end{stm}

\begin{proof}
Функция $\xi(s):=\suml{1}{\bes}\frac{\mu(n)}{n^s}$ аналитична в $\Re s>1$. Поскольку ряды, из которых
составлены функции~$\xi$ и~$\ze$, сходятся абсолютно, то их можно перемножать, поэтому
\eqn{\xi(s)\ze(s)=\suml{n=1}{\bes}\suml{m=1}{\bes}\frac{\mu(n)}{(mn)^s}=
\suml{k=1}{\bes}\frac1{k^s}\sums{n|k}\mu(n).}
Остается доказать, что $\sums{n|k}\mu(n)=\de_{k1}$ (символ Кронекера).
Пусть $k=p_1^{\al_1}\sd p_t^{\al_t}$. Если $n\divs k$, то
${n=p_1^{\be_1}\sd p_t^{\be_t}}$, и $\be_j<\al_j$. Слагаемые, в которых хотя бы одна из степеней $\be_i$ больше $1$,
погибнут сразу (по определению $\mu$). Останутся слагаемые, в которых все степени будут нулевыми (оно будет
всего одно и войдёт со знаком <<$+$>>), слагаемые, в которых ненулевой будет только одна степень (их будет
$\Cb_t^1$ штук, и они войдут со знаком <<$-$>>), и так далее. При $k > 1$ получаем
\eqn{\sums{n|k}\mu(n)=1-\Cb_t^1+\Cb_t^2-\ldots=(1-1)^t=0.}
Таким образом, останется только одно слагаемое при $k=1$. Но это означает, что
$\xi(s) \ze(s) = 1$.

Из полученного соотношения следует, что функции $\xi$ и $\ze$ не имеют нулей в области $\Re s > 1$.
Попутно мы доказали и требуемое равенство $\frac1{\ze(s)} = \xi(s)$.
\end{proof}

\begin{stm}[Связь функции Римана и Мангольдта]
Имеет место формула
\eqn{\frac{\ze'(s)}{\ze(s)}=-\suml{n=2}{\bes}\frac{\La(n)}{n^s}.}
\end{stm}
\begin{proof}
Имеем
\eqn{\suml{n=2}{\bes}\frac{\La(n)}{n^s}\suml{m=1}{\bes}\frac1{m^s}=\suml{k=2}{\bes}\frac1{k^s}\sums{n|k}\La(n).}
Остается доказать, что $\sums{n|k}\La(n)=\ln k$. Действительно, пусть
$k=p_1^{\al_1}\sd p_t^{\al_t}$. Тогда $\ln k = \al_1\ln p_1\spl \al_t \ln p_t$.
Тогда ненулевой вклад в сумму дадут только числа $n$ вида $p_i^{\be_i}$ ($1 \le \be_i \le \al_i$).
Следовательно,
\eqn{\sums{n|k}\La(n)=\al_1\ln p_1\spl \al_t\ln p_t = \ln k.}
Подставляя эту сумму в полученную выше формулу, получаем в точности выражение для производной $\ze$\д функции,
взятой со знаком <<$-$>>.
\end{proof}


\subsubsection{Мультипликативные функции. Формула Эйлера}

\begin{df}
Функцию $f(n)$ назовем \emph{вполне мультипликативной}, если $f(uv)=f(u)f(v)$ при всех $u,v\in\N$.
\end{df}

\begin{ex}
Функции $f(n) = n^s$ вполне мультипликативны.
\end{ex}

\begin{stm}
Пусть $f$ вполне мультипликативна, и $f \nequiv 0$. Тогда $f(1) = 1$.
\end{stm}
\begin{proof}
Пусть $f(n) \neq 0$ для некоторого $n$. Тогда $f(n) = f(n \cdot 1) = f(n) \cdot f(1)$,
а поскольку на $f(n)$ можно сократить, то $f(1) = 1$.
\end{proof}

\begin{lemma}\label{lem:primeDecomposition}
Пусть $f(n)$\т  вполне мультипликативная функция ($f \nequiv 0$), причем ряд
\eqn{S:=\suml{n=1}{\bes}f(n)}
абсолютно сходится. Тогда
\eqn{\prods{p}\br{1-f(p)}^{-1}=S,\text{ то есть }
S(x):=\prods{p\le x}\br{1-f(p)}^{-1} \ra S, \quad x\ra\bes.}
\end{lemma}

\begin{proof}
Мы уже знаем, что $f(1) = 1$. Покажем, что $\hm{f(n)} < 1$ при $n > 1$. В самом деле, если
$\hm{f(n)} \ge 1$, то и $\hm{f(n^k)} = \hm{f(n)}^k \ge 1$, а это противоречит сходимости ряда для~$S$.

Пусть $p$\т простое. Поскольку $|f(p)|<1$, то по формуле для геометрической прогрессии имеем
\eqn{\br{1-f(p)}^{-1}=\suml{k=0}{\bes}f(p)^k=\suml{k=0}{\bes}f(p^k),}
и ряд в правой части абсолютно сходится.
Абсолютно сходящиеся ряды можно перемножать, поэтому
\eqn{S(x)=\prods{p\le x}(1-f(p))^{-1}=\sums{p_j\le x}f(p_1^{k_1}\sd p_r^{k_r})=\sums{*}f(n).}
Сумма по <<$*$>> означает, что суммирование идёт по тем и только тем $n$, у которых все простые делители
не превосходят $x$. Сумма по <<${*}{*}$>> означает, что суммирование ведётся по всем $n$, у которых в разложении
есть простые числа, большие $x$. Легко видеть, что
\eqn{|S-S(x)|=\Bm{\sums{**} f(n)} \le \sums{**}\hm{f(n)}\stackrel{!}{\le}\sums{n\ge x}\hm{f(n)} \ra 0, \quad x \ra \bes,}
как остаток сходящегося ряда. На всякий случай поясним переход, отмеченный <<!>>: в сумме были числа, у которых были
простые делители, большие $x$, а мы добавили туда вообще все числа, большие $x$.
\end{proof}

\begin{imp}[Формула Эйлера]
В области $\Re s>1$ выполняется
\eqn{\ze(s)=\prods{p}\hr{1-\frac1{p^s}}^{-1}.}
\end{imp}

\begin{proof}
В качестве $f(n)$ из предыдущей леммы берем $\frac1{n^s}$ (в области $\Re s>1$ ряд абсолютно сходится).
\end{proof}

\subsubsection{Аналитическое продолжение $\ze$\д функции}

\begin{lemma}[Преобразование Абеля]\label{lem:abelTransform}
Пусть $a_k \in \Cbb$, и $g\cln [1,\bes) \ra \Cbb$\т непрерывно дифференцируемая функция.
Тогда
\eqn{\sums{k\le T}a_kg(k)=A(T)g(T)-\intl{1}{T}A(x)g'(x)dx,\text{ где }A(x):=\sums{k\le x}a_k.}
Если к тому же ряд $\suml{1}{\bes}a_kg(k)$ сходится и $A(T)g(T)\ra 0$ при $T \ra \bes$, то
\eqn{\suml{k=1}{\bes}a_kg(k)=-\intl{1}{\bes}A(x)g'(x)dx.}
\end{lemma}

\begin{proof}
Введём обозначение
\eqn{\al_k(x) := \case{0,& x < k,\\a_k,& x\ge k.}}
Рассмотрим
\begin{multline}
A(T)g(T)-\sums{k\le T}a_kg(k)=\sums{k\le T}a_k(g(T)-g(k))=\sums{k\le T}a_k\intl{k}{T}g'(x)\dx=\\=
\sums{k\le T}\intl{k}{T}\al_k(x)g'(x)\dx=\sums{k\le T}\intl{1}{T}\al_k(x)g'(x)\dx=
\intl{1}{T}\Br{\sums{k\le T}\al_k(x)}g'(x)\dx.
\end{multline}
Остается только заметить, что при $x \in [1,T]$
\eqn{\sums{k\le T}\al_k(x)=\sums{k\le x}\al_k(x)=\sums{k\le x}a_k=A(x).}
Вторая часть леммы получается из первой простым предельным переходом.
\end{proof}

\begin{theorem}[Об аналитическом продолжении $\ze$\д функции]
\label{thm:analDzetaExtension}
$\ze$\д функция аналитически продолжается в область $\hc{\Re s > 0} \wo \hc{s=1}$, причём в точке $s =1$
имеется полюс порядка $1$ c вычетом, равным~$1$.
\end{theorem}
\begin{proof}
Применим предыдущую лемму к случаю, когда
$T=N\in\N$, $a_i\equiv 1$, а $g(x)=x^{-s}$. Тогда $g'(x)=-sx^{-(s+1)}$, а $A(x)=\sums{k\le x}1=[x]$.
Воспользуемся предыдущей леммой и свойством $[x]=x-\hc{x}$:
\begin{multline}
    \suml{k=1}{N}\frac1{k^s}=
    \frac{N}{N^s}+s\intl{1}{N}\frac{[x]}{x^{s+1}}\dx=
    \frac1{N^{s-1}}+s\intl{1}{N}\frac{x}{x^{s+1}}\dx-s\intl{1}{N}\frac{\{x\}}{x^{s+1}}\dx=\\
  = \frac1{N^{s-1}}+\hr{\frac{s}{s-1}-\frac{s}{N^{s-1}(s-1)}}-s\intl{1}{N}\frac{\{x\}}{x^{s+1}}\dx=1+\frac{1}{s-1}-\frac{1}{N^{s-1}
    (s-1)}-s\intl{1}{N}\frac{\{x\}}{x^{s+1}}\dx.
\end{multline}
Пусть сначала $\Re s > 1$. Перейдём к пределу при $N\ra\bes$. Получим
\eqn{\ze(s) = 1+\frac1{s-1}-s\intl{1}{\bes}\frac{\{x\}}{x^{s+1}}\dx.}
Нам бы хотелось принять эту формулу за определение $\ze$\д функции при $\Re s >0$.
Для этого надо доказать аналитичность правой части (точнее, интегрального слагаемого).
Пусть $s=\si+it$. Разобьём интеграл на части. Пусть
\eqn{\ph_n(s) :=\intl{n}{n+1}\frac{\hc{x}}{x^{s+1}}\dx=\intl{n}{n+1}\frac{x-n}{x^{s+1}}\dx.}
Очевидно, что $\ph_n(s)$ аналитичны в $\Re s>0$.
Значит, интеграл в правой части есть сумма ряда из аналитических функций. Этот ряд равномерно сходится,
так как при $\si\ge\de>0$ он мажорируется числовым рядом:
\eqn{|\ph_n(s)|\le\intl{n}{n+1}\frac{\hc{x}}{x^{\si+1}}\dx\le\frac1{n^{\si+1}}\le\frac1{n^{\de+1}}.}
Осталось применить теорему Вейерштрасса.

Далее видно, что у полученного продолжения есть полюс в точке $s=1$ с вычетом, равным $1$.
\end{proof}

\begin{imp}\label{imp:dzetaExtension}Попутно мы получили такое выражение для $\ze$\д функции:
\eqn{\ze(s)=\suml{n=1}{N}\frac1{n^s}+\frac1{(s-1)N^{s-1}}-s\intl{N}{\bes}\frac{\{x\}}{x^{s+1}}dx,\quad \Re s>0.}
\end{imp}

\subsubsection{Оценки $\ze$\д функции и её производной}

\begin{lemma}
Пусть $s=\si+it$. Пусть $\si \in [1,2]$, а $|t|\ge 3$. Тогда
\eqn{|\ze(s)|\le5\ln|t|,\qquad  |\ze'(s)|\le8\ln^2|t|.}
\end{lemma}
\begin{proof}
Оценим сначала саму $\ze$\д функцию. Воспользуемся следствием \ref{imp:dzetaExtension} и оценим каждое слагаемое.
Пусть $N=[|t|]$. Тогда
\eqn{\hm{\suml{n=1}{N}\frac1{n^s}}\le\suml{n=1}{N}\frac1{n^{\si}}\le\suml{n=1}{N}\frac1n\le1+\intl{1}{N}\frac{dx}x=1+\ln N\le2\ln N\le2\ln|t|.}
\eqn{\hm{\frac1{(s-1)N^{s-1}}}=\frac1{|s-1|N^{\si-1}}\le\frac1{|s-1|}\le\frac13.}
\eqn{\hm{s\intl{N}{\bes}\frac{\{x\}}{x^{s+1}}\dx} \le
(|t|+2)\intl{N}{\bes}\frac1{x^{\si+1}}\dx \le (|t|+2)\intl{N}{\bes}\frac{dx}{x^2}=\frac{|t|+2}{N} \le
\frac{|t|+2}{|t|-1}=1+\frac3{|t|-1}\le\frac52.}
Итого:
\eqn{|\ze(s)|\le2\ln|t|+\frac13+\frac52<5\ln|t|.}
Дифференцируя выражение для $\ze$, получаем
\eqn{\ze'(s)=-\suml{1}{N}\frac{\ln n}{n^s}-\frac1{(s-1)^2 N^{s-1}}-\frac{\ln N}{(s-1)N^{s-1}}-
\intl{N}{\bes}\frac{\{x\}}{x^{s+1}}\dx+s\intl{N}{\bes}\frac{\{x\}\ln x}{x^{s+1}}\dx.}
Оценим слагаемые в этом выражении (начнём с более простых):
\eqn{\hm{\frac1{(s-1)^2N^{s-1}}}\le\frac19\frac1{N^{s-1}}\le\frac19.}
\eqn{\hm{\frac{\ln N}{(s-1)N^{s-1}}}\le\frac13\ln N\le\frac13\ln|t|.}
\eqn{\hm{\intl{N}{\bes}\frac{\{x\}}{x^{s+1}}\dx}\le\intl{N}{\bes}\frac1{x^2}\dx=\frac1N\le\frac12.}
\mln{\bbm{s\intl{N}{\bes}\frac{\hc{x}\ln x}{x^{s+1}}\dx} \le
\br{|t|+2}\intl{N}{\bes}\frac{\ln x}{x^2}dx=
\br{|t|+2}\bbr{-\frac{\ln x}{x}\biggr|_N^{\bes}+\intl{N}{\bes}\frac1{x^2}dx}=\\
=\frac{\ln N+1}{N}(|t|+2) \le 2\ln|t|\frac{|t|+2}{|t|-1} \le 5\ln |t|.}
При оценке первого слагаемого мы воспользуемся монотонным убыванием функции $\frac{\ln x}{x}$ при $x\in(e,\bes)$
и применим интегральный признак сходимости:
\mln{
\suml{1}{N} \hm{\frac{\ln n}{n^s}} =
\suml{2}{N} \frac{\ln n}{n^\si} \le
\suml{2}{N} \frac{\ln n}{n} = \frac{\ln 2}{2} + \suml{3}{N}\frac{\ln n}{n} \le
\frac{\ln2}2+\frac{\ln3}3+\intl{3}{N}\frac{\ln x}{x}\dx\le\\ \le
\frac{\ln2}2+\frac{\ln3}3+\frac12\ln^2N-\frac12\ln^23\le
\frac{\ln2}2+\frac12\ln^2N<\ln^2N\le\ln^2|t|.}
Итого:
\eqn{|\ze'(s)|\le\frac19+\frac13\ln|t|+\frac12+5\ln|t|+\ln^2|t|\le\hr{\frac12+\frac12+\frac19+5+1}\ln^2|t|\le8\ln^2|t|.}
\hfill\end{proof}

\subsubsection{Гипотеза Римана и теоремы о нулях $\ze$\д функции}

Для $\ze$\д функции имеет место следующее соотношение:
\eqn{\ze(1-s) = 2^{1-s}\pi^{-s} \cos\frac{\pi s}{2}\cdot\Ga(s)\ze(s),}
где $\Ga$\т гамма\д функция  Эйлера.
Таким образом, значения функции $\ze$ слева и справа от прямой $\Re s = \frac12$ связаны некоторым уравнением.
%Имеет место соотношение,
%Введем новую функцию
%\eqn{\eta(s):=\pi^{-1/2}\Gamma\hr{\frac{s}2}\ze(s).}
%Она замечательна тем, что при $\Re s \in (0,1)$ выполняется равенство $\eta(1-s)=\eta(s)$,
%\те эта функция симметрична относительно прямой $\hc{\Re s=\frac12}$.
Этот факт установил Б.\,Риман в 1859 году.

\begin{petit}
В лекциях на этом месте было написано что\д то несуразное про функцию $\eta$,
поэтому этот фрагмент воспроизведён по книге~\cite{gnsh}.
\end{petit}

Риман также высказал предположение о том, что все нули функции $\ze(s)$, лежащие в области $0<\Re s<1$,
расположены на этой прямой (\emph{гипотеза Римана}).
В 1914~году Харди доказал, что на $\Re s=\frac12$ лежит бесконечное число нулей $\ze$\д функции; после
этого было доказано, что как минимум треть нулей лежит на этой прямой.

\dmvnpiclh{pictures.3}{0}
Большинство\dmvnpicla{pictures.3}{1}{0pc} математиков верят, что гипотеза верна. На сегодняшний день проверены первые $1\,500\,000\,000$
решений. Гипотеза Римана входит в число семи главных нерешенных математических проблем. За её доказательство
Институт математики Клея (Кембридж, штат Массачусетс) выплатит приз в \$1 млн.

\begin{lemma}\label{lem:prodBound}
\dmvnpiclh{pictures.3}{0}
Пусть $r\in(0,1)$, $\ph\in\R$. Тогда
\eqn{\Pi:=(1-r)^3\cdot |1-re^{i\ph}|^4\cdot |1-re^{2i\ph}|\le1.}
\end{lemma}

\begin{proof}
\dmvnpiclh{pictures.3}{0}
Возьмем $-\ln$ от левой и правой частей доказываемого неравенства. Получим
\eqn{\begin{aligned}
        -\ln\Pi&=-3\ln(1-r)-4\ln|1-re^{i\ph}|-\ln|1-re^{2i\ph}|=\\
        &=-3\Re \ln(1-r)-4\Re \ln(1-re^{i\ph})-\Re\ln(1-re^{2i\ph})=\\
        &=3\Re\suml{1}{\bes}\frac{r^n}n+4\Re \suml{1}{\bes}\frac{r^n}ne^{in\ph}+\Re \suml{1}{\bes}\frac{r^n}ne^{2in\ph}=\\
        &=\Re\suml{1}{\bes}\frac{r^n}n(3+4e^{in\ph}+e^{2in\ph})=\suml{1}{\bes}\frac{r^n}n(3+4\cos n\ph+\cos 2n\ph)\ge0,
    \end{aligned}}
потому что $3+4\cos x+\cos 2x=2\cos^2x+4\cos x+2=2(\cos x+1)^2\ge0$.
\end{proof}

\begin{lemma}\label{lem:ProdZetasGeOne}
Если $\si=\Re s > 1$, то
\eqn{P:=|\ze^3(\si)\ze^4(\si+it)\ze(\si+2it)|\ge1.}
\end{lemma}
\begin{proof}
Воспользуемся формулой Эйлера: $\ze(s)=\prods{p}\hr{1-\frac1{p^s}}^{-1}$. Применим предыдущую лемму, взяв $r = \frac{1}{p^\si}$ и $\ph = -t\ln p$.
Имеем:
\eqn{P=\prods{p}\hm{\hr{1-\frac1{p^{\si}}}^3\hr{1-\frac1{p^{\si}}\frac1{p^{it}}}^4\hr{1-\frac1{p^\si}\frac1{p^{2it}}}}^{-1}\ge1.}
\hfill\end{proof}

\begin{lemma}
Если $\Re s=1$, то $\ze(s)\neq0$.
\end{lemma}
\begin{proof}
Предположим противное: найдется такая точка $s_0=1+it$, что $\ze(s_0)=0$. Пусть $s=\si+it$, и $\si>1$. Имеем
\eqn{|\ze(s)|=|\ze(s)-\ze(s_0)|=O(|s-s_0|)=O(\si-1).}
\eqn{\ze(\si)=\suml{1}{\bes}\frac1{n^{\si}}\le1+\intl{1}{\bes}\frac{dx}{x^{\si}}=1+\frac1{\si-1}=\frac{\si}{\si-1}=O\hr{\frac1{\si-1}}.}
Далее, имеем $|\ze(\si+2it)|=O(1)$, так как $\si+2it\ra 1+2it$ при $\si\ra1$, а в~точке $1+2it$ дзета\д функция
аналитична, и потому в этой точке у неё есть конечный предел.

Теперь оценим порядок функции $P$ из предыдущей леммы. Имеем
\eqn{P=O\hr{\frac1{(\si-1)^3}(\si-1)^4}=O(\si-1)\ra 0, \quad \si\ra1,}
что противоречит предыдущей лемме.
\end{proof}

\begin{lemma}
\dmvnpiclh{pictures.2}{0}
Пусть\dmvnpicla{pictures.2}{2}{-1pc} $\si \in [1,2]$, а $|t|\ge 3$. Тогда
\eqn{\hm{\frac{\ze'(s)}{\ze(s)}}\le C\ln^9|t|.}
\end{lemma}

\begin{proof}
\dmvnpiclh{pictures.2}{0}
Оценка сверху на $|\ze'(s)|$ у нас уже была. Получим оценку снизу на $|\ze(s)|$.
Положим $\si_1(t) := 1+\frac1{C\ln^9|t|}$, где $C=2^{23}$.
Разобьем нашу область на две части (см.~рис.~2). Первая\т $\si\ge \si_1(t)$, а
вторая\т $1\le\si\le\si_1(t)$.

\dmvnpiclh{pictures.2}{-6}
\pt{1} Пусть сначала выполнено неравенство $\si \ge \si_1(t)$. Из доказательства предыдущей леммы мы знаем, что
\eqn{\ze(\si)\le\frac2{\si-1}\le2C\ln^9|t|.}
Из известной оценки для модуля $\ze$-функции получаем:
\eqn{|\ze(\si+2it)|\le5\ln(2|t|)\le16\ln|t|.}
Теперь применяем лемму \ref{lem:ProdZetasGeOne}:
\eqn{1\le|\ze(s)|^4(2C\ln^9|t|)^3 16\ln|t|\Ra|\ze(s)|\ge(2C)^{-3/4}\frac12\ln^{-7}|t|=2^{-19}\ln^{-7} |t|=16C^{-1}\ln^{-7}|t|.}

\pt{2} Пусть теперь $1 \le \si \le \si_1(t)$. Имеем
\eqn{|\ze(s)-\ze(\si_1+it)|=\bbm{\intl{\si}{\si_1}\ze'(u+it)\,du}\le|\si_1-\si|\cdot 8\ln^2|t|\le8 C^{-1}\ln^{-7}|t|.}
Значит, по неравенству треугольника имеем
\eqn{|\ze(s)|\ge|\ze(\si_1+it)|-8C^{-1}\ln^{-7}|t|\ge8C^{-1}\ln^{-7}|t|.}
Здесь мы воспользовались оценкой для $|\ze(\si_1+it)|$, полученной выше.
Итого получаем
\eqn{\hm{\frac{\ze'(s)}{\ze(s)}}\le\frac{8\ln^2|t|}{8C^{-1}\ln^{-7}|t|}=C\ln^9|t|.}
\hfill\end{proof}

\subsection{Доказательство асимптотического закона простых чисел}


\begin{lemma}
Пусть $a,b>0$, тогда
\eqn{\frac1{2\pi i}\intl{a-i\bes}{a+i\bes}\frac{b^s}{s^2}\ds=
\case{\ln b,&b\ge1,\\0,&0< b<1.}}
\end{lemma}

\begin{proof}
\dmvnpiclh{pictures.1}{0}
Обозначим\dmvnpicla{pictures.1}{3}{-1.5pc} $r := |s|$.
Пусть сначала $b \ge 1$. Будем интегрировать по контуру, отмеченному на рис.~3.
\eqn{\hm{\frac{b^s}{s^2}}=\frac{b^{\si}}{r^2}\le\frac{b^a}{r^2}\text{ (здесь
мы воспользовались тем, что $b\ge1$)}.}
Пусть $C$\т дуга окружности, входящая в контур интегрирования. Следовательно,
\eqn{\Bm{\frac1{2\pi i}\ints{C}\frac{b^s}{s^2}\ds}\le r\frac{b^a}{r^2} = \frac{b^a}{r} \ra0, \quad r\ra\bes.}

\dmvnpiclh{pictures.1}{0}
По теореме Коши о вычетах
\eqn{\frac1{2\pi i}\ints{\Gamma}\frac{b^s}{s^2}\ds=\res_{s=0}\frac{b^s}{s^2}=\ln b.}
Остается перейти к пределу при $r\ra\bes$.

\dmvnpiclh{pictures.6}{0}
Пусть\dmvnpicla{pictures.6}{4}{-.5pc} теперь $0<b<1$. В этом случае будем интегрировать по другому пути, показанному на рис.~4.
Получаем
\eqn{\label{eqn:intEval}\Bm{\frac1{2\pi i}\ints{C}\frac{b^s}{s^2}\ds}\le
\frac1{2\pi}2\pi r\frac{b^a}{r^2}\ra 0, \quad r \ra \bes.}
Здесь мы воспользовались тем, что для $b\le1$ верно неравенство $b^{\si}\le b^a$, $\si\ge a$.

\dmvnpiclh{pictures.6}{0}
Внутри контура $\Ga$ особенностей у подынтегральной функции нет. Остаётся применить теорему Коши и перейти к пределу.
\end{proof}

\dmvnpiclh{pictures.6}{0}
Далее для сокращения выкладок введём обозначение:
\eqn{\xi(s) := -\frac{\ze'(s)}{\ze(s)}.}

\begin{lemma}
Пусть $x>1$. Тогда функция $\om(x)$ представляется абсолютно сходящимся интегралом
\eqn{\om(x) = J(x):=\frac1{2\pi i}\intl{2-i\bes}{2+i\bes}\xi(s)\cdot \frac{x^s}{s^2}\ds.}
\end{lemma}
\begin{proof}
Докажем абсолютную сходимость. Интегрирование ведется по прямой $s=2+it$.
Вспомним, что
\eqn{\xi(s) = \suml{n=2}{\bes} \frac{\La(n)}{n^s}.}
Очевидно, что $\La(n) \le \ln n$. Следовательно,
\eqn{|\xi(s)| \le\suml{n=2}{\bes}\frac{\La(n)}{n^2}\le\suml{n=2}{\bes}\frac{\ln n}{n^2} \le C.}
Поэтому
\eqn{\hm{\xi(s)\cdot\frac{x^s}{s^2}}\le\frac{Cx^2}{4+t^2}.}
Значит, $J(x)$ оценивается сходящимся интегралом и потому сходится абсолютно.

Докажем, что $J(x)=\om(x)$. Разделим сумму ряда для $\xi(s)$ на два слагаемых:
\eqn{\xi(s)=\suml{n=2}{N}\frac{\La(n)}{n^s}+R_N(s).}
\eqn{\label{eqn:restBound}|R_N(s)|\le\suml{N+1}{\bes}\frac{\La(n)}{n^2}\le
\suml{N+1}{\bes}\frac{\ln n}{n^2}=:\rho_N\ra 0,\quad N \ra \bes.}
\eqn{J(x)=\suml{n=2}{N}\La(n)\frac1{2\pi i}\intl{2-i\bes}{2+i\bes}\frac{\hr{\frac{x}{n}}^s}{s^2}\ds+
\frac1{2\pi i}\intl{2-i\bes}{2+i\bes}R_N(s)\frac{x^s}{s^2}\ds.}
К интегралу в первом слагаемом применим предыдущую
лемму, а во втором слагаемом заменим $R_N(s)$ на его оценку~\eqref{eqn:restBound} сверху:
\eqn{J(x)=\sums{n\le x}\La(n)\ln \hr{\frac{x}{n}}+I,\qquad
|I|\le\frac{\rho_N x^2}{2\pi}\intl{-\bes}{+\bes}\frac{dt}{4+t^2}\ra 0,\quad N\ra\bes.}
Теперь применим преобразование Абеля (лемма~\ref{lem:abelTransform}) для $a_n=\La(n)$, $g(t)=\ln\hr{\frac{x}{t}}$. Согласно
установленной ранее формуле~\eqref{eqn:mangoldtEq}, $A(x)=\sums{n\le x}\La(n)=\psi(x)$. Поэтому
\eqn{J(x)=\psi(x)\cdot0+\intl{1}{x}\frac{\psi(t)}{t}\,dt=\om(x).}
Лемма доказана.
\end{proof}

Введём еще одно обозначение:
\eqn{\xi_x(s) := \hr{-\frac{\ze'(s)}{\ze(s)}}\cdot\frac{x^{s-1}}{s^2}.}

\begin{lemma}
\dmvnpiclh{pictures.4}{0}
Пусть\dmvnpicla{pictures.4}{5}{-1pc} $0<\eta<1$, $T\ge3$ и в области $\si \in [\eta,1]$, $|t|\le T$ у дзета\д функции нет нулей.
Тогда
\eqn{\om(x)=x\br{1+R(x)},\quad R(x)=
\frac1{2\pi i}\ints{\Ga}\xi_x(s)\ds\ra 0, \quad x\ra\bes.}
\end{lemma}
\begin{proof}
\dmvnpiclh{pictures.4}{0}
\eqn{\frac1{2\pi i}\ints{\Ga}\xi_x(s)\cdot x \ds= \res_{s=1}\br{\xi_x(s)\cdot x}.}

\dmvnpiclh{pictures.4}{-4}
Посчитаем, чему равен этот вычет. Мы знаем (теорема~\ref{thm:analDzetaExtension}), что
$\ze(s)=\frac{f(s)}{s-1}$ где $f(1)=1$.
Поэтому
\eqn{-\xi(s)=-\frac1{s-1}+\frac{f'(s)}{f(s)}, \text{ откуда } \res_{s=1}\br{\xi_x(s)\cdot x}=x.}

\dmvnpiclh{pictures.5}{-13}
Оценим\dmvnpicla{pictures.5}{6}{-1pc} теперь подынтегральную функцию на отрезке $BC$:
\eqn{\hm{\xi_x(s)\cdot x}\le C\ln^9|t|\frac{x^{\si}}{\si^2+t^2}.}
Поэтому справедлива следующая оценка для интеграла по $BC$:
\eqn{\Bm{\frac1{2\pi i}\ints{BC}\xi_x(s)\cdot x \ds}\le
\frac1{2\pi}\intl{1}{2}\frac{C\ln^9|t|}{t^2}x^{\si}\,d\si\le\frac{C}{2\pi}\frac{\ln^9|t|}{t^2}x^2\xra{t\ra\bes}0.}
Все необходимые оценки получены. По теореме Коши
\eqn{\om(x)= x+x\cdot\frac1{2\pi i}\ints{-\Ga}\xi_x(s)\ds=x\br{1+R(x)}.}
Теперь докажем, что $R(x)\ra0$ при $x\ra\bes$. Зафиксируем число $\ep>0$ и покажем, что найдётся $x_0$ такое, что при
$x>x_0$ будет выполнена оценка $|R(x)|<\ep$. Имеем
\eqn{\label{eqn:intBoundBC}\hm{\frac1{2\pi i}\intl{1+iT}{1+i\bes}\xi_x(s)\ds}\le
\frac1{2\pi}\intl{T}{\bes}\frac{C\ln^9|t|}{1+t^2}dt\le\frac{\ep}5.}
Такой оценки мы добились именно за счет выбора~$T$.

Остается подобрать нужное $\eta$. Поскольку на отрезке $[1-iT,1+iT]$ функция $\ze(s)$ в нуль не обращается, то для
каждой точки этого отрезка найдется некая её окрестность в которой $\ze(s)\neq 0$. Отрезок\т компакт, поэтому
можно выбрать конечное подпокрытие такими кружочками. Значит, найдётся $\eta$ столь близкое к~$1$, что отрезок
$[\eta-iT,\eta+iT]$ окажется покрытым этими кружочками.

Пусть $\max_{DEFG}\hm{\frac{\ze'(s)}{\ze(s)}\frac1{s^2}}=M$. Тогда за счет увеличения $x$ можно добиться
следующих оценок:
\eqn{\label{eqn:intBoundDEFG}
\hm{\frac1{2\pi i}\ints{ED}\xi_x(s)\ds}\le
\frac1{2\pi}\intl{-\bes}{1}Mx^{\si-1}d\si=\frac{M}{2\pi}\frac{x^{\si-1}}{\ln x}\Biggl|_{-\bes}^{1}\le
\frac{M}{2\pi\ln x}<\frac{\ep}5.}
Аналогично
\eqn{\label{eqn:intBoundFG}\hm{\frac1{2\pi i}\ints{FG}\xi_x(s)\ds}<\frac{\ep}5.}
\eqn{\label{eqn:intBoundEF}\hm{\frac1{2\pi i}\ints{EF}\xi_x(s)\ds}\le
\frac1{2\pi}\intl{-T}{T}Mx^{\eta-1}dt=\frac{MT}{\pi}x^{\eta-1}<\frac{\ep}5.}
Теперь соединяем вместе оценки \eqref{eqn:intBoundBC}, \eqref{eqn:intBoundDEFG}, \eqref{eqn:intBoundEF} и \eqref{eqn:intBoundFG} и получаем, что $|R(x)|<\ep.$
\end{proof}

Итак, мы доказали, что $\om(x) \sim x$. Это завершает доказательство асимптотического закона.

\section{Теорема Дирихле}

Основным результатом данной главы будет теорема о простых числах в арифметических прогрессиях,
доказанная Дирихле в 1839 году. Она утверждает, что если первый член и разность арифметической
прогрессии суть взаимно простые натуральные числа, то такая прогрессия содержит бесконечно много простых чисел.

\subsection{Частные случаи теоремы Дирихле. Сравнения по модулю}

\subsubsection{Простейший частный случай: $a_n = 4n+3$}

В качестве разминки докажем нашу теорему в частном случае.

\begin{stm}
В последовательности $\hc{4n+3}$ бесконечно много простых чисел.
\end{stm}
\begin{proof}
Предположим, что это не так. Пусть $p_1,p_2\sco p_r$\т  все простые числа вида $4n+3$. Рассмотрим число
$N:=4p_1\sd p_r+3$. Разложим $N$ в произведение простых: $N=q_1\sd q_s$. Очевидно, числа $q_i$
не могут быть чётными, поэтому либо $q_i=4k_i+1$, либо $q_i=4k_i+3$. Если бы все числа $q_i$ были вида
$4k_i+1$, то и их произведение тоже имело бы такой вид, а это не так. С другой стороны, ни одно из $q_i$
не может совпадать с каким\д либо из чисел $p_j$ по соображениям делимости. Противоречие.
\end{proof}

\begin{petit}
В 1775~г. Эйлер доказал бесконечность количества простых вида $100n+1$. Общее доказательство утверждения о
бесконечности простых вида $an\pm 1$ можно найти в~\cite{hass}. Теорема Дирихле\т куда более общий факт.
\end{petit}

\subsubsection{Сравнения по модулю и их простейшие свойства}

\begin{df}
Два целых числа $a$ и $b$ называются \emph{сравнимыми по модулю $m$} (обозначается $a\equiv b \pmod{m}$), если
$m\divs(a-b)$ или, что то же самое, если $a$ и $b$ имеют одинаковые остатки при делении на $m$.
\end{df}

\textbf{Свойства:}
\begin{nums}{-2}
\item Если $a\equiv b\pmod{m}$, $c\equiv d\pmod{m}$, то $a\pm c\equiv b\pm d\pmod{m}$.
\item $a\equiv b\pmod{m}$, $c\equiv d\pmod{m}$, то $ac\equiv bd\pmod{m}$, ибо $ac-bd=ac-bc+bc-bd\bw=c(a-b)\bw+b(c-d)\dv m$.
\item Если $ab\equiv ac\pmod{m}$ и $(a,m)=1$, то $b\equiv c\pmod{m}$. Действительно, если
      $m\mid(ab-ac)$ и $m$ не делит $a$, то по лемме~\ref{lem:div} получаем, что $m\mid(b-c)$, а это и требовалось.
\end{nums}

\begin{note}
Условие $(a,m)=1$ в последнем свойстве существенно: $4\equiv 0\pmod{4}$, но из этого не следует, что $2\equiv 0\pmod{4}$.
\end{note}

Итак, мы разбили все множество $\Z$ на классы сравнимых по модулю $m$ элементов (классы вычетов по модулю~$m$):
$\ol{0},\ol{1}\sco \ol{m-1}$. Из свойств, указанных выше, сразу следует, что классы вычетов по модулю $m$ образуют кольцо,
обозначаемое $\Z/m\Z$.

\begin{petit}
Чтобы не доказывать эти свойства вычетов, можно было бы сослаться на алгебру и сказать: рассмотрим факторкольцо $\Z/m\Z$.
Очень часто для кольца $\Z/m\Z$ используется более короткое обозначение: $\Z_m$. Мы тоже будем его использовать.
\end{petit}

\begin{lemma}
    При всех $a\in\Z$, для которых $(a,m)=1$, уравнение $ax\equiv b\pmod{m}$ имеет единственное решение в~$\Z_m$.
\end{lemma}
\begin{proof}
Рассмотрим отображение $S\cln \Z_m \ra \Z_m$, определённое по правилу $S\cln\ol{k}\ra \ol{ak}$.
Заметим, что $S$ инъективно: (если $ak_1\equiv ak_2\pmod{m}$, то $k_1\equiv k_2\pmod{m}$).
Но инъективное отображение конечных множеств одинаковой мощности биективно, значит, найдётся ровно одно $x\in \Z_m$, для которого $\ol{ax} = \ol{b}$.
\end{proof}

\begin{imp}
Если $(a,m)=1$, то элемент $\ol a$ обратим в $\Z_m$.
\end{imp}
\begin{proof}
Уравнение $ax\equiv1\pmod{m}$ разрешимо. А это и означает, что $x = a^{-1}$.
\end{proof}

\begin{df}
Количество натуральных чисел, не превосходящих $m$ и взаимно простых с $m$, называется \emph{функцией Эйлера} и обозначается~$\ph(m)$.
\end{df}

\begin{stm}
Количество обратимых элементов в $\Z_m$ равно~$\ph(m)$. Они образуют группу, которая обозначается~$\Z_m^*$.
\end{stm}
\begin{proof}
Первое сразу следует из предыдущего следствия и определения функции Эйлера. Доказательство второго утверждения
предоставляется читателю.
\end{proof}

Как вычислять функцию $\ph(m)$? Вообще говоря, вычисление $\ph$ для произвольного $m$\т это алгоритмически трудоёмкая задача.
Однако, если $p$\т простое, то ясно, что $\ph(p)=p-1$. Легко также видеть, что $\ph(p^k)\bw=p^k-p^{k-1}=p^{k-1}(p-1)$,
поскольку лишь числа, кратные $p$, не~взаимно просты с~$p^k$.

\begin{df}
Функция $f$ называется \emph{мультипликативной}, если для любых взаимно простых чисел $a$ и~$b$ имеет место равенство $f(ab)=f(a)f(b)$.
\end{df}

\begin{note}
Всякая вполне мультипликативная функция является мультипликативной, но не~наоборот.
\end{note}

\begin{lemma}
Функция Эйлера мультипликативна.
\end{lemma}
\begin{proof}
Пусть $(a,b) =1$.
Рассмотрим множество чисел
\eqn{M := \hc{m = ak+bl \bvl \quad k=0\sco b-1, \quad  l=0\sco a-1}.}
Докажем, что все они различны. Действительно, если $ak_1+bl_1=ak_2+bl_2$, то $b\divs (ak_1-ak_2)$, значит, $b\divs(k_1-k_2)$.
Воспользовавшись тем, что $|k_1-k_2|<b$, заключаем, что $k_1=k_2$, поэтому и $l_1=l_2$.

Теперь докажем, что $(ak+bl,ab)=1$ тогда и только тогда, когда $(a,l)=1$ и $(k,b)=1$.

\framebox{$\Ra$} Предположим противное: найдётся $p$ такое, что $p\mid a$ и $p \mid l$.
Но тогда $p\mid (ak+bl)$ и $p\mid ab$\т противоречие. Взаимная простота $k$ и $b$ доказывается симметрично.

\framebox{$\Lar$} Предположим противное: найдётся $p$ такое, что $p\mid(ak+bl)$ и $p \mid ab$. Тогда $p$ делит либо $a$, либо $b$. Пусть,
для определённости, $p\mid a$. Тогда $p\mid bl$. Но $p$ не может делить $b$ (\тк $a$ и $b$ взаимно просты),
значит $p\mid l$. Но это противоречит тому, что $(a,l)=1$.

Докажем, что $ak+bl$ лежат в разных классах вычетов по модулю $ab$. Действительно, если
$ak_1+bl_1\bw\equiv ak_2+bl_2 \pmod{ab}$, тогда $a\mid(a(k_1-k_2)+b(l_1-l_2))$, поэтому $a\mid (l_1-l_2)$, значит, $l_1=l_2$.
Аналогично, $k_1=k_2$.

Итак, чисел, взаимно простых с $ab$ и меньших $ab$, столько, сколько существует в $M$ чисел $ak+bl$,
для которых $(a,l)=1$, и $(k,b)=1$. А это и означает, что $\ph(ab)=\ph(a)\ph(b)$.
\end{proof}

\begin{theorem}[Эйлер]
Если $(a,m)=1$, то $a^{\ph(m)}\equiv1 \pmod{m}$.
\end{theorem}
\begin{proof}
\pt{1} Группа $\Z_m^*$ имеет порядок $\ph(m)$. А по теореме Лагранжа порядок элемента делит
порядок группы, значит $\ol{a}^{\ph(m)}=\ol{1}\Ra a^{\ph(m)}\equiv1 \pmod{m}$.

\pt{2} Можно доказать эту теорему, не ссылаясь явно на алгебру. Пусть $b_1\sco b_{\ph(m)}$\т
все числа из отрезка $[1,m]$, взаимно простые с $m$. Легко проверить,
что числа $ab_1\sco ab_{\ph(m)}$ тоже взаимно просты с модулем и лежат в разных классах вычетов. Значит,
существует биекция между $\hc{b_1\sco b_{\ph(m)}}$ и $\hc{ab_1\sco ab_{\ph(m)}}$. Поэтому
\eqn{b_1\sd b_{\ph(m)}\equiv ab_1\sd ab_{\ph(m)}\pmod{m},}
откуда $a^{\ph(m)}\equiv1\pmod{m}$.
\end{proof}

\begin{imp}[Малая теорема Ферма]
Если простое число $p$ не делит $a$, то $a^p\equiv a\pmod{p}$.
\end{imp}

\begin{stm}
Множество простых вида $4n+1$ бесконечно.
\end{stm}
\begin{proof}
Предположим противное: $p_1\sco p_r$\т  все простые такого вида. Составим число
$N:=4p_1^2\sd p_r^2+1\bw=a^2+1$, где $a=2p_1\sd p_r$. Пусть $q$\т простой делитель $N$. Очевидно, что
$q\neq p_1\sco p_r$. Имеем $a^2+1\equiv 0 \pmod{q}$, то есть $a^2\equiv-1\pmod{q}$. Возведем левую и правую части
последнего выражения в степень $\frac{q-1}{2}$ (она целая, поскольку $q$, очевидно, нечётное). Получим
\eqn{a^{q-1}\equiv(-1)^{\frac{q-1}{2}}\pmod{q}.}
Пользуемся теоремой Ферма и получаем, что $q=4k+1$. Противоречие.
\end{proof}

\subsubsection{Ещё одно доказательство бесконечности\\множества простых чисел вида $4n\pm1$}

Хотя этот факт нами уже установлен в предыдущих параграфах, не лишним будет узнать доказательство,
предложенное Эйлером. Похожий метод будет использован в общем случае, функции $\chi$ назовутся \emph{характерами},
а функции $L_i$\т \emph{$L$\д функциями Дирихле.}

\begin{theorem}
\eqn{\label{eqn::Lseries}\sums{p\equiv1(4)}\frac1p=\bes,\qquad\sums{p\equiv3(4)}\frac1p=\bes.}
\end{theorem}
\begin{proof}
Для вещественного параметра $s$ введем новые функции
\eqn{\begin{aligned}
L_0(s)&:=1+\frac1{3^s}+\frac1{5^s}+\ldots=\suml{0}{\bes}\frac1{(2n+1)^s},\\
L_1(s)&:=1-\frac1{3^s}+\frac1{5^s}-\ldots=\suml{0}{\bes}\frac{(-1)^n}{(2n+1)^s}.
\end{aligned}}
Оба ряда абсолютно сходятся в области $s>1$. Пусть
\eqn{\chi_0(n):=\case{0, & n=2k,\\
1, & n=2k+1,}}
\eqn{\chi_1(n):=\case{
\phantom{-}0, & n=2k,\\
\phantom{-}1, & n\equiv1 \pmod{4},\\
-1,& n\equiv3\pmod{4}.}}
Тогда
\eqn{L_0(s)=\suml{1}{\bes}\frac{\chi_0(n)}{n^s},\qquad L_1(s)=\suml{1}{\bes}\frac{\chi_1(n)}{n^s}.}
Легко проверить, что $\chi_0$ и $\chi_1$\т вполне мультипликативные функции. Значит, для функции
$\frac{\chi(n)}{n^s}$, где $\chi=\chi_0$ или $\chi =\chi_1$, можно применить
лемму~\ref{lem:primeDecomposition} и получить, что
\eqn{L_0(s)=\prods{p\ge3}\hr{1-\frac{\chi_0(p)}{p^s}}^{-1},\qquad L_1(s)=\prods{p\ge3}\hr{1-\frac{\chi_1(p)}{p^s}}^{-1}.}
Логарифмируя левую и правую части полученных выражений, получаем
\eqn{\ln L(s)=-\sums{p\ge3}\ln\hr{1-\frac{\chi(p)}{p^s}}=\sums{p\ge3}\hr{\frac{\chi(p)}{p^s}+r_p(s)}.}
Оценим $r_p(s)$. Если $|x|<\frac12$, то $\ln(1-x)=-\hr{x+\frac{x^2}2+\ldots}$, поэтому
\eqn{|\ln(1-x)+x|\le\frac{|x|^2}2+\frac{|x|^3}3+\ldots\le\frac12(|x|^2+|x|^3+\ldots)=\frac{|x|^2}{2(1-|x|)}\le|x^2|.}
Поэтому $r_p(s)\le\hm{\frac{\chi(p)}{p^s}}^2\le\frac1{p^2}$. Значит, $\bm{\sums{p} r_p(s)}\le\sum\frac1{n^2}\le\bes$. Посему
\eqn{\ln L(s)=\sums{p\ge3}\frac{\chi(p)}{p^s}+O(1),\quad s>1.}
Значит, получаем
\eqn{\begin{aligned}
\sums{p\equiv1(4)}\frac1{p^s}+\sums{p\equiv3(4)}\frac1{p^s}&=\ln L_0(s)+O(1),\\
\sums{p\equiv1(4)}\frac1{p^s}-\sums{p\equiv3(4)}\frac1{p^s}&=\ln L_1(s)+O(1).
\end{aligned}}
Складывая и вычитая эти равенства, получаем
\eqn{\begin{aligned}
\sums{p\equiv1(4)}\frac1{p^s}&=\frac12\br{\ln L_0(s)+\ln L_1(s)}+O(1).\\
\sums{p\equiv3(4)}\frac1{p^s}&=\frac12\br{\ln L_0(s)-\ln L_1(s)}+O(1).
\end{aligned}}
Пусть $s \in \R$.
Перейдём к пределу $s\ra1+$. Если бы ряды из формулировки теоремы сходились, то это бы означало, что
существуют пределы правых частей равенств. Покажем, что они не существуют.
\eqn{L_0(s)=\suml{0}{\bes}\frac1{(2n+1)^s}\ge\frac1{2^s}\suml{0}{\bes}\frac1{(n+1)^s}=\frac{\ze(s)}{2^s}\ra\bes,\quad s\ra1,}
поскольку $\ze$\д функция в точке $1$ имеет полюс.

Теперь оценим $L_1(s)$. Применим признак Дирихле равномерной сходимости ряда:
если $\hm{\sum a_n(s)}\le C$ и $b_n\rra0$, то ряд $\sum a_n(s)b_n(s)$ сходится равномерно.
В нашем случае $a_n(s)=(-1)^n$, а $b_n(s)=\frac1{(2n+1)^s}$. Значит, $L_1(s)$ сходится равномерно на
множестве $s\ge1$. Поэтому сумма ряда для $L_1(s)$ непрерывна на $s\ge1$, и
\eqn{\liml{s\ra1}\ln L_1(s)=\ln L_1(1)=\ln\hr{1-\frac13+\frac15-\frac17+\ldots}<\bes.}
Значит, ряды \eqref{eqn::Lseries} расходятся.
\end{proof}

\subsection{Характеры Дирихле}

\subsubsection{Определение и простейшие свойства}

Зафиксируем некоторое $m\ge2$.

\begin{df}
Вполне мультипликативная $m$\д периодическая функция $\chi\cln\Z\ra\Cbb$ называется \emph{характером Дирихле}, если
$\chi(n)\neq 0$ тогда и только тогда, когда $(n,m)=1$.
Характер
\eqn{\chi_0(n) = \case{1,&(n,m)=1,\\0,&(n,m)\neq 1}}
называется \emph{главным характером}.
\end{df}

Сформулируем некоторые очевидные свойства характеров.

\pt{1} Если $\chi_1$ и $\chi_2$\т характеры, то и $\chi_1\chi_2$\т характер.

\pt{2} $\chi_0\chi=\chi$ для любого $\chi$.

Далее мы покажем, что характеры образуют группу (пока не доказано существование обратного элемента).

\pt{3} По определению $\chi(1)\neq0$. Из мультипликативности следует, что $\chi(1)=1$.

\pt{4} Пусть $(n,m)=1$. Тогда $n^{\ph(m)}\equiv1\pmod{m}$. Поэтому $1=\chi(1)=\chi\br{n^{\ph(m)}}=\chi^{\ph(m)}(n)$.
Значит, ненулевые значения характера\т это просто корни из единицы степени $\ph(m)$.

\pt{5} Очевидным следствием \pt{4} является следующий полезный факт: $|\chi(n)|\le 1$ для всех~$n$.

\begin{theorem}
Для каждого $m\ge 2$ существует в точности $\ph(m)$ характеров.
\eqn{\suml{n=1}{m}\chi(n)=\case{\ph(m),& \chi=\chi_0,\\
0,& \chi\neq \chi_0.}}
\eqn{\sums{\chi}\chi(n)=\case{\ph(m), & n\equiv1\pmod{m},\\0, & n\nequiv1\pmod{m}.}}
\end{theorem}
\begin{proof}
\pt{1}
Группа $\Z_m^*$ абелева, а потому единственным образом разлагается в прямое произведение циклических подгрупп:
\eqn{\Z_m^*=H_1\st H_r,\qquad H_j=\ha{\ol c_j},\quad |H_j|=d_j, \quad d_1\sd d_r=\ph(m).}
Иначе говоря, если $\ol n\in \Z_m^*$, то $\ol{n}=\ol{c}_1^{k_1}\sd\ol{c}_r^{k_r}$.
На языке сравнений это звучит так: если $(n,m)=1$, то $n=c_1^{k_1}\sd c_r^{k_r}\pmod{m}$.

Пусть $\xi_k$\т корень из $1$ степени $d_k$, $k=1\sco r$. Обозначим $\xi:=(\xi_1\sco \xi_r)$.
Построим функцию
\eqn{\chi_\xi(n):=\case{0, & (n,m)\neq1,\\\xi_1^{k_1}\sd\xi_r^{k_r}, & n \equiv c_1^{k_1}\sd c_r^{k_r} \pmod{m}.}}
Очевидно, что это характер.
Покажем теперь, что разным наборам $\xi$ соответствуют разные характеры. Действительно, если
$\xi_i\neq \nu_i$, то $\xi_i=\chi_\xi(c_i)\neq \chi_\nu(c_i)=\nu_i$.
Значит, количество построенных характеров совпадает с количеством наборов $\xi$, а их $d_1\sd d_r=\ph(m)$.

Теперь докажем, что других характеров нет.
Пусть $\chi$\т произвольный характер. Покажем, что он определяется значениями на образующих группы.
Пусть $\tau_i := \chi(c_i)$. Поскольку $c_j^{d_j}\equiv1\pmod{m}$, то $1\bw=\chi(1)\bw=\chi(c_j^{d_j})\bw=\chi^{d_j}(c_j)$.
Таким образом $\tau:=(\tau_1\sco \tau_r)$\т это набор корней из единицы соответствующих степеней.
Поэтому, если $(n,m)=1$, то $\chi(n)=\chi\br{c_1^{k_1}\sd c_r^{k_r}}=\tau_1^{k_1}\sd\tau_r^{k_r}$.
Значит, $\chi$ уже содержится в построенном множестве  характеров и первое утверждение полностью доказано.

\pt{2} Если $\chi=\chi_0$, то очевидно, что $\suml{n=1}{m}\chi(n)=\ph(m)$. Если же $\chi\neq \chi_0$,
то сопоставим этому характеру набор $\xi$ такой, что если $(n,m)=1$ и $n\equiv c_1^{k_1}\sd c_r^{k_r}\pmod{m}$,
то $\chi(n)=\xi_1^{k_1}\sd\xi_r^{k_r}$. Поэтому
\eqn{\suml{n=1}{m}\chi(n)=\sums{n\cln (n,m)=1}\chi(n)=\sums{k_1\sco k_r}\xi_1^{k_1}\sd\xi_r^{k_r}=
\prodl{j=1}{r}\hr{\suml{k=0}{d_j-1}\xi_j^{k_j}}=0.}
Последнее равенство обосновано тем, что найдётся $k$ такое, что $\xi_k\neq 1$ и поэтому один из сомножителей равен~$0$.

\begin{petit}
Поясним, почему один из сомножителей равен нулю. Пусть $\xi$\т корень степени $d$ из~$1$,
причём $\xi \neq 1$. Покажем, что $1 + \xi + \xi^2\spl \xi^{d-1} = 0$. Умножение этой суммы на $\xi$ означает поворот,
и при этом повороте фигура из $d$ векторов, торчащих из нуля в вершины правильного $d$\д угольника, переходит в себя.
Значит, сумма тоже не поменяется. Такое может быть только в случае, когда сумма равна нулю.
\end{petit}

\pt{3} Если $n\equiv1\pmod{m}$, то в силу периодичности имеем $\chi(n)=\chi(1)=1$, поэтому $\sums{\chi}\chi(n)=\ph(m)$. Пусть теперь
$n\nequiv1 \pmod{m}$. Если $(n,m)\neq1$, то для всех $\chi$ имеем $\chi(n)=0$, и потому $\sums{\chi}\chi(n)=0$.
Если же $(n,m)=1$, то найдётся ненулевой показатель степени $k_j$ в разложении~$n$. Поэтому
\eqn{\sums{\chi}\chi(n)=\sums{\xi_1\sco \xi_r}\xi_1^{k_1}\sd\xi_r^{k_r}=\prodl{j=1}{r}\hr{\sums{\xi_i}\xi_i^{k_j}}=0.}
\hfill\end{proof}

\begin{imp}
Пусть $\chi\neq \chi_0$ и $S(N):=\suml{k=1}{N}\chi(k)$. Тогда $|S(N)|\le m$.
\end{imp}
\begin{proof}
Разделим $N$ на $m$ с остатком: $N = mq+r$. Разобьём сумму на две части: неполное частное и остаток.
Из теоремы следует, что неполное частное равно нулю (в силу $m$\д периодичности), а остаток не больше $m$.
\end{proof}
\begin{ex}
Пусть $m=4$, тогда $\ph(m)=2$. Если $n$ нечетно, то $n\equiv3^k\pmod{4}$, $k=0,1$, поэтому $d=2$.
Мы приходим к функции из предыдущего параграфа:
\eqn{\chi(n)=\case{1,&n\equiv1\pmod{4},\\0, &n\equiv 0\pmod{2},\\-1,&n\equiv3\pmod{4}.}}
\end{ex}

\subsubsection{$L$\д функции Дирихле}

\begin{df}
Зафиксируем $m\ge2$ и характер $\chi$. \emph{$L$\д функцией Дирихле} назовем функцию
\eqn{L(s,\chi):=\suml{n=1}{\bes}\frac{\chi(n)}{n^s}.}
\end{df}

\begin{lemma}
Если $\chi=\chi_0$, то ряд для $L(s,\chi)$ абсолютно сходится в $\Re s>1$ и $L(s,\chi)$ аналитична в $\Re s>1$.
Если $\chi\neq \chi_0$, то ряд для $L(s,\chi)$ сходится в $\Re s>0$ и $L(s,x)$ аналитична в $\Re s>0$.
\end{lemma}
\begin{proof}
Первое утверждение очевидным образом следует из свойств $\ze$\д функции и свойства $|\chi(n)|\le1$.
Докажем второе утверждение. Пусть $s=\si+it$, положим $S(N):=\suml{k=1}{N}\chi(k)$. Полагая по определению $S(0) := 0$, имеем
\eqn{\suml{n=1}{N}\frac{\chi(n)}{n^s}=\suml{n=1}{N}\frac{S(n)-S(n-1)}{n^s}=
\suml{n=1}{N}\frac{S(n)}{n^s}-\suml{n=1}{N-1}\frac{S(n)}{(n+1)^s}=
\frac{S(N)}{N^s}+\suml{n=1}{N-1}S(n)\hr{\frac1{n^s}-\frac1{(n+1)^s}}.}
Мы знаем, что $|S(n)|\le m$, если $\chi\neq \chi_0$. Поэтому
$\hm{\frac{S(N)}{N^s}}\le\frac{m}{N^{\si}}\ra 0$ при $N\ra\bes$.
Оценим выражение в скобке:
\eqn{\hm{\frac1{n^s}-\frac1{(n+1)^s}}=\hm{s\intl{n}{n+1}x^{-(s+1)}dx}\le |s|\cdot n^{-(\si+1)} \quad \Ra \quad
\hm{S(n)\hr{\frac1{n^s}-\frac1{(n+1)^s}}}\le\frac{m|s|}{n^{\si+1}}.}
Поэтому ряд сходится равномерно на любом компакте из правой полуплоскости, значит сходится в правой
полуплоскости. Значит, в ней он задаёт аналитическую функцию.
\end{proof}

Нам хотелось бы продолжить функцию $L(s,\chi_0)$ в область $\Re s>0$. В этом нам поможет следующая
\begin{lemma}[Формула Эйлера для $L$\д функций] Для любого характера $\chi$ в области $\Re s>1$ справедливо тождество
\eqn{L(s,\chi)=\prods{p}\hr{1-\frac{\chi(p)}{p^s}}^{-1}.}
\end{lemma}
\begin{proof}
Доказательство аналогично тому, как мы это делали для $\ze$\д функции. Здесь пользуемся вполне
мультипликативностью функции $\frac{\chi(n)}{n^s}$ и леммой~\ref{lem:primeDecomposition}.
\end{proof}

Таким образом,
\eqn{L(s,\chi_0)=\prods{p}\hr{1-\frac{\chi_0(p)}{p^s}}^{-1}=\prods{p\nmid m}\hr{1-\frac{\chi_0(p)}{p^s}}^{-1} =
\prods{p\nmid m}\hr{1-\frac{1}{p^s}}^{-1}.}
Поэтому
\eqn{L(s,\chi_0)=\ze(s)\prods{p\mid m}\hr{1-\frac{1}{p^s}},\quad\Re s>1.}

В первой главе мы продолжали $\ze$-функцию в область $\Re s>0$. При этом у нее был полюс первого порядка в $s=1$.
Заметим, что функция $\prods{p\mid m}\hr{1-\frac{1}{p^s}}$\т целая функция, не обращающаяся в $0$
в точке $s=1$. Поэтому можно считать, что $L(s,\chi_0)$ мы определили всюду в правой полуплоскости,
и она имеет полюс первого порядка в $s=1$.

\begin{theorem}
\label{thm:noZeroLFunction}
Если $\chi\neq \chi_0$, то $L(1,\chi)\neq0$.
\end{theorem}
\begin{proof}
Доказательство разобьём на две части.

\pt{1} Пусть сначала $\chi$\т не действительный характер (то есть принимает не только действительные значения).
Это равносильно тому, что $\chi^2\neq \chi_0$. Настало время ещё раз применить лемму~\ref{lem:prodBound}. Рассмотрим
\eqn{P:=\hm{L(s,\chi_0)^3L(s,\chi)^4L(s,\chi^2)}=
\prods{p\nmid m}\hr{\hm{1-\frac{\chi_0(p)}{p^s}}^3\hm{1-\frac{\chi(p)}{p^s}}^4\hm{1-\frac{\chi^2(p)}{p^s}}}^{-1}.}
Если считать, что $s$\т действительное число, большее $1$, а $r:=\frac{1}{p^s}$, $\chi(p)= e^{i\ph}$, тогда
$\chi^2(p)=e^{2i\ph}$. По лемме каждый множитель в произведении $P$ не меньше 1, а значит, $P\ge1$.

Предположим теперь, что $L(1,\chi)=0$. Тогда по непрерывности $L(s,\chi)=O(s-1)$ при $s\ra1$.
Так как $\chi^2\neq \chi_0$, то $L(s,\chi^2)=O(1)$. Кроме того,
\eqn{L(s,\chi_0)=O\hr{\frac1{s-1}}.}
Из этих оценок следует, что
\eqn{P=O\hr{\frac1{(s-1)^3}(s-1)^4\cdot1}=O(s-1),}
а это противоречит ранее полученному свойству $P\ge1$.

\pt{2} Пусть теперь $\chi$\т действительный характер, то есть $\chi^2=\chi_0$. Рассмотрим функцию
$F(s):=\ze(s)L(s,\chi)$. Дальнейшему доказательству предпошлём лемму.
\begin{lemma}
В области $\Re s>1$ функция $F(s) = \ze(s) L(s,\chi)$ представима в виде
\eqn{F(s)=\suml{n=1}{\bes}\frac{a_n}{n^s}, \qquad a_n \in \Z_+,\qquad a_{k^2}\ge1,}
причём в точке $s = \frac12$ ряд расходится.
\end{lemma}
\begin{proof}
В силу абсолютной сходимости рядов для $\ze$ и $L$, их можно перемножать. Значит,
\eqn{F(s)=\suml{u=1}{\bes}\frac1{u^s}\suml{v=1}{\bes}\frac{\chi(v)}{v^s} =
\sums{u,v\ge1}\frac{\chi(v)}{(uv)^s}=\suml{v=1}{\bes}\frac1{n^s}\sums{v\mid n}\chi(v)=
\suml{n=1}{\bes}\frac{a_n}{n^s},}
где $a_n=\sums{v\mid n}\chi(v)$. Поскольку $\chi(n)=\pm 1$, то очевидно, что $a_n\in\Z$. Осталось проверить
неотрицательность. Пусть $n=p_1^{\al_1}\sd p_r^{\al_r}$, тогда $v=p_1^{\be_1}\sd p_r^{\be_r}$ ($\be_j\le\al_j$),
тогда
\eqn{a_n=\sums{\be_1\sco \be_r}\chi(p_1^{\be_1}\sd p_r^{\be_r})=
\prodl{j=1}{r}\hr{\suml{\be_j=0}{\al_j}\chi(p_j)^{\be_j}}=a_{n1}\sd a_{nr},}
где
\eqn{a_{nj}=\suml{\be=0}{\al_j}\chi(p_j)^\be =\case{
\al_j+1,&\chi(p_j)=1,\\
1,&\chi(p_j)=0,\\
0,&\chi(p_j)=-1,\text{ и }\al_j\text{ нечётно},\\
1,&\chi(p_j)=-1,\text{ и }\al_j\text{ чётно}.}}
При $n=k^2$ степени чётны, поэтому $a_{nj}\neq0$, значит, $a_n\ge1$.

Если предположить, что ряд сходится при $s = \frac12$, то есть $\suml{n=1}{\bes}\frac{a_n}{n^{1/2}}<\bes$,
то, тем более, $\suml{k=1}{\bes}\frac{a_{k^2}}{k}<\bes$, а поскольку $a_{k^2} \ge 1$, то получаем,
что гармонический ряд тоже должен сходиться, что нелепо.

Кроме того, стандартными рассуждениями получаем, что ряд для $F(s)$ равномерно сходится на всяком компакте в области
$\Re s>1$, откуда следует аналитичность и возможность почленного дифференцирования.
\end{proof}

Вернёмся к доказательству второй части теоремы. Предположим, что $L(1,\chi)=0$. Тогда
функция $F(s)$ аналитична в $\Re s>0$ (полюс исчезнет). Значит, можно разложить функцию $F(s)$
в круге с центром в точке~$s=2$ радиуса~$2$:
\mln{
F(s)=\suml{k=0}{\bes}\frac{F^{(k)}(2)}{k!}(s-2)^k=
\suml{k=0}{\bes}\frac{(s-2)^k}{k!}(-1)^k\suml{n=1}{\bes}\frac{a_n\ln^k n}{n^2}=\\=
\suml{k=0}{\bes}\suml{n=1}{\bes}\frac{(2-s)^ka_n\ln^k n}{n^2k!}=
\suml{n=1}{\bes}a_n\suml{k=0}{\bes}\frac{(2-s)^k\ln^k n}{n^2k!}\stackrel{!}{=}\suml{n=1}{\bes}\frac{a_n}{n^s}.}
Поясним переход, отмеченный знаком <<!>>: мы свернули тейлоровское разложение функции $\frac{1}{n^s}$ в точке~${s=2}$.
Этот ряд расходится при $s=\frac12$, а мы получили, что он сходится в силу аналитичности функции.
Противоречие.

Итак, мы доказали, что $L(1,\chi) \neq 0$ для неглавных характеров. Теорема доказана полностью.
\end{proof}

\subsubsection{Доказательство теоремы Дирихле}

\begin{lemma}
В области $\Re s>1$ имеет место равенство
\eqn{-\frac{L'(s,\chi)}{L(s,\chi)}=\suml{n=1}{\bes}\frac{\La(n)\chi(n)}{n^s},}
где $\La$\т функция Мангольдта. Ряд сходится абсолютно и $L(s,\chi)\neq 0$ в этой области.
\end{lemma}
\begin{proof}
Имеет место очевидная оценка
\eqn{\hm{\frac{\La(n)\chi(n)}{n^s}}\le\frac{\ln n}{n^{\Re s}}.}
Отсюда легко следует абсолютная сходимость и аналитичность. Остается проверить выполнение
равенства, заявленного в лемме. В самом деле,
\eqn{L(s,\chi)\suml{n=1}{\bes}\frac{\La(n)\chi(n)}{n^s}=
\suml{l=1}{\bes}\frac{\chi(l)}{l^s}\suml{n=1}{\bes}\frac{\La(n)\chi(n)}{n^s}=
\suml{n=1}{\bes}\frac{\chi(n)}{n^s}\sums{k\mid n}\La(k)\stackrel{!}{=}-L'(s,\chi).}
В последнем переходе, отмеченном <<!>>, мы воспользовались тем, что $\sums{k\mid n}\La(k)=\ln n$.

Докажем отсутствие нулей у~$L$. При дифференцировании порядок нуля падает на единицу.
Значит, если бы у~$L$ был нуль порядка $r > 0$, то порядок нуля слева был бы не меньше $r$,
а справа\т в точности равен $r-1$. Противоречие.
\end{proof}

Теперь можно приступить к доказательству того, ради чего мы заварили всю эту кашу с характерами.

\begin{theorem}[Дирихле]
Если $(m,l)=1$, то последовательность $\hc{mn+l}$ содержит бесконечно много простых чисел.
\end{theorem}
\begin{proof}
Из предыдущей леммы следует, что
\eqn{-\frac{L'(s,\chi)}{L(s,\chi)}=\sums{k\ge1,p}\frac{\ln p\cdot \chi(p^k)}{p^{ks}}=
\sums{p}\frac{\ln p\cdot \chi(p)}{p^s}+\sums{k\ge2,p}\frac{\ln p\cdot \chi(p^k)}{p^{ks}}.}
Докажем, что последнее слагаемое ограничено константой, не зависящей от $s$.
\eqn{\Bm{\sums{k\ge2,p}\frac{\ln p \cdot \chi(p^k)}{p^{ks}}}\le\sums{k,p}\frac{\ln p}{p^k}=
\sums{p}\ln p\suml{k=2}{\bes}\frac1{p^k}=\sums{p}\frac1{p^2-p}\ln p\le\suml{n=2}{\bes}\frac1{n^2-n}\ln n<\bes.}
Таким образом, установлено соотношение:
\eqn{\label{eqn:dzetaSimL}
\sums{p}\frac{\ln p\cdot \chi(p)}{p^s}=-\frac{L'(s,\chi)}{L(s,\chi)}+O(1),\quad \Re s>1.
}
Числа $m$ и $l$ взаимно просты по условию, значит, уравнение $lx\equiv1\pmod{m}$ имеет единственное
решение из~$\Z_m^*$. Пусть $ld\equiv1\pmod{m}$. Тогда умножим равенство \eqref{eqn:dzetaSimL} на $\chi(d)$ и
просуммируем по всем $\chi$:
\eqn{\sums{p}\frac{\ln p}{p^s}\sums{\chi}\chi(pd)=-\sums{\chi}\chi(d)\frac{L'(s,\chi)}{L(s,\chi)}+O(1).}
Вспомним, что
\eqn{\sums{\chi}\chi(n)=\case{\ph(m),&n\equiv1\pmod{m},\\0,&n\nequiv1\pmod{m}.}}
Поэтому в $\sums{p}\chi(pd)$ ненулевыми будут лишь слагаемые у которых $p\equiv l\pmod{m}$.
Что касается правой части, то в ней особым является только слагаемое главного характера,
все остальные в силу теоремы~\ref{thm:noZeroLFunction} не имеют полюсов и потому их можно загнать в $O(1)$.
Итого получаем
\eqn{\ph(m)\cdot\sums{p\equiv l(m)}\frac{\ln p}{p^s}=-\chi_0(d)\frac{L'(s,\chi_0)}{L(s,\chi_0)}+O(1).}
А мы знаем, что $L(s,\chi_0)=\frac{f(s)}{s-1}$, причём $f(1) \neq 0$.
Стало быть, $\frac{L'}{L}=-\frac1{s-1}+\frac{f'}{f}$. Кроме того,
$\chi_0(d)=1$. Поэтому
\eqn{\frac1{s-1}+O(1)=\ph(m)\sums{p\equiv l(m)}\frac{\ln p}{p^s}.}
Слева стоит функция, стремящаяся к бесконечности при $s\ra1$, а справа\т  некоторая сумма, которая стремится
к бесконечности лишь в случае, когда слагаемых в ней бесконечное количество. Теорема Дирихле доказана.
\end{proof}

\section{Алгебраические и трансцендентные числа}

\subsection{Алгебраические числа}

\subsubsection{Свойства алгебраических чисел}

\begin{df}
Комплексное число $\al$ называется \emph{алгебраическим}, если найдется не тождественно нулевой многочлен
$f(x)\in\Q[x]$, для которого $f(\al)=0$. Многочлен $f$ называется \emph{аннулирующим} для данного элемента~$\al$.
\end{df}

Легко видеть, что множество всех многочленов, аннулирующих данный элемент $\al$,
образует идеал в $\Q[x]$.


\begin{df}
Многочлен минимальной степени со старшим коэффициентом~$1$, аннулирующий число $\al$, называется
\emph{минимальным многочленом} числа $\al$. Мы обычно будем обозначать его $f_{\al}(x)$.
Степень многочлена $f_{\al}(x)$ называется \emph{степенью числа}~$\al$ и обозначается
$\deg\al$.
\end{df}

Множество всех алгебраических чисел будем обозначать через~$\A$.

\begin{ex}
Алгебраические числа степени $1$\т это в точности все рациональные числа.
\end{ex}

\begin{ex}
Пусть $\al=\sqrt{2}$. Очевидно, что $f_{\al}(x)=x^2-2$ является минимальным многочленом,
поскольку $\al$ иррационально, и его степень никак не может быть меньше $2$.
\end{ex}

\begin{ex}
Пусть $\al=i$. Тогда $f_{\al}(x)=x^2+1$.
\end{ex}

\begin{ex}
Пусть $\al=\sqrt[3]{2}$. Можно показать, что $f_{\al}(x)=x^3-2$ является минимальным многочленом, но
сделать это сложнее, чем в случае квадратного корня.
\end{ex}

\begin{stm}
Минимальный многочлен $f_\al$ элемента $\al$ неприводим над $\Q$, и все его корни различны.
\end{stm}
\begin{proof}
Он неприводим (иначе это бы означало, что его степень не минимальна).
Докажем, что все корни у $f_{\al}(x)$ разные. Действительно, наличие кратного корня
равносильно тому, что $(f_\al, f_\al') \neq 1$.
\begin{petit}
Контрольный вопрос: почему НОД $f_\al$ и $f_\al'$\т это многочлен с рациональными коэффициентами?
Ведь их общий корень может быть и комплексным! Ответ: потому что алгоритм Евклида не выводит из м\'еньшего
поля (в нашем случае\т из $\Q$).
\end{petit}
Но поскольку многочлен $f_\al$ неприводим, многочлен $f_\al'$ должен
делиться на $f_\al$, а это невозможно, поскольку $\deg f'_\al\bw<\deg f_\al$. Стало быть, кратных корней не бывает.
\end{proof}

\begin{df}
Корни многочлена $f_{\al}$ называются \emph{сопряженными} с~$\al$. Их ровно $\deg\al$ штук.
\end{df}

Нам потребуется одна теорема из курса алгебры:

\begin{theorem}\label{thm:symPoly}
Пусть $R$\т  коммутативное кольцо с единицей. Тогда для любого симметрического многочлена $A\in
R[x_1\sco x_m]$ найдется многочлен $Q\in R[x_1\sco x_m]$, такой что $A(x_1\sco x_m)=Q(\si_1\sco \si_m)$, где
$\si_i$\т элементарные симметрические многочлены.
\end{theorem}

\begin{lemma}\label{lem:rationalPoly}
Пусть $P(x,y)\in R[x,y]$. Тогда
\eqn{\prodl{i=1}{m}P(x,x_i)=Q(x,\si_1\sco \si_m),\quad Q\in R[x,x_1\sco x_m].}
\end{lemma}
\begin{proof}
Пусть
\eqn{\label{eqn:prodCircleDivPoly}\prodl{i=1}{m}P(x,x_i)=A_N x^N\spl A_1x+A_0,\quad A_j\in R[x_1\sco x_m].}
Поменяем $x_i$ местами, при этом левая часть равенства никак не изменится, значит, и правая не изменится.
Стало быть, многочлены $A_j$ являются симметрическими.
Тогда по предыдущей теореме найдутся многочлены $Q_1\sco Q_N$, такие что $A_j=Q_j(\si_1\sco \si_m)$.
Остаётся только подставить $Q_i$ вместо $A_i$ в \eqref{eqn:prodCircleDivPoly}.
\end{proof}

\begin{imp}\label{imp:rationalPoly}
Пусть $P(x,y)\in\Q[x,y]$, $\be \in\A$ и $\deg\be = m$, а $\be_1\sco \be_m$
сопряжены с $\be$. Тогда
\eqn{\label{eqn:Polynom}\prodl{j=1}{m}P(x,\be_j)\in\Q[x].}
\end{imp}
\begin{proof}
По предыдущей лемме, многочлен \eqref{eqn:Polynom} равен
$Q(x,\si_1\sco \si_m)$, $Q\in\Q[x,x_1\sco x_m]$. Пусть
$f_\be (x)=x^m+b_{m-1}x^{m-1}\spl b_0$ ($b_i \in \Q$). Но тогда по формулам Виета
$\si_i(\be)=(-1)^ib_i\in\Q$.
\end{proof}

\begin{theorem}\label{thm:algClosureField}
Множество $\A$ замкнуто относительно алгебраических операций. Иначе говоря, если
числа $\al$ и $\be$ алгебраические, то числа $\al\pm\be$, $\al\cdot\be$, $\frac{\al}{\be}$ тоже
алгебраические.
\end{theorem}
\begin{proof}
Для доказательства нам достаточно предъявить соответствующие аннулирующие многочлены.

\pt{1} Сумма: пусть $\be_1\sco \be_m$\т числа, сопряженные с $\be$. Докажем, что искомым многочленом (из
определения алгебраического числа) будет
\eqn{H_1(x):=\prodl{i=1}{m}f_{\al}(x-\be_i).}
Действительно, $H_1(x)\in\Q[x]$ по предыдущей лемме. Ясно, что $H_1(\al+\be)=0$
(поскольку $\be$ содержится во множестве $\{\be_i\}$).

\pt{2} Разность: аналогично \pt{1}.

\pt{3} Произведение: здесь хочется взять функцию $\prodl{i=1}{m}f_{\al}\hr{\frac{x}{\be_i}}$, но
$f_{\al}\hr{\frac{x}{y}}$\т не многочлен, поэтому его надо подправить, взяв многочлен $y^n\cdot f_\al\hr{\frac xy}$:
\eqn{H_3(x):=\prodl{i=1}{m}\be_i^nf_{\al}\hr{\frac{x}{\be_i}},\quad n=\deg\al.}

\pt{4} Частное: берем многочлен $f_\al(xy)$, и получаем, что многочлен
\eqn{H_4(x):=\prodl{i=1}{m}f_{\al}(x\cdot\be_i)}
аннулирует $\frac\al\be$.
\end{proof}

\begin{imp}
Множество алгебраических чисел образует поле.
\end{imp}

\begin{petit}
Это можно было бы доказать и проще, с применением теории конечных расширений полей (она нам всё равно понадобится).
Возьмём основное поле $K$ и докажем, что множество всех элементов, алгебраических над $K$ (обозначим его $\ol K$)
образует поле.

Пусть $\al, \be \in \ol K$. Рассмотрим (конечное) расширение $K(\al, \be)$. Все конечные расширения\т алгебраические.
Действительно, пусть $[E:K] =d$, то есть конечное расширение $E$\т это $d$\д мерное векторное пространство над $K$.
Стало быть, различные степени любого элемента $x \in E$ будут линейно зависимы над $K$, если их взять более $d$ штук.
Это и будет искомый аннулирующий многочлен.

Элементы $\al$ и $\be$, конечно, лежат в $E:=K(\al,\be)$. Но это поле, значит, $\al\pm \be \in E$, $\al\cdot \be \in E$
и $\frac{\al}{\be} \in E$. Значит, все они, в частности, алгебраичны над $K$.
\end{petit}


\subsubsection{Целые алгебраические числа}

\begin{df}
Алгебраическое число $\al$ называется \emph{целым алгебраическим}, если $f_{\al}(x)\in\Z[x]$.
\end{df}

\begin{ex}
$\al=\frac1{\sqrt{2}}$: $f_{\al}(x)=x^2-\frac12\notin\Z[x]$.
\end{ex}

\begin{ex}
$\al=\frac{1+\sqrt{5}}2$: $f_{\al}(x)=x^2-x-1\in\Z[x]$.
\end{ex}

\begin{ex}
Рациональное число является целым алгебраическим тогда и только тогда, когда оно целое.
\end{ex}

\begin{df}
Многочлен с целыми коэффициентами называется \emph{примитивным}, если его коэффициенты взаимно просты в совокупности.
\end{df}

\begin{lemma}[Гаусс]
Произведение примитивных многочленов есть примитивный многочлен.
\end{lemma}
\begin{proof}
Пусть $A(x)=\suml{0}{n}a_ix^i$, $B(x)=\suml{0}{m}b_ix^i$. Тогда $A(x)B(x)=C(x)=\suml{0}{n+m}c_ix^i$. Предположим, что
многочлен $C$ не примитивный, то есть найдётся $p$, делящее все его коэффициенты.
В силу примитивности многочленов $A$ и $B$, у них найдутся коэффициенты, не делящиеся на $p$.
Выберем среди них коэффициенты с минимальными номерами, пусть это будут $a_u$ и $b_v$.
Имеем $c_{u+v}=\sums{k+l=u+v}a_kb_l$. В силу минимальности $u$ и $v$, если $k<u$, то $p\mid a_k$, а если
$l<v$, то $p\mid b_l$. Поэтому $c_{u+v}\equiv a_ub_v\nequiv0\pmod{p}$. Противоречие.
\end{proof}

\begin{petit}
Алгебраическое доказательство: допустим, что простое число $p$ делит все коэффициенты многочлена $C$.
Тогда проекция многочлена $C$ на $\F_p[x]$\т это нулевой многочлен.
С другой стороны, в силу примитивности $A$ и $B$, их проекции на $\F_p[x]$ суть  ненулевые многочлены.
Но в $\F_p[x]$ нет делителей нуля, поэтому равенство $0 = \ol C = \ol A\cdot \ol B$ в $\F_p[x]$ невозможно.
\end{petit}


\begin{imp}
Если $A(x)=x^m+a_{m-1}x^{m-1}\spl a_0\in\Z[x]$ и $A(\al)=0$, то $\al$\т  целое
алгебраическое.
\end{imp}
\begin{proof}
Многочлены $A(x)$ и $f_{\al}(x)$ имеют общий корень $\al$. Многочлен $f_{\al}$ неприводим и посему
$f_{\al}\mid A$ в $\Q[x]$, то есть $A(x)=f_{\al}(x)Q(x)$, где $Q\in \Q[x]$. У
$A(x)$ и $f_{\al}(x)$ старшие коэффициенты равны~$1$, значит, и у $Q(x)$ он равен единице. Пусть $q_r$\т
наименьший общий знаменатель коэффициентов многочлена $Q$, то есть
\eqn{Q(x)=x^r+\frac{q_{r-1}}{q_r}x^{r-1}\spl \frac{q_0}{q_r}.}
Очевидно, что числа $q_0\sco q_{r-1},q_r$ взаимно просты в совокупности.
Тогда многочлен $Q$ представ\'им в виде $Q(x)\bw=\frac1{q_r} U(x)$, где $U$\т примитивный многочлен.
Аналогично $f_{\al}(x)\bw=\frac1{p_s} V(x)$, где $V$\т  примитивный. Но
тогда $A(x)\bw=\frac1{q_rp_s}U(x)V(x)$, значит, $U(x)V(x)=q_r p_s A(x)$. Пользуясь леммой Гаусса,
получаем, что $q_r p_s=1$, поэтому $p_s=q_r=1$. Значит, $f_{\al}(x)\in\Z[x]$.
\end{proof}

\begin{theorem}
Множество целых алгебраических чисел замкнуто относительно операций сложения, вычитания и умножения.
\end{theorem}
\begin{proof}
Достаточно доказать, что в условиях данной теоремы построенные в теореме~\ref{thm:algClosureField} полиномы $H_j$ лежат в~$\Z[x]$
(равенство единице старшего коэффициента очевидно). Это будет означать (в силу предыдущего следствия)
справедливость теоремы. Само доказательство проходит дословно, с заменой кольца $\Q$ на кольцо $\Z$,
потому что для него тоже выполнены теорема~\ref{thm:symPoly} и лемма~\ref{lem:rationalPoly}).
\end{proof}

\begin{lemma}
Если $\al$\т алгебраическое число, то для него найдется $d\in\N$, для которого $d\al$\т целое
алгебраическое.
\end{lemma}
\begin{proof}
Пусть $f_{\al}(x)=x^n+p_{n-1}x^{n-1}\spl p_0\in\Q[x]$. Пусть $d$\т  общий знаменатель
всех $p_i$. Домножим многочлен на $d^n$ и выделим в мономе степени $k$ множитель $(d\al)^k$:
\eqn{0=d^n\cdot f_{\al}(\al)=(d\al)^n+dp_{n-1}(d\al)^{n-1}\spl d^np_0.}
Значит, многочлен $A(x):=x^n+dp_{n-1}x^{n-1}\spl d^np_0$ аннулирует $d\al$,
а в силу выбора $d$ имеем $A\in\Z[x]$.
\end{proof}

\subsubsection{Теорема о примитивном элементе}

Пусть $\xi_1\sco \xi_n$\т  алгебраические числа, а $\Q(\xi_1\sco \xi_n)$\т  минимальное поле,
содержащее $\Q$ и $\{\xi_i\}$. Оно имеет вид
\eqn{\Q(\xi_1\sco \xi_n)=\hc{\frac{A(\xi_1\sco \xi_n)}{B(\xi_1\sco \xi_n)}\;\biggl|\;B(\xi_1\sco \xi_n)\neq0;\;A,B\in\Q[x_1\sco x_n]}.}
Поля такого вида называются \emph{конечнопорождёнными}.

\begin{petit}
В курсе алгебры расширения полей строились несколько иначе. Напомним эту конструкцию.
Пусть $K$\т основное поле, и пусть $p \in K[x]$. Факторкольцо $E := K[x]/(p)$ будет полем
тогда и только тогда, когда $p$ неприводим над $K$. Заметим, что если $\al$\т
корень неприводимого многочлена $p$, то в $E$ этот многочлен уже имеет корень $\ol x$.
Поле $E$ будет векторным пространством над $K$ степени $d := \deg p$ с базисом $1, \ol x, \ol x^2\sco \ol x^{d-1}$.
Расширение $E$ поля $K$ называется простым алгебраическим расширением поля $K$,
полученным присоединением корня $\al$ неприводимого многочлена $p$.
\end{petit}

\begin{lemma}[Про освобождение от знаменателя]
Пусть $\xi\in \A$, и $\deg\xi=n$. Тогда каждый элемент $\al\in\Q(\xi)$ единственным
образом представляется в виде
\eqn{\al=r_0+r_1\xi\spl r_{n-1}\xi^{n-1},\quad r_i\in\Q.}
Иначе говоря, $\Q(\xi)$\т  $n$\д мерное векторное пространство над $\Q$ с базисом
$1,\xi\sco \xi^{n-1}$.
\end{lemma}
\begin{proof}
Пусть $\al=\frac{A(\xi)}{B(\xi)}$. Наша цель\т построить элемент $B^{-1}(\xi)$. Многочлены $f_{\xi}$ и $B$ взаимно просты (иначе бы $f_{\xi}\mid B$,
а тогда $B(\xi)=0$). Поэтому в силу леммы о линейном представлении НОД существуют многочлены
$U(x)$ и $V(x)$ из $\Q[x]$ такие, что
\eqn{B(x)V(x)+f_{\xi}(x)U(x)=1.}
Значит, в частности, $B(\xi)V(\xi)+f_{\xi}(\xi)U(\xi)=1$, а поскольку $f_\xi(\xi) = 0$, то
\eqn{\frac1{B(\xi)}=V(\xi),\text{ то есть } \al=A(\xi)V(\xi).}
Разделим многочлен $A(x)V(x)$ на $f_{\xi}(x)$ с остатком:
\eqn{A(x)V(x)=Q(x)f_{\xi}(x)+R(x),\quad R(x)=r_0\spl r_{n-1}x^{n-1}.}
Тогда $\al=A(\xi)V(\xi)=R(\xi)$.

Осталось проверить, что такое представление единственно. Допустим, что это не так, и
$\al\bw=R(\xi)\bw=S(\xi)$, причём $\deg R,S<n$. Тогда многочлен $F=R-S$ аннулирует $\xi$,
что невозможно, ибо $\deg\xi=n$, а $\deg F < n$.
\end{proof}

\begin{ex}
$\Q(\sqrt{2},\sqrt{3})=\Q(\sqrt{2}+\sqrt{3})=\Q(\ta)$. Действительно, решая
уравнения
\eqn{(\ta-\sqrt{2})^2=3, \quad  (\ta-\sqrt{3})^2=2,}
получаем, что
$\sqrt{2}=\frac{\ta^2-1}{2\ta}$ и $\sqrt{3}=\frac{\ta^2+1}{2\ta}$. Поэтому
$\Q(\sqrt{2},\sqrt{3})\subset\Q(\ta)$, значит,
$\Q(\sqrt{2},\sqrt{3})=\Q(\ta)$.
\end{ex}

\begin{theorem}[О примитивном элементе]
Пусть $\xi_1\sco \xi_m \in \A$ и $E=\Q(\xi_1\sco \xi_m)$. Тогда найдется
$\ta\in E$ такое, что $E=\Q(\ta)$.
\end{theorem}
\begin{proof}
Пусть сначала $m=2$. Для удобства обозначим $\al=\xi_1$, $\be=\xi_2$, Пусть $\al_1\sco \al_p$ и
$\be_1\sco \be_q$\т числа, сопряженные с $\al$ и $\be$ соответственно.
Пусть, для определённости, $\be_1 = \be$ и $\al_1 = \al$.

Положим
$\ta:=\al_1+c\be_1$, где $c$\т  целое число, такое что
$\al_1+c\be_1\neq\al_i+c\be_j$ ($(i,j)\neq(1,1)$). Такое число~$c$ обязательно найдётся,
поскольку имеется лишь конечное количество ограничений. Пусть $K:=\Q(\ta)$, тогда ясно, что $K \subs E$.

Рассмотрим многочлены $h(x):=f_\al(\ta-cx)\in K[x]$ и $f_\be \in\Q[x]\subset K[x]$.
Имеем $f_\be(\be)=0$, $h(\be)=0$, поэтому степень многочлена $d:=\hr{h,f_\be}$ не меньше единицы.
Все корни $d(x)$ различны (поскольку $d$ делит $f_\be$, а у него все корни различны). Пусть
$\ga$\т корень многочлена $d$, не совпадающий с $\be$. Тогда  $\ga=\be_j$, причём $j\neq1$.
Имеем $h(\ga) = h(\be_j)=f_\al(\ta-c\be_j)=0$, но все корни многочлена $f_\al(x)$\т это $\al_i$,
поэтому $\ta-c\be_j=\al_i$ для некоторого $i$, значит, $\ta=\al_i+c\be_j$. Но мы специально выбирали $\ta$
так, что это возможно лишь в случае $i=j=1$, поэтому на самом деле $\ga=\be$.
Значит, у многочлена $d$ есть только один корень $\be$, поэтому он имеет вид $d(x)=ax+b\in K[x]$.
Поэтому $\be=-\frac ba\in K$ и $\al=\ta-c\be\in K$. Таким образом, мы доказали,
что $E=\Q(\al,\be)\subs K$.
Но ранее мы уже доказали обратное включение. Значит, на самом деле $E=K$.

Для произвольного~$m$ доказываем индукцией,
с учётом равенства $\Q(\xi_1\sco \xi_m) \bw= \Q(\xi_1\sco \xi_{m-1})(\xi_m)$.
\end{proof}

\begin{imp}
Каждое поле, порожденное конечным количеством алгебраических чисел, есть конечномерное линейное пространство
над $\Q$.
\end{imp}

\begin{df}
\emph{Степенью расширения} $E$ поля рациональных чисел называется размерность $E$ над $\Q$ и
обозначается $[E:\Q]$, то есть $[E:\Q]=\dim_{\Q}E=\deg\ta$.
\end{df}

\subsubsection{Алгебраическая замкнутость поля алгебраических чисел}

\begin{theorem}\label{thm:algClosureClosed}
Если $P\in\A[x]$ и $\xi$\т корень $P(x)$, то $\xi\in\A$. Иначе говоря, поле алгебраических чисел
алгебраически замкнуто.
\end{theorem}
\begin{proof}
Пусть $P(x)=\al_n x^n\spl \al_0$, $\al_i\in\A$, $\al_n\neq0$, $P(\xi)=0$. Можно считать,
что $\al_n=1$ (в любом случае можно на $\al_n$ поделить). Пусть
$E=\Q(\al_{n-1}\sco \al_0)=\Q(\ta)$, $m=\deg \ta=[E:\Q]$. Поэтому найдутся
многочлены $F_j\in\Q[x]$ степени не выше $m$: $\al_j=F_j(\ta)$. Составим новый многочлен от
двух переменных:
\eqn{F(x,y)=x^n+F_{n-1}(y)x^{n-1}\spl F_0(y)\in\Q[x,y].}
Пусть $\ta_i$\т корни, сопряжённые с $\ta$, и пусть $\ta = \ta_1$.
Рассмотрим теперь
\eqn{H(x)=\prodl{i=1}{m}F(x,\ta_i).}
В силу следствия~\ref{imp:rationalPoly} получаем $H\in\Q[x]$. Но $F(\xi,\ta_1)=P(\xi)=0$, то есть
$H(\xi)=0$. Значит $\xi\in\A$.
\end{proof}

\begin{petit}
Мы уже приводили более краткие аргументы того, что множество $\ol K$ всех корней многочленов над некоторым полем~$K$
само является полем. Докажем его алгебраическую замкнутость без использования тяжёлой артиллерии (теоремы о примитивном элементе).
Рассмотрим произвольный многочлен $f \in \ol K[x]$,
и пусть $a_i$\т его коэффициенты. Рассмотрим поле $E := K(a_0\sco a_n)$. Это конечное расширение поля $K$.
Итак, имеем такую башню полей: $K \subs E \subs \ol K$. Присоединим к полю $E$ корень $\ta$ многочлена $f$.
Тогда у нас есть ещё одна башня: $K \subs E \subs E(\ta)$, причём оба расширения конечны. Как мы знаем, в этом случае
$E(\ta)$\т конечное расширение поля $K$, значит, все его элементы алгебраичны над $K$, в частности, корень $\ta$.
Стало быть, $\ta \in \ol K$ по определению. Итак, мы показали, что многочлен $f$ имеет корень в $\ol K$.
\end{petit}


\subsection{Проблема квадратуры круга}
Мы хотим получить ответ на вопрос, можно ли при помощи циркуля и линейки построить квадрат, по площади равный
единичному кругу. Определимся для начала, какие простейшие операции мы можем совершать циркулем и линейкой.
Они известны нам со школы: мы можем соединять точки и строить окружности, находить их точки пересечения,
а также выбирать произвольную точку где\д нибудь.

\begin{df}
\emph{Алгебраические точки}\т это все точки плоскости, обе координаты которых\т алгебраические числа.
\emph{Алгебраические прямые}\т это все прямые, заданные в виде $ax+by+c=0$, где $a,b,c\in\A$.
\emph{Алгебраические окружности}\т это все окружности, чей центр\т алгебраическая точка и радиус\т
алгебраическое число.
\end{df}

\begin{theorem}
Каждая операция над алгебраическими элементами (точками, прямыми и окружностями) приводит к алгебраическим элементам.
\end{theorem}
\begin{proof}
В самом деле,
\begin{nums}{-1}
\item Если мы проводим прямую через две алгебраические точки $A=(x_1,y_1)$ и $B=(x_2,y_2)$, то она
      алгебраическая, так как её уравнение имеет вид $(x_2-x_1)(y-y_1)=(y_2-y_1)(x-x_1)$.
\item Если мы строим окружность рационального радиуса и с рациональным центром, то она будет
      алгебраической по определению.
\item Точка пересечения двух алгебраических прямых\т алгебраическая, поскольку для ее нахождения мы решаем
      систему из двух линейных уравнений и значит совершаем лишь операции сложения и умножения над
      алгебраическими числами.
\item Точки пересечения окружности и прямой\т алгебраические, так как для их нахождения нужно решать
      квадратное уравнение.
\item Точки пересечения двух окружностей\т алгебраические. Действительно, если нам дано
      \eqn{\case{(x-x_1)^2+(y-y_1)^2=R_1^2\\
      (x-x_2)^2+(y-y_2)^2=R_2^2,
      }}
      то можно вычесть из первого уравнения второе, получить систему из линейного и квадратного
      уравнений и свести задачу к предыдущей.
\end{nums}
Теорема доказана.
\end{proof}

Поскольку множество рациональных чисел всюду плотно, то и множество алгебраических чисел всюду плотно, значит
никто нам не мешает считать, что начинать решать проблему квадратуры круга мы будем с выбора алгебраических
объектов. Тогда, как показано выше, дальше мы будем получать только алгебраические объекты. Поэтому и
квадрат, который мы получим, если решим проблему, будет иметь стороны алгебраической длины, то есть $\sqrt{\pi}$\т
число алгебраическое, значит, по теореме~\ref{thm:algClosureClosed}, число $\pi$ тоже алгебраическое. Таким образом,
если мы докажем трансцендентность $\pi$, то мы докажем и неразрешимость проблемы квадратуры круга.

\begin{petit}
История решения этой проблемы такова: в 1830~г. проблему квадратуры свели к проблеме трансцендентности~$\pi$,
которую в свою очередь решил в 1882~г. Линдеман.
\end{petit}

\subsection{Расширения полей}

\subsubsection{Нормальные расширения}

\begin{df}
Пусть $E$ и $F$\т поля. Инъективный гомоморфизм $\si\cln E\ra F$ называется \emph{вложением}.
Мы будем обозначать это так: $\si\cln E \inj F$.
\end{df}

\begin{ex}
Для поля $\Cbb$ существует всего два автоморфизма, сохраняющих подполе $\R$.
Это тождественный автоморфизм и комплексное сопряжение $z \mapsto \ol z$.
Это связано с тем, что $\Cbb=\R(i)$, и $[\Cbb:\R] = 2$.
\end{ex}

\begin{theorem}
Пусть $E$\т конечное расширение $\Q$ и $[E:\Q] =:\nu$. Пусть $E=\Q(\ta)$
и $\ta_1\sco \ta_\nu$\т сопряженные с~$\ta$, а
\eqn{\si_j\cln \al=r(\ta)\mapsto r(\ta_j), \quad j = 1\sco \nu.}
Тогда все отображения $\si_j$ являются попарно различными вложениями $E \inj \Cbb$,
и других нет.
\end{theorem}
\begin{proof}
Чтобы сократить обозначения, не будем каждый раз указывать, что многочлены от $\ta$,
представляющие элементы расширения $E$, имеют степень строго меньше~$\nu$ и имеют рациональные
коэффициенты.

\pt{1} Сначала проверим, что $\si_j$\т гомоморфизм. Пусть $\al=r(\ta)$, $\be=s(\ta)$.

Сумма: пусть $t:=r+s$. Тогда $\al+\be=t(\ta)$, поэтому
\eqn{\si_j(\al+\be)=t(\ta_j)=r(\ta_j)+s(\ta_j)=\si_j(\al)+\si_j(\be).}

Произведение: пусть $t := rs$. Разделим $t$ с остатком на $f_\ta$,
получим $t=f_\ta q+u$. Тогда
\eqn{\si_j(\al\be)=u(\ta_j)=r(\ta_j)s(\ta_j)=\si_j(\al)\si_j(\be).}

Частное:
\eqn{\ga=\frac{\al}\be \Ra\al=\be\ga\Ra\si_j(\al)=\si_j(\be)\si_j(\ga)\Ra\si_j(\ga)=\frac{\si_j(\al)}{\si_j(\be)}.}

\pt{2} Теперь проверим инъективность~$\si_j$. Для этого покажем, что $\Ker \si_j = 0$.
Допустим, что $\si_j(\al) = 0$, тогда $r(\ta_j) = 0$, а это невозможно, поскольку минимальный многочлен $\ta_j$
имеет степень $\nu$, а $\deg r < \nu$.

Все $\si_j$ различны, потому что $\si_j(\ta)=\ta_j\neq\ta_i=\si_i(\ta)$ ($i\neq j$).

\pt{3} Теперь докажем, что других вложений нет. Пусть $\si$\т произвольное вложение. Тогда,
поскольку это инъективный гомоморфизм, то $\si(0)=0$, $\si(1)=1$.
Далее, очевидно, что $\si(n)= \si(1\spl 1) = 1\spl 1 = n$, и $\si(-n)=-n$ (поскольку $\si(0)=\si(n)+\si(-n)$),
и, наконец, $\si\hr{\frac ab}=\frac{\si(a)}{\si(b)}=\frac ab$,
то есть $\si$ сохраняет $\Q$.

Воспользуемся неподвижностью подполя $\Q$. Имеем
\eqn{f_{\ta}(\ta)=\ta^{\nu}+a_{\nu-1}\ta^{\nu-1}\spl a_0=0.}
Подействуем на это равенство вложением $\si$, получим:
\eqn{\br{\si(\ta)}^\nu+a_{\nu-1}\br{\si(\ta)}^{\nu-1}\spl a_0=0.}
Итак, $\si(\ta)$\т тоже корень многочлена $f_{\ta}$, значит, он совпадает с одним из $\ta_j$.
Но это значит, что $\si$ совпадает с одним $\si_j$. Действительно,
пусть $\al=r(\ta)$, тогда $\si(\al)=r\br{\si(\ta)}=r(\ta_j)$.
\end{proof}

Изучим вопрос, как устроены образы фиксированного алгебраического числа под действием вложений.
\begin{theorem}\label{thm:algImage}
Пусть $E \sups \Q$, $\nu=[E:\Q]$, $\al\in E$, $\deg\al=m$. Тогда $m\divs\nu$ и
$\hc{\si_1(\al)\sco \si_\nu(\al)}$ состоит из сопряженных числа $\al$ и каждое их них повторяется ровно
$\frac{\nu}{m}$ раз.
\end{theorem}
\begin{proof}
Пусть $E=\Q(\ta)$ и $\al=r(\ta)$. Тогда многочлен
$g(x)=\prodl{j=1}{\nu}(x-\si_j(\al))=\prodl{j=1}{\nu}(x-r(\ta_j))$ по следствию~\ref{imp:rationalPoly} лежит в~$\Q[x]$.
Разделим $g$ на $f_{\al}$ столько раз, сколько сможем:
\eqn{g(x)=f_{\al}^k(x)d(x),\quad f \nmid d, \quad k\ge0.}
и докажем, что $d(x)\equiv \const$. Предположим противное: $\deg d\ge1$ и $\be$\т корень $d$.
Тогда тем более $g(\be) = 0$, значит, найдется $j$, для которого $\be=r(\ta_j)=\si_j(\al)$.
Имеем $f_{\al}(\al)=0$. Подействуем на это равенство
вложением~$\si_j$, получим $f_\al\br{\si_j(\al)}=0$, то есть $f_{\al}(\be)=0$.
Стало быть, многочлены $f_{\al}$ и $d$ имеют общий корень, а так как $f_\al$ неприводим, то $d\dv f_\al$,
а мы вроде договорились, что это не так.
Значит, на самом деле, $d(x)\equiv \const$, но поскольку старшие коэффициенты у $g$ и~$f_\al$ равны~$1$,
то эта константа на самом деле равна~$1$.

Итак, $g(x)= f_{\al}^k(x)$. Слева стоит многочлен степени $\nu$, а справа\т степени
$km$, отсюда и следует, что, во\д первых, $m\mid \nu$, а во\д вторых, множество корней $g(x)$ состоит из
$k$ комплектов корней многочлена $f_{\al}$.
\end{proof}

\begin{imp}
Число $\al\in E$ рационально $\Lra$ оно неподвижно при всех вложениях.
\end{imp}
\begin{proof}
Ясно, что нужно доказать только в обратную сторону. Предположим, что $\deg\al\ge2$. Тогда в силу предыдущей теоремы
в наборе $\hc{\si_1(\al)\sco \si_\nu(\al)}$ должно быть хотя бы два различных числа, а это не
так по условию\т противоречие.
\end{proof}

\begin{df}
Расширение $E\sups\Q$ называется \emph{нормальным}, если для любого вложения $\si\cln E\inj\Cbb$
выполняется равенство $\si(E)=E$.
\end{df}

\begin{ex}
$E=\Q(\sqrt{2})$. Тут только два вложения: тождественное и <<сопряжение>> ($a+b\sqrt{2}\mapsto
a-b\sqrt{2}$). Ясно, что для обоих $\si(E)=E$. Поэтому такое $E$\т  нормальное.
\end{ex}

\begin{ex}
$E=\Q(\sqrt[3]{2})$. Многочлен $x^3-2$ имеет три корня:
$\sqrt[3]{2}$, $\xi\sqrt[3]{2}$, $\xi^2\sqrt[3]{2}$, ($\xi=e^\frac{2\pi i}{3}$), то
$\al=a+b\sqrt[3]{2}\bw+c\sqrt[3]{4}\mapsto a\bw+b\xi\sqrt[3]{2}\bw+c\xi^2\sqrt[3]{4}$\т число, вообще говоря,
комплексное. Поэтому $\si(E)\neq E$, значит такое $E$\т не нормальное.
\end{ex}
\begin{theorem}[Достаточное условие нормальности]
Пусть $E=\Q(\xi_1\sco \xi_m)$, где $\xi_i\in\A$ и все сопряженные каждому $\xi_i$
принадлежат $E$. Тогда $E$ нормально.
\end{theorem}
\begin{proof}
Пусть $\al\in E$. Тогда допустимо представление:
\eqn{\al=\frac{A(\xi_1\sco \xi_m)}{B(\xi_1\sco \xi_m)},\quad
A,B\in\Q[x_1\sco x_m], \quad B(\xi_1\sco \xi_m)\neq0.}
Подействуем вложением $\si$ на это равенство:
\eqn{\si(\al)=\frac{A(\si(\xi_1)\sco \si(\xi_m))}{B(\si(\xi_1)\sco \si(\xi_m))}.}
Как мы уже знаем $\si(\xi_i)$ сопряжено с $\xi_i$, поэтому $\si(\xi_i)\in E\Ra\si(\al)\in
E\Ra\si(E)\subset E$.

Докажем теперь обратное включение. Пусть
$E=\Q(\ta)$, $\ta_1\sco \ta_{\nu}$\т  сопряженные с $\ta$. Как мы знаем (теорема~\ref{thm:algImage}),
образ $\ta$ при отображении $\si$\т это один из сопряжённых с числом $\ta$ корней, пусть это будет $\ta_k$.
Легко видеть, что числа $1,\ta_k\sco \ta_k^{\nu-1}\in E$ образуют базис $E$ (действительно, наличие нетривиальной
линейной зависимости противоречило бы тому, что $\si$\т вложение), и потому всякое $\be\in E$ можно представить в виде
\eqn{\be=r_0+r_1\ta_k\spl r_{\nu-1}\ta_k^{\nu-1}.}
Найдём к этому элементу прообраз: именно, положим
\eqn{\al:=r_0+r_1\ta\spl r_{\nu-1}\ta^{\nu-1}\in E.}
Ясно что $\si(\al)=\be$. А это и означает, что $E\subs\si(E)$.
\end{proof}

\begin{imp}
Если $E$\т нормальное расширение, то отображение, обратное к вложению, тоже является вложением,
и композиция двух вложений\т снова вложение. Иначе говоря, вложения образуют группу.
\end{imp}

\begin{df}
Группа автоморфизмов нормального расширения называется \emph{группой Галуа}.
\end{df}

\subsubsection{Норма в конечных расширениях}

Пусть $E$\т расширение $\Q$ и $\nu=[E:\Q]$, $\si_1\sco \si_\nu$\т вложения.

\begin{df}
\emph{Нормой} элемента $\al$ из $E$ называется число $N(\al):=\prodl{j=1}{\nu}\si_j(\al)$.
\end{df}

\begin{ex}
Если $\al \in \Q$, то $N(\al)=\al^{\nu}$.
\end{ex}

\begin{theorem}[Свойства нормы]
\begin{nums}{-2}
\item Пусть $\deg\al=d$, $f_{\al}=x^d\spl a_d$. Тогда $N(\al)=(-1)^\nu a_d^{\nu/d}$.
\item $N(\al)\in\Q$, а если $\al\in \Z_E$, то $N(\al)\in\Z$.
\item $N(\al)=0\Lra\al=0$.
\item $N(\al\cdot\be)=N(\al)N(\be)$.
\end{nums}
\end{theorem}
\begin{proof}
Все доказательства будут несложными:
\begin{nums}{-2}
\item Рассмотрим набор $\hc{\si_1(\al)\sco \si_\nu(\al)}$. Пусть $\al_1\sco \al_d$\т все сопряжённые с $\al$.
      Применяя теорему~\ref{thm:algImage} и обозначая $k = \frac{\nu}{d}$, получаем, что
      \eqn{N(\al) = (\al_1\sd \al_d)^k \stackrel{!}{=} \br{(-1)^d a_d}^k = (-1)^\nu a_d^k.}
      Здесь переход, отмеченный <<!>>, следует из формул Виета.
\item Следует из первого свойства и того факта, что $a_d\in\Q$. А в случае $\al\in\Z_E$ имеем $a_d\in\Z$.
\item $N(\al)=0\Lra\exi j\cln \si_j(\al)=0\Lra\al=0$.
\item $N(\al\cdot\be)=\prodl{j=1}{\nu}\si_j(\al\cdot\be)=\prodl{j=1}{\nu}\si_j(\al)\si_j(\be)=N(\al)N(\be)$.
\end{nums}
Теорема доказана.
\end{proof}

\subsection{Приближение иррациональных чисел рациональными}

\subsubsection{Приближение действительных чисел рациональными}

Мы знаем, что $\Q$ плотно в $\R$, поэтому для любого $\al\in\R$ и для любого $\ep>0$
найдутся $p$ и $q$ такие, что ${\bm{\al-\frac{p}q}<\ep}$, то есть любое действительное число сколь угодно
точно приближается рациональными. Нас интересует вопрос: а насколько маленьким
можно взять $q$, чтобы такая оценка всё ещё выполнялась.

\begin{theorem}[Дирихле]
Пусть $\al\in\R$ и $n\in\N$, тогда найдётся рациональное число $\frac{p}q$, для которого
\eqn{\hm{\al-\frac{p}q}<\frac1{qn}, \quad 1\le q\le n.}
\end{theorem}
\begin{proof}
Рассмотрим полуинтервал $[0,1)$, поделим его на $n$ равных частей.
Рассмотрим числа $x_k := \hc{\al k}$, $k \bw= 0\sco n$. По принципу Дирихле,
в одну из частей разбиения попадут хотя бы два числа $x_k$ и $x_m$.
Без ограничения общности, $k > m$. Тогда
\eqn{\hm{x_k - x_m} = \bm{\hc{\al k} - \hc{\al m}} = \Bm{ \br{\al k - \hs{\al k}} - \br{\al m - \hs{\al m}}} < \frac1n.}
Положим $q := k - m$, а $p := \hs{\al k} - \hs{\al m}$. Тогда
$\hm{x_k - x_m} = \hm{\al q - p} < \frac1n$, то есть $\hm{\al - \frac pq} < \frac1{qn}$.
При этом, очевидно, $1 \le q \le n$.
\end{proof}

\begin{imp}
Если $\al$\т  иррациональное число, то существует бесконечно много $p$ и $q$:
$|\al-\frac{p}q|<\frac1{q^2}$.
\end{imp}
\begin{proof}
Во-первых, заметим, что такие $p$ и $q$ существуют, поскольку по предыдущей теореме найдутся $p$ и $q$:
${|\al-\frac{p}q|<\frac1{qn}\le\frac1{q^2}}$. Нужно доказать, что их бесконечно много. Предположим
противное: $\frac{p_1}{q_1}\sco \frac{p_N}{q_N}$\т все приближения. Тогда выберем $n$ так,
что $\frac1n<\minl{j}\bm{\al-\frac{p_j}{q_j}}$ и по этому числу~$n$
найдем такие $p$ и $q$, как в предыдущей теореме. Очевидно, это приближение не содержится в нашем конечном
наборе, поскольку оно <<лучше>> каждого из имеющихся. Противоречие.
\end{proof}

Итак, мы видим, что можем приблизить число <<с квадратичной точностью>>. Возникает резонный вопрос,
а верна ли теорема для степени~3?  Ответ такой: вообще говоря, нет.
Например для числа~$\sqrt{2}$ оценка со степенью~3 неверна. Причина этого кроется в том, что <<хорошо>> приближаются дробями
только трансцендентные числа.

\begin{stm}[Пример Лиувилля]
Для числа $\al=\suml{n=0}{\bes}2^{-n!}$ и для любого $m\ge2$ найдется бесконечно много рациональных
чисел $\frac{p}{q}$: $0<|\al-\frac{p}q|<\frac1{q^m}$.
\end{stm}
\begin{proof}
Представим $\al$ в виде
\eqn{\al=\suml{n=0}{N}\frac1{2^{n!}}+\suml{N+1}{\bes}\frac1{2^{n!}}.}
Пусть $\frac{p_N}{q_N}=\suml{n=0}{N}\frac1{2^{n!}}$, $q_N=2^{N!}$. Тогда
\eqn{0<\al-\frac{p_N}{q_N}=\suml{N+1}{\bes}\frac1{2^{n!}}<
\frac1{2^{(N+1)!}}\hr{1+\frac12+\frac14+\ldots}=\frac2{2^{(N+1)!}}=\frac2{q_N^{N+1}}\le\frac1{q_N^m} \text{ при всех }N \ge m,}
что и требовалось доказать.
\end{proof}

\begin{petit}
Этот пример был построен Лиувиллем в 1840~г. Причина того, что данное $\al$ хорошо приближается, заключается в
том, что оно трансцендентно. Собственно, пример Лиувилля создавался как демонстрация того, что существуют
трансцендентные числа. Это уже потом, в 1873~г. Кантор придумал простое доказательство существования
трансцендентных ($\A$ счётно, $\R$ несчётно, значит существуют трансцендентные числа).
\end{petit}

\subsubsection{Приближение алгебраических чисел рациональными}

Речь в этом параграфе пойдет, разумеется, о действительных алгебраических числах, ибо понятно, что числа из
$\Cbb\wo\R$ не приблизишь рациональными. Докажем сейчас теорему о том, что
алгебраические числа <<плохо>> приближаются рациональными.

\begin{theorem}[Лиувилль]
Пусть $\al\in\A\cap\R$, и  $d := \deg\al\ge2$. Тогда найдётся константа $C$, зависящая от $\al$, что для всех $\frac pq$
\eqn{\Bm{\al-\frac{p}q}\ge\frac{C}{q^d}.}
\end{theorem}
\begin{proof}
Пусть $D$\т  общий знаменатель коэффициентов многочлена $f_{\al}(x)$. Пусть также
$\al=\al_1,\al_2\sco \al_d$\т  числа, сопряженные с~$\al$. Ясно, что $f_{\al}(x)$ не
имеет рациональных корней (так как он неприводим), поэтому
\eqn{\label{eqn:fAlphaDiff}
0\neq\hm{f_{\al}\hr{\frac{p}q}}=\frac{A}{Dq^{d}}\ge\frac1{Dq^d}.}
В последнем неравенстве мы воспользовались тем, что $A$\т  целое неотрицательное и не равно нулю. Рассмотрим
теперь два случая:

\pt{1} Пусть $\bm{\al-\frac{p}q}\le1$. Тогда
$\bm{\al_j-\frac{p}q}\le\bm{\al_j-\al}+\bm{\al-\frac{p}q}\le\bm{\al_j-\al} + 1$. Отделив от произведения первый
множитель, получаем
\eqn{\hm{f_{\al}\hr{\frac{p}q}}=\prodl{j=1}{d}\Bm{\al_j-\frac{p}q}\le
\Bm{\al-\frac{p}q}\cdot\prodl{j=2}{d}(1+|\al_j-\al|).}
Отсюда, с использованием оценки~\eqref{eqn:fAlphaDiff} получаем
\eqn{\frac1{Dq^d} \le \Bm{\al-\frac{p}q}\cdot\prodl{j=2}{d}(1+|\al_j-\al|),}
а затем полагаем
$C:=\Br{D\cdot\prodl{j=2}d(1+|\al_j-\al|)}^{-1}<1$.

\pt{2}
Если же $\bm{\al-\frac{p}q}>1$, то можно взять ту же константу $C$.
\end{proof}

\begin{imp}
Если $\al\in\R$ и для любого $m\ge2$ неравенство $0<\bm{\al-\frac{p}q}<\frac1{q^m}$ имеет
бесконечно много решений, то $\al$ трансцендентно.
\end{imp}
\begin{proof}
\pt{1} Докажем, что $\al\notin\Q$. Предположим, что это не так, и
$\al=\frac{a}b$, $a,b\in\Z$, $b>0$. Возьмём $m = 2$. По условию существует
бесконечно много решений неравенства $0<\bm{\al-\frac{p}q}<\frac1{q^2}$. С другой стороны,
\eqn{0\neq\Bm{\frac{a}b-\frac{p}q}=\frac{A}{bq}\ge\frac1{bq}}
Сопоставляя два неравенства, получаем, что $\frac1{q^2}>\frac1{bq}\Ra q < b$. Поэтому $q$ мы можем выбрать лишь
конечным числом способов, значит, решений вида $\frac{p}q$ конечное число. Противоречие.

\pt{2} Докажем, что $\al \notin \A$. Предположим противное, и пусть $\deg\al=d\ge2$.
Пусть $m=d+1$, тогда по условию теоремы существует бесконечно
много решений неравенства $0<\bm{\al-\frac{p}q}<\frac1{q^m}$. А с другой стороны по предыдущей теореме:
$\bm{\al-\frac{p}q}\ge\frac{C}{q^d}$. Сопоставляя эти неравенства, получаем:
$\frac{C}{q^d}<\frac1{q^{d+1}}\Ra q<\frac1C$. Поэтому у нас в распоряжении лишь конечный набор $q$, а значит
и чисел $\frac{p}q$ конечное множество\т  противоречие.
\end{proof}

\subsection{Теорема Линдемана\ч Вейерштрасса и её следствия}

\subsubsection{Трансцендентность $e$}

Докажем вначале иррациональность числа~$e$. Впервые это сделал Эйлер в середине XVIII~века (он разложил число $e$ в
непрерывную дробь).

\begin{theorem}[Фурье, 1815~г.]
Число $e$ иррационально.
\end{theorem}
\begin{proof} Представим число $e$ в виде ряда и умножим его на $n!$. Получим
\eqn{n!e=n!\suml{k=0}{\bes}\frac1{k!}=\suml{k=0}{n}\frac{n!}{k!}+\suml{k=n+1}{\bes}\frac{n!}{k!}=p_n+r_n.}
Поэтому
\eqn{0<n!e-p_n=r_n=\frac1{n+1}\hr{1+\frac1{n+2}+\frac1{(n+2)(n+3)}+\ldots}<
\frac1{n+1}\hr{1+\frac12+\frac14+\ldots}=\frac2{n+1}.}
Предположим, что $e=\frac ab$. Положим $n = b$. Тогда слева стоит целое число (факториал убьёт знаменатель),
а~справа\т ненулевое число, меньшее $1$.
\end{proof}

\begin{lemma}[Тождество Эрмита]
Пусть $f(x)\in\Cbb[x]$. Пусть
$F(x):=f(x)+f'(x)+f''(x)+\ldots$ Тогда
\eqn{\label{eqn:hermiteEquality}e^kF(0)-F(k)=e^k\intl{0}{k}e^{-x}f(x)dx.}
\end{lemma}
\begin{proof}
Легко проверить, что
\eqn{\int f(x)e^{-x}dx=-F(x)e^{-x}+c.}
Поэтому
\eqn{\intl{0}{k}e^{-x}f(x)dx=-F(k)e^{-k}+F(0).}
Умножим левую и правую часть на $e^k$, получим
\eqn{e^kF(0)-F(k)=e^k\intl{0}{k}e^{-x}f(x)\,dx,}
что и требовалось доказать.
\end{proof}

\begin{lemma}\label{divDerLemma}
Пусть $f \in \Z[x]$. Тогда $\frac{1}{p!}f^{(p)}(x) \in \Z[x]$ при всех $p \in \N$.
\end{lemma}
\begin{proof}
Имеем
\eqn{(x^r)^{(p)}=\case{0, \quad p>r,\\r(r-1)\sd\br{r-(p-1)}x^{r-p}=p!\cdot \Cb_r^p x^{r-p}, \quad p\le r.}}
В любом случае производная каждого монома делится на $p!$. Значит, $\frac1{p!}f^{(p)}(x)\in\Z[x]$.
\end{proof}

\begin{theorem}[Эрмит, 1873~г.]
Число $e$ трансцендентно.
\end{theorem}
\begin{proof}
Предположим, что $e\in\A$. Тогда найдется многочлен
$f_e(x)=a_mx^m+a_{m-1}x^{m-1}\spl a_0$, $a_i\in\Z$ такой, что $f_e(e)=0$.
Умножим теперь~\eqref{eqn:hermiteEquality} на $a_k$ и просуммируем по всем
$k=0\sco m$, получим:
\eqn{\label{eqn:sumHermiteEq}
F(0)\ub{\suml{k=0}{m}a_ke^k}_{0} -\suml{k=0}{m}a_kF(k)=\suml{k=0}{m}a_ke^k\intl{0}{k}f(x)e^{-x}dx.}
Пусть
\eqn{\ta_k(x):=\case{1, & x\le k,\\0, & x>k,}\qquad\Phi(x):=\suml{k=0}{m}a_k e^k \ta_k(x),}
тогда правая часть уравнения \eqref{eqn:sumHermiteEq} равна
\eqn{J:=\intl{0}{m}f(x)\Phi(x)e^{-x}\dx.}

Теперь подберём функцию $f(x)$ так, чтобы $J\in(0,1)$, а числа $F(k)$ были целыми.

Будем искать $f(x)$ в виде
\eqn{f(x):=\frac1{n!}x^{n+r_0}(x-1)^{n+r_1}\sd(x-m)^{n+r_m}, \qquad r_i \in \hc{0,1}.}
Поскольку $\Phi(x)$\т ступенчатая функция с разрывами в точках $1\sco m$, а $f(x)$ обращается в нуль в этих же точках,
функция $f(x)\Phi(x)$ будет непрерывной с нулями в точках $1\sco m$.
Числа $r_i$ мы выберем так, чтобы в этих точках не было перемены знака у функции $f(x)\Ph(x)$,
и она была бы неотрицательной на $[0,m]$. Тогда $J>0$. Теперь выберем такое большое $n$, чтобы интеграл $J$ был меньше~$1$.
Оценим числитель $f(x)$: имеем $|x-k|\le m$, значит, числитель $f(x)$ не превосходит $m^{(m+1)(n+1)}$, а в знаменателе стоит $n!$,
который задавит любую показательную функцию. Кроме того, $|\Ph|\le C$. Значит, $J\in(0,1)$ при достаточно большом $n$.

Докажем, что $F(k)\in\Z$. Действительно, $f(x)=\frac{g(x)}{n!}$, где $g(x)\in\Z[x]$. По предыдущей лемме $\frac{1}{p!}g^{(p)}(x) \in\Z[x]$.
Следовательно,
\eqn{F(k)=\sums{p\ge0}f^{(p)}(k)=\sums{p\ge n}f^{(p)}(k)=\sums{p\ge n}\ub{\frac1{p!}g^{(p)}(k)}_{\in\Z} \cdot \ub{\frac{p!}{n!}}_{\in\Z}\in\Z.}

Возвращаемся к уравнению \eqref{eqn:sumHermiteEq} и видим, что в правой его части стоит $J\in(0,1)$, а в левой\т сумма целых чисел.
Противоречие.
\end{proof}

\subsubsection{Иррациональность $\pi$}

\begin{theorem}[Эрмит] Число $\pi$ иррационально.
\end{theorem}
\begin{proof}
Пусть $f(x)\in\Cbb[x]$ и $F(x):=f(x)-f^{(2)}(x)+f^{(4)}(x)-\ldots$, тогда легко проверить, что
\eqn{\int f(x)\sin x\dx=F'(x)\sin x-F(x)\cos x +C.}
Поэтому
\eqn{J:=\intl{0}{\pi}f(x)\sin x\dx=F(\pi)+F(0).}
Предположим, что $\pi=\frac ab$. Тогда положим
\eqn{f(x)=\frac{b^nx^n(\pi-x)^n}{n!}=\frac{x^n(a-bx)^n}{n!}.}
Так как $f(x)\ge0$ и $\sin x\ge0$ на отрезке $[0,\pi]$, то $J>0$. Кроме того,
\eqn{J<\pi\frac{b^n\pi^{2n}}{n!}=\frac{b^n\pi^{2n+1}}{n!}\ra 0, \quad n\ra\bes.}
Поэтому можно выбрать $n$ таким большим, чтобы $J\in(0,1)$. Далее, $f(\pi-x)=f(x)$, поэтому $f^{(2p)}(\pi-x)\bw=f^{(2p)}(x)$,
в частности, $f^{(2p)}(\pi)=f^{(2p)}(0)$. Значит, $F(\pi)=F(0)$.
Осталось показать, что $F(0)\in\Z$. В самом деле,
\eqn{F(0)= \sums{p\ge 0} (-1)^p f^{(2p)}(0) = \sums{2p \ge n} (-1)^p f^{(2p)}(0).}
Аналогично предыдущей теореме показывается, что это выражение целое.
Снова получаем противоречие, связанное с тем, что в интервале $(0,1)$ нет целых чисел.
\end{proof}

\subsubsection{Доказательство теоремы Линдемана\ч Вейерштрасса}

Ключевым в доказательстве этой теоремы будет некоторое утверждение, на первый взгляд кажущееся диким.

\begin{prop}\label{lem:uglyProp}
Пусть $A(t):=a_0e^{\al_0t}\spl a_me^{\al_mt}$, и $a_j,\al_j\in\A$. Если
$A(t)\nequiv0$ и ряд Тейлора функции $A(t)$ в точке $0$ имеет рациональные коэффициенты, то $A(1)\neq0$.
\end{prop}

\begin{ex}
Чтобы осознать, что это действительно так, рассмотрим такой пример:
\eqn{e^{\sqrt{2}t}+e^{-\sqrt{2}t}=\suml{k=0}{\bes}\frac{\sqrt{2}^k+(-\sqrt{2})^k}{k!}t^k=\suml{k=0}{\bes}\frac{2^{k+1}}{k!}t^{2k}.}
\end{ex}

Это вселяет некоторую надежду. Прежде, чем начать доказывать это предложение, сделаем несколько простых
замечаний, которые упростят жизнь.

\pt{1} Можно считать, что все $\al_j$ различны.

\pt{2} Можно считать, что все $a_j$\т  целые алгебраические (то есть $a_j \in \Z_E$). Это действительно так,
поскольку по свойству целых алгебраических всегда можно найти такое $d\in\N$, что $da_j\in\Z_E$.

\pt{3} Можно считать, что все $a_j\neq0$.

\begin{proof}
Будем доказывать от противного. Предположим, что $A(1)=0$. Тогда будет верна
\begin{lemma}\label{lem:seriesAtOneZero}
Пусть $A(1)=0$, $n\in\N$,  и
\eqn{f(x) :=(x-\al_0)^n(x-\al_1)^{n+1}\sd(x-\al_m)^{n+1}, \quad g(x):=\frac1{n!}\sums{l\ge n}f^{(l)}(x).}
Тогда
\eqn{|a_0g(\al_0)\spl a_mg(\al_m)|\le\frac{C^{n+1}}{n!},\quad C=C(a_j,\al_j)>0.}
\end{lemma}
\begin{proof}
Пусть $F(x)=\sums{l\ge0}f^{(l)}(x)$. Легко проверить, что
\eqn{e^a\intl{0}{a}f(z)e^{-z}dz=e^aF(0)-F(a).}
Подставим вместо $a$ число $\al_j$, умножим на $a_j$ и просуммируем по всем $j$ от $0$ до $m$:
\eqn{\label{eqn:sumJ0M}-\suml{j=0}{m}a_jF(\al_j)=\suml{j=0}{m}a_je^{\al_j}\intl{0}{\al_j}f(z)e^{-z}\,dz.}
\eqn{F(\al_j)=\sums{l\ge0}f^{(l)}(\al_j)=\sums{l\ge n}f^{(l)}(\al_j)=n!g(\al_j)}
Поэтому \eqref{eqn:sumJ0M} можно переписать так:
\eqn{\label{eqn:sumGalpha}
\suml{j=0}{m}a_jg(\al_j)=-\frac1{n!}\suml{j=0}{m}a_je^{\al_j}\intl{0}{\al_j}f(z)e^{-z}\,dz.}

Пусть $r=\maxl{j}|\al_j|$. Тогда если $|z-\al_j|\le 2r$, то $|f(z)|\le(2r)^{n+m(n+1)}$ и $|e^{-z}|\le
e^r$. Из этих простеньких оценок следует, что модуль правой части \eqref{eqn:sumGalpha} меньше $\frac{C^{n+1}}{n!}$ для
некоторого $C$\т  что и требовалось доказать.
\end{proof}


Пусть $D\in\N$\т такое число, что все числа $D\al_j$ уже являются целыми алгебраическими. Рассмотрим число
\eqn{I:=D^{m(n+1)}\br{a_0g(\al_0)\spl a_mg(\al_m)}.}
Следующая лемма утверждает, что

\begin{lemma}
Число $I$ является целым алгебраическим.
\end{lemma}
\begin{proof}
Фактически достаточно доказать, что при всех $l\ge n$ выполнено
\eqn{\label{eqn:dInteger}D^{m(n+1)}\frac1{l!}f^{(l)}(\al_j)\in\Z_E.}
Действительно, тогда нетрудно убедиться, что
\eqn{D^{m(n+1)}g(\al_j)=\sums{l\ge n}\frac{l!}{n!}\frac{D^{m(n+1)}}{l!}f^{(l)}(\al_j)\in\Z_E,}
откуда и следует утверждение теоремы. Итак, докажем~\eqref{eqn:dInteger}.
\eqn{D^{m(n+1)}f(x)=\frac1{D^n}(Dx-D\al_0)^n(Dx-D\al_1)^{n+1}\sd(Dx-D\al_m)^{n+1}=D^{-n}h(Dx),}
где
\eqn{h(t)=(t-D\al_0)^n(t-D\al_1)^{n+1}\sd(t-D\al_m)^{n+1}\in \Z_E[t].}
Поэтому, применяя лемму~\ref{divDerLemma}, получаем
\eqn{D^{m(n+1)}\frac1{l!}f^{(l)}(\al_j)=D^{-n+l}\frac1{l!}h^{(l)}(D\al_j)\in\Z_E.}
Лемма доказана.
\end{proof}

\begin{lemma}
Найдется сколь угодно большое число $n$ такое, что $I\neq0$ и $|N(I)|\ge1$.
\end{lemma}
\begin{proof}
Посчитаем $I\pmod{(n+1)}$:
\eqn{\frac{D^{m(n+1)}}{n!}\sums{l\ge n+1}f^{(l)}(\al_j)=\sums{l\ge
n+1}\frac{l!}{n!}\frac{D^{m(n+1)}}{l!}f^{(l)}(\al_j)=(n+1)\ga_j,\quad\ga_j\in\Z_E.}
Поэтому
\mln{I\equiv D^{m(n+1)}a_0g(\al_0)\equiv
D^{m(n+1)}a_0\frac1{n!}f^{(n)}(\al_0)=\\=D^{m(n+1)}a_0(\al_0-\al_1)^{n+1}\sd(\al_0-\al_m)^{n+1}=a_0\prodl{j=1}{m}(D\al_0-D\al_j)^{n+1}.}
Пусть
\eqn{S:=\Bm{N(a_0)\cdot \prodl{j=1}{m}N(D\al_0-D\al_j)}.}
Ясно, что $S\in \Z$. Подберем $n$ так, чтобы $(n+1,S)=1$. Докажем, что тогда $I\nequiv0\pmod{(n+1)}$. Предположим противное,
то есть что $I \dv (n+1)$. Тогда свойство делимости верно и для их норм, поэтому
\eqn{S^{n+1} \dv \br{N(n+1)} = (n+1)^{\nu}.}
Но это невозможно в силу выбора $n$. Значит, $I \nequiv 0\pmod{(n+1)}$ и тем более $I\neq0$. Поэтому
$|N(I)|\ge1$ (ибо $I\in\Z_E$).
\end{proof}

\medskip

\textbf{Доказательство предложения.} Пусть $n=Sr$, значит, $n+1=Sr+1$. Тогда $(n+1,S)=1$. Пусть $\si$\т вложение $E$ в $\Cbb$.
Если предположить, что наше предложение неверно, то
$\si(a_0)e^{\si(\al_0)}\spl \si(a_m)e^{\si(\al_m)}=0$ (тут мы воспользовались рациональностью коэффициентов). Тогда мы попадаем в условия леммы~\ref{lem:seriesAtOneZero}, где в роли
$f(x)$ и $g(x)$ выступают $\si(f)(x)$ и $\si(g)(x)$ соответственно. Поэтому
\eqn{\bm{\si(a_0)\si(g)\br{\si(\al_0)}\spl \si(a_m)\si(g)\br{\si(\al_m)}}\le\frac{C_{\si}^{n+1}}{n!} \Ra|\si(I)|\le\frac{\la_{\si}^{n+1}}{n!}}
для всех $\si$.
Поэтому
\eqn{1\le|N(I)|=\prodl{j=1}{\nu}|\si_j(I)|\le \Br{\prodl{j=1}{\nu}\la_{\si_j}}^{n+1}\cdot \frac1{(n!)^{\nu}}<1,}
получаем противоречие.
\end{proof}

\begin{theorem}[Линдеман\ч Вейерштрасс]
Пусть $\al_0\sco \al_n$\т различные алгебраические числа, тогда числа $e^{\al_0}\sco e^{\al_m}$ линейно независимы
над полем $\A$.
\end{theorem}
\begin{proof}
Предположим противное: существуют $a_0\sco a_m\in\A$, для которых $\suml{i=0}{m}a_ie^{\al_i}=0$. Пусть
\eqn{A(t):=a_0e^{\al_0t}\spl a_me^{\al_mt}.}
 Тогда $A(1)=0$ по предположению, а $A(t)\nequiv 0$, ибо
$\al_j$ различны (в курсе дифференциальных уравнений показывалось, что экспоненты с различными
показателями линейно независимы). Разложим $A(t)$ в ряд Тейлора:
\eqn{A(t)=\suml{k=0}{\bes}(a_0\al_0^k\spl a_m\al_m^k)\frac{t^k}{k!}.}
Казалось бы, что тут-то и нужно применить предложение~\ref{lem:uglyProp}, но коэффициенты в ряде Тейлора функции $A(t)$,
вообще говоря, не являются рациональными. Поэтому попытаемся из $A(t)$ изготовить функцию с рациональными
коэффициентами.

Введем обозначения: пусть $E$\т конечное расширение поля $\Q$, полученное присоединением
всех коэффициентов $a_i$ и всех сопряжённых к ним, а также всех чисел $\al_j$ и всех сопряжённых к ним.
Пусть $\nu=[E:\Q]$. Тогда существует $\nu$ штук автоморфизмов поля~$E$, обозначим их
$\si_1\sco \si_{\nu}$.

Рассмотрим формальный ряд: $f(t)=\sum\ga_kt^k$, $\ga_k\in E$. Обозначим через $\si(f)(t)$ следующий
ряд: $\sum\si(\ga_k)t^k$. Заметим, что операция $\si(f)$ обладает следующими очевидными свойствами
\begin{nums}{-2}
\item $\si(f\pm g)=\si(f)\pm\si(g);$
\item $\si(fg)=\si(f)\si(g);$
\item $\si(f')=\si(f)'$.
\end{nums}

Поскольку $A(t)\nequiv0$, то в~ряде Тейлора для $A(t)$ есть ненулевые коэффициенты и поэтому автоморфизм
$\si$ от них тоже будет не нулём, значит
\eqn{\label{eqn:sigmaAofT}
\si(A)(t)=\suml{k=0}{\bes}\hr{\si(a_0)\si(\al_0)^k\spl \si(a_m)\si(\al_m)^k}\frac{t^k}{k!}=
\si(a_0)e^{\si(\al_0)t}\spl \si(a_m)e^{\si(\al_m)t}\nequiv0.}

Рассмотрим функцию $B(t)$, определенную следующим образом:
\eqn{B(t)=\prodl{j=1}{\nu}\si_j(A)(t)=b_0e^{\be_0t}\spl b_Me^{\be_Mt},\quad b_i,\be_i\in E.}
В силу \eqref{eqn:sigmaAofT}, $B(t)\nequiv 0$. Пусть $\si$\т  произвольный автоморфизм~$E$. Но множество всех
автоморфизмов образует группу и поэтому
\eqn{\si(B)(t)=\prodl{j=1}{\nu}\si\si_j(A)(t)=\prodl{j=1}{\nu}\si_j(A)(t)=B(t).}
Это означает, что коэффициенты Тейлора функции $B(t)$ не меняются под действием любого автоморфизма, что
возможно лишь в том случае, когда они все рациональны. Итак, как и обещалось, мы построили функцию такого же
вида, как и в предложении~\ref{lem:uglyProp}, у неё рациональные коэффициенты, и она в единице равна нулю
(поскольку один из $\si_j$\т  тождественный)\т  противоречие с утверждением предложения.
\end{proof}

\subsubsection{Следствия из теоремы Линдемана\ч Вейерштрасса}

\begin{imp}
Если $\al\neq0$ и $\al\in\A$, то $e^{\al}$ трансцендентно.
\end{imp}
\begin{proof}
Допустим, что $\be := e^\al$\т алгебраическое число. Тогда $1 \cdot e^\al - \be \cdot 1 = 0$.
Мы получили линейную комбинацию (с коэффициентами из $\A$) экспонент с показателями $\al$ и~$0$,
равную нулю. Это противоречит теореме Линдемана\ч Вейерштрасса. Значит, $e^\al$ трансцендентно.
\end{proof}

\begin{ex}
Число $e^{\sqrt{2}}$  трансцендентно.
\end{ex}

\begin{imp}
Число $\pi$ трансцендентно.
\end{imp}
\begin{proof}
Пусть $\pi\in\A$, тогда и $\pi i\in\A$. Но тогда и $e^{\pi i}=-1$ трансцендентно, что неверно.
\end{proof}

\begin{imp}
Пусть $\be\in\A$, $\be\neq0,1$. Тогда $\ln\be$ трансцендентно при любом выборе ветви логарифма.
\end{imp}
\begin{proof}
Если $\al=\ln\be$\т  алгебраическое, то $e^{\al}=\be$\т  трансцендентное. Противоречие.
\end{proof}

\begin{imp}
Если $\al\in\A$, $\al\neq0$, то $\sin\al$, $\cos\al$, $\tg\al$\т трансцендентные.
\end{imp}
\begin{proof}
Предположим $\be=\sin\al\in\A$. Тогда $2ie^{i\al}\be=e^{2i\al}-1$. Значит, $e^{i\al}$\т корень многочлена
с алгебраическими коэффициентами, значит $e^{i\al}\in\A$\т противоречие.
\end{proof}

\begin{imp}[Вейерштрасс]
Если $\be_1\sco \be_r$\т алгебраические и линейно независимые над $\Q$, то
числа $e^{\be_1}\sco e^{\be_r}$ \textbf{алгебраически} независимы над $\A$, то есть
не существует такого ненулевого многочлена $P \in\A[x_1\sco x_r]$, что $P(e^{\be_1}\sco e^{\be_r})=0$.
\end{imp}
\begin{proof}
Пусть $P(x_1\sco x_r)=\sums{\ol{k}}a_{\ol{k}} x_1^{k_1}\sd x_r^{k_r}$. Предположим, что
$P(e^{\be_1}\sco e^{\be_r})=0$, то есть
\eqn{\sums{\ol{k}}a_{\ol{k}}e^{(k_1\be_1\spl k_r\be_r)}=0.}
Поскольку $\be_i$ линейно независимы над~$\Q$, числа $\sum k_i\be_i$ различны. По теореме
Линдемана\ч Вейерштрасса, числа $e^{(k_1\be_1\spl k_r\be_r)}$ линейно независимы над~$\A$, что противоречит
написанному линейному соотношению.
\end{proof}

\end{document}
