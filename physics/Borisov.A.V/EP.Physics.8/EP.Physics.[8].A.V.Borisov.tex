\documentclass[a4paper,draft]{article}
\usepackage[simple]{dmvn}

\def\iof.#1#2.#3.{#1.\,#2.\,#3}

\title{Программа и задачи к экзамену по физике}
\author{Лектор\т \iof.АВ.Борисов.}
\date{VIII семестр, 2006 г.}
\begin{document}
\maketitle

\section*{Программа экзамена}

\begin{nums}{-2}
\item Преобразования Лоренца.
\item Релятивистская кинематика.
\item Динамика релятивистской частицы: вариационный принцип и уравнения Лагранжа.
\item Уравнения Гамильтона для релятивистской частицы во внешнем поле.
\item Канонические преобразования. Релятивистское уравнение Гамильтона\ч Якоби.
\item Теорема Э.\,Нётер в механике.
\item Вариационный принцип и уравнения Лагранжа в теории поля.
\item Группа Лоренца и её генераторы.
\item Теорема Э.\,Нётер в теории поля.
\item Скалярное поле. Калибровочные преобразования.
\item Электромагнитное поле как абелево калибровочное поле.
\item Уравнения Максвелла.
\item Энергия, импульс и момент электромагнитного поля.
\item Действие системы <<поле$+$частицы>>. Система уравнений Максвелла\ч Лоренца.
\item Закон сохранения энергии\д импульса в электродинамике.
\item Функция Грина волнового уравнения.
\item Поле произвольно движущейся заряженной частицы.
\item Электромагнитное излучение.
\item Спектр и поляризация излучения.
\item Синхротронное излучение.
\end{nums}

\begin{thebibliography}{5}
\setlength\itemsep{-2pt}
\bibitem{gal-gr}
    \iof.ДВ.Гальцов., \iof.ЮВ.Грац., \iof.ВЧ.Жуковский.. \emph{Класические поля.}\т М.: Изд-во Моск. ун-та, 1991.
\bibitem{gal}
    \iof.ДВ.Гальцов.. \emph{Теоретическая физика для студентов\д математиков.}\т М.: Изд-во Моск. ун-та, 2003.
\bibitem{ll}
    \iof.ЛД.Ландау., \iof.ЕМ.Лифшиц.. \emph{Теория поля.}\т М.: ФизМатЛит, 2001.
\bibitem{arnold}
    \iof.ВИ.Арнольд.. \emph{Математические основы классической механики.}\т М.: Наука, 1989.
\bibitem{dnf}
    \iof.БА.Дубровин., \iof.СП.Новиков., \iof.АТ.Фоменко.. \emph{Современная геометрия: методы и приложения.}\т 5-е изд., испр.\т М.: УРСС, 2001.
\end{thebibliography}

\medskip\dmvntrail

\pagebreak

\section*{Экзаменационные задачи}

\begin{nums}{-2}
\item Используя преобразование Лоренца для 4\д скорости, найти закон сложения 3\д скоростей в релятивистской кинематике.
\item Показать, что 4\д ускорение\т пространственноподобный 4\д вектор.
\item Найти решение уравнения Гамильтона\ч Якоби для свободной релятивистской частицы.
\item Найти закон движения релятивистской заряженной частицы в постоянном однородном электрическом поле.
\item Найти закон движения релятивистской заряженной частицы в постоянном однородном магнитном поле.
\item Найти закон движения релятивистской заряженной частицы в кулоновом поле.
\item Найти явный вид генераторов группы Лоренца.
\item По теореме Нётер найти энергию и импульс комплексного скалярного поля.
\item По теореме Нётер найти момент комплексного скалярного поля.
\item По теореме Нётер найти заряд комплексного скалярного поля, связанный с инвариантностью
      лагранжиана относительно калибровочных преобразований.
\item Записать систему уравнений Гамильтона для вещественного скалярного поля и показать
      её эквивалентность уравнению Лагранжа.
\item Вывести преобразования Лоренца для напряжённостей электромагнитного поля.
\item Найти электромагнитное поле неподвижной заряженной частицы.
\item Найти электромагнитное поле неподвижного равномерно заряженного шара.
\item Найти электромагнитное поле бесконечного прямого цилиндрического проводника с постоянным током.
\item Найти электромагнитное поле равномерно движущейся заряженной частицы, используя  преобразование Лоренца
      из системы покоя частицы.
\item Найти электромагнитное поле равномерно движущейся заряженной частицы, интегрируя уравнения Максвелла.
\item Показать, что уравнения Максвелла без источников имеют решения в виде плоских электромагнитных волн.
\item Найти мощность излучения заряженной релятивистской частицы, движущееся в постоянном однородном электрическом поле.
\item Найти мощность излучения заряженной релятивистской частицы, движущееся в постоянном однородном магнитном поле.
\end{nums}


\medskip\dmvntrail

\end{document}
