\documentclass[unicode,10pt]{article}
\usepackage{dmvn}
\usepackage{polyglossia}
\usepackage{unicode-math}
\usepackage{fontspec}
\defaultfontfeatures{Scale=MatchLowercase, Mapping=tex-text}
\setmainfont{CMU Serif}
\setsansfont{CMU Sans Serif}
\setmonofont{CMU Typewriter Text}
\setmathfont{xits-math.otf}

\begin{document}
{\bfseries
\center{Программа экзамена по физике, 8-семестр.}

\center{Лектор -- Гальцов Дмитрий Владимирович.}
}

\begin{enumerate}
\item Электростатика. Потенциал, энергия электрического поля.
\item Магнитное поле, вектор-потенциал, сила Лоренца, энергия
  магнитного поля.
\item Электромагнитная индукция. Сохранение заряда и ток смещения Максвелла.
\item Уравнения Максвелла в трехмерной форме. Теорема Умова-Пойнтинга.
\item Электромагнитные волны. Волновой вектор.
\item Преобразования Лоренца. Кинематические эффекты СТО. Сложение скоростей.
\item Пространство Минковского. Мировая линия. 4-скорость и 4-ускорение.
\item \textit{4-потенциал и тензор электромагнитного
  поля. Преобразование компонент поля. Уравнение Максвелла в 4-х
  мерной форме}
\item Принцип наименьшего действия в релятивисткой механике.
\item Обобщенный импульс. Гамильтониан. Уравнения Гамильтона -- Якоби
  для заряда в электромагнитном поле.
\item Движение заряда в кулоновом поле.
\end{enumerate}
\end{document}
