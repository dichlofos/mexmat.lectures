\documentclass[unicode,10pt]{article}
\usepackage{dmvn}
\usepackage{polyglossia}
\usepackage{unicode-math}
\usepackage{fontspec}
\usepackage{epsfig}
\usepackage{ulem}
\usepackage{wrapfig}
\defaultfontfeatures{Scale=MatchLowercase, Mapping=tex-text}
\setmainfont{CMU Serif}
\setsansfont{CMU Sans Serif}
\setmonofont{CMU Typewriter Text}
\setmathfont{xits-math.otf}
\newcommand{\questionably}[1]{{
    \setmainfont{URW Chancery L}
    #1}}
\newcommand{\contrib}[1]{
  {
  \setmainfont{URW Chancery L}
  \uwave{#1}
  }}
\newcommand{\veca}{\vec A}
\newcommand{\vecb}{\vec B}
\newcommand{\vece}{\vec E}
\newcommand{\eX}{\vec e_x}
\newcommand{\eY}{\vec e_y}
\newcommand{\eZ}{\vec e_z}
\newcommand{\eR}{\vec e_r}
\newcommand{\unsafeIO}{
  \begin{petit}
    {\Large \bf\#!} Творчество верстальщика.
  \end{petit}
}
\begin{document}
Сей документ -- затеханная версия решений задач по физике к экзамену
Гальцова Д.В за авторством \contrib{Маниты Оксаны}. Обратите внимание
на утверждения, выделенные другим шрифтом -- как то
\questionably{«Очевидно, что $a³+b³=c³$ неразрешимо в натуральных
    числах.»}.  За сим верстальщик,\contrib{Богатов Дмитрий}
указывает, что ему эти утверждения не очевидны. Это может иметь или не
иметь значения. Такой способ иногда изящнее, чем окружение
\verb'petit'.  В любом случае, \questionably{all mistakes are my own.}

Также, некоторые задачи решены верстальщиком. Они помечены, как
требующие большей осторожности в использовании.

\begin{tproblem}
  Построить электростатический потенциал (решение уравнения Пуассона)
  в $R^n$, $n≥2$ для
  \begin{itemize}
  \item Точечного заряда
  \item Заряженного шара (внутри и снаружи)
  \item Заряда, равномерно распределенного по двумерной плоскости ($n≥3$)
  \end{itemize}
\end{tproblem}
\begin{solution}
  \begin{itemize}
  \item Плотность заряда $ρ=eδ^n(\vec r -\vec r_0)$. Без ограничения
    общности, можно считать, что $\vec r_0 = \vec 0$.
    \begin{denote}
      $S_n^R$ -- площадь $n$-мерной сферы радиуса $R$.
    \end{denote}
    \begin{denote}
      $V_n^R$ -- объем $n$-мерного шара радиуса $R$.
    \end{denote}

    По электростатической теореме Гаусса (в учебнике коэффицент
    $4π$ для $n=3$, в общем случае он равен $S^1_{n-1}$)
    \begin{equation*}
      \displaystyle
      ∮\limits_{∂V}\vec E·d\!\vec S = S¹_{n-1}\!∫\limits_V ρ(\vec r)\,d^nx
    \end{equation*}
    для любого шара с центром в заряде, имеем
    \begin{equation*}
      S¹_{n-1}e = ES^r_{n-1}
    \end{equation*}
    т.к из соображений симметрии, $\vec E |\!| \vec n$.  Т.к
    $S^r_{n-1} = r^{n-1}S¹_{n-1}$, то
    \begin{equation}
      E= \frac{e}{r^{n-1}} = -\grad φ
    \end{equation}
    Откуда $φ = \frac{e}{(n-2)r^{n-2}} +c$. Константу полагаем нулю, что бы $φ(∞) = 0$.
  \item Равномерно заряженный с плотностью $ρ_0$ шар радиуса $R$,
    \begin{enumerate}
    \item Вне шара $r>R$. Из соображений симметрии, $φ=φ(r)$. По т. Гаусса
      \begin{equation*}
        \begin{array}{c}
          \displaystyle S¹_{n-1}(ρ_0V^R_n) = ES^r_{n-1} \\
          \displaystyle E = \frac{ρ_0V^R_n}{r^{n-1}} = -\grad φ \\
          \displaystyle  φ = \frac{ρ_0V^R_n}{(n-2)r^{n-2}}
        \end{array}
      \end{equation*}
    \item Внутри шара, $r<R$ также по т.Гаусса
      \begin{equation*}
        \begin{array}{c}
          \displaystyle S¹_{n-1}V¹_nr^nρ_0 = ES^r_{n-1} = ES¹_{n-1}r^{n-1}\\
          \displaystyle \vec E= ρ_0V¹_n\vec r = V¹_nρ_0\grad{\frac{r^2}{2}} = -\grad φ \\
          \displaystyle φ = -V¹_nρ_0\frac{r^2}{2}+φ_0
        \end{array}
      \end{equation*}
      Выберем $φ_0$ по непрерывности в $r = R$:
      \begin{equation*}
        -V¹_nρ_0\frac{r^2}{2}+φ_0 = \frac{ρ_0V^R_n}{(n-2)r^{n-2}}
      \end{equation*}
      Отсюда, $φ_0 = R^2V¹_nρ_0(\frac 1{n-2} + \frac 12)$.
      Получаем, полное выражение потенциала
      \begin{equation*}
        φ =
        \begin{aligned}
          \bcase{
            \frac{n}{2(n-2)}R^2V¹_nρ_0 -V¹_nρ_0\frac{r^2}{2}&,r≤R\\
            \frac{ρ_0V^R_n}{(n-2)r^{n-2}}&,r>R
          }
        \end{aligned}
      \end{equation*}
    \end{enumerate}
  \item $σ$ -- плотность заряда на плоскости.(Картинка будет
    когда-нибудь.) Рассмотрим сечение $R^n$ пространством
    $R^{n-2}=R(x_3,…,x_n)$. От плоскости будет точка и $φ=\frac
    1r^{n-4}$. Из  симметрии, $φ = (x²_3+…+x²_n)^{\frac{4-n}2}$
  \end{itemize}
\end{solution}

\input{task2.tex}
\begin{tproblem}
  Нерелятивистский заряд падает под углом на плоскость $z$, несущую
  заряд с плотностью
  \begin{equation*}
    ρ=λ\frac{d}{dz}δ(z)
  \end{equation*}
  (двойной электрический слой). Под каким углом заряд выйдет с другой
  стороны плоскости? (механического взаимодейтвия нет)
\end{tproblem}
\begin{solution}
  \newcommand{\Eres}{4πλδ(z)}
  \newcommand{\fd}[2]{\frac{d#1}{d#2}}

  Из соображений симметрии, $E$ не зависит от $x$,$y$.
  Запишем уравнение Пуассона
  \begin{equation*}
    4πλδ'(z) = \Div E = \pf{E}{z}
  \end{equation*}
  \begin{wrapfigure}{l}{0.16\textwidth}
    \begin{center}
      \includegraphics[width=0.16\textwidth]{t3image.eps}
    \end{center}
  \end{wrapfigure}
  Проинтегрировав, получаем $4πλδ(z) + c= E$. \questionably{Вне двойного слоя поля
    нет, поэтому $c=0$.} Это поле изменяет импульс частицы по координате
  $z$, т.к не влияет на $x$, $y$ координаты. Т.к заряд нерелятивистский,
  то верны формулы классической механики:
  \begin{equation*}
    \begin{array}{ccc}
      \displaystyle
      \vec p = m\vec v & \frac{d\vec p}{dt} = \vec F = e\vec E & \frac{d\vec p_z}{dt} = m\frac{d\vec v_z}{dt} = \Eres e
    \end{array}
  \end{equation*}
  После соударения
  \begin{equation*}
    \fd pt = m\pf{v_z}{z}\pf zt = m\pf{v_z}{z}v_z = \frac m2 \frac{∂}{∂z}v²_z
  \end{equation*}
  Отсюда, $p(+0) - p(-0) = \frac 12 mv²_z\Big{|}_{\text{
      до}}^{\text{после}} = 4πeλ$. Т.е $v²_z(+0) - v²_z(-0) = \frac{8πeλ}{m}$. Отсюда получаем соотношение на углы
  \begin{equation*}
    v_z(+0) = \sqrt{\frac{8πeλ}{m}+v²_z(-0)} ⇒ \cos β = \sqrt{\frac{8πeλ}{m}+\cos^2α}
  \end{equation*}
\end{solution}
\begin{tproblem}
  Равномерно заряженная сфера радиуса $R$ и заряда $Q$ разрезана на
  две половины. Найти силу взаимодействия половин.
\end{tproblem}
\begin{solution}
  Т.к сфера осталась, её внутреннее поле не изменилось и равно нулю по
  принципу максимума. Поместим внтрь отриц ательно заряженную
  полусферу. Силы взаимодействия половин равны ( '$+$' и '$-$'
  притягиваяются с той же силой, что и '$+$' и '$+$' отталкиваются).
  Поэтому, достаточно найти взаимодействие двух вложенных полусфер.
  ($F_{21} = F_{31} = F_{32}$).  \questionably{«Разогнем
      задачу»}. Горизонтальные составляющие скомпенсируются по
  симметрии, вертикальне -- проинтегрируем. $φ_{zz} = -4πeδ(z)$ $⇒$ $φ
  = -4πe|z|$.
  \begin{petit}
    В исходниках коэффицент 2, но этого совершенно непонятно.
  \end{petit}
  $E = -4πq_1θ(z)|_{0-}^{0+} = -4πq_1$. Тогда сила взаимодействия
  элементарных зарядов $F= Eq_2 = -4πq_1q_2$. У нас $q_1 = q_2 = σ$.
  \questionably{Сила взаимодействия полусфер тогда равна}
  \begin{equation*}
    F = \intl{0}{2π}\intl{0}{\frac π2}R²F\sin θ\cos θdθdφ = σ²2πR²π = \frac{Q²}{8R²}
  \end{equation*}
  где $σ$ -- поверхностная плотность заряда, $Q = 4πR²σ$.
\end{solution}
\input{task5.tex}
\begin{tproblem}
  Найти индукционное электрическое поле, порождаемое магнитным полем тока
  \begin{equation*}
    \begin{array}{cc}
      j_x = iδ(z)\cos ωt, &j_y = j_z = 0
    \end{array}
  \end{equation*}
\end{tproblem}
\begin{solution}
  \newcommand{\vim}{\vec\imath}
  \newcommand{\vjm}{\vec\jmath}
  \begin{petit}
    Куда дели мнимую единицу? Впрочем, туда ей и дорога.
  \end{petit}
  Введем обозначение -- вектор $\vim = (1,0,0)$. Тогда, $\vjm =
  \vim\cos ωt δ(z)$.  $\rot\vecb = \frac{4π}c\vjm$,
  \questionably{поэтому} $\vecb ⊥ \vec e_x$, что означает $B_x =0$, и,
  по закону Био-Савара-Лапласа, $B_z = 0$.
  \begin{petit}
    Странно как-то получается. Несимметрично. Чем координата $z$ лучше
    координаты $y$?
  \end{petit}
  \tbk
\end{solution}
\input{task7.tex}
\begin{tproblem}
  Рассчитать магнитное поле равномерно вращающейся сферы.
\end{tproblem}
\begin{solution}
  Обозначим в отсутсвии картинки угловую скорость $\vec Ω$. Тогда
  $\vec v = \vec Ω ×\vec r$, $\vec\jmath = ρ\vec v = σδ(r-R)\vec
  v$. Запишем известные уравнения на магнитное поле:

  \begin{displaymath}
    \begin{array}{cc}
      \Div\vecb = 0 & \rot\vecb = \frac{4π}c\vec\jmath
    \end{array}
  \end{displaymath}
  Введем вектор-потенциал $\veca: \Div\veca  = 0, \rot\veca = \vecb$. Тогда $Δ\veca = -\frac{4π}c\vec\jmath$.
  \questionably{ Значит, т.к} $\frac 1r$ -- функция Грина, $Δ\frac 1r = -4πδ(r)$,
  \newcommand{\hmr}{\hm{r-r'}}
  \newcommand{\Rreq}{{\vec r' = R\vec n}}
  \newcommand{\sRreq}{\evn\Rreq}
  \newcommand{\vecd}{\vec\nabla}
  \begin{eqnarray*}
    \veca(r) = \frac 1c∫\frac\jmath(r')\hmr d³x'
    = \frac 1c∫\frac{σδ(r'-R)[\vec Ω×\vec r']}\hmr d³x' = \hc{d³x' = dS'dr'} = \frac σc∫\frac{δ(r-r')[\vec Ω×\vec r']}\hmr dS'dr \bw= \\
    = \hc\Rreq = \frac σc∫_{S²}\frac{Ω×r'}{\hmr}\sRreq dS'
    = \frac{Rσ}c \vec Ω×∫_{S²}\frac{dS'}\hmr \sRreq = \text{\questionably{Почему??}}
    = -\frac{Rσ}c \vec Ω×\vecd ∫\frac{dV}\hmr
  \end{eqnarray*}
  \begin{petit}
    Ваш верстальщик ничего дальше не понял. Дабы не плодить в этом
    мире лажу, дальше смотрите в оригинале.
  \end{petit}
\end{solution}
\input{task9.tex}
\input{task10.tex}
\begin{tproblem}
  Найти связь между средним магнитным полем в круге и значением поля
  на окружности в бетатроне (ускорителе электронов, в котором
  ускоряющее электрическое поле создается благодаря электро-магнитной
  индукции неоднородным, зависящим от времени магнитным полем
  $B_z(ρ,t)$, $ρ² = x² + y²$, причем электроны ускоряются, оставаясь на окружности постоянного радиуса)
\end{tproblem}
\begin{solution}
  \newcommand{\vecec}{\vec\Ec}
  \newcommand{\δ}{\,d²\!}
  \newcommand{\Α}[1]{\!<\!\!#1\!\!>}
  По классическим определениям импульса и энергии,
  \begin{equation*}
    \begin{array}{c}
      \frac{d\vec p}{dt} = e\hr{\vece + \frac 1c[\vec v×\vecb]}\\
      \frac{d\vecec}{dt} = e(\vece, \vec v) = eEv
    \end{array}
  \end{equation*}
    Запишем закон индукции Фарадея ($D_2$ -- это круг)
    \begin{equation*}
      \begin{array}{c}
       \displaystyle  \intl {D²}{}\rot\vece d²x = ∮\limits_{S¹}\vece ds = 2πRE\\
       \displaystyle \intl {D²}{}\rot\vece\δ x  = -\frac 1c\frac{∂}{∂t}\intl{D²}{}\vecb \δ x = -\frac 1c πR²\frac{∂}{∂t}\Α{B}
      \end{array}
    \end{equation*}
    где $\Α{B}$ мы обозначили среднее магнитное поле в круге. Совмещая два равенства, получаем,
    \begin{equation*}
      E = -\frac{πR²}{2πRc}\frac{∂}{∂t}\Α{B} = -\frac{1}{2c}R\frac d{dt}\Α{B}.
    \end{equation*}
  Также нам понадобится равенство неясной природы
  \begin{equation*}
    pΩ = \hm{\hs{p×Ω}} = f_ν = \frac ec |[\vec v, \vecb\evn{R}]| = \frac ec VB_R
  \end{equation*}
  где $B_R$ -- среднее значение поля на окружности.

  Подставим полученное выражение для $E$ в имеющиеся формулы:
  \begin{equation*}
    \begin{array}{c}
      \displaystyle pΩ = \frac ec VB_R\\
      \displaystyle eE = \frac{dp}{dt}\frac 1R =\frac ec\dot B_R
    \end{array}
  \end{equation*}
  Приравнивая, получаем, что
  \begin{equation*}
    \begin{array}{ccc}
      \frac {R\dot B_R}c = \frac 1{2c}R\frac{d}{dt}\Α{B} & ⇒ &
      B_R = \frac 12\Α{B}
    \end{array}
  \end{equation*}
\end{solution}
\input{task13.tex}
\input{task14.tex}


\begin{tproblem}
  \newcommand{\ω}{ω\hr{t - z/c}}
  С помощью метода Гамильтона-Якоби найти закон движения и траекторию
  релятивистского заряда в поле плоской волны круговой поляризации,
  описываемой потенциалом
  \begin{equation*}
    \veca = A_0 (\cos\hr\ω, \sin\hr\ω, 0)
  \end{equation*}
\end{tproblem}
\begin{solution}
  \newcommand{\ω}{ω\hr{t - z/c}}
  \newcommand{\ρ}[1]{p_{0#1}}
  Запишем уравнение Гамильтона-Якоби (стр. 64). Полагаем, что $φ=0$,
  т.к \questionably{электрическое поле отсутствует}.
  \begin{equation*}
    \begin{array}{c}
      \frac 1{c²}\hr{\pf St + eφ}^2 - \hr{\nabla S -\frac ec\veca}² = m²c²\\
      \displaystyle \frac 1{c²}\hr{\pf St}² - \hr{\pf Sx - \frac{eA_0\cos\hr\ω}c}²
      + \hr{\pf Sy - \frac{eA_0\sin\hr\ω}c}² + \hr{\pf Sz}² = m²c²
    \end{array}
  \end{equation*}
  Сделаем замену $u = t-z/c$, $v= t+z/c$.
  Тогда
  \begin{equation*}
    \begin{array}{ccc}
      \displaystyle
      \pf St = \pf Su + \pf Sv & &
      \displaystyle
      \pf Sz = \frac 1c\hr{\pf Sv - \pf Su}
    \end{array}
  \end{equation*}
  Подставим в уравнение:
  \begin{equation*}
    \begin{array}{c}
      \displaystyle
      \frac 1{c²}\hr{\pf Su + \pf Sv}² -\hr{\pf Sx -\frac{eA_0\cos ωu}c}^2
      -\hr{\pf Sy - \frac{eA_0\sin ωu}c}² - \frac 1{c²}\hr{\pf Sv - \pf Su}² =  m²c² \\
      \displaystyle
      4\pf Su\pf Sv = (c\pf Sx -eA_0\cos ωu)² + (c\pf Sy - eA_0\sin ωu) + m²c² \\
      \displaystyle
      4\pf Su\pf Sv = c²\hr{\hr{\pf Sx}²+\hr{\pf Sy}²} -2eA_0c(\pf Sx\cos ωu + \pf Sy\sin ωu) +m²с^4 +e²A_0²
    \end{array}
  \end{equation*}
Ищем решение в виде $S = S_0(u) + S_1(x) + S_2(y) + S_3(v)$. В таком
предположении, уравнение переписывается
 \begin{equation*}
   m²c^4 + e²A²_0 = 4S'_0(u)S'_3(v) - c²\hr{S'_1(x)}² -c²\hr{S'_2(y)}²
   + 2S'_1(x)ecA_0\cos ωu + 2S'_2(y)ecA_0\sin ωu
 \end{equation*}
 Выбираем константы первых интегралов (как разделяющиеся переменные)
 \begin{equation*}
   \begin{array}{ccc}
     сS'_1(x) = \ρ x =\const &
     cS'_2(y) = \ρ y = \const &
     S'_3(v) = \ρ v = \const
   \end{array}
 \end{equation*}
 Переписываем уравнение
 \begin{equation*}
   m²c^4 + e²A²_0 = 4S'_0(u)\ρ v - {\ρ x}² -{\ρ y}² + 2\ρ xeA_0\cos ωu + 2\ρ yeA_0\sin ωu
 \end{equation*}
 \begin{denotes}
   $\Ec² ≔ m²c^4 + e²A²_0 + \ρ x² + \ρ y²$, $λ = \sqrt{\ρ x² + \ρ y²}$, $ξ =ω(u-u_0)$
 \end{denotes}
 Интегрируем, разворачивая $u_0$ обратно. Авторский замысел, видимо.
 \begin{equation*}
   4\ρ vS_0(u) = \Ec²u - \frac{2eA_0}ω(\ρ x\sin ωu -\ρ y\cos ωu)
 \end{equation*}
 Отсюда, суммируя, получаем выражение для $S$:
 \begin{equation*}
   S =\frac 1{4\ρ v}\hr{\Ec ²u - \frac{2λeA_0}ω\sin (ω(u-u_0))} +
   \ρ xx + \ρ yy + \ρ vv
 \end{equation*}
Теперь получим уравнения движения:
\begin{petit}
  Это можно оформить типографически аккуратней, но, в другой жизни.
\end{petit}
\begin{equation*}
  \begin{array}{clc}
    \displaystyle
    &\pf S{\ρ x} = x_0 = x + \frac 1{4\ρ v}\hr{-\frac{2eA_0}ω\sin ωu + 2\ρ xu} &
    \text{вращение и перемещение вдоль $u$}\\
    \displaystyle
    &\pf S{\ρ y} = y_0 = y+ \frac 1{4\ρ v}\hr{\frac{2eA_0}ω\cos ωu + 2\ρ yu} & \\
    \displaystyle
    &\pf S{\ρ v} = v_0 = v - \frac 1{4\ρ v}\hr{c²u -\frac{2eA_0}ω\hr{\ρ x\sin ωu + \ρ y\cos ωu}}  &
  \end{array}
\end{equation*}
Вернемся к старым координатам
\tbk
\end{solution}

\begin{tproblem}
  Найти время жизни электрона, движущегося по круговой орбите в атоме
  водорода, при учете излучения. То же для заряда, вращающегося в
  однородном магнитном поле.
\end{tproblem}
\begin{solution}
  Полагаем, что излучение мало. Тогда $\frac{d\Ec}{dt} = -I(R)$, где
  $I$ -- интенсивность излучения.

  Уравнение движения электрона $m\ddot{\vec r} = -\frac{e²\vec r}{r³}$
  -- центральное поле.  \questionably{Существует} круговая орбита, что
  $\vec a = -a\vec n$, $v = \const$.  Далее, $a = v²/r$, второй закон
  Ньютона $\hm{\vec F_\text{цент}} =\hm{\frac{mv²}r} = \frac{e²}{r²}$.
  \questionably{Полная энергия частицы}: $\Ec = \frac{mv²}2 - \frac{e²}r =
  -\frac{e²}{2r}$. Подставим в уравнение излучения:
  \begin{petit}
    Подходящую формулу я не нашел. Впрочем, я почти верю.
  \end{petit}
  \begin{equation*}
    \frac{e²\dot r}{2r²} = -\frac 23 e²a² = -\frac 23 \frac{e^6}{m²r^4}
  \end{equation*}
  Отсюда, $a²\dot r = -\frac 43ρ²$, $ρ ≝ e²/m$.
  Интегрируя, получаем систему
  \begin{equation*}
    \begin{array}{cc}
      \displaystyle\bcase{r³ = -4ρ²t + C\\ r(0) = r_0} &
      \displaystyle\bcase{r = \sqrt[3]{-4ρ²t + r³_0}}
    \end{array}
  \end{equation*}
  Конец жизни электрона означает, что $r = 0$. Отсюда $t = \frac{r³_0}{4ρ^2}$.
  \questionably{При учете того, что $T=t/c =\frac{r³_0}{4ρ^2c}$} получаем
  для $ ρ≈ 10^{-13}cm$, $r_0 ≈10^{-8}cm$, $c ≈3·10^{10}cm$ имеем $T≈10^{-9}cm$.

  Пункт про магнитное поле отстутствует. \tbk
\end{solution}

\input{task20.tex}
\input{task21.tex}
\input{task22.tex}
\end{document}
