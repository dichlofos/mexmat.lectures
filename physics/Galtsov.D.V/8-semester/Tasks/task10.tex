\begin{tproblem}
  Найти магнитное поле постоянного тока плотности $j$, текущего по
  поверхности цилиндра радиуса $R$
  \begin{enumerate}
  \item паралельно оси
  \item вдоль окружностей.
  \end{enumerate}
\end{tproblem}
\unsafeIO{}
\begin{solution}
Заметим, что ввиду симметричности задачи, вектор магнитного поля имеет
вид $\vecb = B_r\vec e_r + B_z\vec e_z$. По формуле Стокса(стр. 23)  имеем
\begin{equation*}
   ∮_{S¹_r}(B_r\vec e_r + B_z\vec e_z)dl = \frac{4π}c\intl{D²_r}{}\vec\jmath dS
\end{equation*}
В первом случае, $\vec\jmath = jδ(r-R)\vec e_z$, откуда
\begin{equation*}
  \intl{D²_r}{}\vec\jmath dS = 2πj\vec e_zδ_{r≥R}
\end{equation*}
откуда заключаем,что
\begin{equation*}
   ∮_{S¹_r}(B_r\vec e_r + B_z\vec e_z)dl = \frac{4π}c 2πj\vec e_zδ_{r≥R}
\end{equation*}
что означает, что
\begin{equation*}
  \begin{array}{cc}
    B_r ≡ 0 &  2πrB_z =\frac{4π}c j\vec e_z2πRδ_{r≥R}
  \end{array}
\end{equation*}
  Упрощая, $B_z = \frac{4π}cj\frac Rrδ_{r≥R}$.  Во втором случае,
  вектор $\vec\jmath$ направлен по касательной к окружности, и в
  правой части уравнения будет $0$ вне зависимости от радиуса круга $r$.
  Потому во втором случае магнитное поле равно 0.
\end{solution}
