\begin{tproblem}
  Найти связь между средним магнитным полем в круге и значением поля
  на окружности в бетатроне (ускорителе электронов, в котором
  ускоряющее электрическое поле создается благодаря электро-магнитной
  индукции неоднородным, зависящим от времени магнитным полем
  $B_z(ρ,t)$, $ρ² = x² + y²$, причем электроны ускоряются, оставаясь на окружности постоянного радиуса)
\end{tproblem}
\begin{solution}
  \newcommand{\vecec}{\vec\Ec}
  \newcommand{\δ}{\,d²\!}
  \newcommand{\Α}[1]{\!<\!\!#1\!\!>}
  По классическим определениям импульса и энергии,
  \begin{equation*}
    \begin{array}{c}
      \frac{d\vec p}{dt} = e\hr{\vece + \frac 1c[\vec v×\vecb]}\\
      \frac{d\vecec}{dt} = e(\vece, \vec v) = eEv
    \end{array}
  \end{equation*}
    Запишем закон индукции Фарадея ($D_2$ -- это круг)
    \begin{equation*}
      \begin{array}{c}
       \displaystyle  \intl {D²}{}\rot\vece d²x = ∮\limits_{S¹}\vece ds = 2πRE\\
       \displaystyle \intl {D²}{}\rot\vece\δ x  = -\frac 1c\frac{∂}{∂t}\intl{D²}{}\vecb \δ x = -\frac 1c πR²\frac{∂}{∂t}\Α{B}
      \end{array}
    \end{equation*}
    где $\Α{B}$ мы обозначили среднее магнитное поле в круге. Совмещая два равенства, получаем,
    \begin{equation*}
      E = -\frac{πR²}{2πRc}\frac{∂}{∂t}\Α{B} = -\frac{1}{2c}R\frac d{dt}\Α{B}.
    \end{equation*}
  Также нам понадобится равенство неясной природы
  \begin{equation*}
    pΩ = \hm{\hs{p×Ω}} = f_ν = \frac ec |[\vec v, \vecb\evn{R}]| = \frac ec VB_R
  \end{equation*}
  где $B_R$ -- среднее значение поля на окружности.

  Подставим полученное выражение для $E$ в имеющиеся формулы:
  \begin{equation*}
    \begin{array}{c}
      \displaystyle pΩ = \frac ec VB_R\\
      \displaystyle eE = \frac{dp}{dt}\frac 1R =\frac ec\dot B_R
    \end{array}
  \end{equation*}
  Приравнивая, получаем, что
  \begin{equation*}
    \begin{array}{ccc}
      \frac {R\dot B_R}c = \frac 1{2c}R\frac{d}{dt}\Α{B} & ⇒ &
      B_R = \frac 12\Α{B}
    \end{array}
  \end{equation*}
\end{solution}