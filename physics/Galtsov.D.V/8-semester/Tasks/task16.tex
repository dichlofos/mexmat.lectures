

\begin{tproblem}
  \newcommand{\ω}{ω\hr{t - z/c}}
  С помощью метода Гамильтона-Якоби найти закон движения и траекторию
  релятивистского заряда в поле плоской волны круговой поляризации,
  описываемой потенциалом
  \begin{equation*}
    \veca = A_0 (\cos\hr\ω, \sin\hr\ω, 0)
  \end{equation*}
\end{tproblem}
\begin{solution}
  \newcommand{\ω}{ω\hr{t - z/c}}
  \newcommand{\ρ}[1]{p_{0#1}}
  Запишем уравнение Гамильтона-Якоби (стр. 64). Полагаем, что $φ=0$,
  т.к \questionably{электрическое поле отсутствует}.
  \begin{equation*}
    \begin{array}{c}
      \frac 1{c²}\hr{\pf St + eφ}^2 - \hr{\nabla S -\frac ec\veca}² = m²c²\\
      \displaystyle \frac 1{c²}\hr{\pf St}² - \hr{\pf Sx - \frac{eA_0\cos\hr\ω}c}²
      + \hr{\pf Sy - \frac{eA_0\sin\hr\ω}c}² + \hr{\pf Sz}² = m²c²
    \end{array}
  \end{equation*}
  Сделаем замену $u = t-z/c$, $v= t+z/c$.
  Тогда
  \begin{equation*}
    \begin{array}{ccc}
      \displaystyle
      \pf St = \pf Su + \pf Sv & &
      \displaystyle
      \pf Sz = \frac 1c\hr{\pf Sv - \pf Su}
    \end{array}
  \end{equation*}
  Подставим в уравнение:
  \begin{equation*}
    \begin{array}{c}
      \displaystyle
      \frac 1{c²}\hr{\pf Su + \pf Sv}² -\hr{\pf Sx -\frac{eA_0\cos ωu}c}^2
      -\hr{\pf Sy - \frac{eA_0\sin ωu}c}² - \frac 1{c²}\hr{\pf Sv - \pf Su}² =  m²c² \\
      \displaystyle
      4\pf Su\pf Sv = (c\pf Sx -eA_0\cos ωu)² + (c\pf Sy - eA_0\sin ωu) + m²c² \\
      \displaystyle
      4\pf Su\pf Sv = c²\hr{\hr{\pf Sx}²+\hr{\pf Sy}²} -2eA_0c(\pf Sx\cos ωu + \pf Sy\sin ωu) +m²с^4 +e²A_0²
    \end{array}
  \end{equation*}
Ищем решение в виде $S = S_0(u) + S_1(x) + S_2(y) + S_3(v)$. В таком
предположении, уравнение переписывается
 \begin{equation*}
   m²c^4 + e²A²_0 = 4S'_0(u)S'_3(v) - c²\hr{S'_1(x)}² -c²\hr{S'_2(y)}²
   + 2S'_1(x)ecA_0\cos ωu + 2S'_2(y)ecA_0\sin ωu
 \end{equation*}
 Выбираем константы первых интегралов (как разделяющиеся переменные)
 \begin{equation*}
   \begin{array}{ccc}
     сS'_1(x) = \ρ x =\const &
     cS'_2(y) = \ρ y = \const &
     S'_3(v) = \ρ v = \const
   \end{array}
 \end{equation*}
 Переписываем уравнение
 \begin{equation*}
   m²c^4 + e²A²_0 = 4S'_0(u)\ρ v - {\ρ x}² -{\ρ y}² + 2\ρ xeA_0\cos ωu + 2\ρ yeA_0\sin ωu
 \end{equation*}
 \begin{denotes}
   $\Ec² ≔ m²c^4 + e²A²_0 + \ρ x² + \ρ y²$, $λ = \sqrt{\ρ x² + \ρ y²}$, $ξ =ω(u-u_0)$
 \end{denotes}
 Интегрируем, разворачивая $u_0$ обратно. Авторский замысел, видимо.
 \begin{equation*}
   4\ρ vS_0(u) = \Ec²u - \frac{2eA_0}ω(\ρ x\sin ωu -\ρ y\cos ωu)
 \end{equation*}
 Отсюда, суммируя, получаем выражение для $S$:
 \begin{equation*}
   S =\frac 1{4\ρ v}\hr{\Ec ²u - \frac{2λeA_0}ω\sin (ω(u-u_0))} +
   \ρ xx + \ρ yy + \ρ vv
 \end{equation*}
Теперь получим уравнения движения:
\begin{petit}
  Это можно оформить типографически аккуратней, но, в другой жизни.
\end{petit}
\begin{equation*}
  \begin{array}{clc}
    \displaystyle
    &\pf S{\ρ x} = x_0 = x + \frac 1{4\ρ v}\hr{-\frac{2eA_0}ω\sin ωu + 2\ρ xu} &
    \text{вращение и перемещение вдоль $u$}\\
    \displaystyle
    &\pf S{\ρ y} = y_0 = y+ \frac 1{4\ρ v}\hr{\frac{2eA_0}ω\cos ωu + 2\ρ yu} & \\
    \displaystyle
    &\pf S{\ρ v} = v_0 = v - \frac 1{4\ρ v}\hr{c²u -\frac{2eA_0}ω\hr{\ρ x\sin ωu + \ρ y\cos ωu}}  &
  \end{array}
\end{equation*}
Вернемся к старым координатам
\tbk
\end{solution}
