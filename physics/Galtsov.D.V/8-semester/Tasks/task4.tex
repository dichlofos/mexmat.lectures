\begin{tproblem}
  Равномерно заряженная сфера радиуса $R$ и заряда $Q$ разрезана на
  две половины. Найти силу взаимодействия половин.
\end{tproblem}
\begin{solution}
  Т.к сфера осталась, её внутреннее поле не изменилось и равно нулю по
  принципу максимума. Поместим внтрь отриц ательно заряженную
  полусферу. Силы взаимодействия половин равны ( '$+$' и '$-$'
  притягиваяются с той же силой, что и '$+$' и '$+$' отталкиваются).
  Поэтому, достаточно найти взаимодействие двух вложенных полусфер.
  ($F_{21} = F_{31} = F_{32}$).  \questionably{«Разогнем
      задачу»}. Горизонтальные составляющие скомпенсируются по
  симметрии, вертикальне -- проинтегрируем. $φ_{zz} = -4πeδ(z)$ $⇒$ $φ
  = -4πe|z|$.
  \begin{petit}
    В исходниках коэффицент 2, но этого совершенно непонятно.
  \end{petit}
  $E = -4πq_1θ(z)|_{0-}^{0+} = -4πq_1$. Тогда сила взаимодействия
  элементарных зарядов $F= Eq_2 = -4πq_1q_2$. У нас $q_1 = q_2 = σ$.
  \questionably{Сила взаимодействия полусфер тогда равна}
  \begin{equation*}
    F = \intl{0}{2π}\intl{0}{\frac π2}R²F\sin θ\cos θdθdφ = σ²2πR²π = \frac{Q²}{8R²}
  \end{equation*}
  где $σ$ -- поверхностная плотность заряда, $Q = 4πR²σ$.
\end{solution}