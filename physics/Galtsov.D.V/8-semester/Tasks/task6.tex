\begin{tproblem}
  Найти индукционное электрическое поле, порождаемое магнитным полем тока
  \begin{equation*}
    \begin{array}{cc}
      j_x = iδ(z)\cos ωt, &j_y = j_z = 0
    \end{array}
  \end{equation*}
\end{tproblem}
\begin{solution}
  \newcommand{\vim}{\vec\imath}
  \newcommand{\vjm}{\vec\jmath}
  \begin{petit}
    Куда дели мнимую единицу? Впрочем, туда ей и дорога.
  \end{petit}
  Введем обозначение -- вектор $\vim = (1,0,0)$. Тогда, $\vjm =
  \vim\cos ωt δ(z)$.  $\rot\vecb = \frac{4π}c\vjm$,
  \questionably{поэтому} $\vecb ⊥ \vec e_x$, что означает $B_x =0$, и,
  по закону Био-Савара-Лапласа, $B_z = 0$.
  \begin{petit}
    Странно как-то получается. Несимметрично. Чем координата $z$ лучше
    координаты $y$?
  \end{petit}
  \tbk
\end{solution}