\documentclass[a4paper,draft]{article}
\usepackage[simple]{dmvn}

\def\iof.#1#2.#3.{#1.\,#2.\,#3}

\title{Программа и задачи к экзамену по физике}
\author{Лектор\т Дмитрий Владимирович Гальцов}
\date{8--9 семестр, 2006 г.}
\begin{document}
\maketitle

\section{8 семестр}

\subsection{Экзаменационные билеты (2005, 2006)}

\emph{К каждому билету прилагается одна задача из списка.}

\begin{nums}{-2}
\item Закон Кулона. Потенциал, уравнение Пуассона. Энергия электрического поля.
\item Магнитное поле. Вектор\д потенциал. Сила Лоренца. Энергия магнитного поля.
\item Закон электромагнитной индукции. Сохранение заряда и ток смещения Максвелла.
\item Уравнения Максвелла в трехмерной форме. Теорема Умова\ч Пойнтинга. Тензор натяжений.
\item Электромагнитные волны. Волновой вектор, поляризация.
\item Преобразования Лоренца. Кинематические эффекты СТО. Сложение скоростей.
\item Пространство Минковского. Мировая линия, 4\д скорость и 4\д ускорение.
\item 4\д потенциал и тензор электромагнитного поля. Преобразование компонент поля. Уравнение Максвелла в четырёхмерной форме.
\item Уравнения движения релятивистского заряда. Функция Лагранжа и действие. Принцип наименьшего действия в релятивистской механике.
\item Обобщенный импульс. Гамильтониан. Уравнение Гамильтона\ч Якоби для заряда в электромагнитном поле.
\item Движение релятивистского заряда в кулоновом поле.
\item Действие для электромагнитного поля. Тензор энергии-импульса.
\item Функции Грина уравнения Д'Аламбера. Запаздывающие потенциалы.
\item Излучение ускоренного заряда. Рассеяние электромагнитных волн зарядами.
\item Гравитационное поле. Принцип эквивалентности. Метрика и связность.
\item Уравнение геодезических и уравнение Гамильтона\ч Якоби частицы в гравитационном поле. Симметрии и законы сохранения.
\item Решение Шварцшильда. Чёрные дыры. Отклонение луча, смещение перигелия.
\item Тензор кривизны. Уравнения Эйнштейна. Принцип наименьшего действия. Линейное приближение.
\item Однородные и изотропные космологические модели.
\end{nums}

\begin{thebibliography}{5}
\setlength\itemsep{-2pt}
\bibitem{gal-gr}
    \iof.ДВ.Гальцов., \iof.ЮВ.Грац., \iof.ВЧ.Жуковский.. \emph{Класические поля.}\т М.: Изд-во Моск. ун-та, 1991.
\bibitem{gal}
    \iof.ДВ.Гальцов.. \emph{Теоретическая физика для студентов\д математиков.}\т М.: Изд-во Моск. ун-та, 2003.
\bibitem{ll}
    \iof.ЛД.Ландау., \iof.ЕМ.Лифшиц.. \emph{Теория поля.}\т М.: ФизМатЛит, 2001.
\bibitem{arnold}
    \iof.ВИ.Арнольд.. \emph{Математические основы классической механики.}\т М.: Наука, 1989.
\bibitem{dnf}
    \iof.БА.Дубровин., \iof.СП.Новиков., \iof.АТ.Фоменко.. \emph{Современная геометрия: методы и приложения.}\т 5-е изд., испр.\т М.: УРСС, 2001.
\end{thebibliography}

\subsection{Задачи}

\subsubsection{Электростатика и электродинамика. Магнитное поле}

\begin{nums}{-2}
\item Найти ёмкость $C = \frac{q}{\ph_1-\ph_2}$\\
1) плоского конденсатора (не заполненного диэлектриком), состоящего из двух параллельных пластин
площади $S$ на расстоянии $d$ ($d^2 \ll S$);\\
2) конденсатора, образованного двумя концентрическими сферами радиусов $R_1$ и $R_2$.

Здесь $q$\т заряд одной из обкладок, $\ph_1$ и $\ph_2$\т потенциалы обкладок.

\item Найти электростатический потенциал и электрическое поле шара радиуса $R$ (внутри и снаружи),
заряженного по объёму с плотностью $\rho$.

\item Рассчитать распределение поверхностного заряда на бесконечной проводящей заземлённой
плоскости, индуцированное точечным зарядом $q$, находящимся на расстоянии $h$ от неё.

\item Рассчитать распределение поверхностного заряда на проводящей сфере радиуса $R$,
индуцированное точечным зарядом $q$, находящимся на расстоянии $h > R$ от её центра.

\item Равномерно заряженная сфера (заряд $Q$, радиус $R$) разрезана на две половины.
Найти силу взаимодействия половин.

\item По тонкому проводнику, образующему правильный $n$\д угольник со стороной $a$, течёт постоянный ток $J$.
Найти магнитное поле в центре.

\item Найти силу, разрывающую на две половины равномерно заряженный по объёму шар (заряд $Q$, радиус $R$).

\item Диск радиуса $R$, равномерно заряженный с поверхностной плотностью $\si$, вращается с постоянной
угловой скоростью $\om$. Найти магнитное поле на оси симметрии.

\item Шар радиуса $R$, равномерно заряженный с плотностью $\rho$, вращается с постоянной
угловой скоростью $\om$. Найти магнитное поле в центре шара.

\item Определить потенциал и электростатическое поле, создаваемое плотностью заряда $\rho(\vec r) = \rho_0 \cos(\vec k, \vec r)$,
где $\vec k$\т заданный постоянный вектор.

\item Рассчитать вектор\д потенциал и магнитное поле, создаваемое равномерно заряженной сферой
(полный заряд $Q$, радиус $R$), вращающейся с постоянной угловой скоростью $\om$.

\item Найти магнитное поле, создаваемое постоянным и однородным поверхностным током, текущим по бесконечной плоскости.

\item Найти магнитное поле постоянного тока плотности $i$, текущего по поверхности цилиндра радиуса $R$\\
1) параллельно оси;\\
2) вдоль окружностей.
\end{nums}

\subsubsection{Специальная теория относительности}

\begin{nums}{-2}
\item  Получить формулы релятивистского сложения скоростей
в векторном виде, считая скорость движущейся системы отсчёта
произвольно направленной.
\item Получить формулы преобразования трёхмерного ускорения
при переходе к движущейся системе отсчёта.
\item Записать преобразования Лоренца в форме $\SL(2,\Cbb)$.
\item Показать, что композиция двух преобразований Лоренца
с непараллельными скоростями не является чистым преобразованием Лоренца.
\item Найти закон движения частицы, испытывающей постоянное ускорение
в собственной системе отсчёта.
\item Рассчитать движение релятивистского заряда в постоянных
и однородных электрическом и магнитном полях равной величины\\
1) ортогональных,\\
2) параллельных.
\item С помощью уравнения Гамильтона\ч Якоби найти закон движения
и траекторию релятивистского заряда в поле плоской электромагнитной
волны круговой поляризации, описываемой вектор\д потенциалом
$$A = A_0 \hc{\cos\hr{\om\hr{t-\frac{z}{c}}}, \sin\hr{\om\hr{t - \frac{z}{c}}},0}$$
\item Рассчитать движение релятивистского заряда в кулоновом поле.
Рассмотреть все случаи финитного и инфинитного движений.
\item Исходя из релятивистского закона сохранения энергии и импульса,
доказать, что электрон\д позитронная пара не может аннигилировать в один фотон.
Возможна ли аннигиляция с образованием двух фотонов?
\item Определить скорость системы отсчёта, в которой постоянные и однородные
электрическое и магнитное поля параллельны. Всегда ли существует решение
и единственно ли оно?
\item Найти закон отражения и частоту отражённой волны при наклонном падении
на плоское зеркало, движущееся со скоростью $V$\\
1) параллельно своей плоскости,\\
2) перпендикулярно ей.
\item Показать, что сумма тензоров энергии\д импульса системы релятивистских зарядов
и электромагнитного поля имеет нулевую дивергенцию в силу уравнений
движения и уравнений Максвелла.
\end{nums}

\subsubsection{Излучение. Общая теория относительности}

\begin{nums}{-2}
\item Вычислить угловое распределение и полную интенсивность излучения
нерелятивистского заряда, совершающего гармонические колебания вдоль оси с амплитудой
$z_0$ и частотой $\om$ ($\om z_0 \ll c$).
\item Найти время жизни электрона, движущегося по круговой орбите в атоме водорода, при учёте излучения.
\item Найти полную интенсивность и угловое распределение излучения релятивистского заряда, движущегося в
постоянном и однородном магнитном поле в плоскости, перпендикулярной полю. Какова доля излучения,
поляризованного в плоскости движения (вектор электрического поля лежит в этой плоскости)?
\item Получить сечение рассеяния электромагнитной волны частоты $\om$ на связанном электроне,
совершающем колебания частоты $\om_0$ вдоль оси, направленной по вектору поляризации волны. Движение считать
нерелятивистским.
\item Вывести формулу для смещения полуоси эллиптической орбиты за счёт релятивистских поправок при движении
в поле Шварцшильда.
\item Показать, что радиальные геодезические в поле Шварцшильда приближаются
к горизонту событий за бесконечное время по часам удалённого наблюдателя. Каково время падения
по собственным часам?
\item Рассчитать параметры всех круговых орбит массивных и безмассовых частиц в поле Шварцшильда.
\item Рассчитать космологическое красное смещение в однородных и изотропных моделях Вселенной.
\end{nums}


\medskip\dmvntrail

\newpage

\section{9 семестр}

\subsection{Программа экзамена (2006)}

\begin{nums}{-2}
\item Волновой пакет. Фазовая и групповая скорость. Расплывание.
\item Состояния и наблюдаемые. Непрерывный спектр. Квантовые
      скобки Пуассона. Представления.
\item Одновременно измеримые
      величины. Соотношение неопределенностей. Полные наборы
      наблюдаемых.
\item Картины Гейзенберга и Шредингера. Уравнение
      Шредингера. Оператор эволюции.
\item Квазиклассическое приближение. Сшивание волновых функций в точках поворота для
      одномерного движения. Формула Бора\ч Зоммерфельда.
\item Стационарная теория возмущений. Случай вырождения невозмущенного
      спектра. Эффект Штарка.
\item Общие свойства спектра одномерного
      оператора Шредингера. Потенциальная яма и потенциальный барьер.
      Периодический потенциал.
\item Гармонический осциллятор. Когерентные состояния.
\item Движение в центральном поле. Падение на центр.
\item Атом водорода. Симметрия $SO(4)$.
\item Рассеяние в центральном поле. Фазы и сечение рассеяния. Формула Резерфорда.
\item Заряд в электромагнитном поле. Калибровочная инвариантность
      уравнения Шредингера. Спектр энергий в однородном магнитном поле.
\item Квантовые переходы в дискретном и непрерывном спектре.
      Борновское приближение в теории рассеяния.
\item Спин. Уравнение Паули.
\item Тождественные частицы. Обменное взаимодействие.
\item Понятие о периодической системе элементов. Модель Томаса-Ферми.
\item Матрица плотности. Уравнение эволюции.
\item Энтропия и температура. Термодинамические потенциалы. Химический потенциал.
\item Микроканоническое и каноническое распределения. Вычисление
      статистической суммы в квазиклассическом приближении.
\item Большое каноническое распределение. Больцмановский идеальный газ.
\item Статистика Ферми. Вырожденный идеальный Ферми\д газ.
      Уравнение состояния в нерелятивистском и
      ультрарелятивистском случаях.
\item Статистика Бозе. Конденсация идеального Бозе\д газа.
      Химический потенциал вблизи точки конденсации.
\item Черное излучение.
\end{nums}

\subsection{Задачи}

\begin{nums}{-2}
\item
Записать стационарное уравнение Шредингера {\em в импульсном
представлении} для одномерного движения в поле
$$
U=-\alpha \;\delta(x),\quad \alpha>0
$$
и построить решения, принадлежащие дискретному и непрерывному
участкам спектра.

\item
<<Одномерная молекула>>. Найти спектр энергий связанных
состояний $E_n,$ (занумерованных в порядке возрастания начиная
с $n=0$) в поле
$$
U=-\alpha (\delta(x+a/2) + \delta(x-a/2)),\quad \alpha>0.
$$
Показать что <<энергия молекулы>> в основном состоянии как
функция расстояния между центрами
$$
V(a)=\frac{q^2}{a}+E_0(a),\quad q={\rm const}
$$
имеет минимум.


\item
Построить гейзенберговские операторы координаты и импульса для
частицы в поле
$$
U(x)=a+bx+cx^2,\quad a,\,b,\,c={\rm const}.
$$

\item
Найти комплексные собственные значения энергии квазистационарных
состояний в поле
$$
U=\infty \; \theta(-x) + \alpha\delta(x-a),\quad \alpha>0,
$$
предполагая, что
$$
\psi(x)=A\exp(ikx),\quad k\in C
$$
при $x>a$.

\item
Найти спектр энергий в поле
$$ U=g\sum_{n=-\infty}^{n=\infty}(-1)^n\delta(x+nl)
$$
\item
Доказать формулу Бейкера\ч Хаусдорфа,
$$
{\rm e}^X\,{\rm e}^Y\,={\rm e}^{Z/2}\,{\rm e}^{(X+Y)},
$$
где $X,\,Y,\,Z$ операторы, удовлетворяющие соотношениям
$$
 [X,\,Y]=Z,\quad [X,\,Z]= 0,\quad [Y,\,Z]=0.
$$

\item
Вычислить коммутаторы
$$
[x,\,f(p)],\quad [p,\,f(x)],\quad [a,\,f(a^+)],
$$
где $x,\,p,\,a,\,a^+$ операторы, удовлетворяющие перестановочным
соотношениям
$$
[p,\,x]=\frac{\hbar}{i}\;,\quad [a,a^+]=1
$$
и $f$ -- аналитическая функция.
\item
Имеет ли оператор $id/dx$ самосопряженные расширения в
$L_2([0,\,\infty))$? То же для оператора $i(d/dx+1/x)$.
Построить все самосопряженные расширения оператора $id/dx$ в
$L_2([0,\,1])$.


\item
Построить решения задачи на собственные значения для
гамильтониана
 $$
H=\frac{p^2}{2m}+\frac{m \omega^2 x^2}{2}+\alpha \delta(x)
 $$
для всех значений вещественных параметров $\omega,\, \alpha$.

\item Найти коэффициенты в рекуррентных соотношениях для
нормированных собственных векторов $|l,m >$ операторов
$L^2=L_x^2+L_y^2+L_z^2\,$ и $ L_z$
$$
 L_+|l,m>=C |l,m+1>,\quad L_-|l,m>=D |l,m-1>,
$$
где $L_\pm=L_x\pm iL_y$. \item Найти спектр энергий частицы в
<<непробиваемой консервной банке>>
$$U=0, \quad \mbox{при} \quad \sqrt{x^2+y^2}<a,\;\; 0<z<h,$$
и $U=\infty$ во внешней области. \item Найти волновые функции
стационарных состояний и уровни энергии частицы, движущейся в {\em
трехмерном} пространстве в поле
$$U=-\frac{\alpha}{\sqrt{x^2+y^2}}.$$



\item Построить интегралы движения для электрона в постоянном и
однородном магнитном поле (описываемом уравнением Паули) включая
спиновый. Доказать, что угловая скорость прецессии спина совпадает
с угловой скоростью орбитального движения.

\item Построить
волновые функции стационарных состояний электрона в параллельных
(постоянных и однородных) электрическом и магнитном полях
напряженности $E,B$.

\item В рамках теории возмущений рассчитать
расщепление уровней атома водорода (с учетом спина электрона) в
слабом однородном магнитном поле.
Спин-орбитальным взаимодействием пренебречь.

 \item В борновском приближении
вычислить дифференциальное сечение рассеяния на потенциале Юкавы:
$$V(r)= g\frac{e^{-ar}}{r}$$

\item Нейтральная частица со спином $S=\frac{1}{2}$ и магнитным
моментом $\mu$ находится при $t=0$ в состоянии с проекцией спина
на некоторое направление, равной $ \frac{1}{2}$. Рассмотреть
прецессию магнитного момента в магнитном поле, перпендикулярном
этому направлению и имеющем напряженность $B$. Решить задачу в
представлении  Гейзенберга. Найти направление, вдоль которого
ориентирован спин в момент времени $t$.

\item Найти собственные значения и собственные  векторы оператора
$\mbox{exp}  ({ \sigma}_k  {  a}_k)  $, где $ {\sigma}_k$ --
матрицы Паули, а $ a_1 , a_2 , a_3 $ -- действительные числа.
\item В рамках теории возмущений вычислить поправку к энергии
орто- и парагелия в основном состоянии за счет взаимодействия
между электронами.

\item
 Вычислить вероятности квантовых переходов одномерного
гармонического осциллятора под действием возмущения
$$V=\alpha \;x\;\delta(t). $$
\item Вычислить давление и теплоемкость идеального Ферми - газа
при $T < \epsilon_F$ (энергия Ферми). \vskip5pt Указание
$$ \int \frac{\epsilon^{\nu} d\epsilon}{\exp
(\frac{\epsilon - \mu}{T} ) + 1} \ =\ \frac{\mu^{\nu +1}}{\nu + 1}
\biggl( 1+ \frac{1}{6} \pi^2 \nu \biggl( \nu + 1\biggr)
\biggl(\frac{T}{\mu}\biggr)^2 + \ldots \biggr)
$$

\item Доказать, что для идеального Бозе / Ферми газа
$$\ha{(n_{\vec p } \ - \ha{ n_{\vec p} } )^2 } \ = \ha{ n_{\vec p} } (1 \pm \ha{ n_{\vec p} })\ . $$
\item Вычислить давление и теплоемкость идеального Бозе\д газа при
$T < T_0$ (температура вырождения). Показать, что производная
теплоемкости по температуре испытывает скачок при $T=T_0$.   \item
Получить распределение Планка для спектра излучения черного тела.
Вычислить энтропию и свободную энергию. \item Найти КПД цикла
Карно и доказать его максимальность для замкнутых циклов. \item
Найти свободную энергию системы гармонических осцилляторов в
термостате при температуре $T$. \item Матрица плотности
осциллятора равна:
$$
\wh{\rho} = \frac{\suml{n=0}{\infty} \bs{|n\rangle\langle n|}
\hs{\exp\hr{ -\frac{E_n}{kT}}}}
{\suml{i=0}{\infty} \exp\hr{-\frac{E_i}{kT}}}$$
Найти плотность распределения координаты.
\end{nums}



\medskip\dmvntrail


\end{document}
