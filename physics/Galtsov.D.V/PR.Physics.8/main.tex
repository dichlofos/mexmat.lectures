\documentclass[unicode,10pt]{article}
\usepackage{dmvn}
\usepackage{polyglossia}
\usepackage{unicode-math}
\usepackage{fontspec}
\usepackage{epsfig}
\usepackage{ulem}
\usepackage{wrapfig}
\defaultfontfeatures{Scale=MatchLowercase, Mapping=tex-text}
\setmainfont{CMU Serif}
\setsansfont{CMU Sans Serif}
\setmonofont{CMU Typewriter Text}
\setmathfont{xits-math.otf}
\newcommand{\questionably}[1]{{
    \setmainfont{URW Chancery L}
    #1}}
\newcommand{\contrib}[1]{
  {
  \setmainfont{URW Chancery L}
  \uwave{#1}
  }}
\newcommand{\veca}{\vec A}
\newcommand{\vecb}{\vec B}
\newcommand{\vece}{\vec E}
\newcommand{\eX}{\vec e_x}
\newcommand{\eY}{\vec e_y}
\newcommand{\eZ}{\vec e_z}
\newcommand{\eR}{\vec e_r}
\newcommand{\unsafeIO}{
  \begin{petit}
    {\Large \bf\#!} Творчество верстальщика.
  \end{petit}
}
\begin{document}
\begin{tproblem}
  Построить электростатический потенциал (решение уравнения Пуассона)
  в $R^n$, $n≥2$ для
  \begin{itemize}
  \item Точечного заряда
  \item Заряженного шара (внутри и снаружи)
  \item Заряда, равномерно распределенного по двумерной плоскости ($n≥3$)
  \end{itemize}
\end{tproblem}
\begin{solution}
  \begin{itemize}
  \item Плотность заряда $ρ=eδ^n(\vec r -\vec r_0)$. Без ограничения
    общности, можно считать, что $\vec r_0 = \vec 0$.
    \begin{denote}
      $S_n^R$ -- площадь $n$-мерной сферы радиуса $R$.
    \end{denote}
    \begin{denote}
      $V_n^R$ -- объем $n$-мерного шара радиуса $R$.
    \end{denote}

    По электростатической теореме Гаусса (в учебнике коэффицент
    $4π$ для $n=3$, в общем случае он равен $S^1_{n-1}$)
    \begin{equation*}
      \displaystyle
      ∮\limits_{∂V}\vec E·d\!\vec S = S¹_{n-1}\!∫\limits_V ρ(\vec r)\,d^nx
    \end{equation*}
    для любого шара с центром в заряде, имеем
    \begin{equation*}
      S¹_{n-1}e = ES^r_{n-1}
    \end{equation*}
    т.к из соображений симметрии, $\vec E |\!| \vec n$.  Т.к
    $S^r_{n-1} = r^{n-1}S¹_{n-1}$, то
    \begin{equation}
      E= \frac{e}{r^{n-1}} = -\grad φ
    \end{equation}
    Откуда $φ = \frac{e}{(n-2)r^{n-2}} +c$. Константу полагаем нулю, что бы $φ(∞) = 0$.
  \item Равномерно заряженный с плотностью $ρ_0$ шар радиуса $R$,
    \begin{enumerate}
    \item Вне шара $r>R$. Из соображений симметрии, $φ=φ(r)$. По т. Гаусса
      \begin{equation*}
        \begin{array}{c}
          \displaystyle S¹_{n-1}(ρ_0V^R_n) = ES^r_{n-1} \\
          \displaystyle E = \frac{ρ_0V^R_n}{r^{n-1}} = -\grad φ \\
          \displaystyle  φ = \frac{ρ_0V^R_n}{(n-2)r^{n-2}}
        \end{array}
      \end{equation*}
    \item Внутри шара, $r<R$ также по т.Гаусса
      \begin{equation*}
        \begin{array}{c}
          \displaystyle S¹_{n-1}V¹_nr^nρ_0 = ES^r_{n-1} = ES¹_{n-1}r^{n-1}\\
          \displaystyle \vec E= ρ_0V¹_n\vec r = V¹_nρ_0\grad{\frac{r^2}{2}} = -\grad φ \\
          \displaystyle φ = -V¹_nρ_0\frac{r^2}{2}+φ_0
        \end{array}
      \end{equation*}
      Выберем $φ_0$ по непрерывности в $r = R$:
      \begin{equation*}
        -V¹_nρ_0\frac{r^2}{2}+φ_0 = \frac{ρ_0V^R_n}{(n-2)r^{n-2}}
      \end{equation*}
      Отсюда, $φ_0 = R^2V¹_nρ_0(\frac 1{n-2} + \frac 12)$.
      Получаем, полное выражение потенциала
      \begin{equation*}
        φ =
        \begin{aligned}
          \bcase{
            \frac{n}{2(n-2)}R^2V¹_nρ_0 -V¹_nρ_0\frac{r^2}{2}&,r≤R\\
            \frac{ρ_0V^R_n}{(n-2)r^{n-2}}&,r>R
          }
        \end{aligned}
      \end{equation*}
    \end{enumerate}
  \item $σ$ -- плотность заряда на плоскости.(Картинка будет
    когда-нибудь.) Рассмотрим сечение $R^n$ пространством
    $R^{n-2}=R(x_3,…,x_n)$. От плоскости будет точка и $φ=\frac
    1r^{n-4}$. Из  симметрии, $φ = (x²_3+…+x²_n)^{\frac{4-n}2}$
  \end{itemize}
\end{solution}

\begin{tproblem}
  Построить электростатический потенциал $φ(x,y,z,θ)$ точечного заряда
  в пространстве $M=R³×S¹$ с компактными дополнительным измерением:
  $dl² = dx² + dy² + dz² + ρ²_0dθ²$, $ρ_0=\const$, $0≤ θ ≤ 2π$,
  $(x,y,z) ∈R³$. Считать, что заряд покоится в точке
  $x=y=z=θ=0$. Рассмотреть предельные случаи $φ(x,y,z,0)$ при $r <\!<
  ρ_0$ и $r >\!> ρ_0$, где $r²=x²+y²+z²$.
\end{tproblem}
\begin{hint}
  Искомый потенциал является решением уравнения Пуассона в $M$
  \begin{equation*}
    ∂²_x +∂²_y+∂²_z+\frac{1}{ρ²_0}∂²_θ = \frac{q}{ρ_0}δ(x)δ(y)δ(z)δ(θ)
  \end{equation*}
\end{hint}
\begin{tproblem}
  Нерелятивистский заряд падает под углом на плоскость $z$, несущую
  заряд с плотностью
  \begin{equation*}
    ρ=λ\frac{d}{dz}δ(z)
  \end{equation*}
  (двойной электрический слой). Под каким углом заряд выйдет с другой
  стороны плоскости? (механического взаимодейтвия нет)
\end{tproblem}
\begin{solution}
  \newcommand{\Eres}{4πλδ(z)}
  \newcommand{\fd}[2]{\frac{d#1}{d#2}}

  Из соображений симметрии, $E$ не зависит от $x$,$y$.
  Запишем уравнение Пуассона
  \begin{equation*}
    4πλδ'(z) = \Div E = \pf{E}{z}
  \end{equation*}
  \begin{wrapfigure}{l}{0.16\textwidth}
    \begin{center}
      \includegraphics[width=0.16\textwidth]{t3image.eps}
    \end{center}
  \end{wrapfigure}
  Проинтегрировав, получаем $4πλδ(z) + c= E$. \questionably{Вне двойного слоя поля
    нет, поэтому $c=0$.} Это поле изменяет импульс частицы по координате
  $z$, т.к не влияет на $x$, $y$ координаты. Т.к заряд нерелятивистский,
  то верны формулы классической механики:
  \begin{equation*}
    \begin{array}{ccc}
      \displaystyle
      \vec p = m\vec v & \frac{d\vec p}{dt} = \vec F = e\vec E & \frac{d\vec p_z}{dt} = m\frac{d\vec v_z}{dt} = \Eres e
    \end{array}
  \end{equation*}
  После соударения
  \begin{equation*}
    \fd pt = m\pf{v_z}{z}\pf zt = m\pf{v_z}{z}v_z = \frac m2 \frac{∂}{∂z}v²_z
  \end{equation*}
  Отсюда, $p(+0) - p(-0) = \frac 12 mv²_z\Big{|}_{\text{
      до}}^{\text{после}} = 4πeλ$. Т.е $v²_z(+0) - v²_z(-0) = \frac{8πeλ}{m}$. Отсюда получаем соотношение на углы
  \begin{equation*}
    v_z(+0) = \sqrt{\frac{8πeλ}{m}+v²_z(-0)} ⇒ \cos β = \sqrt{\frac{8πeλ}{m}+\cos^2α}
  \end{equation*}
\end{solution}
\begin{tproblem}
  Равномерно заряженная сфера радиуса $R$ и заряда $Q$ разрезана на
  две половины. Найти силу взаимодействия половин.
\end{tproblem}
\begin{solution}
  Т.к сфера осталась, её внутреннее поле не изменилось и равно нулю по
  принципу максимума. Поместим внтрь отриц ательно заряженную
  полусферу. Силы взаимодействия половин равны ( '$+$' и '$-$'
  притягиваяются с той же силой, что и '$+$' и '$+$' отталкиваются).
  Поэтому, достаточно найти взаимодействие двух вложенных полусфер.
  ($F_{21} = F_{31} = F_{32}$).  \questionably{«Разогнем
      задачу»}. Горизонтальные составляющие скомпенсируются по
  симметрии, вертикальне -- проинтегрируем. $φ_{zz} = -4πeδ(z)$ $⇒$ $φ
  = -4πe|z|$.
  \begin{petit}
    В исходниках коэффицент 2, но этого совершенно непонятно.
  \end{petit}
  $E = -4πq_1θ(z)|_{0-}^{0+} = -4πq_1$. Тогда сила взаимодействия
  элементарных зарядов $F= Eq_2 = -4πq_1q_2$. У нас $q_1 = q_2 = σ$.
  \questionably{Сила взаимодействия полусфер тогда равна}
  \begin{equation*}
    F = \intl{0}{2π}\intl{0}{\frac π2}R²F\sin θ\cos θdθdφ = σ²2πR²π = \frac{Q²}{8R²}
  \end{equation*}
  где $σ$ -- поверхностная плотность заряда, $Q = 4πR²σ$.
\end{solution}
\begin{tproblem}
  Найти силу взаимодействия пластин с током в магнитном конденсаторе
  на единицу площади (противоположные токи в паралельных плоскостях).
\end{tproblem}
\begin{tproblem}
  Найти индукционное электрическое поле, порождаемое магнитным полем тока
  \begin{equation*}
    \begin{array}{cc}
      j_x = iδ(z)\cos ωt, &j_y = j_z = 0
    \end{array}
  \end{equation*}
\end{tproblem}

\begin{tproblem}
  Полупространство $z > 0$ заполнено однородным магнитным полем,
  направленным паралельно плоскост $z = 0$. Заряд падает из области $z
  < 0$ под углом $α$ к плоскости. Найти траекторию дальнейшего движения.
\end{tproblem}
\unsafeIO{}
\begin{solution}
  \newcommand{\vecr}{{\vec r}}
  Пусть положение частицы описывается её координатами $\vec r =
  x\eX+y\eY+z\eZ$. Не ограничивая общности, полагаем, что $\vecr(0) =
  0$, $\dot\vecr(0) = ξ\cos α\eX + ξ\sin α\eZ$, $ξ = \const$, $\vecb =
  B\eX$ На частицу действует сила Лоренца $F$, вычисляемая по формуле
  \begin{equation*}
    \vec F = \frac ec\mbmat{\eX & \eY & \eZ \\ \dot x & \dot y & \dot z \\ B & 0 & 0} = B\frac ec(\dot z\eX - \dot y\eZ)
  \end{equation*}
  Записываем второй закон Ньютона
  \begin{equation*}
    m(\ddot x\eX +\ddot y\eY + \ddot z\eZ) = B\frac ec(\dot z\eY - \dot y\eZ)
  \end{equation*}
  Покоординатно приравниваем, вводя обозначение $ ω  ≝ B\frac{e}{cm}$
  \begin{equation*}
    \begin{array}{ccc}
      \ddot x  = 0 & \ddot y = ω\dot z & \ddot z = -ω\dot y
    \end{array}
  \end{equation*}
  Решим систему на $y$,$z$, введя замену $ u = \dot y$, $v = \dot z$.
  Система переписывается как
  \begin{equation*}
    \bcase{\dot u = ωv\\\dot v = -ωu}
  \end{equation*}
  Продифференцировав первое уравнение и подставив второе, получаем уравнение на $u$:
  \begin{equation*}
    \ddot u + ω²u = 0
  \end{equation*}
  откуда, $u = a\cos ωt + b\sin ωt$, используя систему, $ v = -a\sin
  ωt + b\cos ωt $.  Возвращаемся к исходным переменным $x$,$y$,$z$
  (т.к $a$,$b$ -- произвольные констатны, то в них можно загнать $ω$
  при интегрировании)
  \begin{equation}
    \begin{array}{l}
      x = εt + β \\
      y = a\sin ωt - b\cos ωt + c_1 \\
      z = a\cos ωt + b\sin ωt +c_2
    \end{array}
  \end{equation}
  Подставим начальные условия. Из $\vecr = 0$ следует,
  что $β = 0$, $ c_1 = b$, $ c_2 = -a$. Тогда у нас остается три неизвестные:
  \begin{equation*}
    \begin{array}{l}
       x = εt\\
       y = a\sin ωt - b\cos ωt + b\\
       z = a\cos ωt + b\sin ωt -a
    \end{array}
  \end{equation*}
  Дифференцируем, подставляем $t = 0$ и используем начальную скорость
  \begin{equation*}
    \begin{array}{l}
      \dot x = ε\\
      \dot y = aω\cos ωt + bω\sin ωt\\
      \dot z = -aω\sin ωt + bω\cos ωt
    \end{array}
  \end{equation*}
  Откуда, $ε = ξ\cos α$, $aω = 0$, $bω = ξ\sin α$.
  Записываем ответ
  \begin{equation*}
    \begin{array}{ccc}
      x = ξ\cos α t & y = \frac{ξ\sin α}ω(1- \cos ωt)  & z = \frac{ξ\sin α}{ω}\cos ωt
    \end{array}
  \end{equation*}
  Получается занятный эффект -- частица описывает некую дугу в области
  $ z > 0$ и через время $ t= \frac{π}{2ω}$ вылетает из магнитного поля.
\end{solution}

\begin{tproblem}
  Рассчитать магнитное поле равномерно вращающейся сферы.
\end{tproblem}
\begin{solution}
  Обозначим в отсутсвии картинки угловую скорость $\vec Ω$. Тогда
  $\vec v = \vec Ω ×\vec r$, $\vec\jmath = ρ\vec v = σδ(r-R)\vec
  v$. Запишем известные уравнения на магнитное поле:

  \begin{displaymath}
    \begin{array}{cc}
      \Div\vecb = 0 & \rot\vecb = \frac{4π}c\vec\jmath
    \end{array}
  \end{displaymath}
  Введем вектор-потенциал $\veca: \Div\veca  = 0, \rot\veca = \vecb$. Тогда $Δ\veca = -\frac{4π}c\vec\jmath$.
  \questionably{ Значит, т.к} $\frac 1r$ -- функция Грина, $Δ\frac 1r = -4πδ(r)$,
  \newcommand{\hmr}{\hm{r-r'}}
  \newcommand{\Rreq}{{\vec r' = R\vec n}}
  \newcommand{\sRreq}{\evn\Rreq}
  \newcommand{\vecd}{\vec\nabla}
  \begin{eqnarray*}
    \veca(r) = \frac 1c∫\frac\jmath(r')\hmr d³x'
    = \frac 1c∫\frac{σδ(r'-R)[\vec Ω×\vec r']}\hmr d³x' = \hc{d³x' = dS'dr'} = \frac σc∫\frac{δ(r-r')[\vec Ω×\vec r']}\hmr dS'dr \bw= \\
    = \hc\Rreq = \frac σc∫_{S²}\frac{Ω×r'}{\hmr}\sRreq dS'
    = \frac{Rσ}c \vec Ω×∫_{S²}\frac{dS'}\hmr \sRreq = \text{\questionably{Почему??}}
    = -\frac{Rσ}c \vec Ω×\vecd ∫\frac{dV}\hmr
  \end{eqnarray*}
  \begin{petit}
    Ваш верстальщик ничего дальше не понял. Дабы не плодить в этом
    мире лажу, дальше смотрите в оригинале.
  \end{petit}
\end{solution}
\begin{tproblem}
  Найти вектор-потенциал магнитного поля, создаваемого прямолинейным тонким током.
\end{tproblem}
\unsafeIO
\begin{solution}
  Сначала найдем магнитное поле $\vecb$. Не ограничивая общности,
  можно считать, что плотность электрического тока равна $\vec\jmath =
  jδ(x)δ(y)\eZ$.  Из соображений симметричности, $\vecb = B_z\eZ +
  B_r\eR$. По теореме Стокса(стр. 23), взяв в качестве $S$ круг
  радиуса $r$ в плоскости $xy$
  \begin{equation*}
    ∮\limits_{S¹_r}(B_z\eZ + B_r\eR)dl =\frac{4π}c\intl{D²_r}{}jδ(x)δ(y)\eZ dS = \frac{4π}cj\eZ
  \end{equation*}
  откуда $B_r = 0$, $B_z = \frac{2j}{cr}$.  Т.е $\vecb = (0, 0,
  \frac{2j}{cr})$.  Вычислим поток магнитного поля через круг
  радиуса $r$ в плоскости $xy$.
  \begin{equation*}
    Φ ≡ \intl{D²_r}{}\vecb ·dS = \intl{S¹_r}{}\veca ·dl
  \end{equation*}
  Откуда заключаем, что $\veca = A(r)\eZ$. Вычислим интегралы.
  \begin{eqnarray*}
    \intl{D²_r}{}\frac{2j}{cr'}dS =\hc{\mat{ x = r'\sin φ\\y = r'\cos φ}} = \intl or\intl 0{2π} \frac{2j}c dr'dφ
  \end{eqnarray*}
  Бредово получается, что $\veca = \const$. \tbk
\end{solution}

\begin{tproblem}
  Найти магнитное поле постоянного тока плотности $j$, текущего по
  поверхности цилиндра радиуса $R$
  \begin{enumerate}
  \item паралельно оси
  \item вдоль окружностей.
  \end{enumerate}
\end{tproblem}
\unsafeIO{}
\begin{solution}
Заметим, что ввиду симметричности задачи, вектор магнитного поля имеет
вид $\vecb = B_r\vec e_r + B_z\vec e_z$. По формуле Стокса(стр. 23)  имеем
\begin{equation*}
   ∮_{S¹_r}(B_r\vec e_r + B_z\vec e_z)dl = \frac{4π}c\intl{D²_r}{}\vec\jmath dS
\end{equation*}
В первом случае, $\vec\jmath = jδ(r-R)\vec e_z$, откуда
\begin{equation*}
  \intl{D²_r}{}\vec\jmath dS = 2πj\vec e_zδ_{r≥R}
\end{equation*}
откуда заключаем,что
\begin{equation*}
   ∮_{S¹_r}(B_r\vec e_r + B_z\vec e_z)dl = \frac{4π}c 2πj\vec e_zδ_{r≥R}
\end{equation*}
что означает, что
\begin{equation*}
  \begin{array}{cc}
    B_r ≡ 0 &  2πrB_z =\frac{4π}c j\vec e_z2πRδ_{r≥R}
  \end{array}
\end{equation*}
  Упрощая, $B_z = \frac{4π}cj\frac Rrδ_{r≥R}$.  Во втором случае,
  вектор $\vec\jmath$ направлен по касательной к окружности, и в
  правой части уравнения будет $0$ вне зависимости от радиуса круга $r$.
  Потому во втором случае магнитное поле равно 0.
\end{solution}

\begin{tproblem}
  Найти связь между средним магнитным полем в круге и значением поля
  на окружности в бетатроне (ускорителе электронов, в котором
  ускоряющее электрическое поле создается благодаря электро-магнитной
  индукции неоднородным, зависящим от времени магнитным полем
  $B_z(ρ,t)$, $ρ² = x² + y²$, причем электроны ускоряются, оставаясь на окружности постоянного радиуса)
\end{tproblem}
\begin{solution}
  \newcommand{\vecec}{\vec\Ec}
  \newcommand{\δ}{\,d²\!}
  \newcommand{\Α}[1]{\!<\!\!#1\!\!>}
  По классическим определениям импульса и энергии,
  \begin{equation*}
    \begin{array}{c}
      \frac{d\vec p}{dt} = e\hr{\vece + \frac 1c[\vec v×\vecb]}\\
      \frac{d\vecec}{dt} = e(\vece, \vec v) = eEv
    \end{array}
  \end{equation*}
    Запишем закон индукции Фарадея ($D_2$ -- это круг)
    \begin{equation*}
      \begin{array}{c}
       \displaystyle  \intl {D²}{}\rot\vece d²x = ∮\limits_{S¹}\vece ds = 2πRE\\
       \displaystyle \intl {D²}{}\rot\vece\δ x  = -\frac 1c\frac{∂}{∂t}\intl{D²}{}\vecb \δ x = -\frac 1c πR²\frac{∂}{∂t}\Α{B}
      \end{array}
    \end{equation*}
    где $\Α{B}$ мы обозначили среднее магнитное поле в круге. Совмещая два равенства, получаем,
    \begin{equation*}
      E = -\frac{πR²}{2πRc}\frac{∂}{∂t}\Α{B} = -\frac{1}{2c}R\frac d{dt}\Α{B}.
    \end{equation*}
  Также нам понадобится равенство неясной природы
  \begin{equation*}
    pΩ = \hm{\hs{p×Ω}} = f_ν = \frac ec |[\vec v, \vecb\evn{R}]| = \frac ec VB_R
  \end{equation*}
  где $B_R$ -- среднее значение поля на окружности.

  Подставим полученное выражение для $E$ в имеющиеся формулы:
  \begin{equation*}
    \begin{array}{c}
      \displaystyle pΩ = \frac ec VB_R\\
      \displaystyle eE = \frac{dp}{dt}\frac 1R =\frac ec\dot B_R
    \end{array}
  \end{equation*}
  Приравнивая, получаем, что
  \begin{equation*}
    \begin{array}{ccc}
      \frac {R\dot B_R}c = \frac 1{2c}R\frac{d}{dt}\Α{B} & ⇒ &
      B_R = \frac 12\Α{B}
    \end{array}
  \end{equation*}
\end{solution}
\begin{tproblem}
  Записать преобразование Лоренца в форме $SL(2,\Cbb)$.
\end{tproblem}

\begin{tproblem}
  Показать, что композиция двух преобразований Лоренца с
  непаралельными скоростями не является чистым преобразованием Лоренца.
\end{tproblem}

\begin{tproblem}
  \newcommand{\ω}{ω\hr{t - z/c}}
  С помощью метода Гамильтона-Якоби найти закон движения и траекторию
  релятивистского заряда в поле плоской волны круговой поляризации,
  описываемой потенциалом
  \begin{equation*}
    \veca = A_0 (\cos\hr\ω, \sin\hr\ω, 0)
  \end{equation*}
\end{tproblem}

\begin{tproblem}
  Найти время жизни электрона, движущегося по круговой орбите в атоме
  водорода, при учете излучения. То же для заряда, вращающегося в
  однородном магнитном поле.
\end{tproblem}
\begin{solution}
  Полагаем, что излучение мало. Тогда $\frac{d\Ec}{dt} = -I(R)$, где
  $I$ -- интенсивность излучения.

  Уравнение движения электрона $m\ddot{\vec r} = -\frac{e²\vec r}{r³}$
  -- центральное поле.  \questionably{Существует} круговая орбита, что
  $\vec a = -a\vec n$, $v = \const$.  Далее, $a = v²/r$, второй закон
  Ньютона $\hm{\vec F_\text{цент}} =\hm{\frac{mv²}r} = \frac{e²}{r²}$.
  \questionably{Полная энергия частицы}: $\Ec = \frac{mv²}2 - \frac{e²}r =
  -\frac{e²}{2r}$. Подставим в уравнение излучения:
  \begin{petit}
    Подходящую формулу я не нашел. Впрочем, я почти верю.
  \end{petit}
  \begin{equation*}
    \frac{e²\dot r}{2r²} = -\frac 23 e²a² = -\frac 23 \frac{e^6}{m²r^4}
  \end{equation*}
  Отсюда, $a²\dot r = -\frac 43ρ²$, $ρ ≝ e²/m$.
  Интегрируя, получаем систему
  \begin{equation*}
    \begin{array}{cc}
      \displaystyle\bcase{r³ = -4ρ²t + C\\ r(0) = r_0} &
      \displaystyle\bcase{r = \sqrt[3]{-4ρ²t + r³_0}}
    \end{array}
  \end{equation*}
  Конец жизни электрона означает, что $r = 0$. Отсюда $t = \frac{r³_0}{4ρ^2}$.
  \questionably{При учете того, что $T=t/c =\frac{r³_0}{4ρ^2c}$} получаем
  для $ ρ≈ 10^{-13}cm$, $r_0 ≈10^{-8}cm$, $c ≈3·10^{10}cm$ имеем $T≈10^{-9}cm$.

  Пункт про магнитное поле отстутствует. \tbk
\end{solution}

\begin{tproblem}
  Показать, что радиальные геодезические в поле Шварцшильда
  приближаются горизонту событий за бесконечное время по часам
  удаленного наблюдателя. Каково время подения по собственным часам?
\end{tproblem}

\begin{tproblem}
  Рассчитать параметры всеъ круговых орбит массивных и безмассовых
  частиц в поле Шварцшильда.
\end{tproblem}

\begin{tproblem}
   Дать классификацию экваториальны изотропных геодезических в поле
   Шварцшильда, приходящих из бесконечности.
\end{tproblem}

\end{document}
