\begin{tproblem}
  Найти вектор-потенциал магнитного поля, создаваемого прямолинейным тонким током.
\end{tproblem}
\unsafeIO{}
\begin{solution}
  Сначала найдем магнитное поле $\vecb$. Не ограничивая общности,
  можно считать, что плотность электрического тока равна $\vec\jmath =
  jδ(x)δ(y)\eZ$.  Из соображений симметричности, $\vecb = B_z\eZ +
  B_r\eR$. По теореме Стокса(стр. 23), взяв в качестве $S$ круг
  радиуса $r$ в плоскости $xy$
  \begin{equation*}
    ∮\limits_{S¹_r}(B_z\eZ + B_r\eR)dl =\frac{4π}c\intl{D²_r}{}jδ(x)δ(y)\eZ dS = \frac{4π}cj\eZ
  \end{equation*}
  откуда $B_r = 0$, $B_z = \frac{2j}{cr}$.  Т.е $\vecb = (0, 0,
  \frac{2j}{cr})$.  Вычислим поток магнитного поля через круг
  радиуса $r$ в плоскости $xy$.
  \begin{equation*}
    Φ ≡ \intl{D²_r}{}\vecb ·dS = \intl{S¹_r}{}\veca ·dl
  \end{equation*}
  Откуда заключаем, что $\veca = A(r)\eZ$. Вычислим интегралы.
  \begin{eqnarray*}
    \intl{D²_r}{}\frac{2j}{cr'}dS =\hc{\mat{ x = r'\sin φ\\y = r'\cos φ}} = \intl or\intl 0{2π} \frac{2j}c dr'dφ
  \end{eqnarray*}
  Бредово получается, что $\veca = \const$. \tbk
\end{solution}
