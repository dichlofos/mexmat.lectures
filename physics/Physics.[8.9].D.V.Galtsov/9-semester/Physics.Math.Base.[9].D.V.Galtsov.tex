%\documentclass[a4paper,draft,openany]{book}
\documentclass[a4paper
%,12pt
]{article}
\usepackage[cp1251]{inputenc}
\usepackage[russian]{babel}
\usepackage{amsfonts, amssymb, amsmath, amsthm, bm}
%\usepackage{bibunits}

\textheight260mm \textwidth170mm \hoffset-20mm\voffset-30mm
%\topmargin=-1in\oddsidemargin=-0.5in \evensidemargin=-0.5in


\date{версия 29.08.06.}
\begin{document}
\titlepage
\begin{center} {\bf \Huge Математические дополнения \vspace{.5cm}к курсу квантовой механики}
%\author{А.А. Васильева \and  Д.В. Гальцов}
\vskip2cm {\bf\Large А.\,А.\,Васильева \vskip.7cm  Д.\,В.\,Гальцов}
\end{center}\vskip2cm
\begin{abstract}
В этом тексте собраны основные математические утверждения,
используемые в квантовой механике, которые необходимы для более
глубокого понимания ее математического аппарата. В нем можно найти
сведения из теории неограниченных операторов в гильбертовом
пространстве, строгие решения некоторых стандартных задач, а также
ссылки на математическую литературу для дальнейшего изучения.
Изложение следует курсу, читаемому одним из авторов на
механико-математическом факультете МГУ (Д.\,В.\,Гальцов,
Теоретическая физика для студентов-математиков, ч. 2; М.:МГУ,
2003).
\end{abstract} \vskip5cm
\noindent\copyright \vspace{5mm} А.\,А.\,Васильева\\
\noindent\copyright \vspace{5mm} Д.\,В.\,Гальцов

\begin{center}{\bf \large Москва, 2006}  \end{center}
%\maketitle
\newpage
\tableofcontents \newpage
\newcommand{\R}{\mathbb{R}}
\renewcommand{\C}{\mathbb{C}}
\newcommand{\Z}{\mathbb{Z}}
\newcommand{\N}{\mathbb{N}}
\newcommand{\Q}{\mathbb{Q}}
\newcommand{\LL}{\cal{L}}
\newcommand{\M}{{\cal M}}
\newcommand{\G}{{\cal G}}
\newcommand{\K}{\cal{K}}
\newcommand{\HH}{\cal{H}}
\newcommand{\non}{\nonumber}
%\renewcommand{\thebibliography}{Список литературы}
\renewcommand{\le}{\leqslant}
\renewcommand{\ge}{\geqslant}
\renewcommand{\tg}{{\rm tg}}
\renewcommand{\ctg}{{\rm ctg}}
\renewcommand{\sh}{{\rm sh}}
\newcommand{\const}{{\rm const}}
\renewcommand{\Im}{{\rm Im}\,}
\renewcommand{\th}{{\rm th}}
\newcommand{\ran}{{\rm ran}}
\newcommand{\rot}{{\rm rot}}
\newcommand{\supp}{{\rm supp}\,}
\newcommand{\Tr}{{\rm Tr}\,}
\renewcommand{\div}{{\rm div}}
\theoremstyle{plain}
\newtheorem{Trm}{Теорема}[section]
\newtheorem{Def}{Определение}[section]
\newtheorem{Cor}{Следствие}[section]
\newtheorem{Lem}{Лемма}[section]

\newtheorem{Rem}{Замечание}[section]
\newtheorem{Sta}{Утверждение}[section]
\renewcommand{\proofname}{\bf Доказательство}
\numberwithin{equation}{section}
\section{Самосопряженные операторы в гильбертовом пространстве
и связь с квантовой механикой.}
В этой главе вводится понятие самосопряженного оператора, приводятся
условия самосопряженности, способы построения самосопряженных
операторов и две формы спектральной теоремы: в терминах
проекторнозначной меры и в терминах обобщенного преобразования
Фурье. Формулируются основные принципы квантовой механики (в
соответствии с [23]). Вводится понятие обобщенных собственных
векторов, с помощью которых осуществляется обобщенное преобразование
Фурье.
\subsection{Неограниченные операторы в гильбертовом пространстве}
Всюду через ${\cal H}$ обозначается сепарабельное бесконечномерное
гильбертово пространство над полем $\C$ со скалярным произведением
$\langle \cdot, \cdot\rangle$, линейным по правому аргументу.\footnote{Иногда будут
использоваться дираковские обозначения.}
\begin{Def}
Пусть $D(T)\subset {\cal H}$ --- всюду плотное линейное подмногообразие.
Оператором на гильбертовом пространстве ${\cal H}$ называется
линейное отображение $T:D(T)\rightarrow {\cal H}$. Множество $D(T)$
называется областью определения оператора $T$.
\end{Def}
{\bf Пример.} Пусть ${\cal H}=L_2(I)$, где $I\subset \R$ ---
промежуток. Рассмотрим оператор $Tf(x)=f'(x)$ с областью определения
$D(T)=C_0^\infty (I)$. Тогда оператор $T$ является неограниченным.
В самом деле, пусть $[a, \, b]\subset I$, $a<\tilde a<\tilde b<b$.
Пусть $\eta$ --- бесконечно гладкая функция, $\eta|_{I\backslash [a, \, b]}
\equiv 0$, $\eta|_{[\tilde a, \, \tilde b]}\equiv 1$. Положим
$\varphi_n(x)=\eta(x)\sin nx$. Тогда последовательность $\{\|\varphi_n\|
_{L_2(I)}\}$ ограничена, а $\{\|\varphi_n'\|_{L_2(I)}\}$ неограничена.
\par
Если оператор ограничен, то его можно однозначно продолжить по
непрерывности на все гильбертово пространство. Если оператор
неограниченный, то область определения выбирается, вообще говоря,
неоднозначно. В частности, в предыдущем примере в качестве $D(T)$
можно также взять $W^1_2(I)$ или $\mathaccent'27 W^1_2(I)$. В связи
с этим возникают понятия расширения и замыкания оператора.
\begin{Def}
Пусть $T_1$ и $T$ --- операторы в ${\cal H}$. Оператор $T_1$ называется
расширением оператора $T$, если $D(T_1)\supset D(T)$ и $T_1|_{D(T)}=T$.
\end{Def}
Графиком $\Gamma(T)$ линейного оператора $T$ называется множество
пар $$\{(\varphi, \, T\varphi)| \, \varphi\in D(T)\}\subset {\cal H}\times
{\cal H}.$$
В качестве скалярного произведения в ${\cal H}\times {\cal H}$ берется
$$\left\langle(\varphi_1, \, \psi_1),\, (\varphi_2, \, \psi_2)\right\rangle
=\langle \varphi_1, \, \varphi_2\rangle+\langle \psi_1, \, \psi_2\rangle.$$
Оператор $T$ называется {\it замкнутым}, если $\Gamma(T)$ --- замкнутое
подмножество в ${\cal H}\times{\cal H}$. Оператор $T$ {\it замыкаем},
если он имеет замкнутое расширение. Каждый замыкаемый оператор имеет
наименьшее замкнутое расширение, называемое {\it замыканием} и
обозначаемое $\overline{T}$. \par
Заметим, что если оператор ограничен, то его замыкание совпадает с
продолжением по непрерывности на все гильбертово пространство. \par
Найдем замыкание оператора дифференцирования в ${\cal H}=L_2[a, \, b]$,
с областью определения $D(T)=C_0^\infty(a, \, b)$. Утверждается,
что $\overline{T}$ --- это оператор дифференцирования на множестве
$D(\overline{T})=\mathaccent'27 W^1_2([a, \, b])$. В самом деле, если
$f_n\rightarrow f$ и $f_n'\rightarrow g(n\rightarrow \infty)$ в метрике
пространства $L_2[a, \, b]$, то последовательность $\{f_n\}$ фундаментальна
в метрике пространства $W^1_2([a, \, b])$. В силу полноты пространства
Соболева $f_n\rightarrow f$ в метрике $W^1_2([a, \,b])$ при $n\rightarrow \infty$.
Так как $\mathaccent'27 W^1_2([a, \, b])$ --- это замыкание
$C_0^\infty(a, \, b)$ в пространстве $W^1_2([a, \, b])$, то $f\in
\mathaccent'27 W^1_2([a, \, b])$. Обратно, если $f\in \mathaccent'27
W^1_2([a, \, b])$, то существует последовательность $f_n\in C_0^\infty
(a, \, b)$, сходящаяся к $f$ в $W^1_2([a, \, b])$, то есть $f_n
\rightarrow f$ и $f_n'\rightarrow f'$ в $L_2([a, \, b])$ и, значит,
$(f, \, f')$ принадлежит замыканию графика оператора $T$.
\begin{Def}
Пусть $T$ --- оператор в гильбертовом пространстве ${\cal H}$. Областью
определения оператора $T^+$, сопряженного к $T$, называется множество
$D(T^+)$, состоящее из таких векторов $\varphi\in {\cal H}$, для
которых функционал $l_\varphi(\psi)=\langle \varphi, \, T\psi\rangle$,
$\psi\in D(T)$, является ограниченным. Оператор $T^+$, называемый
сопряженным к $T$, определяется с помощью равенства $$\langle T^+\varphi,
\, \psi\rangle=l_\varphi(\psi)=\langle \varphi, \, T\psi\rangle,$$
где $\varphi\in D(T^+)$, $\psi\in D(T)$.
\end{Def}
Так как каждый линейный непрерывный функционал $l(\cdot)$ в гильбертовом
пространстве имеет вид $l(\cdot)=\langle\eta, \, \cdot\rangle$,
где $\eta\in {\cal H}$, то для каждого $\varphi\in D(T^+)$ найдется
вектор $\eta_\varphi$ такой, что $\langle \varphi, \, T\psi\rangle=
\langle \eta_\varphi, \, \psi\rangle$, $\psi\in D(T)$. Значит,
$T^+\varphi=\eta_\varphi$. \par
{\bf Пример.} Пусть ${\cal H}=L_2[a, \, b]$, $D(T)=C_0^\infty([a, \, b])$,
$Tf(x)=if'(x)$. Докажем, что $D(T^+)=W^1_2([a, \, b])$ и $T^+f=if'$.
Действительно, пусть функция $g\in L_2[a, \, b]$ такова, что существует
функция $\eta\in L_2[a, \, b]$ такая, что для любого $f\in C_0^\infty([a, \, b])$
выполнено $\langle g, \, if'\rangle=\langle \eta, \, f\rangle$. Значит,
$\langle ig, \, f'\rangle=-\langle \eta, \, f\rangle$. По определению
пространства Соболева, $g\in W^1_2([a, \, b])$ и $\eta=ig'$ (производная
в обобщенном смысле). Так как на отрезке все функции из $W^1_2$
абсолютно непрерывны и обобщенная производная совпадает с производной,
взятой почти всюду, то $T^+g=\eta=ig'$ почти всюду. \par
Если в качестве области определения оператора $i\frac{d}{dx}$ взять
$\mathaccent'27 W^1_2([a, \, b])$, то область определения сопряженного
оператора содержится в $W^1_2([a, \, b])$. Интегрируя по частям,
можно показать, что она в точности совпадает с $W^1_2([a, \, b])$. \par
В общем случае область определения сопряженного оператора может быть не
всюду плотной и даже может состоять из единственного нулевого вектора
(см. [3], гл. VIII). \par
Введем понятие спектра оператора.
\begin{Def}
Пусть $\, T$ --- замкнутый оператор. Комплексное число $\lambda$
принадлежит резольвентному множеству $\rho(T)$, если оператор
$T-\lambda I$ является биекцией $D(T)$ на ${\cal H}$ с ограниченным
обратным. Оператор $(T-\lambda I)^{-1}$ называется резольвентой оператора $T$.
\end{Def}
Так же, как для случая ограниченных операторов, показывается, что
$\rho(T)$ открыто в $\C$. Дополнение к резольвентному множеству
называется {\it спектром} и обозначается $\sigma(T)$.
\begin{Def}
Оператор $\; T$ называется симметрическим (или эрмитовым), если
$T\subset T^+$. Равносильное условие: $T$ симметричен тогда и
только тогда, когда $$\langle T\varphi, \, \psi\rangle=\langle\varphi, \,
T\psi\rangle$$ для всех $\varphi$, $\psi\in D(T)$.
\end{Def}
Известно, что если оператор является симметрическим и ограниченным,
то его спектр содержится в вещественной прямой. Оказывается, что в
случае неограниченных операторов условия симметричности для этого
недостаточно.
\begin{Def}
Оператор $\; T$ называется самосопряженным, если $T=T^+$, то есть
$T$ симметрический и $D(T)=D(T^+)$. Оператор $\; T$ называется
существенно самосопряженным, если его замыкание является
самосопряженным.
\end{Def}
Как видно из предыдущего примера, оператор $i\frac{d}{dx}$ на $\mathaccent'27
W^1_2([a, \, b])$ является симметрическим, но не самосопряженным, а
тот же оператор на $C_0^\infty([a, \, b])$, следовательно, не является
существенно самосопряженным. \par
\begin{Trm} {\rm [3, т. 2, \S X.1]}
Пусть $T$ --- замкнутый симметрический оператор. Тогда для того, чтобы
$T$ был самосопряженным, необходимо и достаточно, чтобы $\sigma(T)
\subset \R$.
\end{Trm}
Можно показать, что если $T$ существенно самосопряжен, то его единственным
самосопряженным расширением является его замыкание. Отсюда видно, что вместо
самосопряженных операторов достаточно рассматривать существенно
самосопряженные, у которых область определения обычно легче задать. \par
Следующая теорема [3, \S X.1] дает критерий того, что
симметрический оператор имеет самосопряженные расширения и того, что
он является существенно самосопряженным. Сначала введем понятие
индексов дефекта.
\begin{Def}
Пусть $T$ --- симметрический оператор и
$${\cal K_+}=\ker(i-T^+)=\ran(i+T)^{\bot},$$
$${\cal K_-}=\ker(i+T^+)=\ran(-i+T)^{\bot},$$
где $\ran$ обозначает образ оператора. Множества ${\cal K_{\pm}}$
называются дефектными подпространствами оператора $T$. Числа $n_{\pm}=
\dim {\cal K_{\pm}}$ называются индексами дефекта оператора $T$.
\end{Def}
\begin{Trm}
\label{krit_samosopr}
Пусть $T$ --- симметрический оператор с индексами дефекта $n_+$ и
$n_-$. Тогда \par
1) $T$ существенно самосопряжен тогда и только тогда, когда
$n_+=n_-=0$. \par
2) $T$ обладает самосопряженными расширениями тогда и только тогда,
когда $n_+=n_-$. Существует взаимно однозначное соответствие между
самосопряженными расширениями оператора $T$ и унитарными отображениями
из ${\cal K_+}$ на ${\cal K_-}$.
\end{Trm}
{\bf Пример.} Пусть ${\cal H}=L_2(I)$, где $I\subset \R$ --- некоторый
промежуток, $D(T)=C_0^\infty(I)$, $T=i\frac{d}{dx}$. Найдем индексы
дефекта оператора $T$. Пусть $f\in \ran (i+T)^\bot$. Тогда для любого
$g\in D(T)$ выполнено $\langle f, \, ig'+ig\rangle=0$, то есть
$f'-f=0$ в смысле обобщенных функций. Решив это уравнение, получаем
$f(x)=ce^x$. Если промежуток $I$ ограничен сверху, то $f\in L_2(I)$ и $n_+=1$,
в обратном случае $f\notin L_2(I)$ и $n_+=0$. Аналогично находится
$n_-$. Применив теорему \ref{krit_samosopr}, получаем три случая:
\begin{enumerate}
\item если $I=\R$, то $n_+=n_-=0$ и $T$ существенно самосопряжен;
\item если $I$ полуограничен, то $n_+\ne n_-$ и $T$ не имеет самосопряженных
расширений;
\item если $I$ ограничен, то $n_+=n_-=1$, так что $T$ не является
существенно самосопряженным, но имеет самосопряженные расширения.
\end{enumerate}
\par
Из теоремы \ref{krit_samosopr} вытекает следующее утверждение.
\begin{Cor}
\label{cor_sum_oper}
Пусть ${\cal H}=\oplus _{n=1}^{\infty} {\cal H}_n$, $T_n:{\cal H}_n
\rightarrow {\cal H}_n$ --- существенно самосопряженные операторы.
Положим $T|_{{\cal H}_n}=T_n$ и продолжим его по линейности.
Тогда $T$ --- существенно самосопряженный оператор.
\end{Cor}
\begin{proof}
Легко проверяется, что этот оператор симметрический. Пусть $x\in
\ran (i+T)^{\bot}$, $x_n$ --- ортогональная проекция $x$ на ${\cal H}_n$.
Тогда $x_n\in \ran (i+T_n)^{\bot}$, так что $x_n=0$ в силу существенной
самосопряженности $T_n$. Значит, $x=\sum \limits_{n=1}^\infty x_n=0$,
откуда получаем $n_+=0$. Аналогично доказывается, что $n_-=0$.
\end{proof}
Сумма двух существенно самосопряженных операторов, вообще говоря, не
является существенно самосопряженным оператором (потом будут приведены
примеры). Следующая теорема (см. [3], \S X.2) показывает, что ``достаточно
малое'' возмущение самосопряженного оператора будет существенно
самосопряженным.
\begin{Trm}
\label{vust}
(теорема Вюста--Като--Реллиха). Пусть $A$ самосопряжен, а $B$ симметричен,
причем $D(B)\supset D(A)$. Предположим, что для некоторого $b$ и всех
$\varphi\in D(A)$ $$\|B\varphi\|\le \lambda\|A\varphi\|+b\|\varphi\|,$$
где $\lambda\le 1$. Тогда $A+B$ существенно самосопряжен на $D(A)$ или
любой области существенной самосопряженности оператора $A$. Если $\lambda<1$,
то $A+B$ самосопряжен на $D(A)$.
\end{Trm}
В частности, если $B$ ограничен, то $A+B$ существенно самосопряжен. \par
\begin{Def}
Пусть $D_1$, $D_2\subset {\cal H}$ --- всюду плотные линейные подпространства,
$A:D_1\rightarrow D_2$, $B:D_2\rightarrow D_1$ --- линейные операторы.
Коммутатором $A$ и $B$ называется оператор $[A, \, B]=AB-BA$ с областью
определения $D_1\cap D_2$.
\end{Def}
Если $A$ и $B$ симметрические, то легко проверить, что $i[A, \, B]$
тоже симметрический. Однако из того, что $A$, $B$ являются существенно
самосопряженными, не следует, что оператор $i[A, \, B]$ существенно
самосопряженный; он может даже не иметь самосопряженных расширений. \par
{\bf Пример.} Пусть $D_1=D_2=C_0^\infty(0, \, +\infty)$,
$A=x$, $B=-\frac{d^2}{dx^2}+\frac{1}{x^2}$. Тогда $A$ и $B$ являются
существенно самосопряженными (для оператора $B$ это доказывается с
помощью критерия Вейля, который будет приведен в главе \ref{1dimension}).
Однако оператор $i[A, \, B]=2i\frac{d}{dx}$ не имеет самосопряженных
расширений на $C_0^\infty(0, \, +\infty)$. \par
\subsection{Проекторнозначные меры и спектральная теорема}
В этом параграфе приводится обобщение следующего факта: самосопряженный
оператор $A$ в конечномерном евклидовом пространстве представим в виде
$A=\sum \limits _k\lambda_kP_k$, где $\lambda_k$ --- собственные
значения $A$, а $P_k$ --- ортогональные проекторы на подпространство
собственных векторов, соответствующих $\lambda_k$. \par
Напомним, что оператор $A$ является сильным пределом последовательности
операторов $\{A_n\}$ (это обозначается $A=s$-$\lim \limits_{n\rightarrow
\infty}A_n$), если для любого $\varphi\in {\cal H}$ выполнено
$\lim \limits _{n\rightarrow \infty}A_n\varphi=A\varphi$.
\begin{Def}
\label{def_pr_meas}
Пусть каждому борелевскому подмножеству $\Omega$ вещественной прямой сопоставлен
оператор $P_{\Omega}$ и семейство $\{P_{\Omega}\}$ обладает следующими
свойствами: \par
(a) для любого $\Omega$ оператор $P_\Omega$ является ортогональным проектором; \par
(b) $P_{\emptyset}=0$, $P_{(-\infty, \, +\infty)}=I$; \par
(c) если $\Omega=\sqcup _{n=1}^\infty \Omega _n$, то $P_\Omega
=s$-$\lim \limits_{N\rightarrow \infty}
\sum \limits_{n=1}^N P_{\Omega_n}$; \par
(d) $P_{\Omega_1}P_{\Omega_2}=P_{\Omega_1\cap\Omega_2}$. \par
Тогда семейство $\{P_{\Omega}\}$ называется проекторнозначной мерой.
\end{Def}
Для каждого $\psi\in {\cal H}$ рассмотрим функцию множества $p_\psi
(\Omega)=\langle \psi, \, P_\Omega \psi\rangle$.
\begin{Sta}
\label{proekt_measure}
Пусть $\{P_\Omega\}$ --- семейство ортогональных проекторов в гильбертовом
пространстве, где $\Omega\subset \R$ --- борелевские множества. Тогда
для того, чтобы оно являлось проекторнозначной мерой, необходимо и
достаточно, чтобы для любого $\psi\in {\cal H}$ функция $p_\psi(\Omega)$
являлась вероятностной мерой на $\R$.
\end{Sta}
\begin{proof}
Докажем необходимость. Из свойства (a) следует, что $p_\psi(\Omega)\ge 0$
для любого $\Omega$, из свойства (b) --- что $p_\psi(\emptyset)=0$
и $p_\psi(\R)=1$. Счетная аддитивность вытекает из свойства (c). \par
Докажем достаточность. Свойство (a) выполняется автоматически. Так как
$p_\psi(\emptyset)=0$ и $p_\psi(\R)=1$ для любого $\psi\in {\cal H}$,
то выполнено свойство (b). \par
Пусть $\Omega_1\cap \Omega_2=\emptyset$. Докажем, что подпространства
$P_{\Omega_1}{\cal H}$ и $P_{\Omega_2}{\cal H}$ ортогональны.
В самом деле, пусть $\psi\in P_{\Omega_1}{\cal H}$. Тогда $p_\psi(\Omega_1)
=1$, и из свойств вероятностной меры следует, что $p_\psi(\Omega_2)=0$,
то есть $\langle \psi, \, P_{\Omega_2}\psi\rangle=0$. Следовательно,
вектор $\psi$ ортогонален подпространству $P_{\Omega_2}{\cal H}$. \par
Докажем, что $P_{\Omega_1\sqcup\Omega_2}=P_{\Omega_1}+P_{\Omega_2}$.
Из аддитивности вероятностной меры следует, что $\langle \psi, \,
P_{\Omega_1\sqcup \Omega_2}\psi\rangle=\langle \psi, \, P_{\Omega_1}\psi
\rangle+\langle \psi, \, P_{\Omega_2}\psi\rangle$. Поэтому если $\psi\in
P_{\Omega_j}{\cal H}$, то $\langle \psi, \, P_{\Omega_1\sqcup \Omega_2}\psi
\rangle=1$, а если $\psi\in (P_{\Omega_1}{\cal H}+P_{\Omega_2}{\cal H})
^{\bot}$, то $\langle \psi, \, P_{\Omega_1\sqcup \Omega_2}\psi\rangle=0$.
Значит, $P_{\Omega_1\sqcup \Omega_2}=P_{\Omega_1}+P_{\Omega_2}$. \par
Докажем свойство (d). Пусть $\Omega_1$ и $\Omega_2$ --- произвольные
борелевские множества. Тогда $$P_{\Omega_1}P_{\Omega_2}=(P_{\Omega_1
\cap \Omega_2}+P_{\Omega_1\backslash \Omega_2})(P_{\Omega_1\cap \Omega_2}
+P_{\Omega_2\backslash \Omega_1})=$$ $$=P_{\Omega_1\cap \Omega_2}^2+
P_{\Omega_1\cap \Omega_2}P_{\Omega_2\backslash \Omega_1}+P_{\Omega_1
\backslash \Omega_2}P_{\Omega_1\cap \Omega_2}+P_{\Omega_1\backslash
\Omega_2}P_{\Omega_2\backslash \Omega_1}=P_{\Omega_1\cap \Omega_2}.$$ \par
Докажем (c). Пусть $\Omega=\sqcup_{n=1}^\infty \Omega_n$. Из счетной
аддитивности вероятности следует, что $\langle\psi, \, P_{\Omega}\psi
\rangle=\sum \limits_{n=1}^\infty \langle\psi, \, P_{\Omega_n}\psi\rangle$.
Пусть $P=P_{\Omega}$, $P_n=\sum \limits_{k=1}^n P_{\Omega_k}$. Тогда
$\langle\psi, \, (P-P_n)\psi\rangle\rightarrow 0$, $n\rightarrow\infty$.
Из свойства (d) следует, что $P^2=P$, $P_n^2=PP_n=P_nP=P_n$.
Значит, $$\langle (P-P_n)\psi, \, (P-P_n)\psi\rangle=\langle\psi, \, (P-P_n)^2
\psi\rangle=\langle\psi, \, (P^2-PP_n-P_nP+P_n^2)\psi\rangle=$$
$$=\langle\psi, \, (P-P_n)\psi\rangle \underset{n\rightarrow\infty}{\rightarrow} 0,$$ то
есть $P_n$ сильно сходится к $P$.
\end{proof}
Определим интеграл по мере $\{P_\Omega\}$. Пусть $f:\R\rightarrow \R$
--- локально ограниченная борелевская функция. Рассмотрим
квадратичную форму $$q(\psi)=\int f(\lambda)\, dp_\psi(\lambda),$$
где $\psi\in P_{\Omega}{\cal H}$ для некоторого конечного интервала
$\Omega$ (тогда интеграл в правой части существует). Пусть $b$ ---
билинейная форма и $b(\psi, \, \psi)=q(\psi)$. Тогда можно показать, что
для любого конечного интервала $\Omega$ найдется такая константа $C$,
что для любых функций $\varphi$, $\psi\in P_{\Omega}{\cal H}$ выполнено
неравенство $b(\varphi, \, \psi)\le C\|\varphi\|\|\psi\|$. Значит,
на $P_{\Omega}{\cal H}$ определен ограниченный эрмитов оператор
$B_\Omega$ такой, что $\langle \varphi, \, B_\Omega\psi\rangle=b(\varphi, \, \psi)$.
Определим оператор $B$ равенством $B|_{P_\Omega
{\cal H}}=B_\Omega$. Из следствия \ref{cor_sum_oper} получаем, что
$B$ существенно самосопряжен. Его замыкание обозначим символом
$$\int _{-\infty}^{+\infty}f(\lambda)\, dP_\lambda.$$
Если $f\equiv 1$, то $q(\psi)=\langle \psi, \, \psi\rangle$, так что
выполнено равенство
\begin{align}
\label{razl_edinitsy}
I=\int \limits _{-\infty}^{+\infty}dP_\lambda.
\end{align}
Теперь сформулируем спектральную теорему.
\begin{Trm} {\rm [3, \S VIII.3]}
Существует взаимно-однозначное соответствие между самосопряженными
операторами $A$ и проекторнозначными мерами $\{P_\Omega^A\}$ на
${\cal H}$, задаваемое равенством
\begin{align}
\label{sp_teor1}
A=\int \limits _{-\infty}^{+\infty}\lambda \, dP^A_\lambda.
\end{align}
\end{Trm}
При этом $$A^n=\int \limits _{-\infty}^{+\infty}\lambda^n dP^A_\lambda.$$
Проекторнозначная мера, удовлетворяющая (\ref{sp_teor1}), называется
разложением единицы для оператора $A$. \par
С помощью этой теоремы можно определить функцию от оператора формулой
$$f(A)=\int \limits _{-\infty}^{+\infty}f(\lambda)\, dP^A_\lambda.$$
В частности, так определяется экспонента $e^{itA}$. \par
{\bf Пример.} Пусть ${\cal H}=L_2(\R)$. Обозначим через $\chi_\Omega$
характеристическую функцию множества $\Omega$. Тогда операторы
$(P_\Omega\psi)(\lambda)=\chi_\Omega(\lambda)\psi(\lambda)$ задают
проекторнозначную меру (это следует из утверждения \ref{proekt_measure}).
Так как $$p_\psi(\Omega)=\langle \psi, \, P_\Omega\psi\rangle =\int \limits _\Omega
|\psi(\lambda)| ^2\, d\lambda,$$ то $$\int f(\lambda)\, dp_\psi(\lambda)=\int f(\lambda)
|\psi(\lambda)| ^2\, d\lambda=\langle \psi(\cdot), \, f(\cdot)\psi(\cdot)\rangle.$$
В частности, оператор $A=\int \lambda\, dP_\lambda$ имеет вид $(A\psi)
(\lambda)=\lambda\psi(\lambda)$.
\subsection{Основные принципы квантовой механики}
\label{principles}
Принципы квантовой механики формулируются следующим образом.
\begin{itemize}
\item Состояние задается единичным вектором в бесконечномерном сепарабельном
пространстве. \par
Изначально состояния задавались волновыми функциями $\psi(x)$, которым
придавалась следующая вероятностная интерпретация: $|\psi(x)|^2$ ---
плотность вероятности обнаружить частицу в точке $x$. Позже был
сформулирован принцип суперпозиции (см., напр., [13]), откуда следовало,
что состояния --- это векторы в гильбертовом пространстве.
\item Каждой наблюдаемой $A$ соответствует самосопряженный оператор
$\hat A$, при этом для любого борелевского подмножества $\Omega\subset
\R$ и для любого состояния $\psi\in {\cal H}$ вероятность того, что значение наблюдаемой $A$
принадлежит $\Omega$, равна
\begin{align}
\label{veroyatnost}
P(A\in \Omega)=\langle \psi, \, \hat P_{\Omega}\psi\rangle,
\end{align}
где $\{\hat P_{\Omega}\}$ --- проекторнозначная мера из разложения оператора
$\hat A$. \par
{\bf Замечание 1.} Соотношения (\ref{razl_edinitsy}) и (\ref{sp_teor1})
в дираковских обозначениях для наблюдаемой $\hat f$ имеют вид
соответственно $$1=\sum \limits_f|f\rangle \langle f| \; \; \; \; \text{и} \; \;
\; \; \hat f=\sum \limits_f f|f\rangle \langle f|.$$
{\bf Замечание 2.} В некоторой литературе (например, в [16]), сначала
постулируются свойства наблюдаемых, а затем из них выводится,
что каждой наблюдаемой соответствует проекторнозначная мера. В [13]
дается определение наблюдаемых с чисто точечным спектром, из которого
выводится, что для них существует полная ортонормированная система
собственных состояний, так что им соответствует самосопряженный оператор.
Если предположить, что для любой наблюдаемой $A$ таким свойством будет
обладать $P_\Omega=\chi_\Omega\circ A$ для любого борелевского
множества $\Omega$, то оператор $\hat P_{\Omega}$ является ортогональным
проектором (так как он самосопряжен и его собственные значения равны 0 и 1)
и выполнено (\ref{veroyatnost}). Из утверждения \ref{proekt_measure}
следует, что $\{\hat P_{\Omega}\}$ --- проекторнозначная мера. В [17]
дается общее определение наблюдаемых (не только для квантовомеханических
систем) с помощью некоторого набора аксиом (см. также [3, т. 1, стр.
339]), а для квантовой механики задается еще одна аксиома, переформулировка
которой дает соответствие между наблюдаемыми и проекторнозначными мерами.
\par
Из формулы (\ref{veroyatnost}) следует, что среднее значение наблюдаемой
$A$ в состоянии $\psi$ равно
$$\langle \hat A\rangle=\int \limits_{\R}\lambda \langle \psi, \,
dP_\lambda\psi\rangle =\langle \psi, \, \hat A\psi\rangle,$$ а
квадрат дисперсии равен $$(\delta \hat A)^2=\langle \hat A^2\rangle-
\langle \hat A\rangle^2=\langle \psi, \, \hat A^2\psi\rangle-
\langle \psi, \, \hat A\psi\rangle^2.$$
\item Если наблюдаемая имеет классический аналог, то построение
соответствующего оператора основано на принципе соответствия. Классические
наблюдаемые задаются как функции на фазовом пространстве. На их множестве
задается скобка Пуассона $$\{F_1, \, F_2\}=\frac{\partial F_1}{\partial
\vec{p}}\frac{\partial F_2}{\partial \vec{q}}-\frac{\partial F_1}{\partial
\vec{q}}\frac{\partial F_2}{\partial \vec{p}}.$$ В частности, для
координат и импульсов выполнено соотношение $\{p_i, \, q^j\}=\delta_i^j$.
Для операторов $\hat p_i$ и $\hat q^j$ постулируется равенство
$$[\hat{p}_k, \, \hat{q}^j]=\frac{\hbar}{i}\delta_k^j.$$ Дальше
требуется, чтобы в классическом пределе алгебра операторов с умножением
$\{A, \, B\}_{\hbar}:=\frac{i}{\hbar}[A, \, B]$ переходила в алгебру
функций со скобкой Пуассона. Если наблюдаемая имеет вид $f_1(p)+
f_2(q)$, то ей сопоставляется оператор $f_1(\hat p)+f_2(\hat q)$.
Если $f(p, \, q)$ содержит произведения $p$ и $q$, то их симметризуют,
а затем $p$ и $q$ заменяют соответственно на $\hat p$ и $\hat q$.
Подробнее о построении оператора по функции написано, например,
в [12]. \par
В квантовой механике рассматриваются также системы, не имеющие
классических аналогов; для них операторы наблюдаемых должны быть
построены на основании физических соображений непосредственно.
Примером таких наблюдаемых является спин.
\end{itemize}
\subsection{Спектральная теорема фон Неймана}
Введем понятие прямого интеграла гильбертовых пространств [5].
Пусть $(\Lambda, \, \Sigma, \, \mu)$ --- пространство
с положительной мерой $\mu$. Предположим, что задано разбиение
множества $\Lambda$ на непересекающиеся измеримые подмножества
$\Lambda_n$, $n=1, \; 2, \;\dots, \; \infty$. Для каждого $\lambda\in
\Lambda$ рассмотрим гильбертово пространство вида ${\cal H}(\lambda)=l_2^n$, если
$\lambda\in \Lambda_n$ ($l_2^n=\C^n$ при $n<\infty$,
$l_2^\infty:=l_2$). Пусть $N(\lambda)$ --- размерность пространства
${\cal H}(\lambda)$.
Символом $$\int \limits_{\Lambda}{\cal H}(\lambda)\, d\mu(\lambda)$$
обозначим множество векторнозначных функций
$$\Lambda\ni \lambda\mapsto u(\lambda)=(u_k(\lambda))_{k=1}^{N(\lambda)},$$
для которых $$\|u\|^2\stackrel{def}{=} \int \limits_{\Lambda}\sum
\limits _{k=1}^{N(\lambda)}|u_k(\lambda)|^2\, d\mu(\lambda)<\infty.$$
Тогда $\int \limits_{\Lambda}{\cal H}(\lambda)\, d\mu(\lambda)$ является
гильбертовым пространством со скалярным произведением
$$\langle u, \, v\rangle\stackrel{def}{=}\int \limits_{\Lambda}\sum
\limits _{k=1}^{N(\lambda)}u_k^*(\lambda)v_k(\lambda)\,
d\mu(\lambda).$$ При этом две функции считаются эквивалентными, если
они совпадают почти всюду. \par
\begin{Def}
Пространство $\int \limits_{\Lambda}{\cal H}(\lambda)\, d\mu(\lambda)$
называется прямым интегралом пространств ${\cal H}(\lambda)$, $\lambda\in
\Lambda$.
\end{Def}
{\bf Пример 1.} Пусть $\Lambda =\N$, $\mu(\{\lambda\})=1$, $\lambda\in \N$. Тогда
$\int \limits_{\Lambda}{\cal H}(\lambda)\, d\mu(\lambda)=\bigoplus _{k=1}
^\infty {\cal H}_k$ --- обычная прямая сумма пространств. \par
{\bf Пример 2.} Пусть ${\cal H}(\lambda)=\C$, $\lambda\in \Lambda$.
Тогда $\int \limits_{\Lambda}{\cal H}(\lambda)\, d\mu(\lambda)=L_2(\Lambda,
\, \Sigma, \, \mu)$. \par
Так же, как для пространства $L_2(\Lambda, \, \Sigma, \, \mu)$,
доказывается, что если мера $\mu$ сепарабельна, то прямой интеграл
пространств ${\cal H}(\lambda)$ является сепарабельным гильбертовым
пространством. Напомним, что мера называется сепарабельной, если
существует не более чем счетная система подмножеств $C_n\in \Sigma$
такая, что для любого $C\in \Sigma$ и для любого $\varepsilon>0$
найдется $C_n$ такое, что $\mu(C\bigtriangleup C_n)<\varepsilon$.
Например, если $\Lambda=\R^d$ с мерой Лебега, то в качестве множеств
$C_n$ можно взять параллелепипеды, у которых все вершины имеют
рациональные координаты.
\begin{Trm}
\label{sp_teor2}
Пусть $A$ --- самосопряженный оператор в сепарабельном гильбертовом
пространстве ${\cal H}$. Тогда существует унитарный оператор $$F:
{\cal H}\stackrel{\text{\rm на}}{\rightarrow} \hat {\cal H}=\int \limits_{\Lambda}
{\cal H}(\lambda)\, d\mu(\lambda)$$ такой, что для любой борелевской
функции $\varphi$ выполнено $$(F\varphi(A)F^{-1}u)(\lambda)=\varphi
(\lambda)u(\lambda),$$ где $\Lambda\subset \R$ --- спектр оператора $A$,
$\mu$ ---некоторая мера на нем.
\end{Trm}
Число $N(\lambda)$ называется {\it кратностью спектра}. \par
Мера $\mu$ имеет единственное разложение в сумму чисто точечной,
абсолютно непрерывной и сингулярной мер (теорема \ref{leb_st_sum3}):
$\mu=\mu_{{\rm pp}}+
\mu_{{\rm ac}}+\mu_{{\rm sing}}$. Поэтому $$L_2(\R, \, d\mu)=
L_2(\R, \, d\mu_{{\rm pp}})\oplus L_2(\R, \, d\mu_{{\rm ac}})\oplus
L_2(\R, \, d\mu_{{\rm sing}})$$ и такое же разложение имеет место для прямого
интеграла гильбертовых пространств: $$\hat {\cal H}=\hat {\cal H}_{{\rm pp}}
\oplus \hat {\cal H}_{{\rm ac}}\oplus \hat {\cal H}_{{\rm sing}}.$$
Положим ${\cal H}_{{\rm pp}}=F^{-1}\hat {\cal H}_{{\rm pp}}$,
${\cal H}_{{\rm ac}}=F^{-1}\hat {\cal H}_{{\rm ac}}$ и ${\cal H}_{{\rm sing}}
=F^{-1}\hat {\cal H}_{{\rm sing}}$, $\sigma_{{\rm ac}}(A)=\sigma(A|
_{{\cal H}_{{\rm ac}}})$ и $\sigma_{{\rm sing}}(A)=\sigma(A|_{{\cal H}
_{{\rm sing}}})$. \par
{\bf Пример 1.} Пусть $A=-\frac{d^2}{dx^2}$ на $S(\R)$. С помощью
преобразования Фурье он переводится в оператор умножения на $x^2$ в
$S(\R)$. Сделав подходящую замену переменной, получаем, что $\hat {\cal H}
=L_2(\R_+, \, d\lambda; \C^2)$ (то есть $N(\lambda)=2$ для всех $\lambda\ge
0$). Так как мера $\mu$ совпадает с лебеговской, то ${\cal H}={\cal H}
_{{\rm ac}}$. \par
{\bf Пример 2.} В главе \ref{1dimension} будет рассмотрен класс операторов
Штурма--Лиувилля $\hat H=-\frac{d^2}{dx^2}+V(x)$, где $V(x)\rightarrow a$
$(x\rightarrow -\infty)$ и $V(x)\rightarrow b$ $(x\rightarrow +\infty)$
и выполнены некоторые условия на регулярность $V$ и скорость сходимости
к $a$ и $b$. Пусть для определенности $a<b$. Будет показано, что
при $\lambda<a$ спектр однократный и дискретный (то есть он является точечным
и собственные значения изолированы), при $a<\lambda<b$ оператор $\hat H$
имеет однократный лебегов спектр, а при $\lambda>b$ --- двукратный
лебегов спектр. Таким образом, $\sigma_{{\rm ac}}=[a, \, +\infty)$,
$\sigma_{{\rm sing}}=\emptyset$, $\Lambda=\{\lambda_j\}_{j=1}^k\cup
[a, \, +\infty)$, где $\lambda_j<a$, $N(\lambda)=1$ при $\lambda<b$
и $N(\lambda)=2$ при $\lambda>b$. \par
Мера $\mu$ может быть выбрана с точностью до эквивалентности. Точнее,
имеет место
\begin{Trm}
\label{krit_unit_equiv_oper} {\rm [12, Добавление 1]}
Операторы $A_1$ и $A_2$ умножения на $\lambda$ в пространствах
$L_2(\R, \, d\mu_1)$ и $L_2(\R, \, d\mu_2)$ унитарно эквивалентны
тогда и только тогда, когда меры $d\mu_1$ и $d\mu_2$ абсолютно
непрерывны друг относительно друга.
\end{Trm}
Из теоремы \ref{sp_teor2} следует, что существуют пространство с мерой
$(M, \, \sigma)$ и унитарный оператор $U:{\cal H}\rightarrow L_2(M, \,
\sigma)$ такие, что $UAU^{-1}$ является оператором умножения на некоторую
функцию. В самом деле, пусть $\Lambda_n=\{\lambda\in \Lambda:N(\lambda)
=n\}$, $(M_n, \, \sigma_n)=\sqcup _{i=1}^n(\Lambda_n^i, \, \mu)$
--- множество из $n$ экземпляров $\Lambda_n$. В качестве $(M, \, \sigma)$
берется $\sqcup _n (M_n, \, \sigma_n)$, и каждому $\varphi=(\varphi_i
(\lambda))\in \int \limits_{\Lambda}{\cal H}(\lambda)\, d\mu(\lambda)$
сопоставляется функция $\tilde \varphi$ такая, что $\tilde \varphi(\lambda)
=\varphi_i(\lambda)$, если $\lambda\in \Lambda_n^i$.
\subsection{Определение операторов с помощью квадратичных \\ форм}
В некоторых случаях для операторов, описывающих физические
системы, не удается подобрать ``естественным образом'' их область
определения так, чтобы они были существенно самосопряженными.
Примерами таких операторов являются $-\frac{d^2}{dx^2}+\alpha
\delta(x)$ в $L_2(\R)$ или $-\Delta+V(\vec r)$ в $L_2(\R^3)$, где
$V(\vec r)\sim r^{-\alpha}$ при $r\rightarrow 0$ для $\alpha\le
\frac 32$. Иногда удается определить оператор и доказать его
самосопряженность с помощью теории квадратичных форм. \par Пусть
$A$ --- самосопряженный оператор. Перейдем к его спектральному
представлению, так чтобы $A$ стал оператором умножения на $x$ в
пространстве\\ $\bigoplus \limits_{n=1}^N L_2(\R, \, d\mu_n)$.
Положим
$$Q(A)=\left\{\{\psi_n\}_{n=1}^N: \; \sum \limits_{n=1}^N
\int \limits_{-\infty}^\infty|x|\, |\psi_n(x)|^2\, d\mu_n<\infty\right\}$$
и для всех $\varphi$, $\psi\in Q(A)$
$$q_A(\varphi, \, \psi)=\sum \limits _{n=1}^N\int \limits_{-\infty}
^\infty x\overline{\varphi_n(x)}\psi_n(x)\, d\mu_n.$$
В общем случае квадратичной формой называется отображение $q:Q(q)\times
Q(q)\rightarrow \C$ (где $Q(q)$ --- плотное линейное подмножество в
${\cal H}$), такое что $q(\cdot, \, \psi)$ сопряженно-линейно, а
$q(\varphi, \, \cdot)$ линейно. Если $q(\varphi, \, \psi)=\overline
{q(\psi, \, \varphi)}$, то форма $q$ называется {\it симметрической}.
Если существует такое число $M$, что $q(\varphi, \, \varphi)\ge
-M\|\varphi\|^2$ для любого $\varphi\in Q(q)$, то $q$ называется
{\it полуограниченной}; если при этом $M=0$, то $q$ называется
{\it положительной}. Полуограниченная форма $q$ называется {\it замкнутой},
если пространство $Q(q)$ полно относительно нормы $$\|\psi\|_{+1}=\left(q(\psi, \, \psi)+
(M+1)\|\psi\|^2\right)^{1/2}.$$ Можно показать, что
форма, порожденная полуограниченным самосопряженным оператором, замкнута.
Оказывается, верно и обратное.
\begin{Trm}
\label{oper_form}
{\rm [3, т. I, \S VIII.6]} Всякая замкнутая полуограниченная
квадратичная форма порождается некоторым однозначно определенным
самосопряженным оператором.
\end{Trm}
При этом область определения оператора задается как множество тех $\psi
\in {\cal H}_{+1}$, для которых существует вектор $g\in {\cal H}$,
такой что $$q(\varphi, \, \psi)+(M+1)\langle\varphi, \, \psi\rangle
=\langle \varphi, \, g\rangle$$
для любого $\varphi\in Q(q)$. \par
{\bf Пример.} Пусть $q$ задана на $\mathaccent'27 W^1_2[a, \, b]$
по формуле $q(\varphi, \, \psi)=\langle \varphi', \, \psi'\rangle$.
Эта форма замкнута и положительна. Тогда область определения соответствующего
ей самосопряженного оператора --- это функции $\psi\in \mathaccent'27
W^1_2[a, \, b]$, такие что $$\int \limits_a^b \psi'(x)\varphi'^*(x)\, dx
+\int \limits_a^b \psi(x)\varphi^*(x)\, dx=\int \limits_a^b g(x)
\varphi^*(x)\, dx$$ для любого $\varphi\in \mathaccent'27 W^1_2[a, \, b]$.
Интегрируя по частям, получаем $$\int \limits_a^b \left(\psi'(x)+
\int \limits_a^x (\psi(t)-g(t))\, dt\right)\varphi'^*(x)\, dx={\rm const}.$$
В силу произвольности $\varphi$, получаем $\psi'(x)=\int \limits_a^x
(g(t)-\psi(t))\, dt+{\rm const}$, так что $\psi'\in AC[a, \, b]$ и
$\psi''\in L_2[a, \, b]$, то есть $\psi \in W_2^2[a, \, b]$. Интегрируя
по частям, получаем $$q(\psi, \, \varphi)=-\int \limits_a^b\psi''(x)
\varphi(x)\, dx,$$ то есть $q$ порождается оператором $-\frac{d^2}{dx^2}$.
\par
С помощью теоремы \ref{oper_form} доказывается аналог теоремы Като--Реллиха
для квадратичных форм [3, т. 2, \S X.2].
\begin{Trm}
\label{klmn}
(КЛМН-теорема). Пусть $A$ --- положительный самосопряженный
оператор и $\beta(\varphi, \, \psi)$ --- такая симметрическая квадратичная
форма на $Q(A)$, что при некоторых $a<1$, $b\in \R$
$$|\beta(\varphi, \, \varphi)|\le a\langle\varphi, \, A\varphi\rangle+b\langle\varphi, \,
\varphi\rangle$$ для всех $\varphi\in D(A)$. Тогда существует единственный
самосопряженный оператор $C$, для которого $Q(C)=Q(A)$ и
$$\langle\varphi, \, C\psi\rangle =q_A(\varphi, \, \psi)+\beta(\varphi, \, \psi)$$
для всех $\varphi$, $\psi\in D(C)$. Оператор $C$ ограничен снизу числом
$-b$.
\end{Trm}
Пусть $B$ --- самосопряженный оператор, $Q(B)\supset Q(A)$ и для любого
$a>0$ существует $b>0$ такое, что $|\langle\varphi, \, B\varphi\rangle|\le a\langle\varphi, \,
A\varphi\rangle+b\langle\varphi, \, \varphi\rangle$, $\varphi\in Q(A)$. Тогда говорят, что
$B$ бесконечно мал по отношению к $A$ в смысле форм ($B\prec A$). В этом
случае для любого $\lambda\in \R$ форма $\langle\cdot, \, A\cdot\rangle+\langle\cdot, \,
\lambda B\cdot\rangle$ однозначно задает самосопряженный оператор. \par
В случае, когда требуется определить оператор на $L_2(I)$ (где $I\subset
\R$ --- некоторый промежуток), заданный дифференциальным выражением
$l(\psi)=-\psi''+V(x)\psi$ с сингулярным потенциалом $V$, помимо метода
квадратичных форм известны другие способы определения, в частности,
аппроксимация гладкими потенциалами (см. [14], [15]). Об этих способах
будет сказано в главе \ref{1dimension}. \par
Если $A$ --- полуограниченный симметрический оператор, то он обладает
равными индексами дефекта [3, т. 2, стр. 158] и поэтому имеет
самосопряженное расширение. Среди них есть {\it расширение по Фридрихсу},
которое получается из квадратичной формы, порождаемой оператором $A$.
\begin{Trm}
{\rm [3, т. 2, \S X.3]} Пусть $A$ --- полуограниченный симметрический
оператор и $q(\varphi, \, \psi)=\langle \varphi, \, A\psi\rangle$ для
$\varphi$, $\psi\in D(A)$. Тогда $q$ --- замыкаемая квадратичная форма и
ее замыкание $\tilde q$ является квадратичной формой единственного
самосопряженного оператора $\tilde A$. Нижняя граница спектра $\tilde A$
совпадает с нижней гранью формы $q$. Оператор $\tilde A$ является
единственным самосопряженным расширением $A$, область определения которого
содержится в области определения формы $q$.
\end{Trm}
{\bf Пример.} Пусть $A=-\frac{d^2}{dx^2}$ на $C_0^\infty(a, \, b)$.
Это положительный симметрический оператор, порождающий квадратичную
форму $q(\varphi, \, \psi)=\int \limits_a^b \varphi'^*(x)\psi'(x)\, dx$.
Замыкание $\tilde q$ формы $q$ определено на $\mathaccent'27 W^1_2[a, \, b]$
и задается той же формулой. Как было показано, $\tilde q$ порождается
оператором $-\frac{d^2}{dx^2}$ на $\mathaccent'27 W^1_2[a, \, b]\cap
W^2_2[a, \, b]$. \par
Расширение по Фридрихсу будет использоваться при построении оператора
Шредингера $-\Delta+V(x)$ с центрально-симметричным потенциалом, имеющим
особенность в нуле.
\subsection{Принцип минимакса}
Пусть $\hat H$ --- ограниченный снизу самосопряженный оператор,
$Q(\hat H)$ --- область определения формы $\hat H$ и пусть $\lambda\in
\sigma(\hat H)$. Скажем, что $\lambda$ принадлежит дискретному спектру,
если $\lambda$ является изолированной точкой спектра и имеет конечную
кратность. В остальных случаях $\lambda$ принадлежит существенному
спектру. \par
Для каждого $n\in \N$ рассмотрим числа
$$\mu_n=\sup_{\varphi_1, \, \dots, \, \varphi_{n-1}}\inf \{\langle
\psi, \, \hat H\psi\rangle:\psi\in [\varphi_1, \, \dots, \, \varphi_{n-1}]
^{\bot}, \; \|\psi\|=1, \; \psi\in Q(\hat H)\}.$$
Рассмотрим множество $\Lambda$ точек спектра, лежащих ниже края существенного
спектра. Так как оператор $\hat H$ ограничен снизу, то множество
$\Lambda$ ограничено снизу. Кроме того, каждая точка $\lambda\in \Lambda$
является изолированной и кратность собственного значения $\lambda$
конечна (по определению дискретного спектра). Значит, собственные
значения из $\Lambda$ можно занумеровать в порядке возрастания с
учетом их кратности.
\begin{Trm}
{\rm [3, т.4, \S XIII.1]} Для каждого $n\in \N$ либо существуют
$n$ собственных значений (с учетом кратности), лежащих ниже края
существенного спектра, и $\mu_n$ является $n$-м собственным значением,
либо $\mu_n$ является точной нижней гранью существенного спектра и
для любого $m>n$ выполнено $\mu_m=\mu_n$.
\end{Trm}
Из этой теоремы следует, что если $\mu_n\rightarrow \infty$ при $n\rightarrow
\infty$, то спектр оператора $\hat H$ является чисто дискретным.
\subsection{Оснащение гильбертова пространства}
Пусть ${\cal H}$ --- гильбертово пространство. Предположим, что
задано некоторое вложенное в ${\cal H}$ банахово пространство ${\cal
H}_+$ с нормой $\|\cdot\|_+$. Обозначим через $j$ оператор вложения
$j:{\cal H}_+\rightarrow {\cal H}$. Далее предполагаем, что $j$
непрерывен. Введем пространство ${\cal H}_-$ непрерывных антилинейных
функционалов на пространстве ${\cal H}_+$ с операцией сложения и
умножения $$(\lambda h_-'+\mu h_-'', \, h_+)=\lambda (h'_-, \, h_+)+
\mu(h''_-, \, h_+)$$ и с нормой $$\|h_-\|_-=\sup _{\|h_+\|_+=1}
|(h_-, \, h_+)|.$$ Имеется естественное линейное непрерывное отображение
$j^*:{\cal H}\rightarrow {\cal H}_-$, заданное по правилу $$(j^*h, \, h_+)=(h, \, jh_+)=
(h, \, h_+), \; h_+\in {\cal H}_+.$$ Если ${\cal H}_+$
плотно в ${\cal H}$, то $\ker j^*=0$, то есть $j^*$ есть вложение.
Таким образом, мы имеем тройку пространств
\begin{align}
\label{osn}
{\cal H}_+\subset {\cal H}\subset {\cal H}_-.
\end{align}
\begin{Def}
Тройка пространств (\ref{osn}) называется оснащением пространства ${\cal
H}$.
\end{Def}
Если ${\cal H}_+$ --- гильбертово пространство, то ${\cal H}_-$ ---
тоже гильбертово пространство. \par
Приведем одну конструкцию оснащения с помощью некоторого оператора
в пространстве ${\cal H}$. Пусть $T:{\cal H}\rightarrow {\cal H}$ ---
непрерывный линейный оператор такой, что $\ker T=0$, $\ker T^+=0$.
Положим ${\cal H}_+=T{\cal H}$. Так как $\ker T^+=0$, то $T{\cal H}$
плотно в ${\cal H}$. Определим в ${\cal H}_+$ скалярное произведение
$$(Th, \, Th)_+=\langle h, \, h\rangle,$$ то есть $$(h_+, \, h_+)_+=\langle T^{-1}h_+, \,
T^{-1}h_+\rangle .$$ Из непрерывности $T$ следует непрерывность оператора
вложения $j$.
\begin{Def}
Оператор $\,T:{\cal H}\rightarrow {\cal H}$ называется оператором
Гильберта--Шмидта, если для любого ортонормированного базиса
$\{e_n\}$ величина $\|T\|^2_2=\sum \limits_n \|Te_n\|^2$ конечна.
\end{Def}
Можно показать, что $\|T\|$ не зависит от выбора базиса. \par
Всякий оператор Гильберта--Шмидта компактен. Кроме того, в
пространстве $L_2(M, \, \mu)$ он представим в виде интегрального
оператора с ядром $T(\cdot, \, \cdot)\in L_2(M\times M, \, \mu\otimes
\mu)$ (см. [12]). \par
Оснащение (\ref{osn}) называется оснащением Гильберта--Шмидта, если
оно построено по оператору Гильберта\nobreak--\nobreak Шмидта $T:{\cal H}\rightarrow
{\cal H}$. \par
Следующая теорема (см. [5], [7]) дает способ построения отображения
$F$ из теоремы \ref{sp_teor2} (называемого преобразованием Фурье).
\begin{Trm}
\label{trm_maurin1}
Пусть задано оснащение Гильберта--Шмидта гильбертова пространства
${\cal H}$. Тогда преобразование Фурье из теоремы \ref{sp_teor2}
задается формулой $${\cal H}_+\ni \varphi\mapsto \hat{\varphi}_k
(\lambda)=(\varphi, \, e_k(\lambda)), \; k=1, \; \dots, \; \dim {\cal H}
(\lambda),$$ где $e_k(\lambda)$ --- так называемые обобщенные собственные
элементы, принадлежащие пространству ${\cal H}_-$. При этом если $A\varphi
\in {\cal H}_+$, то $(A\varphi, \,
e_k(\lambda))=(\varphi, \, \lambda e_k(\lambda))$.
\end{Trm}
{\bf Замечание.} В [5] приводился более общий способ
построения оснащения: рассматривалась последовательность
${\cal H}_n=K_n{\cal H}$, где $K_n$ --- операторы Гильберта--Шмидта,
брался их индуктивный предел $\Phi$, а элементы $e_k(\lambda)$
принадлежали сопряженному пространству $\Phi^*$. \par
Часто в качестве оператора $A$ рассматривается линейный дифференциальный
оператор в ${\cal H}=L_2(\Omega)$. Во многих случаях удается подобрать
оснащение так, чтобы поиск обобщенных собственных векторов сводился к
решению дифференциального уравнения $A\psi=\lambda\psi$. Пусть ${\cal H}_-
\subset D'(\Omega)$, причем оператор вложения непрерывен. Предположим,
что $A$ --- самосопряженный дифференциальный оператор с гладкими
коэффициентами. Тогда в [5] доказывается, что $e_k(\lambda)$ является
обобщенным решением дифференциального уравнения
\begin{align}
\label{difur}
Ae_k(\lambda)=\lambda e_k(\lambda).
\end{align}
В [6] дается достаточное условие того, чтобы ${\cal H}_-\subset D'(\Omega)$.
Пусть $L$ --- линейный дифференциальный оператор с бесконечно
дифференцируемыми коэффициентами, имеющий самосопряженное расширение
$K$ такое, что $K^{-1}$ существует и является оператором Гильберта--
Шмидта. Тогда полагаем ${\cal H}_+=K^{-1}{\cal H}$. Так как $L$ ---
дифференциальный оператор с бесконечно дифференцируемыми коэффициентами,
то $C_0^\infty(\Omega)\subset {\cal H}_+$. Поскольку для любого
$h_+\in {\cal H}_+$ $$(h_+, \, h_+)_+=\langle Kh_+, \, Kh_+\rangle ,$$ то для любого
$\varphi\in C_0^\infty(\Omega)$ $$(\varphi , \, \varphi)_+
=\langle L\varphi, \, L\varphi\rangle ,$$
и топология в $C_0^\infty(\Omega)$ оказывается
сильнее топологии, индуцированной скалярным произведением в ${\cal H}_+$.
Значит, всякий элемент из ${\cal H}_-$ является элементом $D'(\Omega)$.
\par
Утверждается, что для любого интервала $(\alpha, \, \beta)\subset \R$
(возможно, неограниченного), такой оператор $L$ в $L_2(\alpha, \, \beta)$
существует. В частности, если $(\alpha, \, \beta)=(-\infty, \, +\infty)$,
то в качестве $L$ можно взять $-\frac{d^2}{dx^2}+x^2$. Если $\varphi\in
S(\R)$ (пространству Шварца), то $L(\varphi)=-\frac{d^2\varphi}{dx^2}
+x^2\varphi\in L_2(\R)$, то есть $S(\R)\subset {\cal H}_+$. Кроме того,
топология в $S(\R)$ сильнее, чем в ${\cal H}_+$. Поэтому элементы
${\cal H}_-$ являются элементами пространства $S'(\R)$. В частности,
они не могут экспоненциально возрастать на бесконечности. \par
Существование решения уравнения (\ref{difur}), принадлежащего ${\cal H}_-$,
является необходимым условием того, что $\lambda\in \sigma(A)$. Если
$A=-\Delta+V$, то можно получить достаточное условие.
\begin{Sta}
\label{bound}
Пусть решение $\psi=e_k(\lambda)$ уравнения (\ref{difur}) с
$A=-\Delta+V(\vec{r})$ таково, что $\psi$, $\bigtriangledown \psi\in
L_2^{{\rm loc}}(\R^m)$ и существует $c>0$ такое, что
\begin{align}
\label{bound_cond1}
\int \limits _{n/2\le r\le n}|\psi(\vec{r})|^2\, d^m\vec{r}\le c\int \limits_{0\le r\le
n/2}|\psi(\vec{r})|^2\,d^m\vec{r}
\end{align}
для достаточно больших $n$ и
\begin{align}
\label{bound_cond2}
|\bigtriangledown \psi(\vec{r})|\le c|\psi(\vec{r})|
\end{align}
для любого $\vec{r}\in \R^m\backslash B_R(0)$. Тогда $\lambda$
принадлежит спектру оператора $A$.
\end{Sta}
\begin{proof}
Построим последовательность функций $\psi_n\in L_2(\R)$ таких, что
$\|\psi_n\|_{L_2(\R^m)}\rightarrow \infty\, (n\rightarrow \infty)$ и
$\{\|(A-\lambda)\psi_n\|\}$ ограничено. В самом деле, пусть $\eta \in C^\infty
(\R^m)$, $\eta(\vec{r})\in [0, \, 1]$, $\eta|_{\left\{r\le \frac12\right\}}\equiv 1$,
$\eta|_{\{r\ge 1\}}\equiv 0$. Положим $\eta _n(\vec{r})=\eta\left(\frac
{\vec{r}}{n}\right)$, $\psi_n(\vec{r})=\eta_n(\vec{r})\psi(\vec{r})$.
Тогда $$(A-\lambda)\psi_n(\vec{r})=-\Delta(\psi(\vec{r})\eta_n(\vec{r}))+
V(\vec{r})\eta_n(\vec{r})\psi(\vec{r})-\lambda\eta_n(\vec{r})\psi(\vec{r})=$$
$$=-2\bigtriangledown \eta_n(\vec{r})\bigtriangledown \psi(\vec{r})-
\Delta \eta_n(\vec{r})\psi(\vec{r}).$$
Далее, при достаточно больших $n$ выполнены неравенства
$$\|\psi_n\|^2_{L_2(\R^m)}\ge \int \limits_{r\le n/2}|\psi(\vec{r})|^2
d^m\vec r,$$
$$\int \limits _{\R^m}|\bigtriangledown \eta_n(\vec{r})
\bigtriangledown \psi(\vec{r})|^2\, d^m\vec{r}\le \int \limits
_{n/2\le r\le n}|\bigtriangledown
\eta_n(\vec{r})|^2|\bigtriangledown \psi(\vec{r})|^2d^m\vec r=$$
$$=\int \limits _{n/2\le r \le n}\frac{1}{n^2}|\bigtriangledown \eta\left(\vec{r}/n\right)|^2
|\bigtriangledown\psi(\vec{r})|^2\, d^m\vec r\le \frac{\const}{n^2}\int
\limits _{n/2\le r\le n}|\bigtriangledown\psi(\vec r)|^2\, d^m\vec r\stackrel{(\ref{bound_cond1})}{\le} $$
$$\le \frac{\const}{n^2}\int \limits _{n/2\le r\le n}|\psi (\vec r)|^2\, d^m\vec r
\stackrel{(\ref{bound_cond2})}{\le}\frac{\const}{n^2}\int \limits_{0\le r\le n/2}|\psi(\vec{r})|^2\,
d^m\vec r,$$ $$\int \limits _{\R^m}|\Delta
\eta_n(\vec{r})\psi(\vec{r})|^2\, d^m\vec{r}\le \const \int
\limits _{n/2 \le r\le n}|\Delta \eta_n(\vec{r})\psi(\vec{r})|^2\,
d^m\vec r\le$$
$$\le \frac{\const}{n^4}\int \limits _{0\le r\le n/2}|\psi(\vec{r})|^2
\, d^m\vec r.$$ Значит, $$\|(A-\lambda)\psi_n\|_{L_2(\R^m)}\le \frac{\const}{n^2}
\int \limits _{0\le r\le n/2}|\psi(\vec{r})|^2\, d^m\vec r.$$ Выберем последовательность
$\alpha_n\in \R$ так, чтобы $$\frac{\alpha_n^2}{n^2}\int
\limits_{0\le r\le n/2}|\psi(\vec{r})|^2\, d^m\vec r\in [1, \,
2].$$ Тогда последовательность
$\{\|(A-\lambda)\alpha_n\psi_n\|_{L_2(\R^m)}\}$
ограничена и $\|\alpha_n\psi_n\|\rightarrow \infty$ при
$n\rightarrow \infty$.
\end{proof}
\begin{Cor}
\label{cor_bound}
Пусть функция $\psi\in L_2^{{\rm loc}}(\R)$ из предыдущего утверждения
ограничена сверху вместе со своей производной и отделена от нуля
при достаточно больших $|x|$. Тогда $\lambda\in \sigma(A)$.
\end{Cor}
Пусть $f$, $h\in {\cal H}_+$. Тогда из унитарности преобразования Фурье
получаем равенство
$$\int \limits_{\Lambda}\sum \limits_{k=1}^{N(\lambda)}
\langle e_k(\lambda), \, h\rangle \hat f_k^*(\lambda)\, d\mu(\lambda)=
\int \limits_{\Lambda}\sum \limits_{k=1}^{N(\lambda)}
\hat f_k^*(\lambda)\hat h_k(\lambda)\, d\mu(\lambda)=
\langle \hat f, \, \hat h\rangle=\langle f, \, h\rangle,$$
то есть
$$\int \limits_{\Lambda}\sum \limits_{k=1}^{N(\Lambda)}e_k(\lambda)\hat f_k
(\lambda)\, d\mu(\lambda)=f$$ в слабом смысле. \par
Пусть ${\cal H}=L_2(X)$, где $X$ --- область в $\R^n$ (возможно,
неограниченная). Предположим, что выполнены следующие условия:
\begin{enumerate}
\item множество ограниченных функций из ${\cal H}_+$, имеющих компактный
носитель, плотно в $L_2(X)$;
\item для любых $\lambda\in \Lambda$, $k=1, \, \dots, \, N(\lambda)$
обобщенные собственные векторы $e_k(\lambda)$ являются регулярными
функциями, то есть
\begin{align}
\label{eklphi}
\langle e_k(\lambda), \, \varphi\rangle =\int \limits_X e_k(\lambda, \, x)
\varphi(x)\, dx, \; \varphi\in {\cal H}_+,
\end{align}
где $e_k(\lambda, \, x)$ --- локально интегрируемая функция;
\item в пространстве $\hat{{\cal H}}$ содержится плотное подмножество
$\hat{{\cal H}}_1$, состоящее из ограниченных векторнозначных
функций $\varphi(\lambda)=(\varphi_1(\lambda), \, \dots, \, \varphi
_{N(\lambda)}(\lambda))$ с ограниченным носителем, таких что функция
\begin{align}
\label{obr_preobr_furie}
\tilde \varphi(x)=\int \limits _{\Lambda}\sum \limits _{k=1}^{N(\lambda)}
e^*_k(\lambda, \, x)\varphi _k(\lambda)\, d\mu(\lambda)
\end{align}
принадлежит $L_2(X)$.
\end{enumerate}
Тогда отображение $\varphi\mapsto \tilde \varphi$, заданное формулой
(\ref{obr_preobr_furie}), продолженное по непрерывности на все
$\hat{{\cal H}}$, является обратным к преобразованию Фурье $U$. В самом
деле, пусть $\varphi\in \hat{{\cal H}}_1$ и $\psi\in {\cal H}_+$ ---
ограниченная функция с компактным носителем. Тогда из унитарности $U$
и теоремы Фубини получаем $$\int \limits_X \psi^*(x)(U^{-1}\varphi)(x)\, dx=
\int \limits_{\Lambda}\sum \limits_{k=1}^{N(\lambda)}\left(\int
\limits_X e^*_k(\lambda, \, x)\psi^*(x)\, dx\right)\varphi_k(\lambda)
\, d\mu(\lambda)=$$ $$=\int \limits_X \psi^*(x)\left(\int \limits
_{\Lambda}\sum \limits_{k=1}^{N(\lambda)}e^*_k(\lambda, \, x)
\varphi_k(\lambda)\, d\mu(\lambda)\right)\, dx=\int \limits _X
\psi^*(x)\tilde \varphi(x)\, dx.$$ Так как множество функций $\psi$
плотно в $L_2(X)$, то $\tilde \varphi=U^{-1}\varphi$. \par
Применив к последнему равенству формулу (\ref{eklphi}) для прямого преобразования Фурье,
получаем $$\varphi_{k'}(\lambda')=\int \limits_X e_{k'}(\lambda', \,
x)\left(\int \limits_{\Lambda}\sum \limits_{k=1}^{N(\lambda)}
e_k^*(\lambda, \, x)\varphi_k(\lambda)\, d\mu(\lambda) \right)\, dx'.$$
Если мера $\mu$ эквивалентна лебеговской и функции $\varphi_k(\cdot)$
непрерывны, то последнее равенство можно записать в виде $$\int
\limits_{\Omega}e_k^*(\lambda, \, x)e_{k'}(\lambda', \, x)\, dx=
\delta _{kk'}\delta(\lambda-\lambda').$$
Для того, чтобы существовали регулярные обобщенные собственные функции,
достаточно, чтобы $A$ был эллиптическим дифференциальным оператором
(см. [5]).
\subsection{Представления}
Пусть имеется наблюдаемая $A$. Тогда, по теореме \ref{sp_teor2},
существует унитарный оператор $F:{\cal H}\rightarrow \hat{{\cal H}}=
\int \limits_{\Lambda}{\cal H}(\lambda)\, d\mu(\lambda)$ такой, что
$F\hat{A}F^{-1}$ --- оператор умножения на $\lambda$. Пространство
$\hat{{\cal H}}$ с действующим в нем оператором умножения называется
представлением, соответствующим оператору $A$. \par
Рассмотрим одномерное движение частицы. Как говорилось в \S
\ref{principles}, если состояние задается волновой функцией $\psi(x)$,
то вероятность обнаружить частицу в интервале $\Omega$ равна $\int
\limits_{\Omega}|\psi(x)|^2\, dx=\|\chi_{\Omega}\psi\|_{L_2}$. Таким
образом, если пространство состояний задается как $L_2(\R)$, а
наблюдаемой является координата, то ей соответствует проекторнозначная мера
$P_\Omega\psi(x)=\chi_\Omega(x)\psi(x)$. Отсюда следует, что $\hat x$
--- это оператор умножения на $x$. \par
Если определить оператор импульса в координатном представлении
как $\hat{p}=\frac{\hbar}{i}\frac{d}{dx}$, то $\hat{x}$ и $\hat{p}$
будут удовлетворять каноническим коммутационным соотношениям
\begin{align}
\label{kks}
[\hat{p}, \, \hat{x}]=\frac{\hbar}{i}.
\end{align}
Функции Гамильтона $H=\frac{p^2}{2m}+V(x)$ тогда соответствует оператор
$\hat{H}=-\frac{\hbar ^2}{2m}\frac{d^2}{dx^2}+V(x)$. \par
Возникает вопрос: в каком смысле представление
\begin{align}
\hat{x}=x, \; \hat{p}=\frac{\hbar}{i}\frac{d}{dx}
\label{shre_predst}
\end{align}
является ``единственно возможным'' представлением соотношения (\ref{kks}).
Рассмотрим операторы $U(t)=e^{\frac{i}{\hbar}t\hat{p}}$ и $V(s)=e^{is\hat{x}}$.
Тогда формальные вычисления, использующие (\ref{kks})
и формальные разложения в степенные ряды для экспонент, дают
\begin{align}
\label{kks1}
U(t)V(s)=e^{its}V(s)U(t).
\end{align}
Легко проверяется, что в представлении (\ref{shre_predst}) выполняются
соотношения (\ref{kks1}) (называемые соотношениями Вейля). При этом
$U(t)$ --- левый сдвиг на $t$, а $V(s)$ --- умножение на $e^{isx}$.
С другой стороны, если $U(t)$ --- левый сдвиг на $t$, а $V(s)$ ---
умножение на $e^{isx}$, то $\frac{\hbar}{i}\frac{d}{dt}|_{t=0}U(t)
=\frac{\hbar}{i}\frac{d}{dx}$ и $\frac{1}{i}\frac{d}{ds}|_{s=0}V(s)
=x$. Следующая теорема фон Неймана [3, \S VIII.5] утверждает,
что с точностью до кратности и унитарной эквивалентности соотношения
(\ref{kks1}) имеют единственное решение.
\begin{Trm}
Пусть $U(t)$ и $V(s)$ --- однопараметрические непрерывные унитарные
группы на сепарабельном гильбертовом пространстве ${\cal H}$,
удовлетворяющие соотношениям Вейля. Тогда существуют такие замкнутые
подпространства ${\cal H}_l$, что \par
(a) ${\cal H}=\bigoplus_{l=1}^N {\cal H}_l$, $N\in \N$ или $N=\infty$; \par
(b) $U(t):{\cal H}_l\rightarrow {\cal H}_l$, $V(s):{\cal H}_l\rightarrow
{\cal H}_l$ для всех $s$, $t\in \R$; \par
(c) для каждого $l$ существует такой унитарный оператор $T_l:{\cal H}_l
\rightarrow L_2(\R)$, что $T_lU(t)T_l^{-1}$ --- левый сдвиг на $t$,
а $T_lV(s)T_l^{-1}$ --- умножение на $e^{isx}$.
\end{Trm}
Если операторы $\hat{p}$ и $\hat{x}$ таковы, что существует плотная область
$D\subset {\cal H}$ такая, что
\begin{enumerate}
\item $\hat{p}:D\rightarrow D$, $\hat{x}:D\rightarrow D$,
\item $\hat{p}\hat{x}\psi-\hat{x}\hat{p}\psi=\frac{\hbar}{i}\psi$ для
всех $\psi\in D$,
\item $\hat{p}$ и $\hat{x}$ существенно самосопряжены на $D$,
\end{enumerate}
то отсюда не следует, что порождаемые ими группы удовлетворяют
соотношениям Вейля. Однако если еще известно, что оператор $\hat{p}^2
+\hat{x}^2$ существенно самосопряжен на $D$, то соотношения Вейля
будут выполняться (см. [3], замечания к гл. VIII). \par
Найдем обобщенные собственные векторы оператора $\hat p$,
соответствующие значению $p$. Они удовлетворяют уравнению (\ref{difur})
с $A=\hat{p}$ и $\lambda=p$, то есть $$\frac{\hbar}{i}\frac{d}{dx}e_p(x)
=pe_p(x).$$ Отсюда $e_p(x)=C(p)e^{\frac{ipx}{\hbar}}$. Значит, оператор
перехода от координатного представления к импульсному имеет вид
$$(F\psi)(p)=C(p)\langle\psi(\cdot), \, e_p(\cdot)\rangle=C(p)\int
\limits_{-\infty}^{+\infty}e^{\frac{-ipx}{\hbar}}\psi(x)dx$$ для
любого $\psi\in S(\R)\subset {\cal{H}}_+$. Если положить $C(p)=\frac{1}{\sqrt{2\pi\hbar}}$,
то оператор $F$ будет обычным преобразованием
Фурье, а соответствующая мера на $\R$ в импульсном представлении будет
лебеговской. Оператор координаты в импульсном представлении, следовательно,
задается как $-\frac{\hbar}{i}\frac{d}{dp}$. \par
В многомерном случае операторы координаты и импульса задаются в
координатном представлении как $\hat{x}^k=x^k$, $\hat{p}^k=\frac{\hbar}{i}
\frac{d}{dx^k}$. Оператор Гамильтона имеет вид $\hat{H}=-\frac{\hbar^2}
{2m}\Delta+V(\vec{x})$. \par
Другое часто используемое представление --- энергетическое. Спектр
оператора Гамильтона в типичном случае состоит из дискретного
участка, соответствующего финитному движению, и непрерывного участка,
отвечающего инфинитному движению. В случае чисто дискретного спектра
оператора Гамильтона самосопряженные операторы в энергетическом
представлении задаются эрмитовыми матрицами. \par
Переход от координатного представления к энергетическому осуществляется
с помощью разложения по собственным функциям оператора Гамильтона в
координатном представлении: $$\psi(x)=\sum \limits _{n=1}^\infty
c_n\psi_n(x),$$ $$\hat{H}\psi_n=E_n\psi_n.$$ Последнее уравнение
называется стационарным уравнением Шредингера.
\section{Одновременная измеримость и соотношение \\ неопределенностей}
Если система находится в одном из собственных состояний $|f\rangle$,
то измерение данной величины $F$ с достоверностью покажет соответствующее
собственное значение $f$. В самом деле, $P_{\{f\}}$ --- это проектор на
$|f\rangle$, для любого борелевского множества $\Omega\not\ni f$
оператор $P_\Omega$ является проектором на подпространство, ортогональное
$|f\rangle$, поэтому $${\cal P}(\{F=f\})=\langle f|P_{\{f\}}|f\rangle =
\langle f|f\rangle =1$$ и $${\cal P}(F\in \Omega)=0.$$
Возникает вопрос: существует ли полная система состояний, в которых
несколько наблюдаемых $F_1, \; \dots, \; F_n$ имеют с достоверностью
определенные значения? Более формально: возможен ли изоморфизм
пространства ${\cal H}$ на пространство $L_2(M, \, d\mu)$, при котором
операторы $F_1, \; \dots, \; F_n$ переходят в операторы умножения
на измеримые вещественнозначные функции? \par
\begin{Def}
\label{str_comm}
Самосопряженные операторы называются сильно коммутирующими, если
коммутируют проекторы из разложения единицы этих операторов.
\end{Def}
{\bf Замечание.} Если $D$ --- плотное подпространство в ${\cal H}$,
$A:D\rightarrow D$, $B:D\rightarrow D$ --- существенно самосопряженные
операторы, $AB\varphi-BA\varphi=0$ для любого $\varphi\in D$, то
отсюда еще не следует, что $A$ и $B$ сильно коммутируют (в [3, гл.
VIII] приведен пример Нельсона). Однако если один из операторов
ограничен, то утверждение будет верным.
\begin{Sta} {\rm [12, Добавление 1]}
Пусть $A$ --- самосопряженный оператор, $B$ --- ограниченный симметрический оператор,
для любого $f\in D(A)$ выполнено $Bf\in D(A)$ и $ABf=BAf$. Тогда операторы
$A$ и $B$ сильно коммутируют.
\end{Sta}
\begin{Cor}
\label{cor_nelson}
Пусть $D\subset {\cal H}$ --- плотное подпространство, $A:D\rightarrow
{\cal H}$, $B:{\cal H}\rightarrow {\cal H}$ --- симметрические операторы,
$A$ существенно самосопряжен, $B$ ограничен, $B(D)\subset D$ и $AB\varphi =BA\varphi$
для любого $\varphi\in D$. Тогда
$A$ и $B$ сильно коммутируют.
\end{Cor}
\begin{proof}
Пусть $\overline A$ --- замыкание оператора $A$, $f\in D(\overline A)$.
Докажем, что $Bf\in D(\overline A)$ и $\overline ABf=B\overline Af$. В самом деле, пусть
$f_n\in D$, $f_n\rightarrow f$ и $Af_n\rightarrow \overline Af$. Тогда $ABf_n=BAf_n
\rightarrow B\overline Af$ при $n\rightarrow \infty$. Значит, последовательность
$(Bf_n, \, ABf_n)\in \Gamma _{\overline A}\;$ фундаментальна. Так как
оператор $\overline A$ замкнут, то $Bf_n\rightarrow g$, $ABf_n\rightarrow \overline Ag$,
где $g\in D(\overline A)$. С другой стороны, $Bf_n\rightarrow Bf$ и $ABf_n
\rightarrow B\overline Af$, так что $Bf\in D(\overline A)$ и $\overline ABf=B
\overline Af$. Осталось воспользоваться
предыдущим утверждением.
\end{proof}
Следующая теорема [4, т. 1, гл. 11, \S 5] применима и для того случая, когда оба
оператора не являются ограниченными.
\begin{Trm}
\label{trm_nelson}
(теорема Нельсона). Пусть $D\subset {\cal H}$
--- плотное подпространство, $A$, $B:D\rightarrow D$ --- симметрические
операторы, $AB\varphi =BA\varphi$ для любого $\varphi\in D$ и оператор
$A^2+B^2$ существенно самосопряжен на $D$. Тогда $A$ и $B$ существенно
самосопряжены на $D$ и их замыкания сильно коммутируют.
\end{Trm}
Пусть операторы $A_1$ и $A_2$ в $L_2(M, \, \mu)$ являются операторами
умножения на функции $f_1(m)$ и $f_2(m)$ соответственно. Тогда (см. [3])
проекторы из разложения единицы $P_{\Omega}^j$ определяются как
операторы умножения на $\chi_{\Omega}\circ f_j$ (где $\chi_{\Omega}$
--- характеристическая функция множества $\Omega$). Значит, $P^j_{\Omega}$
снова являются операторами умножения на функцию и, следовательно,
сильно коммутируют. \par
В следующей теореме (см. [5] или [7]) говорится, что верно и обратное
утверждение.
\begin{Trm}
\label{maurin_trm}
Пусть $A_1, \; \dots, \; A_n$ --- сильно коммутирующие операторы,
${\cal H}_+\subset {\cal H}\subset {\cal H}_-$ --- оснащение
Гильберта--Шмидта пространства ${\cal H}$. Тогда существует прямой интеграл
$$\hat{{\cal H}}=\int \limits _{\Lambda}\hat{{\cal H}}(\lambda)\, d\mu
(\lambda)$$ и унитарный оператор $F:{\cal H}\rightarrow \hat{{\cal H}}$,
который удовлетворяет равенству
\begin{align}
\label{4oper_preobr_furie}
(FA_jF^{-1}g)(\lambda)=a_j(\lambda)g(\lambda)
\end{align}
и задается формулой
\begin{align}
\label{4obob_preobr_furie}
{\cal H}_+\ni \varphi\mapsto \hat{\varphi}_k(\lambda)=(\varphi, \,
e_k(\lambda)),
\end{align}
где $k=1, \; \dots, \; \dim \hat{{\cal H}}(\lambda)$
и $e_k(\lambda)\in {\cal H}_-$. Если $\varphi$, $A_j\varphi\in {\cal H}_+$,
то $$(A_j\varphi, \, e_k(\lambda))=a_j(\lambda)(\varphi, \,
e_k(\lambda))$$ для $\mu$-почти всех $\lambda$.
\end{Trm}
Иногда известно, что оператор $A$ коммутирует c некоторым унитарным
оператором (такая ситуация возникает в случае периодического потенциала).
Следующее утверждение дает достаточное условие для того, чтобы операторы
$A$ и $U$ имели общую систему обобщенных собственных векторов.
\begin{Cor}
\label{cor_maurin}
Пусть $A$ существенно самосопряжен на подпространстве $D\subset {\cal H}$,
$U:{\cal H}\rightarrow {\cal H}$ --- унитарный оператор, $D$ ---
инвариантное подпространство для операторов $U$ и $U^{-1}$. Предположим, что
$AU\varphi=UA\varphi$ для любого $\varphi\in D$. Тогда существует
обобщенное преобразование Фурье $F$ пространства ${\cal H}$ на
$$\hat{{\cal H}}=\int \limits _{\Lambda}\hat{{\cal H}}(\lambda)\, d\mu
(\lambda),$$ задаваемое по формуле (\ref{4obob_preobr_furie}),
такое что $FAF^{-1}$ и $FUF^{-1}$ --- операторы умножения на
измеримые функции $a(\lambda)$ и $u(\lambda)$ соответственно. Если
$\varphi\in {\cal H}_+$, $A\varphi\in {\cal H}_+$ ($U\varphi\in
{\cal H}_+$), то $\langle e_k(\lambda), \,
A\psi\rangle=a(\lambda)\langle e_k(\lambda), \, \psi\rangle$
(соответственно $\langle e_k(\lambda), \, U\psi\rangle =u(\lambda)\langle
e_k(\lambda), \, \psi\rangle$).
\end{Cor}
\begin{proof}
Положим $B=U+U^+$, $C=i(U-U^+)$. Тогда $D$ --- общая плотная
инвариантная область определения существенно самосопряженных операторов
$A$, $B$ и $C$. Докажем, что $AU^+=U^+A$ на $D$. Действительно, пусть
$\psi\in D$, $\varphi=U^{-1}\psi$. Тогда $U^{-1}AU\varphi=U^{-1}UA\varphi
=A\varphi$, то есть $U^+A\psi=U^{-1}A\psi=AU^{-1}\psi=AU^+\psi$. Отсюда
следует, что операторы $A$, $B$ и $C$ попарно коммутируют, при этом
$B$ и $C$ ограничены. В силу следствия \ref{cor_nelson}, операторы $A$,
$B$ и $C$ сильно коммутируют. По теореме \ref{maurin_trm}, существует
унитарное отображение $F$ пространства ${\cal H}$ на $\hat{{\cal H}}$,
задаваемое формулой (\ref{4obob_preobr_furie}) и удовлетворяющее равенствам
$FAF^{-1}g(\lambda)=a(\lambda)g(\lambda)$, $FBF^{-1}g(\lambda)=b(\lambda)
g(\lambda)$ и $FCF^{-1}g(\lambda)=c(\lambda)g(\lambda)$. Так как
$U=\frac12(B-iC)$, то $FUF^{-1}g(\lambda)=\frac12(b(\lambda)-ic(\lambda))
g(\lambda)$. Пусть $\varphi\in {\cal H}_+$ и $A\varphi\in {\cal H}_+$.
Тогда $\langle e_k(\lambda), \, A\psi\rangle\psi=a(\lambda)F\psi
=a(\lambda)\langle e_k(\lambda), \, \psi\rangle$. Утверждение для $U$
доказывается аналогично.
\end{proof}
Введем понятие полного набора наблюдаемых (см. [6],  [12]).
Определим функцию от сильно коммутирующих операторов. Пусть $P_\lambda^j$
--- разложение единицы для $A_j$. Определим для каждого $\psi \in
{\cal H}$ в $\R^n$ меру
$$\mu(\Delta_1\times\dots\times\Delta_n)=\langle P^1(\Delta_1)
\dots P^n(\Delta_n)\psi, \, \psi \rangle$$
и продолжим ее по Лебегу. Пусть для любого $\psi \in {\cal H}$
функция $\varphi(\lambda_1, \; \dots, \; \lambda_n)$ измерима и почти
всюду конечна. Так как $A_j$ сильно коммутируют, то ${\cal H}$ можно
изоморфно отобразить на пространство $L_2(M, \, \mu)$ так, чтобы
$A_j$ перешли в операторы умножения на функции $a_j(m)$. При этом
утверждается, что пространство $(M, \, \mu)$ можно выбрать так, чтобы
$\varphi(a_1(m), \; \dots, \; a_n(m))$ была почти всюду конечной.
\begin{Def}
Оператор $\,\varphi(A_1, \; \dots, \; A_n)$ в пространстве ${\cal H}$
--- это оператор, который при описанной выше реализации ${\cal H}$ в
виде $L_2(M, \, \mu)$ переходит в оператор умножения на функцию
$\varphi(a_1(m),\;\dots\break\dots, a_n(m))$.
\end{Def}
Обозначим через $R^j_{z_j}$ резольвенты $A_j$.
\begin{Def}
Вектор $\psi \in {\cal H}$ называется циклическим вектором для системы
операторов $\{A_j\}_{j=1}^n$, если наименьшее замкнутое подпространство в
${\cal H}$, содержащее $\psi$ и инвариантное относительно всех
операторов $R_{z_j}^j$, $j=1, \; \dots, \; n$, совпадает с ${\cal H}$.
\end{Def}
\begin{Def}
Сильно коммутирующие операторы $A_1, \; \dots, \; A_n$ имеют простой
совместный спектр, если существует циклический вектор для этой
системы операторов.
\end{Def}
Следующая теорема дает критерий того, что сильно коммутирующие операторы
имеют простой совместный спектр.
\begin{Trm}
Пусть $A_1, \; \dots, \; A_n$ --- сильно коммутирующие самосопряженные
операторы. Тогда для того, чтобы они имели простой совместный спектр, необходимо
и достаточно, чтобы для любого самосопряженного оператора $B$, сильно коммутирующего
с $A_j$, существовала функция $\varphi$ такая, что $B=\varphi(A_1, \; \dots, \;
A_n)$.
\end{Trm}
Скажем, что наблюдаемые $A_1, \; \dots, \; A_n$ образуют полный
набор, если они одновременно измеримы и всякая наблюдаемая,
измеримая одновременно с $A_1,$ $ \; \dots, \; A_n$, является
функцией от этих величин.
\begin{Trm}
Если $A_1, \; \dots, \; A_n$ --- полный набор наблюдаемых, то
существует изоморфизм пространства состояний на $L_2(\R^n, \, \mu)$
такой, что $A_i$ соответствует оператору умножения на $\lambda_i$.
Если $A_i:L_2(\R^n, \, \mu)\rightarrow L_2(\R^n, \, \mu)$, $(A_if)
(\lambda_1, \; \dots, \; \lambda_n)=\lambda _if(\lambda _1, \;
\dots, \; \lambda _n)$, то $A_1, \; \dots, \; A_n$ имеют совместный
простой спектр.
\end{Trm}
Таким образом, если задано координатное представление некоторой системы,
то операторы $\hat{x}^i$ образуют полный набор и поэтому любая
наблюдаемая, одновременно измеримая с координатами, является функцией от них. \par
В качестве полного набора наблюдаемых можно также взять набор
импульсов $\hat{p}_i$. Если гамильтониан системы с тремя степенями
свободы имеет центрально-симметричный потенциал, то в качестве полного
набора удобно взять $\hat{H}$, $\hat{L}^2$ и $\hat{L}_i$ для некоторого
$i$, где $\vec{L}=\vec{r}\times \vec{p}$, $L^2=L_x^2+L_y^2+L_z^2$
(потом об этом будет сказано подробнее). \par
Пусть имеется классическая система с $n$ степенями свободы и наблюдаемые
$f_j(\vec{p}, \, \vec{x})$, $j=1, \; \dots, \; m$, которые
функционально независимы (то есть векторы $e_j=\bigtriangledown f_j$ линейно
независимы в каждой точке). Пусть $\hat{f}_j$ --- соответствующие
наблюдаемые в квантовой системе. Предположим, что $\hat{f}_j$
образуют полный набор. Тогда $[\hat{f}_k, \, \hat{f}_j]=0$. Значит,
по принципу соответствия, $\{f_k, \, f_j\}=0$. Скобка Пуассона имеет вид
$\{f_k, \, f_j\}=\Omega^{st}e^s_ie^t_j$, где $\Omega$ --- невырожденная
кососимметрическая $(2n)\times(2n)$-матрица. Так как изотропное
подпространство для невырожденной билинейной кососимметрической формы
на $2n$-мерном пространстве имеет размерность не больше $n$, то
число векторов $e_j$ должно быть не больше $n$. Значит, если наблюдаемые
образуют полный набор и их классические аналоги функционально независимы,
то их число не превосходит числа степеней свободы. \par
{\bf Замечание.} Мы предполагали, что скобка Пуассона имеет стандартный
вид. Есть более общие гамильтоновы системы, в которых скобка Пуассона
задается кососимметрической матрицей $\Omega$, зависящей от точки и
вообще говоря вырожденной. \par
Пусть $A^+=A$, $B^+=B$, $[A, \, B]=iC$, $C^+=C$. Определим дисперсии
$$(\delta A)^2=\langle (A-\langle A\rangle)^2\rangle , \; (\delta B)^2=
\langle (B-\langle B\rangle)^2\rangle, \; (\delta C)^2=\langle (C-
\langle C\rangle)^2\rangle,$$ где усреднение берется по состоянию
$\psi$. Построим однопараметрическое семейство векторов
$$\varphi =(A-\langle A\rangle-i\xi(B-\langle B\rangle))
\psi , \; \xi\in \R.$$ Тогда
$$0\le \langle\varphi , \, \varphi\rangle=(\delta A)^2+\xi^2(\delta B)^2
+\xi \langle C\rangle.$$ Так как $\xi\in \R$ произвольно, то
дискриминант не может быть положительным, то есть
\begin{align}
\label{sootn_neopr}
\delta A\cdot \delta B\ge \frac{|\langle C\rangle|}{2}.
\end{align}
Это соотношение называется соотношением неопределенностей. Для операторов
координаты и импульса, входящих в канонически сопряженную пару, $C=
\hbar$ и (\ref{sootn_neopr}) имеет вид
$$\delta \hat{x}\cdot\delta \hat{p}\ge \frac{\hbar}{2}.$$
Для компонент момента импульса $[\hat{L}_x, \, \hat{L}_y]=i\hbar \hat{L}_z$, поэтому
(\ref{sootn_neopr}) записывается в виде
$$\delta \hat{L}_x\cdot \delta \hat{L}_y\ge \hbar \frac{|\langle \hat{L}_z
\rangle|}{2}.$$
\section{Картины Гейзенберга и Шредингера}
В классической механике Гамильтона зависимость от времени наблюдаемой
$F(\vec{p}, \, \vec{q})$ определяется ее скобкой Пуассона с гамильтонианом:
$$\frac{dF}{dt}=\{H, \, F\}=\frac{\partial H}{\partial\vec{p}}
\frac{\partial F}{\partial\vec{q}}-\frac{\partial F}{\partial\vec{p}}
\frac{\partial H}{\partial\vec{q}}.$$ На основании принципа соответствия
получаем для операторов
\begin{align}
\label{evol}
\frac{d\hat{F}(t)}{dt}=\frac{i}{\hbar}[\hat{H}, \, \hat{F}(t)],
\end{align}
где $\hat{F}(t)$ --- оператор физической величины $F$ в момент
времени $t$. Положим $\hat{F}=\hat{F}(0)$. \par
Предположим, что $\hat H$ не зависит от времени явно.
Если бы все операторы в (\ref{evol}) были ограниченными, то можно было
бы проверить, что решение имеет вид $\hat F(t)=e^{\frac{it\hat H}{\hbar}}\hat Fe^{-\frac{it
\hat H}{\hbar}}$. При этом уравнение (\ref{evol}) рассматривалось бы в равномерной
операторной топологии. \par
В следующей теореме [3, т. 1, \S VIII.4] приводятся свойства семейства
операторов $e^{itA}$, $t\in \R$, где $A$ --- самосопряженный оператор
(возможно, неограниченный).
\begin{Trm}
Пусть $A$ --- самосопряженный оператор, $U(t)=e^{itA}$. Тогда \par
(a) для любого $t\in \R$ оператор $U(t)$ унитарен и $U(t+s)=U(t)U(s)$
для любых $t$, $s\in \R$; \par
(b) если $\psi\in {\cal H}$ и $t\rightarrow t_0$, то $U(t)\psi\rightarrow
U(t_0)\psi$; \par
(c) для любого $\psi\in D(A)$ имеем $\frac{U(t)\psi-\psi}{t}\rightarrow
iA\psi\, (t\rightarrow 0)$; \par
(d) если $\exists \lim_{t\rightarrow 0}\frac{U(t)\psi-\psi}{t}$, то
$\psi \in D(A)$.
\end{Trm}
Операторнозначная функция $U(t)$, удовлетворяющая (a) и (b), называется
сильно непрерывной однопараметрической унитарной группой.
\begin{Trm}
(теорема Стоуна). Пусть $U(t)$ --- сильно непрерывная однопараметрическая
унитарной группа. Тогда существует самосопряженный оператор $A$ такой,
что $U(t)=e^{itA}$.
\end{Trm}
Оператор $A$ называется инфинитезимальным генератором группы $U(t)$. \par
Положим $U(t)=e^{\frac{-it\hat{H}}{\hbar}}$ (он называется оператором
эволюции), $$\hat F(t)=U(t)^+\hat FU(t).$$ Это самосопряженный оператор,
определенный на $U(t)^+D(\hat F)$ и имеющий те же спектральные свойства,
что и $\hat F$. \par
В [6] уравнению (\ref{evol}) придается следующий смысл.
Рассмотрим два вектора $\psi_1$ и $\psi_2$. Положим
$$\overline{F}(t)=\langle \hat{F}U(t)\psi_1, \, U(t)\psi_2\rangle.$$
Предположим, что $\psi_1(t)\stackrel{def}{=} U(t)\psi_1\in D(\hat{F})$,
$\hat{F}\psi_2(t)\in D(\hat{H})$ непрерывно зависит от $t$. Докажем, что
тогда $\overline{F}(t)$ дифференцируема. В самом деле,
$$\frac{\overline{F}(s+t)-\overline{F}(s)}{t}=\frac{\langle (\hat{F}
U(t)-U(t)\hat{F})\psi_1(s), \, U(t)\psi_2(s)\rangle}{t}=$$
$$=\left\langle\hat{F}\frac{U(t)-I}{t}\psi_1(s), \, U(t)\psi_2(s)\right\rangle
-\left\langle\frac{U(t)-I}{t}\hat{F}\psi_1(s), \, U(t)\psi_2(s)\right\rangle.$$
Так как $\lim _{t\rightarrow 0}\frac{U(t)-I}{t}\varphi=-\frac{i}{\hbar}
\hat{H}\varphi$ для любого $\varphi\in D(\hat{H})$, то
$$\frac{U(t)-I}{t}\hat{F}\psi_1(s)\rightarrow -\frac{i}{\hbar}\hat{H}
\hat{F}\psi_1(s)\, (t\rightarrow 0)$$ и, значит,
$$\lim\limits_{t\rightarrow 0}\left\langle\frac{U(t)-I}{t}\hat{F}
\psi_1(s), \, U(t)\psi_2(s)\right\rangle=-\frac{i}{\hbar}\langle
\hat{H}\hat{F}\psi_1(s), \, \psi_2(s)\rangle.$$
Так как оператор $\hat{F}$, вообще говоря, является неограниченным, то
$\lim_{t\rightarrow 0}\hat{F}\frac{U(t)-I}{t}\psi_1(s)$ может не
существовать. Воспользовавшись самосопряженностью $\hat{F}$, получаем
$$\left\langle\hat{F}\frac{U(t)-I}{t}\psi_1(s), \, U(t)\psi_2(s)\right\rangle=
\left\langle\frac{U(t)-I}{t}\psi_1(s), \, \hat{F}U(t)\psi_2(s)\right\rangle.$$
По предположению, $\hat{F}U(t)\psi_2(s)=\hat{F}\psi_2(t+s)$ непрерывно
зависит от $t$ и, значит,
$$\lim \limits_{t\rightarrow 0}\left\langle\hat{F}\frac{U(t)-I}{t}\psi_1(s),
\, U(t)\psi_2(s)\right\rangle=-\frac{i}{\hbar}\langle\hat{F}\hat{H}\psi_1(s),
\, \psi_2(s)\rangle.$$ Следовательно, $$\left.\frac{d\overline{F}(t)}{dt}\right|
_{t=s}=\left\langle\frac{i}{\hbar}(\hat{H}\hat{F}-\hat{F}\hat{H})\psi_1(s), \,
\psi_2(s)\right\rangle=\left\langle\frac{i}{\hbar}(\hat{H}\hat{F}(s)-
\hat{F}(s)\hat{H})\psi_1, \, \psi_2\right\rangle=\frac{i}{\hbar}\overline
{[\hat{H}, \, \hat{F}]}.$$
В частности, если $\hat H$ имеет полную систему собственных векторов
$\{\psi_n\}$, при этом $\psi_n(t)\in D(\hat F)$ и $\hat F\psi_n(t)\in
D(\hat H)$ непрерывно зависят от $t$ для любого $n$, то уравнение
(\ref{evol}) можно рассматривать как уравнение для матричных элементов
$\hat F(t)$. \par
Описанная картина эволюции называется гейзенберговской. В ней вектор
состояния не меняется, а меняется наблюдаемая. Математическое
ожидание величины $F$ определяется формулой
$$\langle F(t)\rangle=\langle\psi, \, F(t)\psi\rangle=\langle\psi, \, U(t)^+
FU(t)\psi\rangle=\langle\psi(t), \, F(0)\psi(t)\rangle,$$ где $\psi(t)=
U(t)\psi$. Для $\psi(t)$ получаем уравнение Шредингера
\begin{align}
\label{shre_eq}
i\hbar\frac{d}{dt}\psi(t)=\hat{H}\psi(t)
\end{align}
с начальным условием $\psi(t)|_{t=0}=\psi$. Такое
представление зависимости физических величин от времени называется
представлением Шредингера. Операторы наблюдаемых остаются неизменными,
а вектор состояния эволюционирует. Выбирая координатное представление,
получаем уравнение Шредингера для точечной частицы, движущейся в поле
с потенциалом $U(\vec{r})$:
\begin{align}
\label{shre_coord}
i\hbar \frac{\partial\psi(\vec{r}, \, t)}{\partial t}=-\frac{\hbar^2}{2m}
\Delta \psi(\vec{r}, \, t)+U(\vec{r})\psi(\vec{r}, \, t).
\end{align}
Уравнение (\ref{shre_coord}) называется полным уравнением Шредингера.
Если вектор состояния при $t=0$ является собственным вектором оператора
Гамильтона $$\hat{H}\psi_E=E\psi_E,$$ то зависимость
от времени в соответствии с (\ref{shre_eq}) имеет вид
$$\psi(t)=e^{-\frac{iEt}{\hbar}}\psi_E.$$ В этом
случае среднее значение любой наблюдаемой $F$ не зависит от $t$:
$$\langle F\rangle=\langle \psi(t), \, F(0)\psi(t)\rangle=\langle
\psi_E, \, F(0)\psi_E\rangle.$$ Такое состояние называется стационарным. \par
Для определения изменения математических ожиданий наблюдаемых удобнее
использовать картину Гейзенберга. Для операторов импульса и координаты
точечной частицы уравнение (\ref{evol}) принимает вид
$$\dot{\vec{p}}=\frac{i}{\hbar}\left[H(t), \, \vec{p}(t)\right],$$
$$\dot{\vec{r}}=\frac{i}{\hbar}\left[H(t), \, \vec{r}(t)\right].$$
Пусть $H(\vec{p}, \, \vec{r})=\frac{p^2}{2m}+V(\vec{r})$. Тогда
$$[\hat{H}, \, \hat{\vec{p}}]=[V(\vec{r}), \, \hat{\vec{p}}]=
\left[V(\vec{r}), \, \frac{\hbar}{i}\bigtriangledown\right]=i\hbar\bigtriangledown
V(\vec{r}),$$
$$[\hat{H}, \, \hat{\vec{r}}]=\left[\frac{p^2}{2m}, \, \hat{\vec{r}}\right]=
-i\hbar\frac{\vec{p}}{m}.$$ Значит, $$\frac{i}{\hbar}[H(t), \, \vec{p}(t))]
=\frac{i}{\hbar}U(t)^+[H, \, \vec{p}]U(t)=-U(t)^+\bigtriangledown
V(\vec{r})U(t)=-\bigtriangledown V(\vec{r}(t)).$$
Докажем последнее равенство. Если $A_1, \, \dots, \, A_n$ в некотором
представлении имеют вид операторов умножения на $\lambda_1, \, \dots, \,
\lambda_n$ соответственно, то, по определению, $f(A_1, \, \dots, \, A_n)$
в этом представлении имеет вид оператора умножения на $f(\lambda_1, \,
\dots, \, \lambda_n)$, то есть если $A_j=U_0^+\lambda_jU_0$, то
$f(A_1, \, \dots, \, A_n)=U_0^+f(\lambda_1, \, \dots, \, \lambda_n)U_0$.
Пусть $A_j(t)=U(t)^+A_jU(t)=U(t)^+U_0^+\lambda_jU_0U(t)$. Тогда
$$U(t)^+f(A_1, \, \dots, \, A_n)U(t)=U(t)^+U_0^+f(\lambda_1, \, \dots, \,
\lambda_n)U_0U(t)=f(A_1(t), \, \dots, \, A_n(t)).$$\par
Аналогично
$$\frac{i}{\hbar}[H(t, \, \vec{r}(t))]=\frac{\vec{p}(t)}{m}.$$ Таким образом,
уравнения эволюции имеют вид $$\dot{\vec{p}}=-\bigtriangledown V(\vec{r}(t)),
\; \dot{\vec{r}}=\frac{\vec{p}(t)}{m}.$$ Значит, если состояние $\psi$
таково, что $\psi(t)\in D(\hat p)\cap D(\hat x)$, а $\hat x\psi(t)$
и $\hat p\psi(t)$ принадлежат $D(\hat H)$ и непрерывно зависят от времени,
то для средних от $\hat p$ и $\hat x$ по состоянию $\psi$ имеем
$$\frac{d}{dt}\langle\vec{p}\rangle=-\langle\bigtriangledown V(\vec{r}(t))
\rangle, \; \frac{d}{dt}\langle\vec{r}\rangle=\frac{1}{m}\langle\vec{p}(t)
\rangle.$$ Тем самым для средних получаем классические уравнения движения
(это утверждение называется теоремой Эренфеста).
\section{Теория рассеяния}
\label{scat_th}
Здесь приводятся основные результаты абстрактной теории рассеяния.
\begin{Def}
Пусть $\hat H_0$, $\hat H$ --- самосопряженные операторы на гильбертовом
пространстве ${\cal H}$, ${\cal H}_{{\rm ac}}(\hat H_0)$ --- абсолютно
непрерывное подпространство оператора $\hat H_0$. Положим
$$D_{\pm}=\{\psi\in {\cal H}_{{\rm ac}}(\hat H_0):\exists \, s{\text -}
\lim \limits_{t\rightarrow \mp \infty}e^{it\hat H}e^{-it\hat H_0}\psi\}.$$
Волновыми операторами\footnote{Обозначения взяты из [3]; в [12] операторы
$\Omega_\pm$ обозначаются как $W_{\mp}$.} $\Omega_{\pm}=\Omega_{\pm}
(\hat H, \, \hat H_0)$ называются операторы, заданные на $D_{\pm}$ формулами
$$\Omega_{\pm}\psi=s{\text -}\lim \limits_{t\rightarrow \mp \infty}
e^{it\hat H}e^{-it\hat H_0}\psi.$$
\end{Def}
Обозначим ${\cal H}_{\pm}=\ran \, \Omega_{\pm}$. \par
Волновые операторы обладают следующими свойствами [3, \S XI.3],
[12, \S IV.1, IV.2]:
\begin{enumerate}
\item Подпространства $D_{\pm}$ замкнуты и инвариантны относительно
$\hat H_0$.
\item Операторы $\Omega_{\pm}:D_{\pm}\rightarrow {\cal H}_{\pm}$ являются
унитарными.
\item ${\cal H}_{\pm}\subset {\cal H}_{{\rm ac}}(\hat H)$.
\item ${\cal H}_{\pm}$ являются инвариантными подпространствами $\hat H$,
$\Omega_{\pm}[D_{\pm}\cap D(\hat H_0)]\subset D(\hat H)$ и $(\Omega_{\pm})
^{-1}\hat H|_{{\cal H}_{\pm}}\Omega_{\pm}=\hat H_0|_{D_{\pm}}$.
\end{enumerate}
Из последнего утверждения видно, что $\hat H|_{{\cal H}_{\pm}}$ и $\hat H_0
|_{D_{\pm}}$ унитарно эквивалентны, так что теория рассеяния применяется
в спектральном анализе. В следующей главе она будет использоваться для
нахождения спектральной меры и кратности спектра оператора Штурма--Лиувилля
с потенциалом, сходящимся к некоторым значениям при $x\rightarrow \pm
\infty$. \par
Если $D_+=D_-={\cal H}_{{\rm ac}}(\hat H_0)$, то говорят, что волновые
операторы $\Omega_{\pm}$ существуют. Если ${\cal H}_+={\cal H}_-=
{\cal H}_{{\rm ac}}(\hat H)$, то говорят, что $\Omega_{\pm}$ полны.
Если $\Omega_{\pm}(\hat H, \, \hat H_0)$ существуют, то они полны
тогда и только тогда, когда существуют $\Omega_{\pm}(\hat H_0, \,
\hat H)$ [3, \S XI.3]. Однако обычно доказать полноту труднее,
чем существование, так как оператор $\hat H_0$, как правило, имеет
более простой вид, чем $\hat H$, и легче исследуется. В [3, гл. XI]
приводятся различные достаточные условия существования и полноты
волновых операторов. \par
Дадим физическую интерпретацию волновым операторам. Пусть $\,\hat H_0=-\Delta$
--- гамильтониан свободной частицы, $V$ --- оператор взаимодействия
частицы с полем, $\hat H=\hat H_0+V$. Тогда для любого $t\in \R$
эволюция свободной (взаимодействующей) частицы определяется оператором
$e^{-it\hat H_0}$ (соответственно $e^{-it\hat H}$). Пусть $\psi \in
{\cal H}$. Тогда если существует $\psi_{{\rm in}}=\lim \limits_{t\rightarrow
-\infty}e^{it\hat H}e^{-it\hat H_0}\psi$, то $e^{-it\hat H}\psi_{{\rm in}}
-e^{-it\hat H_0}\psi\rightarrow 0(t\rightarrow -\infty)$, то есть
$\psi$ является итогом эволюции свободной частицы при $t\in (-\infty, \, 0]$
из того состояния, которое при эволюции взаимодействующей с полем
частицы переходит в состояние $\psi_{{\rm in}}$. Существование $\Omega_+$
означает, что любое свободное состояние соответствует некоторому
исходному состоянию взаимодействующей системы. Аналогично существование
$\Omega_-$ означает, что при эволюции взаимодействующей системы в течение
длительного времени можно получить любое свободное состояние. \par
Пусть $\Omega_{\pm}$ полны и $\hat H$ не имеет сингулярного спектра.
Тогда ${\cal H}$ является прямой суммой абсолютно непрерывного
подпространства ${\cal H}_{{\rm ac}}$ и подпространства ${\cal H}_{{\rm p}}$,
натянутого на собственные векторы $\hat H$, которому соответствуют
связанные состояния. В этом случае полнота $\Omega_{\pm}$ означает, что
любое несвязанное взаимодействующее состояние (то есть принадлежащее
${\cal H}_{{\rm ac}}$) в отдаленном прошлом и отдаленном будущем
выглядит как свободное. \par
Следующее утверждение [12, гл. IV, теорема 2.2] дает достаточное условие
существования и полноты волновых операторов для $\hat H_0=-\Delta$, $\hat H=-\Delta+V$.
\begin{Trm}
Пусть $V\in L_\infty(\R^n)$ удовлетворяет неравенству $$|V(x)|\le C(1+|x|)^{-1-\varepsilon},$$
где $\varepsilon>0$. Тогда волновые операторы $\Omega_\pm(\hat H, \, \hat H_0)$ существуют
и полны. Кроме того, сингулярный спектр оператора $\hat H$ пуст.
\end{Trm}
В следующей главе будут явно построены волновые операторы для $\hat H_0=-\frac{d^2}{dx^2}
+U_0\theta(x)$ и $\hat H=\hat H_0+V$ (в предположении, что функция $V$ достаточно быстро
убывает на бесконечности). При этом на конечном отрезке $V$ может иметь сингулярности.
\begin{Def}
Пусть ${\cal H}_+={\cal H}_-$. Оператором рассеяния называется оператор
$S:D_+\rightarrow D_-$, заданный равенством $S=(\Omega_-)^{-1}\Omega_+$.
\end{Def}
Из свойств волновых операторов следует, что $S$ является унитарным.
Кроме того, если $D_+=D_-={\cal H}$, то $S$ перестановочен с $\hat H_0$.
Из последнего утверждения выводится следующая теорема [12, \S IV.2]:
\begin{Trm}
\label{s_matrix_decomp}
Пусть ${\cal H}=L_2(\R^n)$, $\hat H_0=-\Delta$, $\hat H=\hat H_0+V$
и волновые операторы существуют и полны. Тогда существует унитарная
измеримая операторная функция $S(\lambda)$ на $\R_+$, называемая матрицей
рассеяния, такая, что $S$ разлагается в прямой интеграл операторов
$S(\lambda)$ в $L_2(\R_+, \, L_2(S^1))$ ($S^1$ --- единичная сфера), то
есть $S(\lambda)$ --- унитарный оператор в $L_2(S^1)$, измеримо
зависящий от $\lambda\in \R_+$, и оператор $S$ действует по формуле
$$(Sf)(\lambda)=S(\lambda)f(\lambda),$$
где $f(\lambda)$ --- вектор-функции на $\R_+$ со значениями в $L_2(S^1)$.
\end{Trm}
В случае $n=1$ матрица рассеяния --- это матрица $2\times 2$. \par
Пусть $\hat H$ и $\hat H_0$ --- самосопряженные операторы в ${\cal H}$
и $\Omega_{\pm}$ всюду определены. Предположим, что задано оснащение
$B_+\subset {\cal H}\subset B_-$ такое, что $B_-$ содержит полную
систему обобщенных собственных векторов $\psi_0(m)$ оператора $\hat H_0$
с собственными значениями $E(m)$. Для каждого $\varphi_{\pm}\in \Omega
_{\pm}B_+$ положим $$\langle \psi_\pm(m), \, \varphi_{\pm}\rangle =
\langle \psi_0(m), \, (\Omega_\pm)^{-1}\varphi_{\pm}\rangle$$ (формально
это можно записать как $\psi_\pm(m)=\Omega_\pm\psi_0(m)$). Тогда
отображения $\varphi_\pm\mapsto \hat \varphi_\pm(\cdot)=\langle \psi
_\pm(\cdot), \, \varphi_{\pm}\rangle\in L_2(M, \, d\mu)$ являются
изометрическими операторами, определенными на ${\cal H}_\pm$, и
$\psi_{\pm}(m)$ являются обобщенными собственными векторами оператора
$\hat H$ с собственными значениями $E(m)$. \par
Предположим, что $\hat H_0=-\Delta$, $\hat H=-\Delta+V$, где $V$ ---
ограниченная функция, и что волновые операторы существуют. Тогда
[12, \S IV.3] векторы $\psi_\pm(m)$ удовлетворяют уравнению
Липпмана--Швингера $$\psi_\pm(m)=\psi_0(m)+(E(m)\pm i0-\hat H_0)^{-1}
V\psi_\pm(m).$$
\section{Одномерное движение}
\label{1dimension}
В этой главе изучаются спектральные свойства оператора Штурма--Лиувилля.
В простейших случаях (гармонический осциллятор, свободное движение
и потенциальная стенка) обобщенное преобразование Фурье и спектр
описываются явным образом. Далее приводятся способы задания оператора
Штурма--Лиувилля с сингулярными потенциалами в соответствии с [14],
[15]: метод регуляризации, метод квадратичных форм и метод аппроксимации
гладкими потенциалами. После этого исследуются обобщенные собственные
векторы и спектр оператора Штурма--Лиувилля на прямой. Явно строится
оснащение $L_2(\R)$, при котором обобщенные собственные векторы являются
регулярными функциями, удовлетворяющими уравнению Шредингера. Для
операторов с растущим потенциалом доказывается обобщение теоремы о
нулях $k$-й собственной функции на случай сингулярного потенциала.
Затем описывается спектр оператора Штурма--Лиувилля, у которого
потенциал достаточно быстро стремится к некоторому числу при $x\rightarrow
\pm \infty$: доказывается отсутствие сингулярного спектра и с помощью
теории рассеяния описывается абсолютно непрерывный спектр. Также
описывается построение оператора Штурма--Лиувилля с сингулярным
периодическим потенциалом. \par
В конце главы рассказывается о когерентных состояниях для гармонического
осциллятора, строится представление Баргмана--Фока и доказывается
полнота системы состояний $|\alpha _{lm}\rangle$ с $\alpha _{lm}=
\sqrt{\pi}(l+im)$, $l$, $m\in \Z$.
\subsection{Достаточные условия самосопряженности оператора Шредингера}
В координатном представлении гамильтониан задается выражением
$-\frac{\hbar^2}{2m}\frac{d^2}{dx^2}+V(x)$. Нужно задать его область
определения, так чтобы соответствующий оператор был существенно
самосопряжен. \par
{\bf Пример 1.} Докажем, что оператор $-\frac{d^2}{dx^2}$ существенно
самосопряжен на $C_0^\infty(\R)$. Для этого покажем, что его
индексы дефекта равны 0 и воспользуемся теоремой \ref{krit_samosopr}.
Пусть $f\in \ran(\hat H+i)^{\bot}$, то есть для любого $g\in C_0^\infty$
выполнено $\langle f, \, -g''+ig\rangle=0$, то есть $f$ удовлетворяет
уравнению $f''+if=0$ в обобщенном смысле. Отсюда $f=C_1e^{(1-i)x/\sqrt{2}}
+C_2e^{(-1+i)x/\sqrt{2}}$. Так как $f\in L_2(\R)$, то $C_1=C_2=0$ и
$\ran(\hat H+i)^{\bot}=\{0\}$. Аналогично $\ran(\hat H-i)^{\bot}=\{0\}$. \par
{\bf Пример 2.} Рассмотрим оператор $-\frac{d^2}{dx^2}$ в пространстве $L_2(\R_+)$.
Если в качестве его области определения взять $C_0^\infty(0, \, \infty)$,
то $\ran(\hat H+i)^{\bot}=\{Ce^{(-1+i)x/\sqrt{2}}:C\in \C\}$ и $\ran(\hat
H-i)^{\bot}=\{Ce^{(-1-i)x/\sqrt{2}}:C\in \C\}$, то есть $n_{\pm}=1$.
Пусть теперь $g\in C^\infty(\R_+)$, $\supp g\subset [0, \, R]$ для
некоторого $R>0$. Тогда если $f(x)=e^{(-1+i)x/\sqrt{2}}$ удовлетворяет
равенству $\langle f, \, -g''+ig\rangle=0$, то, интегрируя по частям,
получаем
\begin{align}
\label{ex2_part_int}
0=-f^*g'|_0^\infty+{f^*}'g|_0^\infty=f(0)^*g'(0)-f'(0)^*g(0).
\end{align}
Поэтому если в качестве $D(\hat H)$ взять $\{\varphi\in C^\infty(\R_+):
\supp \varphi\subset [0, \, R], \, R>0,\; \alpha \varphi(0)+\beta
\varphi'(0)=0\}$ для некоторых $(\alpha, \, \beta)\in \R^2\backslash\{0\}$
(в этом случае $\hat H$ будет симметрическим), то из (\ref{ex2_part_int})
следует, что $\alpha f(0)-\beta f'(0)=0$, что невозможно при вещественных
$\alpha$ и $\beta$. Аналогично доказывается, что второй индекс дефекта
равен 0, и $\hat H$ существенно самосопряжен на $D(\hat H)$. Если
$\alpha=0$, то получаем граничное условие Неймана $\varphi'(0)=0$, если
$\beta=0$, то получаем граничное условие Дирихле $\varphi(0)=0$. \par
{\bf Пример 3.} Рассмотрим оператор $-\frac{d^2}{dx^2}$ в пространстве $L_2[-1, \, 1]$
(аналогично можно брать любой другой отрезок). Если в качестве его
области определения взять $C_0^\infty(-1, \, 1)$, то индексы дефекта
равны $n_{\pm}=2$. Рассмотрим его расширение с областью определения
$D(\hat H)=\{ \varphi\in C^\infty([-1, \, 1]):\varphi(\pm 1)=0\}$
(это граничные условия Дирихле). С
помощью интегрирования по частям доказывается, что $\hat H$ является
симметрическим. Докажем, что его индексы дефекта равны 0. Пусть
$f\in (\ran(\hat H+i))^{\bot}$, то есть для
любого $g\in D(\hat H)$ выполнено $$\int \limits _{-1}^{1} f(x)^*
(-g''(x)+ig(x))\, dx=0$$ и $f$ удовлетворяет уравнению
$f''+if=0$ в обобщенном смысле. Отсюда следует, что
$$f(x)=C_1e^{(1-i)x/\sqrt{2}}+C_2e^{(-1+i)x/\sqrt{2}}.$$ Пусть $g\in
C^\infty([-1, \, 1])$ и $g(\pm 1)=0$. Тогда
$$0=\int \limits _{-1}^1f(x)^*(g''(x)-ig(x))\, dx=$$
$$=f(x)^*g'(x)|_{-1}^{1}-f'(x)^*g(x)|_{-1}^{1}+\int \limits_{-1}^1
(f''(x)^*-if(x)^*)g(x)\, dx=f(x)^*g'(x)|_{-1}^1,$$ поскольку
$f''(x)+if(x)=0$ для любого $x$ и $g(\pm 1)=0$. Так как $g'(\pm 1)$
может быть любой, то $f(\pm 1)=0$. Таким образом, получаем систему
уравнений на коэффициенты $C_1$, $C_2$ $$\left\{\begin{array}{l}
C_1\exp\left(\frac{1}{\sqrt{2}}-\frac{i}{\sqrt{2}}\right)
+C_2\exp\left(-\frac{1}{\sqrt{2}}+\frac{i}{\sqrt{2}}\right)
=0, \\ C_1\exp\left(-\frac{1}{\sqrt{2}}+\frac{i}
{\sqrt{2}}\right)+C_2\exp\left(\frac{1}{\sqrt{2}}
-\frac{i}{\sqrt{2}}\right)=0,\end{array}\right.$$
откуда $C_1=C_2=0$, так как определитель системы отличен от нуля.
Значит, $f=0$. Аналогично доказывается, что $(\ran(\hat H-i))^{\bot}
=\{0\}$. Таким образом, $n_\pm=0$ и оператор $\hat H$ существенно самосопряжен.
Аналогично можно показать, что если $D(\hat H)$ задается граничными
условиями Неймана $\varphi'(\pm 1)=0$, то $\hat H$ будет существенно
самосопряжен. \par
Теперь приведем достаточные условия существенной самосопряженности
оператора $-\frac{d^2}{dx^2}+V(x)$. В некоторых случаях можно применять
теорему \ref{vust}. В частности, имеет место
\begin{Sta}
Пусть $V=V_1+V_2$, где $V_1\in L_2(\R)$, $V_2\in L_\infty(\R)$. Тогда
$-\frac{d^2}{dx^2}+V(x)$ существенно самосопряжен на $C_0^\infty(\R)$.
\end{Sta}
\begin{proof}
Утверждение доказывается точно так же, как теорема X.15 в [3].
Положим $D(V)=\{\psi\in L_2(\R):V\psi\in L_2(\R)\}$. Тогда $V$ самосопряжен
на $D(V)$. Так как $\|V\psi\|_{L_2}\le \|V_1\|_{L_2}\|\psi\|_{L_\infty}
+\|V_2\|_{L_\infty}\|\psi\|_{L_2}$, то $D(V)\supset C_0^\infty(\R)$.
Для любого $a>0$ существует $b>0$ такое, что $\|\psi\|_{L_\infty}
\le a\|\psi''\|_{L_2}+b\|\psi\|_{L_2}$ [3, т. 2, теорема IX.28],
откуда $\|V\psi\|_{L_2}\le a\|V_1\|_{L_2}\|\psi''\|_{L_2}+b\|V_1\|_{L_2}
\|\psi\|_{L_2}+\|V_2\|_{L_\infty}\|\psi\|_{L_2}$, поэтому при достаточно
малых $a$ выполнено условие теоремы \ref{vust}.
\end{proof}
В [12, гл. II, \S 1] дается следующее достаточное условие самосопряженности.
\begin{Trm}
\label{62trm1}
Пусть $\, Q(x)$ --- положительная четная неубывающая при $x\ge 0$ функция,
удовлетворяющая условию $$\int \limits_{-\infty}^\infty \frac{dx}{\sqrt
{Q(2x)}}=\infty.$$ Тогда если $V(x)\ge -Q(x)$, то $\hat H$ существенно
самосопряжен на $C_0^\infty(\R)$.
\end{Trm}
Для случая полупрямой в [3, т. 2, \S X.1] приводится критерий Вейля
существенной самосопряженности оператора $\hat H=-\frac{d^2}{dx^2}+V(x)$ на
$C_0^\infty(0, \, +\infty)$ (функция $V$ непрерывна на $(0, \, \infty)$).
Скажем, что $V(x)$ удовлетворяет случаю {\it предельной окружности} на
бесконечности (соответственно в нуле), если для некоторого $\lambda\in \C$
все решения уравнения $-\psi''+V\psi=\lambda\psi$ квадратично интегрируемы на
бесконечности (соответственно в нуле). Если $V(x)$ не отвечает условию
предельной окружности на бесконечности (соответственно в нуле), то
$V(x)$ отвечает условию {\it предельной точки}.
\begin{Trm}
(критерий Вейля.) Пусть $\, V$ --- непрерывная функция на $(0, \, +\infty)$.
Оператор $\hat H=-\frac{d^2}{dx^2}+V$ существенно самосопряжен тогда и
только тогда, когда $V$ отвечает условию предельной точки и в нуле, и
на бесконечности.
\end{Trm}
Из этого критерия там же выводятся следующие утверждения:
\begin{enumerate}
\item если в окрестности бесконечности $V(x)\ge -cx^2$, а в окрестности
нуля $V(x)\ge \frac{3}{4x^2}$, то $\hat H$ существенно самосопряжен на
$C_0^\infty(\R)$;
\item если в окрестности нуля $V(x)\le (3/4-\varepsilon)x^{-2}$ при
некотором $\varepsilon>0$, то $\hat H$ не является существенно
самосопряженным;
\item если в окрестности бесконечности  $V(x)=-cx^\alpha$, где $c>0$,
$\alpha>2$, то $\hat H$ не является существенно самосопряженным.
\end{enumerate}
{\bf Пример.} Пусть $\hat H=-\frac{d^2}{dx^2}+\frac{1}{x^2}$, $V(x)=-x^3$. Тогда $\hat H$ и
$V$ являются существенно самосопряженными
на $C_0^\infty(0, \, +\infty)$, а $\hat H+V$ не является существенно
самосопряженным.
\subsection{Гармонический осциллятор}
У гармонического осциллятора гамильтониан имеет вид
$$H=\frac{p^2}{2m}+\frac{m\omega^2 x^2}{2}.$$
Оператор $\hat{H}=-\frac{\hbar^2}{2m}\frac{d^2}{dx^2}+\frac{m\omega^2 x^2}{2}$
с областью определения $S(\R)$ является положительно определенным, то есть
$\langle \psi, \, \hat{H}\psi\rangle>0$ для любого $\psi\in D(\hat{H})
\backslash \{0\}$. Отсюда следует, что собственные значения $\hat{H}$
положительны. \par
Найдем собственные значения и собственные векторы оператора $\hat H$.
Положим $x_0=\sqrt{\frac{\hbar}{m\omega}}$, $p_0=\frac{\hbar}{x_0}$ и
представим гамильтониан в виде
$$H=\frac{\hbar\omega}{2}\left[\left(\frac{x}{x_0}\right)^2+
\left(\frac{p}{p_0}\right)^2\right].$$ Введем операторы
$$a=\frac{1}{\sqrt{2}}\left(\frac{x}{x_0}+i\frac{p}{p_0}\right) \;
(\text{оператор уничтожения}),$$
$$a^+=\frac{1}{\sqrt{2}}\left(\frac{x}{x_0}-i\frac{p}{p_0}\right) \;
(\text{оператор рождения}).$$
Тогда $\hat{H}=\hbar\omega(a^+a+1/2)$. Операторы $a$ и $a^+$ подчиняются
перестановочному соотношению $[a, \, a^+]=1$. \par
\begin{Sta}
Пусть $\psi$ --- собственный вектор оператора $\hat H$ с собственным
значением $E$. Тогда состояния $a\psi$ и $a^+\psi$ являются собственными
векторами оператора $\hat{H}$, соответствующими собственным значениям
$E-\hbar\omega$ и $E+\hbar\omega$ соответственно.
\end{Sta}
\begin{proof}
Докажем, что $[\hat H, \, a]=-\hbar\omega a$ и $[\hat H, \, a^+]=\hbar
\omega a^+$. В самом деле,
$$[\hat H, \, a]=\hat Ha-a\hat H=\hbar\omega a^+aa+\frac{\hbar\omega}{2}a-
\hbar\omega aa^+a-\frac{\hbar\omega}{2}a=$$
$$=\hbar\omega(a^+a-aa^+)a=\hbar\omega[a^+, \, a]a=-\hbar\omega a.$$
Второе равенство доказывается аналогично. \par
Теперь найдем $\hat Ha\psi$ и $\hat Ha^+\psi$. Имеем
\begin{align}
\label{6han}
\hat Ha\psi=a\hat H\psi+[\hat H, \, a]\psi=Ea\psi-\hbar\omega a\psi
=(E-\hbar\omega)a\psi,
\end{align}
$$\hat Ha^+\psi=a^+\hat H\psi+[\hat H, \, a^+]\psi=Ea^+\psi+\hbar
\omega a^+\psi=(E+\hbar\omega)a^+\psi.$$
\end{proof}
Пусть $\psi$ --- собственный вектор $\hat{H}$. Тогда
$$\langle \hbar\omega a^+a\psi, \, \psi\rangle +\frac{\hbar\omega}{2}
\langle\psi , \, \psi\rangle =E\langle \psi , \, \psi\rangle .$$
Так как $\langle a^+\varphi , \, \psi\rangle =\langle \varphi , \, a\psi\rangle$
для любых $\varphi$, $\psi\in S(\R)$, то $\langle \hbar\omega a^+a\psi , \,
\psi\rangle =\hbar\omega\langle a\psi, \, a\psi\rangle\ge 0$. Следовательно,
$E\ge \frac{\hbar\omega}{2}$. \par
Докажем, что существует единственный собственный вектор $\psi_0$ оператора
$\hat{H}$ с собственным значением $\frac{\hbar\omega}{2}$. Действительно,
в координатном представлении
$$a=\frac{1}{\sqrt{2}}\left(q+\frac{d}{dq}\right), \; a^+=\frac{1}{\sqrt{2}}
\left(q-\frac{d}{dq}\right),$$
где $q=\frac{x}{x_0}$. Из условий $E\ge \frac{\hbar\omega}{2}$ и (\ref{6han})
следует, что $a\psi_0=0$, то есть
$$\left(q+\frac{d}{dq}\right)\psi_0=0.$$
Это уравнение имеет единственное решение (с точностью до умножения
на константу) $$\psi_0(q)\sim e^{-q^2/2}.$$ Из условия $a\psi_0=0$ и
равенства $\hat{H}=\hbar\omega(a^+a+1/2)$ следует, что $\hat{H}\psi_0
=\frac{\hbar\omega}{2}\psi_0$. \par
Пусть $E$ --- некоторое собственное значение $\hat{H}$, $\psi$ ---
соответствующая собственная функция. Тогда найдется
такое $l_0\in \Z_+$, что $a^l\psi\ne 0$ для любого $l\le l_0$ и $a^{l_0+1}
\psi=0$. Значит, $a^{l_0}\psi=\const\cdot\psi_0$, поэтому $E=\hbar\omega(l_0+
1/2)$. Функции $\psi_n=N_na^{+n}\psi_0\in S(\R)$ ($N_n$ выбираются из
условия нормировки) являются собственными
для оператора $\hat{H}$ с собственными значениями $\hbar\omega(n+1/2)$.
Они имеют вид $\psi_n(q)=e^{-q^2/2}H_n(q)$, где $H_n$ --- полиномы
Эрмита. Следовательно, множество $\{\psi_n\}_{n=0}^\infty$ образуют
полную ортонормированную систему в $L_2(\R)$ (это факт из функционального
анализа). Таким образом, доказана
\begin{Trm}
Оператор $\hat{H}=\frac{\hat{p}^2}{2m}+\frac{m\omega^2 \hat{x}^2}{2}$
имеет полную ортонормированную систему собственных векторов $|n\rangle$
с собственными значениями $E_n=\hbar\omega(n+1/2)$.
\end{Trm}
В частности, отсюда следует, что собственный вектор $|n\rangle$ с собственным
значением $E_n$ ровно один.
\begin{Cor}
Оператор, являющийся обратным к $\hat H$, является оператором
Гильберта--Шмидта.
\end{Cor}
\begin{Sta}
Выполнены равенства $$a|n\rangle=\sqrt{n}|n-1\rangle, \; a^+|n\rangle =
\sqrt{n+1}|n+1\rangle .$$
\end{Sta}
\begin{proof}
То, что $a|n\rangle=\const|n-1\rangle$, следует из предыдущей теоремы.
Для нахождения константы вычислим $\|a|n\rangle\|$.
Найдем $a^+a|n\rangle$: $$a^+a|n\rangle =\frac{\hat H}{\hbar\omega}|n\rangle
-\frac12|n\rangle =\frac{1}{\hbar\omega}(\hbar\omega(n+1/2))|n\rangle
-\frac12|n\rangle =n|n\rangle .$$ Отсюда $$\langle an|an\rangle =\langle n|a^+a|n\rangle
=n\langle n|n\rangle =n.$$ Направление $|n\rangle$
выбираем по $|n-1\rangle$ так, чтобы константа имела знак +, и получаем первое равенство.
Докажем второе равенство:
$$a^+|n\rangle=\frac{a^+a}{\sqrt{n+1}}|n+1\rangle=\frac{n+1}{\sqrt{n+1}}
|n+1\rangle=\sqrt{n+1}|n+1\rangle.$$
\end{proof}
Отсюда следует, что функции $\psi_n$, соответствующие векторам $|n\rangle$
в координатном представлении, имеют вид $$\psi_n(q)=\frac{1}{\sqrt{2^nn!
\sqrt{\pi}}}\left(q-\frac{d}{dq}\right)^ne^{-q^2/2}.$$ Они имеют $n$
вещественных нулей и являются четными при четных $n$ и нечетными при
нечетных $n$. Следовательно, средние значения импульса и координаты
в состоянии $|n\rangle$ равны 0. \par
Оператор $\hat H$ коммутирует с оператором $F$ преобразования Фурье, поэтому
$\psi_n$ являются также собственными функциями $F$. Так как $F$ унитарный и порождает
группу $\{1, \, F, \, F^2, \, F^3\}$, изоморфную $\Z_4$, то $\hat H$ имеет группу симметрий
$\Z_4$. Оператор $P=F^2$ имеет вид $P\varphi(x)=\varphi(-x)$. Любой оператор Штурма--Лиувилля
с четным потенциалом коммутирует с $P$ и поэтому имеет группу симметрий
$\Z_2$. Таким образом, оператор Гамильтона для одномерного гармонического
осциллятора имеет более широкую группу симметрий, чем в общем случае четного потенциала.
\subsection{Решение спектральной задачи для одномерного оператора
Шредингера (общая схема)} Пусть $\hat H$ --- самосопряженный
оператор, заданный некоторым дифференциальным выражением.
Требуется найти спектр $\hat H$, его кратность и обобщенные
собственные векторы. \par {\it Шаг 1.} Строим оснащение $L_2$ с
помощью оператора $K$ такого, что $K^{-1}$ является оператором
Гильберта--Шмидта. Например, если $\hat H$ --- дифференциальный
оператор с гладкими коэффициентами, то в качестве $K$ можно взять
оператор $-\frac{d^2}{dx^2}+x^2$. \par {\it Шаг 2.} Пользуемся
теоремой \ref{trm_maurin1}: для почти всех точек спектра
существует набор $\{e_k(\lambda)\}_{k=1}^{N(\lambda)}\subset {\cal
H}_-$ такой, что если $\varphi\in {\cal H}_+$ и $\hat H\varphi \in
{\cal H}_+$, то
\begin{align}
\label{6maurin}
\langle e_k(\lambda), \, \hat H\varphi\rangle=\lambda\langle
e_k(\lambda), \, \varphi\rangle
\end{align}
Если $\hat H=-\frac{d^2}{dx^2}+V(x)$, где $V$ --- гладкая функция,
а оснащение построено по оператору $-\frac{d^2}{dx^2}+x^2$, то
$e_k(\lambda)$ удовлетворяют уравнения $-\frac{d^2}{dx^2}e_k(\lambda)
+V(x)e_k(\lambda)=\lambda e_k(\lambda)$ в обобщенном смысле, так как
пространство $D$ основных функций непрерывно вкладывается в ${\cal H}_+$.
Поскольку оператор $\hat H$ эллиптический, то $e_k(\lambda)$ являются классическими
решениями. В случае, когда $V\in L_\infty^{{\rm loc}}$, регулярность
$e_k$ доказана в [12]. Позже будет доказана регулярность $e_k(\lambda)$
для сингулярных потенциалов. \par
В том случае, когда $V$ гладкий всюду, кроме дискретного множества точек,
часто бывает удобно найти общее решение уравнения Шредингера на каждом
участке гладкости, а затем построить решение на всей прямой так, чтобы
в особых точках выполнялось условие склейки (например, если $V$ имеет
скачок, то это условие $C^1$-гладкости, а если $V$ содержит дельта-функцию,
то это условие непрерывности и условие на значение скачка производной). \par
{\it Шаг 3.} Выбираем те решения, которые удовлетворяют условию роста
на бесконечности. Например, если $V$ --- гладкая функция и $K=-\frac{d^2}
{dx^2}+x^2$, то ${\cal H}_+$ содержит пространство Шварца $S(\R)$, при этом
топология на $S(\R)$ сильнее, чем на ${\cal H}_+$, поэтому ${\cal H}_-
\subset S'(\R)$. Отсюда, в частности, следует, что решения не могут
экспоненциально возрастать. В [6], [7] и [12] доказаны более точные
ограничения на рост. \par
{\it Шаг 4.} После того, как найдено множество $\Lambda$ тех $\lambda\in
\R$, для которых удалось найти $e_k(\lambda)$, нужно проверить, что каждое
$\lambda$ действительно принадлежит спектру (до этого проверялось
только необходимое условие). В некоторых случаях это удается проверить
с помощью утверждения \ref{bound} или его следствия. \par
{\it Шаг 5.} На тех участках, где оба решения уравнения подходят,
нужно доказать, что спектр действительно является двукратным. В
некоторых простейших случаях удается явно построить преобразование
Фурье и таким образом точно определить кратность спектра (например,
если $\hat H=-\frac{d^2}{dx^2}$ или $\hat H=-\frac{d^2}{dx^2}+U_0\theta(x)$).
Если $\hat H=-\frac{d^2}{dx^2}+V(x)$, где $V(x)$ достаточно быстро
стремится к некоторым значениям при $x\rightarrow \pm \infty$, то
задачу можно свести к одному из предыдущих случаев, применяя теорию
рассеяния. А именно, можно доказать существование волновых операторов и
отсюда вывести, что кратность не меньше, чем 2. \par
{\it Шаг 6.} В некоторых случаях еще удается доказать отсутствие
сингулярного спектра. Тогда на непрерывном участке спектра меру можно
выбрать равной мере Лебега. \par
Если имеется симметрия, то есть $\hat H$ сильно коммутирует с некоторым
самосопряженным оператором $\hat A$, то поиск обобщенных собственных
векторов может упроститься, так как их можно искать как общие собственные
векторы операторов $\hat H$ и $\hat A$.
\subsection{Свободная частица}
Свободная частица описывается гамильтонианом $\hat H=-\frac{\hbar^2}
{2m}\frac{d^2}{dx^2}$ в $L_2(\R)$. Как было показано, этот оператор
существенно самосопряжен на $C_0^\infty(\R)$. Кроме того, можно показать [3, \S
IX.7], что область определения его замыкания состоит из функций $\varphi\in L_2(\R)$
таких, что $\varphi ''\in L_2(\R)$ (производная в смысле обобщенных функций). \par
Рассмотрим оператор $F$ преобразования Фурье: $$(F\varphi)(k)=\frac{1}{\sqrt{2\pi}}
\int e^{ikx}\varphi(x)\, dx.$$ Это унитарный
оператор из $L_2(\R)$ на $L_2(\R)$, переводящий $\hat H$ в оператор
умножения на $\frac{\hbar^2k^2}{2m}$. Значит, $\hat H$ имеет двукратный
лебегов спектр, совпадающий с $\R_+$. Функции $e^{ikx}$, $k\in \R\backslash
\{0\}$, являются обобщенными собственными векторами $\hat H$,
построенными по оснащению $S(\R)\subset L_2(\R)\subset S'(\R)$, с собственными
значениями $E_k=\frac{\hbar^2 k^2}{2m}$. Из формулы обращения для преобразования
Фурье
$$\int dx\, e^{ikx}\int dk'\, e^{-ik'x}\varphi(k')=\varphi(k)$$
получаем условие ортогональности $$\int e^{i(k-k')x}\, dx=\delta(k-k').$$
Заметим также, что преобразование Фурье переводит оператор импульса $\frac{\hbar}{i}
\frac{d}{dx}$ в оператор умножения на $\hbar k$. \par
Найдем общее решение уравнения Шредингера для свободной частицы:
$$i\hbar\frac{\partial \psi}{\partial t}=-\frac{\hbar^2}{2m}\frac
{\partial^2\psi}{\partial x^2}.$$ Сделаем преобразование Фурье:
$$\psi(x, \, t)=\frac{1}{\sqrt{2\pi}}\int e^{ikx}\tilde \psi(k, \,
t)\, dk, \; \; \; \tilde \psi(k, \, t)=\frac{1}{\sqrt{2\pi}}\int e^{-ikx}
\psi(x, \, t)\, dk.$$
Отсюда получаем уравнение $$\frac{\partial}{\partial t}\tilde \psi(k, \, t)=
-\frac{i\hbar k^2}{2m}\tilde \psi(k, \, t).$$ Его решение, принадлежащее
$L_2(\R)$, имеет вид
$$\tilde \psi(k, \, t)=C(k)e^{-\frac{i\hbar k^2}{2m}t},$$ где
$C(\cdot)\in L_2(\R)$. Значит, $$\psi(x, \, t)=\frac{1}{\sqrt{2\pi}}
\int C(k)e^{-\frac{i\hbar k^2}{2m}t}e^{ikx}dk=\frac{1}{\sqrt{2\pi}}
\int C(k)e^{i(kx-\omega_k t)}dk,$$ где $\omega_k=\frac{\hbar k^2}{2m}=\frac{E_k}{\hbar}$.
Тогда $\psi\in L_2(\R)$ и $\|\psi\|
=\|\tilde \psi\|$. Выберем $C(\cdot)$ так, чтобы $\|\psi\|=1$. Тогда
$C(k)$ является амплитудой вероятности обнаружить частицу имеющей
импульс $p=\hbar k$. Функция $\psi(x, \, t)$ называется волновым пакетом. \par
Найдем изменение во времени для средних значений и дисперсии волнового
пакета. Уравнения для $\hat p(t)$ и $\hat x(t)$ имеют вид
$$\frac{d\hat{p}}{dt}=\frac{i}{\hbar}[\hat{H}, \, \hat{p}]=
\frac{i}{\hbar}\left[\frac{\hbar^2}{2m}\hat{p}^2, \, \hat{p}\right]=0,$$
$$\frac{d\hat{x}}{dt}=\frac{i}{\hbar}[\hat{H}, \, \hat{x}]=
\frac{i}{\hbar}\left[\frac{\hbar^2}{2m}\hat{p}^2, \, \hat{x}\right]=
\frac{\hat{p}}{m},$$ откуда $\hat{p}(t)=\hat{p}(0)$, $\hat{x}(t)=
\hat{x}(0)+\frac{\hat{p}(0)}{m}t$. Усредняя по начальному состоянию,
получаем $$\langle x\rangle_t=\langle x\rangle_0+\frac{\langle p\rangle
_0}{m}t, \; \langle p\rangle_t=\langle p\rangle_0.$$ Дисперсии имеют
вид $$(\delta_t p)^2=\left\langle(p-\langle p\rangle)^2\right\rangle_t
=(\delta_0 p)^2,$$ $$(\delta_t x)^2=\left\langle(x-\langle x\rangle)^2
\right\rangle_t=(\delta_0x)^2+(\delta_0p)^2\frac{t^2}{m^2}+\left(\langle
xp+px\rangle_0-2\langle x\rangle_0\langle p\rangle_0\right)\frac{t}{m},$$
то есть происходит расплывание волнового пакета.
\subsection{Движение на полупрямой и на отрезке}
Как было показано, для задания оператора $\hat H=-\frac{\hbar^2}{2m}
\frac{d^2}{dx^2}$ в $L_2(\R_+)$ нужно граничное условие в 0. Мы
будем рассматривать только условия Дирихле и Неймана. \par
Найдем обобщенные собственные векторы $\hat H$. Для этого сначала
построим оснащение $L_2(\R_+)$. В обоих случаях оно будет задаваться
с помощью оператора $K=-\frac{d^2}{dx^2}+x^2$ на функциях $\varphi\in S(\R)$
таких, что $\varphi(0)=0$ (если $\hat H$ задается граничным условием
Дирихле) или $\varphi'(0)=0$ (если $\hat H$ задается граничным условием
Неймана). Тогда $K^{-1}$ является оператором Гильберта--Шмидта. Это
следует из того, что собственные функции оператора $-\frac{d^2}{dx^2}
+x^2$ в $L_2(\R)$ являются либо четными, либо нечетными. Нечетные
собственные функции образуют базис в подпространстве всех нечетных
функций из $L_2(\R)$, а значит, их ограничение на $\R_+$ образует
базис в пространстве $L_2(\R_+)$, состоящий из функций, удовлетворяющих
граничному условию Дирихле. Аналогично ограничение четных собственных
функций на $\R_+$ образует базис из функций, удовлетворяющих
граничному условию Неймана. Так как в обоих случаях собственные
значения образуют арифметическую прогрессию, то $K^{-1}$ является
оператором Гильберта--Шмидта. \par
Найдем обобщенные собственные функции для $\hat H$. Рассмотрим случай
граничных условий Дирихле. Положим $k^2=\frac{2mE}{\hbar^2}$,
соответствующую обобщенную собственную функцию обозначим $\psi_k$.
Пусть $\varphi$ --- гладкая функция с носителем в $[0, \, R)$ и
$\varphi(0)=0$. Тогда $\varphi\in {\cal H}_+$.
Рассмотрим уравнение $-\eta''-k^2\eta=\varphi$ с начальными условиями
$\varphi(R)=\varphi'(R)=0$. Тогда $\eta$ --- гладкая функция с носителем в
$[0, \, R)$. Если $\eta(0)=0$, то $\eta\in {\cal H}_+$. Тогда из
условий $\eta\in {\cal H}_+$ и $\hat H\eta\in {\cal H}_+$ получаем, что
$\langle \psi_k, \, (\hat H-E)\eta\rangle=0$. \par
Докажем, что $\eta(0)=0$ тогда и только тогда, когда $\int \limits_0^\infty
\sin kx \, \varphi(x)\, dx=0$. В самом деле, $$\int \limits_0^\infty
\sin kx \, \varphi(x)\, dx=\int \limits_0^R\sin kx (-\eta''(x)-k^2
\eta(x))\, dx=$$ $$=-\sin kx\, \eta'(x)|_0^R+\cos kx\, \eta(x)|_0^R+
\int \limits_0^R (-(\sin kx)''-k^2\sin kx)\eta(x)\, dx=\eta(0).$$
Пусть $\varphi_1\in C_0^\infty([0, \, +\infty))$, $\varphi_1(0)=0$ и
$\int \limits_0^\infty \sin kx \, \varphi_1(x)\, dx=1$. Тогда любая
функция $\varphi\in C_0^\infty([0, \, +\infty))$, удовлетворяющая условию
$\varphi(0)=0$, имеет вид $$\varphi=\varphi_0+\varphi_1\int \limits_0^\infty
\varphi(x)\sin kx\, dx,$$ где $\varphi_0(0)=0$ и $$\int \limits_0^\infty
\varphi_0(x)\sin kx \, dx=0.$$ Значит, $\varphi_0=-\eta''-k^2\eta$ для
некоторого $\eta\in {\cal H}_+$,
и $$\langle \psi_k, \, \varphi\rangle=\langle \psi_k, \, -\eta''-k^2\eta
\rangle+\langle \psi_k, \, \varphi_1\rangle \int \limits _0^\infty \sin kx \,
\varphi(x)\, dx=c_k\int \limits _0^\infty \sin kx \, \varphi(x)\, dx,$$
то есть $\psi_k(x)=c_k\sin kx$. \par
Аналогично доказывается, что для граничных условий Неймана $\psi_k(x)=
\tilde c_k\cos kx$. \par
{\bf Замечание.} Точно так же доказывается, что если $V$ ---
гладкая функция на $\R_+$, то оператор $\hat H=-\frac{\hbar^2}{2m}
\frac{d^2}{dx^2}+V$ имеет полную систему обобщенных собственных
функций, каждая из которых является регулярным решением уравнения
$-\frac{\hbar^2}{2m}f''+Vf=Ef$ с начальным условием $f(0)=0$ ($f'(0)=0$)
в случае граничных условий Дирихле (соответственно Неймана). В главе
\ref{3dim_probl} это утверждение будет обобщено на случай, когда $V$
имеет особенности (в том числе и в нуле). \par
Из утверждения \ref{bound} следует, что $\R_+$ совпадает со спектром
$\hat H$. Так как при каждом $k>0$ обобщенные собственные векторы
образуют одномерное пространство, то спектр всюду однократный. \par
Выберем обобщенные собственные векторы так, чтобы они удовлетворяли
условию ортогональности. Рассмотрим случай условий Дирихле. Докажем, что
для любой функции $\varphi\in C_0^\infty(0, \, \infty)$ и для любого $p>0$
выполнено равенство
\begin{align}
\label{intsindelta}
\int \limits_0^\infty \frac{2}{\pi}\sin px\left(
\int \limits _0^\infty\sin p'x \, \varphi(p')\, dp'\right)\, dx=\varphi(p).
\end{align}
Доказательство аналогично доказательству формулы обращения для
преобразования Фурье. Так как функция $\varphi$ имеет компактный носитель,
то по теореме Фубини для любого $R>0$ $$\int \limits_0^R \frac{2}{\pi}
\sin px\left(\int \limits _0^\infty\sin p'x \,\varphi(p')\, dp'\right)\, dx=
\int \limits_0^\infty \frac{2}{\pi}\varphi(p')\left(\int\limits_0^R\sin px
\sin p'x\, dx\right)\, dp'=$$ $$=\frac{2}{\pi}\int \limits_0^\infty
\varphi(p')\frac{\sin R(p-p')}{2(p-p')}\, dp'-\frac{2}{\pi}\int \limits
_0^\infty\varphi(p')\frac{\sin R(p+p')}{2(p+p')}\, dp'.$$ Так как
$p>0$ и $\varphi$ имеет компактный носитель, то по теореме Римана--Лебега
второе слагаемое стремится к 0 при $R\rightarrow \infty$. Выберем $P>p$
так, чтобы $\supp \varphi\subset [0, \, P]$. Тогда $$\frac{1}{\pi}
\int \limits_0^\infty \varphi(p')\frac{\sin R(p'-p)}{p'-p}\, dp'=
\frac{1}{\pi}\int \limits_0^P \varphi(p')\frac{\sin R(p'-p)}{p'-p}\, dp'=$$
$$=\frac{1}{\pi}\int \limits_0^P \frac{\varphi(p')-\varphi(p)}{p'-p}
\sin R(p'-p)\, dp+\frac{1}{\pi}\varphi(p)\int \limits_0^P \frac{\sin
R(p'-p)}{p'-p}\, dp'.$$ Так как функция $\varphi$ гладкая, то первое слагаемое
стремится к 0 в силу теоремы Римана--Лебега. Во втором слагаемом
делаем замену переменной $y=R(p'-p)$ и получаем $$\frac{1}{\pi}\varphi(p)
\int \limits_{-Rp}^{R(P-p)}\frac{\sin y}{y}\, dy\underset{R\rightarrow \infty}
{\rightarrow} \varphi(p),$$ по формуле для интеграла Дирихле. \par
Из этой формулы выводится равенство Парсеваля (так же, как для преобразования
Фурье). Таким образом, отображение $\varphi(p)\mapsto \sqrt{\frac{2}{\pi}}
\int \limits_0^\infty \varphi(p)\sin px\, dp$ продолжается до
изометрического отображения из $L_2(\R_+)$ в $L_2(\R_+)$. \par
Аналогично доказывается, что в случае граничных условий Неймана
$\left\{\sqrt{\frac{2}{\pi}}\cos px:p>0\right\}$ является нормированной
системой обобщенных собственных векторов. \par
Докажем, что продолжения отображений
\begin{align}
\label{r_plus_odd}
\varphi(p)\mapsto \sqrt{\frac{2}{\pi}}\int \limits_0^\infty
\varphi(p)\sin px\, dp
\end{align}
и
\begin{align}
\label{r_plus_even}
\varphi(p)\mapsto \sqrt{\frac{2}{\pi}}\int \limits_0^\infty \varphi(p)
\cos px\, dp
\end{align}
на $L_2(\R_+)$ сюръективны. Для этого достаточно показать, что образы
отображений, заданных на подпространстве нечетных (четных) функций
из $C_0^\infty(\R)$ по формуле (\ref{r_plus_odd}) (соответственно (\ref
{r_plus_even})), будет плотен в подпространстве нечетных (соответственно
четных) функций в $L_2(\R)$. Это вытекает из следующих утверждений:
\begin{enumerate}
\item Пространства четных и нечетных функций взаимно ортогональны.
\item Преобразование Фурье сюръективно.
\item Подпространство четных (нечетных) функций инвариантно относительно
преобразования Фурье.
\end{enumerate}
Теперь рассмотрим действие $\hat H=-\frac{\hbar^2}{2m}\frac{d^2}{dx^2}$ на
$\{\psi\in C^\infty[-a/2, \, a/2]:\psi(\pm a/2)=0\}$ (граничные условия
Дирихле). Было показано, что на этой области определения $\hat H$
существенно самосопряжен. Найдем его спектр. \par
Сначала найдем дискретный спектр и собственные векторы. Можно показать,
что замыканием $\hat H$ является оператор $-\frac{\hbar^2}{2m}\frac{d^2}
{dx^2}$ (дифференцирование почти всюду) на $\{\psi\in W^2_2[-a/2, \,
a/2]:\psi(\pm a/2)=0\}$. Значит, если $\psi$ --- собственная функция
с собственным значением $E$, то $\psi''+\frac{2mE}{\hbar^2}\psi=0$ и
$\psi(\pm a/2)=0$. Решив эту краевую задачу, получаем $\psi_n(x)=
c\sin \frac{\pi n}{a}(x+a/2)$, где $n\in \N$, $c\in \C$. Отсюда
$\frac{2mE_n}{\hbar^2}=\left(\frac{\pi n}{a}\right)^2$ и $E_n=\frac{\pi^2
\hbar^2 n^2}{2ma^2}$. Так как система $\{\psi_n\}$ полна в $L_2(-a/2, \,
a/2)$, то спектр $\hat H$ чисто дискретный и равен $\{E_n\}_{n=1}^\infty$.
\subsection{Потенциальная стенка}
Рассмотрим в $L_2(\R)$ оператор $\hat H=-\frac{\hbar^2}{2m}\frac{d^2}
{dx^2}+U_0\theta(x)$, где $U_0>0$, $\theta(\cdot)$ --- функция Хевисайда.
Так как потенциал ограничен, то по теореме \ref{vust} оператор $\hat H$
существенно самосопряжен на $C_0^\infty(\R)$ и самосопряжен на множестве $D_{\max}$
функций из $L_2(\R)$, вторая обобщенная производная которых также принадлежит
$L_2(\R)$. Найдем его спектр и обобщенные собственные векторы. \par
Так как $\hat H$ является неотрицательным, то его спектр содержится
в $\R_+$. \par
Оснащение ${\cal H}=L_2(\R)$ строим с помощью оператора $K=-\frac{d^2}{dx^2}
+x^2+U_0\theta(x)$. Позже будет доказано, что если потенциал $V$
является локально ограниченной функцией, то обобщенные собственные
векторы абсолютно непрерывны вместе со своими производными и удовлетворяют
уравнению $-\frac{\hbar^2}{2m}\psi''+V\psi=E\psi$. В частности, в
данном примере получаем условие склейки в 0: $\psi(+0)=\psi(-0)$,
$\psi'(+0)=\psi'(-0)$. Кроме того, из вида оператора $K$ следует, что
функции из ${\cal H}_-$ не могут экспоненциально возрастать. \par
Пусть $E>U_0$. Тогда $$\psi(x)=\left\{ \begin{array}{l} c_1e^{ikx}+
c_2e^{-ikx}, \; x<0, \\ c_3e^{iqx}+c_4e^{-iqx}, \; x>0,\end{array}\right.$$
где $k=\sqrt{\frac{2mE}{\hbar^2}}$, $q=\sqrt{\frac{2m(E-U_0)}{\hbar^2}}$,
а $c_3$ и $c_4$ однозначно определяются по $c_1$ и $c_2$ из условий
гладкости. \par
Докажем, что при $E>U_0$ спектр является двукратным. Для этого явно построим
унитарный оператор из $L_2(\R)$ на некоторое подпространство ${\cal H}$,
переводящий оператор умножения в ограничение $\hat H$ на это подпространство.
\par
Положим
\begin{align}
\label{stenka_psij}
\psi_1(x, \, q)=\left\{ \begin{array}{l} c_1(q)\sin qx, \; x>0, \\
c_1(q)\frac{q}{k}\sin kx,\; x<0,\end{array} \right. \; \psi_2(x, \, q)=\left\{
\begin{array}{l} c_2(q)\cos qx, \; x>0, \\ c_2(q)\cos kx,\; x<0,\end{array} \right.
\end{align}
где $k=\sqrt{q^2+\frac{2mU_0}{\hbar^2}}$, $$c_1(q)=\frac{1}{\sqrt{2(1+\frac{|q|}
{k})}}, \; \; \; c_2(q)=\frac{1}{\sqrt{2(1+\frac{k}{|q|})}}.$$ Пусть $\varphi\in
C_0^\infty(0, \, +\infty)$. Докажем, что
\begin{align}
\label{intsincosdelta}
\int \limits_{-\infty}^\infty \int \limits_{-\infty}^\infty \psi_j(x, \, q)
\psi_j(x, \, q')\varphi(q')\, dq'\, dx=\varphi(q),
\end{align}
где $\varphi\in C_0^\infty(\R\backslash \{0\})$
--- нечетная (четная) функция в случае $j=1$ (соответственно 2). Рассмотрим
случай $j=1$ и нечетной функции $\varphi$. При $q>0$
$$\int \limits_{-\infty}^\infty dx\, \psi_1(x, \, q)
\int \limits_{-\infty}^\infty \psi_1(x, \, q')\varphi(q')\, dq'=$$ $$=\int \limits_{-\infty}
^\infty dx\, \psi_1(x, \, q)\int\limits_0^\infty (\psi_1(x, \, -q')\varphi(-q')+\psi_1(x, \,
q')\varphi(q'))\, dq'=$$
$$=2\int \limits_{-\infty}^0 c_1(q)\frac{q}{k}\sin kx \, dx \int \limits_0
^\infty c_1(q')\frac{q'}{k'}\sin k'x\, \varphi(q')\,
dq'+2\int \limits_0 ^\infty c_1(q)\sin qx\, dx\int
\limits_0^\infty c_1(q')\sin q'x\, \varphi(q')\, dq'.$$ По формуле
(\ref{intsindelta}), второе слагаемое равно $2c_1^2(q)\varphi(q)$.
Положив $$\eta(k)=\left\{\begin{array}{l}0, \; 0\le k\le
\sqrt{\frac{2mU_0} {\hbar^2}}, \\ \varphi(q), \;
k>\sqrt{\frac{2mU_0}{\hbar^2}}\end{array} \right.$$ и сделав
замену переменной, получаем, что по формуле (\ref{intsindelta})
первое слагаемое равно $2c_1^2(q)\frac{q}{k}
\eta(k)= 2c_1^2(q)\frac{q}{k}\varphi(q)$, и в сумме получаем
$\varphi(q)$. Если $q<0$, то получаем
$$-2c_1^2\left(1+\frac{|q|}{k}\right)\varphi(|q|)=
-\varphi(|q|)=\varphi(q)$$ в силу нечетности $\varphi$. \par
Аналогично рассматривается случай с $j=2$ и четной функцией
$\varphi$. \par Теперь докажем, что для любой функции $\varphi\in
C_0^\infty(0, \, \infty)$ и для любого $q>0$ выполнено равенство
\begin{align}
\label{intsincos0}
\int\limits_0^\infty \cos qx\, dx\int\limits_0^\infty \sin q'x\, \varphi(q')
\, dq'+\int \limits _{-\infty}^0\cos kx\, dx\int\limits_0^\infty
\frac{q'}{k'}\sin k'x\,\varphi(q')\, dq'=0.
\end{align}
Первое слагаемое равно $$\lim \limits_{N\rightarrow \infty}\int \limits
_0^N \cos qx\, dx\int\limits_0^\infty \sin q'x\, \varphi(q')\, dq'=
\lim \limits_{N\rightarrow \infty}\int\limits_0^\infty \varphi(q')
\, dq'\int\limits_0^N\cos qx \sin q'x\, dx=$$ $$=\lim
\limits_{N\rightarrow \infty}\int\limits_0^\infty
\varphi(q')\left(\frac{1-\cos(q'+q)N}{2(q'+q)} +\frac{1-\cos
(q'-q)N}{2(q'-q)}\right)\, dq'=$$ $$=\lim_{N\rightarrow\infty}
\int \limits_0^\infty \frac12
\varphi(q')\frac{\cos(q'-q)N-\cos(q'+q)N} {q'+q}\,
dq'+\lim\limits_{N\rightarrow \infty}\int \limits_0^\infty
\varphi(q')\sin^2\frac{N(q'-q)}{2}\frac{2q'}{q'^2-q^2}\, dq'.$$
Первое слагаемое равно 0 по теореме Римана--Лебега. \par
Аналогично рассмотрев второе слагаемое в (\ref{intsincos0}),
получаем
$$-\lim\limits_{N\rightarrow\infty}\int \limits_0^\infty \frac{q'}{k'}
\varphi(q')\sin^2\frac{N(k'-k)}{2}\frac{2k'}{k'^2-k^2}\, dq'=
-\lim\limits_{N\rightarrow\infty}\int \limits_0^\infty \varphi(q')
\sin^2\frac{N(k'-k)}{2}\frac{2q'}{q'^2-q^2}\, dq'$$ (так как $q'^2-q^2=
k'^2-k^2$). Итак, нужно доказать, что $$\lim\limits_{N\rightarrow \infty}
\int \limits_0^\infty \left(\sin^2\frac{N(q'-q)}{2}-\sin^2\frac{N(k'-k)}{2}
\right)\frac{2q'}{q'^2-q^2}\varphi(q')\, dq'=$$ $$=\lim\limits_{N
\rightarrow \infty}\int \limits_0^\infty(\cos N(k'-k)-\cos N(q'-q))
\frac{q'}{q'^2-q^2}\varphi(q')\, dq'=0.$$
Для любого $\delta>0$ интеграл по $\R_+\backslash (q-\delta , \, q+\delta)$
стремится к 0 при $N\rightarrow \infty$ по теореме Римана--Лебега.
Так как функция $\varphi$ гладкая, то $$\left|\int \limits_{q-\delta}
^{q+\delta}(\cos N(k'-k)-\cos N(q'-q))\frac{1}{q'-q}\left(\frac{q'\varphi(q')}
{q'+q}-\frac{\varphi(q)}{2}\right)\, dq'\right|\le C\delta,$$ где
$C$ не зависит от $N$. Значит, остается доказать, что при достаточно малых
$\delta>0$ $$\lim \limits_{N\rightarrow \infty}\int \limits_{q-\delta}^{q+\delta}
\frac{\cos N(\sqrt{q'^2+\alpha^2}-\sqrt{q^2+\alpha^2})-\cos N(q'-q)}{q'-q}\, dq'=$$
$$=\lim \limits _{N\rightarrow \infty}\int \limits_{-\delta}^{\delta}\frac{\cos
N(\sqrt{(q+x)^2+\alpha^2}-\sqrt{q^2+\alpha^2})-\cos Nx}{x}\, dx=0$$
(где $\alpha^2=\frac{2mU_0}{\hbar^2}$). Найдем этот предел с помощью вычетов,
показав, что $$\int \limits _{|z|=\delta, \, 0\le {\rm arg}\, z\le \pi}
\frac{\exp(iN(\sqrt{(q+z)^2+\alpha^2}-\sqrt{q^2+\alpha^2}))}{z}\, dz
\underset{N\rightarrow \infty}{\rightarrow 0}.$$ При малых $|z|$ функция
$\sqrt{(q+z)^2+\alpha^2}-\sqrt{q^2+\alpha^2}$ аналитична и представляется
в виде ряда $\sum \limits_{n=1}^\infty a_nz^n$, где все $a_n$
вещественны и $a_1>0$. Поэтому при $|z|=\delta$, ${\rm arg}\, z=\varphi$
$${\rm Im}\left(\sqrt{(q+z)^2+\alpha^2}-\sqrt{q^2+\alpha^2}\right)=
\sum \limits_{n=1}^\infty
a_n\delta^n\sin{n\varphi}=$$ $$=\left(a_1+ \sum
\limits_{n=2}^\infty a_n\delta^{n-1}\frac{\sin n\varphi}{\sin
\varphi}\right)\delta \sin \varphi\ge \frac{a_1}{2}\delta \sin
\varphi$$ при достаточно малых $\delta$. Значит, $$\left|\int
\limits_{|z|=\delta, \, 0\le {\rm arg}\, z\le
\pi}\frac{\exp(iN(\sqrt{(q+z)^2+\alpha^2}-
\sqrt{q^2+\alpha^2}))}{z}\, dz\right|\le \int \limits_0^\pi
e^{-N\frac {a_1}{2}\delta\sin \varphi}\, d\varphi,$$ а эта
величина стремится к 0 при $N\rightarrow \infty$. \par Итак,
(\ref{intsincos0}) доказано. Отсюда следует, что
\begin{align}
\label{intsincos01}
\int\limits_0^\infty \cos qx\, dx\int\limits_{-\infty}^\infty
\sin q'x\, \varphi(q')\, dq'+\int \limits _{-\infty}^0\cos kx\, dx\int
\limits_{-\infty}^\infty \frac{q'}{k'}\sin k'x\, \varphi(q')\, dq'=0
\end{align}
для любого $q\in \R\backslash\{0\}$ и для любой нечетной функции
$\varphi\in C_0^\infty(\R\backslash\{0\})$. \par
Пусть ${\cal H}_1$ --- подпространство нечетных функций из $L_2(\R)$,
${\cal H}_2$ --- подпространство четных функций. Эти подпространства
ортогональны друг другу. Определим операторы $U_j:{\cal H}_j\rightarrow
{\cal H}$ по формуле $$(U_j\varphi)(x)=\int \limits_{-\infty}
^\infty \psi_j(x, \, q)\varphi(q)\, dq$$ для $\varphi\in C_0^\infty
(\R\backslash\{0\})$. Из (\ref{intsincosdelta}) и (\ref{intsincos01})
следует, что эти операторы являются унитарными на свой образ $\tilde{{\cal
H}}_j$ и эти образы ортогональны. \par
С помощью дифференцирования интеграла по параметру и интегрирования по частям
доказывается, что $\tilde {\cal H}_j\subset D_{\max}$ и оператор
$$U_j\left(\frac{\hbar^2}{2m}q^2+U_0\right)U_j^{-1}$$ совпадает
с ограничением $\hat H$ на $\tilde {\cal H}_j$. \par
Следовательно, при $E>U_0$ абсолютно непрерывная часть спектра $\hat H$
двукратна и спектральная мера эквивалентна мере Лебега. \par
Пусть $0<E<U_0$. Тогда при $x>0$ решение уравнения $\hat H\psi=E\psi$
имеет вид $\psi(x, \, q)=c_1e^{qx}+c_2e^{-qx}$, где $q=\sqrt{\frac{2m(U_0-E)}
{\hbar^2}}$. Так как $\psi$ не может экспоненциально возрастать, то
$c_1=0$. При $x<0$ $\psi(x, \, q)=c_3e^{ikx}+c_4e^{-ikx}$, где
$k=\sqrt{\frac{2mE}{\hbar^2}}$, а $c_3$ и $c_4$ однозначно находятся
из условий $C^1$-гладкости. А именно, $$\psi(x)=\left\{ \begin{array}{l}
e^{-qx},\; x>0, \\ \cos kx-\frac{q}{k}\sin kx, \; x<0.\end{array}
\right.$$ Положим $q_0=\sqrt{\frac{2mU_0}{\hbar^2}}$. Докажем, что
существует гладкая функция $c(q)$, не обращающаяся в 0 на $(0, \, q_0)$
такая, что для любой функции $\varphi \in C_0^\infty(0, \, q_0)$
выполнено равенство $$\int \limits_{-\infty}^\infty dx\, \psi(x, \, q)
\int \limits_0^{q_0}\psi(x, \, q')\varphi(q')\, dq'=c(q)\varphi(q).$$
В самом деле, $$\int \limits_0^\infty e^{-qx}e^{-q'x}\, dx=\frac{1}{q'+q},$$
а $$\int \limits_{-N}^0 \psi(x, \, q)\psi(x, \, q')\, dx=\int \limits_0^N
\left(\cos kx+\frac{q}{k}\sin kx\right)\left(\cos k'x+\frac{q'}{k'}
\sin k'x\right)\, dx,$$ так что $$A_N(q):=\int \limits_{-N}^\infty dx\,
\psi(x, \, q)\int \limits_0^{q_0}\psi(x, \, q')\varphi(q')\, dq'=
\int \limits_{0}^{q_0}\frac{\varphi(q')}{q'+q}\, dq+\int \limits
_0^{q_0}dq'\, \varphi(q')\int \limits_0^N \cos kx\cos k'x\, dx+$$
$$+\int \limits _0^{q_0}dq'\, \varphi(q')\int \limits_0^N \frac{q}{k}
\sin kx\, \frac{q'}{k'}\sin k'x\, dx+\int \limits _0^{q_0}dq'\, \varphi(q')
\int \limits_0^N \cos kx\, \frac{q'}{k'}\sin k'x\, dx+$$ $$+\int \limits
_0^{q_0}dq'\, \varphi(q')\int \limits_0^N \frac{q}{k}\sin kx\cos k'x\, dx.$$
Второе и третье слагаемые сходятся
при $N\rightarrow \infty$ к $c_1(q)\varphi(q)$ и $c_2(q)\varphi(q)$,
где $c_1$ и $c_2$ --- некоторые положительные гладкие функции.
Четвертое и пятое слагаемые равны соответственно $$\int \limits_0^{q_0}
\sin^2\frac{N(k'-k)}{2}\frac{2q'}{k'^2-k^2}\varphi(q')\, dq'+\underset{N\rightarrow
\infty}{o(1)}$$ и $$\int \limits_0^{q_0}\sin^2\frac{N(k'-k)}{2}
\frac{2q}{k^2-k'^2}\varphi(q')\, dq'+\underset{N\rightarrow \infty}{o(1)}.$$
Воспользовавшись тем, что $k'^2-k^2=q^2-q'^2$, получаем $$A_N(q)=(c_1(q)+c_2(q))
\varphi(q)+\int \limits_0^{q_0}\left(
2\sin^2\frac{N(k'-k)}{2}\frac{q'-q}{k'^2-k^2}+\frac{1}{q'+q}\right)
\varphi(q')\, dq'+\underset{N\rightarrow \infty}{o(1)}=$$ $$=(c_1(q)+c_2(q))
\varphi(q)+\int \limits_0^{q_0}\left((1-\cos N(k'-k))\frac{(-1)}{q'+q}+
\frac{1}{q'+q}\right)\varphi(q')\, dq'+\underset{N\rightarrow \infty}{o(1)}=$$
$$=(c_1(q)+c_2(q))\varphi(q)+\int \limits_0^{q_0}\frac{\cos N(k'-k)}{q'+q}\varphi(q')
\, dq'+\underset{N\rightarrow \infty}{o(1)}=(c_1(q)+c_2(q))\varphi(q)
+\underset{N\rightarrow \infty}{o(1)}$$ в силу теоремы Римана--Лебега. \par
Таким образом, при $E<U_0$ спектр однократный и на абсолютно непрерывной
части спектра мера эквивалентна мере Лебега. \par
Отсутствие сингулярного спектра при $E>0$ будет доказана позже для
более общего случая.
\begin{Sta}
\label{utv_stenka}
Пусть $\psi_1$, $\psi_2$ заданы формулой (\ref{stenka_psij}). Тогда
отображение $$C_0^\infty(\R\backslash\{0\})\ni\varphi\stackrel{U}
{\mapsto}\int \limits_{-\infty}^\infty (\psi_1(x, \, q)+i\psi_2(x, \, q))
\varphi(q)\, dq$$ продолжается по непрерывности до унитарного оператора
из $L_2(\R)$ на $\tilde{{\cal H}}_1\oplus \tilde{\cal{H}}_2$.
\end{Sta}
\begin{proof}
Пусть $\varphi\in C_0^\infty(\R\backslash\{0\})$. Представим ее в виде
суммы четной и нечетной функций: $\varphi=\varphi_1+\varphi_2$, $\varphi_j
\in {\cal H}_j$, $j=1, \; 2$. Так как $\psi_j$ является нечетной (четной)
по $q$ при $j=1$ (соответственно 2), то $U\varphi=U_1\varphi_1+U_2\varphi_2$.
Так как ${\cal H}_1\bot {\cal H}_2$ и $\tilde{\cal{H}}_1\bot \tilde{\cal{H}}
_2$, то $$\|\varphi\|^2=\|\varphi_1+\varphi_2\|^2=\|\varphi_1\|^2+
\|\varphi_2\|^2=\|U_1\varphi_1\|^2+\|U_2\varphi_2\|^2=\|U_1\varphi_1
+U_2\varphi_2\|^2,$$ поэтому $U$ является изометрией. \par
Пусть $f\in \tilde{\cal{H}}_1\oplus \tilde{\cal{H}}_2$. Тогда $f=U_1\varphi_1
+iU_2\varphi_2$, где $\varphi_j\in {\cal H}_j$, $j=1$, 2. Значит,
$f=U(\varphi_1+\varphi_2)$, так что оператор $U$ сюръективен.
\end{proof}
Это утверждение будет потом использоваться при построении волновых
операторов $\Omega_{\pm}(\hat H, \, \hat H_0)$, где $\hat H_0=
-\frac{\hbar^2}{2m}\frac{d^2}{dx^2}+U_0\theta(x)$, $\hat H=\hat H_0+
V(x)$, где $V$ достаточно быстро убывает на бесконечности.
\subsection{Способы задания операторов Штурма--Лиувилля с сингулярными
потенциалами} В [14] были даны несколько подходов к заданию
оператора Штурма--Лиувилля на отрезке. \par Пусть
$l(\psi)=-\psi''+V(x)\psi$, где $V=W'$ (производная в обобщенном
смысле), $W\in L_2[a, \, b]$.
\begin{Def}
Квазипроизводной функции $\psi\in W^1_1[a, \, b]$ называется функция
$\psi^{[1]}(x)=\psi'(x)-W(x)\psi(x)$.
\end{Def}
Если функция $W$ абсолютно непрерывна, а $\psi$ и $\psi'$ принадлежат
$W^1_1[a, \, b]$, то выражение для $l(\psi)$ можно переписать в виде
\begin{align}
\label{stliouv}
l(\psi)=-(\psi^{[1]})'-W(x)\psi^{[1]}-W^2(x)\psi
\end{align}
Если функция $W$ не является абсолютно непрерывной, то оператор $l$
зададим формулой (\ref{stliouv}) на множестве функций, принадлежащих
$W^1_1[a, \, b]$ вместе со своей квазипроизводной. \par
{\bf Пример.} Пусть $V(x)=\alpha \delta(x-x_0)$. Тогда $W(x)=\alpha
\theta(x-x_0)+c$, и дифференциальное выражение $l(\psi)$ определено на
множестве абсолютно непрерывных функций $\psi$ таких, что $\psi'-
\alpha \theta(x-x_0)\psi(x)$ абсолютно непрерывна (откуда следует, что
$\psi'(x_0+0)-\psi'(x_0-0)=\alpha \psi(x_0)$). \par
В $L_2[a, \, b]$ зададим подпространства $$D(L_M)=\{\psi|\psi, \,
\psi^{[1]}\in W^1_1[a, \, b], \, l(\psi)\in L_2[a, \, b]\},$$
$$D(L_m)=\{\psi\in D(L_M)|\psi(a)=\psi(b)=\psi^{[1]}(a)=\psi^{[1]}
(b)=0\}.$$ Утверждается, что оператор $L_m$, заданный на $D(L_m)$
выражением $l$, является симметрическим и его индексы дефекта равны $n_{\pm}
=2$.
\begin{Trm}
\label{method_regul}
{\rm ([14], [15])}
Самосопряженные расширения $L$ оператора $L_m$ задаются выражением
$l$ на подпространстве $D(L)=\{\psi\in D(L_M)|u_1(\psi)=u_2(\psi)=0\}$,
где
\begin{align}
\label{uj12}
u_j(\psi)=a_{j1}\psi(a)+a_{j2}\psi^{[1]}(a)+b_{j1}\psi(b)+
b_{j2}\psi^{[1]}(b), \; j=1, \, 2
\end{align}
и
\begin{align}
\label{ujab}
a_{j1}\overline{a_{k2}}-a_{j2}\overline{a_{k1}}=b_{j1}\overline{b_{k2}}-
b_{j2}\overline{b_{k1}}, \; j, \; k=1, \; 2.
\end{align}
Обратно, граничные условия (\ref{uj12}), (\ref{ujab}) определяют
самосопряженный оператор $L$, если выполнено одно из условий:
\begin{enumerate}
\item $J_{24}\ne 0$;
\item $J_{24}=0$, $J_{14}-J_{23}=0$;
\item $J_{24}=J_{14}=J_{23}=J_{12}+J_{34}=0$, $J_{13}\ne 0$,
\end{enumerate}
где $J_{sr}$ --- определитель, образованный из $s$-го и $r$-го столбцов
матрицы $$\begin{pmatrix} a_{11} & a_{12} & b_{11} & b_{12} \\
a_{21} & a_{22} & b_{21} & b_{22}\end{pmatrix}.$$
\end{Trm}
{\bf Замечание.} Случай 3) включает граничные условия Дирихле $\psi(a)=
\psi(b)=0$. \par
Такой способ определения оператора Штурма--Лиувилля называется {\it методом
регуляризации}. \par
Второй способ --- это метод квадратичных форм. Пусть $l(\psi)$ задается
формулой (\ref{stliouv}). Тогда
\begin{align}
\label{quad_form_method}
\begin{array}{c}
\langle l(\psi), \, \psi\rangle =-\langle (\psi^{[1]})',\, \psi\rangle
-\langle W(x)\psi^{[1]}, \, \psi
\rangle -\langle W^2(x)\psi , \, \psi\rangle = \\ =\langle \psi ^{[1]}, \,
\psi ^{[1]}\rangle -\langle W^2(x)\psi , \, \psi\rangle +\langle \psi^{\vee},
\, \psi ^{\wedge}\rangle,
\end{array}
\end{align}
где $$\psi^{\wedge}=\begin{pmatrix}
\psi(a) \\ \psi(b) \end{pmatrix}, \; \psi^{\vee}=\begin{pmatrix}
\psi^{[1]}(a) \\ -\psi^{[1]}(b) \end{pmatrix}.$$ Пусть $C$ --- произвольная
матрица $2\times 2$, $A$ --- произвольная самосопряженная матрица $2\times 2$,
$W^1_{2, C}=\{\psi\in W^1_2[a, \, b]:C\psi^{\wedge}=0\}$. На $W^1_{2,C}$
определим квадратичную форму $${\cal L}(\psi, \, \psi)=\langle \psi^{[1]}, \,
\psi^{[1]}\rangle-\langle W^2(x)\psi, \, \psi\rangle+\langle A\psi
^{\wedge}, \, \psi^{\wedge}\rangle.$$ Утверждается, что эта форма
однозначно определяет самосопряженный оператор $L$ и что все самосопряженные
расширения оператора $L_m$ могут быть так построены. Более того, доказано,
что если $L$ --- самосопряженное расширение оператора $L_m$, то
$\tilde D(L)=\{\psi\in W^1_2[a, \, b]:u_j(\psi)=0, \; j=1, \, 2,\;
-\psi''+V(x)\psi\in L_2\}$, где $u_j(\psi)$ задаются формулой (\ref{uj12}),
а $-\psi''+V(x)\psi$ понимается в смысле обобщенных функций. Подпространство
$\tilde D(L)$ совпадает с $D(L)$ из теоремы \ref{method_regul} и
интегрирование по частям в (\ref{quad_form_method}) корректно.\par
Третий способ состоит в аппроксимации потенциала $V(x)$ гладким потенциалом
$V_\varepsilon(x)$, $0\le \varepsilon\le 1$. Этот способ более подробно
рассмотрен в [15], причем не только для случая отрезка, но и для
случая всей прямой. \par
Сначала рассмотрим случай отрезка. Пусть $W_0\in L_2[a, \, b]$,
$W_\varepsilon$ --- гладкие функции,
\begin{align}
\label{approx_potential}
\|W_\varepsilon -W_0\|_{L_2}\underset{\varepsilon \rightarrow 0}{\rightarrow} 0.
\end{align}
Пусть $L_\varepsilon \;$ $(0\le \varepsilon\le 1)$ --- семейство операторов,
порожденных выражением $-\frac{d^2}{dx^2}+W'_\varepsilon(x)$ (в смысле
теоремы \ref{method_regul}) с граничными условиями $u_1(\psi)=u_2(\psi)=0$.
\begin{Def}
Оператор $T_\varepsilon$ сходится к оператору $T_0$ в смысле сильной
(равномерной) резольвентной сходимости (обозначается $T_\varepsilon
\stackrel{R}{\rightarrow} T_0$ и соответственно $T_\varepsilon
\stackrel{R}{\Rightarrow} T_0$), если существует $\mu \in \C$ такое, что
при малых $\varepsilon \ge 0$ $\mu\in \rho(T_\varepsilon)$ и $(T_\varepsilon
-\mu)^{-1}$ сходится к $(T_0-\mu)^{-1}$ сильно (равномерно).
\end{Def}
\begin{Trm}
\label{trm_appr_potent}
{\rm [15]}
Условие (\ref{approx_potential}) влечет равномерную резольвентную сходимость
операторов $L_\varepsilon$ к $L_0$ при $\varepsilon\rightarrow 0$. Оператор
$L_0$ имеет дискретный спектр, а скорость приближения собственных значений
$\lambda_k(\varepsilon)$ оператора $L_\varepsilon$ к собственным
значениям $\lambda_k(0)$ оператора $L_0$ допускает оценку $|\lambda_k
(0)-\lambda_k(\varepsilon)|\le C_k\|W_\varepsilon-W_0\|_{L_2}$.
\end{Trm}
Теперь рассмотрим случай всей оси. Пусть $V(x)=V_1(x)+W'(x)$, где
\begin{align}
\label{uvw}
V_1\in L_{1,\rm{loc}}(\R), \; W\in L_{2,\rm{loc}}(\R), \; \supp W'
\subset (-N, \, N).
\end{align}
Квазипроизводная и дифференциальное выражение $l(\psi)$ определяются
так же, как и в случае отрезка: $\psi^{[1]}(x)=\psi'(x)-W(x)\psi(x)$,
$$l(\psi)=-(\psi^{[1]})'-W(x)\psi^{[1]}-W^2(x)\psi+V_1(x)\psi.$$
Положим $$D(L_m)=\{\psi\in L_2|\, \supp
\psi {\text{ --- компакт,}}\; \psi, \; \psi^{[1]}\in W^1_{1,\rm{loc}}, \;
l(\psi)\in L_2\},$$ $L_m(\psi)=l(\psi)$, $\psi\in D(L_m)$. \par
Определяем оператор $T_m$ так же, как $L_m$, где вместо $V(x)$ стоит
$V_1(x)$. Если выполнено условие
\begin{align}
\label{titchmarsh}
V_1(x)\ge -C-C_1x^2, \; x\in \R, \; C, \; C_1>0,
\end{align}
то оператор $T_m$ существенно самосопряжен на $D(T_m)$. Пусть $T$ ---
замыкание $T_m$.
\begin{Trm}
\label{sss_sing}
{\rm [15]}
Пусть $L$ --- замыкание $L_m$. Если выполнены условия (\ref{uvw}) и
(\ref{titchmarsh}), то оператор $L$ самосопряжен и непрерывные спектры
$T$ и $L$ совпадают.
\end{Trm}
Пусть теперь $W_\varepsilon$, $0\le \varepsilon\le 1$, --- семейство функций
из $L_2$ с носителем в $(-N, \, N)$, при $\varepsilon>0$ функции
$W_\varepsilon$ гладкие и
\begin{align}
\label{r_approx}
\|W_{\varepsilon}-W\|_{L_2}\underset{\varepsilon\rightarrow 0}{\rightarrow 0}.
\end{align}
Пусть $L_{\varepsilon,m}$ --- операторы, порожденные дифференциальным
выражением $l_\varepsilon(\psi)=-\psi''+(V_1+W_\varepsilon')\psi$,
$L_\varepsilon$ --- замыкание $L_{\varepsilon,m}$. По предыдущей теореме,
для любого $\varepsilon\in [0, \, 1]$ оператор $L_\varepsilon$
самосопряжен. \par
Через $O_\gamma(M)$ обозначим $\gamma$-окрестность множества $M$.
\begin{Trm}
\label{unif_res_conv}
{\rm [15]}
Из условия (\ref{r_approx}) следует, что $L_\varepsilon\stackrel{R}
{\Rightarrow} L$ и $\sigma(L_\varepsilon)
\rightarrow \sigma(L)$ при $\varepsilon \rightarrow 0$, то есть для любого
$R>0$ и для любого $\gamma>0$ найдется такое $\varepsilon_\gamma>0$,
что при $0<\varepsilon<\varepsilon_\gamma$ выполнены включения
$[-R, \, R]\cap \sigma(L_\varepsilon)\subset O_\gamma([-R, \, R]\cap
\sigma(L))$ и $[-R, \, R]\cap \sigma(L)\subset O_\gamma([-R, \, R]
\cap \sigma(L_\varepsilon))$.
\end{Trm}
Если функция $V_1$ ограничена снизу, то можно также применять метод
квадратичных форм. Для этого достаточно показать, что для любого $a>0$
существует $b>0$ такое, что
\begin{align}
\label{w_abklmn}
\int \limits _\R W(x)(\psi '^*\psi +\psi '\psi ^*)\, dx\le a\langle T\psi ,
\, \psi\rangle +b\langle \psi , \, \psi \rangle
\end{align}
и воспользоваться теоремой \ref{klmn}. В самом деле,
$$\int \limits_\R W(x)(\psi'^*\psi+\psi'\psi^*)\, dx\le 2\|\psi\|
_{C[-N, \, N]}\|\psi'\|_{L_2[-N, \, N]}\|W\|_{L_2[-N, \, N]}=:A.$$
По лемме Соболева, для любого $\varepsilon>0$ найдется такое $C
_\varepsilon$, что $\|\psi\|_{C[-N, \, N]}\le \varepsilon \|\psi'\|
_{L_2[-N, \, N]}+C_\varepsilon\|\psi\|_{L_2[-N, \, N]}$. Значит,
$$A\le 2\|W\|_{L_2}\varepsilon \|\psi'\|^2_{L_2}+2\|W\|_{L_2}C_\varepsilon
\|\psi\|_{L_2}\|\psi'\|_{L_2}\le$$ $$\le 2\|W\|_{L_2}\varepsilon
\|\psi'\|^2_{L_2}+2\|W\|_{L_2}\varepsilon\|\psi'\|^2_{L_2}+2\|W\|_{L_2}
\frac{C_\varepsilon^2}{\varepsilon}\|\psi\|^2_{L_2}.$$ Таким образом,
для любого $a>0$ существует $\tilde b>0$ такое, что
\begin{align}
\label{w_acklmn}
\int \limits _\R W(x)(\psi '^*\psi +\psi '\psi ^*)\, dx\le a\|\psi '\| ^2_{L_2(\R)}+
\tilde b\| \psi\| ^2_{L_2(\R)}.
\end{align}
Так как функция $V_1$ ограничена снизу, то отсюда следует (\ref{w_abklmn}).
\subsection{Регулярность обобщенных собственных функций\\ оператора
Штурма--Лиувилля} \label{regul_par} Для доказательства
существования регулярных решений уравнения Штурма--Лиувилля
используется следующий результат.
\begin{Sta}
\label{shkalikov}
Пусть $A(x)$ --- матрица $n\times n$, элементами которой являются
функции из $L_1[x_0, \, y_0]$, $f\in L_1([x_0, \, y_0], \, \R^n)$. Тогда
для любого $\xi\in \R^n$ уравнение $$\psi'(x)=A(x)\psi(x)+f(x), \;
\psi(x_0)=\xi$$ имеет единственное решение в классе абсолютно непрерывных
функций. Если $\|A_\varepsilon-A\|_{L_1}\rightarrow 0$ и $\xi_\varepsilon
\rightarrow \xi$ при $\varepsilon\rightarrow 0$, то решения уравнений
$$\psi'_\varepsilon(x)=A_\varepsilon(x)\psi_\varepsilon(x)+f(x), \;
\psi_\varepsilon(x_0)=\xi_\varepsilon$$ сходятся к $\psi$ равномерно
на $[x_0, \, y_0]$. Точнее, выполнена оценка
$$\|\psi-\psi_\varepsilon\|_{C[x_0, \, y_0]}\le C\|f\|_{L_1}\|A-
A_\varepsilon\|_{L_1}+C|\xi-\xi_\varepsilon|,$$ где $C$ не зависит от
$f$ и $\varepsilon$.
\end{Sta}
Для случая, когда $\xi_\varepsilon=\xi$, это утверждение было доказано в
[15]. В общем случае доказательство аналогично. \par
Пусть $V\in L_1^{{\rm loc}}$, $W\in L_2^{{\rm loc}}$, $\eta \in L_1^{{\rm
loc}}$, $k\in \C$, $\varphi^{[1]}=\varphi'-W(x)\varphi$. Рассмотрим
дифференциальное уравнение
\begin{align}
\label{neodn_uravn_st_liouv}
-(\varphi^{[1]})'-W(x)\varphi^{[1]}-W^2(x)\varphi+V(x)\varphi=k^2\varphi
+\eta.
\end{align}
Из предыдущего утверждения получаем
\begin{Cor}
\label{exist_uniq}
На любом отрезке $[x_0, \, y_0]$ и для любых начальных условий $\varphi(x_0)$,
$\varphi^{[1]}(x_0)$ существует единственное решение $\varphi$ уравнения
(\ref{neodn_uravn_st_liouv}) такое, что $\varphi$ и $\varphi^{[1]}$
абсолютно непрерывны на $[x_0, \, y_0]$. При этом $\varphi$ и
$\varphi^{[1]}$ непрерывно (в метрике $C[x_0, \, y_0]$) зависят от
$(\varphi(x_0), \, \varphi^{[1]}(x_0), \, k, \, W)\in $ $\C\times\C\times
\C\times L_2[x_0, \, y_0]$.
\end{Cor}
\begin{proof}
Положим $\psi=\varphi^{[1]}$. Тогда уравнение (\ref{neodn_uravn_st_liouv})
эквивалентно системе $$\varphi'=W(x)\varphi+\psi,$$ $$\psi'=(V(x)-
W^2(x)-k^2)\varphi-W(x)\psi-\eta.$$
Дальше применяем утверждение \ref{shkalikov}.
\end{proof}
{\bf Замечание.} Если $f$ и $g$ --- абсолютно непрерывные функции, то
для почти всех $x$ определен их вронскиан $W(f(x), \, g(x))=f(x)g'(x)-
g(x)f'(x)=f(x)g^{[1]}(x)-g(x)f^{[1]}(x)$. Дифференцируя последнее выражение
по $x$, получаем, что если $f$ и $g$ являются решениями однородного уравнения
(\ref{neodn_uravn_st_liouv}) (то есть с $\eta\equiv 0$), то вронскиан
не зависит от $x$.
\begin{Lem}
\label{hilb_schmidt}
Пусть $V=V_1+V_2$, где функция $V_1\in L_1^{{\rm loc}}(\R)$
ограничена снизу, $V_2=W'$, $W\in L_2[-R_0, \, R_0]$. Тогда для достаточно
больших $c$ оператор $K=-\frac{d^2}{dx^2}+V(x)+c(1+x^2)$ положительно
определен и $K^{-1}$ является оператором Гильберта--Шмидта.
\end{Lem}
\begin{proof}
Положим $\tilde K=-\frac12\frac{d^2}{dx^2}+cx^2$, где $c$ выбирается так,
чтобы оператор $K-\tilde K=-\frac12 \frac{d^2}{dx^2}+V(x)+c$ был положительно
определен (это возможно, так как $V_1$ ограничен снизу и выполнено (\ref{w_acklmn})).
Операторы $K$ и $\tilde K$ задают квадратичные
формы $$q_K(\psi)=\int \limits_{\R} \left(|\psi'|^2+W(x)(\psi'^*\psi+
\psi'\psi^*)+(V_1(x)+c+cx^2)|\psi|^2\right)\, dx,$$
$$q_{\tilde K}(\psi)=\int \limits_{\R} \left(\frac12|\psi'|^2+cx^2|\psi|^2
\right)\, dx$$ с областью определения
$$Q(K)=\Big\{\psi:\forall R>0\; \; \psi\in AC[-R, \, R], \; \int
\limits_{\R} \big(|\psi'|^2+W(x)(\psi'^*\psi+\psi'\psi^*)+$$
$$+(V_1(x)+c+ cx^2)|\psi|^2\big)\, dx<\infty\Big\},$$
$$Q(\tilde K)=\left\{\psi:\forall R>0 \; \; \psi\in AC[-R, \, R], \; \int
\limits_{\R} \left(\frac12|\psi'|^2+cx^2|\psi|^2\right)\,
dx<\infty\right\}.$$ Значит, $Q(K)\subset Q(\tilde K)$ и для
любого $\psi\in Q(K)$ выполнено $q_{\tilde K}(\psi)\le q_K(\psi)$. Положим
$$\mu_n=\sup_{\varphi_1,\dots,\varphi_{n-1}}
\inf\{q_K(\psi):\psi\in [\varphi_1, \, \dots, \,
\varphi_{n-1}]^{\bot}, \; \|\psi\|=1, \; \psi\in Q(K)\},$$
$$\tilde \mu_n=\sup_{\varphi_1,\dots,\varphi_{n-1}}\inf\{ q_{\tilde K}
(\psi):\psi\in [\varphi_1, \, \dots, \, \varphi_{n-1}]^{\bot}, \;
\|\psi\|=1, \; \psi\in Q(\tilde K)\}.$$ Тогда $\tilde \mu_n\le \mu_n$
при всех $n\in \N$. В силу принципа минимакса, $\tilde \mu_n$
является $n$-м собственным значением оператора $\tilde K$, поэтому
последовательность $\{\tilde \mu_n\}$ образует возрастающую
арифметическую прогрессию. Значит, $\mu_n\rightarrow +\infty$ при
$n\rightarrow \infty$ и образует множество всех собственных значений
$K$ с учетом кратности, а спектр $K$ совпадает с $\{\mu_n\}$ (это
также следует из принципа минимакса). Следовательно, $K^{-1}$ является
оператором Гильберта--Шмидта.
\end{proof}
\begin{Sta}
\label{regularnost_sobstv_func}
Пусть $\hat H=-\frac{d^2}{dx^2}+V(x)$, где $V=V_1+V_2$, $V_1\in L_1^{{\rm
loc}}(\R)$, $V_1(x)\ge -c_1-c_2x^2$, $V_2=W'$, $W\in L_2[-R_0, \, R_0]$.
Тогда $\hat H$ имеет полную систему обобщенных собственных векторов
$\psi_E$ таких, что $\psi_E$ и $\psi_E^{[1]}$ абсолютно непрерывны и
$-(\psi_E^{[1]})'-W(x)\psi_E-W^2(x)\psi_E+V_1(x)\psi_E=E\psi_E$. Кратность
спектра всегда не превосходит 2.
\end{Sta}
\begin{proof}
Построим оснащение с помощью оператора $K=-\frac{d^2}{dx^2}+c(1+x^2)+V-
\tilde V$, где $\tilde V$ --- гладкая функция, $V_1-\tilde V$ ограничена
снизу и принадлежит $L_1+L_\infty$, а $c$ выбирается так, чтобы $K$ был
положительным и $K^{-1}$ был оператором Гильберта--Шмидта (функция
$\tilde V$ вычитается для того, чтобы было ограничение на рост
обобщенных собственных функций; например, если функция $V$ гладкая
при больших $x$, то невозможен экспоненциальный рост). Тогда ${\cal H}_+$
плотно в пространстве $L_2$ (так как оператор $K^{-1}$ невырожден).\par
Пусть $\eta \in {\cal H}_+$
имеет компактный носитель, $\varphi$, $\varphi^{[1]}$ абсолютно
непрерывны, $(\hat H-E)\varphi=\eta$ и $\varphi$ имеет компактный носитель.
Тогда $K\varphi=\hat H\varphi+c(1+x^2)\varphi-\tilde V\varphi =
E\varphi+\eta+c(1+x^2)\varphi -\tilde V\varphi\in L_2$, поэтому
$\varphi\in {\cal H}_+$. Отсюда следует, что $\hat H
\varphi=E\varphi+\eta\in {\cal H}_+$. Значит, если $F\in {\cal H}_-$
--- обобщенный собственный вектор, соответствующий точке спектра $E$,
то $F(\eta)=0$. \par
Из следствия \ref{exist_uniq} получаем, что уравнение $\hat Hf = Ef$
имеет два линейно независимых решения $f_1$ и $f_2$ таких, что $f_j$
и $f_j^{[1]}$ абсолютно непрерывны. Докажем, что уравнение $(\hat H-E)
\varphi=\eta$ имеет решение с компактным носителем тогда и только тогда,
когда $\int _{-\infty}^\infty f_j\eta\, dx=0$, $j=1$, $2$. В самом
деле, пусть ${\rm supp}\, \eta\subset (a, \, b)$. Рассмотрим решение
$\varphi$ уравнения $(\hat H-E)\varphi=\eta$ с начальными условиями
$\varphi(a)=0$, $\varphi^{[1]}(a)=0$. Интегрируя по частям, получаем
$$\int \limits_{-\infty}^\infty \eta(x) f_j(x)\, dx=\int \limits_{-\infty}
^\infty (\hat H-E)\varphi(x) f_j(x)\, dx=$$
$$=-\varphi^{[1]}(b)f_j(b)+\varphi(b)f_j^{[1]}(b)+\int \limits_{-\infty}
^\infty (\hat H-E)f_j(x) \varphi(x)\, dx=-\varphi^{[1]}(b)f_j(b)
+\varphi(b)f_j^{[1]}(b).$$ Если $\varphi(b)=\varphi^{[1]}(b)=0$,
то правая часть равна 0. Обратно, система уравнений
$$-\varphi^{[1]} (b)f_j(b)+\varphi(b)f_j^{[1]}(b)=0, \; j=1, \,
2,$$ имеет только нулевое решение, так как определитель этой
системы не равен 0 в силу линейной независимости $f_1$ и $f_2$.
\par Пусть $\eta$ --- произвольная функция из ${\cal H}_+$ с
компактным носителем. Предположим, что существует функция
$\eta_1\in {\cal H}_+$ с компактным носителем такая, что $\int
_{-\infty}^\infty f_1(x)\eta_1(x)\, dx=1$. Пусть $\eta_2\in {\cal
H}_+$ имеет компактный носитель, $$\int _{-\infty}^\infty
f_1(x)\eta_2(x)\, dx=0, \; \; \; \int _{-\infty}^\infty f_2(x)\eta_2(x)
\, dx=1.$$ Положим $$\eta_0(x)=\eta(x)-\eta_1(x)\int
\limits_{-\infty} ^\infty f_1(t)\eta(t)\, dt-\eta_2(x)\int
\limits_{-\infty}^\infty f_2(t)\eta(t)\, dt+$$ $$+\eta_2(x)\int
\limits_{-\infty}^\infty f_2 (t)\eta_1(t)\, dt\int
\limits_{-\infty}^\infty f_1(t)\eta(t)\, dt.$$ Тогда $\eta_0\in
{\cal H}_+$ имеет компактный носитель и $\int _{-\infty} ^\infty
\eta_0(x)f_j(x)\, dx=0$, $j=1$, 2. Значит, $F(\eta_0)=0$ и отсюда
$$F(\eta)=\left(F(\eta_1)-F(\eta_2)\int \limits_{-\infty}^\infty
f_2(x)\eta_1(x)\, dx\right)\int \limits_{-\infty}^\infty f_1(x)
\eta(x)\, dx+F(\eta_2)\int \limits_{-\infty}^\infty f_2(x)
\eta(x)\, dx,$$ то есть $F=c_1f_1+c_2f_2$. \par Докажем, что
множество функций из ${\cal H}_+$ с компактным носителем плотно в $L_2$.
Пусть $\eta\in {\cal H}_+$, $\alpha\in C_0^\infty(\R)$. Тогда $\alpha
\eta$ и $(\alpha\eta)^{[1]}=\alpha'\eta+\alpha \eta^{[1]}$
абсолютно непрерывны, а $K(\alpha\eta)=\alpha
K\eta-2\alpha'\eta'-\alpha''\eta\in L_2$, так как $K\eta\in L_2$ и
$\eta'=\eta^{[1]}+W(x)\eta\in L_2^{{\rm loc}}$. Значит,
$\alpha\eta\in {\cal H}_+$. Множество таких функций плотно в
${\cal H}_+$ относительно метрики $L_2$, а значит, и во всем
$L_2$. Поэтому преобразование Фурье достаточно задать на множестве
функций из ${\cal H}_+$ с компактным носителем, а затем продолжить
по непрерывности.
\par
Теперь докажем, что кратность спектра не больше 2. В самом деле,
преобразование Фурье от функции $\varphi\in {\cal H}_+$ с компактным
носителем при почти всех $\lambda\in \sigma(\hat H)$ имеет вид $(\hat
\varphi_k(\lambda))_{k=1}^{N(\lambda)}$, где $\hat \varphi_k(\lambda)
=\int F_k(x, \, \lambda)\varphi(x)\, dx$, $F_k(x, \, \lambda)$ ---
решение уравнения $\hat HF =\lambda F$, при этом множество $$\left
\{(\hat \varphi_k(\lambda))_{k=1}^{N(\lambda)}:\varphi\in {\cal H}_+
\; \text{имеет компактный носитель}\right\}$$ образует пространство
размерности $N(\lambda)$. Последнее означает, что для любой ненулевой
последовательности $(c_k)_{k=1}^{N(\lambda)}$ найдется функция $\varphi$
такая, что $\sum\limits_{k=1}^{N(\lambda)}c_k\hat \varphi_k(\lambda)\ne 0$.
Отсюда следует, что $\sum \limits_{k=1}^{N(\lambda)}c_kF_k(x, \, \lambda)$
не равна тождественно 0 для любой ненулевой последовательности
$(c_k)_{k=1}^{N(\lambda)}$, а это возможно только при $N(\lambda)\le 2$,
так как уравнение $\hat HF=\lambda F$ имеет два линейно независимых
решения.
\end{proof}
\begin{Cor}
Если функция $V$ кусочно-непрерывна, то обобщенные собственные функции
абсолютно непрерывны вместе со своими производными. Если $$V(x)=
V_0(x)+\sum \limits_{k=1}^n \alpha_n\delta(x-x_n),$$ где $V_0$ ---
кусочно-непрерывная функция, то обобщенные собственные функции $\psi$
непрерывны, а их производные непрерывны всюду, кроме точек $x_k$, в
которых $\psi(x_k+0)-\psi(x_k-0)=\alpha_k\psi(x_k)$.
\end{Cor}
Докажем, что если $F$ --- обобщенный собственный вектор, а оснащение
построено так же, как в предыдущем утверждении, то $F=Kg$, где $g$
и $g^{[1]}$ абсолютно непрерывны на каждом отрезке и $g\in L_2$.
В самом деле, любой элемент ${\cal H}_-$ задается в виде
$F(\eta)=\int g(x)(K\eta(x))\, dx$, где $g\in L_2$. Значит, $$\int
\limits_{\R}g(x)(K\eta(x))\, dx=\int \limits_{\R}F(x)\eta(x)\, dx.$$
Пусть $\tilde g$ --- решение уравнения $K\tilde g=F$, такое что $\tilde g$
и $\tilde g^{[1]}$ абсолютно непрерывны (его существование вытекает
из следствия \ref{exist_uniq}). Если $\eta$ имеет компактный
носитель и $K\eta\in {\cal H}_+$, то с помощью интегрирования по
частям получаем, что $\int (g-\tilde g)(K\eta)\, dx=0$. Значит,
$g-\tilde g$ является регулярным решением уравнения $Kf=0$
(см. доказательство регулярности $F$), так что $g$ и $g^{[1]}$
абсолютно непрерывны и $Kg=F$. \par
Отсюда следует, что если функция $V$ локально ограничена при достаточно
больших $x$, то функция $J^3F(x)$ растет не быстрее некоторого полинома,
где $JF(x)=\int \limits_0^x F(t)dt$ (см., напр., [6]).
\subsection{Общий вид собственных функций одномерного уравнения Шредингера}
Приведем несколько теорем [12, гл. II, \S 3] о поведении решений
уравнения Шредингера на бесконечности.
\begin{Trm}
\label{62trm2}
Пусть $V\in L_\infty^{{\rm loc}}(\R)$ и $V(x)\ge \varepsilon>0$ при
достаточно больших $x$. Тогда для любого решения уравнения
\begin{align}
\label{6trm2shre}
-y''+V(x)y=0
\end{align}
выполнено одно из соотношений:
\begin{enumerate}
\item $y(x)\underset{x\rightarrow +\infty}{\rightarrow} \infty$,
\item $y(x)\underset{x\rightarrow +\infty}{\rightarrow} 0$.
\end{enumerate}
При этом решение, удовлетворяющее условию 2), существует и однозначно
определено с точностью до постоянного множителя.
\end{Trm}
\begin{Sta}
\label{62sta}
Если $V\in L_\infty^{{\rm loc}}(\R)$ и $V(x)\ge 0$ при $x\in [a, \, b]$,
то всякое ненулевое решение уравнения (\ref{6trm2shre}) имеет не более одного
нуля на $[a, \, b]$.
\end{Sta}
Следующая теорема позволяет оценивать скорость возрастания или убывания
решений уравнения (\ref{6trm2shre}).
\begin{Trm}
\label{62trm3}
Пусть потенциалы $V_1(x)$, $V_2(x)\in L_\infty^{{\rm loc}}$ удовлетворяют
условию $V_1(x)\ge V_2(x)\ge 0$. Рассмотрим два уравнения
$$-y_1''+V_1(x)y_1=0, \;\;\; -y_2''+V_2(x)y_2=0.$$
Тогда, если $y_1$, $y_2\rightarrow +\infty$ при $x\rightarrow +\infty$, то
существует такая постоянная $C>0$, что $$y_2(x)\le Cy_1(x)$$ для
достаточно больших $x$. Если $y_1$, $y_2\rightarrow 0$ при $x\rightarrow +\infty$,
$y_1$, $y_2>0$, то существует $C>0$ такое, что $$y_1(x)\le Cy_2(x)$$ при
достаточно больших $x$.
\end{Trm}
В частности, если $V(x)\ge \varepsilon>0$, то решение (\ref{6trm2shre})
возрастает или убывает экспоненциально (из утверждения \ref{62sta}
следует, что при достаточно больших $x$ решение имеет постоянный знак). \par
Пусть $V=V_1+W'$, где $V_1\in L_1^{{\rm loc}}(\R)$, $W\in L_2$ имеет
компактный носитель.
\begin{Cor}
Пусть $\lim \limits _{x\rightarrow -\infty}V(x)=E_1$,
$\lim \limits _{x \rightarrow +\infty}V(x)=E_2$. Тогда при $E<\min
\{E_1, \, E_2\}$ (то есть на уровнях энергии, соответствующих
финитному классическому движению) спектр дискретный и однократный.
\end{Cor}
Пусть теперь $V_1\in L_\infty^{{\rm loc}}(\R)$, $V_1
\rightarrow +\infty$ при $x\rightarrow \infty$. Исследуем спектр
оператора $\hat H$ и свойства собственных функций.
\begin{Trm}
В сделанных выше предположениях спектр оператора $\hat H$ дискретный,
однократный и имеет вид $\{\lambda_k\}_{k=0}^\infty$, где $\lambda_k
<\lambda_{k+1}$ и $\lambda_k\rightarrow +\infty$.
\end{Trm}
\begin{proof}
То, что собственные значения однократны, следует из теоремы \ref{62trm2}.
Докажем, что для достаточно большого $c>0$ оператор $\hat H+cI$ положительно
определен и $(\hat H+cI)^{-1}$ компактен. Отсюда будет следовать остальная
часть теоремы. \par
Если $V\in L_\infty^{{\rm loc}}$, то это доказано в [12, гл. II].
Рассмотрим общий случай. Из неравенства (\ref{w_abklmn}) следует, что
$\hat H$ ограничен снизу и при достаточно большом $c>0$ оператор $\hat H+
cI$ обратим. Пусть $W_n$ --- гладкие функции с общим компактным
носителем и $W_n\stackrel{L_2}{\rightarrow}W$ при $n\rightarrow \infty$.
Положим $\hat H_n=-\frac{d^2}{dx^2}+V_1(x)+W_n'(x)$. Тогда, в силу теоремы
\ref{unif_res_conv}, $\hat H_n$ сходится к $\hat H$ в равномерном
резольвентном смысле. Значит, при достаточно больших $n$ операторы
$\hat H_n+cI$ обратимы и $\|(\hat H_n+cI)^{-1}-(\hat H+cI)^{-1}
\|\rightarrow 0$ при $n\rightarrow \infty$, где норма берется в равномерной
операторной топологии [3, т. 1, теорема VIII.23]. По доказанному,
операторы $(\hat H_n+cI)^{-1}$ компактны. Значит, и их равномерный
предел тоже компактен.
\end{proof}
Найдем число нулей для $k$-й собственной функции оператора $\hat H$.
\begin{Lem}
\label{krit_infty_0}
Пусть $V\in L_\infty^{{\rm loc}}(\R)$ и $V(x)\ge \varepsilon>0$ при $x>a$.
Пусть $\psi(x)$ --- решение уравнения $-\psi''(x)+V(x)\psi(x)=0$.
Тогда для того, чтобы выполнялось условие $\psi(x)\rightarrow +\infty$
($-\infty$) при $x\rightarrow +\infty$, необходимо и достаточно,
чтобы существовала точка $x_0>a$ такая, что $\psi(x_0)>0$ и
$\psi'(x_0)>0$ (соответственно $\psi(x_0)<0$ и $\psi'(x_0)<0$).
\end{Lem}
Это утверждение следует из доказательства теоремы 3.3 в [12, гл. II].
\begin{Lem}
\label{faf1a}
Пусть при $x\ge a$ выполнено неравенство $V(x)\ge E+\varepsilon_0$,
где $\varepsilon_0>0$, и пусть $E_n\rightarrow E$ при $n\rightarrow \infty$.
Рассмотрим функции $\psi$ и $\psi _n$, являющиеся соответственно
решениями уравнения
\begin{align}
\label{st_liouv_0n}
-\psi ''+V\psi =E\psi
\end{align}
и
\begin{align}
\label{st_liouv_0n1}
-\psi _n''+V\psi _n=E_n\psi _n
\end{align}
с граничными условиями $\psi _n(a)=1$, $\psi(a)=1$, $\lim \limits
_{x\rightarrow +\infty}\psi _n(x)=0$
и $\lim \limits _{x\rightarrow +\infty}\psi(x)=0$.
Тогда $\psi _n'(a)\rightarrow \psi '(a)$, $n\rightarrow \infty$.
\end{Lem}
\begin{proof}
Возьмем произвольное $\varepsilon\in (0, \, \varepsilon_0)$.
Докажем, что при достаточно больших $n$ выполнено $\psi_n'(a)
\in [-\varepsilon , \, \varepsilon]$. Пусть $\psi^{\pm \varepsilon}_n$
и $\psi^{\pm \varepsilon}$ --- решения уравнений (\ref{st_liouv_0n})
и (\ref{st_liouv_0n1}) с начальными условиями
$\psi^{\pm \varepsilon}_n(a)=1$, $\psi^{\pm \varepsilon}(a)=1$,
$(\psi_n^{\pm \varepsilon})'(a)=\psi'(a)\pm \varepsilon$ и $(\psi^{\pm
\varepsilon})'(a)=\psi'(a)\pm \varepsilon$. Так как $\psi^{\pm
\varepsilon}_n(a)=\psi_n(a)$, $\psi^{\pm \varepsilon}(a)=\psi(a)$,
$V(x)-E_n\ge 0$ и $V(x)-E\ge 0$ при $x\ge a$ для достаточно больших
$n$, то $\psi_n^{\pm \varepsilon}(x)\ne \psi_n(x)$ и $\psi^{\pm
\varepsilon}(x)\ne \psi(x)$ при $x>a$ (в силу утверждения \ref{62sta}).
Так как решение каждого из уравнений (\ref{st_liouv_0n}), равное 1
при $x=a$ и обращающееся в 0 на бесконечности, единственно, то
$\psi^{\pm \varepsilon}(x)\rightarrow \pm \infty$ при $x\rightarrow
+\infty$. Значит, по лемме \ref{krit_infty_0}, существует точка
$x_0>a$ такая, что $\psi^\varepsilon(x_0)>0$, $(\psi^\varepsilon)'
(x_0)>0$, $\psi^{-\varepsilon}(x_0)<0$ и $(\psi^{-\varepsilon})'(x_0)
<0$. Тогда, в силу следствия \ref{exist_uniq}, для достаточно
больших $n$ выполнено $\psi^\varepsilon_n(x_0)>0$, $(\psi_n^\varepsilon)'
(x_0)>0$, $\psi^{-\varepsilon}_n(x_0)<0$ и $(\psi_n^{-\varepsilon})'(x_0)
<0$. Из неотрицательности функции $V(x)-E_n$ следует, что те же
неравенства выполнены при $x>x_0$. Значит, в силу леммы
\ref{krit_infty_0}, $\psi_n^{\pm \varepsilon}(x)\rightarrow \pm
\infty$ при $x\rightarrow +\infty$. \par
Пусть $\delta>\varepsilon$. Тогда
в некоторой правой полуокрестности точки $a$ выполнены неравенства
\begin{align}
\label{psi_n_delta}
\psi_n^{-\delta}(x)<\psi_n^{-\varepsilon}(x) \;\;\; \text{и} \;\;\;
\psi_n^\delta(x)>\psi_n^\varepsilon(x).
\end{align}
В силу утверждения \ref{62sta}, при $x>a\;$ $\psi_n^{\pm \delta}(x)\ne
\psi_n^{\pm \varepsilon}(x)$, поэтому неравенства (\ref{psi_n_delta})
выполнены для любого $x>a$, так что $\psi_n^{\pm \delta}(x)\rightarrow
\pm \infty$ при $x\rightarrow \infty$. Значит, $\psi_n'(a)\in [-\varepsilon,
\, \varepsilon]$.
\end{proof}
\begin{Sta}
\label{cont_wn}
Пусть $V_1\in L_\infty^{{\rm loc}}(\R)$ и $V(x)\ge E$ при $|x|>a$,
$W\in L_2[-a, \, a]$. Пусть $W_n\stackrel{L_2[-a, \, a]}{\rightarrow}W$
при $n\rightarrow \infty$. Положим $\hat H=-\frac{d^2}{dx^2}+V_1+W'$,
$\hat H_n=-\frac{d^2}{dx^2}+V_1+W'_n$. Пусть $\psi_k$ --- $k$-я
собственная функция оператора $\hat H$ с собственным значением $E_k<E$,
$E_{k,n}<E$ --- $k$-е собственное значение $\hat H_n$. Тогда $k$-ю
собственную функцию $\psi_{n,k}$ оператора $\hat H_n$ можно выбрать так,
чтобы $\psi_{n,k}\rightarrow \psi_k$ и $\psi^{[1]}_{n,k}\rightarrow
\psi^{[1]}_k$ равномерно на каждом отрезке (квазипроизводная определяется
соответственно по функциям $W_n$ и $W$).
\end{Sta}
\begin{proof}
Пусть $b>a$.
Можно считать, что $\psi_k(b)=1$. Выберем $\psi_{n,k}$ так, чтобы
$\psi_{n,k}(b)=1$. По теореме \ref{unif_res_conv}, $\sigma(\hat H_n)
\rightarrow \sigma(\hat H)$, так что $E_{k,n}\rightarrow E_k$, $n\rightarrow
\infty$. По лемме \ref{faf1a}, $\psi_{k,n}'(b)\rightarrow \psi_k'(b)$,
$n\rightarrow \infty$. В силу следствия \ref{exist_uniq}, $\psi_{k,n}
\rightarrow \psi_k$ и $\psi^{[1]}_{k,n}\rightarrow \psi^{[1]}_k$ равномерно
на каждом отрезке.
\end{proof}
\begin{Trm}
Пусть $\hat H=-\frac{d^2}{dx^2}+V_1+W'$, где $W\in L_2$ имеет компактный
носитель, $V_1\in L_\infty^{{\rm loc}}(\R)$ и $V_1\rightarrow +\infty$ при
$x\rightarrow \infty$. Пусть $n_k$ --- число нулей собственной функции,
соответствующей $k$-му собственному значению ($k\in \Z_+$). Тогда
$n_k=k$.
\end{Trm}
\begin{proof}
Случай $W'\in L_\infty$ рассмотрен в [12, гл. II]. Рассмотрим общий случай.
Сначала докажем, что если $\psi$ --- решение уравнения Штурма--Лиувилля,
$\psi(x_0)=0$ и $\psi^{[1]}(x_0)>0$, то $\psi(x)>0(<0)$ в некоторой
правой (левой) полуокрестности точки $x_0$. В самом деле, из непрерывности
функции $\psi^{[1]}$ следует, что $\psi^{[1]}(x)\ge c>0$ в некоторой
окрестности $O(x_0)$. Значит, при $x>x_0$, $x\in O(x_0)$
$$\psi(x)=\int \limits_{x_0}^x\psi'(t)\, dt=\int \limits_{x_0}^x
\left(\psi^{[1]}(t)+W(t)\psi(t)\right)\, dt\ge c(x-x_0)-\|W\|_{L_2[x_0,
\, x]}\|\psi\|_{L_2[x_0, \, x]}.$$ Так как $\|W\|_{L_2[x_0, \, x]}
\rightarrow 0$ при $x\rightarrow x_0$, то достаточно доказать, что
\begin{align}
\label{psi_l2_estim}
\|\psi\|_{L_2[x_0, \, x]}\le C|x-x_0|,
\end{align}
где $C$ не зависит от $x$. Положим $f_1(t)=\int \limits_{x_0}^t
\psi^{[1]}(\tau)\, d\tau$, $f_2(t)=\int \limits_{x_0}^t W(\tau)\psi
(\tau)\, d\tau$. Тогда $$\|\psi\|_{L_2[x_0, \, x]}\le \|f_1\|
_{L_2[x_0, \, x]}+\|f_2\|_{L_2[x_0, \, x]}.$$ Так как функция $\psi^{[1]}$
абсолютно непрерывна, то первое слагаемое оценивается сверху
величиной $C_1|x-x_0|$, где $C_1$ не зависит от $x$. Оценим второе слагаемое:
$$\|f_2\|_{L_2[x_0, \, x]}=\left(\int \limits_{x_0}^x\left|
\int \limits_{x_0}^t W(\tau)\psi(\tau)\, d\tau\right|^2\, dt\right)^{1/2}
\le \left(\int \limits_{x_0}^x\|W\|^2_{L_2[x_0, \, x]}\|\psi\|
_{L_2[x_0, \, t]}^2\, dt\right)^{1/2}\le $$ $$\le \|W\|_{L_2[x_0, \, x]}
\left(\int \limits_{x_0}^xC_2^2(t-x_0)\, dt\right)^{1/2}\le
C_2\|W\|_{L_2[x_0, \, x]}(x-x_0)$$ (мы воспользовались тем, что
функция $\psi$ ограничена некоторой константой $C_2$). Отсюда следует
(\ref{psi_l2_estim}). Аналогично рассматривается случай $x<x_0$. \par
Пусть $W_m$ --- гладкие функции с общим компактным носителем, сходящиеся
к $W$ в $L_2$. Положим $\hat H_m=-\frac{d^2}{dx^2}+V_1+W_m'$. Пусть
$\psi_{m,k}$ и $\psi_k$ --- $k$-е собственные функции операторов $\hat H_m$
и $\hat H$ соответственно. По утверждению \ref{cont_wn}, их можно
выбрать так, чтобы $\psi_{m,k}\rightarrow \psi_k$ и $\psi_{m,k}
\rightarrow \psi_k$ равномерно на каждом отрезке. \par
Так как $W'_m\in L_\infty^{{\rm loc}}$, то функции $\psi_{m,k}$ имеют
ровно $k$ нулей. \par
Пусть $n_k>k$, $\psi_k(x_j)=0$, $1\le j\le n_k$. Так как $\psi_k
\ne 0$, то $\psi_k^{[1]}(x_j)\ne 0$. По доказанному, в каждой точке $x_j$
функция $\psi_k$ меняет знак. Поэтому существует последовательность
точек $y_0<y_1<\dots<y_{n_k}$ такая, что $\psi_k(y_j)$ и $\psi_k(y_{j-1})$
имеют противоположный знак, $1\le j\le n_k$. При достаточно больших
$m$ то же самое выполнено для значений функции $\psi_{m,k}$. Но тогда
$\psi_{m,k}$ имеет $n_k>k$ нулей --- противоречие. \par
Пусть $n_k<k$, $x_j$ --- нули функции $\psi_k$. Так как $V_1(x)
\rightarrow +\infty$ при $x\rightarrow \infty$, то существует $R>0$
такое, что $V(x)>E_k+1$ при $|x|>R$. Можно считать, что $|x_j|<R$
для любого $j=1, \, \dots, \, n_k$ и что $\psi_k(R)>0$. Тогда при
достаточно больших $m$ выполнено $\psi_{m,k}(R)>0$. Пусть существует
$x_*>R$ такое, что $\psi_{m,k}(x_*)=0$. Положим $$x_0=\inf\{x>R:
\psi _{m,k}(x)=0\}.$$ Тогда $\psi'_{m,k}(x_0)<0$, и при некотором
$x>x_0$ выполнено $\psi_{m,k}(x)<0$ и $\psi'_{m,k}(x_0)<0$.
Тогда, по лемме \ref{krit_infty_0}, $\psi_{m,k}(x)\rightarrow -\infty$ при
$x\rightarrow +\infty$ --- противоречие. Значит, при $x>R$ и
достаточно больших $m$ функция $\psi_{m,k}(x)$ не обращается в 0.
Аналогично рассматриваются $x<-R$. \par
Так как $\psi_k^{[1]}(x_j)\ne 0$, то существуют $c>0$ и окрестности
$U_j$ точек $x_j$, такие что $|\psi_k^{[1]}(x)|>c$ при $x\in U_j$ и
$|\psi_k(x)|>c$ при $x\in [-R, \, R]\backslash \cup_{j=1}^{n_k}U_j$.
Так как $\psi_{m,k}\rightarrow \psi_k$ и $\psi_{m,k}\rightarrow
\psi_k$ равномерно на каждом отрезке, то при больших $m$ выполнено
$|\psi_{m,k}^{[1]}(x)|>\frac{c}{2}$ для $x\in U_j$ и $|\psi_{m,k}(x)|
>\frac{c}{2}$ для $x\in [-R, \, R]\backslash \cup_{j=1}^{n_k}U_j$.
Значит, нули функции $\psi_{m,k}$ принадлежат $U_j$. Так их число
равно $k>n_k$, то для некоторого $j$ окрестность $U_j$ содержит два
нуля функции $\psi_{m,k}$. Пусть $y_1$, $y_2\in U_j$ --- два соседних
нуля. Тогда $\psi^{[1]}_{m,k}(y_1)=\psi'_{m,k}(y_1)$, $\psi^{[1]}_{m,k}
(y_2)=\psi'_{m,k}(y_2)$ и знаки этих величин противоположны. Из
непрерывности квазипроизводной следует, что для некоторого $x\in
(y_1, \, y_2)$ выполнено $\psi^{[1]}_{m,k}(x)=0$. Это противоречит
тому, что $|\psi^{[1]}_{m,k}(x)|>\frac{c}{2}$ при $x\in U_j$. Значит,
$n_k=k$.
\end{proof}
\subsection{Решения Йоста}
Пусть $\hat H_0=-\frac{d^2}{dx^2}$, $\hat H=\hat H_0+V$, где $V=V_1+V_2$,
$V_1\in L_1^{{\rm loc}}(\R)$, $V_2=W'$, $W\in L_2(\R)$ имеет
носитель, лежащий в $[-R_0, \, R_0]$. \par
Рассмотрим при $x>R_0$ интегральное уравнение
\begin{align}
\label{jost}
f(x)=e^{-ikx}-\int \limits_x^\infty \frac{\sin k(x-y)}{k}V(y)f(y)\, dy.
\end{align}
Предположим, что $\int \limits_{R_0}^\infty |V(x)|\, dx<\infty$. Тогда
для любого $k\in \R\backslash \{0\}$ уравнение (\ref{jost}) имеет
единственное решение $f(x, \, k)$, при этом
\begin{align}
\label{ostatok_jost}
|f(x, \, k)-e^{-ikx}|\le \left|\exp\left(\frac{4}{|k|}\int \limits_x
^\infty |V(y)|\, dy\right)-1\right|
\end{align}
и
\begin{align}
\label{ostatok_jost_deriv}
|f'(x, \, k)+ike^{-ikx}|\le C(k)\int \limits_x^{+\infty} |V(y)|\, dy,
\end{align}
где $C(k)$ равномерно ограничена по $x\ge R_0$ и по $k\in K$, где
$K$ --- компактное подмножество $\R\backslash \{0\}$
[3, т. 3, гл. XI, \S 8, теорема XI.57]. Функция $f(x, \, k)$
называется {\it решением Йоста}. Дифференцируя (\ref{jost}) по $x$, можно
показать, что $f(x, \, k)$ удовлетворяет уравнению $-f''+V(x)f=k^2f$. \par
Аналогично строятся решения уравнения Шредингера с асимптотиками
$e^{\pm ikx}(1+o(1))$ при $x\rightarrow -\infty$. \par
Продолжение решения Йоста с заданной асимптотикой при $x\rightarrow
+\infty$ на отрезок $[-R_0, \, R_0]$ существует в силу следствия
\ref{exist_uniq}, а на $(-\infty, \, -R_0]$ оно продолжается как
линейная комбинация решений Йоста с асимптотиками $e^{\pm ikx}(1+\underset
{x\rightarrow -\infty}{o(1)})$. \par
Если $W_n\rightarrow W$ в $L_2$, то соответствующие решения Йоста
$f_n(x, \, k)$ с асимптотикой $f_n(x, \, k)=e^{ikx}+\underset{x\rightarrow
+\infty}{o(1)}$, равномерно сходятся к $f(x, \, k)$ по $x\in \R$ и $k\in K$,
где $K$ --- произвольный компакт в $\R\backslash \{0\}$. Действительно,
на $[R_0, \, +\infty)$ эти решения совпадают, на $[-R_0, \, R_0]$
решения равномерно сходятся в силу следствия \ref{exist_uniq}, а при
$x<-R_0$ они имеют вид $f(x, \, k)=c_+(k)f_+(x, \, k)+c_-(k)
f_-(x, \, k)$ и $f_n(x, \, k)=c_+^n(k)f_+(x, \, k)+c_-^n(k)
f_-(x, \, k)$, где $f_{\pm}(x, \, k)=e^{\pm ikx}+\underset{x\rightarrow
-\infty}{o(1)}$ --- решения Йоста.
Поэтому достаточно показать, что $c_\pm^n(k)\rightarrow c_\pm(k)$ при
$n\rightarrow \infty$ равномерно по $k\in K$. Это следует из того,
что $f_n(-R_0, \, k)\rightarrow f(-R_0, \, k)$, $f_n^{[1]}(-R_0,
\, k)\rightarrow f^{[1]}(-R_0, \, k)$ при $n\rightarrow \infty$
равномерно по $k\in K$ и вронскиан $W(f_+, \, f_-)$ отделен от 0
при $k\in K$. Последнее доказывается следующим образом:
$$W(f_+, \, f_-)=\begin{vmatrix} e^{ikx}+o(1) & e^{-ikx}+o(1) \\
ike^{ikx}+o(1) & -ike^{-ikx}+o(1)\end{vmatrix}=-2ik+o(1)=-2ik$$ в силу
постоянства вронскиана. \par
Решение $f(x, \, k)$ уравнения (\ref{jost}) существует также при
комплексных $k$ с отрицательной мнимой частью, и для него $|e^{ikx}f(x,
\, k)-1|\rightarrow 0$ и $|e^{ikx}f'(x, \, k)+ik|
\rightarrow 0$ при $x\rightarrow +\infty$ равномерно по $k\in K$,
где $K$ --- компакт в $A=\{k:{\rm Im}\, k\le 0, \; k\ne 0\}$. При этом
для каждого фиксированного $x$ функции $f(x, \, k)$ и $f'(x, \, k)$
непрерывны на множестве $A$. \par
Если $V(x)$ недостаточно быстро убывает, то часто с помощью некоторых
замен уравнение Шредингера сводят к уравнению с быстро убывающим
потенциалом [12, гл. II, \S 4]. \par
Рассмотрим уравнение
\begin{align}
\label{62shre_eq2}
y''\pm f^2(x)y=0,
\end{align}
где $f(x)>0$ и $\int \limits_0^\infty f(s)\, ds=\infty$. Сделаем замену
$$t=\int \limits _0^x f(s)\, ds, \; z(t)=\sqrt{f(x(t))}y(x(t)).$$
Утверждается, что (\ref{62shre_eq2}) приводится к виду $$-\ddot z+q(x)z=
\pm z, \; \text{где} \; q(x)=\frac{f''}{2f^3}-\frac34 \frac{f'^2}{f^4}.$$
Если $\int \limits_0^\infty |q(x)|\, dx<\infty$, то уравнение имеет пару
решений с асимптотиками $$\frac{1}{\sqrt{f(x)}}\exp\left(\pm i\int
\limits_0^x f(s)\, ds\right)(1+o(1)),$$ если в уравнении стоит $+$, и
$$\frac{1}{\sqrt{f(x)}}\exp \left(\pm \int \limits _0^x f(s)\, ds\right)
(1+o(1)),$$ если в уравнении стоит $-$. В частности, если $V(x)=1/x$, то
решения $$-y''+\frac1x y=k^2y$$ имеют асимптотики $$y_{1, \, 2}(x)\sim
\exp \left[\pm i\left(kx-\frac{1}{2k}\ln x\right)\right].$$
\subsection{Отсутствие сингулярного спектра}
\label{par_ots_sing_sp}
Отсутствие сингулярного спектра доказывается с помощью следующей
теоремы [3, т.4, теорема XIII.20]:
\begin{Trm}
Пусть $\hat H$ --- самосопряженный оператор с резольвентой $$R(\lambda)
=(\hat H-\lambda)^{-1}.$$ Пусть $(a, \, b)$ --- ограниченный интервал.
Предположим, что существует такое плотное множество $D\subset {\cal H}$,
что для каждого $\varphi\in D$ выполнено условие
\begin{align}
\sup _{\varepsilon \in (0, \, 1)}\int \limits_a^b |{\rm Im}\, \langle \varphi,
\, R(\lambda+i\varepsilon)\varphi\rangle|^p\, d\lambda<\infty
\end{align}
с некоторым $p>1$. Тогда $\hat H$ имеет только абсолютно непрерывный
спектр на $(a, \, b)$.
\end{Trm}
В частности, если для любого $\varphi\in D$ множество $$\{\langle \varphi, \,
R(\lambda+i\varepsilon)\varphi\rangle, \; \lambda\in (a, \, b), \;
\varepsilon \in (-1, \, 1)\backslash \{0\}\}$$ ограничено, то $\hat H$
имеет на $(a, \, b)$ только абсолютно непрерывный спектр. В самом деле,
для любого $\varepsilon >0$ выполнено $\langle \varphi , \, R(\lambda +
i\varepsilon)\varphi \rangle =\langle \varphi , \, R(\lambda -
i\varepsilon)\varphi \rangle ^*$ (для доказательства достаточно перейти к
представлению, в котором $\hat H$ --- оператор умножения).
\begin{Sta}
Пусть $\hat H=-\frac{d^2}{dx^2}+V$, где $V(x)=W'(x)$, $x\in (-R_0, \, R_0)$,
$W\in L_2$, и $V\in L_1(\R\backslash (-R_0, \, R_0))$. Тогда
$\hat H$ имеет на $(0, \, +\infty)$ только абсолютно непрерывный спектр.
\end{Sta}
\begin{proof}
В качестве $D$ возьмем $C_0^\infty(\R)$ и рассмотрим произвольный
интервал $(a, \, b)$, где $a$, $b>0$. Докажем, что множество $$\{\langle
\varphi, \, R(\lambda+i\varepsilon)\varphi\rangle, \; \lambda\in (a, \, b),
\; \varepsilon \in (0, \, 1)\}$$ ограничено для любого $\varphi\in
C_0^\infty(\R)$ (случай $\varepsilon \in (-1, \, 0)$ рассматривается
аналогично). Для этого достаточно показать, что множество функций
$R(\lambda+i\varepsilon)\varphi$ будет равномерно ограниченным при
$x\in \supp \varphi$, $\lambda\in (a, \, b)$ и $\varepsilon \in
(0, \, \varepsilon_0)$ для некоторого $\varepsilon _0>0$. \par
Пусть $f=R(\lambda+i\varepsilon)\varphi$. Тогда $f$ --- это решение
уравнения $\hat H f-(\lambda +i\varepsilon)f=\varphi$ и $f\in L_2$.
Пусть $(k+i\mu)^2=\lambda+i\varepsilon$, $k>0$, $\mu>0$, ${\rm supp}\,
\varphi\subset (-x_0, \, x_0)$ (можно считать, что $x_0>R_0$).
Тогда $$f(x_0)=\alpha_{\lambda,\varepsilon}f_+(x_0, \, \lambda, \,
\varepsilon), \;\;\; f(-x_0)=\beta_{\lambda,\varepsilon} f_-(x_0, \, \lambda,
\, \varepsilon),$$ где $f_{\pm}(\cdot, \, \lambda, \, \varepsilon)=f_{\pm}
(\cdot)$ --- решения уравнения $-f''+(V(x)-\lambda-
i\varepsilon)f=0$ с асимптотиками $$f_-(x)=e^{(-ik+\mu)x}+\underset{x\rightarrow
-\infty}{o(1)} \;\;\; \text{и} \;\;\; f_+(x)=e^{(ik-\mu)x}+\underset{x\rightarrow +\infty}
{o(1)}.$$ Функции $f_+(x_0, \, \lambda, \, \varepsilon)$, $f'_+(x_0, \,
\lambda, \, \varepsilon)$, $f_-(x_0, \, \lambda, \, \varepsilon)$ и $f'_-(x_0,
\, \lambda, \, \varepsilon)$ непрерывны по $(\lambda, \, \varepsilon)
\in [a, \, b]\times [0, \, 1]$ (см. [3], т. 3, гл. XI, \S 8, теорема
XI.57). Докажем, что $\alpha_{\lambda,\varepsilon}$ непрерывно зависит от
$(\lambda, \, \varepsilon)\in [a, \, b]\times [0, \, 1]$. В самом деле,
$f=f_{\text{ч}}+f_{\text{о}}$, где $f_{\text{ч}}$ --- частное решение с асимптотикой
$f_{\text{ч}}\underset{x\rightarrow +\infty}{\sim} e^{(ik-\mu)x}$, $f_{\text{о}}$ --- решение
однородного уравнения. Рассмотрим два линейно независимых решения
однородного уравнения $f_1(x)=f_1(x, \, \lambda, \, \varepsilon)$ и
$f_2(x)=f_2(x, \, \lambda, \, \varepsilon)$ такие, что $f_1(x)=
e^{(-ik+\mu)x}(1+o(1))$ при $x\rightarrow -\infty$, а $f_2$ определяется
из условия $f_2(-x_0)=1$, $W(f_1, \, f_2)=1$ (где $W(\cdot, \, \cdot)$
--- вронскиан). При $x<-x_0$ выполнено $f_{\text{ч}}(x)=c_{\lambda,\varepsilon}f_1+
d_{\lambda,\varepsilon}f_2$. Коэффициенты $c_{\lambda,\varepsilon}$
и $d_{\lambda,\varepsilon}$ находятся из системы уравнений
$$\left\{ \begin{array}{l} c_{\lambda,\varepsilon}f_1(-x_0)+
d_{\lambda,\varepsilon}f_2(-x_0)=f_{\text{ч}}(x_0), \\ c_{\lambda,\varepsilon}f'_1
(-x_0)+d_{\lambda,\varepsilon}f'_2(-x_0)=f'_{\text{ч}}(-x_0).\end{array}\right.$$
Так как $f_1(-x_0)$
и $f_1'(-x_0)$ непрерывно зависят от $(\lambda, \, \varepsilon)$, а
$f_2(-x_0)=1$ и $W(f_1, \, f_2)=1$, то $f_2'(-x_0)$ также непрерывно
зависит от $(\lambda, \, \varepsilon)$. Непрерывная зависимость
$f_{\text{ч}}(-x_0)$ от $(\lambda, \, \varepsilon)$ вытекает из следствия
\ref{exist_uniq}. Отсюда получаем, что $c_{\lambda,\varepsilon}$ и
$d_{\lambda,\varepsilon}$ непрерывны по $(\lambda,\, \varepsilon)\in
[a, \, b]\times [0, \, 1]$. \par
Функция $f_{\text{о}}$ имеет вид $\gamma_{\lambda,\varepsilon}f_1-d_{\lambda,
\varepsilon}f_2$ (тогда $f_{\text{o}}+f_{\text{ч}}\in L_2(-\infty, \, 0]$). С другой
стороны, $f_{\text{о}}(x)=\xi_{\lambda,\varepsilon}g(x,\, \lambda, \,
\varepsilon)$, где $g(x, \, \lambda, \, \varepsilon)=e^{(ik-\mu)x}+\underset
{x\rightarrow +\infty}{o(1)}$. Имеем следующую систему уравнений на $\gamma
_{\lambda,\varepsilon}$ и $\xi_{\lambda, \varepsilon}$:
$$\left\{\begin{array}{l} \gamma_{\lambda, \varepsilon}f_1(x_0, \, \lambda,
\, \varepsilon)-d_{\lambda, \, \varepsilon}f_2(x_0, \, \lambda, \,
\varepsilon)=\xi_{\lambda, \, \varepsilon}g(x_0, \, \lambda, \, \varepsilon),
\\ \gamma_{\lambda, \varepsilon}f'_1(x_0, \, \lambda, \, \varepsilon)-
d_{\lambda, \, \varepsilon}f'_2(x_0, \, \lambda, \, \varepsilon)=
\xi_{\lambda, \, \varepsilon}g'(x_0, \, \lambda, \, \varepsilon).
\end{array}\right.$$
Так как все коэффициенты непрерывны по $(\lambda, \, \varepsilon)$, то
достаточно показать, что вронскиан $W(f_1, \, g)$ отделен от 0 при
$\lambda\in [a, \, b]$ и достаточно малых $\varepsilon\ge 0$. Из
непрерывности следует, что достаточно это проверить при $\varepsilon=0$.
Пусть $g$ --- решение Йоста с асимптотиками $$g\underset{x\rightarrow +\infty}{\sim}
e^{ikx}, \;\;\; g\underset{x\rightarrow -\infty}{\sim} pe^{ikx}+qe^{-ikx}.$$ Тогда
$g^*$ --- тоже решение Йоста, и из постоянства вронскиана
$W(g, \, g^*)$ получаем равенство $p^2=q^2+1$, поэтому $|p|\ge 1$. Так как
$f_1\underset{x\rightarrow -\infty}{\sim} e^{-ikx}$, то $W(f_1, \, g)=2pki$,
так что $|W(f_1, \, g)|\ge 2k$. Таким образом, $\xi_{\lambda,\varepsilon}$,
а значит, и $\alpha_{\lambda,\varepsilon}$ непрерывно зависит от
$(\lambda,\varepsilon)$, поэтому $f(x_0)$ и $f'(x_0)$ непрерывно
зависят от $(\lambda, \, \varepsilon)$. Следовательно,
$f(x, \, \lambda, \, \varepsilon)$ непрерывна по $(x, \, \lambda, \,
\varepsilon)\in [-x_0, \, x_0]\times [a, \, b]\times [0, \, \varepsilon_0]$.
\end{proof}
Аналогично доказывается отсутствие сингулярного спектра в случае $\hat H
=-\frac{d^2}{dx^2}+U_0\theta(x)+V(x)$, где функция $V$ такая же,
как в предыдущем случае.
\subsection{Задание волновых операторов в случае быстро убывающих
потенциалов} Пусть $\hat H_0=-\frac{d^2}{dx^2}$, $\hat H=\hat
H_0+V$, где $V= W'+V_1$, $W\in L_2[-R_0, \, R_0]$, $V_1(x)\in L_1(\R)$
и выполнены неравенства
\begin{align}
\label{potent_01}
\int \limits_{-\infty}^x|V(y)|\, dy\le \frac{c}{|x|^{\frac12+\gamma}},
\; x<-R_0,
\end{align}
\begin{align}
\label{potent_02}
\int \limits_x^{+\infty}|V(y)|\, dy\le \frac{c}{x^{\frac12+\gamma}},
\; x>R_0,
\end{align}
для некоторого $\gamma>0$.
Отсюда в силу (\ref{ostatok_jost}) и (\ref{ostatok_jost_deriv}) любое
решение уравнения $-u''+V(x)u=k^2u$ имеет асимптотики
\begin{align}
\label{ostatok_jost1}
u(x, \, k)=c_1e^{ikx}+c_2e^{-ikx}+r(x, \, k),
\; \text{где} \; |r(x, \, k)|\le \frac{c(k)}{|x|^{\frac12+\gamma}}, \;
x\rightarrow \pm \infty,
\end{align}
\begin{align}
\label{ostatok_jost1_deriv}
u'_x(x, \, k)=ikc_1e^{ikx}-ikc_2e^{-ikx}+\tilde r(x, \, k),
\; \text{где} \; |\tilde r(x, \, k)|\le \frac{c(k)}{|x|^{\frac12+
\gamma}}, \; x\rightarrow \pm \infty,
\end{align}
а функция $c(k)$ равномерно ограничена на компактных подмножествах
$\R\backslash \{0\}$. \par
Рассмотрим функцию $$f(x)=\frac{1}{\sqrt{2\pi}}\int \limits_{-\infty}
^\infty e^{ikx}\varphi(k)\, dk,$$
где $\varphi\in C_0^\infty(\R\backslash \{0\})$. Пусть при $k>0$ функция $u(x, \, k)$
--- решение уравнения $-u''+V(x)u=k^2u$, имеющее асимптотики
$$u(x, \, k)\underset{x\rightarrow -\infty}{\sim} \frac{1}{\sqrt{2\pi}}e^{ikx}+
\alpha e^{-ikx},$$ $$u(x, \, k)\underset{x\rightarrow
+\infty}{\sim} \beta e^{ikx}$$ (при $x\in [-R_0, \, R_0]$ здесь подразумевается, что
$u$ --- это решение уравнения $-(u^{[1]})'-W(x)u^{[1]}-W^2(x)u+V_1(x)u=k^2u$).
Коэффициенты $\alpha$ и $\beta$ находятся однозначно.
При $k<0$ функцию $u(x, \, k)$ выбираем как решение с асимптотиками
$$u(x, \, k)\underset{x\rightarrow +\infty}{\sim} \frac{1}{\sqrt{2\pi}}e^{ikx}+\tilde \alpha e^{-ikx},$$
$$u(x, \, k)\underset{x\rightarrow -\infty}{\sim} \tilde \beta e^{ikx}.$$
Положим
\begin{align}
\label{fplus}
f_+(x)=\int \limits_{-\infty}^\infty u(x, \, k)\varphi(k)\, dk.
\end{align}
Эта функция абсолютно непрерывна вместе с квазипроизводной.
Из (\ref{ostatok_jost1}), (\ref{ostatok_jost1_deriv}), условия $W\in L_2$
и свойств преобразования Фурье следует, что $f_+\in L_2(\R)$ и $f'_+
\in L_2(\R)$. Дифференцируя интеграл в (\ref{fplus}) по $x$,
получаем, что
\begin{align}
\label{hf_plus}
\left(-\frac{d^2}{dx^2}+V(x)\right)f_+(x)=\int \limits_{-\infty}
^\infty u(x, \, k)k^2\varphi(k)\, dk.
\end{align}
Эта функция принадлежит $L_2(\R)$. Докажем, что $f_+\in D(\hat H)$.
Пусть $$D_0=\{f\in W^1_2(\R):f^{[1]}\in AC, \; -(f^{[1]})'-W(x)f^{[1]}
-W^2(x)f+V_1f\in L_2(\R)\}.$$
Возьмем гладкую функцию $\eta$ со значениями в $[0, \, 1]$, $\supp \eta
\subset [-2, \, 2]$, $\eta|_{[-1, \, 1]}\equiv 1$. Положим $\eta _n(x)=
\eta\left(\frac{x}{n}\right)$. Тогда $\eta_n f\rightarrow f$ и $\hat H
(\eta_n f)\rightarrow -(f^{[1]})'-W(x)f^{[1]}-W^2(x)f+V_1f$ в $L_2(\R)$
для любого $f\in D_0$. Так как $\eta_nf\in D(\hat H)$ и оператор $\hat H$
замкнут, то $D_0\subset D(\hat H)$. \par
Пусть $D$ --- множество функций $f_+$, заданных равенством (\ref{fplus})
с $\varphi\in C_0^\infty(\R\backslash \{0\})$, $\overline D$ --- его замыкание.
Тогда $\hat H|_D:D\rightarrow \overline D$ существенно самосопряжен. Для этого
достаточно показать, что $\ran (\hat H|_D\pm i)$ плотно в $\overline D$.
В самом деле, пусть $\psi\in C_0^\infty (\R\backslash \{0\})$. Тогда
$\psi(k)=(k^2+i)\varphi(k)$, где $\varphi \in C_0^\infty(\R\backslash \{0\})$.
Значит, $$\int \limits _{-\infty}^\infty u(x, \, k)\psi(k)\, dk=\int \limits
_{-\infty}^\infty u(x, \, k)(k^2+i)\varphi(k)\, dk=(\hat H+i)\int \limits
_{-\infty}^{\infty}u(x, \, k)\varphi(k)\, dk\in \ran (\hat H|_D+i),$$
то есть $D\subset \ran (\hat H|_D+i)$. Аналогично доказывается, что
$D\subset \ran (\hat H|_D-i)$.
\begin{Sta}
Отображение $f\mapsto f_+$ задает волновой оператор $\Omega_+$, то есть
$$\|e^{-it\hat H}f_+-e^{-it\hat H_0}f\|_{L_2}\rightarrow 0, \; t\rightarrow
-\infty.$$
\end{Sta}
\begin{proof}
Ясно, что $$e^{-it\hat H_0}f=\frac{1}{\sqrt{2\pi}}\int \limits_{-\infty}
^\infty e^{ikx-itk^2}\varphi(k)\, dk.$$ Докажем, что $$e^{-it\hat H}f_+
=\int \limits_{-\infty}^\infty u(x, \, k)e^{-itk^2}\varphi(k)\, dk.$$
Определим на множестве $D$ функций $f_+$ вида (\ref{fplus}) оператор
$U(t)f_+=\sum \limits_{n=0}^\infty \frac{(-it\hat H)^n}{n!}f_+$.
Докажем, что $U(t)f_+(x)=\int \limits_{\R}u(x, \, k)e^{-itk^2}\varphi(k)
\, dk$ и что $U(t)=e^{-it\hat H}$. Из (\ref{hf_plus}) следует, что
$$\hat Hf_+(x)=\int \limits_{-\infty}^\infty u(x, \, k)k^2\varphi(k)\, dk.$$
Так как $\varphi\in C_0^\infty(\R\backslash\{0\})$, а $u(x, \, k)$
ограничена по $k$ на компактных подмножествах $\R\backslash\{0\}$, то
по теореме Фубини $$U(t)f_+(x)=\sum \limits_{n=0}^\infty \frac{(-it)^n}
{n!}\int \limits_{-\infty}^\infty u(x, \, k)k^{2n}\varphi(k)\, dk
=\int \limits_{-\infty}^\infty u(x, \, k)e^{-itk^2}\varphi(k)\, dk.$$
Отсюда видно, что $U(t)f_+\in D$ и $U(t+s)=U(t)U(s)$ для любых $t$,
$s\in \R$. Докажем, что $U(t)$ является унитарным, то есть $U(t)^*=
U(-t)$. Пусть $N\in \N$. Тогда для любых $f_+$, $g_+\in D$
$$\langle f_+, \, U(t)^*g_+\rangle=\langle U(t)f_+, \,
g_+\rangle=$$ $$=\sum \limits_{n=0}^N \left \langle \frac{(-it\hat H)^n}
{n!}f_+, \, g_+\right\rangle+\left\langle \sum \limits_{n=N+1}^\infty
\frac{(-it\hat H)^n}{n!}f_+, \, g_+\right\rangle=$$ $$=\sum \limits_{n=0}^N
\left\langle f_+, \, \frac{(it\hat H)^n}{n!}g_+\right\rangle+\left\langle
\sum \limits_{n=N+1}^\infty \frac{(-it\hat H)^n}{n!}f_+, \, g_+
\right\rangle.$$ С другой стороны, $$\langle f_+, \, U(-t)
g_+\rangle=\sum \limits_{n=0}^N \left\langle f_+, \, \frac{(it\hat
H)^n}{n!}g_+\right\rangle+\left\langle \sum \limits_{n=N+1}^\infty f_+, \,
\frac{(it\hat H)^n}{n!}g_+\right\rangle.$$ Таким образом, нужно доказать, что
для любого $f_+\in D$ и $t\in \R$ $$\sum \limits_{n=N+1}^\infty \frac{(-it\hat
H)^n}{n!}f_+\rightarrow 0, \; N\rightarrow \infty .$$ Положим $\eta_N(k)
=e^{-itk^2}-\sum \limits_{n=0}^N\frac{(-itk^2)^n}{n!}$. Тогда
$$f_N(x):=\sum \limits_{n=N+1}^\infty \frac{(-it\hat H)^n}{n!}f_+(x)= \int
\limits _{-\infty}^\infty u(x, \, k)\eta_N(k)\varphi(k)\, dk.$$
Пусть ${\rm supp}\, \varphi\subset K$, где $K$ --- компактное подмножество
$\R\backslash\{0\}$. Так как $\eta_N$ равномерно стремится к 0 на $K$,
а $u(x, \, k)$ равномерно ограничена, то $f_N$ равномерно сходится к 0
на $[-R_0, \, R_0]$. Пусть $x>R_0$. Тогда $$u(x, \, k)=\beta e^{ikx}+
\frac{c(x, \, k)}{x^{\frac12+\gamma}},$$ где $c(x, \, k)$ равномерно
ограничена по $x\ge R_0$ и $k\in K$. Интегрируя по частям, получаем
$$\int \limits_{-\infty}^\infty e^{ikx}\eta_N(k)\varphi(k)\, dk=
\frac{i}{x}\int \limits_{-\infty}^\infty e^{ikx}(\varphi(k)\eta_N(k))'
\, dk=$$ $$=\frac{i}{x}\int \limits_{-\infty}^\infty e^{ikx}(\varphi'(k)
\eta_N(k)+\varphi(k)\eta_{N-1}(k)(-2itk))\, dk=\frac{c(x)\alpha_N}{x},$$
где $\alpha_N\rightarrow 0$ при $N\rightarrow\infty$, а $c(x)$ ограничена при
$x\ge R_0$. Далее, $$\int \limits_{-\infty}^\infty \frac{c(x, \, k)}
{x^{\frac12+\gamma}}\eta_N(k)\varphi(k)\, dk=\frac{\beta_Nd(x)}{x^{\frac12
+\gamma}},$$ где $\beta_N\rightarrow 0$ при $N\rightarrow\infty$ и $d(x)$
ограничена. Значит, $|f_N(x)|\le {\rm const}\left(\frac{\alpha_N}{x}
+\frac{\beta_N}{x^{\frac12+\gamma}}\right)$ при $x\ge R_0$. Аналогично
оценивается $|f_N(x)|$ при $x\le -R_0$. Отсюда следует, что $f_N
\rightarrow 0$ при $N\rightarrow \infty$ в $L_2(\R)$. Унитарность $U(t)$
доказана. \par
Аналогично доказывается, что $U(t)f_+\stackrel{L_2(\R)}{\rightarrow}
f_+$ и $\frac{U(t)-1}{t}f_+\stackrel{L_2(\R)}{\rightarrow}
-i\hat Hf_+$ при $t\rightarrow 0$ для любого $f_+\in D$. Продолжим для каждого
$t\in \R$ оператор $U(t)$ по непрерывности на замыкание $D$. Тогда
$\{U(t)\}$ --- однопараметрическая группа из унитарных операторов.
Докажем, что $U(t)g\rightarrow g$ при $t\rightarrow 0$ для любого $g\in
\overline{D}$. Зафиксируем $\varepsilon>0$. Тогда существует функция
$f_+\in D$ такая, что $\|f_+-g\|<\varepsilon$. Значит, $\|U(t)g-g\|\le
\|U(t)g-U(t)f_+\|+\|U(t)f_+-f_+\|+\|f_+-g\|\le \|U(t)f_+-f_+\|+2\|f_+-g\|
\le 3\varepsilon$ при достаточно малых $t$. \par
Итак, $\{U(t)\}$ --- сильно непрерывная однопараметрическая группа
унитарных операторов. По теореме Стоуна, существует самосопряженный оператор
$A$ такой, что $U(t)=e^{-itA}$. Так как $(U(t)f_+)'|_{t=0}=-i\hat Hf_+$
для $f_+\in D$ и $\hat H|_D$ существенно самосопряжен, то $A=\hat H$ и $U(t)=
e^{-it\hat H}$. \par
Отсюда $$h(t, \, x):=e^{-it\hat H}f_+(x)-e^{-it\hat H_0}f(x)=\int \limits
_{-\infty}^\infty \left(u(x, \, k)-\frac{1}{\sqrt{2\pi}}e^{ikx}\right)
e^{-itk^2}\varphi(k)\, dk.$$
По теореме Римана--Лебега, для любого $x\in \R$ $h(x, \, t)\rightarrow 0$
при $t\rightarrow \infty$. При $x\in [-R_0, \, R_0]$ и $t\in \R$
функция $h(x, \, t)$ равномерно ограничена, поэтому по теореме Лебега
об ограниченной сходимости $$\int \limits_{[-R_0, \, R_0]}|h(x, \, t)|^2
\, dx\rightarrow 0, \; t\rightarrow \infty.$$ При $x<-R_0$
$$u(x, \, k)-\frac{1}{\sqrt{2\pi}}e^{ikx}=\alpha e^{-ikx}+\frac{c(x, \,
k)}{|x|^{\frac12+\gamma}},$$ при $x>R_0$
$$u(x, \, k)-\frac{1}{\sqrt{2\pi}}e^{ikx}=\left(\beta-\frac{1}{\sqrt{2\pi}}
\right) e^{ikx}+\frac{c(x, \, k)}{x^{\frac12+\gamma}}$$ (в случае $k>0$).
По теореме об ограниченной сходимости,
$$\int \limits_{|x|>R_0}\left|\int \limits_{-\infty}^\infty
\frac{c(x, \, k)}{|x|^{\frac12+\gamma}}e^{-itk^2}\varphi(k)\, dk\right|
^2\, dx\rightarrow 0, \; t\rightarrow \infty.$$
Осталось доказать, что $$\int \limits_{x<-R_0}\left|\int \limits_{-\infty}
^0 \left(\tilde \beta-\frac{1}{\sqrt{2\pi}}\right) e^{ikx-itk^2}
\varphi(k)\, dk+\int \limits_0^{+\infty}\alpha e^{-ikx-itk^2}\varphi(k)
\, dk\right|^2\, dx+$$
$$+\int \limits_{x>R_0}\left|\int \limits_{-\infty}^0 \tilde \alpha
e^{-ikx-itk^2}\varphi(k)\, dk+\int \limits_0^{+\infty}\left(\beta-\frac{1}
{\sqrt{2\pi}}\right)e^{ikx-itk^2}\varphi(k)\, dk\right|^2\, dx\rightarrow 0,
\; t\rightarrow -\infty.$$ Это вытекает из следующего утверждения
[3, т. 3, гл. XI, \S 3, лемма 3]: если $s(k)$ --- такая функция на
интервале $(a, \, b)$, что $s''\in L_1$ и $s'>0$, $\varphi\in L_2(a, \, b)$,
то $$\int \limits_0^\infty \left|\int \limits_{(a, \, b)}e^{ikx-its(k)}
\varphi(k)\, dk\right|^2\, dx\rightarrow 0, \; t\rightarrow -\infty.$$
\end{proof}
Аналогично строится волновой оператор $\Omega_-$. \par
Из существования волнового оператора $\Omega_+$ следует, что оператор,
получающийся при ограничении $\hat H$ на некоторое подпространство
${\cal H}_{{\rm ac}}$, унитарно эквивалентен $\hat H_0$. Значит, кратность
абсолютно непрерывного спектра оператора $\hat H$ на $\R_+$ не меньше 2.
С другой стороны, раньше было показано, что кратность спектра не больше 2
и, кроме того, отсутствует сингулярный спектр. В итоге получаем следующее
утверждение.
\begin{Trm}
\label{kratn01}
Пусть $\hat H=-\frac{d^2}{dx^2}+V(x)$, где $V(x)=W'(x)$ при $x\in [-R_0, \,
R_0]$, $W\in L_2[-R_0, \, R_0]$, и $V\in L_1(\{x:|x|>R_0\})$, $V(x)
\rightarrow 0, \; |x|\rightarrow \infty$. Пусть выполнены
неравенства (\ref{potent_01}) и (\ref{potent_02}). Тогда при $E>0$
оператор $\hat H$ имеет двукратный спектр, причем спектральная мера
эквивалентна мере Лебега, а при $E<0$ спектр дискретный.
\end{Trm}
Пусть теперь $\hat H_0=-\frac{d^2}{dx^2}+U_0\theta(x)$, где $\theta(\cdot)$
--- функция Хевисайда, $\hat H=\hat H_0+V$, где потенциал $V$ имеет те
же свойства, что и в теореме \ref{kratn01}. Построим волновые операторы
на $P_{[U_0, \, +\infty)}{\cal H}$, где $P_{[U_0, \, +\infty)}$ ---
спектральный проектор оператора $\hat H$ (в энергетическом представлении
это оператор умножения на характеристическую функцию множества $[U_0,
\, +\infty)$). Если будет доказано, что волновые операторы существуют,
то отсюда будет следовать, что на $P_{[U_0, \, +\infty)}{\cal H}$
спектр $\hat H$ будет иметь кратность не меньше 2. Пусть $u_0(x, \, q)$
--- решение уравнения $-u''+U_0\theta(x)u=(q^2+U_0)u$ такое, что
$u_0(x, \, q)=c(q)e^{iqx}$, $x\ge 0$. Коэффициент $c(q)$ подбирается так,
чтобы отображение $$\varphi(q)\mapsto \int \limits_{-\infty}^\infty
u_0(x, \, q)\varphi(q)\, dq$$ задавало изометрию из $L_2(\R)$ в $L_2(\R)$
(при этом образом будет $P_{[U_0, \, +\infty)}{\cal H}$ в силу утверждения
\ref{utv_stenka}). Пусть $u(x, \, q)$ --- решение уравнения $-u''+
(U_0\theta(x)+V(x))u=(q^2+U_0)u$, асимптотики которого выбраны так, чтобы:
\begin{enumerate}
\item при $q>0$ $u(x, \, q)-u_0(x, \, q)\underset{x\rightarrow +\infty}{\sim}
\alpha(q)e^{iqx}$, $u(x, \, q)-u_0(x, \, q)\underset{x\rightarrow -\infty}
{\sim}\beta(q)e^{-i\sqrt{q^2+U_0}x}$;
\item при $q<0$ $u(x, \, q)-u_0(x, \, q)\underset{x\rightarrow +\infty}{\sim}
\tilde \alpha(q)e^{-iqx}$, $u(x, \, q)-u_0(x, \, q)\underset{x\rightarrow -\infty}
{\sim}\tilde \beta(q)e^{i\sqrt{q^2+U_0}x}$.
\end{enumerate}
Дальше точно так же, как для оператора $-\frac{d^2}{dx^2}$, доказывается,
что $$\Omega_+\int \limits _{-\infty}^{\infty}u_0(x, \, q)\varphi(q)\, dq
=\int \limits_{-\infty}^\infty u(x, \, q)\varphi(q)\, dq.$$
Аналогично строится волновой оператор на $P_{[0, \, U_0]}{\cal H}$.
В итоге получаем
\begin{Trm}
Пусть $\hat H=-\frac{d^2}{dx^2}+U_0\theta(x)+V$, где функция $V$ такая же,
как в теореме \ref{kratn01}. Тогда при $E<U_0$ спектр $\hat H$
однократный, а при $E>U_0$ двукратный; в обоих случаях спектральная
мера эквивалентна мере Лебега. При $E<0$ спектр дискретный.
\end{Trm}
\subsection{Периодический потенциал}
Пусть $V$ --- периодическая обобщенная функция вида $V=W'$, где $W\in
L_2^{{\rm loc}}$. Тогда существует такая константа $\gamma$, что
$W_0(x)=W(x)-\gamma x$ является периодической. В самом деле, пусть
$x_0$ --- период функции $V$. Зафиксируем
функцию $\varphi_1\in C_0^\infty(\R)$ такую, что $\int \varphi_1(x)\, dx
=1$. Тогда для любой функции $\varphi\in C_0^\infty(\R)$ имеет
место разложение $\varphi=\varphi_0+\varphi_1\int \varphi(x)\, dx$,
где $\varphi_0\in C_0^\infty(\R)$ и $\int \varphi_0(x)\, dx=0$.
Значит, $\varphi_0=\psi'$, где $\psi\in C_0^\infty$ и
$$\int W(x+x_0)\varphi(x)\, dx=\int W(x+x_0)\psi'(x)\, dx+\int \varphi(t)
\, dt\int W(x+x_0)\varphi_1(x)\, dx=$$ $$=\int W(x)\psi'(x)\, dx+
\int \varphi(t)\, dt\int W(x+x_0)\varphi_1(x)\, dx=$$ $$=\int W(x)
\varphi(x)\, dx+\int \varphi(x)\, dx\int (W(y+x_0)-W(x))\varphi_1(x)\, dx,$$
то есть $W(x+x_0)=W(x)+c$, так что $W(x)-c\frac{x}{x_0}$ является
периодической функцией. \par
Рассмотрим дифференциальное выражение $l(\varphi)=-(\varphi^{[1]})'-W_0(x)
\varphi^{[1]}-W_0^2(x)\varphi +\gamma \varphi$ и порожденную им квадратичную
форму $$q(\varphi)=-\int [(\varphi ^{[1]})'+W_0(x)\varphi ^{[1]}+W_0^2(x)
\varphi +\gamma \varphi]\varphi ^*\, dx=\int (|\varphi '|^2-W_0(x)(\varphi '^*
\varphi +\varphi ^*\varphi ')+\gamma |\varphi |^2)\, dx.$$ Докажем, что
существует $a<1$ такое, что
$$\left|\int W_0(x)(\varphi'^*\varphi+\varphi^*\varphi')\right|\, dx\le
a\int |\varphi'|^2\, dx+b\int |\varphi|^2\, dx$$
для любого $\varphi\in W^1_2(\R)$. Действительно,
$$\left|\int W_0(x)(\varphi'^*\varphi+\varphi^*
\varphi')\, dx\right|=\left|\sum\limits_{n\in \Z}\int\limits_{[nx_0, \,
(n+1)x_0]} W_0(x)(\varphi'^*\varphi+\varphi^*\varphi')\, dx\right|\le$$
$$\le \sum \limits_{n\in \Z}2\|W_0\|_{L_2[nx_0, \, (n+1)x_0]}\|\varphi\|
_{C[nx_0, \, (n+1)x_0]}\|\varphi'\|_{L_2[nx_0, \, (n+1)x_0]}=:A.$$
По неравенству Соболева, для любого $\varepsilon>0$ существует такая
константа $C>0$ (не зависящая от $n$), что $\|\varphi\|_{C[nx_0, \,
(n+1)x_0]}\le \varepsilon\|\varphi'\|_{L_2[nx_0, \, (n+1)x_0]}+C\|\varphi\|
_{L_2[nx_0, \, (n+1)x_0]}$. Далее, $$\|\varphi\|_{L_2[nx_0, \, (n+1)x_0]}
\|\varphi'\|_{L_2[nx_0, \, (n+1)x_0]}\le \frac{\varepsilon}{C}\|\varphi'\|^2
_{L_2[nx_0, \, (n+1)x_0]}+\frac{C}{\varepsilon}\|\varphi\|^2_{L_2[nx_0, \,
(n+1)x_0]}.$$ Отсюда и из периодичности функции $W_0(x)$ получаем
$$A\le 2\|W_0\|_{L_2[0, \, x_0]}\sum \limits_{n\in \Z}(2\varepsilon
\|\varphi'\|^2_{L_2[nx_0, \, (n+1)x_0]}+\frac{C^2}{\varepsilon}\|\varphi\|^2
_{L_2[nx_0, \, (n+1)x_0]})=$$ $$=4\varepsilon\|W_0\|_{L_2[0, \, x_0]}
\|\varphi'\|^2_2+2\frac{C^2}{\varepsilon}\|W_0\|_{L_2[0, \, x_0]}
\|\varphi\|_2^2.$$
При достаточно малых $\varepsilon$ коэффициент при $\|\varphi'\|_2^2$
будет меньше 1. \par
По теореме \ref{klmn}, область определения формы $q$ совпадает с
$W^1_2(\R)$. При этом $q$ однозначно порождает самосопряженный
оператор $\hat H$ с областью определения $$D(\hat H)=\left\{\psi\in
W^1_2(\R)|\exists g\in L_2(\R):\forall \varphi\in W^1_2(\R)\;
q(\psi, \, \varphi)=\langle g, \, \varphi\rangle_{L_2}\right\}$$
([3], доказательство теоремы VIII.15).
Отсюда выводится, что $D(\hat H)$ состоит из функций, абсолютно
непрерывных вместе с квазипроизводными и таких, что $l(\psi)\in L_2$. \par
Найдем спектр оператора $\hat H$ и его обобщенные собственные векторы.
Оснащение строим с помощью оператора $K\varphi=-(\varphi^{[1]})'-
W_0(x)\varphi^{[1]}-W_0^2(x)\varphi+c_0(1+x^2)\varphi$, где $c_0$ ---
достаточно большое число, такое, что $K$ --- положительно определенный
оператор и обратный к нему является оператором Гильберта--Шмидта. Так же,
как и в \S \ref{regul_par}, доказывается, что обобщенный собственный
вектор имеет вид $F(\varphi)=\int (c_1\varphi_1+c_2\varphi_2)\varphi\,
dx$, где $\varphi_1$, $\varphi_2$ --- линейно независимые решения уравнения
$\hat H\varphi_j=E\varphi_j$, а $\varphi\in {\cal H}_+$ и имеет компактный
носитель. Кроме того, можно показать, что множество таких функций
$\varphi$ плотно в $L_2(\R)$, так что преобразование Фурье однозначно
продолжается на все $L_2$ и кратность спектра не больше 2. \par
Так как потенциал $V$ является периодическим с периодом $x_0$, то $\hat H$
перестановочен с оператором сдвига $T_{x_0}\varphi(x)=\varphi(x-x_0)$.
Этот оператор унитарный, причем он и его обратный оставляют подпространство
$D(\hat H)$ инвариантным. Значит, в силу следствия \ref{cor_maurin},
обобщенные собственные векторы $F$ оператора $\hat H$
можно искать в виде общих обобщенных собственных векторов операторов
$T_{x_0}$ и $\hat H$, то есть $F(x-x_0)=\lambda F(x)$. Докажем, что
$|\lambda|=1$. В самом деле, пусть $\varphi\in {\cal H}_+$.
Тогда $T_{x_0}\varphi\in {\cal H}_+$ и $U(T_{x_0}\varphi)(m)=\lambda(m)
(U\varphi)(m)$, где $U$ --- оператор перехода к представлению, в котором
$T_{x_0}$ и $\hat H$ являются операторами умножения на функцию. Так как
$T_{x_0}$ унитарный, то $|\lambda(m)|=1$. Оператор $U$ определяется по
формуле $U\varphi(m)=\int F(m, \, x)\varphi(x)\, dx$,
где $F(m, \, x)$ --- обобщенный собственный вектор, $\varphi\in {\cal H}_+$
и $$\int F(m, \, x+x_0)\varphi(x)\, dx=\int F(m, \, y)\varphi(y-x_0)\, dy=
U(T_{x_0}\varphi)(m)=\lambda(m)\int F(m, \, x)\varphi(x)\, dx,$$ так что
$F(m, \, x+x_0)=\lambda(m)F(x)$. \par
Итак, если $E$ принадлежит спектру, то существует решение уравнения
$\hat HF(x)P(x)$ такое, что $F(x-x_0)=\lambda(E)F(x)$, $|\lambda(E)|
=1$. Пусть $F_{1,E}$, $F_{2,E}$ --- два линейно независимых решения
уравнения $\hat HFP$ и на $[0, \, x_0]$ выполнено $F=c_1F_{1,E}+c_2F_{2,E}$.
Тогда из непрерывности $F$ и $F^{[1]}$ получаем $$\left\{ \begin{array}{l}
c_1F_{1,E}(0)+c_2F_{2,E}(0)=\lambda(E)(c_1F_{1,E}(x_0)+c_2F_{2,E}(x_0)),
\\ c_1F_{1,E}^{[1]}(0)+c_2F_{2,E}^{[1]}(0)=\lambda(E)(c_1F_{1,E}^{[1]}
(x_0)+c_2F_{2,E}^{[1]}(x_0)).\end{array}\right.$$ Нетривиальное решение
этой системы существует тогда и только тогда, когда
\begin{align}
\label{zone}
\begin{vmatrix} F_{1,E}(0)-\lambda(E)F_{1,E}(x_0) &
F_{2,E}(0)-\lambda(E)F_{2,E}(x_0) \\ F_{1,E}^{[1]}(0)-\lambda(E)F_{1,E}
^{[1]}(x_0) & F_{2,E}^{[1]}(0)-\lambda(E)F_{2,E}^{[1]}(x_0)
\end{vmatrix}=0.
\end{align}
Отсюда получаем уравнение на $E$. Если $E$ принадлежит спектру, то
$|\lambda(E)|=1$. Обратно, если $|\lambda(E)|=1$, то $E$ принадлежит
спектру в силу следствия \ref{cor_bound}. Множество $$\{E:|\lambda(E)|=1\}$$
образует энергетические зоны. \par
В случае, если $\lambda(E)\ne \pm 1$, $F(x)$ и $F^*(x)$ являются двумя
линейно независимыми обобщенными собственными функциями, так что
эта область спектра двукратна. Если $\{E:\lambda(E)=\pm 1\}$ является
дискретным множеством, то спектр и его кратность найдены. \par
{\bf Пример.} Пусть $$\hat H=-\frac{\hbar^2}{2m}\frac{d^2}{dx^2}+
\frac{\hbar^2\varkappa}{m}\sum \limits_{n=-\infty}^\infty \delta(x+na).$$
Тогда $F_{j,E}$ можно выбрать в виде $F_{1,E}(x)=e^{ikx}$, $F_{2,E}(x)=e^{-ikx}$,
где $\hbar^2k^2=2mE$. Условие (\ref{zone}) в этом случае
записывается в виде $$\begin{vmatrix} \lambda-e^{ika} & \lambda-e^{-ika}
\\ \lambda-e^{ika}\left(1-\frac{2i\varkappa}{k}\right) & -\lambda+
e^{-ika}\left(1+\frac{2i\varkappa}{k}\right)\end{vmatrix},$$ откуда
получаем, что $|\cos ka+\frac{\varkappa}{k}\sin ka|\le 1$, то есть
разрешенные значения $k$ образуют бесконечную последовательность
интервалов, расширяющуюся с ростом $k$.
\subsection{Когерентные состояния для гармонического осциллятора}
Пусть $a^+$ и $a$ --- операторы рождения и уничтожения, построенные
по оператору Шредингера для гармонического осциллятора.
\begin{Def}
Когерентными состояниями называются собственные векторы оператора $a$.
\end{Def}
В координатном представлении они удовлетворяют уравнению
$$\frac{1}{\sqrt{2}}\left(q+\frac{d}{dq}\right)\psi_\alpha(q)=\alpha
\psi_\alpha(q).$$
Сделав сдвиг $\tilde q=q-\sqrt{2}\alpha$, получаем уравнение для $\psi_0
(\tilde q)$. Следовательно,
$$\psi_\alpha(q)=N_\alpha e^{-(q-\sqrt{2}\alpha)^2/2},$$ где $N_\alpha$ ---
нормировочная постоянная. Решение, как видно, существует при всех
$\alpha\in \C$. \par
{\bf Замечание.} Собственных векторов оператора $a^+$ не существует
(решением уравнения будет экспонента с положительным показателем). \par
Пусть $|n\rangle$ --- собственные векторы гармонического осциллятора,
$|\alpha\rangle$ --- когерентные состояния. Разложим $|\alpha\rangle$
по базису $|n\rangle$: $$|\alpha\rangle=\sum\limits_{n=0}^\infty
C_n|n\rangle.$$
Подставим это разложение в уравнение
\begin{align}
\label{coher}
a|\alpha\rangle=\alpha|\alpha\rangle
\end{align}
и получим
\begin{align}
\label{coher_razl}
a\left(\sum\limits_{n=0}^\infty C_n|n\rangle\right)=\alpha
\left(\sum\limits_{n=0}^\infty C_n|n\rangle\right).
\end{align}
Если формально внести оператор $a$ под знак суммирования, то получается
соотношение на коэффициенты: $C_{n+1}=\frac{\alpha}{\sqrt{n+1}}C_n$,
откуда $C_n=\frac{\alpha^n}{\sqrt{n!}}C_0$ (это следует из того, что
$a|n\rangle=\sqrt{n}|n-1\rangle$). Так как решение (\ref{coher})
единственно, то коэффициенты $C_n$ находятся однозначно, поэтому
достаточно проверить, что при $C_n=\frac{\alpha^n}{\sqrt{n!}}C_0$ будет
выполнено (\ref{coher_razl}). Покажем, что $a\left(\sum\limits_{n=0}^\infty
C_n|n\rangle\right)=\sum\limits_{n=0}^\infty C_na|n\rangle$. Ряд в
правой части сходится в гильбертовом пространстве, поэтому достаточно
проверить, что $a\left(\sum\limits_{n=0}^\infty C_n|n\rangle\right)=\lim\limits
_{k\rightarrow\infty}\sum\limits_{n=0}^{n_k}C_na|n\rangle$, где
$\{n_k\}$ --- некоторая последовательность в $\N$. \par
Перейдем к координатному представлению. Докажем, что при почти всех $q$
выполнено равенство
\begin{align}
\label{aalpha}
\frac{1}{\sqrt{2}}\left(q+\frac{d}{dq}\right)
\left(\sum\limits_{n=0}^\infty C_n\psi_n(q)\right)=\frac{1}{\sqrt{2}}
\sum\limits_{n=0}^\infty C_n\left(q+\frac{d}{dq}\right)\psi_n(q),
\end{align}
где под суммой в правой части подразумевается предел некоторой
последовательности частичных сумм. Для этого достаточно проверить
локальную равномерную сходимость этих частичных сумм. Так как ряд
$\sum \limits_{n=0}^\infty C_na|n\rangle$ сходится в гильбертовом
пространстве, то ряд в правой части (\ref{aalpha}) сходится в $L_2(\R)$,
откуда следует, что существует последовательность его частичных сумм,
которая сходится почти всюду. Пусть есть сходимость в точке $q$. Тогда
$$\sum\limits_{n=k_1}^{k_2}\frac{|\alpha|^n|C_0|}{\sqrt{n!}}|a(\psi_n
(q+\Delta q)-\psi_n(q))|=\sum\limits_{n=k_1}^{k_2}\frac{|\alpha|^n|C_0|}
{\sqrt{(n-1)!}}|\psi_{n-1}(q+\Delta q)-\psi_{n-1}(q)|=$$
$$=\sum\limits_{n=k_1}^{k_2}\frac{|\alpha|^n|C_0|}
{\sqrt{(n-1)!}}\left|\int\limits_q^{q+\Delta q}\psi'_{n-1}(s)\, ds\right|\le
\const\sum\limits_{n=k_1}^{k_2}\frac{|\alpha|^n|C_0|}{\sqrt{(n-1)!}}
(\max_{s\in [q, \, q+\Delta q]}|s|\|\psi_{n-1}\|_{L_2}\sqrt{|\Delta q|}+$$
$$+\sqrt{|\Delta q|}\sqrt{n-1}\|\psi_{n-2}\|_{L_2})\le C(q)\sqrt{|\Delta q|}
\sum\limits_{n=k_1}^{k_2}\frac{|\alpha|^n|C_0|}{\sqrt{(n-2)!}}$$
(предпоследнее неравенство следует из того, что $\frac{d}{ds}=a\sqrt{2}-s$
и из неравенства Коши--Буняковского). Так как ряд $\sum\limits_{n=0}^\infty
\frac{|\alpha|^n}{\sqrt{(n-2)!}}$ сходится, то из критерия Коши
следует локальная равномерная сходимость последовательности $\sum\limits
_{n=0}^{n_k} C_na\psi_n(q)$ при $k\rightarrow \infty$. \par
Итак, мы получили, что $$|\alpha\rangle=C_0\sum\limits_{n=0}^\infty
\frac{\alpha^n}{\sqrt{n!}}|n\rangle=C_0\sum\limits_{n=0}^\infty
\frac{(\alpha a^+)^n}{n!}|0\rangle=C_0e^{\alpha a^+}|0\rangle.$$
Оператор $e^{\alpha a^+}$ определен на всюду плотном множестве. В
самом деле, для любого $k\in \Z_+$
$$\sum \limits_{n=0}^\infty\frac{(\alpha a^+)^n}{n!}|k\rangle
=\sum \limits_{n=0}^\infty\frac{\alpha^n\sqrt{(n+k)!}}{n!\sqrt{k!}}
|n+k\rangle.$$
Так как коэффициенты этого ряда принадлежат $l_2$, то он сходится в ${\cal
H}$, поэтому $e^{\alpha a^+}$ определен на базисных векторах
${\cal H}$ и их конечных линейных комбинациях. \par
Осталось определить постоянную $C_0$. Из равенства $$1=\langle \alpha|
\alpha\rangle =C_0^2\sum \limits_{n=0}^\infty \frac{|\alpha|^{2n}}{n!}=C_0^2
e^{|\alpha|^2}$$ получаем $C_0=e^{-|\alpha|^2/2}$. Таким образом,
$$|\alpha\rangle=e^{-\alpha\alpha^*/2}e^{\alpha a^+}|0\rangle.$$
Скалярное произведение пары векторов $|\alpha\rangle$ и $|\alpha'\rangle$
равно $$\langle\alpha'|\alpha\rangle=e^{-(|\alpha|^2+|\alpha'|^2)/2}
\sum\limits_{m, \, n=0}^\infty \frac{\alpha'^{*m}\alpha^n}{\sqrt{m!n!}}
\langle m|n\rangle=\exp(\alpha'^*\alpha-\frac12(|\alpha|^2+|\alpha'|^2)),$$
откуда $|\langle\alpha'|\alpha\rangle|=e^{-\frac12|\alpha-\alpha'|^2}\ne 0$,
то есть никакая подсистема векторов $|\alpha\rangle$ не является
ортогональной. \par
\begin{Trm}
Для любого вектора $|\psi\rangle\in {\cal H}$ существует аналитическая
функция $B_{|\psi\rangle}$ такая, что для любого $|\varphi\rangle$
выполнено равенство
\begin{align}
\label{predst_a}
\langle \varphi|\psi\rangle=\int \limits_{\C}e^{-|\alpha|^2}B_{|\psi
\rangle}(\alpha^*)(B_{|\varphi\rangle}(\alpha^*))^*\, d^2\alpha.
\end{align}
Отображение $|\psi\rangle\mapsto B_{|\psi\rangle}$ задает
взаимно-однозначное соответствие между элементами ${\cal H}$ и аналитическими
функциями, квадратично-интегрируемыми по мере $e^{-|\alpha|^2}\, d^2\alpha$.
\end{Trm}
\begin{proof}
Сначала докажем, что $\int \limits_{\C}\frac{\alpha^{*m}\alpha^n}{\sqrt{m!
n!}}e^{-|\alpha|^2}\, d^2\alpha=\pi \delta_{mn}$. В самом деле,
$$\int \limits_{\C}\frac{\alpha^{*m}\alpha^n}{\sqrt{m!n!}}e^{-|\alpha|^2}
\, d^2\alpha=\int \limits _0^\infty r\, dr \int\limits_0^{2\pi}\frac{r^m
e^{-im\varphi}r^ne^{in\varphi}}{\sqrt{m!n!}}e^{-r^2}\, d\varphi=$$
$$=\pi \delta_{mn}\frac{1}{\sqrt{m!n!}}\int \limits_{0}^\infty
r^{(m+n)/2}e^{-r}\, dr=\pi \delta_{mn}\frac{1}{\sqrt{m!n!}}\Gamma
\left(\frac{m+n}{2}+1\right)=\pi \delta_{mn}.$$
Пусть $|\psi\rangle=\sum\limits_{n=0}^\infty C_n|n\rangle$. Положим
$B_{|\psi\rangle}(\alpha^*)=\sum \limits _{n=0}^{\infty}C_n\frac{\alpha^{*n}}
{\sqrt{\pi n!}}$, $|\psi_N\rangle=\sum \limits_{n=0}^N C_n|n\rangle$. Тогда
для любых $N, \; p\in \N$ выполнено
$$\int \limits_{\C}\left(B_{|\psi_{N+p}-\psi_N\rangle}(\alpha^*)\right)^*
B_{|\psi_{N+p}-\psi_N\rangle}(\alpha^*)e^{-|\alpha|^2}\, d^2\alpha=$$
$$=\langle \psi_{N+p}-\psi_N|\psi_{N+p}-\psi_N\rangle=\sum \limits _{n=N+1}
^{N+p}|C_n|^2\le \sum \limits _{n=N+1}^\infty|C_n|^2,$$ поэтому
последовательность $\{B_{|\psi_n\rangle}\}$ фундаментальна в $L_2(\C, \;
e^{-|\alpha|^2}\, d^2\alpha)$. Значит, она сходится к некоторой
функции $B$ в этой метрике, а так как $B_{|\psi_n\rangle}(\alpha^*)
\rightarrow B_{|\psi\rangle}(\alpha^*)(n\rightarrow \infty)$ для любого
$\alpha$, то $B=B_{|\psi\rangle}$. Следовательно, отображение $|\psi
\rangle\mapsto B_{|\psi\rangle}$ является изометрией, и поэтому
выполнено (\ref{predst_a}). \par
Докажем, что система $\{\alpha^n\}_{n=0}^\infty$ полна в пространстве
целых функций, принадлежащих $L_2(\C, \, e^{-|\alpha|^2}\,
d^2\alpha)$. Пусть $f(\alpha)=\sum \limits_{n=0}^\infty C_n\alpha^n$,
$f\in L_2(\C, \, e^{-|\alpha|^2}\, d^2\alpha)$ и $\langle f, \, \alpha^n
\rangle=0$ для любого $n\in \Z_+$. Так как функция $f$ целая, то ее ряд
Тейлора всюду абсолютно сходится, в частности, $\sum \limits _{n=0}^\infty
|C_n|<\infty$. Поэтому
$$0=\int \limits_{\C}\alpha^{*n}f(\alpha)e^{-|\alpha|^2}\, d^2\alpha =\int
\limits _{r>0}dr\, r^{n+1}e^{-r^2}\int \limits_0^{2\pi}
e^{-in\varphi}\sum \limits_{k=0}^\infty C_ke^{ik\varphi}r^k\, d\varphi=$$
$$=2\pi C_n\int \limits_{r>0}dr\, r^{2n+1}e^{-r^2}$$ (последнее равенство
следует из теоремы Фубини), откуда $C_n=0$.
\end{proof}
Заметим, что $$B_{|\psi\rangle}(\alpha^*)e^{-|\alpha|^2/2}=\sum \limits
_{n=0}^\infty \langle n|\psi\rangle\frac{\alpha^{*n}}{\sqrt{\pi n!}}
e^{-|\alpha|^2/2}=\frac{1}{\sqrt{\pi}}\sum \limits_{n=0}^\infty \langle
\alpha|n\rangle\langle n|\psi\rangle=\frac{1}{\sqrt{\pi}}\langle \alpha|
\psi\rangle,$$ и поэтому $$\langle \varphi|\psi\rangle=\frac{1}{\pi}
\int_{\C}\langle \varphi|\alpha\rangle\langle\alpha|\psi\rangle\, d^2
\alpha.$$
Построенное представление называется {\it представлением Баргмана--Фока}.
Операторы $a^+$ и $a$ в этом представлении действуют
следующим образом: $$a^+|\psi\rangle\mapsto \alpha B_{|\psi\rangle}(\alpha),
\; a|\psi\rangle\mapsto \frac{d}{d\alpha}B_{|\psi\rangle}(\alpha).$$
\begin{Trm}
Последовательность $|\alpha_{lm}\rangle$ с $\alpha_{lm}=\sqrt{\pi}(l+im)$,
$l$, $m\in \Z$, образует полную систему в гильбертовом пространстве.
\end{Trm}
\begin{proof}
Докажем, что $\langle \psi|\beta\rangle=0$ тогда и только тогда, когда
$B_{|\psi\rangle}(\beta^*)=0$. \par
Найдем функцию $B_{|\beta\rangle}$. Имеем $$B_{|\beta\rangle}(\alpha^*)
=\sum\limits_{n=0}^\infty\langle n|\beta\rangle\frac{\alpha^{*n}}{\sqrt
{\pi n!}}=\sum \limits_{n=0}^\infty e^{-|\beta|^2/2}\frac{(\alpha^*
\beta)^n}{\sqrt{\pi}n!}=\frac{1}{\sqrt{\pi}}e^{-|\beta|^2/2+\beta
\alpha^*}.$$ Значит, $$\langle \psi|\beta\rangle=\int \limits_{\C}
B_{|\beta\rangle}(\alpha^*)(B_{|\psi\rangle}(\alpha^*))^*e^{-|\alpha|^2}
\, d^2\alpha=\frac{1}{\sqrt{\pi}}\int \limits _{\C}e^{-|\beta|^2/2+
\beta\alpha^*}(B_{|\psi\rangle}(\alpha^*))^*e^{-|\alpha|^2}\, d^2\alpha=$$
$$=\frac{1}{\sqrt{\pi}}\int \limits _{\C}
e^{-|\beta|^2/2+\beta(\alpha^*+\beta^*)}(B_{|\psi\rangle}(\alpha^*+\beta^*))
^*e^{-|\alpha+\beta|^2}\, d^2\alpha=$$$$=\frac{1}{\sqrt{\pi}}\int
\limits _{\C}
e^{-|\beta|^2/2-\beta^*\alpha}(B_{|\psi\rangle}(\alpha^*+\beta^*))
^*e^{-|\alpha|^2}\, d^2\alpha=\int
\limits_{\C}(f(\alpha^*))^*e^{-|\alpha| ^2}\, d^2\alpha,$$ где $f$
--- аналитическая функция. Кроме того,
$$\int \limits_{\C}|f(\alpha)|^2e^{-|\alpha|^2}\, d^2\alpha=\const\int \limits
_{\C}e^{-|\beta|^2-\alpha\beta^*-\beta\alpha^*}|B_{|\psi\rangle}(\alpha^*
+\beta^*)|^2e^{-|\alpha|^2}\, d^2\alpha=$$ $$=\const \int \limits _{\C}
|B_{|\psi\rangle}(\alpha^*+\beta^*)|^2e^{-|\alpha+\beta|^2}\, d^2\alpha<
\infty.$$ Значит, по предыдущей теореме, существует вектор $|\varphi\rangle
\in {\cal H}$ такой, что $f=B_{|\varphi\rangle}$ и $f(\alpha)=\sum
\limits_{n=0}^\infty \langle n|\varphi\rangle\alpha^n$. Поэтому $\langle
\psi|\beta\rangle=\langle\varphi|0\rangle=0$ тогда и только тогда, когда
$B_{|\varphi\rangle}(0)=0$, то есть $B_{|\psi\rangle}(\beta^*)=0$. \par
Таким образом, если система $|\alpha_{lm}\rangle$ не полна, то существует
такой вектор $|\psi\rangle$, что для любых $l$, $m\in\Z$ выполнено $B_{\psi}
(\sqrt{\pi}(l+im))=0$. Воспользуемся следующим фактом из комплексного
анализа ([10], \S 1.4 и Приложение А): для того, чтобы существовала
аналитическая функция $B(\alpha)\in L_2(\C, \, e^{-|\alpha|^2}\, d^2\alpha)$
такая, что ее нули образуют правильную решетку и их число в круге $\{|z|<r\}$
равно $N(r)$, необходимо и достаточно, чтобы $\lim \limits _{r\rightarrow\infty}
\frac{N(r)}{r^2}<1$. В нашем случае при $r\rightarrow \infty$
$\frac{N(r)}{r^2}\sim\frac{S_{\text{\rm круга}}}{S_{\text{\rm клетки}}r^2}=\frac{
\pi r^2}{\pi r^2}=1$, поэтому $B_{|\psi\rangle}$ не является
квадратично-интегрируемой, что противоречит предыдущей теореме.
\end{proof}
Этой теореме можно придать следующий смысл. Каждая клетка $$\{\alpha=x+iy:
\sqrt{\pi}l\le x\le \sqrt{\pi}(l+1), \; \sqrt{\pi}m\le y\le \sqrt{\pi}(m+1)\}$$
соответствует элементарной планковской ячейке, задаваемой соотношением
неопределенности. Если вместо $\sqrt{\pi}$ поставить меньшее число
(ячейка меньше планковской), то система переполнена. Если ячейка
больше планковской, то система является неполной. \par
Рассмотрим изменение во времени средних значений координаты и импульса,
если в начальный момент $t=0$ система находится в состоянии $|\alpha
\rangle$. Получаем $$\frac{da}{dt}=-i\omega a, \; \frac{da^+}{dt}=i\omega
a^+.$$ Отсюда $a(t)= ae^{-i\omega t}$, $a^+(t)=a^+e^{i\omega t}$.
Значит, $$x(t)=\frac{x_0}{\sqrt{2}}(a(t)+a^+(t))=\frac{x_0}{\sqrt{2}}
(ae^{-i\omega t}+a^+e^{i\omega t}),$$ $$p(t)=\frac{p_0}{\sqrt{2}i}(a(t)-
a^+(t))=\frac{p_0}{\sqrt{2}i}(ae^{-i\omega t}-a^+e^{i\omega t}).$$
При усреднении по когерентным состояниям имеем $\langle \alpha|a|\alpha
\rangle=\alpha$, $\langle \alpha|a^+|\alpha\rangle=\alpha^*$, поэтому, положив
$\alpha=\rho e^{i\varphi}$, получим $$\langle x\rangle_t=\sqrt{2}x_0\rho
\cos (\omega t-\varphi), \; \langle p\rangle_t=-\sqrt{2}p_0\rho
\sin (\omega t-\varphi).$$ Эти величины являются классическими решениями
уравнений для осциллятора, причем $\sqrt{2}x_0\rho$ играет роль
амплитуды, а $\varphi$ --- начальной фазы колебаний. Так как $m\omega x_0=p_0$,
то выполнено уравнение $$\langle p\rangle_t=m\frac{d}{dt}\langle
x\rangle_t.$$ Аналогично можно проверить, что дисперсии координат и
импульсов не зависят от времени и, следовательно, минимизируют соотношения
неопределенности в любой момент времени: $\delta_t x=\frac{x_0}{\sqrt{2}}$,
$\delta_t p=\frac{p_0}{\sqrt{2}}$, $\delta_t x\delta_t p=\frac{x_0p_0}{2}
=\frac{\hbar}{2}$.
\section{Трехмерные задачи}
\label{3dim_probl}
В этой главе исследуется оператор Шредингера в $L_2(\R^3)$. Приводятся
достаточные условия самосопряженности оператора Шредингера, соответствующего
движению в потенциальном и электромагнитном поле. Затем исследуется
оператор Шредингера в центрально-симметричном поле. Если $V\in L_\infty ^{\rm{loc}}
(\R^3)$, то существует инвариантное подпространство ${\cal H}_{lm}$,
на котором задача сводится к исследованию оператора Штурма--Лиувилля на
полупрямой. Если потенциал сингулярный, то оператор Шредингера определяется на
${\cal H}_{lm}$ с помощью некоторого оператора Штурма--Лиувилля на
полупрямой и затем замыкается. Доказывается регулярность обобщенных
собственных функций, в том числе в окрестности нуля. \par
Для кулонова поля приводятся два способа нахождения собственных значений
гамильтониана: сведение к уравнению Куммера и метод факторизации. \par
Далее излагается теория рассеяния. Обобщенные собственные векторы для
абсолютно непрерывного спектра можно построить как решения уравнения
Липпмана--Швингера. Их асимптотика на бесконечности дается формулой
(\ref{lippman_asympt}), а оператор $T(\vec k', \, \vec k)$ связан с
оператором рассеяния формулой (\ref{s_and_t}), откуда получается выражение
(\ref{s_of_k}) для матрицы рассеяния в $L_2(S^2)$. Это же выражение получается
другим способом: в (\ref{8ras_asimpt}) выделяется главная часть при
$r\rightarrow \infty$ и получается (\ref{s_matrix}), где $\hat S$ совпадает
с матрицей рассеяния. Затем находится амплитуда рассеяния для
центрально-симметричного потенциала и ее связь с дифференциальным
сечением рассеяния.
\subsection{Достаточные условия самосопряженности оператора Шредингера}
Оператор Лапласа существенно самосопряжен на $S(\R^n)$
(и, следовательно, на $C_0^\infty(\R^n)$). В самом деле, преобразование
Фурье взаимно-однозначно отображает $S(\R^n)$ на себя и переводит оператор
Лапласа в оператор умножения на $-|p|^2$, а этот оператор существенно
самосопряжен. \par
В [12, гл. III], приведен многомерный аналог теоремы \ref{62trm1}:
\begin{Trm}
Пусть $V\in L_\infty^{{\rm loc}}(\R^n)$, $V(x)\ge -Q(x)$, где $Q$ ---
неубывающая положительная непрерывная функция, $$\int \limits_0^\infty
\frac{dr}{\sqrt{Q(2r)}}=\infty.$$ Тогда $\hat H$ существенно
самосопряжен на $C_0^\infty(\R^n)$.
\end{Trm}
В некоторых случаях существенную самосопряженность можно доказать с
помощью теоремы \ref{vust}. В частности, из нее вытекает следующий
результат [3, т. 2, теорема X.15]:
\begin{Trm}
Пусть $V=V_1+V_2$, где $V_1\in L_2(\R^3)$, $V_2\in L_\infty(\R^3)$. Тогда
оператор $-\Delta+V$ существенно самосопряжен на $C_0^\infty(\R^3)$.
\end{Trm}
Например, если $V(r)=\pm cr^{-\alpha}$, где $\alpha<\frac32$, то оператор
Шредингера будет существенно самосопряжен. Однако из физики известно,
что потенциалы вида $V(r)=\pm cr^{-\alpha}$  при $\alpha<2$ также
задают квантовую динамику. Для доказательства существенной
самосопряженности таких операторов используется теория квадратичных
форм: доказывается, что $V\prec -\Delta$, и затем используется теорема
\ref{klmn}. Следующее утверждение [3, т.2, теорема X.19] дает
достаточное условие для этого.
\begin{Trm}
Если $V=V_1+V_2$, где $V_1\in L_\infty(\R^3)$ и $$\int \limits _{\R^6}
\frac{|V_2(x)|\, |V_2(y)|}{|x-y|^2}\, d^3x\, d^3y<\infty,$$
то $V\prec -\Delta$.
\end{Trm}
Условию последней теоремы удовлетворяет, в частности, $V(r)=r^{-\alpha}$
при $\alpha<2$. \par
В некоторых случаях оператор $\hat H$ может быть существенно самосопряженным
на $C_0^\infty(\R)$ за счет того, что потенциал $V$ неотрицательный.
\begin{Trm} {\rm [3, теорема X.29]}
Пусть $V=V_1+V_2$, где $V_1\in L_2^{{\rm loc}}(\R^n)$, $V_1\ge 0$ поточечно
и существуют такие константы $a<1$ и $b>0$, что $$\|V_2\varphi\|
\le a\|\Delta \varphi\|+b\|\varphi\|$$ для любого $\varphi\in C_0^\infty
(\R^3)$. Тогда оператор $-\Delta+V$ существенно самосопряжен на
$C_0^\infty(\R^n)$.
\end{Trm}
Рассмотрим движение нерелятивистского заряда $e$ в электромагнитном поле
$\vec E$, $\vec B$ со скалярным потенциалом $\varphi$ и векторным
потенциалом $\vec A$ ($\vec B=\rot \vec A$, $\vec E=-\bigtriangledown
\varphi-\frac{1}{c}\frac{\partial \vec A}{\partial t}$). Тогда
функция Гамильтона имеет вид
$$H=\frac{1}{2\mu}\left(\vec p-\frac{e}{c}\vec A\right)^2+e\varphi,$$
где $\vec p$ --- обобщенный импульс, $c$ --- скорость света. Соответствующий
оператор в координатном представлении записывается в виде
\begin{align}
\label{magn_hamilt}
\hat H=\frac{1}{2\mu}\left(-\hbar^2\Delta+\frac{i\hbar e}{c}(2\vec A
\bigtriangledown+\div \vec A)+\frac{e^2}{c^2}A^2\right)+e\varphi.
\end{align}
Следующие теоремы дают достаточное условие существенной самосопряженности
оператора $\hat H$ на $C_0^\infty(\R^3)$. Обозначим через $L_p(\R^3)+
L_q(\R^3)$ множество функций $f_1+f_2$ таких, что $f_1\in L_p(\R^3)$,
$f_2\in L_q(\R^3)$.
\begin{Trm} {\rm [3, теорема X.22]}
Пусть каждая компонента $\vec A$ --- вещественнозначная функция из
$L_4(\R^3)+L_\infty(\R^3)$, $\div \vec A\in L_2(\R^3)+L_\infty(\R^3)$ (в
смысле обобщенных функций) и $\varphi$ --- вещественнозначная функция из
$L_2(\R^3)+L_\infty(\R^3)$. Тогда оператор $\hat H$ вида (\ref{magn_hamilt})
существенно самосопряжен на $C_0^\infty(\R^3)$.
\end{Trm}
\begin{Trm}
\label{magn_samosopr}
{\rm [3, теорема X.34]}
Пусть $A_k\in C^1(\R^3)$, $\varphi=\varphi_1+\varphi_2$, где $\varphi_1
\ge 0$, $\varphi_1\in L_2^{{\rm loc}}(\R^3)$, $\varphi_2\in L_2(\R^3)+L_\infty
(\R^3)$. Тогда оператор (\ref{magn_hamilt}) существенно самосопряжен на
$C_0^\infty(\R^3)$.
\end{Trm}
\subsection{Разделение переменных}
Сначала дадим определение тензорного произведения гильбертовых
пространств ([3] или [12]). Пусть ${\cal H}_1, \, \dots, \,
{\cal H}_n$ --- гильбертовы пространства. Выберем в каждом из них
ортонормированный базис $(e_k^j)_{k=1}^\infty$, $j=1, \, \dots, \, n$.
Рассмотрим множество формальных конечных линейных комбинаций $$\sum \limits
_{j_1, \, \dots, \, j_n}\alpha_{j_1, \, \dots, \, j_n}e^1_{j_1}
\otimes\dots\otimes e^n_{j_n}$$ с естественной структурой линейного
пространства. Зададим в этом пространстве скалярное произведение,
положив $$\langle e_{j_1}^1\otimes \dots\otimes e_{j_n}^n, \,
e_{k_1}^1\otimes \dots\otimes e_{k_n}^n\rangle=\delta_{j_1k_1}\dots
\delta_{j_nk_n}.$$ Пополнив пространство по этому скалярному произведению,
получаем гильбертово пространство, которое обозначим ${\cal H}_1
\otimes \dots\otimes {\cal H}_n$ и назовем тензорным произведением
пространств ${\cal H}_1, \, \dots, \, {\cal H}_n$. \par
Пусть $\psi_j=\sum \limits_{k=1}^\infty \alpha_{jk}e_k^j\in {\cal H}_j$,
$j=1, \, \dots, \, n$. Положим $$\psi_1\otimes \dots\otimes\psi_n
=\sum \limits_{k_1, \, \dots, \, k_n}\alpha_{k_1}\dots\alpha_{k_n}
e_{k_1}^1\otimes \dots\otimes e_{k_n}^n.$$ Утверждается, что
$\psi_1\otimes \dots\otimes\psi_n\in {\cal H}_1\otimes \dots\otimes
{\cal H}_n$ и что $$\langle \varphi_1\otimes \dots\otimes \varphi_n, \,
\psi_1\otimes\dots\otimes\psi_n\rangle =\langle \varphi _1, \, \psi _1\rangle
_{{\cal H}_1}\dots \langle \varphi _n \, \psi_n\rangle_{{\cal H}_n}.$$
Если ${\cal H}_j=L_2(M_j, \, \mu_j)$, то пространство ${\cal H}_1\otimes
\dots\otimes {\cal H}_n$ представляется в виде $L_2(M_1\times\dots\times M_n,
\, \mu_1\otimes \dots\otimes \mu_n)$, при этом
$$\psi_1\otimes\dots\otimes \psi_n(x_1, \, \dots,\, x_n)=\psi_1(x_1)
\dots\psi_n(x_n).$$ \par
Пусть ${\cal H}_k$ --- гильбертовы пространства, $\hat H_k$ ---
самосопряженные операторы на плотных подпространствах $D_k\subset
{\cal H}_k$. Рассмотрим оператор $\hat H=\sum \limits_{k=1}^n\hat H_k$
в ${\cal H}={\cal H}_1\otimes\dots \otimes {\cal H}_n$. В качестве
области определения $D$ оператора $\hat H$ возьмем конечные линейные
комбинации элементов $h_1\otimes \dots \otimes h_n$, где $h_k\in D_k$, и положим
$$\hat H(h_1\otimes \dots \otimes h_n)=(\hat H_1h_1)\otimes\dots\otimes
h_n+\dots+h_1\otimes\dots \otimes(\hat H_nh_n).$$ \par Пусть для
каждого $k$ задано оснащение Гильберта--Шмидта ${\cal H}_+^k
\subset {\cal H}_k\subset {\cal H}_-^k$ и
$\{e_{j,k}(\lambda_k)\}_{j=1} ^{N_k(\lambda)}$ --- полная
ортонормированная система обобщенных собственных векторов $\hat
H_k$. Построим оператор $U$ следующим образом. Если $\varphi_k\in
{\cal H}_+^k$, то $$(U(\varphi_1\otimes\dots
\otimes\varphi_n))(\lambda_1, \dots,\, \lambda_n)=$$ $$=\{\langle
e_{j_1,1} (\lambda_1),\, \varphi_1\rangle\dots\langle
e_{j_n,n}(\lambda_n), \, \varphi_n\rangle\}_{j_k=1, \;
k=1,\dots,n}^{N_k(\lambda_k)}\in l_2^{N_1 (\lambda_1)\dots
N_n(\lambda_n)},$$ затем $U$ продолжаем по линейности. Из того,
что $\{e_{j,k}(\lambda_k)\}_{j=1}^{N_k(\lambda)}$ является полной
ортонормированной системой обобщенных собственных векторов $\hat
H_k$, следует, что $U$ изометрично отображает плотное подмножество
${\cal H}$ на плотное подмножество прямого интеграла гильбертовых
пространств $$\int \hat {\cal H}(\lambda_1,\, \dots, \,
\lambda_n)\, d\mu_1 (\lambda_1)\dots d\mu_n(\lambda_n), \;
\text{где} \; \hat {\cal H} (\lambda_1,\, \dots, \,
\lambda_n)=l_2^{N_1(\lambda_1)\dots N_n (\lambda_n)},$$ и переводит
оператор $\hat H$ в оператор умножения на
$\lambda_1+\dots+\lambda_n$. Продолжаем $U$ на все пространство
${\cal H}$ по непрерывности и получаем, что
$\{e_{j_1,1}(\lambda_1)\otimes\dots \otimes
e_{j_n,n}(\lambda_n)\}$ образует полную ортонормированную систему
обобщенных собственных векторов оператора $\hat H$ (более подробно
это написано в [7]). \par Пусть $\hat H=-\Delta+V_1(x_1,\, \dots,
\, x_k)+V_2(x_{k+1}, \, \dots, \, x_n)$ в $L_2(\R^n)$, где
операторы $\hat H_1=-\Delta+V_1$ и $\hat H_2= -\Delta+V_2$
самосопряжены на $D_1\subset L_2(\R^k)$ и $D_2\subset
L_2(\R^{n-k})$. Тогда $L_2(\R^n)=L_2(\R^k)\otimes L_2(\R^{n-k})$ и
$$(\varphi_1\otimes \varphi_2)(x_1, \, \dots, \,
x_n)=\varphi_1(x_1, \, \dots, \, x_k)\varphi_2(x_{k+1}, \, \dots,
\, x_n).$$ Если обобщенные собственные векторы $e_1(\lambda_1)$ и
$e_2(\lambda_2)$ операторов $\hat H_1$ и $\hat H_2$ являются
регулярными функциями, то обобщенные собственные векторы
$e_1(\lambda_1)\otimes e_2(\lambda_2)$ оператора $\hat H$ также
являются регулярными функциями и имеют вид
$$e_1(\lambda_1) (x_1, \, \dots, \, x_k)e_2(\lambda_2)(x_{k+1}, \, \dots,
\, x_n).$$
Таким образом, исследование спектра $\hat H$ сводится к исследованию
спектров операторов $\hat H_1$ и $\hat H_2$, которые действуют в
пространстве функций меньшего числа переменных. \par {\bf Пример
1.} Пусть $\hat H=-\Delta$. Тогда обобщенные собственные функции
имеют вид $e^{ik_1x_1}\dots e^{ik_nx_n}$ и оператор $U$ --- это
обычное преобразование Фурье. \par {\bf Пример 2.} Рассмотрим
оператор Шредингера для гармонического осциллятора: $\hat
H=-\Delta+x_1^2+\dots+x_n^2$, $\psi_k$ --- $k$-я собственная
функция оператора $-\frac{d^2}{dx^2}+x^2$. Тогда функции
$\psi_{k_1}(x_1)\dots \psi_{k_n}(x_n)$ образуют полную
ортонормированную систему собственных функций оператора $\hat H$ с
собственными значениями $k_1+\dots+k_n$. \par {\bf Замечание.}
Отсюда следует, что $\hat H^{-m}$ для некоторого $m$ является
оператором Гильберта--Шмидта. Значит, если $\hat A$ ---
самосопряженный дифференциальный оператор с гладкими
коэффициентами, то оснащение ${\cal H}_+\subset L_2(\R^n)\subset
{\cal H}_-$ можно построить с помощью оператора $\hat H^m$, и
тогда $S(\R^n)\subset {\cal H}_+$ и ${\cal H}_-\subset S'(\R^n)$.
Последнее дает ограничение на рост обобщенных собственных векторов
оператора $\hat A$.
\subsection{Орбитальный момент количества движения}
В классической механике момент импульса материальной точки определяется
как векторное произведение $\vec{L}=\vec{r}\times\vec{p}$. Соответствующее
произведение операторов координат и импульса дают симметрический
оператор $\vec{\hat{L}}=\frac{\hbar}{i}\vec{\hat{r}}\times \vec{\hat{
\bigtriangledown}}=(\hat L_x, \, \hat L_y, \, \hat L_z)$, где
$\hat L_x=\frac{\hbar}{i}(y\partial_z-z\partial_y)$, $\hat L_y=\frac{\hbar}
{i}(z\partial_x-x\partial_z)$, $\hat L_z=\frac{\hbar}{i}(x\partial_y-y
\partial_x)$. Эти операторы удовлетворяют коммутационным соотношениям
\begin{align}
\label{com_moment}
[\hat L_i, \, \hat L_j]=i\hbar\varepsilon_{ijk}\hat L_k,
\end{align}
где $\varepsilon_{ijk}$ --- символ Леви-Чивита. Оператор $\vec{\hat{L}}$
называется орбитальным моментом количества движения (есть еще
спиновый момент, о котором будет говориться позже). \par
Каждому оператору $U\in SO(3)$ сопоставим оператор $\hat{U}$ в $L_2(\R^3)$
по формуле $\hat Uf(\vec{r})=f(U\vec{r})$. Легко проверить, что это
соответствие является гомоморфизмом и оператор $\hat{U}$ унитарный, то
есть отображение $T:U\mapsto \hat{U}$ является унитарным представлением
группы $SO(3)$ в $L_2(\R^3)$. Пусть $U(t)$ --- однопараметрическая
подгруппа в $SO(3)$ и пусть $\vec{\omega}$ --- угловая скорость вращения
(которая однозначно задает генератор группы $U(t)$).
Найдем генератор соответствующей однапараметрической группы $\hat{U}(t)$,
то есть $\frac{d}{dt}\left|_{t=0}\right.\hat U(t)$. Для любой гладкой
функции $f$ имеем
$$\left.\frac{d}{dt}\right|_{t=0}\hat U(t)f(\vec{r})=\left(\bigtriangledown
f,\, \left.\frac{d}{dt}\right|_{t=0}U(t)\vec{r}\right)=\left(
\bigtriangledown f,\, \vec{\omega}\times\vec{r}\right)=$$
$$=\omega_x(y\partial_z-z\partial_y)f(\vec{r})+\omega_y(z\partial_x-
x\partial_z)f(\vec{r})+\omega_z(x\partial_y-y\partial_x)f(\vec{r})
=\left(\vec{\omega}, \, \frac{i}{\hbar}\vec{\hat{L}}\right)f(\vec{r}).$$
В частности, генераторам вращения вокруг оси $x_j$
соответствует оператор $\frac{i}{\hbar}\hat L_{x_j}$, и в полярных
координатах $\hat L_z=\frac{\hbar}{i}\frac{\partial}{\partial \varphi}$.
Из коммутационных соотношений в алгебре $so(3)$ и из (\ref{com_moment})
следует, что отображение $\vec{\omega}\mapsto -\frac{i}{\hbar}
(\vec{\omega}, \, \vec{\hat{L}})$ задает представление алгебры $so(3)$
в $L_2(\R^3)$.\par
Рассмотрим оператор $\hat{L}^2\stackrel{def}{=}\hat{L}_x^2+\hat{L}_y^2
+\hat{L}_z^2$. Прямым вычислением доказывается, что в полярных координатах
$$\hat{L}^2=-\hbar^2\left(\frac{1}{\sin \theta}\partial_\theta(\sin \theta
\partial _\theta)+\frac{1}{\sin^2\theta}\partial^2_\varphi\right),$$
то есть $-\frac{1}{\hbar^2}\hat{L}^2$ совпадает с оператором Лапласа
на единичной сфере. Кроме того, из коммутационных соотношений
(\ref{com_moment}) следует, что $\hat{L}_z$ и $\hat{L}^2$ коммутируют
на множестве гладких функций. \par
Докажем, что существует полная ортонормированная система бесконечно
гладких функций $g(\theta, \, \varphi)$ на двумерной сфере $S^2$ таких,
что
\begin{align}
\label{l2_equation}
-\hbar^2\left(\frac{1}{\sin \theta}\partial_\theta\sin \theta
\partial _\theta+\frac{1}{\sin^2\theta}\partial^2_\varphi\right)
g(\theta, \, \varphi)=ag(\theta, \, \varphi)
\end{align}
и
\begin{align}
\label{lz_equation}
\frac{\hbar}{i}\partial
_\varphi g(\theta, \, \varphi)=bg(\theta, \, \varphi)
\end{align}
для некоторых $a$, $b\in \R$. В самом деле, рассмотрим оператор
$\hat H=-\Delta+r^2$ на $S(\R^3)$. Как было показано, он имеет полную
ортонормированную систему собственных функций, которые являются бесконечно
гладкими, при этом кратность каждого собственного значения конечна.
Пусть ${\cal H}_E$ --- подпространство ${\cal H}$, состоящее из
собственных функций оператора $\hat H$ с собственным значением $E$.
Тогда ${\cal H}_E$ конечномерно и состоит из гладких функций.
Так как потенциал $r^2$ является центрально-симметричным, то на
бесконечно гладких функциях $\hat H$ и $\hat L_{x_j}$ коммутируют. Отсюда
следует, что если $f\in {\cal H}_E$, то $\hat L^2f\in {\cal H}_E$ и
$\hat L_zf\in {\cal H}_E$. Рассмотрим действие $\hat L^2$ и $\hat L_z$
на ${\cal H}_E$. Так как ${\cal H}_E$ конечномерно и состоит из гладких
функций, то на нем операторы ограничены и коммутируют, а значит,
сильно коммутируют и поэтому имеют полную систему общих собственных
векторов. Если $f\in {\cal H}_E$ --- собственный вектор $\hat L^2$ и
$\hat L_z$ с собственными значениями $a$ и $b$ соответственно, то
в силу гладкости она имеет вид $f(\vec r)=\tilde f(r)g(\theta, \, \varphi)$,
где $g$ удовлетворяет уравнениям (\ref{l2_equation}) и (\ref{lz_equation}).
Докажем, что множество $M$ таких $g$ образует полную
систему в $L_2(S^2)$. Действительно, если существует функция $\tilde g$,
ортогональная всем элементам $M$, то функция ${\chi}_{[0, \, 1]}(r)
\tilde g(\theta, \, \varphi)$ будет ортогональна всем собственным
векторам $\hat H$, что противоречит полноте их системы. \par
{\bf Замечание.} Аналогично доказывается, что оператор Лапласа--Бельтрами
на $n-1$-мерной сфере имеет полную систему бесконечно гладких
собственных функций. Другое доказательство приведено в [9]. \par
Из доказанного следует, что операторы $\hat L^2$ и $\hat L_z$ являются
существенно самосопряженными на $S(\R^3)$, их замыкания сильно
коммутируют, а спектр является точечным. \par
Найдем собственные значения и общие собственные векторы операторов
$\hat{L}^2$ и $\hat L_z$. Положим $\hat{L}_{\pm}=\hat{L}_x\pm i\hat L_y$.
Тогда $\hat{L}^2=\hat{L}_-\hat{L}_++\hat{L}_z^2+\hbar \hat{L}_z$. Пусть
$|\lambda, \, m\rangle\in {\cal H}$ --- такой вектор, что
$$\hat{L}^2|\lambda, \, m\rangle=\hbar^2\lambda|\lambda, \, m\rangle,
\; \; \; \hat{L}_z|\lambda, \, m\rangle=\hbar m|\lambda, \, m\rangle.$$ Так как
$[\hat{L}^2, \, \hat{L}_\pm]=0$ и $[\hat{L}_z, \, \hat{L}_{\pm}]=\pm
\hbar \hat L_\pm$, то состояния $\hat L_\pm|\lambda, \, m\rangle$
также являются собственными векторами $\hat{L}^2$ и $\hat{L}_z$ с собственными
значениями $\hbar^2\lambda$ и $\hbar(m\pm 1)$ соответственно:
$$\hat{L}^2(\hat{L}_\pm|\lambda, \, m\rangle)=\hat{L}_\pm(\hat{L}^2
|\lambda, \, m\rangle)=\hbar^2 \lambda\hat L_\pm|\lambda, \, m\rangle,$$
$$\hat{L}_z(\hat{L}_\pm|\lambda, \, m\rangle)=\hat{L}_\pm(\hat{L}_z
|\lambda, \, m\rangle)+[\hat{L}_z, \, \hat{L}_\pm]|\lambda, \, m\rangle=
\hbar (m\pm 1)\hat L_\pm|\lambda, \, m\rangle.$$ Из равенства
$\hat L^2=\hat L^2_x+\hat L^2_y+\hat{L}_z^2$ следует, что при
фиксированном $\lambda$ спектр $\hat{L}_z$ ограничен сверху и снизу,
поэтому существует максимальное положительное значение $m_{\max}=l$ такое,
что $\hat{L}_+|\lambda, \, l\rangle=0$ (где $|\lambda, \, l\rangle$ получен
из $|\lambda, \, m\rangle$ многократным применением $\hat{L}_+$). Применим
оператор $\hat{L}^2$ к вектору $|\lambda, \, l\rangle$:
$$\hat{L}^2|\lambda, \, l\rangle=\hat{L}_-\hat{L}_+|\lambda, \, l\rangle+
\hat{L}_z^2|\lambda, \, l\rangle+\hbar \hat{L}_z|\lambda, \, l\rangle =
\hbar ^2l(l+1)|\lambda, \, l\rangle.$$ Следовательно, $\lambda=l(l+1)$. \par
Аналогично доказывается, что существует минимальное значение $m_{\min}$,
для которого $L_-|\lambda, \, m_{\min}\rangle=0$. При этом $$\hbar^2l(l+1)
|\lambda, \, m_{\min}\rangle=\hat{L}^2|\lambda, \, m_{\min}\rangle =
\hat{L}_+\hat{L}_-|\lambda, \, l\rangle+\hat{L}_z^2|\lambda, \, m_{\min}
\rangle-\hbar \hat{L}_z|\lambda, \, l\rangle=\hbar ^2 m_{\min}(m_{\min}-1)
|\lambda,\, l\rangle.$$ Отсюда и из условия $m_{\min}\le l$ получаем $m_{\min}=-l$.
Значит, состояние $|\lambda, \, -l\rangle$ получается из состояния
$|\lambda, \, l\rangle$ применением $\hat{L}_-$ $2l+1$ раз. Таким образом,
$2l+1$ целое и поэтому $l$ может быть целым или полуцелым. \par
Далее вектор состояния $|\lambda, \, m\rangle$ будем обозначать $|l, \,
m\rangle$. \par
Обозначим через $Y_{lm}(\theta, \, \varphi)$ элементы $|l, \, m\rangle$
в координатном представлении на сфере в $\R^3$. Они удовлетворяют
следующим дифференциальным уравнениям:
\begin{align}
\label{lz}
\frac{\hbar}{i}\partial_\varphi Y_{lm}(\theta, \, \varphi)=\hbar mY_{lm}
(\theta, \, \varphi),
\end{align}
\begin{align}
\label{l2}
-\hbar^2\left(\frac{1}{\sin \theta}\partial_\theta\sin \theta
\partial _\theta+\frac{1}{\sin^2\theta}\partial^2_\varphi\right)Y_{lm}
(\theta, \, \varphi)=\hbar^2l(l+1)Y_{lm}(\theta, \, \varphi).
\end{align}
Из уравнения (\ref{lz}) получаем, что $Y_{lm}(\theta, \, \varphi)=\Theta
_{lm}(\theta)e^{im\varphi}$, $\varphi\in [0, \, 2\pi]$. Отсюда следует,
что $m$ (и, следовательно, $l$) может быть только целым. \par
Подставив в (\ref{l2}) выражение для $Y_{lm}$, имеем уравнение для
функций $\Theta_{lm}(\theta)$: $$\frac{1}{\sin \theta}\partial_\theta
(\sin \theta \partial _\theta\Theta_{lm})-\frac{m^2}{\sin^2\theta}
\Theta_{lm}+l(l+1)\Theta_{lm}=0.$$ Регулярные на сфере решения этого
уравнения называются {\it присоединенными полиномами Лежандра}
$$\Theta_{lm}(\theta)=N_{lm}P_l^m(\cos \theta),$$ где $N_{lm}$ ---
нормировочный коэффициент,
$$P_l^m(\cos \theta)=\frac{(-1)^l}{2^ll!}(\sin \theta)^{|m|}\left(\frac{d}
{d\cos \theta}\right)^{l+|m|}(\sin \theta)^{2l}.$$
Нормировочный коэффициент выбирается так, чтобы система $\{Y_{lm}(\theta,
\, \varphi)\}$ была ортонормирована на единичной сфере:
$$\int \limits_{S^2}Y^*_{l'm'}(\theta, \, \varphi)Y_{lm}(\theta, \, \varphi)
\sin \theta\, d\theta=\delta_{ll'}\delta_{mm'}.$$ Для этого надо
положить $$Y_{lm}(\theta, \, \varphi)=(-1)^mi^l\sqrt{\frac{2l+1}{4\pi}
\frac{(l-|m|)!}{(l+|m|)!}}P_l^m(\cos \theta)e^{im\varphi}.$$ Функции
$Y_{lm}$ называются {\it шаровыми функциями} или {\it сферическими
гармониками}. \par
Пусть $|l, \, l\rangle$ --- один из собственных векторов оператора
$\hat L^2$ в $L_2(\R^3)$. Рассмотрим подпространство $\tilde {\cal H}
\subset L_2(\R^3)$, натянутое на векторы $\{\hat L_-^m|l, \, l\rangle:
m=0, \, \dots, \, 2l+1\}$. Оно инвариантно относительно действия
операторов $\hat L_z$ и $\hat L_\pm$, а значит, относительно любых
линейных комбинаций $\hat L_x$, $\hat L_y$ и $\hat L_z$. Докажем, что
$\tilde{\cal H}$ инвариантно относительно $T(SO(3))$. В самом деле,
если $U\in SO(3)$, то это поворот вокруг некоторого вектора $\vec l$.
Если $U(t)$ --- однопараметрическая подгруппа вращений вокруг вектора
$\vec l$, то генератором $T_{U(t)}$ будет $\langle \vec l, \, \vec{\hat{L}}
\rangle$, поэтому по теореме Стоуна $T(U)=\exp(it\langle \vec l, \,
\vec{\hat{L}}\rangle)$ для некоторого $t\in \R$. Так как $\tilde{\cal H}$
инвариантно относительно оператора $\langle \vec l, \, \vec{\hat L}\rangle$
(который на этом подпространстве ограничен), то оно инвариантно
относительно $\exp(it\langle \vec l, \, \vec{\hat{L}}\rangle)$. \par
С другой стороны, $\tilde{\cal H}$ не содержит собственных инвариантных
подпространств представления $T(SO(3))$. В самом деле, в обратном
случае $\tilde{\cal H}$ разбивалось бы на два ортогональных инвариантных
подпространства, на каждом из которых $\hat L^2$ совпадает в $\hbar^2
l(l+1)I$. Значит, на $\tilde{\cal H}$ существуют два различных вектора,
которые являются собственными векторами $\hat L^2$ и $\hat L_z$ с
одними и теми же собственными значениями, что противоречит построению
$\tilde{\cal H}$. Таким образом, на $\tilde{\cal H}$ задано неприводимое
представление группы $SO(3)$.
\subsection{Представления $SU(2)$ и спиновый момент}
В предыдущем параграфе были построены неприводимые представления
группы $SO(3)$, накрывающей группой которой является $SU(2)$;
при этом $SO(3)=SU(2)/\{\pm 1\}$ [18, стр. 105]. Классификация неприводимых
представлений группы $SU(2)$ связана с понятием спинового момента. \par
Рассмотрим сильно непрерывное унитарное представление
группы $SU(2)$ в сепарабельном гильбертовом пространстве. Так как группа
$SU(2)$ компактна, то ее представление является прямой суммой
(конечномерных) неприводимых унитарных подпредставлений. \par
Пусть $T$ --- неприводимое унитарное представление группы $SU(2)$
в конечномерном пространстве ${\cal H}$. Так как матрицы из группы
$SU(2)$ диагонализуются, то каждый элемент $g\in SU(2)$ имеет вид
$g=e^{X}$, где $X\in su(2)$. Следовательно, $T(g)=e^{T(X)}$. Поэтому
если подпространство инвариантно относительно $T_{su(2)}$, то оно
инвариантно относительно $T_{SU(2)}$, и если оператор коммутирует со
всеми $T(X)$, $X\in su(2)$, то он коммутирует со всеми $T(g)$,
$g\in SU(2)$. Алгебра Ли $su(2)$ группы $SU(2)$ имеет генераторы
$$X_1=\frac12\begin{pmatrix}0 & i \\ i & 0\end{pmatrix}, \; X_2=
\frac12\begin{pmatrix} 0 & 1 \\ -1 & 0\end{pmatrix}, \; X_3=
\frac12\begin{pmatrix} i & 0 \\ 0 & -i\end{pmatrix},$$ удовлетворяющие
коммутационным соотношениям $[X_j, \, X_k]=-\varepsilon_{jkl}X_l$.
Пусть $U_j(t)$ --- однопараметрическая подгруппа, порожденная $X_j$.
Положим $$\hat S_j=\frac{\hbar}{i}T(X_j)=\frac{\hbar}{i}\lim
\limits_{t\rightarrow 0}\frac{TU_j(t)-I}{t}.$$ Тогда $\hat S_j$ ---
самосопряженные операторы, удовлетворяющие коммутационным соотношениям
\begin{align}
\label{comm_sootn_spin}
[\hat S_j, \, \hat S_k]=i\hbar \varepsilon_{jkl}\hat S_l.
\end{align}
Оператор $\hat{\vec S}=(\hat S_1, \, \hat S_2, \, \hat S_3)$ называется
{\it оператором спина}. \par
Положим $\hat S^2=\hat S_1^2+\hat S_2^2+\hat S_3^2$. Этот оператор
коммутирует со всеми $\hat S_j$. Значит, по лемме Шура, $\hat S=\lambda I$.
Найдем все возможные значения $\lambda$. Положим $\hat S_\pm=\hat S_1
\pm i\hat S_2$. Точно так же, как в предыдущем параграфе, доказывается,
что собственные значения оператора $\hat S_3$ равны $-\hbar s, \,
-\hbar (s-1), \, \dots, \, \hbar(s-1), \, \hbar s$, где $s$ --- целое
или полуцелое число, называемое спином, а собственные векторы переводятся
друг в друга с помощью операторов $\hat S_\pm$. При этом значение $\lambda$
равно $\hbar^2 s(s+1)$. Все собственные значения $\hat S$ однократны
(это доказывается так же, как в предыдущем параграфе для группы
$SO(3)$). \par
Обратно, пусть $s$ --- произвольное целое или полуцелое число.
Докажем, что тогда реализуется неприводимое представление группы
$SU(2)$ такое, что $\hat S^2=\hbar^2 s(s+1)I$. В самом деле, рассмотрим
пространство $\C^{2s+1}$ и выберем в нем ортонормированную систему
векторов $\{e_j\}_{j=-s}^{s}$. Положим $\hat S_3e_j=\hbar je_j$,
$$\hat S_+e_j=\left\{ \begin{array}{l} \hbar \sqrt{(s+j+1)(s-j)}e_{j+1},
\; \text{если} \; j<s,\\ 0, \; \text{если}\;
j=s,\end{array}\right. $$$$ \hat S_-e_j=\left\{ \begin{array}{l}
\hbar \sqrt{(s+j)(s-j+1)} e_{j-1}, \; \text{если} \; j>-s,\\ 0, \;
\text{если}\; j=-s,\end{array}\right.$$ $\hat S_1=\frac12 (\hat
S_++\hat S_-)$, $\hat S_2=\frac{1}{2i}(\hat S_+ -\hat S_-)$. Тогда
можно показать, что выполнены коммутационные соотношения
(\ref{comm_sootn_spin}). Значит, построено неприводимое
конечномерное представление алгебры $su(2)$. Это представление
интегрируется до неприводимого представления односвязной группы,
алгеброй Ли которой является $su(2)$, то есть группы $SU(2)$. \par
{\bf Замечание 1.} Представление группы $SU(2)$ связывается со
спином в нерелятивистской теории. В релятивистской теории
рассматривается классификация неприводимых представлений группы
Пуанкаре [4, т. 2, гл. 17], которые характеризуются
значениями массы и спина. \par {\bf Замечание 2.} Представления
группы $SU(2)$ с полуцелыми значениями $s$ иногда рассматриваются
как двузначные представления группы $SO(3)$.
\subsection{Вспомогательные утверждения об операторе Штурма--Лиувилля
в $L_2(\R_+)$}
Сначала докажем неравенство Гильберта [22, стр. 212].
\begin{Sta}
\label{hilbert_ineq}
Пусть $\varphi \in W^1_2(\R_+)$ имеет компактный носитель в $(0, \, +\infty)$.
Тогда $$\int \limits_0^\infty \left(|\varphi'(x)|^2-\frac{1}{4x^2} |\varphi(x)|^2\right)
\, dx\ge 0.$$ Если $c>\frac14$, то для любого $x_0>0$ квадратичная форма $$q(\varphi)=
\int \limits_0^{x_0}\left(|\varphi'(x)|^2-\frac{c}{x^2}|\varphi(x)|^2\right)\, dx$$
не является ограниченной снизу на множестве функций из $W^1_2(0, \, x_0)$ с компактным
носителем.
\end{Sta}
\begin{proof}
Первое утверждение следует из того, что $$\int \limits_0^\infty \left(|\varphi'(x)|^2-
\frac{1}{4x^2}|\varphi(x)|^2\right)\, dx=\int \limits_0^\infty\left|\varphi'-
\frac{\varphi}{2x}\right|^2\, dx\ge 0.$$ Для доказательства второго утверждения
рассмотрим последовательность функций $\varphi_\varepsilon(x)=\eta _\varepsilon(x)\sqrt{x}$,
где $$\eta_\varepsilon(x)=\left\{ \begin{array}{l} 0, \; x\in [0, \, \varepsilon], \\
\frac{x-\varepsilon}{\varepsilon}, \; x\in [\varepsilon, \, 2\varepsilon], \\
1, \; x\in [2\varepsilon, \, x_0/2], \\ 2-\frac{2x}{x_0}, \; x\in [x_0/2, \, x_0].
\end{array}\right.$$ Тогда для любого $M>0$ выполнено $q(\varphi_\varepsilon)+M\langle
\varphi _\varepsilon , \, \varphi _\varepsilon\rangle\rightarrow -\infty$ при
$\varepsilon \rightarrow 0$.
\end{proof}
Рассмотрим в $L_2(\R_+)$ оператор $\hat
H=-\frac{d^2}{dx^2}+V(x)$, где $V=V_1+V_2$, а $V_j$ имеют
следующие свойства:
\begin{enumerate}
\item $V_1\in L_1^{{\rm loc}}(\R_+)$ и существуют такие константы
$c_1\in (0, \, 1/4)$ и $c_2>0$, что в окрестности нуля
$V_1(r)\ge -c_1r^{-2}$, а в окрестности бесконечности $V_2(r)\ge
-c_2r^2$;
\item $V_2=W'$, где $W\in L_2(\R_+)$ и ${\rm supp}\, W\subset [a, \, b]$
($0<a<b<+\infty$).
\end{enumerate}
Положим $f^{[1]}=f'-Wf$. Определим оператор $\hat H$ на множестве
$D_0^{\hat H}$ функций $f$, таких что
\begin{enumerate}
\item $f$ и $f^{[1]}$ абсолютно непрерывны на каждом отрезке в $(0, \,
+\infty)$;
\item $\hat Hf\in L_2(\R_+)$;
\item $f$ имеет компактный носитель в $(0, \, +\infty)$.
\end{enumerate}
На этом множестве $\hat H$ является симметрическим. В силу условия
на $V_1$, выполнено условие предельной точки на бесконечности. Если
выполнено условие предельной точки в нуле, то оператор $\hat H$
существенно самосопряжен. Это доказывается методом расщеплений
[15, теорема 5]. Если в нуле выполнено условие предельной окружности,
то индексы дефекта $\hat H$ равны $(1, \, 1)$ и самосопряженное расширение
строится следующим образом. Пусть при $r<r_0$ выполнено $V(r)\ge
-c_1r^{-2}$. Рассмотрим оператор $\hat H_1=-\frac{d^2}{dr^2}+V(r)$ на
множестве функций из $D_0^{\hat H}$ с носителем в $(0, \, r_0)$ (на
ортогональное дополнение к $D_0^{\hat H}$ его продолжаем нулем). В
силу утверждения \ref{hilbert_ineq}, оператор $\hat H_1$ ограничен снизу в
смысле квадратичных форм. Значит, можно построить его расширение по Фридрихсу
с областью определения $D_1^{\hat H}$. Пусть $$D_2^{\hat H}=\{f\in D_1^{\hat H}:
\supp f\subset [0, \, r_0], f(r_0)=0, \; f^{[1]}(r_0)=0\},$$
$D(\hat H)=D_0^{\hat H}+D_2^{\hat H}$. Оператор $\hat H$ на $D(\hat H)$
является симметрическим. В самом деле, пусть $\varphi_0$, $\psi_0\in
D_0^{\hat H}$, $\varphi_2$, $\psi_2\in D_2^{\hat H}$. Ясно, что
$\langle \psi_j, \, \hat H\varphi_j\rangle=\langle \hat H\psi_j, \,
\varphi_j\rangle$, $j=0, \, 2$. Равенства $\langle \psi_0, \, \hat H
\varphi_2\rangle=\langle \hat H\psi_0, \, \varphi_2\rangle$ и
$\langle \psi_2, \, \hat H\varphi_0\rangle=\langle \hat H\psi_2, \,
\varphi_0\rangle$ доказываются интегрированием по частям. Значит,
$$\langle \psi_0+\psi_2, \, \hat H(\varphi_0+\varphi_2)\rangle=
\langle \hat H(\psi_0+\psi_2), \, \varphi_0+\varphi_2\rangle.$$
Самосопряженность $\hat H$ доказывается методом расщеплений. \par
Заметим, что $D_1^{\hat H}\subset \mathaccent'27 W^1_2[0, \, r_0]$, так как
\begin{align}
\label{q_phi_w12}
q(\varphi)\ge(1-\varepsilon)\langle \varphi', \, \varphi'\rangle
-c_1\langle r^{-1}\varphi, \, \varphi\rangle+\varepsilon\langle
\varphi', \, \varphi'\rangle\ge \varepsilon\|\varphi'\|^2,
\end{align}
если $\frac{c_1}{1-\varepsilon}<\frac14$. \par
Найдем обобщенные собственные векторы оператора $\hat H$. Пусть
$K=-\frac{d^2}{dr^2}+\tilde V(r)$, где $$\tilde V(r)=\left\{ \begin{array}{l}
V(r)+c, \; r\le r_0, \\ V(r)-U(r)+c+c(r-r_0)^2, \; r>r_0, \end{array}\right.$$
где $U(r)\ge 0$ --- гладкая функция, такая что $V(r)-U(r)\ge -c_3$ и
$\int |V-U|\, dx\le c_4$ для некоторых $c_3$, $c_4\ge 0$.
Сначала $K$ определяется на функциях с компактным носителем, а затем
продолжается с помощью расширения по Фридрихсу. С помощью принципа
минимакса (см. лемму \ref{hilb_schmidt}) доказывается, что $K^{-1}$ является
положительно определенным оператором Гильберта--Шмидта при достаточно
больших $c>0$. Оснащение $L_2(\R_+)$ строится с помощью $K$. \par
Так же, как и для оператора Штурма--Лиувилля в $L_2(\R)$, доказывается,
что если $F$ --- обобщенный собственный вектор, соответствующий точке
спектра $E$, и $\varphi\in {\cal H}_+$ имеет компактный носитель в
$(0, \, +\infty)$, то
$$\langle F, \, \varphi\rangle=\int \limits_0^\infty g\varphi\, dx,$$
где $g$ --- регулярное решение уравнения $\hat Hg=Eg$, и так же оценивается
рост функции $g$ на бесконечности. Исследуем поведение функции $g$
в окрестности нуля.
\begin{Lem}
\label{existence_lemma}
Для любого $E\in \R$ уравнение $\hat HfP$ имеет решение $f_E$ такое, что
если $\eta\in C^\infty(\R_+)$, $\supp \eta \subset [0, \, r_0)$,
$\eta(r)=1$ в окрестности 0, то $\eta f_E\in D(\hat H)$.
\end{Lem}
\begin{proof}
Достаточно показать, что для любого $E$ найдется потенциал $V_E$,
совпадающий с $V$ на $[0, \, r_0]$, такой что $E$ является
собственным значением оператора $A=-\frac{d^2}{dr^2}+V_E(r)$ (его
область определения строится так же, как область определения $\hat
H$, при этом множества $D_2^A$ и $D_2^{\hat H}$ совпадают). Тогда
соответствующая собственная функция $f$ удовлетворяет уравнению
$-f''+Vf=Ef$ на $[0, \, r_0]$, а $\eta f\in D(\hat H)$. В самом
деле, существует последовательность функций $f_n=\varphi_n+g_n$,
таких что $\varphi_n$ имеет компактный носитель в $(0, \,
+\infty)$, $g_n\in D_2^A$, $f_n\rightarrow f$, $Af_n\rightarrow
Af$ в $L_2$. Тогда $\eta f_n\rightarrow \eta f$ и $A(\eta
f_n)\rightarrow A(\eta f)$ в $L_2$. Действительно, $$A(\eta f_n-
\eta f)=\eta (Af_n-Af)+\eta''(f_n-f)-2(\eta'(f_n-f))'.$$ Ясно,
что первые два слагаемые стремятся к 0 в $L_2$. Так как $\psi_n:=
\eta'(f_n-f)\in W^1_2[0, \, r_0]$ и имеет компактный носитель в
$(0, \, r_0)$, то в силу (\ref{q_phi_w12}) достаточно доказать, что
$$\int \limits_0^{r_0}\psi_n(x)(A\psi_n)(x)\, dx\underset{n\rightarrow
\infty}{\rightarrow} 0$$ (можно считать, что функции вещественнозначные,
иначе рассматриваем их вещественную и мнимую части). Это следует из
того, что $$A\psi_n=\eta' A(f_n-f)-\eta'''(f_n-f)-2\eta''(f_n'-f'),$$
при этом первые два слагаемые сходятся к 0 в $L_2$ и $$\int \limits
_0^{r_0}2\eta'\eta''(f_n'-f')(f_n-f)\, dx=-\int \limits_0^{r_0}
(\eta'\eta'')'(f_n-f)^2\, dx\underset{n\rightarrow \infty}{\rightarrow} 0.$$
\par
С другой стороны, $\eta f_n=\eta g_n+\eta\varphi_n$, $\eta
g_n=g_n- (1-\eta)g_n\in D_2^A$, $\eta \varphi_n\in D_2^A$, поэтому
$\eta f_n \in D_2^A$. Значит, $\eta f\in D_1^A$, а так как $\eta
f=0$ в окрестности $r_0$, то $\eta f\in D_2^A=D_2^{\hat H}$. \par
Пусть $U_0$ совпадает с $V$ на $[0, \, r_0]$ и $U_0\rightarrow
+\infty$ при $r\rightarrow +\infty$. Тогда, применяя принцип
минимакса, можно показать, что оператор $-\frac{d^2}{dr^2}+U_0$
имеет чисто дискретный спектр $\{\mu_n\}_{n=1}^\infty$,
$\mu_n<\mu_{n+1}$ и $\mu_n\rightarrow +\infty$ при $n\rightarrow
\infty$. Пусть $r_0<a<b<\infty$. Положим $U_M(r)=U_0(r)-M\chi_{[a,
\, b]}(r)$, где $\chi_{[a, \, b]}$ --- характеристическая функция
отрезка $[a, \, b]$. Пусть $\mu_N^M$ --- $N$-е собственное
значение оператора $A_M=-\frac{d^2}{dr^2}+U_M(r)$. Докажем, что
$\mu_N^M\rightarrow -\infty$ при $M\rightarrow +\infty$. Для этого
достаточно доказать, что $\mu_N^M+M$ ограничены сверху. Рассмотрим
оператор $B=-\frac{d^2}{dr^2}+U_0$ в $L_2[a, \, b]$ с граничными
условиями Дирихле. Его спектр является чисто дискретным (см.,
напр., [15]). Пусть $\nu_N$ --- $N$-е собственное значение $B$.
Докажем, что $\mu_N+M\le \nu_N$. В самом деле, согласно принципу
минимакса, $$\mu_N^M+M=\sup_{\varphi_1, \, \dots, \,
\varphi_{N-1}} \inf \{\langle \psi, \, (A_M+M)\psi\rangle:\psi\in
Q(A_M)\cap S_1(0), \; \psi\in [\varphi_1, \, \dots, \,
\varphi_{N-1}]^{\bot}\} \le$$ $$\le \sup_{\varphi_1, \, \dots, \,
\varphi_{N-1}} \inf \{\langle \psi, \, (A_M+M)\psi\rangle:\psi\in
Q(A_M)\cap S_1(0), \; \psi\in [\varphi_1, \, \dots, \,
\varphi_{N-1}]^{\bot}, $$$$ \supp \psi\subset (a, \, b)\}=
\sup_{\begin{array}{l} ^{\varphi_1, \, \dots, \, \varphi_{N-1}:}\\
^{\supp \varphi_j\subset [a, \, b]}\end{array}}\inf \{\langle
\psi, \, (A_M+M)\psi\rangle:\psi\in Q(A_M)\cap S_1(0), $$$$
\psi\in [\varphi_1, \, \dots, \, \varphi_{N-1}]^{\bot}, \; \supp
\psi\subset (a, \, b)\}=$$
$$=\sup_{\varphi_1, \, \dots, \, \varphi_{N-1}\in L_2[a, \, b]}
\inf \{\langle \psi, \, B\psi\rangle:\psi\in Q(B)\cap S_1(0), \;
\psi\in [\varphi_1, \, \dots, \, \varphi_{N-1}]^{\bot}\}=\nu_N$$
($S_1(0)$ обозначает множество единичных векторов).
Предпоследнее равенство следует из того, что если $\supp \psi\subset
(a, \, b)$ и $\psi\bot \varphi_j$, то $\psi\bot (\chi_{[a, \, b]}
\varphi_j)$; последнее равенство следует из того, что оператор
$B$ получается с помощью замыкания формы $\langle \psi, \, B\psi
\rangle$, определенной на функциях с компактным носителем в $(a, \,
b)$. \par
Применяя принцип минимакса, можно доказать, что $\mu_N^M$ непрерывно
зависит от $M$, так что для любого $E\in [\mu_N^M, \, \mu_N]$ найдется $M'$
такое, что $E$ является собственным значением $A_{M'}$. Так как
$\mu_N^M\rightarrow -\infty$ при $M\rightarrow +\infty$ и $\mu_N\rightarrow
+\infty$ при $N\rightarrow \infty$, то в качестве $E$ можно брать
любое число.
\end{proof}
\begin{Lem}
\label{uniqueness_lemma}
Пусть $V(r)\ge -\frac{c}{r^2}$ при $r\in (0, \, r_0]$, где $c<\frac14$.
Тогда уравнение $-\psi''+V\psi=E\psi$ имеет решение, не принадлежащее
множеству $\{f:f|_{[0, \, r_0]}\in W^1_2, \; f(0)=0\}$.
\end{Lem}
\begin{proof}
Пусть $c<\tilde c<\frac14$. Тогда для любых $\varphi\in
C_0^\infty(0, \, r_0)$ выполнено $$\int \limits_0^{r_0}\left(
|\varphi'|^2-\frac{\tilde c}{r^2}|\varphi|^2\right)\, dr\ge 0.$$
Выберем $b\in (0, \, r_0)$ таким, чтобы выполнялось $-\frac{\tilde c}{r^2}+
E-V(r)<0$ при $r\in (0, \, b)$. Пусть $\psi_0(r)=r^{\alpha}$, где
$\alpha=\frac{1-\sqrt{1-4\tilde c}}{2}$. Тогда $\psi_0$ является решением
уравнения $-\psi_0''-\frac{\tilde c}{r^2}\psi_0=0$. Пусть $\psi$ ---
решение уравнения $-\psi''+V\psi=E\psi$ с начальными условиями
$\psi(b)=\psi_0(b)$, $\psi'(b)<\psi_0'(b)$. Тогда в некоторой левой
полуокрестности точки $b$ выполнено $\psi(r)>\psi_0(r)$. Пусть
существует точка $a\in (0, \, b)$ такая, что $\psi(a)=\psi_0(a)$ и
$\psi(r)>\psi_0(r)$ при $a<r<b$. Положим $\varphi(r)=\psi(r)-\psi_0(r)$.
Тогда $\varphi\in \mathaccent'27 W^1_2[a, \, b]$ и для любого $r\in
(a, \, b)$ выполнено $$-\varphi''-\frac{\tilde c}{r^2}\varphi
=-\psi''+(V(r)-E)\psi+\left(-\frac{\tilde c}{r^2}+E-V(r)\right)\psi
+\psi_0''+\frac{\tilde c}{r^2}\psi_0=$$ $$=\left(-\frac{\tilde c}{r^2}
+E-V(r)\right)\psi(r)<0.$$ Значит, $$\int \limits_a^b \left(-\varphi''
-\frac{\tilde c}{r^2}\varphi\right)\varphi\, dr<0.$$ С другой
стороны, левая часть равна $$q(\varphi)=\int \limits_a^b\left(\varphi'^2
-\frac{\tilde c}{r^2}\varphi^2\right)\, dr\ge 0,$$ так как
$\varphi$ приближается в метрике $W^1_2[a, \, b]$ функциями из $C_0^\infty
(a, \, b)$, а $q$ непрерывна в этой метрике. Полученное противоречие
доказывает, что $\psi(r)>\psi_0(r)$ при $0<r<b$. \par
Если $\psi|_{[0, \, r_0]}\in W^1_2$ и $\psi(0)=0$, то $\psi(r)=
O(\sqrt{r})$ в окрестности нуля, что противоречит тому, что $\psi(r)
\ge \psi_0(r)=r^\alpha$, а $\alpha<\frac12$.
\end{proof}
Теперь докажем, что обобщенный собственный вектор оператора $\hat H$
совпадает с решением уравнения $\hat Hg=Eg$ таким, что $\eta g\in
D(\hat H)$, где $\eta$ --- срезающая функция из леммы \ref{existence_lemma}.
Из леммы \ref{uniqueness_lemma} следует, что это решение единственно
с точностью до пропорциональности, так что спектр является однократным.
Пусть $\varphi$ имеет компактный носитель в $(0, \, r_0)$, $\varphi$
и $\varphi'$ абсолютно непрерывны, $K\varphi\in L_2$, $f$ --- решение уравнения
$(\hat H-E)f=\varphi$ с начальными условиями $f(r_0)=0$, $f'(r_0)=0$.
Тогда в окрестности нуля $\hat Hf=Ef$. Имеем
$$\int \limits_0^{r_0}\varphi(r)g(r)\, dr=\int \limits_\varepsilon^{r_0}
\varphi(r)g(r)\, dr=\int \limits_\varepsilon^{r_0}(\hat H-E)fg\, dr=$$
$$=\int \limits_\varepsilon^{r_0}f(\hat H-E)g\, dr-f'(\varepsilon)
g(\varepsilon)+f(\varepsilon)g'(\varepsilon)=-f'(\varepsilon)g(\varepsilon)+
f(\varepsilon)g'(\varepsilon).$$ Таким образом, $\int \limits_0^{r_0}
\varphi(r)g(r)\, dr=0$ тогда и только тогда, когда $W(f, \, g)=0$,
то есть $f=cg$ в окрестности нуля. \par
Функция $\varphi$ принадлежит ${\cal H}_+$. Если $\int \limits_0^{r_0}
g\varphi\, dr=0$, то $f=cg$ в окрестности нуля, и $\hat HfP+\varphi\in
L_2$, так что $f\in D(\hat H)$. Аналогично $f\in D(K)$, то есть
$f\in {\cal H}_+$. Наконец, $\hat HfP+\varphi\in {\cal H}_+$, поэтому
$\langle F, \, \varphi\rangle=\langle F, \, (\hat H-E)f\rangle=0$. \par
Пусть $\varphi_1$ имеет компактный носитель в $(0, \, r_0)$,
$\varphi_1$ и $\varphi'_1$ абсолютно непрерывны, $K\varphi_1\in L_2$ и $\int
\limits_0^{r_0}g\varphi_1\, dr=1$. Тогда $\varphi=\varphi_1\int
\limits_0^{r_0}g\varphi\, dr+\tilde \varphi$, где $\int \limits_0^{r_0}
g\tilde \varphi\, dr=0$. Значит, $\langle F, \, \varphi\rangle=\langle F, \,
\varphi_1\rangle\int \limits_0^{r_0}g\varphi\, dr$, то есть $F=c_0g$
в окрестности нуля, где $c_0$ --- некоторая константа.
\subsection{Движение в центральном поле}
Пусть $\hat H=-\Delta+V$, где $V$ зависит только от радиальной
координаты $r$. Предположим сначала, что $V\in L_\infty^{\rm{loc}}
(\R^3)$ и существуют такие константы $c_1>0$, $c_2>0$, что
$V(r)\ge -c_1-c_2r^2$. Тогда оператор $\hat H$ существенно самосопряжен
на $C_0^\infty(\R^3)$. Рассмотрим подпространства ${\cal H}_{lm}
=\{f(r)Y_{lm}(\theta, \, \varphi):f\in L_2(\R_+, \, r^2\, dr)\}$.
При $l>0$ положим $$D_{lm}=\{f(r)Y_{lm}(\theta, \, \varphi):f\in
C_0^\infty(0, \, +\infty)\}, \; \; \; D_{00}=\{f\in C^\infty(\R_+):\supp f
\; \text{ограничен}\}.$$ Тогда $D_{lm}\subset D(\hat H)$ и $\hat H|_{D_{lm}}$
отображает $D_{lm}$ в ${\cal H}_{lm}$, а $$\hat H|_{D_{lm}}=-\frac{\partial^2}
{\partial r^2}-\frac{2}{r}\frac{\partial}{\partial r}
+\frac{l(l+1)}{r^2}+V(r).$$ Докажем, что $\hat H|_{D_{lm}}$
существенно самосопряжен как оператор в ${\cal H}_{lm}$. Для этого
построим изоморфизм пространств ${\cal H}_{lm}$ и $L_2(\R_+)$ по
формуле $f(r)Y_{lm}(\theta, \, \varphi)\mapsto rf(r)$. При этом
$D_{lm}$ переходит в $C_0^\infty(0, \, +\infty)$ при $l\ne 0$, а
$D_{00}$ переходит в множество бесконечно гладких функций с ограниченным
носителем и обращающихся в 0 при $r=0$. Оператор $\hat H|_{D_{lm}}$
переходит в $\hat H_{lm}=-\frac{d^2}{dr^2}+\frac{l(l+1)}{r^2}+V(r)$.
При $l>0$ оператор $\hat H_{lm}$ существенно самосопряжен на
$C_0^\infty(0, \, +\infty)$ в силу критерия Вейля. При $l=0$ оператор
$\hat H_{00}$ удовлетворяет условию предельной точки на бесконечности,
а в 0 определяется граничным условием Дирихле. Отсюда следует, что
$\hat H_{00}$ существенно самосопряжен (это доказывается методом
расщеплений). Заметим, что граничное условие Дирихле в 0 соответствует
продолжению по Фридрихсу с $C_0^\infty(0, \, +\infty)$. \par
Теперь рассмотрим случай центрально-симметричных потенциалов более
общего вида. Пусть $V=V_1+V_2$, где функция $V_1\in L_1^{{\rm loc}}
(\R_+)$, в окрестности нуля $V_1\ge -cr^{-2}$, где $c<\frac14$, в окрестности
бесконечности $V_1(r)\ge -c_1-c_2r^2$, а $V_2$ имеет компактный
носитель в $(0, \, +\infty)$ и имеет вид $V_2=W'$, где $W\in L_2$.
Определим оператор $\hat H=-\Delta+V(r)$ на каждом из подпространств
${\cal H}_{lm}$. Для этого построим оператор $\hat H_{lm}=-\frac{d^2}
{dr^2}+\frac{l(l+1)}{r^2}+V(r)$ в $L_2(\R_+)$ так, как это было
описано в предыдущем параграфе. Пусть изоморфизм $U:{\cal H}_{lm}
\rightarrow L_2(\R_+, \, dr)$ определен по формуле $\psi(r)Y_{lm}(\theta,
\, \varphi)\mapsto r\psi(r)$. Тогда полагаем $\hat H|_{{\cal H}_{lm}}=U^{-1}
\hat H_{lm}U$. Он задается дифференциальным выражением $l(f)=-\frac{\partial^2f}
{\partial r^2}-\frac{2}{r}\frac{\partial f}{\partial
r}+\frac{l(l+1)}{r^2}f+Vf$. \par
Итак, пусть задан оператор Шредингера $\hat H$, соответствующий гамильтониану
$H(p, \, \vec{r})=\frac{p^2}{2\mu}+V(r)$, где $V$ удовлетворяет указанным
выше условиям, а $\mu$ обозначает массу. Тогда спектр и обобщенные
собственные векторы оператора $\hat H$ находятся следующим образом. \par
{\it Шаг 1.} Рассматриваем ограничение $\hat H$ на ${\cal H}_{lm}$
и исследуем спектр на этом подпространстве. \par
{\it Шаг 2.} Переходим от ${\cal H}_{lm}$ к $L_2(\R_+, \, dr)$
с помощью изоморфизма $U$. При этом $\hat H|_{{\cal H}_{lm}}$ переходит
в оператор $\hat H_{lm}=-\frac{\hbar^2}{2\mu}\frac{d^2}{dr^2}+
V_{\text{\rm эфф}}$, где $V_{\text{\rm эфф}}=\frac{\hbar^2 l(l+1)}{2\mu r^2}+V$. \par
{\it Шаг 3.} Ищем обобщенные собственные векторы $f_E$ оператора $\hat H_{lm}$.
Для этого строим оснащение с помощью оператора $K$, который был определен
в предыдущем параграфе. Как было доказано, $f_E$ --- абсолютно непрерывные
вместе с квазипроизводными функции, удовлетворяющие уравнению
$-\frac{\hbar^2}{2\mu}f''+V_{\text{\rm эфф}}fP$, при этом $f_E\eta\in
D(\hat H_{lm})$, где $\eta$ --- гладкая функция с ограниченным носителем,
равная 1 в окрестности нуля. Такое решение уравнения единственно с
точностью до коэффициента пропорциональности. Кроме того, $f_E=Kg_E$,
где $g_E\in L_2(\R_+)$, что дает ограничение на рост функции $f$ на
бесконечности. \par
{\it Шаг 4.} В пространстве ${\cal H}_{lm}$ соответствующий обобщенный
собственный вектор имеет вид $\frac{f_E(r)}{r}Y_{lm}(\theta, \,
\varphi)$. \par
{\bf Замечание 1.} Азимутальное квантовое
число $m$ не входит в радиальное уравнение, поэтому собственные значения
энергии одинаковы для всех $m$, то есть имеет место вырождение энергии по
$m$. Это можно связать с существованием некоммутирующих между собой
интегралов движения. Пусть $[\hat H, \, \hat A]=0$, $[\hat H, \, \hat B]
=0$ (то есть $A$, $B$ --- интегралы движения) и $[\hat A, \, \hat B]\ne
0$. Тогда существуют общие (обобщенные) собственные векторы пары $\hat H$,
$\hat A$: $\hat H\varphi_{a, \, E}=E\varphi_{a, \, E}$, $\hat A\varphi_{a,
\, E}=a\varphi_{a, \, E}$. Так как $[\hat B, \, \hat H]=0$, то $\hat H
(\hat B\varphi_{a, \, E})=\hat B\hat H\varphi_{a, \, E}=E\hat B\varphi
_{a, \, E}$, то есть $\hat B\varphi_{a, \, E}$ --- собственная функция
$\hat H$ с собственным значением $E$. Так как $\hat A$ и $\hat B$ не
коммутируют, то существует вектор $\varphi_{a, \, E}$ такой, что
$\hat B\varphi_{a, \, E}\ne c\varphi_{a, \, E}$, то есть $\varphi_{a, \, E}$
и $\hat B\varphi_{a, \, E}$ --- различные собственные функции оператора
$\hat H$, соответствующие одному и тому же собственному значению $E$. В
нашем случае в качестве $\hat A$ и $\hat B$ можно взять $\hat L_z$ и
$\hat L_+$ или $\hat L_z$ и $\hat L_-$. \par
{\bf Замечание 2.} Если $V(r)=o(r^{-2})$ при $r\rightarrow 0$, то при $l>0$
в уравнении для $f_E$ есть точка поворота, препятствующая ``классическому''
проникновению частицы в точку $r=0$. Часто $V_{\text{\rm эфф}}$ при $E<0$
имеет две точки поворота, а при $E>0$ --- одну. Соответственно при $E<0$
спектр дискретен и ограничен снизу (он может быть пустым), а при $E>0$
спектр непрерывен. \par
{\bf Замечание 3.} Если $\frac{2\mu}{\hbar^2}V(r)<-\frac{c}{r^2}$ в
окрестности нуля, где $c>\frac14$, то оператор $-\frac{\hbar^2}{2\mu}
\Delta+V(r)$ не является ограниченным снизу в смысле форм на
$C_0^\infty(\R^3\backslash \{0\})$, и стандартной процедуры построения
самосопряженного расширения нет. Если $r^2V(r)\underset{r\rightarrow 0}
{\rightarrow} -\infty$, то $V_{\text{\rm эфф}}\underset{r\rightarrow 0}
{\rightarrow} -\infty$, что соответствует падению на центр. \par
Рассмотрим случай свободного движения, то есть $V=0$. Так как при
преобразовании Фурье оператор $-\Delta$ переходит в положительно
определенный оператор $|p|^2$ с непрерывным спектром, то спектр
оператора Шредингера тоже непрерывный и совпадает с $\R_+$. Уравнение для
радиальной функции имеет вид
\begin{align}
\label{svob_dim3}
R''_{kl}+\frac{2}{r}R'_{kl}+\left(k^2-\frac{l(l+1)}{r^2}\right)R_{kl}=0.
\end{align}
При $l=0$ это уравнение можно написать, как
$$\frac{d^2}{dr^2}(rR_{k0})+k^2rR_{k0}=0,$$ откуда $R_{k0}=c_1\frac
{\sin kr}{r}+c_2\frac{\cos kr}{r}$. Для решения уравнения с $l\ne 0$
делаем подстановку $R_{kl}=r^l\chi_{kl}$. Получаем уравнение для $\chi_{kl}$:
$$\chi''_{kl}+\frac{2(l+1)}{r}\chi'_{kl}+k^2\chi_{kl}=0.$$ Продифференцировав
это уравнение по $r$ и подставив $\tilde \chi=\frac{\chi'_{kl}}{r}$,
получаем, что $\tilde \chi$ удовлетворяет уравнению для $\chi_{k, \, l+1}$.
Отсюда следует, что
$$\chi_{kl}=\left(\frac{1}{r}\frac{d}{dr}\right)^l\chi_{k0}$$ и решение
(\ref{svob_dim3}) имеет вид $$R_{kl}=C_1j_l(kr)+C_2n_l(kr),$$ где
$$j_l(z)=(-1)^lz^l\left(\frac{1}{z}\frac{d}{dz}\right)^l\frac{\sin z}{z},$$
$$n_l(z)=(-1)^{l+1}z^l\left(\frac{1}{z}\frac{d}{dz}\right)^l\frac{\cos z}
{z}$$ --- сферические функции Бесселя и Неймана соответственно. При
малых $z$ выполнено $j_l(z)\sim z^l$, $n_l(z)\sim z^{-(l+1)}$, поэтому $C_2=0$.
При $z\gg l$ имеем $$j_l(z)\sim \frac{1}{z}\cos \left(z-\frac{\pi}{2}
(l+1)\right), \; n_l(z)\sim \frac{1}{z}\sin \left(z-\frac{\pi}{2}(l+1)
\right).$$ Таким образом, общее решение стационарного уравнения Шредингера
имеет вид $$\psi(\vec r)=\sum \limits_{l=0}^\infty \sum \limits_{|m|
\le l}C_{lm}j_l(kr)Y_{lm}(\theta, \, \varphi).$$ Теперь рассмотрим случай
потенциальной ямы:
\begin{align}
\label{yama_dim3}
V(r)=-U_0\theta(a-r), \; U_0>0.
\end{align}
При $-U_0<E<0$ получаем
связанные состояния. Так же, как для (\ref{svob_dim3}), при $r>a$
находим решение в виде $$R_{El}=C_lr^l\left(\frac{1}{r}\frac{d}{dr}\right)^l
R_{E0},$$ где $k=\sqrt{\frac{2\mu|E|}{\hbar^2}}$. Так как собственная
функция не может экспоненциально возрастать на бесконечности, то $R_{E0}=
C\frac{e^{-kr}}{r}$. При $r<a$ $R_{El}=\tilde Cj_l(\varkappa r)$, где
$\varkappa=\sqrt{\frac{2\mu(U_0+E)}{\hbar^2}}$. Условия склейки для
потенциала (\ref{yama_dim3}) --- это условия $C^1$-гладкости. Отсюда
получаем собственные значения оператора Шредингера. При $l=0$ эти
условия имеют вид $$\ctg \varkappa a=-\frac{k}{\varkappa}.$$ При $U_0<
\frac{\pi^2\hbar^2}{8\mu a^2}$ это уравнение не имеет решений, то есть
нет связанных состояний. Это можно объяснить следующим образом: флуктуация
кинетической энергии оказывается больше глубины ямы, и частица не может
удерживаться в яме. Заметим, что в одномерном случае связанное состояние
всегда есть. \par
Рассмотрим случай потенциальной ямы с бесконечными стенками:
$$U(r)=\left\{\begin{array}{l}0, \; r<a, \\ \infty, \; r>a.\end{array}\right.$$
Так же, как для одномерного случая, можно показать, что $R_{kl}(a)=0$.
Отсюда получаем $j_l(ka)=0$, то есть $k=\frac{z_{nl}}{a}$, где $z_{nl}$ ---
$n$-й нуль сферической функции Бесселя $j_l$. Таким образом, дискретный спектр
энергии $E_{nl}=\frac{\hbar^2z^2_{nl}}{2\mu a^2}$ образует двухпараметрическое
семейство. При $l=0$ минимальное значение $k$ равно $\frac{\pi}{a}$, то
есть основное состояние имеет ту же энергию, что и в соответствующей
одномерной задаче.
\subsection{Кулоново поле}
Рассмотрим движение в поле с потенциалом $U=-\frac{\alpha}{r}$, где $\alpha
>0$ отвечает притяжению, $\alpha<0$ --- отталкиванию. При $\alpha=e^2$
это соответствует атому водорода, при $\alpha=Ze^2$ --- одноэлектронному
иону. \par
Радиальное уравнение Шредингера для случая связанных состояний ($E<0$)
может быть записано в виде
\begin{align}
\label{kulon_dim3}
R''_{kl}+\frac{2R'_{kl}}{r}+\left(\frac{2}{r_B}-\frac{l(l+1)}{r}\right)
\frac{R_{kl}}{r}-k^2R_{kl}=0,
\end{align}
где $k=\sqrt{\frac{2\mu|E|}{\hbar^2}}$, $r_B=\frac{\hbar^2}{\mu\alpha}$
--- боровский радиус. Найдем его решения. \par
{\bf Способ 1.} Перейдем к безразмерной переменной $x=2kr$ и введем
функцию $\Phi(x)$, выделив регулярные асимптотики при $x\rightarrow 0$
($x^l$) и $x\rightarrow \infty$ ($e^{-x/2}$): $R=Nx^le^{-x/2}\Phi(x)$.
Подставив это в (\ref{kulon_dim3}), получаем уравнения Куммера для
$\Phi$:
\begin{align}
\label{kummer}
x\Phi''+(\beta-x)\Phi'-\gamma\Phi=0,
\end{align}
где $\beta=2(l+1)$, $\gamma=l+1-\frac{1}{kr_B}$. Регулярным при $x=0$
решением уравнения (\ref{kummer}) является вырожденная гипергеометрическая
функция $$\Phi(\gamma, \, \beta, \, x)=1+\frac{\gamma}{\beta}x+\frac{\gamma
(\gamma+1)}{\beta(\beta+1)}\frac{x^2}{2!}+\dots+\frac{\gamma(\gamma+1)\dots
(\gamma+n-1)}{\beta(\beta+1)\dots(\beta+n-1)}\frac{x^n}{n!}+\dots$$
Если $\gamma+n-1\ne 0$ ни при каком $n$, то $\Phi(x)\sim e^x$ при $x
\rightarrow \infty$, и соответствующее решение не принадлежит оснащению
гильбертова пространства. Если $-\gamma\in \Z_+$, то ряд в некоторый момент
обрывается. Тогда $R\in L_2((0, \, \infty), \, r^2\, dr)$ и решение
является собственным вектором для дискретного спектра. Условие обрыва
ряда $$\gamma=l+1-\frac{1}{kr_B}=-n_r$$ ($n_r$ --- радиальное квантовое
число) приводит к квантованию энергии $E_n=-\frac{\alpha}{2r_Bn^2}$, где
$n=n_r+l+1$ --- главное квантовое число, принимающее натуральные
значения. Так как два целых числа $n_r$ и $l$ входят в выражение для $n$
в виде суммы, то возникает дополнительное вырождение уровней энергии.
При заданном $n$ орбитальное квантовое число $l$ может изменяться от 0
до $n-1$, поэтому полная кратность вырождения $g_n$ уровня с заданным
$n$ равна $$g_n=\sum \limits_{l=0}^{n-1}(2l+1)=n^2.$$
Вырожденная гипергеометрическая функция при условии обрыва ряда становится
полиномом, принадлежащим семейству обобщенных полиномов Лагерра
$$L_p^q(x)=(-1)^q\frac{(p!)^2}{q!(p-q)!}\Phi(q-p, \, q+1, \, x)=
\frac{p!}{(p-q)!}e^x\left(\frac{d}{dx}\right)^ne^{-x}x^{p-q},$$ где
$p-q\in \Z_+$. Таким образом, радиальную функцию дискретного спектра
можно записать в виде $$R_{nl}=N_{nl}x^le^{-x/2}L_{n+1}^{2l+1}(x),$$ где
$N_{nl}$ определяется из условия $\int \limits_0^\infty R_{nl}^2r^2\, dr=1$
и равна $N_{nl}=\frac{2}{n^2}\frac{[(n-l-1)!]^{1/2}}{[(n+l)!]^{3/2}}$. \par
{\bf Способ 2.} Собственные векторы и собственные значения можно найти
с помощью метода факторизации. Пусть оператор $\hat A$ имеет вид
$\hat A=\hat \theta_1^*\hat \theta_1+\lambda_1$. По индукции
положим $\hat A_{i+1}=\hat\theta_i\hat\theta_i^*+\lambda_i$ и допустим,
что существуют $\hat \theta_{i+1}$ и $\lambda_{i+1}$ такие, что $\hat A_{i+1}
=\hat \theta_{i+1}^*\hat \theta_{i+1}+\lambda_{i+1}$. Спектры операторов
$\hat \theta_i\hat\theta_i^*$ и $\hat \theta_i^*\hat\theta_i$ одинаковы
с точностью до $\{0\}$. Предположим, что для всех $i$ оператор $\hat\theta_i^*$
ядра не имеет, а $\hat \theta_i$ имеет нетривиальное ядро, дискретный
спектр оператора $\hat A$ лежит в $\{E<E_0\}$, $E_0\in \overline \R$, и
$\lambda_n\underset{n\rightarrow \infty}{\rightarrow} E_0$. Докажем, что
$\{\lambda_i\}$ --- это последовательность точек дискретного спектра. Пусть
\begin{align}
\label{8spektr}
(\hat \theta_1^*\hat\theta_1+\lambda_1)\psi=E\psi.
\end{align}
Если $\hat\theta_1\psi=0$, то $E=\lambda_1$. Пусть $\hat \theta_1\psi\ne 0$.
Подействуем оператором $\hat\theta_1$ на уравнение (\ref{8spektr}) и
получим $$(\hat\theta_1\hat\theta_1^*+\lambda_1)\hat\theta_1\psi=E\hat\theta_1\psi,$$ то есть
\begin{align}
\label{8spektr1}
\hat A_2\psi_1=E\psi_1,
\end{align}
где $\psi_1=\hat\theta_1\psi$. Значит, $$(\hat\theta_2^*\hat\theta_2+
\lambda_2)\psi_1=E\psi_1.$$ Если $\hat\theta_2\psi_1=0$, то $E=\lambda_2$.
Если $\hat\theta_2\psi_1\ne 0$, то действуем оператором $\hat\theta_2$ на
(\ref{8spektr1}), и т. д. Так как $\langle \hat\theta_i^*\hat\theta_i
\varphi, \, \varphi\rangle=\langle\hat\theta_i\varphi, \, \hat\theta_i\varphi
\rangle\ge 0$, то спектр $\hat\theta_i^*\hat\theta_i$ неотрицательный.
Поскольку $E<E_0$ и $\lambda_n\underset{n\rightarrow \infty}{\rightarrow} E_0$,
то $E-\lambda_n<0$ для некоторого $n\in \N$ и поэтому $\hat\theta_{n-1}
\psi_{n-2}=\psi_{n-1}=0$, то есть $E=\lambda_{n-1}$. Таким образом,
если $E$ принадлежит дискретному спектру $\hat A$, то $E=\lambda_n$ для
некоторого $n$. Обратно, пусть $E=\lambda_n$. По предположению, оператор
$\hat\theta_n$ имеет нетривиальное ядро. Пусть $\psi\in \ker \hat\theta_n$.
Тогда $\hat A_n\psi=E\psi$, откуда $$\hat A_{n-1}\hat\theta_{n-1}^*
\psi=\theta_{n-1}^*\hat A_n\psi=E\hat\theta_{n-1}^*\psi.$$ Так как
ядро оператора $\hat\theta_{n-1}^*$ тривиально, то $\hat\theta_{n-1}^*
\psi\ne 0$ и $E$ является собственным значением $\hat A_{n-1}$. Дальше
по индукции доказывается, что $E$ является собственным значением $\hat A$.
\par
Применим этот метод для оператора $$\hat A=\hat p_r^2+\frac{l(l+1)\hbar
^2}{r^2}-\frac{2c}{r},$$ где $\hat p_r=-i\hbar(\frac{\partial}{\partial r}
+\frac{1}{r})$ --- оператор, канонически
сопряженный к радиальной координате $r$ (в частности, он удовлетворяет
коммутационным соотношениям $[\hat p_r, \, f(r)]=\frac{\hbar}{i}f'(r)$).
Оператор $\hat p_r^2$ равен $-\hbar^2(\frac{d^2}{dr^2}+\frac{2}{r}\frac
{d}{dr})$, то есть совпадает с радиальной частью оператора Лапласа.
Ищем операторы $\hat\theta_j$ в виде $$\hat\theta_j=\hat p_r+i\left(a_j+
\frac{b_j}{r}\right), \; \hat\theta_j^*=\hat p_r-i\left(a_j+\frac{b_j}
{r}\right).$$
Тогда $$\hat\theta_j^*\hat\theta_j=\hat p_r^2+a_j^2+\frac{2a_jb_j}{r}
+\frac{b_j^2}{r^2}-\frac{b_j\hbar}{r^2},$$ откуда находим условия на
$a_1$, $b_1$: $$a_1b_1=-c, \; b_1(b_1-\hbar)=l(l+1)\hbar^2.$$ Решив эту
систему, получаем $b_1=\hbar(l+1)$, $a_1=-\frac{c}{\hbar(l+1)}$ или
$b_1=-l\hbar$, $a_1=\frac{c}{l\hbar}$. Можно показать, что в первом
случае $\hat\theta_1$ имеет нетривиальное ядро, а во втором не имеет
(для этого решаем соответствующее дифференциальное уравнение и оцениваем
поведение решения на бесконечности). Кроме того, в первом случае
$\hat\theta_1^*$ не имеет нетривиального ядра. Таким образом,
$$a_1=-\frac{c}{\hbar(l+1)}, \; b_1=\hbar(l+1), \; \lambda_1=-\frac{c^2}
{(l+1)^2\hbar^2}.$$ Теперь найдем остальные собственные значения. Из
равенства $$\hat\theta_{j+1}^*\hat\theta_{j+1}+\lambda_{j+1}=\hat \theta_j
\hat\theta_j^*+\lambda_j$$ получаем условия на $a_{j+1}$,
$b_{j+1}$ и $\lambda_{j+1}$: $$\left\{\begin{array}{l} a_{j+1}b_{j+1}
=a_jb_j, \\ b_{j+1}(b_{j+1}-\hbar)=b_j(b_j+\hbar), \\ \lambda_{j+1}
+a_{j+1}^2=\lambda_j+a_j^2.\end{array}\right.$$ По индукции доказывается,
что $\hat \theta_{j+1}$ имеет нетривиальное ядро тогда и только тогда,
когда в качестве решения этой системы берется $$a_{j+1}=-\frac{c}{(l+j+1)
\hbar}, \; b_{j+1}=(l+j+1)\hbar, \; \lambda_{j+1}=-\frac{c^2}{(l+j+1)^2
\hbar^2}.$$ Собственный вектор, соответствующий собственному значению
$\lambda_{j+1}$, имеет вид $\hat \theta_j^*\dots\hat \theta_1^*\psi_0$,
где $\psi_0$ --- решение $\hat A\psi_0=\lambda_1\psi_0$.
\subsection{Симметрия $SO(4)$ для кулонова поля}
Как было показано, при движении в кулоновом поле уровни энергии $E_n$
зависят только от $n=n_r+l+1$, то есть имеет место дополнительное
вырождение. Оказывается, что это связано с наличием дополнительного
интеграла движения. \par
При классическом движении частицы в кулоновом поле имеет место специфический
для этого поля закон сохранения. А именно, вектор Лапласа--Рунге--Ленца
$$\vec A=\frac{\vec p\times \vec L}{\mu}-\frac{\alpha\vec r}{r}$$ является
интегралом движения. В квантовой механике этой величине соответствует
симметрический оператор $$\hat{\vec A}=\frac{1}{2\mu}(\vec{\hat p}\times
\vec{\hat L}-\vec{\hat L}\times \vec{\hat p})-\alpha{\frac{\hat{\vec r}}{r}}.$$
Рассмотрим ограничение этого оператора на (конечномерное) подпространство
${\cal H}_E$ собственных функций гамильтониана $\hat H$, соответствующих
значению энергии $E$. На этом подпространстве $\hat{\vec A}$ коммутирует
с $\hat H$. Кроме того, выполнены следующие коммутационные соотношения:
$$[L_j, \, A_k]=i\hbar\varepsilon_{jkl}A_l, \; [A_j, \, A_k]=\frac{2\hbar}
{i\mu}\varepsilon_{jkl}L_lE.$$ Отсюда следует, что $\hat A_j$ коммутирует
с $\hat L_j$, но не коммутирует с $\hat L^2$. Таким образом, имеется
новая сохраняющаяся величина, не измеримая одновременно с другими
сохраняющимися величинами, что и приводит к дополнительному вырождению. \par
Положим $$\hat l_j=\frac{\hat L_j}{\hbar}, \; \;\; \hat u_j=-\hat A_j
\sqrt{\frac{\mu}{-2\hbar^2 E}}.$$
Для этих операторов коммутационные соотношения имеют вид
$$[\hat l_j, \, \hat u_k]=i\varepsilon_{jkl}\hat u_l, \; [\hat u_j, \,
\hat u_k]=i\varepsilon_{jkl}\hat l_l, \; [\hat l_j, \, \hat l_k]=
i\varepsilon_{jkl}\hat l_l.$$
Введем операторы $$\hat {\vec j}_+=\frac12(\hat{\vec l}+\hat{\vec u}),
\; \hat {\vec j}_-=\frac12(\hat{\vec l}-\hat{\vec u}).$$ Для них имеем
$$[\hat j_{+s}, \, \hat j_{+k}]=i\varepsilon_{skl}\hat j_{+l}, \;
[\hat j_{-s}, \, \hat j_{-k}]=i\varepsilon_{skl}\hat j_{-l}, \;
[\hat j_{+s}, \, \hat j_{-k}]=0.$$ Эти коммутационные соотношения
с точностью до умножения на мнимую единицу совпадают с коммутационными
соотношениями алгебры $so(4)$, то есть операторы $ij_{\pm}$ задают
некоторое ее представление. \par
Так как операторы $\hat j_{+z}$, $\hat j_{-z}$, $\hat j_+^2=\hat j_{+x}^2
+\hat j_{+y}^2+\hat j_{+z}^2$ и $\hat j_-^2=\hat j_{-x}^2+\hat j_{-y}^2+
\hat j_{-z}^2$ коммутируют, то ${\cal H}_E$ имеет базис из векторов,
являющихся собственными для всех этих операторов. Собственные значения
$\hat j_{\pm}^2$ равны соответственно $j_{\pm}(j_{\pm}+1)$, где $j_{\pm}$
может быть либо целым, либо полуцелым неотрицательным числом. Имеют
место равенства
$$\hat{\vec l}\hat{\vec u}=\hat{\vec u}\hat{\vec l}=0, \; \hat{\vec l}^2
+\hat{\vec u}^2=-1-\frac{\mu \alpha^2}{2E\hbar^2}.$$ Отсюда
$$\hat j_{\pm}^2=-\frac14\left(1+\frac{\mu \alpha^2}{2E\hbar^2}\right)=
j_{\pm}(j_{\pm}+1),$$ $$E=-\frac{\mu \alpha^2}{2\hbar^2(2j_{\pm}+1)^2}.$$
Ранее было показано, что $E=-\frac{\mu \alpha^2}{2\hbar^2n^2}$, где $n$
--- любое натуральное число, и что кратность вырождения равна $n^2$.
Значит, $j_{\pm}=\frac{n-1}{2}$ принимает все целые и полуцелые
неотрицательные значения и при фиксированном $E$ вектор из ${\cal H}_E$
с заданными собственными значениями $\hat j_{\pm z}$ ровно один (так как
если на него действовать лестничными операторами, то получится $(2j_++1)
(2j_-+1)=n^2$ линейно независимых векторов). Таким образом, на
${\cal H}_E$ задано неприводимое представление алгебры $su(2)\oplus
su(2)$. Так как ${\cal H}_E$ конечномерно, то это представление
единственным образом интегрируется до представления односвязной
группы $SU(2)\times SU(2)$. \par
Докажем, что это представление является однозначным представлением
группы $SO(4)$. Так как $SO(4)=(SU(2)\times SU(2))/\{\pm 1\}$ [18, стр. 105],
то достаточно показать, что $(-1, \, -1)\in SU(2)\times SU(2)$
при данном представлении переходит в единичный оператор. \par
Пусть в $\C^n$ задано неприводимое представление группы $SU(2)$,
и $e_j$, $j=1, \, \dots, \, n$ --- собственные векторы одного
из генераторов $X_3$ алгебры $su(2)$. Тогда в $\C^n\otimes \C^n=\C^{n^2}$
задано неприводимое представление $SU(2)\times SU(2)$:
$(g_1, \, g_2)(e_k^1\otimes e_l^2)=(g_1e_k^1)\otimes (g_2e_l^2)$. При этом
$e_k^1\otimes e_l^2$ являются собственными векторами для $\alpha X_3^1
+\beta X_3^2\in su(2)\oplus su(2)$, то есть это представление
совпадает с построенным представлением на ${\cal H}_E$. Остается
заметить, что $(-1, \, -1)(e_k^1\otimes e_l^2)=(-e_k^1)\otimes(-e_l^2)=
e_k^1\otimes e_l^2$. \par
{\bf Замечание.} Можно показать, что дополнительное вырождение уровней
с различными значениями момента $l$ имеют место и для движения в
центрально-симметричном поле $U=\frac{m\omega^2r^2}{2}$ (пространственный
осциллятор). Специфике кулонова и осцилляторного полей в квантовой
механике отвечает в классической механике специфика, состоящая в том,
что существуют замкнутые траектории частиц в этих (и только этих) полях.
\subsection{Рассеяние}
В главе \ref{scat_th} было получено уравнение Липпмана--Швингера
для функций $\psi_{\pm}=\Omega_{\pm}\psi^0$:
$$\psi_{\pm}=\psi^0+\left(E\pm i0-\hat H_0\right)^{-1}V\psi_{\pm}.$$
Запишем его для случая, когда $\hat H_0=-\Delta$, $\hat H=-\Delta+V$
в $L_2(\R^3)$. Пусть $$\tilde f(k)=\frac{1}{(2\pi)^{3/2}}\int e^{-i\vec k
\vec y}f(\vec y)\, d^3\vec y.$$ Тогда для $z\in \C\backslash \R$ выполнено
$$\left((\hat H_0-z)^{-1}f\right)(\vec x)=\frac{1}{(2\pi)^{3/2}}\int
e^{i\vec k\vec x}\frac{\tilde f(\vec k)}{k^2-z}\, d^3\vec k=\int
G(\vec x, \, \vec y;\, z)f(\vec y)\, d^3\vec y,$$ где
$$G(\vec x, \, \vec y;\, z)=\frac{1}{4\pi}
\frac{e^{i|\vec x-\vec y|\sqrt{z}}}{|\vec x-\vec y|}, \; {\rm Im}\,
\sqrt{z}>0$$ (доказательство см. в [12]). Если $\psi^0=e^{i\vec k\vec x}$,
$E=k^2$, то уравнение Липпмана--Швингера приобретает вид
\begin{align}
\label{lippman_r3}
\psi_{\pm}(\vec x)=e^{i\vec k\vec x}-\frac{1}{4\pi}\int \frac{e^{\pm ik
|\vec x-\vec y|}}{|\vec x-\vec y|}V(\vec y)\psi_{\pm}(\vec y)\, d^3\vec y.
\end{align}
Достаточное условие разрешимости уравнений (\ref{lippman_r3}) дает
следующая теорема [12, гл. IV, \S 3]:
\begin{Trm}
Пусть потенциал $V$ удовлетворяет условию
\begin{align}
\label{cond_ex_l1}
|V(x)|\le C\left(1+|x|\right)^{-5-\delta}, \; \delta>0.
\end{align}
Тогда уравнения (\ref{lippman_r3}) имеют единственное решение в
пространстве $C_b(\R^3)$ непрерывных ограниченных функций.
\end{Trm}
Другое достаточное условие разрешимости приведено в [3, т. 3, теорема
XI.41]: если $V\in L_1(\R^3)$ и
$$\|V\|^2_R=\int \limits_{\R^6}\frac{|V(\vec x)||V(\vec y)|}{|\vec x-\vec
y|^2}\, d^3\vec x\, d^3\vec y<\infty,$$ то уравнения (\ref{lippman_r3})
разрешимы при $k^2\in \R_+\backslash {\cal E}$, где ${\cal E}$ ---
замкнутое множество меры нуль Лебега. Это решение единственно в классе
функций $\{\varphi:|V|^{1/2}\varphi\in L_2\}$. \par
\begin{Trm} {\rm [12, гл. IV, \S 3]}
Пусть потенциал $V$ удовлетворяет условию (\ref{cond_ex_l1}). Тогда
решение $\psi_+(\vec r, \, \vec k)$ уравнения Липпмана--Швингера
имеет асимптотику
\begin{align}
\label{lippman_asympt}
\psi_+(\vec r, \, \vec k)=e^{i\vec k\vec r}
-2\pi^2T(\vec k', \, \vec k)\frac{e^{ikr}}{r}+O\left(\frac{1}{r^{1+
\varepsilon}}\right),
\end{align}
где $\varepsilon=\frac12+\delta$, $\vec k'=k\frac{\vec r}{r}$.
\end{Trm}
При этом
\begin{align}
\label{t_matrix}
T(\vec k', \, \vec k)=\frac{1}{(2\pi)^3}\int
e^{-i\vec k'\vec r}V(\vec r)\psi_+(\vec r, \, \vec k)\, d^3r.
\end{align}
Эта функция называется $T$-матрицей. Через нее выражается оператор
рассеяния $\hat S$ [3, т. 3, теорема XI.42]: для почти всех
$\vec k=k\vec n$ выполнено равенство
\begin{align}
\label{s_and_t}
(\hat S\varphi)(\vec k)=\varphi(\vec k)-\pi ki\int \limits_{S^2}T(k\vec n,
\, k\vec n')\varphi(k\vec n')\, d\Omega',
\end{align}
где $d\Omega'$ --- элемент площади единичной сферы. Так как $$\delta
(k^2-k'^2)\, d^3\vec k'=\delta(k'^2-k^2)k'^2\, dk'\, d\Omega'=
\frac{k'}{2}\delta(k'^2-k^2)\, dk'^2\, d\Omega',$$ то (\ref{s_and_t})
переписывается в виде $$S(\vec k, \, \vec k')=\delta(\vec k-\vec k')
-2\pi iT(\vec k, \, \vec k')\delta(k^2-k'^2),$$ где $S(\vec k, \, \vec k')$
--- интегральное ядро оператора $\hat S$. \par
По теореме \ref{s_matrix_decomp}, оператор $\hat S$ разлагается в
прямой интеграл операторов $S(k):L_2(S^2)\rightarrow L_2(S^2)$,
таких что в импульсном представлении $(\hat S\varphi)(k, \, \vec n)=
(S(k)\varphi(k, \, \cdot))(\vec n)$. Зафиксируем $k$ и положим
$F(\vec n)=\varphi(k, \, \vec n)$, $\tilde S=S(k)$, $f(\vec n', \,
\vec n)=-2\pi^2 T(k\vec n, \, k\vec n')$. Тогда (\ref{s_and_t})
переписывается в виде
\begin{align}
\label{s_of_k}
\tilde SF(\vec n)=F(\vec n)+\frac{ik}{2\pi}\int \limits_{S^2}f(\vec n', \, \vec n)
F(\vec n')\, d\Omega'=(1+2ik\hat f)F(\vec n),
\end{align}
где $\hat fF(\vec \nu)=\frac{1}{4\pi}
\int \limits_{S^2}f(\vec n, \, \vec \nu)F(\vec n)\, d\Omega$. Заметим,
что (\ref{lippman_asympt}) переписывается в виде
\begin{align}
\label{lippman_asympt2}
\psi_+(\vec r)=e^{i\vec k\vec r}+f(\vec n, \, \vec n')\frac{e^{ikr}}{r}
+O(r^{-1-\varepsilon}).
\end{align}
В физической литературе (см., напр., [1]) матрица рассеяния задается
по определению формулой (\ref{s_of_k}). При этом проводятся следующие
рассуждения. Рассматриваются решения $\psi_{\vec n}$ уравнения Шредингера
с асимптотикой (\ref{lippman_asympt2}), являющееся суперпозицией
плоской падающей и сферической рассеянной волн. Здесь $k$ --- волновое
число, $\vec n$ --- единичный вектор вдоль исходного направления
движения частицы (соответствующего падающей волне), $\vec n'$ ---
единичный вектор вдоль направления рассеянной частицы (соответствующего
рассеянной волне). Волновой вектор определяется как $\vec k=k\vec n$,
пространственный вектор --- как $\vec r=r\vec n'$. \par
{\bf Замечание.} Так как в уравнении Шредингера перед оператором Лапласа
стоит коэффициент $\frac{\hbar^2}{2\mu}$, то в соответствующем уравнении
Липпмана--Швингера вместо $V$ пишется $\frac{2\mu}{\hbar^2}V$, а $k=
\sqrt{\frac{2\mu E}{\hbar^2}}$. \par
Функции вида
\begin{align}
\label{psi_int_s2_psin}
\psi=\int \limits_{S^2}\psi_{\vec n}F(\vec n)\, d\Omega
\end{align}
также являются решением стационарного уравнения Шредингера и описывают
некоторый возможный процесс рассеяния. Предположим, что $F$ достаточно
гладкая и найдем асимптотическое значение выражения
\begin{align}
\label{8ras_asimpt}
\int \limits_{S^2}F(\vec n)e^{ikr\vec n\vec n'}\, d\Omega+\frac{e^{ikr}}
{r}\int \limits_{S^2}F(\vec n)f(\vec n, \, \vec n')\, d\Omega
\end{align}
при $r\rightarrow \infty$. Введем полярные координаты, положив в качестве
$\theta$ угол между направлениями $\vec n$ и $\vec n'$, то есть
$\vec n=(\sin \theta \cos \varphi, \, \sin \theta \sin \varphi, \, \cos \theta)$.
Пусть $$\Phi(\cos \theta)=\int \limits _0^{2\pi}F(\sin \theta \cos \varphi,
\, \sin \theta \sin \varphi, \, \cos \theta)\, d\varphi,$$ $\lambda=kr$.
Функция $\Phi$ гладко зависит от $\cos \theta$ всюду, кроме, может быть, $\cos
\theta=\pm 1$; в окрестности этих точек $$\Phi'(\cos \theta)=O\left(
\frac{1}{\sqrt{1-|\cos \theta|}}\right).$$
Тогда (\ref{8ras_asimpt}) имеет вид
$$\int \limits_0^{\pi}\Phi(\cos \theta)e^{i\lambda\cos \theta}\sin \theta
\, d\theta+\frac{e^{ikr}}{r}\int \limits_{S^2}F(\vec n)f(\vec n, \,
\vec n')\, d\Omega=$$ $$=\int \limits_{-1}^1\Phi(\xi)e^{i\lambda\xi}
\, d\xi+\frac{e^{ikr}}{r}\int \limits_{S^2}F(\vec n)f(\vec n, \, \vec n')
\, d\Omega.$$ Проинтегрировав первое слагаемое по частям и воспользовавшись
тем, что $\Phi(\pm 1)=2\pi F(\pm \vec n')$, получаем
$$\frac{2\pi}{i\lambda}(F(\vec n')e^{i\lambda}-F(-\vec n')e^{-i\lambda})-
\frac{1}{i\lambda}\int \limits_{-1}^1\Phi'(\xi)e^{i\lambda\xi}\, d\xi+
\frac{e^{ikr}}{r}\int \limits_{S^2}F(\vec n)f(\vec n, \, \vec n')\, d\Omega.$$
Докажем, что второе слагаемое равно $o(\frac{1}{\lambda})$ при $\lambda
\rightarrow \infty$. В самом деле, $\Phi'(\xi)\sqrt{1+\xi}$ и $\Phi'(\xi)
\sqrt{1-\xi}$ ограничены при $\xi\in [-1, \, 0]$ и $\xi\in [0, \, 1]$
соответственно и поэтому
$$\left|\int \limits_{-1}^1\Phi'(\xi)e^{i\lambda\xi}\, d\xi\right|
\le C\left(\left|\int\limits_{-1}^{-1+1/\lambda}\frac{e^{i\lambda\xi}}
{\sqrt{1+\xi}}\, d\xi\right|+\left|\int \limits _{-1+1/\lambda}^0
\frac{e^{i\lambda\xi}}{\sqrt{1+\xi}}\, d\xi\right|+\right.$$ $$\left.
+\left|\int \limits _0^{1-1/\lambda}\frac{e^{i\lambda\xi}}{\sqrt{1-\xi}}
\, d\xi\right|+\left|\int \limits _{1-1/\lambda}^1 \frac{e^{i\lambda\xi}}
{\sqrt{1+\xi}}\, d\xi\right|\right)=O\left(\frac{1}{\sqrt{\lambda}}\right)$$
при больших значениях $\lambda$
(для доказательства второе и третье слагаемое снова интегрируем по
частям). В итоге получаем, что (\ref{8ras_asimpt}) равно
$$2\pi iF(-\vec n')\frac{e^{-ikr}}{kr}-2\pi iF(\vec n')\frac{e^{ikr}}{kr}
+\frac{e^{ikr}}{r}\int \limits_{S^2}F(\vec n)f(\vec n, \, \vec n')\, d\Omega+
o\left(\frac{1}{r}\right), \; r\rightarrow \infty.$$ Перепишем это в виде
\begin{align}
\label{s_matrix}
\frac{2\pi i}{k}\left(\frac{e^{-ikr}}{r}F(-\vec n')-\frac{e^{ikr}}{r}
\hat SF(\vec n')\right),
\end{align}
где
\begin{align}
\label{s_matr_def}
\hat S=1+2ik\hat f,
\end{align}
а $\hat f$ --- интегральный оператор
\begin{align}
\label{hat_f_def}
\hat fF(\vec n')=\frac{1}{4\pi}\int \limits_{S^2}
f(\vec n, \, \vec n')F(\vec n)\, d\Omega.
\end{align}
Первое слагаемое в (\ref{s_matrix}) --- это сходящаяся к центру волна,
второе слагаемое --- расходящаяся от центра волна. \par
Унитарность $\hat S$ доказывается следующим образом. Пусть
\begin{align}
\label{def_j}
\vec j=\frac{\hbar}{2\mu i}(\psi^*\bigtriangledown \psi-\psi
\bigtriangledown \psi^*),
\end{align}
где $\psi$ задается формулой (\ref
{psi_int_s2_psin}). Так как $\psi$ является решением стационарного
уравнения Шредингера, то $\div \vec{j}=0$ и поэтому
$$\int \limits_{S_r^2}\vec j\, d\vec S=0$$ (где $S^2_r$ --- сфера радиуса
$r$). Поскольку проекция вектора градиента функции на вектор нормали к
сфере равна производной функции по радиальной координате, выполнено
равенство $$\int \limits _{S^2_r}\left(\psi^*\frac{\partial \psi}{\partial
r}-\psi\frac{\partial \psi^*}{\partial r}\right)\, dS=0.$$ Явно вычислив
$\psi^*\frac{\partial \psi}{\partial r}-\psi\frac{\partial \psi^*}
{\partial r}$, получим, что
$$\int \limits _{S^2}|F(-\vec n')|^2\, d\Omega=\int \limits _{S^2}|\hat SF
(\vec n')|^2\, d\Omega,$$ то есть оператор рассеяния унитарный: $\hat S
\hat S^+=1$. Подставим это в (\ref{s_matr_def}):
$$1=\hat S\hat S^+=(1+2ik\hat f)(1-2ik\hat f^+)=1+2ik\hat f-2ik\hat f^+
+4k^2\hat f\hat f^+,$$ откуда $$\hat f-\hat f^+=2ik\hat f\hat f^+.$$
Учитывая (\ref{hat_f_def}), перепишем условие унитарности для рассеяния в
виде
\begin{align}
\label{8unit}
f(\vec n, \, \vec n')-f^*(\vec n', \, \vec n)=\frac{ik}{2\pi}
\int \limits_{S^2}f(\vec n, \, \vec n'')f^*(\vec n', \, \vec n'')\,
d\Omega''.
\end{align}
\par
Введем понятие дифференциального сечения рассеяния. Пусть волновая функция
имеет вид (\ref{lippman_asympt2}), $\vec r=r\vec n'$, $\vec k=k\vec n$.
Положим $\psi_1(\vec r)=f(\vec n, \, \vec n')\frac{e^{ikr}}{r}$
(рассеянная волна), $\psi_2(\vec r)=e^{i\vec{k}\vec{r}}$ (падающая
волна), а $j_l$ задается формулой (\ref{def_j}) с $\psi:=\psi_l$, $l=1, \, 2$.
Тогда поток вероятности рассеянной волны через площадку $r^2\,
d\Omega_{\vec n'}$ равен $$\vec j_1\vec n'r^2\, d\Omega
_{\vec n'}=\frac{\hbar r^2}{2\mu i}\left(\frac{e^{-ikr}}{r}f^*(\vec n,
\, \vec n')\frac{\partial}{\partial r}\frac{e^{ikr}}{r}f(\vec n, \, \vec n')
-\frac{e^{ikr}}{r}f(\vec n, \, \vec n')\frac{\partial}{\partial r}\frac
{e^{-ikr}}{r}f^*(\vec n, \, \vec n')\right)\, d\Omega_{\vec n'}=$$
$$=\frac{\hbar k}{\mu}|f(\vec n, \, \vec n')|^2\, d\Omega_{\vec n'}.$$ Поток
вероятности падающей волны через единичную площадку с вектором нормали $\vec n$ равен
$$\vec j_2\vec n=\frac{\hbar}{2\mu i}\left(e^{-i\vec k\vec r}\bigtriangledown
e^{i\vec k\vec r}-e^{i\vec k\vec r}\bigtriangledown e^{-i\vec k\vec r}\right)
\vec n=\frac{\hbar}{\mu}\vec k\vec n=\frac{\hbar k}{\mu}.$$ Назовем
дифференциальным сечением рассеяния отношение
$$d\sigma\stackrel{def}{=}\frac{\vec j_1\vec nr^2\, d\Omega
_{\vec n'}}{\vec j_2\vec n}=|f(\vec n, \, \vec n')|^2\, d\Omega_{\vec n'}.$$
Тогда полное сечение рассеяния равно $$\sigma=\int \limits_{S^2}d\sigma
=\int \limits_{S^2}|f(\vec n, \, \vec n')|^2\, d\Omega_{\vec n'}.$$
Значит, при $\vec n=\vec n'$ интеграл в правой части (\ref{8unit}) равен
полному сечению рассеяния. Разность в левой части равенства сводится к
мнимой части амплитуды $f(\vec n, \, \vec n)$. Таким образом, получаем
$$\Im f(\vec n, \, \vec n)=\frac{k}{4\pi}\sigma.$$ Это равенство называется
оптической теоремой рассеяния. Приведенные формулы справедливы в любом
потенциальном поле, для которого полное сечение рассеяния конечно. \par
Теперь рассмотрим случай центрально-симметричного поля. Свободная частица,
движущаяся в положительном направлении оси $z$, описывается плоской
волной, которую мы напишем в виде $\psi=e^{ikz}$. Рассеянные частицы
описываются вдали от центра расходящейся сферической волной вида $f(\theta)
\frac{e^{ikr}}{r}$, где $\theta$ --- угол рассеяния; $f(\theta)$ называется
амплитудой рассеяния. Таким образом, решение уравнения Липпмана--Швингера
имеет асимптотику
\begin{align}
\label{rass_cent_sym}
\psi\sim e^{ikz}+f(\theta)\frac{e^{ikr}}{r}.
\end{align}
Дифференциальное сечение рассеяния в интервал углов равно $d\sigma=2\pi
\sin \theta |f(\theta)|^2\, d\theta$. \par
Матрица рассеяния $S(k)$ в $L_2(S^2)$ имеет полную ортонормированную
систему собственных векторов. В самом деле, оператор $S(k)-I$
является оператором Гильберта--Шмидта [3, т. 3, теорема XI.49]
и нормален\footnote{Ограниченный оператор $A$ называется нормальным, если $AA^+=
A^+A$. Если оператор нормальный и компактный, то он имеет полную
ортонормированную систему собственных векторов.} в силу унитарности $S(k)$.
Так как $V$ центрально-симметричен,
то $\hat S$ коммутирует с любым элементом группы $SO(3)$, а значит, $S(k)$
также коммутирует с каждым элементом $SO(3)$. По лемме Шура, $S(k)$
оставляет каждое подпространство ${\cal H}_l=\left\{\sum\limits_{m=-l}
^l c_mY_{lm}:c_m\in \C\right\}$ инвариантным и $S(k)|_{{\cal H}_l}
=s_l(k)I|_{{\cal H}_l}$, где $s_l(k)$ --- некоторые числа. \par
Назовем величины $f_l(k)=\frac{1}{2ik}(s_l(k)-1)$ парциальными амплитудами
рассеяния. Они являются собственными числами оператора $\hat f(k)$
на подпространствах ${\cal H}_l$. Отсюда выводится равенство
\begin{align}
\label{ampl_rass_centr}
f(\theta)=\sum \limits_{l=0}^\infty (2l+1)f_l(k)P_l(\cos \theta),
\end{align}
где $P_l$ --- полиномы Лежандра [3, теорема XI.51]. \par
Из унитарности $S(k)$ следует, что $|s_l(k)|=1$, так что $s_l(k)=e^{2i
\delta_l(k)}$. Величины $\delta_l(k)$ называются фазовыми сдвигами.
\begin{Trm}
\label{thm1153}
{\rm [3, теорема XI.53]}
Пусть $V$ --- центрально-симметричный кусочно-непрерывный потенциал,
такой что $\int \limits_0^1r|V(r)|\, dr<\infty$ и $\int \limits_1^\infty
|V(r)|\, dr<\infty$. Тогда для любого $k>0$ и $l\in \Z_+$
существует единственное $C^1$-гладкое решение $\psi_{l, \, k}$
уравнения $$-\psi''+\left(V(r)+\frac{l(l+1)}{r^2}\right)\psi=k^2\psi$$
такое, что $\psi_{l,\, k}(r)\rightarrow 0$ и $r^{-l-1}\psi_{l,\, k}(r)
\rightarrow 1$ при $r\rightarrow 0$. Это решение удовлетворяет
соотношению $$\lim \limits_{r\rightarrow \infty}\left[c\psi_{l, \, k}(r)-
\sin\left(kr-\frac{\pi l}{2}+\delta_l(k)\right)\right]=0.$$
\end{Trm}
Пусть $\psi$ --- решение уравнения Липпмана--Швингера с асимптотикой
(\ref{rass_cent_sym}). Так как $V$ центрально-симметричен, то уравнение
не меняется при повороте вокруг оси $z$, а значит, решение не зависит
от угла $\varphi$. Кроме того, $\psi$ --- непрерывная функция.
Поэтому $\psi$ имеет разложение по сферическим гармоникам
\begin{align}
\label{8razl_sph_harm}
\psi=\sum \limits_{l=0}^\infty A_lP_l(\cos \theta)R_{kl}(r),
\end{align}
где $R_{kl}$ --- непрерывные функции, удовлетворяющие уравнению
$$\frac{1}{r^2}\frac{d}{dr}\left(r^2\frac{dR_{kl}}{dr}\right)+
\left[k^2-\frac{l(l+1)}{r^2}-\frac{2\mu}{\hbar^2}V(r)\right]R_{kl}=0.$$
Коэффициенты $A_l$ должны быть выбраны так, чтобы функция (\ref
{8razl_sph_harm}) имела на больших расстояниях вид (\ref{rass_cent_sym}). \par
Разложение плоской волны имеет вид $$e^{ikz}=\sum \limits_{l=0}^\infty
(-i)^l(2l+1)P_l(\cos \theta)\left(\frac{r}{k}\right)^l\left(\frac{1}{r}
\frac{d}{dr}\right)^l\frac{\sin kr}{kr} \underset{r\rightarrow \infty}{\sim}$$
$$\sim \frac{1}{kr}\sum \limits_{l=0}^\infty i^l(2l+1)P_l(\cos \theta)
\sin\left(kr-\frac{l\pi}{2}\right)=\frac{1}{2ikr}\sum \limits_{l=0}^\infty
(2l+1)P_l(\cos \theta)[(-1)^{l+1}e^{-ikr}+e^{ikr}].$$
Согласно теореме \ref{thm1153}, существуют константы $c_l$ такие, что функции
$R_{kl}$ имеют асимптотики
$$R_{kl}\sim \frac{c_l}{r}\sin\left(kr-\frac{\pi l}{2}
+\delta_l\right)=\frac{c_l}{2ir}\left((-i)^le^{i(kr+\delta_l)}-i^le^{-i(kr+\delta_l)}
\right).$$ Отсюда видно, что если положить $A_l=\frac{1}{c_lk}(2l+1)i^l
e^{i\delta_l}$, то $$\psi \sim \frac{1}{2ikr}\sum \limits_{l=0}^\infty
(2l+1)P_l(\cos \theta)\left((-1)^{l+1}e^{-ikr}+e^{2i\delta_l}e^{ikr}
\right)$$ и в разности $\psi-e^{ikz}$ все члены, содержащие $e^{-ikr}$,
выпадают. Вычислив коэффициент при $\frac{e^{ikr}}{r}$, снова получаем
(\ref{ampl_rass_centr}). \par
Проинтегрировав $d\sigma$ по всем углам, получаем полное сечение
рассеяния $\sigma$. Подставляя (\ref{ampl_rass_centr}) в интеграл
$$\sigma=2\pi \int \limits_0^\pi |f(\theta)|^2\sin \theta \, d\theta$$
и учитывая, что система $\{P_l\}$ ортогональна и $$\int \limits _0^\pi
P_l^2(\cos \theta)\sin \theta \, d\theta=\frac{2}{2l+1},$$ получим выражение
для полного сечения рассеяния
\begin{align}
\label{ampl_rass_pl1}
\sigma=\frac{4\pi}{k^2}\sum \limits_{l=0}^\infty (2l+1)\sin^2\delta_l.
\end{align}
Каждое слагаемое представляет собой парциальное сечение $\sigma_l$ для
частиц с заданным орбитальным моментом $l$.
\subsection{Состояния непрерывного спектра в случае кулоновского поля}
Для нахождения радиальных функций при $E>0$ надо в (\ref{kulon_dim3})
изменить знак перед последним слагаемым, полагая $k^2=\frac{2\mu E}
{\hbar^2}$. Комплексная замена переменной $y=-2ikr$ и радиальной
функции $R=N(2kr)^le^{ikr}\Phi(y)$ снова приводит к уравнению
Куммера со значениями параметров
$$\gamma=l+1+\frac{i}{kr_B}, \; \beta=2(l+1).$$ Решением снова является
вырожденный гипергеометрический ряд. Но теперь радиальная функция
остается ограниченной при всех вещественных $k$, то есть условие на
бесконечности не накладывает дополнительных ограничений на параметры.
Таким образом, искомая радиальная функция есть $$R_{kl}=N_{kl}(2kr)
^le^{ikr}\Phi\left(l+1+\frac{i}{kr_B}, \, 2l+2, \, -2ikr\right).$$
Вычислим фазы рассеяния в кулоновском поле, представив асимптотику
радиальной функции при больших $r$ в виде $R_{kl}\sim \frac{2}{r}
\sin(kr-\frac{\pi l}{2}+\delta_l)$. Для этого воспользуемся асимптотической
формулой, справедливой при $|x|\gg \beta$, $|x|\gg \gamma$, $-\beta\notin
\Z_+$: $$\Phi(\gamma, \, \beta, \, x)=\frac{\Gamma(\beta)e^{-i\pi\gamma}}
{\Gamma(\beta-\gamma)}x^{-\gamma}+\frac{\Gamma(\beta)}{\Gamma(\gamma)}e^x
x^{\gamma-\beta}.$$
Отсюда находим $$R(r)\sim \frac{2}{r}\sin\left[kr-\frac{\pi l}{2}-
\frac{1}{kr_B}\ln (2kr)+\arg \Gamma\left(l+1+\frac{i}{kr_B}\right)\right].$$
Переход к свободному движению осуществляется при $r_B\rightarrow \infty$.
При конечных $r_B$ фазы рассеяния оказываются логарифмическими функциями,
то есть $\delta_l\rightarrow \infty$. Следовательно, полное сечение
и амплитуда рассеяния вперед будут бесконечны (см. (\ref{ampl_rass_pl1})).
Более того, оказывается, что в случае кулонова потенциала волновые
операторы $\Omega_\pm$ не существуют. В [3, т. 3, \S XI.9] описана
модификация теории рассеяния для этого случая. Пусть $\hat H_0=\Delta$,
$\hat H=-\Delta-\lambda r^{-1}$. Тогда вместо $\Omega_\pm$
рассматриваются $\Omega_\pm^D=s$-$\lim \limits_{t\rightarrow \mp \infty}
e^{it\hat H}U_D(t)$, где $$U_D(t)=\exp\left(-i\int \limits_0^t \hat H_D(s)
\, ds\right),$$ $$\hat H_D(t)=\hat H_0-\frac{\lambda}{2p|t|}\theta(|4t
\hat H_0|-1),$$ $\theta(\cdot)$ --- функция Хевисайда (выражение для
$\hat H_D$ задается в импульсном представлении). Утверждается, что тогда
$\Omega_\pm^D$ существуют и полны и что $\sigma_{{\rm ac}}(\hat H)=[0, \,
+\infty)$. \par
При $\theta \ne 0$ амплитуда конечна и может быть вычислена следующим
образом. В выражении (\ref{ampl_rass_centr}) при $\theta\ne 0$ можно опустить
единицу в разности $e^{2i\delta_l}-1$, так как имеет место формула
$$\frac14 \sum \limits_{l=0}^\infty (2l+1)P_l(\cos \theta)=\delta(1-
\cos \theta).$$ Если $\delta_l$ заменить на $\delta_l-\delta_0$, то
$f(\theta)$ умножится на фазовый множитель, то есть $|f(\theta)|$ не
изменится. Имеем $$\delta_l-\delta_0=-\frac{1}{kr_B}\ln (2kr)+
\arg \Gamma\left(l+1+\frac{i}{kr_B}\right)+\frac{1}{kr_B}\ln (2kr)
-\arg \Gamma\left(1+\frac{i}{kr_B}\right).$$ Так как второе слагаемое
не зависит от $l$, то его также можно опустить. Следовательно, для некоторого
$\xi\in \R$ выполнено
$$f(\theta)=\frac{1}{2ik}\sum \limits_{l=0}^\infty (2l+1)(e^{2i\delta_l}-1)
P_l(\cos \theta)=$$ $$=\frac{e^{i\xi}}{2ik}\sum \limits_{l=0}^\infty
(2l+1)
\frac{\Gamma\left(l+1+\frac{i}{kr_B}\right)}{\Gamma\left(l+1-\frac{i}{kr_B}
\right)}P_l(\cos
\theta)=$$$$=-\frac{e^{i\xi}}{2k^2r_B\sin^2\frac{\theta}{2}}\exp
\left(-\frac{2i}{kr_B}\ln \sin
\frac{\theta}{2}\right)\frac{\Gamma\left(
1+\frac{i}{kr_B}\right)}{\Gamma\left(1-\frac{i}{kr_B}\right)}$$
(здесь использовались формулы для суммирования полиномов
Лежандра). Соответственно дифференциальное сечение рассеяния равно
$$\frac{d\sigma}{d\Omega}=\frac{1}{4k^4r_B^2\sin^4
\frac{\theta}{2}}=\frac{\alpha^2\mu^2}{4p^4
\sin^4\frac{\theta}{2}}$$ (где $p=\hbar k$ --- асимптотическое
значение импульса). Это выражение совпадает с классической
формулой Резерфорда.
\section{Квантование по Бору--Зоммерфельду}
\subsection{Основные понятия}
Рассмотрим квантовую систему, классическим аналогом которой является
гамильтонова система в фазовом пространстве $\R^{2n}=\R^n_p\oplus \R^n_x$
с гамильтонианом $H(p, \, x)$, $p\in \R^n_p$, $x\in \R^n_x$. Если
$H(p, \, x)=\frac{p^2}{2m}+V(x)$, то $\hat H=-\frac{\hbar^2}{2m}\Delta+
V(x)$; в общем случае $\hat H$ --- это некоторый псевдодифференциальный
оператор (см., напр., [11]). Пусть $\hat H$ на некотором интервале
$(E', \, E'')$ имеет чисто дискретный спектр для любого $\hbar$.
Требуется найти асимптотическое распределение собственных значений
оператора $\hat H$ на этом интервале при $\hbar \rightarrow 0$. \par
Напомним, что если задана гамильтонова система, то в фазовом
пространстве задана кососимметрическая форма, которая в канонических
координатах имеет вид $$\omega=\sum \limits_{j=1}^n dx^j\wedge dp_j.$$
\begin{Def}
Многообразие $\Lambda$ в фазовом пространстве называется лагранжевым,
если оно имеет размерность $n$ и ограничение формы $\omega$ на
любое его касательное пространство равно 0 (то есть
в каждой точке касательное пространство является лагранжевым).
\end{Def}
Многообразие $\Lambda$ является лагранжевым тогда и только тогда,
когда для любой точки $\vec r\in \Lambda$ существует такая ее окрестность
$U\subset \Lambda$, что для любой кривой $\gamma \subset U$,
начинающейся в точке $\vec r$, величина $S=\int \limits_\gamma p\, dx$
зависит только от конечной точки $\gamma$. Кроме того, если $\Lambda$
задается в виде $p_j=f_j(x)$, $j=1, \, \dots, \, n$, то $\Lambda$
является лагранжевым тогда и только тогда, когда $f_j(x)=\frac{\partial
S(x)}{\partial x_j}$ (см. [11, \S 4] и [18, \S 35]).
\begin{Def}
Пусть $\Lambda$ --- лагранжево многообразие. Точка $r\in \Lambda$
называется неособой, если некоторая ее окрестность диффеоморфно
проектируется на $\R^n_x$, и особой в обратном случае.
\end{Def}
Множество всех особых точек обозначим $\Sigma(\Lambda)$. Оказывается,
что сколь угодно малым поворотом многообразия $\Lambda$ можно
добиться того, чтобы множество $\Sigma(\Lambda)$ состояло из
открытого $n-1$-мерного многообразия $\Sigma'(\Lambda)$, на котором
ранг производной проектирования на $\R^n_x$ уменьшается на 1, и границы,
имеющей размерность не больше, чем $n-3$ [11, стр. 148]. \par
В окрестности точки $M\in \Sigma'(\Lambda)$ можно выбрать положительную
и отрицательную стороны $\Sigma(\Lambda)$. Обозначим $\hat k=1, \,
\dots, \, k-1, \, k+1, \, \dots, \, n$. Утверждается, что в окрестности
точки $M$ многообразие $\Lambda$ задается $n$ уравнениями $$x_k=x_k
(p_k, \, x_{\hat k}), \; p_{\hat k}=p_{\hat k}(p_k, \, x_{\hat k}),$$
$\Sigma(\Lambda)$ задается уравнением $\frac{\partial x_k}{\partial p_k}=0$
и при переходе через $\Sigma'(\Lambda)$ величина $\frac{\partial x_k}
{\partial p_k}$ меняет знак. За положительную сторону принимается та,
где $\frac{\partial x_k}{\partial p_k}>0$. \par
\begin{Def}
Пусть ориентированная кривая $\gamma\subset \Lambda$ с неособыми концами
пересекает $\Sigma'(\Lambda)$ в конечном числе точек. Обозначим через
$\nu_+$ число переходов с отрицательной стороны на положительную, а
через $\nu_-$ --- число переходов с положительной стороны на
отрицательную. Индексом Маслова кривой $\gamma$ называется величина
$${\rm ind}\, \gamma=\nu_+-\nu_-.$$
\end{Def}
{\bf Пример.} Рассмотрим кривую в $\R^2$, заданную уравнением
$x=\sin t$, $p=\cos t$, $t\in [0, \, 2\pi]$, и пусть $\Lambda$ совпадает
с ее следом. В левой полуокрестности $t=\pi/2$ выполнено $\frac{\partial
x}{\partial p}<0$, а в правой --- $\frac{\partial x}{\partial p}>0$. Значит,
$t=\pi/2$ является точкой перехода с отрицательной стороны на
положительную. Такой же является точка $t=3\pi/2$. Поэтому $\nu_+=2$,
$\nu_-=0$ и ${\rm ind}\, \gamma=2$.
\subsection{Условно-периодическое движение}
Движение называется условно-периодическим, если существуют канонические
координаты
$$(I, \, \varphi)=(I_1, \; \dots,\; I_n, \; \varphi_1, \; \dots, \;
\varphi_n)\in \R^n\times T^n$$ (называемые переменными действие--угол)
такие, что в них уравнения Гамильтона имеют вид
$$\dot{\varphi}_j=\omega_j, \; \dot{I}_j=0, \; j=1, \, \dots, \, n.$$
Здесь $T^n$ обозначает $n$-мерный тор. \par
Достаточное условие того, что движение является условно-периодическим,
дает теорема Лиувилля.
\begin{Trm}
{\rm [2, гл. 10]} Пусть существуют $n$ интегралов движения $F_1, \;
\dots, \; F_n$ таких, что $\{F_i, \, F_j\}=0$ ($\{\cdot, \cdot\}$ ---
скобка Пуассона). Положим $$\Lambda_\alpha=\{(p, \, x):F_i(p, \, x)=\alpha_i,
\; i=1, \dots, \; n\}.$$ Пусть для почти всех $\alpha$
функции $F_j$ независимы, то есть $dF_j$, $j=1, \, \dots, \, n$, линейно
независимы в каждой точке множества $\Lambda_\alpha$, и что множество $\Lambda_\alpha$
связно и компактно. Тогда
\begin{enumerate}
\item для почти всех $\alpha$ множество $\Lambda_\alpha$ является лагранжевым
многообразием, инвариантным относительно действия фазового потока и
диффеоморфным $T^n$;
\item движение на $\Lambda_\alpha$ является условно-периодическим.
\end{enumerate}
\end{Trm}
Значение переменных действия при фиксированном $\alpha$ задается формулой
\begin{align}
\label{1ik}
I_k(\alpha)=\frac{1}{2\pi}\oint \limits_{\gamma_k}\vec p\, d\vec x,
\end{align}
где $\gamma_k$ --- базисные циклы $T^n$ (точнее, соответствующие им кривые
на $\Lambda_\alpha$). \par
В широком классе задач множества $\Lambda_\alpha$ почти всюду диффеоморфно
проектируются на $x$-пространство и поэтому почти всюду локально задаются
уравнением $\vec p=\bigtriangledown S(\vec x)$. Так как $H$ является
первым интегралом, а функции $F_j$ независимы, то $dH$ является
линейной комбинацией $dF_j$ в каждой точке (иначе бы $\bigtriangledown
H$ и $\bigtriangledown F_j$ порождали подпространство размерности
$n+1$ в фазовом пространстве, на котором бы кососимметрическая форма
$\omega$ обращалась в 0). Значит, на $\Lambda_\alpha$ значение $H$
постоянно. Положив $E=H|_{\Lambda_\alpha}$, получаем, что функция $S$
удовлетворяет уравнению Гамильтона--Якоби $$H(\bigtriangledown S(\vec x),
\, \vec x)=E.$$
В качестве $\gamma_k$ часто можно взять связную компоненту пересечения
$\Lambda_\alpha$ с плоскостью $\{(\vec p, \, \vec x):x^i=x^j_0, \,
j\ne k\}$, где $x^j_0$ --- некоторые константы.
\subsection{Правила квантования Бора--Зоммерфельда}
Пусть $U\subset \R^n$ --- открытое множество и при $\alpha\in U$
выполнены условия теоремы Лиувилля. Тогда $\Lambda_\alpha$ ---
гладкие компактные многообразия без края, не пересекающиеся при различных
$\alpha$, гладко зависящие от $\alpha$ и инвариантные относительно
сдвигов вдоль траекторий данной гамильтоновой системы. Предположим,
что размерность $\Sigma(\Lambda_\alpha)$ не превосходит $n-1$ и
$\Sigma(\Lambda_\alpha)$ гладко зависит от $\alpha$. \par
Пусть $\gamma_k(\alpha)$ --- базисные циклы на $\Lambda_\alpha$.
Утверждается, что тогда $l_k={\rm ind}\, \gamma_k(\alpha)$ не зависят
от $\alpha$. Определим величины $I_k(\alpha)$ по формуле (\ref{1ik}).
Для каждого $m=(m_1, \, \dots, \, m_n)\in \Z^n$ и каждого $\hbar>0$
через $\alpha_m(\hbar)$ обозначим такое $\alpha$, что $$I_k(\alpha)=
2\pi\hbar\left(m_k+\frac14 l_j\right)$$ (если оно существует). \par
Пусть $H=\frac{p^2}{2m}+V(x)$ (при этом на потенциал $V$ накладываются
некоторые условия гладкости и роста: см. [11, стр. 268 и 59]). Рассмотрим
семейство операторов $\hat H(\hbar)=-\frac{\hbar^2}{2m}\Delta+V(x)$,
$\hbar\in (0, \, h_0)$. Пусть для любого $\hbar$ спектр $\hat H(\hbar)$
в интервале $(E', \, E'')$ чисто дискретен. Тогда утверждается, что
при малых $\hbar$ собственные значения оператора $\hat H(\hbar)$ в
интервале $(E', \, E'')$ имеют вид $$E_m(\hbar)=E(\alpha_m(\hbar))+O(\hbar^2),$$
где $E(\alpha)$ --- значение функции Гамильтона на $\Lambda_\alpha$,
а $|O(\hbar^2)|\le C_0\hbar^2$, где $C_0$ не зависит от $\hbar$ и $m$
[11, теорема 13.3]. \par
{\bf Замечание.} Эта теорема в [11] приводилась в несколько большей
общности. Во-первых,  не предполагалось, что движение условно-периодическое.
Рассматривалось семейство лагранжевых многообразий $\Lambda_\alpha$, где
число параметров $\alpha$ равнялось размерности одномерной группы
гомологий $\Lambda_\alpha$; при этом предполагалось, что $\Lambda_\alpha$
инвариантны относительно фазового потока и что значение функции
Гамильтона на них постоянно. Во-вторых, гамильтониан имел более общий
вид и $\hat H$ являлся, вообще говоря, псевдодифференциальным оператором.
\subsection{Схема доказательства правил квантования}
Запишем уравнение Шредингера в виде $$-\lambda^{-2}\Delta \psi(x)+V(x)
\psi(x)=E\psi(x),$$ где $\lambda =\frac{1}{\hbar}$ --- большой параметр, и будем искать его решения
в виде формального ряда
\begin{align}
\label{as_ser_k}
\psi(x, \, \lambda)=e^{i\lambda S(x)}\sum \limits_{j=0}
^\infty (i\lambda)^{-j}\varphi_j(x).
\end{align}
Подставив этот ряд в уравнение и приравняв
к нулю коэффициент при $(i\lambda)^0$, получим, что $S(x)$ является решением
уравнения Гамильтона--Якоби. \par
Пусть $\Lambda =\Lambda _\alpha$ для некоторого $\alpha$, $\Omega \subset \Lambda$
--- односвязная область, диффеоморфно проектирующаяся в $\R^n_x$. Тогда $\Omega$
задается как график отображения $x\mapsto \bigtriangledown S(x)$, где $S$ ---
решение уравнения Гамильтона--Якоби. Пусть $d\sigma$ --- объем на $\Lambda$,
$J(x)=\left|\frac{d\sigma}{dx}\right|$ --- производная меры $d\sigma$ по $dx$.
Тогда можно показать, что в разложении (\ref{as_ser_k}) с точностью до числовой
константы $\varphi_0(x)=\sqrt{J(x)}$. \par
Пусть точка $r\in \Lambda$ является особой. Тогда существует разбиение
$(1, \, \dots, \, n)$ на два непересекающихся множества $(\alpha)=(\alpha_1, \, \dots, \,
\alpha_k)$ и $(\beta)=(\beta_1, \, \dots, \, \beta_l)$ такое, что окрестность
точки $r$ диффеоморфно проектируется в лагранжеву плоскость $(p_{(\alpha)}, \,
x_{(\beta)})$. Введем $\lambda$-преобразование Фурье по части переменных:
$$(F_{\lambda, x_{(\alpha)}\rightarrow p_{(\alpha)}}u(x))(p_{(\alpha)}, \, x_{(\beta)})
=\left(\frac{\lambda}{2\pi i}\right)^{k/2}\int \exp (-i\lambda \langle x_{(\alpha)}, \,
p_{(\alpha)}\rangle)u(x)\, dx_{(\alpha)},$$ где $\sqrt{i}=e^{i\pi /4}$. Обратное
преобразование обозначим $F^{-1}_{\lambda, p_{(\alpha)}\rightarrow x_{(\alpha)}}$. \par
Оператор Шредингера строился по функции Гамильтона $H(x, \, p)$ с помощью
преобразования Фурье. Сделав преобразование Фурье по другим переменным,
можно по функции $H$ построить некоторый псевдодифференциальный оператор,
действующий на функциях от $(p_{(\alpha)}, \, x_{(\beta)})$ (в [11] даны
все строгие определения). По соответствующему псевдодифференциальному уравнению
строится формальное асимптотическое решение в виде ряда по $(i\lambda)^j$. Оказывается,
что коэффициент при $(i\lambda)^0$ (с точностью до константы) имеет вид
\begin{align}
\label{pr_p_ax_b}
F^{-1}_{\lambda, p_{(\alpha)}\rightarrow x_{(\alpha)}}\left(\sqrt{J(p_{(\alpha)}, \,
x_{(\beta)})}e^{i\lambda S(p_{(\alpha)}, \, x_{(\beta)})}\right),
\end{align}
где $J(p_{(\alpha)}, \, x_{(\beta)})$ --- производная меры $d\sigma$ по мере $dp_{(\alpha)}
\, dx_{(\beta)}$. \par
Построим функцию, которая при соответствующем $\lambda$-преобразовании Фурье
с точностью до числового множителя и слагаемого $O(\lambda^{-1})$ равна (\ref{pr_p_ax_b}).
Сначала ее построим локально (с помощью разбиения единицы и предканонического оператора),
а затем осуществим склейку (построив канонический оператор, если это возможно). \par
Канонической картой называется односвязная область $\Omega\subset \Lambda$,
которая диффеоморфно проектируется на одну из лагранжевых координатных плоскостей
$(p_{(\alpha)}, \, x_{(\beta)})$.  Неособая карта --- это карта, в которой все точки
являются неособыми. Координаты $(p_{(\alpha)}, \, x_{(\beta)})$ называются
фокальными координатами. Каноническим атласом называется набор $(\Omega_j)$
канонических карт, являющийся не более чем счетным покрытием $\Lambda$, при этом
каждый компакт покрывается конечным числом карт. Можно показать, что канонический
атлас существует. \par
Пусть $r_0$ --- фиксированная неособая точка на $\Lambda$, $\Omega \subset \Lambda$ ---
каноническая карта с фокальными координатами $(p_{(\alpha)}, \, x_{(\beta)})$,
$l(r_0, \, r)$ --- кривая на $\Lambda$, соединяющая точки $r_0$ и $r$. Введем
предканонический оператор $K\equiv K(\Omega, \, (p_{(\alpha)},\, x_{(\beta)})):
C_0^\infty(\Omega)\rightarrow C^\infty(\R^n_x)$ по формуле
$$(K\varphi)(x)=F^{-1}_{\lambda,\, p_{(\alpha)}\rightarrow x_{(\alpha)}}\left(
\varphi(r)|J(p_{(\alpha)}, \, x_{(\beta)})|^{1/2}\exp\left[i\lambda\left( \int \limits
_{l(r_0, \, r)} p\, dx-\langle x_{(\alpha)}(p_{(\alpha)}, \, x_{(\beta)}),
\,p_{(\alpha)}\rangle \right)\right]\right).$$ Утверждается, что если карта $\Omega$
диффеоморфно проектируется
на лагранжевы плоскости $(p_{(\alpha)}, \, x_{(\beta)})$ и $(p_{(\tilde\alpha)}, \,
x_{(\tilde\beta)})$, то при $\lambda\ge 1$ и $\varphi\in C_0^\infty(\Omega)$
выполнено $$(K(\Omega, (p_{(\alpha)}, \, x_{(\beta)}))\varphi)(x)=e^{\frac{i\pi m}{2}}
(K(\Omega, (p_{(\tilde\alpha)}, \, x_{(\tilde\beta)}))\varphi)(x)+O(\lambda^{-1}),$$
где $m\in \Z$ не зависит от $\varphi$ и $r\in \Omega$. Более того, если $r$ ---
неособая точка, то $m={\rm inerdex}\, \frac{\partial x_{(\alpha)}(r)}{\partial p_{(\alpha)}}
-{\rm inerdex}\, \frac{\partial x_{(\tilde\alpha)}(r)}{\partial p_{(\tilde\alpha)}}+4k$,
где $k\in \Z$, ${\rm inerdex}$ --- индекс инерции матрицы (число отрицательных
собственных значений). Это доказывается с помощью метода стационарной фазы. \par
Пусть $\Omega_i$, $\Omega_j$ --- канонические карты с фокальными координатами
$(p_{(\alpha)}, \, x_{(\beta)})$ и $(p_{(\tilde\alpha)}, \, x_{(\tilde\beta)})$
соответственно, $r\in \Omega_i\cap \Omega_j$ --- неособая точка. Индексом пары
карт $\Omega_i$, $\Omega_j$ называется число $$\gamma(\Omega _i\cap \Omega _j)=
{\rm inerdex}\, \frac{\partial x_{(\alpha)}(r)}{\partial p_{(\alpha)}}
-{\rm inerdex}\, \frac{\partial x_{(\tilde\alpha)}(r)}{\partial p_{(\tilde\alpha)}}.$$
Если имеется цепочка карт $\Omega _{i_0}, \, \dots, \, \Omega _{i_s}$ (то есть $\Omega
_{i_j}\cap \Omega _{i_{j+1}}\ne \emptyset$), то индексом этой цепочки называется
$$\gamma (\Omega _{i_0}, \, \dots, \, \Omega _{i_s})=\gamma(\Omega _{i_0}\cap \Omega
_{i_1})+\dots+\gamma(\Omega _{i_{s-1}}\cap \Omega _{i_s}).$$ \par
Пусть на $\Lambda$ задан канонический атлас $\{\Omega _j\}$, причем точка $r_0$
покрывается неособой картой. Определим оператор, отличающийся
от предканонического числовым множителем: $$(K^{r_0}_\lambda(\Omega _i)\varphi)=
e^{-\frac{i\pi}{2}\gamma(C(r))}(K(\Omega, (p_{(\alpha)}, \, x_{(\beta)}))\varphi),$$
где $C(r)$ --- цепочка карт $\{\Omega _{j_k}\}_{k=0}^s$, покрывающая кривую
$l(r_0, \, r)$, $\gamma(C(r))$ --- ее индекс. Такой оператор не зависит от выбора
фокальных координат, а если $\Omega_{j_0}$ и $\Omega _i$ --- неособые карты, то
$$(K^{r_0}_\lambda(\Omega _i)\varphi)(x)=\sqrt{J(x)}\exp \left( i\lambda\int \limits
_{r_0}^r p\, dx-\frac{i\pi}{2}{\rm ind}\, l(r_0, \, r)\right)\varphi (x, \, p(x)).$$
Пусть $e_j(r)$ --- разбиение единицы на $\Lambda$, подчиненное $\{\Omega _j\}$, то есть
$e_j$ --- гладкие неотрицательные функции, ${\rm supp}\, e_j\subset \Omega _j$ и для
любого $r\in \Lambda$ выполнено $\sum \limits _j e_j(r)=1$. Определим канонический оператор
по формуле $$(K^{r_0}_{\Lambda} \varphi)(x)=\sum \limits _j (K^{r_0}_{\Lambda}(\Omega _j)
(e_j\varphi))(x).$$ Для того, чтобы этот оператор был корректно определен (т.е. чтобы
не было зависимости от выбора пути $l(r_0, \, r)$, канонического атласа и разбиения
единицы), необходимо и достаточно, чтобы для любого замкнутого пути $\gamma\subset
\Lambda$ выполнялось $$\lambda \int \limits _{\gamma}p\, dx-\frac{\pi}{2}{\rm ind}\,
\gamma =2\pi k,$$ где $k\in \Z$. Последнее эквивалентно тому, что для базисных циклов
$\gamma _j$ выполнено $$\int \limits _{\gamma _j} p\, dx=2\pi \hbar \left(k_j+\frac14
{\rm ind}\, \gamma _j\right),$$ $k_j \in \Z$. Таким образом, условие существования
канонического оператора --- это правила квантования. \par
Для каждого $\alpha$ выберем на $\Lambda(\alpha)$ точку $r_0=r_0(\alpha)$.
Пусть $\alpha$ таково, что условия квантования выполнены. Положим $E(\alpha)=
H|_{\Lambda_\alpha}$ и возьмем $\varphi \equiv 1$. Тогда утверждается, что $$\hat H
K^{r_0(\alpha)}_{\Lambda _\alpha}\varphi = E(\alpha)K^{r_0(\alpha)}_{\Lambda _\alpha}\varphi
+O(\hbar ^2)$$ (это доказывается с помощью метода стационарной фазы). При этом
$|O(\hbar ^2)|\le C\hbar ^2$, $\|K^{r_0(\alpha)}_{\Lambda _\alpha}\varphi\|
_{L_2(\R^n)}\ge C_1>0$, где $C$, $C_1$ не зависят от $x$, $\alpha$ и $\hbar$. Отсюда
выводится, что расстояние от $E(\alpha)$ до спектра $\hat H$ не превосходит
$\frac{C}{C_1}\hbar^2$.
\subsection{Квантование орбит в атоме водорода}
Пусть гамильтониан имеет вид
$$H(\vec{p}, \, \vec{r})=\frac{p^2}{2m}-\frac{Ze^2}{r}.$$
Найдем асимптотическое распределение собственных значений соответствующего
оператора Шредингера при $\hbar\rightarrow 0$ в соответствии с правилами
квантования Бора--Зоммерфельда. \par
Перейдем к сферическим координатам: $x=r\cos \varphi\sin \theta$,
$y=r\sin \varphi\sin \theta$, $z=r\cos \theta$. Найдем координаты
$p_r$, $p_\theta$ и $p_\varphi$, канонически сопряженные к $r$, $\theta$
и $\varphi$ соответственно. По определению, $p_i=\frac{\partial L}
{\partial \dot{q^i}}=\frac{\partial T}{\partial \dot{q^i}}$, где $L$ ---
лагранжиан, а
$$T=\frac{m|\dot{\vec{r}}|^2}{2}=\frac{m(\dot{r}^2+r^2(\dot{\theta^2}+
\sin^2\theta\dot{\varphi}^2))}{2}$$
--- кинетическая энергия. Отсюда $p_r=m\dot{r}$, $p_\theta=mr^2\dot{\theta}$,
$p_{\varphi}=mr^2\sin^2\theta\dot{\varphi}$,
$$\frac{p^2}{2m}=T=\frac{m}{2}\left(\left(\frac{p_r}{m}\right)^2+
\left(\frac{p_\theta}{rm}\right)^2+\left(\frac{p_\varphi}{rm\sin\theta}
\right)^2\right),$$
$$H=\frac{1}{2m}\left(p_r^2+\frac{p_\theta^2}{r^2}+\frac{p_\varphi^2}
{r^2\sin^2\theta}\right)-\frac{Ze^2}{r}.$$
Ищем решение уравнения Гамильтона--Якоби в виде $S=S_1(\varphi)+S_2(\theta)+
S_3(r)$. Тогда $$\frac{1}{2m}\left(\left(\frac{\partial S_3}
{\partial r}\right)^2+\frac{1}{r^2}\left(\frac{\partial S_2}{\partial
\theta}\right)^2+\frac{1}{r^2\sin^2\theta}\left(\frac{\partial S_1}{\partial
\varphi}\right)^2\right)-\frac{Ze^2}{r}=E.$$
Отсюда следует, что $\frac{\partial S_1}{\partial \varphi}$ не зависит
от $\varphi$, то есть $p_{\varphi}={\rm const}$. Аналогично $L^2:=p_\theta^2+
\frac{p_\varphi^2}{\sin^2\theta}=C^2=\const$. Можно показать, что
$p_\varphi$ и $L^2$ являются первыми интегралами и что $\{p_\varphi, \,
L^2\}=0$, так что выполнены условия теоремы Лиувилля. \par
Многообразие $\Lambda_\alpha$ задаем равенствами $p_j=\frac{\partial S}
{\partial q^j}$. В качестве $\gamma_k$ выбираем сечения $\Lambda_\alpha$
плоскостями $q^j=\const$, $j\ne k$. Тогда $$\oint \limits_{\gamma_\varphi}
p \, dq=\int \limits _0^{2\pi}p_{\varphi}\, d\varphi=2\pi p_{\varphi}.$$
Кривая $\gamma_\varphi$ диффеоморфно проектируется в координатное
пространство, поэтому ${\rm ind}\, \gamma_\varphi=0$.
Из условий квантования получаем $2\pi p_\varphi=2\pi\hbar n_{\varphi}$,
то есть $p_\varphi=\hbar n_{\varphi}$, где $n_\varphi\in \Z$. Далее,
$$\oint \limits_{\gamma_\theta}p \, dq=2\int \limits _{\theta_C}^{\pi-
\theta_C}\sqrt{C^2-\frac{p^2_{\varphi}}{\sin^2\theta}}\, d\theta,$$ где
$\theta_C=\arcsin \frac{p_\varphi}{C}$. Кривая $\gamma_\theta$
диффеоморфна окружности и пересекает ось $\theta$ в двух точках,
а ее индекс равен 2. Значит, правила квантования для $p_\theta$ имеют
вид $$4\int \limits _{\theta_C}^{\pi/2}\sqrt{C^2-\frac{p^2_\varphi}
{\sin^2\theta}}\, d\theta=2\pi\hbar n_{\theta}+\pi\hbar, \;
n_\theta\in \Z_+$$ (отрицательные $n_\theta$ не годятся, так как
левая часть равенства неотрицательна).
Сделав замену $\xi=\frac{C^2}{p_{\varphi}^2}\sin^2\theta$, получаем
$$p_\varphi\int \limits_1^{\frac{C^2}{p_{\varphi}^2}}\frac
{\sqrt{\xi-1}}{\xi\sqrt{1-\frac{p_{\varphi}^2}{C^2}\xi}}\, d\xi=
\pi\hbar n_\theta+\frac{\pi}{2}\hbar.$$ Вычислив интеграл, имеем в итоге
равенство $C=\hbar\left(n_\theta+|n_\varphi|+\frac12\right)$. \par
Теперь напишем условия квантования для $p_r$. Движение происходит по
эллиптическим орбитам (т. е. финитно) в том случае, если $E<0$.
Пусть
$$H=\frac{1}{2m}\left(p_r^2+\frac{1}{r^2}\left(p_\theta^2+\frac{p_\varphi
^2}{\sin^2\theta}\right)\right)-\frac{Ze^2}{r}=E,$$
$r_{\min}$, $r_{\max}$ --- такие значения $r$, при которых $p_r=0$. Тогда
$$\oint \limits_{\gamma_r} p\, dq=2\int \limits_{r_{\min}}^{r_{\max}}
\sqrt{2m\left(E+\frac{Ze^2}{r}\right)-\frac{\hbar^2\left(|n_\varphi|+
n_\theta+\frac12\right)^2}{r^2}}dr=2\pi\hbar n_r+\pi\hbar,\; n_r\in \Z_+$$
(индекс кривой $\gamma_r$ также равен 2). Сделав замену $u=\frac{1}{r}$ и
проинтегрировав по частям, вычисляем этот интеграл. В итоге получаем
$$E_n=-\frac{Z^2me^4}{2\hbar^2(|n_\varphi|+n_\theta+n_r+1)^2}=-\frac{Z^2me^4}
{2\hbar^2n^2},$$
где $n\in \N$. Значения $E_n$ в точности совпадают с собственными значениями
оператора $\hat H$, что связано с его симметрией. \par
{\bf Замечание.} Несмотря на то, что электрон в центральном поле совершает
плоское движение, при квантовании нужно учитывать все три координаты. В
самом деле, если рассмотреть аналогичную двумерную задачу с координатами
$r$ и $\varphi$, то вместо $|n_\varphi|+n_\theta+n_r+1\in \N$ будет стоять
полуцелое число $|n_\varphi|+n_r+\frac12$.
\section{Теория возмущений}
Найти точное решение задачи о нахождении собственных значений и
собственных векторов удается довольно редко. Поэтому применяются приближенные
методы решения. \par
Пусть $\hat H_\beta=\hat H_0+\beta \hat V$, причем операторы $\hat H_\beta$
и $\hat H_0$ самосопряженные, а спектральная задача для $\hat H_0$ имеет
точное решение. Число $\beta$ предполагается достаточно малым. Оператор
$\beta\hat V$ называется возмущением.
\subsection{Регулярная теория возмущений}
Пусть $E_0$ --- собственное значение оператора $\hat H_0$. При некоторых
условиях на оператор $V$ оказывается, что вблизи $E_0$ существует
собственное значение $E_\beta$ оператора $\hat H_\beta$, которое является
аналитической функцией $\beta$, и соответствующий собственный вектор
также аналитически зависит от $\beta$. Определим дискретный спектр оператора.
\begin{Trm}
{\rm [3, теорема XII.5]}
Пусть $A$ --- замкнутый оператор, и пусть $$\{\mu:|\mu-\lambda|<\varepsilon\}
\cap \sigma(A)=\{\lambda\}.$$ Тогда для любого $r\in (0, \, \varepsilon)$
определен оператор $$P_\lambda=-\frac{1}{2\pi i}\int \limits_{|\mu-\lambda|
=r}(A-\mu)^{-1}\, d\mu,$$ который не зависит от $r$ и является проектором
(то есть $P_\lambda^2=P_\lambda$).
\end{Trm}
\begin{Def}
Точка $\lambda\in \sigma(A)$ называется дискретной, если она изолирована и
оператор $P_\lambda$ конечномерен; если $P_\lambda$ одномерен, то $\lambda$
называется невырожденным собственным значением.
\end{Def}
Утверждается, что если $A$ самосопряжен, то точка $\lambda\in \sigma(A)$
является дискретной в смысле этого определения тогда и только тогда, когда
она изолирована и является собственным значением конечной кратности
(или, что то же самое, спектральный проектор $P_{(\lambda-\varepsilon, \,
\lambda+\varepsilon)}$ конечномерен для некоторого $\varepsilon>0$).
\begin{Trm}
\label{regul_teor_vozm}
{\rm [3, теоремы XII.8 и XII.9]}
Пусть $\hat H_0$ --- самосопряженный оператор, $D(\hat V)\supset D(\hat
H_0)$ и существуют такие $a$, $b>0$, что $\|\hat V\psi\|\le a\|\hat H_0
\psi\|+b\|\psi\|$ для любого $\psi\in D(\hat H_0)$. Пусть $E_0$ ---
невырожденное собственное значение оператора $\hat H_0$. Тогда при
$\beta$, близком к 0, существует единственная точка $E_\beta\in \sigma
(\hat H_\beta)$ вблизи $E_0$ и эта точка изолирована и невырождена.
Функция $E_\beta$ аналитична при $\beta$, близких к 0, и существует
аналитический по $\beta$ собственный вектор $\psi_\beta$, такой что $\|\psi_\beta\|
=1$ при вещественных $\beta$.
\end{Trm}
{\bf Замечание.} В [3] эта теорема доказывается для более
общих семейств операторов $T(\beta)$ (аналитических семейств в смысле Като).
То, что $\|\psi_\beta\|$ можно взять равным 1 при вещественных $\beta$,
выполнено при условии самосопряженности $T(\beta)$ при вещественных $\beta$.
Утверждается, что если выполнены условия теоремы \ref{regul_teor_vozm},
то $\hat H_\beta$ является аналитическим семейством в смысле Като.
Самосопряженность $\hat H_\beta$ следует из теоремы \ref{vust}. \par
Пусть $E_\beta$ и $\psi_\beta$ аналитически зависят от $\beta$. Тогда
$$E_\beta=E^0+\beta E^1+\beta^2E^2+\dots,$$ $$\psi_\beta=\psi^0+\beta
\psi^1+\beta^2\psi^2+\dots .$$ Коэффициенты ряда Тейлора для энергии
называются коэффициентами Релея--Шредингера, а сам ряд называется рядом
Релея--Шредингера. Положив $\beta=0$, получаем $E^0=E_0$,
$\psi^0=\psi_0$. Найдем первую поправку к собственному значению.
Из уравнения Шредингера $\hat H_\beta\psi_\beta=E_\beta\psi_\beta$
получаем $$(\hat H_0+\beta \hat V)(\psi_0+\beta \psi^1+o(\beta))=
(E_0+\beta E^1+o(\beta))(\psi_0+\beta \psi^1+o(\beta)),$$ откуда
\begin{align}
\label{popravka1}
\hat V\psi_0+\hat H_0\psi^1=E^1\psi_0+E_0\psi^1.
\end{align}
Умножив скалярно
на $\psi_0$ и воспользовавшись самосопряженностью $\hat H_0$, получаем
$$\langle \psi_0, \, \hat V\psi_0\rangle+\langle \hat H_0\psi_0, \, \psi^1
\rangle=E^1+E_0\langle \psi_0, \, \psi^1\rangle.$$ Так как
$\hat H_0\psi_0=E_0\psi_0$, то отсюда $$E^1=\langle \psi_0, \, \hat V\psi_0\rangle.$$
Предположим, что $\hat H_0$ имеет чисто дискретный спектр, то есть
существует полная ортонормированная система $\{\psi_m^0\}$ собственных
векторов. Пусть $\psi_0=\psi_n^0$, $$\psi_\beta=\sum \limits_{k=0}^\infty
\psi^k_n\beta^k, \;\;\; E_\beta=\sum \limits_{k=0}^\infty E^k_n\beta^k.$$
Разложим $\psi_n^1$ по базису
$\{\psi_m^0\}$. Так как $\langle \psi_n^0, \, \psi_n^0\rangle=\langle
\psi_\beta, \, \psi_\beta\rangle=1$, то $\langle \psi_n^1, \,
\psi_n^0\rangle=o(1)$ при $\beta\rightarrow 0$ и поэтому равно 0, так
как не зависит от $\beta$. Найдем проекции на остальные базисные векторы.
Умножим скалярно (\ref{popravka1}) на $\psi_m^0$: $$\langle \psi^0_m, \,
\hat H_0\psi^1_n\rangle+\langle \psi_m^0, \, \hat V\psi_n^0\rangle=E^0_n
\langle \psi_m^0, \, \psi_n^1\rangle+E_n^1\langle \psi_m^0, \, \psi_n^0
\rangle,$$ откуда $$E_m^0\langle \psi_m^0, \, \psi_n^1\rangle+\langle
\psi_m^0, \, \hat V\psi_n^0\rangle=E_n^0\langle \psi_m^0, \, \psi_n^1
\rangle,$$ то есть $$\langle \psi_m^0, \, \psi_n^1\rangle=\frac{
\langle \psi_m^0, \, \hat V\psi_n^0\rangle}{E_n^0-E_m^0}.$$ Положив
$V_{mn}=\langle \psi_m^0, \, \hat V\psi_n^0\rangle$, получаем
\begin{align}
\label{vec_popr1}
\psi_n^1=\sum \limits_{m\ne n}\frac{V_{mn}}{E^0_n-E^0_m}
\psi_m^0.
\end{align}
Найдем вторую поправку к энергии. Выписав в уравнении
Шредингера коэффициенты при $\beta^2$, получаем $$\hat H_0\psi_n^2
+\hat V\psi_n^1=E_n^0\psi_n^2+E_n^1\psi_n^1+E_n^2\psi_n^0.$$ Умножим скалярно
на $\psi_n^0$: $$\langle \psi_n^0, \, \hat H_0\psi_n^2\rangle+\langle
\psi_n^0, \, \hat V\psi_n^1\rangle=E_n^0\langle \psi_n^0, \, \psi_n^2
\rangle+E_n^1\langle \psi_n^0, \, \psi_n^1\rangle+E_n^2.$$ Из равенств
$\hat H_0\psi_n^0=E_n^0\psi_n^0$, $\langle \psi_n^0, \, \psi_n^1\rangle=0$
и (\ref{vec_popr1}) получаем
$$E_n^2=\sum \limits_{m\ne n}\frac{|V_{mn}|^2}{E_n^0-E_m^0}.$$ Заметим, что
вторая поправка к энергии основного состояния всегда отрицательна. \par
Теперь рассмотрим случай вырожденных собственных значений.
\begin{Trm}
{\rm [3, теорема XII.13 и задача 17 к гл. XII]}
Пусть семейство операторов $\hat H_\beta$ удовлетворяет условиям теоремы
\ref{regul_teor_vozm} и пусть $E_0$ --- дискретное собственное значение
кратности $m$. Тогда существует $m$ (не обязательно различных) однозначных
функций, аналитических вблизи $\beta=0$: $E^{(1)}(\beta), \; \dots$,
$E^{(m)}(\beta)$ с $E^{(k)}(0)=E_0$ таких, что эти функции являются
собственными значениями $\hat H_\beta$ при $\beta$ вблизи нуля. Более того,
это единственные собственные значения вблизи $E_0$. Существуют аналитические
векторнозначные функции $\psi_1(\beta), \; \dots, \; \psi_m(\beta)$ в
окрестности нуля такие, что $\psi_i(\beta)$ --- собственные векторы и
$\langle\psi_i(\beta), \, \psi_j(\beta)\rangle=\delta_{ij}$ для всех вещественных
$\beta$.
\end{Trm}
Пусть $\{\psi^0_i\}_{i=1}^m$ --- ортонормированная система собственных
векторов оператора $\hat H_0$, соответствующих собственному значению
$E_0$. Тогда векторы $\psi_i(\beta)|_{\beta=0}$ представляются в виде
$$\psi_i(0)=\sum \limits_{j=1}^m C_{ij}\psi^0_j.$$ Умножим (\ref{popravka1})
с $\psi_0=\psi_i(0)$ скалярно на $\psi^0_j$ и получим
\begin{align}
\label{popr_vyr}
\sum \limits_k(V_{jk}-E^1\delta_{jk})C_{ik}=0,
\end{align}
где $V_{jk}=\langle \psi^0_j|\hat V|\psi^0_k\rangle$. Система (\ref{popr_vyr})
имеет нетривиальное решение, если $E^1$ --- собственное значение матрицы
$V_{jk}$. Так как эта матрица эрмитова, то собственные значения вещественны
и их число равно $m$, а собственные векторы можно выбрать ортонормированными.
\subsection{Асимптотическая теория возмущений}
Аппарат регулярной теории возмущений не всегда применим, и ряд
Релея--Шредингера может расходиться при всех $\beta\ne 0$. Формальный ряд
можно интерпретировать как асимптотический.
\begin{Def}
Пусть $f$ --- функция, определенная на положительной вещественной
полуоси. Будем говорить, что формальный ряд $\sum \limits_{n=0}^\infty
a_nz^n$ асимптотический для $f$ при $z\rightarrow 0+$, если для каждого
$N\in \Z_+$ $$\lim \limits_{z\rightarrow 0+}\frac{f(z)-\sum \limits
_{n=0}^N a_nz^n}{z^N}=0.$$
\end{Def}
Функция по асимптотическому ряду восстанавливается неоднозначно.
Однозначность имеет место при наложении более сильных условий на функцию
и на ряд.
\begin{Def}
Будем говорить, что функция $f(z)$, аналитическая в секторе $$\{z:
0<|z|<B, \, |\arg z|<\frac{\pi}{2}+\varepsilon\},$$ удовлетворяет
сильному асимптотическому условию и имеет $\sum a_nz^n$ в качестве сильного
асимптотического ряда, если существуют такие $C$ и $\sigma$, что для всех
$N\in \Z_+$ и всех $z$ в данном секторе $$\left|f(z)-\sum \limits_{n=0}^N
a_nz^n\right|\le C\sigma^{N+1}(N+1)!|z|^{N+1}.$$
\end{Def}
\begin{Trm}
{\rm [3, теорема XII.19]}
Если $\sum a_nz^n$ --- сильный асимптотический ряд для двух аналитических
функций $f$ и $g$, то $f=g$.
\end{Trm}
В [3, \S XII.4] приведены достаточные условия того, что ряд Релея--
Шредингера является асимптотическим для $E_\beta$ (которое является
единственным собственным значением оператора $\hat H_\beta$ вблизи
$E_0$) и того, что он является сильным асимптотическим рядом (у этих
теорем довольно длинные формулировки). Примером оператора, для которого
$E_\beta$ задается сильным асимптотическим рядом, является ангармонический
осциллятор $-\frac{\hbar^2}{2m}\frac{d^2}{dx^2}+\frac{m\omega^2x^2}{2}+
\beta x^4$. \par
Сильные асимптотические ряды суммируются методом Бореля: рассматривается
функция $$g(z)=\sum \limits_{n=0}^\infty \frac{a_n}{n!}z^n$$ и полагается
$E(\beta)=\int \limits_0^\infty g(x\beta)e^{-x}\, dx$. Точнее, имеет
место следующее утверждение.
\begin{Trm}
{\rm [3, теорема XII.21]}
Пусть $\sum a_n\beta^n$ есть сильный асимптотический ряд для $E(\beta)$
в секторе $\{\beta:0<|\beta|<B, \; |\arg \beta|\le \frac{\pi}{2}+
\varepsilon\}$. Тогда $g(z)$ аналитична в некотором круге с центром в 0
и аналитически продолжается в область $\{z:|\arg z|<\varepsilon\}$. Если
$|\beta|<B$ и $|\arg \beta|<\varepsilon$, то $\int \limits_0^\infty
|g(x\beta)|e^{-x}\, dx<\infty$ и $E(\beta)=\int \limits_0^\infty
g(x\beta)e^{-x}\, dx$.
\end{Trm}
\subsection{Концентрация спектра}
Мы рассматривали возмущенные системы с изолированным собственным значением,
расположенным вблизи от изолированного собственного значения невозмущенной
системы. Здесь мы рассматриваем ситуацию, когда есть изолированное
невозмущенное собственное значение и почленно конечный ряд теории возмущений,
но нет возмущенного собственного значения. \par
В качестве примера рассмотрим эффект Штарка. Пусть $\hat H_0=-\frac{\hbar
^2}{2\mu}\Delta-\frac{\alpha}{r}$, $\hat V={\cal E}ez$. Можно доказать,
что при вещественных $\beta$ оператор $\hat H_0+\beta \hat V$ существенно
самосопряжен на $C_0^\infty(\R^3)$ и что $\sigma(\hat H_0+\beta \hat V)
=\R$, то есть спектр является непрерывным. Тем не менее ряд теории
возмущений для энергии основного состояния почленно конечен. \par
Рассмотрим последовательность $$V_n(r)=\left\{\begin{array}{l}
{\cal E}ez, \; |z|<n, \\ {\cal E}en, \; z\ge n, \\ -{\cal E}en, \; z\le
-n.\end{array}\right.$$ Тогда
для оператора $\hat H_0+\beta \hat V_n$ выполнены условия теоремы
\ref{regul_teor_vozm}, и при некоторых $B^{(n)}$ ряд $\sum a_m^{(n)}
\beta^m$ для энергии основного состояния $E^{(n)}(\beta)$ сходится к
$E^{(n)}(\beta)$ при $|\beta|\le B^{(n)}$. Пусть $\sum a_m\beta^m$ ---
формальный ряд теории возмущений для энергии основного состояния $\hat
H_0+\beta \hat V$. Тогда $a_m^{(n)}\underset{n\rightarrow \infty}{\rightarrow}
a_m$. Таким образом, $\sum a_m\beta^m$ --- это формальный предел $\sum a_m^{(n)}
\beta^m$. \par
В реальном физическом эксперименте поле однородно только в
конечной области, и потенциал лучше приближается функцией $V_n$, а не
$V$. Поэтому наблюдаемый спектр выглядит как дискретный. \par
С другой стороны, можно придать смысл ряду теории возмущений в случае
непрерывного спектра как асимптотическому ряду для ``центров сгущения''
спектра. Введем формальные определения. \par
\begin{Def}
\label{def_asimpt}
Пусть $\hat H_n$ --- семейство самосопряженных операторов и пусть $\{P_n
(\Omega)\}$ --- семейство спектральных проекторов $\hat H_n$. Будем
говорить, что часть спектра $\hat H_n$, лежащая в $T\subset \R$,
асимптотически попадает в $S_n\subset \R$, если $$s\text{-}\lim \limits
_{n\rightarrow \infty}P_n(T\backslash S_n)=0.$$
Когда $\hat H_n\rightarrow \hat H$ в смысле сильной резольвентной
сходимости, будем говорить, что часть спектра $\hat H_n$ в $S_n$ есть
асимптотическая часть спектра $\hat H$ в $T$, если
$$s\text{-}\lim \limits_{n\rightarrow \infty}P_n(S_n)=P(T),$$ где
$\{P(\Omega)\}$ --- проекторнозначная мера оператора $\hat H$.
\end{Def}
Можно показать, что если $T=(a, \, b)$, $a$, $b$ не принадлежат точечному
спектру $\hat H$ и $S_n\subset T$, то часть спектра $\hat H_n$ в $T$
асимптотически попадает в $S_n$ тогда и только тогда, когда часть спектра
$\hat H_n$, лежащая в $S_n$, асимптотически является частью спектра $\hat
H$ в $T$. \par
Рассмотрим некоторое состояние $\psi$ и спектральные меры $\langle
\psi, \, P_n(\Omega)\psi\rangle$. Тогда определение \ref{def_asimpt} означает,
что вероятность попадания на множество $T\backslash S_n$ стремится к 0,
а вероятность попадания на $S_n$ --- к $\langle \psi, \, P(T)\psi\rangle$,
то есть спектр ``концентрируется'' на $S_n$.
\begin{Def}
Пусть $\hat H_\beta$ --- семейство самосопряженных операторов, определенных
при малых вещественных $\beta$, $\hat H_\beta\underset{\beta\rightarrow 0}
{\rightarrow} \hat H_0$ в сильном резольвентном смысле. Пусть $E_0$ ---
изолированное невырожденное собственное значение и $\psi_0$ ---
соответствующий нормированный собственный вектор. Семейства векторов
$\psi(\beta)$, $\beta\in \R$, и чисел $E_0+\beta E_1$ называются
соответственно псевдособственным вектором первого порядка и псевдособственным
значением первого порядка, если
\begin{enumerate}
\item $\lim \limits_{\beta\rightarrow 0}\|\psi(\beta)-\psi_0\|=0$,
\item $\lim \limits_{\beta\rightarrow 0} \beta^{-1}\|(\hat H_\beta-
E_0-\beta E_1)\psi(\beta)\|=0$.
\end{enumerate}
\end{Def}
\begin{Trm}
{\rm [3, теорема XII.22]}
Пусть $\hat H_\beta\underset{\beta\rightarrow 0}{\rightarrow} \hat H_0$ в сильном
резольвентном смысле и что $\hat H_\beta$ самосопряжен для любого $\beta$.
Пусть $E_0$ --- изолированное невырожденное собственное значение $\hat
H_0$ и $I$ --- такой интервал, что $\overline I\cap \sigma(\hat H_0)
=\{E_0\}$. Тогда для того, чтобы существовала функция $f(\beta)\in
\underset{\beta\rightarrow 0}{o(\beta)}$ такая, что часть спектра $\hat H_\beta$,
лежащая в $I$, асимптотически попадает в $$I_\beta\stackrel{def}{=}
(E_0+\beta E_1-f(\beta), \; E_0+\beta E_1+f(\beta)),$$ необходимо и достаточно,
чтобы $E_0+\beta E_1$ было псевдособственным значением первого порядка
для $\hat H_\beta$.
\end{Trm}
Следующая теорема дает достаточное условие для концентрации спектра на
$I_\beta$.
\begin{Trm}
{\rm [3, теорема XII.23]}
Пусть самосопряженные операторы $\hat H_\beta$ заданы для каждого $\beta$
из некоторой окрестности $N$ нуля в $\R$. Пусть существует такой
симметрический оператор $\hat V$, что
\begin{enumerate}
\item $\hat H_0$ существенно самосопряжен на $D(\hat H_0)\cap D(\hat V)$;
\item если $\psi\in D(\hat H_0)\cap D(\hat V)$, то при любом $\beta\in N$
имеем $\psi\in D(\hat H_\beta)$ и $\hat H_\beta\psi=\hat H_0\psi+\beta
\hat V\psi$;
\item $E_0$ --- изолированное собственное значение $\hat H_0$ с кратностью
1, и соответствующий собственный вектор принадлежит $D(\hat V)$.
\end{enumerate}
Пусть $I$ --- такой открытый интервал, что $\sigma(\hat H_0)\cap
\overline I=\{E_0\}$, и пусть $E_1=\langle \psi_0, \, \hat V\psi_0\rangle$
--- коэффициент Релея--Шредингера первого порядка для возмущения
собственного значения $E_0$. Тогда существует функция $f(\beta)\in o(\beta)$
при $\beta\rightarrow 0$ такая, что часть спектра $\hat H_0+\beta \hat V$,
лежащая в $I$, асимптотически попадает в $(E_0+\beta E_1-f(\beta), \,
E_0+\beta E_1+f(\beta))$.
\end{Trm}
В качестве примера рассмотрим эффект Штарка. Пусть $\hat H_0=-\frac{\hbar^2}
{2\mu}\Delta-\frac{\alpha}{r}$, $\hat V={\cal E}ez$. Можно доказать, что
выполнены условия предыдущей теоремы. Пусть $|\psi_0\rangle$ ---
основное состояние для $\hat H_0$. Так как $E^1=\langle \psi_0|\hat V|
\psi_0\rangle=0$, то часть спектра $\hat H_0+\beta \hat V$, находящаяся
вблизи от $E_0$, концентрируется в $(E_0-\lambda \beta, \, E_0+\lambda
\beta)$ при $\beta\rightarrow 0$ для любого $\lambda>0$. Можно также
доказать, что все коэффициенты Релея--Шредингера конечны. Более того,
для любого $n$ можно найти функцию $f_n(\beta)$ такую, что $|\beta|^{-n}
f_n(\beta)\underset{\beta\rightarrow 0}{\rightarrow} 0$ и часть спектра $\hat H_0
+\beta\hat V$, лежащая вблизи $E_0$, концентрировалась в $(\sum \limits
_{m=0}^n E_m\beta^m-f_n(\beta), \, \sum \limits_{m=0}^n E_m\beta^m+f_n
(\beta))$ при $\beta\rightarrow 0$. \par
Кроме обобщения на порядок $n$, теорема может применяться к уровням
конечной кратности вырождения. Например, в рассмотренном примере
второе собственное значение четырехкратно вырождено ($l=0$, $m=0$ и
$l=1$, $m=\pm 1$, $0$) и расщепляется в первом порядке на одно двукратное
псевдособственное значение и два однократных псевдособственных значения.
В самом деле, $\hat H_0+\beta \hat V$ симметричен относительно поворотов
вокруг оси $z$, и он оставляет инвариантными подпространства $\{f(r, \,
\theta)e^{im\varphi}:f\in L_2\}$. Вычислив матричные элементы $V_{jk}$,
получаем, что поправка к энергии состояний с $m=\pm 1$ равна нулю. Рассмотрим
состояния с $l=0$, $m=0$ и $l=1$, $m=0$. Нормированные волновые функции
этих состояний равны $$\psi_{200}=\frac{1}{\sqrt{2\pi}}\frac{1}{4r_B^{3/2}}
\left(2-\frac{r}{r_B}\right)e^{-r/2r_B},$$ $$\psi_{210}=\frac{1}{\sqrt{2\pi}}
\frac{1}{4r_B^{3/2}}e^{-r/2r_B}\frac{r}{r_B}\cos \theta.$$ Матрица $V_{jk}$
имеет вид $$V_{jk}=-3e{\cal E}r_B\begin{pmatrix}0 & 1 \\ 1 & 0\end{pmatrix},$$
а ее собственные значения равны $\pm 3e{\cal E}r_B$. Вычислив коэффициенты
$C_{\alpha}$ для каждого из собственных значений, находим, что
суперпозициям $$\psi_{\pm}=\frac{1}{\sqrt{2}}(\psi_{200}\pm \psi_{210})$$
соответствуют поправки к энергии $$E_{\pm}^1=\pm 3e{\cal E}r_B.$$
\subsection{Квантовые переходы}
Применим метод возмущений к нестационарному уравнению Шредингера. Будем
считать, что $\hat H_0$ не зависит от времени, а возмущение $\beta\hat V$
зависит. Сначала предположим, что спектр оператора $\hat H_0$ чисто
дискретный, а решение уравнения
\begin{align}
\label{9shre_t}
i\hbar \frac{\partial \psi}{\partial t}=(\hat H_0+\beta \hat V)\psi
\end{align}
принадлежит ${\cal H}$. Значит, $\psi(t)$ можно представить в виде
\begin{align}
\label{9psi_decomp}
\psi=\sum \limits_n C_n(t)\psi_n^0e^{-\frac{iE^0_n}{\hbar}t},
\end{align}
где $\{\psi^0_n\}$ --- полная ортонормированная система собственных
векторов оператора $\hat H_0$, $E^0_n$ --- собственные значения. Подставляя
(\ref{9psi_decomp}) в
(\ref{9shre_t}), учитывая соотношения $\hat H_0\psi^0_n=E^0_n\psi^0_n$
и проектируя на $\psi^0_k$, получаем для коэффициентов разложения
уравнение
\begin{align}
\label{9ck_eq}
\dot C_k=\frac{1}{i\hbar}\sum \limits_n C_n(t)\beta V_{kn}(t)
e^{-i\omega_{nk}t},
\end{align}
где $V_{kn}(t)=\langle \psi^0_k, \, \hat V(t)\psi^0_n\rangle$, а
\begin{align}
\label{omegank}
\omega_{nk}=\frac{E^0_n-E^0_k}{\hbar}
\end{align}
--- частоты переходов. Будем решать систему (\ref{9ck_eq}), разлагая
$C_k$ в формальный ряд по степеням $\beta$: $$C_k=C^0_k+\beta C^1_k+\dots .$$
Подставляем в (\ref{9ck_eq}):
$$\dot C^0_k+\beta \dot C^1_k+\dots=\frac{1}{i\hbar}\sum \limits_n
(C^0_n+\beta C^1_n+\dots)\beta V_{kn}(t)e^{-i\omega_{nk}t}.$$ Выделяем
слагаемые первого порядка малости по $\beta$ и получаем
$$\dot C^1_k=\frac{1}{i\hbar}\sum \limits_n C_n^0 V_{kn}(t)e^{-i\omega_{nk}
t},$$ откуда $$C^1_k(t)=\frac{1}{i\hbar}\int \limits_0^t \sum \limits_n
C^0_nV_{kn}(t')e^{-i\omega_{nk}t'}\, dt'$$ (предполагается, что при $t=0$
система находится в невозмущенном состоянии $C_k(0)=C^0_k(0)$). Пусть
начальное состояние является одним из стационарных состояний, то есть
$C^0_k=\delta_{kn}$. Тогда
\begin{align}
\label{9ck1}
C^1_k(t)=\frac{1}{i\hbar}\int \limits_0^t V_{kn}(t')e^{-i\omega_{nk}t'}
\, dt'.
\end{align}
Теперь рассмотрим общий случай. Пусть ${\cal H}_+\subset {\cal H}\subset
{\cal H}_-$ --- оснащение гильбертова пространства, $\{\psi_E(\cdot)\}
\subset {\cal H}_-$ --- полная система обобщенных собственных векторов для
оператора $\hat H_0$. Предположим, что для любого $t$ выполнено $\psi(t)
\in {\cal H}_-$ и выполнено равенство $$i\hbar \frac{d}{dt}\langle \psi, \,
\varphi\rangle=\langle \psi, \, (\hat H_0+\beta \hat V(t))\varphi\rangle,$$
где $\varphi$, $(\hat H_0+\beta \hat V)\varphi\in {\cal H}_+$ для малых
$\beta$. Обозначим через $F$ оператор преобразования Фурье, построенный по
спектральному разложению $\hat H_0$, а через $d\lambda(E)$ ---
спектральную меру. Пусть $F(\psi(t))(E)=C_E(t)e^{-\frac
{iEt}{\hbar}}$. Тогда из равенств $$i\hbar\frac{d}{dt}\langle F\psi, \,
F\varphi\rangle=\langle F(\psi), \, F((\hat H_0+\beta \hat V)\varphi)
\rangle$$ и $$F(\hat H_0\varphi)(E)=EF(\varphi)(E)$$ следует, что $$\langle
\dot C_E(t)e^{\frac{-iEt}{\hbar}}, \, F(\varphi)\rangle=\frac{\beta}{i\hbar}
\langle F(\psi), \, F(\hat V\varphi)\rangle.$$ Определим обобщенные
функции $V_{E'E}$ и $\int \limits_{\sigma(\hat H_0)}V_{EE'}F(\psi)(E')\, d\lambda(E')$
по формулам
\begin{align}
\label{vee_}
\int \limits _{\sigma(\hat H_0)\times \sigma (\hat H_0)}
V_{E'E}F(\varphi_1)^*(E')F(\varphi_2)(E)\, d\lambda(E')\, d\lambda(E)=
\langle \varphi_1, \, \hat V\varphi_2\rangle,
\end{align}
$$\left\langle \int \limits _{\sigma(\hat H_0)}V_{EE'}F(\psi)(E')\,
d\lambda(E'), \, F(\varphi)(E)\right\rangle=\langle F(\psi)(E'), \,
\langle V_{E'E}, \, F(\varphi)(E)\rangle \rangle=\langle F(\psi), \,
F(\hat V\varphi)\rangle,$$
где $\varphi_1$, $\varphi_2\in {\cal H}_+$ таковы, что $\langle
\varphi_1, \, \hat V\varphi_2\rangle<\infty$, $\varphi\in {\cal H}_+$,
$\hat V\varphi\in {\cal H}_+$ и $\psi \in {\cal H}_-$. Тогда
\begin{align}
\label{9svertka}
\dot C_E=\frac{\beta}{i\hbar}\int \limits _{\sigma(\hat H_0)}V_{EE'}
C_{E'}(t)e^{-\frac{i(E'-E)t}{\hbar}}\, d\lambda(E').
\end{align}
Частоты переходов равны $\omega_{E'E}=\frac{E'-E}{\hbar}$.
Пусть $\psi=\psi^0+\beta \psi^1+\beta^2\psi^2+\dots$ (это асимптотический
ряд из векторов в ${\cal H}_-$). Тогда $$C_E=C_E^0+\beta C_E^1+\beta^2C_E^2
+\dots$$ (асимптотический ряд в $F({\cal H}_-)$). Подставляем этот ряд в
(\ref{9svertka}), выделяем коэффициенты при $\beta$ и получаем $$C^1_E(t)=
\frac{1}{i\hbar}\int \limits_0^t \int \limits_{\sigma(\hat H_0)}
C^0_{E'}V_{EE'}e^{-i\omega_{E'E}t'}\, d\lambda(E')\, dt'.$$ Если начальным
состоянием является стационарное состояние $\psi^0_{E'}$, то
\begin{align}
\label{9ce1}
C^1_E(t)=\frac{1}{i\hbar}\int \limits_0^t V_{EE'}(t')e^{-i\omega_{E'E}t}
\, dt'.
\end{align}
Соотношения (\ref{9ck1}) и (\ref{9ce1}) показывают, что к моменту $t$
система может находиться в состоянии $\psi^0_k$ (соответственно $\psi^0_E$). Таким
образом, нестационарная теория возмущений приводит к представлению о
квантовых переходах между состояниями невозмущенного спектра под
действием возмущения. \par
Рассмотрим предел выражения (\ref{9ck1}) при больших $t$ в случае
$\hat V(t)=\hat W\cos \omega t$, где $\hat W$ --- некоторый оператор,
не зависящий от времени. Тогда в результате интегрирования по времени в
(\ref{9ck1}) получаем
\begin{align}
\label{9cos}
C^1_k=\frac{W_{kn}}{2\hbar}\left[\frac{e^{i(\omega-\omega_{nk})t}-1}
{\omega_{nk}-\omega}+\frac{e^{-i(\omega+\omega_{nk})t}-1}{\omega_{nk}+
\omega}\right]
\end{align}
где $V_{kn}(t)=W_{kn}\cos \omega t$.
Из (\ref{omegank}) видно, что наиболее существенные переходы будут
происходить в состояниях с энергиями $E^0_k=E^0_n\pm \hbar \omega$
(если эти состояния присутствуют в спектре $\hat H_0$). Если $\omega =\omega _{nk}$,
то при больших $t$ первое слагаемое в (\ref{9cos})
доминирует, и $$C^1_k(t)\sim -\frac{i}{2\hbar}W_{kn}t.$$ \par
{\bf Замечание.} Поправка $C^1_k(t)$ будет малой, только если коэффициенты
$W_{kn}$ малы. \par
Вероятность того, что в момент $t$ система будет находиться в состоянии
$|\psi^0_k\rangle$, можно интерпретировать как вероятность перехода из
$|\psi^0_n\rangle$ в $|\psi^0_k\rangle$. Эта величина равна $|C^1_k|^2$.
Скорость перехода определяется формулой $$w_{n\rightarrow k}\stackrel{def}
{=}\frac{d}{dt}|C^1_k|^2.$$ В точном резонансе $w_{n\rightarrow k}=
\frac{|W_{nk}|^2}{2\hbar^2}t$, то есть скорость перехода линейно растет со
временем. \par
Теперь рассмотрим случай непрерывного спектра. Тогда, если коэффициенты
$C_E(t)$ являются регулярными функциями, можно определить обобщенную
функцию $$\langle w_{E'\rightarrow E}, \, \varphi(E)\rangle=\lim \limits
_{t\rightarrow \infty}\frac{d}{dt}\int \limits _{\sigma(\hat H_0)}
|C^1_E(t)|^2\varphi(E)\, d\lambda(E).$$ Если в окрестности точки $E$
мера $d\lambda(E)$ равна $dE$, то это можно записать в виде
дифференциальной вероятности $$dw_{E'\rightarrow E}=\lim _{t\rightarrow
\infty}\frac{d}{dt}|C^1_E(t)|^2\, dE.$$ Пусть снова $\hat V(t)=\hat W\cos \omega t$,
$V_{EE'}=W_{EE'}\cos \omega t$, $E'=E+\hbar\omega$, $\omega\ne 0$.
В окрестности этой точки $$|C^1_E|^2=\frac{|W_{EE'}|^2}{\hbar^2(\omega_{E'E}
-\omega)^2}\sin^2\frac{(\omega-\omega_{E'E})t}{2},$$ поэтому
$$\frac{d}{dt}|C^1_E|^2=\frac{|W_{EE'}|^2}{2\hbar^2}\frac{\sin (\omega-
\omega_{E'E})t}{\omega-\omega_{E'E}}.$$ Так как в пространстве обобщенных
функций $\frac{1}{\pi}\frac{\sin tx}{x}\underset{t\rightarrow \infty}{\rightarrow}
\delta(x)$, то $$dw_{E'\rightarrow E}=\frac{\pi}{2\hbar^2}|W_{EE'}|^2\delta
\left(\frac{E'-E}{\hbar}-\omega\right)\, dE=\frac{\pi}{2\hbar}|W_{EE'}|^2
\delta(E'-E-\hbar\omega)\, dE.$$ Если $\omega=0$, то надо учитывать оба
слагаемых в выражении для $C^1_E$, поэтому получаем значение, в 4 раза
большее:
\begin{align}
\label{dwee_}
dw_{E'\rightarrow E}=\frac{2\pi}{\hbar}|W_{EE'}|^2\delta(E'-E)\, dE.
\end{align}
\subsection{Рассеяние в борновском приближении}
Пусть $\psi_{\vec k}$ --- решение уравнения Липпмана--Швингера
(\ref{lippman_r3}). С помощью итераций построим формальные ряды для
$\psi_{\vec k}$ и подставим их в (\ref{t_matrix}). Получаем
разложение $T$-матрицы в формальный ряд:
$$T(\vec k', \, \vec k)=\sum \limits_{n=0}^\infty T_n(\vec k', \, \vec k),$$
где $$T_0(\vec k', \, \vec k)=\frac{2\mu}{(2\pi)^3\hbar^2}\int
e^{i(\vec k- \vec k')\vec r}V(\vec r)\, d^3r,\qquad T_n(\vec k', \,
\vec k)= $$ $$=\frac{(-1)^n2\mu^{n+1}}{(2\pi)^{n+3}
\hbar^{2n+2}}\int e^{-i\vec k'\vec x_0}V(\vec x_0)\frac{e^{ik|\vec
x_0-\vec x_1|}}{|\vec x_0- \vec x_1|}V(\vec x_1)\dots V(\vec
x_{n-1})\frac{e^{i\vec k|\vec x_{n-1}-\vec x_n|}}{|\vec
x_{n-1}-\vec x_n|}V(\vec x_n)e^{ikx_n}\, d^3\vec x_0\dots d^3\vec x_n.$$
Этот ряд называется {\it рядом Борна}. Выразив дифференциальное
сечение рассеяния через амплитуду рассеяния и вместо $T$ подставив
$T_0$, получаем сечение рассеяния в борновском приближении
$$d\sigma=|f(\vec k, \, \vec k')|^2\, d\Omega\approx\left(\frac{\mu}{2\pi
\hbar^2}\right)^2\left|\int e^{i(\vec k-\vec k')\vec r}V(\vec r)\,
d^3\vec r\right|^2\, d\Omega.$$
Пусть начальное состояние имеет вид плоской волны, отвечающей импульсу
$\vec p$ падающей частицы $$\psi _{\vec p}=\frac{1}{(2\pi \hbar)^{3/2}}
e^{i\vec p\vec r/\hbar},$$ а конечное --- такой же вид, но для импульса
$\vec p'$ $$\psi_{\vec p'}=\frac{1}{(2\pi \hbar)^{3/2}}e^{i\vec p'\vec r/
\hbar}.$$ Потенциал считаем не зависящем от $t$, то есть в формулах
предыдущего раздела полагаем $\omega=0$. Согласно (\ref{vee_}),
$$\int \psi^*_1(\vec r)V(\vec r)\psi_2(\vec r)\, d^3\vec r=
\int V_{\vec p'\vec p}\hat \psi^*_1(\vec p')\hat \psi_2(\vec p)\, d^3
\vec p'\, d^3\vec p,$$ где $$\psi_j(\vec r)=\frac{1}{(2\pi\hbar)^{3/2}}
\int e^{i\vec p\vec r}\hat \psi_j(\vec p)\, d^3\vec p.$$ Если функция
$V$ достаточно быстро убывает на бесконечности, то можно воспользоваться
теоремой Фубини и получить, что $$V_{\vec p'\vec p}=\frac{1}{(2\pi
\hbar)^3}\int e^{i(\vec p-\vec p')\vec r/\hbar}V(\vec r)\, d^3\vec r.$$
Согласно (\ref{dwee_}),
\begin{align}
\label{9wpp}
dw_{\vec p\rightarrow \vec p'}=\frac{2\pi}{\hbar}|V_{\vec p'\vec p}|^2
\delta\left(\frac{p^2}{2\mu}-\frac{p'^2}{2\mu}\right)d^3\vec p'.
\end{align}
\begin{Sta}
Дифференциальное сечение рассеяния равно отношению скорости
перехода \\ (проинтегрированной по $p'$) к $|\vec j|$, где $$\vec j=\frac{\hbar}
{2\mu i}(\psi^*_{\vec p}\bigtriangledown \psi_{\vec p}-\psi_{\vec p}
\bigtriangledown \psi^*_{\vec p})=\frac{\vec p}{\mu}\frac{1}{(2\pi
\hbar)^3}.$$
\end{Sta}
\begin{proof}
Положим $\vec q=\frac{\vec p-\vec p'}{\hbar}$,
$$V_{\vec q}=(2\pi \hbar)^3V_{\vec p'\vec p}=\int e^{i\vec q\vec r}
V(\vec r)\, d^3\vec r.$$ Перейдем к сферическим координатам в импульсном
пространстве и заменим $d^3\vec p'$ на
$$p'^2\, dp'\, d\Omega'=\frac12 p'\, d(p')^2\, d\Omega'.$$
Проинтегрируем (\ref{9wpp}) по $d(p')^2$:
$$\int \limits_0^\infty \frac{2\pi}{\hbar}|V_{\vec p'\vec p}|^2\delta
\left(\frac{p^2}{2\mu}-\frac{p'^2}{2\mu}\right)\frac12 p'\, d(p')^2=
\frac{2\pi}{\hbar}|V_{\vec p'\vec p}|^22\mu \frac12 p=\frac{2\pi}{\hbar}
|V_{\vec p'\vec p}|^2\mu p,$$ откуда $$\frac{\frac{2\pi}{\hbar}|V_{\vec
p'\vec p}|^2\mu p}{\frac{p}{\mu(2\pi\hbar)^3}}\, d\Omega'=|V_{\vec p'
\vec p}|^2\mu^2(2\pi\hbar)^3\frac{2\pi}{\hbar}\, d\Omega'=$$ $$=\mu^2
\frac{1}{(2\pi \hbar)^3}\frac{2\pi}{\hbar}|V_{\vec q}|^2\, d\Omega'=
\left(\frac{\mu}{2\pi\hbar^2}\right)^2|V_{\vec q}|^2\, d\Omega'=d\sigma.$$
\end{proof}
Если потенциал сферически симметричен, то можно разложить $e^{i\vec q
\vec r}$ по шаровым функциям, выполнить интегрирование по углам и
получить для фаз рассеяния выражение в борновском приближении
$$\delta_l=-\frac{\pi \mu}{\hbar^2}\int \limits_0^\infty V(r)J_{l+1/2}
(kr)r\, dr,$$ где $k=p/\hbar$, $\delta_l\ll 1$.
\section{Квантовая статистика}
\subsection{Операторы со следом}
Дадим два эквивалентных определения оператора со следом:
\begin{enumerate}
\item Оператор $A$ называется оператором со следом, если
$A=\sum \limits_{j=1}^N B_jC_j$, где $B_j$ и $C_j$ --- операторы
Гильберта--Шмидта [12, Добавление 1, \S 4].
\item Оператор $A$ называется оператором со следом, если для любого
ортонормированного базиса $(e_n)$ величина $\sum \limits_n \langle e_n, \,
\sqrt{A^*A}e_n\rangle$ конечна [3, \S VI.6].
\end{enumerate}
{\it Следом} оператора $A$ называется величина
$${\rm Tr}\, A=\sum \limits_n \langle e_n, \, Ae_n\rangle,$$ где $(e_n)$ ---
ортонормированный базис. Эта величина не зависит от выбора $(e_n)$.
\begin{Sta}
{\rm [12]}
Пусть $A$ --- оператор со следом, $B$ --- ограниченный оператор. Тогда
$AB$ и $BA$ являются операторами со следом и ${\rm Tr}\, (AB)={\rm Tr}\,
(BA)$.
\end{Sta}
Пусть $$\hat K\varphi(x)=\int \limits_{\R^n}K(x, \, y)\varphi(y)\, dy.$$
Следующая теорема [12] дает необходимое условие того, что
$\hat K$ является оператором со следом и формулу для следа.
\begin{Trm}
\label{trace_int_oper}
Пусть $\hat K$ --- оператор со следом в $L_2(\R^n)$ с ядром $K(x, \, y)$.
Тогда
\begin{enumerate}
\item $K(x, \, x+\xi)\in L_1(\R^n_x)$ для почти всех $\xi\in \R^n$;
\item $K(x, \, x+\xi)$ совпадает почти всюду с непрерывной функцией
по $\xi$ со значениями в $L_1(\R^n_x)$;
\item ${\rm Tr}\, \hat K=\int K(x, \, x)\, dx$, где $K(x, \, x)$ ---
значение этой непрерывной функции при $\xi=0$.
\end{enumerate}
\end{Trm}
\subsection{Матрица плотности}
Предположим, что рассматривается физическая система, которая является частью
некоторой полной квантовой системы, описываемой вектором состояния
$\psi(x, \, y)$, где $x$ --- множество координат выделенной части.
Системы $x$ и $y$ взаимодействуют между собой. Будем описывать
систему с координатами $x$. Пусть наблюдаемая $\hat F$ действует на
функции вида $\psi(x)$. Возможно, перейдя к другому представлению, можно
считать, что $\hat F$ --- это оператор умножения на функцию, то есть
$(\hat F\psi)(x)=F(x)\psi(x)$, где $x$ принадлежит некоторому пространству
с мерой $(X, \, \mu)$. Для функций $\psi(x, \, y)$ полагаем $(\hat F\psi)
(x, \, y)=F(x)\psi(x, \, y)$. Математическое ожидание этой наблюдаемой
равно $$\langle F\rangle =\int \psi^*(x, \, y)F(x)\psi(x, \, y)\, dx\,
dy.$$ Положим $$\rho(x, \, x')=\int \psi^*(x', \, y)\psi(x, \, y)\, dy.$$
Тогда $$\langle F\rangle=\int F(x)\rho(x, \, x)\, dx.$$
Таким образом, незамкнутая система описывается матрицей плотности $\rho
(x, \, x')$. Приведем некоторые ее свойства. Пусть $$\hat \rho:\varphi(x)
\mapsto \int \rho(x, \, x')\varphi(x')\, dx'.$$ Тогда
\begin{enumerate}
\item оператор $\hat \rho$ неотрицательный, то есть $\langle \varphi, \,
\hat \rho \varphi\rangle\ge 0$ для всех $\varphi\in {\cal H}$. В самом деле,
$$\langle \varphi, \, \hat\rho \varphi\rangle=\int \varphi^*(x)\int \rho (x, \, x')
\varphi(x')\, dx'\, dx=$$$$=\int \int \int
\varphi^*(x)\varphi(x')\psi^*(x', \, y)\psi(x, \, y)\, dy\, dx'\,
dx=$$ $$=\int dy \left(\int \varphi^*(x) \psi(x, \, y)\,
dx\right)\left(\int \varphi(x')\psi^*(x', \, y)\, dx' \right)=\int
dy \, f(y)f^*(y)\ge 0.$$ \item Оператор $\hat\rho$ эрмитов, то
есть $\langle \varphi_1, \, \hat \rho \varphi_2\rangle=\langle
\hat \rho\varphi_1, \, \varphi_2\rangle$. Это следует из того, что
$\rho(x', \, x)=\rho(x, \, x')^*$. \item ${\rm Tr} \, \hat\rho=1$.
Действительно, для любого $f\in L_2(X)$ выполнено $$\int
|f(x)|^2\, dx=\sum \limits_n\left|\int \varphi_n(x)f(x)\,
dx\right|^2,$$ где $\{\varphi_n\}$ --- ортонормированный базис.
Значит,
$$\sum \limits _n \langle \varphi_n, \, \hat\rho\varphi_n\rangle=\sum \limits
_n \int \int \int \varphi_n^*(x)\varphi_n(x')\psi(x, \, y)\psi^*(x', \, y)
\, dx\, dx'\, dy=$$ $$=\sum \limits_n\int dy\, \left|\int \varphi_n(x)
\psi^*(x, \, y)\, dx\right|^2=\int dy\, \sum \limits_n \left|\int \varphi_n(x)
\psi^*(x, \, y)\, dx\right|^2=$$ $$=\int dy\, \int dx\, |\psi^*(x, \, y)|^2=
\int |\psi(x, \, y)|^2\, dx\, dy=1.$$ Заметим, что ${\rm Tr}\,
\hat \rho=\int \rho(x, \, x)\, dx$.
\end{enumerate}
Теперь рассмотрим оператор $\hat F\hat \rho$. Он имеет вид
$$(\hat F\hat \rho\varphi)(x)=F(x)\int \rho(x, \, x')\varphi(x')\, dx',$$
то есть это интегральный оператор с ядром $F(x)\rho(x, \, x')$. Если
он имеет след, то, согласно теореме \ref{trace_int_oper},
$${\rm Tr}\, (\hat F\hat \rho)=\int F(x)\rho(x, \, x)\, dx=\langle
F\rangle.$$ \par
Пусть системы $x$ и $y$ не взаимодействуют, то есть гамильтонианы,
которые их описывают, сильно коммутируют. Тогда можно выбрать собственные
состояния так, что $\psi(x, \, y)=\psi_1(x)\psi_2(y)$, $\|\psi_1\|=1$,
$\|\psi_2\|=1$. В этом случае матрица плотности имеет вид
\begin{align}
\label{matr_pure}
\rho(x, \, x')=\int \psi_1^*(x')\psi_2^*(y)\psi_1(x)\psi_2(y)\, dy=
\psi_1^*(x')\psi_1(x).
\end{align}
Диагональные элементы матрицы плотности --- это плотности вероятности
обнаружить систему имеющей координату $x$. Среднее значение $\langle F
\rangle$ при этом имеет стандартный вид $$\langle F\rangle=\int \psi_1^*
(x)F(x)\psi_1(x)\, dx.$$ Пусть матрица плотности имеет вид (\ref{matr_pure}).
Тогда $$(\hat\rho\varphi)(x)=\int \psi_1^*(x')\psi_1(x)\varphi(x')\, dx'=
\langle \psi_1, \, \varphi\rangle\psi_1(x),$$ то есть $\hat\rho$ ---
это оператор ортогонального проектирования на подпространство, натянутое на
вектор $\psi_1$.
\begin{Def}
Пусть $\rho(x, \, x')$ --- матрица плотности, соответствующая некоторому
состоянию. Состояние называется чистым, если существует такой вектор
$|\psi\rangle $, что оператор $\rho$ является оператором ортогонального
проектирования на $|\psi\rangle$. При этом $\psi$ --- та волновая функция,
которая описывает систему. Если состояние не является чистым, то оно
называется смешанным.
\end{Def}
Покажем, что чистые состояния являются крайними точками на множестве всех
состояний, то есть что для любого $\lambda\in (0, \, 1)$ и для любых
двух различных состояний $\rho_1$ и $\rho_2$ состояние $\lambda
\rho_1+(1-\lambda)\rho_2$ не является чистым. В самом деле, состояние
$\rho$ является чистым тогда и только тогда, когда $\rho^2=\rho$.
Пусть $$\left(\lambda\rho_1+(1-\lambda)\rho_2\right)^2=\lambda\rho_1
+(1-\lambda)\rho_2.$$ Тогда $$\lambda \rho_1(1-\rho_1)+(1-\lambda)
\rho_2(1-\rho_2)+\lambda(1-\lambda)(\rho_1-\rho_2)^2=0.$$ Все
слагаемые являются неотрицательно определенными (это доказывается
с помощью перехода к собственному базису), причем третье
не равно нулю, если $\lambda\in (0, \, 1)$
и $\rho_1\ne \rho_2$, так что сумма не может быть равной 0. \par
Уравнение эволюции для матрицы плотности имеет вид
\begin{align}
\label{evol_plotn}
\frac{\partial \hat \rho}{\partial t}=\frac{i}{\hbar}[\hat \rho, \, \hat H].
\end{align}
Это уравнение можно получить двумя способами.
\begin{enumerate}
\item Рассматривается эволюция чистого состояния, при этом соответствующий
вектор изменяется со временем согласно уравнению Шредингера. Отсюда
получается уравнение (\ref{evol_plotn}). Для смешанных состояний
это уравнение постулируется.
\item Если рассмотреть картину Гейзенберга, то состояние не изменяется,
а наблюдаемая эволюционирует по закону $\hat A(t)=U^+(t)\hat AU(t)$, где
$U(t)=e^{\frac{-it\hat H}{\hbar}}$. В представлении Шредингера наблюдаемая не изменяется,
а состояние изменяется. Если потребовать, чтобы среднее значение любой
наблюдаемой было одним и тем же в обоих представлениях, то получаем
$${\rm Tr}\, (\hat \rho(t)\hat A)={\rm Tr}\, (\hat \rho\hat A(t))=
{\rm Tr}\, (\hat \rho U^+(t)\hat AU(t))=
{\rm Tr}\, (U(t)\hat \rho U^+(t)\hat A).$$ В силу произвольности $\hat A$,
получаем $\hat \rho(t)=U(t)\hat \rho U^+(t)$. Значит, $\hat \rho(t)$
удовлетворяет уравнению (\ref{evol_plotn}).
\end{enumerate}
Заметим, что при эволюции свойства 1)--3) матрицы плотности сохраняются,
то есть $\hat \rho(t)$ снова задает состояние. \par
В стационарном состоянии $[\hat \rho, \, \hat H]=0$. Отсюда следует,
что если спектр оператора $\hat H$ чисто дискретный, то в энергетическом
представлении матрица $\hat \rho$ имеет блочно-диагональный вид, где
каждому блоку соответствует собственное значение оператора $\hat H$.
С помощью ортогональной замены базиса каждый блок можно привести к
диагональному виду, при этом новые базисные векторы снова будут
собственными векторами $\hat H$. Таким образом, можно считать,
что $$\rho_{mn}=\delta_{mn}w_n.$$ Величины $w_n$ можно интерпретировать как
вероятность нахождения незамкнутой системы в состоянии $\psi_n$,
являющемся собственным вектором $\hat H$, соответствующем
$n$-му собственному значению. Тогда
$$\langle F\rangle=Tr(\hat F\hat \rho)=\sum \limits_nw_n\langle \psi_n,
\, F\psi_n\rangle,$$ то есть статистическое и квантовомеханическое усреднение
в данном случае разделяются между собой.
\subsection{Равновесные состояния для эргодических систем,\\
близких к неэргодическим} Приведем схему вывода распределения Гиббса для
равновесных состояний, изложенного в [8].
\begin{Def}
Квантовая система, определенная гамильтонианом $\hat H$ и имеющая
коммутирующие между собой первые интегралы $\hat K_1, \; \dots, \; \hat
K_n$, называется эргодической, если каждая сохраняющаяся величина
является функцией $\hat H$ и $\hat K_1, \; \dots, \; \hat K_n$, то есть
совместный спектр $\hat H$ и $\hat K_1, \; \dots, \; \hat K_n$ является
простым.
\end{Def}
Рассмотрим эргодическую систему $L^N(\varepsilon)$, состоящую из $N$ одинаковых слабо
взаимодействующих подсистем $L_i$, являющихся копиями системы $L$.
Пространство состояний системы $L$ обозначим ${\cal H}$, а ее гамильтониан
через $\hat H$. Пространство состояний
${\cal H}^N$ системы $L^N(\varepsilon)$ является $N$-й тензорной степенью
${\cal H}$, а гамильтониан имеет вид $$\hat H^{(N)}(\varepsilon)=\sum
\limits_{k=1}^N \hat H_i+\hat V_\varepsilon,$$ где $\hat H_k$ ---
гамильтониан $L_k$, $\hat V_\varepsilon$ --- взаимодействие. Считаем, что
$V_\varepsilon\rightarrow 0$ при $\varepsilon\rightarrow 0$ и предельная
система $L^N$ состоит из независимых подсистем (ее гамильтониан обозначим
$\hat H^{(N)}$). \par
Пусть $\hat A$ --- наблюдаемая, отнесенная к системе $L$, $\hat A_i$ ---
ее копия, отнесенная к $L_i$. Рассмотрим $$\hat A^{(N)}=\frac{1}{N}
\sum \limits_{i=1}^N \hat A_i.$$ Положим
\begin{align}
\label{vrem_srednee}
\overline A^{(N)}(\varepsilon)=\lim \limits_{T\rightarrow \infty}
\frac{1}{T}\int \limits_0^T e^{\frac{it\hat H^{(N)}(\varepsilon)}{\hbar}}
\hat A^{(N)}e^{-\frac{it\hat H^{(N)}(\varepsilon)}{\hbar}}\, dt.
\end{align}
Пусть система состоит из $N$ одинаковых слабо взаимодействующих подсистем.
Смешанное состояние $\hat \rho$ подсистемы $L_i$ называется {\it состоянием
равновесия}, если $[\hat \rho, \, \hat H_i]=0$ и для любой наблюдаемой
$\hat A$ выполнено равенство $$\left\langle \hat A_i\right\rangle_{\psi_E}=
\left\langle\overline A^{(N)}(\varepsilon)\right\rangle_{\psi_E},$$
где $\psi_E$ --- собственный вектор оператора $\hat H^{(N)}$ с собственным
значением $E$, а $\langle A\rangle_{\psi}$ обозначает среднее наблюдаемой
$A$ в состоянии $\psi$. \par
Пусть $\hat H^{(N)}(\varepsilon)$ имеет простой дискретный спектр $\{\lambda_k\}$
для любого $\varepsilon>0$. Рассмотрим матричный элемент $\overline{a}
_{kl}$ оператора $\overline A^{(N)}(\varepsilon)$ в собственном
базисе $\hat H^{(N)}(\varepsilon)$:
$$\overline a_{kl}=\lim \limits_{T\rightarrow \infty}\frac{1}{T}\int
\limits_0^T a_{kl}e^{\frac{it}{\hbar}(\lambda_k-\lambda_l)}\, dt=
a_{kl}\delta_{\lambda_k\lambda_l}.$$ Отсюда следует, что $\overline
A^{(N)}(\varepsilon)$ коммутирует с $\hat H^{(N)}(\varepsilon)$.
В силу эргодичности, $\overline A^{(N)}(\varepsilon)=f_\varepsilon
(\hat H^{(N)}(\varepsilon))$. Пусть существует предел $f(x)=\lim
\limits_{\varepsilon\rightarrow 0}f_\varepsilon(x)$. Положим $\overline
A^{(N)}=\lim \limits_{\varepsilon\rightarrow 0}\overline A^{(N)}
(\varepsilon)=f(\hat H^{(N)})$.
Оператор $\overline A^{(N)}$ коммутирует с $\hat H^{(N)}$ и кратен единичному
в собственных подпространствах $\hat H^{(N)}$. \par
Пусть $\psi_E$ --- собственный вектор $\hat H^{(N)}$ с собственным
значением $E$. Тогда
\begin{align}
\label{vrem_srednee_proj}
\left\langle \psi_E, \, \overline A^{(N)}\psi_E\right\rangle=\lim \limits
_{\varepsilon\rightarrow 0}\left\langle \psi_E, \, \overline A^{(N)}
(\varepsilon)\psi_E\right\rangle=\left\langle \psi_E, \, f(\hat H^{(N)})
\psi_E\right\rangle=f(E)=\frac{\Tr(\overline A^{(N)} P_E^{(N)})}
{\Tr P_E^{(N)}},
\end{align}
где $P_E^{(N)}$ --- проектор на собственное подпространство $\hat H^{(N)}$
с собственным значением $E$. Покажем, что $\Tr(\overline A^{(N)} P_E^{(N)})=
\Tr(\hat A^{(N)} P_E^{(N)})$. В самом деле, умножим обе части
(\ref{vrem_srednee}) на $P_E^{(N)}$ и возьмем след:
$$\Tr (\overline A^{(N)}(\varepsilon)P_E^{(N)})=\lim \limits_{T\rightarrow
\infty}\frac{1}{T}\int \limits_0^T \Tr (P_E^{(N)}e^{\frac{it\hat H^{(N)}
(\varepsilon)}{\hbar}}\hat A^{(N)}e^{-\frac{it\hat H^{(N)}(\varepsilon)}{\hbar}})
\, dt=$$ $$=\lim\limits _{T\rightarrow \infty}\frac{1}{T}\int \limits_0^T
\Tr (e^{-\frac{it\hat H^{(N)}(\varepsilon)}{\hbar}}P_E^{(N)} e^{\frac{it\hat
H^{(N)}(\varepsilon)}{\hbar}}\hat A^{(N)})\, dt.$$ Так как
$\lim \limits_{\varepsilon \rightarrow 0}\hat H^{(N)}(\varepsilon)=
\hat H^{(N)}$, то $$\lim \limits_{\varepsilon\rightarrow 0}e^{-\frac{it\hat
H^{(N)}(\varepsilon)}{\hbar}}P_E^{(N)} e^{\frac{it\hat H^{(N)}(\varepsilon)}
{\hbar}}=P_E^{(N)}.$$ Значит, $$\Tr\, (\overline A^{(N)}
P_E^{(N)})=\Tr\, (\hat A^{(N)}P_E^{(N)}).$$ Отсюда и из
(\ref{vrem_srednee_proj}) получаем
\begin{align}
\label{microcan_srednee}
\left\langle \psi_E, \, \overline A^{(N)}\psi_E\right\rangle=
\frac{\Tr\, (\hat A^{(N)}P_E^{(N)})}{\Tr\, P_E^{(N)}}.
\end{align}
Так как $L^N(\varepsilon)$ близка к $L^N$ и эргодична, то
\begin{align}
\label{limit_vrem_sredn}
\left\langle\overline A^{(N)}(\varepsilon)\right\rangle_{\psi_E}\approx
\lim _{\varepsilon \rightarrow 0}\left\langle\overline A^{(N)}(\varepsilon)
\right\rangle_{\psi_E}=\left\langle\overline A^{(N)}\right\rangle
_{\psi_E}\stackrel{(\ref{microcan_srednee})}{=}\frac{\Tr (\hat A^{(N)}
P_E^{(N)})}{\Tr P_E^{(N)}}.
\end{align}
Пусть $\psi_s$ --- собственные векторы $\hat H$, $E_s$ ---
соответствующие собственные значения. Утверждается, что
(\ref{limit_vrem_sredn}) имеет предел при $N\rightarrow \infty$ и
$E/N={\cal E}=\const$, равный
\begin{align}
\label{limit_vrem_sredn1}
\sum \limits_s ae^{-\beta E_s}\alpha_s,
\end{align}
где $\alpha_s=\langle \psi_s, \, \hat A\psi_s\rangle$, а $a$ и $\beta$
находятся из соотношений
$$a\sum \limits_s e^{-\beta E_s}=1, \; \;\; \frac{\sum \limits_s E_s e^{-\beta
E_s}}{\sum \limits _s e^{-\beta E_s}}={\cal E}$$ (доказательство довольно
сложное). Таким образом, мы доказали, что в собственном базисе оператора
$\hat H$, в котором матрица $\hat \rho$ диагональна, выполнено
равенство $$\langle \hat A_i\rangle=\sum \limits_s ae^{-\beta E_s}\alpha_s,$$
то есть диагональные элементы $\hat \rho$ равны $ae^{-\beta E_s}$.
Значение (\ref{limit_vrem_sredn1}) является средним $\alpha_s$
по распределению вероятностей $w_s=ae^{-\beta E_s}$. Это
распределение называется {\it распределением Гиббса}. \par
Формула (\ref{limit_vrem_sredn1}) может быть записана в виде
\begin{align}
\label{limit_vrem_sredn2}
\frac{\Tr (\hat A e^{-\beta \hat H})}{{\rm Tr}\, e^{-\beta \hat H}}.
\end{align}
Если $\hat H$ имеет коммутирующие первые интегралы $\hat K_1, \; \dots, \;
\hat K_n$, то (\ref{limit_vrem_sredn2}) заменяется на $$\frac{\Tr (\hat A
e^{-\beta (\hat H+\mu_1 \hat K_1+\dots+\mu_n \hat K_n)})}{\Tr e^{-\beta
(\hat H+\mu_1\hat K_1+\dots+\mu_n\hat K_n)}}.$$
\par
При выводе распределения Гиббса предполагалось, что матрица плотности
незамкнутой системы зависит только от энергии. В частности, число
частиц в ней предполагалось фиксированным. Вообще говоря, может происходить
обмен частиц с термостатом. В этом случае полная система состоит из
подсистем $L_i$, являющимися копиями системы $L$, которой соответствует
гильбертово пространство ${\cal H}=\oplus _{n=0}^\infty {\cal H}_n$,
оператор числа частиц $\hat N$, такой что $\hat N|_{{\cal H}_n}=nI$,
и оператор Гамильтона $\hat H$, сильно коммутирующий с $\hat N$. Тогда,
если полная система эргодична и в ней происходит слабое взаимодействие
между подсистемами, то в состоянии равновесия матрица плотности,
записанная в базисе общих собственных векторов $\hat H$ и $\hat N$,
имеет вид $$w_{nN}=ae^{-\alpha N-\beta E_{nN}}.$$ Такое распределение
называется {\it большим каноническим распределением.}
\subsection{Принцип максимума энтропии}
Пусть матрица плотности имеет диагональный вид $(w_n)$ в энергетическом
представлении.
\begin{Def}
Энтропией называется величина $$S=-\langle \ln w_n\rangle=-\sum \limits_n
w_n\ln w_n.$$
\end{Def}
Пусть задано среднее значение энергии $E$ (называемое внутренней энергией).
Найдем распределение $w_n$, при котором энтропия максимальна, в предположении,
что спектр оператора $\hat H$ ограничен снизу и кратность каждого собственного
значения конечна. Для этого нужно решить следующую экстремальную задачу (см.
\S \ref{sect_conv_prog}):
\begin{align}
\label{max_entropy}
\left\{ \begin{array}{l} \sum \limits_n w_n\ln w_n\rightarrow \inf, \\
\sum \limits_n w_n=1, \\ \sum \limits_n E_nw_n=E, \\ w_n\ge 0.\end{array}
\right.
\end{align}
Докажем, что множество ${\cal W}$ последовательностей $(w_n)$,
$0\le w_n\le 1$, для которых все ряды в (\ref{max_entropy})
сходятся, выпукло. В самом деле, пусть $(w_n)\in {\cal W}$ и
$(\tilde w_n)\in {\cal W}$, $t\in [0, \, 1]$. Достаточно показать,
что ряд $$\sum \limits_n (tw_n+(1-t)\tilde w_n)\ln
(tw_n+(1-t)\tilde w_n)$$ сходится. Пусть $w_n <\tilde w_n$. Тогда
из выпуклости функции $w\ln w$ и неравенства $\ln \tilde w_n\le 0$
следует, что $$w_n-\tilde w_n\le (w_n-\tilde w_n)\ln \tilde
w_n+(w_n-\tilde w_n)\le (tw_n+(1-t)\tilde w_n)\ln
(tw_n+(1-t)\tilde w_n)\le 0.$$
Значит, из критерия Коши следует сходимость подсуммы
ряда по тем $n$, для которых $w_n<\tilde w_n$. Аналогично
рассматривается случай $w_n\ge \tilde w_n$. \par Таким образом,
функционалы в задаче (\ref{max_entropy}) определены на выпуклом
множестве и сами являются выпуклыми, так что это задача выпуклого
программирования. По теореме Куна--Таккера, если $(w_n^0)$ ---
решение (\ref{max_entropy}), то на
этом наборе выполнено условие минимума функции $${\cal
L}=\lambda_0\sum \limits_n w_n\ln w_n+\lambda_1 \sum \limits_n
w_n+\lambda_2 \sum \limits_n E_nw_n$$ на множестве
последовательностей $(w_n)\in {\cal W}$, где $(\lambda_0,
\,\lambda_1, \, \lambda_2)\ne (0, \, 0, \, 0)$. Пусть $w_n^0>0$
для любого $n\in \N$. Тогда для любого $n$ выполнено
$$\lambda_0(1+\ln w_n^0)+\lambda_1+ \lambda_2E_n=0.$$ Если
$\lambda_0=0$, то $\lambda_1=\lambda_2=0$ (так как существуют
различные $E_k$), так что этот случай не годится. Значит, $\ln
w_n^0=\alpha-\beta E_n$, то есть мы получаем распределение Гиббса
$$w_n^0=e^{\alpha-\beta E_n}.$$ Пусть последовательность $E_n$
такова, что $w_n^0 \in {\cal W}$ (для этого достаточно, чтобы
$E_n\ge cn^{\gamma}$, где $\gamma>0$). Докажем, что $(w_n^0)$
задает минимум функции ${\cal L}$. В самом деле, если $(w_n)\in {\cal W}$,
то в силу выпуклости функции $f_c(w)=\lambda _0w\ln w+cw$ выполнено
$$\lambda_0w_n\ln w_n +\lambda_1w_n+\lambda_2E_nw_n\ge \lambda_0w_n^0\ln
w_n^0+\lambda_1w_n^0+\lambda_2E_nw_n^0$$ для любого $n\in \N$.
Просуммировав по $n$, получаем искомое неравенство. Так как
$\lambda_0\ne 0$, то $(w_n^0)$ является точкой минимума в
(\ref{max_entropy}). \par
Единственность решения (\ref{max_entropy}) следует из того, что
для любых двух различных последовательностей $(w_n)$, $(\tilde w_n)
\in {\cal W}$ и для любого $t\in (0, \, 1)$ выполнено строгое
неравенство $$\sum \limits _n (tw_n+(1-t)\tilde w_n)\ln (tw_n+(1-t)\tilde
w_n)<t \sum \limits_n w_n\ln w_n+(1-t)\sum \limits_n \tilde w_n\ln \tilde
w_n.$$ \par
Таким образом, равновесному состоянию
соответствует такое распределение, при котором энтропия
максимальна. \par Теперь предположим, что число частиц в системе
не фиксировано. Найдем распределение, при котором энтропия
максимальна при заданном значении внутренней энергии и среднего
числа частиц. Соответствующая экстремальная задача имеет вид
$$\left\{ \begin{array}{l} \sum \limits_{n, \, N}w_{nN}\ln w_{nN}
\rightarrow \inf, \\ \sum \limits _{n, \, N}E_{nN}w_{nN}=\langle E\rangle,
\\ \sum \limits_{n, \, N}Nw_{nN}=\langle N\rangle, \\ \sum \limits_{n,
\, N}w_{nN}=1, \\ w_{nN}\ge 0,\end{array}\right.$$
Решая ее так же, как для распределения Гиббса, получаем, что
$w_{nN}=e^{\alpha-\gamma N-\beta E_{nN}}$, то есть максимуму энтропии
соответствует большое каноническое распределение.
\section{Дополнение 1: Основные понятия из функционального анализа}
\subsection{Общая топология}
\begin{Def}
Топологическим пространством называется пара $(X, \, \tau)$, где
$X$ --- произвольное множество, $\tau$ --- система подмножеств в
$X$ со следующими свойствами:
\begin{enumerate}
\item $X\in \tau$, $\emptyset\in \tau$;
\item если $U_\alpha\in \tau$ для любого $\alpha\in A$, то
$\cup _{\alpha\in A}U_\alpha\in A$;
\item если $U_k\in \tau$ для любого $k=1, \, \dots, \, n$, то
$\cap _{k=1}^n U_k\in \tau$.
\end{enumerate}
\end{Def}
Множества, принадлежащие $\tau$, называются открытыми. Множество
$F\subset X$ называется замкнутым, если $X\backslash F$ открыто. \par
{\bf Примеры.}
\begin{enumerate}
\item $X=\R^n$, $\tau$ --- система всех открытых множеств $U$ в $\R^n$
(то есть таких, что окрестность любой точки $x\in U$ содержится в
$U$);
\item $X$ --- произвольное множество, $\tau=\{\emptyset, \, X\}$;
\item $X$ --- произвольное множество, $\tau$ --- система всех
подмножеств в $X$.
\end{enumerate}
Пусть $\tau$ и $\tau'$ --- две топологии на $X$. Скажем, что топология
$\tau$ сильнее, чем $\tau'$, если любое множество $U\in \tau'$
принадлежит $\tau$. \par
Система $\beta$ подмножеств множества $X$ называется базой топологии
$\tau$, если $\tau$ --- это совокупность множеств вида
$U=\cup_\alpha U_\alpha$, где $U_\alpha\in \beta$. Например,
система открытых шаров в $\R^n$ задает базу его топологии. \par
Пусть $(X, \, \tau)$ --- топологическое пространство, $Y\subset X$.
Точка $x\in X$ называется точкой прикосновения для множества $Y$,
если для любого открытого множества $U\ni x$ множество $U\cap Y$
непусто (в частности, если $x\in Y$, то $x$ является точкой прикосновения).
Замыканием множества $Y$ называется объединение всех точек прикосновения
множества $Y$. Множество $Y$ называется всюду плотным в $X$, если
его замыкание совпадает с $X$. Пространство $X$ называется
сепарабельным, если в нем существует счетное всюду плотное подмножество.
Например, $\R^n$ сепарабельно, так как множество точек с рациональными
координатами счетно и всюду плотно. \par
Пусть $(X, \, \tau)$ и $(X', \, \tau')$ --- два топологических
пространства. Отображение $F:X\rightarrow X'$ называется непрерывным,
если прообраз любого открытого в $X'$ множества является открытым
множеством в $X$. \par
Система множеств $\{U_\alpha\}_{\alpha\in A}$ называется покрытием
множества $Y$, если для любого $x\in Y$ найдется такое $\alpha\in A$,
что $x\in U_\alpha$. Подмножество $Y$ топологического пространства
$X$ называется компактным, если из любого покрытия множества $Y$
открытыми множествами можно выбрать конечное подпокрытие. Например,
отрезок в $\R$ компактен в силу леммы Гейне--Бореля. Любое замкнутое
и ограниченное подмножество в $\R^n$ также является компактным. \par
\begin{Def}
Метрическим пространством называется пара $(X, \, \rho)$, где
$X$ --- произвольное множество, а $\rho:X\times X\rightarrow \R_+$
обладает следующими свойствами:
\begin{enumerate}
\item $\rho(x, \, y)=0$ тогда и только тогда, когда $x=y$;
\item $\rho(x, \, y)=\rho(y, \, x)$;
\item $\rho(x, \, z)\le \rho(x, \, y)+\rho(y, \, z)$ (неравенство
треугольника).
\end{enumerate}
\end{Def}
Открытым шаром в метрическом пространстве $(X, \, \rho)$ называется множество
$$\{x\in X:\rho(x, \, x_0)<r\}$$ для некоторых $x_0\in X$, $r_0>0$.
Система открытых шаров задает базу топологии в $X$, которая называется
топологией, порожденной метрикой $\rho$. \par
Подмножество $Y$ в метрическом пространстве компактно тогда и только
тогда, когда из любой последовательности $(y_n)\subset Y$ можно
выбрать сходящуюся подпоследовательность. Если $Y$ компактно, то
оно замкнуто и ограничено. Обратное неверно. Рассмотрим произвольное
бесконечное множество и введем на нем метрику $$\rho(x, \, y)=
\left\{ \begin{array}{l} 0, \; \text{если} \; x=y, \\ 1, \;
\text{если} \; x\ne y.\end{array}\right.$$ Тогда $X$ является замкнутым
ограниченным подмножеством в $(X, \, \rho)$, но не является компактным. \par
Пусть $(X, \, \tau)$ --- топологическое пространство. Скажем, что
последовательность $(x_n)$ сходится к $x\in X$, если для любого
открытого множества $U\ni x$ найдется такое $N\in \N$, что для любого
$n\ge N$ выполнено $x_n\in U$. В метрическом пространстве предел
единствен. \par
Последовательность $(x_n)$ называется фундаментальной (или
последовательностью Коши), если для любого $\varepsilon>0$
найдется $N\in \N$ такое, что для любых $n$, $m\ge N$ выполнено
$\rho(x_n, \, x_m)<\varepsilon$. Метрическое пространство называется
полным, если любая фундаментальная последовательность в нем имеет
предел. Например, $\R^n$ является полным метрическим пространством.
\par
Если $(X, \, \rho)$ --- полное метрическое пространство, $Y\subset X$,
то $(Y, \, \rho)$ полно тогда и только тогда, когда $Y$ замкнуто
в $X$. \par
Метрическое пространство $(\tilde X, \, \tilde \rho)$ называется
пополнением $(X, \, \rho)$, если существует инъективное отображение
$i:X\rightarrow \tilde X$ такое, что $\tilde \rho(i(x), \, i(y))=
\rho(x, \, y)$ для любых $x$, $y\in X$ и $i(X)$ плотно в $\tilde X$.
Известно, что любое метрическое пространство имеет пополнение.
\subsection{Теория меры и интеграла}
Пусть $X$ --- произвольное множество, $\Sigma$ --- некоторая система
его подмножеств, обладающая следующими свойствами:
\begin{enumerate}
\item если $A$, $B\in \Sigma$, то $A\backslash B\in \Sigma$;
\item если $\{A_n\}_n$ --- не более чем счетная система подмножеств
из $\Sigma$, то $\cup_n A_n\in \Sigma$.
\end{enumerate}
Тогда $\Sigma$ называется $\sigma$-алгеброй, элементы $\Sigma$
называются измеримыми множествами, а $(X, \, \sigma)$
называется измеримым пространством. Из свойств теоретико-множественных
операций следует, что симметрическая разность $A\bigtriangleup B=
(A\backslash B)\cup (B\backslash A)$ двух измеримых множеств
$A$ и $B$ является измеримым множеством и что пересечение не более
чем счетного числа измеримых множеств является измеримым множеством. \par
Пусть $S$ --- некоторая система подмножеств множества $X$.
Тогда $\sigma$-алгеброй, порожденной системой $S$, называется
наименьшая $\sigma$-алгебра, содержащая $S$. Например, $\sigma$-алгебра,
порожденная системой всех открытых подмножеств в $\R^n$, называется
$\sigma$-алгеброй борелевских множеств. Она же порождается
системой всех замкнутых множеств и системой всех параллелепипедов
$\{(x_1, \, \dots, \, x_n):x_k\in I_k, \; k=1, \, \dots, \, n\}$,
где $I_k\subset \R$ --- промежутки. \par
Пусть $(X, \, \Sigma)$ --- измеримое пространство. Тогда
($\sigma$-аддитивной) мерой на $(X, \, \Sigma)$ называется функция
$\mu:\Sigma\rightarrow [0, \, +\infty]$ такая, что для любого не более
чем счетного числа непересекающихся множеств $A_n\in \Sigma$ выполнено
$\mu(\sqcup_n A_n)=\sum_n \mu(A_n)$ (символ $\sqcup$ обозначает
объединение непересекающихся множеств). Мера называется конечной,
если $\mu(A)<+\infty$ для любого $A\in \Sigma$; мера называется
$\sigma$-конечной, если существует разбиение $X$ на счетное число
непересекающихся множеств $X_n\in \Sigma$ таких, что $\mu (A\cap X_n)
<+\infty$ для любого $A\in \Sigma$ и $n\in \N$. Тройка $(X, \,
\Sigma, \, \mu)$ называется пространством с мерой. Если
$X\in \Sigma$ и $\mu(X)=1$, то мера $\mu$ называется вероятностной. \par
Пусть $\Pi=I_1\times \dots\times I_n$ --- параллелепипед в $\R^n$,
$I_k$ --- промежутки с концами $a_k$, $b_k$. Мера Лебега
параллелепипеда $\Pi$ задается равенством $\mu(\Pi)=\prod_{k=1}^n
|b_k-a_k|$. Пусть $A$ --- ограниченное подмножество в $\R^n$.
Верхней мерой множества $A$ называется величина $$\mu^*(A)=\inf
_{\{\Pi_m\}}\sum \limits_m \mu(\Pi_m),$$ где нижняя грань берется
по всем не более чем счетным покрытиям $\{\Pi_m\}$ множества $A$
параллелепипедами. Множество $A$ называется измеримым по Лебегу,
если для любого $\varepsilon>0$ найдется конечное число
параллелепипедов $\Pi_j$, $j=1, \, \dots, \, k$, таких что
$\mu^*\left(A\bigtriangleup \left(\cup_{j=1}^k \Pi_j\right)\right)
<\varepsilon$. В этом случае мерой Лебега множества $A$ называется
его верхняя мера. Если множество $A\subset \R^n$ неограниченно,
то оно называется измеримым по Лебегу тогда и только тогда, когда
оно представимо в виде счетного числа ограниченных измеримых по
Лебегу множеств $A_m$; его мера равна сумме мер $A_m$ (можно
показать, что определение корректно, то есть не зависит от выбора
разбиения). Утверждается, что измеримые по Лебегу множества
образуют $\sigma$-алгебру, а мера Лебега $\sigma$-аддитивна и
$\sigma$-конечна и является продолжением меры $\mu$, заданной
на параллелепипедах. В частности, каждое борелевское множество
измеримо по Лебегу. \par
Пусть $(X, \, \Sigma)$ --- измеримое пространство. Функция $f:
X\rightarrow \R$ называется измеримой, если для любого $c\in \R$
множество $\{x\in X:f(x)<c\}$ измеримо. Это эквивалентно тому, что
прообраз любого борелевского множества измерим. Измеримые
функции образуют линейное пространство. Если $X=\R^n$, а $\Sigma$
--- это борелевские множества, то измеримые функции называются
борелевскими. В частности, непрерывные функции на $\R^n$ являются
борелевскими. \par
Функция $f:X\rightarrow \R$ называется простой, если она измерима
и принимает не более чем счетное число значений. \par
Пусть $(X, \, \Sigma, \, \mu)$ --- пространство с конечной
$\sigma$-аддитивной мерой. Простая функция $f$, $f|_{X_n}\equiv y_n$,
называется интегрируемой по Лебегу, если ряд $\sum _n y_n\mu(X_n)$
абсолютно сходится. Интегралом Лебега $\int _X f(x)\, d\mu(x)$ от
функции $f$ называется сумма этого ряда. Функция $g:X\rightarrow \R$
называется интегрируемой по Лебегу, если существует последовательность
простых функций $g_n$, равномерно сходящихся к $g$. Интегралом Лебега
от функции $g$ называется величина $$\int \limits_X f(x)\, d\mu(x)=
\lim \limits_{n\rightarrow \infty} g_n(x)\, d\mu(x).$$ Всякая
интегрируемая по Лебегу функция измерима; всякая измеримая ограниченная
функция интегрируема по Лебегу. Если $f$ интегрируема, то $|f|$
также интегрируема. \par
Если мера $\sigma$-конечна, то рассматривается разбиение
$X$ на подмножества $X_n$, на которых мера конечна. Функция $f$
называется интегрируемой, если $f|_{X_n}$ интегрируема и ряд
$\sum _n \int_{X_n}|f|\, d\mu$ сходится; интегралом функции $f$
называется $\sum_n \int _{X_n}f\, d\mu$. \par
Аналогично определяется интеграл Лебега от комплекснозначной
функции. \par
Перечислим свойства интеграла Лебега:
\begin{enumerate}
\item если $f(x)\ge 0$ и $f$ интегрируема, то $\int \limits_X
f(x)\, d\mu(x)\ge 0$;
\item $\left|\int _X f(x)\, d\mu(x)\right|\le \int _X |f(x)|\, d\mu(x)$;
\item если функции $f$ и $g$ интегрируемы и $\alpha$, $\beta\in \R$,
то $\alpha f+\beta g$ интегрируема и $\int (\alpha f+\beta g)\, d\mu
=\alpha \int f\, d\mu+\beta \int g\, d\mu$;
\item если $X=\sqcup_n X_n$, то $f$ интегрируема на $X$ тогда и
только тогда, когда $f|_{X_n}$ интегрируема на $X_n$; при этом
$\int _X f\, d\mu=\sum \limits_n \int _{X_n}f\, d\mu$.
\end{enumerate}
Пусть $(X_1, \, \Sigma_1, \, \mu_1)$ и $(X_2, \, \Sigma_2, \,
\mu_2)$ --- пространства с мерой. Рассмотрим тройку $(X, \,
S, \, \mu)$, где $X=X_1\times X_2$, $S=\{A\in X: A=A_1\times A_2\}$,
$\mu(A_1\times A_2)=\mu_1(A_1)\mu_2(A_2)$. Утверждается, что
$\mu$ можно продолжить до $\sigma$-аддитивной меры на некоторой
$\sigma$-алгебре $\Sigma$, содержащей $S$ (конструкция аналогична построению
меры Лебега в $\R^n$). Эта мера называется произведением мер
$\mu_1$ и $\mu_2$ и обозначается $\mu_1\otimes \mu_2$. \par
Скажем, что некоторое свойство выполнено почти всюду,
если множество, на котором оно не выполнено, имеет меру нуль.
\begin{Trm} (Фубини). Пусть множество $A\subset X_1\times X_2$ измеримо
относительно $\Sigma$. Тогда для почти всех $x_1\in X_1$ множество
$\{x_2\in X_2:(x_1, \, x_2)\in A\}$ принадлежит $\Sigma_2$. Если
$f:X_1\times X_2\rightarrow \R$ интегрируема, то для почти всех
$x_1\in X_1$ функция $f(x_1, \, \cdot)$ интегрируема на $X_2$ и
$$\int \limits_{X_1\times X_2}f(x)\, d(\mu_1\otimes \mu_2)(x_1, \, x_2)=
\int \limits_{X_1}\left(\int \limits
_{X_2}f(x_1, \, x_2)\, d\mu_2(x_2)\right)\, d\mu_1(x_1).$$ Если
функция $f$ измерима и $$\int \limits_{X_1}\left(\int \limits
_{X_2}|f(x_1, \, x_2)|\, d\mu_2(x_2)\right)\, d\mu_1(x_1)<\infty,$$
то $f$ интегрируема на $X_1\times X_2$.
\end{Trm}
Пусть $\mu _1$, $\mu _2$ --- две меры на $(X, \, \Sigma)$. Скажем,
что $\mu_2$ абсолютно непрерывна относительно $\mu_1$, если для
любого множества $A\in \Sigma$ такого, что $\mu_1(A)=0$, выполнено
$\mu_2(A)=0$.
\begin{Trm} (Радон--Никодим). Мера $\mu_2$ абсолютно непрерывна
относительно $\mu_1$ тогда и только тогда, когда существует
неотрицательная интегрируемая функция $g$ такая, что $\mu_2(A)=
\int \limits _A g(x)\, d\mu_1(x)$ для любого $A\in \Sigma$.
\end{Trm}
Пусть $\mu$ --- произвольная $\sigma$-аддитивная мера на борелевских
подмножествах в $\R$. Эта мера называется мерой Лебега--Стильтьеса,
а интеграл по ней --- интегралом Лебега--Стильтьеса. Мера называется
точечной, если существует не более чем счетное число точек $\{x_n\}$
таких, что для любого борелевского множества $A$ выполнено
$\mu(A)=\sum _{x_n\in A}\mu(\{x_n\})$. Мера $\mu$ называется абсолютно
непрерывной, если она абсолютно непрерывна относительно меры Лебега
на $\R$. Мера $\mu$ называется сингулярной, если существует множество
$A$ меры нуль Лебега такое, что $\mu(A)\ne 0$ и $\mu(\R\backslash
A)=0$.
\begin{Trm}
\label{leb_st_sum3}
Любая мера Лебега--Стильтьеса однозначно представляется в виде
суммы точечной, абсолютно непрерывной и сингулярной мер.
\end{Trm}
Функция $f$ называется абсолютно непрерывной на отрезке $[a, \, b]$,
если для любого $\varepsilon>0$ найдется такое $\delta>0$, что
для любых непересекающихся отрезков $[a_k, \, b_k]\subset [a, \, b]$,
$k=1, \, \dots, \, n$, таких, что $\sum \limits_{k=1}^n|b_k-a_n|
<\delta$, выполнено $\sum \limits_{k=1}^n |f(b_k)-f(a_k)|<\varepsilon$.
Функция $f$ абсолютно непрерывна тогда и только тогда, когда
$f(x)=f(a)+\int \limits_{[a, \, x]}g(t)\, dt$, где $g$ ---
интегрируемая функция. При этом $f$ дифференцируема почти всюду и
$f'(x)=g(x)$.
\subsection{Линейные нормированные пространства}
\begin{Def}
Пусть $V$ --- линейное пространство над $\R$ или $\C$. Полунормой
на $V$ называется функция $\|\cdot\|:V\rightarrow \R_+$ со
следующими свойствами:
\begin{enumerate}
\item $\|\alpha x\|=|\alpha|\cdot\|x\|$ для любого вектора $x$ и числа $\alpha$;
\item $\|x+y\|\le \|x\|+\|y\|$.
\end{enumerate}
Если $\|x\|\ne 0$ для любого $x\ne 0$, то $\|\cdot\|$ называется
нормой. В этом случае пара $(V, \, \|\cdot\|)$ называется линейным нормированным
пространством.
\end{Def}
Норма задает метрику в $V$ равенством $\rho(x, \, y)=\|x-y\|$.
Если $V$ полно относительно этой метрики, то оно называется
банаховым пространством. \par
\begin{Def}
Пусть $V$ --- линейное пространство над $\R$ или $\C$. Скалярным
произведением называется отображение $\langle\cdot, \, \cdot\rangle:
V\rightarrow \R$ ($\C$) со следующими свойствами:
\begin{enumerate}
\item $\langle x, \, \alpha y+\beta z\rangle=\alpha \langle x, \, y\rangle+
\beta\langle x, \, z\rangle$ для любых $x$, $y$, $z\in V$,
$\alpha$, $\beta\in \R$ ($\C$);
\item $\langle x, \, y\rangle=\langle y, \, x\rangle^*$ ($^*$ обозначает
комплексное сопряжение);
\item $\langle x, \, x\rangle >0$ для любого $x\ne 0$.
\end{enumerate}
Пара $(V, \, \langle\cdot, \, \cdot\rangle)$ называется евклидовым
пространством.
\end{Def}
Векторы $x$ и $y$ называются ортогональными, если $\langle x, \, y\rangle
=0$. \par
Обозначим $\|x\|=\sqrt{\langle x, \, x\rangle}$. Тогда выполнены
неравенство Коши--Буняковского--Шварца $|\langle x, \, y\rangle|
\le \|x\|\cdot\|y\|$ и неравенство треугольника $\|x+y\|\le \|x\|+\|y\|$.
Таким образом, $\|\cdot\|$ задает норму на $V$. \par
Если евклидово пространство полно относительно метрики, порожденной
скалярным произведением, то оно называется гильбертовым. \par
Пусть $(X, \, \Sigma, \, \mu)$ --- пространство с мерой, $p\ge 1$.
Скажем, что функция $f:X\rightarrow \C$ принадлежит $L_p(X, \, \Sigma, \,
\mu)$, если $|f|^p$ интегрируема по Лебегу. Нормой функции
$f$ называется величина $$\|f\|_p=\left(\int \limits_X |f(x)|^p\,
d\mu(x)\right)^{1/p}.$$ Назовем функции $f$ и $g$ эквивалентными,
если они совпадают почти всюду. Утверждается, что множество
классов эквивалентности функций из $L_p(X, \, \Sigma, \, \mu)$
с нормой $\|\cdot\|_p$ образуют банахово пространство. Если
$p=2$, то норма порождается скалярным произведением $$\langle f, \,
g\rangle=\left(\int \limits_{X}f^*(x)g(x)\, d\mu(x)\right)^{1/2},$$
так что $L_2(X, \, \Sigma, \, \mu)$ --- это гильбертово пространство. \par
{\bf Пример 1.} Пусть $X=\N$, $\Sigma$ --- все подмножества $\N$,
$\mu(\{n\})=1$. Тогда $L_p(X, \, \Sigma, \, \mu)$ --- это
пространство $l_p$ всех последовательностей $a=(a_n)_{n=1}^\infty$
таких, что $\|a\|^p=\sum \limits_{n=1}^\infty |a_n|^p<\infty$. \par
{\bf Пример 2.} Пусть $X=I$ --- промежуток в $\R$, $\Sigma$ ---
измеримые по Лебегу подмножества, $\mu$ --- мера Лебега. Пространство
$L_p(X, \, \Sigma, \, \mu)$ обозначается через $L_p(I)$. \par
Приведем основные свойства сепарабельных гильбертовых пространств.
\begin{enumerate}
\item Пусть ${\cal H}_0$ --- замкнутое подпространство в гильбертовом
пространстве ${\cal H}$. Тогда для любого $x\in {\cal H}$
существует единственный вектор $P(x)\in {\cal H}_0$ такой, что
вектор $x-P(x)$ ортогонален любому элементу ${\cal H}_0$. Вектор
$P(x)$ называется ортогональной проекцией вектора $x$ на подпространство
${\cal H}_0$.
\item Пусть $l$ --- линейный непрерывный функционал на пространстве
${\cal H}$. Тогда существует единственный вектор $y\in {\cal H}$
такой, что $l(x)=\langle y, \, x\rangle$ для любого $x\in {\cal H}$.
\item Каждое сепарабельное гильбертово пространство имеет
не более чем счетный ортонормированный базис, то есть такую систему
векторов $(e_n)$, что $\langle e_k, \, e_n\rangle =\delta_{kn}$
и для любого $x\in {\cal H}$ существует единственный набор
коэффициентов $x_n$, такой что $$x=\sum \limits_{n=1}^\infty x_ne_n=
\lim \limits_{n\rightarrow \infty}\sum \limits_{k=1}^n x_ke_k$$
(предел в метрике пространства ${\cal H}$).
\end{enumerate}
Пусть $X$, $Y$ --- два линейных нормированных пространства,
$A:X\rightarrow Y$ --- линейный оператор. Оператор $A$ называется
ограниченным, если образ единичного шара $B_X=\{x\in X:\|x\|_X\le 1\}$
является ограниченным множеством в $Y$. Нормой ограниченного
оператора $A$ называется величина $\|A\|=\sup_{x\in B_X}\frac{\|Ax\|_Y}
{\|x\|_X}$. Линейный оператор ограничен тогда и только тогда,
когда он непрерывен. \par
Множество ограниченных операторов с определенной выше нормой образует
линейное нормированное пространство. Последовательность операторов
$A_n$ сходится к оператору $A$ равномерно, если $\|A_n-A\|\rightarrow 0$
при $n\rightarrow \infty$. \par
Пусть $X$, $Y$ --- банаховы пространства, $X_0\subset X$, $Y_0
\subset Y$ --- всюду плотные подпространства, $A_0:X_0\rightarrow
Y_0$ --- линейный ограниченный оператор. Тогда существует
единственное его продолжение по непрерывности до оператора
$A:X\rightarrow Y$; при этом $\|A\|=\|A_0\|$. \par
Линейный оператор называется компактным, если пополнение множества
$A(B_X)$ в метрике пространства $Y$ компактно (если $Y$ банахово,
то пополнение --- это замыкание в пространстве $Y$). Любой
компактный оператор ограничен. Обратное, вообще говоря, неверно,
так как в бесконечномерном пространстве пополнение единичного шара
не является компактным. \par
Пусть ${\cal H}$ --- гильбертово пространство, $A:{\cal H}\rightarrow
{\cal H}$ --- ограниченный линейный оператор. Оператор $A^*:{\cal H}
\rightarrow {\cal H}$ называется сопряженным к оператору $A$, если
для любых $x$, $y\in {\cal H}$ выполнено $\langle A^*x, \, y\rangle=
\langle x, \, Ay\rangle$. Утверждается, что сопряженный
оператор существует и единствен и что $\|A^*\|=\|A\|$. Линейный
ограниченный оператор называется самосопряженным, если $A^*=A$. \par
Для компактных самосопряженных операторов выполнена теорема
Гильберта--Шмидта:
\begin{Trm}
Пусть $A:{\cal H}\rightarrow {\cal H}$ --- компактный
самосопряженный оператор. Тогда в ${\cal H}$ существует ортонормированный
базис $(e_n)$, состоящий из собственных векторов оператора $A$:
$Ae_n=\lambda_ne_n$. Все числа $\lambda_n$ вещественны и для любого
$\delta>0$ множество $\R\backslash (-\delta, \, \delta)$ содержит
конечное число точек $\lambda_n$.
\end{Trm}
Пусть ${\cal H}_1$, ${\cal H}_2$ --- гильбертовы пространства.
Оператор $U:{\cal H}_1\rightarrow {\cal H}_2$ называется
унитарным, если он является взаимно-однозначным отображением
${\cal H}_1$ на ${\cal H}_2$ и $\langle Ux, \, Uy\rangle_{{\cal
H}_2}=\langle x, \, y\rangle_{{\cal H}_1}$. Если ${\cal H}_1=
{\cal H}_2$, то это эквивалентно тому, что $U^*=U^{-1}$.
Гильбертовы пространства ${\cal H}_1$ и ${\cal H}_2$ называются
изометрически изоморфными, если существует унитарный оператор
$U:{\cal H}_1\rightarrow {\cal H}_2$. Оказывается, что все
бесконечномерные сепарабельные гильбертовы пространства изометрически
изоморфны друг другу и, в частности, пространству $l_2$.
В самом деле, пусть $(e_n)$ --- ортонормированный базис в ${\cal H}$.
Каждому $x\in {\cal H}$, $x=\sum \limits_{n=1}^\infty x_ne_n$
сопоставим последовательность $(x_n)$. Это и есть искомый унитарный
оператор из ${\cal H}$ на $l_2$. \par
Пусть ${\cal H}_0\subset {\cal H}$ --- замкнутое подпространство.
Каждому вектору $x\in {\cal H}$ сопоставим его ортогональную
проекцию на ${\cal H}_0$. Это отображение $P$ является линейным
ограниченным самосопряженным оператором и $P^2=P$. Если $x\in
{\cal H}_0$, то $Px=x$, если $x\in {\cal H}_0^{\bot}$ (то есть $x$
ортогонален любому вектору из ${\cal H}_0$), то $Px=0$.

\subsection{Обобщенные функции}
Сначала введем понятие топологического векторного пространства. Пусть
поле $\mathbb{K}$ --- это $\R$ или $\C$.
\begin{Def}
Топологическим векторным пространством называется пара $(X, \, \tau)$,
где $X$ --- линейное пространство над полем $\mathbb{K}$, $\tau$
--- топология на $X$, удовлетворяющая следующему условию: операции
сложения векторов и умножения вектора на число являются непрерывными
отображениями $X\times X\rightarrow X$ и $\mathbb{K}\times X\rightarrow
X$ соответственно.
\end{Def}
Пусть $X$ --- линейное пространство, $\{p_{\alpha}\}_{\alpha\in A}$ ---
система полунорм на $X$. Введем на $X$ топологию, базой которой
является система множеств вида $$\{x\in X:p_{\alpha_k}(x-x_0)<\varepsilon_k,
\; k=1, \, \dots, \, n\},$$ где $x_0\in X$, $n\in \N$, $\varepsilon_k
>0$. Тогда $X$ называется полинормированным пространством. В
полинормированном пространстве сложение векторов и умножение
вектора на число является непрерывной операцией, так что это
частный случай топологического векторного пространства.
\begin{Sta}
Пусть на линейном пространстве $X$ заданы две системы полунорм
$P=\{p_\alpha\}$ и $Q=\{q_\beta\}$. Тогда для того, чтобы
топология, порожденная системой $P$, была сильнее топологии,
порожденной системой $Q$, необходимо достаточно, чтобы для любого
$\beta$ нашлись $C>0$ и $\{\alpha_i\}_{i=1}^n$ такие, что
$q_\beta(x)\le C\sum \limits_{i=1}^n p_{\alpha_i}(x)$ для любого
$x\in X$.
\end{Sta}
Пусть $\Omega \subset \R^m$ --- область, $K_n\subset \Omega$ ---
компактные подмножества, $K_n\subset K_{n+1}$, $\cup_{n=1}^\infty
K_n=\Omega$. Рассмотрим пространство $C_0^\infty(\Omega)$
бесконечно дифференцируемых функций с компактным носителем и
зададим в нем топологию с помощью системы допустимых полунорм.
Полунорма $p$ называется допустимой, если для любого $n\in \N$
найдутся числа $C(n)>0$ и $j(n)\in \Z_+$ такие, что для
любой функции $\varphi\in C_0^\infty(\Omega)$ с носителем в
$K_n$ выполнено $p(\varphi)\le
C(n) \max_{|\alpha|<j(n)}\max_{x\in K_n}|D^{\alpha}\varphi(x)|$.
Здесь $\alpha=(\alpha_1, \, \dots, \, \alpha_m)$, $\alpha_j\in
\Z_+$, --- мультииндекс, $|\alpha|=\alpha_1+\dots+\alpha_m$,
$D^\alpha\varphi(x)= \frac{\partial^{|\alpha|}\varphi}{\partial
x_1^{\alpha_1}\dots \partial x_m^{\alpha_m}}(x)$. Обозначим
построенное пространство через $D$. Можно показать, что
$\varphi_n\stackrel{D}{\rightarrow}\varphi$ при $n \rightarrow \infty$
тогда и только тогда, когда существует компакт $K\subset \Omega$
такой, что ${\rm supp}\, \varphi_n\subset K$ для любого $n$ и
$D^\alpha \varphi_n\rightarrow D^\alpha\varphi$ равномерно в
$\Omega$ для любого $\alpha$. \par Пространство $S(\R^m)$ состоит
из бесконечно дифференцируемых функций, таких что \\$\sup_{x\in
\R^m}\|x\|^k|D^\alpha \varphi(x)|<\infty$ для любого $k\in \Z_+$ и
для любого мультииндекса $\alpha$. Топология задается с помощью
системы полунорм $p_{k,\, \alpha}(\varphi) = $ $ \sup_{x\in
\R^m}\|x\|^k|D^\alpha \varphi(x)|$. \par Ясно, что $D(\R^m)\subset
S(\R^m)$; кроме того, топология в $D(\R^m)$ сильнее, чем в
$S(\R^m)$. \par Пространства обобщенных функций $D'$ и $S'$ ---
это пространства линейных непрерывных функционалов на $D$ и $S$
соответственно. Топология задается системой полунорм
$p_\varphi(F)=|\langle F, \, \varphi\rangle|$, $\varphi\in D$
(соответственно $\varphi\in S$). Так как топология в $D$ сильнее,
чем в $S$, то $S'\subset D'$ и вложение $S'$ в $D'$ непрерывно. При
этом включение $S'\subset D'$ строгое: любая локально
интегрируемая функция задает обобщенную функцию из $D'$ по формуле
$\langle F, \, \varphi\rangle=\int F(x)\varphi(x)\, dx$, но она не
обязательно принадлежит $S'$. Например, если $|F(x)|\ge
Ce^{\|x\|^\gamma}$ при достаточно больших $x$, где $C>0$,
$\gamma>0$, то $F\notin S'$. \par Умножение на гладкую функцию и
дифференцирование задаются соответственно по формулам \\$\langle
\eta(x)F(x), \, \varphi(x)\rangle=\langle F(x), \,
\eta(x)\varphi(x)\rangle$ и $\langle F'_{x_k}, \, \varphi\rangle=
-\langle F, \, \varphi'_{x_k}\rangle$. \par Пусть $L=\sum
\limits_{|\alpha|\le n}a_{\alpha}(x)D^\alpha+f(x)$ --- линейный
дифференциальный оператор с гладкими коэффициентами. Скажем, что
$F$ удовлетворяет дифференциальному уравнению $LF=0$ в обобщенном
смысле, если для любого $\varphi\in D$ выполнено $\langle LF, \,
\varphi\rangle=0$. Если $L=\sum \limits_{k=0}^n
c_k(x)\frac{d^k}{dx^k}+f(x)$ в $D(a, \, b)$, где $c_k$ и $f$
бесконечно дифференцируемы и $c_n(x)\ne 0$ для любого $x\in (a,
\, b)$, то обобщенное решение уравнения $LF=0$ является гладкой
функцией и удовлетворяет этому уравнению в классическом смысле
[20]. В многомерном случае это, вообще говоря, не
выполняется: например, функция $f(x, \, y)=\delta(x-y)$, задаваемая
равенством $\langle
f, \, \varphi\rangle =\int \varphi(t, \, t)\, dt$, является
обобщенным решением уравнения $f'_x+f'_y=0$. Для того, чтобы
решение было регулярным, достаточно потребовать, чтобы
дифференциальный оператор был эллиптическим. В частности,
обобщенное решение уравнения $(-\Delta +V)u=Eu$, где $E\in \C$,
$V\in C^\infty(\Omega)$, является бесконечно дифференцируемой
функцией [3, \S IX.6]. \par Теперь дадим определение
пространства Соболева. Пусть $\Omega\subset \R^m$ --- область.
Пространством Соболева $W_2^r(\Omega)$ называется множество
функций $f\in L_2(\Omega)$ таких, что их обобщенные производные
$D^\alpha f$, $|\alpha|\le r$, принадлежат $L_2(\Omega)$. В этом
пространстве вводится норма $$\|f\|^2_{W^r_2(\Omega)}= \sum
\limits_{|\alpha|\le r}\int \limits_{\Omega}|D^\alpha f(x)|^2\,
dx,$$ и $W^r_2(\Omega)$ является гильбертовым пространством
относительно этой нормы. \par Рассмотрим случай $m=1$ и $r=1$.
Тогда пространство $W^1_2(a, \, b)$ совпадает с множеством
абсолютно непрерывных функций $f$, таких что $f'\in L_2(a, \, b)$
(где $f'$ --- производная функции $f$ почти всюду). При этом $f'$
совпадает с обобщенной производной функции $f$. Если $(a, \, b)$
--- конечный интервал, то норма функции $f$ в пространстве $C[a,
\, b]$ оценивается следующим образом: для любого $\varepsilon>0$
существует константа $C=C(\varepsilon, \, b-a)$ такая, что $$\|f\|
_{C[a, \, b]}\le \varepsilon \|f'\|_{L_2[a, \, b]}+C\|f\|_{L_2[a,
\, b]}$$ для любого $f\in W^1_2[a, \, b]$ (неравенство Соболева).
Из неравенства Коши--Буняковского следует, что для любых $x_1$,
$x_2\in [a, \, b]$ выполнено $$|f(x_2)-f(x_1)|\le
|x_2-x_1|^{1/2}\|f'\|_{L_2[a, \, b]}.$$ \par Через $\mathaccent'27
W^1_2[a, \, b]$ обозначим замыкание $C_0^\infty (a, \, b)$ в
$W^1_2[a, \, b]$. Это подпространство совпадает с множеством
функций $f\in W^1_2[a, \, b]$, таких что $f(a)=f(b)=0$. \par
Подробнее про обобщенные функции написано, например, в [24].
\subsection{Задачи выпуклого программирования.}
\label{sect_conv_prog}
Пусть $X$ --- линейное пространство над $\R$. Множество $A\subset
X$ называется выпуклым, если для любых $x$, $y\in A$ и любого
$\alpha\in [0, \, 1]$ выполнено $\alpha x+(1-\alpha)y\in A$.
Функция $f:A\rightarrow \R$ называется выпуклой, если для любых
$x$, $y\in A$ и для любого $\alpha \in [0, \, 1]$ выполнено
неравенство Йенсена $f(\alpha x+(1-\alpha)y)\le \alpha f(x)+
(1-\alpha)f(y)$. \par
Пусть $A\subset X$ --- выпуклое множество, $f_i:A\rightarrow \R$,
$i=0, \, \dots, \, n+m$, при этом функции $f_i$ выпуклы для любого
$i=0, \, \dots, \, n+m$, а при $i=n+1, \, \dots, \, n+m$ выпуклыми
являются также $-f_i$. Рассмотрим экстремальную задачу
\begin{align}
\label{convex_progr}
\left\{ \begin{array}{l} f_0(x)\rightarrow \inf, \\ f_i(x)\le 0, \;
1\le i\le n, \\ f_i(x)=0, \; n+1\le i\le n+m, \\ x\in A. \end{array}\right.
\end{align}
Запись (\ref{convex_progr}) означает, что ищется минимум функции
$f_0$ на множестве $\tilde A$ точек $x\in A$, удовлетворяющих условиям $f_i(x)
\le 0$ для $i=1, \, \dots, \, n$ и $f_i(x)=0$ для $i=n+1, \, \dots, \, n+m$.
Следующая теорема\footnote{Здесь
применяется модификация теоремы Куна--Таккера для случая, когда
имеются ограничения типа равенств для аффинных функционалов. Это
утверждение отличается тем, что множители Лагранжа перед
функционалами, задающими ограничения типа равенств, не обязательно
неотрицательные. Доказательство такое же, как у теоремы
Куна--Таккера в [19], с небольшими изменениями.} [19] является
аналогом правила множителей Лагранжа для задач на условный экстремум.
\begin{Trm}
(Кун--Таккер)
Пусть $\hat x\in A$ --- решение задачи (\ref{convex_progr}). Тогда
найдутся $\lambda_0$, $\dots,$ $\lambda_{n+m}$, не равные одновременно
нулю, такие, что
\begin{enumerate}
\item $\sum \limits_{i=0}^{n+m} \lambda_i f_i(\hat x)=\min_{x\in A}
\sum \limits_{i=0}^{n+m} \lambda_i f_i(x)$;
\item $\lambda_i\ge 0$, $0\le i\le n$;
\item $\lambda_i f_i(\hat x)=0$, $1\le i\le n$.
\end{enumerate}
Если точка $\hat x\in \tilde A$ удовлетворяет условиям 1)--3) с
$\lambda_0>0$, то $\hat x$ является точкой минимума в (\ref{convex_progr}).
\end{Trm}

\section{Дополнение 2: Группы и алгебры Ли}
Подробное изложение теории групп Ли, ориентированное на приложения
к квантовой механике и теории поля имеется в книге [4]. Построение
представлений некоторых групп, встречающихся в задачах
математической физики, можно найти в [21].
\subsection{Непрерывные группы}
\begin{Def}
Множество $G$ элементов $g$ называется группой, если на нем
определена ассоциативная операция умножения, сопоставляющая каждой
упорядоченной паре $g_1,\, g_2$ элемент $g_3= g_1 g_2$, называемый
произведением, такая, что существуют единица $e$, ($e g=g e$ для
всех $g\in G$) и обратный элемент $g^{-1}:\; gg^{-1}=g^{-1}g=e$.
\end{Def}
Вообще говоря, $g_1 g_2\neq g_2 g_1.$ Группа, для всех элементов
которой $g_1 g_2=g_2 g_1$, называется коммутативной или {\em
абелевой}. Если хотя бы для одной пары элементов операция
умножения некоммутативна, группа называется {\em неабелевой}.
Число элементов группы может быть конечным, либо бесконечным, в
последнем случае элементы группы могут помечаться непрерывно
изменяющимися параметрами.

Группа называется топологической, если множество ее элементов
является топологическим пространством (определена система открытых
множеств), и если групповое умножение является непрерывным
отображением $G\times G\to G$, а $g^{-1}$ --- непрерывным
отображением $G\to G$. Важное значение имеют понятия компактности
и связности. Топологическая группа {\em компактна}, если из любого
покрытия множества ее элементов окрестностями можно выделить
конечное покрытие. В зависимости от топологических свойств
множества, группа может быть {\em односвязной}, если любая
замкнутая кривая может быть непрерывно деформирована в точку, {\em
(много)связной} и {\em несвязной}.\par Теперь напомним определения
дифференцируемого и аналитического многообразий. Рассматривается
хаусдорфово пространство, покрытое системой окрестностей, на
каждой из которых задана карта, т.е. отображение каждой точки
пространства в $\R^d$ (локальная координатная система
$x^i,\,i=1,\ldots, d$). Пространство, снабженное такой структурой,
называется вещественным дифференцируемым (аналитическим)
многообразием размерности $d$, если в областях перекрытия  двух
карт $x^i,\,{x'}^i$ задано преобразование координат
${x'}^i=f^i(x)$, осуществляемое функциями $f^i$ класса $C^\infty$
(аналитическими функциями). Некоторые четномерные многообразия
($d=2n$) допускают введение комплексной структуры. В случае
комплексного многообразия рассматриваются локальные карты
$z^i,\,{\bar z^i}\in \C^n$, а в областях перекрытия карт задаются
голоморфные отображения $z^i\to {z'}^i(z)$. В определении групп Ли
в вещественном случае, как правило, используются аналитические
многообразия, которые наследуют аналитичность от случая
комплексных многообразий.
\begin{Def}
Топологическая группа называется группой Ли, если групповое
множество является  аналитическим многообразием, а операции
умножения и взятие обратного являются аналитическими
отображениями. Вещественная размерность многообразия является
размерностью группы.
\end{Def}
Важным классом групп Ли являются {\em группы преобразований} точек
некоторого многообразия $\M:\; x\to g x$ (левое действие группы).
Например, в $\R^3$ можно рассматривать {\em трансляции} $x^i\to
x^i+a^i$ и {\em вращения} $x^i\to O^i_{\;\; j}x^j,\, O^i_{\;\;
j}O^j_{\;\; k}=\delta^i_k$. Параметрами группы трансляций являются
вещественные числа из $\R^3$, элементами группы вращений
--- ортогональные матрицы, зависящие от непрерывно изменяющихся
параметров (углов). В общем случае, элементы $g(a^i)$
$d$-параметрической группы преобразований зависят от $d$
параметров $a^i$. Удобно принять, что параметры единичного
(тождественного) преобразования имеют нулевые значения $e=g(0)$.
\par
Группа Ли компактна, если компактно групповое многообразие.
Например, группа вращений $SO(N)$ компактна, поскольку область
изменения каждого из ее $N(N-1)/2$ параметров (углов) ---
замкнутое и ограниченное множество в $\R$. Группа Лоренца
$SO(1,3)$ некомпактна, поскольку псевдовращения задаются
параметрами, изменяющимися на всей оси. Группа сдвигов в $\R^n$
также некомпактна (изоморфна $\R^n$). \par Подмножество
$G_0\subset G$, замкнутое относительно заданной на всей группе
операции умножения, называется подгруппой. Возможны непрерывные
подгруппы, которые являются аналитическими подмногообразиями, а
также дискретные подгруппы. Тривиальными подгруппами называются
сама группа и группа, состоящая из одного единичного элемента.
\par Подгруппа $H$ называется {\em инвариантной} (нормальной,
нормальным делителем), если из $h\in H$ и $g\in G$ следует, что,
$ghg^{-1}\in H$. Группа называется {\em простой}, если она не
содержит нетривиальных  связных инвариантных подгрупп, и {\em
полупростой}, если она не содержит нетривиальной инвариантной
связной коммутативной (абелевой) подгруппы. Подчеркнем, что здесь
речь идет о непрерывных подгруппах Ли, простая группа может иметь
нетривиальную дискретную нормальную подгруппу. Существует полная
классификация простых комплексных групп Ли (см. ниже).\par {\bf
Пример.} Группа $O(k)$ ортогональных преобразований в $\R^k$
задается квадратными вещественными матрицами $k\times k$,
удовлетворяющими условию \\$\sum_k V_{ik}V_{jk} = \delta_{ij}$.
Такие матрицы, очевидно, имеют определитель $\pm 1$. Группа $O(k)$
несвязна и состоит из двух связных подгрупп в соответствии со
знаком определителя. Матрицы с определителем единица образуют
инвариантную подгруппу обозначаемую $SO(k)$. Группа $SO(3)$
проста, для высших размерностей это не так: имеется локальный
изоморфизм $SO(4)\cong SO(3)\times SO(3)$, так что группа $SO(4)$
лишь полупроста. Собственные группы $SO(k)$ (не содержащие
отражений) просты при $k\geq 5$.

\begin{Trm}
Пусть $G$ --- топологическая группа, $H$ --- замкнутая подгруппа;
определим множество классов эквивалентности $G/H$: $g_1, g_2$
принадлежат одному и тому же классу, если $g_1=g_2 h,\, h\in H$, и
введем на нем соответствующую топологию. Тогда, если $H$
нормальный делитель, то в $G/H$ можно ввести умножение так, что
$G/H$  будет группой (называемой фактор-группой).
\end{Trm}
Множество элементов, коммутирующих со всеми элементами группы,
является инвариантной абелевой подгруппой и называется {\em
центром}. Рассмотрим группу $SU(n)$ унитарных комплексных матриц
$n\times n$ с единичным определителем. Можно показать, что
$SU(2)/Z_2=SO(3)$, где $Z_2$ -- дискретный центр группы $SU(2)$,
состоящий из двух элементов (единичной и минус-единичной матриц).
Этот пример  демонстрирует накрытие неодносвязной группы $SO(3)$
односвязной группой $SU(2)$. Более общее утверждение формулируется
в виде следующей теоремы об {\em универсальной накрывающей}:
\begin{Trm}
Внутри класса линейно связных, локально односвязных, локально
изоморфных  топологических групп существует единственная с
точностью до изоморфизмов односвязная группа $\tilde G$. Все
остальные группы из этого класса являются фактор-группами $\tilde
G/N$, где $N$ --- дискретная нормальная подгруппа.
\end{Trm}
Поясним понятия прямого и полупрямого произведений групп.
\begin{Def} Группа $G$  является   прямым произведением  $G=H \times
K$ своих нормальных подгрупп $H,\,K$, если ее элементом являются
упорядоченные пары $g=(h,\,k)$  с законом умножения
$gg'=(hh',\,kk')$. \end{Def} \noindent При этом каждая из
групп-сомножителей остается нормальной подгруппой произведения.  В
отличие от этого, {\em полупрямое произведение}  $H\ltimes K$ двух
групп $H,\,K$ определяется таким образом, что только $H$ остается
инвариантной подгруппой произведения. Закон умножения
проиллюстрируем на примере евклидовой группы $E(3)$, состоящей из
трансляций $T(3)$ и вращений $SO(3)$ в $\R^3$. Элементами $E(3)$
являются пары $(a^i,\,O^i_j)$ с законом умножения $( a,\,O)* (
a',\,O')= (a^i+O^i_j a'^j,\,O^i_k O'^k_j)$. Трансляции остаются
инвариантной подгруппой в $G=T(3)\ltimes SO(3)$, но вращения нет.
Аналогичным образом строится группа Пуанкаре как полупрямое
произведение четырехмерных трансляций в пространстве-времени
$\R^{1,3}$ и группы Лоренца $SO(1,3)$.
\par
Действие группы $G$ в многообразии {\em транзитивно}, если для
любой пары точек найдется элемент, переводящий одну точку пары в
другую.\begin{Def}Многообразие, на котором действие группы
является транзитивным, называется однородным пространством этой
группы.\end{Def} {\em Подгруппа изотропии} $H_x$ заданной точки
$x$ однородного пространства состоит из всех элементов $G$,
оставляющих эту точку неподвижной (при этом группы изотропии
разных точек изоморфны друг другу). Нетрудно показать, что
фактор-пространство $G/H$ является однородным пространством группы
$G$. В приведенном выше примере пространство $\R^3$, на котором
действует евклидова группа $E(3)$, можно рассматривать как
однородное фактор-пространство $E(3)/SO(3)$ (здесь $SO(3)$ ---
подгруппа изотропии). Аналогично, пространство Минковского
является фактор-пространством группы Пуанкаре по группе Лоренца.
\par
\subsection{Алгебры Ли}
Абстрактное определение алгебр Ли следующее:
\begin{Def}
Вещественной (комплексной) алгеброй Ли  называется векторное
пространство $L$ над полем вещественных (комплексных) чисел на
котором задана операция умножения, называемая коммутатором,
сопоставляющая каждой паре элементов $A,\, B$ величину $[A,\,B]$,
удовлетворяющую свойствам линейности, антисимметрии и тождеству
Якоби: \end{Def}\vspace{-5mm}
\begin{eqnarray}
 % \nonumber to remove numbering (before each equation)
     & [\alpha A+\beta B,\,C]=\alpha[A,\,C]+\beta[B,\,C],&  \non \\
     &[A,\,B]=-[B,\,A],&  \non \\
     &[A,[B,\,C]]+[B,[C,\,A]]+[C,[A,\,B]]=0.\non&
 \end{eqnarray}
Далее обсуждается случай только конечномерных алгебр. Векторные
поля в многообразии образуют структуру алгебры Ли. Действительно,
пусть $X,\, Y$ --- векторные поля в вещественном аналитическом
$d$-мерном многообразии с локальными координатами $x^i,\,
i=1,...,d$:
$$ X=X^i(x)\frac{\partial}{\partial x^i},\quad
Y=Y^i(x)\frac{\partial}{\partial x^i}.$$ Тогда коммутатор,
определяемый как производная Ли поля $Y$ по полю $X$
$$Z=[X,\,Y]=Z^i \frac{\partial}{\partial x^i},\quad Z^i=X^j
\frac{\partial Y^i}{\partial x^j}-Y^j \frac{\partial X^i}{\partial
x^j}$$ удовлетворяет требуемым свойствам.   Векторное поле
порождает локальную однопараметрическую группу преобразований,
которая каждой точке многообразия $M_0$ и достаточно малому
вещественному параметру $\tau$  сопоставляет точку $M_\tau$,
лежащую на интегральной кривой векторного поля $\gamma(\tau)$,
проходящей при $\tau=0$ через  $M_0$.
\par
Каждая группа Ли имеет соответствующую алгебру Ли. Имеют место
утверждения:
\begin{Trm} Всякая алгебра Ли является алгеброй Ли некоторой
локальной группы Ли. Всякой алгебре Ли соответствует связная
односвязная группа Ли, для которой она является алгеброй.
\end{Trm} Соответствие между группами и алгебрами Ли выглядит
особенно просто для групп преобразований. Пусть $G$
--- $d$-параметрическая группа преобразований  $n$ -мерного
многообразия, задаваемая аналитическими функциями $f^i$,
зависящими от $ n+d$  аргументов
$x'^i=f^i(x^1,\ldots,x^n;\,a^1,\ldots,a^d)$ так, что $f^i(x;\,0)$
является единицей группы. Инфинитезимальное преобразование
$x+dx=f(x;\, da)$, $$dx^i=\frac{\partial f^i(x;\,a)}{\partial
a^j}\Bigg|_{a=0}=\sum_{j=1}^d u^i_j(x)da^j$$ порождает систему $d$
векторных полей $$X_j=u^i_j(x)\frac{\partial}{\partial x^i },$$
называемых генераторами группы $G$. Они образуют алгебру Ли ${\cal
G}$ данной группы. Обратная задача восстановления группы Ли по
заданной алгебре Ли решается с помощью экспоненциального
отображения. Следует иметь в виду, что алгебра Ли является
локальной структурой, не зависящей от глобальной топологии группы.
Иначе, можно построить глобально различающиеся группы имеющие одну
и  ту же алгебру генераторов (например, $SO(3)$ и $SU(2))$. \par
Рассмотрим конечномерную алгебру Ли состоящую из $d$ элементов
$X_\alpha$, образующих замкнутое относительно коммутации
множество:
$$[X_\alpha,\,X_\beta]=C_{\alpha\beta}^{\;\;\;\;\gamma} X_\gamma$$
Величины $C_{\alpha\beta}^{\;\;\;\;\gamma}=-
C_{\beta\alpha}^{\;\;\;\;\gamma}$ называются структурными
константами этой алгебры. В силу тождества Якоби они удовлетворяют
соотношению
\begin{equation}\label{cjak}
C_{\alpha\rho}^{\;\;\;\;\delta}C_{ \beta\gamma}^{\;\;\;\;\rho}+
C_{\beta\rho}^{\;\;\;\;\delta }C_{\gamma\alpha }^{\;\;\;\;\rho}+
C_{\gamma\rho}^{\;\;\;\;\delta}C_{\alpha\beta}^{\;\;\;\;\rho}=0.
\end{equation}
С помощью структурных констант можно построить симметричный тензор
второго ранга
$$g_{\alpha\beta}=C_{ \alpha\gamma}^{\;\;\;\;\delta}C_{\beta\delta}^{\;\;\;\;\gamma},
\quad \alpha,\,\beta=1,...,d,$$ который называется метрикой
Киллинга. Он вводит метрическую структуру на групповом
многообразии. Соответствующая квадратичная форма называется формой
Киллинга.
\par
Алгебра называется абелевой (коммутативной), если все ее элементы
коммутируют между собой. Подпространство $N\subset L$ образует
подалгебру, если $[N,\, N]\subset N$, и идеал, если $[L,\,
N]\subset N$. Максимальный идеал, элементы которого коммутируют со
всеми элементами алгебры, называется центром. Очевидно, центр
коммутативен. \begin{Def}Алгебра Ли называется простой, если она
не имеет нетривиальных идеалов, и полупростой, если она не имеет
коммутативных идеалов.\end{Def}\noindent  Любая полупростая
алгебра Ли представима в виде суммы простых идеалов.
\par Соответствие между структурой группы и ее алгебры Ли
устанавливается следующей теоремой.
\begin{Trm}
Пусть $G$ --- группа Ли, ${\cal G}$ --- ее алгебра Ли, $H$ ---
непрерывная подгруппа и ${\cal H}$
--- множество касательных векторов к единице в $H$ . Тогда
\begin{enumerate}\item
${\cal H}$ --- подалгебра в ${\cal G}$, являющаяся алгеброй Ли
группы $G$, \item если  $H$ --- инвариантная подгруппа, то ${\cal
H}$ --- идеал в ${\cal G}$. \end{enumerate}
\end{Trm}
\noindent Простой критерий, позволяющий установить, является ли
алгебра полупростой, дает
\begin{Trm}
(Картан). Алгебра Ли полупроста тогда и только тогда, когда ее
метрика Киллинга невырождена, $det||g_{\alpha\beta}||\neq0$.
Соответствующая группа Ли также полупроста.
\end{Trm}
Существует простой критерий компактности полупростой группы Ли:
\begin{Trm} (Вейль).
Вещественная полупростая связная группа Ли компактна тогда и
только тогда, когда метрика Киллинга соответствующей алгебры
---  отрицательно определенная квадратичная форма.
\end{Trm}
Это свойство отражает отрицательную определенность следа квадрата
антисимметричной матрицы.
\subsection{Классификация  алгебр и групп Ли}
Классификация алгебр Ли  основана на утверждении:
\begin{Trm} (Адо). Каждая
алгебра Ли над $\C$ изоморфна некоторой матричной алгебре.
\end{Trm}
Очевидно, это остается верным и для вещественных алгебр. Таким
образом любая абстрактная алгебра Ли может рассматриваться как
подалгебра алгебры матриц $gl(n,\,\C),\,n\in\N$.

Дадим теперь определение разрешимых и нильпотентных алгебр. Если
$N$ идеал алгебры $L$, то легко проверить, что $[N,N]$ тоже идеал.
В частности сама алгебра $L$ также идеал, и можно построить
последовательности $L^n$ ({\em производный ряд}) и $L_n$ ({\em
центральный ряд}) идеалов, выбирая $L^0=L,\; L^{n+1}=[L^n,\,L^n]$
и $L_0=L,\; L_{n+1}=[L_n,\,L]$.
\begin{Def}
Алгебра $L $ называется разрешимой, если производный ряд
обрывается для некоторого номера $n, \;L^n=0$, и нильпотентной,
если обрывается центральный ряд $L_n=0$.
\end{Def}
\noindent Нетрудно видеть что $L^n \subset L_n$, так что любая
нильпотентная алгебра разрешима, обратное вообще говоря неверно.
Первая производная подалгебра разрешимой алгебры нильпотентна.
\par
{\bf Пример.} Подалгебра верхних (нижних) треугольных матриц в
$gl(n)$ разрешима. Ее производный ряд обрывается на  $n$-м
элементе. Подалгебра строго верхних (нижних) треугольных матриц в
$gl(n)$ (с нулевой главной диагональю) нильпотентна. Ее
центральный ряд обрывается на $n$-м элементе.

Прямое и полупрямое произведения групп соответствуют прямой и
полупрямой сумме их алгебр Ли. Если группа Ли является прямым
произведением $G=H\times K$, то алгебры образуют {\em прямую
сумму} ${\cal G}={\cal H}\oplus{\cal K}$, при этом любой элемент
${\cal H}$ коммутирует с любым элементом ${\cal K}$, т.е. $[{\cal
H},\,{\cal K}]=0$. {\em Полупрямая сумма} алгебр ${\cal G}={\cal
H}\, +\hspace{-4.6mm}\supset {\cal K}$ определяется так, что
${\cal H}$ является идеалом в ${\cal G}$: т.е. $[{\cal H},\,{\cal
H}]={\cal H},\,[{\cal H},\,{\cal K}]={\cal H},\,[{\cal K},\,{\cal
K}]={\cal K}.$ Имеет место
\begin{Trm}(Леви-Мальцев). Произвольная алгебра Ли ${\cal G}$
представима в виде полупрямой суммы  ${\cal G}={\cal R}\,
+\hspace{-5.2mm}\supset{\cal S}$, где ${\cal S}$  --- полупростая
алгебра и ${\cal R}$ --- максимальный разрешимый идеал (радикал):
$$[{\cal R},\,{\cal R}]\subset{\cal R},\quad
[{\cal R},\,{\cal S}]\subset{\cal R},\quad[{\cal S},\,{\cal
S}]\subset{\cal S}.$$\end{Trm} \noindent В приведенном выше
примере евклидовой трехмерной группы, для соответствующих алгебр
Ли имеем структуру $t(3)\, +\hspace{-4mm}\supset so(3)$  с
генераторами $P_i,\,L_i$, подчиняющимися перестановочным
соотношениям
$$[P_i,\,P_j]=0,\quad
[L_i,\,P_j]=i\epsilon_{ijk}P_k,\quad
[L_i,\,L_j]=i\epsilon_{ijk}L_k.$$
\par
Полной классификации разрешимых и неполупростых алгебр Ли не
существует. Полупростые алгебры представимы в виде прямой суммы
простых алгебр. Для последних имеется полная классификация
основанная на теореме Картана:
\begin{Trm}
Для каждой  простой алгебры Ли ${ \cal G}$ над $\C$  существует
максимальная абелева подалгебра  ${\cal H}\ni H_i,\,i=1,\ldots,
r;\;[H_i,\,H_j]=0$, такая, что для всех остальных элементов $E\in
{\cal G},\,E\notin {\cal H}$  имеем
$$[H_i,\, E_{{\bm \alpha}}]=\alpha^i E_{{\bm \alpha}},$$ где ${\bm
\alpha}$ ---  ненулевые  корневые векторы  в пространстве $\R^r$,
ассоциированном с подалгеброй Картана.
\end{Trm} В матричном представлении элементам подалгебры Картана соответствуют
диагональные матрицы. Размерность $r$ подалгебры Картана называется
{\em рангом} алгебры и соответствующей ей группы Ли. Число
корневых векторов, равное $d-r$, как правило превышает размерность
корневого пространства $\R^r$, так что корни являются линейно
зависимыми. Можно выбрать систему {\em простых корней} ---
векторов, через которые могут быть выражены все остальные, а также
ввести скалярное произведение корней в метрике, индуцируемой
метрикой Киллинга. Доказывается, что простые корни либо
ортогональны, либо располагаются друг относительно друга под
углами $120^0,\, 135^0 ,\,150^0$. Допустимые системы простых
корней изображаются {\em схемами Дынкина}, на основании которых
дается полная классификация всех простых конечномерных алгебр Ли
над $\C$. Она сводится к следующему: имеются четыре бесконечных
серии "классических" алгебр (индекс соответствует рангу алгебры):
\begin{enumerate}\item $A_n$ или $su(n+1)$ --- алгебра Ли группы
унитарных матриц $(n+1)\times(n+1)$ с определителем единица, $
d=n(n+2)$,\item $B_n$ или $so(2n+1)$ --- алгебра Ли группы
ортогональных $(2n+1)\times(2n+1)$ матриц с определителем единица,
$d=n(2n+1),$\item $C_n$ или $sp(2n)$ --- алгебра Ли группы
симплектических $(2n)\times(2n)$ матриц с определителем единица,
$d=n(2n+1)$,\item $D_n,\, n\geq 3$ или $so(2n)$ --- алгебра Ли
группы ортогональных $(2n)\times(2n)$ матриц с определителем
единица, $d=n(2n-1)$ ( для $n=2$ имеется алгебра $so(4)$, которая
полупроста, именно, $so(4)=so(3)\oplus so(3)$),
\end{enumerate} а также пять "исключительных" алгебр:
$G_2,\,F_4,\,E_6,\,E_7,\,E_8$ рангов 2, 4, 6, 7, 8 и размерности
14, 52, 78, 133 и 248 соответственно. Все эти алгебры являются
подалгебрами алгебры Ли $gl(n,\,\C)$ группы  $Gl(n,\,\C)$
обратимых комплексных матриц $n\times n$.\par Каждая полупростая
алгебра Ли может быть представлена в виде прямой суммы простых.
Произвольная алгебра Ли представляется в виде полупрямой суммы
разрешимой и полупростой алгебр.
\par
Классификация  простых алгебр Ли над $\R$ несколько более сложна
(см. [4], гл. 1): различают 12 классических серий и 23
исключительных алгебры.
\subsection{Представления}
Пусть $L$ ---  алгебра Ли над $\R$   ($\C$) и $H$ ---  некоторое
линейное пространство, на котором определено множество линейных
операторов $T$.
\begin{Def}
Представлением $L$ в  $H$   называется гомоморфизм $X\to T(X)$
такой, что для любых $X,\, Y\in L$ и $\alpha, \,\beta \in \R
\;(\C)$ выполняются соотношения
\begin{align}
 \alpha X + \beta Y\to \alpha T(X) + \beta T(Y),\label{r1}\\
 [X,\,Y]\to [T(X),\,T(Y)]= T(X)T(Y)-T(Y)T(X)\label{r2}.
\end{align}
\end{Def}
\noindent Тождество Якоби автоматически следует из (\ref{r2}).\par
{\bf Пример.} Рассмотрим алгебру Ли, состоящую из трех элементов
$Q,\,P,\,W$, удовлетворяющим соотношениям $$[P,Q]=-iW,\quad
[Q,W]=0,\quad [P,W]=0,$$ и пусть $H=L^2(\R)$. Тогда отображение
$$Q\to x,\quad P\to -i\partial_x,\quad W\to I$$ задает
представление этой алгебры в плотной области $D=C_0^\infty(\R)$ в
$H$. \par Структурные константы образуют представление алгебры Ли,
называемое присоединенным.   Действительно, пусть
$[X_\alpha,\,X_\beta]=C_{\alpha\beta}^{\;\;\;\;\gamma} X_\gamma$,
где структурные константы выбраны вещественными. Тогда, в силу
(\ref{cjak}), отображение $X_\alpha\to T(X_\alpha)$, где
$T(X_\alpha)$ --- вещественная матрица $d\times d$ c элементами
$$T(X_\alpha)^\gamma_\beta=-C_{\alpha\beta}^{\;\;\;\;\gamma},$$
является представлением.
\par
При построении представлений алгебр Ли особое значение имеют
полиномы от операторов алгебры, которые коммутируют со всеми ее
элементами (операторы Казимира). Простейший вид имеет квадратичный
оператор Казимира, он строится с помощью обратной метрики Киллинга
\begin{equation}\label{Casimir2}
    C_2=g^{\alpha\beta} X_\alpha X_\beta,\quad [C_2,\,X_\alpha]=0.
\end{equation}
Полное число операторов Казимира равно рангу группы (теорема о
ранге). В силу леммы Шура (см. ниже), операторы Казимира,
коммутирующие со всеми генераторами группы, кратны единичному
оператору в любом неприводимом представлении группы, поэтому
значения  операторов Казимира полностью определяет представление.
\par
{\bf Пример.} Для группы $SO(3)$ ранга 1 имеется единственный
оператор Казимира $C_2=L_1^2+L_2^2+L_3^2$, он коммутирует с тремя
генераторами $L_i$, образующими алгебру $so(3)$:
$[L_i,\,L_j]=i\epsilon_{ijk} L_k$. В неприводимых представлениях
$C_2=l(l+1)I$, где $l$ целое число (разд. 7.3 основного текста).
\par
Операторы Казимира принадлежат так называемой {\em универсальной
обертывающей алгебре}, которая порождается всеми возможными
произведениями элементов исходной алгебры Ли, взятыми во
всевозможном порядке. Такие произведения образуют бесконечномерную
ассоциативную алгебру. Нетрудно проверить, что для нее также
выполняется тождество Якоби.

%\subsection{Представления групп Ли}
В квантовой механике возникает задача отыскания подпространств
$\HH$ гильбертова пространства состояний на которых задано
действие некоторой группы Ли --- {\em представление группы}.
Например, при решении стационарного уравнения Шредингера для
частицы в центральном поле  полезно выбрать базис, реализующий
неприводимые представления группы вращений $SO(3)$. В общем
случае, необходимо построить такую функцию $T(g)$ принимающую
значения на множестве линейных операторов, действующих в $\HH$,
что для любой пары элементов $g_1,\,g_2 \in G$ имеет место
соотношение
\begin{equation}\label{rep}
   T(g_1g_2)=T(g_1)T(g_2).
\end{equation}  Подпространство $\HH$ называется
пространством представления, а операторы $T(g)$ образуют
представление группы $G$. Пространство представления может быть
конечномерным или бесконечномерным, непрерывность функций $T(g)$
может  задаваться  в различных топологиях (слабой, сильной), при
этом каждый оператор $T(g)$ должен иметь непрерывный обратный
оператор. Из (\ref{rep}) следует, что $T(g^{-1})=T^{-1}(g)$ и
оператор, соответствующий единице группы, является  единичным
оператором в $\HH$. Представление называется точным, если
единичный оператор является образом лишь единичного элемента
группы, и тривиальным, если все элементы группы отображаются в
единичный оператор.
\par
Рассмотрим случай конечномерного пространства $\HH$. Тогда,
выбирая в $\HH$ базис $\phi_n$ и разлагая $T(g) \phi_n$ по базису,
будем иметь матричное представление $T_{mn}(g)$, так что действие
операторов представления есть $T_{mn}(g) \phi_n=\phi_m$ (по $n$
суммирование). Умножение на группе порождает матричное умножение в
пространстве представления, при этом все матрицы представления
должны быть невырождены. При преобразовании базиса матрицы
представления претерпевают преобразование подобия. В более общем
случае, если задано линейное отображение $A:\,\HH\to \HH'$,
имеющее непрерывное обратное $A^{-1}$, то, как легко проверить,
операторы $T'(g)=AT(g)A^{-1}$ образуют {\em эквивалентное}
представление. При соответствующем выборе базиса эквивалентные
представления задаются одинаковыми матрицами.   Замена базиса
приводит к изменению матриц представления. Инвариантной
характеристикой, не зависящей от выбора базиса, является {\em
характер} представления $\chi_T(g)$, представляющий собой след
матрицы представления.
\par
В гильбертовом пространстве естественным образом определяется {\em
эрмитово-сопряженное представление}: нетрудно проверить что если
$T(g)$ -представление, то $T^+(g^{-1})$ также
представление.\begin{Def} Представление называется {\em унитарным},
если реализующие его операторы сохраняют скалярное произведение в
гильбертовом пространстве $\langle
T(g)\phi,\,T(g)\psi\rangle=\langle\phi,\,\psi\rangle$.\end{Def}
\noindent Представление унитарно, если оно совпадает с
эрмитово-сопряженным представлением. В ортонормированном базисе
матрица унитарного представления унитарна.  Два унитарных
представления $T(g)$ в $\HH$ и $T'(g)$ в $\HH'$ {\em унитарно
эквивалентны}, если существует унитарный изоморфизм $U:\; \HH\to
\HH'$, такой что $UT(g)=T'(g)U$.
\par
Важной характеристикой представления является {\em приводимость}.
Подпространство $\HH_1\subset\HH$ называется {\em инвариантным},
если из $\phi\in \HH_1$ следует $T(g)\phi \in \HH_1$ для всех
$g\in G$. Для бесконечномерных представлений будем рассматривать
лишь замкнутые подпространства. Тривиальными подпространствами
являются само пространство представления, и нулевое
подпространство. \begin{Def} Представление назвается неприводимым
, если оно содержит лишь тривиальные инвариантные подпространства.
В противном случае представление  приводимо.\end{Def} С каждым
приводимым представлением можно связать два новых представления:
одно является просто сужением на $\HH_1$, второе строится в
фактор-пространстве $\HH_1/\HH$. Если приводимое представление
можно представить в виде прямой суммы неприводимых представлений,
то оно называется {\em вполне приводимым}. Не всякое приводимое
представление вполне приводимо. В случае вполне приводимого
представления можно выбрать базис так, что матрица представления
будет блочно-диагональна (в бесконечномерном случае на диагонали
может быть бесконечно много матриц). Доказывается, что все
конечномерные унитарные представления вполне приводимы ([21], стр.
32).
\par
 Критерий неприводимости конечномерного представления дает
 следующее утверждение, известное как (первая) лемма Шура:
\begin{Trm} Если $T$ --- неприводимое представление в конечномерном
пространстве $\HH$, то единственными операторами, коммутирующими
со всеми $T(g)$, являются операторы, кратные единичному.
\end{Trm}
\noindent Прямым следствием леммы Шура является то, что
неприводимые конечномерные представления абелевых (коммутативных)
групп одномерны. Действительно, для таких групп
$T(g)\,T(h)=T(h)\,T(g)$, так что при фиксированном  $h$ оператор
$T(h)$   перестановочен со всеми операторами  $T(g)$, и  в силу
неприводимости $T(g)$ будем иметь  $T(h)=\lambda(h) I$, где  $I$
единичный оператор. Но такое представление одномерно. Нетрудно
также показать, что любое приводимое представление абелевых групп
является вполне приводимым и разлагается в прямую сумму одномерных
представлений. Соответствующие матрицы диагональны.
\par
Другое важное следствие леммы Шура состоит в том, что операторы
Казимира в любом неприводимом представлении группы  кратны
единичному оператору.
\par
{\bf Пример.} Оператор Казимира $C_2$ группы $SO(3)$ во всех
конечномерных представлениях на $L_2(S^2)$ кратен единичному:
$C_2=L_1^2+L_2^2+L_3^2=(l+1)l I $, где  $l$ - целое число. Это
последнее полностью задает неприводимое представление, базисными
векторами которого являются шаровые функции
$Y_{lm}(\theta,\phi),\, |m|\leq l$. Оператор $C_2$ является
инвариантным оператором на однородном пространстве $S^2$, на
котором группа $SO(3)$ действует транзитивно.
\par
Можно сформулировать и обратное утверждение: если любой оператор
$A$, коммутирующий со всеми операторами некоторого унитарного
представления группы, кратен единичному, то представление
неприводимо.
\par
Для сильно непрерывных представлений компактных групп имеют место
утверждения ([4], гл.7):\begin{Trm}Любое представление компактной
группы ограниченными операторами в гильбертовом пространстве
унитарно относительно некоторого скалярного произведения в
пространстве представления.\end{Trm}\begin{Trm}Каждое неприводимое
унитарное представление компактной группы конечномерно.
\end{Trm}\noindent С учетом полной приводимости конечномерных
унитарных представлений это означает, что все конечномерные
представления компактных групп вполне приводимы. Аналогичное
утверждение верно и для бесконечномерных унитарных представлений
компактных групп: пространство такого представления разлагается в
прямую сумму конечномерных инвариантных подпространств, в которых
индуцируются неприводимые представления группы.
\par
Для групп более общего типа  справедливы следующие утверждения:
\begin{Trm}
Всякое конечномерное неприводимое представление связной
топологической разрешимой группы одномерно.
\end{Trm}
\begin{Trm}
Связная простая некомпактная группа Ли не допускает нетривиальных
конечномерных унитарных представлений.
\end{Trm}
Для полупростых групп последнее утверждение не имеет места
(например некомпактная компонента прямого произведения может иметь
тривиальное представление, а компактная --- конечномерное
унитарное представление). Однако справедлива
\begin{Trm}
Связная полупростая некомпактная группа Ли не имеет точного
унитарного конечномерного представления.
\end{Trm}
По этой причине в релятивистской квантовой теории приходится
рассматривать бесконечномерные унитарные представления простой
группы $SO(1,3)$ (группы Лоренца).
\par
В квантовой механике приходится иметь дело с неограниченными
операторами (см. разд. 3 осн. текста), в этом случае данные выше
простые определения нуждаются в уточнении.
\begin{Def}
Представление алгебры Ли $\LL$  в гильбертовом пространстве $\HH$
--- это любой гомоморфизм из $\LL$  в множество линейных
операторов, действующих  в $\HH$, имеющих общую плотную
инвариантную область определения.
\end{Def}
Любое такое представление может быть расширено до представления
обертывающей алгебры. Стандартное построение общей инвариантной
плотной области определения для представления алгебры Ли,
называемой {\em подпространством Гординга}, описано в [4], гл. 11.
В некоторых случаях общая плотная область определения может быть
построена иначе.
\par Другое важное требование, вытекающее из общих принципов
квантовой механики, состоит в том, что операторы, реализующие
унитарное представление, должны быть самосопряженными. Имеет место
утверждение:
\begin{Trm}
Пусть  $T$ --- унитарное представление произвольной группы Ли, $X$
--- произвольный элемент соответствующей алгебры Ли, и  $p(X)$  ---
любой вещественный полином. Тогда оператор $T(p(iX))$  существенно
самосопряжен на подпространстве Гординга, в частности, $T(iX)$
существенно самосопряжен.
\end{Trm}
Представление алгебры интегрируется до представления группы при
некоторых дополнительных предположениях ([4], гл. 11, теор. 5). В
случае конечномерных представлений это всегда возможно и приводит
к однозначному результату.
\section{Список литературы.}
%\begin{thebibliography}{99}
\begin{enumerate}
\item {\it  Л.Д. Ландау, Е.М. Лифшиц}, Квантовая механика. М.:
Наука, 1974. \item {\it В.И. Арнольд}, Математические методы
классической механики. М.: Наука, 1979. \item {\it М. Рид, Б.
Саймон}, Методы современной математической физики. М.:Мир, 1977.
\item {\it А. Барут, Р. Рончка}, Теория представлений групп и ее
приложения. М.:Мир, 1980. \item {\it К. Морен}, Методы
гильбертовых пространств.  М.:Мир, 1965. \item {\it Ф.А. Березин,
М.А. Шубин}, Лекции по квантовой механике. М., 1972. \item {\it
Ю.М. Березанский,} Разложение по собственным функциям
самосопряженных операторов. Киев, ``Наукова думка'', 1965. \item
{\it  Ф.А. Березин,} Лекции по статистической физике. М., 1972.
\item {\it C. Muller}, Spherical Harmonics. Lecture Notes in
Mathematics, V. 17. 1966. \item {\it  А.М. Переломов}, Обобщенные
когерентные состояния. М.:Наука, 1987. \item {\it  В.П. Маслов,
М.В. Федорюк}, Квазиклассическое приближение для уравнений
квантовой механики. М.:Наука, 1976. \item {\it  Ф.А. Березин, М.А.
Шубин}, Уравнение Шредингера. М.: Изд. МГУ, 1983.
\item Физическая энциклопедия. М.:
Сов. энциклопедия, 1990. \item {\it  А.М. Савчук,  А.А. Шкаликов},
Операторы Штурма--Лиувилля с потенциалами-распределениями// Труды
Моск. Мат. Об-ва. 2003. Т. 64. сс. 159--212. \item {\it  А.М.
Савчук, А.А. Шкаликов},  Операторы Штурма--Лиувилля с сингулярными
потенциалами // Мат. заметки. 1999. Т. 66, вып. 6. сс. 897--912.
\item {\it  Р. Рихтмайер}, Принципы современной математической
физики. М., ``Мир'', 1982. \item {\it Дж. Макки}, Лекции по
математическим основам квантовой механики. М.:Мир, 1965. \item
{\it Б.А. Дубровин, С.П. Новиков, А.Т. Фоменко,} Современная
геометрия: методы и приложения. М.:УРСС, 1998. \item {\it Э.М.
Галеев, В.М. Тихомиров,} Краткий курс теории экстремальных задач.
Изд-во Моск. ун-та, 1989. \item {\it Г.Е. Шилов,} Математический
анализ. Специальный курс. М.:Физматгиз, 1961. \item {\it  Н.Я.
Виленкин,} Специальные функции и теория представлений групп. М.:
Наука, 1965. \item {\it Г.Г. Харди, Дж. Е. Литтльвуд, Г. Полиа,}
Неравенства. М.: ИЛ, 1948.
\item {\it Д.В. Гальцов,} Теоретическая физика. М.: Изд-во Моск.
ун-та, 2003.
\item {\it В.С. Владимиров,} Уравнения математической физики. М.:
``Наука'', 1985.
\end{enumerate}
%\end{thebibliography}
\end{document}
