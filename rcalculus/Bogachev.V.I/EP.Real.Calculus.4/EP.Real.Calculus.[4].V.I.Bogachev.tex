\documentclass[a4paper]{article}
\usepackage[utf,simple]{dmvn}

\title{Программа экзамена по действительному анализу}
\author{Лектор~--- В.\,И.\,Богачёв}
\date{IV семестр, 2004 г.}

\begin{document}
\maketitle

\begin{nums}{-2}
\item Кольца, алгебры и $\si$-алгебры множеств, существование $\si$-алгебры,
      порождённой классом множеств. Структура открытых множеств на прямой.
      Борелевская $\si$-алгебра.
\item Функции, измеримые относительно $\si$-алгебры. Свойства измеримых функций.
\item Аддитивные и счётно-аддитивные меры. Счётная полуаддитивность. Критерий счётной аддитивности.
\item Компактные классы. Счётная аддитивность меры, приближающей компактный класс.
\item Внешняя мера. Определение измеримого множества. Теорема Лебега о
      счётной аддитивности внешней меры на $\si$-алгебре измеримых
      множеств. Единственность продолжения.
\item Построение меры Лебега на прямой и в $\R^n$, основные свойства меры Лебега.
\item Сходимость почти всюду, теорема Егорова.
\item Сходимость по мере и неё связь с сходимостью почти всюду.
      Фундаментальность по мере. Теорема Рисса.
\item Теорема Лузина.
\item Интеграл Лебега для простых функций и его свойства.
\item Общее определение интеграла Лебега. Корректность определения.
\item Основные свойства интеграла Лебега (линейность, монотонность).
\item Неравенство Чебышева. Критерий интегрируемости $f$ в терминах множеств $\bc{x\cln |f| \ge n}$.
\item Теорема Лебега о мажорируемой сходимости. Теорема Беппо~Леви. Теорема Фату.
\item Связь интеграла Лебега с интегралом Римана (собственным и несобственным).
\item Неравенство Гёльдера. Неравенство Минковского.
\item Пространства $L^p(\mu)$ и их полнота. Связь различных видов сходимости измеримых функций.
\item Теорема Радона~-- Никодима.
\item Произведение пространств с мерами. Теорема Фубини.
\item Свёртка интегрируемых функций.
\item Функции ограниченной вариации. Абсолютно непрерывные функции.
      Абсолютная непрерывность первообразной. Связь абсолютно непрерывных
      функций с первообразными интегрируемых функций (без доказательства).
      Формула Ньютона~-- Лейбница и формула интегрирования по частям
      для абсолютно непрерывных функций.
\end{nums}

\medskip
\dmvntrail

\end{document}
