\documentclass[a4paper]{article}
\usepackage[utf,simple]{dmvn}

\title{Программа экзамена по действительному анализу}
\author{Лектор~--- М.\,И.\,Дьяченко}
\date{IV семестр, 2005 г.}

\begin{document}
\maketitle


\begin{nums}{-1}
\item Системы множеств (полукольца, кольца, алгебры, $\si$ -- алгебры и т.д.)
Минимальные кольца и их свойства. Связь между $\si$ -- кольцами и $\de$ -- кольцами.
\item Меры на полукольцах. Классическая мера Лебега на полукольце промежутков в
$\R^n$ и её $\si$ -- аддитивность.
\item Продолжение меры с полукольца на минимальное кольцо.
\item Внешние меры Лебега и Жордана. Их полуаддитивность.
\item Меры Лебега и Жордана. Их свойства.
\item Связь $\sigma $ -- аддитивности и непрерывности. Полнота мер. Мера Бореля.
\item Меры Лебега--Стилтьеса на прямой.
\item $\sigma $ -- конечные меры.
\item Теорема о структуре измеримых множеств. Теорема о структуре открытых множеств на прямой (б/д). Теорема Витали (б/д).
\item Измеримые функции. Их арифметические свойства. Суперпозиции измеримых функций.
\item Измеримые функции и предельный переход. Теорема об измеримости производной непрерывной функции.
\item Сходимость по мере и её свойства.
\item Сходимость почти всюду. Критерий этой сходимости на множествах конечной меры.
\item Связь между сходимостью по мере и сходимостью почти всюду.
\item Теорема Егорова.
\item Теорема Лузина.
\item Интеграл Лебега для простых функций и его свойства.
\item Определение интеграла Лебега в общем случае. Линейность интеграла Лебега
по функции и по множеству для неотрицательных функций.
\item Линейность интеграла Лебега по функции в общем случае. Интегрирование неравенств.
\item Теорема Леви о предельном переходе и её следствия.
\item Теорема Фату и Лебега.
\item Абсолютная непрерывность интеграла Лебега.
\item Критерий интегрируемости по Лебегу на множестве конечной меры. Неравенство Чебышева.
\item Связь между интегралами Римана и Лебега на отрезке в $\R^n$.
\item Заряды. Разложение Хана и Жордана.
\item Теорема Радона-- Никодима.
\item Неравенство Гельдера и Минковского. Пространство $L_p$.
\item Полнота пространств $L_p$.
\item Представление интеграла от $p$-- й степени функции $p\in [1;\bes)$ с помощью функции распределения.
\item Теорема о плотных множествах функций в пространствах $L_p$, $p\in[1; \bes)$.
\item Абсолютно непрерывные функции и их свойства. Теорема Банаха -- Зарецкого (без доказательства достаточности).
\item Теорема о дифференцировании интеграла Лебега по переменному верхнему пределу.
\item Восстановление абсолютно непрерывной функции по её производной с помощью
интеграла Лебега. Замена переменной и интегрирование по частям в интеграле Лебега.
\item Прямые произведения мер. Теорема Фубини (б/д).
\end{nums}

\medskip\dmvntrail
\end{document}
