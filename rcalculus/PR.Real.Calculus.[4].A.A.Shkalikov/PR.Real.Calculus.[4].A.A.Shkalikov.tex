\documentclass[a4paper]{article}
\usepackage[simple]{dmvn}

\title{Задачи, рекомендованные для разбора на семинарских\\
занятиях по курсу <<Действительный анализ>>}
\author{Лектор\т А.\,А.\,Шкаликов}
\date{I поток, IV семестр, 2005--2006 г.}

\begin{document}
\maketitle

\dmvntrail

\section{Мера Лебега}

\begin{enumerate}
\setlength{\itemsep}{-3pt}
\item
Доказать включение
$$
A\swo B\subset (A\swo C)\cup (C\swo B).
$$
\item
Доказать, что любой элемент наименьшего кольца
$\mathcal{R}(\mathcal{S})$, порождённого полукольцом
$\mathcal{S}$, представим в виде
$$
A=\bigsqcup\limits_{k=1}^n A_k,\quad A\in
\mathcal{R}(\mathcal{S}),\quad A_k\in \mathcal{S}.
$$
\item
Доказать, что мера $\mu$, заданная на полукольце $\mathcal{S}$,
однозначно продолжается на кольцо $\mathcal{R}(\mathcal{S})$. При
этом $\mu$ является $\sigma$-аддитивной на
$\mathcal{R}(\mathcal{S})$, если $\mu$ $\sigma$-аддитивна на
$\mathcal{S}$.
\item
Доказать, что функция на полукольце интервалов $[a, b)\subset \R$
$\mu ([a, b))=F(b)-F(a)$ задаёт $\sigma$-конечную меру, если
$F(t)$ --- непрерывная слева неубывающая функция.
\item
Доказать, что мера в задаче 4 является $\sigma$-аддитивной тогда и
только тогда, когда $F(t)$ непрерывна слева.
\item
Доказать, что $A$ измеримо по Лебегу тогда и только тогда, когда
$$
\mu^*(A)+\mu^*(E\wo A)=1
$$
(здесь $E$ --- единица, и $\mu(E)$ = 1).
\item
Доказать, что произвольное ограниченное измеримое множество в
$\R^n$ представимо в виде объединения борелевского множества и
множества меры нуль.
\item
Доказать, что непрерывная функция в $\R^n$ измерима. То же для
монотонной функции.
\item
Доказать, что непрерывная почти всюду функция в $\R^n$ измерима.
\item
Построить нетривиальную меру в $\R^1$, относительно которой все
подмножества в $\R^1$ измеримы.
\item
Доказать, что ограниченное множество $A\subset \R^n$ измеримо
тогда и только тогда, когда найдутся замкнутое и открытое
множества $F_\ep$, и $G_\ep$, такие, что $F_\ep\subs A\subs G_\ep$
и $\mu (G_\ep\wo\F_\ep)<\ep$

\end{enumerate}

\section{Интеграл Лебега}

\begin{enumerate}
\setcounter{enumi}{11}\setlength{\itemsep}{-3pt}
\item
Доказать, что в случае $f(x)\ge 0$ все нижеприведенные определения
эквивалентны:
\begin{nums}{-3}
\item[1.]
Пусть $f(x)$ измерима, $f_n(x):=\frac{k-1}{2^n}$, если $x\in
A_{km}=\left\{y \vl \frac{k-1}{2^n}\le f(y)\le
\frac{k}{2^n}\right\}$, $k=1,2,\ldots $. $f(x)$ называется
суммируемой, если $f_0(x)$ суммируема. Тогда $f_n(x)$ также
суммируемы и $I(f_n)\nearrow I(f)$.
\item[2.]
$f$ суммируема, если существует последовательность суммируемых
простых функций $f_n$, сходящихся к $f(x)$ равномерно. При этом
$I(f_n)\to I(f)$ и не зависит от выбора такой последовательности
$f_n$.
\item[3.]
$f$ суммируема, если $f$ измерима и
$$
I(f):=\sup\limits_{g(x)\le f(x)}I(g)<\infty
$$
где $\sup$ берётся по простым функциям, принимающим конечное число
значений.
\item[4.]
$f$ суммируема если $f$ измерима и
$$
I(f):=\int\limits_0^{\infty}\mu(G_f(t))dt<\infty,
$$
где $G_f:=\{x\vl f(x)\le t\}$. Здесь подразумевается несобственный
интеграл Римана, который корректно определён, так как мера
множества $G_f(t)$ --— монотонная функция от $I$.
\end{nums}
\item
Доказать равенство
$$
(LS)\int\limits_a^bf d\mu(x)=(L)\int\limits_a^b f\mu 'dx
$$
\item
При каких $a$ функция $x^{-a}$ принадлежит $L_p(0, 1)$, $p\ge 1$?
\item
При каких $\alpha$, $\beta$ функция $f(x)=\dfrac{\sin
x^\beta}{x^\alpha}$
\begin{nums}{-3}
\item[a)]
интегрируема по Риману на $(0,1)$ --- возможно, в несобственном
смысле?
\item[б)]
интегрируема по Лебегу (т. е. $\in L_1(0, 1)$)?
\end{nums}
\item
Привести пример ограниченной функции, интегрируемой по Лебегу, но
не интегрируемой по Риману. Привести пример функции, интегрируемой
по Риману в несобственном смысле, но не интегрируемой по Лебегу.
Возможен ли такой пример для неотрицательной функции?
\item
Доказать строгое включение $L_{p_1}(0, 1)\subs L_p(0, 1)$ при
$p_1>p$. Доказать, что $L_{p_1}(\R )$ не вложено в $L_{p}(\R )$ ни
при каких $p\neq p_1$.
\item
Показать, что в условиях теоремы Фату предельный переход под
знаком интеграла гарантировать нельзя.
\item
Следует ли из суммируемости $f(x)$ суммируемость $|f(x)|$?
Наоборот, следует ли из суммируемости $|f(x)|$ суммируемость
$f(x)$?
\end{enumerate}

\section{Сходимость}

\begin{enumerate}
\setcounter{enumi}{19}\setlength{\itemsep}{-3pt}
\item
Привести пример последовательности функций $\{f_n\}$, сходящейся
по мере, но не сходящейся почти всюду.
\item Выяснить связь между сходимостями
\begin{items}{-3}
\item
в $L_1(0, 1)$
\item
в $L_2(0, 1)$
\item
по мере
\item
почти всюду
\end{items}
(связь между сходимостями по мере и почти всюду прояснена на
лекциях). Построить последовательность функций, сходящуюся в
$L_1(\R )$, по не сходящуюся в $L_2(\R)$, и наоборот.
\item
Доказать, что в пространстве измеримых функций нельзя ввести
метрику, сходимость в которой эквивалентна сходимости почти всюду.
\end{enumerate}

\section{$AC$ и $BV$ функции:}
\begin{enumerate}
\setcounter{enumi}{22}\setlength{\itemsep}{-3pt}
\item
Найти полную вариацию функций $f(x) = \cos x$ на $[0,4\pi]$,
$f(x)=x-x^2$  на $[0,1]$.
\item
Доказать, что канторовская лестница --- непрерывная, но не
абсолютно непрерывная функция.
\item
При каких а функция $\al$ функция $x^\al\sin \frac{1}{x}$ является
функцией ограниченной вариации на $(0,1)$?
\item
Какая связь между $AC-$ и $BV-$ классами?
\item
Пусть $f\in BV(0, 1)$. Пояснить, почему существует $f'(x)$, причём
$f'(x)\in L_1(0, 1)$. Доказать
$$
\int f'(t)dt\le f(x)-f(0).
$$
\end{enumerate}

\end{document}
