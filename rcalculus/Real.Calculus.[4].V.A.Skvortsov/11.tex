
% 11 лекция по действлану.

\begin{center} \textbf{Лекция 11.} \end{center}

\begin{center} \textbf{Сравнение интеграла Римана--Стилтьеса
с интегралом Лебега--Стильтьеса} \end{center}

$(RS) \int_a^b f d g$

$a = x_0 < x_1 < x_2 < \ldots < x_{i-1} < x_i < x_i < \ldots < x_n
= b$

$P$ --- разбиение отрезка

$M_i = sup_{x \in [x_{i-1}, x_i]} f(x), \; m_i = inf_{x \in
[x_{i-1}, x_i]} f(x)$

$g$ берем непрерывной слева.

\textbf{Верхняя сумма Дарбу.} \quad $U(f, g, P) = \sum_{i=1}^n M_i
g(\triangle_i)$

\textbf{Нижняя сумма Дарбу.} \quad $L(f, g, P) = \sum_{i=1}^n m_i
g(\triangle_i)$

Если $\inf U = \sup L$, то $\exists$ интеграл Римана--Стильтьеса,
равный этой величине.

Если $\exists \: \int_a^b f d g$, то можно взять
последовательность разбиений $P_k, P_k \subset P_{k+1}$
 и получить интеграл как $\sigma(P_k) \rightarrow 0 $ предел при $k
\rightarrow \infty$ сумм Дарбу.

\textbf{Теорема.} \quad В случае существования интеграла
Римана--Стильтьеса $(LS) \int_a^b f d \mu_g$

%
%
%
%
%
%
%%%%%%%%%%%%%%%%%%%%%%%%%%%%%%%%%%%%%%%%%%%%%%%%%%%%%%%%%%%%%%%%%%
%%%%%%%%%%%%%%%%%%%%%%%%%%%%%%%%%%%%%%%%%%%%%%%%%%%%%%%%%%%%%%%%%%
%%%%%%%%%%%%%%%%%          стр 20.0    %%%%%%%%%%%%%%%%%%%%%%%%%%%%
%%%%%%%%%%%%%%%%%%%%%%%%%%%%%%%%%%%%%%%%%%%%%%%%%%%%%%%%%%%%%%%%%%
%%%%%%%%%%%%%%%%%%%%%%%%%%%%%%%%%%%%%%%%%%%%%%%%%%%%%%%%%%%%%%%%%%
%
%
%
%
%
\textbf{Доказательство.} \quad

$U_k(x) = M_i, \; x \in [x_{i-1}, x_i)$

$L_k(x) = m_i, \; x \in [x_{i-1}, x_i)$

$U(f, g, P_k) = (LS) \int_a^b U_k d \mu_g \rightarrow (RS)
\int_a^b f d f$

$L(f, g, P_k) = (LS) \int_a^b L_k d \mu_g\nearrow$

$U_k \searrow U$

$L_k \nearrow L$

$U_k (x) \geqslant U(x) \geqslant (x) \geqslant f(x) \geqslant
L(x) \geqslant L_k(x)$

По теореме Б. Леви можно переходить к пределу под знаком интеграла
Лебега.

$(LS) \int_a^b U d \mu_g = (LS) \int_a^b L d \mu_g $

$(LS) \int_a^b (U - L)d \mu_g = 0 \Rightarrow U - L = 0$ $\mu_g$
$\Rightarrow$ $U(x) = f(x) = L(x)$

$(LS) \int_a^b f d \mu_g = (LS)\int_a^b U d \mu_g = (LS) \int_a^b
\mu_g = (RS) \int_a^b f d g$

Теорема будет верна для $g \in VB$ (педст. в виде разности двух
монотонных функций)

Задача (*). Интеграл не зависит от представления $g$ (но будем
ист. представление \={V} - \b{V})

\begin{center} \textbf{Теорема Фубини} \end{center}

$(X, S, \mu_x)  \quad (Y, T, \mu_y)$

$X \times Y$

$A \in S, B \in T \quad C = A \times B$

$\mu(C) = ^{\!\!\!\!\!\! def} \mu_x A \times \mu_y B$

Задача $C = A \times B, A \in S, B \in T, \{C\}$ образует
полукольцо.

Мера определена на полукольце $\rightarrow$ определим её на
минимальном кольце $\rightarrow$ с помощью конструкции Лебега
определим $\mu^* \rightarrow$ с помощью определения Каратеодори
определим класс измеримых множеств.

Получим меру, называемую производной мер $\mu_x$ и $\mu_y$ на$X
\times Y, \; \mu = \mu_x \times \mu_y$


%
%
%
%
%
%
%%%%%%%%%%%%%%%%%%%%%%%%%%%%%%%%%%%%%%%%%%%%%%%%%%%%%%%%%%%%%%%%%%
%%%%%%%%%%%%%%%%%%%%%%%%%%%%%%%%%%%%%%%%%%%%%%%%%%%%%%%%%%%%%%%%%%
%%%%%%%%%%%%%%%%%          стр 20.1    %%%%%%%%%%%%%%%%%%%%%%%%%%%%
%%%%%%%%%%%%%%%%%%%%%%%%%%%%%%%%%%%%%%%%%%%%%%%%%%%%%%%%%%%%%%%%%%
%%%%%%%%%%%%%%%%%%%%%%%%%%%%%%%%%%%%%%%%%%%%%%%%%%%%%%%%%%%%%%%%%%
%
%
%
%
%

Проверим $\sigma$--аддитивность меры на полукольце $C =
\bigcup_{k=1}^\infty, \quad C_k = A_k \times B_k$

$\mu C = \mu_x A \times \mu_y B = \int_x \chi_A(x) \mu_y B d \mu_x
= \int_x \sum_k \chi_{A_k} \mu_y B_k d \mu_x = \sum_k \mu_x A_k
\mu_y B_k$

$f(x,y)$ определена на $X \times Y$ --- измерим., неотриц.

Тогда $\int_{X \times Y} f(x,y) d \mu = \int_X( \int_Y f(x,y) d
\mu_y) d \mu_x$, если имеет смысл левая часть.

Считаем: $f(x,y)$ измерима при фиксированном $x$ на $Y$ для п.в.
$y$

Тогда $F(x) = \int_Y f(x,y) d \mu_y$ имеет смысл и предполагается
измеримой.

Сначала докажем теорему для характеристической функции
$\chi_A(x,y)$ измеримого множества относительно $\mu = \mu_x
\times \mu_y$.

Сначала рассмотрим $A = A' \times A"$.

Для такой $\chi_A$ --- очевидно. ($\chi_{\text{элемента кольца}} =
\sum \chi_{\text{элемента полукольца}}$)


\textbf{Лемма.} \quad $\forall$ измеримого $A \; \exists B_{n_k}$
--- последовательность элементов кольца, $B_{n_k} \uparrow B_n, \; B_{n_k} \downarrow B, \; n
\rightarrow \infty$ и $\mu B = \mu A, \; B supset A$

\textbf{Доказательство.} \quad

$\mu A = \inf_{\bigcup P_i \supset A} \sum_{i=1}^\infty \mu P_i$

$\mu A + 1/n > \sum_{i=1}^\infty \mu P_i \geqslant \mu (\bigcup_i
P_i)$

Хотим, чтобы $B_n \supset B_{n+1}$

 $(\bigcup_i P_i) \bigcap (\bigcup_k P_k') =
\bigcup_{i,k}(P_i \bigcap P_k')$

Пользуясь этим, получим $\mu A + 1/n > \mu(B_n)$

$\lim_{n \rightarrow \infty} \mu B_n = \mu(\bigcap_{n=1}^\infty
D_n) = \mu B = \mu A$

$B_{n_k} = \bigcup_{i=1}^{k} P_i$

%
%
%
%
%
%
%%%%%%%%%%%%%%%%%%%%%%%%%%%%%%%%%%%%%%%%%%%%%%%%%%%%%%%%%%%%%%%%%%
%%%%%%%%%%%%%%%%%%%%%%%%%%%%%%%%%%%%%%%%%%%%%%%%%%%%%%%%%%%%%%%%%%
%%%%%%%%%%%%%%%%%          стр 21.0    %%%%%%%%%%%%%%%%%%%%%%%%%%%%
%%%%%%%%%%%%%%%%%%%%%%%%%%%%%%%%%%%%%%%%%%%%%%%%%%%%%%%%%%%%%%%%%%
%%%%%%%%%%%%%%%%%%%%%%%%%%%%%%%%%%%%%%%%%%%%%%%%%%%%%%%%%%%%%%%%%%
%
%
%
%
%

Итак, для $\chi_{B_{n_k}}$ теорема Фубини верна.

$A_x = \{y: (x,y) \in A\}$

$\int_{X \times Y} \chi_{B_n} (x,y) d \mu = $ $\lim_{k \rightarrow
\infty} \int_{X \times Y} \chi_{B_{n_k}} (x,y) d \mu = \lim_{k
\rightarrow \infty} \\ \int_X \underbrace{(\int_Y \chi_{B_{n_k}}
(x,y) d \mu_y)}_{\text{монот. посл. функций}} d \mu_x = $
(Применяем теорему Б. Леви)  $= \int_X (\lim_{k \rightarrow
\infty} \int_Y \chi_{B_{n_k}} (x,y)  \mu_y) d \mu_x$

$\int_{X \times Y} \chi_{B_n} (x,y) d \mu = \int_X \lim_{k
\rightarrow \infty} \mu_y (B_{n_k}) f \mu_x = \int_X \mu_y (B_n)_x
d \mu_x = \int_X (\int_Y \chi_{B_n} (x,y) d \mu_y) d \mu_x$

Докажем для $\chi_{B_n}$. Аналогичен переход $\chi_{B_n}
\rightarrow \chi_B$

$A = B \setminus(B \setminus A)$

$\chi_A = \chi_B - \chi_{B \setminus A}, \quad \mu(B \setminus A)
= 0$

Докажем теорему Фубини для любого измеримого множества меры 0.

$C \subset B', \mu C = \mu B' = 0$ ( аналогично $B'$ - изм.
оболочка)

Для $B'$ теорема доказана.

$\int_{X \times Y} \chi_C (x,y) d \mu = \int_{X \times Y}
\chi_{B'} d \mu = \int_X (\int_Y \chi_{B'} d \mu_y) d \mu_x = 0
\Rightarrow \int_Y \chi_{B'} d \mu_y = 0$ $x$

$\mu_y B_x ' = 0$ $\Rightarrow \mu C_x = 0$

$\int_{X \times Y} \chi_C (x, y) d \mu = 0$

$\int_X(\int_Y \chi_C (x,y) d \mu_y) d \mu_x = 0$
