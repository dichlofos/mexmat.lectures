

\begin{center} \textbf{Лекция 7.} \end{center}


%
%
%
%
%
%
%
%%%%%%%%%%%%%%%%%%%%%%%%%%%%%%%%%%%%%%%%%%%%%%%%%%%%%%%%%%%%%%%%%%
%%%%%%%%%%%%%%%%%%%%%%%%%%%%%%%%%%%%%%%%%%%%%%%%%%%%%%%%%%%%%%%%%%
%%%%%%%%%%%%%%%%%          стр 12.1    %%%%%%%%%%%%%%%%%%%%%%%%%%%%
%%%%%%%%%%%%%%%%%%%%%%%%%%%%%%%%%%%%%%%%%%%%%%%%%%%%%%%%%%%%%%%%%%
%%%%%%%%%%%%%%%%%%%%%%%%%%%%%%%%%%%%%%%%%%%%%%%%%%%%%%%%%%%%%%%%%%
%
%
%
%
%

\textbf{Определение.} \quad $f \in N[a, b]$ интегрируема по
Ньютону, если существует $F(x) : F'(x) = f(x)$

$(N) \int_a^b f dx = F(b) - F(a)$

Интеграл Лебега не покрывает интеграл Ньютона.

%%%%%%%%%%%%%%%%%%%%%%%%%%%%%%%%%%%%%%%%%%%%%%%%%%%%%%%%%%%%%%%%
%%%%%%%%%%%%%%%%%%%%%%%%%%%%%%%%%%%%%%%%%%%%%%%%%%%%%%%%%%%%%%%%
%    картинка




\textbf{Теорема.} \quad Неопределенный интеграл Лебега дифф. п.в.
$F(x) = (L) \int_a^x f d\mu \Rightarrow F'(x) = f(x)$ (п.в.)

$f \in L[a,b]$

$L (X, \EuScript{M}, \mu)$ --- пространство Лебега, состоящее из
классов эквивалентных функций.

$f \thicksim g \Leftrightarrow \mu\{x \in X: \: f(x) \neq g(x)\} =
0$

$\rho (f, g) = 0 \Leftrightarrow f = g$

$\rho (f,g) = \in_x |f - g| d \mu $ --- метрика $\Rightarrow L(X,
\EuScript{M}, \mu)$

$\chi \in \EuScript{L}$

$\int_x |f - f^N| d \mu \xrightarrow[N \rightarrow \infty]{} 0$,
то есть каждая измеримая функция может быть с любой точностью
приближена ограниченной функцией (по метрике).

$\forall \varepsilon > 0 \; \forall f \in \EuScript{L} \;\;
\exists |g_\varepsilon (x)| < A: \: \rho(f, g_\varepsilon) <
\varepsilon$

Если верна теорема Лузина, то ограниченную функцию можно
приближать непрерывными, ограниченными той же константой. То есть
$\exists \varphi$ непрерывная, $\mu \{g_\varepsilon \neq \varphi\}
< \varepsilon/A, $ $|\varphi(x)| < A$

Докажем для отрезка: $\int_a^b |Ђg_\varepsilon(x) - \varphi(x)| d
\mu < 2A\varepsilon / A = 2\varepsilon$

$\forall f \in \EuScript{L} \quad \exists \varphi$ $\rho(f,
\varphi) < 3\varepsilon$


\textbf{Определение.} \quad $f_n \xrightarrow[]{L} f
\Leftrightarrow \rho(f_n, f) \xrightarrow[]{n \rightarrow \infty}$
(сходимость в метрике)

%
%
%
%
%
%
%
%%%%%%%%%%%%%%%%%%%%%%%%%%%%%%%%%%%%%%%%%%%%%%%%%%%%%%%%%%%%%%%%%%
%%%%%%%%%%%%%%%%%%%%%%%%%%%%%%%%%%%%%%%%%%%%%%%%%%%%%%%%%%%%%%%%%%
%%%%%%%%%%%%%%%%%          стр 13.0    %%%%%%%%%%%%%%%%%%%%%%%%%%%%
%%%%%%%%%%%%%%%%%%%%%%%%%%%%%%%%%%%%%%%%%%%%%%%%%%%%%%%%%%%%%%%%%%
%%%%%%%%%%%%%%%%%%%%%%%%%%%%%%%%%%%%%%%%%%%%%%%%%%%%%%%%%%%%%%%%%%
%
%
%
%
%



\textbf{Теорема. (неравенство Чебышева)} \quad

$f \in L(E) \Rightarrow \mu \{x \in E: \; |f(x)|  \geqslant C\}
\leqslant \frac{1}{c} (L) \int_E |f| d \mu$

$E_c = {x \in E: |f(x) \geqslant C|}$

$c \mu E_c = \int_{E_c} c d\mu \leqslant \int_{E_0} |f| d \mu
\leqslant \int_E |f| d\mu$

\textbf{Утверждение.} \quad $f_n \in L(E), \int_E |f - f_n| d \mu
\xrightarrow[n \rightarrow \infty]{} 0 \Rightarrow f_n
\xrightarrow[]{\mu} f$

\textbf{Доказательство.} \quad $\forall \varepsilon > 0 \;
\mu\{x\in E: |f(x) - f_n(x)| \geqslant \varepsilon\} \leqslant
\frac{1}{\varepsilon} \int_E |f- f_n|d\mu \xrightarrow[n
\rightarrow \infty]{} 0$

\textbf{Задача.} \quad Связь сходимости в метрике с другими видами
сходимости

\textbf{Задача.} \quad $\mu \{x \in E: |f(x) \geqslant C|\} =
\b{O}(\frac{1}{c}), c \rightarrow \infty$

Если $f(x)$ суммируема на $E$, то $\mu \{x \in E: |f(x) \geqslant
C|\} = $ \b{O} $ (\frac{1}{c}), c \rightarrow \infty$

Пример измеримой несуммируемой функции, для которой это выполнено.


$L(X, \EuScript{M}, \mu)$ --- нормированное пространство с нормой
$\|f\| = \int_E |f| d\mu$, при условии, что элементы пространства
--- классы эквивалентности (иначе нет условия $\|x\| = 0 \Leftrightarrow x =
0$, 0 в нашем пространстве соответствует классу
эквивалентности)

Можно рассматривать $L^p$ --- пространство функций, интегрируемых
р раз по Лебегу.

$f \in L(E) \Rightarrow f \in L(E') \; \forall E' \subset E, E'
\in \EuScript{M}$

$\varphi(E) = \int_{E_1} f d \mu$ --- неопределенный интеграл
Лебега

$\varphi$ --- аддитивная функция

\textbf{Определение.} \quad $(X, \EuScript{M}), \; \EuScript{M} -
\delta$-алгебра

$\varphi$ --- аддитивная функция множества, если

$\varphi: \EuScript{M} \rightarrow \mathbb{R}$ и является
 $\delta$--аддитивной : $\varphi(\bigcup_{k=1}^{\infty} E_k) =
\sum_{k=1}^{\infty}\varphi(E_k)$

$E_1 \subset E_2 \subset \ldots \subset E_k \subset_{E_{k+1}}
\subset \ldots$

$\varphi(\bigcup_{k=1}^\infty E_k) = \lim_{k \rightarrow \infty}
\varphi (E_k)$ --- доказывается так же, как непрерывность меры.

$A_1 = E_1, \: A_2 = E_2 \ E_1, \: \ldots, \: A_k = E_k \ E_{k-1}$

$\bigcup_{k=1}^\infty E_k = \bigcup_{k=1}^\infty A_k$

$\varphi(\bigcup_{k=1}^\infty) = \sum_{k=1}^\infty \varphi(A_k)$

$E_1 \supset E_2 \supset \ldots \supset E_k \supset E_{k+1}
\supset \ldots$

$\varphi(\bigcap_{k=1}^\infty E_k) = \lim_{k \rightarrow \infty}
\varphi(E_k)$


%
%
%
%
%
%
%
%%%%%%%%%%%%%%%%%%%%%%%%%%%%%%%%%%%%%%%%%%%%%%%%%%%%%%%%%%%%%%%%%%
%%%%%%%%%%%%%%%%%%%%%%%%%%%%%%%%%%%%%%%%%%%%%%%%%%%%%%%%%%%%%%%%%%
%%%%%%%%%%%%%%%%%          стр 13.1    %%%%%%%%%%%%%%%%%%%%%%%%%%%%
%%%%%%%%%%%%%%%%%%%%%%%%%%%%%%%%%%%%%%%%%%%%%%%%%%%%%%%%%%%%%%%%%%
%%%%%%%%%%%%%%%%%%%%%%%%%%%%%%%%%%%%%%%%%%%%%%%%%%%%%%%%%%%%%%%%%%
%
%
%
%
%
Док. переходом к дополнению (не нужно оговаривать конечность меры)

 $\varphi \geqslant 0 \quad E_k$ --- последовательность множеств.

$\varliminf_{k \rightarrow \infty} E_k = \bigcup_k \bigcap_{n
\geqslant k} E_n$

$\varphi(\varliminf_{k \rightarrow \infty}) \leqslant
\varliminf_{k \rightarrow \infty} \varphi(E_k) \leqslant
\varlimsup_{k \rightarrow \infty} \varphi(E_k) \leqslant
\varphi(\varlimsup_{k \rightarrow \infty} E_k)$


$A_k = \bigcap_{n \geqslant k} E_n \nearrow_{k \rightarrow
\infty}$

$\varphi(\varliminf_{k \rightarrow \infty} E_k) = \lim_{k
\rightarrow} \varphi(A_k) \leqslant \varliminf_{k \rightarrow
\infty}(\varphi(E_k))$

Вторая часть доазательства аналогично переходом к $\searrow$
последовательности или переходом к дополнению.

$(X, \EuScript{M}), \varphi$ --- аддитивная функция.

\textbf{Определение.} \quad \={V}$(E, \varphi)$ --- верхняя
вариация относительно $\varphi$

\={V} $= \sup_{A \subset E} \varphi(A), A \in \EuScript{M}$

\b{V} $= - \inf_{A \subset E} \varphi(A), A \in \EuScript{M}$ ---
нижняя вариация

Полная вариация: $V = $ \={V}$(E) + $ \b{V}$(E)$

Из определения следует, что $\varphi(\varnothing) = 0$

\b{V} (E, q) = \={V}$(E, -\varphi)$

$\varphi(E) = \int_E f d \mu$, тогда \={V}$E = \int_E f^+ d\mu$,
\b{V}$(E) = \int_E f^- d\mu$

$V(E) = \int_E |f| d \mu$

Покажем, что \b{V}, \={V}, $V$ --- аддитивные функции множества

$1. \quad E = \bigcup_{k=1}^\infty, \; E_k \in \EuScript{M}$

$H_k = E_k \ (\bigcup_{n=1}^{k-1} E_n) \quad E =
\bigcup_{k=1}^\infty H_k$

$A \subset E \; A = \bigcup_{k=1}^\infty (A \bigcap H_k) \; A
\bigcap H_k \subset E_k \Rightarrow$

$\varphi(A) = \sum_{k=1}^\infty \varphi(A \bigcap H_k) \leqslant
\sum_{k=1}^\infty$ \={V}$ E_k $, переходя к $sup$ по $A$,
получаем:  \b{V}$(E) \leqslant \sum_{k=1}^\infty$ \={V} $(E_k)$
--- полуаддитивность.

То же вернодля \={V}$(E) \Rightarrow$ верно для $V(E)$

2. Проверим конечность \quad \b{V}, \={V}, $V$. Достаточно
доказать для $V$:

Предположим, что $\exists E: \; V(E) = +\infty$

$E = E_1 \supset E_2 \supset E_3 \supset \ldots \supset E_k
\supset \ldots , |\varphi(E_k)| \geqslant k - 1$

$V(E_k) = + \infty$

Строим: для $E_1$ это выполнено.

%
%
%
%
%
%
%
%%%%%%%%%%%%%%%%%%%%%%%%%%%%%%%%%%%%%%%%%%%%%%%%%%%%%%%%%%%%%%%%%%
%%%%%%%%%%%%%%%%%%%%%%%%%%%%%%%%%%%%%%%%%%%%%%%%%%%%%%%%%%%%%%%%%%
%%%%%%%%%%%%%%%%%          стр 14.0    %%%%%%%%%%%%%%%%%%%%%%%%%%%%
%%%%%%%%%%%%%%%%%%%%%%%%%%%%%%%%%%%%%%%%%%%%%%%%%%%%%%%%%%%%%%%%%%
%%%%%%%%%%%%%%%%%%%%%%%%%%%%%%%%%%%%%%%%%%%%%%%%%%%%%%%%%%%%%%%%%%
%
%
%
%
%
Так как $V(E_k) = + \infty \Rightarrow \exists \: A \subset E_k:
|\varphi(E_k)| \geqslant |\varphi(E_k)| + k$

$|\varphi(A \ E_k)| \geqslant |\varphi(A)| - |\varphi(E_k)|
\geqslant k$

$V(E_k) \leqslant V(A) + V(E_k \ A) \Rightarrow V(A)$ $V(E_k \ A)
= \infty$

В качестве $E_k$ возьмем то из них, на котором вариация ---
$\infty$

Так как это $\searrow$ последовательность, $\varphi(\bigcap_k E_k)
= \lim_{k \rightarrow \infty} \varphi(E_k) = \infty \Rightarrow$
 противоречие с конечностью $\varphi$

3. \quad Докажем противоположное неравенство:

Полагаем $E = \bigcup_{k=1}^\infty E_k$

докажем: \={V}$(E) \geqslant \sum_{k=1}^\infty$ \={V}$(E_k)$ -
$\varepsilon$

Устремим $\varepsilon$ к нулю, получим:

\={V}$(E) \geqslant \sum_{k=1}^\infty$ \={V}$(E_k)$

\textbf{Разложение Жордана.}

$\varphi(E) =$ \={V}$(E) -$ \b{V}$(E)$

\textbf{Доказательство.}

$A \subset E \quad \varphi(E) = \varphi(A) + \varphi(E \setminus
A)$

$\varphi(A) = \varphi(E) - \varphi(E \setminus A)$

Переходим к $sup$ и получаем

\={V} $(E) = \varphi(E) - \inf_{A \subset E} \varphi(E \setminus
A) = \varphi(E) +$ \b{V} $(E)$
