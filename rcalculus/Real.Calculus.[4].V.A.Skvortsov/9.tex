
% девятая лекция по действлану.

\begin{center} \textbf{Лекция 9.} \end{center}

Теорема верна и в случае, когда $\mu (E) \quad \delta$ конечна

$E = \bigcup_i E_i, \; \mu E_i < \infty$

Применим первую часть теоремы к $E_i$.

$A \subset E, \varphi(A) = \sum_i \varphi(A \bigcap E_i) = \sum_i
\int_{A \bigcap E_i} f_i d \mu + \varphi(Z \bigcap A), \; Z =
\bigcup_i Z_i, \; f_i$ --- суммируема, $\varphi$ ---
неотрицательная.

Если $\varphi$ --- любого знака $\Rightarrow$ пользуемся
разложением Жордана и применяем предыдущую часть теоремы

$\varphi =$ \={V} $-$ \b{V}

$\varphi(A) = \alpha(A) + \sigma(A)$

$\sigma(A \bigcap Z) = \varphi(A \bigcap Z) \Rightarrow \sigma(A)
= \varphi(A \bigcap Z)$

%
%
%
%
%
%
%
%%%%%%%%%%%%%%%%%%%%%%%%%%%%%%%%%%%%%%%%%%%%%%%%%%%%%%%%%%%%%%%%%%
%%%%%%%%%%%%%%%%%%%%%%%%%%%%%%%%%%%%%%%%%%%%%%%%%%%%%%%%%%%%%%%%%%
%%%%%%%%%%%%%%%%%          стр 16.1    %%%%%%%%%%%%%%%%%%%%%%%%%%%%
%%%%%%%%%%%%%%%%%%%%%%%%%%%%%%%%%%%%%%%%%%%%%%%%%%%%%%%%%%%%%%%%%%
%%%%%%%%%%%%%%%%%%%%%%%%%%%%%%%%%%%%%%%%%%%%%%%%%%%%%%%%%%%%%%%%%%
%
%
%
%
%

$f$ опр. однозначно

$\int_A f_1 d \mu = \int_A f_2 d \mu \Rightarrow \int_A (f_1 - f2)
d \mu = 0 \Rightarrow$


$( A^+ = \{x: \; f_1 - f_2 \geqslant 0\}$,\quad $A^- = \{x: \; f_1
- f_2 \leqslant 0\} $ + используем неравенство Чебышева)

\textbf{Теорема Радона-Никодима.} \quad

Каждый заряд $\varphi$ представим однозначным образом как $\int_A
f d\mu = \varphi(A), \; f$ --- суммируемая функция.

Заряд --- абсолютно непрерывная $\delta$--аддитивная функция. На
прямой  $f$ --- производная $\varphi$. В общем случае $f$ называют
производной Радона-Никодима от  $\varphi: \; f d \mu = d \varphi$


Абсолютная непрерывность $\forall \varepsilon > 0 \;\exists \delta
> 0 \; \mu A < \delta \Rightarrow |\int_A f d \mu| < \varepsilon$

\textbf{Задача.} \quad Построить $\varphi (A)^A = \int_A f d \mu$
и доказать непрерывность построения $\delta$ по $\varepsilon$ и
$f$.


Рассмотрим случай прямой и меры Лебега:

 \textbf{Определение.} \quad $F \in VB$ (ограниченной вариации), если конечна $V_a^b F = sup_p \sum_{i=1}^n |F(x_i) - F(x_{i-1})|
\quad P: \; a = x_0 < x_1 < \ldots < x_n = b$

$F = V_a^x(F) - (V_a^x (F) - F(x))$

$V_a^x(F)$   $|\triangle F(I)| \leqslant V_I(F)$

$\phantom{a}$

$\dfrac{\text{(\={V} + \b{V} + (\={V} - \b{V}))} }{2}$

$F = \dfrac{V_a^x(F) + F(x)}{2} - \dfrac{V_A^x(F) - F(x)}{2}$

Абсолютно - непр. функция точки $(AC)$

$F \in AC(E)$, если $\forall \varepsilon \; \exists \delta > 0: \;
\forall \{(\alpha_i, \beta_i)\}, \alpha_i, \beta_i \in E,\;
(\alpha_i, \beta_i)\bigcap(\alpha_j, \beta_j) = \varnothing, i
\neq j $

$\sum_i (\beta_i - \alpha_i) < \delta \Rightarrow \sum_i
|F(\beta_i) - F(\alpha_i)| < \varepsilon$

Опр. для конечного набора интервалов и для счетного равносим.
(если 1 интервал $\Rightarrow$ равномерн. непрерывность)

$F \in AC \Rightarrow F$ равномерно непрерывна на $E \;
\Rightarrow$ непрерывна в каждой точке $E$ (по множеству)

\textbf{Задача.} \quad  $f$ непрерывна на  $[a,b], \; E \subset
[a,b]$ $f \in AC(E) \Rightarrow f \in AC(\overline{E})$


%
%
%
%
%
%
%
%%%%%%%%%%%%%%%%%%%%%%%%%%%%%%%%%%%%%%%%%%%%%%%%%%%%%%%%%%%%%%%%%%
%%%%%%%%%%%%%%%%%%%%%%%%%%%%%%%%%%%%%%%%%%%%%%%%%%%%%%%%%%%%%%%%%%
%%%%%%%%%%%%%%%%%          стр 17.0    %%%%%%%%%%%%%%%%%%%%%%%%%%%%
%%%%%%%%%%%%%%%%%%%%%%%%%%%%%%%%%%%%%%%%%%%%%%%%%%%%%%%%%%%%%%%%%%
%%%%%%%%%%%%%%%%%%%%%%%%%%%%%%%%%%%%%%%%%%%%%%%%%%%%%%%%%%%%%%%%%%
%
%
%
%
%
$AC \subset VB$

$\varepsilon = 1$. Нашли $\delta_1$ из определения $AC$. Разбили
на конечное число отрезков длины  $< \delta$.

На каждом отрезке функция ограниченной вариации, значит,
ограниченной вариации и на $[a,b]$.

\textbf{Теорема} \quad $F \in AC([a,b]) \Rightarrow V_a^x(F) \in
AC([a,b]) \quad \{\alpha_i, \beta_i\}_i \quad \sum_i |F(\beta_i) -
F(\alpha_i)| < \delta \quad V_{\alpha_i}^{\beta_i}(F) <
\frac{\varepsilon}{2^i}$

Находим разбиение $p_i$ внутри $(\alpha_i, \beta_i)$.

$V_{\alpha_i}^{\beta_i} (F) - \frac{\varepsilon}{2^i} < \sum p_i$

$\sum_i V_{\alpha_i}^{\beta_i} - \varepsilon < \sum_i \sum p_i <
\varepsilon$

$\sum_i V_{\alpha_i}^{\beta_i} < 2\varepsilon \Rightarrow V_a^x
\in AC([a,b])$

\textbf{Следствие.} \quad $F \in AC \Rightarrow \varphi_F$
абсолютно непрерывна

\textbf{Задача} \quad Линейная комбинация функций из $AC$ ---
функция из $AC$

$F$ --- разность монотонных функций из $AC$. Значит, док. для
монотонн. $F$

$\mu E = 0$, $E$ покр. $(\alpha_i, \beta_i) \quad \sum(\beta_i -
\alpha_i) < \delta$

$\varphi(E) \leqslant \varphi(\bigcup_i(\alpha_i, \beta_i))
\leqslant \sum_i \varphi([\alpha_i, \beta_i]) = \sum_i F(\beta_i)
- F(\alpha_i) < \varepsilon \Rightarrow \varphi(E) = 0$

\textbf{Теорема.} \quad $F(x) = \int_a^x f d \mu \Leftrightarrow
F(x) \in AC$

Необходимость.  $F'(x) = f(x)$ п.в., если есть предст.  $F(x) =
\int_a^x f d \mu$ (в обратную сторону докажем позднее)

$F(x) - F(a) = \int_a^x f d \mu \Rightarrow F(x) \in AC$

Достаточность. $F \rightarrow \varphi_F \quad \varphi$ абсолютно
непрерывна $\Rightarrow \varphi([0, x]) = F(x) - F(a) = \int_a^x f
d \mu$

\textbf{Определение.} \quad $F \in ACG[a, b]$, если $[a, b] =
\bigcup_{i=1}^\infty$ и $F \in AC(E_i)$ и $F$ непрерывна на
$[a,b]$

\textbf{Задача.} \quad $F'(x)$ существует на $[a, b] \Rightarrow F
\in ACG[a,b]$


%
%
%
%
%
%
%
%%%%%%%%%%%%%%%%%%%%%%%%%%%%%%%%%%%%%%%%%%%%%%%%%%%%%%%%%%%%%%%%%%
%%%%%%%%%%%%%%%%%%%%%%%%%%%%%%%%%%%%%%%%%%%%%%%%%%%%%%%%%%%%%%%%%%
%%%%%%%%%%%%%%%%%          стр 17.1    %%%%%%%%%%%%%%%%%%%%%%%%%%%%
%%%%%%%%%%%%%%%%%%%%%%%%%%%%%%%%%%%%%%%%%%%%%%%%%%%%%%%%%%%%%%%%%%
%%%%%%%%%%%%%%%%%%%%%%%%%%%%%%%%%%%%%%%%%%%%%%%%%%%%%%%%%%%%%%%%%%
%
%
%
%
%

Задача. $F'(x)$ существует на $[a,b]$ $\not \Rightarrow F \in AC,
F \in VB$ (привести соответствующие примеры)

$f \in L[a,b] \Leftrightarrow \exists F \in AC: \quad F'(x) =
f(x)$

$f$ на $[a,b]$ --- интегрируемая в смысле
Дантуа--Хинчина.$\Leftrightarrow \exists F \in ACG[a,b] : F'(x) =
f(x)$ п.в.

Задача (*)

1. \quad $f \in \EuScript{H} [a,b] \Rightarrow f$ интегрируема по
 Дантуа--Хинчину

2. \quad Показать, что этот класс шире класса интегрируемых по К
-х функций.

 \textbf{Определение.} \quad Интеграл Дантуа--Хинчина $(D) \int_a^b f d \mu =
\phantom{x}^{\!\!\!\!\!\!\!\!\!\!\!\! def} F(b) - F(a)$

Надо доказать корректность определения, т.е. если  $F'(x) = 0$
п.в.  $\Rightarrow F(x) \equiv C$

\textbf{N--свойство Лузина.} \quad $F$ (действ.) опр. на $E$--изм
обладает  $N$--свойством, если $\mu E = 0 \Rightarrow \mu(F (E_1))
= 0, E_1 \subset E$

\textbf{Теорема} \quad $F \in AC(E) \Rightarrow F$ обладает
$N$--свойством, $E \subset [a,b]$

Достаточно доказать теорему для замкнутого множества $\mu A = 0$

$A \subset \bigcup_i (\alpha_i, \beta_i) \quad \sum_i[\beta_i -
\alpha_i] < \delta$

$F([\alpha_i, \beta_i] \bigcap A) \subset [\inf\limits_{x \in
[\alpha_i, \beta_i] \bigcap E} F(x), \sup\limits_{x \in [\alpha_i,
\beta_i] \bigcap E} F(x)] = |F(x_i) - F(y_i)|, \\ \quad x_i, y_i
\in [\alpha_i, \beta_i]$

$\mu (F(A)) \subset \sum_i |F(x_i) - F(y_i)| < \varepsilon
\Rightarrow \mu(F(A)) = 0$
