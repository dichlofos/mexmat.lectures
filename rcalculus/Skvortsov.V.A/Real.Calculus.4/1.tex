


%%%%%%%%%%%%%%%%%%%%%%%%%%%%%%%%%%%%%%%%%%%%%%%%%%%%%%%%%%%%%%%%%%
%%%%%%%%%%%%%                стр. 1         %%%%%%%%%%%%%%%%%%%%%%
%%%%%%%%%%%%%%%%%%%%%%%%%%%%%%%%%%%%%%%%%%%%%%%%%%%%%%%%%%%%%%%%%%

\begin{center}
\textbf{Лекция 1.}
\end{center}



\textbf{Определение.}\quad $S$ --- полукольцо множеств, если:

$\o \in S$; $S$ замкнуто относительно операции $\bigcap\:;$ и если
$A_1 \in A$, и $A_1, A \in S$, то $A = \bigcup_{i=1}^n{A_i}$,
$A_i\in S.$

\textbf{Определение.}\quad Кольцо --- непустое семейство множеств,
замкнутое относительно~$\bigcap\:.$ Обозначается $R,\:\Delta,\:
\bigcap\:.$

\textbf{Задача 1.}\quad Замкнутость относительно $\backslash\:,$
$\bigcap.$

\textbf{Задача 2.}\quad Кольцо является полукольцом.

\textbf{Задача 3.}\quad Какие пары операций на множествах дают
определения, эквивалентные кольцу.

Все множества семейства --- подмножества множества Х.

Если Х входит в класс, назовем его единицей.

\textbf{Определение.}\quad Кольцо с единицей --- алгебра множеств.


\textbf{Определение.}\quad Если кольцо замкнуто относительно
счётных объединений, назовем его $\delta$ - кольцом. $\delta$ -
кольцо с единицей --- $\delta$ - алгебра.

\textbf{Определение.}\quad Кольцо, порожденное данным семейством
--- минимальное кольцо, содержащее данное семейство.

$R(S)$ --- минимальное кольцо, порожденное $S$.

\textbf{Теорема.}\quad $\forall A \in R(S) $

$$ A = \bigcup_{j=1}^n B_j$$
$\triangle$

$B = \bigcup_{i=1}^k C_i, \; C_i \in S.$

$A\bigcap B = \bigcup_{i,j}\;(B_j\bigcap C_i)$

$A\setminus B = \bigcup _j \;(B_j \setminus B)$ $= \bigcup\_j
\bigcap_i\; (B_j \setminus C_i),\; B_j \setminus C_i$ принадлежит
рассматриваемому семейству.

$A\Delta B = (A\setminus B) \bigcup\: (B\setminus A) \Rightarrow$
семейство замкнуто относительно $\Delta.$



\begin{center}
\textbf{Мера}
\end{center}


$m: \EuScript{A} \rightarrow [\,0;+\infty);\; \EuScript{A}$ ---
семейство множеств,
%
%
%
%%%%%%%%%%%%%%%%%%%%%%%%%%%%%%%%%%%%%%%%%%%%%%%%%%%%%%%%%%%%%%%%%%
%%%%%%%%%%%%%%%%%%%%%%%%%%%% стр 2. %%%%%%%%%%%%%%%%%%%%%%%%%%%%%%
%%%%%%%%%%%%%%%%%%%%%%%%%%%%%%%%%%%%%%%%%%%%%%%%%%%%%%%%%%%%%%%%%%
%
%
%
%

$m (A \bigcup B) = m(A) + m(B)$

$m(\bigcup_{i=1}^\infty A_i) = \sum_{i=1}^\infty m(A_i)$
($\delta$-аддитивность)

$m: S \rightarrow [\,0;+\infty), \; S $ --- полукольцо

\textbf{Теорема.}\quad Существует единственное продолжение меры
$m: S \rightarrow [\,0;+\infty)$~на~$R(S)\; m'$, причем если $m$
$\delta$-аддитивна, то $m' - \delta$-аддитивна на кольце.


$A \in R(S), \; A = \bigcup_{j=1}^n B_j, \; B_j \in S$

$m'(A) \doteq \sum_{j=1}^n m\:B_j$

Проверим, что $m'$ не зависит от представления.

$A = \bigcup_{i=1}^k C_i$ --- другое представление.

$A = \bigcup_{j,i}\: (B_j \bigcap\: C_i),\; B_j \bigcap C_i \in S$

$m'(A) = \sum_{j=1}^n m(B_j) = \sum_{j=1}^n \sum_{i=1}^k m(B_j
\bigcap C_i) = \sum_{i=1}^k m(C_i) \; \Rightarrow \sum_{i=1}^k
m(C_i)~=~m'(A)$

Пусть есть второе продолжение $m''$

$m''(A) = \sum_{j=1}^n m''(B_j) = \sum_{j=1}^n m(B_j) = m'(A)
\Rightarrow m''$ совпадает с $m'.$

Проверка аддитивности:

$A = \bigcup_{j=1}^n A_j$

$A = \bigcup_{i=1}^m B_i,\; Bi \in S$

$A_j = \bigcup_{k=1}^{k_j} B_{j,k}$

$m'(A) = \sum_i \sum_{j,k} m(B_{j,k} \bigcap\: B_i) = \sum_j
\sum_{k,i} m(B_{j,k}\bigcap\: B_i) = \sum m'(A_j) \;\;\; (*)$

Проверим $\delta$-аддитивность.

$A = \bigcup_{j=1}^{\infty} A_j, \; A_j = \bigcup_{k=1}^{k_j}
B_{j,k}$


%
%
%
%
%%%%%%%%%%%%%%%%%%%%%%%%%%%%%%%%%%%%%%%%%%%%%%%%%%%%%%%%%%%%%%%%%%
%%%%%%%%%%%%%%%%%%%%%%  стр 2.1 %%%%%%%%%%%%%%%%%%%%%%%%%%%%%%%%%%
%%%%%%%%%%%%%%%%%%%%%%%%%%%%%%%%%%%%%%%%%%%%%%%%%%%%%%%%%%%%%%%%%%

Далее пишем (*), только там пользуемся $\delta$-аддитивностью $m$.

\textbf{Задача 4.}\quad На полукольце прямоугольников площадь ---~
$\delta$-аддитивная~мера.

\textbf{Теорема (полуаддитивность меры).}

$A \subset \bigcup_{i=1}^{\infty} A_i,$ тогда $m(A) \leqslant \sum
_{i=1}^{\infty} m(A_i)$

$(m$ --- $\delta$-аддитивна)


$\triangle$

$B_1 = A \bigcap\: A_1, \; B_2 = (A \bigcap A_2)\setminus B_1$

$B_i = (A \bigcap\: A_i)\backslash \bigcup_{k=1}^{i-1} B_k$

$A = \bigcup_{i=1}^{\infty}$

$ mA = \sum_{i=1}^m\,B_i \leqslant mA_i$ ($B_i \subset A_i$ в силу
монотонности меры)

\begin{center}
\textbf{Внешняя мера}
\end{center}

Х - основное множество, $S$ - полукольцо.

$E \subset \bigcup_i P_i, \; P_i \in S$

Тогда $\mu ^* (E) = \inf\limits_{\bigcup\limits_i P_i \supset E}
\sum_{j=1}^{\infty}mP_i$

Свойства внешней меры:

1) полуаддитивность

$E \subseteq \bigcup_{j=1}^{\infty} E_j \Rightarrow \mu ^* (E)
\leqslant \sum_{j=1}^{\infty} \mu ^* (E_j)$

Доказательство: приближенно с точностью $\varepsilon / 2^j \;
\forall\; E_j$

$\mu ^* (E_j) + \varepsilon / 2^i > \sum_{i=1}^{\infty}mP_{j,i}\;;
 \; \bigcup_{j,i} P{j,i}$ --- покрытие $Е$.

$\mu ^* (E_j) \leqslant \sum_{i,j}mP_{j,i} \leqslant \sum_j \mu ^*
(E_j) + \varepsilon$

Так как это верно $\forall\,\varepsilon$, то $\mu ^* (E_j)
\leqslant \sum_j \mu ^* (E_j)$

\textbf{Определение.}\quad \textbf{Внутренняя мера} \quad $\mu _*
(E) = mX - \mu ^* (X \setminus E)$

\textbf{Определение.}\quad \textbf{Измеримое множество}

(1) $\mu _* (E) = \mu ^* (E)$

(2) $\mu ^* (A) = \mu ^* (A \bigcap E) + \mu ^* (A \setminus E) \;
\forall \; A$
