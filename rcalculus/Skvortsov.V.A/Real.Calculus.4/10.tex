
% 10 лекция по действлану.

\begin{center} \textbf{Лекция 10.} \end{center}

\textbf{Определение.} \quad $F$ --- действительная функция, опр. в
окрестности точки $x$

$\varlimsup_{n \rightarrow \infty} \dfrac{F(x+h) - F(x)}{h} =
\overline{D} F(x)$

$\varliminf_{n \rightarrow \infty} \dfrac{F(x+h) - F(x)}{h} =
\underline{D} F(x)$


%
%
%
%
%
%
%%%%%%%%%%%%%%%%%%%%%%%%%%%%%%%%%%%%%%%%%%%%%%%%%%%%%%%%%%%%%%%%%%
%%%%%%%%%%%%%%%%%%%%%%%%%%%%%%%%%%%%%%%%%%%%%%%%%%%%%%%%%%%%%%%%%%
%%%%%%%%%%%%%%%%%          стр 18.0    %%%%%%%%%%%%%%%%%%%%%%%%%%%%
%%%%%%%%%%%%%%%%%%%%%%%%%%%%%%%%%%%%%%%%%%%%%%%%%%%%%%%%%%%%%%%%%%
%%%%%%%%%%%%%%%%%%%%%%%%%%%%%%%%%%%%%%%%%%%%%%%%%%%%%%%%%%%%%%%%%%
%
%
%
%
%


\textbf{Производные числа Дини} \quad --- это правый и левый
верхний и нижние пределы $(F(x + h) - F(x))/h$


$D^+ F(x) = \varlimsup_{n \rightarrow +0} \dfrac{F(x+h) - F(x)}{h}
$

$D_+ F(x) = \varliminf_{n \rightarrow +0} \dfrac{F(x+h) - F(x)}{h}
$

$D^- F(x) = \varlimsup_{n \rightarrow -0} \dfrac{F(x+h) - F(x)}{h}
$

$D_- F(x) = \varliminf_{n \rightarrow -0} \dfrac{F(x+h) - F(x)}{h}
$

\textbf{Задача.} \quad Пример непрерывной $F: \: D^+ F = D^- F = +
\infty, \; D_+ F(X) = D_-F(x) = -\infty$ на множестве
положительной меры (изобразить график)

\textbf{Теорема.} \quad $F \in ACG([a,b]), D_+ F(x) \geqslant C$
п.в.  $\Rightarrow F$ $[a,b]$

\textbf{Доказательство.} \quad

Фиксируем произвольное $\varepsilon > 0$. Введем функцию $g(x) =
F(x) + \varepsilon x$

$D_+ g(x) = D_+ F(x) + \varepsilon > 0$

\textbf{Задача.} \quad Линейная комбинация $ACG$--функций ---
$ACG$--функция.

Значит, (факт из задачи) $g \in ACG([a,b]) \Rightarrow g$ обладает
$N$--свойством.

Пусть $E$--множество, где не выполнено условие $D: \: g(x) > 0$

$\mu E = 0$ (т.к. условие выполнено п.в.) $\Rightarrow g$ $N$

Предположим, что $g$ не является монотонно возрастающей $\exists
x_1, x_2 \in [a,b], \; x_1<x_2, $ $g(x_1)
> g(x_2)$

%
%
%
%
%
%
%%%%%%%%%%%%%%%%%%%%%%%%%%%%%%%%%%%%%%%%%%%%%%%%%%%%%%%%%%%%%%%%%%
%%%%%%%%%%%%%%%%%%%%%%%%%%%%%%%%%%%%%%%%%%%%%%%%%%%%%%%%%%%%%%%%%%
%%%%%%%%%%%%%%%%%          стр 18.1    %%%%%%%%%%%%%%%%%%%%%%%%%%%%
%%%%%%%%%%%%%%%%%%%%%%%%%%%%%%%%%%%%%%%%%%%%%%%%%%%%%%%%%%%%%%%%%%
%%%%%%%%%%%%%%%%%%%%%%%%%%%%%%%%%%%%%%%%%%%%%%%%%%%%%%%%%%%%%%%%%%
%
%
%
%
%

$\exists y_0 \not \in g(E)$ $\mu g(E) = 0$

$\{x: g(x) = y_0\}$ --- замкнут., т.к.

$g$ --- непрерывная $\Rightarrow \exists x_0$ --- самая прав.,

$\dfrac{g(x_0 + h) - g(x_0)}{h} \leqslant 0$

$D_+ g(x_0) \leqslant 0$ --- противоречие

То есть, если $x_1 < x_2 \Rightarrow g(x_1) < g(x_2)$

$F(x_1) + \varepsilon x_1 \leqslant F(x_2) + \varepsilon x_2$

$F(x_1) \leqslant F(x_2)$ (в пределе при $\varepsilon \rightarrow
0$)

\textbf{Замечание.} \quad Можно восп. $D^+$

\textbf{Следствие.} \quad $F \in ACG([a,b])$

$F'(x) = 0$ п.в. $\Rightarrow F(x) = const$

Значит, интеграл Дантуа--Хинчина с помощью формулы $(D) \int_a^b f
= F(b) - F(a)$ определен однозначно.

Если $G'(x) = f(x)$ $\Rightarrow (F(x) - G(x))' = 0$ п.в.
$\Rightarrow F(b) - F(a) = G(b) - G(a)$ $(F(x) - G(x) = const$
п.в.)

Доказать, что если $F \in VB$, то $F'$ кон. существует п.в.
Достаточно показать это для монотонно возрастающих функций.

\textbf{Теорема.} \quad $f$ не убывает на $[a, b] \Rightarrow f'$
сущ. п.в. $f' \in L[a,b]$ и $\int_a^b f' d \mu \leqslant f(b - 0)
- f(a + 0)$

$\{I\}$ --- семейство, покр. $E$ в смысле Витали

$\exists \{I_i\}$ --- конечное число, $I_i$: (если $E$ ---
огранич.)

Нужно доказать совпадение производных чисел Дини. Покажем, что
совпадают $D^+ f(x)$ и $D_- f(x)$, то есть $\mu^* (A \bigcap
(\bigcup_i I_i)) > \mu^*(E) - \varepsilon$

$D^+ f(x)$ $D_- f(x)$ $\mu\{x: D^+ f(x) > D^- f(x)\} = 0$

$A = A_{rs} = \{x: D^+ f(x) > r > s > D_- f(x)\}$

Докажем, что $\mu A = 0$.

Предположим, что $\mu^* A > 0$. Найдем $G$ --- откр..

$G \supset A, \; \mu G < \mu^* A + \varepsilon$

%
%
%
%
%
%
%%%%%%%%%%%%%%%%%%%%%%%%%%%%%%%%%%%%%%%%%%%%%%%%%%%%%%%%%%%%%%%%%%
%%%%%%%%%%%%%%%%%%%%%%%%%%%%%%%%%%%%%%%%%%%%%%%%%%%%%%%%%%%%%%%%%%
%%%%%%%%%%%%%%%%%          стр 19.0    %%%%%%%%%%%%%%%%%%%%%%%%%%%%
%%%%%%%%%%%%%%%%%%%%%%%%%%%%%%%%%%%%%%%%%%%%%%%%%%%%%%%%%%%%%%%%%%
%%%%%%%%%%%%%%%%%%%%%%%%%%%%%%%%%%%%%%%%%%%%%%%%%%%%%%%%%%%%%%%%%%
%
%
%
%
%

$\dfrac{-f(x) + f(x-h)}{-h} < S$ по некоторой
подпоследовательности $h$

$[x-h, x]$ для таких $h$ обр. покрытие Витали

$[x - h, x] \subset G    --//--//--$

Выберем конечное число отрезков $[x_k - h_k, x_k]$

$\dfrac{-f(x_k) + f(x_k - h_k)}{-h_k} < S$ $\mu^* (A \bigcap
(\bigcup_k [x_k - h_k, x_k])) > \mu^*(A) - \varepsilon$

$f(x_k) - f(x_k - h_k) < sh_k$

$\sum_k (f(x_k) - f(x_k - h_k)) < s \sum_k h_k \leqslant s \mu G
\leqslant s(\mu^* A + \varepsilon)$

Используем второе неравенство: $B = A \bigcap (\bigcup_k [x_k -
h_k, x_k])$


$\dfrac{f(x+k) - f(x)}{k} > r$   $\exists$ подпоследовательность
$h_k$

$[x, x + k] \subset \bigcup_k [x_k - h_k,x_k]$

Берем эти отрезки. Ои покрывают $B$ в смысле Витали. Выберем
$[x_i', x_i'+ k_i]$ --- конечное число отрезков, для которых
выполнено:
 $\dfrac{f(x_i' + k_i) - f(x_i')}{k_i} > r$

$f(x_i' + k_i) - f(x_i') > k_i r$

$\sum_i (f(x_i' + k_i) - f(x_i')) > r \sum_i k_i \fbox{>}$

$\mu* (B \bigcap (\bigcup[x_i', x_i' + k_i])) > \mu^*(B) -
\varepsilon > \mu^*(A) - 2\varepsilon$

$\fbox{>} r (\mu^*A - 2\varepsilon)$

$\sum_k (f(x_k) - f(x_k - n_k)) > \sum_i (f(x_i + k_i) - f(x_i)),$
так как $f$ --- монотонная

$s \mu^* (A) + s \varepsilon > r \mu^* (A) + 2 r \varepsilon$

$s \mu^* (A) \geqslant r \mu^* (A) \quad s \geqslant r$ ---
противоречие.

$\Rightarrow \mu^* A = 0$

Это верно для любой пары производных чисел (доказывается
аналогично) $\Rightarrow$ все производные числа Дини совпадают
п.в.

Докажем неравенство теоремы: $f_k(x) = \dfrac{f(x + 1/k) -
f(x)}{1/k} \xrightarrow[k \rightarrow \infty]{f'(x)}$

$f(x)$ продолж. справа от в конст. $f(b-0)$

%
%
%
%
%
%
%%%%%%%%%%%%%%%%%%%%%%%%%%%%%%%%%%%%%%%%%%%%%%%%%%%%%%%%%%%%%%%%%%
%%%%%%%%%%%%%%%%%%%%%%%%%%%%%%%%%%%%%%%%%%%%%%%%%%%%%%%%%%%%%%%%%%
%%%%%%%%%%%%%%%%%          стр 19.1    %%%%%%%%%%%%%%%%%%%%%%%%%%%%
%%%%%%%%%%%%%%%%%%%%%%%%%%%%%%%%%%%%%%%%%%%%%%%%%%%%%%%%%%%%%%%%%%
%%%%%%%%%%%%%%%%%%%%%%%%%%%%%%%%%%%%%%%%%%%%%%%%%%%%%%%%%%%%%%%%%%
%
%
%
%
%

$f(x) = f(b-0) \; \forall x \geqslant b$

$f_k(x) \geqslant 0$

По теореме Фату  $\int_a^b f' d \mu \leqslant \varliminf_{k
\rightarrow \infty} \int_a^b f_k d \mu = \lim_{k \rightarrow
\infty} k (\int_b^{b + 1/k} f d x - \int_a^{a + 1/k} f dx) = f(b -
0) - f(a - 0)$

Равенство  $\int_a^b f'd \mu = f(b) - f(a)$ выполняется тогда и
только тогда, когда $f$ --- абсолютно непрерывна.

$f \in AC, \; f'(x) = g(x)$ п.в. ( сущ. п.в. , т.к. $f \in AC$)
$\int_a^b f' d \mu = f(b) - f(a) \Leftrightarrow g$ интегрируема
по Лебегу (можно взять за эквивалентное определение)
