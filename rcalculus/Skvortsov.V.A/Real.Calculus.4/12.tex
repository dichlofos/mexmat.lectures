
% двенадцатая лекция по действлану.

\begin{center} \textbf{Лекция 12.} \end{center}

$X \times Y$ $\mu_x \times \mu_y$ $f(x,y) \geqslant 0$

$\int_{X \times Y} f(x,y) d \mu = \int_X (\int_Y f(x,y) d \mu_y) d
\mu_x$

Уже  доказано для $\chi_A$, $A$ измеримо.

Меры $\mu_x$ и $\mu_y$ предполагаются полными. Так как доказано
для характреристических функций $\Rightarrow$ доказано для простых
функций.

%
%
%
%
%
%
%%%%%%%%%%%%%%%%%%%%%%%%%%%%%%%%%%%%%%%%%%%%%%%%%%%%%%%%%%%%%%%%%%
%%%%%%%%%%%%%%%%%%%%%%%%%%%%%%%%%%%%%%%%%%%%%%%%%%%%%%%%%%%%%%%%%%
%%%%%%%%%%%%%%%%%          стр 21.1    %%%%%%%%%%%%%%%%%%%%%%%%%%%%
%%%%%%%%%%%%%%%%%%%%%%%%%%%%%%%%%%%%%%%%%%%%%%%%%%%%%%%%%%%%%%%%%%
%%%%%%%%%%%%%%%%%%%%%%%%%%%%%%%%%%%%%%%%%%%%%%%%%%%%%%%%%%%%%%%%%%
%
%
%
%
%

$f_n(x,y) \nearrow f(x,y), \; f_n$ --- простые функции

$\int_{X \times Y} f_n(x,y) d \mu = \int_X (\int_Y f_n (x,y) d
\mu_y) d \mu_x$

Выбрасываем множество меры 0 для каждого $n$, где $f_n(x,y)$
неизм. по $y$ и берем их объединение. Затем применяем теорему
Леви.

Для неотрицательных функций торема Фубини доказана.

Теорема Фубини: $f$ --- измеримая функция, $f$ интегрируема по
Лебегу на $X \times Y$ (интеграл конечен). Тогда $\int_{X \times
Y} f(x,y) d \mu = \int_X(\int_Y f (x,y) d \mu_y) d \mu_x$

$f(x,y)$ изм. для п.в. $x$, и интеграл по у конечен для п.в. х

$\int_X f(x,y) d \mu_y$ интегрируема по $x$

(в предположении конечности левой части формулы)

$f = f^+ f^-$


\textbf{Пример существенности условия конечности левой части
формулы.} \quad

$f(x,y) = \frac{xy}{(x^2 + y^2)^2}, \; x\in [-1, 1], y \in [-1,
1]$

Эта функция не интегрируема по Лебегу как функция двух переменных.

Пространство $L_p (X, \EuScript(M), \mu)$

$1 \leqslant p \leqslant +\infty$

$\| f_p\| = (\int_X |f|^p d\mu)^{1/p} < + \infty$

Уже рассматривали  $L_1 = L$

$\| f\| = 0 \Leftrightarrow f = 0$ --- это все функции, равные 0
п.в. (эквивалентны нулевой функции)

Элементы $L_p$ --- классы эквивалентности функций.

Проверим свойства нормы: $\|\lambda f\|_p = |\lambda| \|f\|_p$ ---
очевидно.

Если $1 < p < + \infty$, то сопр. показатель $q$ п определению ---
число, удовлетворяющее свойству $1/p + 1/q = 1$

%
%
%
%
%
%
%%%%%%%%%%%%%%%%%%%%%%%%%%%%%%%%%%%%%%%%%%%%%%%%%%%%%%%%%%%%%%%%%%
%%%%%%%%%%%%%%%%%%%%%%%%%%%%%%%%%%%%%%%%%%%%%%%%%%%%%%%%%%%%%%%%%%
%%%%%%%%%%%%%%%%%          стр 22.0    %%%%%%%%%%%%%%%%%%%%%%%%%%%%
%%%%%%%%%%%%%%%%%%%%%%%%%%%%%%%%%%%%%%%%%%%%%%%%%%%%%%%%%%%%%%%%%%
%%%%%%%%%%%%%%%%%%%%%%%%%%%%%%%%%%%%%%%%%%%%%%%%%%%%%%%%%%%%%%%%%%
%
%
%
%
%

$q = p/(p-1)$

Докажем \textbf{неравенство Гёльдера.}

$f \in L_p, g \in L_1$, тогда $\int_X |fg| d \mu \leqslant \|f\|_p
\|g\|_q$

$ab \leqslant \int_0^a x^{p-1}dx + \int_0^b y^{q-1}dy $

$ab \leqslant q^p/p + b^q/q$

Применим это к доказательству неравенства Гёльдера.

$a = \dfrac{|f(x)|}{\|f\|_p}, b = \dfrac{|g(x)|}{\|g\|_q}$

$\int_X \dfrac{|f(x)| \cdot |g(x)|}{\|f\|_p \|g\|_q} \leqslant
\dfrac{1}{p} \int_X \dfrac{|f(x)|^p}{\|f\|_p}d \mu +
\dfrac{1}{q}\int_X \dfrac{|g(x)|^q}{\|g\|_q}d \mu $

Упрощаем и получаем неравенство Гёльдера.


\textbf{Доказательство неравенства Минковского.} \quad

$\|f + g\|_p \leqslant \|f\|_p + \|g\|_p, \; p \geqslant 1$


$|f + g|^p \leqslant \int_X |f + g|^{p-1}|f|d \mu + \int_X |f +
g|^{p-1}|g| d \mu \leqslant $

$\leqslant (\int_X |f + g|^{(p-1)\cdot \dfrac{p}{p-1}})^{1 -
1/p}\cdot(\int_X |f|^p)^{1/p} + (\int_X
|f+g|^p)^{1-1/p}\cdot(\int_X |g|^p)^{1/p}$

Если $\int_X |f + g| d \mu = 0$, то это очевидно.

$|f(x) + g(x)|^p \leqslant 2^p \max{|f(x)^p, |g(x)^p||} \leqslant
2^p(|f(x)|^p + |g(x)|^p)$

Тогда, сокращая на $\int_X |f + g|^p d \mu$, получаем неравенство
Минковского (оно же неравенство треугольника для $\| \cdot \|$)

$f_n \xrightarrow[]{L_p} f$

\textbf{Связь разных сходимостей.} \quad

$\mu{x \in X: |(f_n - f)(x)| > \varepsilon} \leqslant
\dfrac{\int_X |f_n - f| d \mu}{\varepsilon^p} = $
%
%
%
%
%
%
%%%%%%%%%%%%%%%%%%%%%%%%%%%%%%%%%%%%%%%%%%%%%%%%%%%%%%%%%%%%%%%%%%
%%%%%%%%%%%%%%%%%%%%%%%%%%%%%%%%%%%%%%%%%%%%%%%%%%%%%%%%%%%%%%%%%%
%%%%%%%%%%%%%%%%%          стр 22.1    %%%%%%%%%%%%%%%%%%%%%%%%%%%%
%%%%%%%%%%%%%%%%%%%%%%%%%%%%%%%%%%%%%%%%%%%%%%%%%%%%%%%%%%%%%%%%%%
%%%%%%%%%%%%%%%%%%%%%%%%%%%%%%%%%%%%%%%%%%%%%%%%%%%%%%%%%%%%%%%%%%
%
%
%
%
%
$\dfrac{\|f_n - f\| p^p}{\varepsilon}$

Значит, сходимость в $L_p \Rightarrow$ сходимость по мере
(обратное неверно)

\textbf{Задача.} \quad Выяснить связь сходимости в $L_p$ с другими
сходимостями

\textbf{Теорема.} \quad $L_p$ --- полное пространство

Докажем, что $L_p$ --- полное пространство (в предположении, что
$\mu - \delta$--конечна)

Пусть $f_n$ удовлетворяет условию Коши. Покажем, что $f_n
\rightarrow f$ поточечно п.в.

Найдем $n_k : \|f_{n_k} - f_n\|_p < 1/2^k \forall n \geqslant n_k$

Определим подпоследовательности $f_{n_k}; n_k < n_{k+1} < \ldots$

Возьмем множество, на котором $\mu$ конечна.

$(X = \bigcup_{k=1}^\infty E_k, \mu E_k < \infty)$. Возьмем $E_k)$

Тогда $\int |f_{n_k} - f_{n_{k+1}}| d \mu \leqslant \|f_{n_k} -
f_{n_{k+1}}\|\cdot c \leqslant c \cdot \dfrac{1}{2^k}$

(Неравенство Гёльдера)

Ряд из интегралов сходится $\Rightarrow$

$|f_{n_1} (x)| + \sum_{k=1}^\infty |f_{n_{k+1}}(x) - f_{n_k}(x)|$
сходится п.в.

Тогда ряд $f_{n_1}(x) + \sum_{k=1}^\infty (f_{n_{k+1}}(x) -
f_{n_k}(x))$ сходится абсолютно.

Его частичные суммы --- $f_{n_k}$.

Значит, $f_{n_k}(x) \rightarrow \int_X |f_{n_k} - f_{n_l}| d \mu
\xrightarrow[l \rightarrow \infty]{l>k} \int_X |f_{n_k} - f|^p d
\mu \leqslant \varepsilon$ (по теореме Фату)

$ \| f_{n_k} - f_{n_l}\|_p < 1/2^k, \; n_k < n \leqslant n_{k+1}$

Тогда $\|f_n - f\|_p \leqslant \|f_n - f_{n_k}\|_p + \|f_{n_k} -
f\| < 2\varepsilon$
