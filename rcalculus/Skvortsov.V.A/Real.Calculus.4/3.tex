

\begin{center}
\textbf{Лекция 3.}
\end{center}

$m, m'$

$\mu^* (E) = \inf\limits_{\bigcup P_k \supset E}
\sum_{k=1}^{\infty} m(P_k)$

$\mu^* = m' \mbox{ для элементов кольца}$

$E \subset R(S)$

Покажем, что $E$ входит в класс измеримых множеств.

Обозначим~$\EuScript{M}$~---~класс~измеримых~множеств.

$\forall A \; \mu^*(A) = \mu^*(A \bigcap E) + \mu^*(A \setminus
E)$ (надо доказать)

%%%%%%%%%%%%%%%%%%%%%%%%%%%%%%%%%%%%%%%%%%%%%%%%%%%%%%%%%%%%%%%%%
%%%%%%%%%%%%%%%%%%%%%%%%%%%%%%%%%%%%%%%%%%%%%%%%%%%%%%%%%%%%%%%%%
%%%%%%%%%%%%%%%%%%          стр 5         %%%%%%%%%%%%%%%%%%%%%%%
%%%%%%%%%%%%%%%%%%%%%%%%%%%%%%%%%%%%%%%%%%%%%%%%%%%%%%%%%%%%%%%%%
%%%%%%%%%%%%%%%%%%%%%%%%%%%%%%%%%%%%%%%%%%%%%%%%%%%%%%%%%%%%%%%%%
$A \subset \bigcup_{i=1}^{\infty} P_k \quad \sum_{k=1}^{\infty}
mP_k < \mu^*(A) + \varepsilon$ (выбираем такое покрытие)

$\mu^*(A) \leqslant \mu^*(A \bigcap E) + \mu^*(A \setminus E)$ в
силу полуаддитивности

$\mu^*(A \bigcap E) + \mu^*(A \setminus E) \leqslant
\sum_{k=1}^{\infty} \mu^*(P_k \bigcap E) + \mu^*(P_k \setminus E)
=  \sum_{k=1}^{\infty} (m'(P_k \bigcap E) + m'(P_k \setminus E)) =
\sum_{k=1}^{\infty} mP_k \leqslant \mu^*(A) + \varepsilon$

$\mu^*(A \bigcap E) + \mu^*(A \setminus E) \leqslant \mu^*(A) +
\varepsilon$

В пределе при $\varepsilon \rightarrow 0 \quad \mu^*(A \bigcap E)
+ \mu^*(A \setminus E) \leqslant \mu^*(A)$

Значит, выполнено равенство Каратеодори.

\textbf{Пример неизмеримого множества}

$E = \{x_\alpha\}$, где $x_\alpha$ --- рациональная точка на
единичной окружности, повёрнутая на~угол~$\alpha$.

Покажем, что $E$ --- неизмеримо. Построенная мера инвариантна
относительно сдвига $r_n$, причем $\bigcup_{n=1}^{\infty}{E + r_n}
= T\;$.  Если $E$ измеримо и имеет меру $a$, $\{E~+~r_n\}$~также
имеет меру $a$. Т.е. $\sum_{n=1}^{\infty} a = 1$, а это
невозможно.

\textbf{Непрерывность меры Лебега}

$\{E_k\} \quad E_k \subset E_{k+1}$

\textbf{Определение.} \quad Мера непрерывна, если $\lim_{k
\rightarrow \infty} \mu E_k = \mu(\bigcup_{k=1}^{\infty})$

$\bigcup_{k=1}^{\infty} E_k = E_1 \bigsqcup \;(E_2 \setminus E_1)
\bigsqcup \ldots \bigsqcup\; (E_k \setminus E_{k-1})$

$\mu(\bigcup_{k=1}^{\infty}) = \mu E_1 + \sum_{k=1}^{\infty} \mu
(E_k \setminus E_{k-1}) = \mu E_1 + \sum_{k=1}^{\infty} (\mu E_k -
\mu E_{k-1}) = \lim_{k \rightarrow \infty} \mu (E_k)$

Свойство непрерывности $\Leftrightarrow$ $\delta$-аддитивности.




%%%%%%%%%%%%%%%%%%%%%%%%%%%%%%%%%%%%%%%%%%%%%%%%%%%%%%%%%%%%%%%%%%%
%%%%%%%%%%%%%%%%%%%%%%%%%%%%%%%%%%%%%%%%%%%%%%%%%%%%%%%%%%%%%%%%%%%
%%%%%%%%%%%%%%%%%%%%%    стр 5.1         %%%%%%%%%%%%%%%%%%%%%%%%%%
%%%%%%%%%%%%%%%%%%%%%%%%%%%%%%%%%%%%%%%%%%%%%%%%%%%%%%%%%%%%%%%%%%%
%%%%%%%%%%%%%%%%%%%%%%%%%%%%%%%%%%%%%%%%%%%%%%%%%%%%%%%%%%%%%%%%%%%



\textbf{Задача.} \quad Пусть есть $\delta$-кольцо с единицей, Х
--- единица, $\mu X < + \infty$; $\quad \{E_k\}, E_k \supset
E_{k+1}$; тогда $\lim_{k \rightarrow \infty} \mu \:(E_k) = \mu\:
(\bigcap_{k=1}^{\infty} E_k)$

(доказывается переходом к дополнению)

\textbf{Задача.} Пример, показывающий существенность условия $\mu
E < +\infty$






\begin{center}
\textbf{Регулярность внешней меры}
\end{center}

Внешняя мера $\lambda$ \textbf{регулярна}, если $\forall \; A
\subset
X \; \exists B \in M: \mbox{B --- измерима по Каратеодори и} \\
A \subset B \mbox{ т.ч. } \lambda(A) = \lambda(B).$

Покажем, что мера, построенная конструкцией Лебега (мера $\mu^*$),
регулярна.

$\mu^*(A) + \frac{1}{i} \geqslant \sum_{k=1}^{\infty} mP_{i,k}
\geqslant
\\ (\forall \:i \;\exists \:\bigcup_k P_{i,k} \supset A \quad
B = \bigcap_{i=1}^{\infty}\bigcup_{k=1}^{\infty} P_{i,k}) \\
\geqslant \; \mu(\bigcup_{k=1}^{\infty}P_{i,k}) \geqslant \mu^*(B)
\quad \forall i \in \mathbb{N} \; \Rightarrow \; \mu^*A \geqslant
\mu B$

Очевидно, что $\mu^*A \leqslant \mu^*B \Rightarrow \mu^*A = \mu B$

В случае $\mathbb{R}^n \quad B =
\bigcap_{i=1}^{\infty}\bigcup_{k=1}^{\infty} P_{i,k} \in
\EuScript{Y}_{\delta}$

(если в качестве элементов полукольца --- интервалы)

Для приближения с точностью до $\varepsilon$ можно взять открытое
множество $E \subset \bigcup_{k} P_k$

$\mu F_{\varepsilon} + \varepsilon \leqslant \mu E \leqslant \mu
G_{\varepsilon} - \varepsilon\;$ для Е --- измеримого множества

И в $\mathbb{R}_n$ это --- достаточное условие измеримости

Внутренняя мера $E \subset X : \mu_*(E) = mX - \mu^*(X \setminus
E)$

Множество Е \textbf{измеримо по Лебегу}, если $\mu^*(E)=
\mu_*(E)$, то есть $mX = \mu^*(E) + \mu^*(X \setminus E)$

Покажем, что это определение эквивалентно определению Каратеодори:

1) опр. Каратеодори $\Rightarrow$ опр. Лебега (очевидно)

2) опр. Каратеодори $\Leftarrow$ опр. Лебега:




%%%%%%%%%%%%%%%%%%%%%%%%%%%%%%%%%%%%%%%%%%%%%%%%%%%%%%%%%%%%%%%%%%%%
%%%%%%%%%%%%%%%%%%%%%%%%%%%%%%%%%%%%%%%%%%%%%%%%%%%%%%%%%%%%%%%%%%%%
%%%%%%%%%%%%%%%%%%%%%%%%%%    стр 6    %%%%%%%%%%%%%%%%%%%%%%%%%%%%%
%%%%%%%%%%%%%%%%%%%%%%%%%%%%%%%%%%%%%%%%%%%%%%%%%%%%%%%%%%%%%%%%%%%%
%%%%%%%%%%%%%%%%%%%%%%%%%%%%%%%%%%%%%%%%%%%%%%%%%%%%%%%%%%%%%%%%%%%%




$mX = \mu^*(E) + \mu^*(X \setminus E)$

$A \in \EuScript{M}, \quad \mu^*(E) = \mu(A) \quad (\exists \mbox{
такое } \; E \subset A)$

$X \setminus E \subset B, \quad B \in \EuScript{M}, \quad \mu^*(X
\setminus E) = \mu B$

$A \bigcup B = X$

$mX = \mu^*(E) + \mu^*(X \setminus E) = \mu A + \mu B$

$\mu X = \mu A + \mu B - \mu(A \bigcap B) \Rightarrow \mu(A
\bigcap B) = 0$

$A \setminus E \subset A \bigcap B \Rightarrow A \setminus E$ ---
измеримо

$E = A \setminus (A \setminus E)$ как разность измеримых множеств

$F$ - монотонная неубывающая функция

В качестве кольца: $[\alpha; \beta)$

$m([\alpha, \beta)) = F(\beta) - F(\alpha)$



%%%%%%%%%%%%%%%%%%%%%%%%%%%%%%%%%%%%%%%%%%%%%%%%%%%%%%%%%%%%%%%%%
%
%
%                       исправление 1
%
%
%
%%%%%%%%%%%%%%%%%%%%%%%%%%%%%%%%%%%%%%%%%%%%%%%%%%%%%%%%%%%%%%%%%





$F(x) = \{ m([0;x)), \quad x > 0 \\
           0, \quad x = 0 \\
           m([x;0)), \quad  x < 0))$

\textbf{Теорема.} \quad Мера $m$, таким образом,
$\delta$-аддитивна на полукольце $[\alpha, \beta) \Leftrightarrow
F \mbox{ непрерывна слева } (F(t - 0) - F(t))$

\textbf{Необх.} \quad


%%%%%%%%%%%%%%%%%%%%%%%%%%%%%%%%%%%%%%%%%%%%%%%%%%%%%%%%%%%%%%%%%
%
%
%                       исправление 2
%
%
%
%%%%%%%%%%%%%%%%%%%%%%%%%%%%%%%%%%%%%%%%%%%%%%%%%%%%%%%%%%%%%%%%%

$\mu E_n = F(t) - F(t - \frac{1}{n}) \xrightarrow[k \rightarrow
\infty]{} 0$

$\lim_{n \rightarrow \infty} \mu E_n = 0$ (т.е. $F(t)$ непрерывна
слева по опр. Гейне)



\textbf{Дост.} \quad $[\alpha; \beta) = \bigsqcup_{i=1}^{\infty}\:
[\alpha_n, \beta_n)$

$\sum_{n=1}^{k} (F(\beta_n) - F(\alpha_n)) \leqslant F(b) - F(a)$

%%%%%%%%%%%%%%%%%%%%%%%%%%%%%%%%%%%%%%%%%%%%%%%%%%%%%%%%%%%%%%%%%
%
%
%                       исправление 3
%
%
%
%%%%%%%%%%%%%%%%%%%%%%%%%%%%%%%%%%%%%%%%%%%%%%%%%%%%%%%%%%%%%%%%%

$F(b) - F(b - \delta) < \varepsilon$ (в силу непрерывности слева)

$F(\alpha_n) - F(\alpha_n - \delta_n) < \frac{\varepsilon}{2^n}$

Выберем конечное покрытие $[\alpha, b - \delta]$

$F(b - \delta) - F(a) \leqslant \sum_{n=1}^N (F(\beta_n) -
F(\alpha_n - \beta_n)) \leqslant \sum_{n=1}^{\infty} (F(\beta_n) -
F(\alpha_n - \beta_n))$

$F(b) - \varepsilon - F(a) \leqslant \sum_{n=1}^{\infty}
(F(\beta_n) - F(\alpha_n)) + \sum_{n=1}^{\infty}
\frac{\varepsilon}{2^n}$

В пределе при $\varepsilon \rightarrow 0$

$\sum_{n=1}^{\infty} (F(\beta_n) - F(\alpha_n)) \geqslant F(b) -
F(a)$



%%%%%%%%%%%%%%%%%%%%%%%%%%%%%%%%%%%%%%%%%%%%%%%%%%%%%%%%%%%%%%%%%%%
%%%%%%%%%%%%%%%%%%%%%%%%%%%%%%%%%%%%%%%%%%%%%%%%%%%%%%%%%%%%%%%%%%%
%%%%%%%%%%%%%%%%%%%%%    стр 6.1         %%%%%%%%%%%%%%%%%%%%%%%%%%
%%%%%%%%%%%%%%%%%%%%%%%%%%%%%%%%%%%%%%%%%%%%%%%%%%%%%%%%%%%%%%%%%%%
%%%%%%%%%%%%%%%%%%%%%%%%%%%%%%%%%%%%%%%%%%%%%%%%%%%%%%%%%%%%%%%%%%%



Полученная конструкцией Лебега в этом случае мера называется
\textbf{мерой Лебега --- Стильтьеса}.

\begin{center}
\textbf{Измеримые функции}
\end{center}


%%%%%%%%%%%%%%%%%%%%%%%%%%%%%%%%%%%%%%%%%%%%%%%%%%%%%%%%%%%%%%%%%
%
%
%                       исправление 4
%
%
%
%%%%%%%%%%%%%%%%%%%%%%%%%%%%%%%%%%%%%%%%%%%%%%%%%%%%%%%%%%%%%%%%%

$\sum_{k} y_k \mu E_k$

$E_k = \{x: y_{k-1} \leqslant f(x) \leqslant y_k\}$

Надо, чтобы $E_k$ были измеримыми.

\textbf{Определение.} \quad $\mu \:-\:  \delta$-аддитивная мера,
$\: \EuScript{M}$ --- класс измеримых множеств; $f: X \rightarrow
\mathbb{R}$ --- измерима, если $\: \forall c \; E_c = \{ x \in X,
f(x) < C\} \in \EuScript(M)$

\textbf{Задача.} \quad Доказать, что будут измеримы множества, у
которых $f(X) \leqslant C, f(X) > C, f(X) \geqslant C\;$ для
$f(x)$ -- измеримых.
