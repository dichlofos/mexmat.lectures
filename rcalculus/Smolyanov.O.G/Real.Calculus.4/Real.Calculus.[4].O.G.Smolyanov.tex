\documentclass[12pt,titlepage]{article}
\usepackage[cp1251]{inputenc}
\usepackage[russian]{babel}
\usepackage{amsmath, amsthm, amsfonts, amssymb}
\usepackage{graphicx}

\newcounter{lec}
\renewcommand{\thelec}{\arabic{lec}}
\newcommand*{\lecture}{\refstepcounter{lec}\vspace{20pt}
\begin{center}{\rmfamily\textsc{Лекция \thelec.}}
\end{center}}

\newcounter{tema}
\renewcommand{\thetema}{\arabic{tema}}
\newcommand*{\tema}[1]{\vspace{10pt}
\begin{center}{\textbf{\refstepcounter{tema}
\textsc{\thetema. #1.}}}\vspace{7pt}
\end{center}}

\newcommand{\prim}{\vspace{5pt}\noindent\textbf{Примеры.}}
\newcommand{\svoy}{\vspace{5pt}\noindent\textbf{Свойства.}}

\newtheorem{theorem}{Теорема}[tema]
\renewcommand{\thetheorem}{\thetema.\arabic{theorem}}

\newtheorem{sled}{Следствие}[tema]
\renewcommand{\thesled}{\thetema.\arabic{sled}}

\newtheorem{lemm}{Лемма}[tema]
\renewcommand{\thelemm}{\thetema.\arabic{lemm}}

\newtheorem{predl}{Предложение}[tema]
\renewcommand{\thepredl}{\thetema.\arabic{predl}}

\theoremstyle{definition}
\newtheorem{defen}{Определение}[tema]
\newtheorem*{zam}{Замечание}

\newcommand*{\p}[1]{#1\nobreak\discretionary{}{\hbox{$\mathsurround=0pt #1$}}{}}

\begin{document}
\begin{titlepage}
\begin{center}
\vspace*{-20pt}
\textsf{%
{\Large МОСКОВСКИЙ ГОСУДАРСТВЕННЫЙ УНИВЕРСИТЕТ \\%
\vspace{5pt}%
имени М.В.ЛОМОНОСОВА}\\%
\vspace{25pt} %
{\Large Механико--математический факультет}\\%
\vspace{15pt} %
{\large Кафедра Теории функций и функционального анализа\\}%
\vspace{40pt}%
\includegraphics{mmlogo.2}\\ %
\vspace{40pt}%
{\LARGE\textbf{Курс лекций по действительному анализу\\} }%
\vspace{25pt} %
Лектор --- Олег Георгиевич Смолянов\\ %
\vspace{20pt}%
Летописец --- Бибиков Павел Витальевич (группа 212)\\
телефон: 137-45-97\\
e-mail: {\normalfont \verb"tsdtp4u@proc.ru"}\\
\vspace{40pt}}%
II курс, 3 семестр, 2 поток (2006 -- 2007 гг.)
\end{center}
\end{titlepage}
%---------------------------Lecture 1-----------------------------------------------%
\lecture

\vspace{-25pt}

\tema{Кольца и полукольца}

\begin{defen}
Пусть $\Omega$ --- фиксированное множество, тогда \emph{кольцом $S$
подмножеств $\Omega$} называется всякая непустая совокупность
подмножеств $\Omega$ со следующими свойствами:

(1) $A, B\in S\Rightarrow A\cap B\in S$,

(2) $A, B\in S\Rightarrow A\bigtriangleup B\in S$.
\end{defen}

Если положить $A\bigtriangleup B=A+B$, $A\cap B=A\cdot B$, то будут
выполнены все аксиомы кольца. Такие кольца называются
\emph{булевыми}.

Есть другие эквивалентные условия:

(3) $A, B\in S\Rightarrow A\cup B\in S$,

(4) $A, B\in S\Rightarrow A\setminus B\in S$.

\begin{predl}
$\{(1), (2)\}\Leftrightarrow \{(3), (4)\}$.
\end{predl}

\begin{proof}
$\{(1), (2)\}\Rightarrow \{(3), (4)\}$:
\begin{align*}
A\cup B&=(A\bigtriangleup B)\bigtriangleup(A\cap B),\\
A\setminus B&=(A\cup B)\bigtriangleup B.
\end{align*}

$\{(3), (4)\}\Rightarrow \{(1), (2)\}$:
\begin{align*}
A\bigtriangleup B&=(A\setminus B)\cup (B\setminus A), \\
A\cap B&=(A\cup B)\bigtriangleup (A\bigtriangleup B).
\end{align*}
\end{proof}

Можно ослабить условие $(3)$:

$(3')$ $A, B\in S, A\cap B=\varnothing \Rightarrow A\cup B\in S$.

\begin{predl}
$\{(3), (4)\}\Leftrightarrow \{(3'), (4)\}$.
\end{predl}

\begin{proof}
$\{(3), (4)\}\Rightarrow \{(3'), (4)\}$ --- очевидно.

$\{(3'), (4)\}\Rightarrow \{(3), (4)\}$: $A\cup B=(A\setminus B)\cup
B$.
\end{proof}

\begin{defen}
\emph{Полукольцо $\mathcal{P}$ в $\Omega$} --- это совокупность
подмножеств $\Omega$ со следующими свойствами:

(1) $\varnothing\in \mathcal{P}$,

(2) $A,B\in \mathcal{P}\Rightarrow A\cap B\in\mathcal{P}$,

(3) $A,B\in\mathcal{P}, B\subset A\Rightarrow \exists\,
n\in\mathbb{N}, \; A_1,\ldots,A_n\in\mathcal{P}: A\setminus
B=\bigsqcup\limits_{j=1}^n A_j$.
\end{defen}

1. Если $S$ --- кольцо, то $S$ --- полукольцо.

2. Пусть $S$ --- кольцо, тогда $\varnothing\in S$: $\exists\, A\in
S\Rightarrow \varnothing=A\setminus A\in S$.

Если $\Omega\in S$, то $\Omega$ играет роль 1 в том смысле, что
$\forall\, A\in S\; A\cap \Omega=A$.

\begin{defen}
Полукольцо с единицей называется \emph{полуалгеброй по\-дмножеств},
а кольцо с единицей --- \emph{алгеброй}.

\emph{$\sigma$-кольцо} --- это кольцо $S$, обладающее свойством:
$A_1,A_2,\ldots\in S\p\Rightarrow \bigcup\limits_{j=1} A_j\in S$.

Алгебра, которая является $\sigma$-кольцом, называется
\emph{$\sigma$-алгеброй подмножеств}.

\emph{$\delta$-кольцо} --- это кольцо $S$, обладающее свойством:
$A_1,A_2,\ldots\in S\p\Rightarrow \bigcap\limits_{j=1} A_j\in S$.
\end{defen}

\prim

1. $\Omega=\mathbb{R}^1$, $\mathcal{P}=\{(a;b), [a;b), (a;b],
[a;b]\mid a\leqslant b\in\mathbb{R}^1\}$ --- полукольцо, но не
кольцо и не полуалгебра.

2. $\Omega=\mathbb{R}^1$, $S$ --- множество конечных объединений
элементов из $\mathcal{P}$, где $\mathcal{P}$ --- полукольцо.

3. Если $\mathfrak{a}$ --- алгебра, являющаяся $\sigma$-кольцом, то
$\mathfrak{a}$ --- $\sigma$-алгебра.

4. Всякое $\sigma$-кольцо является $\delta$-кольцом. Обратное
неверно: пусть $\Omega\p=\mathbb{R}^1$, $S$ --- множество всех
ограниченных подмножеств из $\Omega$. Тогда $S$ --- $\delta$-кольцо,
но не $\sigma$-кольцо.

5. Не всякое $\sigma$-кольцо является алгеброй: пусть $S$ ---
множество всех не более чем счетных подмножеств из $\Omega$. Тогда
$S$ --- $\sigma$-кольцо, но не алгебра.

\tema{Мера и ее счетно аддитивность}

\begin{defen}
\emph{Мерой $\nu$} называется функция, область определения которой
является полукольцом $\mathcal{P}$ подмножеств некоторого множества,
принимающая числовые значения и обладающая свойством: $\forall\,
A_j\p\in\mathcal{P}, j=1,\ldots,n$, если $A_i\cap A_k=\varnothing$
при $i\neq k$ и $\bigsqcup\limits_{j=1}^n A_j\in\mathcal{P}$, то
$$\nu\Big(\bigsqcup\limits_{j=1}^n A_j\Big)=\sum\limits_{j=1}^n \nu
A_j.$$ Это свойство меры называется \emph{аддитивным}.
\end{defen}

\begin{zam}
1. $n$ --- произвольное число (его нельзя заменить на 2).

2. Если $\mathcal{P}$ --- кольцо, то достаточно потребовать, чтобы
$\forall\, A_1, A_2\in\mathcal{P}: A_1\cap A_2=\varnothing$ имеем
$\nu(A_1\sqcup A_2)=\nu A_1+\nu A_2$.
\end{zam}

\begin{defen}
Мера $\nu$ называется \emph{счетно аддитивной}, если $$\forall\,
A_1,A_2,\ldots\in\mathcal{P} : \bigsqcup\limits_{j=1}^\infty
A_j\in\mathcal{P} \Rightarrow \nu\Big(\bigsqcup\limits_{j=1}^\infty
A_j\Big)=\sum\limits_{j=1}^\infty\nu A_j.$$

Мера \emph{неотрицательна}, если ее значения неотрицательны.
\end{defen}

\prim

1. Пусть $\mathcal{P}$ --- полукольцо подмножеств $\mathbb{R}^1$
(см. выше) и $\nu((a;b))\p=\nu([a;b))=\nu((a;b])=\nu([a;b])=b-a$,
если $b\geqslant a$. Такая мера называется \emph{мерой Лебега}.

2. \emph{Мера Лебега-Стильтьеса}: пусть $f$ --- неубывающая на
$\mathbb{R}^1$ функция, тогда
\begin{align*}
\nu_{LS}^f([a;b))&=f(b-0)-f(a-0),\\
\nu_{LS}^f((a;b])&=f(b+0)-f(a+0),\\
\nu_{LS}^f([a;b])&=f(b+0)-f(a-0),\\
\nu_{LS}^f((a;b))&=f(b-0)-f(a+0).
\end{align*}
%-----------------------------------------------------------------------------------%

%---------------------------Lecture 2-----------------------------------------------%
\lecture

\begin{defen}
\emph{Кольцо, порожденное полукольцом $\mathcal{P}$} --- это
минимальное кольцо, содержащее $\mathcal{P}$:
$S(\mathcal{P})=\bigcap\limits_{S\supset\mathcal{P}}S$.
\end{defen}

\begin{predl}
Пусть $\mathcal{P}$ --- полукольцо. Тогда порожденное им кольцо
$S(\mathcal{P})$ --- это множество всевозможных множеств вида
$\bigsqcup\limits_{j=1}^n A_j$, где $n\in\mathbb{N}$,
$A_j\in\mathcal{P}$.
\end{predl}

\begin{proof}
1. Докажем, что семейство множеств $\Big\{\bigsqcup\limits_{j=1}^n
A_j\mid n\p\in\mathbb{N}, j=1,\ldots,n \Big\}$ --- это кольцо.
Имеем:
\begin{align*}
\Big(\bigsqcup\limits_{j=1}^n
A_j\Big)\sqcup\Big(\bigsqcup\limits_{k=1}^r
B_k\Big)&=\bigsqcup\limits_{l=1}^m C_l,\\
\Big(\bigsqcup\limits_{j=1}^n A_j\Big)\setminus
\Big(\bigsqcup\limits_{k=1}^r B_k\Big)&=
\bigsqcup\limits_{j=1}^n\Big(A_j\setminus\bigsqcup\limits_{k=1}^r
B_k\Big)=\\
&=\bigsqcup\limits_{j=1}^n (((A_j\setminus B_1)\setminus
B_2)\setminus\ldots\setminus
B_r)=\\
&=\bigsqcup\limits_{j=1}^n\Big(\bigsqcup\limits_{p=1}^s
E_p\Big),\quad E_s\in\mathcal{P}.
\end{align*}

2. Докажем минимальность: пусть $S_0\supset \mathcal{P}$, тогда
$\forall\, n, \forall \,A_j\in\mathcal{P}, j\p=1,\ldots,n$ \;
$\bigcup\limits_{j=1}^n A_j\in S_0\Rightarrow S_0\supset
S(\mathcal{P})$.
\end{proof}

\begin{predl}
Пусть $\nu\colon \mathcal{P}\to\mathbb{R}^+$ --- мера на
$\mathcal{P}$. Тогда $\exists !\, \bar{\nu}\colon
S(\mathcal{P})\p\to\mathbb{R}^+$ --- мера на $S(\mathcal{P})$,
такая, что $\bar{\nu}\mid_\mathcal{P}=\nu$.
\end{predl}

\begin{proof}
Вначале докажем, что существует не более чем одно продолжение. Пусть
$A\in S(\mathcal{P})$, тогда $A=\bigsqcup\limits_{j=1}^n P_j$,
$P_j\in\mathcal{P}$. Отсюда
$\bar{\nu}A\p=\sum\limits_{j=1}^n\bar{\nu}P_j=\sum\limits_{j=1}^n
\nu P_j$.

Теперь проверим, что введенная выше функция $\bar{\nu}$ является
мерой. Во-первых, докажем, что значение $\bar{\nu}A$ не зависит от
множеств, на которые раскладывается $A$. Пусть
$A=\bigsqcup\limits_{j=1}^n P_j=\bigsqcup\limits_{k=1}^m B_k$, тогда
$A=\bigsqcup\limits_{j,k}^{n,m}P_j\cap B_k$ и поскольку
$P_j=\bigsqcup\limits_{k=1}^n (P_j\cap B_k)$, то
$\sum\limits_{j=1}^n \nu
P_j=\sum\limits_{j=1}^n\sum\limits_{k=1}^m\nu(P_j\cap B_k)$.
Аналогично, $\sum\limits_{k=1}^m \nu
B_k=\sum\limits_{j=1}^n\sum\limits_{k=1}^m\nu(P_j\cap B_k)$. Поэтому
$\sum\limits_{j=1}^n\nu P_j=\sum\limits_{k=1}^m\nu B_k$.

Во-вторых, докажем аддитивность. По определению,
\begin{align*}
\bar{\nu}\Big(\Big(\bigsqcup\limits_{j=1}^n A_j\Big)\sqcup
\Big(\bigsqcup\limits_{k=1}^r B_k\Big)\Big)&=\sum\limits_{j=1}^n\nu
A_j+\sum\limits_{k=1}^r \nu
B_k=\\
&=\sum\limits_{j=1}^n\bar{\nu}A_j+\sum\limits_{k=1}^r\bar{\nu}B_k=
\bar{\nu}\Big(\bigsqcup\limits_{j=1}^n
A_j\Big)+\bar{\nu}\Big(\bigsqcup\limits_{k=1}^r B_k\Big).
\end{align*}
\end{proof}

\begin{theorem}
Если исходная мера счетно аддитивна, то ее продолжение тоже счетно
аддитивно.
\end{theorem}

\begin{proof}
Пусть $A\in S(\mathcal{P})$, $A=\bigsqcup\limits_{j=1}^\infty A_j$,
$A_j\in S(\mathcal{P})$. Тогда $\forall\, j$
$A_j=\bigsqcup\limits_{k=1}^{k(j)}A_{jk}$, $A_{jk}\in\mathcal{P}$.
Значит,
$\bar{\nu}A=\bar{\nu}\Big(\bigsqcup\limits_{j,k}A_{jk}\Big)$,
поэтому $$\bar{\nu}A=\sum\limits_{j,k}\nu
A_{jk}=\sum\limits_{j=1}^\infty \sum\limits_{k=1}^{k(j)}\nu
A_{jk}=\sum\limits_{j=1}^\infty\bar{\nu}A_j.$$
\end{proof}

\begin{predl}
Пусть $\nu$ --- мера на кольце $S$. Тогда $\nu$ счетно аддитивна
$\Leftrightarrow$ для всех $A, A_j\in S$ выполняется следующее
свойство: если $\bigcup\limits_{j=1}^\infty A_j\supset A$, то
$\sum\limits_{j=1}^\infty \nu A_j\geqslant \nu A$.
\end{predl}

\begin{proof}
Пусть $\nu$ счетно аддитивна и $\bigcup\limits_{j=1}^\infty
A_j\supset A$, где $A, A_j\in S$. Тогда $$A=(A\cap
A_1)\sqcup(A\cap(A_2\setminus A_1))\sqcup(A\cap(A_3\setminus(A_1\cup
A_2)))\sqcup\ldots,$$ поэтому
\begin{multline*}
\nu A=\nu(A\cap A_1)+\nu(A\cap(A_2\setminus
A_1))+\nu(A\cap(A_3\setminus(A_1\cup
A_2)))+\ldots\leqslant\\
\leqslant \nu A_1+\nu A_2+\nu A_3+\ldots=\sum\limits_{j=1}^\infty
\nu A_j.
\end{multline*}

Обратно, пусть $B=\bigsqcup\limits_{k=1}^\infty B_k$, где $B, B_k\in
S$. Докажем, что $\nu B\p=\sum\limits_{k=1}^\infty \nu B_k$.
Заметим, что для любого $n$ выполняется равенство
$$B=\Big(\bigsqcup\limits_{k=1}^n
B_k\Big)\sqcup\Big(B\setminus\Big(\bigsqcup\limits_{k=1}^n
B_k\Big)\Big),$$ поэтому в силу аддитивности меры $$\nu
B=\sum\limits_{k=1}^n \nu
B_k+\nu\Big(B\setminus\Big(\bigsqcup\limits_{k=1}^n
B_k\Big)\Big)\geqslant \sum\limits_{k=1}^\infty \nu B_k.$$ В то же
время $\nu B\leqslant\sum\limits_{k=1}^\infty \nu B_k$. Отсюда
получаем, что $\nu B=\sum\limits_{k=1}^\infty \nu B_k$ и мера $\nu$
счетно аддитивна.
\end{proof}

Рассмотрим полукольцо $\mathcal{P}$ подмножеств $\mathbb{R}^1$ с
мерой Лебега $\nu$. Тогда верна следующая

\begin{theorem}
Мера Лебега счетно аддитивна.
\end{theorem}

\begin{proof}
Вначале заметим, что если $(a;b)=(a;c]\cup(c;b)$, то
$\nu((a;b))=b-a=(c-a)+(b-c)=\nu((a;c])+\nu((c;b))$. Аналогичное
равенство можно записать и для произвольного конечного количества
интервалов разбиения отрезка $(a;b)$.

Пусть теперь $\mathcal{P}\ni A=\bigsqcup\limits_{j=1}^\infty A_j$,
где $A_j\in\mathcal{P}$. Зафиксируем произвольное число
$\varepsilon>0$. Будем считать, что $A=(a;b]$ --- полуинтервал. В
таком случае $\exists\, [\alpha;\beta]\subset A:
\nu([\alpha;\beta])>\nu A-\varepsilon$ и $\exists\,
(\alpha_j;\beta_j)\supset A_j: \nu((\alpha_j;\beta_j))\p<\nu
A_j+\frac{\varepsilon}{2^j}$. Отсюда $[\alpha;\beta]\subset A\subset
\bigcup\limits_{j=1}^\infty(\alpha_j;\beta_j)$. Значит, $\exists\,
n: [\alpha;\beta]\subset\bigcup\limits_{j=1}^n (\alpha_j;\beta_j)$,
поэтому $\nu([\alpha;\beta])\leqslant \sum\limits_{j=1}^n
\nu((\alpha_j;\beta_j))<\sum\limits_{j=1}^n\big(\nu
A_j+\frac{\varepsilon}{2^j}\big)<\sum\limits_{j=1}^\infty \nu A_j +
2\varepsilon$ и $\nu A\leqslant \sum\limits_{j=1}^\infty \nu A_j$.
\end{proof}

\begin{defen}
Рассмотрим алгебру $S$ со счетно аддитивной мерой $\nu$. $T(\Omega)$
--- это множество всех подмножеств $\Omega$. Легко видеть, что
$T(\Omega)$ является $\sigma$-алгеброй. Определим \emph{внешнюю меру
$\nu^*\colon T(\Omega)\to \mathbb{R}^1$}, $\nu^*
A\p=\inf\limits_{A\subset\bigcup\limits_j
A_j}\sum\limits_{j=1}^\infty \nu A_j$, где $A\in T(\Omega)$.

Множество $A\subset \Omega$ называется \emph{$\nu$-измеримым}, если
$\forall\, \varepsilon>0 \quad\exists\, B\in S:
\nu^*(A\bigtriangleup B)<\varepsilon$.
\end{defen}
%-----------------------------------------------------------------------------------%

%---------------------------Lecture 3-----------------------------------------------%
\lecture

\vspace{-30pt}

\tema{Теорема Каратеодори}

\begin{theorem}[Каратеодори]
Пусть $\nu$ --- счетно аддитивная неотрицательная мера на алгебре
$S$ подмножеств $\Omega$, $\sigma(S)$ --- $\sigma$-алгебра,
порожденная $S$. Тогда $\exists !\,
\bar{\nu}\colon\sigma(S)\to\mathbb{R}^1$
--- счетно аддитивная мера на $\sigma(S)$, такая, что
$\bar{\nu}\mid_S=\nu$.
\end{theorem}

\begin{proof}
Пусть $\mathfrak{a}$ --- множество $\nu$-измеримых подмножеств
$\Omega$. Докажем, что на самом деле $\bar{\nu}=\nu^*$, где $\nu^*$
--- внешняя мера. Для этого мы докажем следующие утверждения:

\begin{itemize}
  \item $S\subset\mathfrak{a}$;
  \item $\mathfrak{a}$ --- $\sigma$-алгебра;
  \item сужение $\nu^*$ на $\mathfrak{a}$ счетно аддитивно;
  \item $\nu^*\mid_S=\nu$;
  \item $\nu^*$ --- единственное продолжение, удовлетворяющее
  условиям теоремы.
\end{itemize}

Введем на множестве $T(\Omega)$ \emph{полуметрику}:
$\rho(A,B)=\nu^*(A\bigtriangleup B)$. Функция $\rho$ удовлетворяет
следующим свойствам:

1) $\rho(A,B)=\rho(B,A)$ (очевидно).

2) $\rho(A, B)\geqslant 0$, причем $\rho(A,B)=0\nLeftrightarrow
A=B$.

3) $\rho(A,C)\leqslant\rho(A,B)+\rho(B,C)$ (неравенство
треугольника): действительно, из определения внешней меры следует,
что если $A\subset\bigcup\limits_{j=1}^\infty A_j$, то
$\nu^*A\leqslant\sum\limits_{j=1}^\infty \nu^*A_j$, а т.к.
$A\bigtriangleup C\subset(A\bigtriangleup B)\cup(B\bigtriangleup
C)$, то $$\rho(A,C)=\nu^*(A\bigtriangleup
C)\leqslant\nu^*(A\bigtriangleup B)+\nu^*(B\bigtriangleup
C)=\rho(A,B)+\rho(B,C).$$

Заметим, что из свойства 3) следует, что
$$|\rho(A,C)-\rho(A,B)|\leqslant\rho(C,B).$$ В самом деле, положим
$A=\varnothing$, тогда $$|\rho(\varnothing,C)-\rho(\varnothing,
B)|\leqslant\rho(C,B)\Leftrightarrow
|\nu^*C-\nu^*B|\leqslant\nu^*(C\bigtriangleup B).$$

1. Докажем, что $S\subset\mathfrak{a}$. Действительно,
$A\in\mathfrak{a}\Leftrightarrow\forall\,\varepsilon>0\quad\exists\,
C\in S:\rho(A,C)<\varepsilon$. Если $A\in S$, то положим $C=A$,
тогда $\rho(A,C)=0<\varepsilon$ и $S\subset\mathfrak{a}$.

2. Докажем, что $\mathfrak{a}$ --- алгебра. Понятно, что
$\Omega\in\mathfrak{a}$, т.к. $\Omega\in S$. Необходимо проверить,
что если $A,B\in\mathfrak{a}$, то $A\setminus B\in\mathfrak{a}$ и
$A\cup B\in\mathfrak{a}$. Проверим только первую импликацию (вторая
проверятся аналогично). Т.к. $A,B\in\mathfrak{a}$, то
$\forall\,\varepsilon>0\quad\exists\, A_S,B_S\in S:
\nu^*(A\bigtriangleup A_S)<\varepsilon/2$ и $\nu^*(B\bigtriangleup
B_S)<\varepsilon/2$. Поскольку $(A\setminus
B)\bigtriangleup(A_S\setminus B_S)\subset(A\bigtriangleup
A_S)\cup(B\bigtriangleup B_S)$, то $$\nu^*((A\setminus
B)\bigtriangleup(A_S\setminus B_S))\leqslant\nu^*(A\bigtriangleup
A_S)+\nu^*(B\bigtriangleup B_S)<\varepsilon.$$

3. Докажем\footnote{Счетно аддитивность меры $\nu$ используется
только в этом пункте.}, что $\nu^*\mid_S=\nu$. Очевидно, что если
$A\in S$, то $\nu^*A\p\leqslant\nu A$. Докажем обратное неравенство.
Если $A\subset\bigcup\limits_{j=1}^\infty C_j$, то $\nu
A\leqslant\sum\limits_{j=1}^\infty\nu C_j$, поэтому $\nu
A\leqslant\inf\limits_{A\subset\bigcup\limits_{j=1}^\infty
C_j}\sum\limits_{j=1}^\infty\nu C_j=\nu^*A$. Отсюда получаем, что
$\nu A=\nu^*A$.

Прежде чем двигаться дальше, докажем следующую лемму.

\begin{lemm}
Функция $\mu\colon S\to\mathbb{R}^+$ является мерой тогда и только
тогда, когда $\forall\,A_1,A_2\in S\quad \mu(A_1\cup
A_2)+\mu(A_1\cap A_2)=\mu A_1+\mu A_2$ и $\mu(\varnothing)=0$.
\end{lemm}

\begin{proof}
Сначала установим, что функция $\mu$, удовлетворяющая условию леммы,
будет мерой. Действительно, если $A_1\cap A_2=\varnothing$, то
$\mu(A_1\p\cup A_2)=\mu A_1+\mu A_2$.

Обратно, поскольку $\mu(A_1\cup A_2)=\mu(A_1\setminus A_2)+\mu A_2$,
то $$\mu(A_1\cup A_2)+\mu(A_1\cap A_2)=\mu(A_1\setminus A_2)+\mu
A_2+\mu(A_1\cap A_2)=\mu A_1+\mu A_2.$$
\end{proof}

4. Докажем, что $\nu^*$ является мерой на $\mathfrak{a}$. Согласно
лемме 3.1, достаточно доказать, что
$\forall\,A,C\in\mathfrak{a}\quad \nu^*(A\cup C)+\nu^*(A\cap
C)=\nu^* A+\nu^* C$. По определению, $\forall\,
\varepsilon>0\quad\exists\,A_S,C_S\in S: \nu^*(A\bigtriangleup
A_S)<\varepsilon$ и $\nu^*(C\bigtriangleup C_S)<\varepsilon$. Т.к.
$(A\cup C)\bigtriangleup(A_S\cup C_S)\subset (A\bigtriangleup
A_S)\cup(C\bigtriangleup C_S)$, то $$\nu^*((A\cup
C)\bigtriangleup(A_S\cup C_S))\leqslant\nu^*(A\bigtriangleup
A_S)+\nu^*(C\bigtriangleup C_S)<2\varepsilon.$$ Аналогично,
$$\nu^*((A\cap C)\bigtriangleup(A_S\cap
C_S))\leqslant\nu^*(A\bigtriangleup A_S)+\nu^*(C\bigtriangleup
C_S)<2\varepsilon.$$ По неравенству треугольника
$$\text{$|\nu^*(A\cup C)-\nu^*(A_S\cup C_S)|<2\varepsilon$ и $|\nu^*
(A\cap C)-\nu^*(A_S\cap C_S)|<2\varepsilon$},$$
$$\text{$|\nu^*A-\nu^*A_S|<\varepsilon$ и
$|\nu^*C-\nu^*C_S|<\varepsilon$}.$$ Кроме того, поскольку
$\nu^*A_S=\nu A_S$ и $\nu^*C_S=\nu C_S$, а также $$\nu(A_S\cup
C_S)+\nu(A_S\cap C_S)=\nu A_S+\nu C_S,$$ то $$|\nu^*(A\cup
C)+\nu^*(A\cap C)-\nu^*A-\nu^*C|<6\varepsilon,$$ откуда $\nu^*(A\cup
C)+\nu^*(A\cap C)=\nu^*A+\nu^*C$.

Теперь докажем счетно аддитивность. Очевидно, что если $\mu$ ---
мера на кольце $S$ и при $A,A_j\in S$ из того, что
$A\subset\bigcup\limits_{j=1}^\infty A_j$ следует, что $\mu
A\p\leqslant\sum\limits_{j=1}^\infty\mu A_j$, то мера $\mu$ счетно
аддитивна. При $\mu=\nu^*$ получаем требуемое.

5. Докажем, что $\mathfrak{a}$ --- $\sigma$-алгебра. Для этого
достаточно проверить, что если $A_j\in\mathfrak{a}$, то
$\bigsqcup\limits_{j=1}^\infty A_j\in\mathfrak{a}$. Поскольку
$\forall\, j\quad A_j\subset\Omega\in S\subset\mathfrak{a}$ и
$$\forall\, n\quad\Omega=\Big(\bigsqcup\limits_{j=1}^n
A_j\Big)\sqcup\Big(\Omega\setminus\bigsqcup\limits_{j=1}^n
A_j\Big),$$ то в силу аддитивности получаем, что
$$\nu^*\Omega=\nu^*\Big(\bigsqcup\limits_{j=1}^n A_j\Big)+\nu^*\Big
(\Omega\setminus\bigsqcup\limits_{j=1}^n
A_j\Big)\geqslant\sum\limits_{j=1}^n\nu^* A_j,$$ откуда
$\sum\limits_{j=1}^\infty \nu^*A_j<\infty$.

Зафиксируем произвольное $\varepsilon>0$. Тогда $$\exists\,
n:\nu^*\Big(\bigsqcup\limits_{j=n}^\infty
A_j\Big)\leqslant\sum\limits_{j=n}^\infty\nu^*A_j<\varepsilon.$$
Т.к. $\bigsqcup\limits_{j=1}^n A_j\in\mathfrak{a}$, то $\exists\,
C\in S:\nu^*\Big(\Big(\bigsqcup\limits_{j=1}^n
A_j\Big)\bigtriangleup C\Big)<\varepsilon$. По неравенству
треугольника получаем, что
$$\rho\Big(\bigsqcup\limits_{j=1}^\infty A_j,C\Big)\leqslant
\rho\Big(\bigsqcup\limits_{j=1}^\infty A_j,\bigsqcup\limits_{j=1}^n
A_j\Big)+\rho\Big(\bigsqcup\limits_{j=1}^n A_j,
C\Big)<2\varepsilon,$$ поэтому $\bigsqcup\limits_{j=1}^\infty
A_j\in\mathfrak{a}$.

6. Докажем единственность. Пусть $\mu$ --- другое продолжение меры
$\nu$ с $S$ на $\mathfrak{a}$, удовлетворяющее условиям теоремы.
Введем <<расстояние>>\footnote{На самом деле, это никакое не
расстояние.} на $\mathfrak{a}$: $\rho_\mu(A,B)=\mu(A\bigtriangleup
B)$. Тогда $|\mu A-\mu B|<\mu(A\bigtriangleup B)$.

Пусть $A\in\mathfrak{a}$, тогда
$\forall\,\varepsilon>0\quad\exists\, C\in S:
|\nu^*A-\nu^*C|<\rho(A,C)<\varepsilon$. Поскольку
$\rho_\mu(A,C)\leqslant\rho(A,C)$, то $$|\mu A-\mu
C|<\rho_\mu(A,C)\leqslant\rho(A,C)<\varepsilon.$$ Т.к. $\mu C=\nu
C$, то по неравенству треугольника $|\nu^* A-\mu A|<2\varepsilon$,
поэтому $\mu A=\nu^*A$.

Осталось доказать неравенство
$\rho_\mu(A_1,A_2)\leqslant\rho_\nu(A_1,A_2)$. В самом деле,
$$\rho_\nu(A_1,A_2)=\inf\limits_{\bigcup\limits_j C_j\supset A_1
\bigtriangleup A_2} \sum\nu C_j,$$ но
$\rho_\nu(A_1,A_2)=\sum\limits_{j=1}^\infty \nu C_j$, поэтому
$\mu(A_1\bigtriangleup A_2)\leqslant\sum\limits_{j=1}^\infty \mu
C_j$ и $$\rho_\mu(A_1,A_2)=\mu(A_1\bigtriangleup
A_2)\leqslant\inf\sum\limits_j \nu C_j=\nu^*(A_1\bigtriangleup
A_2)=\rho_\nu(A_1,A_2).$$
\end{proof}

\clearpage
%-----------------------------------------------------------------------------------%

%---------------------------Lecture 4-----------------------------------------------%
\lecture

\vspace{-30pt}

\tema{Измеримые пространства и пространства с мерой}

\begin{defen}
Пара $(\Omega,\mathfrak{a})$ называется \emph{измеримым
пространством}, если $\mathfrak{a}$ --- это $\sigma$-алгебра
подмножеств $\Omega$. Элементы $\mathfrak{a}$ называются
\emph{измеримыми множествами}.

Тройка $(\Omega,\mathfrak{a},\nu)$ называется \emph{пространством с
мерой}, если $\nu$ --- это счетно аддитивная мера на
$\sigma$-алгебре $\mathfrak{a}$.
\end{defen}

\begin{zam}
Далее \emph{все} меры мы будем считать счетно аддитивными и
неотрицательными.
\end{zam}

\begin{defen}
Пусть $(\Omega,\mathfrak{a},\nu)$ --- пространство с мерой. Мера
$\nu$ называется \emph{$\sigma$-конечной}, если $\exists\, \Omega_j:
\Omega=\bigcup\limits_{j=1}^n \Omega_j$, причем
$\Omega_j\in\mathfrak{a}$ и $\nu\Omega_j<\infty$.
\end{defen}

\begin{defen}
Мера $\nu$  на $\sigma$-алгебре $\mathfrak{a}$ называется
\emph{полной}, если выполнено следующее условие: $\forall\,
A\in\mathfrak{a}$ если $\nu A=0$, то всякое подмножество $A$ само
измеримо (и тогда его мера также равна 0).
\end{defen}

\begin{predl}
Продолженная мера $\bar{\nu}$ в теореме Каратеодори полна.
\end{predl}

\begin{proof}
Пусть $A\in\mathfrak{a}$ и $\bar{\nu}A=0$, тогда
$\forall\,\varepsilon>0\quad\exists\, B\in S:
\rho_\nu(A,B)<\varepsilon$ и $\nu^*A=0$. Проверим, что если
$C\subset A$, то $C\in\mathfrak{a}$ и $\nu^*C=0$. Заметим, что если
$C$ измеримо, то $\nu^*C=\bar{\nu}C=0$, т.к.
$$\forall\,\varepsilon>0\quad \nu^*C=\nu^*(C\bigtriangleup\varnothing)
=\rho(C,\varnothing)<\varepsilon.$$
\end{proof}

\begin{predl}
Если $(\Omega,\mathfrak{a},\nu)$ --- пространство с мерой, то
существует такое пространство с мерой
$(\Omega,\bar{\mathfrak{a}},\bar{\nu})$, что
$\mathfrak{a}\subset\bar{\mathfrak{a}}$ и $\bar{\nu}$ ---
продолжение $\nu$ на $\bar{\mathfrak{a}}$, причем $\bar{\nu}$ ---
полная мера.
\end{predl}

\begin{proof}
В самом деле, $$A\in\bar{\mathfrak{a}}\Leftrightarrow \exists\,
C\in\mathfrak{a}: \nu^*(A\bigtriangleup C)=0,\quad \nu
C=\bar{\nu}A.$$
\end{proof}

\begin{defen}
В таком случае $\bar{\nu}$ называется \emph{пополнением} $\nu$.
\end{defen}

Пусть $(\Omega,\mathfrak{b},\mu)$ --- пространство с полной мерой,
причем $\mathfrak{b}\supset\mathfrak{a}$, $\mu$ --- продолжение
$\nu$ на $\mathfrak{b}$, тогда
$\mathfrak{b}\supset\bar{\mathfrak{a}}$ и $\mu$ --- это продолжение
меры $\bar{\nu}$. Т.е. пополнение меры является \emph{минимальным}
продолжением.

Пусть $\Omega$ --- метрическое пространство.

\begin{defen}
\emph{$\sigma$-алгеброй $\mathfrak{B}$ борелевских подмножеств
метрического пространства} называется $\sigma$-алгебра, порожденная
всеми открытыми множествами (или, что то же самое, $\sigma$-алгебра,
порожденная замкнутыми множествами).
\end{defen}

\begin{predl}
$\sigma$-алгебра борелевских подмножеств $\mathbb{R}^1$

1) совпадает с $\sigma$-алгеброй, порожденной всеми интервалами
$(\alpha;\beta)$;

2) совпадает с $\sigma$-алгеброй, порожденной всеми отрезками
$[\alpha;\beta]$;

3) совпадает с $\sigma$-алгеброй, порожденной всеми полуинтервалами
вида $(\alpha;\beta]$ или $[\alpha;\beta)$;

4) совпадает с $\sigma$-алгеброй, порожденной всеми лучами
$(-\infty;\alpha)$ или $(\alpha;+\infty)$;

5) совпадает с $\sigma$-алгеброй, порожденной всеми лучами
$(-\infty;\alpha]$ или $[\alpha;+\infty)$.
\end{predl}

\begin{proof}
Достаточно доказать, что $\sigma$-алгебра, порожденная одним из
указанных пяти способов, содержит все открытые интервалы.

1) $(\alpha;+\infty)=\bigcup\limits_{n=1}^\infty\left[\alpha+\frac 1
n;+\infty\right)$;

2) $(-\infty;\beta)=\mathbb{R}^1\setminus[\beta;+\infty)$;

3) $(\alpha;\beta)=(-\infty;\beta)\cap(\alpha;+\infty)$;

4) $V\subset\mathbb{R}^1$, $V$ открыто.

$\forall\, x\in V\quad\exists\, (\alpha;\beta)\subset V,
x\in(\alpha;\beta)$. Немного уменьшив этот интервал, можно считать,
что $\alpha,\beta\in\mathbb{Q}$, поэтому $V$ есть объединение
(счетных) интервалов, т.е. $\sigma$-алгебра, порожденная множествами
вида $[\beta;+\infty)$, совпадает с $\sigma$-алгеброй борелевский
подмножеств.
\end{proof}

Пусть мера $\nu$ задана на кольце $S$. Пусть $\mathfrak{a}$ --- это
совокупность $\nu^*$-измеримых подмножеств. Можно проверить, что
$\mathfrak{a}$ --- это $\delta$-кольцо. Построим
$\bar{\mathfrak{a}}$: $A\in\bar{\mathfrak{a}}\Leftrightarrow
\forall\, C\in\mathfrak{a}\quad C\cap A\in\mathfrak{a}$.

\begin{defen}
Введем следующую функцию:
$\bar{\bar{\nu}}A=\sup\limits_{C\in\mathfrak{a}}\bar{\nu}(A\cap C)$.
Правда, $\bar{\bar{\nu}}\colon S\to [0;+\infty]$, т.е.
$\bar{\bar{\nu}}$ может равняться бесконечности.
\end{defen}

\prim

$\Omega=\mathbb{R}^1$, $\mathcal{P}$ --- полукольцо промежутков.
Тогда
$\bar{\bar{\nu}}\mathbb{R}^1=\bar{\bar{\nu}}(\alpha;+\infty)=+\infty$.

\vspace{5pt}

Введем следующие обозначения: $\mathfrak{a}_L$ --- это
$\sigma$-алгебра измеримых по Лебегу множеств, $\nu_L$ --- мера
Лебега.

\begin{predl}
$\forall\, A\in\mathfrak{a}_L$ $\nu_L A>0$. $\exists\, B\subset A:
B\not \in \mathfrak{a}_L$.
\end{predl}

\begin{predl}
$\forall\, A\in\mathfrak{a}_L, \nu_L>0$ $\exists\, B\subset
A:\nu^*B=\nu_L A$ и $\nu^*(A\setminus B)\p=\nu_L A$.
\end{predl}
%-----------------------------------------------------------------------------------%

%---------------------------Lecture 5-----------------------------------------------%
\lecture

Рассмотрим пространство с мерой $(\mathbb{R}^1, \mathfrak{a}, \nu)$.

\begin{lemm}
\label{rashional.points}$\forall\, A\in\mathfrak{a}$, $\nu A>0\quad
\exists\, a\neq b\in A: |a-b|\in\mathbb{Q}$.
\end{lemm}

\begin{proof}
Поскольку $\mathbb{R}^1=\bigcup\limits_{n=1}^\infty [-n;n]$, то
$A=\bigcup\limits_{n=1}^\infty A\cap[-n;n]$ и $$0<\nu
A\leqslant\sum\limits_{n=1}^\infty \nu(A\cap[-n;n]).$$ Значит,
найдется такое $n\in\mathbb{N}$, что $\nu(A\cap[-n;n])>0$. Положим
$A_n\p=A\cap[-n;n]$.

Допустим, что в $A_n$ нет точек с рациональным расстоянием. Тогда
верно следующее утверждение: $\forall\, a\neq b\in \mathbb{Q}\quad
(A_n+a)\cap(A_n+b)=\varnothing$. В самом деле, если $\exists\,
c\in(A_n+a)\cap(A_n+b)$, то $\exists\, r_1,r_2\in A_n: r_1=c-a$ и
$r_2=c-b$, откуда $r_1-r_2=b-a\in\mathbb{Q}$ --- противоречие.

Поскольку $\bigsqcup\limits_{r\in\mathbb{Q}\cap[0;n]}(A_n+r)\subset
[-n;2n]$, то $$\infty=\sum\nu
A_n=\sum\limits_{r\in\mathbb{Q}\cap[0;n]}\nu(A_n+r)\leqslant
\nu([-n;2n])=3n$$ --- противоречие.
\end{proof}

\begin{predl}
$\forall\, C\in\mathfrak{a}$, $\nu C>0\quad \exists\, D\subset
C:D\not \in \mathfrak{a}$.
\end{predl}

\begin{proof}
Введем на множестве $C$ отношение эквивалентности: $a_1\sim
a_2\Leftrightarrow |a_1-a_2|\in\mathbb{Q}$. Тогда множество $C$
разбивается на классы эквивалентности: $C=\bigcup\limits_\alpha
C_\alpha$. Выберем в каждом $C_\alpha$ по одной точке и обозначим
это множество точек через $D$, т.е. $\forall\,\alpha\quad D\cap
C_\alpha=\{a_\alpha\}$.

Докажем, что множество $D$ неизмеримо. Предположим противное: пусть
$D\in\mathfrak{a}$. Ясно, что расстояние между любыми его точками
иррационально, поэтому по лемме~\ref{rashional.points} $\nu D=0$.

Поскольку $\forall\, a\neq b\in \mathbb{Q}\quad
(D+a)\cap(D+b)=\varnothing$, то $C\subset
\bigsqcup\limits_{r\in\mathbb{Q}}(D+r)$, поэтому $\nu C\leqslant
\sum\limits_{r\in\mathbb{Q}}\nu(D+r)=0$, т.к. $\forall\,
r\in\mathbb{Q}\quad \nu(D+r)=\nu D=0$. Но $\nu C>0$ по условию ---
противоречие.
\end{proof}

\tema{Измеримые функции}

%\vspace{10pt}

%\begin{center}
%\textsc{Измеримые функции}
%\end{center}

%\vspace{7pt}

\begin{defen}
Пусть $(\Omega_1, \mathfrak{a}_1)$ и $(\Omega_2, \mathfrak{a}_2)$
--- измеримые пространства. Функция $f\colon \Omega_1\to\Omega_2$
называется \emph{измеримой}, если $\forall\,A\in\mathfrak{a}_2\quad
f^{-1}A\in\mathfrak{a}_1$.

Возьмем в качестве $(\Omega_2,\mathfrak{a}_2)$ пару
$(\mathbb{R}^1,\mathfrak{B})$. Функция $f\colon
\Omega\to\mathbb{R}^1$ называется \emph{измеримой}, если $\forall\,
B\in\mathfrak{B}\quad f^{-1}B\in\mathfrak{a}$.
\end{defen}

\begin{zam}
В последнем определении множество $\mathfrak{B}$ нельзя заменить на
$\mathfrak{a}_L$.
\end{zam}

\begin{predl}[Критерий измеримости]
Функция $f\colon \Omega\to\mathbb{R}^1$ измерима $\Leftrightarrow$
$\forall\, a\in\mathbb{R}^1\quad
\{x:f(x)<a\}\in\mathfrak{a}$.\footnote{В условии этой теоремы можно
заменить знак <<$<$>> на знаки <<$\leqslant$>>, <<$>$>>,
<<$\geqslant$>>, но НЕ на знак <<$=$>>.}
\end{predl}

\begin{proof}
Пусть функция $f$ измерима. Тогда $\{x:
f(x)<a\}\p=f^{-1}((-\infty;a))$. Поскольку $\forall\, a\quad
(-\infty;a)\in\mathfrak{B}$, то $\{x:f(x)<a\}\in\mathfrak{a}$.

Обратно, рассмотрим множество
$\sigma=\{B\subset\mathbb{R}^1:f^{-1}B\in\mathfrak{a}\}$. Легко
видеть, что $\sigma$ --- это $\sigma$-алгебра: если
$B_1,B_2\in\sigma\Rightarrow f^{-1}(B_1\cup B_2)=f^{-1}B_1\p\cup
f^{-1}B_2\in\mathfrak{a}\Rightarrow B_1\cup B_2\in\sigma$;
$f^{-1}(\mathbb{R}^1)=\Omega\in\mathfrak{a}\Rightarrow
\mathbb{R}^1\in\sigma$.

$\forall\, a\quad f^{-1}((-\infty;a))\in\mathfrak{a}\Rightarrow
\forall\, a\quad (-\infty;a)\in\sigma$. Поскольку
$\sigma(\{(-\infty;a):a\in\mathbb{R}^1\})=\mathfrak{B}=\sigma(\mathbb{R}^1)$,
то $\sigma\supset \mathfrak{B}$, а значит, $\forall\,
B\in\mathfrak{B}\quad f^{-1}B\in\mathfrak{a}$.
\end{proof}

\begin{defen}
Если $\Omega=\mathbb{R}^1$, то удобно взять
$\mathfrak{a}=\mathfrak{a}_L$. Тогда функция $f\colon
\mathbb{R}^1\to\mathbb{R}^1$ называется \emph{измеримой по Лебегу},
если $\forall\, a\in\mathbb{R}^1\quad
\{x\in\mathbb{R}^1:f(x)<a\}\in\mathfrak{a}_L$.\footnote{Множество
$\mathfrak{a}_L$ нельзя заменить на $\mathfrak{B}$.}
\end{defen}

Пусть теперь $f\colon
(\Omega,\mathfrak{a})\to(\mathbb{R}^1,\mathfrak{B})$.

\begin{predl}
Пусть $\forall\, \omega\in\Omega\quad\exists\,
\lim\limits_{n\to\infty}f_n(\omega)=f(\omega)$, где $f_n$ ---
измеримые функции. Тогда функция $f$ тоже измерима.
\end{predl}

\begin{proof}
Положим $\varphi_k(\omega)=\sup\limits_{n\geqslant k}f_n(\omega)$,
тогда $f(\omega)=\inf\limits_k\varphi_k(\omega)$. Докажем, что
$\forall\, a\in\mathbb{R}^1\quad
\{\omega\in\Omega:\varphi_k(\omega)<a\}\in\mathfrak{a}$. В самом
деле, $$\{\omega:\varphi_k(\omega)<a\}=\bigcap\limits_{n\geqslant
k}\{\omega:f_n(\omega)<a\}\in\mathfrak{a},$$ поэтому все $\varphi_k$
измеримы. Поскольку $f(\omega)=\inf\limits_k \varphi_k(\omega)$ и
$\{\omega: \inf\limits_k
\varphi_k(\omega)\p>a\}=\bigcap\limits_k\{\omega:\varphi_k(\omega)\p
>a\}\in\mathfrak{a}$, то $f(\omega)$ измерима.
\end{proof}
%-----------------------------------------------------------------------------------%

%---------------------------Lecture 6-----------------------------------------------%
\lecture

\begin{defen}
Пусть фиксировано измеримое пространство $(\Omega,\mathfrak{a})$.
Функция на измеримом пространстве называется \emph{простой}, если
она принимает конечное число значений.

Если простая функция определена на пространстве с мерой, то
$\nu\{\omega\p\in\Omega\mid f(\omega)\neq 0\}<\infty$.
\end{defen}

\begin{predl}
Функция $f$ является простой $\Leftrightarrow$ $$\exists\,\Omega_j:
\Omega=\bigsqcup\limits_{j=1}^n \Omega_j \quad\text{и}\quad
\forall\,\omega \quad f(\omega)=\sum\limits_{j=1}^n
a_j\gamma_{\Omega_j}(\omega),$$ где
$$\gamma_{\Omega_j}(\omega)=\begin{cases}1,&\text{если $\omega\in\Omega_j$};\\ 0,&\text{иначе}
\end{cases}$$ --- \emph{индикаторная функция множества} $\Omega_j$.
\end{predl}

\begin{proof}
В самом деле, обратная ипмликация очевидна. Пусть теперь функция $f$
проста и $a_1$, \ldots, $a_n$ --- различные ненулевые значения $f$.
Тогда множества $\Omega_j=\{\omega\in\Omega\mid f(\omega)=a_j\}$
удовлетворяют условию.
\end{proof}

\begin{zam}
Условие <<$\Omega=\bigsqcup\limits_{j=1}^n \Omega_j$>> можно
заменить на условие <<$\forall\, j\p\neq k\quad
\Omega_j\cap\Omega_k=\varnothing$>>.
\end{zam}

Очевидно, что если $f_1$ и $f_2$ --- простые функции, то их линейная
комбинация $\alpha f_1+\beta f_2$ и произведение $f_1f_2$ также
будут простыми функциями.

\begin{predl}
Если функция $f(\omega)=\sum\limits_{j=1}^n
a_j\gamma_{\Omega_j}(\omega)$, причем $\forall\, j\p\neq k\quad
\Omega_j\cap \Omega_k=\varnothing$ и $a_j\neq a_k$, то $f$ измерима
$\Leftrightarrow$ все $\Omega_j$ измеримы.
\end{predl}

\begin{proof}
Достаточно провести доказательство для простейшей функции
$f(\omega)=\gamma_{\Omega_1}(\omega)$. Имеем:
$\Omega_1=\{\omega\in\Omega \mid f(\omega)>0\}$, поэтому
$$\forall\, a\in\mathbb{R}^1\quad \{\omega\in\Omega\mid f(\omega)>a\}
=\begin{cases}\Omega_1,&\text{если $a\geqslant 0$};\\
\Omega,& \text{если $a<0$},\end{cases}$$ а значит, $f$ измерима
$\Leftrightarrow$ $\Omega_1$ измеримо.
\end{proof}

\begin{theorem}
\label{fun.approx}Для каждой измеримой функции $f\colon \Omega\to
\bar{\mathbb{R}}$ найдутся простые и измеримые функции $f_n\colon
\Omega\to\mathbb{R}^1$, $n=1,2,\ldots,$ такие, что $\forall\,
\omega\in\Omega\quad f_n(\omega)\to f(\omega)$ при $n\to\infty$.
Если при этом $f\geqslant 0$, то можно добиться того, что
$f_n\geqslant0$ и $f_n\nearrow f$, а если $f$ ограничена, то
$f_n\rightrightarrows f$.
\end{theorem}

\begin{proof}
Нетрудно убедиться, что функции
\begin{multline*}
f_n(x)=n\cdot\gamma_{\{\omega\mid f(\omega)\geqslant n\}}(x)+(-n)
\cdot \gamma_{\{\omega\mid f(\omega)\leqslant
-n\}}(x)+\\
+\frac{1}{2^n}
\sum\limits_{k=-n2^n+1}^{n2^n}(k-1)\gamma_{\{\omega\mid
\frac{k-1}{2^n}\leqslant f(\omega)<\frac{k}{2^n}\}}(x)
\end{multline*}
удовлетворяют условию теоремы.
\end{proof}

\begin{predl}
Если функции $f$ и $g$ измеримы, то их линейная комбинация $\alpha
f+\beta g$ измерима.
\end{predl}

\begin{proof}
В самом деле, если функции $h_1$ и $h_2$ просты и измеримы, то и
функция $h=h_1+h_2$ тоже проста и измерима. Выберем теперь функции
$f_n$ и $g_n$, удовлетворяющие условиям теоремы~\ref{fun.approx},
тогда $f_n\to f$ и $g_n\to g$, поэтому $\alpha f_n+\beta g_n\to
\alpha f+\beta g$ при $n\to\infty$, что и требовалось.
\end{proof}

\tema{Интегрируемые функции}

%\vspace{10pt}

%\begin{center}
%\textsc{Интегрируемые функции}
%\end{center}

%\vspace{7pt}

\begin{defen}
Пусть $(\Omega,\mathfrak{a},\nu)$ --- пространство с
$\sigma$-конечной мерой и $f$ --- простая измеримая функция. Тогда
$f(\omega)=\sum\limits_{j=1}^n a_j\gamma_{\Omega_j}(\omega)$, причем
$\forall\, j\quad \nu\Omega_j<\infty$ и $a_j\in\mathbb{R}^1$.
\emph{Интегралом Лебега функции $f$ по пространству $\Omega$}
называется величина $$\int\limits_\Omega
\!f(\omega)\,\nu(d\omega)=\int\limits_\Omega
\!f(\omega)\,d\nu=\sum\limits_{j=1}^na_j\nu\Omega_j.$$
\end{defen}

\begin{defen}
Пусть теперь функция $f\geqslant 0$ --- измерима. Тогда найдутся
такие простые измеримые функции $f_n$, что $f_n\nearrow f$. Положим
$$\int\limits_\Omega\! f(\omega)\,\nu(d\omega)=\lim\limits_{n\to\infty}
\int\limits_\Omega f_n(\omega)\,\nu(d\omega).$$
\end{defen}

\begin{defen}
Если $f$ --- произвольная измеримая функция, то верно равенство
$f(\omega)=f_+(\omega)-f_-(\omega)$, где
$$f_+(\omega)=\begin{cases}f(\omega),&\text{если $f(\omega)\geqslant 0$};\\
0,& \text{иначе},\end{cases}\quad\text{и}\quad
f_-(\omega)=\begin{cases}-f(\omega),&\text{если $f(\omega)\leqslant 0$};\\
0,& \text{иначе}.\end{cases}$$

Легко проверить, что функции $f_\pm$ измеримы. Положим
$$\int\limits_\Omega \! f(\omega)\,\nu(d\omega)=\int\limits_\Omega \!
f_+(\omega)\,\nu(d\omega)-\int\limits_\Omega \!
f_-(\omega)\,\nu(d\omega),$$ если хотя бы один из интегралов
конечен.

Если интеграл конечен, то функция $f$ называется
\emph{интегрируемой}, а если равен $\infty$, то
\emph{квазиинтегрируемой}.
\end{defen}

\begin{defen}
Если $A\in\mathfrak{a}$, то $$\int\limits_A \!
f(\omega)\,\nu(d\omega)=\int\limits_\Omega \!
\gamma_A(\omega)f(\omega)\,\nu(d\omega).$$
\end{defen}

\begin{lemm}
\label{chetno.add}Пусть $S$ --- кольцо, $(\Omega,S,\nu)$ --
пространство с конечной мерой. Тогда мера $\nu$ счетно аддитивна
$\Leftrightarrow$
$$\forall\, A_1\supset A_2\supset\ldots \in S: \bigcap\limits_{j=1}
^\infty A_j=\varnothing\Rightarrow \nu A_j\to 0\quad\text{при
$j\to\infty$}.$$
\end{lemm}

\begin{proof}
Пусть $S\ni B=\bigsqcup\limits_{j=1}^\infty B_j$, где $B_j\in S$.
Докажем, что $\nu B\p=\sum\limits_{j=1}^\infty \nu B_j$. Положим
$A_k=\bigsqcup\limits_{j=k}^\infty B_j$, тогда
$\bigcap\limits_{k=1}^\infty A_k=\varnothing$. Значит $\nu A_k\to 0$
при $k\to\infty$, и т.к. $$\nu B=\sum\limits_{j=1}^{k-1}\nu B_j+\nu
A_k,$$ то $$\sum\limits_{j=1}^{k-1} \nu B_j\to\nu B\quad\text{при
$k\to\infty$}.$$

Обратно, пусть мера $\nu$ счетно аддитивна. Тогда
$$\bigsqcup\limits_{j=1}^\infty (A_j\setminus
A_{j+1})=A\setminus\bigcap\limits_{j=1}^\infty A_j=A_1,$$ поэтому
$$\nu A_1=\sum\limits_{j=1}^\infty \nu(A_j\setminus A_{j+1})=\sum
\limits_{j=1}^{k-1}\nu(A_j\setminus
A_{j+1})+\sum\limits_{j=k}^\infty\nu(A_j\setminus A_{j+1}),$$ а
значит,
$$\sum\limits_{j=k}^\infty\nu(A_j\setminus A_{j+1})\to 0\quad\text{при $k\to\infty$}.$$
Поскольку
$$\forall\,k\in\mathbb{N}\quad \bigsqcup\limits_{j=k}^\infty (A_j\setminus
A_{j+1})=A_k,$$ то $$\nu
A_k=\sum\limits_{j=k}^\infty\nu(A_j\setminus A_{j+1})\to
0\quad\text{при $k\to\infty$},$$ что и требовалось.
\end{proof}
%-----------------------------------------------------------------------------------%

%---------------------------Lecture 7-----------------------------------------------%
\lecture

\begin{zam}
Далее, всюду, где мы будем писать $\int\limits_\Omega\!
f(\omega)\,\nu(d\omega)$, мы будем предполагать функцию $f$
измеримой.
\end{zam}

Пусть фиксировано пространство с мерой $(\Omega, \mathfrak{a},\nu)$.
Сейчас мы будем рассматривать только те простые функции $f$, для
которых $\nu\{\omega\in\Omega\mid f(\omega)\neq 0\}<\infty$.

После того, как мы ввели понятие интеграла Лебега, необходимо
доказать корректность этого определения.

Вначале докажем корректность определения интеграла Лебега для
простых функций. А именно, пусть $f(\omega)=\sum\limits_{j=1}^n
a_j\gamma_{A_j}(\omega)$ --- простая функция, причем $A_i\cap
A_j=\varnothing$. Возьмем другое представление функции $f$:
$f(\omega)=\sum\limits_{k=1}^m b_k\gamma_{B_k}(\omega)$. Если
$A_j\cap B_k\neq\varnothing$, то $a_j=b_k$, поэтому
$$\sum\limits_{j=1}^n a_j
\nu A_j=\sum\limits_{j=1}^n a_j\sum\limits_{k=1}^m \nu(A_j\cap
B_k)=\sum\limits_{k=1}^m\sum\limits_{j=1}^n b_k\nu(A_j\cap
B_k)=\sum\limits_{k=1}^m b_k\nu B_k.$$

\svoy\footnote{В свойствах 1--3 функции $f$ и $g$ считаются
неотрицательными и простыми.}

1. $\int\limits_\Omega\! (f(\omega)+g(\omega))\,\nu(d\omega)=
\int\limits_\Omega\! f(\omega)\,\nu(d\omega)+ \int\limits_\Omega\!
g(\omega)\,\nu(d\omega)$. Действительно, пусть
$f(\omega)=\sum\limits_{j=1}^n a_j\gamma_{A_j}(\omega)$ и
$g(\omega)=\sum\limits_{k=1}^m b_k\gamma_{B_k}(\omega)$. Тогда,
поскольку $A_j\p=\bigsqcup\limits_{k=1}^m A_j\cap B_k$ и
$\gamma_{A_j}(\omega)=\sum\limits_{k=1}^m \gamma_{A_j\cap
B_k}(\omega)$, то
$$f(\omega)=\sum\limits_{j=1}^n\sum\limits_{k=1}^m a_j\gamma_{A_j
\cap B_k}(\omega)\quad\text{и}\quad
g(\omega)=\sum\limits_{k=1}^m\sum\limits_{j=1}^n b_k\gamma_{A_j \cap
B_k}(\omega).$$ Отсюда $$\int\limits_\Omega\!
f(\omega)\,\nu(d\omega)=\sum\limits_{j=1}^n\sum\limits_{k=1}^m
a_j\nu(A_j\cap B_k),\quad \int\limits_\Omega\!
g(\omega)\,\nu(d\omega)=\sum\limits_{j=1}^n\sum\limits_{k=1}^m
b_k\nu(A_j\cap B_k),$$ поэтому
\begin{multline*}
\int\limits_\Omega\!
(f(\omega)+g(\omega))\,\nu(d\omega)=\sum\limits_{j=1}^n\sum\limits_{k=1}^m
(a_j+b_k)\nu(A_j\cap B_k)= \\ =\int\limits_\Omega\!
f(\omega)\,\nu(d\omega)+ \int\limits_\Omega\!
g(\omega)\,\nu(d\omega).
\end{multline*}

2. $\int\limits_\Omega\!\alpha
f(\omega)\,\nu(d\omega)=\alpha\int\limits_\Omega\!
f(\omega)\,\nu(d\omega)$.

3. Если $\nu A=0$, то $\int\limits_A\! f(\omega)\,\nu(d\omega)=0$.
Действительно, $\int\limits_A\!
f(\omega)\,\nu(d\omega)\p=\int\limits_\Omega\!
f(\omega)\gamma_A(\omega)\,\nu(d\omega)$. Если
$f(\omega)\p=\sum\limits_{j=1}^n a_j\gamma_{A_j}(\omega)$, то тогда
$f(\omega)\gamma_A(\omega)\p=\sum\limits_{j=1}^n a_j\gamma_{A\cap
A_j}(\omega)$ и $\int\limits_A\!
f(\omega)\,\nu(d\omega)=\sum\limits_{j=1}^n a_j\nu(A\cap A_j)=0$.

4. Если $f(\omega)\geqslant 0$, то $\int\limits_\Omega\!
f(\omega)\,\nu(d\omega)\geqslant 0$. Отсюда следует, что если
$g_1\leqslant g_2$, то $\int\limits_\Omega\!
g_1(\omega)\,\nu(d\omega)\leqslant\int\limits_\Omega\!
g_2(\omega)\,\nu(d\omega)$. Действительно. $g_2=(g_2-g_1)+g_1$,
поэтому $$\int\limits_\Omega\!
g_2(\omega)\,\nu(d\omega)=\int\limits_\Omega\!
(g_2-g_1)(\omega)\,\nu(d\omega)+\int\limits_\Omega\!
g_1(\omega)\,\nu(d\omega)\geqslant \int\limits_\Omega\!
g_1(\omega)\,\nu(d\omega).$$

5. Если $\mathfrak{a}\ni B=B_1\sqcup B_2$, то $\int\limits_B\!
f(\omega)\,\nu(d\omega)=\int\limits_{B_1}\!
f(\omega)\,\nu(d\omega)+\int\limits_{B_2}\!
f(\omega)\,\nu(d\omega)$. Действительно, это следует из того, что
\begin{align*}
\int\limits_B\! f(\omega)\,\nu(d\omega)&=\int\limits_\Omega\!
f(\omega)\gamma_B(\omega)\,\nu(d\omega), \\
\int\limits_{B_1}\! f(\omega)\,\nu(d\omega)&=\int\limits_\Omega\!
f(\omega)\gamma_{B_1}(\omega)\,\nu(d\omega), \\
\int\limits_{B_2}\! f(\omega)\,\nu(d\omega)&=\int\limits_\Omega\!
f(\omega)\gamma_{B_2}(\omega)\,\nu(d\omega),
\end{align*}
а также из того, что $\gamma_{B_1\sqcup
B_2}=\gamma_{B_1}+\gamma_{B_2}$.

6. Если $g_n(\omega)\nearrow \varphi(\omega)$ и
$\lim\limits_{n\to\infty}g_n(\omega)\geqslant \varphi(\omega)$, где
функция $\varphi\geqslant 0$ проста, а функции $g_n$ являются
простыми и измеримыми, то
$\lim\limits_{n\to\infty}\int\limits_\Omega\!
g_n(\omega)\,\nu(d\omega)\p\geqslant \int\limits_\Omega\!
\varphi(\omega)\,\nu(d\omega)$.

\begin{proof}
Пусть $\Omega_\varphi=\{\omega\in\Omega\mid \varphi(\omega)>0\}$,
тогда $\nu\Omega_\varphi<\infty$. Пусть
$0<\varepsilon<\min\limits_{\omega\in\Omega_\varphi}\{\varphi(\omega)\}$.
Тогда положим
$$F_\varphi(\omega)=\begin{cases}\varphi(\omega)-\varepsilon,&
\text{если $\omega\in\Omega_\varphi$,}\\ 0,& \text{если
$\varphi(\omega)=0$}.
\end{cases}$$
Введем также обозначение
$A_n^\varepsilon=\{\omega\in\Omega_\varphi\mid
g_n(\omega)>F_\varphi(\omega)\}$. Тогда понятно, что
$\Omega_\varphi\setminus
A_n^\varepsilon=\{\omega\in\Omega_\varphi\mid g_n(\omega)\leqslant
F_\varphi(\omega)\}$. Легко видеть, что $A_1^\varepsilon\subset
A_2^\varepsilon\subset\ldots$, поэтому $(\Omega_\varphi\setminus
A_1^\varepsilon)\supset (\Omega_\varphi\setminus
A_2^\varepsilon)\supset\ldots$ и $\bigcap\limits_{n=1}^\infty
(\Omega_\varphi\setminus A_n^\varepsilon)=\varnothing$. Отсюда в
силу счетной аддитивности и леммы~\ref{chetno.add}
$\nu(\Omega_\varphi\setminus A_n^\varepsilon)\to 0$ при
$n\to\infty$.

Теперь выпишем следующую цепочку неравенств:

\begin{align*}
\int\limits_\Omega\! g_n(\omega)\,\nu(d\omega)&\geqslant
\int\limits_{\Omega_\varphi}\! g_n(\omega)\,\nu(d\omega)=
\int\limits_{A_n^\varepsilon}\! g_n(\omega)\,\nu(d\omega) +
\int\limits_{\Omega_\varphi\setminus A_n^\varepsilon}\!
g_n(\omega)\,\nu(d\omega)\geqslant\\
&\geqslant\int\limits_{A_n^\varepsilon}\!
F_\varphi(\omega)\,\nu(d\omega) +
\int\limits_{\Omega_\varphi\setminus A_n^\varepsilon}\!
g_n(\omega)\,\nu(d\omega)=\\
&= \int\limits_{\Omega_\varphi}\! F_\varphi(\omega)\,\nu(d\omega)-
\int\limits_{\Omega_\varphi\setminus A_n^\varepsilon}\!
F_\varphi(\omega)\,\nu(d\omega)+\int\limits_{\Omega_\varphi\setminus
A_n^\varepsilon}\! g_n(\omega)\,\nu(d\omega)\geqslant\\
&\geqslant \int\limits_{\Omega_\varphi}\!
F_\varphi(\omega)\,\nu(d\omega)-\int\limits_{\Omega_\varphi\setminus
A_n^\varepsilon}\! F_\varphi(\omega)\,\nu(d\omega)=\\
&= \int\limits_{\Omega_\varphi}\!
\varphi(\omega)\,\nu(d\omega)-\varepsilon\nu\Omega_\varphi-
\int\limits_{\Omega_\varphi\setminus A_n^\varepsilon}\!
F_\varphi(\omega)\,\nu(d\omega)\geqslant\\
&\geqslant \int\limits_\Omega\!
\varphi(\omega)\,\nu(d\omega)-\varepsilon\nu\Omega_\varphi
-\nu(\Omega_\varphi\setminus A_n^\varepsilon)\cdot \max
\varphi(\omega).
\end{align*}
Но $\nu(\Omega_\varphi\setminus A_n^\varepsilon)\cdot\max
\varphi(\omega)\to 0$ при $n\to\infty$, поэтому
$\lim\limits_{n\to\infty}\int\limits_\Omega\!
g_n(\omega)\,\nu(d\omega)\p\geqslant \int\limits_\Omega\!
\varphi(\omega)\,\nu(d\omega)-\varepsilon\nu\Omega_\varphi$, откуда
следует требуемое неравенство.
\end{proof}

Теперь докажем корректность определения интеграла для произвольных
неотрицательных измеримых функций. Пусть $f\geqslant 0$ --- данная
функция, $g_n^1(\omega)\nearrow f(\omega)$ и $g_n^2(\omega)\nearrow
f(\omega)$. Покажем, что
$\lim\limits_{n\to\infty}\int\limits_\Omega\!
g_n^1(\omega)\,\nu(d\omega)\p=\lim\limits_{k\to\infty}\int\limits_\Omega\!
g_k^2(\omega)\,\nu(d\omega)$. Действительно, при каждом $n$ по
свойству 6 имеем: $\int\limits_\Omega\!
g_n^1(\omega)\,\nu(d\omega)\leqslant
\lim\limits_{k\to\infty}\int\limits_\Omega\!
g_k^2(\omega)\,\nu(d\omega)$, откуда получаем
$\lim\limits_{n\to\infty}\int\limits_\Omega\!
g_n^1(\omega)\,\nu(d\omega)\p\leqslant\lim\limits_{k\to\infty}\int\limits_\Omega\!
g_k^2(\omega)\,\nu(d\omega)$. Аналогично можно получить обратное
неравенство, поэтому оба предела совпадают.

Отсюда легко получить корректность определения интеграла Лебега для
произвольных измеримых функций.

\begin{defen}
Две измеримые функции $f$ и $g$ на пространстве с мерой называются
\emph{эквивалентными}, если $\nu\{\omega\in\Omega\mid f(\omega)\neq
g(\omega)\}=0$.
\end{defen}

\begin{defen}
Говорят, что какое-то свойство \emph{имеет место почти всюду}, если
оно верно всюду, кроме множества меры 0. Например, если $f=g$ почти
всюду, то $\int\limits_\Omega\!
f(\omega)\,\nu(d\omega)=\int\limits_\Omega\!
g(\omega)\,\nu(d\omega)$.
\end{defen}

\begin{zam}
В дальнейшем мы обычно будем опускать слова <<почти всюду>>, но
принципиально от этого ничего не изменится.
\end{zam}

\begin{theorem}[Леви]
Пусть $g_n\geqslant 0$ и $g_n\nearrow f$ почти всюду. Тогда
$\exists\,\lim\limits_{n\to\infty}\int\limits_\Omega\!
g_n(\omega)\,\nu(d\omega)=\int\limits_\Omega\!
f(\omega)\,\nu(d\omega)$.
\end{theorem}

\begin{zam}
В условии теоремы можно считать, что $g_n\geqslant g$, где функция
$g$ интегрируема.
\end{zam}

\begin{proof}
По теореме~\ref{fun.approx} $\forall\, g_i\quad \exists\,
g_{ij}\nearrow g_i$ при $j\to\infty$, причем функции
$g_{ij}\geqslant 0$ просты и измеримы. Пусть
$\psi_n(\omega)=\max\{g_{ij}(\omega)\mid 1\leqslant i\p\leqslant
n,1\leqslant j\leqslant n\}$. Докажем. что $\psi_n(\omega)\nearrow
f(\omega)$.

Ясно, что $\psi_n(\omega)\leqslant g_n(\omega)\leqslant g_k(\omega)$
при $k>n$. Отсюда получаем, что $\psi_n(\omega)\leqslant
\lim\limits_{k\to\infty} g_k(\omega)=f(\omega)$, а значит,
$\lim\limits_{n\to\infty} \psi_n(\omega)\leqslant f(\omega)$.

Обратно, $\forall\, n\geqslant k\quad \psi_n(\omega)\geqslant
g_{kn}(\omega)$, поэтому
$\lim\limits_{n\to\infty}\psi_n(\omega)\geqslant\lim\limits_{n\to\infty}
g_{kn}(\omega)\p=g_k(\omega)$. Значит,
$\lim\limits_{n\to\infty}\psi_n(\omega)\geqslant\lim\limits_{k\to\infty}
g_k(\omega)=f(\omega)$.

Теперь, т.к. $\psi_n(\omega)\leqslant g_n(\omega)$, то
$\int\limits_\Omega\! \psi_n(\omega)\,\nu(d\omega)\leqslant
\int\limits_\Omega\! g_n(\omega)\,\nu(d\omega)$ и $\psi_n(\omega)
\p\nearrow f(\omega)$, то по определению интеграла Лебега получаем
$\int\limits_\Omega\!
f(\omega)\,\nu(d\omega)\p=\lim\limits_{n\to\infty}
\int\limits_\Omega\! \psi_n(\omega)\,\nu(d\omega)$. Но
$$\int\limits_\Omega\! f(\omega)\,\nu(d\omega)\geqslant
\int\limits_\Omega\! g_n(\omega)\,\nu(d\omega)\geqslant
\int\limits_\Omega\! \psi_n(\omega)\,\nu(d\omega),$$ откуда по
теореме о двух милиционерах следует требуемое.
\end{proof}
%-----------------------------------------------------------------------------------%

%---------------------------Lecture 8-----------------------------------------------%
\lecture

Прежде чем двигаться дальше, введем следующие важные обозначения:

1) $\mathcal{\bar{L}}_0(\Omega,\mathfrak{a},\nu)$  --- множество
почти всюду определенных измеримых функций;

2) $\mathcal{L}_0(\Omega,\mathfrak{a},\nu)$  --- множество классов
почти всюду определенных измеримых функций;

3) $\mathcal{\bar{L}}_1(\Omega,\mathfrak{a},\nu)$  --- множество
почти всюду оп\-ределенных интегрируемых функций;

4) $\mathcal{L}_1(\Omega,\mathfrak{a},\nu)$  --- множество классов
почти всюду определенных интегрируемых функций.

Здесь необходимо дать несколько новых определений.

\begin{defen}
Пусть $A\in\mathfrak{a}$ и $\nu A>0$. Рассмотрим пространство с
мерой $(A,\mathfrak{a}\cap A=\{C\cap A\mid
C\in\mathfrak{a}\},\nu\mid_{\mathfrak{a}\cap A})$. Оно называется
\emph{подпространством с мерой} пространства с мерой $(\Omega,
\mathfrak{a}, \nu)$.

Пусть теперь функция $f$ определена почти всюду, $f\colon A\to
\mathbb{R}^1$, где $A\p\in\mathfrak{a}$ и $\nu(\Omega\setminus
A)=0$. Функция $f$ называется \emph{измеримой}, если она измерима на
$(A, \mathfrak{a}\cap A, \nu\mid_{\mathfrak{a}\cap A})$.

Функции $f$ и $g$ называются \emph{эквивалентными} (обозначение:
$f\sim g$), если $$\nu\{\omega\mid \text{$f(\omega)$ и $g(\omega)$
существуют и $f(\omega)\neq g(\omega)$}\}=0.$$
\end{defen}

\begin{zam}
В дальнейшем мы будем писать просто $\mathcal{L}$, без указания
аргументов. Кроме того, когда мы будем писать $f\in\mathcal{L}_1$,
то чаще всего под этим мы будем подразумевать не класс
эквивалентности, а конкретный представитель этого класса.
\end{zam}

\begin{theorem}[Фату--Лебег]
1. Пусть $\varphi\in\mathcal{\bar{L}}_1$ и $\varphi(\omega)\leqslant
f_n(\omega)$. Тогда
$\int\limits_\Omega\!\underline{\lim}f_n(\omega)\,\nu(d\omega)
\leqslant\underline{\lim} \int\limits_\Omega\!
f_n(\omega)\,\nu(d\omega)$.

2. Пусть $\psi\in\mathcal{\bar{L}}_1$ и $\psi(\omega)\geqslant
f_n(\omega)$ при всех $n$. Тогда
$\int\limits_\Omega\!\overline{\lim}f_n(\omega)\,\nu(d\omega)
\p\geqslant\overline{\lim} \int\limits_\Omega\!
f_n(\omega)\,\nu(d\omega)$.
\end{theorem}

\begin{proof}
1. Положим $g_n(\omega)=\inf\limits_{k\geqslant n}f_k(\omega)$.
Тогда $\varphi(\omega)\leqslant g_1(\omega)\p\leqslant
g_2(\omega)\leqslant\ldots$ и $\lim\limits_{n\to\infty}
g_n(\omega)=\underline{\lim}f_n(\omega)$, откуда по теореме Леви
получаем, что
$\int\limits_\Omega\!\underline{\lim}f_n(\omega)\,\nu(d\omega)=
\lim\limits_{n\to\infty}\int\limits_\Omega\!
g_n(\omega)\,\nu(d\omega)$.

Проверим, что
$\lim\limits_{n\to\infty}\int\limits_\Omega\!g_n(\omega)\,\nu(d\omega)
\leqslant \underline{\lim} \int\limits_\Omega\!
f_n(\omega)\,\nu(d\omega)$. $\forall\,n\leqslant k\quad g_n(\omega)
\p\leqslant f_k(\omega)$, откуда $\int\limits_\Omega\! g_n(\omega)\,
\nu(d\omega)\leqslant \int\limits_\Omega\!
f_k(\omega)\,\nu(d\omega)$. Значит, $\int\limits_\Omega\!
g_n(\omega)\, \nu(d\omega)\p\leqslant \inf\limits_{k\geqslant
n}\int\limits_\Omega\! f_k(\omega)\,\nu(d\omega)$ и
$$\lim\limits_{n\to\infty}\int\limits_\Omega\! g_n(\omega)\,\nu(d\omega)
\leqslant\lim\limits_{n\to\infty}\inf\limits_{k\geqslant n}
\int\limits_\Omega \! f_k(\omega)\,\nu(d\omega)= \underline{\lim}
\int\limits_\Omega\! f_n(\omega)\,\nu(d\omega).$$

2. Положим $f_n^1(\omega)=-f_n(\omega)$ и
$\varphi(\omega)=-\psi(\omega)$. Тогда утверждение легко следует из
п.1.
\end{proof}

\begin{theorem}[Лебег]
Пусть $f_n(\omega)\to f(\omega)$ и $|f_n(\omega)|\leqslant
\varphi(\omega)$, где $\varphi\in\mathcal{L}_1$. Тогда
$f\in\mathcal{L}_1$ и $\int\limits_\Omega\! f(\omega)\,\nu(d\omega)
=\lim\limits_{n\to\infty}\int\limits_\Omega\!
f_n(\omega)\,\nu(d\omega)$.
\end{theorem}

\begin{zam}
Условие теоремы можно ослабить, а именно достаточно потребовать
только сходимость функций $f_n$ по мере.
\end{zam}

\begin{proof}
Очевидно, что $|f(\omega)|\leqslant \varphi(\omega)$, поэтому
$f\in\mathcal{L}_1$. Поскольку
$\underline{\lim}f_n(\omega)=\overline{\lim}f_n(\omega)=f(\omega)$,
то $$\int\limits_\Omega\!
\underline{\lim}f_n(\omega)\,\nu(d\omega)=\int\limits_\Omega\!
\overline{\lim}f_n(\omega) \,\nu(d\omega)=
\int\limits_\Omega\!f(\omega)\,\nu(d\omega),$$ и по теореме
Фату--Лебега
\begin{align*}
\overline{\lim}\int\limits_\Omega\!f_n(\omega)\,\nu(d\omega)&\leqslant
\int\limits_\Omega\! \overline{\lim}f_n(\omega)\,\nu(d\omega)=
\int\limits_\Omega\! f(\omega)\,\nu(d\omega)= \\
&=\int\limits_\Omega\!
\underline{\lim}f_n(\omega)\,\nu(d\omega)\leqslant
\underline{\lim}\int\limits_\Omega\!f_n(\omega)\,\nu(d\omega),
\end{align*}
откуда и получаем требуемое.
\end{proof}

\begin{theorem}[Фату]
Пусть $0\leqslant f_n(\omega)\to f(\omega)$ и $\int\limits_\Omega \!
f_n(\omega)\, \nu(d\omega)\leqslant C$, где $C>0$. Тогда
$f\in\mathcal{L}_1$ и $\int\limits_\Omega \! f(\omega)\,
\nu(d\omega)\leqslant C$.
\end{theorem}

\begin{proof}
По теореме Фату--Лебега получаем $\int\limits_\Omega \!
\underline{\lim}f_n(\omega)\, \nu(d\omega)\p\leqslant
\underline{\lim}\int\limits_\Omega \! f_n(\omega)\, \nu(d\omega)$,
поэтому $$\int\limits_\Omega \! f(\omega)\, \nu(d\omega)=
\int\limits_\Omega \! \underline{\lim}f_n(\omega)\,
\nu(d\omega)\leqslant \underline{\lim}\int\limits_\Omega \!
f_n(\omega)\, \nu(d\omega) \leqslant C,$$ что и требовалось.
\end{proof}

\begin{zam}
В теореме Фату нельзя переходить к пределу так же, как и в теореме
Лебега: например, можно взять $\Omega=[0;1]$ и
$$f_n(x)=\begin{cases}2n-4n^2|x-\frac{1}{2n}|,& \text{если $x\in[0;1/n]$},
\\ 0,& \text{если $x\in[1/n;1]$}.\end{cases}$$

Кроме того, ограничение <<$f_n\geqslant 0$>> существенно: опять
можно взять $\Omega=[0;1]$ и <<перевернуть>> функции $f_n$, сдвинув
их немного вверх.
\end{zam}
%-----------------------------------------------------------------------------------%

%---------------------------Lecture 9-----------------------------------------------%
\lecture

\begin{theorem}[неравенство Чебышева]
Пусть функция $f(\omega)\geqslant 0$, тогда $\nu\{\omega\mid
f(\omega)\geqslant c\}\leqslant\frac 1 c\int\limits_\Omega\!
f(\omega)\,\nu(d\omega)$.
\end{theorem}

\begin{proof}
Имеем следующую цепочку неравенств: $$\int\limits_\Omega\!
f(\omega)\,\nu(d\omega)\geqslant \int\limits_{\{\omega\mid
f(\omega)\geqslant c\}}\! f(\omega)\,\nu(d\omega)\geqslant
\int\limits_{\{\omega\mid f(\omega)\geqslant c\}}\!
c\,\nu(d\omega)=c\nu\{\omega\mid f(\omega)\geqslant c\},$$ что и
требовалось.
\end{proof}

\begin{sled}
Пусть $f_n(\omega)\geqslant 0$ и $\exists\,
f(\omega)=\sum\limits_{n=1}^\infty f_n(\omega)$. Тогда
$$\exists\,\int\limits_\Omega\!
f(\omega)\,\nu(d\omega)=\sum\limits_{n=1}^\infty
\int\limits_\Omega\! f_n(\omega)\,\nu(d\omega).$$ В частности, если
$\sum\limits_{n=1}^\infty \int\limits_\Omega\!
f_n(\omega)\,\nu(d\omega)<\infty$, то сумма ряда
$\sum\limits_{n=1}^\infty f_n(\omega)$ почти всюду конечна.
\end{sled}

\begin{proof}
Первое утверждение следует из теоремы Леви. Докажем второе
утверждение. Для этого достаточно доказать, что если
$g(\omega)\geqslant 0$ и $C_g=\int\limits_\Omega\!
g(\omega)\,\nu(d\omega)<\infty$, то функция $g$ конечна почти всюду.
Поскольку $\{\omega\mid
g(\omega)=\infty\}=\bigcap\limits_{n=1}^\infty \{\omega\mid
g(\omega)\geqslant n\}$ и $\nu\{\omega\mid g(\omega)\geqslant
n\}\p\leqslant\frac 1 n\int\limits_\Omega\!
g(\omega)\,\nu(d\omega)=\frac{C_g}{n}\to 0$ при $n\to\infty$ (по
неравенству Чебышева), то
$$\nu\{\omega\mid g(\omega)=\infty\}=\nu\Big(\bigcap\limits_{n=1}^\infty
\{\omega\mid g(\omega)\geqslant n\}\Big)=0,$$ что и требовалось.
\end{proof}

\prim

Пусть $\Omega=[0;1]$ и $r_n\in\mathbb{Q}\cap[0;1]$ --- $n$-е
рациональное число из $\Omega$. Возьмем функции
$f_n(x)=\frac{1}{\sqrt{|x-r_n|}}$ и рассмотрим ряд
$\sum\limits_{n=1}^\infty \frac{1}{2^n} f_n(x)$. Очевидно, что
$\exists\, C>0: \int\limits_\Omega\! f_n(\omega)\,\nu(d\omega)<C$,
поэтому $\sum\limits_{n=1}^\infty \int\limits_\Omega\!
\frac{1}{2^n}f_n(\omega)\,\nu(d\omega)<C<\infty$ и ряд
$\sum\limits_{n=1}^\infty \frac{1}{2^n} f_n(x)$ сходится почти
всюду.

\tema{Сходимость по мере}

\begin{defen}
Пусть фиксировано пространство с мерой $(\Omega, \mathfrak{a},\nu)$
и функции $f_n$ измеримы. Тогда $f_n$ \emph{сходится к функции $f$
по мере}, если $\forall\, \varepsilon>0\quad \nu\{\omega\mid
|f_n(\omega)-f(\omega)|>\varepsilon\}\to 0$ при $n\to\infty$.
Обозначение: $f_n\mathop{\to}\limits_\nu f$.
\end{defen}

\begin{theorem}
Если $f_n\mathop{\to}\limits_\nu f$, то $\exists\, f_{n_k}:
f_{n_k}\to f$ почти всюду. Если мера $\nu$ конечна, то всякая почти
всюду сходящаяся последовательность сходится по мере.
\end{theorem}

\begin{proof}
Если $f_j(\omega)\to f(\omega)$, то $$\forall\, n\;\exists\, k:
\forall\, r>k\quad |f_r(\omega)-f(\omega)|<\frac 1 n,$$ поэтому
$$A=\{\omega\mid f_j(\omega)\to f(\omega)\}=\bigcap\limits_n
\bigcup\limits_k \bigcap\limits_{r>k} \{\omega\mid
|f_r(\omega)-f(\omega)|<\frac 1 n\}$$ и
$$B=\Omega\setminus A=\{\omega\mid f_j(\omega)\nrightarrow
f(\omega)\}=\bigcup\limits_n \bigcap\limits_k
\bigcup\limits_{r\geqslant k} \{\omega\mid
|f_r(\omega)-f(\omega)|\geqslant\frac 1 n\}.$$ Необходимо выбрать
такие $f_{r_j}$, чтобы $\nu B=0$. Но если $\nu B=0$, то
$$\forall\, n\quad \nu\Big(\bigcap\limits_{k=1}^\infty \bigcup\limits_
{j\geqslant k}\{\omega\mid |f_{r_j}(\omega)-f(\omega)|\geqslant
\frac 1 n\}\Big)=0,$$ т.е. $$\nu\Big(\bigcup\limits_ {j\geqslant
k}\{\omega\mid |f_{r_j}(\omega)-f(\omega)|\geqslant \frac 1
n\}\Big)\to 0\quad \text{при $k\to\infty$}.$$

Пусть теперь $$\forall\, n\quad \nu\{\omega\mid
|f_r(\omega)-f(\omega)|\geqslant\frac 1 n\}\to 0\quad\text{при
$r\to\infty$}.$$ Выберем $\varepsilon_j$ так, чтобы
$\sum\limits_{j=1}^\infty \varepsilon_j<\infty$ (например, можно
взять $\varepsilon_j=2^{-j}$). Для каждого $j$ выберем $r_j$ так,
чтобы $\nu\{\omega\mid |f_{r_j}(\omega)-f(\omega)|\geqslant\frac 1
n\}<\varepsilon_j$, тогда
\begin{align*}
\nu\Big(\bigcup\limits_{j\geqslant k}\{\omega\mid |f_{r_j}(\omega)-
f(\omega)|\geqslant\frac 1 n\}\Big)&\leqslant\sum\limits_{j\geqslant
k}\nu\{\omega\mid |f_{r_j}(\omega)-f(\omega)|\geqslant\frac 1
n\}<\\
&<\sum\limits_{j\geqslant k}\varepsilon_j\to 0 \quad\text{при
$k\to\infty$},
\end{align*}
т.к. ряд $\sum\limits_{j=1}^\infty \varepsilon_j$ сходится. Поэтому
если $\nu\Big(\bigcup\limits_{j\geqslant k}\{\omega\mid
|f_{r_j}(\omega)- f(\omega)|\geqslant\frac 1 n\}\Big)\to 0$, то
$\nu\{\omega\mid |f_k(\omega)-f(\omega)|\geqslant\frac 1 n\}\to 0$
при $k\to\infty$, откуда следует сходимость по мере.
\end{proof}

\prim

1. Пусть $\Omega=\mathbb{R}^1$, а функции $f_n$ таковы, что
$$f_n(x)=\begin{cases}0,& \text{если $x\leqslant n$}; \\ 1,&
\text{если $x> n$}\end{cases}.$$ Тогда $f_n(x)\to 0$, но $\nu\{x\mid
|f_n(x)-0|>\frac 1 2\}=\nu[n;\infty)=\infty$, поэтому сходимости по
мере нет.

2. Пусть $\Omega=\mathbb{T}$ и $\varepsilon_j>0$, причем
$\sum\limits_{j=1}^\infty \varepsilon_j=\infty$ и $\varepsilon_j\to
0$ (например, можно взять $\varepsilon_j=j^{-1}$). Положим
$$f_n(x)=\begin{cases}1,& \text{если $x\in[\varepsilon_1+\ldots+\varepsilon_
{n-1};\varepsilon_1+\ldots+\varepsilon_n]$},\\
0,&\text{иначе}.\end{cases}$$ Тогда $f_n(x)$ сходится по мере, но
нигде не сходится.

\vspace{5pt}

\begin{theorem}
Если функция $f$ интегрируема по Риману на множестве $\Omega=[0;1]$,
то она интегрируема и по Лебегу, причем значения интегралов
совпадают.
\end{theorem}

\begin{proof}
Введем следующие обозначения:
$$f_n^s=\sum\limits_{j=1}^{2^n}\frac{s_j}{2^n}, \quad
f_n^i=\sum\limits_{j=1}^{2^n}\frac{i_j}{2^n},$$ где
$$s_j=\sup\limits_{t\in\left[\frac{j-1}{2^n};\frac{j}{2^n}\right)}f(t), \quad
i_j=\inf\limits_{t\in\left[\frac{j-1}{2^n};\frac{j}{2^n}\right)}f(t).$$
Положим также
$$\bar{f}_n(t)=\sum\limits_{j=1}^{2^n}s_j\gamma_{\left[\frac{j-1}{2^n};
\frac{j}{2^n}\right)}(t), \quad
\underline{f}_n(t)=\sum\limits_{j=1}^{2^n}i_j\gamma_{\left[\frac{j-1}{2^n};
\frac{j}{2^n}\right)}(t).$$ Легко видеть, что
$$\sup\limits_{t\in[0;1]} f(t)\geqslant
\bar{f}_1\geqslant\bar{f}_2\geqslant\ldots\geqslant
f\geqslant\ldots\geqslant\underline{f}_2\geqslant\underline{f}_1\geqslant\inf
\limits_{t\in[0;1]}f(t).$$ Значит,
$\exists\,\lim\limits_{j\to\infty} \bar{f}_j=\bar{f}$ и
$\exists\,\lim\limits_{j\to\infty} \underline{f}_j=\underline{f}$,
причем $\sup\limits_{t\in[0;1]}f(t)\geqslant \bar{f}\geqslant
f\geqslant\underline{f}\p\geqslant \inf\limits_{t\in[0;1]}f(t)$.
Поэтому по теореме Лебега можно перейти к пределу под интегралами
$\int\limits_{[0;1]}\!\bar{f}_n(t)\,dt=f_n^s$ и
$\int\limits_{[0;1]}\!\underline{f}_n(t)\,dt=f_n^i$:
$$(R)\int\limits_{[0;1]}\!f(t)\,dt=\int\limits_{[0;1]}\!\bar{f}_n(t)\,dt\to
\int\limits_{[0;1]}\!f(t)\,dt\leftarrow\int\limits_{[0;1]}\!\underline{f}_n(t)\,dt
=(R)\int\limits_{[0;1]}\!f(t)\,dt,$$ откуда
$$\int\limits_{[0;1]}\!\bar{f}_n(t)\,dt=(R)\int\limits_{[0;1]}\!f(t)\,dt=
\int\limits_{[0;1]}\!\underline{f}_n(t)\,dt.$$ Поскольку
$\bar{f}(t)-\underline{f}(t)\geqslant 0$ почти всюду и
$\int\limits_{[0;1]}\!(\bar{f}(t)-\underline{f}(t))\,dt=0$, то
$\bar{f}\p=\underline{f}$ почти всюду. Значит,
$f=\bar{f}=\underline{f}$ почти всюду. Поэтому функция $f$ измерима
(как предел простых измеримых функций), $f\in\mathcal{\bar{L}}_1$ и
$$\int\limits_{[0;1]}\!f(t)\,dt=\int\limits_{[0;1]}\!\bar{f}(t)\,dt=
\int\limits_{[0;1]}\!\underline{f}(t)\,dt=(R)\int\limits_{[0;1]}\!f(t)\,dt,$$
что и требовалось.
\end{proof}
%-----------------------------------------------------------------------------------%

%---------------------------Lecture 10-----------------------------------------------%
\lecture

\begin{defen}
Пусть $(E, \mathfrak{a}, \nu)$ --- пространство с полной мерой.
Тогда функции\footnote{Не обязательно измеримые!} $f_1$ и $f_2$
\emph{равны почти всюду}, если множество точек, где они не равны,
есть подмножество измеримого множества меры 0.

Например, если $f$ --- $\mathfrak{a}$-измеримая функция, то
$\exists\, \bar{f}$ --- $\bar{\mathfrak{a}}$-измеримая функция,
совпадающая с $f$ почти всюду.
\end{defen}

\begin{theorem}
Пусть функция $f$ интегрируема на отрезке $[0;1]$ в несобственном
смысле (с особенностью в точке 0) и абсолютно. Тогда
$f\in\mathcal{\bar{L}}_1$ и
$\exists\,\int\limits_{[0;1]}\!f(t)\,dt=(R)\int\limits_0^1\!f(t)\,dt$.
\end{theorem}

\begin{proof}
По определению несобственного интеграла Римана,
$$(R)\int\limits_0^1\!f(t)\,dt=\lim\limits_{\varepsilon\to 0}\;(R)\int
\limits_\varepsilon^1\!f(t)\,dt
=\lim\limits_{\varepsilon\to
0}\int\limits_{[\varepsilon;1]}\!f(t)\,dt.$$ Пусть
$$f_n(t)=\begin{cases}
f(t),&\text{при $t\geqslant\frac 1 n$;}\\
0,&\text{при $t<\frac 1 n$},
\end{cases}$$ тогда $\lim\limits_{n\to\infty} f_n(t)=f(t)$ при $t\neq
0$ и $|f_1(t)|\leqslant|f_2(t)|\leqslant\ldots$. Отсюда
$$\int\limits_{[0;1]}\!|f_n(t)|\,dt=\int\limits_{[1/n;1]}\!|f(t)|\,dt=(R)\int
\limits_{1/n}^1\!|f(t)|\,dt\leqslant(R)\int\limits_0^1\!|f(t)|\,dt<\infty,$$
поэтому по теореме Леви
$$\int\limits_{[0;1]}\!|f(t)|\,dt=\inf\limits_{[0;1]}\int\!|f(t)|\,dt=\lim
\limits_{n\to\infty}\;
(R)\int\limits_{1/n}^1\!|f(t)|\,dt=(R)\int\limits_0^1\!|f(t)|\,dt<\infty.$$
По теореме Лебега получаем $$\exists\,
\lim\limits_{n\to\infty}\int\limits_{[0;1]}\!f_n(t)\,dt=\int\limits_{[0;1]}\!
f(t)\,dt=\lim\limits_{n\to\infty}\; (R)\int\limits_0^1\!f_n(t)\,dt=
(R)\int\limits_0^1\!f(t)\,dt,$$ что и требовалось.
\end{proof}

\begin{zam}
Условие абсолютной сходимости нельзя отбросить: рассмо\-трим,
например, функцию $g(x)=(-1)^n\frac{2^n}{n}$ при
$x\in[2^{-n};2^{1-n})$. Тогда $g\in\mathcal{R}[0;1]$ и
$\int\limits_0^1\!g(x)\,dx=\sum\limits_{n=1}^\infty
\frac{(-1)^n}{n}<\infty$, но $g\not\in\mathcal{\bar{L}}_1$.
\end{zam}

\tema{Формула замены переменных}

\begin{defen}
Пусть $(\Omega_1,\mathfrak{a}_1, \nu_1)$ --- пространство с мерой,
$(\Omega_2, \mathfrak{a_2})$ --- измеримое пространство и $f\colon
(\Omega_1, \mathfrak{a_1},\nu_1)\to(\Omega_2, \mathfrak{a}_2)$ ---
измеримое отображение, т.е. $\forall\, A\in\mathfrak{a}_2\quad
f^{-1}A\in\mathfrak{a}_1$. Тогда \emph{образ меры $\nu_1$
относительно $f$} есть мера $\nu_2$ на измеримом пространстве
$(\Omega_2, \mathfrak{a_2})$, обозначаемая $f_*\nu_1$ или
$\nu_1f^{-1}$ и определяемая по формуле
$(f_*\nu_1)A=\nu_1(f^{-1}A)$.
\end{defen}

\begin{predl}
Образ счетно аддитивной меры счетно аддитивен.
\end{predl}

\begin{proof}
Пусть $A_j\in\mathfrak{a}_2$, тогда
\begin{multline*}
(\nu_1f^{-1})\Big(\bigsqcup\limits_{j=1}^\infty
A_j\Big)=\nu_1\Big(f^{-1}\bigsqcup\limits_{j=1}^\infty A_j\Big)=
\nu_1\Big(\bigsqcup\limits_{j=1}^\infty
f^{-1}A_j\Big)=\\
=\sum\limits_{j=1}^\infty \nu_1(f^{-1}A_j)= \sum\limits_{j=1}^\infty
(\nu_1f^{-1})A_j,
\end{multline*} что и требовалось.
\end{proof}

\begin{theorem}
Пусть $G\colon \Omega_2\to\mathbb{R}^1$ --- измеримая функция. Тогда
$$\int\limits_{\Omega_1}\!G(f(\omega_1))\,\nu_1(d\omega_1)=\int\limits_
{\Omega_2}\!G(\omega_2)(\nu_1f^{-1})(d\omega_2).$$
\end{theorem}

\begin{proof}
Легко видеть, что достаточно проверить утверждение теоремы для
индикаторной функции $G(\omega_2)=\gamma_A(\omega_2)$, где
$A\in\mathfrak{a}_2$. Тогда по определению образа меры
\begin{multline*}
\int\limits_{\Omega_2}\!\gamma_A(\omega_2)\,(\nu_1f^{-1})(d\omega_2)=
(\nu_1f^{-1})(A)=\nu_1(f^{-1}A)=\\
=\int\limits_{\Omega_1}\!\gamma_{f^{-1}A}(\omega_1)
\,\nu_1(d\omega_1)=\int\limits_{\Omega_1}\!\gamma_A(f(\omega_1))\,\nu_1(d\omega_1),
\end{multline*} что и требовалось.
\end{proof}
%-----------------------------------------------------------------------------------%

%---------------------------Lecture 11-----------------------------------------------%
\lecture

\vspace{-25pt}

\tema{Абсолютная непрерывность}

\begin{defen}
Рассмотрим меры $\nu$ и $\mu$ на $\sigma$-алгебре $\mathfrak{a}$.
Мера $\nu$ называется \emph{абсолютно непрерывной относительно меры
$\mu$}, если $\forall\, A\p\in\mathfrak{a}\quad \mu A=0\Rightarrow
\nu A=0$. Обозначение: $\nu\ll \mu$.

Если $\nu\ll \mu$ и $\nu \ll \nu$, то меры $\nu$ и $\mu$ называются
\emph{эквивалентными}. Обозначение: $\nu\sim\mu$.
\end{defen}

\begin{predl}[критерий абсолютной непрерывности]
$\nu\ll\mu$ $\p\Leftrightarrow$
$\forall\,\varepsilon>0\quad\exists\, \delta>0: \forall\,
A\in\mathfrak{a} \quad \mu A<\delta\p\Rightarrow \nu A<\varepsilon$.
\end{predl}

\begin{zam}
В этом критерии существенна неотрицательность меры!
\end{zam}

\begin{proof}
Достаточность очевидна: если $\mu A=0$, то $\forall\,\delta\quad \mu
A<\delta$ поэтому $\forall\,\varepsilon\quad \nu A<\varepsilon$, а
значит, $\nu A=0$.

Докажем необходимость. Для этого предположим противное:
$\exists\,\varepsilon_0\p>0: \forall\,\delta>0\quad\exists\,
A_\delta\in\mathfrak{a}:\mu A<\delta$ и $\nu A\geqslant
\varepsilon_0$. Возьмем такую последовательность $\{\delta_n\}$,
чтобы $\delta_n>0$ и $\sum\limits_{n=1}^\infty \delta_n<\infty$.
Тогда для каждого $\delta_n$ найдется такое
$A_n=A_{\delta_n}\in\mathfrak{a}$, что $\mu A_n<\delta_n$ и $\nu
A_n\geqslant\varepsilon_0$. Положим $C_k=\bigcup\limits_{n=k}^\infty
A_n$, тогда $C_1\supset C_2\supset\ldots$ и в силу счетной
аддитивности меры $\mu$ имеем: $$\mu
C_k\leqslant\sum\limits_{n=k}^\infty \mu
A_n<\sum\limits_{n=k}^\infty \delta_n\to 0\quad\text{при
$k\to\infty$},$$ поэтому $\mu C_k\to 0$ при $k\to\infty$, а значит,
$\mu\bigcap\limits_{k=1}^\infty C_k=0$, откуда
$\nu\bigcap\limits_{k=1}^\infty C_k=0$. Но $\forall\, k\quad \nu
C_k\geqslant\varepsilon_0$, поэтому $\nu\bigcap\limits_{k=1}^\infty
C_k\geqslant\varepsilon_0$ --- противоречие.
\end{proof}

\begin{defen}
Пусть $(\Omega, \mathfrak{a},\mu)$ --- пространство с мерой,
$f\in\mathcal{L}_1$. Функция $\nu$ называется \emph{произведением
функции $f$ и меры $\mu$}, если $\nu
A\p=\int\limits_A\!f(x)\,\mu(dx)$.
\end{defen}

\begin{predl}
Функция $\nu$ является счетно аддитивной мерой.
\end{predl}

\begin{proof}
Из свойств интеграла Лебега следует, что функция $\nu$ является
мерой. Проверим счетно аддитивность. Пусть
$A=\bigsqcup\limits_{k=1}^\infty A_k$, тогда
$\gamma_A(\omega)=\sum\limits_{k=1}^\infty\gamma_{A_k}(\omega)$ и
$f(x)\gamma_A(x)=\lim\limits_{n\to\infty}\sum\limits_{k=1}^n\gamma_{A_k}(x)f(x)$.
Отсюда $$\Big|\sum\limits_{k=1}^n\gamma_{A_k}(x)f(x)\Big|\leqslant
\sum\limits_{k=1}^n\gamma_{A_k}(x)|f(x)|\leqslant|f(x)|\in\mathcal{L}_1,$$
и по теореме Лебега
\begin{multline*}
\nu
A=\int\limits_A\!f(x)\,\mu(dx)=\int\limits_\Omega\!\gamma_A(x)f(x)\,\mu(dx)=\\
=\lim\limits_{n\to\infty}\sum\limits_{k=1}^n\int\limits_\Omega\!\gamma_{A_k}(\omega)
f(\omega)\,\mu(d\omega)=\lim\limits_{n\to\infty}\sum\limits_{k=1}^n\nu
A_k=\sum\limits_{k=1}^\infty \nu A_k,
\end{multline*}
что и требовалось.
\end{proof}

Отсюда следует, что $\nu\ll \mu$: если $\mu A=0$, то $\nu
A=\int\limits_A\!f(x)\,\mu(dx)=0$.

\begin{theorem}[свойство абсолютной непрерывности интеграла Лебега]
$\forall\,\varepsilon>0\quad\exists\,\delta>0:\forall\,A\in\mathfrak{a}\quad
\mu
A<\delta\Rightarrow\big|\int\limits_A\!f(x)\,\mu(dx)\big|<\varepsilon$.
\end{theorem}

\begin{proof}
$\big|\int\limits_A\!f(x)\,\mu(dx)\big|\leqslant\int\limits_A\!|f(x)|\,\mu(dx)$.
Рассмотрим \emph{вариацию меры $\nu$}:
$\|\nu\|A=\int\limits_A\!|f(x)|\,\mu(dx)$, тогда по критерию
абсолютной непрерывности получаем
$$\Big|\int\limits_A\!f(x)\,\mu(dx)\Big|\leqslant\int\limits_A\!|f(x)|\,\mu(dx)<\varepsilon,$$
что и требовалось.
\end{proof}

\begin{predl}
Пусть $f\geqslant 0$, тогда
$g\in\mathcal{L}_1(\Omega,\mathfrak{a},\nu)$ $\Leftrightarrow$
$g-f\p\in\mathcal{L}_1(\Omega,\mathfrak{a},\mu)$ и тогда
$\int\limits_\Omega\!g(\omega)\,\nu(d\omega)=\int\limits_\Omega\!g(\omega)f(\omega)\,\mu(d\omega)$.
\end{predl}

\begin{proof}
Достаточно доказать это утверждение для случая, когда $g=\gamma_C$
--- индикатор некоторого множества $C$. Но тогда по определению
произведения функции и меры
$$\int\limits_\Omega\!g(\omega)\,\nu(d\omega)=\nu C=\int\limits_C\!f(\omega)\,\mu(d\omega)=
\int\limits_\Omega\!g(\omega)f(\omega)\mu(d\omega),$$ что и
требовалось.
\end{proof}

Рассмотрим теперь важный пример. Пусть $F\colon [a;b]\to [c;d]$ ---
произвольная монотонно возрастающая непрерывная функция, такая, что
$F(a)=c$ и $F(b)=d$. Тогда на $\sigma$-алгебре $\mathfrak{B}$
борелевских множеств отрезка $[a;b]$ есть мера $\nu_F$, такая, что
$\nu_F(\alpha;\beta)=F(\beta)-F(\alpha)$. Можно рассмотреть образ
меры $\nu_F$ на отрезке $[c;d]$, а именно меру
$F_*\nu_F=\nu_FF^{-1}$. Это
--- стандартная мера Лебега на $[c;d]$: $$(\nu_FF^{-1})(\gamma;\delta)=\nu_F(F^{-1}
(\gamma;\delta))=F(F^{-1}(\delta))-F(F^{-1}(\gamma))=\delta-\gamma.$$

Пусть теперь $g\in\mathcal{L}_1([c;d],\mathfrak{a},\nu_L)$, тогда
$$\int\limits_{[c;d]}\!g(x)\,\nu_L(dx)=\int\limits_{[c;d]}\!g(x)(\nu_FF^{-1})(dx)=\int\limits
_{[a;b]}\!g(F(x))\,\nu_F(dx).$$ Но
$\nu_F(\alpha;\beta)=\int\limits_{[\alpha;\beta]}\!F'(x)\,dx$,
поэтому $\nu_F(A)=\int\limits_A\!F'(x)\,dx$, и поскольку
$\nu_F\p=F'\nu_L$, то
$$\int\limits_{[c;d]}\!g(x)\,\nu_L(dx)=\int\limits_{[a;b]}\!g(F(x))F'(x)\,dx$$
--- стандартная формула замены переменной.

 \tema{Пространство $\mathcal{L}_p$}

\begin{defen}
\emph{Пространство $\mathcal{\bar{L}}_p$} --- это пространство всех
измеримых функций $f$, таких, что $|f|^p\in\mathcal{L}_1$. При этом
$1\leqslant p<\infty$.
\end{defen}

\begin{predl}
Пространство $\mathcal{\bar{L}}_p$ является линейным.
\end{predl}

\begin{proof}
Достаточно доказать, что если $f,g\in\mathcal{\bar{L}}_p$, то
$|f+g|^p\p\in\mathcal{L}_1$. Для этого докажем следующее
неравенство:
$$|f+g|^p\leqslant C_p(|f|^p+|g|^p),$$ где $C_p$ --- некоторая
константа.

Рассмотрим функцию $\psi(t)=\frac{(1+t)^p}{1+t^p}$. Эта функция
непрерывна на луче $[0;\infty)$, $\psi(0)=1$,
$\lim\limits_{t\to\infty} \psi(t)=1$. Отсюда следует, что функция
$\psi(t)$ ограничена: $\exists\, C_p: |\psi(t)|\leqslant C_p$.
Подставим $t=\frac a b$: $(a+b)^p\leqslant C_p(a^p+b^p)$, откуда
$|f+g|^p\leqslant(|f|+|g|)^p\leqslant C_p(|f|^p+|g|^p)$, что и
требовалось.
\end{proof}

\begin{defen}
\emph{Пространство
$\mathcal{L}_p=\mathcal{\bar{L}}_p/\mathcal{\bar{L}}_p^0$}, где
$\mathcal{\bar{L}}_p^0=\{f\in\mathcal{\bar{L}}_p\mid
\int\limits_\Omega\!|f(\omega)|^p\,\nu(d\omega)=0\}$.

Величина
$\|f\|_p=\big(\int\limits_\Omega\!|f(\omega)|^p\,\nu(d\omega)\big)^{1/p}$
является \emph{нормой} в $\mathcal{L}_p$.
\end{defen}

\begin{theorem}[неравенство Гельдера]
$\int\limits_\Omega\!|f(\omega)||g(\omega)|\,\nu(d\omega)\leqslant\|f\|_p\|g\|_q$,
где $f\in\mathcal{L}_p$, $g\in\mathcal{L}_q$ и $\frac 1 p+\frac 1
q=1$.
\end{theorem}

\begin{proof}
Воспользуемся неравенством Гельдера в виде
$ab\p\leqslant\frac{a^p}{p}\p+\frac{b^q}{q}$ при $\frac 1 p+\frac 1
q=1$, доказанному в курсе математического анализа. Подставим в него
$a=\frac{|f|}{\|f\|_p}$ и $b=\frac{|g|}{\|g\|_q}$. Имеем:
$\frac{|f|}{\|f\|_p}\cdot
\frac{|g|}{\|g\|_q}\leqslant\frac{|f|^p}{p\|f\|_p^p}+\frac{|g|^q}{q\|g\|_q^q}$,
откуда $$\int\limits_\Omega\!\frac{|f|\cdot|g|}{\|f\|_p\cdot
\|g\|_q}\,\nu(dx)\leqslant\frac 1
p\int\limits_\Omega\!\frac{|f|^p}{\|f\|_p^p}\,\nu(dx)+\frac 1
q\int\limits_\Omega\!\frac{|g|^q}{\|g\|_q^q}\,\nu(dx)=\frac 1
p+\frac 1 q=1,$$ что и требовалось.
\end{proof}

\begin{theorem}[неравенство Минковского]
$\|f+g\|_p\leqslant\|f\|_p+\|g\|_p$, где $f,g\in\mathcal{L}_p$.
\end{theorem}

\begin{proof}
Пусть $q=\frac{p}{p-1}$. Воспользуемся неравенством Гельдера, тогда
\begin{multline*}
\int\limits_\Omega\!|f+g|^p\,\nu(dx)\leqslant\int\limits_\Omega\!(|f|+|g|)^p\,\nu(dx)
=\int\limits_\Omega\!(|f|+|g|)^{p-1}|f|\,\nu(dx)+\\+\int\limits_\Omega\!(|f|+|g|)^{p-1}|g|
\,\nu(dx)\leqslant\|f\|_p\|(|f|+|g|)^{p/q}\|_q+\|g\|_p\|(|f|+|g|)^{p/q}\|_q.
\end{multline*}
Поскольку
$$\|(|f|+|g|)^{p/q}\|_q=\Big(\int\limits_\Omega\!(|f|+|g|)^{p}\,\nu(dx)\Big)^{1/q}=\|(|f|+|g|)\|_p^{p/q},$$
то, поделив обе части предыдущего неравенства на
$\|(|f|+|g|)\|_p^{p/q}$, получаем
$$\|f+g\|_p^{p-1/q}=\|f+g\|_p\leqslant\|f\|_p+\|g\|_p,$$ что и
требовалось.
\end{proof}

\begin{sled}
Величина
$\|f\|_p=\big(\int\limits_\Omega\!|f(\omega)|^p\,\nu(d\omega)\big)^{1/p}$
является нормой в $\mathcal{L}_p$.\hfill{$\square$}
\end{sled}
%-----------------------------------------------------------------------------------%

%---------------------------Lecture 12-----------------------------------------------%
\lecture

\begin{theorem}
Пространство $\mathcal{L}_p$ полно.
\end{theorem}

\begin{proof}
Пусть $\varphi_n\in\mathcal{L}_p$ --- фундаментальная
последовательность, т.е. $\forall\,\varepsilon>0\quad\exists\,
n_0:\forall\,k,m> n_0\quad \|\varphi_k-\varphi_m\|_p<\varepsilon$.
Необходимо доказать, что $\exists\,\varphi\in\mathcal{L}_p:
\|\varphi_n-\varphi\|_p\to 0$ при $n\to\infty$.

Мы будем считать, что $\nu\Omega<\infty$. Тогда если
$\psi\in\mathcal{L}_p$, то $\psi\in\mathcal{L}_1$:
$$\int\limits_\Omega\!|\psi(\omega)|\,\nu(d\omega)\leqslant \int\limits_\Omega
\!|\psi(\omega)|\cdot \mathbf{1}\,\nu(d\omega)\leqslant
C\|\psi\|_p<\infty$$ согласно неравенству Гельдера\footnote{Здесь и
далее считается, что $\frac 1 p+\frac 1 q=1$.} (здесь $\mathbf{1}$
--- это единичная функция, а $C\p=\|\mathbf{1}\|_q$).

Выберем последовательность $\{\varepsilon_n\}$, такую, что
$\varepsilon_n>0$ и $\sum\limits_{n=1}^\infty \varepsilon_n<\infty$.
Тогда $\forall\,\varepsilon_j\quad\exists\,n_j=n(\varepsilon_j):
\forall\,k,m\geqslant n_j\quad
\|\varphi_k-\varphi_m\|_p<\varepsilon_j$.

Рассмотрим ряд
$\varphi_{n_1}+(\varphi_{n_2}-\varphi_{n_1})+(\varphi_{n_3}-\varphi_{n_2})+
(\varphi_{n_4}-\varphi_{n_3})+\ldots$. Поскольку
$\|\varphi_{n_{j+1}}-\varphi_{n_j}\|_p<\varepsilon_j$, то
$$\int\limits_\Omega\!|\varphi_{n_1}|\,d\nu+\int\limits_\Omega\!|\varphi_{n_2}-
\varphi_{n_1}|\,d\nu+\ldots\leqslant C\sum\limits_{n=1}^\infty
\varepsilon_n+\int\limits_\Omega\!|\varphi_{n_1}|\,d\nu<\infty,$$ и
по теореме Леви
$$\int\limits_\Omega\!\Big(|\varphi_{n_1}+\sum\limits_{j=1}^\infty
|\varphi_{n_{j+1}}-\varphi_{n_j}||\Big)\,d\nu<\infty,$$ откуда
получаем, что ряд
$\varphi_{n_1}+(\varphi_{n_2}-\varphi_{n_1})+\ldots$ сходится
абсолютно почти всюду. Значит,
$\varphi_{n_1}+\sum\limits_{j=1}^k(\varphi_{n_{j+1}}-\varphi_{n_j})=\varphi_{n_{k+1}}$
и $\varphi_{n_k}(\omega)\to\varphi(\omega)$ почти всюду при
$k\to\infty$. Докажем, что $\varphi\in\mathcal{L}_p$ и
$\|\varphi-\varphi_{n_j}\|_p\to 0$ при $j\to\infty$.

Поскольку
$$\|\varphi_{n_{j+k}}-\varphi_{n_j}\|_p^p=\int\limits_\Omega\! |\varphi_{n_{j+k}} -
\varphi_{n_j}|^p\,d\nu<\varepsilon_j^p,$$ то по теореме Фату
$\|\varphi_{n_j}-\varphi\|_p^p=\int\limits_\Omega\!
|\varphi_{n_j}-\varphi|^p\,d\nu<\varepsilon_j^p$, откуда
$\|\varphi-\varphi_{n_j}\|_p\p<\varepsilon_j$. Значит,
$\varphi-\varphi_{n_j}\in\mathcal{L}_p$ и в силу линейности
$\varphi\in\mathcal{L}_p$. Кроме того, $\varepsilon_j\to 0$ при
$j\to\infty$, откуда следует, что $\|\varphi-\varphi_{n_j}\|_p\to 0$
при $j\to\infty$.

Из неравенства треугольника немедленно следует, что
$\varphi_n\to\varphi$ в пространстве $\mathcal{L}_p$, что и
требовалось.
\end{proof}

\begin{zam}
Приведенное доказательство справедливо при $p>1$, однако легко
видеть, что при $p=1$ теорема также верна.
\end{zam}

Пусть теперь $\varphi\in\mathcal{L}_p$ и $\psi\in\mathcal{L}_q$.
Тогда в силу неравенства Гельдера
$$\Big|\int\limits_\Omega\! \varphi\psi\,d\nu\Big|\leqslant \int\limits_
\Omega \!|\varphi\psi|\,d\nu\leqslant
\|\varphi\|_p\cdot\|\psi\|_q.$$ Отсюда следует, что линейный
функционал $F_\psi\colon \varphi\mapsto
\int\limits_\Omega\!\varphi\psi\,d\nu$ на пространстве
$\mathcal{L}_p$ непрерывен в 0: в самом деле, если
$\|\varphi_n\|_p\to 0$, то
$|F_\psi(\varphi_n)|\p\leqslant\|\varphi_n\|_p\cdot\|\psi\|_q\to 0$.
В силу линейности этот функционал непрерывен на всем пространстве
$\mathcal{L}_p$.

\begin{theorem}[Рис]
Пусть $F$ --- непрерывный линейный функционал на пространстве
$\mathcal{L}_p$. Тогда $\exists\, \psi_F\in\mathcal{L}_q:
\forall\,\varphi\in\mathcal{L}_p\quad
F(\varphi)=\int\limits_\Omega\!\varphi\psi_F\,d\nu$.\footnote{Доказательство
этой теоремы будет дано в курсе функционального анализа независимо
от теоремы Радона-Никодима.}
\end{theorem}

\begin{theorem}[Радон-Никодим]
Пусть $(\Omega,\mathfrak{a})$ --- измеримое пространство, $\nu$ и
$\mu$ --- меры на $\mathfrak{a}$, причем $\mu\ll \nu$. Тогда
$\exists\,\psi\in\mathcal{L}_1(\Omega,\mathfrak{a},\nu):
\mu=\psi\cdot\nu$.
\end{theorem}

\begin{proof}
Рассмотрим меру $\eta=\mu+\nu$ (т.е.
$\forall\,A\in\mathfrak{a}\quad\eta A=\mu A\p+\nu A$) и линейный
функционал $F$ на пространстве
$\mathcal{L}_2(\Omega,\mathfrak{a},\eta)$, такой, что
$F(\varphi)=\int\limits_\Omega\!\varphi(\omega)\,\mu(d\omega)$. По
неравенству Гельдера при $p=q=2$
$$\|\varphi\|_1=\int\limits_\Omega\!|\varphi(\omega)|\,\eta(d\omega)\leqslant
\|\varphi\|_2\cdot\|\mathbf{1}\|_2,$$ откуда
$|\varphi|\in\mathcal{L}_1(\Omega,\mathfrak{a},\eta)$. Поскольку
$$\infty>\int\limits_\Omega\!|\varphi(\omega)|\,\eta(d\omega)=
\int\limits_\Omega\!|\varphi(\omega)|\,\mu(d\omega)+\int\limits_\Omega\!|\varphi(\omega)|\,\nu(d\omega),$$
то $\int\limits_\Omega\!|\varphi(\omega)|\,\mu(d\omega)<\infty$ и
$$\int\limits_\Omega\!|\varphi(\omega)|\,\eta(d\omega)\geqslant
\int\limits_\Omega\!|\varphi(\omega)|\,\mu(d\omega)\geqslant
\Big|\int\limits_\Omega\!\varphi(\omega)\,\mu(d\omega)\Big|=|F(\varphi)|,$$
а значит, $|F(\varphi)|\leqslant\|\varphi\|_2\cdot\|\mathbf{1}\|_2$,
откуда следует, что функционал $F$ непрерывен на пространстве
$\mathcal{L}_2(\Omega,\mathfrak{a},\eta)$. Поэтому по теореме Риса
получаем, что $\exists\,g\in\mathcal{L}_2(\Omega,\mathfrak{a},\eta):
\forall\,\varphi\in\mathcal{L}_2(\Omega,\mathfrak{a},\eta)\quad
F(\varphi)=\int\limits_\Omega\!|\varphi(\omega)|\,\mu(d\omega)\p=
\int\limits_\Omega\!|\varphi(\omega)|g(\omega)\,\eta(d\omega)=
\int\limits_\Omega\!|\varphi(\omega)|g(\omega)\,\mu(d\omega)+
\int\limits_\Omega\!|\varphi(\omega)|g(\omega)\,\nu(d\omega)$.

Проверим, что $\eta A_0=\eta\{\omega\mid g(\omega)\geqslant 1\}=0$.
Пусть $\varphi=\gamma_{A_0}$ --- индикатор множества $A_0$. Тогда
$$\int\limits_\Omega\!\gamma_{A_0}(\omega)\,\mu(d\omega)=\mu A_0=
\int\limits_{A_0}\!g(\omega)\,\mu(d\omega)+\int\limits_{A_0}\!g(\omega)\,\nu(d\omega)
\geqslant \mu A_0+\nu A_0,$$ откуда следует, что $\nu A_0=0$.
Поскольку $\mu\ll\nu$, то $\mu A_0=0$, а значит, и $\eta A_0=0$.

Проверим теперь, что $\eta A_1=\eta\{\omega\mid g(\omega)<0\}$.
Аналогично,
$$0\leqslant\int\limits_\Omega\!\gamma_{A_1}(\omega)\,\mu(d\omega)=\mu A_1=
\int\limits_{A_1}\!g(\omega)\,\eta(d\omega)\leqslant 0,$$ откуда
$\eta A_1=0$.

Таким образом,
\begin{multline*}
F(\varphi)=\int\limits_{\Omega\setminus
A_0}\!\varphi(\omega)\,\mu(d\omega)=\int\limits_{\Omega\setminus
A_0}\!\varphi(\omega)g(\omega)\,\eta(d\omega)=\\=\int\limits_{\Omega\setminus
A_0}\!\varphi(\omega)g(\omega)\,\mu(d\omega)+\int\limits_{\Omega\setminus
A_0}\!\varphi(\omega)g(\omega)\,\nu(d\omega).
\end{multline*}
По теореме Леви получаем, что это равенство верно при всех
$\varphi\geqslant 0$. Отсюда $\int\limits_{\Omega\setminus
A_0}\!(\mathbf{1}-g(\omega))\varphi(\omega)\,\mu(d\omega)=\int\limits_{\Omega\setminus
A_0}\!\varphi(\omega)g(\omega)\,\nu(d\omega)$.

Пусть $A\in\mathfrak{a}$. Положим
$\varphi=\frac{\gamma_A}{\mathbf{1}-g}$ и
$\psi=\frac{g}{\mathbf{1}-g}\geqslant 0$, тогда
$\int\limits_\Omega\!\gamma_A(\omega)\,\mu(d\omega)\p=\int\limits_\Omega\!\gamma_A(\omega)
\psi(\omega)\,\nu(d\omega)$, т.е. $\mu
A=\int\limits_A\!\psi(\omega)\,\nu(d\omega)$.

Осталось доказать, что
$\psi\in\mathcal{L}_1(\Omega,\mathfrak{a},\nu)$. Для этого положим
$A=\Omega$. Получаем, что
$\infty>\mu\Omega=\int\limits_\Omega\!\psi(\omega)\,\nu(d\omega)$,
откуда все и следует.
\end{proof}

\tema{Теоремы Фубини, Хана-Жордана и Лебега}

Пусть $(\Omega_1,\mathfrak{a}_1,\nu_1)$ и
$(\Omega_2,\mathfrak{a}_2,\nu_2)$ --- пространства с мерой.

\begin{defen}
\emph{Произведением измеримых пространств} называется измеримое
пространство $(\Omega_1\times\Omega_2,
\mathfrak{a}_1\otimes\mathfrak{a}_2)$, где $\Omega_1\times
\Omega_2=\{(\omega_1,\omega_2)\mid \omega_j\in\Omega_j;j=1,2\}$ и
$\mathfrak{a}_1\otimes\mathfrak{a}_2$ --- $\sigma$-алгебра,
определяемая следующим образом. Возьмем полукольцо
$\mathcal{P}=\{A_1\times A_2\mid A_j\in\mathfrak{a}_j;j=1,2\}$
(элементы которого называются \emph{прямоугольниками}), тогда
$\mathfrak{a}_1\otimes \mathfrak{a}_2=\sigma(\mathcal{P})$.

\emph{Произведение пространств с мерой} --- это произведение
измеримых пространств, на котором задана мера $\nu_1\otimes\nu_2$
(называющаяся \emph{произведением мер $\nu_1$ и $\nu_2$}),
определяемая следующим образом. Если $A_1,A_2\in\mathcal{P}$, то
$(\nu_1\otimes \nu_2)(A_1\times A_2)=\nu_1 A_1\cdot\nu_2 A_2$.

\begin{predl}
Произведение мер $\nu_1\otimes\nu_2$ счетно аддитивно.
\end{predl}

\begin{proof}
Пусть $A_1\times A_2=\bigsqcup\limits_{k=1}^\infty A_1^k\times
A_2^k$ и $f_{A_1A_2}(\omega_1)=\nu_2
A_2\p\cdot\gamma_{A_1}(\omega_1)$, тогда в силу счетной аддитивности
меры $\nu_2$ имеем: $f_{A_1A_2}(\omega_1)\p=\sum\limits_{k=1}^\infty
f_{A_1^kA_2^k}(\omega_1)$. Поэтому по теореме Леви получаем, что
\begin{multline*}
(\nu_1\otimes\nu_2)(A_1\times A_2)=\int\limits_{\Omega_1}\!f_{A_1
A_2}(\omega_1)\,\nu_1(d\omega_1)=\\
=\sum\limits_{k=1}^\infty\int\limits_{\Omega_1}
\!f_{A_1^kA_2^k}(\omega_1)\,\nu_1(d\omega_1)=\sum\limits_{k=1}^\infty
(\nu_1\otimes\nu_2)(A_1^k\times A_2^k),
\end{multline*}
что и требовалось.
\end{proof}

Отсюда следует, что полученную счетно аддитивную функцию можно
продолжить на всю $\sigma$-алгебру
$\mathfrak{a}_1\otimes\mathfrak{a}_2$. Полученная мера и называется
\emph{произведением мер $\nu_1$ и $\nu_2$}.
\end{defen}

\begin{zam}
Если меры $\nu_1$ и $\nu_2$ полны, то мера $\nu_1\otimes\nu_2$ может
не быть полной, поэтому иногда бывает удобно рассматривать
пополнение произведения пространств с мерой:
$(\Omega_1\times\Omega_2,\overline{\mathfrak{a}_1\otimes\mathfrak{a}_2},
\overline{\nu_1\otimes\nu_2})$.
\end{zam}

\begin{zam}
Если функция $f$ измерима на
$(\Omega_1\times\Omega_2,\mathfrak{a}_1\otimes\mathfrak{a}_2,\nu_1\otimes\nu_2)$,
то $\forall\,\omega_1\in\Omega_1$ функция $\omega\mapsto
f(\omega_1,\omega)$ измерима на $(\Omega_2,\mathfrak{a}_2,\nu_2)$. В
случае, когда рассматриваются пополненное пространство, необходимо
заменить выражение <<$\forall\,\omega_1\in\Omega_1$>> на <<для почти
всех $\omega_1\in\Omega_1$>>.
\end{zam}
%-----------------------------------------------------------------------------------%

%---------------------------Lecture 13-----------------------------------------------%
\lecture

\begin{theorem}[Фубини]
Пусть
$f\in\mathcal{L}_1(\Omega_1\times\Omega_2,\mathfrak{a}_1\otimes\mathfrak{a}_2,\nu_1\otimes\nu_2)$,
тогда для $\nu_1$-почти всех $\omega_1\in\Omega_1$ функция
$\omega\mapsto f(\omega_1,\omega)$ интегрируема на
$(\Omega_2,\mathfrak{a}_2,\nu_2)$, причем функция
$\omega_1\mapsto\int\limits_{\Omega_2}\!f(\omega_1,\omega_2)\,\nu_2(d\omega_2)$
является $\nu_1$-интегрируемой на $\Omega_1$ и
$$\int\limits_{\Omega_1\times\Omega_2}\!f(\omega_1,\omega_2)\,(\nu_1\otimes\nu_2)
(d\omega_1\times
d\omega_2)=\int\limits_{\Omega_1}\!\Big(\int\limits_{\Omega_2}\!f(\omega_1,\omega_2)
\,\nu_2(d\omega_2)\Big)\,\nu_1(d\omega_1).$$
\end{theorem}

\begin{proof}
Как обычно, достаточно доказать теорему для индикаторов: пусть
$f=\gamma_{A}$, где $A\in\mathfrak{a}_1\otimes\mathfrak{a}_2$. Пусть
$$g(\omega_2)=
\begin{cases}
\nu_1A_1,&\text{если $\omega_2\in A_2$;}\\
0,&\text{иначе},
\end{cases}$$
тогда, если $A=A_1\times A_2$, где $A_1\in\mathfrak{a_1}$ и
$A_2\in\mathfrak{a_2}$, то
\begin{multline*}
\int\limits_{\Omega_1\times\Omega_2}\!\gamma_A(\omega_1,\omega_2) \,
(\nu_1\otimes\nu_2)(d\omega_1\times
d\omega_2)=(\nu_1\otimes\nu_2)(A_1\times A_2)=\\
=\int\limits_{\Omega_2}\Big(\int\limits_{\Omega_1}\!\gamma_A(\omega_1,
\omega_2)\,\nu_1(d\omega_1)\Big)\,\nu_2(d\omega_2)=\int\limits_{\Omega_2}\!
g(\omega_2)\,\nu_2(d\omega_2)=\nu_1A_1\cdot\nu_2A_2.
\end{multline*}

Назовем множество \emph{допустимым}, если для его индикатора верна
теорема Фубини. Обозначим множество всех допустимых множеств через
$\mathcal{Q}$. Нам необходимо доказать, что
$\mathcal{Q}=\mathfrak{a}_1\otimes\mathfrak{a}_2$. Укажем некоторые
свойства множества $\mathcal{Q}$.

\svoy

1. $\Omega=\Omega_1\times\Omega_2\in\mathcal{Q}$ (очевидно).

2. $\forall\,B_j\in\mathcal{Q}:B_1\subset B_2\ldots\Rightarrow
\bigcup\limits_{j=1}^\infty B_j\in\mathcal{Q}$ (следует из теоремы
Леви).

3. $B_1\subset B_2\in\mathcal{Q}\Rightarrow B_2\setminus
B_1\in\mathcal{Q}$ (следует из линейности).

4. $\mathcal{P}\subset\mathcal{Q}$ (см. предыдущее равенство).

\begin{defen}
Совокупность множеств, удовлетворяющая свойствам 1, 2, 3 называется
\emph{$D$-системой множеств}.
\end{defen}

Докажем, что $D(\mathcal{P})=\mathfrak{a}_1\otimes\mathfrak{a}_2$
(легко видеть, что этого будет достаточно для завершения
доказательства). Это утверждение эквивалентно тому, что
$D(\mathcal{P})$ --- это $\sigma$-алгебра. Для этого нужно доказать,
что если $B_1,B_2\in \mathcal{Q}$, то $B_1\cap B_2\in\mathcal{Q}$.

Пусть $\mathcal{Q}_1=\{A\in D(\mathcal{P})\mid \forall\,B\in
\mathcal{P}\quad A\cap B\in D(\mathcal{P})\}$ и
$\mathcal{Q}_2=\{A\in D(\mathcal{P})\mid \forall\,B\in
D(\mathcal{P})\quad A\cap B\in D(\mathcal{P})\}$. Легко видеть, что
$\mathcal{Q}_1$ и $\mathcal{Q}_2$ --- $D$-системы, причем
$\mathcal{P}\subset\mathcal{Q}_1,\mathcal{Q}_2$. Но
$\mathcal{Q}_1\subset D(\mathcal{P})$, поэтому
$\mathcal{Q}_1=D(\mathcal{P})$ и $\mathcal{Q}_2=D(\mathcal{P})$.
Отсюда получаем, что $D(\mathcal{P})$ --- это $\sigma$-алгебра и
$D(\mathcal{P})=\mathfrak{a}_1\otimes\mathfrak{a}_2$.
\end{proof}

%-----------------------------------------------------------------------------------%

%---------------------------Lecture 14-----------------------------------------------%
\lecture

\begin{theorem}[Хан-Жордан]
Пусть $(\Omega,\mathfrak{a})$ --- измеримое пространство, $\nu$ ---
возможно знакопеременная\footnote{Обратите внимание!} мера.

1. $\exists\,\Omega_+,\Omega_-: \Omega_+\sqcup \Omega_-=\Omega$,
причем $\forall\, A\subset \Omega_+:A\in\mathfrak{a}\quad \nu
A\geqslant 0$ и $\forall\, A\subset \Omega_-:A\in\mathfrak{a}\quad
\nu A\leqslant 0$.

2. $\nu=\nu^+-\nu^-$, где меры $\nu^+$ и $\nu^-$ неотрицательны.
\end{theorem}

\begin{defen}
Разложение пространства $\Omega$, указанное в п.1, называется
\emph{разложением Хана}.
\end{defen}

\begin{defen}
Мера $\nu^+$ называется \emph{положительной}, а мера $\nu^-$ ---
\emph{отрицательной} вариацией меры $\nu$. Величина
$\|\nu\|=\nu^++\nu^-$ называется \emph{вариацией меры
$\nu$}\footnote{Иногда \emph{вариацией} называется величина
$\|\nu\|\Omega$. В таком случае отображение
$\nu\p\mapsto\|\nu\|\Omega$ является нормой.}.
\end{defen}

\begin{proof}
Вначале докажем п.2, исходя из п.1. Положим $\nu^+A\p=\nu(A\cap
\Omega_+)$ и $\nu^-A=-\nu(A\cap \Omega_-)$. Как легко видеть, что
$\nu^+$ и $\nu^-$, определенные таким образом, дают искомое
разложение.

\begin{zam}
Разложение меры в п.2 не является единственным. Однако, если
$\nu=\mu^+-\mu^-$, то $\mu^+A\geqslant\nu^+A$ и
$\mu^-A\geqslant\nu^-A$.
\end{zam}

Докажем теперь п.1. Назовем множество $C\in\mathfrak{a}$
\emph{отрицательным}, если $\forall\,A\in\mathfrak{a}\quad\nu(A\cap
C)\leqslant 0$, и \emph{положительным}, если
$\forall\,A\in\mathfrak{a}\quad\nu(A\cap C)\geqslant 0$.

Пусть $\alpha=\inf\{\nu C\mid C \text{--- отрицательно}\}$ и
$\{C_n\}$ --- последовательность измеримых множеств, таких, что $\nu
C_n\searrow \alpha$. Положим $A_0=\bigcup\limits_{n=1}^\infty C_n$.
Очевидно, что $A_0$ отрицательно. Докажем, что множество
$\Omega_+=\Omega\setminus A_0$ является положительным (тогда можно
взять $\Omega_-=A_0$).

Предположим противное: пусть найдется такое $\bar{A}\in\Omega_+$,
что $\nu\bar{A}<0$. Выберем последовательность $\{\varepsilon_j\}$,
такую, что $\varepsilon_j>0$ и $\varepsilon_j\searrow 0$. Тогда
множество $\bar{A}$ не может быть отрицательным, иначе его можно
присоединить к $A_0$, и тогда получится, что $\nu A_0<\alpha$.
Поэтому $\exists\,A_1\subset \bar{A}:\nu A_1>\varepsilon_{j_1}$.
Рассмотрим множество $\bar{A}\setminus A_1$. Проводя аналогичные
рассуждения, получаем, что $\exists\,A_2\subset\bar{A}\setminus
A_1:\nu A_2>\varepsilon_{j_2}$.

В результате мы получим последовательность попарно непересекающихся
множеств $\{A_k\}$, где $\nu A_k>\varepsilon_{j_k}$. Кроме того,
$\nu\Big(\bigsqcup\limits_{k=1}^\infty
A_k\Big)=\sum\limits_{k=1}^\infty \nu A_k\p<\infty$, поэтому $\nu
A_k\to0$ и $\varepsilon_{j_k}\to0$.

Имеем: $\nu\Big(\bar{A}\setminus\bigsqcup\limits_{k=1}^\infty
A_k\Big)<0$. Докажем, что множество
$\bar{A}\setminus\bigsqcup\limits_{k=1}^\infty A_k$ отрицательно.
Действительно, если бы это было не так, то нашлось бы множество
$D\subset\bar{A}\setminus\bigsqcup\limits_{k=1}^\infty A_k$, такое,
что $\nu D=\varepsilon>0$, а значит, нашлось бы такое $k$, что
$\varepsilon_{j_k}<\varepsilon$ --- противоречие с выбором $A_k$
(тогда надо было выбрать $D$).

Значит, множество $\bar{A}\setminus\bigsqcup\limits_{k=1}^\infty
A_k$ отрицательно, и его можно присоединить к $A_0$, в результате
чего получится, что $\nu A_0<\alpha$ --- противоречие.

Таким образом, можно выбрать $\Omega_-=A_0$ и
$\Omega_+=\Omega\setminus A_0$.
\end{proof}

Пусть $f$ --- интегрируемая функция,
$F(t)=\int\limits_{-\infty}^t\!f(\tau)\,d\tau$ и
$\mu=f\cdot\lambda$, где $\lambda$ --- стандартная мера Лебега.

\begin{theorem}[Лебег]
Функция $F(t)=\int\limits_{-\infty}^t\!f(\tau)\,d\tau$
дифференцируема почти всюду по $t$ и $F'(t)=f(t)$.
\end{theorem}

\begin{proof}

Для доказательства нам понадобятся следующие две леммы.

\begin{lemm}\label{lemm1}
Если $\{I_\alpha\}$ --- семейство интервалов,
$G=\bigcup\limits_\alpha I_\alpha$ и $\lambda G<\infty$, то можно
выбрать подсемейство попарно непересекающихся интервалов
$\{I_{\alpha_k}\}$, таких, что $\sum\limits_{k=1}^n\lambda
I_{\alpha_k}\geqslant \frac 1 4 \lambda G$.
\end{lemm}

\begin{proof}
Вначале докажем, что существует такой компакт $K\p\subset G$, что
$\lambda K\geqslant\frac 3 4\lambda G$. Поскольку
$G=\bigcup\limits_{n=1}^\infty((-n;n)\cap G)$, то можно считать, что
множество $G$ ограничено. Положим $K_\alpha=\{x\in G\mid
\rho(x,\mathbb{R}^1\setminus G)\geqslant\alpha\}$. Тогда все
$K_\alpha$ --- компакты, и $\bigcup\limits_{n=1}^\infty K_{1/n}=G$.
Значит, $\exists\,K=K_{n_0}:\lambda K\p\geqslant\frac 3 4\lambda G$.

Из любого покрытия компакта $K$ можно выбрать конечное подпокрытие
$J_1$,\ldots, $J_n$. Пусть $I_{\alpha_1}$ --- это то из множеств
$\{J_k\}$, у которого мера Лебега максимальна, $I_{\alpha_2}$ ---
множество с наибольшей мерой Лебега из тех, что не пересекаются с
$I_{\alpha_1}$, и т.д.

Докажем, что подпокрытие $\{I_{\alpha_k}\}$ --- искомое. Заменим
множества $I_{\alpha_i}$ интервалами $\hat{I}_{\alpha_i}$ с той же
серединой, но втрое большей длиной. Тогда $\bigcup\limits_r
\hat{I}_{\alpha_r}\supset K$, поэтому $$\lambda
K\leqslant\sum\limits_r\lambda
\hat{I}_{\alpha_r}=3\sum\limits_r\lambda I_{\alpha_r},$$ откуда
$\frac 1 4\lambda G\leqslant\sum\limits_r\lambda I_{\alpha_r}$, что
и требовалось.
\end{proof}

\begin{lemm}\label{lemm2}
Если $A\in\mathfrak{B}(\mathbb{R}^1)$ и $\mu A=0$, то $F'(t)=0$ для
почти всех $t\in A$.
\end{lemm}

\begin{proof}
Пусть $\delta,\varepsilon>0$. Т.к. $A\in\mathfrak{B}(\mathbb{R}^1)$,
то $\exists\,G_\delta\supset A:\mu G_\delta<\delta$. Положим
$I_h=(x-h;x+h)$ и $A_\varepsilon=\{x\in A\mid \lim\limits_{h\to
0}\sup\frac{\mu I_h}{h}>\varepsilon\}$. Тогда $\forall\,x\in
A_\varepsilon\quad\exists\,h:I_h\subset G_\delta$ и $\mu
I_h>\varepsilon h=\varepsilon\frac{\lambda I_h}{2}$, т.е. $2\mu
I_h>\varepsilon\lambda I_h$. По лемме~\ref{lemm1} объединение
$V=\bigcup\limits_h I_h$ содержит $A_\varepsilon$ и содержится в
$G_\delta$, причем $$\lambda A_\varepsilon\leqslant\lambda
V\leqslant 4\sum\limits_k\lambda I_{h_k}<\frac{8}{\varepsilon}
\sum\limits_k\mu I_{h_k}\leqslant\frac{8}{\varepsilon}\mu
G_\delta<\frac{8\delta}{\varepsilon}.$$ Отсюда получаем, что
$\lambda A_\varepsilon=0$, а значит, для почти всех $\varepsilon$
функция $F(t)$ дифференцируема, причем $F'(x)=\lim\limits_{h\to
0}\sup\frac{F(x+h)-F(x-h)}{2h}=0$, что и требовалось.
\end{proof}

Теперь докажем теорему Лебега. Пусть $\hat{F}(t)=\lim\limits_{h\to
0}\sup\frac{F(t+h)-F(t)}{h}$ и $\check{F}(t)=\lim\limits_{h\to
0}\inf\frac{F(t+h)-F(t)}{h}$. Согласно лемме~\ref{lemm2},
$\hat{F}(t)=0$ для почти всех $t\in\{x\mid f(x)=0\}$.

Для каждого $r\in\mathbb{R}^1$ положим
$F_r(t)=\int\limits_{-\infty}^t\!(f(\tau)-r)_+\,d\tau$, тогда
$\hat{F}_r(t)=0$ почти всюду на множестве $\{x\mid f(x)\leqslant
r\}$.

Т.к. $f\leqslant (f-r)_++r$, то $\hat{F}(t)\leqslant r$ на множестве
$\{x\mid f(x)\leqslant r\}$. Но
\begin{multline*}
\lambda\{x\mid f(x)<\hat{F}(x)\}=\lambda\bigcup\limits_r\{x\mid
f(x)\leqslant r< \hat{F}(x)\}\leqslant\\
\leqslant\sum\limits_r\lambda\{\{x\mid f(x)\leqslant r\}\cap\{x\mid
r<\hat{F}(x)\}\}=0,
\end{multline*}
откуда $\hat{F}(t)\leqslant f(t)$ почти всюду. Аналогично,
$\check{F}(t)\geqslant f(t)$ почти всюду. Таким образом,
$\check{F}(t)\geqslant
f(t)\geqslant\hat{F}(t)\geqslant\check{F}(t)$, а значит, почти всюду
$\exists\,F'(t)=f(t)$, что и требовалось.
\end{proof}

\tema{Интеграл Лебега-Стильтьеса}

Пусть $(\mathbb{R}^1,\mathfrak{B}(\mathbb{R}^1))$ --- измеримое
пространство.

\begin{defen}
Функция $F$ \emph{имеет ограниченную вариацию}, если
$\exists\,\nu\colon
\mathfrak{B}(\mathbb{R}^1)\to\mathbb{R}^1:\forall\,t\quad
F(t)=\nu(-\infty;t]$. Тогда $\nu(a;b]=F(b)-F(a)$ --- это \emph{мера
Лебега-Стильтьеса}. Функция $F$ называется \emph{абсолютно
непрерывной}, если $\exists\,f\in\mathcal{L}_1:\forall\,t\quad
F(t)=\int\limits_{-\infty}^t\!f(\tau)\,d\tau$.
\end{defen}

\begin{zam}
Абсолютно непрерывная функция всегда имеет ограниченную вариацию:
положим $\nu A=\int\limits_A\!f(x)\,dx$ и $F(t)=\nu(-\infty;t]$.

Из теоремы Лебега следует, что если мы знаем производную $F'(t)$
абсолютно непрерывной функции $F(t)$, то сама функция $F(t)$
однозначно восстанавливается:
$F(t)=\int\limits_{-\infty}^t\!F'(\tau)\,d\tau$.
\end{zam}

\begin{defen}
Пусть $f,g\colon \mathbb{R}^1\to\mathbb{R}^1$. \emph{Интегралом
Стильтьеса} называется величина
$\int\limits_a^b\!g(x)\,df(x)=\lim\limits_{n\to\infty}\sum\limits_{k=1}^n
g(\xi_k)(f(x_{k+1})-f(x_k))$, где $\{(x_k,\xi_k)\}$ --- отмеченное
разбиение отрезка $[a;b]$.
\end{defen}

\begin{theorem}[без доказательства]
Пусть $f$ --- функция ограниченной вариации \textup{(}т.е.
$f(t)=\nu_f(-\infty;t]$\textup{)}, а $g$ --- ограниченная функция.
Тогда интеграл Стильтьеса $\int\limits_a^b\!g(t)\,df(t)$ существует
тогда и только тогда, когда значение меры $\nu_f$ на множестве точек
разрыва функции $g$ равно 0. В этом случае
$\int\limits_a^b\!g(t)\,df(t)=\int\limits_a^b\!g(t)\,\nu_f(dt)$.\hfill{$\square$}
\end{theorem}

\begin{theorem}[без доказательства]
Интегралы $\int\limits_a^b\!f(t)\,dg(t)$ и
$\int\limits_a^b\!g(t)\,df(t)$ существуют или не существуют
одновременно, причем если они существуют, то верна формула
интегрирования по частям: $$\int\limits_a^b\!g(t)\,df(t)=
f(b)g(b)-f(a)g(a)-\int\limits_a^b\!f(t)\,dg(t).$$\hfill{$\square$}
\end{theorem}

\begin{theorem}
Пусть $f$ и $g$ --- функции ограниченной вариации, а также
$\mu_f(a;b]=f(b)-f(a)$ и $\nu_g(a;b]=g(b)-g(a)$. Тогда верна формула
интегрирования по частям:
$$\int\limits_{(a;b]}\!f(t)\,\nu_g(dt)=g(b)f(b)-g(a)f(a)-\int\limits_
{(a;b]}\!g(t-0)\,\mu_f(dt).$$
\end{theorem}

\begin{proof}
Теорема верна в силу следующей цепочки равенств:

\begin{align*}
\int\limits_{(a;b]}\!f(t)\,\nu_g(dt)&=\int\limits_{(a;b]}\!(f(a)+\mu_f(a;t])\,\nu_g(dt)=
f(a)g(t)\!\!\mid_a^b+\int\limits_{(a;b]}\!\mu_f(a;t]\,\nu_g(dt)=\\
&=f(a)(g(b)-g(a))+\int\limits_a^b\!\!\int\limits_a^b\!\gamma_E(t_1,t_2)\,(\nu_g\otimes\mu_f)(dt_1\times
dt_2)=\footnotemark[11]\\
&=f(a)(g(b)-g(a))-\int\limits_{(a;b]}\!g(t-0)\,\mu_f(dt)+g(t)f(t)\!\!\mid_a^b=\\
&=g(b)f(b)-g(a)f(a)-\int\limits_ {(a;b]}\!g(t-0)\,\mu_f(dt),
\end{align*}\footnotetext[11]{Здесь $E$ --- это треугольник на координатной плоскости $t_1Ot_2$
с вершинами в точках $(a;a)$, $(b;b)$ и $(b;a)$.}%
что и требовалось.
\end{proof}

\begin{zam}
Если функции $f$ и $g$ абсолютно непрерывны, то формула
интегрирования по частям записывается в следующем виде:
$$\int\limits_a^b\!f'(t)g(t)\,dt=f(b)g(b)-f(a)g(a)-\int\limits_a^b\!g'(t)f(t)\,dt.$$
Из этой формулы, в частности, следует, что функция $fg$ также
абсолютно непрерывна.
\end{zam}
%-----------------------------------------------------------------------------------%

\clearpage

\begin{center}
\textbf{\textsc{Приложение. \\ Экзаменационные вопросы.}}
\end{center}

\vspace{7pt}

\noindent 1. Продолжение конечно аддитивной меры с полукольца на
порожденное кольцо.

\noindent 2. Доказательство неравенства $|\nu^*A-\nu^*B|\leqslant
\nu^*(A\bigtriangleup B)$.

\noindent 3. Совпадение $\nu^*$ и $\nu$ на области определения меры
$\nu$.

\noindent 4. Доказательство того, что множество $\nu$-измеримых
подмножеств является $\sigma$-алгеброй.

\noindent 5. Доказательство того, что внешняя мера $\nu^*$ счетно
аддитивна на $\sigma$-алгебре $\nu$-измеримых подмножеств.

\noindent 6. Доказательство единственности продолжения счетно
аддитивной меры с алгебры на порожденную $\sigma$-алгебру.

\noindent 7. Счетная аддитивность меры Лебега (на полукольце
полуинтервалов вещественной прямой).

\noindent 8. Два определения измеримости вещественной функции на
измеримом пространстве и их эквивалентность.

\noindent 9. Доказательство того, что всякая измеримая функция
является пределом последовательности простых функций.

\noindent 10. Доказательство измеримости функции, являющейся
пределом сходящейся последовательности измеримых функций.

\noindent 11. Доказательство существования неизмеримого по Лебегу
подмножества вещественной прямой.

\noindent 12. Связь между сходимостью почти всюду и сходимостью по
мере.

\noindent 13. Критерий счетно аддитивной конечной меры на кольце
множеств.

\noindent 14. Определение интеграла Лебега для измеримых функций.

\noindent 15. Неравенство Чебышева.

\noindent 16. Теорема Беппо Леви.

\noindent 17. Теорема Фату-Лебега.

\noindent 18. Теорема Лебега о предельном переходе под знаком
интеграла.

\noindent 19. Теорема Фату.

\noindent 20. Доказательство того, что интегрируемая по Риману
функция интегрируема по Лебегу и интегралы совпадают.

\noindent 21. Связь между интегралом Лебега и несобственным
интегралом Римана.

\noindent 22. Замена переменной в интеграле Лебега.

\noindent 23. Равносильность двух определений абсолютной
непрерывности одной меры относительно другой.

\noindent 24. Счетная аддитивность и абсолютная непрерывность
интеграла Лебега.

\noindent 25. Теорема Радона-Никодима.

\noindent 26. Теорема Хана-Жордана.

\noindent 27. Неравенство Гельдера.

\noindent 28. Неравенство Минковского.

\noindent 29. Полнота пространства $\mathcal{L}_p$ при $p\geqslant
1$.

\noindent 30. Счетная аддитивность произведения двух счетно
аддитивных мер.

\noindent 31. Теорема Фубини.

\noindent 32. Дифференцирование интеграла Лебега с переменным
верхним пределом.

\noindent 33. Восстановление абсолютно непрерывной функции по ее
производной.

\noindent 34. Связь между интегралами Римана-Стильтьеса и
Лебега-Стильть\-еса (без доказательства).
\end{document}
